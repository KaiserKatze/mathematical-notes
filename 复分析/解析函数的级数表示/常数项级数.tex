\section{常数项级数}
\subsection{常数项级数的概念}
\begin{definition}
设有复数列\[
	\alpha_1,\alpha_2,\dotsc,\alpha_n,\dotsc,
\]
其中\(\alpha_n=a_n+\iu b_n\ (n=1,2,\dotsc)\),则表达式\[
	\alpha_1+\alpha_2+\dotsb+\alpha_n+\dotsb
\]
称为\DefineConcept{复常数项无穷级数},
简称\DefineConcept{复数项级数}或\DefineConcept{复级数},
记作\(\sum_{i=1}^\infty \alpha_i\),
即\[
	\sum_{i=1}^\infty \alpha_i = \alpha_1+\alpha_2+\dotsb+\alpha_n+\dotsb,
\]
其中第\(n\)项\(\alpha_n\)叫做级数的\DefineConcept{一般项}.

作复数项级数的前\(n\)项的和\[
	s_n = \alpha_1+\alpha_2+\dotsb+\alpha_n = \sum_{i=1}^n{u_i},
\]
称\(s_n\)为级数的\DefineConcept{部分和}.

如果级数\(\sum_{i=1}^\infty \alpha_i\)的部分和数列\(\{s_n\}\)有极限\(s=a+\iu b\),
即\[
	\lim_{n\to\infty}s_n = s,
\]
则称无穷级数\(\sum_{i=1}^\infty \alpha_i\) \DefineConcept{收敛},
这时极限\(s\)叫做这级数的\DefineConcept{和},并写成\[
	s = \alpha_1+\alpha_2+\dotsb+\alpha_n+\dotsb;
\]
如果\(\{s_n\}\)没有极限,
则称无穷级数\(\sum_{i=1}^\infty \alpha_i\) \DefineConcept{发散}.
\end{definition}

\subsection{常数项级数的性质}
\begin{theorem}\label{theorem:解析函数的级数表示.复级数与其实部及虚部级数的关系}
设\(\alpha_n=a_n+\iu b_n\ (n=1,2,\dots)\),则\[
	\sum_{i=1}^\infty \alpha_i = s = a + \iu b
	\iff
	\sum_{i=1}^\infty a_i = a,
	\sum_{i=1}^\infty b_i = b.
\]
\end{theorem}

根据\cref{theorem:解析函数的级数表示.复级数与其实部及虚部级数的关系}
以及\cref{theorem:无穷级数.收敛级数性质1,theorem:无穷级数.收敛级数性质2,theorem:无穷级数.收敛级数性质3,theorem:无穷级数.收敛级数性质4,theorem:无穷级数.级数收敛的必要条件},
可知收敛的复级数具有与实数项无穷级数相似的性质.
\begin{property}
收敛的复数项级数具有的性质包括:
\begin{enumerate}
	\item \(\sum_{i=1}^\infty \alpha_i\ \text{收敛}
		\implies
		\lim_{n\to\infty}\alpha_n=0\);

	\item \(\sum_{i=1}^\infty \alpha_i\ \text{收敛}
		\implies
		(\exists M > 0)[\abs{\alpha_n} \leq M \quad(n=1,2,\dotsc)]\);

	\item 若\(\sum_{i=1}^\infty \alpha_i=S_1\),
	\(\sum_{i=1}^\infty \beta_i=S_2\),
	则\[
		\sum_{i=1}^\infty (\alpha_i\pm\beta_i)=S_1+S_2,
	\]\[
		\sum_{i=1}^\infty c\alpha_i
		=c\sum_{i=1}^\infty \alpha_i
		=cS
		\quad(c\in\mathbb{C});
	\]

	\item 在复级数\(\sum_{i=1}^\infty \alpha_i\)中去掉、加上或改变有限项,
	不会改变其收敛性.
\end{enumerate}
\end{property}

\begin{definition}
若级数\(\sum_{n=1}^\infty \abs{\alpha_n}\)收敛,
则称“级数\(\sum_{n=1}^\infty \alpha_n\)~\DefineConcept{绝对收敛}(converge absolutely)”.
%@see: https://mathworld.wolfram.com/AbsoluteConvergence.html
不绝对收敛的收敛级数,称为\DefineConcept{条件收敛级数}(conditionally convergent series).
%@see: https://mathworld.wolfram.com/ConditionalConvergence.html
\end{definition}

参考\cref{theorem:无穷级数.绝对收敛级数必定收敛},
显然,在复级数中也有“绝对收敛级数一定收敛”.
级数\(\sum_{n=1}^\infty \abs{\alpha_n}\)的各项均为非负实数,
因此\(\sum_{n=1}^\infty \abs{\alpha_n}\)为实正项级数,
可按实正项级数的收敛性判别法则,
如\hyperref[theorem:无穷级数.正项级数的比较审敛法]{比较审敛法}、
\hyperref[theorem:无穷级数.正项级数的比值审敛法]{比值审敛法}、
\hyperref[theorem:无穷级数.正项级数的根值审敛法]{根值审敛法}等,
判断其收敛性.

参考\cref{theorem:无穷级数.绝对收敛级数的可交换性},
和实级数一样,
在复级数中也有“绝对收敛级数的各项可以重排顺序而不改变其绝对收敛性与和”.

\begin{definition}
设复级数\(\sum_{n=1}^\infty \alpha_n = S_1\)
和\(\sum_{n=1}^\infty \beta_n = S_2\),
称复级数\[
	\sum_{n=1}^\infty (
		\alpha_1 \beta_n + \alpha_2 \beta_{n-1} + \dotsc + \alpha_n \beta_1
	)
	= \sum_{n=1}^\infty
		\sum_{k=1}^n \alpha_k \beta_{n+1-k}
\]为级数\(\sum_{n=1}^\infty \alpha_n\)
和级数\(\sum_{n=1}^\infty \beta_n\)的\DefineConcept{柯西乘积},
记作\(\left( \sum_{n=1}^\infty \alpha_n \right) \cdot \left( \sum_{n=1}^\infty \beta_n \right)\).
\end{definition}

\begin{theorem}
若复级数\(\sum_{n=1}^\infty \alpha_n\)
和\(\sum_{n=1}^\infty \beta_n\)绝对收敛,
且分别收敛到\(S_1\)和\(S_2\),
则\[
	\left( \sum_{n=1}^\infty \alpha_n \right)
	\cdot \left( \sum_{n=1}^\infty \beta_n \right)
	= S_1 \cdot S_2.
\]
\end{theorem}

\begin{example}[复等比级数]\label{example:解析级数的级数表示.复等比级数}
讨论复级数\(\sum_{n=0}^\infty z^n\)的收敛性.
\begin{solution}
分两种情形讨论:\begin{enumerate}
	\item 当\(\abs{z} < 1\)时,
	正项级数\(\sum_{n=0}^\infty \abs{z}^n\)收敛,
	故原级数绝对收敛.
	原级数的部分和为\[
		S_n
		= 1 + z + z^2 + \dotsb + z^{n-1} + z^n
		= \frac{1-z^{n+1}}{1-z},
	\]
	那么\[
		\sum_{n=0}^\infty z^n
		= \lim_{n\to\infty} S_n
		= \lim_{n\to\infty} \frac{1-z^{n+1}}{1-z}
		= \frac{1}{1-z}.
	\]

	\item 当\(\abs{z} \geq 1\)时,
	\(\abs{z}^n \geq 1\),
	所以一般项\(z^n \not\to 0\ (n\to\infty)\),
	级数发散.
\end{enumerate}
\end{solution}
\end{example}
