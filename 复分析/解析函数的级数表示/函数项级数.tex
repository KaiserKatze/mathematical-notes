\section{函数项级数}
\subsection{函数项级数的概念}
参照\hyperref[definition:无穷级数.实函数项级数的概念]{实函数项级数的定义},可以定义出复函数项级数的概念如下:
\begin{definition}\label{definition:解析函数的级数表示.收敛级数}
设点集\(E \subseteq \mathbb{C}\).
如果给定一个定义在点集\(E\)上的复变函数列\begin{equation*}
	f_1(z), f_2(z), \dotsc, f_n(z), \dotsc,
\end{equation*}
则由该函数列构成的表达式\begin{equation*}
	f_1(z) + f_2(z) + \dotsb + f_n(z) + \dotsb
\end{equation*}
称为定义在点集\(E\)上的\DefineConcept{复函数项无穷级数},
简称\DefineConcept{复函数项级数}.

如果\(\sum_{i=1}^\infty f_i(z)\)
对于\(\forall z_0 \in E\)
都有\(\sum_{i=1}^\infty f_i(z_0)\)收敛,
则称“\(\sum_{i=1}^\infty f_i(z)\)在点集\(E\)上收敛”,
称复变函数\(F(z) = \sum_{i=1}^\infty f_i(z)\)为它的\DefineConcept{和函数}.
\end{definition}
用“\(\epsilon-\delta\)”语言来说,
级数\(\sum_{i=1}^\infty f_i(z)\)
在点\(z \in E\)处
收敛于\(F(z)\)的充分必要条件是:\begin{equation*}
	(\forall \epsilon > 0)
	(\exists N \in \mathbb{N}^+)
	[
		n > N
		\implies
		\abs{F(z) - \sum_{i=1}^n f_i(z)} < \epsilon
	].
\end{equation*}

\subsection{级数的一致收敛}
通常来说,复函数项级数是否收敛依赖于其定义域\(E\).
\begin{definition}\label{definition:解析函数的级数表示.一致收敛级数}
设定义在点集\(E\)上的复函数项级数\(\sum_{i=1}^\infty f_i(z)\)的部分和函数为\(S_n(z)\),
若在点集\(E\)存在函数\(F(z)\),使得\begin{equation*}
	(\forall\epsilon>0)
	(\exists N \in \mathbb{N})
	(\forall z \in E)
	[
		n>N \implies \abs{F(z) - S_n(z)} < \epsilon
	],
\end{equation*}
则称“\(\sum_{i=1}^\infty f_i(z)\)在点集\(E\)上\DefineConcept{一致收敛}于\(F(z)\)”.
\end{definition}

比较\cref{definition:解析函数的级数表示.收敛级数,definition:解析函数的级数表示.一致收敛级数} 描述的两种收敛,
前者定义中的\(N\)不仅与\(\epsilon\)有关,
通常也与所讨论的点\(z \in E\)有关;
也就是说,当正数\(\epsilon\)取定后,
随点\(z\)不同,
正整数\(N\)通常也随之变化;
后者定义中的\(N = N(\epsilon)\),
却只随\(\epsilon\)变化而变化,
并不随\(z\)变化而变化.
因此,前者描述的收敛反映的是级数在\(E\)的局部性质,
后者描述的一致收敛反映的是级数在\(E\)中的整体性质.

\begin{theorem}[柯西一致收敛准则]\label{theorem:无穷级数.柯西一致收敛准则}
复函数项级数\(\sum_{i=1}^\infty f_i(z)\)在点集\(E\)上一致收敛于某函数的充分必要条件是:\begin{equation*}
	\begin{array}{r}
		(\forall\epsilon>0)
		(\exists N \in \mathbb{N}) \\
		(\forall z \in E)
		(\forall p \in \mathbb{N})
	\end{array}
	[
		\begin{array}{l}
			n > N \implies \\
			\hspace{20pt}
			\abs{f_{n+1}(z) + f_{n+2}(z) + \dotsb + f_{n+p}(z)} < \epsilon
		\end{array}
	].
\end{equation*}
\end{theorem}

类似于\cref{theorem:无穷级数.魏尔斯特拉斯判别法},
由\hyperref[theorem:无穷级数.柯西一致收敛准则]{柯西一致收敛准则}可得复函数项级数一致收敛的一个充分条件.
\begin{corollary}[Weierstrass M - 判别法,优级数准则]\label{theorem:无穷级数.优级数准则}
若复函数序列\(\{f_n(z)\}\)在点集\(E\)上有定义,
且存在正数列\(\{M_n\}\),
使得对\(\forall z \in E\),
有\begin{equation*}
	\abs{f_n(z)} \leq M_n
	\quad(n=1,2,\dotsc),
\end{equation*}
且正项级数\(\sum_{i=1}^\infty M_i\)收敛,
则复函数项级数\(\sum_{i=1}^\infty f_i(z)\)
在点集\(E\)上绝对收敛且一致收敛.
\begin{proof}
记复函数项级数\(\sum_{i=1}^\infty f_i(z)\)的部分和为\(S_n(z)\).
取\(M > N\),那么部分和\(S_n(z)\)满足\begin{equation*}
	\abs{S_M(z) - S_N(z)}
	= \abs{\sum_{i=N+1}^M f_i(z)}
	\leq \sum_{i=N+1}^M M_i.
\end{equation*}
因为正项级数\(\sum_{i=1}^\infty M_i\)收敛,
所以当\(N,M \to \infty\)时,
\(\sum_{i=N+1}^M M_i \to 0\),
可知\(\{S_n\}\)是一个一致收敛的柯西序列,
同时它也绝对收敛.
\end{proof}
\end{corollary}
这样的正项级数\(\sum_{i=1}^\infty M_i\)
称为复函数项级数\(\sum_{i=1}^\infty f_i(z)\)的\DefineConcept{优级数}.

\begin{example}
证明:级数\(\sum_{n=1}^\infty z^n\)
在闭圆\(\abs{z} \leq r\ (r<1)\)上一致收敛.
\begin{proof}
显然成立,因为级数\(\sum_{n=1}^\infty z^n\)有收敛的优级数\(\sum_{n=1}^\infty r^n\).
\end{proof}
\end{example}

\begin{definition}
设函数\(f_n(z)\ (n=1,2,\dots)\)在区域\(D\)内有定义.
若\(\sum_{n=1}^\infty f_n(z)\)在含于\(D\)内的任意一个有界闭区域\(d\)上都一致收敛,
则称级数\(\sum_{n=1}^\infty f_n(z)\)在\(D\) \DefineConcept{内闭一致收敛}.
\end{definition}
显然,若\(\sum_{i=1}^\infty f_i(z)\)在区域\(D\)内闭一致收敛,
则\(\sum_{i=1}^\infty f_i(z)\)在\(D\)内每一点都是收敛的,
但不一定在\(D\)上一致收敛.
自然,若\(\sum_{i=1}^\infty f_i(z)\)在\(D\)上一致收敛,
则\(\sum_{i=1}^\infty f_i(z)\)在\(D\)内闭一致收敛.

\begin{theorem}\label{theorem:解析函数的级数表示.一致收敛级数的基本性质1}
设级数\(\sum_{i=1}^\infty f_i(z)\)的各项在区域\(D\)内连续,
并且一致收敛于\(F(z)\),
则和函数\(F(z)\)也在\(D\)连续.
\end{theorem}

\begin{theorem}\label{theorem:解析函数的级数表示.一致收敛级数的基本性质2}
设级数\(\sum_{i=1}^\infty f_i(z)\)的各项在曲线\(C\)上连续,
并且在\(C\)上一致收敛于\(F(z)\),
则沿\(C\)可以逐项积分:\begin{equation*}
	\int_C F(z) \dd{z}
	= \sum_{i=1}^\infty \int_C f_i(z) \dd{z}.
\end{equation*}
\end{theorem}

\begin{theorem}[魏尔斯特拉斯定理]\label{theorem:解析函数的级数表示.魏尔斯特拉斯定理}
设级数\(\sum_{i=1}^\infty f_i(z)\)的各项在区域\(D\)内解析,
且\(\sum_{i=1}^\infty f_i(z)\)在\(D\)内闭一致收敛于\(F(z)\),
则\begin{enumerate}
	\item \(F(z)\)在\(D\)内解析;

	\item 在\(D\)内可以逐项求任意阶导数:\begin{equation*}
		F^{(m)}(z) = \sum_{i=1}^\infty f_i^{(m)}(z)
		\quad(m=1,2,\dotsc);
	\end{equation*}

	\item \(\sum_{i=1}^\infty f_i^{(m)}(z)\)在\(D\)内一致收敛于\(F^{(m)}(z)\).
\end{enumerate}
\begin{proof}
因为\(\sum_{i=1}^\infty f_i(z)\)在\(D\)内收敛,
所以和函数\(F(z)\)在\(D\)内有定义.
根据\cref{theorem:解析函数的级数表示.一致收敛级数的基本性质1},
\(F(z)\)在\(D\)内连续.

设\(K\)是\(D\)内的任一圆周,其内部属于\(D\).
又设\(\gamma\)是\(K\)内部的任一围线,
则由假定可知\(\sum_{i=1}^\infty f_i(z)\)在\(\gamma\)上一致收敛;
那么根据\cref{theorem:解析函数的级数表示.一致收敛级数的基本性质2},
得\begin{equation*}
	\int_\gamma F(z) \dd{z} = \sum_{i=1}^\infty \int_\gamma f_i(z) \dd{z}.
\end{equation*}
因为\(f_i(z)\)在\(K\)的内部解析,
根据\hyperref[equation:解析函数的积分表示.柯西积分公式]{柯西积分公式},
有\(\int_\gamma f_i(z) \dd{z} = 0\),
从而有\(\int_\gamma F(z) \dd{z} = 0\).
根据\hyperref[theorem:解析函数的积分表示.莫雷拉定理]{莫雷拉定理}可知,
\(f(z)\)在\(K\)的内部解析.
因为\(K\)是\(D\)内的任一圆周,
所以\(f(z)\)在\(D\)内解析.

\def\f#1{f\ifx\relax#1\relax\else_{#1}\fi^{(m)}(z_0) = \frac{m!}{2\pi\iu} \int_\Gamma \frac{f\ifx\relax#1\relax\else_{#1}\fi(\zeta)}{(\zeta-z_0)^{m+1}} \dd{\zeta}}
设\(z_0\)是\(D\)内任一点,
则\begin{equation*}
	(\exists \rho > 0)
	[\Set{ z \given \abs{z-z_0} \leq \rho } \subseteq D].
\end{equation*}
根据上述结果,
\(f(z)\)在闭圆\(\overline{K}\)上解析.
应用\cref{equation:解析函数的积分表示.柯西高阶导数公式},
得\begin{equation*}
	\def\arraystretch{1.5}
	\begin{array}{l}
	\f{} \\
	\f{n}
	\end{array}
	\quad(m=1,2,\dotsc),
\end{equation*}
其中\(\Gamma: \abs{\zeta-z_0}=\rho\).
由于\(\sum_{i=1}^\infty f_i(z)\)在\(D\)内闭一致收敛于\(f(z)\),
从而在圆周\(\Gamma\)上\begin{equation*}
	\frac{f(\zeta)}{(\zeta-z_0)^{m+1}}
	= \sum_{i=1}^\infty \frac{f_i(\zeta)}{(\zeta-z_0)^{m+1}}
\end{equation*}是一致收敛的.
于是由\cref{theorem:解析函数的级数表示.一致收敛级数的基本性质2} 得\begin{equation*}
	\int_\Gamma \frac{f(\zeta)}{(\zeta-z_0)^{m+1}} \dd{\zeta}
	= \sum_{i=1}^\infty \int_\Gamma \frac{f_i(\zeta)}{(\zeta-z_0)^{m+1}} \dd{\zeta},
\end{equation*}
两端同乘以\(\frac{m!}{2\pi\iu}\)就得到所要证明的结果.
\end{proof}
\end{theorem}

\begin{example}
试证函数\(f(z) = \sum_{n=1}^\infty \frac{1}{n^z}\)
在区域\(D = \Set{ z \given \Re z > 1 }\)内解析.
\begin{proof}
显然\(\sum_{n=1}^\infty \frac{1}{n^z}\)是解析函数项级数.
任取有界闭区域\(D' \subseteq D\).
设\(D'\)到直线\(\Re z = 1\)的距离为\(d\ (d > 0)\),
于是\begin{equation*}
	(\forall z \in D')
	[\Re z > 1+d],
\end{equation*}
从而\begin{equation*}
	\abs{n^{-z}} = \abs{e^{-z \ln n}}
	= e^{-(\Re z) \ln n} = n^{-\Re z}
	< n^{-(1+d)}.
\end{equation*}
因正项级数\(\sum_{n=1}^\infty n^{-(1+d)}\)收敛,
根据\hyperref[theorem:无穷级数.优级数准则]{优级数准则},
级数\(\sum_{n=1}^\infty n^{-z}\)在\(D' \subseteq D\)上一致收敛,
从而在\(D\)内闭一致收敛.
故由\hyperref[theorem:解析函数的级数表示.魏尔斯特拉斯定理]{魏尔斯特拉斯定理},
\(f(z) = \sum_{n=1}^\infty n^{-z}\)在\(D\)内解析.
\end{proof}
\end{example}
