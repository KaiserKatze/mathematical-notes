\section{解析函数在孤立奇点附近的性态}
\subsection{有限孤立奇点的情形}
\subsubsection{孤立奇点的分类}
现在我们利用罗朗展开对解析函数的孤立奇点进行分类,并讨论解析函数在各类孤立奇点附近的性态.
设\(a\)为函数\(f(z)\)的有限孤立奇点,
\(f(z)\)在去心邻域\(0<\abs{z-a}<R\)内的罗朗展式为\[
	f(z) = \sum_{n=-\infty}^\infty c_n (z-a)^n
	= \sum_{n=0}^\infty c_n (z-a)^n
	+ \sum_{n=1}^\infty c_{-n} (z-a)^{-n}.
\]
我们已经知道,
级数\(\sum_{n=0}^\infty c_n (z-a)^n\)
称为\(f(z)\)在点\(a\)的解析部分,
其和函数\(\phi(z)\)在包括点\(a\)的邻域\(\abs{z-a}<R\)内是解析的,
故\(f(z)\)在点\(a\)的奇异性质完全体现在\(f(z)\)的罗朗展式的复幂项部分.
这就是为什么级数\(\sum_{n=1}^\infty c_{-n} (z-a)^{-n}\)又被称为奇异部分.

现在根据奇异部分仅可能出现的三种情况,
将\(f(z)\)的有限孤立奇点作如下分类:
\begin{definition}
设\(a\)是\(f(z)\)的有限孤立奇点.
\begin{enumerate}
	\item 若\(f(z)\)在点\(a\)的奇异部分为零,
	则称“\(a\)是\(f(z)\)的\DefineConcept{可去奇点}”.

	{\footnotesize
	函数\(f(z)\)在可去奇点的去心邻域内的罗朗展式只有非负幂项的解析部分\[
		c_0 + c_1 (z-a) + c_2 (z-a)^2 + \dotsb,
	\]
	而这解析部分的和函数\(\phi(z)\)在包括点\(a\)在内的邻域\(K\)内解析,
	且\(\phi(a) = c_0\).
	于是,在这去心邻域内,\(f(z) = \phi(z)\).
	只要我们适当补充或改变\(f(z)\)在点\(a\)的定义,
	使得\(f(a) = \phi(a)\),
	那么\(f(z)\)在邻域\(N_R(a)\)内就没有奇点了.
	}

	\item 若\(f\)在点\(a\)的奇异部分为有限多项,
	并假设最小复幂项为\((z-a)\)的\((-m)\)次幂项,
	即\[
		\frac{c_{-m}}{(z-a)^m} + \frac{c_{-(m-1)}}{(z-a)^{m-1}}
		+ \dotsb + \frac{c_{-1}}{z-a}
		\quad(c_{-m}\neq0),
	\]
	则称“\(a\)是\(f\)的\(m\)阶\DefineConcept{极点}(pole)”.
	特别地,一阶极点又称为\DefineConcept{简单极点}(simple pole).

	\item 若\(f\)在点\(a\)的奇异部分有无限多项,
	则称“\(a\)是\(f\)的\DefineConcept{本性奇点}(essential singularity)”.
\end{enumerate}
\end{definition}

\begin{example}
点\(z=0\)是函数\(f(z) = \frac{\sin z}{z}\)的可去奇点.
只要补充定义\(f(z) = \lim_{z\to0} \frac{\sin z}{z} = 1\),
即规定\[
	f(z) = \left\{ \begin{array}{cc}
		\frac{\sin z}{z}, & z\neq0, \\
		1, & z=0,
	\end{array} \right.
\]
则\(f(z)\)在\(z=0\)就解析了.
\end{example}

\subsubsection{可去奇点的判定}
\begin{theorem}\label{theorem:解析函数的级数表示.可去奇点的特征}
若点\(a\)为\(f(z)\)的孤立奇点,
则下列几个命题是等价的:
\begin{enumerate}
	\item 点\(a\)是\(f(z)\)的可去奇点,\(f(z)\)在点\(a\)的奇异部分为零;
	\item \(\lim_{z \to a} f(z) = c_0\ (\abs{c_0} < +\infty)\);
	\item \(f(z)\)在点\(a\)的某去心邻域内有界.
\end{enumerate}
\end{theorem}

\subsubsection{极点的判定}
\begin{theorem}\label{theorem:解析函数的级数表示.极点的特征}
若\(a\)为\(f(z)\)的孤立奇点,则下列几个命题是等价的:
\begin{enumerate}
	\item 点\(a\)是\(f(z)\)的\(m\)阶极点,
	\(f(z)\)在点\(a\)的奇异部分为\[
		\frac{c_{-m}}{(z-a)^m}
		+ \frac{c_{-(m-1)}}{(z-a)^{m-1}}
		+ \dotsb + \frac{c_{-1}}{z-a}
		\quad(c_{-m}\neq0);
	\]

	\item \(f(z)\)在点\(a\)的某去心邻域内能表成\[
		f(z) = \frac{\lambda(z)}{(z-a)^m},
	\]
	其中\(\lambda(z)\)在点\(a\)的邻域内解析,
	且\(\lambda(a)\neq0\);

	\item 点\(a\)是\(g(z) = \frac{1}{f(z)}\)的可去奇点,
	将\(a\)作为\(g(z)\)的解析点看待时,
	点\(a\)为\(g(z)\)的\(m\)阶零点.
\end{enumerate}
\end{theorem}

\begin{corollary}\label{theorem:解析函数的级数表示.孤立奇点成为极点的充分必要条件1}
\(f(z)\)的孤立奇点\(a\)是极点的充分必要条件是:\[
	\lim_{z \to a} f(z) = \infty.
\]
\begin{proof}
“点\(a\)是\(f(z)\)的(\(m\)阶)极点”
等价于“点\(a\)是\(g(z) = \frac{1}{f(z)}\)的(\(m\)阶)零点”,
也就等价于“\(\lim_{z \to a} f(z) = \infty\)”.
\end{proof}
\end{corollary}
\cref{theorem:解析函数的级数表示.孤立奇点成为极点的充分必要条件1} 的缺点是不能指明极点的阶数.

\subsubsection{本性奇点的判定}
\begin{theorem}
\(f(z)\)的孤立奇点\(a\)是本性奇点的充分必要条件是:
不存在有限或无限的极限\(\lim_{z \to a} f(z)\).
\end{theorem}

\subsubsection{皮卡定理}
\begin{theorem}[皮卡定理]
解析函数在本性奇点的去心邻域内无穷多次地取到每个有限复值,
至多可能除一个值(称为\DefineConcept{皮卡例外值}).
\end{theorem}
皮卡定理深刻、准确地揭示了解析函数在其本性奇点邻近取值的奇异性.
上述定理还有另一种表述方式,即“任何不为常数的整函数,除去可能的一个值外,取遍所有(有限)的复数值.”

又因为整函数是在复平面上不取无穷远点的亚纯函数,
从而皮卡定理可以推广到任意的亚纯函数:
\begin{theorem}
\(\mathbb{C}\)上的任意不为常数的亚纯函数,
除去可能的两个值外,
取遍\(\overline{\mathbb{C}}\)的所有值.
\end{theorem}

\begin{example}
点\(z=0\)是函数\(f(z) = e^{1/z}\)的一个孤立奇点.
在去心邻域\(0<\abs{z}<+\infty\)内,
\(f(z) = e^{1/z}\)的罗朗展式为\[
	e^{1/z}
	= 1 + \frac{1}{z}
	+ \frac{1}{2!} \frac{1}{z^2}
	+ \dotsb
	+ \frac{1}{n!} \frac{1}{z^n}
	+ \dotsb.
\]
该展开式中含有无穷多个负次幂项,
因此\(z=0\)是\(e^{1/z}\)的本性奇点.
又因为\(e^{1/z}\neq0\),
所以\(z=0\)是\(e^{1/z}\)的皮卡例外值.
对任意有限非零复数\(A\),
若取\(z_n = \frac{1}{\ln A + 2n\pi\iu}\),
则\(\lim_{n\to\infty} z_n = 0\),
且\[
	f(z_n) = A
	\quad(n=1,2,\dotsc).
\]
这表示在\(z=0\)附近,
\(f(z) = e^{1/z}\)可以无穷多次地取到事先给定的任何有限非零复数\(A\).
于是可以想见函数\(e^{1/z}\)在\(z=0\)附近取值的奇异性了.
\end{example}

总之,上述定理,
不仅反映了解析函数在孤立奇点附近的性态,
同时也指出了判别奇点类型的方法.
下面就如何使用它们判别奇点类型举几个例子.

\begin{example}
证明\(z=0\)是\(f(z) = \frac{1}{z} (e^z-1)\)的可去奇点.
\begin{proof}[证法一]
用罗朗展开证明.

显然,\(z=0\)是\(f(z) = \frac{1}{z} (e^z-1)\)的孤立奇点.
在去心邻域\(0<\abs{z}<+\infty\)内将\(f(z)\)作罗朗展开,
得\[
	\frac{e^z-1}{z}
	= \frac{1}{z} \left( \sum_{n=0}^\infty \frac{z^n}{n!} - 1 \right)
	= \frac{1}{z} \sum_{n=1}^\infty \frac{z^n}{n!}
	= \sum_{n=1}^\infty \frac{z^{n-1}}{n!}
	= 1 + \frac{z}{2!} + \frac{z^2}{3!} + \dotsb,
\]
由此可见\(\frac{1}{z} (e^z-1)\)在点\(z=0\)的奇异部分为零,
也就是说点\(z=0\)是它的可去奇点.
\end{proof}
\begin{proof}[证法二]
用极限关系证明.

由于\(\lim_{z\to0} \frac{1}{z} (e^z-1) = \lim_{z\to0} e^z = 1\)是有限复数,
那么根据\cref{theorem:解析函数的级数表示.可去奇点的特征},
\(z=0\)就是\(\frac{e^z-1}{z}\)的可去奇点.
\end{proof}
\end{example}

\subsection{解析函数在无穷远点附近的性态}
由于函数\(f(z)\)在无穷远点\(\infty\)总是没有定义的,
所以无穷远点\(\infty\)总是函数\(f(z)\)的奇点.

\begin{definition}
若函数\(f(z)\)在无穷远点\(\infty\)的去心邻域\[
	0 \leq r < \abs{z} < +\infty
\]内解析,
则称点\(\infty\)为\(f(z)\)的一个孤立奇点.
若点\(\infty\)是\(f(z)\)的奇点的聚点,
则称点\(\infty\)为\(f(z)\)的非孤立奇点.
\end{definition}

设点\(\infty\)是\(f(z)\)的孤立奇点,
则作倒变换\(z = 1/\zeta\),
得函数\[
	\phi(\zeta) = f(1/\zeta) = f(z)
\]
在\(\zeta\)平面上原点\(\zeta=0\)的去心邻域\(D: 0<\abs{\zeta}<\frac{1}{r}\)内解析,
\(\zeta=0\)为函数\(\phi(\zeta)\)的孤立奇点.
于是,通过对函数\(\phi(\zeta)\)在有限孤立奇点\(\zeta=0\)的去心邻域\(D\)内性态的讨论,
我们就可以得到函数\(f(z)\)在无穷远点\(\infty\)附近的性态.
\begin{definition}\label{definition:解析函数的级数表示.无穷远处孤立奇点的分类}
若\(\zeta=0\)为\(\phi(\zeta)=f(1/\zeta)\)的可去奇点(视为解析点),
则称\(z=\infty\)为\(f(z)\)的可去奇点(解析点);
若\(\zeta=0\)为\(\phi(\zeta)\)的\(m\)阶极点,
则称\(z=\infty\)为\(f(z)\)的\(m\)阶极点;
若\(\zeta=0\)为\(\phi(\zeta)\)的本性奇点,
则称\(z=\infty\)为\(f(z)\)的本性奇点.
\end{definition}

前面在介绍分式线性变换时,
我们曾按广义连续性来定义函数在点\(\infty\)处的值:
定义\(f(\infty) = \lim_{z\to\infty} f(z)\).
但在点\(\infty\)没有定义差商,
因此我们没有定义函数在无穷远点处的可微性.
现在依据\cref{definition:解析函数的级数表示.无穷远处孤立奇点的分类},
我们可以将“函数\(f(z)\)在点\(\infty\)解析”定义为:
点\(\infty\)为\(f(z)\)的可去奇点,
且定义\(f(\infty) = \lim_{z\to\infty} f(z)\).

设由方程\[
	\phi(\zeta) = f(1/\zeta) = f(z)
\]
确定的函数\(\phi(\zeta)\)在去心邻域\(0<\abs{\zeta}<1/r\)内的罗朗展式为\[
	\phi(\zeta) = \sum_{n=-\infty}^\infty c_n \zeta^n.
\]
将变量\(\zeta\)换为\(z\),
就得到\(f(z)\)在无穷远点去心邻域\(r<\abs{z}<+\infty\)内的罗朗展式\[
	f(z) = \sum_{n=-\infty}^\infty c_n (1/z)^n
	= \sum_{n=-\infty}^\infty b_n z^n,
\]
其中\(b_n = c_{-n}\ (n=0,\pm1,\pm2,\dotsc)\).
也就是说,\(\phi(\zeta)\)在原点去心邻域的展式中的复幂项系数,
与\(f(z)\)在无穷远点去心邻域的展式中的相应正幂项系数相等,
而前者展式中的正幂项系数与后者相应复幂项系数相等.
根据这个关系,应用对有限孤立奇点的讨论结果,我们知道:
\begin{align*}
	\text{\(z=\infty\)是\(f(z)\)的可去奇点}
	&\iff
	\text{\(f(z)\)的罗朗展式中不含\(z\)的正次幂} \\
	&\iff
	\lim_{z\to\infty} f(z) = A~\text{是有限复数}, \\
	\text{\(z=\infty\)是\(f(z)\)的\(m\)阶极点}
	&\iff
	\text{\(f(z)\)的罗朗展式中只有有限个正次幂且其最高次幂是\(m\)} \\
	&\iff
	\lim_{z\to\infty} f(z) = \infty, \\
	\text{\(z=\infty\)是\(f(z)\)的本性奇点}
	&\iff
	\text{\(f(z)\)的罗朗展式中有无限多个正次幂} \\
	&\iff
	\lim_{z\to\infty} f(z)~\text{不存在}.
\end{align*}

\begin{example}
多项式函数\[
	P(z) = a_0 z^n + a_1 z^{n-1} + \dotsb + a_n
	\quad(a_0\neq0)
\]以无穷远点\(\infty\)为\(n\)阶极点.
\end{example}

\begin{example}
有理分式函数\[
	f(z) = \frac{a_0 z^n + a_1 z^{n-1} + \dotsb + a_n}{b_0 z^m + b_1 z^{m-1} + \dotsb + b_m}
	\quad(a_0\neq0,b_0\neq0)
\]
当\(m \geq n\)时,
\(\lim_{z\to\infty} f(z) = 0\)
或\(\lim_{z\to\infty} f(z) = \frac{a_0}{b_0}\),
因此\(\infty\)是\(f(z)\)的解析点;
当\(m<n\)时,
利用多项式的降幂除法,
可求得\(f(z)\)在点\(\infty\)的罗朗展式为\[
	f(z) = \frac{a_0}{b_0} z^{n-m} + \dotsb
	\quad(\frac{a_0}{b_0}\neq0),
\]
因此\(\infty\)是\(f(z)\)的\(n-m\)阶极点.
\end{example}

\begin{example}
无穷远点\(\infty\)是指数函数\(f(z) = e^z\)的本性奇点.
\end{example}
