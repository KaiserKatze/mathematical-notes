\section{解析函数的罗朗展开式}
\subsection{罗朗级数、罗朗定理}
\begin{definition}
形如\begin{equation}\label{equation:解析函数的级数表示.罗朗级数}
\begin{split}
	\sum_{n=-\infty}^\infty c_n (z-a)^n
	&\defeq \sum_{n=-\infty}^{-1} c_n (z-a)^n + \sum_{n=0}^\infty c_n (z-a)^n \\
	&\defeq \sum_{n=1}^\infty c_{-n} (z-a)^{-n} + \sum_{n=0}^\infty c_n (z-a)^n
\end{split}
\end{equation}的级数
称为\DefineConcept{罗朗级数}(Laurent series),
%@see: https://mathworld.wolfram.com/LaurentSeries.html
其中\(a\)及\(c_n\ (n=0,\pm1,\dotsc)\)是复常数.
\end{definition}
显然,当\(c_{-n}=0\ (n=1,2,\dotsc)\)时,
形式上罗朗级数 \labelcref{equation:解析函数的级数表示.罗朗级数} 就化为了幂级数.
可以说罗朗级数是由两个幂级数组成的.
若级数\[
	\sum_{n=0}^\infty c_n (z-a)^n
	\quad\text{和}\quad
	\sum_{n=1}^\infty c_{-n} (z-a)^{-n}
\]都在点\(z_0\)收敛,
则称罗朗级数 \labelcref{equation:解析函数的级数表示.罗朗级数} 在点\(z_0\)收敛.
记罗朗级数 \labelcref{equation:解析函数的级数表示.罗朗级数} 的和函数为\(f(z)\).
称级数\[
	\sum_{n=0}^\infty c_n (z-a)^n
\]为“\(f(z)\)在点\(a\)的\DefineConcept{解析部分}或\DefineConcept{正则部分}”.
称级数\[
	\sum_{n=1}^\infty c_{-n} (z-a)^{-n}
\]为“\(f(z)\)在点\(a\)的\DefineConcept{主要部分}(principal part)
%@see: https://mathworld.wolfram.com/PrincipalPart.html
或\DefineConcept{奇异部分}”.

设级数\(\sum_{n=0}^\infty c_n (z-a)^n\)的收敛半径为\(R \in (0,+\infty)\),
则它在圆\(\abs{z-a}<R\)内绝对收敛、内闭一致收敛,
它的和函数\(\phi(z)\)在圆\(\abs{z-a}<R\)内解析.

记\(\zeta = \frac{1}{z-a}\).
设级数\(\sum_{n=1}^\infty c_{-n} (z-a)^{-n} = \sum_{n=1}^\infty c_{-n} \zeta^n\)的
收敛半径为\(\lambda \in (0,+\infty)\),
则它在圆\(\abs{\zeta} < \lambda\)内绝对收敛、内闭一致收敛.
换言之,级数\(\sum_{n=1}^\infty c_{-n} (z-a)^{-n}\)
在\(r = \frac{1}{\lambda} < \abs{z-a} < +\infty\)内绝对收敛、内闭一致收敛,
它的和函数\(\psi(z)\)在\(r < \abs{z-a} < +\infty\)内解析.

结合上面的论述,我们来讨论一下级数 \labelcref{equation:解析函数的级数表示.罗朗级数} 的敛散性:
\begin{enumerate}
	\item 当\(r > R\)时,
	级数 \labelcref{equation:解析函数的级数表示.罗朗级数} 处处发散;

	\item 当\(r = R\)时,
	级数 \labelcref{equation:解析函数的级数表示.罗朗级数} 在区域\(\abs{z-a} \neq R\)上是发散的,
	而它在圆\(\abs{z-a}=R\)上则有三种可能性:
	\begin{enumerate}
		\item 它在圆上处处收敛.
		例如级数\(\sum_{\substack{n=-\infty \\ n\neq0}}^\infty \frac{z^n}{n^2}\)
		在圆\(\abs{z}=1\)上处处收敛;

		\item 它在圆上处处发散.
		例如级数\(\sum_{n=-\infty}^\infty z^n\)
		在圆\(\abs{z}=1\)上处处发散;

		\item 它在圆上有的点收敛,有的点发散.
		例如级数\(\sum_{\substack{n=-\infty \\ n\neq0}}^\infty \frac{z^n}{n}\)
		在圆\(\abs{z}=1\)上除点\(z=1\)外处处收敛;
	\end{enumerate}

	\item 当\(r < R\)时,
	级数 \labelcref{equation:解析函数的级数表示.罗朗级数}
	在圆环\(H: r < \abs{z-a} < R\)内绝对收敛且内闭一致收敛,
	在\(H\)外发散.
	圆环\(H\)称为罗朗级数的\DefineConcept{收敛圆环}.

	特别地,当\(r = 0\)且\(R = +\infty\)时,
	级数 \labelcref{equation:解析函数的级数表示.罗朗级数}
	在复平面\(\mathbb{C}\)上除点\(a\)外处处收敛.

	根据\hyperref[theorem:解析函数的级数表示.魏尔斯特拉斯定理]{魏尔斯特拉斯定理},
	罗朗级数 \labelcref{equation:解析函数的级数表示.罗朗级数} 的和函数在其收敛圆环\(H\)内是解析的,
	且可在\(H\)内逐项求任意阶导数.
\end{enumerate}

现在我们把罗朗级数 \labelcref{equation:解析函数的级数表示.罗朗级数}
在点\(a\)的解析部分与主要部分的和函数\(\phi(z)\)与\(\psi(z)\)分别表述为\[
	\def\arraystretch{2}
	\begin{array}{ll}
		\phi(z) = \sum_{n=0}^\infty c_n (z-a)^n & (\abs{z-a}<R), \\
		\psi(z) = \sum_{n=1}^\infty c_{-n} (z-a)^{-n} & (r < \abs{z-a} < +\infty).
	\end{array}
\]
显然,\(\phi(z)\)在\(\abs{z-a}<R\)内解析,
\(\psi(z)\)在\(r<\abs{z-a}<+\infty\)内解析.
当\(r \leq R\)时,
罗朗级数 \labelcref{equation:解析函数的级数表示.罗朗级数} 的和函数\(f(z)\)满足\[
	f(z) = \phi(z) + \psi(z),
\]
且\(f(z)\)在\(r<\abs{z-a}<R\)内解析.

综上所述,我们有以下定理.
\begin{theorem}
设罗朗级数 \labelcref{equation:解析函数的级数表示.罗朗级数} 的收敛圆环为\(H: r < \abs{z-a} < R\),
则它在\(H\)内绝对收敛且内闭一致收敛,
它的和函数\(f(z)\)在\(H\)内解析,
且\[
	f(z) = \sum_{n=-\infty}^\infty c_n (z-a)^n
\]在\(H\)内可逐项求任意阶导数.
\end{theorem}

上述定理的逆定理也成立.
\begin{theorem}[罗朗定理]\label{theorem:解析函数的级数表示.罗朗定理}
在圆环\(H: 0 \leq r < \abs{z-a} < R < +\infty\)内解析的函数\(f(z)\)
必定可以展成罗朗级数\begin{equation}\label{equation:解析函数的级数表示.罗朗展式}
	f(z) = \sum_{n=-\infty}^\infty c_n (z-a)^n,
\end{equation}
其中\begin{equation}\label{equation:解析函数的级数表示.罗朗系数}
	c_n = \frac{1}{2\pi\iu} \int_{\abs{z-a}=\rho} \frac{f(\zeta)}{(\zeta-a)^{n+1}} \dd{\zeta}
	\quad(r<\rho<R).
\end{equation}
并且展式 \labelcref{equation:解析函数的级数表示.罗朗展式} 是唯一的
(即\(f(z)\)和圆环\(H\)唯一地决定了系数\(c_n\)).
\end{theorem}
\cref{equation:解析函数的级数表示.罗朗展式}
称为“函数\(f(z)\)在圆环\(H\)内的\DefineConcept{罗朗展式}”,
\cref{equation:解析函数的级数表示.罗朗系数}
称为该展式的\DefineConcept{罗朗系数}.

在\cref{theorem:解析函数的级数表示.罗朗定理} 中,
当给定函数\(f(z)\)在点\(a\)解析时,
收敛圆环\(H\)就退化成收敛圆\(K: \abs{z-a}<R\).
这时,罗朗定理就退化为\hyperref[theorem:解析函数的级数表示.泰勒定理]{泰勒定理},
而罗朗系数 \labelcref{equation:解析函数的级数表示.罗朗系数} 就是泰勒系数.
也只有这时,罗朗系数除了有积分形式以外,
还有微分形式\[
	c_n = \frac{f^{(n)}(a)}{n!}.
\]
也只有这时,罗朗级数才退化为泰勒级数.
因此,泰勒级数是罗朗级数的特殊情形,
即\(c_{-n} = 0\ (n=1,2,\dotsc)\)的情形.

在求一些初等函数的罗朗展式时,
一般来说,不采用通过罗朗系数公式 \labelcref{equation:解析函数的级数表示.罗朗系数} 计算\(c_n\)的“直接法”,
而主要采用“间接法”,
即根据罗朗展式的唯一性,
通过各种代数运算或分析运算以及变量代换等方法,
应用已知的一些初等函数的泰勒展式,
来求出所给函数的罗朗展式.
所以,把函数展开成罗朗级数时,泰勒级数仍然是基础.

\begin{example}
求函数\(f(z) = \frac{1}{(z-1)(z-2)}\)在适当区域内的展式.
\begin{solution}
\(f(z)\)在\(z\)平面上只有两个奇点:\(z=1\)及\(z=2\).
因此,就\(f(z)\)的解析区域来说,
\(z\)平面可分成如下三个互不相交的解析区域:
\begin{enumerate}
	\item 圆\(D_1: \abs{z}<1\);
	\item 圆环\(D_2: 1<\abs{z}<2\);
	\item 圆环\(D_3: 2<\abs{z}<+\infty\).
\end{enumerate}
其中\(D_1\)是单连通区域,
\(D_2\)、\(D_3\)都是以\(z=0\)为中心的不同的圆环.
下面在这三个区域内分别求\(f(z)\)的展开式.

首先将\(f(z)\)分解成部分分式:\[
	f(z) = \frac{1}{z-2} - \frac{1}{z-1}.
\]
那么\begin{enumerate}
	\item 在\(D_1\)内,
	\(\abs{\frac{z}{2}}<1\),
	利用几何级数公式得\[
		f(z) = \frac{1}{1-z} - \frac{1}{2 (1-z/2)}
		= \sum_{n=0}^\infty \left(1 - \frac{1}{2^{n+1}}\right) z^n,
	\]
	这就是\(f(z)\)在\(D_1\)内的泰勒展式.

	\item 在\(D_2\)内,
	\(\abs{\frac{1}{z}}<1\),
	\(\abs{\frac{z}{2}}<1\),
	\begin{align*}
		f(z) &= -\frac{1}{2}\cdot\frac{1}{1-z/2}
			- \frac{1}{z}\cdot\frac{1}{1-1/z} \\
		&= -\frac{1}{2} \sum_{n=0}^\infty \frac{z^n}{2^n}
			- \frac{1}{z} \sum_{n=1}^\infty \frac{1}{z^{n-1}} \\
		&= -\sum_{n=0}^\infty \frac{z^n}{2^{n+1}}
			- \sum_{n=1}^\infty \frac{1}{z^n},
	\end{align*}
	这就是\(f(z)\)在\(D_2\)内的罗朗展式.

	\item 在\(D_3\)内,
	\(\abs{\frac{1}{z}}<1\),
	\(\abs{\frac{2}{z}}<1\),
	故\begin{align*}
		f(z) &= \frac{1}{z}\cdot\frac{1}{1-2/z}
			- \frac{1}{z}\cdot\frac{1}{1-1/z} \\
		&= \frac{1}{z} \sum_{n=0}^\infty \frac{2^n}{z^n}
			- \frac{1}{z} \sum_{n=0}^\infty \frac{1}{z^n} \\
		&= \sum_{n=2}^\infty \frac{2^{n-1}-1}{z^n},
	\end{align*}
	这就是\(f(z)\)在\(D_3\)内的罗朗展式.
\end{enumerate}
\end{solution}
由此例我们看到,
当函数\(f(z)\)及其解析的圆环取定以后,
根据罗朗展式的唯一性,
虽然其罗朗展式的系数\(c_n\ (n=0,\pm1,\dotsc)\)是唯一确定的,
但计算\(c_n\)的方法并不唯一,
不一定非得用\cref{equation:解析函数的级数表示.罗朗系数} 来计算罗朗系数.
\(c_n\)的唯一性保证我们可以用任何简便的方法来计算\(c_n\).
另外,此例还表明:
同一函数在不同圆环内的罗朗展式不同.
\end{example}

\subsection{在孤立奇点去心邻域内的罗朗展式}
\begin{definition}
若\(f(z)\)在点\(a\)的某一去心邻域内解析,
但在点\(a\)不解析,
则称\(a\)为\(f(z)\)的\DefineConcept{孤立奇点}.
若\(a\)是\(f(z)\)的一个奇点,
且在点\(a\)的任意邻域内\(f(z)\)总还有除点\(a\)以外的其他奇点,
则称点\(a\)为\(f(z)\)的\DefineConcept{非孤立奇点}.
\end{definition}

\begin{example}
点\(z=0\)是函数\(f(z) = \frac{1}{z}\)的孤立奇点,
是函数\(g(z) = \frac{1}{\sin(1/z)}\)的非孤立奇点.
\end{example}

\begin{example}
函数\(f(z) = \frac{1}{(z-1)(z-2)}\)
在\(z\)平面只有两个孤立奇点\(z=1\)和\(z=2\).
分别求\(f(z)\)在这两个点的去心邻域内的罗朗展式.
\begin{solution}
在去心邻域\(0<\abs{z-1}<1\)内\[
	f(z) = -\frac{1}{z-1} + \frac{1}{(z-1)-1} \\
	= -\frac{1}{z-1} - \sum_{n=0}^\infty (z-1)^n.
\]

在去心邻域\(0<\abs{z-2}<1\)内\[
	f(z) = \frac{1}{z-2} - \frac{1}{1+(z-2)}
	= \frac{1}{z-2} - \sum_{n=0}^\infty (-1)^n (z-2)^n.
\]
\end{solution}
\end{example}

在用间接法进行罗朗展开时,常常要用到罗朗级数的加法和乘法:
\begin{theorem}[罗朗级数的加法]
设\(F(z)\)在环域\(H: r < \abs{z-a} < R\)内解析,
且\(F(z) = f(z) + g(z)\),
\(f(z)\)与\(g(z)\)在环域\(H\)内的罗朗展式分别为\[
	f(z) = \sum_{n=-\infty}^\infty a_n (z-a)^n
	\quad\text{和}\quad
	g(z) = \sum_{n=-\infty}^\infty b_n (z-a)^n,
\]
则在\(H\)内\[
	F(z) = \sum_{n=-\infty}^\infty (a_n+b_n) (z-a)^n.
\]
\end{theorem}

\begin{theorem}[罗朗级数的乘法]
设\(F(z) = f(z) g(z)\)在环域\(H: r < \abs{z-a} < R\)内解析,
且\(f(z)\)与\(g(z)\)在\(H\)内的罗朗展式分别为\[
	f(z) = \sum_{n=-\infty}^\infty a_n (z-a)^n
	\quad\text{和}\quad
	g(z) = \sum_{n=-\infty}^\infty b_n (z-a)^n,
\]
则在\(H\)内\[
	F(z) = \sum_{n=-\infty}^\infty c_n (z-a)^n,
\]
其中\[
	c_n = \sum_{k=-\infty}^\infty a_k b_{n-k},
	\quad n=0,\pm1,\dotsc.
\]
\end{theorem}

\begin{example}
\def\fe{e^{\frac{t}{2} \left(z-\frac{1}{z}\right)}}
设\(t\)是实参数,
求函数\(f(z) = \fe\)在点\(z=0\)的罗朗展式.
\def\s#1{\sum_{#1}^\infty }%
\def\sk{\s{k=0} \frac{1}{k!} \left(\frac{t}{2}\right)^k z^k}%
\def\sl{\s{l=0} \frac{1}{l!} \left(-\frac{t}{2}\right)^l \left(\frac{1}{z}\right)^l}%
\begin{solution}
除\(z=0\)外,
\(f\)在\(z\)平面解析,
在去心邻域\(0<\abs{z}<+\infty\)内,
有\[
	\fe = \left[ \sk \right] \cdot \left[ \sl \right].
	\eqno(1)
\]
记\(A = \sk\),\(B = \sl\).

对于任意固定的\(t\),
(1)式右边的两个级数当\(\abs{z}>0\)都绝对收敛,
所以应用罗朗级数的乘法公式,可以任意方式合并同幂项.

为了得到乘积中某个正幂项\(z^n\ (n\geq0)\),
应将\(B\)中的各项分别与\(A\)中的第\(k=l+n\)项相乘,
即得\[
	\s{n=0} \left[ \s{l=0} \frac{(-1)^l}{l!(l+n)!} \left(\frac{t}{2}\right)^{2l+n} \right] z^n;
\]

而为了得到乘积中某个负幂项\(z^{-m}\ (m>0)\),
应将\(A\)中的各项分别与\(B\)中的第\(l=k+m\)项相乘,
即得\[
	\s{m=1} \left[ (-1)^m \s{k=0} \frac{(-1)^k}{k! (k+m)!} \left(\frac{t}{2}\right)^{2k+m} \right] z^{-m}.
\]
把\(-m\)改记为\(n\),
\(k\)改记为\(l\),
则\begin{align*}
	\fe &= \s{n=0}
	\left[
		\s{l=0}
		\frac{(-1)^l}{l!(l+n)!}
		\left(\frac{t}{2}\right)^{2l+n}
	\right]
	z^n \\
	&\hspace{20pt}
	\sum_{n=-1}^{-\infty}
	\left[
		(-1)^n
		\s{l=0}
		\frac{(-1)^l}{l!(l-n)!}
		\left(\frac{t}{2}\right)^{2l-n}
	\right]
	z^n \\
	&= \sum_{n=-\infty}^\infty J_n(t) z^n,
\end{align*}
其中\begin{equation}
	J_n(t) = \left\{ \begin{array}{rl}
		\s{l=0} \frac{(-1)^l}{l!(l+n)!} \left(\frac{t}{2}\right)^{2l+n},
			& n=0,1,2,\dotsc, \\
		(-1)^n \s{l=0} \frac{(-1)^l}{l!(l-n)!} \left(\frac{t}{2}\right)^{2l-n},
			& n=-1,-2,\dotsc.
	\end{array} \right.
\end{equation}
可以说\[
	J_{-n}(t) = (-1)^n J_n(t),
	\quad n=1,2,\dotsc.
\]

把\(J_n(t)\)称为“\(n\)阶\DefineConcept{贝塞尔函数}”
把\(J_{-n}(t)\)称为“\(-n\)阶\DefineConcept{贝塞尔函数}”.
使用达朗贝尔公式,
不难求得表示\(J_n(t)\)的幂级数的收敛半径为\(+\infty\).
这样,如果把\(t\)看成复变数,
则\(J_n(t)\)和\(J_{-n}(t)\)都在全平面\(\mathbb{C}\)上解析.
\end{solution}
\end{example}
