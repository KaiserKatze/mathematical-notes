\chapter{复数}
\section{复数的形式与运算}
\subsection{复数的三角形式}
\begin{definition}%[复数的三角形式]
由于平面直角坐标系上的点\((x,y)\)也可以用极坐标\(\opair{r,\theta}\)表示,
因此根据极坐标和直角坐标间的关系,
复数\(z = x + \iu y\)也可以表示成\begin{equation*}
	z = r(\cos\theta+ \iu \sin\theta).
\end{equation*}
这就是复数\(z\)的\DefineConcept{三角形式},
其中\begin{equation*}
	r = \sqrt{x^2+y^2} = \abs{z},
\end{equation*}\begin{equation*}
	\tan\theta = \frac{y}{x}.
\end{equation*}
我们把\(r\)称为复数\(z\)的\DefineConcept{模长}(complex modulus),
%@see: https://mathworld.wolfram.com/ComplexModulus.html
把\(\theta\)称作复数\(z\)的\DefineConcept{辐角}(complex argument).
%@see: https://mathworld.wolfram.com/ComplexArgument.html
\end{definition}

对于复数\(z = x + \iu y=r(\cos\theta+ \iu \sin\theta)\),
使得\(\tan\theta=\frac{y}{x}\)成立的\(\theta=\Arg z\)值有无穷多个.
规定落在区间\((-\pi,\pi]\)内的辐角值为\(\Arg z\)的\DefineConcept{主值},
或称之为\DefineConcept{主辐角},
记作\(\arg{z}\),
即\begin{equation*}
	\Arg z \defeq \arg{z} + 2k\pi, \quad k\in\mathbb{Z},
\end{equation*}
则主辐角\(\arg{z}\)是唯一确定的.
有时候主辐角\(\arg{z}\)被规定落在区间\([0,2\pi)\)内.

注意:\begin{enumerate}
	\item 对复数\(z = x + \iu y\),若\(z \neq 0\)且\(z \neq \infty\),
	则满足\(\tan \theta = \frac{y}{x}\)的\(\theta\)的取值有无穷多个,
	因此辐角\(\Arg z\)有无穷多个值,
	但主辐角\(\arg z\)只有一个值;
	\item 当复数\(z = 0\)或\(z = \infty\)时,
	辐角\(\Arg z\)、主辐角\(\arg{z}\)均无意义.
\end{enumerate}

\begin{theorem}
设非零复数\(z=r(\cos\theta+ \iu \sin\theta)\)的主辐角为\(\arg{z} \in (-\pi,\pi]\),则有\begin{equation*}
\def\arraystretch{1.5}
\arg{z} = \left\{ \begin{array}{lc}
\arctan{\frac{y}{x}}, &\quad x > 0, \\
\frac{\pi}{2}, &\quad x = 0 \land y > 0, \\
\arctan{\frac{y}{x}} + \pi, &\quad x < 0 \land y \geq 0, \\
\arctan{\frac{y}{x}} - \pi, &\quad x < 0 \land y < 0, \\
-\frac{\pi}{2}, &\quad x = 0 \land y < 0,
\end{array} \right.
\end{equation*}
\end{theorem}

\begin{theorem}
设非零复数\(z = x + \iu y=r(\cos\theta+ \iu \sin\theta)\)的主辐角为\(\arg{z} = \alpha\),则\begin{equation*}
\tan{\frac{\alpha}{2}}
= \frac{\sin\alpha}{1+\cos\alpha}
= \frac{r\sin\alpha}{r+r\cos\alpha}
= \frac{y}{r+x}
= \frac{y}{x+\sqrt{x^2+y^2}}
\end{equation*}所以\begin{equation*}
\arg{z} = \alpha
= 2 \arctan{ \frac{y}{x+\sqrt{x^2+y^2}} }
\end{equation*}
\end{theorem}

\begin{theorem}
设复数\(
	z_1 = r_1 (\cos\theta_1 + \iu \sin\theta_1),
	z_2 = r_2 (\cos\theta_2 + \iu \sin\theta_2)
\),
则\begin{gather*}
	z_1 z_2
	% = (\cos\theta_1+ \iu \sin\theta_1)(\cos\theta_2+ \iu \sin\theta_2)
	= r_1 r_2 [\cos(\theta_1+\theta_2) + \iu \sin(\theta_1+\theta_2)], \\
	\frac{z_1}{z_2}
	% = \frac{\cos\theta_1+ \iu \sin\theta_1}{\cos\theta_2+ \iu \sin\theta_2}
	= \frac{r_1}{r_2} [\cos(\theta_1-\theta_2) + \iu \sin(\theta_1-\theta_2)]
	\quad(z_2\neq0), \\
	z_1^n = r_1^n (\cos n\theta_1 + \iu \sin n\theta_1), \\
	z_1^{\frac1n} = \sqrt[n]{r_1} \left( \cos\frac{2k\pi+\theta}{n} + \iu \sin\frac{2k\pi+\theta}{n} \right)
	\quad(k=0,1,2,\dotsc,n-1).
\end{gather*}
\end{theorem}

\subsection{复数的指数形式}
根据以上引理,只要定义
\begin{equation}\label{equation:复数.欧拉公式}
	e^{\iu \theta}
	\defeq
	\cos\theta+ \iu \sin\theta,
\end{equation}
就有\begin{equation*}
	e^{\iu \theta_1} e^{\iu \theta_2} = e^{\iu(\theta_1+\theta_2)}
	\quad \text{和} \quad
	\frac{e^{\iu \theta_1}}{e^{\iu \theta_2}} = e^{\iu(\theta_1-\theta_2)}
\end{equation*}成立,
继而可以将任意非零复数\(z = x + \iu y\)写成指数形式,即\begin{equation*}
	z = r e^{\iu \theta}.
\end{equation*}

复数0的辐角无意义,不能写成指数形式.

\begin{theorem}%[指数形式下的复数的乘除法]
设非零复数\(z_1 = r_1 e^{\iu \theta_1}\)、\(z_2 = r_2 e^{\iu \theta_2}\),则\begin{equation*}
z_1 z_2 = r_1 e^{\iu \theta_1} \cdot r_2 e^{\iu \theta_2} = r_1 r_2 e^{\iu(\theta_1+\theta_2)}
\end{equation*}\begin{equation*}
\frac{z_1}{z_2} = \frac{r_1 e^{\iu \theta_1}}{r_2 e^{\iu \theta_2}} = \frac{r_1}{r_2} e^{\iu(\theta_1-\theta_2)}
\end{equation*}
\end{theorem}

\begin{property}
设复数\(z_1\)、\(z_2\),则有\begin{align*}
\Arg{z_1 z_2} &= \Arg{z_1} + \Arg{z_2} \\
\Arg{\frac{z_1}{z_2}} &= \Arg{z_1} - \Arg{z_2} \\
\Arg{\ComplexConjugate{z}} &= -\Arg z
\end{align*}
注意以上等式两边各是无穷多个角度值的集合.

同样地,存在\(k_1,k_2 \in \mathbb{Z}\),使得\begin{align*}
\arg{z_1 z_2} &= \arg{z_1} + \arg{z_2} + 2 k_1 \pi \\
\arg{\frac{z_1}{z_2}} &= \arg{z_1} - \arg{z_2} + 2 k_2 \pi
\end{align*}
成立.
\end{property}

\begin{theorem}%[指数形式下复数相等条件]
非零复数\(z_1=r_1 e^{\iu \theta_1}\)、\(z_2=r_2 e^{\iu \theta_2}\)相等的充分必要条件是:\begin{equation*}
\left\{ \begin{array}{l}
r_1 = r_2 \\
\theta_1 = \theta_2 + 2k\pi
\end{array} \right. \quad (k \in \mathbb{Z})
\end{equation*}
\end{theorem}

我们可以利用指数形式的复数作出对三角函数和积互化公式的证明.
由于\begin{equation*}
e^{\iu(\alpha+\beta)}
= e^{\iu \alpha} e^{\iu \beta}
\end{equation*}
可以写为代数形式
\begin{equation*}\begin{aligned}
\cos(\alpha+\beta)+ \iu \sin(\alpha+\beta)
&= (\cos\alpha+ \iu \sin\alpha)(\cos\beta+ \iu \sin\beta) \\
&= (\cos\alpha \cos\beta - \sin\alpha \sin\beta)
    + \iu(\cos\alpha \sin\beta + \sin\alpha \cos\beta),
\end{aligned}\end{equation*}
那么,稍加比较便知\begin{equation*}
\cos(\alpha+\beta) = \cos\alpha \cos\beta - \sin\alpha \sin\beta,
\end{equation*}\begin{equation*}
\sin(\alpha+\beta) = \cos\alpha \sin\beta + \sin\alpha \cos\beta.
\end{equation*}

\begin{example}
证明:\begin{gather}
\sum_{k=1}^n \cos kx
    = \cos(\frac{n+1}{2}x)
    \csc(\frac{1}{2}x)
    \sin(\frac{n}{2}x). \\
\sum_{k=1}^n \sin kx
    = \sin(\frac{n+1}{2}x)
    \csc(\frac{1}{2}x)
    \sin(\frac{n}{2}x).
\end{gather}
\end{example}

\subsection{复数的乘幂与方根}
\begin{definition}
设\(n\)是正整数,则\(z^n\)表示\(n\)个\(z\)的乘积.
规定:\(0^n=0\),\(z^0=1\),\(z^{-n}=\frac{1}{z^n}\).
当\(z=re^{\iu \theta} \neq 0\)时,\begin{equation*}
z^n = r^n e^{in\theta} = r^n(\cos n\theta+ \iu \sin n\theta),
\quad n \in \mathbb{Z}.
\end{equation*}
\end{definition}

\begin{theorem}%[棣莫弗公式]
\begin{equation}\label{equation:复数.棣莫弗公式}
	(\cos\theta+i\sin\theta)^n = \cos{n\theta}+ \iu \sin{n\theta}
\end{equation}
\end{theorem}
我们称\cref{equation:复数.棣莫弗公式} 为\DefineConcept{棣莫弗公式}.

\begin{definition}%[复数的方根]
已知\(z\in\mathbb{C}\),
关于复数\(w\)的方程\begin{equation*}
	w^n = z \quad (n \geq 2, n \in \mathbb{Z})
\end{equation*}的根
称为\(z\)的~\DefineConcept{\(n\)次方根}.
记所有根的总体为\(\sqrt[n]{z}\).

当\(z=0\)时,以上方程只有唯一解\(w = 0\).

当\(z \neq 0\)时,设\(z=re^{\iu \theta}\),\(w=\rho e^{\iu \phi}\)(其中\(r,\theta,\rho,\phi\in\mathbb{R}^+\))代入原方程,得\begin{equation*}
\rho^n e^{\iu n \phi} = r e^{\iu \theta}
\end{equation*}根据复数相等的充分必要条件,得\begin{equation*}
\left\{ \begin{array}{l}
\rho^n = r \\
n\phi = \theta + 2k\pi
\end{array} \right. \quad (k \in \mathbb{Z})
\end{equation*}由于\(\rho\)和\(r\)都是正实数,故可由\(\rho^n=r\)得出唯一确定的算术根\(\rho=\sqrt[n]{r}\);
又可由\(n\phi=\theta+2k\pi\)得出\(\phi=\frac{\theta+2k\pi}{n}\).
也就是说,\(z\)的\(n\)次方根为\begin{equation*}
w_k = (\sqrt[n]{z})_k = \sqrt[n]{r} e^{\iu \frac{\theta + 2k\pi}{n}} \quad (k=0,1,\dotsc,n-1)
\end{equation*}
\end{definition}

在复平面上,\(\sqrt[n]{z}\)的不同值\(w_k\)表示为\begin{equation*}
w_k = w_0 e^{\iu \frac{2k\pi}{n}} \quad (k \in \mathbb{Z}),
\end{equation*}对给定的复数\(z\),由\(z\)的模\(r\)和辐角\(\theta\)先在复平面上确定\(w_0 = \sqrt[n]{r} e^{\iu \frac{\theta}{n}}\),然后将\(w_0\)依次绕原点旋转\begin{equation*}
\def\f{\frac{2\pi}{n}}
\f,\,2\cdot\f,\,3\cdot\f,\,\dots,\,(n-1)\cdot\f,
\end{equation*}包括\(w_0\)在内,就在复平面上得到\(\sqrt[n]{z}\)的总共\(n\)个值的几何表示.它们均匀地分布在中心在原点、半径为\(\sqrt[n]{r}\)的圆周上,即它们是内接于此圆周的正\(n\)角形的\(n\)个顶点.

\section{复平面上的几何图形}
复数乘除法的几何意义可由指数形式下的乘除法运算公式得到.

复数\(z=z_1 z_2\)对应的向量是把复数\(z_1\)对应的向量先伸缩\(r_2 = \abs{z_2}\)倍,
再旋转一个角度\(\theta_2 = \arg z_2\)得到的.

直角坐标平面上任意一条用隐函数方程\(F(x,y)=0\)表示的曲线,经过变量代换即可得到其复方程为\begin{equation*}
F\left(\frac{z+\ComplexConjugate{z}}{2},\frac{z-\ComplexConjugate{z}}{2\iu}\right)=0.
\end{equation*}

\begin{example}%[射线]
从点\(z_0\)出发,与正实轴夹角为\(\theta_0\)的射线的复变数方程为\begin{equation*}
\arg(z-z_0) = \theta_0.
\end{equation*}
\end{example}

\begin{example}%[线段]
连接\(z_1\)、\(z_2\)两点的线段的参数方程为\begin{equation*}
z = z_1 + t(z_2 - z_1), \qquad t \in [0,1].
\end{equation*}
\end{example}

\begin{example}%[直线]
过\(z_1\)、\(z_2\)两点的直线的参数方程为\begin{equation*}
z = z_1 + t(z_2 - z_1), \qquad t \in (-\infty,+\infty).
\end{equation*}

实轴的方程为\(\Im z = 0\);
虚轴的方程为\(\Re z = 0\).
\end{example}

\begin{example}%[三点共线的充分必要条件]
三点\(z_1\)、\(z_2\)、\(z_3\)共线的充分必要条件为\begin{equation*}
\frac{z_3 - z_1}{z_2 - z_1} = t \neq 0, \qquad t \in \mathbb{R}.
\end{equation*}
\end{example}

\begin{example}%[圆]
以\(z_0\)为圆心,\(R\)为半径的圆周的方程为\begin{equation*}
\abs{z - z_0} = R.
\end{equation*}而\(\abs{z-z_0}<R\)表示圆的内部,\(\abs{z-z_0}>R\)表示圆的外部.

复平面上圆周的一般方程为\begin{equation*}
A z \ComplexConjugate{z} + \beta \ComplexConjugate{z} + \ComplexConjugate{\beta} z + C = 0
\end{equation*}其中\(A,C\in\mathbb{R}\),\(A \neq 0\),\(\beta\in\mathbb{C}\),且\begin{equation*}
\abs{\beta}^2 > AC
\end{equation*}

以\(z_0\)为圆心,\(R\)为半径的圆周的方程还可表示为\begin{equation*}
z - z_0 = R e^{\iu \theta}.
\end{equation*}
\end{example}

\begin{definition}
设圆\(C: \abs{z - a} = R\).
如果点\(z_1,z_2\)都在从圆心\(a\)出发的同一条射线上,且满足\begin{equation*}
\abs{z_1 - a} \abs{z_2 - a} = R^2,
\end{equation*}则称点\(z_1,z_2\)关于圆周\(C\)对称.

特别地,规定圆心\(a\)与无穷远点\(\infty\)关于圆周\(C\)对称.
\end{definition}

\begin{theorem}
点\(z_1,z_2\)关于圆周\(C: \abs{z - a} = R\)对称的充分必要条件是:\begin{equation*}
\ComplexConjugate{z_a - a} (z_2 - a) = R^2.
\end{equation*}
\end{theorem}

\begin{theorem}
点\(z_1,z_2\)关于圆周\(C: \abs{z - a} = R\)对称的充分必要条件是:通过\(z_1,z_2\)的任意圆周都与\(C\)正交.
\end{theorem}
