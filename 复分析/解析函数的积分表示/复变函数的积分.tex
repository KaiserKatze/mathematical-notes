\section{复变函数的积分}
\subsection{积分的定义和计算方法}
为了叙述简便而又不妨碍实际应用,
今后除特别声明外,
我们谈到曲线时一律是指光滑或逐段光滑的曲线.
其中,逐段光滑的简单闭曲线
简称为\DefineConcept{围线}(contour).
%@see: https://mathworld.wolfram.com/Contour.html

\begin{definition}
设有向曲线\(C: z = z(t)\ (\alpha \leq t \leq \beta)\)
以\(a = z(\alpha)\)为起点,
\(b = z(\beta)\)为终点.
又设函数\(f(z)\)沿\(C\)有定义.
在\(C\)上沿着\(C\)从\(a\)到\(b\)的方向
(此为实参数增大的方向,通常作为\(C\)的正方向)
任取\(n-1\)个分点\begin{equation*}
	z_0 = a,\ z_1,\dots,\ z_{n-1},\ z_n = b
\end{equation*}
把曲线\(C\)分成\(n\)个小弧段.
在每个小弧段\(\Arc{z_{k-1} z_k}\)上任取一点\(\zeta_k\),
求和得\begin{equation*}
	S_n = \sum_{k=1}^n{f(\zeta_k) \increment z_k},
\end{equation*}
其中\(\increment z_k = z_k - z_{k-1}\).
记\(\lambda = \max\{\abs{\increment z_1},\abs{\increment z_2},\dots,\abs{\increment z_n}\}\).
若\(\lambda\to0\)(分点无限增多,且这些弧段长度均趋于零)时,
上述和式的极限\(\lim_{n\to\infty}S_n\)存在且趋于确定的极限\(J\),
那么称函数\(f(z)\)沿\(C\) \DefineConcept{可积},
称\(J\)为函数\(f(z)\)沿\(C\)(从\(a\)到\(b\))的\DefineConcept{积分},
记作\(\int_C f(z) \dd{z}\),
即\begin{equation*}
	\int_C f(z) \dd{z} = J = \lim_{\lambda\to0} \sum_{k=1}^n{f(\zeta_k) \increment z_k}.
\end{equation*}
其中\(C\)称为\DefineConcept{积分路径}.
\(\int_C f(z) \dd{z}\)表示沿\(C\)的正方向的积分,
而\(\int_{C^-}{f(z)\dd{z}}\)表示沿\(C\)的负方向的积分.
\end{definition}

如果积分路径是一条围线,
那么把这个积分特别称为\DefineConcept{围线积分}(contour integral).
%@see: https://mathworld.wolfram.com/ContourIntegral.html
%@see: https://mathworld.wolfram.com/ContourIntegration.html

\begin{theorem}
如果函数\(f(z)=u(x,y)+\iu v(x,y)\)沿曲线\(C\)连续,
则\(f(z)\)沿\(C\)可积,
且\begin{equation*}
	\int_C f(z) \dd{z}
	= \int_C u(x,y)\dd{x} - v(x,y)\dd{y}
	+ \iu \int_C v(x,y)\dd{x} + u(x,y)\dd{y}.
\end{equation*}

为了便于记忆,上式也可写成\begin{equation*}
\int_C f(z) \dd{z} = \int_C{(u+\iu v)(\dd{x}+\iu\dd{y})}.
\end{equation*}
\end{theorem}

\begin{corollary}
如果\(\int_C f(z) \dd{z}\)沿有向光滑曲线\(C\)连续.曲线\(C\)的方程为\begin{equation*}
z = z(t) = x(t) + \iu y(t), \quad \alpha \leq t \leq \beta,
\end{equation*}那么有\begin{equation*}
\int_C f(z) \dd{z} = \int_\alpha^\beta f[z(t)] z'(t) \dd{t}.
\end{equation*}称上式为复积分的\DefineConcept{变量代换公式}.
\end{corollary}

\begin{example}
设\(n\in\mathbb{Z}\),\(C\)是以\(a\)为心,\(R\)为半径的圆周.求:\begin{equation*}
I = \int_C \frac{\dd{z}}{(z-a)^n}.
\end{equation*}
\begin{solution}
由于圆周\(C\)的参数方程为\begin{equation*}
z-a=Re^{\iu\theta}, \quad 0 \leq \theta \leq 2\pi,
\end{equation*}所以\(\dd{z}=\iu R e^{\iu\theta} \dd{\theta}\).

当\(n=1\)时,\begin{equation*}
\int_C \frac{\dd{z}}{z-a}
= \int_0^{2\pi}{\frac{\iu R e^{\iu\theta} \dd{\theta}}{R e^{\iu\theta}}}
= \iu \int_0^{2\pi}\dd{\theta}
= 2\pi\iu.
\end{equation*}

当\(n \neq 1\)时,\begin{align*}
\int_C{\frac{\dd{z}}{(z-a)^n}}
&= \int_0^{2\pi}{
	\frac{
		\iu R e^{\iu\theta} \dd{\theta}
	}{
		R^n e^{\iu n \theta}
	}
}
= \frac{\iu}{R^{n-1}} \int_0^{2\pi}{e^{-\iu(n-1)\theta}\dd{\theta}} \\
&= \frac{\iu}{R^{n-1}} \left[
	\int_0^{2\pi}{\cos(n-1)\theta\dd{\theta}}
	-\iu \int_0^{2\pi}{\sin(n-1)\theta\dd{\theta}}
	\right]
= 0.
\end{align*}
也就是说,\begin{equation}\label{equation:解析函数的积分表示.重要积分1}
\int_C \frac{\dd{z}}{(z-a)^n} = \left\{ \begin{array}{cl}
2\pi\iu, & n=1, \\
0, & n\in\mathbb{Z}-\{1\}.
\end{array} \right.
\end{equation}
\end{solution}
\end{example}

\subsection{复积分的基本性质}
\begin{property}
设\(f(z)\)和\(g(z)\)沿曲线\(C\)连续,则\begin{enumerate}
\item \(\int_C a f(z) \dd{z} = a \int_C f(z) \dd{z}, a\in\mathbb{C}\);
\item \(\int_C [f(z) \pm g(z)] \dd{z} = \int_C f(z) \dd{z} \pm \int_C g(z) \dd{z}\);
\item \(\int_C f(z) \dd{z} = \int_{C_1} f(z) \dd{z} + \int_{C_2} f(z) \dd{z}\),其中\(C\)由曲线\(C_1\)和\(C_2\)衔接而成;
\item \(\int_{C^-} f(z) \dd{z} = -\int_C f(z) \dd{z}\).
\end{enumerate}
\end{property}

\begin{theorem}
设\(f(z)\)和\(g(z)\)沿曲线\(C\)连续,则\begin{equation*}
\abs{\int_C f(z) \dd{z}}
\leq \int_C{\abs{f(z)}\abs{\dd{z}}}
= \int_C{\abs{f(z)}\dd{s}},
\end{equation*}其中\(\abs{\dd{z}}=\dd{s}=\sqrt{(\dd{x})^2+(\dd{y})^2}\)表示弧长的微分.
\end{theorem}

\begin{corollary}[积分估值定理]\label{theorem:解析函数的积分表示.积分估值定理}
若存在\(M > 0\),使在曲线\(C\)上\(\abs{f(z)} \leq M\),曲线\(C\)的长为\(L\),则有\begin{equation}\label{equation:解析函数的积分表示.长大不等式}
\abs{\int_C f(z) \dd{z}} \leq ML.
\end{equation}\rm
因为\cref{equation:解析函数的积分表示.长大不等式} 是关于弧长\(L\)和函数\(f\)的模(大小)\(\abs{f(z)}\)的不等式,所以又称之为\DefineConcept{长大不等式}.
\end{corollary}
