\section{柯西积分定理}
\subsection{柯西积分定理}
\begin{theorem}\label{theorem:解析函数的积分表示.柯西积分定理}
设函数\(f\)在单连通区域\(D\)内解析,
则对\(D\)内的任意一条围线\(C\),
有\[
	\int_C f(z) \dd{z} = 0.
\]
\end{theorem}
\cref{theorem:解析函数的积分表示.柯西积分定理}
常被称为\DefineConcept{柯西积分定理}(Cauchy integral theorem).
%@see: https://mathworld.wolfram.com/CauchyIntegralTheorem.html

\begin{theorem}\label{theorem:解析函数的积分表示.柯西积分定理.闭区域的情形}
设\(C\)是一条围线,
\(D\)是\(C\)的内部区域,
\(f(z)\)在闭区域\(\overline{D}=D \cup C\)上解析,
则\[
	\int_C f(z) \dd{z} = 0.
\]
\end{theorem}

\begin{corollary}\label{theorem:解析函数的积分表示.柯西积分定理.非简单闭曲线的情形}
设\(f(z)\)在单连通区域\(D\)内解析,
\(C\)为\(D\)内任一闭曲线(不必是简单闭曲线),
则\[
	\int_C f(z) \dd{z} = 0.
\]
\end{corollary}

\begin{corollary}\label{theorem:解析函数的积分表示.解析函数在解析区域内的积分与路径无关}
若\(f(z)\)在单连通区域\(D\)内解析,
则\(f(z)\)在\(D\)内的积分与路径无关.
\end{corollary}

我们约定,当复积分\(\int_C f(z) \dd{z}\)与路径\(C\)无关,
只与积分的起点\(z_0\)和终点\(z_1\)有关时,
这个积分就可以表示为\[
	\int_{z_0}^{z_1} f(z) \dd{z}.
\]

下面对柯西积分定理从两个方面进行推广:
一方面是被积函数的解析范围;
另一方面是解析区域的连通性.
这两个方面的推广分别表现在下面两个定理中.
\begin{theorem}
设\(C\)是一条围线,
区域\(D\)是\(C\)的内部,
\(f(z)\)在\(D\)内解析,
在\(\overline{D}=D \cup C\)上连续\footnote{也称这种情况为“\(f(z)\)从\(D\)内连续到边界”.},
则\[
	\int_C f(z) \dd{z} = 0.
\]
\end{theorem}

\begin{definition}
设有\(n+1\)条围线\(C_0,C_1,\dots,C_n\),
其中\(C_1,\dots,C_n\)中每一条都在其余各条的外部,
而它们又全都在\(C_0\)的内部.
在\(C_0\)内部且在\(C_1,\dots,C_n\)外部的点集构成有界的多连通区域\(D\).
\(D\)以\(C_0,C_1,\dots,C_n\)为边界.
我们把区域\(D\)的边界称为\DefineConcept{复围线},
记作\(C=C_0+C_1^-+\dots+C_n^-\).
规定:
在外边界\(C_0\)上,
\(C\)的正方向为逆时针方向;
在内边界\(C_1,\dots,C_n\)上,
\(C\)的正方向为顺时针方向.
\end{definition}

\begin{theorem}[多连通区域的柯西积分定理]\label{theorem:解析函数的积分表示.多连通区域的柯西积分定理}
设\(C\)是由复围线\(C=C_0+C_1^-+\dots+C_n^-\)所围成的有界多连通区域,
\(f(z)\)在\(D\)内解析,
在\(\overline{D}=D \cup C\)上连续,
则\[
	\int_C f(z) \dd{z} = 0
\]或\begin{equation}
	\def\Ic#1{\int_{C_{#1}} f(z) \dd{z}}
	\Ic{0} = \Ic{1} + \dots + \Ic{n}.
\end{equation}
\end{theorem}

在掌握上述柯西积分定理的两种等价形式、两个推论和两个推广结论后,
可以使复积分的计算变得极为简便.

\begin{example}
设\(a\)是任意围线\(C\)内部一点,试证:\[
	\int_C \frac{\dd{z}}{(z-a)^n}
	= \left\{ \begin{array}{cl}
		2\pi\iu, & n=1, \\
		0, & n\in\mathbb{Z}-\{1\}.
	\end{array} \right.
\]
\begin{proof}
记围线\(C\)所围成的区域为\(D\).
取\(r\in\mathbb{R}^+\)
使得圆周\(C_0: z - a = r e^{\iu\theta}\ (0 \leq \theta \leq 2\pi)\)
所围成的区域\(D_0 \subseteq D\).
根据\hyperref[theorem:解析函数的积分表示.多连通区域的柯西积分定理]{多连通区域的柯西积分定理}有\[
	\int_C \frac{\dd{z}}{(z-a)^n} = \int_{C_0} \frac{\dd{z}}{(z-a)^n};
\]
再应用重要积分 \labelcref{equation:解析函数的积分表示.重要积分1} 即得要证结论.
\end{proof}
\end{example}
由本例结果还可以得到一个时常用到的重要公式:
对任一围线\(C\),
都有\begin{equation}\label{equation:解析函数的积分表示.重要积分2}
	\frac{1}{2\pi\iu}
	\int_C \frac{\dd{z}}{z-a}
	= \left\{ \begin{array}{ll}
		1, & a \in I(C); \\
		0, & a \in E(C).
	\end{array} \right.
\end{equation}

\begin{example}
试讨论在不同围线\(C\)上的复积分\[
	I = \frac{1}{2\pi\iu} \int_C \frac{\dd{z}}{(z-a)(z-b)}
\]的取值,
其中\(a \neq b\)且\(a,b \notin C\).
\begin{solution}
由于\[
	\frac{1}{(z-a)(z-b)}
	= \frac{1}{a-b} \left(\frac{1}{z-a}-\frac{1}{z-b}\right),
\]
应用\cref{equation:解析函数的积分表示.重要积分2}
即得\[
	I = \left\{\begin{array}{cl}
		0, & \text{\(a,b\)同时在\(C\)内部或\(C\)外部}, \\
		\frac{1}{a-b}, & \text{\(a\)在\(C\)内部而\(b\)在\(C\)外部}, \\
		\frac{1}{b-a}, & \text{\(a\)在\(C\)外部而\(b\)在\(C\)内部}.
	\end{array}\right.
\]
\end{solution}
\end{example}

\begin{example}%例3.2.4
试证:若函数\(f(z)\)在区域\(D: 0<\abs{z-a}<R\)内解析,
且\[
	\lim_{z \to a}(z-a)f(z)=A,
\]
则\[
	\int_\gamma f(z) \dd{z} = 2\pi\iu A,
\]
其中\(\gamma(\theta) = a+re^{\iu\theta}\ (0 \leq \theta \leq 2\pi, 0 < r < R)\).
\begin{proof}
由\(\lim_{z \to a}(z-a)f(z) = A\),
设\[
	(z-a)f(z) = A + \epsilon(z),
\]
其中\(\epsilon(z)\)满足\(\lim_{z \to a} \epsilon(z) = 0\).
那么有\[
	f(z) = \frac{1}{z-a}[A+\epsilon(z)].
\]

由于\(r\to0\)时,
\(z=\gamma(\theta) \to a\),
所以\[
	\lim_{r\to0} \abs{\epsilon[\gamma(\theta)]} = 0,
	\quad 0 \leq \theta \leq 2\pi.
\]

又因为\[
	\abs{\frac{\epsilon(z)}{z-a}}
	= \abs{\frac{\epsilon[\gamma(\theta)]}{\gamma(\theta)-a}}
	= \frac{\abs{\epsilon[\gamma(\theta)]}}{\abs{re^{\iu\theta}}}
	= \frac{\abs{\epsilon[\gamma(\theta)]}}{r},
\]
则根据积分估值定理可得\[
	\abs{\int_\gamma \frac{\epsilon(z)}{z-a} \dd{z}}
	\leq \abs{\frac{\epsilon(z)}{z-a}} \cdot 2\pi r
	= \frac{\abs{\epsilon[\gamma(\theta)]}}{r} \cdot 2\pi r
	= 2\pi \abs{\epsilon[\gamma(\theta)]},
\]
且由函数\(f(z) = \frac{1}{z-a}[A+\epsilon(z)]\)在区域\(D\)内解析
可知函数\(\frac{\epsilon(z)}{z-a}\)也同样在区域\(D\)内解析,
再根据\hyperref[theorem:解析函数的积分表示.多连通区域的柯西积分定理]{多连通区域的柯西积分定理}可得\[
	\int_\gamma \frac{\epsilon(z)}{z-a} \dd{z}
	= \lim_{r\to0} \int_\gamma \frac{\epsilon(z)}{z-a} \dd{z}
	= 0.
\]

那么有\[
	\int_\gamma f(z) \dd{z}
	= \int_\gamma \frac{A}{z-a} \dd{z}
	+ \int_\gamma \frac{\epsilon(z)}{z-a} \dd{z}
	= 2\pi\iu A.
	\qedhere
\]
\end{proof}
\end{example}

\subsection{原函数}
在实积分中,
我们学习过\hyperref[theorem:定积分.原函数存在定理]{原函数存在定理},
现在我们将这个定理推广到复积分的情形.
\begin{theorem}\label{theorem:解析函数的积分表示.原函数1}
设\(f(z)\)在单连通区域\(D\)内连续,
且对全含于\(D\)内的任一围线\(C\),
有\[
	\int_C f(z) \dd{z} = 0,
\]
则由变上限积分所确定的函数\[
	F(z) = \int_{z_0}^z f(\zeta) \dd{\zeta}
\]在\(D\)内解析,
且\(F'(z) = f(z)\),
其中\(z_0, z \in D\).
\begin{proof}
由于\(D\)是单连通区域,
又由\(\int_C f(z) \dd{z}=0\)
可知\(f(z)\)在\(D\)内从点\(z_0\)到\(z\)的积分与路径无关,
因此\(F(z)\)是\(D\)内的单值函数.

以点\(z\)为心作一个含于\(D\)内的圆,
在此圆内任取一点\(z+\increment z\),
于是\[
	\frac{F(z+\increment z)-F(z)}{\increment z}
	= \frac{1}{\increment z}
		\left[
			\int_{z_0}^{z+\increment z} f(\zeta) \dd{\zeta}
			-\int_{z_0}^z f(\zeta) \dd{\zeta}
		\right]
	= \frac{1}{\increment z} \int_z^{z+\increment z} f(\zeta) \dd{\zeta}.
\]
又因为\[
	\frac{1}{\increment z} \int_z^{z+\increment z} f(z) \dd{\zeta}
	= \frac{f(z)}{\increment z} \int_z^{z+\increment z} \dd{\zeta} = f(z),
\]
所以\[
	\frac{F(z+\increment z)-F(z)}{\increment z} - f(z)
	= \frac{1}{\increment z} \int_z^{z+\increment z} [f(\zeta)-f(z)]\dd{\zeta}.
\]
由于\(f(\zeta)\)在点\(z\)连续,
故对任给的\(\epsilon > 0\),
存在\(\delta > 0\),
当\(\abs{\zeta - z} < \delta\)时,
便有\(\abs{f(\zeta)-f(z)} < \epsilon\).
这样,若取\(0<\abs{\increment z}<\delta\),
则由\hyperref[theorem:解析函数的积分表示.积分估值定理]{积分估值定理}可得\[
	\abs{\frac{F(z+\increment z)-F(z)}{\increment z} - f(z)}
	= \frac{1}{\abs{\increment z}} \abs{ \int_z^{z+\increment z} [f(\zeta)-f(z)]\dd{\zeta} }
	\leq \frac{1}{\abs{\increment z}} \epsilon \abs{\increment z}
	= \epsilon.
\]
也就是说\[
	\lim_{\increment z\to0} \frac{F(z+\increment z)-F(z)}{\increment z} = f(z)
\]或\[
	F'(z) = f(z).
	\qedhere
\]
\end{proof}
\end{theorem}

\begin{corollary}\label{theorem:解析函数的积分表示.原函数2}
若\(f(z)\)在单连通区域\(D\)内解析,
则由变上限积分确定的函数\[
	F(z) = \int_{z_0}^z f(\zeta) \dd{\zeta}
\]在\(D\)内解析,
且\(F'(z) = f(z)\).
\end{corollary}

\begin{definition}
若在区域\(D\)内有\(F'(z)=f(z)\),
则称\(F(z)\)为\(f(z)\)在区域\(D\)内的一个\DefineConcept{原函数}.
\end{definition}

\begin{corollary}\label{theorem:解析函数的积分表示.原函数3}
若\(f(z)\)在单连通区域\(D\)内解析
(或在\cref{theorem:解析函数的积分表示.原函数1} 的前提条件下),
\(\Phi(z)\)为\(f(z)\)在\(D\)内的任一原函数,
则有牛顿--莱布尼茨公式
\begin{equation}\label{equation:解析函数的积分表示.牛顿莱布尼茨公式}
	\int_{z_0}^z f(\zeta) \dd{\zeta}
	= \eval{\Phi(\zeta)}_{z_0}^z
	= \Phi(z)-\Phi(z_0)
\end{equation}成立.
\end{corollary}
事实上,由于\(F(z) = \int_{z_0}^z f(\zeta) \dd{\zeta}\)是\(f(z)\)在\(D\)内的一个原函数,
按定义对任意\(z \in D\),
有\([\Phi(z) - F(z)]'=0\),
从而\[
	\Phi(z) - F(z) = C,
\]
即\[
	\Phi(z) = F(z) + C.
\]
令\(z=z_0\),
得\(C = \Phi(z_0)\),
于是\cref{equation:解析函数的积分表示.牛顿莱布尼茨公式} 成立.

\begin{example}
在单连通区域\(D: -\pi<\arg z<\pi\)内,
函数\(\ln z\)是\(f(z) = \frac{1}{z}\)的一个原函数,
而\(f(z)\)在单连通区域\(D\)内解析,
故由\cref{theorem:解析函数的积分表示.原函数3} 有\[
	\int_1^z \frac{\dd{\zeta}}{\zeta}
	= \ln z - \ln 1
	= \ln z \quad(z \in D).
\]
\end{example}

\begin{example}
计算积分\[
	\int_C (z^2 - \sin z) \dd{z},
\]
其中\(C\)为摆线\[
	\left\{ \begin{array}{l}
	x = a(\theta-\sin\theta) \\
	y = a(1\cos\theta)
	\end{array} \right.
	\quad(0\leq\theta\leq2\pi).
\]
\begin{solution}
因为被积函数\(f(z) = z^2 - \sin z\)在\(z\)平面\(\mathbb{C}\)上解析,
\(\mathbb{C}\)是单连通区域,
所以积分只与路径的起点、终点有关,
而与路径无关.
当\(\theta=0\)时,\(z=0\);
当\(\theta=2\pi\)时,\(z=2\pi a\).
因此要么将沿\(C\)的积分简化为沿实轴的积分,
要么直接应用\hyperref[equation:解析函数的积分表示.牛顿莱布尼茨公式]{牛顿--莱布尼茨公式}.
\begin{align*}
	\int_C (z^2 - \sin z) \dd{z}
	&= \int_0^{2\pi a} (x^2 + \sin x) \dd{x} \\
	&= \left(\frac{1}{3} x^3 - \cos x\right)_0^{2\pi a} \\
	&= \frac{8}{3} \pi^3 a^3 - \cos(3\pi a+1).
\end{align*}
\end{solution}
\end{example}

必须指出的是:
\cref{theorem:解析函数的积分表示.原函数2,theorem:解析函数的积分表示.原函数3} 中,
区域\(D\)的单连通性对于推论的结论来说是不可缺少的.
若函数\(f(z)\)的解析区域\(D\)是复连通区域,
则\(f(z)\)沿\(D\)内任意围线上的积分可能不为零,
因而若仍用\(\int_{z_0}^z f(\zeta) \dd{\zeta}\)来
表示\(f(z)\)沿点\(z_0\)到点\(z\)的\(D\)内某曲线上的积分,
则\(\int_{z_0}^z f(\zeta) \dd{\zeta}\)一般来说
不仅与\(z_0,z\)在\(D\)中的位置有关,
也与由\(z_0\)到\(z\)的积分路径有关.
随着\(z_0\)到\(z\)的积分路径不同,
\(\int_{z_0}^z f(\zeta) \dd{\zeta}\)的值也不同,
所以,一般来说\(\int_{z_0}^z f(\zeta) \dd{\zeta}\)是一个多值函数,
\(f(z)\)在多连通区域\(D\)内就不可能有原函数.

若\(f(z)\)在\(D\)内有原函数\(F(z)\),
而\(C(t)\ (\alpha\leq t \leq\beta)\)是\(D\)内从\(z_0\)到\(z\)的光滑曲线,
则由\(F'(z) = f(z)\),
\begin{align*}
	\int_C f(z) \dd{z}
	&= \int_\alpha^\beta F'[C(t)] \cdot C'(t) \dd{t}
	= \eval{F[C(t)]}_\alpha^\beta \\
	&= F[C(\beta)] - F[C(\alpha)]
\end{align*}
与路径无关,只与\(C(t)\)的起点\(z_0\)和终点\(z\)有关,这就产生矛盾.
例如,函数\(f(z) = \frac{1}{z}\)在\(D=\mathbb{C}-\{0\}\)内解析,但它在\(D\)内没有原函数.
\begin{example}%例3.2.7
\def\f{\frac{\dd{\zeta}}{\zeta}}
试证:在两连通区域\(G: z\neq0,\infty\)内\[
	\int_1^z \f
	= \Ln z
	\quad (z \in G),
\]
其中积分路径\(C\)是不经过原点,
且连接点\(z_0=1\)和点\(z\)的任意逐段光滑曲线.
\begin{proof}
考虑这样两条路径的积分,
其中一条\(L\)沿着正方向或负方向绕原点若干周,
另一条则既不过原点也不穿过负实轴.
%由例3.2.5和例3.2.2可得
\[
	\int_L \f
	= \left(\int_{ABCbDA} + \int_{ADaCEA} + \int_{AEF}\right) \f
	= 2 \int_\gamma \f + \int_l \f.
\]
一般,\begin{align*}
	\int_L \f
	&= \int_l \f + n \int_\gamma \f \\
	&= \ln z + 2n\pi\iu \quad(n=0,\pm1,\dotsc) \\
	&= \Ln z.
\end{align*}
因此,所给变上限积分\(F(z) = \int_1^z \f\)
在两连通区域\(G\)内是对数函数\(\Ln z\)的一个积分表达式.
\end{proof}
\end{example}
