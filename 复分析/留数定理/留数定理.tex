\section{留数定理}
\subsection{留数的定义及计算方法}
我们知道,若有限点\(a\)是函数\(f(z)\)的解析点,
围线\(\Gamma\)全在点\(a\)的某邻域内并包围点\(a\),
则根据\hyperref[theorem:解析函数的积分表示.柯西积分定理]{柯西积分定理},
有\begin{equation*}
	\int_\Gamma f(z) \dd{z} = 0.
\end{equation*}

若有限点\(a\)是\(f(z)\)的一个孤立奇点,
则当围线\(\Gamma\)全在点\(a\)的某去心邻域\(U(a,R)\)内且包围点\(a\)时,
积分\(\int_\Gamma f(z) \dd{z}\)的值通常不为零.
事实上,设在点\(a\)的某去心邻域\(\mathring{U}(a,R)\)内,
\(f(z)\)的罗朗展式为\begin{equation*}
	f(z) = \sum_{n=-\infty}^{+\infty} C_n (z-a)^n,
\end{equation*}
其中\(C_n\ (n=0,\pm1,\pm2,\dotsc)\)是复常数.
将上式等号两端沿\(\Gamma\)积分,
并利用\hyperref[theorem:解析函数的级数表示.一致收敛级数的基本性质2]{逐项积分定理}及
重要积分 \labelcref{equation:解析函数的积分表示.重要积分1} \begin{equation*}
	\int_\Gamma (z-a)^n \dd{z} = \left\{ \begin{array}{cl}
		2\pi\iu, & n=-1, \\
		0,& n\in\mathbb{Z}-\{-1\},
	\end{array} \right.
\end{equation*}
可得\begin{equation*}
	\int_\Gamma f(z) \dd{z} = 2\pi\iu C_{-1}.
\end{equation*}
也就是说,将\(f(z)\)的罗朗展式沿围线\(\Gamma\)积分后
将会只留下\((z-a)\)的负一次幂项积分不为零,
其余所有项积分通通为零.

由此可见,\(C_{-1}\)在罗朗展式的各个系数中具有独特的地位.
若不计因子\(2\pi\iu\),
它实质上表现的是围线积分\(\int_\Gamma f(z) \dd{z}\)的值.

\begin{definition}
设有限点\(a\)为函数\(f(z)\)的一个孤立奇点,
设\(f(z)\)在点\(a\)的去心邻域\(\mathring{U}(a,R)\)内解析,
取围线\(\Gamma: \abs{z-a}=\rho\in(0,R)\),
把积分\begin{equation*}
	\frac{1}{2\pi\iu} \int_\Gamma f(z) \dd{z}
\end{equation*}
称为“\(f(z)\)在点\(a\)的\DefineConcept{留数}(residue)”,
记作\(\Res_{z=a} f(z)\),
即\begin{equation}\label{equation:留数定理.留数定义1}
	\Res_{z=a} f(z)
	\defeq
	\frac{1}{2\pi\iu} \int_\Gamma f(z) \dd{z}
	\quad(0<\rho<R).
\end{equation}
\end{definition}

根据\hyperref[theorem:解析函数的积分表示.多连通区域的柯西积分定理]{多连通区域上的柯西积分定理}以及前面的讨论可知,\begin{equation}\label{equation:留数定理.留数定义2}
	\Res_{z=a} f(z) = C_{-1},
\end{equation}
其中\(C_{-1}\)为\(f(z)\)在点\(a\)去心邻域内罗朗展式的负一次幂系数.
显然,\cref{equation:留数定理.留数定义2} 也可作为\(f(z)\)在其孤立奇点\(a\)处的留数定义.

\begin{definition}
设\(z=\infty\)是函数\(f(z)\)的孤立奇点,
设\(f(z)\)在点\(\infty\)的去心邻域\(0 \leq r < \abs{z} < +\infty\)内解析.
取围线\(\Gamma: \abs{z} = \rho\in(r,+\infty)\),
把积分\begin{equation*}
	\frac{1}{2\pi\iu} \int_{\Gamma^-} f(z) \dd{z}
\end{equation*}称为“\(f(z)\)在无穷远点\(\infty\)的\DefineConcept{留数}”,
记为\(\Res_{z=\infty} f(z)\),
即\begin{equation}\label{equation:留数定理.留数定义3}
	\Res_{z=\infty} f(z)
	\defeq
	\frac{1}{2\pi\iu} \int_{\Gamma^-} f(z) \dd{z}.
\end{equation}
\end{definition}

假设\(f(z)\)在孤立奇点\(\infty\)的去心邻域内可展成罗朗级数\begin{equation*}
	f(z) = \sum_{n=-\infty}^\infty C_n z^n,
\end{equation*}
其中\(C_n\ (n=0,\pm1,\pm2,\dotsc)\)是复常数.
由\hyperref[theorem:解析函数的级数表示.一致收敛级数的基本性质2]{逐项积分定理}及重要积分 \labelcref{equation:解析函数的积分表示.重要积分1} 可知,
\begin{equation}\label{equation:留数定理.留数定义4}
	\Res_{z=\infty} f(z) = - C_{-1}.
\end{equation}
也就是说,\(\Res_{z=\infty} f(z)\)等于\(f(z)\)在点\(\infty\)的罗朗展式中\(\frac{1}{z}\)项系数的相反数.

这样,当我们知道点\(a\)是\(f(z)\)的孤立奇点时,
不论点\(a\)是有限点还是\(\infty\),
也不论点\(a\)是三类孤立奇点中的哪一种,
我们总可以用罗朗展开的方法,
按\cref{equation:留数定理.留数定义2}
或\cref{equation:留数定理.留数定义4}
求出\(f(z)\)在点\(a\)的留数.

\begin{property}
若\(a\)是可去奇点,
且\(a\neq\infty\),
则有\(\Res_{z=a} f(z) = 0\).
\end{property}

\begin{example}
函数\(f(z) = 1 + \frac{1}{z}\)在无穷远点\(\infty\)处解析,
但\(\Res_{z=\infty} f(z) = -1\).
此例说明,无穷远点\(\infty\)作为任意复变函数\(f(z)\)的可去奇点时,
\(f(z)\)的留数不一定是零.
\end{example}

\begin{property}
若\(a\neq\infty\)是\(f(z)\)的\(m\ (m\geq1)\)阶极点,
则\(f(z)\)在点\(a\)的留数为
\begin{equation}\label{equation:留数定理.函数在极点的留数1}
	\Res_{z=a} f(z)
	= \frac{1}{(m-1)!} \lim_{z \to a} \dv[m-1]{z} \{ (z-a)^m f(z) \}.
\end{equation}
特别地,当\(m=1\)时,
\begin{equation}\label{equation:留数定理.函数在极点的留数2}
	\Res_{z=a} f(z)
	= \lim_{z \to a} [(z-a) f(z)].
\end{equation}
\begin{proof}
当有限点\(a\)为\(f(z)\)的\(m\)阶极点时,
\(f(z)\)在点的去心邻域内可表成
\begin{equation}\label{equation:留数定理.函数在极点的留数.辅助1式}
	f(z) = \frac{1}{(z-a)^m} g(z),
\end{equation}
其中\(g(z)\)在点\(a\)是解析的,
且\(g(a)\neq0\),
\(g(z)\)在点\(a\)的邻域内可展成泰勒级数\begin{equation*}
	g(z) = \sum_{n=0}^\infty \frac{1}{n!} g^{(n)}(a) (z-a)^n,
\end{equation*}
于是由\hyperref[equation:解析函数的积分表示.柯西高阶导数公式]{柯西高阶导数公式}
\begin{equation}\label{equation:留数定理.函数在极点的留数.辅助2式}
	\Res_{z=a} f(z) = C_{-1}
	= \frac{1}{2\pi\iu} \int_{\abs{z-a}=\rho} \frac{g(z)}{(z-a)^m} \dd{z}
	= \frac{1}{(m-1)!} g^{(m-1)}(a).
\end{equation}
由于\cref{equation:留数定理.函数在极点的留数.辅助1式} 中的\(g(z)\)在点\(a\)解析,
所以\begin{equation*}
	g^{(m-1)}(a) = \lim_{z \to a} g^{(m-1)}(z)
	= \lim_{z \to a} [(z-a)^m f(z)],
\end{equation*}
因此\cref{equation:留数定理.函数在极点的留数.辅助2式}
即\cref{equation:留数定理.函数在极点的留数1}.

当\((z-a)^m f(z)\)中的\((z-a)^m\)因子能从\(f(z)\)中消去时,
就在消去后直接用\cref{equation:留数定理.函数在极点的留数.辅助2式}
计算\(\Res_{z=a} f(z)\),
否则才用极限等式 \labelcref{equation:留数定理.函数在极点的留数1} 计算.
这样做较为省事一些.

若\(f(z) = \frac{\phi(z)}{\psi(z)}\),
\(\phi(z)\)及\(\psi(z)\)在点\(a\)解析,
且\(\phi(a)\neq0\),
\(\psi(a)=0\),
\(\psi'(a)\neq0\),
则\(a\)为\(f(z)\)的一阶极点,
于是\begin{equation*}
	\Res_{z=a} f(z)
	= \lim_{z \to a} (z-a) \frac{\phi(z)}{\psi(z)}
	= \lim_{z \to a} \frac{\phi(z)}{\frac{\psi(z)-\psi(a)}{z-a}}
	= \frac{\phi(a)}{\psi'(a)},
\end{equation*}
即\begin{equation}\label{equation:留数定理.函数在极点的留数3}
	\Res_{z=a} f(z)
	= \frac{\phi(a)}{\psi'(a)}.
\end{equation}
这也是计算一阶极点留数的一个简便公式.
\end{proof}
\end{property}

\subsection{留数定理}
%@see: https://mathworld.wolfram.com/ResidueTheorem.html
\begin{theorem}[柯西留数定理]\label{theorem:留数定理.柯西留数定理}
若\(f(z)\)在围线或复围线\(\Gamma\)所围区域\(D\)内
除有限个奇点\(\AutoTuple{a}{n}\)外解析,
在闭域\(\overline{D}=D+\Gamma\)上除\(\AutoTuple{a}{n}\)外连续,
则\begin{equation}\label{equation:留数定理.柯西留数定理}
	\int_\Gamma f(z) \dd{z}
	= 2\pi\iu \sum_{k=1}^n \Res_{z=a_k} f(z).
\end{equation}
\begin{proof}
在\(D\)内以\(a_k\ (k=1,2,\dotsc,n)\)为中心作半径充分小的圆周\(\Gamma_k\),
使得每个\(\Gamma_k\)都在其余小圆周的外部,
即\begin{equation*}
	I(\Gamma_i) \cap I(\Gamma_j) = \emptyset\ (i \neq j),
\end{equation*}
而所有的小圆周又都在\(D\)的内部,
即\begin{equation*}
	\bigcup_k I(\Gamma_k) \subseteq I(\Gamma).
\end{equation*}
由\hyperref[theorem:解析函数的积分表示.多连通区域的柯西积分定理]{多连通区域上的柯西积分定理},
有\begin{equation*}
	\int_\Gamma f(z) \dd{z}
	= \sum_{k=1}^n \int_{\Gamma_k} f(z) \dd{z}.
\end{equation*}
但按留数定义\begin{equation*}
	\int_{\Gamma_k} f(z) \dd{z} = 2\pi\iu \Res_{z=a_k} f(z).
\end{equation*}
所以\cref{theorem:留数定理.柯西留数定理} 成立.
\end{proof}
\end{theorem}
柯西留数定理把计算围线积分的整体问题,
化成了计算围线内部各孤立奇点处留数的局部问题.
这是留数定理的实质.
这里也表现了解析函数的特性.
利用这种特性,一方面可以用来计算围线积分;
另一方面,我们后面将看到,可以用来考查解析函数的零点分布状况.

\begin{theorem}\label{theorem:留数定理.孤立奇点的留数之和}
若\(f(z)\)在全平面\(\mathbb{C}\)上除有限个点\(\AutoTuple{a}{n}\)外解析,
点\(\infty\)也为\(f(z)\)的孤立奇点,
则\begin{equation}\label{equation:留数定理.孤立奇点的留数之和}
	\sum_{k=1}^n \Res_{z=a_k} f(z) + \Res_{z=\infty} f(z) = 0.
\end{equation}
也就是说,\(f(z)\)所有孤立奇点的留数之和为零.
\begin{proof}
以原点为心,作半径\(R\)充分大的圆周\(\Gamma\),
使得\(\Gamma\)的内部包括\(\AutoTuple{a}{n}\).
由\cref{theorem:留数定理.柯西留数定理},\begin{equation*}
	\int_\Gamma f(z) \dd{z}
	= 2\pi\iu \sum_{k=1}^n \Res_{z=a_k} f(z).
\end{equation*}
而\begin{equation*}
	\frac{1}{2\pi\iu} \int_\Gamma f(z) \dd{z}
	= -\Res_{z=\infty} f(z),
\end{equation*}
所以立即得到\cref{equation:留数定理.孤立奇点的留数之和}.
\end{proof}
\end{theorem}

\cref{theorem:留数定理.柯西留数定理}
有时也称为“有界区域上的留数定理”,
而\cref{theorem:留数定理.孤立奇点的留数之和}
则称为“扩充平面\(\mathbb{C}_\infty\)上的留数定理”.
当点\(\AutoTuple{a}{n}\)上的留数容易求出,
而点\(\infty\)处的留数难于计算时,
可以用\cref{theorem:留数定理.孤立奇点的留数之和}
来计算点\(\infty\)处的留数.

除用\cref{equation:留数定理.柯西留数定理}
和\cref{equation:留数定理.孤立奇点的留数之和}
计算在点\(\infty\)处的留数外,
还可以用公式\begin{equation}
	\Res_{z=\infty} f(z)
	= -\Res_{\zeta=0}
	\left[
		f\left(\frac{1}{\zeta}\right) \frac{1}{\zeta^2}
	\right]
\end{equation}来计算.
这是因为经变量代换\(z = \frac{1}{\zeta}\)之后,
\(z\)沿\(\abs{z}=\rho\)正方向行进时,
\(\zeta\)沿\(\abs{\zeta}=\frac{1}{\rho}\)的负方向行进,
故\begin{align*}
	\Res_{z=\infty} f(z)
	&= -\frac{1}{2\pi\iu}
		\int_{\abs{z}=\rho} f(z) \dd{z} \\
	&= -\frac{1}{2\pi\iu}
		\int_{\abs{\zeta}=1/\rho}
		f\left(\frac{1}{\zeta}\right)
		\frac{1}{\zeta^2} \dd{\zeta} \\
	&= -\Res_{\zeta=0} \left[ f\left(\frac{1}{\zeta}\right) \frac{1}{\zeta^2} \right].
\end{align*}

\begin{example}
求函数\begin{equation*}
	f(z) = \frac{e^{\iu mz}}{1+z^2} \quad(m\in\mathbb{R}^*)
\end{equation*}在所有孤立奇点处的留数.
\begin{solution}
容易看出,\(z=\pm\iu\)是\(f(z) = \frac{e^{\iu mz}}{1+z^2}\)的一阶极点,
\(z=\infty\)是\(f(z)\)的本性奇点,
使用\cref{equation:留数定理.函数在极点的留数.辅助2式},
得\begin{equation*}
	\Res_{z=\iu} f(z)
	= \eval{\frac{e^{\iu mz}}{z+\iu}}_{z=\iu}
	= \frac{e^{-m}}{2\iu}
	= -\frac{\iu}{2} e^{-m},
\end{equation*}\begin{equation*}
	\Res_{z=-\iu} f(z)
	= \eval{\frac{e^{\iu mz}}{z-\iu}}_{z=-\iu}
	= \frac{\iu}{2} e^m.
\end{equation*}
应用\cref{theorem:留数定理.孤立奇点的留数之和} 得\begin{equation*}
	\Res_{z=\infty} f(z)
	= -\left[ \Res_{z=\iu} f(z) + \Res_{z=-\iu} f(z) \right]
	= -\iu \sinh m.
\end{equation*}
\end{solution}
\end{example}

\begin{example}
求函数\begin{equation*}
	f(z) = \frac{\sin 2z}{(z+1)^3}
\end{equation*}在所有孤立奇点处的留数.
\begin{solution}
可以求出,点\(z=-1\)是\(f(z)\)的3阶极点,
点\(z=\infty\)是\(f(z)\)的本性奇点.
于是\begin{equation*}
	\Res_{z=-1} f(z)
	= \frac{1}{2!} \left[ \dv[2]{z}(\sin 2z) \right]_{z=-1}
	= 2 \sin 2,
\end{equation*}
从而\(\Res_{z=\infty} f(z) = -2 \sin 2\).
\end{solution}
\end{example}

\begin{example}
求函数\begin{equation*}
	f(z) = \frac{(z^2-1)^2}{z^2(z-\alpha)(z-\beta)}
	\quad(\alpha\beta=1, \alpha\neq\beta)
\end{equation*}在所有孤立奇点处的留数.
\begin{solution}
可以求出,点\(\alpha,\beta\)是\(f(z)\)的一阶极点,
点\(z=0\)是\(f(z)\)的二阶极点,
点\(z=\infty\)是\(f(z)\)的可去奇点.
令\(f(\infty)=1\)后,
\(f(z)\)在\(z=\infty\)解析.
从而\begin{equation*}
	\Res_{z=\alpha} f(z)
	= \eval{ \frac{(z^2-1)^2}{z^2(z-\beta)} }_{z=\alpha}
	= \frac{(\alpha^2-1)^2}{\alpha^2(\alpha-\beta)}
	= \alpha-\beta,
\end{equation*}\begin{equation*}
	\Res_{z=\beta} f(z)
	= \eval{ \frac{(z^2-1)^2}{z^2(z-\alpha)} }_{z=\beta}
	= \frac{(\beta^2-1)^2}{\beta^2(\beta-\alpha)}
	= \beta-\alpha.
\end{equation*}
为了求出\(f(z)\)在点\(z=0\)的留数,
将\(f(z)\)表成\begin{equation*}
	f(z) = \frac{1}{z^2} \cdot \frac{(z^2-1)^2}{(z-\alpha)(z-\beta)}
	= \frac{1}{z^2} g(z),
\end{equation*}
其中\(g(z) = \frac{(z^2-1)^2}{(z-\alpha)(z-\beta)}\).
我们看到,只要求出\(g(z)\)在\(z=0\)的泰勒展式中的一次幂系数\(g'(0)\),
就可求得\(\Res_{z=0} f(z)\).
这是因为这时在\(z=0\)的去心邻域内,
\(f(z)\)的罗朗展式为\begin{align*}
	f(z) &= \frac{1}{z^2}
		\left[
			g(0) + g'(0) z
			+ \frac{g''(0)}{2!} z^2
			+ \frac{g'''(0)}{3!} z^3
			+ \dotsb
		\right] \\
	&= \frac{g(0)}{z^2}
		+ \frac{g'(0)}{z}
		+ \frac{g''(0)}{2!}
		+ \frac{g'''(0)}{3!} z
		+ \dotsb,
\end{align*}
所以\(\Res_{z=0} f(z) = g'(0)\).
现在将\(g(z)\)在\(z=0\)作泰勒展开,
得\begin{align*}
	g(z) &= \frac{(z^2-1)^2}{\alpha-\beta}
		\left(\frac{1}{z-\alpha}-\frac{1}{z-\beta}\right) \\
	&= (z^2-1)^2 \frac{1}{\alpha-\beta}
		\left[
			\frac{1}{\beta} \left(1+\frac{z}{\beta}+\dotsb\right)
			- \frac{1}{\alpha} \left(1+\frac{z}{\alpha}+\dotsb\right)
		\right] \\
	&= (z^2-1)^2 \frac{1}{\alpha-\beta}
		\left[
			\left(\frac{1}{\beta}-\frac{1}{\alpha}\right)
			+ \left(\frac{1}{\beta^2}-\frac{1}{\alpha^2}\right) z
			+ \dotsb
		\right].
\end{align*}

由此可见\(z\)的一次幂系数是
\(\frac{1}{\alpha-\beta} \left(\frac{1}{\beta^2}-\frac{1}{\alpha^2}\right)
= \alpha+\beta\),
所以\begin{equation*}
	\Res_{z=0} f(z) = \alpha+\beta.
\end{equation*}
\end{solution}
\end{example}

\begin{example}
计算:\begin{equation*}
	\int_{\abs{z}=n} \tan \pi z \dd{z}
	\quad(n\in\mathbb{N}^+).
\end{equation*}
\begin{solution}
由于函数\(f(z) = \tan \pi z = \frac{\sin \pi z}{\cos \pi z}\)
只以\(z=k+\frac{1}{2}\ (k=0,\pm1,\dotsc)\)为一阶极点.
由\cref{equation:留数定理.函数在极点的留数3},得\begin{equation*}
	\Res_{z=k+1/2} f(z)
	= \eval{\frac{\sin \pi z}{(\cos \pi z)'}}_{z=k+1/2}
	= -\frac{1}{\pi}.
\end{equation*}

在圆周\(\abs{z}=n\)的内部,\(\tan \pi z\)有\(2n\)个极点,
即\(k + \frac{1}{2}\ (k=0,\pm1,\dotsc,\pm(n+1),-n)\),
故由留数定理,得\begin{equation*}
	\int_{\abs{z}=n} f(z) \dd{z}
	= 2\pi\iu \sum_{\abs{k+\frac{1}{2}}<n} \Res_{z=k+1/2} f(z)
	= 2\pi\iu \left(-\frac{2n}{\pi}\right)
	= -4n\iu.
\end{equation*}
\end{solution}
\end{example}
