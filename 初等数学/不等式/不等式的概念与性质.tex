\section{不等式的概念与性质}
\begin{definition}
设\(a,b\in\mathbb{R}\).

如果\(a-b\)是正数,
则称“\(a\) \DefineConcept{大于} \(b\)”,
记作\(a>b\),
即\[
	a-b>0
	\iff
	a>b.
\]

如果\(a-b\)是负数,
则称“\(a\) \DefineConcept{小于} \(b\)”,
记作\(a<b\),
即\[
	a-b<0
	\iff
	a<b.
\]

如果\(a-b\)是零,
则称“\(a\) \DefineConcept{等于} \(b\)”,
记作\(a=b\),
即\[
	a-b=0
	\iff
	a=b.
\]

如果\(a-b\)是非负数,
则称“\(a\) \DefineConcept{大于或等于} \(b\)”,
记作\(a \geq b\),
即\[
	a-b\geq0
	\iff
	a \geq b.
\]

如果\(a-b\)是非正数,
则称“\(a\) \DefineConcept{小于或等于} \(b\)”,
记作\(a \leq b\),
即\[
	a-b\leq0
	\iff
	a \leq b.
\]

如果\(a-b\)不是零,
则称“\(a\) \DefineConcept{不等于} \(b\)”,
记作\(a \neq b\),
即\[
	a-b\neq0
	\iff
	a \neq b.
\]

我们把\[
	>, \qquad
	<, \qquad
	\neq, \qquad
	\geq, \qquad
	\leq,
\]这五个符号统称为\DefineConcept{不等号}.

用不等号连接两个解析式所得的式子,称为\DefineConcept{不等式}.
\end{definition}
应该注意到:\begin{gather*}
	a \geq b \iff a = b \lor a > b, \\
	a \leq b \iff a = b \lor a < b.
\end{gather*}

\begin{property}\label{theorem:不等式.不等式的对称性和传递性}
不等式具有以下性质:\begin{itemize}
	% 对称性
	\item \(a>b \iff b<a\).
	% 传递性1
	\item \(a>b \land b>c \implies a>c\).
	% 传递性2
	\item \(a<b \land b<c \implies a<c\).
\end{itemize}
\begin{proof}
由于正数的相反数是负数,负数的相反数是正数,得\[
	a > b \iff a-b > 0 \iff -(a-b) < 0 \iff b-a < 0 \iff b < a.
\]
根据两个正数的和仍是正数,得\[
	\left. \begin{array}{c}
		a > b \iff a-b > 0 \\
		b > c \iff b-c > 0
	\end{array} \right\}
	\implies (a-b)+(b-c) > 0
	\implies a-c > 0
	\implies a > c.
\]
同理可得\(a<b \land b<c \implies a<c\).
\end{proof}
\end{property}
我们为了书写简便,通常会把合取词\(\land\)省略,
把\(a>b \land b>c\)写成\(a>b>c\),
把\(a<b \land b<c\)写成\(a<b<c\),
把\(a=b \land b=c\)写成\(a=b=c\),
把\(a \geq b \land b \geq c\)写成\(a \geq b \geq c\),
把\(a \leq b \land b \leq c\)写成\(a \leq b \leq c\).

可以证明:\begin{gather*}
	a \leq b \leq c \implies a \leq c, \\
	a \geq b \geq c \implies a \geq c, \\
	a \leq b < c \implies a < c, \\
	a < b \leq c \implies a < c, \\
	a \geq b > c \implies a > c, \\
	a > b \geq c \implies a > c. \\
\end{gather*}

\begin{theorem}\label{theorem:不等式.加法的单调性}
如果\(a>b\),那么\(a+c>b+c\).
\begin{proof}
显然有\[
	a>b
	\iff a-b>0
	\iff (a+c)-(b+c)>0
	\iff a+c>b+c.
	\qedhere
\]
\end{proof}
\end{theorem}

\begin{corollary}
如果\(a+b>c\),那么\(a>c-b\).
\begin{proof}
显然有\[
	a+b>c
	\iff a+b+(-b)>c+(-b)
	\iff a>c-b.
	\qedhere
\]
\end{proof}
\end{corollary}
一般地说,不等式中任何一项的符号变成相反的符号后,应把它从一边移到另一边.

\begin{corollary}
如果\(a>b\)且\(c>d\),那么\(a+c>b+d\).
\begin{proof}
显然有\[
	\left. \begin{array}{c}
		a>b \iff a+c>b+c \\
		c>d \iff b+c>b+d
	\end{array} \right\}
	\implies a+c>b+d.
	\qedhere
\]
\end{proof}
\end{corollary}
这就是说,若干个同向不等式两边分别相加,所得不等式与原不等式同向.

\begin{example}
证明:如果\(a > b\)且\(c < d\),那么\(a - c > b - d\).
\begin{proof}
因为\(c < d\),所以\(-c > -d\).
又因为\(a > b\),\(a + (-c) > b + (-d)\),所以\(a - c > b - d\).
\end{proof}
\end{example}

\begin{example}
证明:\((n-1)(n+1)<n^2\).
\begin{proof}
因为\((n-1)(n+1)=n^2-1\),
而\(-1<0\),\(n^2-1<n^2\),
所以\((n-1)(n+1)<n^2\).
\end{proof}
\end{example}

\begin{theorem}\label{theorem:不等式.不等式与非零数相乘}
设\(a>b\).
如果\(c>0\),
那么\(ac>bc\);
如果\(c<0\),
那么\(ac<bc\).
\begin{proof}
根据同号相乘得正,
异号相乘得负,
有\[
	\left. \begin{array}{r}
		a>b \iff a-b>0 \\
		c>0
	\end{array} \right\}
	\implies (a-b)c>0
	\iff ac-bc>0
	\iff ac>bc;
\]
同理有\[
	\left. \begin{array}{r}
		a>b \iff a-b> 0 \\
		c<0
	\end{array} \right\}
	\implies (a-b)c<0
	\iff ac-bc<0
	\iff ac<bc.
	\qedhere
\]
\end{proof}
\end{theorem}

\begin{corollary}\label{theorem:不等式.不等式与负一相乘}
设\(a>b\),则\(-a<-b\).
\begin{proof}
在\cref{theorem:不等式.不等式与非零数相乘} 中
令\(c = -1\)便得.
\end{proof}
\end{corollary}

\begin{corollary}\label{theorem:不等式.同向不等式相乘}
如果\(a>b>0,c>d>0\),那么\(ac>bd>0\).
\begin{proof}
显然有\[
	\left. \begin{array}{r}
		a>b,c>0 \implies ac>bc \\
		c>d,b>0 \implies bc>bd
	\end{array} \right\}
	\implies ac>bd.
	\qedhere
\]
\end{proof}
\end{corollary}
这就是说,若干个两边都是正数的同向不等式两边分别相乘,所得不等式与原不等式同向.

\begin{corollary}
如果\(a>b>0,d>c>0\),那么\(\frac{a}{c}>\frac{b}{d}>0\).
\end{corollary}

\begin{example}
证明:如果\(a > b\)且\(ab > 0\),那么\(\frac1a < \frac1b\).
\begin{proof}
因为\(ab > 0\),\(\frac1{ab} > 0\),
所以\(b \cdot \frac1{ab} < a \cdot \frac1{ab}\),
\(\frac1a < \frac1b\).
\end{proof}
\end{example}

\begin{corollary}
如果\(a>b>0\),那么\(a^n>b^n>0 \quad (n\in\mathbb{N}^+)\).
\begin{proof}
由\cref{theorem:不等式.同向不等式相乘} 归纳可得.
\end{proof}
\end{corollary}
\begin{theorem}
设\(a>b\geq0\).
\begin{itemize}
	\item 若\(c>0\),则\(a^c>b^c\).
	\item 若\(c<0\),则\(a^c<b^c\).
\end{itemize}
%TODO proof
\end{theorem}
\begin{theorem}
设\(x>y\).
\begin{itemize}
	\item 若\(a>1\),
	则\(a^x>a^y\).
	\item 若\(0<a<1\),
	则\(a^x<a^y\).
\end{itemize}
%TODO proof
\end{theorem}
\begin{theorem}
设\(x>y>0\).
\begin{itemize}
	\item 若\(a>1\),则\(\log_a x > \log_a y\).
	\item 若\(0<a<1\),则\(\log_a x < \log_a y\).
\end{itemize}
%TODO proof
\end{theorem}

\begin{example}
证明:如果\(a > b > 0\)且\(c < d < 0\),那么\(ac < bd < 0\).
\begin{proof}
因为\(c < d < 0\),\(-c > -d > 0\),\(a(-c) > b(-d) > 0\),
所以\(ac < bd < 0\).
\end{proof}
\end{example}

\begin{corollary}\label{theorem:不等式.正整数次幂的序}
设\(m,n\in\mathbb{N}^+\)且\(m>n\).
\begin{itemize}
	\item 当\(a>1\)时,\(a^m > a^n \geq a > 1\).
	\item 当\(a=1\)时,\(a^m = a^n = 1\).
	\item 当\(0<a<1\)时,\(0 < a^m < a^n \leq a < 1\).
	\item 当\(a=0\)时,\(a^m = a^n = 0\).
\end{itemize}
\begin{proof}
根据幂的定义,第2、4种情形是显然的.
现在来证第1种情形,因为\[
	a > 1
	\implies
	a^2 = a \cdot a > 1 \cdot a = a
	\implies
	a^3 > a^2 \geq a,
\]
运用数学归纳法可得\[
	a>1
	\implies
	(\forall m,n\in\mathbb{N}^+)
	[m>n \implies a^m > a^n \geq a > 1].
\]

再证第3种情形,因为\[
	1>a>0
	\implies
	a = 1 \cdot a > a \cdot a = a^2
	\implies
	a \geq a^2 > a^3,
\]
运用数学归纳法可得\[
	0<a<1
	\implies
	(\forall m,n\in\mathbb{N}^+)
	[m>n \implies 0 < a^m < a^n \leq a < 1].
	\qedhere
\]
\end{proof}
\end{corollary}

\begin{theorem}
如果\(a>b>0\),那么\(\sqrt[n]{a} > \sqrt[n]{b} \quad (n\in\mathbb{N}^+)\).
\begin{proof}
用反证法.
假设当\(a>b>0\)时,\(\sqrt[n]{a} \ngtr \sqrt[n]{b}\),
那么\[
	\sqrt[n]{a} < \sqrt[n]{b}
	\quad\lor\quad
	\sqrt[n]{a} = \sqrt[n]{b}
\]成立.
但是\[
	\sqrt[n]{a} < \sqrt[n]{b} \implies a<b,
	\qquad
	\sqrt[n]{a} = \sqrt[n]{b} \implies a=b.
\]矛盾!
故\(\sqrt[n]{a}>\sqrt[n]{b}\)成立.
\end{proof}
\end{theorem}

\begin{theorem}
设\(m,n\in\mathbb{N}^+\)且\(m>n\).
\begin{itemize}
	\item 当\(a>1\)时,\(a \geq a^{\frac1n} > a^{\frac1m} > 1\).
	\item 当\(0<a<1\)时,\(0 < a^{\frac1n} < a^{\frac1m} \leq a < 1\).
\end{itemize}
%TODO proof
% \begin{proof}
% 由\(m>n\)得\(1\leq\frac1m<\frac1n\),
% \end{proof}
\end{theorem}

\begin{example}
证明:\(\abs{a} \geq 0\).
\begin{proof}
我们可以按\(a\)的取值分为两种情况讨论:
\begin{itemize}
	\item 当\(a \geq 0\)时,\(\abs{a}=a\),
	原式化为\(a \geq 0\),成立.
	\item 当\(a < 0\)时,\(\abs{a}=-a\),
	原式化为\(-a \geq 0\)即\(a \leq 0\),成立.
	\qedhere
\end{itemize}
\end{proof}
\end{example}
\begin{example}\label{example:不等式.数及其绝对值的序}
证明:\(-\abs{a} \leq a \leq \abs{a}\).
\begin{proof}
我们可以按\(a\)的取值分为两种情况讨论:
\begin{itemize}
	\item 当\(a \geq 0\)时,\(\abs{a}=a\),
	原式化为\(-a \leq a \leq a\),成立.
	\item 当\(a < 0\)时,\(\abs{a}=-a\),
	原式化为\(a \leq a \leq -a\),成立.
	\qedhere
\end{itemize}
\end{proof}
\end{example}
\begin{example}\label{example:不等式.数的绝对值及其上界的关系1}
证明:\(\abs{a} \leq b
\iff
-b \leq a \leq b\).
\begin{proof}
假设\(\abs{a} \leq b\),
那么由\cref{theorem:不等式.不等式与负一相乘}
可得\(-b \leq -\abs{a}\).
又因为 \hyperref[example:不等式.数及其绝对值的序]{\(-\abs{a} \leq a \leq \abs{a}\)},
所以由\hyperref[theorem:不等式.不等式的对称性和传递性]{传递性}可知
\(-b \leq a \leq b\).

假设\(-b \leq a \leq b\).
用反证法.
如果\(\abs{a} > b\),
那么由\cref{theorem:不等式.不等式与负一相乘}
可得\(-\abs{a} < -b\),
于是由\hyperref[theorem:不等式.不等式的对称性和传递性]{传递性}可知\begin{equation*}
	-\abs{a} < a < \abs{a}.
\end{equation*}
当\(a\geq0\)即\(\abs{a} = a\)时,上式化为\(-a < a < a\),不成立;
当\(a<0\)即\(\abs{a} = -a\)时,上式化为\(a < a < -a\),也不成立.
因此\(\abs{a} \leq b\).
\end{proof}
\end{example}
\begin{example}\label{example:不等式.数的绝对值及其上界的关系2}
设\(b>0\).
证明:\(\abs{a} \geq b
\iff
a \geq b \lor a \leq -b\).
\begin{proof}
这是\cref{example:不等式.数的绝对值及其上界的关系1} 的逆否命题.
\end{proof}
\end{example}
\begin{example}\label{example:不等式.数的上下界}
设\(a < x < b\).
证明:\(\abs{x} < \abs{a}\)或\(\abs{x} < \abs{b}\).
%TODO proof
% \begin{proof}
% 用反证法.
% 假设\(\abs{x} \geq \abs{a}\)且\(\abs{x} \geq \abs{b}\),
% 那么
% \end{proof}
\end{example}
