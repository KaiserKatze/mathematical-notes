\chapter{复数概论}
\section{复数的形式与运算}
\subsection{复数的代数形式}
\begin{definition}[虚数单位]
规定:满足\(\iu^2=-1\)的数\(\iu\)称为\DefineConcept{虚数单位}.
\end{definition}

\begin{definition}
把由有序实数对\((x,y)\)作代数组合所确定的数\(z=x+\iu y\)称为代数形式的\DefineConcept{复数}.
实数\(x\)、\(y\)分别称为复数\(z=x+\iu y\)的\DefineConcept{实部}、\DefineConcept{虚部},记作\(x=\Re z\)、\(y=\Im z\).

特别地,当\(\Im z=0\)时,\(z=\Re z=x\)是实数;
当\(\Re z=0\)时,\(z=\iu\Im z=\iu y\),称为\DefineConcept{纯虚数}.
\end{definition}

\begin{definition}[代数形式下复数相等条件]
设\(z_1\)和\(z_2\)都是复数,则当\(\Re z_1 = \Re z_2\)和\(\Im z_1 = \Im z_2\)同时成立时,则称\(z_1 = z_2\).
特别地,对于复数\(z\),则当且仅当\(\Re z=0\)且\(\Im z=0\)时,\(z=0\).
\end{definition}

\begin{definition}[共轭复数]
设\(z=x + \iu y \in\mathbb{C}\),其中\(x,y\in\mathbb{C}\),称复数\(\ComplexConjugate{z}=x - \iu y\)为\(z\)的\DefineConcept{共轭复数}(conjugate).
\end{definition}

\subsection{复数的模}
\begin{definition}[复数的模]
设复数\(z = x + \iu y\),称\begin{equation*}
\abs{z} = \sqrt{x^2 + y^2}
\end{equation*}为\(z\)的\DefineConcept{模}或\DefineConcept{绝对值}.
\end{definition}

\begin{property}
共轭复数以及复数的模具有以下性质:
\begin{gather*}
	\Re \ComplexConjugate{z} = \Re z,
	\Im \ComplexConjugate{z} = -\Im z,
	\ComplexConjugate{(\ComplexConjugate{z})} = z,
	z+\ComplexConjugate{z} = 2 \Re z,
	z-\ComplexConjugate{z} = 2\iu \Im z,
	z\ComplexConjugate{z} = \abs{z}^2,
	\abs{\ComplexConjugate{z}}=\abs{z},
	\abs{z_1 z_2} = \abs{z_1} \abs{z_2},
	\abs{\frac{z_1}{z_2}} = \frac{\abs{z_1}}{\abs{z_2}} \quad(z_2 \neq 0),
	\ComplexConjugate{z_1 \pm z_2} = \ComplexConjugate{z_1} \pm \ComplexConjugate{z_2},
	\ComplexConjugate{z_1 z_2} = \ComplexConjugate{z_1} \cdot \ComplexConjugate{z_2},
	\ComplexConjugate{\left(\frac{z_1}{z_2}\right)} = \frac{\ComplexConjugate{z_1}}{\ComplexConjugate{z_2}} \quad (z_2 \neq 0),
	-\abs{z} \leq \Re z \leq \abs{z} \leq \abs{\Re z} + \abs{\Im z},
	-\abs{z} \leq \Im z \leq \abs{z} \leq \abs{\Re z} + \abs{\Im z}.
\end{gather*}
\end{property}

\subsection{复数的代数运算}
\begin{definition}[复数加法]
设\(z_1=x_1+\iu y_1\),\(z_2=x_2+\iu y_2\),
定义\begin{equation*}
z_1+z_2=(x_1+x_2)+\iu(y_1+y_2)
\end{equation*}为复数\(z_1\)和\(z_2\)的加法运算.
\end{definition}

\begin{definition}[复数减法]
设\(z_1=x_1+\iu y_1\),\(z_2=x_2+\iu y_2\),定义\begin{equation*}
z_1-z_2=(x_1-x_2)+\iu(y_1-y_2)
\end{equation*}为复数的\(z_1\)和\(z_2\)的减法运算.
显然,复数减法是加法的逆运算.
\end{definition}

\begin{definition}[复数乘法]
设\(z_1 = x_1 + \iu y_1\),\(z_2 = x_2 + \iu y_2\),定义\begin{equation*}
z_1 \cdot z_2
= (x_1 + \iu y_1)(x_2 + \iu y_2)
= (x_1 x_2 - y_1 y_2)+\iu(x_1 y_2 + x_2 y_1)
\end{equation*}为复数\(z_1\)和\(z_2\)的乘法运算.
\end{definition}

\begin{definition}[复数除法]
设\(z_1 = x_1 + \iu y_1\),\(z_2 = x_2 + \iu y_2 \neq 0\),定义满足\begin{equation*}
z_1 = z \cdot z_2
\end{equation*}的复数\(z = x + \iu y\)为复数\(z_1\)和\(z_2\)的商,记作\begin{equation*}
z = \frac{z_1}{z_2},
\end{equation*}称为复数\(z_1\)和\(z_2\)的除法运算.

显然,复数的除法是乘法的逆运算.
\end{definition}

\begin{theorem}
设\(z_1 = x_1 + \iu y_1\),\(z_2 = x_2 + \iu y_2 \neq 0\),则\begin{equation*}
\frac{z_1}{z_2}
= \frac{z_1 \cdot \ComplexConjugate{z_2}}{z_2 \cdot \ComplexConjugate{z_2}}
= \frac{z_1 \cdot \ComplexConjugate{z_2}}{\abs{z_2}^2}
= \frac{x_1 x_2 + y_1 y_2}{x_2^2 + y_2^2}
+ \iu \frac{x_2 y_1 - x_1 y_2}{x_2^2 + y_2^2}.
\end{equation*}
\begin{proof}
设\(z = x + \iu y = \frac{z_1}{z_2}\),则\begin{equation*}
z \cdot z_2 = (x + \iu y)(x_2 + \iu y_2)
= (x_2 x - y_2 y) + \iu(y_2 x + x_2 y)
= x_1 + \iu y_1,
\end{equation*}从而有方程组\begin{equation*}
\left\{ \begin{array}{l}
x_2 x - y_2 y = x_1, \\
y_2 x + x_2 y = y_1,
\end{array} \right.
\end{equation*}解得\begin{equation*}
x = \frac{x_1 x_2 + y_1 y_2}{x_2^2 + y_2^2},
\quad
y = \frac{x_2 y_1 - x_1 y_2}{x_2^2 + y_2^2}.
\end{equation*}
\end{proof}
\end{theorem}

\begin{theorem}
若\(z,w \in \mathbb{C}\),则有\begin{equation}
\abs{z \pm w}^2 = \abs{z}^2 + \abs{w}^2 \pm 2 \Re(z \ComplexConjugate{w}).
\end{equation}
\begin{proof}
\(
\abs{z + w}^2
= (z + w) (\ComplexConjugate{z + w})
= z\ComplexConjugate{z} + z\ComplexConjugate{w} + w\ComplexConjugate{z} + w\ComplexConjugate{w}
= \abs{z}^2 + 2 \Re(z\ComplexConjugate{w}) + \abs{w}^2
\).
\end{proof}
\end{theorem}

\begin{theorem}
若\(z,w \in \mathbb{C}\),则有\begin{equation}
\abs{z + w}^2 \leq (\abs{z} + \abs{w})^2.
\end{equation}
\begin{proof}
\(
\abs{z + w}^2
= \abs{z}^2 + 2 \Re(z \ComplexConjugate{w}) + \abs{w}^2
\leq \abs{z}^2 + 2 \abs{z}\abs{\ComplexConjugate{w}} + \abs{w}^2
= (\abs{z} + \abs{w})^2
\).
\end{proof}
\end{theorem}

\begin{theorem}[三角不等式]
若\(z,w \in \mathbb{C}\),则有\begin{equation}
\abs{\abs{z}-\abs{w}} \leq \abs{z \pm w} \leq \abs{z} + \abs{w}.
\end{equation}
\begin{proof}
因为\begin{equation*}
\abs{z + w}^2 \leq (\abs{z} + \abs{w})^2,
\end{equation*}所以\(\abs{z + w} \leq \abs{z} + \abs{w}\).

又因为\begin{equation*}
\abs{z} = \abs{z + w + (-w)} \leq \abs{z+w} + \abs{-w} = \abs{z+w} + \abs{w},
\end{equation*}所以\(\abs{z}-\abs{w} \leq \abs{z+w}\).

同样地,有\(\abs{w}-\abs{z} \leq \abs{z+w}\).

综上所述,\(\abs{\abs{z}-\abs{w}}\leq\abs{z+w}\).
\end{proof}
\end{theorem}

\begin{example}
证明:\(\abs{z_1+z_2}^2 + \abs{z_1-z_2}^2 = 2 (\abs{z_1}^2 + \abs{z_2}^2)\).
\begin{proof}
记\(z_1 = x_1 + \iu y_1, z_2 = x_2 + \iu y_2\).那么\begin{equation*}
z_1+z_2 = (x_1+x_2) + \iu(y_1+y_2),
\end{equation*}\begin{equation*}
\abs{z_1+z_2}^2 = (x_1+x_2)^2 + (y_1+y_2)^2;
\end{equation*}同理\(\abs{z_1-z_2}^2 = (x_1-x_2)^2 + (y_1-y_2)^2\).那么\begin{align*}
\abs{z_1+z_2}^2 + \abs{z_1-z_2}^2
&= (x_1+x_2)^2 + (y_1+y_2)^2
+ (x_1-x_2)^2 + (y_1-y_2)^2 \\
&= 2 ( x_1^2 + x_2^2 + y_1^2 + y_2^2 )
= 2 ( \abs{z_1}^2 + \abs{z_2}^2 ).
\qedhere
\end{align*}
\end{proof}
\end{example}

\subsection{复数的几何表示}
\begin{definition}[复数在复平面上的几何表示]
在直角坐标系\(xOy\)上可以用点\((x,y)\)表示复数\(z=x+\iu y\),
也可以用向量\((x,y)\)表示复数\(z=x+\iu y\).
与复数建立了这种对应关系的坐标平面\(xOy\)称为\DefineConcept{复平面}(complex plane),
%@see: https://mathworld.wolfram.com/ComplexPlane.html
记作\(C\).
称\(x\)轴为复平面的\DefineConcept{实轴}(real axis).
%@see: https://mathworld.wolfram.com/RealAxis.html
%@see: https://mathworld.wolfram.com/RealLine.html
称\(y\)轴为复平面的\DefineConcept{虚轴}(imaginary axis).
%@see: https://mathworld.wolfram.com/ImaginaryAxis.html
%@see: https://mathworld.wolfram.com/ImaginaryLine.html

显然,表示复数\(z\)的点与表示其共轭复数\(\ComplexConjugate{z}\)的点关于实轴对称.
\end{definition}

\begin{definition}[复数在复球面上的几何表示]
在\(Ox_1x_2x_3\)坐标系下,考虑单位球面\(S\)(即球心位于原点、半径为1的球面):\begin{equation*}
	x_1^2+x_2^2+x_3^2=1
\end{equation*}
点\((0,0,1)\)称为北极,记作\(N\),
同时\(x_1Ox_2\)平面取为复平面\(C\).
复平面\(C\)交球面\(S\)于单位球的赤道.

对于复平面\(C\)上的每一个点\(z\),它与\(N\)连接的直线必与\(S\)交且只交于一点\(Z \neq N\).
若\(\abs{z} < 1\),则点\(Z\)在下半球面上;
若\(\abs{z} > 1\),则点\(Z\)在上半球面上;
若\(\abs{z} = 1\),则点\(Z\)在赤道上.
反之,取球面上任意一点\(Z \neq N\),连接它与\(N\)的直线也只与复平面\(C\)交于一点\(z\).

可见,除北极\(N=(0,0,1)\)以外,复平面\(C\)和球面\(S\)上的点是一一对应的.
并且当\(\abs{z} \to +\infty\)时,\(Z \to N\).
那么可以假想一个模为无穷大的复数,
称作\DefineConcept{无穷远点}(point at infinity),
%@see: https://mathworld.wolfram.com/PointatInfinity.html
%@see: https://mathworld.wolfram.com/ComplexInfinity.html
记作\(z = \infty\),
作为复平面\(C\)上与复球面北极\(N\)对应的点.

加上无穷远点后的复平面称为\DefineConcept{扩充复平面}(extended complex plane),
%@see: https://mathworld.wolfram.com/ExtendedComplexPlane.html
记作\(C_\infty\),即\begin{equation*}
C_\infty = C \cup \{\infty\}.
\end{equation*}
扩充复平面\(C_\infty\)又称为\DefineConcept{闭平面}.
对应地,复平面\(C\)因为不含无穷远点,所以又称为\DefineConcept{开平面}.

复球面\(S\)与扩充复平面\(C_\infty\)上点之间的映射称为\DefineConcept{球极射影}.
\(S\)又称为\DefineConcept{黎曼复球面}(Riemann sphere).
%@see: https://mathworld.wolfram.com/RiemannSphere.html

另外,对于\(\infty\)还有以下几点值得注意:
\begin{enumerate}
\item \(\infty\)的实部\(\Re\infty\)、虚部\(\Im\infty\)、辐角\(\Arg\infty\)均无意义,其模\(\abs{\infty}=+\infty\);
\item 运算\(\infty \pm \infty\)、\(0 \cdot \infty\)、\(\frac{\infty}{\infty}\)均无意义;
\item 设复数\(z \neq \infty\),有\(z \pm \infty = \infty \pm z = \infty\),\(\frac{z}{\infty} = 0\);
\item 设复数\(z \neq 0\),有\(z \cdot \infty = \infty \cdot z = \infty\),\(\frac{z}{0} = \infty\);
\item 设复数\(z \neq 0\)且\(z \neq \infty\),有\(\frac{\infty}{z} = \infty\);
\item 在扩充复平面\(C_\infty\)上,任一直线都是通过无穷远点\(\infty\)的.同时,没有一个半平面包含点\(\infty\).
\end{enumerate}
\end{definition}

\begin{theorem}
设复数\(z=x+\iu y\),其对应的复球面上的点为\(Z=\opair{x_1,x_2,x_3}\),并满足:

当\(z \neq \infty\)时,\(Z\)的坐标为\begin{equation*}
x_1 = \frac{z + \ComplexConjugate{z}}{\abs{z}^2 + 1}, \qquad
x_2 = \iu\frac{\ComplexConjugate{z} - z}{\abs{z}^2 + 1}, \qquad
x_3 = \frac{\abs{z}^2 - 1}{\abs{z}^2 + 1}
\end{equation*}或\begin{equation*}
x_1 = \frac{2x}{x^2+y^2+1}, \quad
x_2 = \frac{2y}{x^2+y^2+1}, \quad
x_3 = \frac{x^2+y^2-1}{x^2+y^2+1}.
\end{equation*}

当\(z = \infty\)时,\(Z\)的坐标为\(N = (0,0,1)\).
\end{theorem}

\begin{theorem}
已知复球面上一点\(Z=(x_1,x_2,x_3)\),则其对应的复平面上的点为\begin{equation*}
z = x+\iu y = \frac{x_1+\iu x_2}{1-x_3}
\end{equation*}
\end{theorem}
