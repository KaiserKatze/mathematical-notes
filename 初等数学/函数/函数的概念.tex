\section{函数的概念}
\subsection{一元函数的概念}
设\(D\)是\(\mathbb{R}\)的一个非空子集,
称映射\(f\colon D \to \mathbb{R}\)为
“定义在\(D\)上的一个\DefineConcept{一元函数}”,
简称\DefineConcept{函数},通常记为\[
	y = f(x), \qquad x \in D,
\]
其中点集\(D\)称为该函数的\DefineConcept{定义域},
\(x\)称为\DefineConcept{自变量},
\(y\)称为\DefineConcept{因变量}.

函数定义中,对每个\(x \in D\),
按对应法则\(f\),总有唯一确定的值\(y\)与之相应,
这个值称为函数\(f\)在\(x\)处的\DefineConcept{函数值},
记作\(f(x)\),即\(y=f(x)\).
因变量\(y\)与自变量\(x\)之间的这种依赖关系,通常称为\DefineConcept{函数关系}.
函数值\(f(x)\)的全体所构成的集合称为函数\(f\)的\DefineConcept{值域},
记作\(R_f\)或\(f\ImageOfSetUnderRelation{D}\),即\[
	R_f = f\ImageOfSetUnderRelation{D} = \Set{ y \given y = f(x) \land x \in D }.
\]

函数的定义域通常按以下两种情形来确定:
\begin{itemize}
	\item 一种是对有实际背景的函数,
	根据实际背景中变量的实际意义来确定.
	\item 另一种是对抽象地用算式表达的函数,
	通常约定这种函数的定义域是使得算式有意义的一切实数组成的集合,
	这种定义域称为函数的\DefineConcept{自然定义域}.
\end{itemize}

\begin{example}
函数\(y = \frac{1}{x}\)的自然定义域是
\(\Set{ x \in \mathbb{R} \given x \neq 0 }\).
\end{example}

\begin{example}
函数\(y = \sqrt{x}\)的自然定义域是
\(\Set{ x \in \mathbb{R} \given x \geq 0 }\).
\end{example}

如果给定一个对应法则,按这个法则,
对每个\(x \in D\),总有确定但不唯一的\(y\)与之对应.
这样的不符合函数的定义的对应法则,
习惯上称这种法则确定了一个\DefineConcept{多值函数}.
对于多值函数,如果我们附加一些条件,使得在附加条件之下,按照这个法则,
对每个\(x \in D\),总有唯一确定的实数值\(y\)与之对应,那么这就确定了一个函数.
我们称这样得到的函数为多值函数的\DefineConcept{单值分支}.

\subsection{多元函数的概念}
\begin{definition}
设\(D\)是\(\mathbb{R}^2\)的一个非空子集,
称映射\(f\colon D \to \mathbb{R}\)为
“定义在\(D\)上的\DefineConcept{二元函数}”,
通常记为\[
	z = f(x,y),
	\quad \opair{x,y} \in D
\]或\[
	z = f(P),
	\quad P\opair{x,y} \in D
\]
其中点集\(D\)称为该函数的\DefineConcept{定义域},
\(x\)、\(y\)称为\DefineConcept{自变量},
\(z\)称为\DefineConcept{因变量}.
\end{definition}

\begin{definition}
类似地,设\(D\)是\(n\)维空间\(\mathbb{R}^n\)的一个非空子集,
称映射\[
	f\colon D \to \mathbb{R}
\]
为“定义在\(D\)上的\(n\)元\DefineConcept{函数}”,
通常记为\[
	y = f(\AutoTuple{x}{n}),
	\quad \opair{\AutoTuple{x}{n}} \in D
\]或\[
	y = f(\mat{x}),
	\quad \mat{x}=\opair{\AutoTuple{x}{n}} \in D
\]
其中点集\(D\)称为“函数\(f\)的\DefineConcept{定义域}”,
\(x\)、\(y\)称为“函数\(f\)的\DefineConcept{自变量}”,
\(z\)称为“函数\(f\)的\DefineConcept{因变量}”.
\end{definition}
