\section{四元数}
四元数是由1个实数单位与3个虚数单位\(\iu,\ju,\ku\)组成的、配上4个实系数的超复数,
其代数表达式为\begin{equation*}
	\lambda_0 \cdot 1
	+ \lambda_1 \cdot \iu
	+ \lambda_2 \cdot \ju
	+ \lambda_3 \cdot \ku.
\end{equation*}

当\(\lambda_2 = \lambda_3 = 0\)时,四元数就变成了复数.

我们把\begin{equation*}
	\lambda_0
\end{equation*}
称为“四元数\(
	(
		\lambda_0 \cdot 1
		+ \lambda_1 \cdot \iu
		+ \lambda_2 \cdot \ju
		+ \lambda_3 \cdot \ku
	)
\)的\DefineConcept{标量部分}”.
把\begin{equation*}
	\lambda_1 \cdot \iu
	+ \lambda_2 \cdot \ju
	+ \lambda_3 \cdot \ku
\end{equation*}
称为“四元数\(
	(
		\lambda_0 \cdot 1
		+ \lambda_1 \cdot \iu
		+ \lambda_2 \cdot \ju
		+ \lambda_3 \cdot \ku
	)
\)的\DefineConcept{向量部分}”.
把\begin{equation*}
	\sqrt{
		\lambda_1^2
		+ \lambda_2^2
		+ \lambda_3^2
	}
\end{equation*}
称为“四元数\(
	(
		\lambda_0 \cdot 1
		+ \lambda_1 \cdot \iu
		+ \lambda_2 \cdot \ju
		+ \lambda_3 \cdot \ku
	)
\)的向量部分的\DefineConcept{模}”.
把\begin{equation*}
	\sqrt{
		\lambda_0^2
		+ \lambda_1^2
		+ \lambda_2^2
		+ \lambda_3^2
	}
\end{equation*}
称为“四元数\(
	q
	\defeq
	(
		\lambda_0 \cdot 1
		+ \lambda_1 \cdot \iu
		+ \lambda_2 \cdot \ju
		+ \lambda_3 \cdot \ku
	)
\)的\DefineConcept{模}”,
记作\(\norm{q}\).

\begin{theorem}
对于任意四元数\(
	q
	\defeq
	\lambda_0 \cdot 1
	+ \lambda_1 \cdot \iu
	+ \lambda_2 \cdot \ju
	+ \lambda_3 \cdot \ku
\),
存在实数\(\theta\),
使得\begin{equation*}
	q = N (\cos\theta + \xi \sin\theta),
\end{equation*}
其中\begin{equation*}
	N
	\defeq
	\sqrt{
		\lambda_0^2
		+ \lambda_1^2
		+ \lambda_2^2
		+ \lambda_3^2
	},
	\qquad
	\xi
	\defeq
	\frac{
		\lambda_1 \cdot \iu
		+ \lambda_2 \cdot \ju
		+ \lambda_3 \cdot \ku
	}{
		\sqrt{
			\lambda_1^2
			+ \lambda_2^2
			+ \lambda_3^2
		}
	}.
\end{equation*}
% 只要使\(\cos\theta = \lambda_0 / N, \sin\theta = \sqrt{\lambda_1^2 + \lambda_2^2 + \lambda_3^2} / N\)即可.
\end{theorem}

我们把\begin{equation*}
	1 \defeq 1 \cdot 1 + 0 \cdot \iu + 0 \cdot \ju + 0 \cdot \ku
\end{equation*}
称为\DefineConcept{单元}.
把\begin{equation*}
	0 \defeq 0 \cdot 1 + 0 \cdot \iu + 0 \cdot \ju + 0 \cdot \ku
\end{equation*}
称为\DefineConcept{零元}.
把\begin{equation*}
	- \lambda_0 \cdot 1
	- \lambda_1 \cdot \iu
	- \lambda_2 \cdot \ju
	- \lambda_3 \cdot \ku
\end{equation*}
称为“四元数\(
	q
	\defeq
	(
		\lambda_0 \cdot 1
		+ \lambda_1 \cdot \iu
		+ \lambda_2 \cdot \ju
		+ \lambda_3 \cdot \ku
	)
\)的\DefineConcept{负元}”,
记作\((-q)\).
把\begin{equation*}
	\lambda_0 \cdot 1
	- \lambda_1 \cdot \iu
	- \lambda_2 \cdot \ju
	- \lambda_3 \cdot \ku
\end{equation*}
称为“四元数\(
	q
	\defeq
	(
		\lambda_0 \cdot 1
		+ \lambda_1 \cdot \iu
		+ \lambda_2 \cdot \ju
		+ \lambda_3 \cdot \ku
	)
\)的\DefineConcept{共轭元}”,
记作\(q^*\).
把\begin{equation*}
	\frac{
		\lambda_0 \cdot 1
		- \lambda_1 \cdot \iu
		- \lambda_2 \cdot \ju
		- \lambda_3 \cdot \ku
	}{
		\lambda_0^2
		+ \lambda_1^2
		+ \lambda_2^2
		+ \lambda_3^2
	}
\end{equation*}
称为“四元数\(
	q
	\defeq
	(
		\lambda_0 \cdot 1
		+ \lambda_1 \cdot \iu
		+ \lambda_2 \cdot \ju
		+ \lambda_3 \cdot \ku
	)
\)的\DefineConcept{逆元}”,
记作\(q^{-1}\).

\begin{definition}
给定四元数\(
	p
	\defeq
	\lambda_0 \cdot 1
	+ \lambda_1 \cdot \iu
	+ \lambda_2 \cdot \ju
	+ \lambda_3 \cdot \ku
\)和\(
	q
	\defeq
	\mu_0 \cdot 1
	+ \mu_1 \cdot \iu
	+ \mu_2 \cdot \ju
	+ \mu_3 \cdot \ku
\).
\begin{itemize}
	\item 把\begin{equation}
		p + q
		\defeq
		(\lambda_0 + \mu_0) \cdot 1
		+ (\lambda_1 + \mu_1) \cdot \iu
		+ (\lambda_2 + \mu_2) \cdot \ju
		+ (\lambda_3 + \mu_3) \cdot \ku
	\end{equation}
	称为“四元数\(p\)与\(q\)的\DefineConcept{和}”.

	\item 把\begin{equation}
		p - q
		\defeq
		(\lambda_0 - \mu_0) \cdot 1
		+ (\lambda_1 - \mu_1) \cdot \iu
		+ (\lambda_2 - \mu_2) \cdot \ju
		+ (\lambda_3 - \mu_3) \cdot \ku
	\end{equation}
	称为“四元数\(p\)与\(q\)的\DefineConcept{差}”.

	\item 把\begin{equation}
		\begin{aligned}
			p \cdot q
			&\defeq
			(\lambda_0 \mu_0 - \lambda_1 \mu_1 - \lambda_2 \mu_2 - \lambda_3 \mu_3) \cdot 1 \\
			&\hspace{20pt}
				+ (\lambda_1 \mu_0 + \lambda_0 \mu_1 - \lambda_3 \mu_2 + \lambda_2 \mu_3) \cdot \iu \\
			&\hspace{20pt}
				+ (\lambda_2 \mu_0 + \lambda_3 \mu_1 + \lambda_0 \mu_2 - \lambda_1 \mu_3) \cdot \ju \\
			&\hspace{20pt}
				+ (\lambda_3 \mu_0 - \lambda_2 \mu_1 + \lambda_1 \mu_2 + \lambda_0 \mu_3) \cdot \ku,
		\end{aligned}
	\end{equation}
	称为“四元数\(p\)与\(q\)的\DefineConcept{积}”.
\end{itemize}
\end{definition}

\begin{property}
四元数满足结合律.
\end{property}

\begin{property}
四元数满足分配律.
\end{property}

\begin{example}
举例说明:四元数的乘法不满足交换律.
\end{example}

\begin{property}
设\(q\)是四元数,
\(q^*\)是\(q\)的共轭元,
则\(q \cdot q^* = \norm{q}\).
\end{property}

\begin{property}
设\(q\)是四元数,
\(q^*\)是\(q\)的共轭元,
\(q^{-1}\)是\(q\)的逆元,
则\(q^{-1} = q^* / \norm{q}\).
\end{property}

\begin{theorem}
给定四元数\(
	p
	\defeq
	\lambda_0 \cdot 1
	+ \lambda_1 \cdot \iu
	+ \lambda_2 \cdot \ju
	+ \lambda_3 \cdot \ku
\)和\(
	q
	\defeq
	\mu_0 \cdot 1
	+ \mu_1 \cdot \iu
	+ \mu_2 \cdot \ju
	+ \mu_3 \cdot \ku
\),
有\begin{align*}
	p \cdot q
	&= \begin{bmatrix}
		1 & \iu & \ju & \ku \\
	\end{bmatrix}
	\begin{bmatrix}
		\mu_0 & -\mu_1 & -\mu_2 & -\mu_3 \\
		\mu_1 & \mu_0 & -\mu_3 & \mu_2 \\
		\mu_2 & \mu_3 & \mu_0 & -\mu_1 \\
		\mu_3 & -\mu_2 & \mu_1 & \mu_0 \\
	\end{bmatrix}
	\begin{bmatrix}
		\lambda_0 \\
		\lambda_1 \\
		\lambda_2 \\
		\lambda_3 \\
	\end{bmatrix} \\
	&= \begin{bmatrix}
		1 & \iu & \ju & \ku \\
	\end{bmatrix}
	\begin{bmatrix}
		\lambda_0 & -\lambda_1 & -\lambda_2 & -\lambda_3 \\
		\lambda_1 & \lambda_0 & \lambda_3 & -\lambda_2 \\
		\lambda_2 & -\lambda_3 & \lambda_0 & \lambda_1 \\
		\lambda_3 & \lambda_2 & -\lambda_1 & \lambda_0 \\
	\end{bmatrix}
	\begin{bmatrix}
		\mu_0 \\
		\mu_1 \\
		\mu_2 \\
		\mu_3 \\
	\end{bmatrix}.
\end{align*}
\end{theorem}

\begin{definition}
给定四元数\(
	p
	\defeq
	\lambda_0 \cdot 1
	+ \lambda_1 \cdot \iu
	+ \lambda_2 \cdot \ju
	+ \lambda_3 \cdot \ku
\).
把矩阵\begin{equation}
	\begin{bmatrix}
		\mu_0 & -\mu_3 & \mu_2 \\
		\mu_3 & \mu_0 & -\mu_1 \\
		-\mu_2 & \mu_1 & \mu_0 \\
	\end{bmatrix}
\end{equation}
称为“四元数\(p\)的\DefineConcept{核}”.
\end{definition}
\begin{remark}
可以看出:
四元数\(
	p
	\defeq
	\lambda_0 \cdot 1
	+ \lambda_1 \cdot \iu
	+ \lambda_2 \cdot \ju
	+ \lambda_3 \cdot \ku
\)的核是以它的分量\(
	\lambda_0,
	\lambda_1,
	\lambda_2,
	\lambda_3
\)作为元素,
构成的反对称矩阵.
\end{remark}

\begin{table}[hbt]
%@see: 《离散数学》(邓辉文) P151 表5-7
	\centering
	\begin{tblr}{r|*3r}
		\(\times\) & \(\iu\) & \(\ju\) & \(\ku\) \\ \hline
		\(\iu\) & \(-1\) & \(\ku\) & \(-\ju\) \\
		\(\ju\) & \(-\ku\) & \(-1\) & \(\iu\) \\
		\(\ku\) & \(\ju\) & \(-\iu\) & \(-1\) \\
	\end{tblr}
	\caption{四元数的虚数单位的乘法运算}
	\label{equation:四元数.四元数的虚数单位的乘法运算}
\end{table}
