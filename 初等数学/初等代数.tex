\chapter{初等代数}
\section{二项式定理}
利用乘法分配律,我们能够很轻松地得到\DefineConcept{完全平方和公式}%
\begin{equation}
	(a + b)^2 = a^2 + 2ab + b^2,
\end{equation}
和\DefineConcept{完全立方和公式}%
\begin{equation}
	(a + b)^3 = a^3 + 3 a^2 b + 3 a b^2 + b^3.
\end{equation}

我们不禁想要知道一般的二项式\((a+b)^n\)应该如何展开.

\subsection{正整指数}
我们注意到,利用乘法分配律可以得到\begin{align*}
	(x+a_1)(x+a_2)
	&= x(x+a_2) + a_1(x+a_2) \\
	&= x^2 + a_2x + a_1x + a_1a_2 \\
	&= x^2 + (a_1+a_2)x + a_1a_2;
\end{align*}
类似地,有\[
	(x+a_1)(x+a_2)(x+a_3)
	= x^3 + (a_1+a_2+a_3)x^2 + (a_1a_2+a_1a_3+a_2a_3)x + a_1a_2a_3;
\]\begin{align*}
	(x+a_1)(x+a_2)(x+a_3)(x+a_4)
	&= x^4 + (a_1+a_2+a_3+a_4)x^3 \\
		&\hspace{20pt}+ (a_1a_2+a_1a_3+a_1a_4+a_2a_3+a_2a_4+a_3a_4)x^2 \\
		&\hspace{20pt}+ (a_1a_2a_3+a_1a_2a_4+a_1a_3a_4+a_2a_3a_4)x^3 + a_1a_2a_3a_4.
\end{align*}
从这些结果里,我们可以总结出如下经验规律:
\begin{itemize}
	\item 等号右边的项数比左边二项因子的个数多1.
	\item 第一项的\(x\)的指数等于二项因子的个数,
	以后各项\(x\)的指数依次比前面一项小1.
	\item 第一项的系数为1;第二项的系数是数\(a,b,c,\dotsc\)的和;
	第三项的系数是从这\(n\)个数里一次取2个相乘的积的和;
	第四项的系数是从这\(n\)个数里一次取3个相乘的积的和;
	以此类推;最后一项是所有这\(n\)个数的乘积.
\end{itemize}

利用数学归纳法.
假设这些规律在左边有\(n-1\)个二项因子的情况下适用,
即设\[
	(x+a_1)(x+a_2)\dotsm(x+a_{n-1})
	= x^{n-1}+p_1 x^{n-2}+p_2 x^{n-3}+\dotsb+p_{n-1},
\]
其中\[
	p_1 = a_1+a_2+\dotsb+a_{n-1},
	p_2 = a_1a_2+\dotsb+a_2a_3+\dotsb,
	\dotsc,
	p_{n-1} = a_1 a_2 \dotsm a_{n-1}.
\]
在等式两边再乘以一个因子\(x+a_n\),则有\begin{equation}\label{equation:二项式定理.更一般的二项式的展开式}
	\begin{split}
	(x+a_1)(x+a_2)\dotsm(x+a_{n-1})(x+a_n)
	&= x^n + (p_1+a_n)x^{n-1}
		+ (p_2+p_1a_n)x^{n-2} \\
		&\hspace{20pt}+ (p_3+p_2a_n)x^{n-3}
		+ \dotsb + p_{n-1} a_n.
	\end{split}
\end{equation}
由于\(p_1+a_n = (a_1+a_2+\dotsb+a_{n-1})+a_n\)
是\(n\)个数\(\AutoTuple{a}{n}\)的和,
\(p_2+p_1a_n = p_2+(a_1+a_2+\dotsb+a_{n-1})a_n\)
是\(n\)个数\(\AutoTuple{a}{n}\)一次选两个的乘积的和,
\(p_3+p_2a_n = p_3+(a_1a_2+\dotsb+a_2a_3+\dotsb)a_n\)
是\(n\)个数\(\AutoTuple{a}{n}\)一次选三个的乘积的和;
以此类推;\(p_{n-1} a_n = (a_1 a_2 \dotsm a_{n-1})a_n\)
是所有\(n\)个数\(\AutoTuple{a}{n}\)的乘积.
因此,上述规律对任意多个二项因子的乘积都成立.

在\cref{equation:二项式定理.更一般的二项式的展开式} 中,
它的第一项的系数是\(1 = C_n^0\);
它的第二项系数\(p_1+a_n\)的项数等于\(n\),即从\(n\)个元素中取出一个的取法\(C_n^1\);
它的第三项系数\(p_2+p_1a_n\)的项数等于从\(n\)个元素中取出两个的取法\(C_n^2\);
它的第四项系数\(p_3+p_2a_n\)的项数等于从\(n\)个元素中取出三个的取法\(C_n^3\);
以此类推;最后一项的系数\(p_{n-1} a_n\)只有一项,等于从\(n\)个元素中取出全部\(n\)个的取法\(C_n^n\).
因此,如果我们令\[
	a_1 = a_2 = a_3 = \dotsb = a_{n-1} = a_n = a,
\]
那么有
\begin{equation}\label{equation:二项式定理.二项式的展开式}
	(x+a)^n
	= C_n^0 a^0 x^n + C_n^1 a x^{n-1} + C_n^2 a^2 x^{n-2}
	+ \dotsb + C_n^{n-1} a^{n-1} x + C_n^n a^n.
\end{equation}
这就是\DefineConcept{二项式定理}.
我们把\cref{equation:二项式定理.二项式的展开式} 等号右边的部分\[
	\sum_{k=0}^n C_n^k a^k x^{n-k}
\]称为“\((x+a)^n\)的展开式”;
把第\(k+1\)项\[
	C_n^k a^k x^{n-k}
\]称为“\((x+a)^n\)的\DefineConcept{通项}或\DefineConcept{一般项}”.
可以注意到,在展开式的任意一项\[
	C_n^k a^k x^{n-k}
	\quad(k=0,1,2,\dotsc,n)
\]中,\(a\)的指数都等于组合符号\(C_n^k\)的右上标\(k\),
而\(x\)与\(a\)的指数之和等于组合符号\(C_n^k\)的右下标\(n\).

如果我们用\((-a)\)代替\(a\),那么\cref{equation:二项式定理.二项式的展开式} 会变为\[
	(x-a)^n = C_n^0 (-a)^0 x^n + C_n^1 (-a) x^{n-1} + C_n^2 (-a)^2 x^{n-2}
	+ \dotsb + C_n^{n-1} (-a)^{n-1} x + C_n^n (-a)^n.
\]
可以看出,\((x+a)^n\)与\((x-a)^n\)的展开式各项在绝对值上是相等的,即\[
	\abs{C_n^k (-a)^k x^{n-k}}
	= \abs{C_n^k a^k x^{n-k}}.
\]
但是在\((x-a)^n\)的展开式里,各项依次地正负交错,
最后一项的正负号则依\(n\)为奇数还是偶数确定.

\begin{example}
求\((1+x)^n\)展开式的各项系数之和.
\begin{solution}
由\cref{equation:二项式定理.二项式的展开式} 有,\[
	(1+x)^n = 1+C_n^1 x+C_n^2 x^2+\dotsb+C_n^n x^n.
\]
令\(x=1\),上式就变为\[
	1+C_n^1+C_n^2+\dotsb+C_n^n = 2^n.
\]
\end{solution}
这也算是对\cref{theorem:组合数性质3} 的另一种证明方法.
\end{example}

\begin{example}
证明:在\((1+x)^n\)的展开式中,奇位项系数之和等于偶位项系数之和.
\begin{proof}
在恒等式\[
	(1+x)^n = 1+C_n^1 x+C_n^2 x^2+C_n^3 x^3+C_n^4 x^4+C_n^5 x^5+\dotsb+C_n^n x^n
\]中,取\(x=-1\)得\[
	0 = 1-C_n^1+C_n^2-C_n^3+C_n^4-C_n^5\dotsb+(-1)^n C_n^n;
	\eqno(1)
\]
取\(x=1\)得\[
	2^n = 1+C_n^1+C_n^2+C_n^3+C_n^4+C_n^5+\dotsb+C_n^n.
	\eqno(2)
\]

将(1)式与(2)式相加,得\[
	2^n = 2(1+C_n^2+C_n^4+\dotsb)
	= 2 \sum_{\substack{
		0 \leq 2k \leq n \\
		k=0,1,2\dotsc
	}} C_n^{2k},
\]
于是偶位项系数之和为\[
	1+C_n^2+C_n^4+\dotsb = 2^{n-1}.
\]

将(2)式与(1)式相减,得\[
	2^n = 2(C_n^1+C_n^3+C_n^5+\dotsb)
	= 2 \sum_{\substack{
		1 \leq 2k+1 \leq n \\
		k=0,1,2,\dotsc
	}} C_n^{2k+1},
\]
于是奇位项系数之和为\[
	C_n^1+C_n^3+C_n^5+\dotsb = 2^{n-1}.
\]

综上所述,我们有\[
	1+C_n^2+C_n^4+\dotsb
	= C_n^1+C_n^3+C_n^5+\dotsb
	= 2^{n-1},
\]
也就是说,偶位项系数之和、奇位项系数之和两者都等于\(2^{n-1}\).
\end{proof}
\end{example}

\subsection{有理指数}
在上一小节,我们研究了指数为正整数的二项式定理.
现在我们来考虑当指数为负数或分数时,二项式定理是否仍然成立.

设\[
	(1+x)^m
	= 1 + mx + \frac{m(m-1)}{1\cdot2}x^2
	+ \frac{m(m-1)(m-2)}{1\cdot2\cdot3}x^3
	+ \dotsb
\]与\[
	(1+x)^n
	= 1 + nx + \frac{n(n-1)}{1\cdot2}x^2
	+ \frac{n(n-1)(n-2)}{1\cdot2\cdot3}x^3
	+ \dotsb
\]的乘积为\[
	1 + A_1 x + A_2 x^2 + A_3 x^3 + \dotsb.
\]
显然,\(A_1,A_2,A_3,\dotsc\)是\(m\)与\(n\)的函数,
所以,在任何特定的情况下,
\(A_1,A_2,A_3,\dotsc\)的实际取值依赖于\(m\)与\(n\)的值.
但是,在上述两个乘数中,\(x\)的幂的系数结合成\(A_1,A_2,A_3,\dotsc\)的方式,是与\(m,n\)无关的.
换句话说,无论\(m\)与\(n\)取什么值,\(A_1,A_2,A_3,\dotsc\)总保持同样的形式.
所以,如果对任意一组\(m,n\)的值我们能决定\(A_1,A_2,A_3,\dotsc\)的形式,
那么就能做出结论,对于\(m,n\)的所有值,\(A_1,A_2,A_3,\dotsc\)都有同样的形式.

上面阐述的这个原则,作为一个“等价形式的不变性”的例子,经常会被提到.
目前,我们只要承认这样的事实:
在任何代数乘积里,无论所牵涉的量是整数还是分数,是正数还是负数,其结果的形式总不变.
我们将运用这一原理,为有理指数的二项式定理给出一般性的证明.
这一证明是欧拉给出的.

我们先来证明指数为正分数的二项式定理.
%TODO

\section{常见代数公式}

\begin{theorem}[平方差、立方差公式]
\[
a^2 - b^2 = (a-b)(a+b),
\]\[
a^3 - b^3 = (a-b)(a^2+ab+b^2),
\]

推广一下可得,当\(n \in \mathbb{N}^+\)时,有\[
a^n - b^n = (a-b) \sum_{k=0}^{n-1}{a^{n-1-k} b^k}
= (a-b)(a^{n-1} + a^{n-2} b + \dotsb + b^{n-1}).
\]
\end{theorem}

设\(n\)是正整数,则\(x^n-1\)总可被\(x-1\)整除,且\[
	\frac{x^n-1}{x-1}
	= x^{n-1} + \frac{x^{n-1}-1}{x-1}.
\]

\section{初等代数方程}
我们在生活中,经常会遇到这样的问题:
一些量的取值已知,而另一些量的取值未知,
这些已知量和未知量的数量关系已知,
求解未知量的取值范围.
例如,已知一个正常人总有两条腿,
假设一群人站在操场上,
你数出他们的腿共有20条,
要求这群人的人数.
那么你可以用字母\(x\)表示这群人的人数,
因为一个正常人有两条腿,所以这群人的腿一共有\(2x\)条,
也就是说\(2x=20\).
于是我们可以计算得出\(x=\frac{20}{2}=10\),
那么操场上有10个人.
像这样,我们就可以把很多问题归结为求解未知量的取值范围的问题.

下面,我们给出“方程”的定义.

我们知道,对于任意一条合式公式\(\phi\),
我们总是可以用受限变元\(x\)代替\(\phi\)中的某个自由变元\(a\),
得到一个新的公式\(\phi(x)\).
我们把\(\phi(x)\)称为“关于未知量\(x\)的\DefineConcept{方程}(algebraic equation)”,
%@see: https://mathworld.wolfram.com/AlgebraicEquation.html
%@see: https://mathworld.wolfram.com/Unknown.html
把类\[
	\Set{ x \given \phi(x) }
\]称为“方程\(\phi(x)\)的\DefineConcept{解}”.

特别地,如果\(X\)是集合,那么我们把集合\[
	\Set{ x \in X \given \phi(x) }
\]称为“方程\(\phi(x)\)的\(X\)~\DefineConcept{解}”.

例如,对于给定的方程\[
	x(x-1)(x^2-2)(x^2+1)=0,
\]
当\(X=\mathbb{Z}^+\)时,我们可以解出它的\emph{正整数}解\(\{1\}\);
当\(X=\mathbb{Q}\)时,我们可以解出它的\emph{有理数}解\(\{0,1\}\);
当\(X=\mathbb{R}\)时,我们可以解出它的\emph{实数}解\(\{0,1,\pm\sqrt2\}\);
当\(X=\mathbb{C}\)时,我们可以解出它的\emph{复数}解\(\{0,1,\pm\sqrt2,\pm\iu\}\).

如果\[
	\Set{ x \in X \given \phi(x) } = \emptyset,
\]
那么称“方程\(\phi(x)\)没有\(X\)~\DefineConcept{解}”.

%@see: https://mathworld.wolfram.com/VietasFormulas.html
%@see: https://davidaltizio.web.illinois.edu/Vietas%20Formulas.pdf
\subsection{一元二次方程}
一元二次方程的一般形式为:
\begin{equation}\label[quadratic-equation]{equation:一元二次方程.一元二次方程的一般形式}
	ax^2 + bx + c = 0, \quad a \neq 0,
\end{equation}
其中,\(ax^2\)称为\DefineConcept{二次项},
\(bx\)称为\DefineConcept{一次项},
\(c\)称为\DefineConcept{常数项},
把\(a\)、\(b\)、\(c\)统称为“\cref{equation:一元二次方程.一元二次方程的一般形式} 的\DefineConcept{系数}”.

\cref{equation:一元二次方程.一元二次方程的一般形式} 两端同除以\(a\),得\[
	x^2 + \frac{b}{a} x + \frac{c}{a} = 0,
\]
配方,得\[
	\left( x + \frac{b}{2a} \right)^2 + \left( \frac{c}{a} - \frac{b^2}{4a^2} \right) = 0,
\]
移项,再开方,得\[
	x = -\frac{b}{2a} \pm \sqrt{\frac{b^2}{4a^2} - \frac{c}{a}}
	= -\frac{b}{2a} \pm \sqrt{\frac{b^2-4ac}{4a^2}}
	= \frac{-b \pm \sqrt{b^2-4ac}}{2a}.
\]
于是我们得到一元二次方程\(ax^2 + bx + c = 0\ (a\neq0)\)的两个解\[
	x_1 = \frac{-b + \sqrt{b^2-4ac}}{2a},
	\qquad
	x_2 = \frac{-b - \sqrt{b^2-4ac}}{2a}.
\]

\begin{theorem}
记\(\Delta \defeq b^2-4ac\),
称之为\cref{equation:一元二次方程.一元二次方程的一般形式}
的\DefineConcept{判别式}({\rm discriminant}).
\begin{itemize}
	\item 当\(\Delta > 0\)时,它有两个不同的实根\[
		x = \frac{-b \pm \sqrt{\Delta}}{2a}.
	\]
	\item 当\(\Delta = 0\)时,它有两个相同的实根\[
		x = -\frac{b}{2a}.
	\]
	\item 当\(\Delta < 0\)时,它有一对共轭复根\[
		x = \frac{-b \pm \iu \sqrt{-\Delta}}{2a}.
	\]
\end{itemize}
\end{theorem}

\begin{theorem}[韦达定理]
设\(x_1,x_2\)是\cref{equation:一元二次方程.一元二次方程的一般形式} 的两个根,
则有\[
	x_1 + x_2 = -\frac{b}{a},
	\qquad
	x_1 \cdot x_2 = \frac{c}{a}.
\]
\begin{proof}
因为\(x_1,x_2\)是一元二次方程\(ax^2 + bx + c = 0\)的两个根,
所以原方程可化为\[
	a(x - x_1)(x - x_2) = 0
	\quad\text{或}\quad
	a x^2 - a (x_1 + x_2) x + a x_1 x_2 = 0.
\]
将上式与\cref{equation:一元二次方程.一元二次方程的一般形式} 比较可得\[
	b = -a (x_1 + x_2),
	\qquad
	c = a x_1 x_2.
\]
整理得\(x_1 + x_2 = -\frac{b}{a}, x_1 \cdot x_2 = \frac{c}{a}\).
\end{proof}
\end{theorem}

\subsection{一元三次方程}
对于一般的一元三次方程\begin{equation}\label[cubic-equation]{equation:一元三次方程.一元三次方程的一般形式}
	ax^3+bx^2+cx+d=0 \quad(a\neq0),
\end{equation}
我们总可通过以下步骤将其化为标准形式.

首先在等号两边同除以\(a\),得\[
	x^3+\frac{b}{a}x^2+\frac{c}{a}x+\frac{d}{a}=0,
\]
再令\(x=y-\frac{b}{3a}\),得\[
	\left(y-\frac{b}{3a}\right)^3+\frac{b}{a}\left(y-\frac{b}{3a}\right)^2+\frac{c}{a}\left(y-\frac{b}{3a}\right)+\frac{d}{a}=0,
\]
整理得\begin{equation}
	y^3+py+q=0,
\end{equation}
其中\[
	p = \frac{3ac-b^2}{3a^2}, \qquad
	q = \frac{2b^3}{27a^3}-\frac{bc}{3a^2}+\frac{d}{a}.
\]

\begin{theorem}[卡丹公式]
\def\a{-\frac{q}{2}}%
\def\d{\frac{q^2}{4}+\frac{p^3}{27}}
\def\b{\sqrt{\d}}%
\def\c#1{\sqrt[3]{\a#1\b}}%
形如\[
	x^3 + px + q = 0 \quad (p,q \in \mathbb{C})
\]的一元三次方程的解为\[
	x = \c{+}+\c{-}.
\]

令\[
	\alpha=\c{+}, \qquad \beta=\c{-}.
\]
总成立\[
	\alpha \beta = -\frac{p}{3}.
\]

当\(p,q\in\mathbb{R}\)时,判别式\[
	\Delta \defeq -108\left(\d\right) = -27q^2-4p^3
\]的正负号决定了\(x^3+px+q=0\)的根的性质:\begin{enumerate}
	\item 当\(\Delta>0\)时,方程的三个根是各不相同的实根.
	\item 当\(\Delta=0\)时,\begin{enumerate}
		\item 如果\(p=q=0\),则方程有三重实根;
		\item 如果\(p\neq0\)且\(q\neq0\),则方程有一个二重实根和一个与之不同的实根.
		\end{enumerate}
	\item 当\(\Delta<0\)时,方程的三个根各不相同,其中一个是实根,两个是共轭复根.
\end{enumerate}
\end{theorem}

\begin{theorem}[韦达定理]
设\(x_1,x_2,x_3\)是\cref{equation:一元三次方程.一元三次方程的一般形式} 的三个根,
则有\[
	x_1 + x_2 + x_3 = -\frac{b}{a},
	\qquad
	x_1 x_2 + x_2 x_3 + x_3 x_1 = \frac{c}{a},
	\qquad
	x_1 x_2 x_3 = -\frac{d}{a}.
\]
\end{theorem}
