\begingroup
\chapter{初等数论}
数论在一些地方总有让人意想不到的妙用.
有这样一种说法:“自然科学的皇后是数学,数学的皇冠是数论,哥德巴赫猜想则是皇冠上的明珠.”
在这一章,我们介绍一些基础的数论理论,供大家参考.

\section{素数与合数}
在正整数集上,由于没有引入负数、分数的概念,
正整数\(x\)与\(y\)相除的结果除了商\(z\)以外,
还经常是带有余数\(r\)的,于是有关系式\[
	x = y z + r.
\]

\begin{definition}
在正整数集上,如果\(x\)与\(y\)相除余\(r=0\),
我们就称“\(y\)可以\DefineConcept{整除} \(x\)”
或“\(x\)可以被\(y\)整除”
或“\(x\)是\(y\)的\DefineConcept{倍数}”
或“\(y\)是\(x\)的\DefineConcept{因子}”,
记作\(\truediv{x}{y}\),
即\[
%@see: 《具体数学 计算机科学基础(第2版)》 P85 (4.1)
	\truediv{x}{y}
	\defiff
	y > 0
	\land
	(\exists z\in\mathbb{Z})
	[x = y z].
\]
\end{definition}

\begin{property}
%@see: 《具体数学 计算机科学基础(第2版)》 P85
没有任何整数可以被\((-1)\)整除.
\end{property}

\begin{definition}
\(a\)与\(b\)的\DefineConcept{最大公约数}(greatest common divisor),
记作\((a,b)\),
即\[
%@see: 《具体数学 计算机科学基础(第2版)》 P85 (4.2)
	(a,b) \defeq \max\Set{ k \given (\truediv{a}{k}), (\truediv{b}{k}) }.
\]
\end{definition}
\begin{definition}
\(a\)与\(b\)的\DefineConcept{最小公倍数}(least common multiple),
记作\([a,b]\),
即\[
%@see: 《具体数学 计算机科学基础(第2版)》 P86 (4.3)
	[a,b] \defeq \min\Set{ k \given k>0, (\truediv{k}{a}), (\truediv{k}{b}) }.
\]
\end{definition}
在本章以外,为了防止最大公约数、最小公倍数记号与开区间、闭区间符号混淆,
分别记作\(\gcd{p,q}\)和\(\lcm{p,q}\).

\begin{property}
%@see: 《具体数学 计算机科学基础(第2版)》 P86 (4.4)
设\(m,n\)都是整数.
\begin{itemize}
	\item 如果\(n>0\),那么\((0,n)=n\).
	\item 如果\(0<m<n\),那么\((m,n) = (n \mod m,m)\).
\end{itemize}
%TODO proof
\end{property}
\begin{remark}
\(0\)与它自身的最大公约数\((0,0)\)是未定义的.
\end{remark}

\begin{theorem}
\(a\)与\(b\)的最大公约数\((a,b)\)和最小公倍数\([a,b]\)的乘积
等于\(a\)与\(b\)之积,
即\begin{equation}
	(a,b) \cdot [a,b] = ab.
\end{equation}
%TODO proof
\end{theorem}

\begin{definition}
如果正整数\(x\)只能被1或它自己整除\footnote{特别地,我们规定1不是素数.},
那么称\(x\)为\DefineConcept{素数}或\DefineConcept{质数}(prime number);
否则称之为\DefineConcept{合数}(composite number).
\end{definition}
这就是说:\[
	\text{$x$是素数}
	\defiff
	(\forall y)(\forall z)
	[y \cdot z=x \implies y=1 \lor z=1].
\]

\begin{definition}
如果正整数\(x\)与\(y\)除了1以外没有其他公因子,
就称“\(x\)与\(y\) \DefineConcept{互素}(或\DefineConcept{互质})”.
\end{definition}

全体素数的集合,
记作\(\mathbb{S}\),
即\[
	\mathbb{S}
	\defeq
	\Set*{
		x\in\mathbb{N}^+-\{1\}
		\given
		(\forall y\in\mathbb{N}^+)
		[y \neq x \implies (x,y)=1]
	}.
\]

\begin{theorem}
如果一个正整数\(a\)能整除正整数的乘积\(bc\),且\(a\)与其中一个因子\(b\)互素,
则\(a\)必能整除另一个因子\(c\).
\end{theorem}
\begin{corollary}
如果素数\(a\)能整除正整数的乘积\(bcd\dotsm\),则\(a\)必能整除该乘积的一个因子.
\end{corollary}
\begin{corollary}
如果素数\(a\)能整除正整数的乘积\(b^n\ (n\in\mathbb{N}^+)\),那么\(a\)必能整除\(b\).
\end{corollary}

\begin{theorem}
如果正整数\(a\)与正整数\(b,c\)均互素,则它必与乘积\(bc\)互素.
\end{theorem}
\begin{corollary}
如果正整数\(a\)与正整数的乘积\(bc\)互素,则\(a\)与正整数\(b,c\)均互素.
\end{corollary}
\begin{corollary}
如果正整数\(a\)与\(b\)互素,则\(a^m\ (m\in\mathbb{N}^+)\)与\(b^n\ (n\in\mathbb{N}^+)\)互素.
\end{corollary}

\begin{theorem}
如果正整数\(a\)与\(b\)互素,则分数\(\frac{a^m}{b^n}\ (m,n\in\mathbb{N}^+)\)均为既约分数.
\end{theorem}
\begin{corollary}
设\(a,b,c,d\in\mathbb{N}^+\).
如果\(\frac{a}{b} = \frac{c}{d}\),
且\(\frac{a}{b}\)是既约分数,
则\[
	(\forall k\in\mathbb{R}^*)
	[c = ka, d = kb].
\]
\end{corollary}

不难发现,第一个素数是2,它也是素数中唯一一个偶数.
第二个素数是3,接下来是5、7、11……
我们不禁发问,素数的个数究竟是有限的,还是无限的.
\begin{theorem}
素数有无穷多个,
即\[
	(\forall x)(\exists y)[y>x \land \text{$y$是素数}].
\]
\begin{proof}
用反证法\footnote{这个证法是由古希腊数学家欧几里得给出的.}.
假设素数只有\(k\)个,分别是\(\AutoTuple{p}{k}\),且\(p_1 < p_2 < \dotsb < p_k\).
显然,数\(\truediv{p_1 p_2 \dotsm p_k}{p_i}\ (i=1,2,\dotsc,k)\).
但数\[
	N = 1 + p_1 p_2 \dotsm p_k
\]与\(p_i\ (i=1,2,\dotsc,k)\)相除总余1,
这就说明数\(N\)要么是一个素数,
要么是一个可以被区间\((p_k,N)\)内的一个正整数\(M\)整除的合数,
总之素数的个数大于\(k\).
以此类推,可知素数必然有无穷多个.
\end{proof}
\end{theorem}
\begin{definition}\label{definition:初等数论.欧几里得数}
%@see: 《具体数学 计算机科学基础(第2版)》 P90 (4.16)
定义\[
	e_n \defeq \left\{ \begin{array}{cl}
		2, & n=1, \\
		e_1 e_2 \dotsm e_{n-1} + 1, & n\geq2,
	\end{array} \right.
\]
把\(e_n\)称为“第\(n\)个\DefineConcept{欧几里得数}(Euclid number)”.
%@see: https://mathworld.wolfram.com/EuclidNumber.html
\end{definition}
\begin{table}[htb]
	\centering
	\begin{tblr}{c|r|l}
		\hline
		\(n\) & \(e_n\) & \\
		\hline
		\(1\) & \(2\) & 素数 \\
		\(2\) & \(3\) & 素数 \\
		\(3\) & \(7\) & 素数 \\
		\(4\) & \(43\) & 素数 \\
		\(5\) & \(1~807\) & 合数,可以分解为\(13\times139\) \\
		\(6\) & \(3~263~443\) & 素数 \\
		\(7\) & \(10~650~056~950~807\) & 合数,可以分解为\(547\times607\times1~033\times31~051\) \\
		\hline
	\end{tblr}
	\caption{前7个欧几里得数}
%@Mathematica: EuclidNumber[n_] := Product[EuclidNumber[k], {k, 1, n - 1}] + 1
%@Mathematica: Table[{n, EuclidNumber[n], PrimeQ[EuclidNumber[n]]}, {n, 1, 9}] // TableForm
\end{table}
\begin{proposition}
%@see: 《具体数学 计算机科学基础(第2版)》 P90
欧几里得数全都是互素的.
\begin{proof}
因为当\(n > m\)时,有\(e_n \mod e_m = 1\),
所以\((e_m,e_n)
= (1,e_n)
= (0,1)
= 1\).
\end{proof}
\end{proposition}
% 于是,如果假设\(q_j\ (j\geq1)\)是\(e_j\)的最小素因子,
% 则素数\(q_1,q_2,q_3,\dotsc\)全都是不相同的,
% 这是一个含有无穷多个素数的序列.

由\cref{definition:初等数论.欧几里得数} 可知
\(e_n = e_1 e_2 \dotsm e_{n-2} e_{n-1} + 1\),
代入\(e_{n-1} - 1 = e_1 e_2 \dotsm e_{n-3} e_{n-2}\),
便得\(e_n = (e_{n-1} - 1) e_{n-1} + 1\),
整理可得欧几里得数的递归定义:\begin{equation}
	e_n = e_{n-1}^2 - e_{n-1} + 1.
\end{equation}

\begin{theorem}
没有任何一个有理代数式能唯一地表示素数.
\begin{proof}
用反证法.
假设有理代数式\[
	p(x) = a_0 + a_1 x + a_2 x^2 + \dotsb
\]唯一地表示素数,
且\[
	p(m) = a_0 + a_1 m + a_2 m^2 + \dotsb = q.
\]
但\begin{align*}
	p(m+nq) &= a_0 + a_1 (m+nq) + a_2 (m+nq)^2 + \dotsb \\
	&= a_0 + a_1 m + a_2 m^2 + \dotsb + \alpha q \quad(\alpha\in\mathbb{Q}) \\
	&= q + \alpha q = (1+\alpha)q,
\end{align*}
即\(\truediv{p(m+nq)}{q}\),说明\(p(m+nq)\)不是一个素数.
\end{proof}
\end{theorem}

\begin{theorem}[算数基本定理]
一个数只能以一种方式分解素因子.
\begin{proof}
设合数\(N\)可以分解为乘积\(abcd\dotsm\),其中\(a,b,c,d,\dotsc\)是素数;
又设\(N = \alpha\beta\gamma\delta\dotsm\),
其中\(\alpha,\beta,\gamma,\delta,\dotsc\)是另一些素数;
那么\[
	abcd\dotsm = \alpha\beta\gamma\delta\dotsm.
\]
显然,数\(\alpha\)可以整除\(abcd\dotsm\),
所以\(\alpha\)至少能整除它们中的一个因子,
不妨设\(\truediv{a}{\alpha}\),
但是由于\(a,b,c,d,\dotsc\)都是素数,
故可知\(a=\alpha\),
因此\[
	bcd\dotsm = \beta\gamma\delta\dotsm.
\]
以此类推,最后得到\(b=\beta,c=\gamma,d=\delta,\dotsc\),
也就是说,任意合数的素因子分解式是唯一的.
\end{proof}
\end{theorem}

\begin{example}
已知合数\(N = a^p b^q c^r \dotsm\),其中\(a,b,c,\dotsc\)是不同的素数,\(p,q,r,\dotsc\)是正整数.
计算\(N\)的因子个数,\(N\)分解成两个因子的方式数,\(N\)分解成两个互素的因子的方式数,\(N\)的所有因子之和.
\begin{solution}
因为乘积\[
(1+a+a^2+\dotsb+a^p)
(1+b+b^2+\dotsb+b^q)
(1+c+c^2+\dotsb+c^r)\dotsm
\]的展开式的每一项都是数\(N\)的一个因子,
所以,\(N\)的因子个数%
\footnote{这里“因子个数”包括了1和合数\(N\)本身.}%
就是上述乘积的项数\((p+1)(q+1)(r+1)\dotsm\).

当\(N\)不是完全平方数时,\(N\)分解成两个因子的方式显然有\[
\frac{1}{2} (p+1)(q+1)(r+1)\dotsm
\]种.

当\(N\)是完全平方数时,\(N = \sqrt{N}\times\sqrt{N}\)也是一种分解方式,但对应于这种分解方式的因子只有一个\(\sqrt{N}\).
如果不计入这种分解方式,那么\(N\)的分解方式有\[
\frac{1}{2} \left[-1 + (p+1)(q+1)(r+1)\dotsm\right]
\]种;再加上刚刚提到的一种特殊分解方式\(N = \sqrt{N}\times\sqrt{N}\),那么\(N\)的分解方式总共有\[
\frac{1}{2} \left[1 + (p+1)(q+1)(r+1)\dotsm\right]
\]种.

在将数\(N\)分解成互素的两个因子\(\alpha,\beta\)时,其中一个必须包含\(a^p\),否则就会有\(a\)的一些幂在一个因子中,\(a\)的另一些幂在在另一个因子中,于是这两个因子就不互素了.
以此类推,可知“\(N\)分解成两个互素的因子的方式数”与“\(abcd\dotsm\)分解成两个因子的方式数”相等,即为\[
\frac{1}{2}(1+1)(1+1)(1+1)\dotsm.
\]若设\(N\)中共有\(n\)个不同的素因子,那么\(N\)分解成两个互素的因子的方式数是\[
\frac{1}{2} \cdot 2^n = 2^{n-1}.
\]

如前所述,由于乘积\[
(1+a+a^2+\dotsb+a^p)
(1+b+b^2+\dotsb+b^q)
(1+c+c^2+\dotsb+c^r)\dotsm
\]展开式的每一项都是一个因子,所以“\(N\)的所有因子之和”就等于这个乘积,由等比数列的求和公式,便得\(N\)的所有因子之和为\[
\frac{1-a^{p+1}}{1-a}\cdot\frac{1-b^{q+1}}{1-b}\cdot\frac{1-c^{r+1}}{1-c}\dotsm.
\]
\end{solution}
\end{example}

\section{费马小定理}
\begin{lemma}\label{theorem:初等数论.费马小定理.引理0}
证明:任意\(r\)个连续正整数的乘积能被\(r!\)整除.
\begin{proof}
设从正整数\(n\)开始的连续\(r\)个正整数的乘积为\(P_n = n(n+1)(n+2)\dotsm(n+r-1)\),那么\[
P_{n+1} = (n+1)(n+2)\dotsm(n+r),
\]\[
n P_{n+1} = (n+r) P_n = n P_n + r P_n,
\]\[
P_{n+1} = P_n + \frac{r}{n} P_n,
\]\[
P_{n+1} - P_n = r \frac{P_n}{n},
\]上式等号右边是\(r-1\)个连续正整数的\(r\)倍.
因此,如果任意\(r-1\)个连续正整数的乘积能被\((r-1)!\)整除,就有\(P_{n+1} - P_n\)能被\(r!\)整除.
因为\(P_1 = 1 \cdot 2 \dotsm r = r!\),所以\(P_2\)必是\(r!\)的倍数,从而\(P_3,P_4,\dotsc\)也都是\(r!\)的倍数.
这样就证明了“如果\(r-1\)个连续正整数的乘积能被\((r-1)!\)整除,那么\(r\)个连续正整数的乘积便能被\(r!\)整除”.
但是由于每两个连续正整数(必有一个奇数和一个偶数)的乘积能被\(2!=2\)整除,所以每三个连续正整数的乘积也能被\(3!\)整除;以此类推,该命题普遍成立.
\end{proof}
\end{lemma}

\begin{lemma}\label{theorem:初等数论.费马小定理.引理1}
设\(p\)是素数.
证明:除了第一项和最后一项以外,\((a+b)^p\)展开式的每一项的系数,都可以被\(p\)整除.
\begin{proof}
除了第一项与最后一项以外的各项系数为\[
C_p^r = \frac{p(p-1)(p-2)\dotsm(p-r+1)}{r!},
\]其中\(r=1,2,\dotsc,p-1\).
因为\(p\)是素数,所以\(r!\)中(除了1以外)没有一个因子可以整除\(p\);
又因为\(p>r\),所以\(p\)也不能整除\(r!\)中的任一因子;
也就是说,\((p-1)(p-2)\dotsm(p-r+1)\)必能被\(r!\)整除,而系数\(C_p^r\)必能被\(p\)整除.
\end{proof}
\end{lemma}

\begin{lemma}\label{theorem:初等数论.费马小定理.引理2}
设\(p\)是素数.
证明:\[
(a+b+c+d+\dotsb)^p = M + a^p + b^p + c^p + d^p + \dotsb,
\]其中\(M\)是\(p\)的倍数.
\begin{proof}
记\(\beta=b+c+\dotsb\),则由\cref{theorem:初等数论.费马小定理.引理1} 可知\[
(a+\beta)^p = a^p + \beta^p + M_1,
\]其中\(M_1\)是\(p\)的倍数.
接下来,记\(\gamma=c+d+\dotsb\),则同样有\[
\beta^p = (b+\gamma)^p = b^p + \gamma^p + M_2,
\]其中\(M_2\)是\(p\)的倍数.
以此类推,便得要证的结果,且\(M = M_1+M_2+\dotsb\).
\end{proof}
\end{lemma}

\begin{theorem}[费马小定理]\label{theorem:初等数论.费马小定理}
如果\(p\)是素数,且正整数\(n\)与\(p\)互素,则\(n^{p-1}-1\)是\(p\)的倍数.
\begin{proof}
根据\cref{theorem:初等数论.费马小定理.引理2},在\(n\)个正整数的和的\(p\)次幂\[
(a+b+c+d+\dotsb)^p = M + a^p + b^p + c^p + d^p + \dotsb
\]中令\(a=b=c=d=\dotsb=1\),那么有\[
n^p = n + M
\quad\text{或}\quad
n^p - n = n(n^{p-1}-1) = M,
\]即\(\truediv{n^{p-1}-1}{p}\).
\end{proof}
\end{theorem}

\begin{corollary}
如果\(p\)是素数,且\(p\neq2\),那么\(p-1\)是偶数,且对于任意正整数\(N\)总有\[
\left(N^{\frac{p-1}{2}}+1\right)
\left(N^{\frac{p-1}{2}}-1\right)
\]是\(p\)的倍数,或者说\[
N^{\frac{p-1}{2}} = Kp\pm1,
\]其中\(K\)是某个正整数.
\end{corollary}

\begin{corollary}[费马小定理']
如果\(p\)是素数,则\(\truediv{n^p-n}{p}\).
\end{corollary}

\begin{example}
设\(p\)是素数,证明:任意两个正整数的\(p\)次幂的差比这两个数的差大\(p\)的倍数.
\end{example}

\begin{example}
证明:任意完全平方数要么等于\(5n\),要么等于\(5n\pm1\),其中\(n\in\mathbb{N}^+\).
\end{example}

\begin{example}[费马猜想]
费马提出过一个错误的猜想:数\(2^{2^n}+1\ (n=0,1,2,\dotsc)\)都是素数.
这个猜想在当\(n=0,1,2,3,4\)时是正确的,但很遗憾的是,数学家欧拉发现,当\(n=5\)时,数\(2^{2^5}+1 = 4294967297 = 641 \times 6700417\)是合数.
%@Mathematica: FactorInteger[2^(2^5)+1]
\end{example}

\section{无理数}
\begin{proposition}
设\(n\in\mathbb{N}^+\),且\(n\)不是完全平方数,则\(\sqrt{n}\)是无理数.
\begin{proof}
用反证法.
假设\(\sqrt{n} = p/q\),其中\(p,q\in\mathbb{N}^+\).
由于\(n\)不是完全平方数,故有\(m\in\mathbb{N}^+\),使得\(m<p/q<m+1\),
从而\(mq<p<mq+q\),\(0<p-mq<q\).
在等式\(p^2=nq^2\)的两边都减去\(mpq\),
得到\(p^2-mpq=nq^2-mpq\),
提取公因式得\(p(p-mq)=q(nq-mp)\),
整理得\[
	\frac{p}{q} = \frac{nq-mp}{p-mq}.
\]
令\(p_1=nq-mp\),\(q_1=p-mq\).
由于\(q_1\in\mathbb{N}^+\)且\(q_1<q\),
所以\(p_1\in\mathbb{N}^+\)且\(p_1<p\).
对等式\[
	\frac{p}{q} = \frac{p_1}{q_1}
\]反复地进行同样的讨论,
可以得出两串递减的正整数列\[
	p>p_1>p_2>p_3>\dotsb
	\quad\text{与}\quad
	q>q_1>q_2>q_3>\dotsb,
\]
使得\[
	\frac{p}{q}=\frac{p_1}{q_1}=\frac{p_2}{q_2}=\frac{p_3}{q_3}=\dotsb.
\]
这是不可能的,因为从\(p\)或\(q\)开始的正整数不可能无止境地递减下去.
这就证明了\(\sqrt{n}\)不可能是有理数.
\end{proof}
\end{proposition}
\endgroup
