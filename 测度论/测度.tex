\chapter{集族与测度}
%@see: 《Real Analysis Modern Techniques and Their Applications Second Edition》(Gerald B. Folland) P19
在本章我们来学习测度论的基本概念.

确定平面区域的面积和空间区域的体积是几何学的经典问题.
对于区域边界是光滑曲线或曲面的情况,
积分学提供了一套令人满意的解决方案.
但是在处理具有复杂边界的区域时,积分学便束手无策了.
于是,我们希望,对于任意\(n\in\mathbb{N}\),可以找到这么一个映射\(\mu\),
它可以把\(\mathbb{R}^n\)的任一子集\(E\)映射为一个数\(\mu(E) \in [0,+\infty]\).
我们把这个数称为“\(E\)的\(n\)维\DefineConcept{测度}(measure)”.
当\(E\)的长度、面积或体积可以利用积分公式计算时,
我们希望积分值等于测度.
另外,映射\(\mu\)应该具有以下性质:\begin{enumerate}
	\item 如果\(E_1,E_2,\dotsc\)是互斥集序列,
	那么\begin{equation*}
		\mu(E_1 \cup E_2 \cup \dotsb)
		= \mu(E_1) + \mu(E_2) + \dotsb.
	\end{equation*}

	\item 如果\(E\)与\(F\)全等,即\(E\)可以通过平移、旋转、镜像反转这三种变换得到\(F\),
	那么\(\mu(E) = \mu(F)\).

	\item \(\mu(Q)=1\),
	其中\(Q = \Set{ x\in\mathbb{R}^n \given 0\leq x_i<1\ (i=1,2,\dotsc,n) }\).
\end{enumerate}

\section{集族}
在测度论中,我们常把一个非空集合记为\(\Omega\),
研究它的子集、子集族的性质.
我们知道,只要任取\(\mathcal{A} \in \Powerset\Powerset\Omega\),
那么\(\mathcal{A}\)就成为\(\Omega\)的一个子集族.
我们把从自然数集\(\mathbb{N}\)到\(\mathcal{A}\)的映射
称为一个\DefineConcept{集合序列},
简称\DefineConcept{序列}.

\subsection{单调序列}
\begin{example}
%@see: 《实变函数论(第三版)》(周民强) P9 例5
设\(A_n = [n,+\infty)\ (n=1,2,\dotsc)\).
显然\(\{A_n\}\)是一个单调减序列,
容易得出\(\lim_{n\to\infty} A_n = \emptyset\).
\end{example}

\begin{example}
%@see: 《实变函数论(第三版)》(周民强) P9 例6
设函数族\(\{f_n\}_{n\geq1}\)满足
\begin{enumerate}
	\item 对\(\forall x\in\mathbb{R}\),
	总有\(f_n(x) \leq f_{n+1}(x)\ (n=1,2,\dotsc)\),
	\item 对\(\forall x\in\mathbb{R}\),
	总有\(\lim_{n\to\infty} f_n(x) = f(x)\).
\end{enumerate}
给定\(t\in\mathbb{R}\),
取序列\(\{E_n\}_{n\geq1}\),使之满足\begin{equation*}
	E_n = \Set{ x\in\mathbb{R} \given f_n(x) > t }
	\quad(n=1,2,\dotsc).
\end{equation*}
那么就有\begin{equation*}
	E_1 \subseteq E_2 \subseteq \dotsb \subseteq E_n \subseteq \dotsb,
\end{equation*}
并且还有\begin{equation*}
	\lim_{n\to\infty} E_n
	= \bigcup_{n=1}^\infty \Set{ x\in\mathbb{R} \given f_n(x) > t }
	= \Set{ x\in\mathbb{R} \given f(x) > t }.
\end{equation*}
\end{example}

\begin{proposition}
%@see: 《实变函数论(第三版)》(周民强) P11 例8
设\(\{f_n(x)\}\)以及\(f(x)\)是定义在\(\mathbb{R}\)上的实值函数,
则使\(f_n(x)\)不收敛于\(f(x)\)的一切点\(x\)所形成的集合可表示为\begin{equation*}
	\bigcup_{k=1}^\infty
	\bigcap_{N=1}^\infty
	\bigcup_{n=N}^\infty
	\Set{
		x\in\mathbb{R}
		\given
		\abs{f_n(x) - f(x)} \geq \frac1k
	}.
\end{equation*}
%TODO
\end{proposition}

\subsection{特征函数}
\begin{definition}
%@see: 《实变函数论(第三版)》(周民强) P13
设\(X\)是集合,
\(A \subseteq X\).
定义:\begin{equation*}
	\chi_A(x)
	\defeq
	\left\{ \begin{array}{cl}
		1, & x \in A, \\
		0, & x \in X - A.
	\end{array} \right.
\end{equation*}
把\(\chi_A\)称为“定义在\(X\)上的\(A\)的\DefineConcept{特征函数}(characteristic function)”
或“定义在\(X\)上的\(A\)的\DefineConcept{示性函数}(indicator function)”.
%@see: https://mathworld.wolfram.com/CharacteristicFunction.html
%@see: https://mathworld.wolfram.com/IndicatorFunction.html
%@see: https://www.bananaspace.org/wiki/%E6%8C%87%E7%A4%BA%E5%87%BD%E6%95%B0
\end{definition}
由此可见,特征函数\(\chi_A\)在一定意义上可作为\(A\)本身的代表,
从而可以通过对它的研究来了解集合本身.

\begin{proposition}
%@see: 《实变函数论(第三版)》(周民强) P13
设\(X,A,B\)都是集合,\(A \subseteq X\)且\(B \subseteq X\),
则\begin{equation*}
	A \neq B \iff \chi_A \neq \chi_B.
\end{equation*}
\end{proposition}

\begin{proposition}
%@see: 《实变函数论(第三版)》(周民强) P13
设\(X,A,B\)都是集合,\(A \subseteq X\)且\(B \subseteq X\),
则\begin{equation*}
	A \subseteq B
	\iff
	\chi_A(x) \leq \chi_B(x).
\end{equation*}
\end{proposition}

\begin{proposition}
%@see: 《实变函数论(第三版)》(周民强) P13
设\(X,A,B\)都是集合,\(A \subseteq X\)且\(B \subseteq X\),
则\begin{equation*}
	\chi_{A \cup B}(x)
	= \chi_A(x) + \chi_B(x) - \chi_{A \cap B}(x).
\end{equation*}
\end{proposition}

\begin{proposition}
%@see: 《实变函数论(第三版)》(周民强) P13
设\(X,A,B\)都是集合,\(A \subseteq X\)且\(B \subseteq X\),
则\begin{equation*}
	\chi_{A \cap B}(x)
	= \chi_A(x) \cdot \chi_B(x).
\end{equation*}
\end{proposition}

\begin{proposition}
%@see: 《实变函数论(第三版)》(周民强) P13
设\(X,A,B\)都是集合,\(A \subseteq X\)且\(B \subseteq X\),
则\begin{equation*}
	\chi_{A - B}(x)
	= \chi_A(x) \cdot (1 - \chi_B(x)).
\end{equation*}
\end{proposition}

\begin{proposition}
%@see: 《实变函数论(第三版)》(周民强) P13
设\(X,A,B\)都是集合,\(A \subseteq X\)且\(B \subseteq X\),
则\begin{equation*}
	\chi_{A \symdiff B}(x)
	= \abs{\chi_A(x) - \chi_B(x)}.
\end{equation*}
\end{proposition}

\begin{example}
%@see: 《实变函数论(第三版)》(周民强) P14 例1(单调映射的不动点)
设\(X\)是非空集合,
且\(f\colon \Powerset X \to \Powerset X\).
若\begin{equation*}
	(\forall A,B \in \Powerset X)
	[
		A \subseteq B
		\implies
		f\ImageOfSetUnderRelation{A} \subseteq f\ImageOfSetUnderRelation{B}
	],
\end{equation*}
则\begin{equation*}
	(\exists T)
	[
		T \subseteq \Powerset X
		\implies
		f\ImageOfSetUnderRelation{T} = T
	].
\end{equation*}
\begin{proof}
令\begin{equation*}
	S = \Set{ A \in \Powerset X \given A \subseteq f\ImageOfSetUnderRelation{A} },
	\qquad
	T = \bigcup_{A \in S} A,
\end{equation*}
则有\(f\ImageOfSetUnderRelation{T} = T\).

因为\(A \in S\),\(A \subseteq f\ImageOfSetUnderRelation{A}\),
从而由\(A \subseteq T\)
可得\(f\ImageOfSetUnderRelation{A} \subseteq f\ImageOfSetUnderRelation{T}\).
根据\(A \in S\)推出\(A \subseteq f\ImageOfSetUnderRelation{T}\),
这就导致\begin{equation*}
	\bigcup_{A \in S} A \subseteq f\ImageOfSetUnderRelation{T},
	\qquad
	T \subseteq f\ImageOfSetUnderRelation{T}.
\end{equation*}

另一方面,又从\(T \subseteq f\ImageOfSetUnderRelation{T}\)
可知\(f\ImageOfSetUnderRelation{T} \subseteq f\ImageOfSetUnderRelation{f\ImageOfSetUnderRelation{T}}\).
这说明\(f\ImageOfSetUnderRelation{T} \in S\),
我们又有\(f\ImageOfSetUnderRelation{T} \subseteq T\).
\end{proof}
\end{example}
\begin{example}
%@see: 《实变函数论(第三版)》(周民强) P14 思考题 5.
设\(f\colon X \to Y,
g\colon Y \to X\).
证明:若对任意的\(x \in X\),
必有\(g(f(x)) = x\),
则\(f\)是单射,
\(g\)是满射.
%TODO proof
\end{example}

\subsection{集族的封闭性}
%@see: 《测度论讲义(第三版)》(严加安) P2 1.1.6
\begin{definition}[集族的封闭性]
设\(\mathcal{C}\)是一个非空集族.

如果\begin{equation*}
	(\forall A,B\in\mathcal{C})
	[A \cap B \in \mathcal{C}],
\end{equation*}
则称“\(\mathcal{C}\)对有限交封闭”.

如果\begin{equation*}
	(\forall \{A_n\})
	(\forall n\geq1)
	\left[A_n\in\mathcal{C} \implies \bigcap_{k=1}^n A_k \in \mathcal{C}\right],
\end{equation*}
则称“\(\mathcal{C}\)对可列交封闭”.

如果\begin{equation*}
	(\forall A,B\in\mathcal{C})
	[A \cup B \in \mathcal{C}],
\end{equation*}
则称“\(\mathcal{C}\)对有限并封闭”.

如果\begin{equation*}
	(\forall \{A_n\})
	(\forall n\geq1)
	\left[A_n\in\mathcal{C} \implies \bigcup_{k=1}^n A_k \in \mathcal{C}\right],
\end{equation*}
则称“\(\mathcal{C}\)对可列并封闭”.
\end{definition}

\begin{definition}
%@see: 《测度论讲义(第三版)》(严加安) P2
设\(\mathcal{C}\)是一个非空集族.
定义:\begin{equation*}
	\mathcal{C}_{\cap f}
	\defeq
	\Set*{
		A \given
		A = \bigcap_{k=1}^n A_k;
		A_k \in \mathcal{C}, i=1,\dotsc,n;
		n\geq1
	},
\end{equation*}
称其为“用 有限交 运算封闭\(\mathcal{C}\)所得的集族”.
\end{definition}
\begin{definition}
%@see: 《测度论讲义(第三版)》(严加安) P2
设\(\mathcal{C}\)是一个非空集族.
定义:\begin{equation*}
	\mathcal{C}_{\cup f}
	\defeq
	\Set*{
		A \given
		A = \bigcup_{k=1}^n A_k;
		A_k \in \mathcal{C}, i=1,\dotsc,n;
		n\geq1
	},
\end{equation*}
称其为“用 有限并 运算封闭\(\mathcal{C}\)所得的集族”.
\end{definition}
\begin{definition}
%@see: 《测度论讲义(第三版)》(严加安) P2
设\(\mathcal{C}\)是一个非空集族.
定义:\begin{equation*}
	\mathcal{C}_{\sum f}
	\defeq
	\Set*{
		A \given
		A = \bigcup_{k=1}^n A_k;
		A_k \in \mathcal{C}, i=1,\dotsc,n;
		A_i \cap A_j = \emptyset, i,j=1,\dotsc,n, i \neq j;
		n\geq1
	},
\end{equation*}
称其为“用 有限不交并 运算封闭\(\mathcal{C}\)所得的集族”.
\end{definition}
\begin{definition}
%@see: 《测度论讲义(第三版)》(严加安) P2
设\(\mathcal{C}\)是一个非空集族.
定义:\begin{equation*}
	\mathcal{C}_\delta
	\defeq
	\Set*{
		A \given
		A = \bigcap_{k=1}^\infty A_k;
		A_k \in \mathcal{C}, i=1,\dotsc,n
	},
\end{equation*}
称其为“用 可列交 运算封闭\(\mathcal{C}\)所得的集族”.
\end{definition}
\begin{definition}
%@see: 《测度论讲义(第三版)》(严加安) P2
设\(\mathcal{C}\)是一个非空集族.
定义:\begin{equation*}
	\mathcal{C}_\sigma
	\defeq
	\Set*{
		A \given
		A = \bigcup_{k=1}^\infty A_k;
		A_k \in \mathcal{C}, i=1,\dotsc,n
	},
\end{equation*}
称其为“用 可列并 运算封闭\(\mathcal{C}\)所得的集族”.
\end{definition}
\begin{definition}
%@see: 《测度论讲义(第三版)》(严加安) P2
设\(\mathcal{C}\)是一个非空集族.
定义:\begin{equation*}
	\mathcal{C}_\sigma
	\defeq
	\Set*{
		A \given
		A = \bigcup_{k=1}^\infty A_k;
		A_k \in \mathcal{C}, i=1,\dotsc,n;
		A_i \cap A_j = \emptyset, i,j=1,\dotsc,n, i \neq j
	},
\end{equation*}
称其为“用 可列不交并 运算封闭\(\mathcal{C}\)所得的集族”.
\end{definition}
%@see: 《测度论讲义(第三版)》(严加安) P3
此外,我们定义:\begin{gather*}
	\mathcal{C}_{\cap f,\cup f}
	\defeq
	(\mathcal{C}_{\cap f})_{\cup f}, \\
	\mathcal{C}_{\cup f,\cap f}
	\defeq
	(\mathcal{C}_{\cup f})_{\cap f}, \\
	\mathcal{C}_{\sigma,\delta}
	\defeq
	(\mathcal{C}_\sigma)_\delta.
\end{gather*}

\begin{proposition}
设\(\mathcal{C}\)是一个非空集族,
则\(\mathcal{C}_{\cap f}\)对有限交封闭.
\end{proposition}

\begin{proposition}
%@see: 《测度论讲义(第三版)》(严加安) P3 命题1.1.7
设\(\mathcal{C}\)是一个集族,则\begin{itemize}
	\item \(\mathcal{C}_{\cap f,\cup f} = \mathcal{C}_{\cup f,\cap f}\);
	\item 若\(\mathcal{C}\)对有限交封闭,
	则\(\mathcal{C}_{\cup f},
	\mathcal{C}_{\sum f},
	\mathcal{C}_\sigma,
	\mathcal{C}_{\sum \sigma}\)亦然;
	\item 若\(\mathcal{C}\)对有限并封闭,
	则\(\mathcal{C}_{\cap f},
	\mathcal{C}_\delta\)亦然.
\end{itemize}
\begin{proof}
直接从集合的交和并的分配律可得.
\end{proof}
\end{proposition}

\begin{definition}
%@see: 《测度论讲义(第三版)》(严加安) P3
设\(\mathcal{C}\)是一个集族.
\begin{itemize}
	\item 如果\(\mathcal{C}\)中
	任意一个收敛的单调增序列\(\{A_n\}\)的极限\(A\)
	满足\(A\in\mathcal{C}\),
	则称“\(\mathcal{C}\)对单调增序列极限封闭”.
	\item 如果\(\mathcal{C}\)中
	任意一个收敛的单调减序列\(\{A_n\}\)的极限\(A\)
	满足\(A\in\mathcal{C}\),
	则称“\(\mathcal{C}\)对单调减序列极限封闭”.
	\item 如果\(\mathcal{C}\)既对单调增序列极限封闭,又对单调减序列极限封闭,
	则称“\(\mathcal{C}\)对单调序列极限封闭”.
\end{itemize}
\end{definition}

\begin{definition}
%@see: 《测度论讲义(第三版)》(严加安) P3 定义1.1.8
设\(\mathcal{C}\)是一个集族.
\begin{itemize}
	\item 如果\(\mathcal{C}\)对有限交封闭,
	则称“\(\mathcal{C}\)是~\DefineConcept{\(\pi\)类}”.
	\item 如果\(\emptyset\in\mathcal{C}\),
	且\(A,B\in\mathcal{C} \implies A \cap B \in \mathcal{C}, A-B \in \mathcal{C}_{\sum f}\),
	则称“\(\mathcal{C}\)是\DefineConcept{半环}”.
	\item 如果\(\mathcal{C}\)是半环,
	且\(\Omega\in\mathcal{C}\),
	则称“\(\mathcal{C}\)是\DefineConcept{半代数}”.
	\item 如果\(\mathcal{C}\)对有限交及取余集运算封闭,
	且\(\Omega\in\mathcal{C},
	\emptyset\in\mathcal{C}\),
	则称“\(\mathcal{C}\)是\DefineConcept{代数}或\DefineConcept{域}”.
	\item 如果\(\mathcal{C}\)对可列交及取余集运算封闭,
	且\(\Omega\in\mathcal{C},
	\emptyset\in\mathcal{C}\),
	则称“\(\mathcal{C}\)是~\DefineConcept{\(\sigma\)代数}”.
	\item 如果\(\mathcal{C}\)对单调序列极限封闭,
	则称“\(\mathcal{C}\)是\DefineConcept{单调类}”.
	\item 如果\(\Omega\in\mathcal{C}\),
	\((\forall A,B\in\mathcal{C})[B \subseteq A \implies A-B\in\mathcal{C}]\),
	且\(\mathcal{C}\)对单调增序列极限封闭,
	则称“\(\mathcal{C}\)是\DefineConcept{\(\lambda\)类}”.
\end{itemize}
\end{definition}

\begin{proposition}
%@see: 《测度论讲义(第三版)》(严加安) P3
\(\sigma\)代数是\(\lambda\)类.
\end{proposition}
\begin{proposition}
%@see: 《测度论讲义(第三版)》(严加安) P3
\(\lambda\)类是单调类.
\end{proposition}

\begin{example}
%@see: 《测度论讲义(第三版)》(严加安) P3 例1.1.9
设\begin{gather*}
	\mathcal{C}_1 = \Set{ (-\infty,a] \given a\in\mathbb{R} }, \\
	\mathcal{C}_2 = \Set{ (a,+\infty) \given a\in\mathbb{R} }, \\
	\mathcal{C}_3 = \Set{ (a,b] \given a \leq b; a,b\in\mathbb{R} },
\end{gather*}
则\(\mathcal{C}_1,\mathcal{C}_2,\mathcal{C}_3\)是\(\pi\)类,
\(\mathcal{C}_4 = \mathcal{C}_1 \cup \mathcal{C}_2 \cup \mathcal{C}_3\)是半环,
\(\mathcal{C}_4 \cup \{\mathbb{R}\}\)是半代数.
\end{example}

\section{单调类定理}
\begin{proposition}
%@see: 《测度论讲义(第三版)》(严加安) P4
对于\(\Omega\)上的任意非空集族\(\mathcal{C}\),
存在包含\(\mathcal{C}\)的最小\(\sigma\)代数.
\end{proposition}
\begin{definition}
%@see: 《测度论讲义(第三版)》(严加安) P4
把\(\Omega\)上的包含非空集族\(\mathcal{C}\)的最小\(\sigma\)代数
称为“由\(\mathcal{C}\)生成的\(\sigma\)代数”,
记作\(\sigma(\mathcal{C})\).
\end{definition}

\begin{proposition}
%@see: 《测度论讲义(第三版)》(严加安) P4
对于\(\Omega\)上的任意非空集族\(\mathcal{C}\),
存在包含\(\mathcal{C}\)的最小\(\lambda\)类.
\end{proposition}
\begin{definition}
%@see: 《测度论讲义(第三版)》(严加安) P4
把\(\Omega\)上的包含非空集族\(\mathcal{C}\)的最小\(\lambda\)类
称为“由\(\mathcal{C}\)生成的\(\lambda\)类”,
记作\(\lambda(\mathcal{C})\).
\end{definition}

\begin{proposition}
%@see: 《测度论讲义(第三版)》(严加安) P4
对于\(\Omega\)上的任意非空集族\(\mathcal{C}\),
存在包含\(\mathcal{C}\)的最小单调类.
\end{proposition}
\begin{definition}
%@see: 《测度论讲义(第三版)》(严加安) P4
把\(\Omega\)上的包含非空集族\(\mathcal{C}\)的最小单调类
称为“由\(\mathcal{C}\)生成的单调类”,
记作\(m(\mathcal{C})\).
\end{definition}

\begin{proposition}
%@see: 《测度论讲义(第三版)》(严加安) P4
对于\(\Omega\)上的任意非空集族\(\mathcal{C}\),
由\(\mathcal{C}\)生成的\(\sigma\)代数、\(\lambda\)类和单调类满足:\begin{equation*}
	m(\mathcal{C})
	\subseteq \lambda(\mathcal{C})
	\subseteq \sigma(\mathcal{C}).
\end{equation*}
\end{proposition}

%@see: 《测度论讲义(第三版)》(严加安) P4
接下来我们研究在什么条件下成立
\(m(\mathcal{C}) = \sigma(\mathcal{C})\)
或\(\lambda(\mathcal{C}) = \sigma(\mathcal{C})\).

\chapter{勒贝格测度}
\begin{definition}
%@see: 《实变函数论(第三版)》(周民强) P66 定义2.1
设\(E \subseteq \mathbb{R}^n\).
若\(\{I_k\}\)是\(\mathbb{R}^n\)中的可列个开矩体,
且有\(E \subseteq \bigcup_{k\geq1} I_k\),
则称“\(\{I_k\}\)是\(E\)的一个\DefineConcept{L-覆盖}”.
\end{definition}

%TODO 没有“体积”的定义,
% 记\(\abs{I_k}\)为开矩体\(I_k\)的体积,
% 把\begin{equation*}
% 	m^*(E)
% 	\defeq
% 	\inf\Set{
% 		\sum_{k\geq1} \abs{I_k}
% 		\given
% 		\text{\(\{I_k\}\)是\(E\)的L-覆盖}
% 	}
% \end{equation*}称为“\(E\)的\DefineConcept{勒贝格外测度}”,
% 简称\DefineConcept{外测度}.
