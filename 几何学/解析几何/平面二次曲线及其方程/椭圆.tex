\section{椭圆}
\subsection{椭圆的概念}
%@see: 《平面解析几何(甲种本)》 P81
平面内与两个定点的距离之和等于定长且大于两点间距的点的集合
称为\DefineConcept{椭圆}(ellipse).
这两个定点叫做椭圆的\DefineConcept{焦点}(focus).
两焦点的距离叫做\DefineConcept{焦距}.

\subsection{椭圆的标准方程}
%@see: 《平面解析几何(甲种本)》 P82
根据椭圆的定义,我们来求椭圆的方程.
取过焦点\(F_1\)、\(F_2\)的直线为\(x\)轴,
线段\(F_1 F_2\)的垂直平分线为\(y\)轴,
建立直角坐标系.

设\(P(x,y)\)是椭圆上任意一点,椭圆的焦距为\(2c\ (c > 0)\),
\(P\)与\(F_1\)和\(F_2\)的距离之和等于正常数\(2a\),
则\(F_1\)、\(F_2\)的坐标分别是\((-c,0)\)和\((c,0)\).
那么根据定义,椭圆就是点集\begin{equation*}
	\Set{ P \given \LineSegmentLength{P F_1} + \LineSegmentLength{P F_2} = 2a }.
\end{equation*}
由两点间的距离公式,\begin{equation*}
	\LineSegmentLength{P F_1} = \sqrt{(x+c)^2+y^2}, \qquad
	\LineSegmentLength{P F_2} = \sqrt{(x-c)^2+y^2},
\end{equation*}
得\begin{equation*}
	\sqrt{(x+c)^2+y^2} + \sqrt{(x-c)^2+y^2} = 2a,
\end{equation*}
把这个方程移项,两边平方,得\begin{gather*}
	(x+c)^2+y^2 = 4a^2 - 4a\sqrt{(x-c)^2+y^2} + (x-c)^2+y^2, \\
	a^2 - cx = a\sqrt{(x-c)^2+y^2},
\end{gather*}
两边再平方,得\begin{equation*}
	a^4 - 2 a^2 cx + c^2 x^2 = a^2 [(x-c)^2+y^2],
\end{equation*}
整理得\begin{equation*}
	(a^2 - c^2) x^2 + a^2 y^2 = a^2 (a^2 - c^2).
\end{equation*}
由椭圆定义可知,\(2a > 2c\),\(a > c\),\(a^2 - c^2 > 0\).
记\(b \defeq \sqrt{a^2 - c^2}\),
显然\(b > 0\),
并且\begin{equation*}
	b^2 x^2 + a^2 y^2 = a^2 b^2,
\end{equation*}
两边除以\(a^2 b^2\),
得\begin{equation}\label{equation:平面解析几何.椭圆的标准方程1}
%@see: 《平面解析几何(甲种本)》 P83 (1)
	\frac{x^2}{a^2} + \frac{y^2}{b^2} = 1,
	\quad a > b > 0.
\end{equation}

方程 \labelcref{equation:平面解析几何.椭圆的标准方程1} 叫做\DefineConcept{椭圆的标准方程}.
它所表示的椭圆的焦点在\(x\)轴上,
焦点是\(F_1(-c,0)\)和\(F_2(c,0)\),
其中\(c^2 = a^2 - b^2\).

如果椭圆的焦点在\(y\)轴上,焦点是\(F_1(0,-c)\)和\(F_2(0,c)\),
只要将方程 \labelcref{equation:平面解析几何.椭圆的标准方程1} 中的\(x\)、\(y\)互换,就可以得到它的方程.
这时它的方程为\begin{equation}\label{equation:平面解析几何.椭圆的标准方程2}
	\frac{y^2}{a^2} + \frac{x^2}{b^2} = 1,
	\quad a > b > 0.
\end{equation}

方程 \labelcref{equation:平面解析几何.椭圆的标准方程2} 也是椭圆的标准方程.

\begin{figure}[htb]
	\centering
	\begin{tikzpicture}
		%\draw[gray,help lines,dashed] (-5,-3) grid (5,3);
		\draw[thick,->] (-6,0) -- (7,0)node[above]{\(x\)};
		\draw[thick,->] (0,-4) -- (0,4)node[right]{\(y\)};
		\pgfmathsetmacro{\a}{5}
		\pgfmathsetmacro{\b}{3}
		\pgfmathsetmacro{\c}{sqrt(\a*\a-\b*\b)}
		\pgfmathsetmacro{\px}{3}
		\pgfmathsetmacro{\py}{12/5}
		\pgfmathsetmacro{\dx}{\a*\a/\c}  % directrix
		\coordinate (P) at (\px,\py);
		\coordinate (Q) at (\dx,\py);
		\coordinate (R) at (\dx,0);
		\coordinate (M) at (\c,9/5);
		\coordinate (N) at (\c,-9/5);
		\coordinate (F1) at (-\c,0);
		\coordinate (F2) at (\c,0);
		\draw[orange] (0,0)ellipse[x radius=\a,y radius=\b];
		\draw[red] (F1)node[below right,black]{\(F_1\)}
			-- (P)node[above right,black]{\(P\)}
			-- (F2)node[below left,black]{\(F_2\)}
			(P) -- (Q)node[right,black]{\(Q\)}
			pic[draw=gray,-,angle radius=0.3cm]{right angle=P--Q--R}
			pic[draw=gray,-,angle radius=0.3cm]{right angle=N--F2--R};
		\draw[blue] (N)node[right,black]{\(N\)}
			-- (M)node[right,black]{\(M\)};
		\draw[purple] (\dx,-4) -- (\dx,4);
		\draw (-3,2)node[above left]{\(C: \frac{x^2}{a^2}+\frac{y^2}{b^2}=1\)}
			(\dx,-3)node[left]{\(l: x = \frac{a^2}{c}\)};
		\draw (0,0)node[below left]{\(O\)}
			(-5,0)node[above left]{\(A_1\)}
			(5,0)node[above right]{\(A_2\)}
			(0,-3)node[below right]{\(B_1\)}
			(0,3)node[above right]{\(B_2\)};
	\end{tikzpicture}
	\caption{椭圆的图形}
	\label{figure:平面解析几何.椭圆的图形}
\end{figure}

\subsection{椭圆的几何性质}
%@see: 《平面解析几何(甲种本)》 P85
我们根据椭圆的标准方程\begin{equation*}
	\frac{x^2}{a^2} + \frac{y^2}{b^2} = 1,
	\quad a > b > 0,
\end{equation*}
来研究椭圆的几何性质.
\begin{enumerate}
	\item 范围.

	由标准方程可知,椭圆上点的坐标\((x,y)\)都适合不等式\begin{equation*}
		\frac{x^2}{a^2} \leq 1,
		\qquad
		\frac{y^2}{b^2} \leq 1,
	\end{equation*}
	即\begin{equation*}
		x^2 \leq a^2, \qquad y^2 \leq b^2,
	\end{equation*}
	所以\begin{equation*}
		\abs{x} \leq a, \qquad \abs{y} \leq b.
	\end{equation*}
	这说明椭圆位于直线\(x=\pm a\)和\(y=\pm b\)所围成的矩形里.

	\item 对称性.

	在标准方程中,把\(x\)换成\(-x\),或把\(y\)换成\(-y\),
	或把\(x\)、\(y\)同时换成\(-x\)、\(-y\)时,
	方程都不变,所以图形关于\(y\)轴、\(x\)轴和原点都是对称的.
	也就是说,坐标轴是椭圆的对称轴,原点是椭圆的对称中心.
	椭圆的对称中心叫做椭圆的\DefineConcept{中心}.

	\item 顶点.

	在标准方程中,令\(x=0\),得\(y=\pm b\),
	这说明\(B_1(0,-b)\)和\(B_2(0,b)\)是椭圆和\(y\)轴的两个交点.
	同理,令\(y=0\),得\(x=\pm a\),
	说明\(A_1(-a,0)\)和\(A_2(a,0)\)是椭圆和\(x\)轴的两个交点.
	因为\(x\)轴和\(y\)轴是椭圆的对称轴,所以椭圆和它的对称轴共有四个交点.
	这四个交点叫做椭圆的\DefineConcept{顶点}.

	线段\(A_1 A_2\)和\(B_1 B_2\)分别叫做
	椭圆的\DefineConcept{长轴}(major axis)
	和\DefineConcept{短轴}(minor axis).
	它们的长为\begin{equation*}
		\LineSegmentLength{A_1 A_2} = 2a,
		\qquad
		\LineSegmentLength{B_1 B_2} = 2b.
	\end{equation*}
	而\(a\)和\(b\)分别叫做椭圆的长半轴长和短半轴长.

	\item 离心率.

	椭圆的焦距与长轴长的比\(e = \frac{c}{a}\)
	叫做椭圆的\DefineConcept{离心率}(eccentricity).

	因为\(a > c > 0\),所以\(0 < e < 1\).

	\(e\)越接近\(1\),
	则\(c\)越接近\(a\),
	从而\(b = \sqrt{a^2 - c^2}\)越小,因此椭圆越扁.
	反之,\(e\)越接近\(0\),\(c\)越接近\(0\),
	从而\(b\)越接近\(a\),
	这时椭圆就接近于圆.

	如果\(a=b\),
	则\(c=0\),两个焦点重合,
	这时椭圆的标准方程成为\begin{equation*}
		x^2 + y^2 = a^2,
	\end{equation*}
	它的图形就是圆了.
	因此可以把圆\(x^2+y^2=a^2\)看作离心率为\(0\)的椭圆.
\end{enumerate}

\begin{example}
%@see: 《平面解析几何(甲种本)》 P87 例2
我国于1970年4月24日发射的第一颗人造地球卫星(东方红一号卫星)的运行轨道,
是以地球的重心为一个焦点的椭圆,
近地点\(A\)距离地面439公里,
远地点\(B\)距离地面2384公里,
地球半径约为6371公里.求卫星的轨道方程.
\begin{solution}
由题可知\begin{gather*}
	a - c = 6371 + 439 = 6810, \\
	a + c = 6371 + 2384 = 8755.
\end{gather*}
解得\(
	a=7782.5,
	c=972.5
\),
所以\(
	b
	=\sqrt{a^2-c^2}
	=\sqrt{(a+c)(a-c)}
	=7721.5
\).
因此,卫星的轨道方程是\begin{equation*}
	\frac{x^2}{7783^2}+\frac{y^2}{7722^2}=1.
\end{equation*}
\end{solution}
\end{example}

\begin{example}
%@see: 《平面解析几何(甲种本)》 P88 例3
点\(P(x,y)\)与定点\(F(c,0)\)的距离
和它到定直线\(l: x = \frac{a^2}{c}\)的距离的比是
常数\(\frac{c}{a}\ (a > c > 0)\).
求点\(P\)的轨迹.
\begin{solution}
设点\(P\)到直线\(l\)的距离是\(d\),
那么所求轨迹就是点集\begin{equation*}
	\Set*{ P \given \frac{\LineSegmentLength{PF}}{d} = \frac{c}{a} },
\end{equation*}
由此得\begin{equation*}
	\frac{\sqrt{(x-c)^2+y^2}}{\abs{\frac{a^2}{c}-x}} = \frac{c}{a},
\end{equation*}
化简得\begin{equation*}
	(a^2-c^2)x^2 + a^2 y^2 = a^2(a^2-c^2).
\end{equation*}
设\(b^2=a^2-c^2\),
则可将上式进一步化简为\begin{equation*}
	\frac{x^2}{a^2}+\frac{y^2}{b^2}=1.
\end{equation*}
这是椭圆的标准方程,所以点\(P\)的轨迹是椭圆.
\end{solution}
\end{example}
\begin{remark}
由上例可知,点\(P\)与一个定点的距离
和它到一条定直线的距离的比是
常数\(e = \frac{c}{a}\ (0 < e < 1)\)时,
这个点的轨迹是椭圆.
定点是椭圆的焦点,
定直线叫做椭圆的\DefineConcept{准线},
常数\(e\)是椭圆的离心率.
\end{remark}

对于椭圆\(\frac{x^2}{a^2}+\frac{y^2}{b^2}=1\ (a>b>1)\),
相应于焦点\(F(c,0)\)的准线方程是\begin{equation*}
	x = \frac{a^2}{c}
	\quad\text{或}\quad
	x = \frac{a}{e}.
\end{equation*}
根据椭圆的对称性,
相应于焦点\(F'(-c,0)\)的准线方程是\(x=-\frac{a^2}{c}\),
所以椭圆有两条准线.

\begin{example}\label{example:平面解析几何.椭圆的切线}
平面上任意一条直线\(l\)与椭圆\begin{equation*}
	C: \frac{x^2}{a^2} + \frac{y^2}{b^2} = 1 \quad(a>b>1)
\end{equation*}
的位置关系要么是相交于两点(称\(l\)为割线),
要么是相切于一点(称\(l\)为切线),要么没有公共点.
请求出椭圆在其上任意一点\(P_0(x_0,y_0)\)处的切线方程.
\begin{solution}
因为点\(P_0(x_0,y_0)\)在椭圆\(C\)上,
所以\begin{equation*}
	\frac{x_0^2}{a^2} + \frac{y_0^2}{b^2} = 1
	\quad\text{即}\quad
	b^2 x_0^2 + a^2 y_0^2 = a^2 b^2.
\end{equation*}

当切线\(l\)与\(x\)轴相垂直时,
显然只有\((-a,0)\)和\((a,0)\)两点可能是切点,
而它们各自的切线方程分别为\(x=-a\)和\(x=a\).

当切线\(l\)不与\(x\)轴相垂直时,
设切线方程为\(l: y = kx + p\),
其中\(p = y_0 - k x_0\).
联立方程组,得\begin{equation*}
	\begin{cases}
		y = kx + p, \\
		\frac{x^2}{a^2} + \frac{y^2}{b^2} = 1.
	\end{cases}
\end{equation*}
那么有\begin{gather*}
	\frac{1}{a^2} x^2 + \frac{1}{b^2} (kx+p)^2 = 1, \\
	b^2 x^2 + a^2 (k^2 x^2 + 2kpx + p^2) = a^2 b^2, \\
	(a^2 k^2 + b^2) x^2 + 2 a^2 k p x + a^2 (p^2 - b^2) = 0.
\end{gather*}
令判别式\(
	\Delta_1
	= (2 a^2 k p)^2 - 4 (a^2 k^2 + b^2) a^2 (p^2 - b^2)
	= 0
\),
得\begin{gather*}
	4 a^4 k^2 p^2 = 4 (a^2 k^2 + b^2) a^2 (p^2 - b^2), \\
	a^2 k^2 p^2 = (a^2 k^2 + b^2)(p^2 - b^2), \\
	[a^2 (p^2 - b^2) - a^2 p^2] k^2 + b^2 (p^2 - b^2) = 0, \\
	a^2 b^2 k^2 = b^2 (p^2 - b^2), \\
	a^2 k^2 = p^2 - b^2,
\end{gather*}
代入\(p = y_0 - k x_0\),
得\begin{gather*}
	a^2 k^2 = (y_0 - k x_0)^2 - b^2, \\
	y_0^2 - 2k x_0 y_0 + k^2 x_0^2 - b^2 = a^2 k^2, \\
	(x_0^2 - a^2) k^2 - 2 x_0 y_0 k + (y_0^2 - b^2) = 0.
\end{gather*}
上式的判别式\(
	\Delta_2
	= (-2 x_0 y_0)^2 - 4(x_0^2 - a^2)(y_0^2 - b^2)
	= 4(a^2 y_0^2 + b^2 x_0^2 - a^2 b^2)
	= 0
\),
故可解得\begin{equation*}
	k
	= \frac{2 x_0 y_0}{2 (x_0^2 - a^2)}
	= \frac{x_0 y_0}{x_0^2 - a^2}.
\end{equation*}
那么\(l\)的方程为\begin{gather*}
	y - y_0
	= \frac{x_0 y_0}{x_0^2 - a^2} (x - x_0), \\
	(x_0^2 - a^2) y
	= x_0 y_0 x + (x_0^2 - a^2) y_0 - x_0^2 y_0
	= x_0 y_0 x - a^2 y_0, \\
	x_0 y_0 x + (a^2 - x_0^2) y
	= a^2 y_0, \\
	\frac{x_0 x}{a^2} + (a^2 - x_0^2) \frac{y_0 y}{a^2 y_0^2}
	= 1, \\
	\frac{x_0 x}{a^2} + (a^2 - x_0^2) \frac{y_0 y}{(a^2 - x_0^2) b^2}
	= 1,
\end{gather*}
最后化简得\begin{equation}\label{equation:平面解析几何.椭圆的切线}
	\frac{x_0 x}{a^2} + \frac{y_0 y}{b^2} = 1.
\end{equation}
这就是椭圆在其上任意一点\(P_0(x_0,y_0)\)处的切线方程.
\end{solution}
\end{example}

\begin{example}
%@Mathematica: Manipulate[ParametricPlot[{a Cos[t],b Sin[t]},{t,0,2 \[Pi]}],{a,1,100},{b,1,100}]
点\(P(x,y)\)满足参数方程\begin{equation}
	\left\{ \begin{array}{l}
		x = a \cos\theta, \\
		y = b \sin\theta,
	\end{array} \right.
	\quad a > b > 0,
	0 \leq \theta \leq 2\pi.
\end{equation}
求点\(P\)的轨迹.
\begin{solution}
对参数方程各式两边平方,
得\begin{equation*}
	x^2 = a^2 \cos^2\theta,
	\qquad
	y^2 = b^2 \sin^2\theta.
\end{equation*}
因为\(\cos^2\theta + \sin^2\theta = 1\),
所以\begin{equation*}
	\frac{x^2}{a^2} + \frac{y^2}{b^2} = 1.
\end{equation*}%
\begin{figure}[htb]
	\centering
	\begin{tikzpicture}[scale=.7]
		%\draw[gray,help lines,dashed] (-5,-3) grid (5,3);
		\draw[thick,->] (-6,0) -- (6,0)node[above]{\(x\)};
		\draw[thick,->] (0,-6) -- (0,6)node[right]{\(y\)};
		\pgfmathsetmacro{\a}{5}
		\pgfmathsetmacro{\b}{3}
		\pgfmathsetmacro{\c}{sqrt(\a^2-\b^2)}
		\pgfmathsetmacro{\t}{40} % 参数(角度制)
		\pgfmathsetmacro{\cost}{cos(\t)}
		\pgfmathsetmacro{\sint}{sin(\t)}
		\pgfmathsetmacro{\px}{\a*\cost}
		\pgfmathsetmacro{\py}{\b*\sint}
		\pgfmathsetmacro{\ax}{\a*\cost}
		\pgfmathsetmacro{\ay}{\a*\sint}
		\pgfmathsetmacro{\bx}{\b*\cost}
		\pgfmathsetmacro{\by}{\b*\sint}
		\coordinate (O) at (0,0);
		\coordinate (P) at (\px,\py);
		\coordinate (A) at (\ax,\ay);
		\coordinate (B) at (\bx,\by);
		\coordinate (F1) at (-\c,0);
		\coordinate (F2) at (\c,0);
		\draw[blue,dashed] (O)circle(\a)circle(\b);
		\draw[orange] (0,0)ellipse[x radius=\a,y radius=\b];
		\draw[red] (O) -- (B)node[above,black]{\(B\)}
			-- (P)node[right,black]{\(P\)}
			-- (A)node[right,black]{\(A\)} -- (B);
		\fill[red] (B)circle(2pt) (P)circle(2pt) (A)circle(2pt);
		\fill (F1)circle(2pt)node[below right,black]{\(F_1\)}
			(F2)circle(2pt)node[below left,black]{\(F_2\)};
		\draw pic["\(\theta\)",draw=gray,-,angle eccentricity=1.7,angle radius=5mm]{angle=F2--O--B}
		pic[draw=gray,-,angle eccentricity=1.7,angle radius=5mm]{angle=P--B--A};
		\draw (O)node[below left]{\(O\)};
	\end{tikzpicture}
	\caption{椭圆的参数方程}
	\label{figure:平面解析几何.椭圆的参数方程}
\end{figure}
这是椭圆的标准方程,
所以点\(P\)的轨迹是椭圆(见\cref{figure:平面解析几何.椭圆的参数方程}).
\end{solution}
\end{example}

\begin{example}
设椭圆的方程为\begin{equation*}
	\frac{x^2}{a^2}+\frac{y^2}{b^2}=1,
	\quad a>b>1.
\end{equation*}
过椭圆的右焦点\(F_2(c,0)\ (c^2=a^2-b^2)\)
作\(x\)轴的垂线交椭圆于\(M\)、\(N\)两点,
求线段\(MN\)的长.
\begin{solution}
连接左焦点\(F_1(-c,0)\)与点\(M\),
则根据椭圆的定义、勾股定理,
有\begin{equation*}
	\left\{ \begin{array}{l}
		\LineSegmentLength{F_1 F_2} = 2c, \\
		\LineSegmentLength{F_1 M} + \LineSegmentLength{F_2 M} = 2a, \\
		\LineSegmentLength{F_1 M}^2 = \LineSegmentLength{F_1 F_2}^2 + \LineSegmentLength{F_2 M}^2,
	\end{array} \right.
\end{equation*}
整理得关于\(\LineSegmentLength{F_2 M}\)的方程\begin{equation*}
	(2a - \LineSegmentLength{F_2 M})^2 = (2c)^2 + \LineSegmentLength{F_2 M}^2,
\end{equation*}
解得\begin{equation*}
	\LineSegmentLength{F_2 M} = \frac{a^2 - c^2}{a} = \frac{b^2}{a}.
\end{equation*}
由椭圆的对称性可知,
\(\LineSegmentLength{F_2 M} = \LineSegmentLength{F_2 N}\),
故\begin{equation*}
	\LineSegmentLength{MN} = 2\LineSegmentLength{F_2 M} = \frac{2 b^2}{a}.
\end{equation*}
\end{solution}

连接椭圆上任意两点所得的线段叫做椭圆的\DefineConcept{弦}.
若这条弦通过椭圆的焦点,则称之为\DefineConcept{焦点弦}(focal chord).
若焦点弦垂直于椭圆的长轴,则称之为\DefineConcept{通径}(latus rectum);
因为椭圆有两个焦点,所以它有两条通径,通径长总是等于\(\frac{2b^2}{a}\).
\end{example}

\begin{theorem}[椭圆的焦点三角形]
如\cref{figure:平面解析几何.椭圆的焦点三角形} 所示,
设点\(P\)是椭圆\begin{equation*}
	C: \frac{x^2}{a^2} + \frac{y^2}{b^2} = 1
	\quad(a>b>0)
\end{equation*}上任一点,
点\(F_1\)、\(F_2\)是\(C\)的焦点,
我们称\(\triangle P F_1 F_2\)为椭圆\(C\)的焦点三角形.
那么焦点三角形\(\triangle P F_1 F_2\)的面积为\begin{equation*}
	S_{\triangle P F_1 F_2} = b^2 \tan\frac{\theta}{2},
\end{equation*}
其中\(\theta=\angle{F_1 P F_2}\).
\begin{figure}[htb]
	\centering
	\begin{tikzpicture}[scale=.7]
		%\draw[gray,help lines,dashed] (-5,-3) grid (5,3);
		\draw[thick,->] (-6,0) -- (6,0)node[above]{\(x\)};
		\draw[thick,->] (0,-4) -- (0,4)node[right]{\(y\)};
		\pgfmathsetmacro{\a}{5}
		\pgfmathsetmacro{\b}{3}
		\pgfmathsetmacro{\c}{sqrt(\a*\a-\b*\b)}
		\pgfmathsetmacro{\px}{3}
		\pgfmathsetmacro{\py}{12/5}
		\coordinate (P) at (\px,\py);
		\coordinate (F1) at (-\c,0);
		\coordinate (F2) at (\c,0);
		\draw[orange] (0,0)ellipse[x radius=\a,y radius=\b];
		\draw[red] (F1)node[below right,black]{\(F_1\)}
			-- (P)node[above right,black]{\(P\)}
			-- (F2)node[below left,black]{\(F_2\)};
		\draw pic["\(\theta\)",draw=gray,-,angle eccentricity=1.7,angle radius=3mm]{angle=F1--P--F2};
		\draw (-3,2)node[above left]{\(C: \frac{x^2}{a^2}+\frac{y^2}{b^2}=1\)};
		\draw (0,0)node[below left]{\(O\)};
	\end{tikzpicture}
	\caption{椭圆的焦点三角形}
	\label{figure:平面解析几何.椭圆的焦点三角形}
\end{figure}
%TODO proof
\end{theorem}

\begin{theorem}[椭圆的弦长]
设直线\(l: Ax+By+C=0\)
与椭圆\(C: \frac{x^2}{a^2} + \frac{y^2}{b^2} = 1\ (a>b>0)\)
相交于两点\(M\)和\(N\),那么
\begin{equation}
	\LineSegmentLength{MN}
	= \frac{2ab\sqrt{A^2+B^2}
	\sqrt{a^2 A^2 + b^2 B^2 - C^2}}{a^2 A^2 + b^2 B^2}.
\end{equation}

特别地,若\(l\)过\(C\)的焦点
(即\(MN\)是椭圆\(C\)的焦点弦),
那么\begin{equation}
	\LineSegmentLength{MN}
	= \frac{2ab^2}{a^2 - c^2 \cos^2\theta},
\end{equation}
其中\(\theta\)是\(l\)与\(x\)轴的夹角.
%TODO proof
\end{theorem}

\begin{example}
求证两椭圆\(b^2 x^2 + a^2 y^2 - a^2 b^2 = 0\)
和\(a^2 x^2 + b^2 y^2 - a^2 b^2 = 0\)的交点在以原点为中心的圆周上,
并求解这个圆的方程.
\begin{proof}
将椭圆改写为标准方程,得\begin{gather}
	\frac{x^2}{a^2}+\frac{y^2}{b^2}=1, \tag1 \\
	\frac{y^2}{a^2}+\frac{x^2}{b^2}=1. \tag2
\end{gather}
若将(1)式、(2)式联立,看作关于变量\(x^2\)和\(y^2\)的非齐次线性方程组,
则其系数矩阵为\begin{equation*}
	\def\arraystretch{1.5}
	\vb{A}
	= \begin{bmatrix}
		\frac{1}{a^2} & \frac{1}{b^2} \\
		\frac{1}{b^2} & \frac{1}{a^2}
	\end{bmatrix}.
\end{equation*}
由椭圆的定义可知,
\(a \neq b\),
故上述方程的系数行列式为\begin{equation*}
	\def\arraystretch{1.5}
	d
	= \begin{vmatrix}
		\frac{1}{a^2} & \frac{1}{b^2} \\
		\frac{1}{b^2} & \frac{1}{a^2}
	\end{vmatrix}
	= \frac{1}{a^4} - \frac{1}{b^4}
	\neq 0,
\end{equation*}
该方程具有唯一解\begin{equation*}
	\def\arraystretch{1.5}
	\left(
		\frac{1}{d}
		\begin{vmatrix}
			1 & \frac{1}{b^2} \\
			1 & \frac{1}{a^2}
		\end{vmatrix},
		\frac{1}{d}
		\begin{vmatrix}
			\frac{1}{a^2} & 1 \\
			\frac{1}{b^2} & 1
		\end{vmatrix}
	\right)^T,
\end{equation*}
即\begin{equation*}
	x^2
	= y^2
	= \frac{1}{d} \left(\frac{1}{a^2} - \frac{1}{b^2}\right)
	= \frac{a^2 b^2}{a^2 + b^2}.
	\eqno(3)
\end{equation*}
题设的两个椭圆的交点可由(3)式确定.

由椭圆的定义可知,
\(
	a>0,
	b>0
\),
故\(\frac{a^2 b^2}{a^2 + b^2} > 0\),
那么方程\begin{equation*}
	x^2 + y^2
	= \frac{2 a^2 b^2}{a^2 + b^2}
	\eqno(4)
\end{equation*}
符合圆的标准方程,
说明这两个椭圆的交点在以原点为中心,
以\(\sqrt{\frac{2 a^2 b^2}{a^2 + b^2}}\)为半径的圆上.
\end{proof}
\end{example}

\begin{example}
%@see: https://baike.baidu.com/item/%E5%A2%9E%E6%A0%B9/1613679
设椭圆\(C: \frac{x^2}{a^2} + \frac{y^2}{b^2} = 1\ (a>b>0)\),
\(O\)是坐标原点,
\(A\)是\(C\)的右顶点.
如果\(C\)上存在异于\(A\)的一点\(P\),
使得\(OP \perp PA\),
求\(C\)的离心率的范围.
\begin{solution}
假设\(C'\)是以\(OA\)为直径的圆,
那么“\(C\)上存在一点\(P\),使得\(OP \perp PA\)”的充分必要条件是“\(C'\)与\(C\)相交于不同两点”.
由题意有,点\(O\)的坐标为\((0,0)\),点\(A\)的坐标为\((a,0)\),
那么由\cref{equation:平面解析几何.圆的直径式方程} 可知
\(C'\)的方程为\((x-0)(x-a) + (y-0)(y-0) = 0\),
即\(x^2 - a x + y^2 = 0\).
联立方程组\begin{equation*}
	\begin{cases}
		x^2 - a x + y^2 = 0, \\
		\frac{x^2}{a^2} + \frac{y^2}{b^2} = 1.
	\end{cases}
\end{equation*}
消去\(y^2\),得\(b^2 x^2 + a^2 (a x - x^2) = a^2 b^2\),
再整理得\begin{equation*}
	(b^2 - a^2) x^2 + a^3 x - a^2 b^2 = 0.
	\eqno(1)
\end{equation*}
容易注意到判别式\(
	\Delta
	\defeq (a^3)^2 - 4 (b^2 - a^2) (-a^2 b^2)
	= a^6 - 4 a^4 b^2 + 4 a^2 b^4
	= a^2 (a^2 - 2 b^2)^2
	\geq 0
\),
说明方程(1)一定有两个实根.
%@credit: {c791c701-7186-45bd-86fa-47bac11b3e12} 提出可以利用韦达定理直接解出另一个根
由于\(C\)和\(C'\)都过点\(A\),
所以\(x=a\)是方程(1)的一个根.
考虑到\(C\)与\(C'\)应该交于不同两点,
不妨设\(x=\xi\)是方程(1)的异于\(a\)另一个根,
那么由\hyperref[theorem:一元二次方程.韦达定理]{韦达定理}有
\(a \xi = \frac{- a^2 b^2}{b^2 - a^2}\),
于是\(\xi = \frac{a b^2}{a^2 - b^2}\).
因为圆\(C'\)上点的横坐标的取值范围是\([0,a]\),
并且,点\(O(0,0)\)虽然在圆\(C'\)上却不在椭圆\(C\)上,
所以\(0 < \xi < a\),
从而有\(0 < \frac{b^2}{a^2 - b^2} < 1\),
即\(\frac{a^2 - b^2}{b^2} > 1\),
亦即\(\frac{a^2}{b^2} > 2\),
那么由\(
	e = \frac{c}{a}
	= \frac{\sqrt{a^2-b^2}}{a}
	= \sqrt{1 - \frac{b^2}{a^2}}
\)可知,
离心率的取值范围是\(\frac1{\sqrt2} < e < 1\).
%@Mathematica: PlotEllipse[a_, e_] := Module[
%				{b, c},
%				c = e a;
%				b = Sqrt[a^2 - c^2];
%				{
%					{"e", e},
%					{"plot",
%					Plot[{-Sqrt[((a^2 b^2 - b^2 x^2)/a^2)], Sqrt[(
%						a^2 b^2 - b^2 x^2)/a^2], -Sqrt[a x - x^2], Sqrt[
%						a x - x^2]}, {x, -a, a}, AspectRatio -> 1,
%					PlotRange -> {-3, 3}]},
%					{"intersect above axis x",
%					NSolve[x (x - a) + y^2 == 0 && x^2/a^2 + y^2/b^2 == 1 &&
%						y > 0, {x, y}]}
%					} // TableForm
%				];
%@Mathematica: Manipulate[PlotEllipse[3, e], {e, 0.1, .9, .01}]
\end{solution}
\end{example}
