\section{圆}
\subsection{圆的概念}
%@see: 《平面解析几何(甲种本)》 P72
平面内与定点距离等于定长的点的集合
称为\DefineConcept{圆}.
定点称为\DefineConcept{圆心},
定长称为\DefineConcept{半径}.

\subsection{圆的标准方程}
%@see: 《平面解析几何(甲种本)》 P72
根据圆的定义,我们来求圆心是\(C(a,b)\),半径是\(r\)的圆的方程.

设点\(P(x,y)\)是圆上任意一点,根据定义,点\(P\)到圆心\(C\)的距离等于\(r\).
圆就是点集\begin{equation*}
	\Set{ P \given \LineSegmentLength{PC} = r }.
\end{equation*}

由两点间的距离公式,
点\(M\)适合的条件,
可以表示为\begin{equation*}
%@see: 《平面解析几何(甲种本)》 P72 (1)
	\LineSegmentLength{PC} = \sqrt{(x-a)^2+(y-b)^2} = r.
\end{equation*}
把上式等号两边分别平方,
得\begin{equation}\label{equation:平面解析几何.圆的标准方程}
%@see: 《平面解析几何(甲种本)》 P72 (2)
	(x-a)^2+(y-b)^2 = r^2.
\end{equation}

\cref{equation:平面解析几何.圆的标准方程}
就是圆心是\(C(a,b)\),半径是\(r\)的圆的方程,
我们称之为\DefineConcept{圆的标准方程}.

特别地,如果圆心在坐标原点,
这时\(a=b=0\),
那么圆的方程就是\begin{equation}
	x^2 + y^2 = r^2.
\end{equation}

\begin{example}
%@see: 《平面解析几何(甲种本)》 P80 习题五 3.
已知一个圆的直径端点是\(A(x_1,y_1)\)和\(B(x_2,y_2)\).
证明:圆的方程是\begin{equation}\label{equation:平面解析几何.圆的直径式方程}
	(x-x_1)(x-x_2) + (y-y_1)(y-y_2) = 0.
\end{equation}
\begin{proof}
显然线段\(AB\)的中点\(C(x_0,y_0)\)就是圆心,
其中\(
	x_0 \defeq (x_1+x_2)/2,
	y_0 \defeq (y_1+y_2)/2
\),
并且圆的半径为\begin{equation*}
	r = \frac{\sqrt{(x_1-x_2)^2+(y_1-y_2)^2}}{2}.
\end{equation*}
那么所求圆的标准方程为\begin{equation*}
	(x-x_0)^2 + (y-y_0)^2 = r^2.
\end{equation*}
代入\(x_0,y_0,r\),得\begin{equation*}
	\left( x - \frac{x_1+x_2}{2} \right)^2
	+ \left( y - \frac{y_1+y_2}{2} \right)^2
	= \frac{(x_1-x_2)^2+(y_1-y_2)^2}{4},
\end{equation*}
整理得\begin{equation*}
	(x-x_1)(x-x_2) + (y-y_1)(y-y_2) = 0.
	\qedhere
\end{equation*}
\end{proof}
\end{example}

\subsection{圆的一般方程}
%@see: 《平面解析几何(甲种本)》 P76
把圆的标准方程\begin{equation*}
	(x-a)^2 + (y-b)^2 = r^2
\end{equation*}
展开,得\begin{equation*}
	x^2 + y^2 - 2ax - 2by + a^2 + b^2 - r^2 = 0.
\end{equation*}
可见,任何一个圆的方程都可以写成下面的形式\begin{equation}\label{equation:平面解析几何.圆的一般方程}
%@see: 《平面解析几何(甲种本)》 P76 (1)
	x^2 + y^2 + Dx + Ey + F = 0.
\end{equation}
反过来,我们来研究上式确定的曲线是不是圆.
将上式左边配方,
得\begin{equation}\label{equation:平面解析几何.圆的一般方程.配方后}
%@see: 《平面解析几何(甲种本)》 P77 (2)
	\left(x+\frac{D}{2}\right)^2+\left(y+\frac{E}{2}\right)^2 = \frac{D^2+E^2-4F}{4}.
\end{equation}
下面分情况讨论.
\begin{enumerate}
	\item 当\(D^2+E^2-4F > 0\)时,
	比较方程 \labelcref{equation:平面解析几何.圆的一般方程.配方后} 与圆的标准方程,
	可知上述曲线是以\((-\frac{D}{2},-\frac{E}{2})\)为圆心、
	\(\frac{1}{2} \sqrt{D^2+E^2-4F}\)为半径的圆;

	\item 当\(D^2+E^2-4F = 0\)时,
	方程 \labelcref{equation:平面解析几何.圆的一般方程.配方后} 只有实数解\(x=-\frac{D}{2}\),\(y=-\frac{E}{2}\),
	所以上述曲线是一个点\((-\frac{D}{2},-\frac{E}{2})\);

	\item 当\(D^2+E^2-4F < 0\)时,
	方程 \labelcref{equation:平面解析几何.圆的一般方程.配方后} 没有实数解,因而它不表示任何图形.
\end{enumerate}
%@see: 《平面解析几何(甲种本)》 P77
综上所述,
当\(D^2+E^2-4F > 0\)时,
方程 \labelcref{equation:平面解析几何.圆的一般方程.配方后} 表示一个圆,
叫做\DefineConcept{圆的一般方程}.

圆的标准方程的优点在于它明确地指出了圆心和半径,
而一般方程突出了方程形式上的特点:\begin{enumerate}
	\item \(x^2\)和\(y^2\)的系数相同,且均不为零;
	\item 没有\(xy\)这样的二次项.
\end{enumerate}
以上两点是二元二次方程\begin{equation*}
	A x^2 + B xy + C y^2 + D x + E y + F = 0
\end{equation*}
表示圆的必要不充分条件.

要求圆的一般方程,只要求出\(D,E,F\)这三个系数就可以了.

\begin{example}
%@see: 《平面解析几何(甲种本)》 P77 例1
求过三点\(O(0,0),M_1(1,1),M_2(4,2)\)的圆的方程.
\begin{solution}
设所求圆的方程为\(x^2 + y^2 + D x + E y + F = 0\).
用待定系数法,建立关于\(D,E,F\)的三元一次方程组\begin{equation*}
	\def\PtEqx#1#2{#1^2 + #2^2 + #1 D + #2 E + F = 0}
	\left\{ \begin{array}{l}
		\PtEqx{0}{0}, \\
		\PtEqx{1}{1}, \\
		\PtEqx{4}{2}.
	\end{array} \right.
\end{equation*}
解得\(D = -8, E = 6, F = 0\),
于是所求圆的方程为\(x^2 + y^2 - 8x + 6y = 0\).
\end{solution}
\end{example}

\begin{example}
%@see: 《平面解析几何(甲种本)》 P78 例2
已知一曲线是与两个定点\(O(0,0),A(3,0)\)距离之比为\(\frac12\)的点的轨迹.
求这个曲线的方程.
\begin{solution}
设\(M(x,y)\)是所求曲线\(S\)上任意一点,即\begin{equation*}
	S = \Set*{
		M
		\given
		\frac{\LineSegmentLength{OM}}{\LineSegmentLength{AM}} = \frac12
	}.
\end{equation*}
由两点间的距离公式,
点\(M\)所适合的条件可以表示为\begin{equation*}
%@see: 《平面解析几何(甲种本)》 P78 (1)
	\frac{\sqrt{x^2+y^2}}{\sqrt{(x-3)^2+y^2}} = \frac12.
\end{equation*}
将上式两边平方,得\begin{equation*}
	\frac{x^2+y^2}{(x-3)^2+y^2} = \frac14.
\end{equation*}
化简得\begin{equation*}
%@see: 《平面解析几何(甲种本)》 P79 (2)
	x^2+y^2+2x-3=0.
\end{equation*}
这就是所求的曲线方程.
\end{solution}
\end{example}

\subsection{圆的参数方程}
方程\begin{equation}
	\left\{ \begin{array}{l}
		x = a + r \cos\theta, \\
		y = b + r \sin\theta
	\end{array} \right.
\end{equation}
表示以\((a,b)\)为圆心、\(r\)为半径的圆,
叫做\DefineConcept{圆的参数方程}.

\subsection{圆与点的位置关系}
已知圆\(C: (x-a)^2+(y-b)^2=r^2\).
任给一点\(P(x_0,y_0)\),令\(z_0 = (x_0-a)^2+(y_0-b)^2\).
若\(z_0 < r^2\),则称点\(P\)在圆\(C\)的内部;
若\(z_0 = r^2\),则称点\(P\)在圆\(C\)上;
若\(z_0 > r^2\),则称点\(P\)在圆\(C\)的外部.

若圆\(C\)的方程是\(x^2+y^2+Dx+Ey+F=0\),
则令\(u_0 = x_0^2+y_0^2+Dx_0+Ey_0+F\).
只要\(u_0 < 0\),则称点\(P\)在圆\(C\)内;
只要\(u_0 = 0\),则称点\(P\)在圆\(C\)上;
只要\(u_0 > 0\),则称点\(P\)在圆\(C\)外.

\subsection{圆与直线的位置关系}
\begin{example}
%@see: 《平面解析几何(甲种本)》 P74 例3
已知圆的方程是\(x^2+y^2 = r^2\),
求经过圆上一点\(P(x_0,y_0)\)的切线的方程.
\begin{solution}
设切线的斜率为\(k\),半径\(OP\)的斜率为\(k_1\).
因为圆的切线垂直于过切点的半径,
于是\(k = -\frac{1}{k_1}\).
又因为\(k_1 = \frac{y_0}{x_0}\),
所以\(k = -\frac{x_0}{y_0}\),
那么经过点\(P\)的切线方程是\begin{equation*}
	y-y_0 = -\frac{x_0}{y_0} (x-x_0),
\end{equation*}
整理得\begin{equation*}
	x_0 x + y_0 y = x_0^2 + y_0^2.
\end{equation*}
因为\(P(x_0,y_0)\)在圆上,
所以\(x_0^2 + y_0^2 = r^2\),
所求切线方程是\begin{equation}
	x_0 x + y_0 y = r^2.
\end{equation}
\end{solution}
\end{example}

\begin{example}
已知平面直线\(l: x + y = a\)与圆\(C: x^2 + y^2 = r^2\).
求当直线\(l\)与圆\(C\)相交、相切或相离时,
参数\(a\)与\(r\)的关系.
%TODO
\end{example}

\subsection{圆与圆的位置关系}
已知两个圆:\begin{equation*}
	(x-x_1)^2+(y-y_1)^2=R^2
	\quad\text{和}\quad
	(x-x_2)^2+(y-y_2)^2=r^2,
\end{equation*}其中\(R \geq r\).
设两圆的圆心距为\(d\).
记方程组\begin{equation*}
	M: \begin{cases}
		(x-x_1)^2+(y-y_1)^2=R^2, \\
		(x-x_2)^2+(y-y_2)^2=r^2.
	\end{cases}
\end{equation*}

\begin{table}[htb]
	\centering
	\begin{tblr}{*3c}
		\hline
		位置关系 & 几何特征 & 代数特征 \\ \hline
		外离 & \(d>R+r\) & \(M\)无实数解 \\
		外切 & \(d=R+r\) & \(M\)有一组实数解 \\
		相交 & \(R-r<d<R+r\) & \(M\)有两组实数解 \\
		内切 & \(d=R-r\) & \(M\)有一组实数解 \\
		内含 & \(d<R-r\) & \(M\)无实数解 \\
		\hline
	\end{tblr}
	\caption{圆与圆的位置关系}
\end{table}

\subsection{圆系的方程}
%@see: https://www.bilibili.com/video/BV1UuChYJE1d/
假设有两个不重合的定点\(P_1(x_1,y_1)\)和\(P_2(x_2,y_2)\).
我们想求出经过这两个点的所有的圆的方程.
设某个圆的方程为\begin{equation*}
	x^2 + y^2 + Cx + Dy + E = 0.
\end{equation*}
依次代入\(P_1,P_2\)两点的坐标,得\begin{gather*}
	x_1 C + y_1 D + E = -(x_1^2 + y_1^2), \\
	x_2 C + y_2 D + E = -(x_2^2 + y_2^2),
\end{gather*}
或者\begin{equation}\label{equation:平面解析几何.过不重合两点的圆系方程}
	\begin{bmatrix}
		x_1 & y_1 & 1 \\
		x_2 & y_2 & 1
	\end{bmatrix}
	\begin{bmatrix}
		C \\ D \\ E
	\end{bmatrix}
	= \begin{bmatrix}
		-(x_1^2 + y_1^2) \\
		-(x_2^2 + y_2^2)
	\end{bmatrix}.
\end{equation}
由于\(P_1,P_2\)不重合,
\((x_1,y_1) \neq (x_2,y_2)\),
所以\((x_1,y_1,1),(x_2,y_2,1)\)线性无关,
从而有\begin{equation*}
	\rank\begin{bmatrix}
		x_1 & y_1 & 1 \\
		x_2 & y_2 & 1
	\end{bmatrix}
	= \rank\begin{bmatrix}
		x_1 & y_1 & 1 & -(x_1^2 + y_1^2) \\
		x_2 & y_2 & 1 & -(x_2^2 + y_2^2)
	\end{bmatrix}
	= 2 < 3.
\end{equation*}
方程 \labelcref{equation:平面解析几何.过不重合两点的圆系方程} 有无穷多解,
它的解空间是\(1\)维的,
其中每一个解都确定了一个圆.
不妨设方程 \labelcref{equation:平面解析几何.过不重合两点的圆系方程} 的通解为\begin{equation*}
	(C,D,E)^T
	= (C_0,D_0,E_0)^T
	+ \lambda (C_1,D_1,E_1)^T,
\end{equation*}
其中\(\lambda\)是任意常数.
那么过\(P_1(x_1,y_1),P_2(x_2,y_2)\)两点的圆系方程为\begin{equation*}
	x^2 + y^2 + C_0 x + D_0 y + E_0 + \lambda (C_1 x + D_1 y + E_1) = 0.
\end{equation*}
可以注意到,
\((C_0,D_0,E_0)^T\)是非齐次方程\begin{equation*}
	\begin{cases}
		x_1 C + y_1 D + E = -(x_1^2 + y_1^2), \\
		x_2 C + y_2 D + E = -(x_2^2 + y_2^2)
	\end{cases}
\end{equation*}
的一个解,
而\((C_1,D_1,E_1)^T\)是齐次方程\begin{equation*}
	\begin{cases}
		x_1 C + y_1 D + E = 0, \\
		x_2 C + y_2 D + E = 0
	\end{cases}
\end{equation*}
的解(或者说\((C_1,D_1,E_1)^T\)是由\(P_1(x_1,y_1),P_2(x_2,y_2)\)两点确定的直线的方程的参数).
