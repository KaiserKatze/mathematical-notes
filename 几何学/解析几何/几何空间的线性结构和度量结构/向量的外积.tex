\section{向量的外积}
\subsection{向量的外积的定义}
\begin{definition}
%@see: 《解析几何》(丘维声) P31 定义4.1
给定两个向量\(\vb{a},\vb{b}\).
若向量\(\vb{c}\)满足
\begin{equation}\label{equation:解析几何.向量外积的长度关系}
	\VectorLengthA{\vb{c}}
	= \VectorLengthA{\vb{a}} \VectorLengthA{\vb{b}} \sin\angle(\vb{a},\vb{b}),
\end{equation}
且\(\vb{c}\)与\(\vb{a},\vb{b}\)均垂直,\((\vb{a},\vb{b},\vb{c})\)成右手系,
则把\(\vb{c}\)称为“\(\vb{a}\)与\(\vb{b}\)的\DefineConcept{外积}(outer product)”
“\(\vb{a}\)与\(\vb{b}\)的\DefineConcept{向量积}(vector product)”
或“\(\vb{a}\)与\(\vb{b}\)的\DefineConcept{叉乘}(cross product)”,
记作\(\VectorOuterProduct{\vb{a}}{\vb{b}}\).

特别地,若\(\vb{a}\)与\(\vb{b}\)中至少有一个是零向量,则\(\VectorOuterProduct{\vb{a}}{\vb{b}}=\vb{0}\).
\end{definition}

\begin{figure}[htb]
	\centering
	\def\subwidth{.4\textwidth}
	\begin{subfigure}[b]{\subwidth}
		\centering
		\begin{tikzpicture}[scale=.9]
			\coordinate(O)at(0,0);
			\fill[black!30,rotate=-7](-2.5,-2)--++(5,0)--++(2,3.5)--++(-5,0)--(-2.5,-2);
			\begin{scope}[->,>=Stealth,ultra thick]
				\draw(O)--(1.5,-1.5)coordinate(A)node[right]{\(\vb{a}\)};
				\draw(O)--(2,.7)coordinate(B)node[right]{\(\vb{b}\)};
				\draw[red](O)--(0,3)node[black,right]{\(\VectorOuterProduct{\vb{a}}{\vb{b}}\)};
			\end{scope}
			\draw pic["\(\theta\)",draw=orange,->,angle eccentricity=1.5,angle radius=1cm]
				{angle=A--O--B};
		\end{tikzpicture}
		\caption{}
	\end{subfigure}
	\begin{subfigure}[b]{\subwidth}
		\centering
		\begin{tikzpicture}[scale=.9]
			\coordinate(O)at(0,0);
			\fill[black!30,rotate=-7,name path=u]
				(-2.5,-2)--++(5,0)--++(2,3.5)--++(-5,0)--(-2.5,-2);
			\path[name path=v](O)--(0,-3);
			\draw[name intersections={of=u and v}]
				[red,dashed,ultra thick](O)--(intersection-1);
			\begin{scope}[->,>=Stealth,ultra thick]
				\draw(O)--(1.5,-1.5)coordinate(A)node[right]{\(\vb{a}\)};
				\draw(O)--(2,.7)coordinate(B)node[right]{\(\vb{b}\)};
				\draw[red](intersection-1)--(0,-3)node[black,right]{\(\VectorOuterProduct{\vb{b}}{\vb{a}}\)};
			\end{scope}
			\draw pic["\(\theta\)",draw=orange,<-,angle eccentricity=1.5,angle radius=1cm]
				{angle=A--O--B};
		\end{tikzpicture}
		\caption{}
	\end{subfigure}
	\caption{}
\end{figure}

从定义可以看出,\(\VectorOuterProduct{\vb{a}}{\vb{b}}=\vb{0}\)的充分必要条件是\(\vb{a},\vb{b}\)共线.
因此要特别注意:若\(\VectorOuterProduct{\vb{a}}{\vb{b}}=\vb{0}\),
不能断定\(\vb{a},\vb{b}\)中必有一个为\(\vb{0}\).
这是与数的乘法很不一样的地方.

\subsection{向量的外积的几何意义、有向平面}
当向量\(\vb{a}\)与\(\vb{b}\)不共线时,
从\cref{equation:解析几何.向量外积的长度关系} 容易看出\(\VectorOuterProduct{\vb{a}}{\vb{b}}\)的几何意义,
即\(\VectorOuterProduct{\vb{a}}{\vb{b}}\)表示以\(\vb{a},\vb{b}\)为邻边的平行四边形的面积.
但要说明\(\VectorOuterProduct{\vb{a}}{\vb{b}}\)的方向的几何意义,
我们需要先给出所谓的“平面的定向”的概念.

平面的定向,就是平面上的旋转方向.
在平面几何中,常用“逆时针方向”与“顺时针方向”来描述平面上的两个旋转方向.
对于放在三维空间中的平面,这种说法不足以描述平面上的旋转方向.
从一侧看来是逆时针的旋转方向,从另一侧看就成了顺时针的.
因此我们需要采用另外的方法来描述.

我们知道,给定三个不共线的点\(O,A,B\),
我们可以确定一个平面\(OAB\),
且向量\(\vb{a}=\vec{OA},
\vb{b}=\vec{OB}\)不共线.
如果规定了这两个向量的先后顺序,
则从第一个向量到第二个向量的转角小于\(\pi\)的旋转方向,
就称为平面\(OAB\)的一个定向.
例如,如果规定“先\(\vb{a}\)后\(\vb{b}\)”的顺序,
则从\(\vb{a}\)到\(\vb{b}\)的转角小于\(\pi\)的旋转方向是平面\(OAB\)的一个定向;
但是,如果规定“先\(\vb{b}\)后\(\vb{a}\)”的顺序,
则从\(\vb{b}\)到\(\vb{a}\)的转角小于\(\pi\)的旋转方向是平面\(OAB\)的另一个定向,
它与前述方向恰好相反.

平面的两个定向,也可以用平面的两侧来代表.
如果右手四指沿平面上取定的旋转方向弯曲,拇指必指向平面的一侧.
这样,平面的两个定向就对应于平面的两侧,
而平面的两侧又可用垂直于该平面的两个方向(或单位向量)来刻画,
因此通常也用垂直于平面的方向来表示平面的定向.
设\(\vb{e}_1\)是与平面\(OAB\)垂直的单位向量,
如果右手四指从\(\vb{a}\)弯向\(\vb{b}\)(转角小于\(\pi\))时,
右手拇指的指向是\(\vb{e}_1\)的方向,
则\(\vb{e}_1\)表示的平面\(OAB\)的定向就是由\(\vb{a}\)到\(\vb{b}\)的旋转方向.
设单位向量\(\vb{e}_2\)与\(\vb{e}_1\)方向相反,
则\(\vb{e}_2\)表示平面\(OAB\)的定向就是由\(\vb{b}\)到\(\vb{a}\)的旋转方向.

现在再回头来看外积\(\VectorOuterProduct{\vb{a}}{\vb{b}}\)的方向的几何意义.
\(\VectorOuterProduct{\vb{a}}{\vb{b}}\)的方向给出了
以\(\vb{a},\vb{b}\)为邻边的平行四边形的边界的一个绕行方向:
即让右手的拇指指向\(\VectorOuterProduct{\vb{a}}{\vb{b}}\)的方向,
右手其余四指的弯向就是这个方向.
对于一个平行四边形,如果给它的边界指定了一个绕行方向,则称它是\DefineConcept{定向平行四边形}.
因此,\(\VectorOuterProduct{\vb{a}}{\vb{b}}\)的方向的几何意义
就是它给以\(\vb{a},\vb{b}\)为邻边的平行四边形确定了一个定向.

假定我们已经用单位向量\(\vb{e}\)规定了平面\(OAB\)的定向.
对于这个平面上的定向平行四边形,
可以给它的面积赋予一个正号或负号:
如果它的方向与平面的定向一致,则规定它的面积是正的;
如果不一致,则规定它的面积是负的.
这就叫定向平行四边形的\DefineConcept{定向面积}.
以\(\vb{a},\vb{b}\)为邻边,
并且定向为\(\VectorOuterProduct{\vb{a}}{\vb{b}}\)的平行四边形的定向面积
用\(S(\vb{a},\vb{b})\)表示.
于是,当\(\VectorOuterProduct{\vb{a}}{\vb{b}}\)与\(\vb{e}\)同向时,
\(S(\vb{a},\vb{b}) > 0\);
当\(\VectorOuterProduct{\vb{a}}{\vb{b}}\)与\(\vb{e}\)反向时,
\(S(\vb{a},\vb{b}) < 0\).
又因为\(
	\VectorLengthA{\VectorOuterProduct{\vb{a}}{\vb{b}}}
	= \abs{S(\vb{a},\vb{b})}
\),
所以\begin{equation}\label{equation:解析几何.向量外积与定向平行四边形的定向面积之间的关系}
%@see: 《解析几何》(丘维声) P33 (4.2)
	\VectorOuterProduct{\vb{a}}{\vb{b}}
	= S(\vb{a},\vb{b})
	\vb{e}.
\end{equation}

\subsection{向量的外积的运算规则}
\begin{theorem}
%@see: 《解析几何》(丘维声) P33 命题4.1
若\(\vb{a}\neq\vb{0}\),
则\(\VectorOuterProduct{\vb{a}}{\vb{b}}=\VectorOuterProduct{\vb{a}}{\vb{b}_2}\),
其中\(\vb{b}_2\)是\(\vb{b}\)沿方向\(\vb{a}\)下的外射影.
\end{theorem}

\begin{theorem}
%@see: 《解析几何》(丘维声) P33 命题4.2
设\(\vb{e}\)是单位向量,\(\vb{b}\perp\vb{e}\),
则\(\VectorOuterProduct{\vb{e}}{\vb{b}}\)等于
\(\vb{b}\)按右手螺旋法则绕\(\vb{e}\)旋转\(\frac{\pi}{2}\)得到的向量\(\vb{b}_1\).
\end{theorem}

\begin{corollary}
%@see: 《解析几何》(丘维声) P34 推论4.1
若\([O;\vb{e}_1,\vb{e}_2,\vb{e}_3]\)为右手直角坐标系,则有\begin{equation*}
	\VectorOuterProduct{\vb{e}_1}{\vb{e}_2} = \vb{e}_3, \qquad
	\VectorOuterProduct{\vb{e}_2}{\vb{e}_3} = \vb{e}_1, \qquad
	\VectorOuterProduct{\vb{e}_3}{\vb{e}_1} = \vb{e}_2.
\end{equation*}
\end{corollary}

\begin{theorem}
%@see: 《解析几何》(丘维声) P34 定理4.1
对于任意向量\(\vb{a},\vb{b},\vb{c}\),任意实数\(\lambda\),有
\begin{itemize}
	\item 反交换律,即\begin{equation*}
		\VectorOuterProduct{\vb{a}}{\vb{b}}
		= -\VectorOuterProduct{\vb{b}}{\vb{a}};
	\end{equation*}
	\item 线性性,即\begin{equation*}
		\VectorOuterProduct{(\lambda \vb{a})}{\vb{b}}
		= \lambda(\VectorOuterProduct{\vb{a}}{\vb{b}})
		= \VectorOuterProduct{\vb{a}}{(\lambda \vb{b});}
	\end{equation*}
	\item 左分配律,即\begin{equation*}
		\VectorOuterProduct{\vb{a}}{(\vb{b}+\vb{c})}
		= \VectorOuterProduct{\vb{a}}{\vb{b}}
		+ \VectorOuterProduct{\vb{a}}{\vb{c}};
	\end{equation*}
	\item 右分配律,即\begin{equation*}
		\VectorOuterProduct{(\vb{b}+\vb{c})}{\vb{a}}
		= \VectorOuterProduct{\vb{b}}{\vb{a}}
		+ \VectorOuterProduct{\vb{c}}{\vb{a}}.
	\end{equation*}
\end{itemize}
\end{theorem}

\subsection{利用坐标计算向量的外积}
首先取一个仿射标架\([O;\vb{d}_1,\vb{d}_2,\vb{d}_3]\),
设向量\(\vb{a},\vb{b}\)的坐标分别是\begin{equation*}
	(a_1,a_2,a_3)^T, \qquad
	(b_1,b_2,b_3)^T,
\end{equation*}
则\begin{equation}
\begin{split}
	\VectorOuterProduct{\vb{a}}{\vb{b}}
	&= \VectorOuterProduct
		{(a_1 \vb{d}_1 + a_2 \vb{d}_2 + a_3 \vb{d}_3)}
		{(b_1 \vb{d}_1 + b_2 \vb{d}_2 + b_3 \vb{d}_3)} \\
	&= (a_1 b_2 - a_2 b_1) \VectorOuterProduct{\vb{d}_1}{\vb{d}_2}
	+ (a_3 b_1 - a_1 b_3) \VectorOuterProduct{\vb{d}_3}{\vb{d}_1}
	+ (a_2 b_3 - a_3 b_2) \VectorOuterProduct{\vb{d}_2}{\vb{d}_3}.
\end{split}
\end{equation}
由此可见,只要知道基向量之间的外积,就可求出\(\VectorOuterProduct{\vb{a}}{\vb{b}}\).

如果我们取定的是直角标架\([O;\vb{e}_1,\vb{e}_2,\vb{e}_3]\),则
\begin{equation}
	\VectorOuterProduct{\vb{a}}{\vb{b}}
	= \begin{vmatrix}
		a_2 & b_2 \\
		a_3 & b_3
	\end{vmatrix}
	\vb{e}_1
	- \begin{vmatrix}
		a_1 & b_1 \\
		a_3 & b_3
	\end{vmatrix}
	\vb{e}_2
	+ \begin{vmatrix}
		a_1 & b_1 \\
		a_2 & b_2
	\end{vmatrix}
	\vb{e}_3.
\end{equation}

于是我们有下述定理.
\begin{theorem}
%@see: 《解析几何》(丘维声) P36 定理4.2
设\(\vb{a},\vb{b}\)在右手直角坐标系中的坐标分别为\begin{equation*}
	(a_1,a_2,a_3)^T, \qquad
	(b_1,b_2,b_3)^T,
\end{equation*}
则
\begin{equation}
	\VectorOuterProduct{\vb{a}}{\vb{b}}
	= \left( \begin{vmatrix}
		a_2 & b_2 \\
		a_3 & b_3
	\end{vmatrix},
	- \begin{vmatrix}
		a_1 & b_1 \\
		a_3 & b_3
	\end{vmatrix},
	\begin{vmatrix}
		a_1 & b_1 \\
		a_2 & b_2
	\end{vmatrix} \right)^T.
\end{equation}
\end{theorem}
为了便于记忆,我们常常将上式写作\begin{equation}
	\VectorOuterProduct{\vb{a}}{\vb{b}}
	= \begin{vmatrix}
		\vb{e}_1 & \vb{e}_2 & \vb{e}_3 \\
		a_1 & a_2 & a_3 \\
		b_1 & b_2 & b_3
	\end{vmatrix}.
\end{equation}

\subsection{二重外积}
向量的外积是否满足结合律?
首先让我们探索\(\VectorOuterProduct{\vb{a}}{(\VectorOuterProduct{\vb{b}}{\vb{c}})}\)的计算结果.
设\(\vb{b}=\vec{OB},\vb{c}=\vec{OC}\)不共线,
从外积的定义可知,
\(\VectorOuterProduct{\vb{b}}{\vb{c}}\)垂直于由\(\vb{b},\vb{c}\)确定的平面\(OBC\).
又由于\(\VectorOuterProduct{\vb{a}}{(\VectorOuterProduct{\vb{b}}{\vb{c}})}\)与\(\VectorOuterProduct{\vb{b}}{\vb{c}}\)垂直,
因此\(\VectorOuterProduct{\vb{a}}{(\VectorOuterProduct{\vb{b}}{\vb{c}})}\)在平面\(OBC\)内,
从而\(\VectorOuterProduct{\vb{a}}{(\VectorOuterProduct{\vb{b}}{\vb{c}})} = k_1 \vb{b} + k_2 \vb{c}\),
其中\(k_1,k_2\)待定.

\begin{theorem}
%@see: 《解析几何》(丘维声) P37 命题4.3
对任意向量\(\vb{a},\vb{b},\vb{c}\),有
\begin{equation}\label{equation:解析几何.二重外积公式}
	\VectorOuterProduct{\vb{a}}{(\VectorOuterProduct{\vb{b}}{\vb{c}})}
	= (\VectorInnerProductDot{\vb{a}}{\vb{c}}) \vb{b}
	- (\VectorInnerProductDot{\vb{a}}{\vb{b}}) \vb{c}.
\end{equation}
\end{theorem}
\cref{equation:解析几何.二重外积公式}
称为\DefineConcept{二重外积公式}.

由\cref{equation:解析几何.二重外积公式}
和外积的反交换律可以得到\begin{equation*}
	\VectorOuterProduct{(\VectorOuterProduct{\vb{a}}{\vb{b}})}{\vb{c}}
	= -\VectorOuterProduct{\vb{c}}{(\VectorOuterProduct{\vb{a}}{\vb{b}})}
	= -(\VectorInnerProductDot{\vb{c}}{\vb{b}}) \vb{a}
	+ (\VectorInnerProductDot{\vb{c}}{\vb{a}}) \vb{b}.
\end{equation*}
从而在一般情况下,
\(\VectorOuterProduct{\vb{a}}{(\VectorOuterProduct{\vb{b}}{\vb{c}})}
\neq
\VectorOuterProduct{(\VectorOuterProduct{\vb{a}}{\vb{b}})}{\vb{c}}\),
即向量的外积不适合结合律.

容易证明下述\DefineConcept{雅克比等式}:
\begin{equation}\label{equation:解析几何.雅克比等式}
	\VectorOuterProduct{\vb{a}}{(\VectorOuterProduct{\vb{b}}{\vb{c}})}
	+ \VectorOuterProduct{\vb{b}}{(\VectorOuterProduct{\vb{c}}{\vb{a}})}
	+ \VectorOuterProduct{\vb{c}}{(\VectorOuterProduct{\vb{a}}{\vb{b}})}
	= \vb{0}.
\end{equation}

\begin{example}
%@see: 《解析几何》(丘维声) P38 习题1.4 1.
证明:\begin{equation*}
	\abs{\VectorOuterProduct{\vb{a}}{\vb{b}}}^2
	= \VectorLengthA{\vb{a}}^2 \VectorLengthA{\vb{b}}^2 - (\VectorInnerProductDot{\vb{a}}{\vb{b}})^2.
\end{equation*}
\begin{proof}
\def\t{\angle(\vb{a},\vb{b})}%
直接计算得\begin{align*}
	\abs{\VectorOuterProduct{\vb{a}}{\vb{b}}}^2
	&= (\VectorLengthA{\vb{a}}\VectorLengthA{\vb{b}}\sin\t)^2 \\
	&= \VectorLengthA{\vb{a}}^2 \VectorLengthA{\vb{b}}^2 (1-\cos^2\t) \\
	&= \VectorLengthA{\vb{a}}^2 \VectorLengthA{\vb{b}}^2
	- (\VectorLengthA{\vb{a}}\VectorLengthA{\vb{b}}\cos\t)^2 \\
	&= \VectorLengthA{\vb{a}}^2 \VectorLengthA{\vb{b}}^2
	- (\VectorInnerProductDot{\vb{a}}{\vb{b}})^2.
	\qedhere
\end{align*}
\end{proof}
\end{example}

\begin{example}
%@see: 《解析几何》(丘维声) P38 习题1.4 5.
证明:\begin{equation*}
	\VectorOuterProduct{(\vb{a}-\vb{b})}{(\vb{a}+\vb{b})}
	= 2(\VectorOuterProduct{\vb{a}}{\vb{b}}).
\end{equation*}
\begin{proof}
直接计算得\begin{align*}
	\VectorOuterProduct{(\vb{a}-\vb{b})}{(\vb{a}+\vb{b})}
	&= \VectorOuterProduct{(\vb{a}-\vb{b})}{\vb{a}}
		+ \VectorOuterProduct{(\vb{a}-\vb{b})}{\vb{b}} \\
	&= \VectorOuterProduct{\vb{a}}{\vb{a}}-\VectorOuterProduct{\vb{b}}{\vb{a}}
	+ \VectorOuterProduct{\vb{a}}{\vb{b}}-\VectorOuterProduct{\vb{b}}{\vb{b}} \\
	&= \vb{0}+\VectorOuterProduct{\vb{a}}{\vb{b}}
	+ \VectorOuterProduct{\vb{a}}{\vb{b}}-\vb{0} \\
	&= 2(\VectorOuterProduct{\vb{a}}{\vb{b}}).
	\qedhere
\end{align*}
\end{proof}
\end{example}
