\section{向量的内积}
关于角的度量问题如何利用向量来解决?

\subsection{射影与分量}
几何空间\(V\)中,给定一个单位向量\(\vb{e}\),
过点\(O\)作直线\(l\),其方向向量为\(\vb{e}\);
过点\(O\)做一个平面\(\pi\)与\(l\)垂直,
在平面\(\pi\)上取两个互相垂直的单位向量\(\vb{e}_1,\vb{e}_2\),
则\([O;\vb{e}_1,\vb{e}_2,\vb{e}]\)是几何空间\(V\)的一个直角坐标系.
于是,如\cref{figure:解析几何.内射影与外射影},
任给一个向量\(\vb{a}\),它总可唯一地分解成\begin{equation*}
	\vb{a}
	= x \vb{e}_1 + y \vb{e}_2 + z \vb{e}
	= \vb{a}_2 + \vb{a}_1,
\end{equation*}
其中\(\vb{a}_2 = x \vb{e}_1 + y \vb{e}_2\),
\(\vb{a}_1 = z \vb{e}\).

可见\(\vb{a}_2 \perp \vb{e}\),\(\vb{a}_1\)与\(\vb{e}\)共线.
我们把\(\vb{a}_1\)称为“\(\vb{a}\)在方向\(\vb{e}\)上的\DefineConcept{内射影}”
或“\(\vb{a}\)在方向向量为\(\vb{e}\)的轴\(l\)上的\DefineConcept{正投影}”,
记作\(\Prj_{\vb{e}}(\vb{a})\);
把\(\vb{a}_2\)称为“\(\vb{a}\)沿方向\(\vb{e}\)下的\DefineConcept{外射影}”.

\begin{figure}[htb]
	\centering
	\begin{tikzpicture}
		\coordinate(O)at(0,0);
		\coordinate(A)at(2,0);
		\coordinate(B)at(1,2);
		\draw[red,dashed](B)--(O-|B)coordinate(C)
			(B)--(B-|O)coordinate(D);
		\draw pic[draw=gray,-,angle radius=0.2cm]{right angle=A--O--D};
		\begin{scope}[->,>=Stealth]
			\draw[ultra thick](O)--(A)node[right]{\(\vb{e}\)};
			\draw[ultra thick](O)--(B)node[right]{\(\vb{a}\)};
			\draw[red](O)--(C)node[below]{\(\vb{a}_1\)};
			\draw[red](O)--(D)node[left]{\(\vb{a}_2\)};
		\end{scope}
		\fill(O)circle(1pt);
	\end{tikzpicture}
	\caption{}
	\label{figure:解析几何.内射影与外射影}
\end{figure}

\begin{figure}[htb]
	\centering
	\def\subwidth{.4\linewidth}
	\begin{subfigure}[b]{\subwidth}
		\centering
		\begin{tikzpicture}
			\coordinate(O)at(0,0);
			\coordinate(A)at(2,0);
			\coordinate(B)at(1,2);
			\draw[red,dashed](B)--(O-|B)coordinate(C);
			\begin{scope}[->,>=Stealth]
				\draw[ultra thick](O)--(A)node[right]{\(\vb{v}\)};
				\draw[ultra thick](O)--(B)node[left]{\(\vb{u}\)};
				\draw[red](O)--(C)node[below]{\(\Prj_{\vb{v}}\vb{u}\)};
			\end{scope}
			\draw pic["\(\theta\)",draw=orange,-,angle eccentricity=2,angle radius=0.3cm]
				{angle=A--O--B};
			\fill(O)circle(1pt);
		\end{tikzpicture}
		\caption{}
	\end{subfigure}
	\begin{subfigure}[b]{\subwidth}
		\centering
		\begin{tikzpicture}
			\coordinate(O)at(0,0);
			\coordinate(A)at(2,0);
			\coordinate(B)at(-1,2);
			\draw[red,dashed](B)--(O-|B)coordinate(C);
			\begin{scope}[->,>=Stealth]
				\draw[ultra thick](O)--(A)node[right]{\(\vb{v}\)};
				\draw[ultra thick](O)--(B)node[left]{\(\vb{u}\)};
				\draw[red](O)--(C)node[below]{\(\Prj_{\vb{v}}\vb{u}\)};
			\end{scope}
			\draw pic["\(\theta\)",draw=orange,-,angle eccentricity=2,angle radius=0.3cm]
				{angle=A--O--B};
				\fill(O)circle(1pt);
		\end{tikzpicture}
		\caption{}
	\end{subfigure}
	\caption{}
\end{figure}

\begin{theorem}
%@see: 《解析几何》(丘维声) P24 命题3.1
对于几何空间中的任意两个向量\(\vb{a},\vb{b}\),任意实数\(\lambda\),有\begin{gather}
	\Prj_{\vb{e}}(\vb{a}+\vb{b})
	= \Prj_{\vb{e}}(\vb{a})
	+ \Prj_{\vb{e}}(\vb{b}), \\
	\Prj_{\vb{e}}(\lambda \vb{a})
	= \lambda \Prj_{\vb{e}}(\vb{a}).
\end{gather}
\end{theorem}

由于\(\vb{a}\)在方向\(\vb{e}\)上的内射影\(\vb{a}_1\)与\(\vb{e}\)共线,
因此存在唯一的实数\(\mu\),使得\(\vb{a}_1 = \mu \vb{e}\).
把这个实数\(\mu\)称为“\(\vb{a}\)在方向\(\vb{e}\)上的\DefineConcept{分量}(component)”,
记作\((\vb{a})_{\vb{e}}\).

\begin{theorem}
%@see: 《解析几何》(丘维声) P25 命题3.2
几何空间中任一向量\(\vb{a}\)在方向\(\vb{e}\)上的分量为\begin{equation}
	(\vb{a})_{\vb{e}}
	= \VectorLengthA{\vb{a}} \cos\angle(\vb{a},\vb{e}).
\end{equation}
\end{theorem}

从内射影和分量的定义立即得到以下命题:
\begin{theorem}
%@see: 《解析几何》(丘维声) P25 命题3.3
对几何空间中任一向量\(\vb{a}\),有\begin{equation*}
	\Prj_{\vb{e}}(\vb{a})
	= (\vb{a})_{\vb{e}} \vb{e}.
\end{equation*}
\end{theorem}

\begin{theorem}
%@see: 《解析几何》(丘维声) P25 命题3.4
对几何空间中任意两个向量\(\vb{a},\vb{b}\),有\begin{gather}
	(\vb{a}+\vb{b})_{\vb{e}}
	= (\vb{a})_{\vb{e}}
	+ (\vb{a})_{\vb{e}}, \\
	(\lambda \vb{a})_{\vb{e}}
	= \lambda (\vb{a})_{\vb{e}},
	\quad\lambda\in\mathbb{R}.
\end{gather}
\end{theorem}

\subsection{向量的夹角}

\begin{definition}
给定两个非零向量\(\vb{a},\vb{b}\).
如\cref{figure:解析几何.向量的夹角},
任取一点\(O\),作\(\vec{OA}=\vb{a},\vec{OB}=\vb{b}\).
称\(\angle{AOB}\)为“向量\(\vb{a}\)和\(\vb{b}\)的夹角%
\footnote{%
在不强调“由向量\(\vb{a}\)转动到向量\(\vb{b}\)所扫过的角度”时,
也就是说当没有规定角度的正负时,
两向量间夹角\(\theta\)始终是区间\([0,\pi]\)中的一个值,
那么有\(\angle(\vb{a},\vb{b}) \equiv \angle(\vb{b},\vb{a})\);%
否则,规定\(\angle(\vb{a},\vb{b}) \equiv -\angle(\vb{b},\vb{a})\).%
}”,
记作\(\angle(\vb{a},\vb{b})\)或\(\angle(\vb{b},\vb{a})\).

如果\(\angle(\vb{a},\vb{b}) = 0\)或\(\angle(\vb{a},\vb{b}) = \pi\),
就称“向量\(\vb{a}\)与\(\vb{b}\) \DefineConcept{平行}(parallel)”,
记作\(\vb{a}\parallel\vb{b}\).

如果\(\angle(\vb{a},\vb{b}) = \pi/2\),
就称“向量\(\vb{a}\)与\(\vb{b}\) \DefineConcept{垂直}(perpendicular)”,
%@see: https://mathworld.wolfram.com/Perpendicular.html
记作\(\vb{a}\perp\vb{b}\).
\end{definition}

\begin{figure}[htb]
	\centering
	\begin{tikzpicture}
		\pgfmathsetmacro{\by}{sqrt(3)}
		\coordinate (O) at (0,0);
		\coordinate (A) at (2,0);
		\coordinate (B) at (1,\by);
		\draw[->] (O)node[left]{\(O\)}
			-- (A)node[right]{\(A\)}node[midway,below]{\(\vb{a}\)};
		\draw[->] (O) -- (B)node[above]{\(B\)}node[midway,above left]{\(\vb{b}\)};
		\draw pic["\(\theta\)",draw=orange,-,angle eccentricity=1.5,angle radius=4mm]
			{angle=A--O--B};

		\coordinate (O) at (5,0);
		\coordinate (A) at (7,0);
		\coordinate (B) at (4,\by);
		\draw[->] (O)node[left]{\(O\)}
			-- (A)node[right]{\(A\)}node[midway,below]{\(\vb{a}\)};
		\draw[->] (O) -- (B)node[above]{\(B\)}node[midway,left]{\(\vb{b}\)};
		\draw pic["\(\theta\)",draw=orange,-,angle eccentricity=1.7,angle radius=3mm]
			{angle=A--O--B};
	\end{tikzpicture}
	\caption{}
	\label{figure:解析几何.向量的夹角}
\end{figure}

由于零向量与任一向量的夹角\(\theta\)可以在\(0\)到\(\pi\)之间任意取值,
因此可以认为零向量与任何向量都平行,
也可以认为零向量与任何向量都垂直.

\subsection{向量的内积的定义与性质}
\begin{definition}
%@see: 《解析几何》(丘维声) P26 定义3.1
给定两个向量\(\vb{a},\vb{b}\),
称实数\(\VectorLengthA{\vb{a}} \VectorLengthA{\vb{b}} \cos\angle(\vb{a},\vb{b})\)
为“\(\vb{a}\)与\(\vb{b}\)的\DefineConcept{内积}(inner product)”
%@see: https://mathworld.wolfram.com/InnerProduct.html
或“\(\vb{a}\)与\(\vb{b}\)的\DefineConcept{数量积}(scalar product)”,
记作\(\VectorInnerProductDot{\vb{a}}{\vb{b}}\),即
\begin{equation}
	\VectorInnerProductDot{\vb{a}}{\vb{b}}
	\defeq
	\VectorLengthA{\vb{a}} \VectorLengthA{\vb{b}} \cos\angle(\vb{a},\vb{b}).
\end{equation}
\end{definition}

\begin{definition}
若向量\(\vb{a},\vb{b}\)满足\(\VectorInnerProductDot{\vb{a}}{\vb{b}}=0\),
则称“\(\vb{a}\)与\(\vb{b}\) \DefineConcept{正交}(orthogonal)”.
\end{definition}

\begin{proposition}
若零向量\(\vb{0}\)与任意一个\(\vb{a}\)的内积为零,
即\(\VectorInnerProductDot{\vb{0}}{\vb{a}}=0\).
\end{proposition}

若\(\vb{b}\neq\vb{0}\),则有
\begin{equation}
	\VectorInnerProductDot{\vb{a}}{\vb{b}}
	= (\vb{a})_{\vb{b}} \VectorLengthA{\vb{b}},
\end{equation}
若\(\vb{a}\neq\vb{0}\),则有
\begin{equation}
	\VectorInnerProductDot{\vb{a}}{\vb{b}}
	= (\vb{b})_{\vb{a}} \VectorLengthA{\vb{a}},
\end{equation}
由此可见,两向量的内积等于其中一个向量的长度,
和另一个向量在前者方向上的分量的乘积.
这表明了向量的内积与分量的关系.

由向量内积的定义式可得
\begin{gather}
	\VectorInnerProductDot{\vb{a}}{\vb{a}}=\VectorLengthA{\vb{a}}^2, \\
	\VectorLengthA{\vb{a}} = \sqrt{\VectorInnerProductDot{\vb{a}}{\vb{a}}}, \\
	\cos\angle(\vb{a},\vb{b}) = \frac{\VectorInnerProductDot{\vb{a}}{\vb{b}}}{\VectorLengthA{\vb{a}} \VectorLengthA{\vb{b}}},
	\quad \vb{a}\neq\vb{0},\vb{b}\neq\vb{0}.
\end{gather}
以上两式表明,可以利用向量的内积来解决几何上长度和角度的问题.

由向量内积的定义还可得到:
\(\vb{a}\perp\vb{b}\)的充分必要条件是\(\VectorInnerProductDot{\vb{a}}{\vb{b}}=0\).

\begin{theorem}
%@see: 《解析几何》(丘维声) P26 定理3.1
对于任意向量\(\vb{a},\vb{b},\vb{c}\),任意实数\(\lambda\),有\begin{enumerate}
	\item {\rm\bf 对称性},
	即\begin{equation}\label{equation:解析几何.内积的对称性}
		\VectorInnerProductDot{\vb{a}}{\vb{b}} = \VectorInnerProductDot{\vb{b}}{\vb{a}}.
	\end{equation}

	\item {\rm\bf 线性性},
	即\begin{gather}
		\VectorInnerProductDot{(\lambda\vb{a})}{\vb{b}}
		= \lambda(\VectorInnerProductDot{\vb{a}}{\vb{b}}),
			\label{equation:解析几何.内积的线性性1} \\
		\VectorInnerProductDot{(\vb{a}+\vb{c})}{\vb{b}}
		= \VectorInnerProductDot{\vb{a}}{\vb{b}}+\VectorInnerProductDot{\vb{c}}{\vb{b}};
			\label{equation:解析几何.内积的线性性2} \\
	\end{gather}

	\item {\rm\bf 正定性},
	即\begin{equation}\label{equation:解析几何.内积的正定性}
		\VectorInnerProductDot{\vb{a}}{\vb{a}}\geq0,
	\end{equation}
	这里当且仅当\(\vb{a}=\vb{0}\)时有\(\VectorInnerProductDot{\vb{a}}{\vb{a}}=0\)成立.
\end{enumerate}
\end{theorem}

由内积的对称性和线性性还可以得到\begin{gather*}
	\VectorInnerProductDot{\vb{a}}{(\lambda\vb{b})}
	= \lambda(\VectorInnerProductDot{\vb{a}}{\vb{b}}), \\
	\VectorInnerProductDot{\vb{a}}{(\vb{b}+\vb{c})}
	= \VectorInnerProductDot{\vb{a}}{\vb{b}}
	+ \VectorInnerProductDot{\vb{a}}{\vb{c}}.
\end{gather*}

\begin{example}
证明:三角形的三条高线交于一点.
\begin{proof}
设\(\triangle ABC\)的三条高线\(BE,CF\)交于点\(M\),连接\(AM\).
因为\(BE \perp AC\),所以\(\VectorInnerProductDot{\vec{BM}}{\vec{AC}}=0\),即\begin{equation*}
	\VectorInnerProductDot{(\vec{AM}-\vec{AB})}{\vec{AC}}=0,
\end{equation*}
亦即\begin{equation*}
	\VectorInnerProductDot{\vec{AM}}{\vec{AC}}=\VectorInnerProductDot{\vec{AB}}{\vec{AC}}.
\end{equation*}

因为\(CF \perp AB\),所以\(\VectorInnerProductDot{\vec{CM}}{\vec{AB}}=0\),从而\begin{equation*}
	\VectorInnerProductDot{\vec{AM}}{\vec{AB}}=\VectorInnerProductDot{\vec{AC}}{\vec{AB}}.
\end{equation*}
于是有\(\VectorInnerProductDot{\vec{AM}}{\vec{AC}}=\VectorInnerProductDot{\vec{AM}}{\vec{AB}}\),
即\(\VectorInnerProductDot{\vec{AM}}{\vec{BC}}=0\),\(AM \perp BC\).
延长\(AM\)交\(BC\)于点\(D\),则\(AD\)为\(BC\)边上的高.
综上所述,\(\triangle ABC\)的三条高线交于一点\(M\).
\end{proof}
\end{example}

\begin{example}
设\(\triangle ABC\)的三边长分别为\begin{equation*}
	\abs{BC} = a, \qquad
	\abs{CA} = b, \qquad
	\abs{AB} = c.
\end{equation*}
证明\DefineConcept{海伦公式}(Heron's formula):\begin{equation}
	S_{\triangle ABC}
	= \sqrt{p (p-a) (p-b) (p-c)},
\end{equation}
其中\(p = \frac12(a+b+c)\).
\begin{proof}
由余弦定理可知\begin{equation*}
	\cos C = \frac{a^2+b^2-c^2}{2ab},
\end{equation*}
那么\begin{equation*}
	\sin C = \sqrt{1 - \cos^2 C}
	= \frac{\sqrt{4a^2b^2 - (a^2+b^2-c^2)^2}}{2ab}.
\end{equation*}
于是\(\triangle ABC\)的面积为\begin{align*}
	A &= \frac12 a b \sin C \\
	&= \frac14 \sqrt{4a^2b^2 - (a^2+b^2-c^2)^2} \\
	&= \frac14 \sqrt{(a+b+c)(b+c-a)(a+c-b)(a+b-c)} \\
	&= \sqrt{p (p-a) (p-b) (p-c)}.
	\qedhere
\end{align*}
\end{proof}
\end{example}

\subsection{利用坐标计算向量的内积}
首先取一个仿射标架\([O;\vb{d}_1,\vb{d}_2,\vb{d}_3]\),
设\(\vb{a},\vb{b}\)的坐标分别是\begin{equation*}
	(a_1,a_2,a_3)^T, \qquad
	(b_1,b_2,b_3)^T,
\end{equation*}
则\begin{align*}
	\VectorInnerProductDot{\vb{a}}{\vb{b}}
	&= \VectorInnerProductDot{(a_1 \vb{d}_1 + a_2 \vb{d}_2 + a_3 \vb{d}_3)}
			{(b_1 \vb{d}_1 + b_2 \vb{d}_2 + b_3 \vb{d}_3)} \\
	&= (a_1 b_1) \VectorInnerProductDot{\vb{d}_1}{\vb{d}_1}
		+ (a_1 b_2) \VectorInnerProductDot{\vb{d}_1}{\vb{d}_2}
		+ (a_1 b_3) \VectorInnerProductDot{\vb{d}_1}{\vb{d}_3} \\
	&\hspace{20pt}
		+ (a_2 b_1) \VectorInnerProductDot{\vb{d}_2}{\vb{d}_1}
		+ (a_2 b_2) \VectorInnerProductDot{\vb{d}_2}{\vb{d}_2}
		+ (a_2 b_3) \VectorInnerProductDot{\vb{d}_2}{\vb{d}_3} \\
	&\hspace{20pt}
		+ (a_3 b_1) \VectorInnerProductDot{\vb{d}_3}{\vb{d}_1}
		+ (a_3 b_2) \VectorInnerProductDot{\vb{d}_3}{\vb{d}_2}
		+ (a_3 b_3) \VectorInnerProductDot{\vb{d}_3}{\vb{d}_3}.
\end{align*}
可见,只要知道基向量\(\vb{d}_1,\vb{d}_2,\vb{d}_3\)之间的内积
(看起来有9个数,实际上由于内积的对称性,至多有6个不同的数)
就可以求出任意两个向量的内积.
这9个数\begin{equation*}
	\VectorInnerProductDot{\vb{d}_i}{\vb{d}_j}
	\quad(i,j=1,2,3)
\end{equation*}
称为“仿射标架\([O;\vb{d}_1,\vb{d}_2,\vb{d}_3]\)的\DefineConcept{度量参数}”.

如果我们取定的是直角标架\([O;\vb{e}_1,\vb{e}_2,\vb{e}_3]\),
则\begin{equation*}
	\VectorInnerProductDot{\vb{e}_i}{\vb{e}_j}
	= \left\{ \begin{array}{ll}
		1, & i = j, \\
		0, & i \neq j.
	\end{array} \right.
\end{equation*}
于是我们有下述定理.
\begin{theorem}\label{theorem:解析几何.直角坐标系下向量内积的坐标计算式}
%@see: 《解析几何》(丘维声) P28 定理3.2
在直角坐标系中,两个向量的内积等于它们的对应坐标的乘积之和,即
\begin{equation}
	\VectorInnerProductDot{\vb{a}}{\vb{b}}
	= a_1 b_1 y + a_2 b_2 + a_3 b_3,
\end{equation}
其中\(\vb{a}=(a_1,a_2,a_3)^T\),\(\vb{b}=(b_1,b_2,b_3)^T\).
\end{theorem}

由\cref{theorem:解析几何.直角坐标系下向量内积的坐标计算式} 可知,
在直角坐标系下,向量\(\vb{a}=(a_1,a_2,a_3)^T\)的长度为\begin{equation}
	\VectorLengthA{\vb{a}} = \sqrt{a_1^2+a_2^2+a_3^2};
\end{equation}
而两点\(A(x_1,y_1,z_1)^T\)和\(B(x_2,y_2,z_2)^T\)之间的距离为\begin{equation}
	\abs{\vec{AB}} = \sqrt{(x_2-x_1)^2+(y_2-y_1)^2+(z_2-z_1)^2}.
\end{equation}

\subsection{方向角与方向余弦}
在直角坐标系中,还可以用向量\(\vb{a}\)与基向量的内积来计算\(\vb{a}\)的坐标.
设\(\vb{a}\)在直角标架\([O;\vb{e}_1,\vb{e}_2,\vb{e}_3]\)中的坐标为\((a_1,a_2,a_3)^T\),
则有\begin{equation*}
	\vb{a}=a_1 \vb{e}_1 + a_2 \vb{e}_2 + a_3 \vb{e}_3.
\end{equation*}
上式两边用\(\vb{e}_1\)作内积,得\begin{equation*}
	\VectorInnerProductDot{\vb{a}}{\vb{e}_1} = a_1.
\end{equation*}
同理可得\begin{equation*}
	\VectorInnerProductDot{\vb{a}}{\vb{e}_2} = a_2, \qquad
	\VectorInnerProductDot{\vb{a}}{\vb{e}_3} = a_3.
\end{equation*}
这说明向量\(\vb{a}\)与基向量\(\vb{e}_j\)的内积就是
\(\vb{a}\)的第\(j\ (j=1,2,3)\)个直角坐标.

特别地,单位向量\(\vb{a}^0\)的直角坐标为\begin{equation*}
	\left( \cos\angle(\vb{a},\vb{e}_1),
	\cos\angle(\vb{a},\vb{e}_2),
	\cos\angle(\vb{a},\vb{e}_3) \right).
\end{equation*}

我们把一个向量\(\vb{a}\)与直角标架的基向量\(\vb{e}_1,\vb{e}_2,\vb{e}_3\)所成的角\begin{equation*}
	\alpha=\angle(\vb{a},\vb{e}_1), \qquad
	\beta=\angle(\vb{a},\vb{e}_2), \qquad
	\gamma=\angle(\vb{a},\vb{e}_3),
\end{equation*}称为“方向\(\vb{a}\)的\DefineConcept{方向角}”;
把方向角的余弦\(\cos\alpha,\cos\beta,\cos\gamma\)称为
“方向\(\vb{a}\)的\DefineConcept{方向余弦}”.
由上可知,\begin{equation*}
	\cos\alpha
	= \frac{\VectorInnerProductDot{\vb{a}}{\vb{e}_1}}{\VectorLengthA{\vb{a}}\VectorLengthA{\vb{e}_1}}
	= \frac{a_1}{\VectorLengthA{\vb{a}}},
	\qquad
	\cos\beta
	= \frac{\VectorInnerProductDot{\vb{a}}{\vb{e}_2}}{\VectorLengthA{\vb{a}}\VectorLengthA{\vb{e}_2}}
	= \frac{a_2}{\VectorLengthA{\vb{a}}},
	\qquad
	\cos\gamma
	= \frac{\VectorInnerProductDot{\vb{a}}{\vb{e}_3}}{\VectorLengthA{\vb{a}}\VectorLengthA{\vb{e}_3}}
	= \frac{a_3}{\VectorLengthA{\vb{a}}},
\end{equation*}
也就是说\(\vb{a}\)的方向余弦就是与\(\vb{a}\)同向的单位向量\(\vb{a}^0\)的直角坐标,即\begin{equation*}
	(\cos\alpha,\cos\beta,\cos\gamma)^T
	= \frac{1}{\VectorLengthA{\vb{a}}} (a_1,a_2,a_3)^T
	= \frac{1}{\VectorLengthA{\vb{a}}} \vb{a},
\end{equation*}
从而有\begin{equation*}
	\cos^2\alpha+\cos^2\beta+\cos^2\gamma=1.
\end{equation*}
