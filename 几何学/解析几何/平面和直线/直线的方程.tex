\section{直线的方程,直线与平面的相关位置}

\subsection{直线的方程}
一个点和一个非零向量可以决定一条直线.

给定一条直线\(l\)和一个向量\(\vb{a}\),
如果直线\(l\)与用来表示向量\(\vb{a}\)的有向线段所在的直线平行,
则称“向量\(\vb{a}\)是直线\(l\)的\DefineConcept{方向向量}”.

取一个仿射标架\([O;\vb{d}_1,\vb{d}_2,\vb{d}_3]\).
已知一点\(M_0(x_0,y_0,z_0)\),
非零向量\(\vb{\nu}=(X,Y,Z)\).
现在来求经过点\(M_0\),且方向向量为\(\vb{\nu}\)的直线\(l\)的方程.

点\(M(x,y,z)\)在直线\(l\)上的充分必要条件是\(\vec{M_0 M}\)与\(\vb{\nu}\)共线,
即\(\vec{M_0 M} = t \vb{\nu}\),其中\(t\)是实数.
设\(M_0,M\)的定位向量分别用\(\vb{r}_0,\vb{r}\)表示,
则由上式得
\begin{equation}\label{equation:解析几何.直线的向量式参数方程}
	\vb{r} = \vb{r}_0 + t \vb{\nu}.
\end{equation}
\cref{equation:解析几何.直线的向量式参数方程} 称为“直线的\DefineConcept{向量式参数方程}”,
其中\(t\)是实参数.
我们可以这样考虑参数\(t\)的几何意义:
它表示点\(M\)在直线\(l\)上的仿射标架\([M_0;\vb{\nu}]\)中的坐标.

将\cref{equation:解析几何.直线的向量式参数方程} 用坐标写出,可得
\begin{equation}\label{equation:解析几何.直线的参数方程}
	\left\{ \begin{array}{l}
		x = x_0 + t X, \\
		y = y_0 + t Y, \\
		z = z_0 + t Z.
	\end{array} \right.
\end{equation}
\cref{equation:解析几何.直线的参数方程} 称为“直线的\DefineConcept{参数方程}”,其中\(t\)是实参数.

下面消去参数\(t\).
因为\(\vb{\nu}\neq\vb{0}\),
必有\(X,Y,Z\)不同时为零.
若\(X\neq0\),则\begin{equation*}
	t = \frac{x - x_0}{X};
\end{equation*}
否则\(x - x_0 = 0\).
若\(Y\neq0\),则\begin{equation*}
	t = \frac{y - y_0}{Y};
\end{equation*}
否则\(y - y_0 = 0\).
若\(Z\neq0\),则\begin{equation*}
	t = \frac{z - z_0}{Z};
\end{equation*}
否则\(z - z_0 = 0\).
于是,
若\(X,Y\neq0\),\(Z=0\),则有
\begin{equation}
	\left\{ \begin{array}{l}
		\frac{x - x_0}{X}
		= \frac{y - y_0}{Y}, \\
		z - z_0 = 0;
	\end{array} \right.
\end{equation}
若\(Y,Z\neq0\),\(X=0\),则有
\begin{equation}
	\left\{ \begin{array}{l}
		\frac{y - y_0}{Y}
		= \frac{z - z_0}{Z}, \\
		x - x_0 = 0;
	\end{array} \right.
\end{equation}
若\(Z,X\neq0\),\(Y=0\),则有
\begin{equation}
	\left\{ \begin{array}{l}
		\frac{x - x_0}{X}
		= \frac{z - z_0}{Z}, \\
		y - y_0 = 0;
	\end{array} \right.
\end{equation}
若\(X,Y,Z\neq0\),则有
\begin{equation}\label{equation:解析几何.直线的点向式方程}
	\frac{x - x_0}{X}
	= \frac{y - y_0}{Y}
	= \frac{z - z_0}{Z}.
\end{equation}
\cref{equation:解析几何.直线的点向式方程}
称为“直线的\DefineConcept{点向式方程}”,
或“直线的\DefineConcept{对称式方程}”,
或“直线的\DefineConcept{标准方程}”.
它实际上是联立两个方程而得的方程组.

如果已知直线\(l\)上两点\(M_1(x_1,y_1,z_1)\)和\(M_2(x_2,y_2,z_2)\),
则\(\vec{M_1 M_2}\)是\(l\)的一个方向向量,从而\(l\)的方程为
\begin{equation}\label{equation:解析几何.直线的两点式方程}
	\frac{x - x_1}{x_2 - x_1}
	= \frac{y - y_1}{y_2 - y_1}
	= \frac{z - z_1}{z_2 - z_1}.
\end{equation}
\cref{equation:解析几何.直线的两点式方程}
称为“直线\(l\)的\DefineConcept{两点式方程}”.

任意一条直线可以看成某两个相交平面的交线.
设直线\(l\)是相交平面\(\alpha_1\)和\(\alpha_2\)的交线,它们的方程分别为\begin{equation*}
	A_1 x + B_1 y + C_1 z + D_1 = 0
	\quad\text{和}\quad
	A_2 x + B_2 y + C_2 z + D_2 = 0.
\end{equation*}
这两个方程的一次项系数不成比例
(如果这两个方程的一次项系数成比例,则这两个平面要么平行要么重合,总之它们不是相交平面),
则方程组\begin{equation}
	\left\{ \begin{array}{l}
		A_1 x + B_1 y + C_1 z + D_1 = 0, \\
		A_2 x + B_2 y + C_2 z + D_2 = 0
	\end{array} \right.
\end{equation}
就是直线\(l\)的方程,称为“直线\(l\)的\DefineConcept{一般方程}”.

由于直角坐标系是特殊的仿射坐标系,
所以上述一切结论在直角坐标系中都成立.
利用右手直角坐标系的特殊性,
由直线的一般方程写出它的标准方程时,
求直线\(l\)的方向向量\(\vb{\nu}\)可以更加直观:
因为\(\vb{\nu}\perp\vb{n}_i\),
其中\(\vb{n}_i\)是\(\alpha_i\ (i=1,2)\)的法向量,
所以可以取\(\vb{\nu}=\VectorOuterProduct{\vb{n}_1}{\vb{n}_2}\).
由于\(\vb{n}_i = (A_i,B_i,C_i)\),
所以\begin{equation*}
	\vb{\nu}
	= \begin{vmatrix}
		\vb{i} & \vb{j} & \vb{k} \\
		A_1 & B_1 & C_1 \\
		A_2 & B_2 & C_2
	\end{vmatrix}.
\end{equation*}

特别地,平面直角坐标系\(Oxy\)上的直线方程\begin{equation*}
	A x + B y + D = 0
\end{equation*}
相当于空间直角坐标系\(Oxyz\)中的直线方程\begin{equation*}
	\left\{ \begin{array}{l}
		A x + B y + 0 z + D = 0, \\
		0 x + 0 y + 1 z + 0 = 0,
	\end{array} \right.
\end{equation*}
它的方向向量为\begin{equation*}
	\vb\nu
	= \begin{vmatrix}
		\vb{i} & \vb{j} & \vb{k} \\
		A & B & 0 \\
		0 & 0 & 1
	\end{vmatrix}
	= \begin{bmatrix}
		B \\ -A \\ 0
	\end{bmatrix}.
\end{equation*}

\subsection{两条直线的相关位置}
%@see: 《解析几何》(丘维声) P63
在仿射坐标系中,设直线\(l_i\)经过点\(M_i(x_i,y_i,z_i)\),
方向向量为\(\vb{\nu}=(X_i,Y_i,Z_i)\),\(i=1,2\).

\(l_1\)与\(l_2\)平行的充分必要条件是
\(\vb{\nu}_1\)与\(\vb{\nu}_2\)共线,
但\(\vec{M_1 M_2}\)与\(\vb{\nu}_1\)不共线,
即存在实数\(\lambda\),使得\(\vb{\nu}_2 = \lambda \vb{\nu}_1\);
但对一切实数\(\mu\),都有\(\vec{M_1 M_2} \neq \mu \vb{\nu}_1\).

\(l_1\)与\(l_2\)重合的充分必要条件是
\(\vb{\nu}_1,\vb{\nu}_2,\vec{M_1 M_2}\)三个向量共线,
即存在实数\(\lambda,\mu\),使得\begin{equation*}
	\vb{\nu}_2 = \lambda \vb{\nu}_1,
	\qquad
	\vec{M_1 M_2} = \mu \vb{\nu}_1.
\end{equation*}

\(l_1\)与\(l_2\)相交的充分必要条件是
\(\vb{\nu}_1,\vb{\nu}_2,\vec{M_1 M_2}\)三个向量共面,
且\(\vb{\nu}_1\)与\(\vb{\nu}_2\)不共线,即\begin{equation*}
	\begin{vmatrix}
		x_2 - x_1 & X_1 & X_2 \\
		y_2 - y_1 & Y_1 & Y_2 \\
		z_2 - z_1 & Z_1 & Z_2
	\end{vmatrix} = 0,
\end{equation*}
且对一切实数\(\lambda\),都有\(\vb{\nu}_2 \neq \lambda \vb{\nu_1}\).

\(l_1\)与\(l_2\)异面的充分必要条件是
\(\vb{\nu}_1,\vb{\nu}_2,\vec{M_1 M_2}\)三个向量不共面,即\begin{equation*}
	\begin{vmatrix}
		x_2 - x_1 & X_1 & X_2 \\
		y_2 - y_1 & Y_1 & Y_2 \\
		z_2 - z_1 & Z_1 & Z_2
	\end{vmatrix} \neq 0.
\end{equation*}

\begin{example}
设直线\(l_1: y=k_1 x+b_2\)与\(l_2: y=k_2 x+b_2\)垂直.
证明:\(k_1 \cdot k_2 = -1\).
\begin{proof}
由题意有,直线\(l_1,l_2\)的方向向量分别为\begin{equation*}
	\vb{\nu}_1 = \lambda (1,k_1), \qquad
	\vb{\nu}_2 = \mu (1,k_2),
\end{equation*}
其中\(\lambda,\mu\)取任意实数.
因为直线\(l_1\)与\(l_2\)垂直,所以它们的方向向量垂直,那么\begin{equation*}
	\VectorInnerProductDot{\vb{\nu}_1}{\vb{\nu}_2}
	= (\lambda \cdot 1) \cdot (\mu \cdot 1)
	+ (\lambda \cdot k_1) \cdot (\mu \cdot k_2) = 0,
\end{equation*}
化简得\(k_1 \cdot k_2 = -1\).
\end{proof}
\end{example}

\subsection{直线和平面的相关位置}
在仿射坐标系中,设直线\(l\)经过点\(M_0(x_0,y_0,z_0)\),方向向量为\(\vb{\nu}=(X,Y,Z)\),
平面\(\alpha\)的方程为\(A x + B y + C z + D = 0\).

\(l\)与\(\alpha\)平行的充分必要条件是\(\vb{\nu}\)与\(\alpha\)平行,且\(M_0\)不在\(\alpha\)上,
即\begin{equation*}
	AX+BY+CZ=0
	\quad\land\quad
	Ax_0+By_0+Cz_0+D\neq0.
\end{equation*}

\(l\)在平面\(\alpha\)上的充分必要条件是\(\vb{\nu}\)与\(\alpha\)平行,且\(M_0\)在\(\alpha\)上,
即\begin{equation*}
	AX+BY+CZ=0
	\quad\land\quad
	Ax_0+By_0+Cz_0+D=0.
\end{equation*}

\(l\)与\(\alpha\)相交的充分必要条件是\(\vb{\nu}\)与\(\alpha\)不平行,即\begin{equation*}
	AX+BY+CZ\neq0.
\end{equation*}
此时,将\(l\)的参数方程\begin{equation*}
	\left\{ \begin{array}{l}
		x = x_0 + tX, \\
		y = y_0 + tY, \\
		z = z_0 + tZ
	\end{array} \right.
\end{equation*}
代入\(\alpha\)的方程中,得\begin{equation*}
	A(x_0+tX) + B(y_0+tY) + C(z_0+tZ) + D = 0,
\end{equation*}
解得\begin{equation}
	t = - \frac{Ax_0+By_0+Cz_0+D}{AX+BY+CZ}.
\end{equation}
将\(t\)代回到\(l\)的参数方程中,即可求出\(l\)与\(\alpha\)的交点的坐标.
