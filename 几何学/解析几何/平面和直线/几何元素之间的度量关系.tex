\section{点、直线和平面之间的度量关系}

\subsection{点到直线的距离}
如\cref{figure:解析几何.点到直线的距离},
设直线\(l\)经过点\(M_0\),方向向量为\(\vb{\nu}\),
则\(l\)外一点\(M\)到直线\(l\)的距离\(d\),
是以\(\vec{M_0 M},\vb{\nu}\)为邻边的平行四边形的底边\(\vb{\nu}\)上的高,
因此有\begin{equation}\label{equation:解析几何.点到直线的距离}
	d = \frac{\VectorLengthA{\VectorOuterProduct{\vec{M_0 M}}{\vb{\nu}}}}{\VectorLengthA{\vb{\nu}}}.
\end{equation}

\begin{figure}[htb]
	\centering
	\begin{tikzpicture}
		\draw(-1,0)--(3,0)node[right]{\(l\)};
		\begin{scope}[>=Stealth,->]
			\draw(0,0)node[below]{\(M_0\)}--(1,2)node[above]{\(M\)};
			\draw(0,0)--(2,0)node[below]{\(\vb{\nu}\)};
		\end{scope}
		\draw(2,0)--++(1,2)--(1,2);
		\draw[dashed](1,2)--(1,0)node[midway,right]{\(d\)};
	\end{tikzpicture}
	\caption{}
	\label{figure:解析几何.点到直线的距离}
\end{figure}

\subsection{两条直线的距离}
两条直线上的点之间的最短距离,称为这两条直线的距离.

如果\(l_1 \parallel l_2\),
则\(l_1\)上任意一点到\(l_2\)的距离就是\(l_1\)与\(l_2\)的距离;
如果\(l_1\)与\(l_2\)相交或重合,
则\(l_1\)与\(l_2\)的距离为零.

\begin{figure}[htb]
	\centering
	\begin{tikzpicture}
		\draw(0,0)node[below]{\(P_1\)}--(0,2)node[above]{\(P_2\)}node[midway,left]{\(l\)}
			node[midway,right]{\(\VectorOuterProduct{\vb{\nu}_1}{\vb{\nu}_2}\)};
		\draw[name path=a](0,0)--(5,-1)node[above left]{\(\alpha_1\)}node[below right]{\(l_1\)}
			--++(0,2)--(0,2);
		\path[name path=b](4,1)--++(0,2);
		\draw[dashed][name intersections={of=a and b}]
			(0,0)--(4,1)--(intersection-1);
		\draw(intersection-1)
			--(4,3)node[above right]{\(l_2\)}node[below left]{\(\alpha_2\)}
			--(0,2);
		\begin{scope}[>=Stealth,->]
		\draw(0,0)--(0,1);
		\draw(1,-.2)node[below]{\(M_1\)}--(2,-.4)node[below]{\(\vb{\nu}_1\)};
		\draw(1,2.25)node[above]{\(M_2\)}--(2,2.5)node[above]{\(\vb{\nu}_2\)};
		\end{scope}
		\fill(1,-.2)circle(1pt) (1,2.25)circle(1pt);
	\end{tikzpicture}
	\caption{}
	\label{figure:解析几何.异面直线的公垂线段}
\end{figure}

如\cref{figure:解析几何.异面直线的公垂线段},
设\(l_1\)与\(l_2\)异面,且\(\vb{\nu}_1\)与\(\vb{\nu}_2\)不共线,\(M_1 \neq M_2\).
又设\(l_i\)经过点\(M_i\),方向向量为\(\vb{\nu}_i\ (i=1,2)\).
我们称同时与两条异面直线\(l_1,l_2\)垂直相交的直线\(l\)为\(l_1,l_2\)的\DefineConcept{公垂线},
两个垂足间的线段\(P_1 P_2\)称为\DefineConcept{公垂线段}.

我们首先证明任意两条异面直线的公垂线存在且唯一.
\begin{enumerate}
	\item 因为\(\vb{\nu}_1\)与\(\vb{\nu}_2\)不共线,
	所以\(\vb{\nu}_1\)与\(\VectorOuterProduct{\vb{\nu}_1}{\vb{\nu}_2}\)不共线.
	于是点\(M_1\)与两向量\(\vb{\nu}_1\)、\(\VectorOuterProduct{\vb{\nu}_1}{\vb{\nu}_2}\)决定一个平面\(\alpha_1\).
	同理,\(M_2,\vb{\nu}_2,\VectorOuterProduct{\vb{\nu}_1}{\vb{\nu}_2}\)决定一个平面\(\alpha_2\).
	又因为\(\VectorOuterProduct{\vb{\nu}_1}{(\VectorOuterProduct{\vb{\nu}_1}{\vb{\nu}_2})}\)
	与\(\VectorOuterProduct{\vb{\nu}_2}{(\VectorOuterProduct{\vb{\nu}_1}{\vb{\nu}_2})}\)不共线,
	而这两个向量分别是\(\alpha_1\)和\(\alpha_2\)的法向量,
	于是\(\alpha_1\)与\(\alpha_2\)必相交,记交线为\(l\),其方向向量为\begin{equation*}
		\VectorOuterProduct
		{[\VectorOuterProduct{\vb{\nu}_1}{(\VectorOuterProduct{\vb{\nu}_1}{\vb{\nu}_2})}]}
		{[\VectorOuterProduct{\vb{\nu}_2}{(\VectorOuterProduct{\vb{\nu}_1}{\vb{\nu}_2})}]}
		= \abs{\VectorOuterProduct{\vb{\nu}_1}{\vb{\nu}_2}}^2
			(\VectorOuterProduct{\vb{\nu}_1}{\vb{\nu}_2});
	\end{equation*}
	也就是说,\(\VectorOuterProduct{\vb{\nu}_1}{\vb{\nu}_2}\)是\(l\)的方向向量.
	由于\begin{equation*}
		\VectorOuterProduct{\vb{\nu}_1}{\vb{\nu}_2} \perp \vb{\nu}_i,
		\quad i=1,2,
	\end{equation*}
	所以\(l \perp l_i\ (i=1,2)\).
	因为\(l\)与\(l_i\)都在\(\alpha_i\)内,
	所以\(l\)与\(l_i\)都相交(\(i=1,2\));
	这就是说\(\alpha_1\)与\(\alpha_2\)的交线\(l\)就是\(l_1\)与\(l_2\)的公垂线.

	\item 假设另有一条不同于\(l\)的直线\(l'\)也是\(l_1\)与\(l_2\)的公垂线,
	则\(l'\)的方向向量垂直于\(\vb{\nu}_i\ (i=1,2)\),
	从而\(\VectorOuterProduct{\vb{\nu}_1}{\vb{\nu}_2}\)就是\(l'\)的一个方向向量.
	因为\(l'\)在由\(l_i\)和\(\VectorOuterProduct{\vb{\nu}_1}{\vb{\nu}_2}\)决定的平面\(\alpha_i\ (i=1,2)\)内,
	所以\(l'\)是\(\alpha_1\)与\(\alpha_2\)的交线,即\(l'\)与\(l\)重合,矛盾!
\end{enumerate}

现在我们来证明两条异面直线\(l_1,l_2\)的公垂线段的长就是\(l_1\)与\(l_2\)的距离.
如\cref{figure:解析几何.异面直线的距离就是其公垂线段的长},
在\(l_i\)上任取一点\(Q_i\ (i=1,2)\).
作出由\(M_1,\vb{\nu}_1,\vb{\nu}_2\)决定的平面\(\alpha\),
于是公垂线\(P_1 P_2 \perp \alpha\).
由\(Q_2\)作\(\alpha\)的垂线,设垂足为\(N\).
因为\(l_2 \parallel \alpha\),
所以\(\abs{P_1 P_2} = \abs{Q_2 N}\);
又因为\(\abs{Q_2 Q_1} \geq \abs{Q_2 N}\),
所以\(\abs{P_1 P_2}\)是\(l_1\)与\(l_2\)上的点之间的最短距离,即\(l_1\)与\(l_2\)的距离.

由上可知,我们现在只要计算出公垂线段\(P_1 P_2\)的长,就得到了\(l_1\)与\(l_2\)的距离.
我们已经知道,\(\vec{P_1 P_2}\)与\(\VectorOuterProduct{\vb{\nu}_1}{\vb{\nu}_2}\)共线,
若记\(\vb{e} = (\VectorOuterProduct{\vb{\nu}_1}{\vb{\nu}_2})^0\),则\begin{align*}
	d &= \abs{\vec{P_1 P_2}}
	= \abs{\VectorInnerProductDot{\vec{P_1 P_2}}{\vb{e}}} \\
	&= \abs{
			\VectorInnerProductDot
			{(\vec{P_1 M_1} + \vec{M_1 M_2} + \vec{M_2 P_2})}
			{\vb{e}}
		} \\
	&= \abs{\VectorInnerProductDot{\vec{M_1 M_2}}{\vb{e}}}
	= \abs{
			\VectorInnerProductDot
			{\vec{M_1 M_2}}
			{\frac{\VectorOuterProduct{\vb{\nu}_1}{\vb{\nu}_2}}{\abs{\VectorOuterProduct{\vb{\nu}_1}{\vb{\nu}_2}}}}
		} \\
	&= \frac{
			\abs{
				\VectorMixedProductDC{\vec{M_1 M_2}}{\vb{\nu}_1}{\vb{\nu}_2}
			}
		}{
			\abs{
				\VectorOuterProduct{\vb{\nu}_1}{\vb{\nu}_2}
			}
		}.
\end{align*}
我们可以从上述计算过程了解到其中隐含的几何意义:
两条异面直线\(l_1\)与\(l_2\)的距离\(d\)等于以\(\vec{M_1 M_2},\vb{\nu}_1,\vb{\nu}_2\)为棱的平行六面体的体积除以以\(\vb{\nu}_1,\vb{\nu}_2\)为邻边的平行四边形的面积.

综上所述,\(l_1\)与\(l_2\)的距离为
\begin{equation}
	d = \frac{
			\abs{
				\VectorMixedProductDC{\vec{M_1 M_2}}{\vb{\nu}_1}{\vb{\nu}_2}
			}
		}{
			\abs{
				\VectorOuterProduct{\vb{\nu}_1}{\vb{\nu}_2}
			}
		}.
\end{equation}

\begin{figure}[htb]
	\centering
	\begin{tikzpicture}
		\draw(0,0)--(5,0)node[above left]{\(\alpha\)}--++(1,2)--++(-5,0)--(0,0);
		\filldraw(1,1)--(1.5,1)circle(1pt)node[below]{\(M_1\)}--(2,1)coordinate(P1)node[below]{\(P_1\)}
			--(3.5,1)coordinate(Q1)circle(1pt)node[below]{\(Q_1\)}--(4.5,1)node[right]{\(l_1\)};
		\draw(P1)--++(0,2)coordinate(P2)node[above]{\(P_2\)};
		\filldraw(1,2.6)--(1.5,2.8)circle(1pt)node[above]{\(M_2\)}--(P2)--(3,3.4)coordinate(Q2)node[above]{\(Q_2\)}
			--(4.5,4)node[right]{\(l_2\)};
		\draw(Q1)--(Q2);
		\filldraw(Q2)--++(0,-2)circle(1pt)coordinate(N)node[left]{\(N\)};
		\draw[dashed](N)--++(.3,0)coordinate(N1);
		\draw pic[draw=gray,-,angle radius=0.2cm]{right angle=Q1--P1--P2}
			pic[draw=gray,-,angle radius=0.2cm]{right angle=P1--P2--Q2}
			pic[draw=gray,-,angle radius=0.2cm]{right angle=N1--N--Q2};
		\begin{scope}[>=Stealth,->]
			\draw(3.9,1)--(4,1)node[above]{\(\vb{\nu}_1\)};
			\draw(3.9,3.76)--(4,3.8)node[below]{\(\vb{\nu}_2\)};
		\end{scope}
	\end{tikzpicture}
	\caption{}
	\label{figure:解析几何.异面直线的距离就是其公垂线段的长}
\end{figure}

%@see: 《高等代数与解析几何(上册)》(盛为民、李方) P146 定理5.4.3
设两异面直线\(l_1\)与\(l_2\)的对称式方程为\begin{gather*}
%@see: 《高等代数与解析几何(上册)》(盛为民、李方) P146 (5.4.6)
%@see: 《高等代数与解析几何(上册)》(盛为民、李方) P146 (5.4.7)
	l_1: \frac{x - x_1}{X_1}
	= \frac{y - y_1}{Y_1}
	= \frac{z - z_1}{Z_1}, \\
	l_2: \frac{x - x_2}{X_2}
	= \frac{y - y_2}{Y_2}
	= \frac{z - z_2}{Z_2},
\end{gather*}
则两直线的距离为\begin{equation}
%@see: 《高等代数与解析几何(上册)》(盛为民、李方) P147 (5.4.8)
	d = \frac{
		\abs{
			\begin{vmatrix}
				x_2 - x_1 & y_2 - y_1 & z_2 - z_1 \\
				X_1 & Y_1 & Z_1 \\
				X_2 & Y_2 & Z_2
			\end{vmatrix}
		}
	}{
		\sqrt{
			\begin{vmatrix}
				Y_1 & Z_1 \\
				Y_2 & Z_2
			\end{vmatrix}^2
			+ \begin{vmatrix}
				Z_1 & X_1 \\
				Z_2 & X_2
			\end{vmatrix}^2
			+ \begin{vmatrix}
				X_1 & Y_1 \\
				X_2 & Y_2
			\end{vmatrix}^2
		}
	}.
\end{equation}

\subsection{两条直线所成的角}
我们规定,两条直线所成的角是指它们的方向向量夹角中不大于\(\frac{\pi}{2}\)的那个角.

设空间直线\(l_1\)和\(l_2\)的方向向量分别为\(\vb{\nu}_1\)和\(\vb{\nu}_2\),
那么\(l_1\)和\(l_2\)的夹角等于它们的方向向量的夹角,满足\begin{equation*}
	\cos\angle(l_1,l_2)
	= \cos\angle(\vb{\nu}_1,\vb{\nu}_2)
	= \frac{\abs{\VectorInnerProductDot{\vb{\nu}_1}{\vb{\nu}_2}}}{\VectorLengthA{\vb{\nu}_1}\VectorLengthA{\vb{\nu}_2}}.
\end{equation*}

\subsection{直线和平面所成的角}
当直线\(l\)不垂直于平面\(\alpha\)时,
规定它与\(\alpha\)所成的角为\(l\)与它在\(\alpha\)上的射影所成的角;
当它垂直于\(\alpha\)时,
规定它与\(\alpha\)所成的角为\(\frac{\pi}{2}\).

如\cref{figure:解析几何.直线与平面的夹角} 所示,
设\(\vb{n}\)是平面\(\alpha\)的一个法向量,\(\vb{\nu}\)为\(l\)的一个方向向量,
则\(l\)与\(\alpha\)所成的角\(\theta\)满足\begin{equation*}
	\theta = \left\{ \def\arraystretch{2} \begin{array}{ll}
		\frac{\pi}{2} - \angle(\vb{\nu},\vb{n}),
			& 0 \leq \angle(\vb{\nu},\vb{n}) < \frac{\pi}{2}, \\
		\angle(\vb{\nu},\vb{n}) - \frac{\pi}{2},
			& \frac{\pi}{2} \leq \angle(\vb{\nu},\vb{n}) < \pi.
	\end{array} \right.
\end{equation*}
或\begin{equation*}
	\sin\theta = \abs{\cos\angle(\vb{\nu},\vb{n})}.
\end{equation*}

\begin{figure}[hb]
	\centering
	\begin{tikzpicture}
		\path[name path=l](-1,-1)--(0,0);
		\draw[name path=a](-2,-1)--++(4,0)node[above left]{\(\alpha\)}--++(1,2)--++(-4,0)--(-2,-1);
		\draw[name intersections={of=a and l}]
			(-1.3,-1.3)--(intersection-1);
		\draw[dashed](intersection-1)--(0,0);
		\filldraw(0,0)coordinate(O)circle(1pt)--(1.8,1.8)coordinate(L)node[right]{\(l\)};
		\draw(0,0)--++(1.4,0)coordinate(T)node[right]{\(l'\)};
		\begin{scope}[>=Stealth,->]
			\draw(0,0)--++(0,2)node[left]{\(\vb{n}\)};
			\draw(0,0)--++(1.5,1.5)node[left]{\(\vb{\nu}\)};
		\end{scope}
		\draw pic["\(\theta\)",draw=orange,-,angle eccentricity=1.5,angle radius=4mm]{angle=T--O--L};
	\end{tikzpicture}
	\caption{}
	\label{figure:解析几何.直线与平面的夹角}
\end{figure}

\begin{example}
%@see: 《解析几何》(丘维声) P75 习题2.4 7.
设两条异面直线的方程分别为\begin{equation*}
	l_1: \frac{x-x_1}{X_1} = \frac{y-y_1}{Y_1} = \frac{z-z_1}{Z_1},
\end{equation*}\begin{equation*}
	l_2: \frac{x-x_2}{X_2} = \frac{y-y_2}{Y_2} = \frac{z-z_2}{Z_2},
\end{equation*}
求与\(l_1,l_2\)等距离的平面的方程.
\begin{solution}
在\(M_1,\vb{\nu}_1,\vb{\nu}_2\)决定的平面上任取一点\(Q_1\),
在\(M_2,\vb{\nu}_1,\vb{\nu}_2\)决定的平面上任取一点\(Q_2\),
可以证明线段\(Q_1 Q_2\)的中点\(Q\)在所求的与\(l_1,l_2\)等距离的平面\(\alpha\)上.

由题意有,\(l_1,l_2\)分别过点\(M_1(x_1,y_1,z_2)\)和\(M_2(x_2,y_2,z_2)\),
它们的方向向量分别为\begin{equation*}
	\vb{\nu}_1 = (X_1,Y_1,Z_1), \qquad
	\vb{\nu}_2 = (X_2,Y_2,Z_2);
\end{equation*}
那么由\(M_1,M_2\)连线段的中点\(M\left(
	\frac{x_1+x_2}{2},
	\frac{y_1+y_2}{2},
	\frac{z_1+z_2}{2}
\right)\)
和\(\vb{\nu}_1,\vb{\nu}_2\)决定的平面的方程为\begin{equation*}
	\alpha: \begin{vmatrix}
		x - (x_1+x_2)/2 & y - (y_1+y_2)/2 & z - (z_1+z_2)/2 \\
		X_1 & Y_1 & Z_1 \\
		X_2 & Y_2 & Z_2
	\end{vmatrix} = 0.
\end{equation*}
\end{solution}
\end{example}

\begin{example}
%@see: 《解析几何》(丘维声) P75 习题2.4 9.
在给定的直角坐标系中,点\(P\)不在坐标平面上,
从点\(P\)到\(Ozx\)平面、\(Oxy\)平面分别作垂线,垂足为\(M\)和\(N\).
设直线\(OP\)与平面\(OMN,Oxy,Oyz,Ozx\)所成的角分别为\(\theta,\alpha,\beta,\gamma\).
证明:\begin{equation*}
	\csc^2\theta = \csc^2\alpha+\csc^2\beta+\csc^2\gamma.
\end{equation*}
%TODO
\end{example}
