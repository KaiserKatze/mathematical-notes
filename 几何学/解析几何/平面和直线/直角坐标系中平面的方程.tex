\section{直角坐标系中平面的方程,点到平面的距离}

\subsection{直角坐标系中平面方程的系数的几何意义}
确定一个平面的条件还可以是:
一个点和一个与这个平面垂直的非零向量.
与平面垂直的非零向量称为这个平面的\DefineConcept{法向量}.

取一个直角标架\([O;\vb{e}_1,\vb{e}_2,\vb{e}_3]\).
我们来求经过点\(M_0(x_0,y_0,z_0)\),
且法向量为\(\vb{n}=(a,b,c)\)的平面\(\alpha\)的方程.
点\(M(x,y,z)\)在平面\(\alpha\)上的充分必要条件是\begin{equation*}
	\vec{M_0 M} \perp \vb{n},
\end{equation*}
于是有\(\VectorInnerProductDot{\vec{M_0 M}}{\vb{n}} = 0\),
即\begin{equation}\label{equation:解析几何.平面的点法式方程}
	a(x-x_0) + b(y-y_0) + c(z-z_0) = 0,
\end{equation}
或\begin{equation*}
	a x + b y + c z + h = 0,
\end{equation*}其中\(h=-(a x_0 + b y_0 + c z_0)\).
上式就是所求平面\(\alpha\)的方程.
我们称\cref{equation:解析几何.平面的点法式方程} 为“平面的\DefineConcept{点法式方程}”.

由此可见,在直角坐标系中,
平面方程的一次项系数组成的向量就是这个平面的一个法向量的坐标.

\subsection{点到平面的距离}
\begin{theorem}
%@see: 《解析几何》(丘维声) P56 定理2.1
在直角坐标系中,点\(P_1(x_1,y_1,z_1)\)到平面\begin{equation*}
	\alpha: A x + B y + C z + D = 0
\end{equation*}的距离为
\begin{equation}
	d = \frac{\abs{A x_1 + B y_1 + C z_1 + D}}{\sqrt{A^2 + B^2 + C^2}}.
\end{equation}
\end{theorem}

\subsection{三元一次不等式的几何意义}
取定一个直角坐标系.
我们已经知道,坐标适合方程\begin{equation*}
	A x + B y + C z + D = 0
\end{equation*}的点在此方程表示的平面\(\alpha\)上.
显然,坐标合适不等式
\begin{equation}\label{equation:解析几何.平面的侧1}
	A x + B y + C z + D > 0
\end{equation}
的点\(P(x,y,z)\)就不在平面\(\alpha\)上.
设\(P\)到平面\(\alpha\)引的垂线的垂足为\(P_0\),
则\(\vec{P_0 P}\)与\(\vb{n}=(A,B,c)\)同向.
因此,所有坐标适合不等式 \labelcref{equation:解析几何.平面的侧1} 的点,
都在平面\(\alpha\)的同一侧,即\(\vb{n}\)所指的那一侧.
同理,所有坐标适合不等式
\begin{equation}\label{equation:解析几何.平面的侧2}
	A x + B y + C z + D < 0
\end{equation}
的点,也都在平面\(\alpha\)的同一侧,只不过是\((-\vb{n})\)所指的那一侧.

由上可知,平面\(\alpha\)把空间中的所有不在\(\alpha\)上的点分成了两部分:
一部分点的坐标都适合不等式 \labelcref{equation:解析几何.平面的侧1},
另一部分点的坐标都适合不等式 \labelcref{equation:解析几何.平面的侧2}.

于是,我们可以得到这样的结论:
两个点\(P_1(x_1,y_1,z_1)\)和\(P_2(x_2,y_2,z_2)\)在平面\(\alpha\)同侧的充分必要条件是\begin{equation*}
	(A x_1 + B y_1 + C z_1 + D) (A x_2 + B y_2 + C z_2 + D) > 0.
\end{equation*}
可以证明,这个结论在仿射坐标系中也成立.

\subsection{两个平面的夹角}
两个相交平面的夹角,是指两个平面交成的四个二面角中的任意一个.
易知其中两个等于两个平面的法向量\(\vb{n}_1,\vb{n}_2\)的夹角\(\angle(\vb{n}_1,\vb{n}_2)\),
另外两个等于\(\angle(\vb{n}_1,\vb{n}_2)\)的补角.
规定两个平行或重合平面的夹角为它们的法向量\(\vb{n}_1,\vb{n}_2\)的夹角或其补角,
从而等于\(0\)或\(\pi\).

设在直角坐标系中,两个平面的方程分别是\begin{equation*}
	\alpha_i:
	A_i x + B_i y + C_i z + D_i = 0,
	\quad i=1,2.
\end{equation*}
则\(\alpha_1\)与\(\alpha_2\)的一个夹角满足\begin{equation*}
	\angle(\alpha_1,\alpha_2)
	= \frac{\VectorInnerProductDot{\vb{n}_1}{\vb{n}_2}}{\VectorLengthA{\vb{n}_1} \VectorLengthA{\vb{n}_2}}
	= \frac{A_1 A_2 + B_1 B_2 + C_1 C_2}{\sqrt{A_1^2 + B_1^2 + C_1^2} \sqrt{A_2^2 + B_2^2 + C_2^2}},
\end{equation*}
由此可得,两个平面\(\alpha_1\)与\(\alpha_2\)垂直的充分必要条件是
\begin{equation}
	A_1 A_2 + B_1 B_2 + C_1 C_2 = 0.
\end{equation}
