\section{空间的直角坐标变换}
\cref{theorem:解析几何.空间的仿射坐标变换公式,%
theorem:解析几何.空间仿射坐标系同定向的充分必要条件}
在直角坐标系中当然也成立,
现在我们进一步研究直角坐标变换的特殊性.

\begin{theorem}
设 I \([O;\vb{e}_1,\vb{e}_2,\vb{e}_3]\)
和 II \([O';\vb{e}_1',\vb{e}_2',\vb{e}_3']\)
都是直角坐标系,
则 I 到 II 的过渡矩阵\(\vb{A}\)是正交矩阵,
从而 II 到 I 的过渡矩阵是\(\vb{A}^T\).
\end{theorem}

由于正交矩阵的行列式等于\(\pm1\),
因此空间的两个直角坐标系同定向的充分必要条件就是它们的过渡矩阵的行列式等于\(+1\).
