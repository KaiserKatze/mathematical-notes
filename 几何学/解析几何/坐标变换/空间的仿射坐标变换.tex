\section{空间的仿射坐标变换}
\begin{theorem}\label{theorem:解析几何.空间的仿射坐标变换公式}
设 I \([O;\vb{d}_1,\vb{d}_2,\vb{d}_3]\)
和 II \([O';\vb{d}_1',\vb{d}_2',\vb{d}_3']\)
都是空间仿射坐标系,
\(O'\)的 I 坐标是\((x_0,y_0,z_0)\),
\(\vb{d}_j'\)的 I 坐标是\((a_{1j},a_{2j},a_{3j})\ (j=1,2,3)\),
则 I 到 II 的点的仿射坐标变换公式为
\begin{equation}
	\begin{bmatrix}
		x \\ y \\ z
	\end{bmatrix} = \begin{bmatrix}
		a_{11} & a_{12} & a_{13} \\
		a_{21} & a_{22} & a_{23} \\
		a_{31} & a_{32} & a_{33}
	\end{bmatrix} \begin{bmatrix}
		x' \\ y' \\ z'
	\end{bmatrix} + \begin{bmatrix}
		x_0 \\ y_0 \\ z_0
	\end{bmatrix},
\end{equation}
I 到 II 的向量的仿射坐标变换公式为
\begin{equation}
	\begin{bmatrix}
		u_1 \\ u_2 \\ u_3
	\end{bmatrix} = \begin{bmatrix}
		a_{11} & a_{12} & a_{13} \\
		a_{21} & a_{22} & a_{23} \\
		a_{31} & a_{32} & a_{33}
	\end{bmatrix} \begin{bmatrix}
		u_1' \\ u_2' \\ u_3'
	\end{bmatrix},
\end{equation}
其中\begin{equation*}
	\vb{A} = \begin{bmatrix}
		a_{11} & a_{12} & a_{13} \\
		a_{21} & a_{22} & a_{23} \\
		a_{31} & a_{32} & a_{33}
	\end{bmatrix}
\end{equation*}称为“I 到 II 的过渡矩阵”,
它的第\(j\)列是\(\vb{d}_j'\ (j=1,2,3)\)的 I 坐标.
\end{theorem}

\begin{theorem}\label{theorem:解析几何.空间仿射坐标系同定向的充分必要条件}
仿射坐标系 I \([O;\vb{d}_1,\vb{d}_2,\vb{d}_3]\)
到 II \([O';\vb{d}_1',\vb{d}_2',\vb{d}_3']\)的过渡矩阵\(\vb{A}\)是非奇异的,
并且 I 与 II 同定向的充分必要条件是\(\abs{A}>0\).
\end{theorem}

\begin{corollary}
平面上的两个仿射坐标系 I \([O;\vb{d}_1,\vb{d}_2]\)
和 II \([O';\vb{d}_1',\vb{d}_2']\)同定向的充分必要条件是:
I 到 II 的过渡矩阵\(\vb{A}\)的行列式\(\abs{A}>0\).
\end{corollary}
