\section{平面的仿射坐标变换}
\subsection{点的仿射坐标变换公式}
平面上给了两个仿射坐标系:
\([O;\vb{d}_1,\vb{d}_2]\)
和\([O';\vb{d}_1',\vb{d}_2']\).
为方便起见,前一个称为旧坐标系,简记为 I;
后一个称为新坐标系,简记为 II.
点或向量在 I 中的坐标称为它的旧坐标,
在 II 中的坐标称为它的新坐标.
为了研究同一个点的新旧坐标的关系,
首先要明确上述两个坐标系的相对位置.

设 II 的原点\(O'\)的旧坐标是\((x_0,y_0)\),
基向量\(\vb{d}_1',\vb{d}_2'\)的旧坐标分别是\((a_{11},a_{21})\)和\((a_{12},a_{22})\).
现在我们来求某一点\(M\)的旧坐标\((x,y)\)与它的新坐标\((x',y')\)之间的关系.
因为\begin{align*}
	\vec{OM}
	&= \vec{O O'} + \vec{O' M}
	= (x_0 \vb{d}_1 + y_0 \vb{d}_2)
	+ (x' \vb{d}_1' + y' \vb{d}_2') \\
	&= (x_0 \vb{d}_1 + y_0 \vb{d}_2)
	+ x' (a_{11} \vb{d}_1 + a_{21} \vb{d}_2)
	+ y' (a_{12} \vb{d}_1 + a_{22} \vb{d}_2) \\
	&= (a_{11} x' + a_{12} y' + x_0) \vb{d}_1
	+ (a_{21} x' + a_{22} y' + y_0) \vb{d}_2,
\end{align*}
所以有\begin{equation}\label{equation:解析几何.平面坐标系的点的仿射坐标变换公式I到II}
	\left\{ \begin{array}{l}
		x = a_{11} x' + a_{12} y' + x_0, \\
		y = a_{21} x' + a_{22} y' + y_0.
	\end{array} \right.
\end{equation}
\cref{equation:解析几何.平面坐标系的点的仿射坐标变换公式I到II}
称为“平面上坐标系 I 到 II 的点的\DefineConcept{仿射坐标变换公式}”.
它是一个把任意一点\(M\)的旧坐标\(x,y\)表示成它的新坐标\(x',y'\)的一次多项式.

利用矩阵,我们可以将\cref{equation:解析几何.平面坐标系的点的仿射坐标变换公式I到II}
化为\begin{equation}\label{equation:解析几何.平面坐标系的点的仿射坐标变换公式I到II.矩阵形式1}
	\begin{bmatrix}
		x \\ y
	\end{bmatrix}
	= \begin{bmatrix}
		a_{11} & a_{12} \\
		a_{21} & a_{22}
	\end{bmatrix} \begin{bmatrix}
		x' \\ y'
	\end{bmatrix} + \begin{bmatrix}
		x_0 \\ y_0
	\end{bmatrix}.
\end{equation}
我们称矩阵\begin{equation*}
	\vb{A} = \begin{bmatrix}
		a_{11} & a_{12} \\
		a_{21} & a_{22}
	\end{bmatrix}
\end{equation*}为“坐标 I 到 II 的\DefineConcept{过渡矩阵}”,
它的第一列是 II 的第一个基向量\(\vb{d}_1'\)的 I 坐标,
第二列是 II 的第二个基向量\(\vb{d}_2'\)的 I 坐标.

\begin{theorem}
平面上点的仿射坐标变换公式 \labelcref{equation:解析几何.平面坐标系的点的仿射坐标变换公式I到II}
中的系数行列式不等于零,即\begin{equation*}
	\begin{vmatrix}
		a_{11} & a_{12} \\
		a_{21} & a_{22}
	\end{vmatrix} \neq 0.
\end{equation*}
\end{theorem}

根据上述定理,如果我们把\cref{equation:解析几何.平面坐标系的点的仿射坐标变换公式I到II}
看作关于\(x',y'\)的方程组,可以求得唯一解
\begin{equation}\label{equation:解析几何.平面坐标系的点的仿射坐标变换公式II到I}
	\left\{ \begin{array}{l}
		x' = \frac{1}{D} \begin{vmatrix}
			x - x_0 & a_{12} \\
			y - y_0 & a_{22}
		\end{vmatrix}, \\
		y' = \frac{1}{D} \begin{vmatrix}
			a_{11} & x - x_0 \\
			a_{21} & y - y_0
		\end{vmatrix}.
	\end{array} \right.
\end{equation}
其中\begin{equation*}
	D = \begin{vmatrix}
		a_{11} & a_{12} \\
		a_{21} & a_{22}
	\end{vmatrix}.
\end{equation*}
\cref{equation:解析几何.平面坐标系的点的仿射坐标变换公式II到I}
是把平面上任意一点\(M\)的新坐标\(x',y'\)表示成它的旧坐标\(x,y\)的一次多项式,
我们称其为“平面上坐标系 II 到 I 的点的仿射坐标变换公式”.

\subsection{向量的仿射坐标变换公式}
现在我们来看平面上的向量\(\vb{m}\)的旧坐标\((u,v)\)与它的新坐标\((u',v')\)之间的关系.
设\(\vb{m}=\vec{M_1 M_2}\),
其中点\(M_i\)的旧坐标为\((x_i,y_i)\),新坐标为\((x_i',y_i')\ (i=1,2)\),
则有\begin{align*}
	u &= x_2 - x_1
	= (a_{11} x_2' + a_{12} y_2' + x_0)
	- (a_{11} x_1' + a_{12} y_1' + x_0) \\
	&= a_{11} (x_2' - x_1')
	+ a_{12} (y_2' - y_1')
	= a_{11} u' + a_{12} v', \\
	v &= y_2 - y_1
	= (a_{21} x_2' + a_{22} y_2' + y_0)
	- (a_{21} x_1' + a_{22} y_1' + y_0) \\
	&= a_{21} u' + a_{22} v',
\end{align*}
即\begin{equation}\label{equation:解析几何.平面坐标系的向量的仿射坐标变换公式I到II}
	\left\{ \begin{array}{l}
		u = a_{11} u' + a_{12} v', \\
		v = a_{21} u' + a_{22} v'.
	\end{array} \right.
\end{equation}
\cref{equation:解析几何.平面坐标系的向量的仿射坐标变换公式I到II}
称为“平面上坐标系 I 到 II 的向量的仿射坐标变换公式”.
它是把任意一个向量\(\vb{m}\)的旧坐标\(u,v\)表示成它的新坐标\(u',v'\)的一次齐次多项式,
这时与点的坐标变换公式不同的地方.
平面上的点和向量是有本质区别的两种对象,
如果只从一个坐标系来看,
似乎点和向量的坐标都是有序实数对,
看不出它们的区别;
但是,如果取两个原点不重合的仿射坐标系,
通过坐标变换,
就可以看出两者的显著区别:
点的坐标变换公式 \labelcref{equation:解析几何.平面坐标系的点的仿射坐标变换公式I到II} 中有常数项,
而向量的坐标变换公式 \labelcref{equation:解析几何.平面坐标系的向量的仿射坐标变换公式I到II} 中没有常数项.

利用矩阵,我们可以将\cref{equation:解析几何.平面坐标系的向量的仿射坐标变换公式I到II}
化为\begin{equation}\label{equation:解析几何.平面坐标系的向量的仿射坐标变换公式I到II.矩阵形式1}
	\begin{bmatrix}
		u \\ v
	\end{bmatrix}
	= \begin{bmatrix}
		a_{11} & a_{12} \\
		a_{21} & a_{22}
	\end{bmatrix} \begin{bmatrix}
		u' \\ v'
	\end{bmatrix}.
\end{equation}

由于\cref{equation:解析几何.平面坐标系的向量的仿射坐标变换公式I到II}
中的系数行列式仍然不为零,因此可以反解出
\begin{equation}\label{equation:解析几何.平面坐标系的向量的仿射坐标变换公式II到I}
	\left\{ \begin{array}{l}
		u' = \frac{1}{D} \begin{vmatrix}
			u & a_{12} \\
			v & a_{22}
		\end{vmatrix}, \\
		v' = \frac{1}{D} \begin{vmatrix}
			a_{11} & u \\
			a_{21} & v
		\end{vmatrix}.
	\end{array} \right.
\end{equation}
这是“平面上坐标系 II 到 I 的向量的仿射坐标变换公式”.
由\cref{equation:解析几何.平面坐标系的向量的仿射坐标变换公式II到I} 看出,
I 的基向量\(\vb{d}_1,\vb{d}_2\)的 II 坐标分别是\begin{equation*}
	\frac{1}{D} (a_{22},-a_{21}), \qquad
	\frac{1}{D} (-a_{12},a_{11}).
\end{equation*}
