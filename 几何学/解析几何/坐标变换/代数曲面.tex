\section{代数曲面、代数曲线及其次数}
空间(或平面)的任一点对于不同不同坐标系的坐标是不同的,
因而作为点的轨迹的图形在不同坐标系中的方程也就不同,但是有
\begin{theorem}\label{theorem:解析几何.坐标变换时代数图形的不变性}
%@see: 《解析几何》(丘维声) P141 定理4.4
若图形\(S\)在仿射坐标系 I 中的方程\(F(x,y,z)=0\)的左端是关于\(x,y,z\)的\(n\)次多项式,
则\(S\)在任意一个仿射坐标系 II 中的方程\(G(x',y',z')=0\)的左端是关于\(x',y',z'\)的\(n\)次多项式.
\begin{proof}
因为 I 到 II 的点的坐标变换公式中\(x,y,z\)均表示成\(x',y',z'\)的一次多项式,
所以若\(F(x,y,z)\)是多项式,
则用 I 到 II 的坐标变换公式代入\(F(x,y,z)\)中得到的\(G(x',y',z')\)必是\(x',y',z'\)的多项式,
并且\(G(x',y',z')\)的次数\(m\)不超过\(F(x,y,z)\)的次数\(n\).
同理,因为用 II 到 I 的点的坐标变换公式代入\(G(x',y',z')\)中即得\(F(x,y,z)\),
所以\(n \leq m\).
于是\(m = n\).
\end{proof}
\end{theorem}
%@see: 《解析几何》(丘维声) P142
\cref{theorem:解析几何.坐标变换时代数图形的不变性} 说明,
一个图形的方程的左端是否为多项式,以及这个多项式的次数与坐标系的选择无关,
它们都是图形本身的性质.

若图形\(S\)的左端是多项式,则称\(S\)为\DefineConcept{代数曲面},
并且把这个多项式的次数称为这个代数曲面的\DefineConcept{次数}.

平面上的图形有类似的性质.
若平面上图形\(S\)的方程\(F(x,y)=0\)的左端是多项式,
则称\(S\)是\DefineConcept{代数曲线},
并且把这个多项式的次数称为这个代数曲线的\DefineConcept{次数}.
例如,椭圆、双曲线、抛物线在直角坐标系中的标准方程是二次的,
由此可知,它们不论在哪一个仿射坐标系中的方程也都是二次的,
所以它们才都被称为二次曲线.

%@see: 《解析几何》(丘维声) P144 习题4.4 12.
设 I \([O;\vb{e}_1,\vb{e}_2,\vb{e}_3]\)
和 II \([O';\vb{e}_1',\vb{e}_2',\vb{e}_3']\)
是有相同原点的右手直角坐标系,
则 I 到 II 的坐标变换可以分三个阶段来完成:\begin{gather*}
	\left\{ \begin{array}{l}
		x = x'' \cos\psi - y'' \sin\psi, \\
		y = x'' \sin\psi + y'' \cos\psi, \\
		z = z'';
	\end{array} \right. \\
	\left\{ \begin{array}{l}
		x'' = x''', \\
		y'' = y''' \cos\theta - z''' \sin\theta, \\
		z'' = y''' \sin\theta + z''' \cos\theta;
	\end{array} \right. \\
	\left\{ \begin{array}{l}
		x''' = x' \cos\phi - y' \sin\phi, \\
		y''' = x' \sin\phi + y' \cos\phi, \\
		z''' = z',
	\end{array} \right.
\end{gather*}
其中\(\psi,\theta,\phi\)称为\DefineConcept{欧拉角},
它们完全确定了 I 到 II 的坐标变换.
这里,第一步绕\(z\)轴旋转,
使得\(x\)轴转至\(x''\)轴(\(x''\)轴是\(Ox'y'\)平面与\(Oxy\)平面的交线),
旋转角为\(\psi\ (0\leq\psi<2\pi)\);
第二步绕\(x''\)轴旋转,
使得\(z\)轴(即\(z''\)轴)转至\(z'\)轴位置(即\(z'''\)轴与\(z'\)轴重合),
旋转角为\(\theta\ (0\leq\theta<2\pi)\);
第三步绕\(z'\)轴(即\(z'''\)轴)旋转,
使得\(x'''\)轴(即\(x''\)轴)转至\(x'\)轴位置,
旋转角为\(\phi\ (0\leq\phi<2\pi)\).
