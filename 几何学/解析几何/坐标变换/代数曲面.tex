\section{代数曲面、代数曲线及其次数}
空间(或平面)的任一点对于不同不同坐标系的坐标是不同的,
因而作为点的轨迹的图形在不同坐标系中的方程也就不同,但是有
\begin{theorem}\label{theorem:解析几何.坐标变换时代数图形的不变性}
若图形\(S\)在仿射坐标系 I 中的方程\(F(x,y,z)=0\)的左端是关于\(x,y,z\)的\(n\)次多项式,
则\(S\)在任意一个仿射坐标系 II 中的方程\(G(x',y',z')=0\)的左端是关于\(x',y',z'\)的\(n\)次多项式.
\end{theorem}
\cref{theorem:解析几何.坐标变换时代数图形的不变性} 说明,
一个图形的方程的左端是否为多项式,以及这个多项式的次数与坐标系的选择无关,
它们都是图形本身的性质.

若图形\(S\)的左端是多项式,则称\(S\)为\DefineConcept{代数曲面},
并且把这个多项式的次数称为这个代数曲面的\DefineConcept{次数}.

平面上的图形有类似的性质.
若平面上图形\(S\)的方程\(F(x,y)=0\)的左端是多项式,
则称\(S\)是\DefineConcept{代数曲线},
并且把这个多项式的次数称为这个代数曲线的\DefineConcept{次数}.
例如,椭圆、双曲线、抛物线在直角坐标系中的标准方程是二次的,
由此可知,它们不论在哪一个仿射坐标系中的方程也都是二次的,
所以它们才都被称为二次曲线.
