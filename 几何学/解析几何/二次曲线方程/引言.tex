\begingroup
平面上的二次曲线,除了椭圆(包括圆)、双曲线、抛物线以外,还有没有别的类型?
如果从所给的二次方程判别它代表什么二次曲线?
它的形状和位置如何?
二次曲线有哪些几何性质?
这些就是本章所要研究的问题.
本章所取的坐标系都是右手直角坐标系.

平面上二次曲线的一般方程是\begin{equation}\label{equation:二次曲线方程.平面二次曲线的一般方程}
%@see: 《解析几何》(丘维声) P145 (0.1)
	a_{11} x^2 + 2 a_{12} x y + a_{22} y^2
	+ 2 a_1 x + 2 a_2 y + a_0 = 0,
\end{equation}
其中\(a_{11},a_{12},a_{22}\)不全为零.

方程 \labelcref{equation:二次曲线方程.平面二次曲线的一般方程} 的左边
是关于\(x,y\)的一个二次多项式,
记作\(F(x,y)\),
即\begin{equation*}
	F(x,y)
	\defeq
	a_{11} x^2 + 2 a_{12} x y + a_{22} y^2
	+ 2 a_1 x + 2 a_2 y + a_0.
\end{equation*}
于是方程 \labelcref{equation:二次曲线方程.平面二次曲线的一般方程} 可以改写为\begin{equation*}
	F(x,y) = 0.
\end{equation*}

我们把\(F(x,y)\)的二次项部分记作\(\phi(x,y)\),
即\begin{equation}
%@see: 《解析几何》(丘维声) P145 (0.2)
	\phi(x,y)
	\defeq
	a_{11} x^2 + 2 a_{12} x y + a_{22} y^2.
\end{equation}
利用矩阵乘法,可以把\(\phi(x,y)\)写成下述形式:\begin{equation}
%@see: 《解析几何》(丘维声) P145 (0.3)
	\phi(x,y)
	= \begin{bmatrix}
		x & y
	\end{bmatrix}
	\begin{bmatrix}
		a_{11} & a_{12} \\
		a_{12} & a_{22}
	\end{bmatrix}
	\begin{bmatrix}
		x \\ y
	\end{bmatrix}.
\end{equation}
把矩阵\begin{equation}
%@see: 《解析几何》(丘维声) P145 (0.4)
	\begin{bmatrix}
		a_{11} & a_{12} \\
		a_{12} & a_{22}
	\end{bmatrix}
\end{equation}
称为“\(\phi(x,y)\)的矩阵”.
可以注意到,\(\phi(x,y)\)的矩阵是一个对称矩阵,
它的主对角元是\(\phi(x,y)\)中\(x^2,y^2\)项的系数,
它的\((1,2)\)元是\(\phi(x,y)\)中\(xy\)项的系数的一半.
类似地,利用矩阵乘法,可以\(F(x,y)\)写成如下形式:\begin{equation}
%@see: 《解析几何》(丘维声) P145 (0.5)
	F(x,y)
	= \begin{bmatrix}
		x & y & 1
	\end{bmatrix}
	\begin{bmatrix}
		a_{11} & a_{12} & a_1 \\
		a_{12} & a_{22} & a_2 \\
		a_1 & a_2 & a_0
	\end{bmatrix}
	\begin{bmatrix}
		x \\ y \\ 1
	\end{bmatrix}.
\end{equation}
把\begin{equation}
%@see: 《解析几何》(丘维声) P146 (0.6)
	\begin{bmatrix}
		a_{11} & a_{12} & a_1 \\
		a_{12} & a_{22} & a_2 \\
		a_1 & a_2 & a_0
	\end{bmatrix}
\end{equation}
称为“\(F(x,y)\)的矩阵”.
可以注意到,\(F(x,y)\)的矩阵也是一个对称矩阵.
如果把\(\phi(x,y)\)的矩阵记作\(\vb{A}\),
即\begin{equation*}
	\vb{A} \defeq \begin{bmatrix}
		a_{11} & a_{12} \\
		a_{12} & a_{22}
	\end{bmatrix},
\end{equation*}
把\(F(x,y)\)的矩阵记作\(\vb{P}\),
即\begin{equation*}
	\vb{P} \defeq \begin{bmatrix}
		a_{11} & a_{12} & a_1 \\
		a_{12} & a_{22} & a_2 \\
		a_1 & a_2 & a_0
	\end{bmatrix},
\end{equation*}
又令\(\vb\delta \defeq (a_1,a_2)^T\),
则\(\vb{P}\)可以分块写成\begin{equation*}
%@see: 《解析几何》(丘维声) P146 (0.7)
	\vb{P}
	= \begin{bmatrix}
		\vb{A} & \vb\delta \\
		\vb\delta^T & a_0
	\end{bmatrix}.
\end{equation*}
再令\(\vb\alpha = (x,y)^T\),
则\(F(x,y)\)可以表示成\begin{equation*}
%@see: 《解析几何》(丘维声) P146 (0.8)
	F(x,y)
	= \begin{bmatrix}
		\vb\alpha^T & 1
	\end{bmatrix}
	\begin{bmatrix}
		\vb{A} & \vb\delta \\
		\vb\delta^T & a_0
	\end{bmatrix}
	\begin{bmatrix}
		\vb\alpha \\ 1
	\end{bmatrix}.
\end{equation*}
于是方程 \labelcref{equation:二次曲线方程.平面二次曲线的一般方程} 可以写成
\begin{equation}\label{equation:二次曲线方程.平面二次曲线的一般方程.矩阵形式}
%@see: 《解析几何》(丘维声) P146 (0.9)
	\begin{bmatrix}
		\vb\alpha^T & 1
	\end{bmatrix}
	\begin{bmatrix}
		\vb{A} & \vb\delta \\
		\vb\delta^T & a_0
	\end{bmatrix}
	\begin{bmatrix}
		\vb\alpha \\ 1
	\end{bmatrix}
	= 0.
\end{equation}
\endgroup
