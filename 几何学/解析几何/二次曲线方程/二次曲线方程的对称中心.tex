\section{二次曲线方程的对称中心}
上一节我们利用二次曲线的不变量和半不变量直接从原方程的系数确定了曲线的类型和形状,
那么能否直接从原方程确定曲线的位置呢?
譬如,对于椭圆,只要能从原方程的系数求出它的对称中心和两根对称轴(长轴和短轴),则椭圆的位置就确定了.
因此,从本节开始,我们来讨论如何直接从原方程判别二次曲线有没有对称中心和对称轴,
如果有的话,如何求出它们.
我们还要讨论如何直接从原方程求出二次曲线的切线和法线,求出双曲线的渐近线等.
由于二次曲线的对称轴、切线、渐近线都是直线,因此我们就从讨论直线与二次曲线的相关位置入手.

\subsection{直线与二次曲线的相关位置}
设二次曲线\(S\)的方程是\cref{equation:二次曲线方程.平面二次曲线的一般方程},
直线\(l\)经过点\(M_0(x_0,y_0)\),方向向量是\(\vb{v} = (\mu,\nu)\),
则\(l\)的参数方程为\begin{equation*}
%@see: 《解析几何》(丘维声) P170 (3.1)
	\left\{ \begin{array}{l}
		x = x_0 + \mu t, \\
		y = y_0 + \nu t,
	\end{array} \right.
	\quad -\infty < t < +\infty.
\end{equation*}
为了讨论直线\(l\)与二次曲线\(S\)的相关位置,
我们把上式代入\cref{equation:二次曲线方程.平面二次曲线的一般方程},
整理得\begin{equation*}
%@see: 《解析几何》(丘维声) P170 (3.2)
	\phi(\mu,\nu) t^2
	+ 2 (
		F_1(x_0,y_0) \mu
		+ F_2(x_0,y_0) \nu
	) t
	+ F(x_0,y_0)
	= 0,
\end{equation*}
其中\(F(x,y)\)是\cref{equation:二次曲线方程.平面二次曲线的一般方程} 中等号左边的多项式,
\(\phi(x,y)\)是\(F(x,y)\)的二次项部分,
而\begin{gather*}
	F_1(x,y) \defeq a_{11} x + a_{12} y + a_1, \\
	F_2(x,y) \defeq a_{12} x + a_{22} y + a_2.
\end{gather*}
