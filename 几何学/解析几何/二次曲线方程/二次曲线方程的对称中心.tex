\section{二次曲线方程的对称中心}
上一节我们利用二次曲线的不变量和半不变量直接从原方程的系数确定了曲线的类型和形状,
那么能否直接从原方程确定曲线的位置呢?
譬如,对于椭圆,只要能从原方程的系数求出它的对称中心和两根对称轴(长轴和短轴),则椭圆的位置就确定了.
因此,从本节开始,我们来讨论如何直接从原方程判别二次曲线有没有对称中心和对称轴,
如果有的话,如何求出它们.
我们还要讨论如何直接从原方程求出二次曲线的切线和法线,求出双曲线的渐近线等.
由于二次曲线的对称轴、切线、渐近线都是直线,因此我们就从讨论直线与二次曲线的相关位置入手.

\subsection{直线与二次曲线的相关位置}
设二次曲线\(S\)的方程是\cref{equation:二次曲线方程.平面二次曲线的一般方程},
直线\(l\)经过点\(M_0(x_0,y_0)\),方向向量是\(\vb{v} = (\mu,\nu)\),
则\(l\)的参数方程为\begin{equation*}
%@see: 《解析几何》(丘维声) P170 (3.1)
	\left\{ \begin{array}{l}
		x = x_0 + \mu t, \\
		y = y_0 + \nu t,
	\end{array} \right.
	\quad -\infty < t < +\infty.
\end{equation*}
为了讨论直线\(l\)与二次曲线\(S\)的相关位置,
我们把上式代入\cref{equation:二次曲线方程.平面二次曲线的一般方程},
%@Mathematica: Collect[F[Subscript[x, 0] + \[Mu] t, Subscript[y, 0] + \[Nu] t] // Expand, {t, \[Mu], \[Nu]}]
整理得\begin{equation}\label{equation:二次曲线方程的对称中心.在曲线方程中代入直线方程}
%@see: 《解析几何》(丘维声) P170 (3.2)
	\phi(\mu,\nu) t^2
	+ 2 (
		F_1(x_0,y_0) \mu
		+ F_2(x_0,y_0) \nu
	) t
	+ F(x_0,y_0)
	= 0,
\end{equation}
其中\(F(x,y)\)是\cref{equation:二次曲线方程.平面二次曲线的一般方程} 中等号左边的多项式,
\(\phi(x,y)\)是\(F(x,y)\)的二次项部分,
而\begin{equation*}
	F_1(x,y) \defeq a_{11} x + a_{12} y + a_1,
	\qquad
	F_2(x,y) \defeq a_{12} x + a_{22} y + a_2.
\end{equation*}

下面我们按\(\phi(\mu,\nu)\)的取值是否为零,分情况讨论.
\begin{enumerate}
	\item 当\(\phi(\mu,\nu) \neq 0\)时,
	\cref{equation:二次曲线方程的对称中心.在曲线方程中代入直线方程} 是
	关于\(t\)的一元二次方程,
	它的判别式为\begin{equation*}
		\Delta \defeq 4 (F_1(x_0,y_0) \mu + F_2(x_0,y_0) \nu)^2 - 4 \phi(\mu,\nu) F(x_0,y_0).
	\end{equation*}
	\begin{enumerate}
		\item 如果\(\Delta > 0\),则\(l\)与\(S\)有两个不同交点;

		\item 如果\(\Delta = 0\),则\(l\)与\(S\)有两个重合交点;

		\item 如果\(\Delta < 0\),则\(l\)与\(S\)没有交点
		(由于此时方程 \labelcref{equation:二次曲线方程的对称中心.在曲线方程中代入直线方程} 有两个共轭复根,
		因此称\(l\)与\(S\)有两个虚交点).
	\end{enumerate}

	\item 当\(\phi(\mu,\nu) = 0\)时,
	我们又可以按照\(F_1(x_0,y_0) \mu + F_2(x_0,y_0) \nu\)是否等于零,分情况讨论.
	\begin{enumerate}
		\item 如果\(F_1(x_0,y_0) \mu + F_2(x_0,y_0) \nu \neq 0\),
		则方程 \labelcref{equation:二次曲线方程的对称中心.在曲线方程中代入直线方程} 是
		关于\(t\)的一元一次方程,
		说明\(l\)与\(S\)有且仅有一个交点;

		\item 如果\(F_1(x_0,y_0) \mu + F_2(x_0,y_0) \nu = 0\),
		我们还可以按照\(F(x_0,y_0)\)是否等于零,分情况讨论.
		\begin{enumerate}
			\item 如果\(F(x_0,y_0) = 0\),
			则方程 \labelcref{equation:二次曲线方程的对称中心.在曲线方程中代入直线方程} 是一个恒等式,有无穷多解,
			任意一个实数\(t\)都是方程 \labelcref{equation:二次曲线方程的对称中心.在曲线方程中代入直线方程} 的一个解,
			说明整条直线\(l\)都在\(S\)上,\(l\)是\(S\)的一个子集;

			\item 如果\(F(x_0,y_0) \neq 0\),
			则方程 \labelcref{equation:二次曲线方程的对称中心.在曲线方程中代入直线方程} 无解,
			说明\(l\)与\(S\)没有交点.
		\end{enumerate}
	\end{enumerate}
\end{enumerate}

\begin{definition}
%@see: 《解析几何》(丘维声) P171 定义3.1
设二次曲线\(S\)的方程为 \labelcref{equation:二次曲线方程.平面二次曲线的一般方程},
\(\vb{v}\)是一个非零向量,
\(\vb{v}\)的坐标是\((\mu,\nu)\).
\begin{itemize}
	\item 如果\(\phi(\mu,\nu) = 0\),
	则称“向量\(\vb{v}\)是曲线\(S\)的\DefineConcept{渐进方向}”.

	\item 如果\(\phi(\mu,\nu) \neq 0\),
	否则称“向量\(\vb{v}\)是曲线\(S\)的\DefineConcept{非渐进方向}”.
\end{itemize}
\end{definition}

\begin{theorem}
%@see: 《解析几何》(丘维声) P171 定理3.1
椭圆型曲线没有渐进方向.
%TODO proof
\end{theorem}

\begin{theorem}
%@see: 《解析几何》(丘维声) P171 定理3.1
双曲型曲线有两个渐进方向.
%TODO proof
\end{theorem}

\begin{theorem}
%@see: 《解析几何》(丘维声) P171 定理3.1
抛物型曲线只有一个渐进方向.
%TODO proof
\end{theorem}

\subsection{二次曲线的对称中心}
我们已经知道,椭圆和双曲线都有一个对称中心,而抛物线没有对称中心.
其中几种二次曲线有没有对称中心?
如果一条二次曲线有对称中心,如何直接从原方程求出它的对称中心?
本小节就来讨论这些问题.

\begin{definition}
%@see: 《解析几何》(丘维声) P172 定义3.2
设\(S\)是一个二次曲线.
如果\(S\)上任意一点\(M_1\)关于定点\(O'\)的对称点\(M_2\)仍在\(S\)上,
则称“点\(O'\)是曲线\(S\)的一个\DefineConcept{对称中心}”.
\end{definition}

\begin{theorem}\label{theorem:二次曲线方程的对称中心.一点是二次曲线方程的对称中心的充分必要条件}
%@see: 《解析几何》(丘维声) P172 定理3.2
点\(O'(x_0,y_0)\)是二次曲线\(S\)的对称中心的充分必要条件
是\begin{equation}\label{equation:二次曲线方程的对称中心.对称中心方程组}
%@see: 《解析几何》(丘维声) P173 (3.4)
	F_1(x_0,y_0) = F_2(x_0,y_0) = 0.
\end{equation}
\begin{proof}
必要性.
设\(O'(x_0,y_0)\)是二次曲线\(S\)的一个对称中心.
任取\(S\)的一个非渐进方向\((\mu,\nu)\),
则经过\(O'\)且方向为\((\mu,\nu)\)的直线\(l\)
与\(S\)一定有两个交点\(M_1(x_1,y_1)\)和\(M_2(x_2,y_2)\).
由于\(O'\)是\(S\)的对称中心,
所以\(O'\)是线段\(M_1 M_2\)的中点.
于是\begin{equation*}
%@see: 《解析几何》(丘维声) P173 (3.5)
	x_0 = \frac{x_1 + x_2}{2},
	\qquad
	y_0 = \frac{y_1 + y_2}{2}.
	\eqno(1)
\end{equation*}
设\(M_i\)对应的参数值为\(t_i\),则有\begin{equation*}
%@see: 《解析几何》(丘维声) P173 (3.6)
	\left\{ \begin{array}{l}
		x_i = x_0 + \mu t_i, \\
		y_i = y_0 + \nu t_i,
	\end{array} \right.
	\quad i=1,2.
	\eqno(2)
\end{equation*}
将(2)式代入(1)式,得\begin{equation*}
	2 x_0 = 2 x_0 + \mu (t_1 + t_2),
	\qquad
	2 y_0 = 2 y_0 + \nu (t_1 + t_2),
\end{equation*}
于是有\begin{equation*}
	\mu (t_1 + t_2) = \nu (t_1 + t_2) = 0.
\end{equation*}
% 由非渐进方向的定义,向量\(\vb{v} = (\mu,\nu)\)不是零向量
由于\(\mu,\nu\)不全为零,
因此\(t_1 + t_2 = 0\).
因为\(M_1,M_2\)是\(l\)与\(S\)的交点,
所以它对应的参数值\(t_1,t_2\)是
关于\(t\)的一元二次方程 \labelcref{equation:二次曲线方程的对称中心.在曲线方程中代入直线方程} 的根.
根据\hyperref[theorem:一元二次方程.韦达定理]{韦达定理},
有\begin{equation*}
	t_1 + t_2
	= \frac{-2}{\phi(\mu,\nu)} (
		F_1(x_0,y_0) \mu
		+ F_2(x_0,y_0) \nu
	).
\end{equation*}
再由\(t_1 + t_2 = 0\)得\begin{equation*}
%@see: 《解析几何》(丘维声) P173 (3.7)
	\mu F_1(x_0,y_0) + \nu F_2(x_0,y_0) = 0.
\end{equation*}
上式对于\(S\)的任意非渐进方向\((\mu,\nu)\)都成立.
取\(S\)的两个不共线的非渐进方向\(\vb{v}_1(\mu_1,\nu_1)\)和\(\vb{v}_2(\mu_2,\nu_2)\),
则有\begin{equation*}
%@see: 《解析几何》(丘维声) P173 (3.8)
	\mu_1 F_1(x_0,y_0) + \nu_1 F_2(x_0,y_0) = 0,
	\qquad
	\mu_2 F_1(x_0,y_0) + \nu_2 F_2(x_0,y_0) = 0.
\end{equation*}
这说明\((F_1(x_0,y_0),F_2(x_0,y_0))\)是关于\(x,y\)的齐次线性方程组\begin{equation*}
%@see: 《解析几何》(丘维声) P173 (3.9)
	\left\{ \begin{array}{l}
		\mu_1 x + \nu_1 y = 0, \\
		\mu_2 x + \nu_2 y = 0
	\end{array} \right.
	\eqno(3)
\end{equation*}的解.
由于\(\vb{v}_1\)与\(\vb{v}_2\)不共线,
因此\begin{equation*}
	\begin{vmatrix}
		\mu_1 & \mu_2 \\
		\nu_1 & \nu_2
	\end{vmatrix}
	\neq 0,
\end{equation*}
从而齐次线性方程组(3)只有零解,
即\begin{equation*}
	F_1(x_0,y_0) = F_2(x_0,y_0) = 0.
\end{equation*}

充分性.
假设点\(O'\)的坐标\((x_0,y_0)\)是方程 \labelcref{equation:二次曲线方程的对称中心.对称中心方程组} 的解,
即\begin{equation*}
	F_1(x_0,y_0) = F_2(x_0,y_0) = 0.
\end{equation*}
作移轴,使\(O'\)为新坐标系的原点,移轴公式为\begin{equation*}
	\left\{ \begin{array}{l}
		x = x' + x_0, \\
		y = y' + y_0.
	\end{array} \right.
\end{equation*}
代入\(S\)的原方程 \labelcref{equation:二次曲线方程.平面二次曲线的一般方程},
得到\(S\)的新方程为\begin{equation*}
%@see: 《解析几何》(丘维声) P174 (3.10)
	a_{11} (x')^2
	+ 2 a_{12} x' y'
	+ a_{22} (y')^2
	+ F(x_0,y_0)
	= 0.
\end{equation*}
在此方程中,用\(-x'\)代\(x'\),用\(-y'\)代\(y'\),方程不变,
所以\(O'\)是\(S\)的对称中心.
\end{proof}
\end{theorem}

\begin{theorem}
%@see: 《解析几何》(丘维声) P174 定理3.3
椭圆型或双曲型曲线有唯一的对称中心.
%TODO proof
\end{theorem}

\begin{theorem}
%@see: 《解析几何》(丘维声) P174 定理3.3
\(I_3 = 0\)的抛物型曲线有无穷多个对称中心,
且这些对称中心组成一条直线(方程为\(F_1(x,y) = 0\)或\(F_2(x,y) = 0\)).
%TODO proof
\end{theorem}

\begin{theorem}
%@see: 《解析几何》(丘维声) P174 定理3.3
\(I_3 \neq 0\)的抛物型曲线没有对称中心.
%TODO proof
\end{theorem}

我们把具有唯一对称中心的二次曲线称为\DefineConcept{中心型曲线},
把没有对称中心的二次曲线称为\DefineConcept{无心曲线},
把具有无穷多个对称中心的二次曲线称为\DefineConcept{线心曲线},
把无心曲线和线心曲线统称为\DefineConcept{非中心型曲线}.

我们把线心曲线的全体对称中心组成的直线称为它的\DefineConcept{中心直线}.

椭圆型和双曲型曲线都是中心型曲线,
抛物型曲线都是非中心型曲线,
其中抛物线是无心曲线,一对平行直线、一对虚平行直线、一对重合直线都是线性曲线.
