\section{二次曲线方程的直径和对称轴}
我们已经知道,椭圆、双曲线和抛物线都有对称轴.
如何直接从它们的原方程求出它们的对称轴?

\begin{definition}
%@see: 《解析几何》(丘维声) P176 定义4.1
设\(S\)是一个二次曲线.
如果\(S\)上任意一点\(M_1\)关于定直线\(l\)的对称点\(M_2\)仍在\(S\)上,
则称“直线\(l\)是曲线\(S\)的一个\DefineConcept{对称轴}”.
\end{definition}

由定义可知,如果直线\(l\)是二次曲线\(S\)的对称轴,
则\(l\)必定经过\(S\)的某一组平行弦的中点,
并且\(l\)跟这组平行弦方向垂直.
因此,为了研究二次曲线的对称轴,我们首先需要研究它的一组平行弦各自中点所在的直线.

\subsection{二次曲线的直径}
\begin{theorem}\label{theorem:二次曲线方程的直径和对称轴.二次曲线沿非渐进方向的平行弦的中垂线}
%@see: 《解析几何》(丘维声) P176 定理4.1
二次曲线\(S\)的沿非渐进方向\((\mu,\nu)\)的平行弦的中点
都在一条直线\begin{equation}\label{equation:二次曲线方程的直径和对称轴.二次曲线沿非渐进方向的平行弦的中垂线1}
	\mu F_1(x,y) + \nu F_2(x,y) = 0.
\end{equation}上.
\begin{proof}
任取沿非渐进方向\((\mu,\nu)\)的一条弦\(M_1 M_2\).
由\cref{theorem:二次曲线方程的对称中心.弦中点方程} 可知,
弦\(M_1 M_2\)的中点的坐标\((x,y)\)满足方程\begin{equation*}
	\mu F_1(x,y) + \nu F_2(x,y) = 0,
\end{equation*}
即\begin{equation*}
	\mu (a_{11} x + a_{12} y + a_1) + \nu (a_{12} x + a_{22} y + a_2) = 0,
\end{equation*}
亦即\begin{equation}\label{equation:二次曲线方程的直径和对称轴.二次曲线沿非渐进方向的平行弦的中垂线2}
%@see: 《解析几何》(丘维声) P177 (4.2)
	(a_{11} \mu + a_{12} \nu) x
	+ (a_{12} \mu + a_{22} \nu) y
	+ a_1 \mu + a_2 \nu
	= 0.
\end{equation}
方程 \labelcref{equation:二次曲线方程的直径和对称轴.二次曲线沿非渐进方向的平行弦的中垂线2} 的
一次项系数一定不全为零.
事实上,假如\begin{equation*}
	\left\{ \begin{array}{l}
		a_{11} \mu + a_{12} \nu = 0, \\
		a_{12} \mu + a_{22} \nu = 0,
	\end{array} \right.
\end{equation*}
则\begin{equation*}
	\mu (a_{11} \mu + a_{12} \nu) + \nu (a_{12} \mu + a_{22} \nu) = 0,
\end{equation*}
即\begin{equation*}
	a_{11} \mu^2 + 2 a_{12} \mu \nu + a_{22} \nu^2 = 0,
\end{equation*}
亦即\begin{equation*}
	\phi(\mu,\nu) = 0.
\end{equation*}
这说明\((\mu,\nu)\)是\(S\)的渐进方向,矛盾.
因此方程 \labelcref{equation:二次曲线方程的直径和对称轴.二次曲线沿非渐进方向的平行弦的中垂线2}
是关于\(x,y\)的一次方程,它表示一条直线.
于是,沿\((\mu,\nu)\)方向的平行弦的中点都在这条直线上.
\end{proof}
\end{theorem}

\begin{definition}
%@see: 《解析几何》(丘维声) P177 定义4.2
二次曲线\(S\)的沿非渐进方向\((\mu,\nu)\)的平行弦中点所在的直线,
称为“\(S\)的\DefineConcept{共轭于方向\((\mu,\nu)\)的直径}”.
\end{definition}

由\cref{theorem:二次曲线方程的直径和对称轴.二次曲线沿非渐进方向的平行弦的中垂线} 可知,
二次曲线\(S\)的共轭于方向\((\mu,\nu)\)的直径的方程
就是\cref{equation:二次曲线方程的直径和对称轴.二次曲线沿非渐进方向的平行弦的中垂线1},
也可以写成\cref{equation:二次曲线方程的直径和对称轴.二次曲线沿非渐进方向的平行弦的中垂线2} 的形式.

\begin{corollary}
%@see: 《解析几何》(丘维声) P177 推论4.1
中心型曲线、线心曲线的直径一定经过它的对称中心.
%TODO proof
\end{corollary}

从\cref{equation:二次曲线方程的直径和对称轴.二次曲线沿非渐进方向的平行弦的中垂线2} 看出,
二次曲线\(S\)的共轭于非渐进方向\((\mu,\nu)\)的直径\(l\)的方向\((\mu^*,\nu^*)\)为\begin{equation*}
	\begin{bmatrix}
		\mu^* \\ \nu^*
	\end{bmatrix}
	= \vb{B}
	\begin{bmatrix}
		\mu \\ \nu
	\end{bmatrix},
\end{equation*}
其中\begin{equation*}
	\vb{B} \defeq \begin{bmatrix}
		-a_{12} & -a_{22} \\
		a_{11} & a_{12}
	\end{bmatrix}.
\end{equation*}

\begin{proposition}\label{theorem:二次曲线方程的直径和对称轴.直径方向}
%@see: 《解析几何》(丘维声) P177 命题4.1
设二次曲线\(S\)的方程为 \labelcref{equation:二次曲线方程.平面二次曲线的一般方程},
其二次项部分\(\phi(x,y)\)的矩阵为\(\vb{A}\),
\(S\)的共轭于非渐进方向\((\mu,\nu)\)的直径\(l\)的方向为\((\mu^*,\nu^*)\),
则\begin{equation*}
	\phi(\mu^*,\nu^*) = I_2 \phi(\mu,\nu).
\end{equation*}
%TODO proof
\end{proposition}

对于中心型曲线\(S\),
由于\(I_2 \neq 0\),
因此由\cref{theorem:二次曲线方程的直径和对称轴.直径方向} 可得\begin{equation*}
	\phi(\mu^*,\nu^*) = 0
	\iff
	\phi(\mu,\nu) = 0,
\end{equation*}
从而\((\mu^*,\nu^*)\)是\(S\)的非渐进方向,
当且仅当\((\mu,\nu)\)是非渐进方向.
于是,中心型曲线\(S\)的共轭于非渐进方向\((\mu_1,\nu_1)\)的直径\(l_1\)的方向\((\mu^*_1,\nu^*_1)\)必定是非渐进方向.
因此存在共轭于方向\((\mu^*_1,\nu^*_1)\)的直径\(l'\).
我们把\(l\)和\(l'\)称为“曲线\(S\)的一对\DefineConcept{共轭直径}”.

\subsection{圆锥曲线的对称轴}
设\(l\)是圆锥曲线\(S\)的对称轴,根据上述讨论可知,\(l\)是\(S\)的一条直径,
并且\(l\)与它所共轭的非渐进方向\((\mu,\nu)\)垂直.
因此,\(l\)的方程是\cref{equation:二次曲线方程的直径和对称轴.二次曲线沿非渐进方向的平行弦的中垂线1}
或\cref{equation:二次曲线方程的直径和对称轴.二次曲线沿非渐进方向的平行弦的中垂线2}.
因为\(l\)与\((\mu,\nu)\)垂直,
所以有\begin{equation*}
	-\mu (a_{12} \mu + a_{22} \nu) + \nu (a_{11} \mu + a_{12} \nu) = 0.
\end{equation*}
