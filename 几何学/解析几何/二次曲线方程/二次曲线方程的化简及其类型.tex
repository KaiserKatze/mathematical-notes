\section{二次曲线方程的化简及其类型}
为了判别在右手直角坐标系 I \([O;\vb{e}_1,\vb{e}_2]\)中的
二次方程 \labelcref{equation:二次曲线方程.平面二次曲线的一般方程}
当系数取各种值时,
它能表示哪一种曲线,
容易想到的办法是:
作坐标变换,
使得二次曲线 \labelcref{equation:二次曲线方程.平面二次曲线的一般方程}
在新坐标系中的方程比较简单,易于辨认出它表示的曲线的种类.
由于任意一个直角坐标变换都可以由移轴公式和转轴公式得到,
所以我们首先来研究在转轴时
二次方程 \labelcref{equation:二次曲线方程.平面二次曲线的一般方程} 的系数变化规律.

\subsection{转轴消去交叉项}
设 II \([O';\vb{e}_1',\vb{e}_2']\)是由 I 经过转轴得到的,
转角为\(\theta\),
则 I 到 II 的点的坐标变换公式
就是\cref{equation:解析几何.平面坐标系的点的右手直角坐标变换公式I到II.矩阵形式3},
即\begin{equation}
%@see: 《解析几何》(丘维声) P146 (1.1)
	\begin{bmatrix}
		x \\ y
	\end{bmatrix}
	= \begin{bmatrix}
		\cos\theta & -\sin\theta \\
		\sin\theta & \cos\theta
	\end{bmatrix}
	\begin{bmatrix}
		x' \\ y'
	\end{bmatrix}.
\end{equation}
令\begin{equation*}
	\vb{R} \defeq \begin{bmatrix}
		\cos\theta & -\sin\theta \\
		\sin\theta & \cos\theta
	\end{bmatrix},
	\qquad
	\vb\alpha \defeq \begin{bmatrix}
		x \\ y
	\end{bmatrix},
	\qquad
	\vb\alpha' \defeq \begin{bmatrix}
		x' \\ y'
	\end{bmatrix},
\end{equation*}
则转轴公式 \labelcref{equation:解析几何.平面坐标系的点的右手直角坐标变换公式I到II.矩阵形式3}
可以写成\begin{equation*}
	\vb\alpha = \vb{R} \vb\alpha'.
\end{equation*}

由转轴公式 \labelcref{equation:解析几何.平面坐标系的点的右手直角坐标变换公式I到II.矩阵形式3}
可以得到\begin{equation*}
	\begin{bmatrix}
		x \\ y \\ 1
	\end{bmatrix}
	= \begin{bmatrix}
		\cos\theta & -\sin\theta & 0 \\
		\sin\theta & \cos\theta & 0 \\
		0 & 0 & 1
	\end{bmatrix}
	\begin{bmatrix}
		x' \\ y' \\ 1
	\end{bmatrix},
\end{equation*}
即\begin{equation*}
%@see: 《解析几何》(丘维声) P147 (1.2)
	\begin{bmatrix}
		\vb\alpha \\ 1
	\end{bmatrix}
	= \begin{bmatrix}
		\vb{R} & \vb0 \\
		\vb0 & 1
	\end{bmatrix}
	\begin{bmatrix}
		\vb\alpha' \\ 1
	\end{bmatrix}.
\end{equation*}
把上式代入方程 \labelcref{equation:二次曲线方程.平面二次曲线的一般方程.矩阵形式} 中,
可以得到二次曲线 \labelcref{equation:二次曲线方程.平面二次曲线的一般方程} 的新方程:
\begin{equation}
%@see: 《解析几何》(丘维声) P147 (1.3)
	\begin{bmatrix}
		(\vb\alpha')^T & 1
	\end{bmatrix}
	\begin{bmatrix}
		\vb{R}^T & \vb0 \\
		\vb0 & 1
	\end{bmatrix}
	\begin{bmatrix}
		\vb{A} & \vb\delta \\
		\vb\delta^T & a_0
	\end{bmatrix}
	\begin{bmatrix}
		\vb{R} & \vb0 \\
		\vb0 & 1
	\end{bmatrix}
	\begin{bmatrix}
		\vb\alpha' \\ 1
	\end{bmatrix}
	= 0,
\end{equation}
即\begin{equation*}
%@see: 《解析几何》(丘维声) P147 (1.3)'
	\begin{bmatrix}
		(\vb\alpha')^T & 1
	\end{bmatrix}
	\begin{bmatrix}
		\vb{R}^T \vb{A} \vb{R} & \vb{R}^T \vb\delta \\
		\vb\delta^T \vb{R} & a_0
	\end{bmatrix}
	\begin{bmatrix}
		\vb\alpha' \\ 1
	\end{bmatrix}
	= 0.
\end{equation*}
由于\begin{equation*}
	(\vb{R}^T \vb{A} \vb{R})^T
	= \vb{R}^T \vb{A}^T \vb{R}
	= \vb{R}^T \vb{A} \vb{R},
\end{equation*}
所以\(\vb{R}^T \vb{A} \vb{R}\)仍是对称矩阵.
于是新方程的二次项部分\(\phi'(x',y')\)的矩阵为\(\vb{R}^T \vb{A} \vb{R}\),
一次项系数的一半组成的列是\(\vb{R}^T \vb\delta\),
常数项是\(a_0\).
这说明,经过转轴,
新方程的二次项系数只与原方程的二次项系数及转角\(\theta\)有关,
新方程的一次项系数只与原方程的一次项系数及转角\(\theta\)有关,
常数项不变.

转轴后二次曲线\(S\)的新方程的二次项部分的矩阵为\(\vb{R}^T \vb{A} \vb{R}\).
新方程中不出现交叉项(即\(x' y'\)项),
当且仅当\begin{equation*}
%@see: 《解析几何》(丘维声) P147 (1.4)
	\vb{R}^T \vb{A} \vb{R}
	= \begin{bmatrix}
		a'_{11} & 0 \\
		0 & a'_{22}
	\end{bmatrix}.
\end{equation*}
换句话说,
新方程中不出现交叉项,
当且仅当\(\vb{A}\)可以合同对角化.
%@see: 《解析几何》(丘维声) P149 定理1.1
鉴于\(\vb{A}\)是实对称矩阵,
那么由\cref{theorem:实对称矩阵.实对称矩阵可以正交对角化} 可知,
\(\vb{A}\)一定可以合同对角化.

假设\(\vb{A}\)的两个特征值相同,
那么\(\vb{A}\)的属于特征值\(\lambda\)的特征子空间\(V_\lambda\)就是坐标平面\(\mathbb{R}^2\).

假设\(\lambda_1,\lambda_2\)是\(\vb{A}\)的两个不同的特征值,
\(\vb\gamma_1\)是\(\vb{A}\)的属于特征值\(\lambda_1\)的特征向量,
\(\vb\gamma_2\)是\(\vb{A}\)的属于特征值\(\lambda_2\)的特征向量,
则\begin{equation*}
	\vb{A} \vb\gamma_1 = \lambda_1 \vb\gamma_1,
	\qquad
	\vb{A} \vb\gamma_2 = \lambda_2 \vb\gamma_2,
\end{equation*}
从而有\begin{align*}
	\vb\gamma_1^T \vb{A} \vb\gamma_2 &= \lambda_2 \vb\gamma_1^T \vb\gamma_2, \\
	\vb\gamma_2^T \vb{A} \vb\gamma_1 &= \lambda_1 \vb\gamma_2^T \vb\gamma_1
									= \lambda_1 (\vb\gamma_2^T \vb\gamma_1)^T
									= \lambda_1 \vb\gamma_1^T \vb\gamma_2.
\end{align*}
由于\begin{equation*}
	\vb\gamma_1^T \vb{A} \vb\gamma_2
	= (\vb\gamma_1^T \vb{A} \vb\gamma_2)^T
	= \vb\gamma_2^T \vb{A}^T = \vb\gamma_1
	= \vb\gamma_2^T \vb{A} \vb\gamma_1,
\end{equation*}
因此\(
	\lambda_2 \vb\gamma_1^T \vb\gamma_2
	= \lambda_1 \vb\gamma_1^T \vb\gamma_2
\),
从而有\(
	(\lambda_2 - \lambda_1) \vb\gamma_1^T \vb\gamma_2 = 0
\).
因为\(\lambda_2 \neq \lambda_1\),
所以\(\vb\gamma_1^T \vb\gamma_2 = 0\),
即\(\vb\gamma_1\)与\(\vb\gamma_2\)正交.
显然\(\Span\{\vb\gamma_1,\vb\gamma_2\} = \mathbb{R}^2\).
这就说明\(\vb{A}\)的属于特征值\(\lambda_1\)的特征子空间\(V_{\lambda_1}\)
和\(\vb{A}\)的属于特征值\(\lambda_2\)的特征子空间\(V_{\lambda_2}\)互为正交补.
