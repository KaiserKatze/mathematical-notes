\section{二次曲线方程的化简及其类型}
为了判别在右手直角坐标系 I \([O;\vb{e}_1,\vb{e}_2]\)中的
二次方程 \labelcref{equation:二次曲线方程.平面二次曲线的一般方程}
当系数取各种值时,
它能表示哪一种曲线,
容易想到的办法是:
作坐标变换,
使得二次曲线 \labelcref{equation:二次曲线方程.平面二次曲线的一般方程}
在新坐标系中的方程比较简单,易于辨认出它表示的曲线的种类.
由于任意一个直角坐标变换都可以由移轴公式和转轴公式得到,
所以我们首先来研究在转轴时
二次方程 \labelcref{equation:二次曲线方程.平面二次曲线的一般方程} 的系数变化规律.

\subsection{转轴消去交叉项}
设 II \([O';\vb{e}_1',\vb{e}_2']\)是由 I 经过转轴得到的,
转角为\(\theta\),
则 I 到 II 的点的坐标变换公式
就是\cref{equation:解析几何.平面坐标系的点的右手直角坐标变换公式I到II.矩阵形式3},
即\begin{equation}
%@see: 《解析几何》(丘维声) P146 (1.1)
	\begin{bmatrix}
		x \\ y
	\end{bmatrix}
	= \begin{bmatrix}
		\cos\theta & -\sin\theta \\
		\sin\theta & \cos\theta
	\end{bmatrix}
	\begin{bmatrix}
		x' \\ y'
	\end{bmatrix}.
\end{equation}
令\begin{equation*}
	\vb{R} \defeq \begin{bmatrix}
		\cos\theta & -\sin\theta \\
		\sin\theta & \cos\theta
	\end{bmatrix},
	\qquad
	\vb\alpha \defeq \begin{bmatrix}
		x \\ y
	\end{bmatrix},
	\qquad
	\vb\alpha' \defeq \begin{bmatrix}
		x' \\ y'
	\end{bmatrix},
\end{equation*}
则转轴公式 \labelcref{equation:解析几何.平面坐标系的点的右手直角坐标变换公式I到II.矩阵形式3}
可以写成\begin{equation*}
	\vb\alpha = \vb{R} \vb\alpha'.
\end{equation*}

由转轴公式 \labelcref{equation:解析几何.平面坐标系的点的右手直角坐标变换公式I到II.矩阵形式3}
可以得到\begin{equation*}
	\begin{bmatrix}
		x \\ y \\ 1
	\end{bmatrix}
	= \begin{bmatrix}
		\cos\theta & -\sin\theta & 0 \\
		\sin\theta & \cos\theta & 0 \\
		0 & 0 & 1
	\end{bmatrix}
	\begin{bmatrix}
		x' \\ y' \\ 1
	\end{bmatrix},
\end{equation*}
即\begin{equation*}
%@see: 《解析几何》(丘维声) P147 (1.2)
	\begin{bmatrix}
		\vb\alpha \\ 1
	\end{bmatrix}
	= \begin{bmatrix}
		\vb{R} & \vb0 \\
		\vb0 & 1
	\end{bmatrix}
	\begin{bmatrix}
		\vb\alpha' \\ 1
	\end{bmatrix}.
\end{equation*}
把上式代入方程 \labelcref{equation:二次曲线方程.平面二次曲线的一般方程.矩阵形式} 中,
可以得到二次曲线 \labelcref{equation:二次曲线方程.平面二次曲线的一般方程} 的新方程:
\begin{equation}
%@see: 《解析几何》(丘维声) P147 (1.3)
	\begin{bmatrix}
		(\vb\alpha')^T & 1
	\end{bmatrix}
	\begin{bmatrix}
		\vb{R}^T & \vb0 \\
		\vb0 & 1
	\end{bmatrix}
	\begin{bmatrix}
		\vb{A} & \vb\delta \\
		\vb\delta^T & a_0
	\end{bmatrix}
	\begin{bmatrix}
		\vb{R} & \vb0 \\
		\vb0 & 1
	\end{bmatrix}
	\begin{bmatrix}
		\vb\alpha' \\ 1
	\end{bmatrix}
	= 0,
\end{equation}
即\begin{equation*}
%@see: 《解析几何》(丘维声) P147 (1.3)'
	\begin{bmatrix}
		(\vb\alpha')^T & 1
	\end{bmatrix}
	\begin{bmatrix}
		\vb{R}^T \vb{A} \vb{R} & \vb{R}^T \vb\delta \\
		\vb\delta^T \vb{R} & a_0
	\end{bmatrix}
	\begin{bmatrix}
		\vb\alpha' \\ 1
	\end{bmatrix}
	= 0.
\end{equation*}
由于\begin{equation*}
	(\vb{R}^T \vb{A} \vb{R})^T
	= \vb{R}^T \vb{A}^T \vb{R}
	= \vb{R}^T \vb{A} \vb{R},
\end{equation*}
所以\(\vb{R}^T \vb{A} \vb{R}\)仍是对称矩阵.
于是新方程的二次项部分\(\phi'(x',y')\)的矩阵为\(\vb{R}^T \vb{A} \vb{R}\),
一次项系数的一半组成的列是\(\vb{R}^T \vb\delta\),
常数项是\(a_0\).
这说明,经过转轴,
新方程的二次项系数只与原方程的二次项系数及转角\(\theta\)有关,
新方程的一次项系数只与原方程的一次项系数及转角\(\theta\)有关,
常数项不变.

转轴后二次曲线\(S\)的新方程的二次项部分的矩阵为\(\vb{R}^T \vb{A} \vb{R}\).
注意到旋转矩阵\(\vb{R}\)中有一个参数\(\theta\)尚未确定,
我们希望新方程中不出现交叉项(即\(x' y'\)项的系数\(a'_{12}\)等于零),
为此需要给\(\theta\)的取值范围增加限制条件.
首先,我们可以断言:
新方程中不出现交叉项,
当且仅当\begin{equation*}
%@see: 《解析几何》(丘维声) P147 (1.4)
	\vb{R}^T \vb{A} \vb{R}
	= \begin{bmatrix}
		a'_{11} & 0 \\
		0 & a'_{22}
	\end{bmatrix}.
\end{equation*}
注意到旋转矩阵\(\vb{R}\)是一个正交矩阵(即\(\vb{R}^T \vb{R} = \vb{E}\)),
于是我们可以说:
新方程中不出现交叉项,
当且仅当\(\vb{A}\)可以正交对角化.
%@see: 《解析几何》(丘维声) P149 定理1.1
我们知道\hyperref[theorem:实对称矩阵.实对称矩阵可以正交对角化]{实对称矩阵可以正交对角化},
恰巧\(\vb{A}\)就是一个实对称矩阵,
因此一定存在一个正交矩阵\(\vb{Q}\),
使得\(\vb\Lambda \defeq \vb{Q}^T \vb{A} \vb{Q}\)是一个对角矩阵.
接下来我只需要证明存在一个旋转矩阵\(\vb{R}\)和一个正交矩阵\(\vb{P}\),
使得\(\vb{R} \vb{P} = \vb{Q}\)
且\(\vb{R}^T \vb{A} \vb{R}\)是对角矩阵
(这里之所以引入矩阵\(\vb{P}\),
是因为旋转矩阵\(\vb{R}\)的行列式为\(
	\DeterminantA{\vb{R}}
	= \cos^2\theta + \sin^2\theta
	= 1
\),
但是\(\vb{Q}\)作为一般的正交矩阵,
根据\cref{equation:正交矩阵.正交矩阵的行列式},
我们只能保证\(\vb{Q}\)的行列式的绝对值为\(
	\abs{
		\DeterminantA{\vb{Q}}
	}
	= 1
\),
也就是说\(\vb{Q}\)的行列式可正可负,
为了在\(\vb{R}\)与\(\vb{Q}\)之间构造一个等式,
就必须额外添上一个矩阵\(\vb{P}\)作为因子,
使\(\sgn\DeterminantA{\vb{Q}} = \sgn\DeterminantA{\vb{P}}\),
像这样,根据\cref{theorem:行列式.矩阵乘积的行列式},
\(
	\DeterminantA{\vb{R} \vb{P}}
	= \DeterminantA{\vb{R}} \DeterminantA{\vb{P}}
	= \sgn\DeterminantA{\vb{P}}
	= \sgn\DeterminantA{\vb{Q}}
	= \DeterminantA{\vb{Q}}
\);
之所以\(\vb{P}\)必须是正交矩阵,
是因为我们考虑到\(
	\vb{P} (\vb{Q}^T \vb{A} \vb{Q}) \vb{P}^T
	= (\vb{P} \vb{P}^T) \vb{R}^T \vb{A} \vb{R} (\vb{P} \vb{P}^T)
	= \vb{R}^T \vb{A} \vb{R}
\)).
下面我们按照\(\DeterminantA{\vb{Q}}\)的符号,分两种情况讨论:\begin{itemize}
	\item 当\(\DeterminantA{\vb{Q}} > 0\)时,
	取\(
		\vb{R} \defeq \vb{Q},
		\vb{P} \defeq \vb{E}
	\),
	显然有\(\vb{R} \vb{P} = \vb{Q}\)成立,
	并且\(\vb{R}^T \vb{A} \vb{R}\)是对角矩阵;

	\item 当\(\DeterminantA{\vb{Q}} < 0\)时,
	取\(
		\vb{P} \defeq \begin{bmatrix}
			1 & 0 \\
			0 & -1
		\end{bmatrix},
		\vb{R} \defeq \vb{Q} \vb{P}
	\),
	则\begin{equation*}
		\vb{R} \vb{P}
		= (\vb{Q} \vb{P}) \vb{P}
		= \vb{Q} \vb{P}^2
		= \vb{Q}
		\begin{bmatrix}
			1^2 & 0 \\
			0 & (-1)^2
		\end{bmatrix}
		= \vb{Q},
	\end{equation*}
	并且,由\(
		\vb{P}^T
		\begin{bmatrix}
			a & 0 \\
			0 & b
		\end{bmatrix}
		\vb{P}
		= \begin{bmatrix}
			a & 0 \\
			0 & b
		\end{bmatrix}
	\)
	可知\(
		\vb{R}^T \vb{A} \vb{R}
		= \vb{P}^T (\vb{Q}^T \vb{A} \vb{Q}) \vb{P}
	\)还是对角矩阵.
\end{itemize}
综上所述,对于任意一个实对称矩阵\(\vb{A}\),
一定存在实数\(\theta\),
使得旋转矩阵\(
	\vb{R} = \begin{bmatrix}
		\cos\theta & -\sin\theta \\
		\sin\theta & \cos\theta
	\end{bmatrix}
\)
满足\(\vb{R}^T \vb{A} \vb{R}\)是对角矩阵.

在证明\(\theta\)的存在性以后,
我们终于可以尝试具体解出\(\theta\)的取值了.
假设\(
	\vb{A} = \begin{bmatrix}
		a & b \\
		b & c
	\end{bmatrix}
\),
%@Mathematica: A = ({ {a, b}, {b, c} })
%@Mathematica: R = {{Cos[t], -Sin[t]}, {Sin[t], Cos[t]}}
那么\begin{equation*}
%@Mathematica: Transpose[R].A.R // FullSimplify
	\vb{R}^T \vb{A} \vb{R}
	= \begin{bmatrix}
		a \cos^2\theta + c \sin^2\theta + b \sin2\theta
		& b \cos2\theta + (c-a) \cos\theta \sin\theta \\
		b \cos2\theta + (c-a) \cos\theta \sin\theta
		& c \cos^2\theta - 2 b \cos\theta \sin\theta + a \sin^2\theta
	\end{bmatrix},
\end{equation*}
%@Mathematica: (Transpose[R].A.R)[[1, 2]] // TrigReduce
从而\(
	b \cos2\theta - \frac{a-c}{2} \sin2\theta
	= 0
\),
因此\begin{equation*}
	\theta = \frac12 \arctan\frac{2b}{a-c}
	\quad\text{或}\quad
	\theta = \frac12 \arccot\frac{a-c}{2b}.
\end{equation*}

如果我们将\(\vb{R}\)按列分块为\((\vb\eta_1,\vb\eta_2)\),
则由\begin{equation*}
	\vb{R}^T \vb{A} \vb{R}
	= \begin{bmatrix}
		a'_{11} & 0 \\
		0 & a'_{22}
	\end{bmatrix}
\end{equation*}
可得\begin{equation*}
%@see: 《解析几何》(丘维声) P147 (1.5)
	\vb{A} (\vb\eta_1,\vb\eta_2)
	= (\vb\eta_1,\vb\eta_2)
	\begin{bmatrix}
		a'_{11} & 0 \\
		0 & a'_{22}
	\end{bmatrix}
\end{equation*},
即\begin{equation*}
	(\vb{A} \vb\eta_1,\vb{A} \vb\eta_2)
	= (a'_{11} \vb\eta_1,a'_{22} \vb\eta_2),
\end{equation*}
也即\begin{equation*}
%@see: 《解析几何》(丘维声) P148 (1.6)
	\vb{A} \vb\eta_1
	= a'_{11} \vb\eta_1,
	\qquad
	\vb{A} \vb\eta_2
	= a'_{22} \vb\eta_2,
\end{equation*}
由此可见\(\vb{R}\)的列向量是\(\vb{A}\)的属于某个特征值的特征向量.

假设\(\vb{A}\)的两个特征值相同,
那么\(\vb{A}\)的属于特征值\(\lambda\)的特征子空间\(V_\lambda\)就是坐标平面\(\mathbb{R}^2\).

假设\(\lambda_1,\lambda_2\)是\(\vb{A}\)的两个不同的特征值,
\(\vb\gamma_1\)是\(\vb{A}\)的属于特征值\(\lambda_1\)的特征向量,
\(\vb\gamma_2\)是\(\vb{A}\)的属于特征值\(\lambda_2\)的特征向量,
则\begin{equation*}
	\vb{A} \vb\gamma_1 = \lambda_1 \vb\gamma_1,
	\qquad
	\vb{A} \vb\gamma_2 = \lambda_2 \vb\gamma_2,
\end{equation*}
从而有\begin{align*}
	\vb\gamma_1^T \vb{A} \vb\gamma_2 &= \lambda_2 \vb\gamma_1^T \vb\gamma_2, \\
	\vb\gamma_2^T \vb{A} \vb\gamma_1 &= \lambda_1 \vb\gamma_2^T \vb\gamma_1
									= \lambda_1 (\vb\gamma_2^T \vb\gamma_1)^T
									= \lambda_1 \vb\gamma_1^T \vb\gamma_2.
\end{align*}
由于\begin{equation*}
	\vb\gamma_1^T \vb{A} \vb\gamma_2
	= (\vb\gamma_1^T \vb{A} \vb\gamma_2)^T
	= \vb\gamma_2^T \vb{A}^T = \vb\gamma_1
	= \vb\gamma_2^T \vb{A} \vb\gamma_1,
\end{equation*}
因此\(
	\lambda_2 \vb\gamma_1^T \vb\gamma_2
	= \lambda_1 \vb\gamma_1^T \vb\gamma_2
\),
从而有\(
	(\lambda_2 - \lambda_1) \vb\gamma_1^T \vb\gamma_2 = 0
\).
因为\(\lambda_2 \neq \lambda_1\),
所以\(\vb\gamma_1^T \vb\gamma_2 = 0\),
即\(\vb\gamma_1\)与\(\vb\gamma_2\)正交.
显然\(\Span\{\vb\gamma_1,\vb\gamma_2\} = \mathbb{R}^2\).
这就说明\(\vb{A}\)的属于特征值\(\lambda_1\)的特征子空间\(V_{\lambda_1}\)
和\(\vb{A}\)的属于特征值\(\lambda_2\)的特征子空间\(V_{\lambda_2}\)互为正交补.

在确定旋转矩阵\(\vb{R}\)的待定参数\(\theta\)以后,
作坐标变换\begin{equation*}
%@see: 《解析几何》(丘维声) P150 (1.8)
	\begin{bmatrix}
		x \\ y
	\end{bmatrix}
	= \vb{R}
	\begin{bmatrix}
		x' \\ y'
	\end{bmatrix},
\end{equation*}
则\(Ox'y'\)是右手直角坐标系,
上述坐标变换是转轴.
在\(Ox'y'\)中,
曲线\(S\)的方程的二次项部分的矩阵为\begin{equation*}
	\vb{R}^T \vb{A} \vb{R}
	= \begin{bmatrix}
		\lambda_1 & 0 \\
		0 & \lambda_2
	\end{bmatrix},
\end{equation*}
从而\(S\)的新方程的二次项部分为\begin{equation*}
	\lambda_1 (x')^2 + \lambda_2 (y')^2,
\end{equation*}
一次项部分为\begin{equation*}
%@see: 《解析几何》(丘维声) P150 (1.9)
	2 (\vb{R}^T \vb\delta)^T
	\begin{bmatrix}
		x' \\ y'
	\end{bmatrix}
	= 2 \vb\delta^T \vb{R}
	\begin{bmatrix}
		x' \\ y'
	\end{bmatrix}
	= 2 (\vb\delta^T \vb\eta_1,\vb\delta^T \vb\eta_2)
	\begin{bmatrix}
		x' \\ y'
	\end{bmatrix}
	= 2 \vb\delta^T \vb\eta_1 x'
		+ 2 \vb\delta^T \vb\eta_2 y'.
\end{equation*}
记\(
	a'_1 \defeq \vb\delta^T \vb\eta_1,
	a'_2 \defeq \vb\delta^T \vb\eta_2
\),
则\(S\)在\(Ox'y'\)中的方程为\begin{equation*}
%@see: 《解析几何》(丘维声) P150 (1.10)
	\lambda_1 (x')^2 + \lambda_2 (y')^2 + 2 a'_1 x' + 2 a'_2 y' + a_ 0 = 0.
\end{equation*}

\begin{example}
%@see: 《解析几何》(丘维声) P150 例1.1
设在右手直角坐标系\(Oxy\)中,
二次曲线\(S\)的方程为\begin{equation*}
	5x^2 + 4xy + 2y^2 - 24x - 12y + 18 = 0.
\end{equation*}
作转轴消去交叉项,写出\(S\)在转轴后的新方程.
\begin{solution}
\(S\)在\(Oxy\)中的方程的二次项部分的矩阵为\begin{equation*}
	\vb{A} \defeq \begin{bmatrix}
		5 & 2 \\
		2 & 2
	\end{bmatrix}.
\end{equation*}
%@Mathematica: A = {{5, 2}, {2, 2}}
%@Mathematica: Eigenvalues[A]
%@Mathematica: NullSpace[6 IdentityMatrix[2] - A]
%@Mathematica: NullSpace[IdentityMatrix[2] - A]
\(\vb{A}\)的特征值为\(
	\lambda_1 = 6,
	\lambda_2 = 1
\).

对于特征值\(\lambda_1 = 6\),
解齐次线性方程组\((\lambda_1 \vb{E} - \vb{A}) \vb{x} = \vb0\),
它的基础解系为\(\vb\gamma_1 = (2,1)^T\).
令\begin{equation*}
	\vb\eta_1
	\defeq \frac1{\VectorLengthA{\vb\gamma_1}} \vb\gamma_1
	= \begin{bmatrix}
		\frac2{\sqrt5} & \frac1{\sqrt5}
	\end{bmatrix}^T.
\end{equation*}

对于特征值\(\lambda_2 = 1\),
解齐次线性方程组\((\lambda_2 \vb{E} - \vb{A}) \vb{x} = \vb0\),
它的基础解系为\(\vb\gamma_2 = (-1,2)^T\).
令\begin{equation*}
	\vb\eta_2
	\defeq \frac1{\VectorLengthA{\vb\gamma_2}} \vb\gamma_2
	= \begin{bmatrix}
		-\frac1{\sqrt5} & \frac2{\sqrt5}
	\end{bmatrix}^T.
\end{equation*}

于是\begin{equation*}
	\vb{R} \defeq \begin{bmatrix}
		\frac2{\sqrt5} & -\frac1{\sqrt5} \\
		\frac1{\sqrt5} & \frac2{\sqrt5}
	\end{bmatrix}
\end{equation*}
是正交矩阵,且\(\VectorLengthA{\vb{R}} = 1\).
作转轴\begin{equation*}
	\begin{bmatrix}
		x \\ y
	\end{bmatrix}
	= \vb{R}
	\begin{bmatrix}
		x' \\ y'
	\end{bmatrix}.
\end{equation*}
\(S\)在\(Ox'y'\)中的方程的一次项系数的一半分别为\begin{gather*}
	a'_1 \defeq \vb\delta^T \vb\eta_1
	= \begin{bmatrix}
		-12 & -6
	\end{bmatrix}
	\begin{bmatrix}
		\frac2{\sqrt5} \\ \frac1{\sqrt5}
	\end{bmatrix}
	= -6\sqrt5, \\
	a'_2 \defeq \vb\delta^T \vb\eta_2
	= \begin{bmatrix}
		-12 & -6
	\end{bmatrix}
	\begin{bmatrix}
		-\frac1{\sqrt5} & \frac2{\sqrt5}
	\end{bmatrix}
	= 0,
\end{gather*}
于是\(S\)在\(Ox'y'\)中的方程为\begin{equation*}
	6(x')^2 + (y')^2 - 12\sqrt5 x' + 18 = 0.
\end{equation*}
\end{solution}
\end{example}
