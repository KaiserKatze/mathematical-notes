\section{二次曲线方程的化简及其类型}
为了判别在右手直角坐标系 I \([O;\vb{e}_1,\vb{e}_2]\)中的
二次方程 \labelcref{equation:二次曲线方程.平面二次曲线的一般方程}
当系数取各种值时,
它能表示哪一种曲线,
容易想到的办法是:
作坐标变换,
使得二次曲线 \labelcref{equation:二次曲线方程.平面二次曲线的一般方程}
在新坐标系中的方程比较简单,易于辨认出它表示的曲线的种类.
由于任意一个直角坐标变换都可以由移轴公式和转轴公式得到,
所以我们首先来研究在转轴时
二次方程 \labelcref{equation:二次曲线方程.平面二次曲线的一般方程} 的系数变化规律.

\subsection{转轴消去交叉项}
设 II \([O';\vb{e}_1',\vb{e}_2']\)是由 I 经过转轴得到的,
转角为\(\theta\),
则 I 到 II 的点的坐标变换公式为
\cref{equation:解析几何.平面坐标系的点的右手直角坐标变换公式I到II.矩阵形式3}
\begin{equation}
%@see: 《解析几何》(丘维声) P146 (1.1)
	\begin{bmatrix}
		x \\ y
	\end{bmatrix}
	= \begin{bmatrix}
		\cos\theta & -\sin\theta \\
		\sin\theta & \cos\theta
	\end{bmatrix}
	\begin{bmatrix}
		x' \\ y'
	\end{bmatrix}.
\end{equation}
令\begin{equation*}
	\vb{T} \defeq \begin{bmatrix}
		\cos\theta & -\sin\theta \\
		\sin\theta & \cos\theta
	\end{bmatrix},
	\qquad
	\vb\alpha \defeq \begin{bmatrix}
		x \\ y
	\end{bmatrix},
	\qquad
	\vb\alpha' \defeq \begin{bmatrix}
		x' \\ y'
	\end{bmatrix},
\end{equation*}
则转轴公式 \labelcref{equation:解析几何.平面坐标系的点的右手直角坐标变换公式I到II.矩阵形式3}
可以写成\begin{equation*}
	\vb\alpha = \vb{T} \vb\alpha'.
\end{equation*}

由转轴公式 \labelcref{equation:解析几何.平面坐标系的点的右手直角坐标变换公式I到II.矩阵形式3}
可以得到\begin{equation*}
	\begin{bmatrix}
		x \\ y \\ 1
	\end{bmatrix}
	= \begin{bmatrix}
		\cos\theta & -\sin\theta & 0 \\
		\sin\theta & \cos\theta & 0 \\
		0 & 0 & 1
	\end{bmatrix}
	\begin{bmatrix}
		x' \\ y' \\ 1
	\end{bmatrix},
\end{equation*}
即\begin{equation*}
%@see: 《解析几何》(丘维声) P147 (1.2)
	\begin{bmatrix}
		\vb\alpha \\ 1
	\end{bmatrix}
	= \begin{bmatrix}
		\vb{T} & \vb0 \\
		\vb0 & 1
	\end{bmatrix}
	\begin{bmatrix}
		\vb\alpha' \\ 1
	\end{bmatrix}.
\end{equation*}
把上式代入方程 \labelcref{equation:二次曲线方程.平面二次曲线的一般方程.矩阵形式} 中,
可以得到二次曲线 \labelcref{equation:二次曲线方程.平面二次曲线的一般方程} 的新方程:
\begin{equation}
%@see: 《解析几何》(丘维声) P147 (1.3)
	\begin{bmatrix}
		(\vb\alpha')^T & 1
	\end{bmatrix}
	\begin{bmatrix}
		\vb{T}^T & \vb0 \\
		\vb0 & 1
	\end{bmatrix}
	\begin{bmatrix}
		\vb{A} & \vb\delta \\
		\vb\delta^T & a_0
	\end{bmatrix}
	\begin{bmatrix}
		\vb{T} & \vb0 \\
		\vb0 & 1
	\end{bmatrix}
	\begin{bmatrix}
		\vb\alpha' \\ 1
	\end{bmatrix}
	= 0,
\end{equation}
即\begin{equation*}
%@see: 《解析几何》(丘维声) P147 (1.3)'
	\begin{bmatrix}
		(\vb\alpha')^T & 1
	\end{bmatrix}
	\begin{bmatrix}
		\vb{T}^T \vb{A} \vb{T} & \vb{T}^T \vb\delta \\
		\vb\delta^T \vb{T} & a_0
	\end{bmatrix}
	\begin{bmatrix}
		\vb\alpha' \\ 1
	\end{bmatrix}
	= 0.
\end{equation*}
由于\begin{equation*}
	(\vb{T}^T \vb{A} \vb{T})^T
	= \vb{T}^T \vb{A}^T \vb{T}
	= \vb{T}^T \vb{A} \vb{T},
\end{equation*}
所以\(\vb{T}^T \vb{A} \vb{T}\)仍是对称矩阵.
于是新方程的二次项部分\(\phi'(x',y')\)的矩阵为\(\vb{T}^T \vb{A} \vb{T}\),
一次项系数的一半组成的列是\(\vb{T}^T \vb\delta\),
常数项是\(a_0\).
这说明,经过转轴,
新方程的二次项系数只与原方程的二次项系数及转角\(\theta\)有关,
新方程的一次项系数只与原方程的一次项系数及转角\(\theta\)有关,
常数项不变.

转轴后二次曲线\(S\)的新方程的二次项部分的矩阵为\(\vb{T}^T \vb{A} \vb{T}\).
新方程中不出现交叉项(即\(x' y'\)项)
当且仅当\begin{equation*}
%@see: 《解析几何》(丘维声) P147 (1.4)
	\vb{T}^T \vb{A} \vb{T}
	= \begin{bmatrix}
		a'_{11} & 0 \\
		0 & a'_{22}
	\end{bmatrix}.
\end{equation*}
设\(\vb{T}\)的列向量组为\(\vb\eta_1,\vb\eta_2\).
由于\(\vb{T}^{-1} = \vb{T}^T\),
因此上式两边左乘\(\vb{T}\),得\begin{equation*}
%@see: 《解析几何》(丘维声) P147 (1.5)
	\vb{A} (\vb\eta_1,\vb\eta_2)
	= (\vb\eta_1,\vb\eta_2)
	\begin{bmatrix}
		a'_{11} & 0 \\
		0 & a'_{22}
	\end{bmatrix},
\end{equation*}
从而\begin{equation*}
	(\vb{A} \vb\eta_1,\vb{A} \vb\eta_2)
	= (a'_{11} \vb\eta_1,a'_{22} \vb\eta_2),
\end{equation*}
即\begin{equation*}
%@see: 《解析几何》(丘维声) P148 (1.6)
	\vb{A} \vb\eta_1
	= a'_{11} \vb\eta_1,
	\qquad
	\vb{A} \vb\eta_2
	= a'_{22} \vb\eta_2.
\end{equation*}
从上述讨论得出,
转轴后的新方程中不出现交叉项
当且仅当\(\vb{A}\)有两个特征值\(a'_{11}\)和\(a'_{22}\)(这两个特征值可以相等),
并且\(\vb{A}\)有两个正交的特征向量.
