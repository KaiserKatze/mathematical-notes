\section{二次曲线方程的切线,双曲线的渐近线}\label{section:解析几何.二次曲线方程的切线}
\subsection{二次曲线经过其上一点的切线}
\begin{definition}
%@see: 《解析几何》(丘维声) P185 定义5.1
如果直线\(l\)与二次曲线\(S\)有两个重合的交点
或者\(l\)在\(S\)上,
则称“\(l\)是\(S\)的\DefineConcept{切线}”,
\(l\)与\(S\)的交点称为\DefineConcept{切点}.
\end{definition}

首先来讨论
怎样求经过二次曲线\(S\)上的一点\(M_0(x_0,y_0)\)的切线\(l\).
设切线\(l\)的方向是\((\mu,\nu)\),
则\(l\)的参数方程是\begin{equation*}
%@see: 《解析几何》(丘维声) P170 (3.1)
	\left\{ \begin{array}{l}
		x = x_0 + \mu t, \\
		y = y_0 + \nu t,
	\end{array} \right.
	\quad -\infty < t < +\infty.
\end{equation*}
代入二次曲线\(S\)的方程 \labelcref{equation:二次曲线方程.平面二次曲线的一般方程} 中,
得到方程 \labelcref{equation:二次曲线方程的对称中心.在曲线方程中代入直线方程},
即\begin{equation}
%@see: 《解析几何》(丘维声) P170 (3.2)
%@see: 《解析几何》(丘维声) P185 (5.1)
	\phi(\mu,\nu) t^2
	+ 2 (
		F_1(x_0,y_0) \mu
		+ F_2(x_0,y_0) \nu
	) t
	+ F(x_0,y_0)
	= 0,
\end{equation}
既然\(l\)是\(S\)的切线,
因此,要么\(l\)与\(S\)有两个重合的交点,要么\(l\)在\(S\)上.
结合\cref{section:解析几何.直线与二次曲线的相关位置} 的讨论,我们可以得出以下结论.
在前一种情形中,有\(\phi(\mu,\nu) \neq 0\),
且关于\(t\)的一元二次方程 \labelcref{equation:二次曲线方程的对称中心.在曲线方程中代入直线方程} 的
判别式\(\Delta = 0\).
由于\(M_0(x_0,y_0)\)在\(S\)上,
于是点\(M_0\)满足方程\(F(x,y) = 0\),
即\(F(x_0,y_0) = 0\),
从而有\begin{align*}
	\Delta
	&= 4 (F_1(x_0,y_0) \mu + F_2(x_0,y_0) \nu)^2 - 4 \phi(\mu,\nu) F(x_0,y_0) \\
	&= 4 (F_1(x_0,y_0) \mu + F_2(x_0,y_0) \nu)^2.
\end{align*}
因此\begin{equation}\label{equation:二次曲线方程的切线.切向量方程}
%@see: 《解析几何》(丘维声) P185 (5.2)
	F_1(x_0,y_0) \mu + F_2(x_0,y_0) \nu = 0.
\end{equation}
在后一种情形中,有\(\phi(\mu,\nu) = 0\),
%@see: 《解析几何》(丘维声) P185 (5.3)
且同样成立方程 \labelcref{equation:二次曲线方程的切线.切向量方程}.
总之,如果经过二次曲线\(S\)上的点\(M_0(x_0,y_0)\)的直线\(l\)是\(S\)的切线,
则\(l\)的方向\((\mu,\nu)\)应满足方程 \labelcref{equation:二次曲线方程的切线.切向量方程}.
反之,如果这样的直线\(l\)的方向\((\mu,\nu)\)满足方程 \labelcref{equation:二次曲线方程的切线.切向量方程},
那么,要么\(l\)在\(S\)上(即\(\phi(\mu,\nu) = 0\)),
要么\(l\)与\(S\)有两个重合的交点(\(\phi(\mu,\nu) \neq 0\)),
从而\(l\)是\(S\)的切线.

\begin{theorem}
%@see: 《解析几何》(丘维声) P186 定理5.1
设二次曲线\(S\)的方程为 \labelcref{equation:二次曲线方程.平面二次曲线的一般方程},
\(M_0(x_0,y_0)\)是\(S\)上的一个点.
\begin{itemize}
	\item 如果\(F_1(x_0,y_0)\)与\(F_2(x_0,y_0)\)不全为零,
	则存在\(S\)的唯一一条切线经过\(M_0\),
	其方程为\begin{equation*}
		(x-x_0) F_1(x_0,y_0) + (y-y_0) F_2(x_0,y_0) = 0.
	\end{equation*}

	\item 如果\(F_1(x_0,y_0) = F_2(x_0,y_0) = 0\),
	则经过点\(M_0(x_0,y_0)\)的每一条直线都是\(S\)的切线.
\end{itemize}
\begin{proof}
如果\(F_1(x_0,y_0)\)与\(F_2(x_0,y_0)\)不全为零,
那么由\cref{equation:二次曲线方程的切线.切向量方程} 可得\begin{equation*}
	\frac{\mu}{\nu}
	= \frac{-F_2(x_0,y_0)}{F_1(x_0,y_0)},
\end{equation*}
因此经过二次曲线\(S\)上的点\(M_0(x_0,y_0)\)的切线\(l\)的方程为\begin{equation*}
%@see: 《解析几何》(丘维声) P186 (5.4)
	\frac{x-x_0}{-F_2(x_0,y_0)}
	= \frac{y-y_0}{F_1(x_0,y_0)}
	\quad\text{或}\quad
	(x-x_0) F_1(x_0,y_0) + (y-y_0) F_2(x_0,y_0) = 0.
\end{equation*}

如果\(F_1(x_0,y_0) = F_2(x_0,y_0) = 0\),
那么任意一个方向\((\mu,\nu)\)都满足\cref{equation:二次曲线方程的切线.切向量方程},
从而经过点\(M_0(x_0,y_0)\)的任意一条直线都是\(S\)的切线.
\end{proof}
\end{theorem}

\begin{definition}
%@see: 《解析几何》(丘维声) P186 定义5.2
设\(M_0\)是曲线\(S\)上的一个点.
如果经过\(M_0\)的每一条直线都是\(S\)的切线,
则称“点\(M_0\)是曲线\(S\)的\DefineConcept{奇异点}”
或“点\(M_0\)是曲线\(S\)的\DefineConcept{奇点}”.
\end{definition}

\subsection{二次曲线经过其外一点的切线}
现在来讨论怎么求过二次曲线外一点\(M_1(x_1,y_1)\)的切线\(l\)(如果存在的话).
此时\(l\)不可能整条直线都在\(S\)上,
因此\(l\)必与\(S\)有两个重合的交点.
设\(l\)的方向是\((\mu,\nu)\),
则它应满足\(\phi(\mu,\nu) \neq 0\),
并且\begin{equation*}
	\Delta
	= 4 (F_1(x_1,y_1) \mu + F_2(x_1,y_1) \nu)^2 - 4 \phi(\mu,\nu) F(x_1,y_1)
	= 0,
\end{equation*}
即\begin{equation}\label{equation:二次曲线方程的切线.曲线外一点1}
%@see: 《解析几何》(丘维声) P186 (5.5)
	(F_1(x_1,y_1) \mu + F_2(x_1,y_1) \nu)^2 - \phi(\mu,\nu) F(x_1,y_1)
	= 0.
\end{equation}
因为\(l\)的方程为\begin{equation*}
	\frac{x-x_1}{\mu} = \frac{y-y_1}{\nu},
\end{equation*}
所以对于切线\(l\)上的任意一个点\((x,y)\),
都有\begin{equation*}
	\frac{\mu}{\nu} = \frac{x-x_1}{y-y_1}.
\end{equation*}
不妨就取\(\mu = x-x_1, \nu=y-y_1\),
代入\cref{equation:二次曲线方程的切线.曲线外一点1},
得\begin{equation}\label{equation:二次曲线方程的切线.曲线外一点2}
%@see: 《解析几何》(丘维声) P187 (5.6)
	(F_1(x_1,y_1) (x-x_1) + F_2(x_1,y_1) (y-y_1))^2 - \phi(x-x_1,y-y_1) F(x_1,y_1)
	= 0.
\end{equation}
上式左端是关于\((x-x_1),(y-y_1)\)的二次齐次多项式或零多项式.
\begin{enumerate}
	\item 假设\cref{equation:二次曲线方程的切线.曲线外一点2} 左端是
	关于\((x-x_1),(y-y_1)\)的二次齐次多项式,
	不是零多项式.
	\begin{enumerate}
		\item 假设\cref{equation:二次曲线方程的切线.曲线外一点2} 左端
		可以分解成\((x-x_1),(y-y_1)\)的两个实系数一次齐次多项式的乘积.
		此时从\cref{equation:二次曲线方程的切线.曲线外一点2} 得到
		一对相交或重合的直线\(l_1,l_2\).
		\begin{enumerate}
			\item 若\(l_i\)的方向\((\mu_i,\nu_i)\)满足\(\phi(\mu_i,\nu_i) \neq 0\),
			则\(l_i\)是经过点\(M_1\)的\(S\)的切线.
			\item 若\(l_i\)的方向\((\mu_i,\nu_i)\)满足\(\phi(\mu_i,\nu_i) = 0\),
			则经过点\(M_1\)的\(S\)的切线不存在.
		\end{enumerate}
		\item 假设\cref{equation:二次曲线方程的切线.曲线外一点2} 左端
		不可以分解成\((x-x_1),(y-y_1)\)的两个实系数一次齐次多项式.
		由于直线\(l\)上任意一点的坐标
		都是\cref{equation:二次曲线方程的切线.曲线外一点2} 左端的多项式的零点,
		因此这种情况不可能出现.
	\end{enumerate}
	\item 假设\cref{equation:二次曲线方程的切线.曲线外一点2} 的左端是零多项式,
	那么有\begin{gather*}
		(F_1(x_1,y_1))^2 - a_{11} F(x_1,y_1) = 0, \\
		(F_2(x_1,y_1))^2 - a_{22} F(x_1,y_1) = 0, \\
		F_1(x_1,y_1) F_2(x_1,y_1) - a_{12} F(x_1,y_1) = 0.
	\end{gather*}
	由于\(F(x_1,y_1) \neq 0\),
	因此可以推出\(a_{11} a_{22} - a_{12}^2 = 0\),
	即\(I_2 = 0\),
	从而\(S\)是抛物型曲线.
	\begin{enumerate}
		\item 假设\(S\)是抛物线.
		\begin{enumerate}
			\item 如果点\(M_1\)在抛物线外,
			则经过点\(M_1\)有\(S\)的两条切线.
			\item 如果点\(M_1\)在抛物线内,
			则不存在经过点\(M_1\)的\(S\)的切线.
		\end{enumerate}
		\item 假设\(S\)是一对平行直线或虚平行直线,
		则不存在经过点\(M_1\)的\(S\)的切线.
		\item 假设\(S\)是一对重合直线,
		则经过点\(M_1\)与\(S\)相交的每一条直线都是\(S\)的切线.
	\end{enumerate}
\end{enumerate}

\subsection{二次曲线的法线}
\begin{definition}
%@see: 《解析几何》(丘维声) P188 定义5.3
经过曲线\(S\)上一点\(M_0\)且垂直于过该点的切线的直线
称为“曲线\(S\)在点\(M_0\)的\DefineConcept{法线}”.
\end{definition}

设\(M_0(x_0,y_0)\)是二次曲线\(S\)上一点.
当\(F_1(x_0,y_0)\)与\(F_2(x_0,y_0)\)不全为零时,
\begin{equation*}
	\begin{bmatrix}
		-F_2(x_0,y_0) \\
		F_1(x_0,y_0)
	\end{bmatrix},
\end{equation*}
是经过\(M_0\)的切线的一个方向,
从而\begin{equation*}
	\begin{bmatrix}
		F_1(x_0,y_0) \\
		F_2(x_0,y_0)
	\end{bmatrix},
\end{equation*}
是过点\(M_0\)的法线的一个方向,
因此过点\(M_0\)的法线方程为\begin{equation}
	\frac{x-x_0}{F_1(x_0,y_0)}
	= \frac{y-y_0}{F_2(x_0,y_0)}.
\end{equation}

\subsection{双曲线的渐近线}
\begin{definition}
%@see: 《解析几何》(丘维声) P188 定义5.4
沿渐进方向且与双曲线\(S\)没有交点的直线
称为“双曲线\(S\)的一条渐近线”.
\end{definition}

设\((\mu,\nu)\)是双曲线\(S\)的渐进方向,
\(l\)是方向为\((\mu,\nu)\)的渐近线.
因为\(l\)与\(S\)无交点,
所以\(l\)上任意一点\(M(x,y)\)满足方程\begin{equation}
%@see: 《解析几何》(丘维声) P188 (5.7)
	\mu F_1(x,y) + \nu F_2(x,y) = 0.
\end{equation}
上式就是双曲线\(S\)的渐近线\(l\)的方程.

因为双曲线的中心\(O_1\)的坐标\((x_1,y_1)\)满足\begin{equation*}
	F_1(x_1,y_1) = 0,
	\qquad
	F_2(x_1,y_1) = 0,
\end{equation*}
所以\(O_1\)必在渐近线\(l\)上.
于是,双曲线的渐近线一定是经过双曲线的中心且方向为双曲线的渐进方向的一条直线.

\begin{remark}
最后我们指出,
\cref{section:解析几何.二次曲线方程的对称中心,section:解析几何.二次曲线方程的直径和对称轴,section:解析几何.二次曲线方程的切线} 的内容,
除了涉及对称轴和法线的内容以外,
其余内容均可在仿射坐标系中进行讨论,
并且可以得到同样的有关结论.
\end{remark}
