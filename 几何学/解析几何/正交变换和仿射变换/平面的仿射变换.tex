\section{平面的仿射变换}

\begin{definition}
%@see: 《解析几何》(丘维声) P207 定义3.1
如果平面上的一个点变换\(\tau\)使得对应线段之比为一个非零常数\(k\),
那么称“\(\tau\)是一个\DefineConcept{相似变换}”,
其中\(k\)称为\DefineConcept{相似比}.
\end{definition}

容易看出,相似变换把线段映成线段,
把直线映成直线,
把射线映成射线.
相似变换保持任意两条线段的比值不变.

\begin{definition}
%@see: 《解析几何》(丘维声) P208 定义3.2
如果有一个相似比为\(k\)的相似变换\(\tau\),
使得一个图形\(E\)的像是图形\(E'\),
那么称“\(E\)和\(E'\)是\DefineConcept{相似图形}”,
其中\(k\)称为这两个图形的\DefineConcept{相似比}.
\end{definition}

\begin{definition}
%@see: 《解析几何》(丘维声) P208 定义3.3
平面上取定一个点\(O\),
把平面上每一个点\(P\)对应到点\(P'\),
使得\(\vec{OP'} = k \vec{OP}\),
其中\(k\)是一个非零常数,
点\(O\)对应到它自身,
平面上的这个点变换称为\DefineConcept{位似变换},
其中点\(O\)称为\DefineConcept{位似中心},
\(k\)称为\DefineConcept{位似比}.
\end{definition}

\begin{definition}
%@see: 《解析几何》(丘维声) P208 定义3.4
如果有一个位似比为\(k\)的位似变换,
使得图形\(E\)的像是图形\(E'\),
那么称“\(E\)和\(E'\)是\DefineConcept{位似图形}”,
其中\(k\)称为这两个图形的\DefineConcept{位似比}.
\end{definition}

\begin{proposition}
%@see: 《解析几何》(丘维声) P208 命题3.1
位似比为\(k\)的位似变换
把线段映成线段,
且像线段与原线段的长度之比等于\(\abs{k}\).
\end{proposition}

\begin{corollary}
%@see: 《解析几何》(丘维声) P209 推论3.1
位似比为\(k\)的位似变换是相似比为\(\abs{k}\)的相似变换.
\end{corollary}

\begin{proposition}
%@see: 《解析几何》(丘维声) P209 命题3.2
位似变换是可逆变换.
位似比为\(k\)、位似中心为\(O\)的位似变换\(\tau\)的逆变换
是位似比为\(k^{-1}\)、位似中心为\(O\)的位似变换.
\end{proposition}

\begin{proposition}
%@see: 《解析几何》(丘维声) P210 命题3.3
相似变换可以分解成一个位似变换与一个正交点变换的乘积.
\end{proposition}

\begin{proposition}
%@see: 《解析几何》(丘维声) P210 命题3.4
相似变换是可逆变换.
\end{proposition}

\begin{definition}
%@see: 《解析几何》(丘维声) P211 定义3.5
平面上给定一条直线\(l\)和一个非零向量\(\vb{d}\),
其中\(\vb{d}\)不是\(l\)的方向向量.
如果平面上的一个变换\(\tau\)
把每一个点\(P\)对应到点\(P'\),
使得\begin{itemize}
	\item \(\vec{PP'}\)与\(\vb{d}\)共线;
	\item 点\(P'\)与点\(P\)在\(l\)的同侧;
	\item \(\LineSegmentLength{AP'} = k \LineSegmentLength{AP}\),
	其中\(A\)是直线\(PP'\)与\(l\)的交点,
	\(k\)是非零常数,
\end{itemize}
那么称变换\(\tau\)是沿方向\(\vb{d}\)(或\(-\vb{d}\))向着直线\(l\)的\DefineConcept{压缩变换},
其中\(l\)称为\DefineConcept{压缩轴},
\(\vb{d}\)称为\DefineConcept{压缩方向},
\(k\)称为\DefineConcept{压缩系数}.
当\(\vb{d}\)与\(l\)垂直时,称\(\tau\)是\DefineConcept{正压缩变换}.
当\(k>1\)时,也称\(\tau\)是\DefineConcept{拉伸变换}.
\end{definition}

\begin{proposition}
%@see: 《解析几何》(丘维声) P211 命题3.5
在向着\(y\)轴且压缩系数为\(k\)的正压缩变换下,
平面上曲线\(F(x,y) = 0\)的像是曲线\(F(k^{-1}x,y) = 0\).
\end{proposition}

\begin{corollary}
%@see: 《解析几何》(丘维声) P212 推论3.2
在向着\(y\)轴且压缩系数为\(k\)的正压缩变换下,
函数\(y = f(x)\)的图像的像是\(y = f(k^{-1} x)\)的图像.
\end{corollary}

\begin{proposition}
%@see: 《解析几何》(丘维声) P212 命题3.6
压缩变换把共线三点映成共线三点,
把不共线三点映成不共线三点.
\end{proposition}

\begin{proposition}
%@see: 《解析几何》(丘维声) P213 命题3.7
压缩变换是可逆变换.
\end{proposition}

\begin{definition}
%@see: 《解析几何》(丘维声) P213 定义3.6
平面上给定一条直线\(l\)和\(l\)的一个方向向量\(\vb{v}\).
如果平面上的一个变换\(\sigma\)
把\(l\)上的每一点映成它自身,
把不在\(l\)上的点映成点\(P'\),
使得\begin{itemize}
	\item \(PP'\)与\(l\)平行;
	\item 对在\(l\)的一侧的点\(P\),有\(\vec{PP'}\)与\(\vec{v}\)同向;
	对在\(l\)的另一侧的点\(Q\),有\(\vec{QQ'}\)与\(\vec{v}\)反向;
	\item 从点\(P\)向直线\(l\)作垂线,垂足为\(M\),
	有\(\LineSegmentLength{PP'} = k \LineSegmentLength{PM}\),
	其中\(k\)是非零常数,
\end{itemize}
那么称变换\(\sigma\)是\DefineConcept{错切变换},
其中\(l\)称为\DefineConcept{错切轴},
\(k\)称为\DefineConcept{错切系数}.
\end{definition}

\begin{proposition}
%@see: 《解析几何》(丘维声) P213 命题3.8
错切变换把共线三点映成共线三点,
把不共线三点映成不共线三点.
\end{proposition}

\begin{proposition}
%@see: 《解析几何》(丘维声) P214 命题3.9
错切变换是可逆变换.
\end{proposition}

相似变换、位似变换、压缩变换和错切变换
都把共线三点映成共线三点,
并且它们都是可逆变换.
抓住它们的共同特征,我们抽象出下述概念.
\begin{definition}
%@see: 《解析几何》(丘维声) P215 定义3.7
如果平面到自身的双射\(\sigma\)把共线三点映成共线三点,
则称\(\sigma\)是平面上的一个\DefineConcept{仿射变换}.
\end{definition}

相似变换、位似变换、压缩变换和错切变换都是平面上的仿射变换.

\subsection{仿射变换的性质}
\begin{property}
%@see: 《解析几何》(丘维声) P215 性质1
仿射变换\(\sigma\)把不共线三点映成不共线三点.
\end{property}

\begin{property}
%@see: 《解析几何》(丘维声) P215 性质2
仿射变换\(\sigma\)的逆变换\(\sigma^{-1}\)也是仿射变换.
\end{property}

\begin{property}
%@see: 《解析几何》(丘维声) P216 性质3
仿射变换的乘积还是仿射变换.
\end{property}

\begin{property}
%@see: 《解析几何》(丘维声) P216 性质4
仿射变换\(\sigma\)把直线映成直线.
\end{property}

\begin{property}
%@see: 《解析几何》(丘维声) P216 性质5
仿射变换\(\sigma\)把平行直线映成平行直线.
\end{property}

\begin{lemma}
%@see: 《解析几何》(丘维声) P220 引理3.1
设映射\(f\colon \mathbb{R} \to \mathbb{R}\)满足\begin{equation*}
	f(x+y) = f(x) + f(y), \\
	f(xy) = f(x) f(y),
\end{equation*}
且\(f\)不是零映射,
那么\(f(x) = x\).
\end{lemma}

\begin{property}
%@see: 《解析几何》(丘维声) P222 性质8
仿射变换\(\sigma\)把线段映成线段.
\end{property}

\begin{property}
%@see: 《解析几何》(丘维声) P223 性质11
仿射变换\(\sigma\)保持线段的分比不变.
\end{property}

\begin{theorem}
%@see: 《解析几何》(丘维声) P223 定理3.1(仿射变换基本定理)
设\(\sigma\)是平面上的一个变换,
I \([O;\vb{d}_1,\vb{d}_2]\)是仿射坐标系,
\(
	\sigma(O) = O',
	\sigma(\vb{d}_1) = \vb{d}'_1,
	\sigma(\vb{d}_2) = \vb{d}'_2
\),
则\(\sigma\)是仿射变换,
当且仅当 II \([O';\vb{d}'_1,\vb{d}'_2]\)也是仿射坐标系,
且点\(P\)的 I 坐标等于它的像点\(P'\)的 II 坐标.
\end{theorem}

\begin{corollary}
%@see: 《解析几何》(丘维声) P223 推论3.3
如果平面上的仿射变换\(\sigma_1,\sigma_2\)把仿射坐标系 I 映成同一个仿射坐标系 II,
那么\(\sigma_1 = \sigma_2\).
\end{corollary}

\begin{corollary}
%@see: 《解析几何》(丘维声) P224 推论3.4
平面上任给两组不共线三点\(\AutoTuple{A}{3}\)和\(\AutoTuple{B}{3}\),
则存在唯一的仿射变换\(\sigma\),
使得\(
	\sigma(A_i) = B_i
	\ (i=1,2,3)
\).
\end{corollary}

\begin{corollary}
%@see: 《解析几何》(丘维声) P224 推论3.5
对平面上任给的两个仿射坐标系 I 和 II,
存在唯一的仿射变换把 I 映成 II.
\end{corollary}

\begin{theorem}
%@see: 《解析几何》(丘维声) P224 定理3.2
设\(\sigma\)是平面上的一个变换,
I \([O;\vb{d}_1,\vb{d}_2]\)是仿射坐标系,
\(
	\sigma(O) = O',
	\sigma(\vb{d}_1) = \vb{d}'_1,
	\sigma(\vb{d}_2) = \vb{d}'_2
\),
\(O',\vb{d}'_1,\vb{d}'_2\)的 I 坐标分别为\begin{equation*}
	(x_0,y_0),
	\qquad
	(a_{11},a_{21}),
	\qquad
	(a_{12},a_{22}),
\end{equation*}
点\(P\)和像点\(P'\)的 I 坐标分别为\begin{equation*}
	(x,y),
	\qquad
	(x',y'),
\end{equation*}
则\(\sigma\)是仿射变换的充分必要条件是\begin{equation*}
%@see: 《解析几何》(丘维声) P224 (3.7)
	\begin{bmatrix}
		x' \\ y'
	\end{bmatrix}
	= \begin{bmatrix}
		a_{11} & a_{12} \\
		a_{21} & a_{22}
	\end{bmatrix}
	\begin{bmatrix}
		x \\ y
	\end{bmatrix}
	+ \begin{bmatrix}
		x_0 \\ y_0
	\end{bmatrix},
\end{equation*}
且矩阵\begin{equation*}
	\begin{bmatrix}
		a_{11} & a_{12} \\
		a_{21} & a_{22}
	\end{bmatrix}
\end{equation*}
是可逆矩阵.
\end{theorem}

\subsection{仿射变换的变积系数}
正交变换保持点之间的距离不变,保持向量之间的夹角不变,从而保持图形的面积不变.
而一般的仿射变换会改变点之间的距离,会改变向量之间的夹角,从而也会改变图形的面积.
下面我们来讨论仿射变换改变图形面积的规律.

设在平面上规定了一个定向,用平面的单位法向量\(\vb{e}\)表示.
用\(S(\vb{a},\vb{b})\)表示以\(\vb{a},\vb{b}\)为邻边,
并且边界的环行方向为\(\vb{a}\)到\(\vb{b}\)的旋转方向的定向平行四边形的定向面积,
由\cref{equation:解析几何.向量外积与定向平行四边形的定向面积之间的关系} 得\begin{equation*}
	\VectorOuterProduct{\vb{a}}{\vb{b}}
	= S(\vb{a},\vb{b})
	\vb{e}.
\end{equation*}

\begin{theorem}\label{theorem:平面的仿射变换.仿射变换的变积系数}
%@see: 《解析几何》(丘维声) P225 定理3.3
设仿射变换\(\tau\)在仿射标架 I \([O;\vb{d}_1,\vb{d}_2]\)中的公式为\begin{equation*}
%@see: 《解析几何》(丘维声) P225 (3.10)
	\begin{bmatrix}
		x' \\ y'
	\end{bmatrix}
	= \begin{bmatrix}
		a_{11} & a_{12} \\
		a_{21} & a_{22}
	\end{bmatrix}
	\begin{bmatrix}
		x \\ y
	\end{bmatrix}
	+ \begin{bmatrix}
		x_0 \\ y_0
	\end{bmatrix},
\end{equation*}
对于任意不共线的向量\(\vb{a},\vb{b}\),
有\(
	\tau(\vb{a}) = \vb{a}',
	\tau(\vb{b}) = \vb{b}'
\),
则有\begin{equation*}
%@see: 《解析几何》(丘维声) P225 (3.11)
	\frac{S(a',b')}{S(a,b)}
	= \begin{vmatrix}
		a_{11} & a_{12} \\
		a_{21} & a_{22}
	\end{vmatrix}.
\end{equation*}
\end{theorem}

%@see: 《解析几何》(丘维声) P226
易知,如果仿射变换\(\tau\)在仿射坐标系 I 中的公式的系数矩阵为\(\vb{A}\),
那么\(\tau\)在仿射坐标系 II 中的公式的系数矩阵为\(\vb{H}^{-1} \vb{A} \vb{H}\),
其中\(\vb{H}\)是 I 到 II 的过渡矩阵.
因为\begin{equation*}
	\DeterminantA{\vb{H}^{-1} \vb{A} \vb{H}}
	= \DeterminantA{\vb{H}^{-1}}
	\DeterminantA{\vb{A}}
	\DeterminantA{\vb{H}}
	= \DeterminantA{\vb{H}}^{-1}
	\DeterminantA{\vb{A}}
	\DeterminantA{\vb{H}}
	= \DeterminantA{\vb{A}},
\end{equation*}
所以仿射变换\(\tau\)的公式中系数矩阵的行列式与仿射标架的选择无关.

\begin{definition}
%@see: 《解析几何》(丘维声) P226 定义3.8
仿射变换\(\tau\)的公式中系数矩阵的行列式
称为“仿射变换\(\tau\)的\DefineConcept{行列式}”,
记作\(\det\tau\).
如果\(\det\tau>0\),则称\(\tau\)是\DefineConcept{第一类仿射变换}.
如果\(\det\tau<0\),则称\(\tau\)是\DefineConcept{第二类仿射变换}.
\end{definition}

\cref{theorem:平面的仿射变换.仿射变换的变积系数} 表明:
仿射变换\(\tau\)按照同一个比值\(\det\tau\)来改变所有平行四边形的定向面积.

\begin{corollary}\label{theorem:平面的仿射变换.仿射变换的变积系数.推论}
%@see: 《解析几何》(丘维声) P226 推论3.6
若平面上任意一块有面积的区域\(D\)
经过仿射变换\(\tau\)映成区域\(D'\),
则有\begin{equation*}
	\frac{S(D')}{S(D)} = \abs{\det\tau},
\end{equation*}
其中\(S(D'),S(D)\)分别表示\(D',D\)的面积.
\end{corollary}

\cref{theorem:平面的仿射变换.仿射变换的变积系数.推论} 说明:
仿射变换\(\tau\)按照同一个比值\(\abs{\det\tau}\)来改变平面上所有(有面积的)图形的面积.
因此,把\(\abs{\det\tau}\)称为仿射变换\(\tau\)的\DefineConcept{变积系数}.

\begin{example}
%@see: 《解析几何》(丘维声) P228 习题6.3 6.
如果一条直线与它在仿射变换\(\tau\)下的像重合,
则称这条直线是\(\tau\)的\DefineConcept{不变直线}.
求下述仿射变换的不变直线:\begin{equation*}
	\begin{cases}
		x' = 7x - y + 1, \\
		y' = 4x + 2y + 4.
	\end{cases}
\end{equation*}
%TODO
\end{example}

\begin{example}
%@see: 《解析几何》(丘维声) P228 习题6.3 8.
证明:如果一个仿射变换有两个不动点\(M_1\)和\(M_2\),
则直线\(M_1 M_2\)上的每一个点在这个仿射变换下不变.
%TODO
\end{example}

\begin{example}
%@see: 《解析几何》(丘维声) P228 习题6.3 9.(1)
证明:相似保持角的大小不变.
%TODO
\end{example}

%@see: 《解析几何》(丘维声) P228 习题6.3 9.(2)
% 写出相似在一个直角坐标系中的公式.
%TODO

%@see: 《解析几何》(丘维声) P228 习题6.3 10.(1)
% 适当选取坐标系,求出位似的公式.
%TODO

\begin{example}
%@see: 《解析几何》(丘维声) P228 习题6.3 10.(2)
证明:位似可以分解成某两个压缩的乘积.
%TODO
\end{example}

\begin{example}
%@see: 《解析几何》(丘维声) P228 习题6.3 11
将仿射变换\begin{equation*}
	\begin{cases}
		x' = 4x, \\
		y' = 2y
	\end{cases}
\end{equation*}
分解成一个压缩和一个位似的乘积.
%TODO
\end{example}
