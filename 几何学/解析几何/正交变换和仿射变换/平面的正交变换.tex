\section{平面的正交变换}

%@see: 《解析几何》(丘维声) P193
设\(V\)是一个2维或3维几何空间.

如果映射\(\sigma\colon V \to V\)使得
任意一点\(P\)到它的像\(P'\)的指向是给定的一个方向,
并且点\(P\)与\(P'\)的距离等于给定的长度,
则称“映射\(\sigma\)是几何空间\(V\)的一个\DefineConcept{平移}”.

%@see: 《解析几何》(丘维声) P193
取定一个点\(O\),给定一个角\(\alpha\),
如果映射\(\sigma\colon V \to V\)使得
任意一点\(P\)和它的像\(P'\)满足\begin{equation*}
	\LineSegmentLength{OP'}
	= \LineSegmentLength{OP},
	\qquad
	\angle P'OP = \alpha,
\end{equation*}
则称“映射\(\sigma\)是几何空间\(V\)绕坐标原点\(O\)的一个\DefineConcept{旋转}”.
把点\(O\)称为\DefineConcept{旋转中心}.
把\(\alpha\)称为\DefineConcept{旋转角}.

%@see: 《解析几何》(丘维声) P193
如果映射\(\tau\colon V \to V\)使得
不在直线\(l\)上的任意一点\(P\)与它的像\(P'\)的连线段被直线\(l\)垂直平分,
\(l\)上任意一点的像是它本身,
则称“映射\(\tau\)是几何空间\(V\)关于直线\(l\)的\DefineConcept{轴反射}”.
此时,称“点\(P\)与点\(P'\)关于直线\(l\) \DefineConcept{对称}”,
其中一个点叫作另一个点关于直线\(l\)的\DefineConcept{对称点}.

上述三种点变换(平移、旋转、反射)有一个共同的特点:保持任意两点的距离不变.

\begin{definition}
%@see: 《解析几何》(丘维声) P199 定义2.1
平面上的一个点变换,
如果保持任意两点的距离不变,
则称之为\DefineConcept{正交点变换}或\DefineConcept{保距变换}.
\end{definition}

\begin{property}
%@see: 《解析几何》(丘维声) P199 性质1
正交变换的乘积是正交变换.
\end{property}

\begin{property}
%@see: 《解析几何》(丘维声) P199 性质2
恒等变换是正交变换.
\end{property}

\begin{property}\label{theorem:平面的正交变换.正交变换保持共线性}
%@see: 《解析几何》(丘维声) P199 性质3
正交变换
把共线的三点映成共线的三点,并且保持它们的顺序不变;
把不共线的三点映成不共线的三点.
\begin{proof}
设\(A,B,C\)是共线的三点,
且点\(B\)在线段\(AC\)上,
则\begin{equation*}
	\LineSegmentLength{AB} + \LineSegmentLength{BC}
	= \LineSegmentLength{AC}.
\end{equation*}
又设\(\sigma\)是正交变换,
它把\(A,B,C\)分别映成\(A',B',C'\),
则\begin{equation*}
	\LineSegmentLength{A'B'}
	= \LineSegmentLength{AB},
	\qquad
	\LineSegmentLength{A'C'}
	= \LineSegmentLength{AC},
	\LineSegmentLength{B'C'}
	= \LineSegmentLength{BC}.
\end{equation*}
于是\begin{equation*}
	\LineSegmentLength{A'B'}
	+ \LineSegmentLength{B'C'}
	= \LineSegmentLength{A'C'}.
\end{equation*}
这表明\(A',B',C'\)三点共线,并且\(B'\)在线段\(A'C'\)上.

设\(D,E,F\)是不共线三点,则\begin{equation*}
	\LineSegmentLength{DE}
	+ \LineSegmentLength{EF}
	> \LineSegmentLength{DF}.
\end{equation*}
设正交变换\(\sigma\)把\(D,E,F\)分别映成\(D',E',F'\),
则有\begin{equation*}
	\LineSegmentLength{D'E'}
	= \LineSegmentLength{DE},
	\qquad
	\LineSegmentLength{D'F'}
	= \LineSegmentLength{DF},
	\LineSegmentLength{E'F'}
	= \LineSegmentLength{EF}.
\end{equation*}
于是\begin{equation*}
	\LineSegmentLength{D'E'}
	+ \LineSegmentLength{E'F'}
	> \LineSegmentLength{D'F'},
\end{equation*}
这表明\(D',E',F'\)三点不共线.
\end{proof}
\end{property}

\begin{property}
%@see: 《解析几何》(丘维声) P199 性质4
正交变换把直线映成直线,把线段映成线段,并且保持线段的分比不变.
\begin{proof}
设\(\sigma\)是正交变换.
任取一条直线\(l\),
在\(l\)上取两点\(A,B\).
设\(\sigma\)把\(A,B\)分别映成\(A',B'\).
由\cref{theorem:平面的正交变换.正交变换保持共线性} 可知,
\(l\)上任意一点\(C\)的像\(C'\)一定在直线\(A'B'\)上.
反之,在直线\(A'B'\)上任取一点\(Q'\),
要证\(Q'\)有原像,并且其原像在\(l\)上.
不妨设\(Q'\)在线段\(A'B'\)上,
于是\(
	\LineSegmentLength{A'Q'}
	< \LineSegmentLength{A'B'}
	= \LineSegmentLength{AB}
\).
在线段\(AB\)上可以找到一点\(P\),
使得\(
	\LineSegmentLength{AP}
	= \LineSegmentLength{A'Q'}
\).
设\(P\)在\(\sigma\)下的像为\(P'\),
则\(
	\LineSegmentLength{A'P'}
	= \LineSegmentLength{AP}
	= \LineSegmentLength{A'Q'}
\),
并且\(P'\)在线段\(A'B'\)上.
于是得出\(P'=Q'\).
这表明\(P\)是\(Q'\)的一个原像.
这证明了\(\sigma\)把直线\(l\)映成直线\(A'B'\).

任取一条线段\(AB\).
设\(A,B\)在\(\sigma\)下的像分别是\(A',B'\).
在线段\(AB\)上任取一点\(C\),
则由\cref{theorem:平面的正交变换.正交变换保持共线性} 可知,
\(C\)在\(\sigma\)下的像\(C'\)在线段\(A'B\)上.
反之,在线段\(A'B'\)上任取一点\(D'\),
\(D'\)有原像\(D\),
并且\(D\)在直线\(AB\)上.
由\cref{theorem:平面的正交变换.正交变换保持共线性} 可知,
\(D\)必在线段\(AB\)上.
这证明了\(\sigma\)把线段\(AB\)映成线段\(A'B'\).

设\(C\)是线段\(AB\)的一个内分点,
并且\(\frac{\LineSegmentLength{AC}}{\LineSegmentLength{CB}} = \lambda > 0\).
又设\(\sigma\)把\(A,B,C\)分别映成\(A',B',C'\),
则\(C'\)在线段\(A'B'\)上,
并且\begin{equation*}
	\frac{\LineSegmentLength{A'C'}}{\LineSegmentLength{C'B'}}
	= \frac{\LineSegmentLength{AC}}{\LineSegmentLength{CB}}
	= \lambda.
\end{equation*}
对于线段\(AB\)的外分点,可以类似证得结论.
\end{proof}
\end{property}

\begin{property}
%@see: 《解析几何》(丘维声) P200 性质5
正交变换是可逆的,并且它的逆变换也是正交变换.
\end{property}

\begin{property}
%@see: 《解析几何》(丘维声) P201 性质6
正交变换把平行直线映成平行直线.
\end{property}

我们把平面上所有点组成的集合记作\(S\),
把平面上所有向量组成的集合记作\(\overline{S}\).
\begin{property}
%@see: 《解析几何》(丘维声) P201 性质7
正交点变换\(\sigma\)诱导了集合\(\overline{S}\)上的一个变换\(\overline{\sigma}\),
即\begin{equation*}
%@see: 《解析几何》(丘维声) P201 (2.1)
	\overline{\sigma}(\vec{AB}) = \vec{A'B'},
\end{equation*}
其中\(A',B'\)是\(A,B\)在\(\sigma\)下的像.
\end{property}

我们把正交点变换\(\sigma\)诱导的、集合\(\overline{S}\)上的变换\(\overline{\sigma}\),
称为\DefineConcept{正交向量变换}.
今后在谈到正交(点)变换\(\sigma\)在向量上的作用时,
指的就是\(\sigma\)诱导的向量变换\(\overline{\sigma}\)在该向量上的作用.

\begin{property}
%@see: 《解析几何》(丘维声) P201 性质8
正交变换\(\sigma\)还具有以下性质:\begin{enumerate}
	\item 保持向量的加法,即\begin{equation*}
		\overline{\sigma}(\vb\alpha+\vb\beta)
		= \overline{\sigma}(\vb\alpha)
		+ \overline{\sigma}(\vb\beta);
	\end{equation*}

	\item 保持向量的数乘,即\begin{equation*}
		\overline{\sigma}(\lambda \vb\alpha)
		= \lambda \overline{\sigma}(\vb\alpha);
	\end{equation*}

	\item 保持向量的长度不变;

	\item 保持向量的夹角不变;

	\item 保持向量的内积不变.
\end{enumerate}
\end{property}

\begin{theorem}
%@see: 《解析几何》(丘维声) P202 定理2.1(正交变换第一基本定理)
平面上的正交变换\(\sigma\)把任意一个直角标架 I \([O;\vb{e}_1,\vb{e}_2]\)变成一个直角标架 II,
并且使得任意一点\(P\)的 I 坐标等于它的像\(P'\)的 II 坐标;
反之,如果平面上的一个点变换\(\tau\)
使得任意一点\(Q\)在直角标架 I 中的坐标等于\(Q\)的像\(Q'\)在直角标架 II 中的坐标,
则\(\tau\)是正交变换.
\end{theorem}

\begin{corollary}
%@see: 《解析几何》(丘维声) P203 推论2.1
如果平面上的两个正交变换\(\sigma\)和\(\tau\)
把直角标架 I 变成同一个直角标架 II,
则\(\sigma = \tau\).
\end{corollary}
\begin{remark}
这个推论说明:平面上的正交变换被它在一个直角标架上的作用所确定.
\end{remark}

\begin{theorem}
%@see: 《解析几何》(丘维声) P203 定理2.2(正交变换第二基本定理)
平面上的正交变换,或者是平移,或者是旋转,或者是反射,或者是它们的乘积.
\end{theorem}

平移、旋转,以及它们的乘积,称为\DefineConcept{刚体运动}.

上述定理表明:
平面上的正交变换,或者是刚体运动,或者是反射,或者是刚体运动与反射的乘积.

\begin{theorem}
%@see: 《解析几何》(丘维声) P204 定理2.3
设平面上的正交点变换\(\sigma\)
把直角标架 I \([O;\vb{e}_1,\vb{e}_2]\)
映成直角标架 II \([O';\vb{e}'_1,\vb{e}'_2]\),
其中\(O',\vb{e}'_1,\vb{e}'_2\)的 I 坐标分别是\begin{equation*}
	(a_1,a_2),
	\qquad
	(a_{11},a_{21}),
	\qquad
	(a_{12},a_{22}),
\end{equation*}
则\(\sigma\)在直角标架 I 中的公式是\begin{equation*}
%@see: 《解析几何》(丘维声) P204 (2.2)
	\begin{bmatrix}
		x' \\ y'
	\end{bmatrix}
	= \begin{bmatrix}
		a_{11} & a_{12} \\
		a_{21} & a_{22}
	\end{bmatrix}
	\begin{bmatrix}
		x \\ y
	\end{bmatrix}
	+ \begin{bmatrix}
		a_1 \\ a_2
	\end{bmatrix},
\end{equation*}
其中\((x,y)\)是任意一点\(P\)的 I 坐标,
\((x',y')\)是\(P\)在\(\sigma\)下的像\(P'\)的 I 坐标,
并且矩阵\begin{equation*}
	\vb{A} \defeq \begin{bmatrix}
		a_{11} & a_{12} \\
		a_{21} & a_{22}
	\end{bmatrix}
\end{equation*}
是正交矩阵.
反之,如果平面上的一个点变换\(\tau\)
在直角标架 I \([O;\vb{e}_1,\vb{e}_2]\)中的公式是\begin{equation*}
%@see: 《解析几何》(丘维声) P204 (2.4)
	\begin{bmatrix}
		x' \\ y'
	\end{bmatrix}
	= \begin{bmatrix}
		b_{11} & b_{12} \\
		b_{21} & b_{22}
	\end{bmatrix}
	\begin{bmatrix}
		x \\ y
	\end{bmatrix}
	+ \begin{bmatrix}
		b_1 \\ b_2
	\end{bmatrix},
\end{equation*}
其中\begin{equation*}
	\vb{B} \defeq \begin{bmatrix}
		b_{11} & b_{12} \\
		b_{21} & b_{22}
	\end{bmatrix}
\end{equation*}
是正交矩阵,
则\(\tau\)是正交变换.
\end{theorem}

\begin{example}
%@see: 《解析几何》(丘维声) P206 习题6.2 5.
证明:若平面上的正交变换\(\sigma\)有两个不动点\(A,B\),
则直线\(AB\)上每一个点都是\(\sigma\)的不动点.
%TODO proof
\end{example}

\begin{example}
%@see: 《解析几何》(丘维声) P206 习题6.2 6.
设\(\tau_1\)和\(\tau_2\)分别是平面关于直线\(l_1\)和\(l_2\)的反射,
\(l_1\)与\(l_2\)交于点\(O\),夹角为\(\theta\).
证明:\(\tau_2 \tau_1\)是绕点\(O\)的旋转,旋转角为\(2\theta\).
%TODO proof
\end{example}
