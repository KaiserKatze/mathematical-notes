\section{图形的度量性质和仿射性质}
\subsection{度量性质和仿射性质的概念}
%@see: 《解析几何》(丘维声) P230 定义4.1
在任意正交变换下不变的几何性质称为\DefineConcept{度量性质}.
在任意正交变换下不变的几何量称为\DefineConcept{正交不变量}.
在任意正交变换下不变的几何概念称为\DefineConcept{度量概念}.
在任意仿射变换下不变的几何性质称为\DefineConcept{仿射性质}.
在任意仿射变换下不变的几何量称为\DefineConcept{仿射不变量}.
在任意仿射变换下不变的几何概念称为\DefineConcept{仿射概念}.

因为正交变换都是仿射变换,
所以,在仿射变换下不变的性质,在正交变换下当然也不变.
这说明,仿射性质都是度量性质,仿射概念都是度量概念,仿射不变量都是正交不变量.
但是,反过来,度量性质不一定是仿射性质.

\begin{table}[hbt]
%@see: 《解析几何》(丘维声) P230
	\centering
	\begin{tblr}{*2{p{5cm}|}p{5cm}}
		\hline\hline
		度量性质 & 正交不变量 & 度量概念
		\\ \hline
		垂直,轴对称
		& 点与点之间的距离,向量的长度,两向量的夹角,图形的面积,二次曲线的不变量
		& 距离(长度),角度,面积,对称轴
		\\ \hline\hline
		放射性质 & 仿射不变量 & 仿射概念
		\\ \hline
		共线,平行,相交,共线点的顺序,中心对称
		& 线段的分比,代数曲线的次数
		& 直线,线段,线段的中点,对称中心,代数曲线,
		二次曲线的渐进方向、非渐进方向,二次曲线的直径,
		中心型二次曲线的共轭直径,二次曲线的切线
		\\ \hline\hline
	\end{tblr}
	\caption{}
\end{table}

\subsection{变换群}
度量性质是在所有正交变换下不变的性质,
因此需要讨论平面上的所有正交变换组成的集合\(H\).
从正交变换的性质知道,
\(H\)具有以下性质:\begin{enumerate}
	\item 对于任意\(\sigma,\tau \in H\),有\(\sigma \tau \in H\);
	\item 恒等变换\(I \in H\);
	\item 任给\(\sigma \in H\),则\(\sigma\)可逆,且\(\sigma^{-1} \in H\).
\end{enumerate}

类似地,平面上所有仿射变换组成的集合也具有这三条性质.
\begin{definition}
%@see: 《解析几何》(丘维声) P231 定义4.2
设\(S\)是一个集合,
\(G\)是\(S\)到自身的一些双射组成的集合.
如果\begin{itemize}
	\item 对于任意\(\sigma,\tau \in G\),有\(\sigma \tau \in G\);
	\item 恒等变换\(I \in G\);
	\item 任给\(\sigma \in G\),则\(\sigma\)可逆,且\(\sigma^{-1} \in G\).
\end{itemize}
则称“\(G\)是集合\(S\)的一个\DefineConcept{变换群}”.
\end{definition}

于是,由平面上的所有正交变换组成的集合\(H\)是平面的一个变换群,称之为平面的\DefineConcept{正交变换群};
由平面上的所有仿射变换组成的集合\(H_0\)也是平面的一个变换群,称之为平面的\DefineConcept{仿射变换群}.
集合\(S\)到自身的所有双射组成的集合显然是\(S\)的变换群,称之为“\(S\)的\DefineConcept{全变换群}”.

德国数学家克莱因在1872年运用变换群的思想来区分各种几何学.
他提出:每一种几何都是研究图形在一定的变换群下不变的性质的.
这就是著名的\DefineConcept{爱尔兰根纲领}(Erlange Program).
于是,研究图形在正交变换群下不变的性质(即度量性质)的几何学称为\DefineConcept{欧氏几何学};
研究图形在仿射变换群下不变的性质(即仿射性质)的几何学称为\DefineConcept{仿射几何学}.
