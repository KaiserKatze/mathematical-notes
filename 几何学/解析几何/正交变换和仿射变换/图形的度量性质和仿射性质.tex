\section{图形的度量性质和仿射性质}
\subsection{度量性质和仿射性质的概念}
%@see: 《解析几何》(丘维声) P230 定义4.1
在任意正交变换下不变的几何性质称为\DefineConcept{度量性质}.
在任意正交变换下不变的几何量称为\DefineConcept{正交不变量}.
在任意正交变换下不变的几何概念称为\DefineConcept{度量概念}.
在任意仿射变换下不变的几何性质称为\DefineConcept{仿射性质}.
在任意仿射变换下不变的几何量称为\DefineConcept{仿射不变量}.
在任意仿射变换下不变的几何概念称为\DefineConcept{仿射概念}.

因为正交变换都是仿射变换,
所以,在仿射变换下不变的性质,在正交变换下当然也不变.
这说明,仿射性质都是度量性质,仿射概念都是度量概念,仿射不变量都是正交不变量.
但是,反过来,度量性质不一定是仿射性质.

\begin{table}[hbt]
%@see: 《解析几何》(丘维声) P230
%@see: 《解析几何》(丘维声) P236 习题6.4 1.
%@see: 《解析几何》(丘维声) P236 习题6.4 2.
%@see: 《解析几何》(丘维声) P236 习题6.4 3.
	\centering
	\begin{tblr}{*2{p{5cm}|}p{5cm}}
		\hline\hline
		度量性质 & 正交不变量 & 度量概念
		\\ \hline
		% 度量性质
		垂直,轴对称
		% 正交不变量
		& 点与点之间的距离,向量的长度,两向量的夹角,图形的面积,二次曲线的不变量
		% 度量概念
		& 距离(长度),
		角度,三角形的角平分线,
		面积,对称轴
		\\ \hline\hline
		仿射性质 & 仿射不变量 & 仿射概念
		\\ \hline
		% 仿射性质
		共线,平行,相交,共线点的顺序,中心对称,
		一个点是否属于封闭图形内部
		% 仿射不变量
		& 线段的分比,代数曲线的次数
		% 仿射概念
		& 直线,线段,线段的中点,三角形的中线,
		对称中心(例如三角形的重心),
		代数曲线,
		二次曲线的渐进方向、非渐进方向,
		二次曲线的直径,
		中心型二次曲线的共轭直径,
		二次曲线的切线
		\\ \hline\hline
	\end{tblr}
	\caption{}
\end{table}

\subsection{变换群}
度量性质是在所有正交变换下不变的性质,
因此需要讨论平面上的所有正交变换组成的集合\(H\).
从正交变换的性质知道,
\(H\)具有以下性质:\begin{enumerate}
	\item 对于任意\(\sigma,\tau \in H\),有\(\sigma \tau \in H\);
	\item 恒等变换\(I \in H\);
	\item 任给\(\sigma \in H\),则\(\sigma\)可逆,且\(\sigma^{-1} \in H\).
\end{enumerate}

类似地,平面上所有仿射变换组成的集合也具有这三条性质.
\begin{definition}
%@see: 《解析几何》(丘维声) P231 定义4.2
设\(S\)是一个集合,
\(G\)是\(S\)到自身的一些双射组成的集合.
如果\begin{itemize}
	\item 对于任意\(\sigma,\tau \in G\),有\(\sigma \tau \in G\);
	\item 恒等变换\(I \in G\);
	\item 任给\(\sigma \in G\),则\(\sigma\)可逆,且\(\sigma^{-1} \in G\).
\end{itemize}
则称“\(G\)是集合\(S\)的一个\DefineConcept{变换群}”.
\end{definition}

于是,由平面上的所有正交变换组成的集合\(H\)是平面的一个变换群,称之为平面的\DefineConcept{正交变换群};
由平面上的所有仿射变换组成的集合\(H_0\)也是平面的一个变换群,称之为平面的\DefineConcept{仿射变换群}.
集合\(S\)到自身的所有双射组成的集合显然是\(S\)的变换群,称之为“\(S\)的\DefineConcept{全变换群}”.

德国数学家克莱因在1872年运用变换群的思想来区分各种几何学.
他提出:每一种几何都是研究图形在一定的变换群下不变的性质的.
这就是著名的\DefineConcept{爱尔兰根纲领}(Erlange Program).
于是,研究图形在正交变换群下不变的性质(即度量性质)的几何学称为\DefineConcept{欧氏几何学};
研究图形在仿射变换群下不变的性质(即仿射性质)的几何学称为\DefineConcept{仿射几何学}.

\subsection{图形的正交等价和仿射等价}
利用所给的变换群可以把平面上所有图形进行分类.
\begin{definition}
%@see: 《解析几何》(丘维声) P234 定义4.3
给定两个平面图形\(C_1,C_2\),
如果存在一个正交变换\(\sigma\)把\(C_1\)映成\(C_2\),
则称“\(C_1\)与\(C_2\) \DefineConcept{正交等价}”,
记作\(C_1 \sim C_2\).
\end{definition}

%@see: 《解析几何》(丘维声) P234
正交等价是平面上图形之间的一种关系,它具有反身性、对称性、传递性.
对于平面上每个图形\(C\),所有与\(C\)正交等价的图形组成的集合,
称为“图形\(C\)的\DefineConcept{正交等价类}”.

根据等价类的性质可知:
只要两个图形正交等价,则它们的正交等价类相等;
平面上每个图形属于且只属于一个正交等价类;
同一正交等价类里的任意两个图形必正交等价;
不同正交等价类里的两个图形不正交等价.

既然同一个正交等价类里的任意两个图形都是正交等价的,
那么同一个正交等价类里的图形的共同性质就是在任意正交变换下不变的性质,即度量性质.
这样,我们在研究某个图形的度量性质时,就可以在它的正交等价类里挑一个最简单的图形来研究.
这就是研究图形的正交分类的目的.

%@see: 《解析几何》(丘维声) P235
同样地,平面的仿射变换群把平面上所有图形分成了一些\DefineConcept{仿射等价类}.
同一个仿射等价类里的任意两个图形是仿射等价的;
不同仿射等价类里的两个图形是不仿射等价的.
因此,同一个仿射等价类里图形的共同性质就是仿射性质.
我们在研究某个图形的仿射性质时,就可以在它的仿射等价类里挑一个最简单的图形来研究.

\begin{example}
%@see: 《解析几何》(丘维声) P235 例4.1
平面上所有平行四边形恰好组成一个仿射等价类.
\end{example}
