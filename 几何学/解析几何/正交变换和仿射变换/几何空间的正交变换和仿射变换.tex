\section{几何空间的正交变换和仿射变换}
\begin{definition}
%@see: 《解析几何》(丘维声) P241 定义6.1
几何空间的一个点变换,如果保持任意两点的距离不变,
那么称之为\DefineConcept{正交点变换}或\DefineConcept{保距点变换}.
\end{definition}

常见的正交点变换有平移、绕一条定直线旋转、关于给定平面的镜面反射等.

\begin{property}
%@see: 《解析几何》(丘维声) P241 性质1
正交点变换的乘积是正交点变换.
\end{property}

\begin{property}
%@see: 《解析几何》(丘维声) P241 性质2
恒等变换是正交点变换.
\end{property}

\begin{property}
%@see: 《解析几何》(丘维声) P241 性质3
正交点变换是单射.
\end{property}

\begin{property}
%@see: 《解析几何》(丘维声) P241 性质4
正交点变换把共线三点映成共线三点,并且保持共线点的顺序.
正交点变换把不共线三点映成不共线三点.
\end{property}

\begin{property}
%@see: 《解析几何》(丘维声) P241 性质5
正交点变换把直线映成直线.
正交点变换把线段映成线段,并且保持线段的分比不变.
\end{property}

\begin{property}
%@see: 《解析几何》(丘维声) P241 性质6
正交变换把平面映成平面.
\end{property}

\begin{property}
%@see: 《解析几何》(丘维声) P242 性质7
正交点变换把平行平面映成平行平面.
\end{property}

\begin{property}
%@see: 《解析几何》(丘维声) P242 性质8
正交点变换把平行直线映成平行直线.
\end{property}

我们把几何空间中所有点组成的集合记作\(S\),
把几何空间中所有向量组成的集合记作\(\overline{S}\).
\begin{property}
%@see: 《解析几何》(丘维声) P242 性质9
正交点变换\(\sigma\)诱导了集合\(\overline{S}\)上的一个变换\(\overline{\sigma}\),
即\begin{equation*}
	\overline{\sigma}(\vec{AB}) = \vec{A'B'},
\end{equation*}
其中\(A',B'\)是\(A,B\)在\(\sigma\)下的像.
\end{property}

我们把正交点变换\(\sigma\)诱导的、集合\(\overline{S}\)上的上述变换\(\overline{\sigma}\)
称为几何空间的\DefineConcept{正交向量变换}.
今后在谈到正交(点)变换\(\sigma\)在向量上的作用时,
指的就是\(\sigma\)诱导的向量变换\(\overline{\sigma}\)的作用.

\begin{property}
%@see: 《解析几何》(丘维声) P242 性质10
正交变换保持向量的加法,
保持向量的数量乘法,
保持向量的长度不变,
保持向量的夹角不变,
保持向量的内积不变.
\end{property}

\begin{theorem}
%@see: 《解析几何》(丘维声) P243 定理6.1
正交变换\(\sigma\)把任意一个直角标架 I \([O;\vb{e}_1,\vb{e}_2,\vb{e}_3]\)
映成一个直角标架(记作 II),
并且使得任意一个点\(P\)的 I 坐标等于它的像\(P'\)的 II 坐标;
反之,如果空间的一个点变换\(\tau\)使得任意一个点\(Q\)在直角标架 I 中的坐标
等于\(Q\)的像\(Q'\)在直角标架 II 中的坐标,
则\(\tau\)是正交变换.
\end{theorem}

\begin{corollary}
%@see: 《解析几何》(丘维声) P243 推论6.1
如果几何空间的两个正交变换\(\sigma\)和\(\tau\)把直角标架 I 映成同一个直角标架 II,
则\(\sigma = \tau\).
\end{corollary}

\begin{property}
%@see: 《解析几何》(丘维声) P243 性质11
正交变换是满射.
正交变换是可逆的.
正交变换的逆变换也是正交变换.
\end{property}

\begin{property}
%@see: 《解析几何》(丘维声) P243 性质12
正交变换把相交平面映成相交平面.
\end{property}

\begin{theorem}
%@see: 《解析几何》(丘维声) P243 定理6.2
空间的正交点变换\(\sigma\)在一个直角坐标系中的公式为\begin{equation*}
%@see: 《解析几何》(丘维声) P243 (6.1)
	\begin{bmatrix}
		x' \\ y' \\ z'
	\end{bmatrix}
	= \begin{bmatrix}
		a_{11} & a_{12} & a_{13} \\
		a_{21} & a_{22} & a_{23} \\
		a_{31} & a_{32} & a_{33}
	\end{bmatrix}
	\begin{bmatrix}
		x \\ y \\ z
	\end{bmatrix}
	+ \begin{bmatrix}
		a_1 \\ a_2 \\ a_3
	\end{bmatrix},
\end{equation*}
其中\((x,y,z)\)是任意一点\(P\)的 I 坐标,
\((x',y',z')\)是\(P\)在\(\sigma\)下的像\(P'\)的 I 坐标,
并且矩阵\begin{equation*}
	\vb{A} \defeq \begin{bmatrix}
		a_{11} & a_{12} \\
		a_{21} & a_{22}
	\end{bmatrix}
\end{equation*}
是正交矩阵.
反之,如果平面上的一个点变换\(\tau\)
在直角标架 I \([O;\vb{e}_1,\vb{e}_2]\)中的公式是\begin{equation*}
	\begin{bmatrix}
		x' \\ y' \\ z'
	\end{bmatrix}
	= \begin{bmatrix}
		a_{11} & a_{12} & a_{13} \\
		a_{21} & a_{22} & a_{23} \\
		a_{31} & a_{32} & a_{33}
	\end{bmatrix}
	\begin{bmatrix}
		x \\ y \\ z
	\end{bmatrix}
	+ \begin{bmatrix}
		a_1 \\ a_2 \\ a_3
	\end{bmatrix},
\end{equation*}
其中\begin{equation*}
	\begin{bmatrix}
		b_{11} & b_{12} \\
		b_{21} & b_{22}
	\end{bmatrix}
\end{equation*}
是正交矩阵,
则\(\tau\)是正交点变换.
\end{theorem}

%TODO 以下几条暂时省略
%@see: 《解析几何》(丘维声) P244 定义6.2
%@see: 《解析几何》(丘维声) P244 命题6.1
%@see: 《解析几何》(丘维声) P245 命题6.2
%@see: 《解析几何》(丘维声) P245 定理6.3

\begin{definition}
%@see: 《解析几何》(丘维声) P245 定义6.3
几何空间的一个点变换\(\tau\),
如果它在一个仿射坐标系中的公式为\begin{equation*}
%@see: 《解析几何》(丘维声) P246 (6.4)
	\begin{bmatrix}
		x' \\ y' \\ z'
	\end{bmatrix}
	= \vb{A}
	\begin{bmatrix}
		x \\ y \\ z
	\end{bmatrix}
	+ \begin{bmatrix}
		a_1 \\ a_2 \\ a_3
	\end{bmatrix},
\end{equation*}
其中\(\vb{A}\)是非奇异矩阵,
则称\(\tau\)是几何空间的\DefineConcept{仿射点变换}.
\end{definition}

%@see: 《解析几何》(丘维声) P246
几何空间的仿射变换具有以下性质:\begin{enumerate}
	\item 两个仿射变换的乘积仍是仿射变换.
	\item 恒等变换是仿射变换.
	\item 仿射变换是可逆的.
	仿射变换的逆变换也是仿射变换.
	\item 仿射变换把平面映成平面.
	\item 仿射变换把平行平面映成平行平面.
	\item 仿射变换把相交平面映成相交平面.
	\item 仿射变换把直线映成直线.
	\item 仿射变换把线段映成线段.
	\item 仿射变换保持线段的分比不变.
	\item 仿射变换把平行直线映成平行直线.
	\item 几何空间的仿射点变换诱导了空间的一个仿射向量变换.
	\item 仿射变换保持向量的加法,保持向量的数量乘法.
	\item 仿射变换把任意一个仿射标架 I 映成仿射标架 II,且任意一点\(P\)的 I 坐标等于它的像\(P'\)的 II 坐标;
	反之,若变换\(\tau\)仿射标架 I 映成仿射标架 II,且任意一点\(P\)的 I 坐标等于它的像\(P'\)的 II 坐标,则\(\tau\)是仿射变换.
	\item 若两个仿射变换在同一个仿射标架上的作用相同,则它们相等.
\end{enumerate}

\begin{theorem}
%@see: 《解析几何》(丘维声) P246 定理6.4
几何空间中任给两组不共面的四点\(\AutoTuple{A}{4}\)和\(\AutoTuple{B}{4}\),
必然存在唯一的仿射变换,
把\(A_1\)映成\(B_1\),
把\(A_2\)映成\(B_2\),
把\(A_3\)映成\(B_3\),
把\(A_4\)映成\(B_4\).
\end{theorem}

\begin{theorem}
%@see: 《解析几何》(丘维声) P246 定理6.5
设几何空间的仿射变换\(\tau\)在仿射标架 I 中的公式为\begin{equation*}
%@see: 《解析几何》(丘维声) P246 (6.4)
	\begin{bmatrix}
		x' \\ y' \\ z'
	\end{bmatrix}
	= \vb{A}
	\begin{bmatrix}
		x \\ y \\ z
	\end{bmatrix}
	+ \begin{bmatrix}
		a_1 \\ a_2 \\ a_3
	\end{bmatrix},
\end{equation*}
又设几何空间中任意三个不共面的向量\(\vb{a},\vb{b},\vb{c}\)在\(\overline{\tau}\)下的像
分别是\(\vb{a}',\vb{b}',\vb{c}'\),
则有\begin{equation*}
	\frac{
		\VectorInnerProductDot{\VectorOuterProduct{\vb{a}'}{\vb{b}'}}{\vb{c}'}
	}{
		\VectorInnerProductDot{\VectorOuterProduct{\vb{a}}{\vb{b}}}{\vb{c}}
	}
	= \DeterminantA{\vb{A}}.
\end{equation*}
\end{theorem}
易知仿射变换\(\tau\)的公式中的系数矩阵的行列式与坐标系的选择无关,
因此可以称\(\DeterminantA{\vb{A}}\)是\(\tau\)的行列式.

上述定理表明,几何空间的一个仿射变换\(\tau\)按同一比值来改变所有平行六面体的定向体积,
从而按同一比值改变几何空间中任意区域的体积.
