\section{二次曲线的正交分类和仿射分类}
我们已经知道,平面上的二次曲线有9种,它们是:椭圆、虚椭圆、一个点、双曲线、
一对相交直线、抛物线、一对平行直线、一对虚平行直线、一对重合直线.
平面上的二次曲线恰好也可以划分为9个不同的仿射等价类.

\begin{table}[hbt]
	\centering
	\begin{tblr}{c|c}
		& 代表图形的方程 \\ \hline
		圆类
		& \(x^2 + y^2 = 1\) \\
		虚圆类
		& \(x^2 + y^2 = -1\) \\
		点类
		& \(x^2 + y^2 = 0\) \\
		双曲线类
		& \(x^2 - y^2 = 1\) \\
		相交直线类
		& \(x^2 - y^2 = 0\) \\
		抛物线类
		& \(y^2 = x\) \\
		平行直线类
		& \(y^2 = 1\) \\
		虚平行直线类
		& \(y^2 = -1\) \\
		重合直线类
		& \(y^2 = 0\) \\
	\end{tblr}
	\caption{仿射等价类}
\end{table}

\begin{theorem}
%@see: 《解析几何》(丘维声) P238 定理5.1
平面的任意一个仿射变换均可分解成一个正交变换与两个沿互相垂直方向的压缩或拉伸变换的乘积.
%TODO proof
\end{theorem}

利用二次曲线的仿射分类,可以解决本章开头提出的问题:
\begin{example}
%@see: 《解析几何》(丘维声) P239 例5.1
证明:椭圆的任意一对共轭直径把椭圆分成4块面积相等的部分.
%TODO proof
\end{example}

\begin{example}
%@see: 《解析几何》(丘维声) P240 习题6.5 1.
证明:长半轴长为\(a\)、短半轴长为\(b\)的椭圆的面积是\(\pi a b\).
%TODO proof
\end{example}

\begin{example}
%@see: 《解析几何》(丘维声) P240 习题6.5 2.
证明:以椭圆的一对共轭半径为边的平行四边形的面积是一个常数.
%TODO proof
\end{example}

\begin{example}
%@see: 《解析几何》(丘维声) P240 习题6.5 3.
证明:椭圆的过共轭直径与椭圆的交点的切线围成的平行四边形的面积是一个常数.
%TODO proof
\end{example}

\begin{example}
%@see: 《解析几何》(丘维声) P240 习题6.5 4.
证明:椭圆的任意一个外切平行四边形的两条对角线所在的直线是椭圆的一对共轭直径.
%TODO proof
\end{example}

\begin{example}
%@see: 《解析几何》(丘维声) P240 习题6.5 5.
证明:以椭圆\(
	\frac{x^2}{a^2} + \frac{y^2}{b^2} = 1
\)的任意一对共轭直径和椭圆的交点为顶点的平行四边形的面积都等于\(2ab\).
%TODO proof
\end{example}

\begin{example}
%@see: 《解析几何》(丘维声) P240 习题6.5 6.
证明:所有内接于椭圆的四边形中,面积最大的时以一对共轭直径和椭圆的交点为顶点的平行四边形.
%TODO proof
\end{example}
