\section{直纹面}
我们看到,柱面和锥面都是由直线组成的.
那么有没有别的也由直线组成的曲面呢?

\begin{definition}
给定一个曲面\(S\),如果存在一族直线,使得这一族中的每一条直线都在\(S\)上,
并且\(S\)上的每一个点都在这一族的某一条直线上,
那么我们称曲面\(S\)为\DefineConcept{直纹面}(ruled surface),
%@see: https://mathworld.wolfram.com/RuledSurface.html
%@see: https://mathworld.wolfram.com/DoublyRuledSurface.html
称这一族直线为\(S\)的一族\DefineConcept{直母线}.
\end{definition}

二次曲面中哪些是直纹面呢?
9种二次曲面和1种二次锥面都是直纹面.
3种椭球面不是直纹面,因为它有界.
双叶双曲面不是直纹面,
因为当它由方程 \labelcref{equation:解析几何.双叶双曲面} 给出时,
平行于\(Oxy\)平面的直线不可能全在\(S\)上,
与\(Oxy\)平面相交的直线也不会全在\(S\)上.
同理可知,椭圆抛物面不是直纹面.
剩下的2种二次曲面,单叶双曲面和双曲抛物面,可以证明都是直纹面.

\begin{theorem}
单叶双曲面是直纹面.
\begin{proof}
设单叶双曲面\(S\)的方程是\begin{equation*}
	\frac{x^2}{a^2}+\frac{y^2}{b^2}-\frac{z^2}{c^2}=1.
\end{equation*}
点\(M_0(x_0,y_0,z_0)\)在单叶双曲面\(S\)上的充分必要条件是\begin{equation*}
	\frac{x_0^2}{a^2}+\frac{y_0^2}{b^2}-\frac{z_0^2}{c^2}=1.
\end{equation*}
移项并且分解因式,得\begin{equation*}
	\left(\frac{x_0}{a}+\frac{z_0}{c}\right)
	\left(\frac{x_0}{a}-\frac{z_0}{c}\right)
	= \left(1+\frac{y_0}{b}\right)
	\left(1-\frac{y_0}{b}\right),
\end{equation*}
即\begin{equation*}
	\def\arraystretch{1.5}
	\begin{vmatrix}
		\frac{x_0}{a}+\frac{z_0}{c} & 1+\frac{y_0}{b} \\
		1-\frac{y_0}{b} & \frac{x_0}{a}-\frac{z_0}{c}
	\end{vmatrix} = 0.
\end{equation*}
因为\(1\pm\frac{y_0}{b}\)不全为零,所以方程组\begin{equation*}
	\left\{ \def\arraystretch{1.5} \begin{array}{l}
		\left(\frac{x_0}{a}+\frac{z_0}{c}\right) X
		+ \left(1+\frac{y_0}{b}\right) Y = 0, \\
		\left(1-\frac{y_0}{b}\right) X
		+ \left(\frac{x_0}{a}-\frac{z_0}{c}\right) Y = 0
	\end{array} \right.
\end{equation*}
是\(X,Y\)的一次齐次方程组.
如前所述,它的系数行列式为零,所以它有非零解,
即存在不全为零的实数\(\mu_0,\nu_0\),使得\begin{equation*}
	\left\{ \def\arraystretch{1.5} \begin{array}{l}
		\mu_0 \left(\frac{x_0}{a}+\frac{z_0}{c}\right)
		+ \nu_0 \left(1+\frac{y_0}{b}\right) = 0, \\
		\mu_0 \left(1-\frac{y_0}{b}\right)
		+ \nu_0 \left(\frac{x_0}{a}-\frac{z_0}{c}\right) = 0.
	\end{array} \right.
\end{equation*}
这表明点\(M_0\)在直线\begin{equation*}
	l_0: \left\{ \def\arraystretch{1.5} \begin{array}{l}
		\mu_0 \left(\frac{x}{a}+\frac{z}{c}\right)
		+ \nu_0 \left(1+\frac{y}{b}\right) = 0, \\
		\mu_0 \left(1-\frac{y}{b}\right)
		+ \nu_0 \left(\frac{x}{a}-\frac{z}{c}\right) = 0
	\end{array} \right.
\end{equation*}上.
现在考虑一族直线:\begin{equation*}
	\mathcal{L}:
	\left\{ \def\arraystretch{1.5} \begin{array}{l}
		\mu \left(\frac{x}{a}+\frac{z}{c}\right)
		+ \nu \left(1+\frac{y}{b}\right) = 0, \\
		\mu \left(1-\frac{y}{b}\right)
		+ \nu \left(\frac{x}{a}-\frac{z}{c}\right) = 0,
	\end{array} \right.
\end{equation*}
其中\(\mu,\nu\)取所有不全为零的实数.
若\((\mu_1,\nu_1)\)与\((\mu_2,\nu_2)\)成比例,
则它们确定直线族\(\mathcal{L}\)中的同一条直线;
若它们不成比例,则它们确定不同的直线.
所以直线族\(\mathcal{L}\)实际上只依赖于一个参数,即\(\mu\)与\(\nu\)的比值.
上面就证明了:
单叶双曲面\(S\)上的任一点\(M_0\)在直线族\(\mathcal{L}\)的某一条直线\(l_0\)上.
现在从直线族\(\mathcal{L}\)中任取一条直线\(l_1\),它对应于\((\mu_1,\nu_1)\),
且在\(l_1\)上任取一点\(M_1(x_1,y_1,z_1)\),则有\begin{equation*}
	\left\{ \def\arraystretch{1.5} \begin{array}{l}
		\mu_1 \left(\frac{x_1}{a}+\frac{z_1}{c}\right)
		+ \nu_1 \left(1+\frac{y_1}{b}\right) = 0, \\
		\mu_1 \left(1-\frac{y_1}{b}\right)
		+ \nu_1 \left(\frac{x_1}{a}-\frac{z_1}{c}\right) = 0.
	\end{array} \right.
\end{equation*}
因为\(\mu_1,\nu_1\)不全为零,所以上式说明方程组\begin{equation*}
	\left\{ \def\arraystretch{1.5} \begin{array}{l}
		\left(\frac{x_1}{a}+\frac{z_1}{c}\right) X
		+ \left(1+\frac{y_1}{b}\right) Y = 0, \\
		\left(1-\frac{y_1}{b}\right) X
		+ \left(\frac{x_1}{a}-\frac{z_1}{c}\right) Y = 0
	\end{array} \right.
\end{equation*}有非零解,从而它的系数行列式等于零.
于是可知,\(M_1(x_1,y_1,z_1)\)在单叶双曲面\(S\)上,
也就是说,\(S\)是直纹面,且直线族\(\mathcal{L}\)是它的一族直母线.

考虑到\begin{equation*}
	\def\arraystretch{1.5}
	\begin{vmatrix}
		\frac{x_0}{a}+\frac{z_0}{c} & 1-\frac{y_0}{b} \\
		1+\frac{y_0}{b} & \frac{x_0}{a}-\frac{z_0}{c}
	\end{vmatrix}
	= \begin{vmatrix}
		\frac{x_0}{a}+\frac{z_0}{c} & 1+\frac{y_0}{b} \\
		1-\frac{y_0}{b} & \frac{x_0}{a}-\frac{z_0}{c}
	\end{vmatrix},
\end{equation*}
我们还可以类似地得到\(S\)的另一族直母线\begin{equation*}
	\left\{ \def\arraystretch{1.5} \begin{array}{l}
		\mu \left(\frac{x}{a}+\frac{z}{c}\right)
		+ \nu \left(1-\frac{y}{b}\right) = 0, \\
		\mu \left(1+\frac{y}{b}\right)
		+ \nu \left(\frac{x}{a}-\frac{z}{c}\right) = 0,
	\end{array} \right.
\end{equation*}
其中\(\mu,\nu\)取所有不全为零的实数.
\end{proof}
\end{theorem}

\begin{theorem}
双曲抛物面是直纹面.
\begin{proof}
设双曲抛物面\(S\)的方程是\begin{equation*}
	\frac{x^2}{p}-\frac{y^2}{q}=2z,
\end{equation*}
则它有两族直母线\begin{equation*}
	\left\{ \def\arraystretch{1.5} \begin{array}{l}
		\left(\frac{x}{\sqrt{p}}+\frac{y}{\sqrt{q}}\right)+2\lambda=0, \\
		z+\lambda\left(\frac{x}{\sqrt{p}}-\frac{y}{\sqrt{q}}\right)=0
	\end{array} \right.
	\quad\text{和}\quad
	\left\{ \def\arraystretch{1.5} \begin{array}{l}
		\lambda\left(\frac{x}{\sqrt{p}}+\frac{y}{\sqrt{q}}\right)+z=0, \\
		2\lambda\left(\frac{x}{\sqrt{p}}-\frac{y}{\sqrt{q}}\right)=0,
	\end{array} \right.
\end{equation*}
其中\(\lambda\)取所有实数.
\end{proof}
\end{theorem}
