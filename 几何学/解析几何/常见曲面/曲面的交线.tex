\section{曲面的交线,曲面围成的区域}
\subsection{画空间图形常见的三种方法}
在纸上描绘空间图形时,我们常用以下三种方法:
\begin{enumerate}
	\item 斜二测法(或称斜二等轴测投影法).
	让\(z\)轴垂直向上,\(y\)轴水平向右,
	\(x\)轴与\(y\)轴、\(z\)轴分别成 135\textdegree 角.
	规定\(y\)轴与\(z\)轴的单位长度相等,
	而\(x\)轴的单位长度为\(y\)轴单位长度的一半.

	\item 正等测法(即正等轴测投影法).
	让\(z\)轴垂直向上,
	\(x\)轴、\(y\)轴、\(z\)轴两两成 120\textdegree 角.
	规定三根轴的单位长度相等.

	\item 正二测法(即正二等轴测投影法).
	具体地说,有以下两种作图方案:
	\begin{itemize}
		\item 让\(z\)轴垂直向上,
		\(x\)轴与\(z\)轴的夹角为\(90^\circ + \alpha\),
		其中\(\alpha\)是锐角,且\(\tan\alpha\approx\frac{7}{8}\);
		\(y\)轴与\(z\)轴的夹角为\(90^\circ + \beta\),
		其中\(\beta\)是锐角,且\(\tan\beta\approx\frac{1}{8}\).
		规定\(z\)轴和\(y\)轴的单位长度相等,
		而\(x\)轴的单位长度为\(y\)轴的单位长度的一半.

		\item 仍然让\(z\)轴垂直向上,
		\(x\)轴与\(z\)轴夹角为\(90^\circ + \beta\),
		其中\(\tan\beta\approx\frac{1}{8}\);
		\(y\)轴的负向与\(z\)轴的夹角为\(90^\circ + \alpha\),
		其中\(\tan\alpha\approx\frac{7}{8}\).
		规定\(x\)轴与\(z\)轴的单位长度相等,
		\(y\)轴的单位长度为\(z\)轴的单位长度的一半.
	\end{itemize}
	一般来说,采用正二测法画出的图形较为逼真.
\end{enumerate}

\subsection{曲线在坐标平面上的投影,曲面的交线的画法}
如\cref{figure:解析几何.点在坐标平面上的投影},
对于空间中任一点\(M\)以及它在三个坐标平面上的投影点\(M_1,M_2,M_3\)这四个点,
只要知道了其中两个点就可以画出另外两个点.
譬如,若知道了\(M_2,M_3\)两个点,则只要分别过\(M_2,M_3\)画出投影线(平行于相应坐标轴的直线),
它们的交点就是点\(M\),再过\(M\)画投影线(平行于\(z\)轴),
它与\(Oxy\)平面的交点就是点\(M_1\).

\begin{figure}[htb]
	\centering
	\begin{tikzpicture}
		\def\spacecoordinate(#1,#2,#3,#4){\coordinate(#1)at(#3-#2*0.35355,#4-#2*0.35355);}%
		\spacecoordinate(O,0,0,0)
		\spacecoordinate(A,1,2,3)
		\spacecoordinate(A1,1,2,0)
		\spacecoordinate(A2,0,2,3)
		\spacecoordinate(A3,1,0,3)

		\begin{scope}[->]
			\draw(O)--(3,0)node[right]{\(y\)};
			\draw(O)--(0,4)node[right]{\(z\)};
			\draw(O)--(-1,-1)node[left]{\(x\)};
		\end{scope}
		\filldraw
			(O)node[above right]{\(O\)}
			(A)circle(2pt)node[right]{\(M\)}
			(A1)circle(2pt)node[right]{\(M_1\)}
			(A2)circle(2pt)node[right]{\(M_2\)}
			(A3)circle(2pt)node[left]{\(M_3\)};
		\draw[dashed]
			(A)--(A1)
			(A)--(A2)
			(A)--(A3);
	\end{tikzpicture}
	\caption{}
	\label{figure:解析几何.点在坐标平面上的投影}
\end{figure}

因此,为了画出两个曲面的交线\(\Gamma\),
只需要先画出\(\Gamma\)上每个点在某两个坐标面上的投影.
\(\Gamma\)上所有点在某个平面\(\alpha\)
(这个平面可以是\(Oxy\)平面、\(Oyz\)平面或\(Ozx\)平面)
上的投影组成的曲线,
称为“\(\Gamma\)在\(\alpha\)上的\DefineConcept{投影曲线}”,
简称为“\(\Gamma\)在\(\alpha\)上的\DefineConcept{投影}”.
如果我们把\(\alpha\)的法线记为\(l\),
把以\(\Gamma\)准线、母线平行于\(l\)的柱面记为\(S\),
那么曲线\(\Gamma\)在\(\alpha\)上的投影就是\(S\)与\(\alpha\)的交线,
我们称柱面\(S\)为“\(\Gamma\)沿\(l\)的\DefineConcept{投影柱面}”.

\subsection{曲面所围成的区域的画法}
几个曲面或平面所围成的空间的区域可用几个不等式联立起来表示.
如何画出这个区域呢?
关键是要画出相应曲面的交线,随之可得所求空间区域.
