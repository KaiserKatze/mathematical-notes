\section{球面和旋转面}
\subsection{球面的一般方程}
我们来求球心为\(M_0(x_0,y_0,z_0)\),半径为\(R\)的球面的方程.
点\(M(x,y,z)\)在这个球面上的充分必要条件是\(\abs{\vec{M_0 M}} = R\),即
\begin{equation}\label{equation:解析几何.球面的标准方程}
	(x-x_0)^2+(y-y_0)^2+(z-z_0)^2=R^2,
\end{equation}
展开得
\begin{equation}\label{equation:解析几何.球面的一般方程}
	x^2 + y^2 + z^2 + 2 b_1 x + 2 b_2 y + 2 b_3 z + c = 0,
\end{equation}
其中\(b_1 = -x_0\),
\(b_2 = -y_0\),
\(b_3 = -z_0\),
\(c = x_0^2 + y_0^2 + z_0^2 - R^2\).

\cref{equation:解析几何.球面的标准方程}
和\cref{equation:解析几何.球面的一般方程}
就是所求球面的方程,
它是一个三元二次方程,
没有交叉项(即含\(xy,xz,yz\)的项),
平方项的系数相同.
反之,任一形如\cref{equation:解析几何.球面的一般方程} 的方程也可经过配方得到
\begin{equation}\label{equation:解析几何.形如球面的一般方程的方程经过配方所得方程}
	(x+b_1)^2 + (y+b_2)^2 + (z+b_3)^2 + (c-b_1^2-b_2^2-b_3^2) = 0.
\end{equation}
当\(b_1^2+b_2^2+b_3^2>c\)时,
\cref{equation:解析几何.形如球面的一般方程的方程经过配方所得方程}
表示一个球心在\((-b_1,-b_2,-b_3)\),半径为\(\sqrt{b_1^2+b_2^2+b_3^2-c}\)的球面;
当\(b_1^2+b_2^2+b_3^2=c\)时,
它表示一个点\((-b_1,-b_2,-b_3)\);
当\(b_1^2+b_2^2+b_3^2<c\)时,
它没有轨迹(或者说它表示一个虚球面).

\subsection{球面的参数方程,点的球面坐标}
如果球心在原点,半径为\(R\),在球面上任取一点\(M(x,y,z)\),
从\(M\)作\(Oxy\)平面的垂线,垂足为\(N\),
连接\(OM,ON\),
设\(x\)轴的正半轴到\(\vec{ON}\)的角度为\(\phi\),
\(\vec{ON}\)到\(\vec{OM}\)的角度为\(\theta\)
(\(M\)在\(Oxy\)平面上方时,\(\theta>0\);否则\(\theta<0\)),
则有
\begin{equation}\label{equation:解析几何.球心在原点半径为R的球面的参数方程}
	\left\{ \begin{array}{l}
		x = R \cos\theta \cos\phi, \\
		y = R \cos\theta \sin\phi, \\
		z = R \sin\theta,
	\end{array} \right.
	\quad
	-\frac{\pi}{2} \leq \theta \leq \frac{\pi}{2},
	-\pi < \phi \leq \pi.
\end{equation}
\cref{equation:解析几何.球心在原点半径为R的球面的参数方程}
称为“球心在原点、半径为R的球面的\DefineConcept{参数方程}”;
它有两个参数\(\theta,\phi\),
其中\(\theta\)称为\DefineConcept{纬度},
\(\phi\)称为\DefineConcept{经度}.
球面上的每一个点(除去它与\(z\)轴的交点)对应唯一的实数对\((\theta,\phi)\),
因此\((\theta,\phi)\)称为“球面上的点的\DefineConcept{曲纹坐标}”.

因为几何空间中任一点\(M(x,y,z)\)必在以原点为球心,yi
以\(R=\abs{\vec{OM}}\)为半径的球面上,
而球面上的点(除去它与\(z\)轴的交点外)
又由它的曲纹坐标\((\theta,\phi)\)唯一确定,
因此,除去\(z\)轴外,几何空间中的点\(M\)由有序三元实数组\((R,\theta,\phi)\)唯一确定.
我们把\((R,\theta,\phi)\)称为“几何空间中的点的\DefineConcept{球面坐标}或\DefineConcept{空间极坐标}”,
其中\begin{equation*}
	R \geq 0,
	\qquad
	-\frac{\pi}{2} \leq \theta \leq \frac{\pi}{2},
	\qquad
	-\pi < \phi \leq \pi.
\end{equation*}

\subsection{曲面和曲线的一般方程、参数方程}
从球面的方程 \labelcref{equation:解析几何.球面的一般方程}
和球面的参数方程 \labelcref{equation:解析几何.球心在原点半径为R的球面的参数方程} 看到,
一般来说,曲面的一般方程是一个三元方程
\begin{equation}\label{equation:解析几何.曲面的一般方程}
	F(x,y,z) = 0,
\end{equation}
曲面的参数方程是含两个参数的方程
\begin{equation}\label{equation:解析几何.曲面的参数方程}
	\left\{ \begin{array}{l}
		x = x(u,v), \\
		y = y(u,v), \\
		z = z(u,v),
	\end{array} \right.
	\quad
	a \leq u \leq b,
	c \leq v \leq d.
\end{equation}
对于\((u,v)\)的每一对值,
由方程 \labelcref{equation:解析几何.曲面的参数方程} 确定的点\((x,y,z)\)都在此曲面上;
而此曲面上任一点的坐标都可由\((u,v)\)的某一对值
通过方程 \labelcref{equation:解析几何.曲面的参数方程} 表示.
于是,通过曲面的参数方程 \labelcref{equation:解析几何.曲面的参数方程},
曲面上的点(可能要除去个别点)便可由数对\((u,v)\)来确定,
于是称\((u,v)\)为“曲面上的点的\DefineConcept{曲纹坐标}”.

我们可以给出如下定义.
\begin{definition}
如果曲面\(S\)与三元方程\begin{equation*}
	F(x,y,z)=0
\end{equation*}满足
\begin{enumerate}
	\item 曲面\(S\)上任一点的坐标都满足这个三元方程,
	\item 不在曲面\(S\)上的点的坐标都不满足这个三元方程,
\end{enumerate}
那么,方程\(F(x,y,z)=0\)就叫做“曲面\(S\)的\DefineConcept{方程}”,
而曲面\(S\)就叫做“方程\(F(x,y,z)=0\)的\DefineConcept{图形}”.
\end{definition}

由于几何空间中曲线可以看成两个曲面的交线,
所以它的一般方程就是联立两个三元方程所得的方程组\begin{equation}\label{equation:解析几何.曲线的一般方程}
	\left\{ \begin{array}{l}
		F(x,y,z) = 0, \\
		G(x,y,z) = 0;
	\end{array} \right.
\end{equation}
而它的参数方程则是含有一个参数的方程
\begin{equation}\label{equation:解析几何.曲线的参数方程}
	\left\{ \begin{array}{l}
		x = x(t), \\
		y = y(t), \\
		z = z(t),
	\end{array} \right.
	\quad
	a \leq t \leq b.
\end{equation}
对于\(t\)的每一个值,
由方程 \labelcref{equation:解析几何.曲线的参数方程} 确定的点\((x,y,z)\)都在此曲线上;
而此曲线上任一点的坐标都可由\(t\)的某个值
通过方程 \labelcref{equation:解析几何.曲线的参数方程} 表示.

\subsection{旋转面}
球面可以看成一个半圆绕它的直径旋转一周所形成的曲面.
现在来研究更一般的情形.

我们称一条曲线\(\Gamma\)绕一条直线\(l\)旋转所得的曲面为\DefineConcept{旋转面},
其中\(l\)称为\DefineConcept{轴},\(\Gamma\)称为\DefineConcept{母线}.

母线\(\Gamma\)上的任一点\(M_0\)绕\(l\)旋转会得到一个圆,我们称之为\DefineConcept{纬圆}.
纬圆所在的平面与轴\(l\)垂直.
过\(l\)的半平面与旋转面的交线称为\DefineConcept{经线}.
经线可以作为母线,但母线不一定是经线.

已知轴\(l\)经过点\(M_1(x_1,y_1,z_1)\),
方向向量为\(\vb{\nu}=(A,B,C)\),
母线\(\Gamma\)的方程为\begin{equation*}
	\left\{ \begin{array}{l}
		F(x,y,z) = 0, \\
		G(x,y,z) = 0.
	\end{array} \right.
\end{equation*}
点\(M(x,y,z)\)在旋转面上的充分必要条件是
\(M\)在经过母线\(\Gamma\)上某一点\(M_0(x_0,y_0,z_0)\)的纬圆上,
即有母线\(\Gamma\)上的一点\(M_0\),
使得\(M\)和\(M_0\)到轴\(l\)的距离相等
(或到轴上一点\(M_1\)的距离相等),
并且\(\vec{M_0 M} \perp l\).
因此有\begin{equation*}
	\left\{ \begin{array}{l}
		F(x_0,y_0,z_0) = 0, \\
		G(x_0,y_0,z_0) = 0, \\
		\abs{\VectorOuterProduct{\vec{M M_1}}{\vb{\nu}}} = \abs{\VectorOuterProduct{\vec{M_0 M_1}}{\vb{\nu}}}, \\
		A(x-x_0) + B(y-y_0) + C(z-z_0) = 0.
	\end{array} \right.
\end{equation*}
从这个方程组中消去参数\(x_0,y_0,z_0\),就得到关于\(x,y,z\)的方程,它就是所求旋转面的方程.

现在设旋转面的轴是\(z\)轴,
母线\(\Gamma\)在\(Oyz\)平面上,
其方程为\begin{equation*}
	\left\{ \begin{array}{l}
		f(y,z) = 0, \\
		x = 0,
	\end{array} \right.
\end{equation*}
则点\(M(x,y,z)\)在旋转面上的充分必要条件是\begin{equation*}
	\left\{ \begin{array}{l}
		f(y_0,z_0) = 0, \\
		x_0 = 0, \\
		x^2+y^2=x_0^2+y_0^2, \\
		1\cdot(z-z_0) = 0.
	\end{array} \right.
\end{equation*}
消去参数\(x_0,y_0,z_0\),得
\begin{equation}\label{equation:解析几何.Oyz平面上的曲线绕z轴旋转所得旋转面的方程}
	f(\pm\sqrt{x^2+y^2},z) = 0.
\end{equation}
\cref{equation:解析几何.Oyz平面上的曲线绕z轴旋转所得旋转面的方程} 就是所求旋转面的方程.
由此可见,为了得到由\(Oyz\)平面上的曲线\(\Gamma\)绕\(z\)轴旋转所得旋转面的方程,
只要将母线\(\Gamma\)在\(Oyz\)平面上的方程中的\(y\)改为\(\pm\sqrt{x^2+y^2}\),保持\(z\)不变.
坐标平面上的曲线绕坐标轴旋转所得旋转面的方程都有类似的规律.

\begin{table}[htb]
	\centering
	\begin{tabular}{|c|c|c|}
	\hline
	曲线方程 & 旋转轴 & 旋转曲面方程 \\ \hline
	\multirow{2}{*}{\(f(x,y)=0\)} & \(x\) & \(f(x,\pm\sqrt{y^2+z^2})=0\) \\ \cline{2-3}
		& \(y\) & \(f(\pm\sqrt{x^2+z^2},y)=0\) \\ \hline
	\multirow{2}{*}{\(f(y,z)=0\)} & \(y\) & \(f(y,\pm\sqrt{x^2+z^2})=0\) \\ \cline{2-3}
		& \(z\) & \(f(\pm\sqrt{x^2+y^2},z)=0\) \\ \hline
	\multirow{2}{*}{\(f(x,z)=0\)} & \(x\) & \(f(x,\pm\sqrt{y^2+z^2})=0\) \\ \cline{2-3}
		& \(z\) & \(f(\pm\sqrt{x^2+y^2},z)=0\) \\
	\hline
	\end{tabular}
	\caption{}
\end{table}


\begin{example}
母线\begin{equation*}
	\Gamma: \left\{ \begin{array}{l}
		y^2 = 2pz, \\
		x = 0,
	\end{array} \right.
	\quad p>0
\end{equation*}
绕\(z\)轴旋转所得旋转面的方程为\begin{equation*}
	x^2+y^2=2pz.
\end{equation*}
这个曲面称为\DefineConcept{旋转抛物面}.
\end{example}

\begin{example}
母线\begin{equation*}
	\Gamma: \left\{ \begin{array}{l}
		\frac{x^2}{a^2}-\frac{y^2}{b^2}=1, \\
		z=0
	\end{array} \right.
\end{equation*}
绕\(x\)旋转所得旋转面的方程为\begin{equation*}
	\frac{x^2}{a^2}-\frac{y^2+z^2}{b^2}=1.
\end{equation*}
这个曲面称为\DefineConcept{旋转双叶双曲面}.

母线\(\Gamma\)绕\(y\)轴旋转所得旋转面的方程为\begin{equation*}
	\frac{x^2+z^2}{a^2}-\frac{y^2}{b^2}=1.
\end{equation*}
这个曲面称为\DefineConcept{旋转单叶双曲面}.
\end{example}

\begin{example}
圆\begin{equation*}
	\left\{ \begin{array}{l}
		(x-a)^2+z^2=r^2, \\
		y=0,
	\end{array} \right.
	\quad 0<r<a
\end{equation*}
绕\(z\)轴旋转所得旋转面的方程为\begin{equation*}
	(\pm\sqrt{x^2+y^2}-a)^2+z^2=r^2,
\end{equation*}
即\begin{equation*}
	(x^2+y^2+z^2+a^2-r^2)^2=4a^2(x^2+y^2).
\end{equation*}
这个曲面称为\DefineConcept{环面}.
\end{example}
