\section{射影平面上的交比}
%@see: 《解析几何》(丘维声) P269
我们知道,在中心投影下,线段的分比会改变,
但是,射影平面上共线四点的交比保持不变.
我们接下来对这个结论予以证明.

\subsection{交比的定义和性质}
%@see: 《平面解析几何(甲种本)》 P7
%@see: 《解析几何》(丘维声) P269
设\(A,B,C\)是欧氏平面\(\pi_0\)上共线三点,\(C \neq B\)(即\(C\)与\(B\)不重合),
我们把线段长\(\LineSegmentLength{AC}\)和\(\LineSegmentLength{CB}\)之比
称为“点\(C\)分线段\(AB\)所成的\DefineConcept{比}”,记作\(R(A,B,C)\),
即\begin{equation*}
%@see: 《解析几何》(丘维声) P269 (3.1)
	R(A,B,C)
	\defeq
	\frac{\LineSegmentLength{AC}}{\LineSegmentLength{CB}}.
\end{equation*}
把点\(C\)称为“线段\(AB\)的\DefineConcept{定比分点}”.
这就是过去我们熟知的定比分点的概念.
根据上述定义,显然不存在点\(C\)使得\(R(A,B,C) = -1\).
因此,我们需要将定比分点从通常点推广为无穷远点.

设\(A,B,C\)是射影平面\(\overline{\pi_0}\)上共线三点,\(A,B\)是通常点,\(C \neq B\).
当\(C\)是通常点时,规定\(R(A,B,C)\)的定义不变.
当\(C\)是无穷远点时,规定:\begin{equation*}
	R(A,B,C) \defeq -1.
\end{equation*}

\begin{table}[hbt]
	\centering
	\begin{tblr}{c|c}
		\hline
		定比分点\(C\)相对于线段端点\(A,B\)的位置
		& 分比\(\lambda \defeq R(A,B,C)\) \\
		\hline
		\(C\)在\(AB\)的延长线长
		& \(\lambda<-1\) \\
		{\color{red}\(C\)是无穷远点}
		& \(\lambda=-1\) \\
		\(C\)在\(BA\)的延长线上
		& \(-1<\lambda<0\) \\
		\(C\)与\(A\)重合
		& \(\lambda=0\) \\
		\(C\)在\(AB\)的中点与\(A\)之间
		& \(0<\lambda<1\) \\
		\(C\)与\(AB\)的中点重合
		& \(\lambda=1\) \\
		\(C\)在\(AB\)的中点与\(B\)之间
		& \(\lambda>1\) \\
		\hline
	\end{tblr}
	\caption{}
\end{table}

接下来定义\(\overline{\pi_0}\)上共线四点的交点.
\begin{definition}%\label{definition:解析几何.射影平面上的交比.四点的交比}
%@see: 《解析几何》(丘维声) P269 定义3.1
设\(A,B,C,D\)是射影平面\(\overline{\pi_0}\)上共线四点,
\(A,B\)是通常点,
\(A \neq B\),
\(C \neq B\),
\(D \neq A\),
\(D \neq B\).
将\(R(A,B,C)\)与\(R(A,B,D)\)之比
称为“点\(A,B,C,D\)的\DefineConcept{交比}”
或“点\(A,B,C,D\)的\DefineConcept{二重比}”,
记作\(R(A,B,C,D)\),
即\begin{equation*}
%@see: 《解析几何》(丘维声) P270 (3.2)
	R(A,B,C,D)
	\defeq
	\frac{R(A,B,C)}{R(A,B,D)}
	= \frac{\LineSegmentLength{AC} \LineSegmentLength{DB}}{\LineSegmentLength{CB} \LineSegmentLength{AD}}.
\end{equation*}
若\(A,B,C\)各不相同,而\(D = B\),则规定:\begin{equation*}
	R(A,B,C,D) \defeq 0.
\end{equation*}
\end{definition}

类似地,我们也可以给射影平面上共点四线规定它们的交比.
\begin{definition}\label{definition:解析几何.射影平面上的交比.四线的交比}
设\(\AutoTuple{l}{4}\)是射影平面\(\overline{\pi_0}\)上经过点\(O\)的四条直线,
\(l_1 \neq l_2\).
任取一条不经过点\(O\)的直线\(l\),
与\(\AutoTuple{l}{4}\)分别交于\(\AutoTuple{A}{4}\).
如果\(R(\AutoTuple{A}{4})\)有定义,
则将点\(\AutoTuple{A}{4}\)的交比\(R(\AutoTuple{A}{4})\)
称为“直线\(\AutoTuple{l}{4}\)的\DefineConcept{交比}”,
记作\(R(\AutoTuple{l}{4})\),
即\begin{equation*}
%@see: 《解析几何》(丘维声) P270 (3.3)
	R(\AutoTuple{l}{4})
	\defeq
	R(\AutoTuple{A}{4}).
\end{equation*}
\end{definition}

下面来证明\cref{definition:解析几何.射影平面上的交比.四线的交比} 是有意义的,
即这样定义的交比与直线\(l\)(称为\DefineConcept{截线})的选取无关.
为此,在\(\pi_0\)上取定一个仿射标架\([O;\AutoTuple{d}{2}]\),
设直线\(l_i\)的齐次坐标为\((\eta_{i1},\eta_{i2},\eta_{i3})\),
点\(A_i\)的齐次坐标为\((a_{i1},a_{i2},a_{i3})\ (i=1,2,3,4)\).
因为\(l_1 \neq l_2\),
由\hyperref[example:解析几何.射影平面上的对偶原理.三线共点的条件]{三线共点的条件}可知\begin{equation*}
	% \eta_{3j} = \lambda_1 \eta_{1j} + \mu_1 \eta_{2j},
	% \qquad
	% \eta_{4j} = \lambda_2 \eta_{1j} + \mu_2 \eta_{2j}
	% \quad(j=1,2,3).
	\begin{bmatrix}
		\eta_{31} \\
		\eta_{32} \\
		\eta_{33}
	\end{bmatrix}
	= \lambda_1
	\begin{bmatrix}
		\eta_{11} \\
		\eta_{12} \\
		\eta_{13}
	\end{bmatrix}
	+ \mu_1
	\begin{bmatrix}
		\eta_{21} \\
		\eta_{22} \\
		\eta_{23}
	\end{bmatrix},
	\qquad
	\begin{bmatrix}
		\eta_{41} \\
		\eta_{42} \\
		\eta_{43}
	\end{bmatrix}
	= \lambda_1
	\begin{bmatrix}
		\eta_{11} \\
		\eta_{12} \\
		\eta_{13}
	\end{bmatrix}
	+ \mu_1
	\begin{bmatrix}
		\eta_{21} \\
		\eta_{22} \\
		\eta_{23}
	\end{bmatrix},
\end{equation*}
即\begin{equation*}
	\begin{bmatrix}
		\eta_{31} & \eta_{41} \\
		\eta_{32} & \eta_{42} \\
		\eta_{33} & \eta_{43}
	\end{bmatrix}
	= \begin{bmatrix}
		\eta_{11} & \eta_{21} \\
		\eta_{12} & \eta_{22} \\
		\eta_{13} & \eta_{23}
	\end{bmatrix}
	\begin{bmatrix}
		\lambda_1 & \lambda_2 \\
		\mu_1 & \mu_2
	\end{bmatrix}.
\end{equation*}
在仿射坐标系中,
若点\(M_1(x_1,y_1)\)与\(M_2(x_2,y_2)\)的连线
与直线\(A x + B y + C = 0\)的交点\(M\),
则\begin{equation*}
	R(M_1,M_2,M)
	= \frac{\LineSegmentLength{M_1M}}{\LineSegmentLength{MM_2}}
	= -\frac{A x_1 + B y_1 + C}{A x_2 + B y_2 + C}.
\end{equation*}
因此,我们得到\begin{equation*}
	R(A_1,A_2,A_3)
	= \frac{A_1A_3}{A_3A_2}
	= -\frac{\eta_{31} \dfrac{a_{11}}{a_{13}} + \eta_{32} \dfrac{a_{12}}{a_{13}} + \eta_{33}}
			{\eta_{31} \dfrac{a_{21}}{a_{23}} + \eta_{32} \dfrac{a_{22}}{a_{23}} + \eta_{33}}
	= -\frac{\mu_1 \left( \eta_{21} \dfrac{a_{11}}{a_{13}} + \eta_{22} \dfrac{a_{12}}{a_{13}} + \eta_{23} \right)}
			{\lambda_1 \left( \eta_{11} \dfrac{a_{12}}{a_{23}} + \eta_{12} \dfrac{a_{22}}{a_{23}} + \eta_{13} \right)},
\end{equation*}
同理可得\begin{equation*}
	R(A_1,A_2,A_4)
	= -\frac{\mu_2 \left( \eta_{21} \dfrac{a_{11}}{a_{13}} + \eta_{22} \dfrac{a_{12}}{a_{13}} + \eta_{23} \right)}
			{\lambda_2 \left( \eta_{11} \dfrac{a_{21}}{a_{23}} + \eta_{12} \dfrac{a_{22}}{a_{23}} + \eta_{13} \right)}.
\end{equation*}
于是\begin{equation*}
%@see: 《解析几何》(丘维声) P271 (3.4)
	R(l_1,l_2,l_3,l_4)
	= R(A_1,A_2,A_3,A_4)
	= \frac{\LineSegmentLength{A_1A_3} \LineSegmentLength{A_4A_2}}
			{\LineSegmentLength{A_3A_2} \LineSegmentLength{A_1A_4}}
	= \frac{\mu_1 \lambda_2}{\lambda_1 \mu_2}
	= \frac{\lambda_2}{\mu_2} \bigg/ \frac{\lambda_1}{\mu_1}.
\end{equation*}
由此可见,\cref{definition:解析几何.射影平面上的交比.四线的交比} 中
共线四点的交比只与这四条直线的相互位置有关,而与截线\(l\)的选取无关.

\begin{figure}[hbt]
%@see: 《解析几何》(丘维声) P271 图7.8
	\centering
	\begin{tikzpicture}[
		label distance=2pt,
	]
		% 依赖 tkz-euclide 宏包
		\coordinate[label=right:$O$](O)at(0,0);
		\coordinate[label=above left:$A_1$](A1)at(-3,-3);
		\coordinate[label=above left:$A_2$](A2)at(-1,-3);
		\coordinate[label=above left:$A_3$](A3)at(1,-3);
		\coordinate[label=above left:$A_4$](A4)at(2,-3);

		\coordinate[label=above left:$A_1'$](A1')at($(O)!.6!(A1)$);
		\coordinate[label=above left:$A_2'$](A2')at($(O)!.55!(A2)$);
		\coordinate[label=above left:$A_3'$](A3')at($(O)!.5!(A3)$);
		\coordinate[label=above right:$A_4'$](A4')at($(O)!.45!(A4)$);

		\tkzDrawLines(A1,A4 A1',A4' O,A1 O,A2 O,A3 O,A4)

		\draw($(A3)!2!(A4)$)node[above]{$l$};
		\draw($(A3')!2!(A4')$)node[below]{$l'$};
		\draw($(A1')!1.2!(A1)$)node[below right]{$l_1$};
		\draw($(A2')!1.2!(A2)$)node[below right]{$l_2$};
		\draw($(A3')!1.2!(A3)$)node[below left]{$l_3$};
		\draw($(A4')!1.2!(A4)$)node[below left]{$l_4$};
	\end{tikzpicture}
	\caption{}
	\label{figure:解析几何.射影平面上的交比.四线的交比1}
\end{figure}

如\cref{figure:解析几何.射影平面上的交比.四线的交比1} 所示,
在以\(O\)为中心的中心投影下,
设\(l\)的像为\(l'\),
\(l'\)与\(l_i\)的交点为\(A_i'\ (i=1,2,3,4)\),
则在这个中心投影下,
\(A_i\)的像是\(A_i'\ (i=1,2,3,4)\).
根据上面的讨论,共点四线的交比与截线的选取无关,得\begin{equation*}
	R(A_1,A_2,A_3,A_4)
	= R(l_1,l_2,l_3,l_4)
	= R(A_1',A_2',A_3',A_4').
\end{equation*}
因此在中心投影下交比保持不变.

由于中心投影可以分解成射影和截影两个步骤,
因此交比在射影和截影下保持不变.
据此可以推广前述交比的定义,
免除前面定义交比时所加上的某些元素不能是无穷远元素的限制.
给定共线四点\(A,B,C,D\),
若\(A,B,C\)各不相同,
并且\(D \neq A\),
则可以把它们的交比\(R(A,B,C,D)\)规定为
它们在某个点\(O\)上的射影\(OA,OB,OC,OD\)的交比;
给定共点四线\(l_1,l_2,l_3,l_4\),
若\(l_1,l_2,l_3\)各不相同,
并且\(l_4 \neq l_1\),
则可以把它们的交比\(R(l_1,l_2,l_3,l_4)\)规定为
它们在某条直线\(l\)上的截影\(P_1,P_2,P_3,P_4\)的交比.
这样规定的交比仍然包含前面的定义作为特例,并且具有性质:
交比在上述射影和截影下保持不变.
这样一来,我们在讨论交比的性质时,
总可以假设它们是共线的四个通常点的交比,
因为如果共线的四个点中有无穷远点,
我们总可以经过射影和截影,把它们变成共线的四个通常点.
如\cref{figure:解析几何.射影平面上的交比.四线的交比2} 所示,
点\(A_1\)是无穷远点,
但是它在点\(O\)上的射影\(OA_1\)在直线\(l'\)上的截影\(A_1'\)是一个通常点.

\begin{figure}[hbt]
%@see: 《解析几何》(丘维声) P271 图7.8
	\centering
	\begin{tikzpicture}[
		label distance=2pt,
	]
		% 依赖 tkz-euclide 宏包
		\coordinate[label=above right:$O$](O)at(0,0);
		\coordinate[label=above left:$A_1$](A1)at(-3,0);
		\coordinate[label=above left:$A_2$](A2)at(-1,-3);
		\coordinate[label=above left:$A_3$](A3)at(1,-3);
		\coordinate[label=above left:$A_4$](A4)at(2,-3);

		\coordinate[label=above right:$A_1'$](A1')at($(O)!.6!(A1)$);
		\coordinate[label=above right:$A_4'$](A4')at($(O)!.45!(A4)$);

		\tkzDrawLines(A2,A4 A1',A4' O,A1 O,A2 O,A3 O,A4)
		\tkzInterLL(A1',A4')(O,A2) \tkzGetPoint{A2'}
		\tkzInterLL(A1',A4')(O,A3) \tkzGetPoint{A3'}

		\draw(A2')+(200:15pt)node[]{$A_2'$};
		\draw(A3')node[below left]{$A_3'$};

		\draw($(A3)!2!(A4)$)node[above]{$l$};
		\draw($(A1')!1.2!(A4')$)node[right]{$l'$};
	\end{tikzpicture}
	\caption{}
	\label{figure:解析几何.射影平面上的交比.四线的交比2}
\end{figure}
