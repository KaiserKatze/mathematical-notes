\section{射影坐标和射影坐标变换}
\subsection{点的射影坐标}
之前我们在扩充欧氏平面\(\overline{\pi_0}\)上定义点的齐次坐标时,
用了欧氏平面\(\pi_0\)上的仿射标架\([O_1;\AutoTuple{d}{2}]\).
我们自然希望能够直接用射影平面\(\overline{\pi_0}\)上的“点”作为标架,
在射影平面\(\overline{\pi_0}\)上建立射影坐标系.

为了解决上述问题,我们首先回顾把\(O\)这个模型,
借助它将上述问题转化成:
能不能用把\(O\)中的直线(对应扩充欧氏平面上的点)作为标架,建立射影坐标系?
我们已经知道,把\(O\)中的任意一条直线\(l\)完全被它的方向向量所决定,
但是\(l\)的方向向量不唯一,它们可以相差一个非零倍数.
基于这一事实,我们分两步来回答上述问题.

首先,我们在把\(O\)中取一个仿射标架\([O;\AutoTuple{d}{3}]\).
这是把\(O\)中的每一条直线\(l\)的方向向量\(v\)在这个仿射标架下就有仿射坐标\((\AutoTuple{x}{3})\).
我们自然会想到,将\((\AutoTuple{x}{3})\)称为“直线\(l\)的坐标”.
这种坐标可以相差一个非零倍数.
这说明:只要在把\(O\)中取定一个仿射标架,那么把\(O\)中的每一条直线就都有了坐标.
但是我们的目的是要以把\(O\)中的直线作为标架,
因此我们要进一步分析:在把\(O\)中取定一个仿射标架\([O;\AutoTuple{d}{3}]\)后可以确定把\(O\)中的几条直线?
显然,\(\AutoTuple{d}{3}\)这三个向量分别确定了把\(O\)中的三条直线\(\AutoTuple{l}{3}\),
其中\(d_i\)就是\(l_i\)的方向向量,
于是\(l_1\)的坐标是\((1,0,0)\),\(l_2\)的坐标是\((0,1,0)\),\(l_3\)的坐标是\((0,0,1)\).
这三条直线\(\AutoTuple{l}{3}\)称为“把\(O\)中的\DefineConcept{基本直线}”
此外,若令\(d = d_1 + d_2 + d_3\),
则\(d\)也确定了把\(O\)中的一条直线\(l_4\),它的坐标是\((1,1,1)\).
直线\(l_4\)称为“把\(O\)中的\DefineConcept{单位直线}”.
由于\(\AutoTuple{d}{3},d\)中任意三个向量都不共面,
因此\(\AutoTuple{l}{4}\)中任意三条直线都不共面.
这样的四条直线称为“把\(O\)中\DefineConcept{一般位置}的四条直线”.
上述讨论说明:把\(O\)中取定一个仿射标架\([O;\AutoTuple{d}{3}]\)后,
就确定了把\(O\)中一般位置的四条直线\(\AutoTuple{l}{4}\),
其中前三条是基本直线,第四条是单位直线.

其次,我们容易想到把上述把\(O\)中一般位置的四条直线\(\AutoTuple{l}{4}\)
合在一起作为把\(O\)中的射影标架.
但是,在这样做之前还要说明一件事,
如果我们在单位直线\(l_4\)上取另一个方向向量\(d'\),
在基本直线\(l_i\)上取方向向量\(d'_i\ (i=1,2,3)\),
使得\(d' = d'_1 + d'_2 + d'_3\),
那么我们又得到把\(O\)中的一个仿射标架\([O;\AutoTuple{d'}{3}]\).
于是,把\(O\)中每一条直线\(l\)在这个仿射标架下也有坐标\((\AutoTuple{x'}{3})\),
它是\(l\)的方向向量\(v\)在标架\([O;\AutoTuple{d'}{3}]\)中的坐标.
而且对于这个仿射标架,直线\(\AutoTuple{l}{4}\)的坐标分别是\(
	(1,0,0),
	(0,1,0),
	(0,0,1),
	(1,1,1)
\).
因此,对于这个仿射标架,\(\AutoTuple{l}{3}\)仍是基本直线,\(l_4\)仍是单位直线.
既然对于三条基本直线和一条单位直线可以得到不同的仿射标架\([O;\AutoTuple{d}{3}]\)和\([O;\AutoTuple{d'}{3}]\),
那么我们必须说明把\(O\)中每一条直线\(l\)的方向向量\(v\)
分别在这两个仿射标架下的坐标\((\AutoTuple{x}{3})\)和\((\AutoTuple{x'}{3})\)是成比例的,
这样才能把上述三条基本直线和一条单位直线合在一起作为把\(O\)中的射影标架.
现在就来说明\((\AutoTuple{x}{3})\)和\((\AutoTuple{x'}{3})\)为什么是成比例的.
这是因为\(d'\)和\(d\)的\(l_4\)的方向向量,
所以可设\(d' = \lambda d\).
同理,可设\(d'_i = \lambda_i d_i\ (i=1,2,3)\).
于是\begin{equation*}
	d'
	= d'_1 + d'_2 + d'_3
	= \lambda_1 d_1 + \lambda_2 d_2 + \lambda_3 d_3,
	\qquad
	d'
	= \lambda d
	= \lambda (d_1 + d_2 + d_3),
\end{equation*}
因此\begin{equation*}
	\lambda_1 d_1 + \lambda_2 d_2 + \lambda_3 d_3
	= \lambda d_1 + \lambda d_2 + \lambda d_3,
\end{equation*}
比较可得\(\lambda_i = \lambda\ (i=1,2,3)\).
所以\begin{equation*}
	v = \sum_{i=1}^3 x'_i d'_i
	= \sum_{i=1}^3 x'_i \lambda d_i.
\end{equation*}
由此得出\(x_i = \lambda x'_i\ (i=1,2,3)\).

综上所述,我们可以给出下述定义.
\begin{definition}\label{definition:解析几何.射影坐标及其变换.把的射影坐标系}
%@see: 《解析几何》(丘维声) P279 定义4.1
设\(l_1,l_2,l_3,l_4\)是把\(O\)中一般位置的四条直线,
在\(l_4\)上取一个方向向量\(d\),
在\(l_i\)上取方向向量\(d_i\ (i=1,2,3)\),
使得\(d = d_1 + d_2 + d_3\),
则称“\([l_1,l_2,l_3,l_4]\)是把\(O\)的一个\DefineConcept{基底}”
或“\([l_1,l_2,l_3,l_4]\)是把\(O\)的一个\DefineConcept{射影标架}”;
同时称“\(l_1,l_2,l_3\)是把\(O\)中的\DefineConcept{基本直线}”
“\(l_4\)是把\(O\)中的\DefineConcept{单位直线}”;
把\(O\)中每一条直线\(l\)的方向向量\(v\)在仿射标架\([O;\AutoTuple{d}{3}]\)下的坐标\((\AutoTuple{x}{3})\)
称为“直线\(l\)在基底\([l_1,l_2,l_3,l_4]\)中的\DefineConcept{齐次射影坐标}”.
\end{definition}

显然,把\(O\)中直线\(l\)在基底\([l_1,l_2,l_3,l_4]\)中的齐次射影坐标不唯一,但是它们成比例;
不同直线在同一基底中的齐次射影坐标不成比例.

既然我们已经在把\(O\)中建立了射影坐标系,
下面就来研究如何在扩充欧氏平面\(\overline{\pi_0}\)上引进射影坐标系.

在欧氏平面\(\pi_0\)外取一点\(O\),
于是扩充欧氏平面\(\overline{\pi_0}\)上的点、直线
分别与把\(O\)中的直线、平面一一对应.
显然,把\(O\)中一般位置的四条直线
对应\(\overline{\pi_0}\)上一般位置的四个点(其中任意三点不共线).
于是容易想到下述定义.
\begin{definition}\label{definition:解析几何.射影坐标及其变换.扩充欧氏平面的射影坐标系}
%@see: 《解析几何》(丘维声) P280 定义4.2
在扩充欧氏平面\(\overline{\pi_0}\)上取定一般位置的四个点\(A_1,A_2,A_3,E\),
在\(\pi_0\)外取一点\(O\).
对于\(\overline{\pi_0}\)上的每一个点\(M\),
将直线\(OM\)在把\(O\)的基底\([OA_1,OA_2,OA_3,OE]\)中的齐次射影坐标\((x_1,x_2,x_3)\)
称为“点\(M\)在扩充欧氏平面\(\overline{\pi_0}\)的基底\([A_1,A_2,A_3,E]\)中的\DefineConcept{齐次射影坐标}”.
点\(A_1,A_2,A_3\)称为\DefineConcept{基本点}.
点\(E\)称为单位点.
\end{definition}

显然,基本点\(A_1,A_2,A_3\)的射影坐标分别为\((1,0,0),(0,1,0),(0,0,1)\),
单位点\(E\)的射影坐标为\((1,1,1)\).
基本点\(A_1,A_2,A_3\)的连线\(A_1A_2,A_2A_3,A_3A_1\)
构成的三角形\(\triangle A_1A_2A_3\)称为\DefineConcept{坐标三角形}.

现在来说明上述定义与\(\pi_0\)外一点\(O\)的取法无关.
在\(OE\)上取一个方向向量\(d\),
在\(OA_i\)上取方向向量\(d_i\ (i=1,2,3)\),
使得\(d = d_1 + d_2 + d_3\).
设\(\vec{OM}\)在仿射标架\([O;\AutoTuple{d}{3}]\)中的坐标为\((\AutoTuple{x}{3})\),
由\cref{definition:解析几何.射影坐标及其变换.把的射影坐标系,definition:解析几何.射影坐标及其变换.扩充欧氏平面的射影坐标系} 可知,
\((x_1,x_2,x_3)\)就是点\(M\)在基底\([A_1,A_2,A_3,E]\)中的射影坐标.
设直线\(A_3E\)与\(A_1A_2\)交于点\(A_{12}\),
直线\(A_3M\)与\(A_1A_2\)交于点\(M_{12}\).
由于\(\vec{OA_{12}},\vec{OA_1},\vec{OA_2}\)共面,以及\(\vec{OA_{12}},\vec{OA_3},\vec{OE}\)共面,
于是可以得到\(\vec{OA_{12}} = \lambda (1,1,0)\).
类似地,由\(\vec{OM_{12}},\vec{OA_1},\vec{OA_2}\)共面,以及\(\vec{OM_{12}},\vec{OA_3},\vec{OM}\)共面,
可得\(\vec{OM_{12}} = \mu (x_1,x_2,0)\).
于是\(OA_1,OA_2,OA_{12},OM_{12}\)的方向向量分别为\(
	d_1,
	d_2,
	d_1 + d_2,
	x_1 d_1 + x_2 d_2
\).
在平面\(OA_1A_2\)上取仿射标架\([O;\AutoTuple{d}{2}]\),
则\(O,A_1,A_2,A_{12},M_{12}\)的齐次坐标分别为\(
	(0,0,1),
	(1,0,t_1),
	(0,1,t_2),
	(1,1,t_3),
	(x_1,x_2,t_4)
\).
容易求出直线\(OA_1,OA_2,OA_{12},OM_{12}\)在齐次坐标中的方程,
从而可得它们的齐次坐标依次为\(
	(0,1,0),
	(-1,0,0),
	(-1,1,0),
	(-x_2,x_1,0)
\).
显然,有\begin{equation*}
	\begin{bmatrix}
		-1 \\ 1 \\ 0
	\end{bmatrix}
	= \begin{bmatrix}
		0 \\ 1 \\ 0
	\end{bmatrix}
	+ \begin{bmatrix}
		-1 \\ 0 \\ 0
	\end{bmatrix},
	\qquad
	\begin{bmatrix}
		-x_2 \\ x_1 \\ 0
	\end{bmatrix}
	= x_1 \begin{bmatrix}
		0 \\ 1 \\ 0
	\end{bmatrix}
	+ x_2 \begin{bmatrix}
		-1 \\ 0 \\ 0
	\end{bmatrix}.
\end{equation*}
于是,由\cref{equation:解析几何.射影平面上的交比.四线的交比1} 可得\begin{equation*}
	R(OA_1,OA_2,OA_{12},OM_{12})
	= \frac{x_1}{x_2} \bigg/ \frac11
	= \frac{x_1}{x_2}.
\end{equation*}
那么,假如点\(M\)不在直线\(A_1A_3\)上,就有\begin{equation*}
	R(A_1,A_2,A_{12},M_{12})
	= x_1 / x_2.
\end{equation*}
类似地,假如点\(M\)不在直线\(A_1A_2\)上,就有\begin{equation*}
	R(A_2,A_3,A_{23},M_{23})
	= x_2 / x_3;
\end{equation*}
假如点\(M\)不在直线\(A_2A_3\)上,就有\begin{equation*}
	R(A_3,A_1,A_{31},M_{31})
	= x_3 / x_1.
\end{equation*}
由此可见,对于不在直线\(A_1A_2,A_2A_3,A_3A_1\)上的点\(M\),
我们可以利用它的射影坐标\(x_1,x_2,x_3\)计算出上述上个交比;
而直线\(A_1A_3\)上的点\((x_1,0,x_3)\),则交比等于\(x_3 / x_1\);
直线\(A_1A_2\)或\(A_2A_3\)上的点类似.
因此\(\overline{\pi_0}\)上每一个点在基底\([A_1,A_2,A_3,E]\)中的射影坐标与点\(O\)取法无关
(唯一的要求就是点\(O\)不在欧氏平面\(\pi_0\)上),
并且点\(M\)的射影坐标的几何意义为\begin{gather*}
	x_1 / x_2
	= R(A_1,A_2,A_{12},M_{12}), \\
	x_2 / x_3
	= R(A_2,A_3,A_{23},M_{23}), \\
	x_3 / x_1
	= R(A_3,A_1,A_{31},M_{31}).
\end{gather*}

不难看出,之前定义的“点\(M\)在\([O_1;\AutoTuple{d}{2}]\)下的一个齐次仿射坐标”
是在特殊基底\([\AutoTuple{A}{3},E]\)中的齐次射影坐标,
其中\(A_1,A_2\)是无穷远点,\(A_3,E\)是通常点.

应该注意到,在一般的基底下,不能用点\(M\)的射影坐标中\(x_3\)是否为零来判定\(M\)是不是无穷远点.

设点\(M\)在基底\([A_1,A_2,A_3,E]\)下的射影坐标为\((x_1,x_2,x_3)\),
那么显然有\begin{equation*}
	\begin{bmatrix}
		x_1 \\ x_2 \\ x_3
	\end{bmatrix}
	= x_1 \begin{bmatrix}
		1 \\ 0 \\ 0
	\end{bmatrix}
	+ x_2 \begin{bmatrix}
		0 \\ 1 \\ 0
	\end{bmatrix}
	+ x_3 \begin{bmatrix}
		0 \\ 0 \\ 1
	\end{bmatrix},
\end{equation*}
即任意一点\(M\)的射影坐标等于三个基本点的射影坐标的线性组合.

\subsection{射影坐标变换公式}
设\(\overline{\pi_0}\)上有两个基底I\([A_1,A_2,A_3,E]\)和II\([A'_1,A'_2,A'_3,E']\),
\(M\)是\(\overline{\pi_0}\)上一点,
\(M\)在基底 I 中的射影坐标为\((x_1,x_2,x_3)\),
\(M\)在基底 II 中的射影坐标为\((x'_1,x'_2,x'_3)\).
我们来讨论\((x_1,x_2,x_3)\)与\((x'_1,x'_2,x'_3)\)之间的关系.

在\(\pi_0\)外取一点\(O\),
把\(O\)中相对地有两个基底\([OA_1,OA_2,OA_3,OE]\)和\([OA'_1,OA'_2,OA'_3,OE']\),
对于这两个基底分别取它们所对应的一个仿射标架
\(\overline{I}[d_1,d_2,d_3,d]\)和\(\overline{II}[d'_1,d'_2,d'_3,d']\),
于是\(\vec{OM}\)在仿射标架\(\overline{I}\)中的坐标为\((\lambda x_1,\lambda x_2,\lambda x_3)\),
\(\vec{OM}\)在仿射标架\(\overline{II}\)中的坐标为\(\lambda' x'_1,\lambda' x'_2,\lambda' x'_3\).
由几何空间中向量的仿射坐标变换公式可知,
若\(d'_j\)的\(\overline{I}\)坐标为\((a_{1j},a_{2j},a_{3j})\ (j=1,2,3)\),
则\begin{equation*}
	\begin{bmatrix}
		\lambda x_1 \\ \lambda x_2 \\ \lambda x_3
	\end{bmatrix}
	= \begin{bmatrix}
		a_{11} & a_{12} & a_{13} \\
		a_{21} & a_{22} & a_{23} \\
		a_{31} & a_{32} & a_{33}
	\end{bmatrix}
	\begin{bmatrix}
		\lambda' x'_1 \\ \lambda' x'_2 \\ \lambda' x'_3
	\end{bmatrix},
\end{equation*}
即\begin{equation}\label{equation:解析几何.射影坐标及其变换.射影坐标变换公式}
%@see: 《解析几何》(丘维声) P283 (4.1)
	\rho \begin{bmatrix}
		x_1 \\ x_2 \\ x_3
	\end{bmatrix}
	= \begin{bmatrix}
		a_{11} & a_{12} & a_{13} \\
		a_{21} & a_{22} & a_{23} \\
		a_{31} & a_{32} & a_{33}
	\end{bmatrix}
	\begin{bmatrix}
		x'_1 \\ x'_2 \\ x'_3
	\end{bmatrix},
\end{equation}
其中\(\rho\)是非零常数.
对于不同的点,\(\rho\)的值不同.
\cref{equation:解析几何.射影坐标及其变换.射影坐标变换公式} 就是点的射影坐标变换公式.
由于\(d'_j,d'\)分别是\(\vec{OA'_j},\vec{OE'}\)的方向向量,
因此\(
	d'_j = \lambda_j \vec{OA'_j},
	d' = \lambda \vec{OE'}
	\ (j=1,2,3)
\).
由于\(d' = d'_1 + d'_2 + d'_3\),
因此\(
	\lambda \vec{OE'}
	= \lambda_1 \vec{OA'_1}
	+ \lambda_2 \vec{OA'_2}
	+ \lambda_3 \vec{OA'_3}
\).
又因为\(\vec{OA'_j},\vec{OE'}\)的\(\overline{I}\)坐标分别于\(A'_j,E'\)的 I 坐标成比例,
那么可以求出\(\lambda_j\),
进而求出\(d'_j\)的\(\overline{I}\)坐标,
最后写出射影坐标变换公式.

\subsection{直线的射影坐标方程}
设相异两点\(P,Q\)在基底\([A_1,A_2,A_3,E]\)中的射影坐标分别为\((p_1,p_2,p_3),(q_1,q_2,q_3)\).
点\(M(\AutoTuple{x}{3})\)在直线\(PQ\)上的充分必要条件是把\(O\)中相应的直线\(OM,OP,OQ\)共面,
从而存在不全为零的数\(\lambda,\mu\),使得它们的方向向量\(v_M,v_P,v_Q\)使得\begin{equation}
	v_M = \lambda v_P + \mu v_Q,
\end{equation}
即\begin{equation}\label{equation:解析几何.射影坐标及其变换.射影平面上相异两点确定的直线的射影坐标参数方程}
%@see: 《解析几何》(丘维声) P284 (4.3)
	\begin{bmatrix}
		x_1 \\ x_2 \\ x_3
	\end{bmatrix}
	= \lambda \begin{bmatrix}
		p_1 \\ p_2 \\ p_3
	\end{bmatrix}
	+ \mu \begin{bmatrix}
		q_1 \\ q_2 \\ q_3
	\end{bmatrix},
\end{equation}
亦即\begin{equation*}
	\begin{vmatrix}
		x_1 & p_1 & q_1 \\
		x_2 & p_2 & q_2 \\
		x_3 & p_3 & q_3
	\end{vmatrix}
	= 0,
\end{equation*}
展开得\begin{equation}\label{equation:解析几何.射影坐标及其变换.射影平面上相异两点确定的直线的射影坐标方程}
%@see: 《解析几何》(丘维声) P284 (4.4)
	\eta_1 x_1 + \eta_2 x_2 + \eta_3 x_3 = 0,
\end{equation}
其中\begin{equation*}
	\eta_1 \defeq \begin{vmatrix}
		p_2 & q_2 \\
		p_3 & q_3
	\end{vmatrix},
	\qquad
	\eta_2 \defeq \begin{vmatrix}
		p_3 & q_3 \\
		p_1 & q_1
	\end{vmatrix},
	\qquad
	\eta_3 \defeq \begin{vmatrix}
		p_1 & q_1 \\
		p_2 & q_2
	\end{vmatrix}
\end{equation*}
不全为零(因为\(P,Q\)是相异两点,所以\(v_P,v_Q\)不共线).
这说明,直线\(PQ\)的射影坐标方程
\labelcref{equation:解析几何.射影坐标及其变换.射影平面上相异两点确定的直线的射影坐标方程}
是一个三元一次齐次方程.
反过来,容易看出,任意一个三元一次齐次方程(在射影坐标系中)都表示射影平面上的一条直线.

\cref{equation:解析几何.射影坐标及其变换.射影平面上相异两点确定的直线的射影坐标参数方程}
称为“直线\(PQ\)在给定射影坐标系中的\DefineConcept{参数方程}”.

容易看出,两个方程\begin{equation*}
	\eta_1 x_1 + \eta_2 x_2 + \eta_3 x_3 = 0
	\quad\text{与}\quad
	\omega_1 x_1 + \omega_2 x_2 + \omega_3 x_3 = 0
\end{equation*}
表示同一条直线的充分必要条件是\((\eta_1,\eta_2,\eta_3)\)与\((\omega_1,\omega_2,\omega_3)\)成比例.
因此,我们将\((\eta_1,\eta_2,\eta_3)\)
称为“直线\(\eta_1 x_1 + \eta_2 x_2 + \eta_3 x_3 = 0\)的\DefineConcept{齐次射影坐标}”.

从上述讨论还可以看出:\begin{enumerate}
	\item 在射影平面上,设三点\(P,Q,R\)的射影坐标分别为\(
		(p_1,p_2,p_3),
		(q_1,q_2,q_3),
		(r_1,r_2,r_3)
	\),
	则\(P,Q,R\)三点共线的充分必要条件是\begin{equation*}
		\begin{vmatrix}
			p_1 & q_1 & r_1 \\
			p_2 & q_2 & r_2 \\
			p_3 & q_3 & r_3
		\end{vmatrix}
		= 0;
	\end{equation*}

	\item 在射影平面上,设三点\(P,Q,R\)的射影坐标分别为\(
		(p_1,p_2,p_3),
		(q_1,q_2,q_3),
		(r_1,r_2,r_3)
	\),
	\(P\)与\(Q\)不重合,
	则点\(R\)在直线\(PQ\)上的充分必要条件是,
	存在不全为零的数\(\lambda,\mu\),
	使得\begin{equation*}
		\begin{bmatrix}
			r_1 \\ r_2 \\ r_3
		\end{bmatrix}
		= \lambda \begin{bmatrix}
			p_1 \\ p_2 \\ p_3
		\end{bmatrix}
		+ \mu \begin{bmatrix}
			q_1 \\ q_2 \\ q_3
		\end{bmatrix}.
	\end{equation*}
\end{enumerate}

\subsection{用射影坐标计算交比}
之前我们得到了用齐次坐标计算交比的公式,
即\cref{equation:解析几何.射影平面上的交比.四线的交比1,equation:解析几何.射影平面上的交比.四点的交比1}.
现在我们来说明这两个公式在射影坐标下仍然成立.

\begin{theorem}
%@see: 《解析几何》(丘维声) P286 定理4.1
设\(A,B,C,D\)是射影平面\(\overline{\pi_0}\)上共线四点,
其中\(A,B,C\)各不相同,
并且\(D \neq A\).
又设\(A,B,C,D\)在任一基底\([A_1,A_2,A_3,E]\)中的射影坐标分别为\(
	(\AutoTuple{a}{3}),
	(\AutoTuple{b}{3}),
	(\AutoTuple{c}{3}),
	(\AutoTuple{d}{3})
\),
且\begin{equation*}
	\begin{bmatrix}
		c_1 & d_1 \\
		c_2 & d_2 \\
		c_3 & d_3
	\end{bmatrix}
	= \begin{bmatrix}
		a_1 & b_1 \\
		a_2 & b_2 \\
		a_3 & b_3
	\end{bmatrix}
	\begin{bmatrix}
		\lambda_1 & \lambda_2 \\
		\mu_1 & \mu_2
	\end{bmatrix},
\end{equation*}
则\begin{equation}%\label{equation:解析几何.射影坐标及其变换.四点的交比1}
%@see: 《解析几何》(丘维声) P286 (4.5)
	R(A,B,C,D)
	= \frac{\lambda_2}{\mu_2} \bigg/ \frac{\lambda_1}{\mu_1}.
\end{equation}
\end{theorem}

类似地,关于用齐次坐标计算共点四线的交比的公式 \labelcref{equation:解析几何.射影平面上的交比.四线的交比1} 在射影坐标中仍然成立.

此外,由\cref{equation:解析几何.射影平面上的交比.四点的交比2} 的推到过程可以看出,
既然\cref{equation:解析几何.射影平面上的交比.四点的交比1} 对于射影坐标也成立,
那么\cref{equation:解析几何.射影平面上的交比.四点的交比2} 对于射影坐标同样成立.
并且,共点四线的交比的类似于\cref{equation:解析几何.射影平面上的交比.四点的交比2} 的公式对于射影坐标也成立.

\subsection{点的非齐次射影坐标}
\begin{definition}
%@see: 《解析几何》(丘维声) P287 定义4.3
在射影平面上取一个基底I\([A_1,A_2,A_3,E]\),
对于不在直线\(A_1A_2\)上的点\(M(x_1,x_2,x_3)\),
记\(
	x \defeq x_1 / x_3,
	y \defeq x_2 / x_3
\),
把\((x,y)\)
称为“点\(M\)的\DefineConcept{非齐次射影坐标}”.
\end{definition}

我们之前提到过一个点的齐次射影坐标的几何意义:\begin{equation*}
	x_1 / x_3 = R(A_1,A_3,A_{13},M_{13}),
	\qquad
	x_2 / x_3 = R(A_2,A_3,A_{23},M_{23}),
\end{equation*}
因此点\(M\)的非齐次射影坐标实际上就是交比(如\cref{definition:解析几何.射影坐标及其变换.点的非齐次射影坐标} 所示).

\begin{figure}[hbt]
%@see: 《解析几何》(丘维声) P287 图7.14
	\centering
	\begin{tikzpicture}[
		label distance=2pt,
	]
		% 依赖 tkz-euclide 宏包
		\coordinate[label=above:$A_1$](A1)at(0,0);
		\coordinate[label=left:$A_2$](A2)at(-2.4,-3.4);
		\coordinate[label=right:$A_3$](A3)at(2,-3.4);
		\coordinate[label=left:$E$](E)at(0,-2);
		\coordinate[label=right:$M$](M)at(-.8,-3);

		\tkzInterLL(A1,E)(A2,A3) \tkzGetPoint{A23} \draw(A23)node[below]{$A_{23}$};
		\tkzInterLL(A1,M)(A2,A3) \tkzGetPoint{M23} \draw(M23)node[below]{$M_{23}$};
		\tkzInterLL(A2,E)(A1,A3) \tkzGetPoint{A13} \draw(A13)node[right]{$A_{13}$};
		\tkzInterLL(A2,M)(A1,A3) \tkzGetPoint{M13} \draw(M13)node[right]{$M_{13}$};

		\tkzDrawPoints(E,M)

		\draw(A1)--(A2)--(A3)--cycle;
		\draw(A13)--(A2)--(M13);
		\draw(A23)--(A1)--(M23);
	\end{tikzpicture}
	\caption{}
	\label{definition:解析几何.射影坐标及其变换.点的非齐次射影坐标}
\end{figure}

取另一基底II\([A'_1,A'_2,A'_3,E']\),
对于既不在直线\(A_1A_2\)上,
又不在直线\(A'_1A'_2\)上的任意一点\(M\),
假设它在基底 I 、II 中的齐次射影坐标分别是\((x_1,x_2,x_3)\)、\((x'_1,x'_2,x'_3)\),
非齐次射影坐标分别是\((x,y)\)、\((x',y')\),
则\begin{equation*}
	x = x_1 / x_3,
	y = x_2 / x_3,
	x' = x'_1 / x'_3,
	y' = x'_2 / x'_3.
\end{equation*}
由齐次射影坐标变换公式 \labelcref{equation:解析几何.射影坐标及其变换.射影坐标变换公式} 可得\begin{equation*}
	\rho x_i = \sum_{j=1}^3 a_{ij} x'_j
	\quad(i=1,2,3),
\end{equation*}
从而有\begin{equation}\label{equation:解析几何.射影坐标及其变换.点的非齐次射影坐标变换公式}
%@see: 《解析几何》(丘维声) P288 (4.6)
	\begin{bmatrix}
		x \\ y
	\end{bmatrix}
	= \frac1{a_{31} x' + a_{32} y' + a_{33}}
	\begin{bmatrix}
		a_{11} x' + a_{12} y' + a_{13} \\
		a_{21} x' + a_{22} y' + a_{23}
	\end{bmatrix}.
\end{equation}
\cref{equation:解析几何.射影坐标及其变换.点的非齐次射影坐标变换公式}
是点的非齐次射影坐标变换公式,
它表明一个点在基底 I 中的非齐次射影坐标\((x,y)\)
可以用这个点在基底 II 中的非齐次射影坐标\((x',y')\)的分式线性函数来表达.
