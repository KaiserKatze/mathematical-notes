%@see: 《解析几何》(丘维声) P251
迄今为止,我们介绍了解析几何的主要研究方法
--- 坐标法、向量法、坐标变换法以及点变换(正交变换和仿射变换)法,
并且利用这些方法研究了一些图形的度量性质和仿射性质.
所有这些都是在几何空间中进行的.

我们已经看到平面仿射变换的重要特征是:
仿射变换是双射,它把共线三点映成共线三点.
在实际生活中,我们还会遇到更一般的映射,
它将一个平面的点映成另一个平面的点,
在此过程中保持点的共线关系不变.
譬如,在航空摄影时,我们要把地面(假定是平坦的)上的景物摄到底片上,
这可以看成地平面\(\pi_0\)到底片\(\pi_1\)的一种保持点的共线关系不变的映射.
通常来说,摄像机镜头不会恰好垂直地对着大地,
因此\(\pi_0\)与\(\pi_1\)不平行,
或者说\(\pi_0\)与\(\pi_1\)是两个相交平面.
假设点\(O\)是平面\(\pi_0\)和\(\pi_1\)外一点.
将平面\(\pi_0\)上的每一个点\(P\),
映成平面\(\pi_1\)与直线\(OP\)的交点\(P'\)的映射,
称为从平面\(\pi_0\)到\(\pi_1\)、以点\(O\)为中心的\DefineConcept{中心投影}.
显然,在中心投影下,点的共线关系仍然是保持的.
但是,\(\pi_0\)上的点\(M\)如果满足\(OM \parallel \pi_1\),
则\(M\)在\(\pi_1\)上没有像;
同样地,\(\pi_1\)上的点\(N\)如果满足\(ON \parallel \pi_0\),
则\(N\)在\(\pi_0\)上没有原像.
为了弥补这些“缺陷”,使中心投影成为映射,并且为双射,
就需要在平面\(\pi_1\)上添加一些新的点,
使得\(\pi_0\)上类似\(M\)这样的点(它们形成一条直线,是两个平面的交线,
其中一个平面经过点\(O\)且与\(\pi_1\)平行,
另一个平面就是\(\pi_0\))都有像;
而在平面\(\pi_0\)上也添加一些新的点,
使得\(\pi_1\)上\(N\)这样的点(它们也形成一条直线,也是两个平面交线,
其中一个平面经过点\(O\)且与\(\pi_0\)平行,
另一个平面就是\(pi_1\))都有原像.
像这样,添加一些点以后,就形成了射影平面.

本章的主要研究内容是:
射影平面,
射影平面导自身(或到另一个射影平面)的具有保持点的共线关系不变的特性的双射(即射影映射、射影变换),
以及研究射影平面上的一次曲线和二次曲线.
