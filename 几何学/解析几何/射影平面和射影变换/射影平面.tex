\section{射影平面,齐次坐标}
\subsection{以点为中心的把,扩充欧氏平面}
%@see: 《解析几何》(丘维声) P252
本节研究一个平面上的各个几何元素之间的关联关系.
由\cref{theorem:欧氏几何.定理2}
可知过一条直线\(l\)和不在\(l\)上的一点\(P\),有且仅有一个平面(记为\(\pi\)).
于是平面\(\pi\)就可以看成是一些点和直线的集合.
用同样的观点来看点\(O\),那么我们看到的就是:
所有与点\(O\)关联的平面\(\lambda\)和直线\(p\).
今后我们将与一个点\(O\)关联的所有平面和直线的集合
称为“以点\(O\)为中心的\DefineConcept{把}”.

接下来,继续按这个观点,考察中心投影的构成.
我们会发现,把与平面之间存在简单而且自然的对应关系.
设点\(O\)不在平面\(\pi\)上,则平面\(\pi\)上的每一个点\(P\)决定把\(O\)中的一条直线\(OP\);
平面\(\pi\)上的每一条直线\(l\)决定把\(O\)中的一个平面\(Ol\).
这样就得到一个从平面\(\pi\)到把\(O\)的一个映射,
称为“平面\(\pi\)在把\(O\)上的\DefineConcept{射影}”.
反过来,从把\(O\)到平面\(\pi\)也有一个对应关系.
把\(O\)中的每一个平面\(\lambda\)与平面\(\pi\)交于一条直线(用\(\pi\lambda\)表示),
把\(O\)中的每一条直线\(p\)与平面\(\pi\)交于一个点(用\(\pi p\)表示).
我们将这个对应关系称为“把\(O\)在平面\(\pi\)上的\DefineConcept{截影}”.

容易看出,经过射影和截影,关联关系是不变的.

我们还可以看出,中心投影可以分解成射影、截影两个步骤:
从平面\(\pi_0\)到\(\pi_1\)、以点\(O\)为中心的中心投影,
实际上就是\(\pi_0\)在把\(O\)上的射影,
与把\(O\)在\(\pi_1\)上的截影,
这两个映射的复合.
在\cref{figure:解析几何.射影平面和射影变换.中心射影} 中,
平面\(\pi_0\)上的点\(M\)在中心投影下没有像,
这是因为把\(O\)的直线\(OM\)在截影下没有像;
平面\(\pi_1\)上的点\(N\)在中心投影下没有原像,
这是因为把\(O\)的直线\(ON\)在射影下没有原像.
为什么把\(O\)中有的直线在截影下没有像,
有的直线在射影下没有原像,
其原因就在于欧几里得平面的结构与把的结构是不同的.
弥补这一缺陷的方法,应当是将把作为模型,对欧几里得平面加以扩充.

现在我们就将把\(O\)作为模型,来扩充欧几里得平面,
使得射影和截影能够成为它们之间的、保持关联关系不变的双射.
取一个欧几里得平面\(\pi_0\),使点\(O\)不在\(\pi_0\)上.
容易看出,在从\(\pi_0\)到把\(O\)的射影下,
把\(O\)中与\(\pi_0\)平行的每一个平面\(\pi_0'\)
以及\(\pi_0'\)上的每一条直线都没有原像;
而在从把\(O\)到\(\pi_0\)的截影下,
把\(O\)中的每一个平面\(\pi_0'\)以及\(\pi_0'\)上的每一条直线都没有像.
为了使射影和截影都成为双射,
就应当在\(\pi_0\)上加进一条直线,让它和把\(O\)中的平面\(\pi_0'\)对应,
还应当在\(\pi_0\)上添上一些点,让它们和把\(O\)中的平面\(\pi_0'\)上的直线对应,
而且像这样添加了新的元素(直线和点)以后,
应该让射影和截影仍旧保持关联关系不变.
为此,采取如下的扩充方法:
考虑到\(\pi_0\)上的每一条直线\(l\)在把\(O\)中的对应平面\(Ol\)
与\(\pi_0'\)有唯一的交线\(l'\)(如\cref{figure:解析几何.射影平面和射影变换.无穷远点} 所示),
规定在\(\pi_0\)的每一条直线\(l\)上加进唯一的一个点,来与\(l'\)相对应.
我们将新添加的这个与\(l'\)对应的点称为\DefineConcept{无穷远点}.
相对地,为了以示区别,我们将原有的、不是无穷远点的那些点称为\DefineConcept{通常点}.
由于\(\pi_0\)上平行的直线在把\(O\)中的对应平面与\(\pi_0'\)的交线都相同
(与\(l\)平行的直线\(m\)在把\(O\)中对应平面\(Om\)与\(\pi_0'\)的交线还是\(l'\)),
而\(\pi_0\)上不平行的直线在把\(O\)中的对应平面与\(\pi_0'\)的交线都不同,
因此规定:
\(\pi_0\)上平行的直线有相同的无穷远点,
\(\pi_0\)上不平行的直线有不同的无穷远点.
最后,为了与把\(O\)中的平面\(\pi_0'\)对应,
我们在\(\pi_0\)上还要加进一条直线作为\(\pi_0\)与\(\pi_0'\)的交线.
我们将新添加的这条直线称为\DefineConcept{无穷远直线}.
于是无穷远直线对应于把\(O\)中的平面\(\pi_0'\).
由于无穷远点所对应的把\(O\)中的直线都在\(\pi_0'\)上,
因此规定无穷远点都在无穷远直线上.

\begin{figure}[hbt]
%@see: 《解析几何》(丘维声) P254 图7.2
	\centering
	\begin{tikzpicture}
		\draw(0,0)--(4,0)--++(1,{sqrt(3)})node[below left]{$\pi_0'$}--++(-4,0)--(0,0);
		\draw(.6,0)coordinate(l1a)--(3,{sqrt(3)})coordinate(l1b)node[above]{$l'$};
		\fill($.7*(l1a) + .3*(l1b)$)coordinate(O)circle(2pt)node[above left]{$O$};
		\pgfmathsetmacro{\kp}{1.2}
		\draw(-1.5,-4)coordinate(p0a)
			--(4.5,-4)coordinate(p0b)node[above left]{$\pi_0$}
			--++(1.2,{\kp*sqrt(3)})coordinate(p0c);
		\draw($(p0c) + (p0a) - (p0b)$)coordinate(p0d)--(p0a);
		\draw[name path=l1al2a](l1a)--(-.5,-4)coordinate(l2a);
		\draw[name path=lal1a](l1a)--(1.5,-4)coordinate(la);
		\draw(la)--($(la) + \kp*(l1b) - \kp*(l1a)$)coordinate(lb)node[pos=.7,below]{$l$};
		\path[name path=l2](l2a)--($(l2a) + \kp*(l1b) - \kp*(l1a)$)coordinate(l2b);
		\draw[dashed](l1b)--(l2b);
		\draw(l1b)--(lb);
		\draw(p0c)--(lb);
		\path[name path=p0cd](p0c)--(p0d);
		\begin{scope}[name intersections={of=p0cd and l1al2a}]
			\draw[dashed](intersection-1)--(lb);
			\draw(intersection-1)--(p0d);
		\end{scope}
		\begin{scope}[name intersections={of=lal1a and l2}]
			% \fill[color=red](intersection-1)circle(2pt);
			\draw(l2a)--(intersection-1)node[midway,above]{$m$};
			\draw[dashed](intersection-1)--(l2b);
		\end{scope}
	\end{tikzpicture}
	\caption{}
	\label{figure:解析几何.射影平面和射影变换.无穷远点}
\end{figure}

把上面的讨论总结一下:
对于欧几里得平面\(\pi_0\),
在它的每一条直线上都加进了一个无穷远点\(P_\infty\),
平行的直线有相同的无穷远点,
不平行的直线有不同的无穷远点
(这样一来,与每个无穷远点对应的时一个完全确定的方向),
所有无穷远点组成一条无穷远直线\(l_\infty\).
补充了这样一些无穷远点和一条无穷远直线的欧几里得平面\(\pi_0\),
就称为一个\DefineConcept{扩充欧几里得平面},
记作\(\overline{\pi_0}\).

显然,扩充欧氏平面\(\overline{\pi_0}\)
与把\(O\)之间的射影和截影都是保持关联关系不变的双射,
也就是说:
\(\overline{\pi_0}\)上的所有点
与把\(O\)中的所有直线一一对应;
\(\overline{\pi_0}\)上的所有直线
与把\(O\)中的所有平面一一对应;
\(\overline{\pi_0}\)上一点\(P\)与一直线\(l\)关联,
当且仅当把\(O\)中与\(P\)对应的直线\(OP\)和与\(l\)对应的平面\(Ol\)关联.
因此,当我们只考虑基本几何元素之间的关联关系时,
扩充欧氏平面的结构与把的结构在本质上是一样的.
由此我们抽象出射影平面的概念.

\subsection{射影平面的定义和几何模型}
%@see: 《解析几何》(丘维声) P255 定义1.1
由两类分别称为“点”和“直线”的元素所构成的集合\(S\),
如果在其中的“点”和“直线”之间规定了某种名为“关联”的关系,
并且\(S\)中所有“点”和所有“直线”
可以分别与欧氏空间中的一个把\(O\)中的所有直线和所有平面建立一一对应关系,
使得对应关系保持关联性,
则称\(S\)是一个\DefineConcept{射影平面}.

%@see: 《解析几何》(丘维声) P255 例1.1
对于几何空间中的任意一个把\(O\),
如果将其中的直线称为“点”,将其中平面称为“直线”,
那么它就是一个射影平面.

%@see: 《解析几何》(丘维声) P255 例1.2
扩充欧氏平面\(\overline{\pi_0}\)是一个射影平面.

%@see: 《解析几何》(丘维声) P255 例1.3
取定一个球面,球心为\(O\).
将球面上每一对{对径点}(即位于直径两端的两个点)看成一个“点”.
将球面上每一个大圆(即球面与过球面的平面的交线)看成一条“直线”.
如果一对对径点在一个大圆上,那么我们说“点”在“直线”上.
这样我们就得到一个射影平面.

%@see: 《解析几何》(丘维声) P256
任意两个射影平面,由于它们的“点”“直线”分别有一一对应关系,且该对应保持关联性,
因此它们的有关“点”和“直线”的关联关系方面的性质是相同的.
于是,我们在研究这方面的性质时,
只要取一个射影平面作为代表来研究就可以了.
今后我们常常取把\(O\)或者扩充欧氏平面\(\overline{\pi_0}\)作为代表,
这是因为把\(O\)中“点”和“直线”的关联关系非常直观,
并且由于把\(O\)在几何空间中,
这样我们可以运用几何空间的有关知识来研究把\(O\)的性质;
至于扩充欧氏平面\(\overline{\pi_0}\),
由于它是由欧氏平面扩充得到的,
所以可以通过\(\overline{\pi_0}\)来解决欧氏平面上的一些问题.

现在我们就利用把\(O\)来研究射影平面上点与直线的关联关系.
从把\(O\)看,两条不同的直线唯一地决定一个经过它们的平面;
反过来,两个不同的平面相交于唯一的一条直线.
所以,在射影平面上有以下两条性质:
\begin{enumerate}
	\item 两个点必与唯一一条直线关联;
	\item 两条直线必与唯一一个点关联.
\end{enumerate}

接着我们在扩充欧氏平面\(\overline{\pi_0}\)上检验一下上述事实.
欧氏平面上的一条直线(简称欧氏直线)加进一个无穷远点后
就成为扩充欧氏平面上的一条直线,
这样的直线称为扩充欧氏平面上的射影直线.
如上所述,\(\overline{\pi_0}\)上的一条射影直线是由\(\pi_0\)的一条欧氏直线唯一确定的,
从而在\(\overline{\pi_0}\)的全体射影直线与\(\pi_0\)的全体直线之间存在一个一一对应关系.
但是要注意,\(\overline{\pi_0}\)上除了射影直线外,还有一条无穷远直线.

下面来看\(\overline{\pi_0}\)上两个点必与唯一一条直线关联的意义.
如果这两个点都是\(\pi_0\)上的通常点,
则它们决定一条欧氏直线,
从而决定了一条射影直线.
如果这两个点中有一个是通常点,另一个是无穷远点,
那么它们也决定一条欧氏直线
(就是经过这个通常点,而与给定的无穷远点所对应的方向平行的直线),
因而它们也就决定了一条射影直线.
如果这两个点都是无穷远点,
则它们决定了唯一一条无穷远直线.

再看\(\overline{\pi_0}\)上两条直线必与唯一一个点关联的意义.
两条射影直线交于一个通常点(它们对应的两条欧氏直线不平行)
或者交于一个无穷远点(它们对应的两条欧氏直线平行);
一条射影直线和无穷远直线交于一个无穷远点.

由此可见,平行的概念在射影平面上是没有意义的,因为射影平面上任意两条直线总相交.

射影平面与欧氏平面还有一些不同之处.
例如,射影平面上的直线是封闭的,好比圆周一样.
从按照把的模型引进无穷远点的规定可以直观地看出这一点:
考虑\(\overline{\pi_0}\)上两个通常点在一条直线\(l\)上沿着相反的方向跑向无穷远处,
在把\(O\)中与这两个点对应的两条直线都将趋于把\(O\)中与\(l\)平行的那条唯一的直线,
这说明\(\overline{\pi_0}\)上两个点在一条直线上沿着相反的方向远离时将达到同一个无穷远点.
又如,在欧氏平面\(\pi_0\)上,一条直线将平面分成两部分,
但是射影平面\(\overline{\pi_0}\)上的一条直线不能分割射影平面.
譬如,在\(\overline{\pi_0}\)上把无穷远直线去掉,
亦即除掉所有的无穷远点,
我们得到的时一个欧氏平面\(\pi_0\),
可见无穷远直线并没有把\(\overline{\pi_0}\)分成两块.
由于从把的结构上看,
射影平面上的无穷远直线与其他直线并无本质的不同,
因此\(\overline{\pi_0}\)上的任意一条直线都不能把\(\overline{\pi_0}\)分成两块.

\subsection{实射影平面}
%@see: 《解析几何》(丘维声) P257
几何空间是由所有点组成的集合,
也可看成以定点\(O\)为起点的所有定位向量组成的集合.
把几何空间看成后者,则它是实数域\(\mathbb{R}\)上的三维线性空间,
经过点\(O\)的直线是它的一维子空间,
经过点\(O\)的平面是它的二维子空间,
于是,把\(O\)中的直线就是几何空间的一维子空间,把\(O\)中的平面就是几何空间的二维子空间.
由于把\(O\)本身就是一个射影平面,
于是受此启发,我们考虑下述几何模型.

设\(V\)是实数域\(\mathbb{R}\)上的三维线性空间,
将\(V\)的每一个一维子空间称为“点”,
将\(V\)的每一个二维子空间称为“直线”,
并且规定若向量\(\alpha\)生成的一维子空间\(\Span\{\alpha\}\)
包含于向量\(\beta,\gamma\)生成的二维子空间\(\Span\{\beta,\gamma\}\),
则称“点”\(\Span\{\alpha\}\)与“直线”\(\Span\{\beta,\gamma\}\)关联,
那么由这样的“点”和“直线”构成的集合,记作\(\mathcal{P}(\mathbb{R}^2)\).
由于\(V\)和几何空间都是实数域\(\mathbb{R}\)上的三维线性空间,
因此它们同构,从而\(V\)的每一个一维子空间与几何空间中经过点\(O\)的直线,
即把\(O\)中的直线有一个一一对应,
\(V\)的每一个二维子空间与几何空间中经过点\(O\)的平面,
即把\(O\)中的平面也有一个一一对应,
并且当\(V\)中\(\Span\{\alpha\} \subseteq \Span\{\beta,\gamma\}\),
则\(\Span\{\alpha\}\)对应的把\(O\)中的直线
与\(\Span\{\beta,\gamma\}\)对应的把\(O\)中的平面关联.
于是,根据射影平面的定义,\(\mathcal{P}(\mathbb{R}^2)\)是一个射影平面,
我们称其为\DefineConcept{实射影平面}.

\begin{property}
%@see: 《解析几何》(丘维声) P258 性质1
任意给定实射影平面\(\mathcal{P}(\mathbb{R}^2)\)中两个不同的点,
有且仅有一条直线与这两个点关联.
%TODO proof
\end{property}

\begin{property}
%@see: 《解析几何》(丘维声) P258 性质2
任意给定实射影平面\(\mathcal{P}(\mathbb{R}^2)\)中两条不同的直线,
有且仅有一个点与这两条直线关联.
%TODO proof
\end{property}
