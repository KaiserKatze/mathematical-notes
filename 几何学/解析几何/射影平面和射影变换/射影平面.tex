\section{射影平面,齐次坐标}
\subsection{以点为中心的把,扩充欧氏平面}
%@see: 《解析几何》(丘维声) P252
本节研究一个平面上的各个几何元素之间的关联关系.
由\cref{theorem:欧氏几何.定理2}
可知过一条直线\(l\)和不在\(l\)上的一点\(P\),有且仅有一个平面(记为\(\pi\)).
于是平面\(\pi\)就可以看成是一些点和直线的集合.
用同样的观点来看点\(O\),那么我们看到的就是:
所有与点\(O\)关联的平面\(\lambda\)和直线\(p\).
今后我们将与一个点\(O\)关联的所有平面和直线的集合
称为“以点\(O\)为中心的\DefineConcept{把}”.

接下来,继续按这个观点,考察中心投影的构成.
我们会发现,把与平面之间存在简单而且自然的对应关系.
设点\(O\)不在平面\(\pi\)上,则平面\(\pi\)上的每一个点\(P\)决定把\(O\)中的一条直线\(OP\);
平面\(\pi\)上的每一条直线\(l\)决定把\(O\)中的一个平面\(Ol\).
这样就得到一个从平面\(\pi\)到把\(O\)的一个映射,
称为“平面\(\pi\)在把\(O\)上的\DefineConcept{射影}”.
反过来,从把\(O\)到平面\(\pi\)也有一个对应关系.
把\(O\)中的每一个平面\(\lambda\)与平面\(\pi\)交于一条直线(用\(\pi\lambda\)表示),
把\(O\)中的每一条直线\(p\)与平面\(\pi\)交于一个点(用\(\pi p\)表示).
我们将这个对应关系称为“把\(O\)在平面\(\pi\)上的\DefineConcept{截影}”.

容易看出,经过射影和截影,关联关系是不变的.

我们还可以看出,中心投影可以分解成射影、截影两个步骤:
从平面\(\pi_0\)到\(\pi_1\)、以点\(O\)为中心的中心投影,
实际上就是\(\pi_0\)在把\(O\)上的射影,
与把\(O\)在\(\pi_1\)上的截影,
这两个映射的复合.
在\cref{figure:解析几何.射影平面和射影变换.中心射影} 中,
平面\(\pi_0\)上的点\(M\)在中心投影下没有像,
这是因为把\(O\)的直线\(OM\)在截影下没有像;
平面\(\pi_1\)上的点\(N\)在中心投影下没有原像,
这是因为把\(O\)的直线\(ON\)在射影下没有原像.
为什么把\(O\)中有的直线在截影下没有像,
有的直线在射影下没有原像,
其原因就在于欧几里得平面的结构与把的结构是不同的.
弥补这一缺陷的方法,应当是将把作为模型,对欧几里得平面加以扩充.

现在我们就将把\(O\)作为模型,来扩充欧几里得平面,
使得射影和截影能够成为它们之间的、保持关联关系不变的双射.
取一个欧几里得平面\(\pi_0\),使点\(O\)不在\(\pi_0\)上.
容易看出,在从\(\pi_0\)到把\(O\)的射影下,
把\(O\)中与\(\pi_0\)平行的每一个平面\(\pi_0'\)
以及\(\pi_0'\)上的每一条直线都没有原像;
而在从把\(O\)到\(\pi_0\)的截影下,
把\(O\)中的每一个平面\(\pi_0'\)以及\(\pi_0'\)上的每一条直线都没有像.
为了使射影和截影都成为双射,
就应当在\(\pi_0\)上加进一条直线,让它和把\(O\)中的平面\(\pi_0'\)对应,
还应当在\(\pi_0\)上添上一些点,让它们和把\(O\)中的平面\(\pi_0'\)上的直线对应,
而且像这样添加了新的元素(直线和点)以后,
应该让射影和截影仍旧保持关联关系不变.
为此,采取如下的扩充方法:
考虑到\(\pi_0\)上的每一条直线\(l\)在把\(O\)中的对应平面\(Ol\)
与\(\pi_0'\)有唯一的交线\(l'\)(如\cref{figure:解析几何.射影平面和射影变换.无穷远点} 所示),
规定在\(\pi_0\)的每一条直线\(l\)上加进唯一的一个点,来与\(l'\)相对应.
我们将新添加的这个与\(l'\)对应的点称为\DefineConcept{无穷远点}.
由于\(\pi_0\)上平行的直线在把\(O\)中的对应平面与\(\pi_0'\)的交线都相同
(与\(l\)平行的直线\(m\)在把\(O\)中对应平面\(Om\)与\(\pi_0'\)的交线还是\(l'\)),
而\(\pi_0\)上不平行的直线在把\(O\)中的对应平面与\(\pi_0'\)的交线都不同,
因此规定:
\(\pi_0\)上平行的直线有相同的无穷远点,
\(\pi_0\)上不平行的直线有不同的无穷远点.
最后,为了与把\(O\)中的平面\(\pi_0'\)对应,
我们在\(\pi_0\)上还要加进一条直线作为\(\pi_0\)与\(\pi_0'\)的交线.
我们将新添加的这条直线称为\DefineConcept{无穷远直线}.
于是无穷远直线对应于把\(O\)中的平面\(\pi_0'\).
由于无穷远点所对应的把\(O\)中的直线都在\(\pi_0'\)上,
因此规定无穷远点都在无穷远直线上.

\begin{figure}[hbt]
%@see: 《解析几何》(丘维声) P254 图7.2
	\centering
	\begin{tikzpicture}
		\draw(0,0)--(4,0)--++(1,{sqrt(3)})node[below left]{$\pi_0'$}--++(-4,0)--(0,0);
		\draw(.6,0)coordinate(l1a)--(3,{sqrt(3)})coordinate(l1b)node[above]{$l'$};
		\fill($.7*(l1a) + .3*(l1b)$)coordinate(O)circle(2pt)node[above left]{$O$};
		\pgfmathsetmacro{\kp}{1.2}
		\draw(-1.5,-4)coordinate(p0a)
			--(4.5,-4)coordinate(p0b)node[above left]{$\pi_0$}
			--++(1.2,{\kp*sqrt(3)})coordinate(p0c);
		\draw($(p0c) + (p0a) - (p0b)$)coordinate(p0d)--(p0a);
		\draw[name path=l1al2a](l1a)--(-.5,-4)coordinate(l2a);
		\draw[name path=lal1a](l1a)--(1.5,-4)coordinate(la);
		\draw(la)--($(la) + \kp*(l1b) - \kp*(l1a)$)coordinate(lb)node[pos=.7,below]{$l$};
		\path[name path=l2](l2a)--($(l2a) + \kp*(l1b) - \kp*(l1a)$)coordinate(l2b);
		\draw[dashed](l1b)--(l2b);
		\draw(l1b)--(lb);
		\draw(p0c)--(lb);
		\path[name path=p0cd](p0c)--(p0d);
		\begin{scope}[name intersections={of=p0cd and l1al2a}]
			\draw[dashed](intersection-1)--(lb);
			\draw(intersection-1)--(p0d);
		\end{scope}
		\begin{scope}[name intersections={of=lal1a and l2}]
			% \fill[color=red](intersection-1)circle(2pt);
			\draw(l2a)--(intersection-1)node[midway,above]{$m$};
			\draw[dashed](intersection-1)--(l2b);
		\end{scope}
	\end{tikzpicture}
	\caption{}
	\label{figure:解析几何.射影平面和射影变换.无穷远点}
\end{figure}

把上面的讨论总结一下:
对于欧几里得平面\(\pi_0\),
在它的每一条直线上都加进了一个无穷远点\(P_\infty\),
平行的直线有相同的无穷远点,
不平行的直线有不同的无穷远点
(这样一来,与每个无穷远点对应的时一个完全确定的方向),
所有无穷远点组成一条无穷远直线\(l_\infty\).
补充了这样一些无穷远点和一条无穷远直线的欧几里得平面\(\pi_0\),
就称为一个\DefineConcept{扩充欧几里得平面},
记作\(\overline{\pi_0}\).

显然,扩充欧氏平面\(\overline{\pi_0}\)
与把\(O\)之间的射影和截影都是保持关联关系不变的双射,
也就是说:
\(\overline{\pi_0}\)上的所有点
与把\(O\)中的所有直线一一对应;
\(\overline{\pi_0}\)上的所有直线
与把\(O\)中的所有平面一一对应;
\(\overline{\pi_0}\)上一点\(P\)与一直线\(l\)关联,
当且仅当把\(O\)中与\(P\)对应的直线\(OP\)和与\(l\)对应的平面\(Ol\)关联.
因此,当我们只考虑基本几何元素之间的关联关系时,
扩充欧氏平面的结构与把的结构在本质上是一样的.
由此我们抽象出射影平面的概念.
