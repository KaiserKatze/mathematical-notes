\section{射影映射和射影变换}
本章开头介绍了中心投影,这一节我们要来推广中心投影的概念.

\subsection{射影映射的定义和性质}
\begin{definition}
%@see: 《解析几何》(丘维声) P290 定义5.1
设\(\mathscr{P}_1,\mathscr{P}_2\)是两个射影平面,
\(f\colon \mathscr{P}_1 \to \mathscr{P}_2\)是一个双射.
如果\(f\)把共线三点映成共线三点,
则称\(f\)是一个\DefineConcept{射影映射}.
\end{definition}

\begin{definition}
设\(\mathscr{P}\)是一个射影平面,
\(f\colon \mathscr{P} \to \mathscr{P}\)是一个双射.
如果\(f\)把共线三点映成共线三点,
则称\(f\)是一个\DefineConcept{射影变换}.
\end{definition}

\begin{example}
%@see: 《解析几何》(丘维声) P290
中心投影是射影映射.
\end{example}

由定义立即得到射影映射的下述性质.
\begin{property}
%@see: 《解析几何》(丘维声) P290 性质1
射影映射的乘积还是射影映射.
\end{property}

\begin{property}
%@see: 《解析几何》(丘维声) P290 性质2
射影映射把不共线三点映成不共线三点.
\end{property}

\begin{property}
%@see: 《解析几何》(丘维声) P291 性质3
射影映射都是可逆的,并且它的逆映射也是射影映射.
\end{property}

\begin{property}
%@see: 《解析几何》(丘维声) P291 性质4
射影映射把直线映成直线.
\end{property}

\begin{property}
%@see: 《解析几何》(丘维声) P291 性质5
射影映射把一般位置的四个点映成一般位置的四个点.
\end{property}

\begin{theorem}\label{theorem:解析几何.射影映射.射影映射基本定理1}
%@see: 《解析几何》(丘维声) P291 定理5.1(射影映射基本定理之一)
从射影平面\(\overline{\pi_0}\)到\(\overline{\pi_1}\)的射影映射\(\tau\)
把\(\overline{\pi_0}\)上的一般位置的四个点\(\AutoTuple{A}{3},E\)
映成\(\overline{\pi_1}\)上的一般位置的四个点\(\AutoTuple{A'}{3},E'\),
并且\(\overline{\pi_0}\)上任意一点\(M\)在基底\([\AutoTuple{A}{3},E]\)中的射影坐标
等于\(M\)的像\(M'\)在基底\([\AutoTuple{A'}{3},E']\)中的射影坐标.
\end{theorem}

\begin{theorem}
%@see: 《解析几何》(丘维声) P293 定理5.2(射影映射基本定理之二)
设\(\AutoTuple{A}{3},E\)是射影平面\(\overline{\pi_0}\)上一般位置的四个点,
\(\AutoTuple{A'}{3},E'\)是射影平面\(\overline{\pi_1}\)上一般位置的四个点,
则存在从\(\overline{\pi_0}\)到\(\overline{\pi_1}\)的唯一的射影映射\(\tau\)
把\(\AutoTuple{A}{3},E\)分别映成\(\AutoTuple{A'}{3},E'\).
\end{theorem}

\begin{theorem}
%@see: 《解析几何》(丘维声) P293 定理5.3
设\(\tau\)是射影平面\(\overline{\pi_0}\)到\(\overline{\pi_1}\)的一个映射,
在\(\overline{\pi_0}\)、\(\overline{\pi_1}\)上
分别取基底 I \([\AutoTuple{A}{3},E]\)、II \([\AutoTuple{B}{3},F]\),
\(\overline{\pi_0}\)上任意一点\(M\)在 I 中的射影坐标为\((x_1,x_2,x_3)\),
\(\tau(M)\)在 II 中的射影坐标为\((x'_1,x'_2,x'_3)\).
\begin{itemize}
	\item
	如果\(\tau\)是射影映射,
	则存在非零常数\(\rho\)和非奇异矩阵\begin{equation*}
		A
		\defeq
		\begin{bmatrix}
			a_{11} & a_{12} & a_{13} \\
			a_{21} & a_{22} & a_{23} \\
			a_{31} & a_{32} & a_{33}
		\end{bmatrix}
	\end{equation*}
	使得\begin{equation}\label{equation:解析几何.射影映射.射影映射的基底变换公式}
	%@see: 《解析几何》(丘维声) P293 (5.5)
		\rho
		\begin{bmatrix}
			x'_1 \\ x'_2 \\ x'_3
		\end{bmatrix}
		= A
		\begin{bmatrix}
			x_1 \\ x_2 \\ x_3
		\end{bmatrix}.
	\end{equation}

	\item
	如果存在非零常数\(\rho\)和非奇异矩阵\begin{equation*}
		A
		\defeq
		\begin{bmatrix}
			a_{11} & a_{12} & a_{13} \\
			a_{21} & a_{22} & a_{23} \\
			a_{31} & a_{32} & a_{33}
		\end{bmatrix}
	\end{equation*}
	使得\cref{equation:解析几何.射影映射.射影映射的基底变换公式} 成立,
	则\(\tau\)一定是射影映射.
\end{itemize}
%TODO
% \begin{proof}
% 设\(\tau\)是从\(\overline{\pi_0}\)到\(\overline{\pi_1}\)的一个射影映射,
% 它把\(\AutoTuple{A}{3},E\)分别映成\(\AutoTuple{A'}{3},E'\),
% 则\(\AutoTuple{A'}{3},E'\)是\(\overline{\pi_1}\)上的一般位置的四个点.
% 取基底 I' \([\AutoTuple{A'}{3},E']\).
% 根据\cref{theorem:解析几何.射影映射.射影映射基本定理1},
% \(\tau(M)\)在 I' 中的射影坐标等于\(M\)在 I 中的射影坐标,
% II 到 I' 的射影坐标变换公式中的系数矩阵
% \end{proof}
\end{theorem}

我们把\cref{equation:解析几何.射影映射.射影映射的基底变换公式}
称为“射影映射\(\tau\)关于基底 I 和 II 的公式”.

\begin{property}
%@see: 《解析几何》(丘维声) P295 性质6
射影映射保持共线四点的交比不变.
\end{property}

\begin{property}
%@see: 《解析几何》(丘维声) P295 性质7
射影映射保持共点四线的交比不变.
\end{property}
