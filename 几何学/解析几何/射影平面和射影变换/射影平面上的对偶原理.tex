\section{射影平面上的对偶原理}
%@see: 《解析几何》(丘维声) P265
我们已经知道,射影平面上点和直线的地位是对等的.
现在我们来进一步讨论这个问题.

%@see: 《解析几何》(丘维声) P266
设\(\phi(\text{点},\text{线})\)是关于射影平面上的一些点和一些直线的关联关系的一个命题.
我们把该命题中的点都改写成直线,把直线都改写成点,
并且保持关联关系亦即其他一切表述不变,
就能得到一个新命题\(\phi(\text{线},\text{点})\).
我们把这个新命题称为原命题的\DefineConcept{对偶命题}.

\begin{table}[hbt]
%@see: 《解析几何》(丘维声) P266
	\centering
	\begin{tblr}{p{8cm}|p{8cm}}
		\hline
		原命题 & 对偶命题 \\
		\hline
		射影平面上三点共线的充分必要条件是它们的齐次坐标组成的3阶行列式等于\(0\).
		& 射影平面上三线共点的充分必要条件是它们的齐次坐标组成的3阶行列式等于\(0\). \\
		射影平面上,若三点\(\AutoTuple{P}{3}\)不共线,则三线\(P_1P_2,P_2P_3,P_3P_1\)不共点.
		& 射影平面上,若三线\(\AutoTuple{p}{3}\)不共点,则三线\(p_1p_2,p_2p_3,p_3p_1\)不共线. \\
		德萨格定理
		& 德萨格定理的逆定理 \\
		直线(看成点集)的点方程是三元一次齐次方程.
		& 点(看成线系)的线方程是三元一次齐次方程. \\
		\hline
	\end{tblr}
	\caption{}
\end{table}

对于射影平面上的每个命题与它的对偶命题,有下述重要性质.
\begin{theorem}%[射影平面上的对偶原理]
%@see: 《解析几何》(丘维声) P266 射影平面上的对偶原理
射影平面上,如果可以证明命题\(\phi(\text{点},\text{线})\)是一条定理,
那么可以证明它的对偶命题\(\phi(\text{线},\text{点})\)也是一条定理.
\end{theorem}
