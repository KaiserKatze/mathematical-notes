\section{射影平面上的对偶原理}
%@see: 《解析几何》(丘维声) P265
我们已经知道,射影平面上点和直线的地位是对等的.
现在我们来进一步讨论这个问题.

%@see: 《解析几何》(丘维声) P266
设\(\phi(\text{点},\text{线})\)是关于射影平面上的一些点和一些直线的关联关系的一个命题.
我们把该命题中的点都改写成直线,把直线都改写成点,
并且保持关联关系亦即其他一切表述不变,
就能得到一个新命题\(\phi(\text{线},\text{点})\).
我们把这个新命题称为原命题的\DefineConcept{对偶命题}.

\begin{table}[hbt]
%@see: 《解析几何》(丘维声) P266
	\centering
	\begin{tblr}{p{8cm}|p{8cm}}
		\hline
		原命题 & 对偶命题 \\
		\hline
		射影平面上三点共线的充分必要条件是它们的齐次坐标组成的3阶行列式等于\(0\).
		& 射影平面上三线共点的充分必要条件是它们的齐次坐标组成的3阶行列式等于\(0\). \\
		射影平面上,若三点\(\AutoTuple{P}{3}\)不共线,则三线\(P_1P_2,P_2P_3,P_3P_1\)不共点.
		& 射影平面上,若三线\(\AutoTuple{p}{3}\)不共点,则三线\(p_1p_2,p_2p_3,p_3p_1\)不共线. \\
		德萨格定理
		& 德萨格定理的逆定理 \\
		直线(看成点集)的点方程是三元一次齐次方程.
		& 点(看成线系)的线方程是三元一次齐次方程. \\
		\hline
	\end{tblr}
	\caption{}
\end{table}

对于射影平面上的每个命题与它的对偶命题,有下述重要性质.
\begin{theorem}[对偶原理]\label{theorem:解析几何.射影平面上的对偶原理.对偶原理}
%@see: 《解析几何》(丘维声) P266 射影平面上的对偶原理
射影平面上,如果可以证明命题\(\phi(\text{点},\text{线})\)是一条定理,
那么可以证明它的对偶命题\(\phi(\text{线},\text{点})\)也是一条定理.
\end{theorem}

根据射影平面上的对偶原理,
我们只要证明一个命题\(\phi(\text{点},\text{线})\)成立,
那么它的对偶命题\(\phi(\text{线},\text{点})\)就必然成立.
譬如,我们已经证明了“射影平面上三点共线的充分必要条件是它们的齐次坐标组成的3阶行列式等于\(0\).”成立,
那么它的对偶命题“射影平面上三线共点的充分必要条件是它们的齐次坐标组成的3阶行列式等于\(0\).”也成立.

\begin{figure}[hbt]
%@see: 《解析几何》(丘维声) P267 图7.6
	\centering
	% \begin{tikzpicture}[
	% 	dot/.style={draw,fill,circle,inner sep=0pt,minimum size=2pt},
	% 	label distance=4pt,
	% ]
	% 	\coordinate[dot,label=below left:$A$](A)at(0,0);
	% 	\coordinate[dot,label=above left:$B$](B)at(1,2);
	% 	\coordinate[dot,label=below left:$C$](C)at(2,-2);
	% 	\coordinate[dot,label=right:$A'$](A1)at(6,-1);
	% 	\coordinate[dot,label=above right:$B'$](B1)at(4,3);
	% 	\coordinate[dot,label=below right:$C'$](C1)at(3,-2.5);
	% 	\draw(A)--(B)--(C)--(A);
	% 	\draw(A1)--(B1)--(C1)--(A1);
	% 	\path[name path=lineAB](A)--($(A)!2.8!(B)$);
	% 	\path[name path=lineA1B1](A1)--($(A1)!1.7!(B1)$);
	% 	\draw[name intersections={of=lineAB and lineA1B1,by={P}}]
	% 		(P)node[dot,label=right:$P$]{};
	% 	\path[name path=lineAC](A)--($(A)!2!(C)$);
	% 	\path[name path=lineA1C1](A1)--($(A1)!2!(C1)$);
	% 	\draw[name intersections={of=lineAC and lineA1C1,by={Q}}]
	% 		(Q)node[dot,label=below:$Q$]{};
	% 	\path[name path=lineBC](B)--($(B)!1.7!(C)$);
	% 	\path[name path=lineB1C1](B1)--($(B1)!1.4!(C1)$);
	% 	\draw[name intersections={of=lineBC and lineB1C1,by={R}}]
	% 		(R)node[dot,label=right:$R$]{};
	% 	\path[name path=lineAA1](A1)--($(A1)!1.6!(A)$);
	% 	\path[name path=lineBB1](B1)--($(B1)!2.5!(B)$);
	% 	\draw[name intersections={of=lineAA1 and lineBB1,by={D}}]
	% 		(D)node[dot,label=left:$D$]{};
	% 	\begin{scope}[dashed]
	% 		\draw(B)--(P)--(B1);
	% 		\draw(C)--(Q)--(C1);
	% 		\draw(C)--(R)--(C1);
	% 		\draw(A1)--(D)--(B1);
	% 		\draw(D)--(C1);
	% 	\end{scope}
	% \end{tikzpicture}
	\begin{tikzpicture}[
		label distance=2pt,
	]
		% 依赖 tkz-euclide 宏包
		\coordinate[label=below left:$A$](A)at(0,0);
		\coordinate[label=above left:$B$](B)at(1,2);
		\coordinate[label=below left:$C$](C)at(2,-2);
		\coordinate[label=below:$A'$](A1)at(6,-1);
		\coordinate[label=above right:$B'$](B1)at(4,3);
		\coordinate[label=below right:$C'$](C1)at(3,-2.5);

		\draw(A)--(B)--(C)--(A);
		\draw(A1)--(B1)--(C1)--(A1);

		\tkzInterLL(A,B)(A1,B1) \tkzGetPoint{P}
		\tkzInterLL(A,C)(A1,C1) \tkzGetPoint{Q}
		\tkzInterLL(B,C)(B1,C1) \tkzGetPoint{R}
		\tkzInterLL(A,A1)(B,B1) \tkzGetPoint{D}

		\tkzDrawPoints(A,B,C,A1,B1,C1,D,P,Q,R)

		\draw(P)node[label=right:$P$]{};
		\draw(Q)node[label=below left:$Q$]{};
		\draw(R)node[label=right:$R$]{};
		\draw(D)node[label=above:$D$]{};

		\begin{scope}[dashed]
			\draw(B)--(P)--(B1);
			\draw(C)--(Q)--(C1);
			\draw(C)--(R)--(C1);
			\draw(A1)--(D)--(B1);
			\draw(D)--(C1);
			\draw(P)--(R);
		\end{scope}
	\end{tikzpicture}
	\caption{}
	% \label{figure:解析几何.射影平面上的对偶原理.德萨格定理}
\end{figure}

\begin{theorem}[德萨格定理]\label{theorem:解析几何.射影平面上的对偶原理.德萨格定理}
%@see: 《解析几何》(丘维声) P267 德萨格定理
射影平面上,如果两个三角形的对应顶点的连线共点,那么它们的对应边的交点共线.
\end{theorem}

\hyperref[theorem:解析几何.射影平面上的对偶原理.德萨格定理]{德萨格定理}说明:
给定一点\(D\)和一个三角形\(ABC\),
当\(D\)在\(\triangle ABC\)以外时,
如果\(\triangle A'B'C'\)是\(\triangle ABC\)在以\(D\)为中心的中心投影下的像,
那么\(\triangle ABC\)和\(\triangle A'B'C'\)的对应边的交点共线.
于是由\hyperref[theorem:解析几何.射影平面上的对偶原理.对偶原理]{对偶原理}可知
德萨格定理的逆定理“射影平面上,如果它们的对应边的交点共线,那么两个三角形的对应顶点的连线共点.”也成立.

\begin{example}
%@see: 《解析几何》(丘维声) P268 习题7.2 1.
证明:射影平面上,若\(P_1,P_2,P_3\)三点不共线,
则\(P_1P_2,P_2P_3,P_3P_1\)三线不共点.
%TODO
\end{example}

\begin{example}\label{example:解析几何.射影平面上的对偶原理.三线共点的条件}
%@see: 《解析几何》(丘维声) P269 习题7.2 3.
设射影平面上直线\(l_i\)的齐次坐标为\((\eta_{i1},\eta_{i2},\eta_{i3})\ (i=1,2,3)\),
并且\(l_1 \neq l_2\).
证明:\(\AutoTuple{l}{3}\)共点的充分必要条件是
存在不全为零的数\(\lambda\)和\(\mu\),
使得\begin{equation*}
	% \eta_{3j} = \lambda \eta_{1j} + \mu \eta_{2j}\ (j=1,2,3).
	\begin{bmatrix}
		\eta_{31} \\
		\eta_{32} \\
		\eta_{33}
	\end{bmatrix}
	= \lambda
	\begin{bmatrix}
		\eta_{11} \\
		\eta_{12} \\
		\eta_{13}
	\end{bmatrix}
	+ \mu
	\begin{bmatrix}
		\eta_{21} \\
		\eta_{22} \\
		\eta_{23}
	\end{bmatrix}
\end{equation*}
%TODO
\end{example}
