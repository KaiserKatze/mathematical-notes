\section{配极,二次曲线的射影分类}
\subsection{射影平面上的二次曲线}
射影平面上,射影坐标\((x_1,x_2,x_3)\)满足
二次齐次方程\begin{equation}\label{equation:配极.二次曲线的齐次射影坐标方程}
%@see: 《解析几何》(丘维声) P301 (6.1)
	a_{11} x_1^2
	+ a_{22} x_2^2
	+ a_{33} x_3^2
	+ 2 a_{12} x_1 x_2
	+ 2 a_{13} x_1 x_3
	+ 2 a_{23} x_2 x_3
	= 0
\end{equation}
的点构成的集合
称为\DefineConcept{二次曲线}.

显然,若点\(M(x_1,x_2,x_3)\)
在二次曲线 \labelcref{equation:配极.二次曲线的齐次射影坐标方程} 上,
则\((\lambda x_1,\lambda x_2,\lambda x_3)\)
也满足方程 \labelcref{equation:配极.二次曲线的齐次射影坐标方程};
并且二次曲线 \labelcref{equation:配极.二次曲线的齐次射影坐标方程}
在任意基底下的方程仍然是二次齐次方程.

记\begin{equation*}
	F(x_1,x_2,x_3)
	\defeq
	a_{11} x_1^2
	+ a_{22} x_2^2
	+ a_{33} x_3^2
	+ 2 a_{12} x_1 x_2
	+ 2 a_{13} x_1 x_3
	+ 2 a_{23} x_2 x_3,
\end{equation*}
则\begin{equation*}
	F(x_1,x_2,x_3)
	= \vb{X}^T \vb{A} \vb{X},
\end{equation*}
其中\begin{equation*}
	\vb{A}
	\defeq
	\begin{bmatrix}
		a_{11} & a_{12} & a_{13} \\
		a_{12} & a_{22} & a_{23} \\
		a_{13} & a_{23} & a_{33} \\
	\end{bmatrix},
	\qquad
	\vb{X}
	\defeq
	\begin{bmatrix}
		x_1 \\ x_2 \\ x_3
	\end{bmatrix}.
\end{equation*}
显然\(\vb{A}\)是一个对称矩阵.
如果\(\vb{A}\)是非奇异的,
则称“二次曲线 \labelcref{equation:配极.二次曲线的齐次射影坐标方程} 是\DefineConcept{非退化的}”;
否则称“二次曲线 \labelcref{equation:配极.二次曲线的齐次射影坐标方程} 是\DefineConcept{退化的}”.

现在我们可以把方程 \labelcref{equation:配极.二次曲线的齐次射影坐标方程}
改写成\begin{equation}
%@see: 《解析几何》(丘维声) P301 (6.1)'
	\vb{X}^T \vb{A} \vb{X} = 0
\end{equation}
或\begin{equation}
%@see: 《解析几何》(丘维声) P301 (6.1)''
	\sum_{i=1}^3 \sum_{j=1}^3 a_{ij} x_i x_j = 0
	\quad(x_{ij} = x_{ji}).
\end{equation}

现在考虑扩充欧氏平面\(\overline{\pi_0}\).
在\(\pi_0\)上取一个直角标架\([O_1;e_1,e_2]\),
相应的\(\overline{\pi_0}\)的基底记作 I,
则点\(M\)在 I 中的射影坐标\((x_1,x_2,x_3)\)就是\(M\)的齐次坐标.
于是\(M\)在\([O_1;e_1,e_2]\)中的仿射坐标\((x,y)\)满足\begin{equation*}
	x = \frac{x_1}{x_3},
	\qquad
	y = \frac{x_2}{x_3}.
\end{equation*}

假设二次曲线\(\overline{S}\)在 I 中的方程是\cref{equation:配极.二次曲线的齐次射影坐标方程}.
\begingroup
\setlist{%
    itemindent=3.5em,%修改enumitem全局设置,使得item序号对齐悬挂缩进2个字
}
\begin{enumerate}[label={{\rm\bf 情形}\arabic*.}]
	\item 设\(a_{11},a_{22},a_{12}\)不全为零.
	通常点\(M(x_1,x_2,x_3)\)在\(\overline{S}\)上的充分必要条件是
	\(M\)的仿射坐标\((x,y)\)适合\begin{equation}\label{equation:配极.二次曲线的非齐次仿射坐标方程1}
	%@see: 《解析几何》(丘维声) P302 (6.2)
		a_{11} x^2
		+ a_{22} y^2
		+ 2 a_{12} x y
		+ 2 a_{13} x
		+ 2 a_{23} y
		+ a_{33}
		= 0.
	\end{equation}
	无穷远点\(N(x_1,x_2,0)\)在\(\overline{S}\)上的充分必要条件是
	\(N\)相应的\(\pi_0\)上的方向\(v = (x_1,x_2)\)适合\begin{equation}
	%@see: 《解析几何》(丘维声) P302 (6.3)
		a_{11} x_1^2
		+ a_{22} x_2^2
		+ 2 a_{12} x_1 x_2
		= 0.
	\end{equation}
	显然,\(v\)是\(\pi_0\)上的二次曲线 \labelcref{equation:配极.二次曲线的非齐次仿射坐标方程1} 的渐进方向.

	这就说明:如果\(a_{11},a_{22},a_{12}\)不全为零,
	那么\(\overline{\pi_0}\)上的二次曲线\(\overline{S}\)
	是由欧氏平面\(\pi_0\)上的二次曲线 \labelcref{equation:配极.二次曲线的非齐次仿射坐标方程1}
	以及它的渐进方向所对应的无穷远点所组成.

	\item 设\(a_{11} = a_{22} = a_{12} = 0\).
	这时,通常点\(M(x_1,x_2,x_3)\)在\(\overline{S}\)上的充分必要条件是
	它的仿射坐标\((x,y)\)满足方程\begin{equation}\label{equation:配极.二次曲线的非齐次仿射坐标方程2}
	%@see: 《解析几何》(丘维声) P302 (6.4)
		2 a_{13} x
		+ 2 a_{23} y
		+ a_{33} = 0.
	\end{equation}
	显然,任意一个无穷远点的坐标都满足
	方程 \labelcref{equation:配极.二次曲线的齐次射影坐标方程},
	从而整条无穷远直线都在\(\overline{S}\)上.

	这就说明:如果\(a_{11} = a_{22} = a_{12} = 0\),
	那么\(\overline{\pi_0}\)上的二次曲线\(\overline{S}\)
	要么是由一条射影直线 \labelcref{equation:配极.二次曲线的非齐次仿射坐标方程2}
	和一条无穷远直线组成,
	要么是由一对重合的无穷远直线组成.
\end{enumerate}
\endgroup

总之,射影平面\(\overline{\pi_0}\)上的二次曲线\(\overline{S}\)有11种,
其中9种分别是欧氏平面\(\pi_0\)上的二次曲线附加它的渐进方向所对应的无穷远点,
另外2种分别是
一条射影直线 \labelcref{equation:配极.二次曲线的非齐次仿射坐标方程2} 和一条无穷远直线,
一对重合的无穷远直线.

\subsection{二次曲线的切线}
%@see: 《解析几何》(丘维声) P302
现在研究直线与二次曲线相交的情况.
设直线\(PQ\)的参数方程为\begin{equation}\label{equation:配极.直线的齐次射影坐标方程}
%@see: 《解析几何》(丘维声) P302 (6.5)
	\begin{bmatrix}
		x_1 \\ x_2 \\ x_3
	\end{bmatrix}
	= \lambda
	\begin{bmatrix}
		p_1 \\ p_2 \\ p_3
	\end{bmatrix}
	+ \mu
	\begin{bmatrix}
		q_1 \\ q_2 \\ q_3
	\end{bmatrix},
\end{equation}
其中\(\lambda,\mu\)不全为零.
代入二次曲线\(\overline{S}\)的方程 \labelcref{equation:配极.二次曲线的齐次射影坐标方程} 中,
得\begin{equation*}
	\left(
		\lambda
		\begin{bmatrix}
			p_1 \\ p_2 \\ p_3
		\end{bmatrix}
		+ \mu
		\begin{bmatrix}
			q_1 \\ q_2 \\ q_3
		\end{bmatrix}
	\right)^T
	\vb{A}
	\left(
		\lambda
		\begin{bmatrix}
			p_1 \\ p_2 \\ p_3
		\end{bmatrix}
		+ \mu
		\begin{bmatrix}
			q_1 \\ q_2 \\ q_3
		\end{bmatrix}
	\right)
	= 0,
\end{equation*}
即\begin{equation}\label{equation:配极.直线与二次曲线的交点的齐次射影坐标方程}
%@see: 《解析几何》(丘维声) P303 (6.6)
	\lambda^2 F(p_1,p_2,p_3)
	+ \mu^2 F(q_1,q_2,q_3)
	+ 2 \lambda \mu
	\overline{F}(p_1,p_2,p_3;q_1,q_2,q_3)
	= 0,
\end{equation}
其中\begin{equation*}
	\overline{F}(p_1,p_2,p_3;q_1,q_2,q_3)
	\defeq
	\begin{bmatrix}
		p_1 \\ p_2 \\ p_3
	\end{bmatrix}^T
	\vb{A}
	\begin{bmatrix}
		q_1 \\ q_2 \\ q_3
	\end{bmatrix}.
\end{equation*}

如果\(
	F(p_1,p_2,p_3)
	= F(q_1,q_2,q_3)
	= \overline{F}(p_1,p_2,p_3;q_1,q_2,q_3)
	= 0
\),
则\(\lambda,\mu\)可以取任意不全为零的实数,
从而直线\(PQ\)的所有点都在二次曲线\(\overline{S}\)上.

如果\(
	F(p_1,p_2,p_3),
	F(q_1,q_2,q_3),
	\overline{F}(p_1,p_2,p_3;q_1,q_2,q_3)
\)不全为零,
则方程 \labelcref{equation:配极.直线与二次曲线的交点的齐次射影坐标方程}
可以确定\(\lambda/\mu\)的两个值(可能是不同的实数,可能是相同的实数,可能是不同的复数),
从而得到直线\(PQ\)与二次曲线\(\overline{S}\)有两个交点.

\begin{definition}
%@see: 《解析几何》(丘维声) P303 定义6.1
若直线\(L\)与二次曲线\(\overline{S}\)有重合的两个交点,
或\(L\)整个在\(\overline{S}\)上,
则称\(L\)是\(\overline{S}\)的\DefineConcept{切线},
它们的交点称为\DefineConcept{切点}.
\end{definition}

设直线\(L\)是\(\overline{S}\)的切线,
切点为\(P\).
在\(L\)上任意取定一点\(Q\ (\neq P)\),
设\(P,Q\)的射影坐标分别是\((p_1,p_2,p_3),(q_1,q_2,q_3)\),
\(L\)的参数方程为\cref{equation:配极.直线的齐次射影坐标方程}.
因为\(L\)与\(\overline{S}\)有二重交点\(P\),
所以\cref{equation:配极.直线与二次曲线的交点的齐次射影坐标方程}
确定\(\lambda/\mu\)的两个相同的值,
从而有\begin{equation*}
	\overline{F}^2(p_1,p_2,p_3;q_1,q_2,q_3)
	- F(p_1,p_2,p_3) \cdot F(q_1,q_2,q_3) = 0.
\end{equation*}
由于\(P\)在\(\overline{S}\)上,
所以\(F(p_1,p_2,p_3) = 0\),
那么由上式可得\begin{equation}\label{equation:配极.二次曲线的切线的齐次射影坐标方程}
%@see: 《解析几何》(丘维声) P303 (6.7)
%@see: 《解析几何》(丘维声) P303 (6.8)
	\overline{F}(p_1,p_2,p_3;q_1,q_2,q_3)
	= 0.
\end{equation}
切线\(L\)上任意一点\(Q(q_1,q_2,q_3)\)均适合
方程 \labelcref{equation:配极.二次曲线的切线的齐次射影坐标方程}.
若\((p_1,p_2,p_3) \vb{A} \neq \vb0\),
则\cref{equation:配极.二次曲线的切线的齐次射影坐标方程}
是一次齐次方程,
它就是切线\(L\)的方程.
若\((p_1,p_2,p_3) \vb{A} = \vb0\),
则\cref{equation:配极.二次曲线的切线的齐次射影坐标方程}
中\((q_1,q_2,q_3)\)可以取任意不全为零的实数值.
这意味着扩充欧氏平面上任意一点与点\(P\)的连线都是\(\overline{S}\)的切线.

设\(P(p_1,p_2,p_3)\)是二次曲线\(\overline{S}\)上的一点.
如果\((p_1,p_2,p_3) \vb{A} = \vb0\),
则称\(P\)是\(\overline{S}\)的一个\DefineConcept{奇点}.

显然,非退化的二次曲线没有奇点.

如果直线\(L\)整个在\(\overline{S}\)上,
那么仍然可以推得\cref{equation:配极.二次曲线的切线的齐次射影坐标方程},
因此仍有上述结论.

\subsection{极点,极线}
在二次曲线\(\overline{S}\)外
取一点\(P(p_1,p_2,p_3)\),
过点\(P\)引任意一条直线\(L\),
使得\(L\)与\(\overline{S}\)有两个不同的交点\(G,H\),
作点\(P\)关于点\(G,H\)的调和共轭点\(Q\).
用这样的方法所做的点\(Q\)的轨迹
称为“点\(P\)关于二次曲线\(\overline{S}\)的\DefineConcept{配极}”,
记作\(\mathcal{P}(P,\overline{S})\);
相对地,把点\(P\)称为
“配极\(\mathcal{P}(P,\overline{S})\)关于\(\overline{S}\)的\DefineConcept{极点}”.

现在来求点\(P\)关于二次曲线\(\overline{S}\)的配极的方程.
设\(Q(q_1,q_2,q_3)\)是配极上任意一点,
则直线\(PQ\)的参数方程是\cref{equation:配极.直线的齐次射影坐标方程},
它与\(\overline{S}\)的交点\(G,H\)对应的参数值\(\lambda_i,\mu_i\)
满足\cref{equation:配极.直线与二次曲线的交点的齐次射影坐标方程},
点\(G\)的坐标为\begin{equation*}
	(
		\lambda_1 p_1 + \mu_1 q_1,
		\lambda_1 p_2 + \mu_1 q_2,
		\lambda_1 p_3 + \mu_1 q_3
	),
\end{equation*}
点\(H\)的坐标为\begin{equation*}
	(
		\lambda_2 p_1 + \mu_2 q_1,
		\lambda_2 p_2 + \mu_2 q_2,
		\lambda_2 p_3 + \mu_2 q_3
	),
\end{equation*}
从而\(P,Q,G,H\)的交比为\begin{equation*}
	R(P,Q,G,H)
	= \frac{\lambda_2}{\mu_2} \bigg/ \frac{\lambda_1}{\mu_1}.
\end{equation*}
因为\(
	R(P,Q,G,H)
	= R(G,H,P,Q)
	= -1
\),
所以\begin{equation*}
	\frac{\lambda_1}{\mu_1}
	+ \frac{\lambda_2}{\mu_2}
	= 0.
\end{equation*}
于是由\cref{equation:配极.直线与二次曲线的交点的齐次射影坐标方程} 可得\begin{equation*}
	\overline{F}(p_1,p_2,p_3;q_1,q_2,q_3)
	= 0.
\end{equation*}
这说明,配极\(\mathcal{P}(P,\overline{S})\)上的任意一点\(Q(q_1,q_2,q_3)\)
满足\begin{equation}\label{equation:配极.配极的齐次射影坐标方程}
%@see: 《解析几何》(丘维声) P305 (6.9)
	\begin{bmatrix}
		p_1 \\ p_2 \\ p_3
	\end{bmatrix}^T
	\vb{A}
	\begin{bmatrix}
		q_1 \\ q_2 \\ q_3
	\end{bmatrix}
	= 0.
\end{equation}

\begingroup
\setlist{%
    itemindent=3.5em,%修改enumitem全局设置,使得item序号对齐悬挂缩进2个字
}
\begin{enumerate}[label={{\rm\bf 情形}\arabic*.}]
	\item
	若点\(P\)不在\(\overline{S}\)上,
	所以\((p_1,p_2,p_3) \vb{A} \neq \vb0\),
	从而\cref{equation:配极.配极的齐次射影坐标方程}
	是关于\(q_1,q_2,q_3\)的一次方程.
	显然,点\(P\)关于\(\overline{S}\)的配极\(\mathcal{P}(P,\overline{S})\)上的任意一点\(Q\)
	都在直线 \labelcref{equation:配极.配极的齐次射影坐标方程} 上,
	也就是说,配极\(\mathcal{P}(P,\overline{S})\)是直线 \labelcref{equation:配极.配极的齐次射影坐标方程} 的子集.
	为简便起见,我们干脆把整条直线 \labelcref{equation:配极.配极的齐次射影坐标方程}
	称为“点\(P\)关于\(\overline{S}\)的一条\DefineConcept{极线}”.

	\item
	若点\(P\)在\(\overline{S}\)上,
	但\(P\)不是\(\overline{S}\)的奇点,
	则以\(P\)为切点的切线方程 \labelcref{equation:配极.二次曲线的切线的齐次射影坐标方程}
	与方程 \labelcref{equation:配极.配极的齐次射影坐标方程} 形式一样.
	因此,这时我们把以\(P\)为切点的切线
	称为“点\(P\)关于\(\overline{S}\)的一条\DefineConcept{极线}”.

	\item
	若点\(P\)是\(\overline{S}\)的奇点,
	即\((p_1,p_2,p_3) \vb{A} = \vb0\),
	那么\(\overline{\pi_0}\)上任意一点都满足
	方程 \labelcref{equation:配极.配极的齐次射影坐标方程}.
	因此,这时我们把每一条直线
	称为“点\(P\)关于\(\overline{S}\)的一条\DefineConcept{极线}”.
\end{enumerate}
\endgroup

极线具有两条重要性质.
\begin{property}
%@see: 《解析几何》(丘维声) P305
点\(P\)的极线\(\mathcal{P}(P,\overline{S})\)上的
任意一点\(Q\)的极线\(\mathcal{P}(Q,\overline{S})\)经过点\(P\).
\end{property}

\begin{property}
%@see: 《解析几何》(丘维声) P306
点\(P\)的极线\(\mathcal{P}(P,\overline{S})\)
经过点\(P\)的充分必要条件是
\(P\)在\(\overline{S}\)上.
\end{property}

\begin{definition}
%@see: 《解析几何》(丘维声) P306 定义6.2
设\(\overline{S}\)是非退化的二次曲线,
\(\overline{\pi_0}\)上的点的集合与直线的集合之间有一个双射\(\tau\),
将点\(P\)映成\(P\)关于\(\overline{S}\)的极线,
将直线\(L\)映成\(L\)关于\(\overline{S}\)的极点,
我们把\(\tau\)称为
“扩充欧氏平面\(\overline{\pi_0}\)关于\(\overline{S}\)的\DefineConcept{配极映射}”.
\end{definition}

配极映射保持关联性.
这是因为如果点\(P\)在直线\(L\)上,
假设\(L\)是\(Q\)的极线,
则点\(P\)的极线必定经过\(Q\).

\subsection{自配极三角形}
\begin{definition}
%@see: 《解析几何》(丘维声) P306 定义6.3
设\(\overline{S}\)是射影平面\(\overline{\pi_0}\)上的一条二次曲线,
\(\triangle ABC\)是\(\overline{\pi_0}\)上的一个三角形.
如果\(\triangle ABC\)的每一条边
都是对顶点的极线,
那么称“\(\triangle ABC\)是关于\(\overline{S}\)的一个\DefineConcept{自配极三角形}”.
\end{definition}

\begin{theorem}
%@see: 《解析几何》(丘维声) P306 定理6.1
对于射影平面\(\overline{\pi_0}\)上
每一条二次曲线\(\overline{S}\),
总存在关于\(\overline{S}\)的一个自配极三角形.
%TODO proof
\end{theorem}

\subsection{二次曲线的射影分类}
任意给定一条二次曲线\(\overline{S}\),
假设它在基底 I 中的方程为\cref{equation:配极.二次曲线的齐次射影坐标方程}.
取关于\(\overline{S}\)的一个自配极三角形\(ABC\)作为坐标三角形
建立一个基底 II \([A,B,C,F]\).
假设\(\overline{S}\)在 II 中的方程为\begin{equation*}
%@see: 《解析几何》(丘维声) P307 (6.11)
	\sum_{i=1}^3 \sum_{j=1}^3 a^*_{ij} x^*_i x^*_j = 0.
\end{equation*}
假设\(\triangle ABC\)的三条边\(BC,AC,AB\)在 II 中的方程分别是\begin{equation*}
	x^*_1 = 0,
	\qquad
	x^*_2 = 0,
	\qquad
	x^*_3 = 0.
\end{equation*}
因为\(BC\)是点\(A\)的极线,
所以\(BC\)在 II 中的方程应该是\begin{equation*}
	\begin{bmatrix}
		1 \\ 0 \\ 0
	\end{bmatrix}^T
	\begin{bmatrix}
		a^*_{11} & a^*_{12} & a^*_{13} \\
		a^*_{12} & a^*_{22} & a^*_{23} \\
		a^*_{13} & a^*_{23} & a^*_{33}
	\end{bmatrix}
	\begin{bmatrix}
		x^*_1 \\
		x^*_2 \\
		x^*_3 \\
	\end{bmatrix}
	= 0,
\end{equation*}
即\begin{equation*}
	a^*_{11} x^*_1
	+ a^*_{12} x^*_2
	+ a^*_{13} x^*_3
	= 0.
\end{equation*}
于是有\(
	a^*_{12}
	= a^*_{13}
	= 0
\).
同理,因为\(AC\)是点\(B\)的极线,
所以有\(
	a^*_{13}
	= a^*_{23}
	= 0
\).
于是\(\overline{S}\)在 II 中的方程为\begin{equation}\label{equation:配极.二次曲线的齐次射影坐标方程-自配极坐标三角形}
%@see: 《解析几何》(丘维声) P307 (6.12)
	a^*_{11} (x^*_1)^2
	+ a^*_{22} (x^*_2)^2
	+ a^*_{33} (x^*_3)^2
	= 0.
\end{equation}

适当选取自配极三角形的顶点的次序,
\cref{equation:配极.二次曲线的齐次射影坐标方程-自配极坐标三角形}
可以分为以下几种情形.
\begingroup
\setlist{%
    itemindent=3.5em,%修改enumitem全局设置,使得item序号对齐悬挂缩进2个字
}
\begin{enumerate}[label={{\rm\bf 情形}\arabic*.}]
	\item \begin{equation*}
		\lambda_1 (x^*_1)^2
		+ \lambda_2 (x^*_2)^2
		+ \lambda_3 (x^*_3)^2
		= 0,
	\end{equation*}
	其中\(\lambda_1 \lambda_2 \lambda_3 \neq 0\);

	\item \begin{equation*}
		\lambda_1 (x^*_1)^2
		+ \lambda_2 (x^*_2)^2
		= 0,
	\end{equation*}
	其中\(\lambda_1 \lambda_2 \neq 0\);

	\item \begin{equation*}
		\lambda_1 (x^*_1)^2
		= 0,
	\end{equation*}
	其中\(\lambda_1 \neq 0\).
\end{enumerate}
\endgroup
再取基底 III,使得 II 到 III 的坐标变换公式为\begin{equation*}
	\rho x^*_i
	= \frac1{\sqrt{\abs{\lambda_i}}}
	\overline{x}_i
	\quad(i=1,2,3).
\end{equation*}
那么从情形1可以得到\begin{equation*}
	\overline{x}_1^2
	+ \overline{x}_2^2
	+ \overline{x}_3^2
	= 0,
	\qquad
	\overline{x}_1^2
	+ \overline{x}_2^2
	- \overline{x}_3^2
	= 0;
\end{equation*}
从情形2可以得到\begin{equation*}
	\overline{x}_1^2
	+ \overline{x}_2^2
	= 0,
	\qquad
	\overline{x}_1^2
	- \overline{x}_2^2
	= 0;
\end{equation*}
从情形3可以得到\begin{equation*}
	\overline{x}_1^2
	= 0.
\end{equation*}
因此,任意给定一条二次曲线\(\overline{S}\),
我们总可以适当选取一个基底,
使得\(\overline{S}\)的方程是上述五个方程中的某一个.
由于射影坐标变换公式与射影变换的公式在形式上类似,
因此由上述结论可知,
对于任给的一条二次曲线\(\overline{S}\),
我们总可以作一个适当的射影变换,
使得\(\overline{S}\)变成一条新的二次曲线\(\overline{S}'\),
而\(\overline{S}'\)在原来基底中的方程是下述之一:\begin{gather*}
%@see: 《解析几何》(丘维声) P308 (6.13)
	x_1^2 + x_2^2 + x_3^2 = 0, \\
	x_1^2 + x_2^2 - x_3^2 = 0, \\
	x_1^2 + x_2^2 = 0, \\
	x_1^2 - x_2^2 = 0, \\
	x_1^2 = 0.
\end{gather*}
其中,前两条二次曲线是非退化的,后三条是退化的.
不难看出,这五条二次曲线彼此不射影等价
(所谓“两条曲线射影等价”,
意思是存在一个射影变换把其中一条映成另一条).
于是我们说:
%@see: 《解析几何》(丘维声) P308 定理6.2
射影平面上所有二次曲线分成五个射影类,
它们的代表是上述五个方程分别所代表的曲线.

容易看出,第一类曲线\begin{equation*}
	x_1^2 + x_2^2 + x_3^2 = 0
\end{equation*}
都是没有轨迹的二次曲线,
第二类曲线\begin{equation*}
	x_1^2 + x_2^2 - x_3^2 = 0
\end{equation*}
都是欧氏平面上的椭圆、双曲线、抛物线附加它们的渐进方向(如果有的话)所对应的无穷远点,
第三类曲线\begin{equation*}
	x_1^2 + x_2^2 = 0
\end{equation*}
都是一个点,
第四类曲线\begin{equation*}
	x_1^2 - x_2^2 = 0
\end{equation*}
都是一对相交直线,
第五类直线\begin{equation*}
	x_1^2 = 0
\end{equation*}
都是一对重合直线.

射影平面上的椭圆、双曲线、抛物线都在同一个射影类里,这是不奇怪的.
这是因为椭圆、双曲线、抛物线是不同平面与圆锥面的交线,
于是在某个中心投影下,这三种曲线可以互相转化.
例如,假设椭圆\(\overline{S}_1\)是平面\(\pi_0\)与圆锥面的交线,
假设抛物线\(\overline{S}_2\)是平面\(\pi_1\)与圆锥面的交线,
那么以圆锥面顶点为中心、从\(\overline{\pi_0}\)到\(\overline{\pi_1}\)的中心投影
把椭圆\(\overline{S}_1\)映成抛物线\(\overline{S}_2\).

\subsection{斯坦纳定理,帕斯卡定理,布里昂香定理}
本小节讨论有轨迹、非退化的第二类二次曲线\(
	x_1^2 + x_2^2 - x_3^2 = 0
\)的三个定理.

\begin{theorem}
%@see: 《解析几何》(丘维声) P309 斯坦纳(Steiner)定理
设\(\overline{S}\)是一条非退化二次曲线,
\(A_1,A_2,A_3,A_4\)是\(\overline{S}\)上给定四个不同的点,
则\begin{itemize}
	\item \(\overline{S}\)上任意一点\(P\)
	与它们的连线的交比\(R(PA_1,PA_2,PA_3,PA_4)\)是一个常数
	(与点\(P\)在\(\overline{S}\)上的位置无关);

	\item 如果点\(P\)与\(A_1,A_2,A_3,A_4\)中某一个点重合,
	则\(P\)与其余三个点的连线\(PA_{i_1},PA_{i_2},PA_{i_3}\)
	与过点\(P\)的切线\(L\)的交比\(R(PA_{i_1},PA_{i_2},PA_{i_3},L)\)
	仍等于上述常数.
\end{itemize}
%TODO proof
\end{theorem}

\begin{theorem}
%@see: 《解析几何》(丘维声) P310 帕斯卡(Pascal)定理
一条非退化二次曲线的内接六边形的三对对边的交点一定共线.
%TODO proof
\end{theorem}

\begin{theorem}
%@see: 《解析几何》(丘维声) P311 布里昂香(Brianchon)定理
一条非退化二次曲线的内接六边形的三对对顶点的连线一定共点.
%TODO proof
\end{theorem}
