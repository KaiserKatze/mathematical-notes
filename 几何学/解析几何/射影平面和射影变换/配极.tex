\section{配极,二次曲线的射影分类}
\subsection{射影平面上的二次曲线}
射影平面上,射影坐标\((x_1,x_2,x_3)\)满足
二次齐次方程\begin{equation}\label{equation:配极.二次曲线的齐次射影坐标方程}
%@see: 《解析几何》(丘维声) P301 (6.1)
	a_{11} x_1^2
	+ a_{22} x_2^2
	+ a_{33} x_3^2
	+ 2 a_{12} x_1 x_2
	+ 2 a_{13} x_1 x_3
	+ 2 a_{23} x_2 x_3
	= 0
\end{equation}
的点构成的集合
称为\DefineConcept{二次曲线}.

显然,若点\(M(x_1,x_2,x_3)\)
在二次曲线 \labelcref{equation:配极.二次曲线的齐次射影坐标方程} 上,
则\((\lambda x_1,\lambda x_2,\lambda x_3)\)
也满足方程 \labelcref{equation:配极.二次曲线的齐次射影坐标方程};
并且二次曲线 \labelcref{equation:配极.二次曲线的齐次射影坐标方程}
在任意基底下的方程仍然是二次齐次方程.

记\begin{equation*}
	F(x_1,x_2,x_3)
	\defeq
	a_{11} x_1^2
	+ a_{22} x_2^2
	+ a_{33} x_3^2
	+ 2 a_{12} x_1 x_2
	+ 2 a_{13} x_1 x_3
	+ 2 a_{23} x_2 x_3,
\end{equation*}
则\begin{equation*}
	F(x_1,x_2,x_3)
	= \vb{X}^T \vb{A} \vb{X},
\end{equation*}
其中\begin{equation*}
	\vb{A}
	\defeq
	\begin{bmatrix}
		a_{11} & a_{12} & a_{13} \\
		a_{12} & a_{22} & a_{23} \\
		a_{13} & a_{23} & a_{33} \\
	\end{bmatrix},
	\qquad
	\vb{X}
	\defeq
	\begin{bmatrix}
		x_1 \\ x_2 \\ x_3
	\end{bmatrix}.
\end{equation*}
显然\(\vb{A}\)是一个对称矩阵.
如果\(\vb{A}\)是非奇异的,
则称“二次曲线 \labelcref{equation:配极.二次曲线的齐次射影坐标方程} 是\DefineConcept{非退化的}”;
否则称“二次曲线 \labelcref{equation:配极.二次曲线的齐次射影坐标方程} 是\DefineConcept{退化的}”.

现在我们可以把方程 \labelcref{equation:配极.二次曲线的齐次射影坐标方程}
改写成\begin{equation}
%@see: 《解析几何》(丘维声) P301 (6.1)'
	\vb{X}^T \vb{A} \vb{X} = 0
\end{equation}
或\begin{equation}
%@see: 《解析几何》(丘维声) P301 (6.1)''
	\sum_{i=1}^3 \sum_{j=1}^3 a_{ij} x_i x_j = 0
	\quad(x_{ij} = x_{ji}).
\end{equation}

现在考虑扩充欧氏平面\(\overline{\pi_0}\).
在\(\pi_0\)上取一个直角标架\([O_1;e_1,e_2]\),
相应的\(\overline{\pi_0}\)的基底记作 I,
则点\(M\)在 I 中的射影坐标\((x_1,x_2,x_3)\)就是\(M\)的齐次坐标.
于是\(M\)在\([O_1;e_1,e_2]\)中的仿射坐标\((x,y)\)满足\begin{equation*}
	x = \frac{x_1}{x_3},
	\qquad
	y = \frac{x_2}{x_3}.
\end{equation*}

假设二次曲线\(\overline{S}\)在 I 中的方程是\cref{equation:配极.二次曲线的齐次射影坐标方程}.
\begingroup
\setlist{%
    itemindent=3.5em,%修改enumitem全局设置,使得item序号对齐悬挂缩进2个字
}
\begin{enumerate}[label={{\rm\bf 情形}\arabic*.}]
	\item 设\(a_{11},a_{22},a_{12}\)不全为零.
	通常点\(M(x_1,x_2,x_3)\)在\(\overline{S}\)上的充分必要条件是
	\(M\)的仿射坐标\((x,y)\)适合\begin{equation}\label{equation:配极.二次曲线的非齐次仿射坐标方程1}
	%@see: 《解析几何》(丘维声) P302 (6.2)
		a_{11} x^2
		+ a_{22} y^2
		+ 2 a_{12} x y
		+ 2 a_{13} x
		+ 2 a_{23} y
		+ a_{33}
		= 0.
	\end{equation}
	无穷远点\(N(x_1,x_2,0)\)在\(\overline{S}\)上的充分必要条件是
	\(N\)相应的\(\pi_0\)上的方向\(v = (x_1,x_2)\)适合\begin{equation}
	%@see: 《解析几何》(丘维声) P302 (6.3)
		a_{11} x_1^2
		+ a_{22} x_2^2
		+ 2 a_{12} x_1 x_2
		= 0.
	\end{equation}
	显然,\(v\)是\(\pi_0\)上的二次曲线 \labelcref{equation:配极.二次曲线的非齐次仿射坐标方程1} 的渐进方向.

	这就说明:如果\(a_{11},a_{22},a_{12}\)不全为零,
	那么\(\overline{\pi_0}\)上的二次曲线\(\overline{S}\)
	是由欧氏平面\(\pi_0\)上的二次曲线 \labelcref{equation:配极.二次曲线的非齐次仿射坐标方程1}
	以及它的渐进方向所对应的无穷远点所组成.

	\item 设\(a_{11} = a_{22} = a_{12} = 0\).
	这时,通常点\(M(x_1,x_2,x_3)\)在\(\overline{S}\)上的充分必要条件是
	它的仿射坐标\((x,y)\)满足方程\begin{equation}\label{equation:配极.二次曲线的非齐次仿射坐标方程2}
	%@see: 《解析几何》(丘维声) P302 (6.4)
		2 a_{13} x
		+ 2 a_{23} y
		+ a_{33} = 0.
	\end{equation}
	显然,任意一个无穷远点的坐标都满足
	方程 \labelcref{equation:配极.二次曲线的齐次射影坐标方程},
	从而整条无穷远直线都在\(\overline{S}\)上.

	这就说明:如果\(a_{11} = a_{22} = a_{12} = 0\),
	那么\(\overline{\pi_0}\)上的二次曲线\(\overline{S}\)
	要么是由一条射影直线 \labelcref{equation:配极.二次曲线的非齐次仿射坐标方程2}
	和一条无穷远直线组成,
	要么是由一对重合的无穷远直线组成.
\end{enumerate}
\endgroup

总之,射影平面\(\overline{\pi_0}\)上的二次曲线\(\overline{S}\)有11种,
其中9种分别是欧氏平面\(\pi_0\)上的二次曲线附加它的渐进方向所对应的无穷远点,
另外2种分别是
一条射影直线 \labelcref{equation:配极.二次曲线的非齐次仿射坐标方程2} 和一条无穷远直线,
一对重合的无穷远直线.
