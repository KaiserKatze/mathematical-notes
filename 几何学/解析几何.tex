\section{平面二次曲线及其方程}

\subsection{平面曲线在极坐标系下方程}
在平面内取一个定点\(O\),称为\DefineConcept{极点}.
过极点引一条射线\(Ox\),称为\DefineConcept{极轴}.
再选一个长度单位和计算角度的正方向(通常取逆时针方向),
这样的坐标系称为\DefineConcept{极坐标系}.
连接平面内任意一点\(P\)与极点\(O\)所得的线段\(OP\),称为“点\(P\)的\DefineConcept{极径}”.
这条线段\(OP\)与极轴所得的夹角\(\angle POx\),称为“点\(P\)的\DefineConcept{极角}”.

\begin{table}[htb]
	\centering
	\begin{tblr}{c|c|c}
		\hline
		& 直角坐标方程
		& 极坐标方程 \\ \hline
		过极点的直线的方程
			& \(y = x \tan\theta_0\)
			& \(\theta=\theta_0\ (\text{$\theta_0$是常数})\) \\
		垂直于极轴的直线的方程
			& \(x = a\)
			& \(\rho\cos\theta=a\) \\
		平行于极轴的直线的方程
			& \(y = a\)
			& \(\rho\sin\theta=a\) \\
			& \(x + y = a\)
			& \(\rho = \frac{a}{\cos\theta + \sin\theta}\) \\
		中心在极点、半径为\(r\)的圆
			& \(x^2+y^2=r^2\)
			& \(\rho=r\) \\
		中心在\((r,0)\)、半径为\(r\)的圆
			& \((x-r)^2+y^2=r^2\)
			& \(\rho=2r\cos\theta\) \\
		中心在\((r,\pi)\)、半径为\(r\)的圆
			& \((x+r)^2+y^2=r^2\)
			& \(\rho=-2r\cos\theta\) \\
		中心在\((\rho_0,\theta_0)\)、半径为\(r\)的圆
			&
			& \(\rho^2+\rho_0^2-2\rho_0\rho\cos(\theta-\theta_0)=r^2\) \\
		\hline
	\end{tblr}
	\caption{}
\end{table}


\section{本章总结}

\subsection*{常见的坐标系}
极坐标系是一种平面坐标系.
极坐标系上的点\((\rho,\theta)\)与平面直角坐标系上的点\((x,y)\)存在以下的一一对应关系:\begin{equation*}
\left\{ \begin{array}{l}
x = \rho\cos\theta, \\
y = \rho\sin\theta. \\
\end{array} \right.
\end{equation*}

柱面坐标系是一种空间坐标系.
柱面坐标系上的点\((\rho,\theta,z)\)与空间直角坐标系上的点\((x,y,z)\)存在以下的一一对应关系:\begin{equation*}
\left\{ \begin{array}{l}
x = \rho\cos\theta, \\
y = \rho\sin\theta, \\
z = z. \\
\end{array} \right.
\end{equation*}

球极坐标系是一种空间坐标系.
球极坐标系上的点\((r,\theta,\phi)\)与空间直角坐标系上的点\((x,y,z)\)存在以下的一一对应关系:\begin{equation*}
\left\{ \begin{array}{l}
x = r \sin\phi \cos\theta, \\
y = r \sin\phi \sin\theta, \\
z = r \cos\phi. \\
\end{array} \right.
\end{equation*}

\begin{figure}[h]
	\centering
	\begin{tikzpicture}
		\coordinate (O) at (0,0);
		\coordinate (M) at (2,2);
		\coordinate (P) at (2,-1.2);
		\coordinate (A) at (-0.8,-1.2);
		\coordinate (Z) at (0,1);
		\draw (O)node[left]{\(O\)}
				-- (M)node[right]{\(M\)} node[midway,above]{\(r\)}
				-- (P)node[right]{\(P\)} node[midway,right]{\(z\)}
				-- (A)node[left]{\(A\)} node[midway,below]{\(y\)}
				-- (O)node[midway,left]{\(x\)}
			(O) -- (P)
			pic["\(\phi\)",draw=orange,<-,angle eccentricity=1.3,angle radius=0.6cm]{angle=M--O--Z}
			pic["\(\theta\)",draw=orange,->,angle eccentricity=1.5,angle radius=0.4cm]{angle=A--O--P};
		\begin{scope}[>=Stealth,->,ultra thick]
			\draw(0,0) -- (-1.5,-2.25) node[right]{\(x\)};
			\draw(0,0) -- (3,0) node[right]{\(y\)};
			\draw(0,0) -- (0,3) node[above]{\(z\)};
		\end{scope}
	\end{tikzpicture}
	\caption{球极坐标系}
\end{figure}
