\section{第五组公理:连续公理}
\begin{axiom}[连续公理]
关于线段、角的度量,有如下公理:
\begin{enumerate}
	\item (度量公理,阿基米德公理).
	若\(AB\)和\(CD\)是任意两线段,则必存在一个数\(n\)使得沿\(A\)到\(B\)的射线上,
	自\(A\)作首尾相接的\(n\)个线段\(CD\),必将越过\(B\)点.
	\item (直线完备公理).
	一直线上的点集联通其顺序关系与合同关系不可能再这样扩充,
	使得这直线上原来元素之间所具有的关系,
	从第一组、第二组、第三组公理所推出的直线顺序与合同的基本性质\footnote{%
	所谓“基本性质”是指第二组第一至三条公理和\cref{theorem:欧氏几何.定理5} 中所叙述的顺序性质,
	以及第三组第一至第三条公理中所叙述的合同性质连同迁移线段的唯一性.}
	以及度量公理都仍旧保持\footnote{%
	所谓“仍旧保持”是指,当点集扩充后,顺序关系及合同关系也将延续到扩充后的点集中去.
	我们注意到第一组第三条公理在各种扩充后,不言而喻地仍然保持,
	至于在所考虑的扩充下,\cref{theorem:欧氏几何.定理3} 仍能成立,
	则是保持阿基米德公理的结果.}.
\end{enumerate}
\end{axiom}

完备公理中所要求保持的诸公理之一是阿基米德公理,
这时完备公理从本质上能以建立所不可缺少的一个条件.
其实我们能够证明:
若直线上的一个点集能满足上面所列举的关于顺序公理和定理以及合同公理和定理,
这点集就恒能够增加新点,使扩充后的点集还满足这里所提到的诸公理;
也就是说,如果一条完备公理,只要求保持这里所提到的诸公理和定理,
但不要求保持阿基米德公理或一条等价的公理,就要产生矛盾.

这两条连续公理都是直线公理.

下面所述更普遍的定理,主要根据直线完备公理.
\begin{theorem}[完备定理]\label{theorem:欧氏几何.定理32}
几何元素(即点、直线、平面)形成一个集合,
它在保持关联公理、顺序公理、合同公理和阿基米德公理,
从而更不用说,在保持全体公理的条件之下,
不可能经由点、直线和平面再行扩充.
\end{theorem}

完备定理还能表成较强的形式.
也就是在完备定理内缩要求保持的诸公理中,有些并不是绝对需要的,为了定理能够成立,
重要的倒是在所要求保持的诸公理中包含有第一组第七条公理.
其实我们能证明:
对于满足第一至第第五组公理的元素集合,恒能给添加新的点、直线和平面,
使得扩充后的心机和能满足除去第一组第七条公理之外的全体公理;
这就是说,一条完备定理若不包含第一组第七条公理或一条等价的公理,就将引出矛盾.

完备公理不是阿基米德公理的一个推论.
实际上,只有阿基米德公理,连同第一至第四组公理,
并不足以证明我们的几何和通常的笛卡尔解析几何完全相同.
但是加上了完备公理(虽然这条公理并没有直接提到收敛的概念),
就能证明(相当于戴德金分割的)确界的存在,
和关于聚点存在的波尔查诺定理,
从而才证明欧氏几何和笛卡尔几何相同.

从上文可见,连续的要求,在本质上,分成两个不同的部分:
阿基米德公理和完备公理;
前者的作用是替连续的要求做准备,
后者为完成整个公理系统作基础.

在本章后面的研究中,我们主要只用阿基米德公理作根据,而普遍地不假设完备公理.
