\section{几何元素和五组公理}
\begin{definition}\label{definition:欧氏几何.几何元素.基本几何元素}
在平面几何中,有两种基本研究对象:
\begin{enumerate}
	\item \DefineConcept{点}(我们常用大写拉丁字母\(A,B,C,\dotsc\)表示);
	\item \DefineConcept{直线}(我们常用小写拉丁字母\(a,b,c,\dotsc\)表示).
\end{enumerate}
在空间几何中,除了上述两种以外,还多了一种研究对象:
\begin{enumerate}
\setcounter{enumi}{2}
	\item \DefineConcept{平面}(我们常用小写希腊字母\(\alpha,\beta,\gamma,\dotsc\)表示).
\end{enumerate}
点和直线统称为“平面几何的\emph{元素}”;
点、直线和平面统称为“空间几何的\emph{元素}”.
\end{definition}

我们设想,点、直线、平面这三类几何元素之间总是存在某种关系.
根据这些关系,我们提出以下五组命题,并认定它们恒为真,特别地称它们为“公理(axiom)”.
