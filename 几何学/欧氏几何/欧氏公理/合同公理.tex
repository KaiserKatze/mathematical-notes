\section{第三组公理:合同公理}
本组公理规定“合同”这个概念,利用它就可以规定运动的概念.

我们首先介绍角的概念.
\begin{definition}\label{definition:欧氏几何.几何元素.角}
设\(\alpha\)是任意平面,
而且\(h\)和\(k\)是\(\alpha\)上的、%
从一点\(A\)起始的、%
不属于同一直线的%
两条射线,
我们把这一对射线\(h\)和\(k\)所成的射线组叫做一个\DefineConcept{角},
记作\(\angle(h,k)\)或\(\angle(k,h)\).
称射线\(h\)和\(k\)为这个角的\DefineConcept{边}.
称点\(A\)为这个角的\DefineConcept{顶点}.

如果在\(h\)上任取一点记为\(B\),在\(k\)上任取一点记为\(C\),
那么也称角\(\angle(h,k)\)为\(\angle BAC\)或\(\angle A\).
不致混淆时,也可以用小写希腊字母表记角.

记射线\(h\)所在的直线为\(p\),射线\(k\)所在的直线为\(q\).
射线\(h\)与\(k\)(包括点\(A\))把平面\(\alpha\)上其余点分成两个区域:
在\(q\)的\(h\)侧(即\(h\)的点所在的那一侧)的,且%
在\(p\)的\(k\)侧(即\(k\)的点所在的那一侧)的区域,
叫做“角\(\angle(h,k)\)的\DefineConcept{内部}”,
或者说是在\DefineConcept{角内};
其他区域叫做“角\(\angle(h,k)\)的\DefineConcept{外部}”,
或者说是在\DefineConcept{角外}.
\end{definition}

\begin{property}
根据第一组和第二组公理,
易知任意角(不妨设为交于点\(A\)的射线\(h\)、\(k\)所成的角,即\(\angle(h,k)\))具有以下性质:
\begin{enumerate}
\item 角内、角外两个区域各含有点,连结角内两点的线段完全在角内.
\item 若点\(H\)在射线\(h\)上,点\(K\)在射线\(k\)上,则线段\(HK\)完全在角内.
\item 一条从点\(A\)起始的射线,要么完全在角内,要么完全在角外.
\item 一条完全在角内的射线与线段\(HK\)有交点.
\item 若\(B\)是一个区域的一点,而且\(C\)是另一个区域的一点,
则每一条连接\(B\)和\(C\)的折线段,要么通过点\(A\),要么与\(h\)或\(k\)至少有一个交点;
反之,若\(B\)和\(D\)是同一个区域的两点,则恒有一条连接\(B\)和\(D\)的折线段,
它既不通过点\(A\),又与\(h\)、\(k\)无交点.
\end{enumerate}
\end{property}


\begin{axiom}[合同公理]\label{axiom:欧氏几何.合同公理}
线段与线段之间、角与角之间都有一定的相互关系,我们用“合同”或“相等”这个词来描述.
\begin{enumerate}
\item 设\(A\)和\(B\)是直线\(a\)上的两点,
\(P\)是直线\(b\)上的点,
而且给定了直线\(b\)上\(P\)的一侧,
则在直线\(b\)上\(P\)的这一侧,
恒有一点\(Q\),
使得线段\(AB\)和线段\(PQ\)合同.
我们将上述关系记为\(AB \equiv PQ\).

\item 若两线段\(PQ\)和\(MN\)都和线段\(AB\)合同,
则\(PQ\)和\(MN\)也合同.

\item 设两线段\(AB\)和\(BC\)在同一直线\(a\)上,无公共点,
而且两线段\(PQ\)和\(QR\)在同一直线\(b\)上,亦无公共点.
若\(AB \equiv PQ\)且\(BC \equiv QR\),
则\(AC \equiv PR\).

\item 设给定了一个平面\(\alpha\)上的一个角\(\angle(h,k)\),
一平面\(\beta\)上的一直线\(b\),
和在\(\beta\)上\(b\)的一侧.
设\(p\)是\(b\)上的、从点\(B\)起始的一条射线,
则平面\(\beta\)上恰有一条射线\(q\),
使得\(\angle(h,k)\)与\(\angle(p,q)\)合同,
而且使得\(\angle(p,q)\)的内部在\(b\)的这给定了的一侧.
我们将上述关系记为\(\angle(h,k) \equiv \angle(p,q)\).

\item 若两个三角形\(ABC\)和\(PQR\)有下列合同式
\begin{equation*}
AB \equiv PQ, \qquad
AC \equiv PR, \qquad
\angle BAC \equiv \angle QPR,
\end{equation*}
则也恒有合同式\footnote{%
只需要交换记号,还可以同时得到另一个合同式
\(\angle ACB \equiv \angle PRQ\)
也同时成立.
}
\begin{equation*}
\angle ABC \equiv \angle PQR.
\end{equation*}
\end{enumerate}
\end{axiom}

我们在前面用点\(A\)、\(B\)所成的点组规定一条线段,并用\(AB\)或\(BA\)表示;
我们在线段的定义里,并不考虑这两点的顺序;
因此下列四个合同式的意义相同:
\begin{equation*}
AB \equiv PQ, \qquad
AB \equiv QP, \qquad
BA \equiv PR, \qquad
BA \equiv QP.
\end{equation*}

如同线段我们不考虑它的方向,在角的定义中我们也不考虑旋转方向.
因此下列四个合同式的意义也相同:
\begin{equation*}
\angle(h,k) \equiv \angle(p,q), \qquad
\angle(h,k) \equiv \angle(q,p), \qquad
\angle(k,h) \equiv \angle(p,q), \qquad
\angle(k,h) \equiv \angle(q,p).
\end{equation*}

\cref{axiom:欧氏几何.合同公理} 第3条要求线段能够相加.

\cref{axiom:欧氏几何.合同公理} 第4条可以表述为:
每一个角都能用唯一确定的方式迁移到一个给定了的平面上,
使得它沿着一条给定了的射线,并且在这射线的给定了的一侧.
藉此,我们直接保证了角的迁移的可能性与唯一性.

\cref{axiom:欧氏几何.合同公理} 第1条、第2条、第3条只论及线段的合同,
因此可以叫做“第三组公理中的直线公理”.

\cref{axiom:欧氏几何.合同公理} 第4条论及角的合同.
\cref{axiom:欧氏几何.合同公理} 第5条则把线段的合同和角的合同这两个概念联系起来.
这两条概念论及平面几何的几何元素,
因此可以叫做“第三组公理中的平面公理”.

\cref{axiom:欧氏几何.合同公理} 第1条要求线段平移的可能性,
但它还没有保证这种平移的唯一性.
只有结合\cref{axiom:欧氏几何.合同公理} 第5条,
从角的迁移的唯一性出发予以证明.
具体地,我们应用反证法,
假设把线段\(PQ\)迁移到一条从\(A\)起始的射线上可以得到不同的两点\(B\)、\(D\);
在直线\(AB\)外取一点\(C\),于是有下列合同式
\begin{equation*}
AB \equiv AD, \qquad
AC \equiv AC, \qquad
\angle BAC \equiv \angle DAC;
\end{equation*}
那么根据\cref{axiom:欧氏几何.合同公理} 第5条,得
\begin{equation*}
\angle ACB \equiv \angle ACD;
\end{equation*}
这和\cref{axiom:欧氏几何.合同公理} 第4条中要求的角的迁移的唯一性矛盾,
因此线段平移也是唯一的.

\begin{property}
线段的合同关系具有自反性、对称性和传递性,即
\begin{enumerate}
\item {\rm\bf 自反性},
\(AB \equiv AB\).

\item {\rm\bf 对称性},
\(AB \equiv PQ \implies PQ \equiv AB\)
\footnote{%
正因线段的合同关系具有对称性,
我们才能说“某两条线段互相合同”.
}.

\item {\rm\bf 传递性},
\(AB \equiv PQ \land PQ \equiv MN \implies AB \equiv MN\).
\end{enumerate}
\end{property}

\begin{property}
角的合同关系具有自反性、对称性和传递性,即
\item {\rm\bf 自反性},
\(\angle(h,k) \equiv \angle(h,k)\).

\item {\rm\bf 对称性},
\(\angle(h,k) \equiv \angle(p,q)
\implies
\angle(p,q) \equiv \angle(h,k)\).

\item {\rm\bf 传递性},
\(\angle(h,k) \equiv \angle(m,n)
\land
\angle(m,n) \equiv \angle(p,q)
\implies
\angle(h,k) \equiv \angle(p,q)\).
\end{property}
角的合同关系的自反性是显然的,
至于它的对称性、传递性和可加性,
则留待以后予以证明.
