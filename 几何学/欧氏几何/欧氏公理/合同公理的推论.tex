\section{合同公理的推论}
\begin{definition}
两角共顶点,共一边,而且不公共的两边合成一条直线的,叫做\DefineConcept{邻补角}.
\end{definition}
\begin{definition}
两角共顶点,而且它们的边合成两条直线的,叫做\DefineConcept{对顶角}.
\end{definition}
\begin{definition}
一个角和它的邻补角合同的,叫做\DefineConcept{直角}.
\end{definition}

\begin{theorem}\label{theorem:欧氏几何.定理11}
若一个三角形中的两边合同,和这两边相对的两角就也合同\footnote{%
换言之,等腰三角形的底角相等.
}.
\end{theorem}

\begin{definition}
若两个三角形\(ABC\)和\(PQR\)满足下列所有的合同式
\begin{equation*}
\begin{split}
AB \equiv PQ, \qquad
AC \equiv PR, \qquad
BC \equiv QR, \\
\angle A \equiv \angle P, \qquad
\angle B \equiv \angle Q, \qquad
\angle C \equiv \angle R,
\end{split}
\end{equation*}
就说“三角形\(ABC\)合同于三角形\(PQR\)”,
或者说“三角形\(ABC\)和三角形\(PQR\)是全等三角形”,
或者说“三角形\(ABC\)、\(PQR\)全等”,
记为\(\triangle ABC \cong \triangle PQR\).
\end{definition}

\begin{theorem}[三角形的合同定理1]\label{theorem:欧氏几何.定理12}
若两个三角形\(ABC\)、\(PQR\)有下列合同式
\begin{equation*}
AB \equiv PQ, \qquad
AC \equiv PR, \qquad
\angle A \equiv \angle P,
\end{equation*}
则\(\triangle ABC \cong \triangle PQR\).
\end{theorem}

\begin{theorem}[三角形的合同定理2]\label{theorem:欧氏几何.定理13}
若两个三角形\(ABC\)、\(PQR\)有下列合同式
\begin{equation*}
AB \equiv PQ, \qquad
\angle A \equiv \angle P,
\angle B \equiv \angle Q,
\end{equation*}
则\(\triangle ABC \cong \triangle PQR\).
\end{theorem}

\begin{theorem}\label{theorem:欧氏几何.定理14}
设\(\angle ABC\)的邻补角为\(\angle CBD\),
\(\angle PQR\)的邻补角为\(\angle RQS\).
若\(\angle ABC \equiv \angle PQR\),
则\(\angle CBD \equiv \angle RQS\).
\end{theorem}

\begin{corollary}\label{theorem:欧氏几何.对顶角合同}
任意一个角和它的对顶角合同.
\end{corollary}

\begin{corollary}\label{theorem:欧氏几何.直角存在}
直角存在.
\begin{proof}
把任意一个角迁移到沿着一条从点\(O\)起始的射线\(OA\),
而且迁移到这射线的两侧.
在新得到的这两个角的另外两条边上,
取线段\(OB \equiv OC\),
线段\(BC\)交射线\(OA\)于一点\(D\).

若点\(D\)就是点\(O\),
\(\angle BOA\)和\(\angle COA\)是合同的邻补角,所以是直角.

若点\(D\)在射线\(OA\)上,
或在与\(OA\)恰好反向的射线上,
总有\(\angle DOB \equiv \angle DOC\);
根据\cref{axiom:欧氏几何.合同公理} 第2条,
每一条线段都和它自己合同,即\(OD \equiv OD\);
再根据\cref{axiom:欧氏几何.合同公理} 第5条,
就有\(\angle ODB \equiv \angle ODC\).
\end{proof}
\end{corollary}

\begin{theorem}\label{theorem:欧氏几何.定理15}
设\(h\)、\(k\)和\(l\)是一平面\(\alpha\)上的、从一点\(M\)起始的三条射线,
而且\(p\)、\(q\)和\(r\)是一平面\(\beta\)上的、从一点\(N\)起始的三条射线;%
又设\(h\)和\(k\)分别在\(l\)的同侧(或异侧),
且\(p\)和\(q\)也分别在\(r\)的同侧(或异侧).
若
\begin{equation*}
\angle(h,l) \equiv \angle(p,r)
\quad\text{且}\quad
\angle(k,l) \equiv \angle(q,r),
\end{equation*}
则
\begin{equation*}
\angle(h,k) \equiv \angle(p,q).
\end{equation*}
\end{theorem}

\begin{theorem}\label{theorem:欧氏几何.定理16}
设平面\(\alpha\)上的\(\angle(h,k)\)合同于平面\(\beta\)上的\(\angle(p,q)\),
而且\(l\)是平面\(\alpha\)上的、从\(\angle(h,k)\)的顶点起始的、在\(\angle(h,k)\)角内的一条射线.
这时平面\(\beta\)上恒恰有一条从\(\angle(p,q)\)的顶点起始的、在\(\angle(p,q)\)角内的一条射线\(r\),
使得
\begin{equation*}
\angle(h,l) \equiv \angle(p,r), \qquad
\angle(k,l) \equiv \angle(q,r).
\end{equation*}
\end{theorem}

\begin{theorem}\label{theorem:欧氏几何.定理17}
若两点\(C\)和\(D\)在直线\(AB\)的异侧,
而且\(AC \equiv AD\)、\(BC \equiv BD\),
则\(\angle ABC \equiv \angle ABD\).
\end{theorem}

\begin{theorem}[三角形的合同定理3]\label{theorem:欧氏几何.定理18}
若两个三角形\(ABC\)和\(PQR\)的每对对应边合同,即
\begin{equation*}
AB \equiv PQ, \qquad
AC \equiv PR, \qquad
BC \equiv QR,
\end{equation*}
则\(\triangle ABC \cong \triangle PQR\).
\end{theorem}

\begin{theorem}\label{theorem:欧氏几何.定理19}
若两个角\(\angle(a,b)\)和\(\angle(c,d)\)都合同于第三个角\(\angle(e,f)\),
则\(\angle(a,b)\)也合同于\(\angle(c,d)\).
\end{theorem}
由此,我们证明了角的合同关系具有对称性、传递性.

现在我们就可以比较角的大小了.

\begin{theorem}\label{theorem:欧氏几何.定理20}
如\cref{figure:欧氏几何.图20},
给定任意两个角\(\angle(h,k)\)和\(\angle(p,r)\).
设迁移\(\angle(h,k)\)到沿着\(p\),而且在\(p\)的\(r\)侧时,所得到的射线是\(q\);
又迁移\(\angle(p,r)\)到沿着\(h\),而且在\(h\)的\(k\)侧时,所得到的射线是\(l\);
这时,若\(q\)在\(\angle(p,r)\)内,则\(l\)在\(\angle(h,k)\)外.
反之也成立.
\end{theorem}

\begin{figure}[htb]
	\centering
	\begin{tikzpicture}
		\pgfmathsetmacro{\r}{3}
		\begin{scope}
			\draw(0,0)--(\r,0)node[right]{\(h\)};
			\pgfmathsetmacro{\kx}{\r*cos(45)}
			\pgfmathsetmacro{\ky}{\r*sin(45)}
			\draw(0,0)--(\kx,\ky)node[right]{\(k\)};
			\pgfmathsetmacro{\lx}{\r*cos(60)}
			\pgfmathsetmacro{\ly}{\r*sin(60)}
			\draw(0,0)--(\lx,\ly)node[above]{\(l\)};
		\end{scope}
		\begin{scope}[xshift=6cm]
			\draw(0,0)--(\r,0)node[right]{\(p\)};
			\pgfmathsetmacro{\kx}{\r*cos(45)}
			\pgfmathsetmacro{\ky}{\r*sin(45)}
			\draw(0,0)--(\kx,\ky)node[right]{\(q\)};
			\pgfmathsetmacro{\lx}{\r*cos(60)}
			\pgfmathsetmacro{\ly}{\r*sin(60)}
			\draw(0,0)--(\lx,\ly)node[above]{\(r\)};
		\end{scope}
	\end{tikzpicture}
	\caption{}
	\label{figure:欧氏几何.图20}
\end{figure}

\begin{definition}
在\cref{theorem:欧氏几何.定理20} 中,
若\(q\)在\(\angle(p,r)\)内,则称“\(\angle(h,k)\)小于\(\angle(p,r)\)”,记为\(\angle(h,k) < \angle(p,r)\).
若\(q\)在\(\angle(p,r)\)外,则称“\(\angle(h,k)\)大于\(\angle(p,r)\)”,记为\(\angle(h,k) > \angle(p,r)\).
\end{definition}

因此,两个角\(\alpha,\beta\)恒恰适合以下三种情形之一:
\begin{itemize}
	\item \(\alpha<\beta\)和\(\beta>\alpha\).
	\item \(\alpha\equiv\beta\).
	\item \(\alpha>\beta\)和\(\beta<\alpha\).
\end{itemize}

角的大小的比较有传递性.
若有下列三种情形
\begin{itemize}
	\item \(\alpha>\beta,\beta>\gamma\),
	\item \(\alpha>\beta,\beta\equiv\gamma\),
	\item \(\alpha\equiv\beta,\beta>\gamma\),
\end{itemize}
之一,则\begin{equation*}
	\alpha>\gamma.
\end{equation*}

\begin{theorem}\label{theorem:欧氏几何.定理21}
所有的直角都互相合同.
\end{theorem}

\begin{definition}
一个角大于它的邻补角的,也就是大于一直角的,叫做\DefineConcept{钝角};
小于它的邻补角的,也就是小于一直角的,叫做\DefineConcept{锐角}.
\end{definition}

\begin{definition}
\(\triangle ABC\)的\(\angle ABC\)、\(\angle BCA\)和\(\angle CAB\)
叫做这个三角形的\DefineConcept{内角},简称为这个三角形的\DefineConcept{角};
它们的邻补角叫做这个三角形的\DefineConcept{外角}.
\end{definition}

\begin{theorem}[外角定理]\label{theorem:欧氏几何.定理22}
在三角形中,一个外角大于其任一不相邻的内角.
\end{theorem}

下列定理是外角定理的重要推论.

\begin{theorem}\label{theorem:欧氏几何.定理23}
在三角形中,长边所对的角大于短边所对的角.
\end{theorem}

\begin{theorem}\label{theorem:欧氏几何.定理24}
若三角形有两角合同,则有两边合同.
\end{theorem}
这是\cref{theorem:欧氏几何.定理11} 的逆定理,
也是\cref{theorem:欧氏几何.定理23} 的直接推论.

从\cref{theorem:欧氏几何.定理22},
还能很简单地证得下述对三角形的合同定理二的补充.
\begin{theorem}\label{theorem:欧氏几何.定理25}
若\(\triangle ABC\)和\(\triangle DEF\)有下列合同式\begin{equation*}
	AB \equiv DE, \qquad
	\angle A \equiv \angle D, \qquad
	\angle C \equiv \angle F,
\end{equation*}
则这两个三角形合同.
\end{theorem}

\begin{theorem}\label{theorem:欧氏几何.定理26}
每一线段都能二等分.
\end{theorem}

类似地,从\cref{theorem:欧氏几何.定理11} 和\cref{theorem:欧氏几何.定理26},能直接推证下列事实:
\begin{theorem}
每一角都能二等分.
\end{theorem}

合同的概念可以推广应用到任意的图形上去.

\begin{definition}
设\(A,B,C,D,\dotsc,K,L\)是直线\(\alpha\)上的一个点列,
\(A',B',C',D',\dotsc,K',L'\)是直线\(\alpha'\)上的一个点列,
而且所有的对应线段都两两合同,
那么称这两个点列互相合同.
\(A\)和\(A'\)、\(B\)和\(B'\),一直到\(L\)和\(L'\),叫做这\emph{合同点列}的对应点.
\end{definition}

\begin{theorem}\label{theorem:欧氏几何.定理27}
两个合同的点列的点的顺序相同.
\end{theorem}

\begin{definition}
任意有限个点叫做一个\DefineConcept{图形}.
一个图形的点,若都在一个平面上,这图形就叫做一个\DefineConcept{平面图形}.
\end{definition}

两个图形的点之间若有一个一一对应的关系,
使得由此规定的每对对应的线段都互相合同,
且每对对应的角都互相合同,
那么这两个图形合同.

由\cref{theorem:欧氏几何.定理14}
和\cref{theorem:欧氏几何.定理27},
可知合同图形有下述性质:
若一个图形中的三个点在一条直线上,
则每一个和它合同图形中的对应的三个点也在一条直线上.
合同图形中的、对应平面上的对应点,
对于对应直线而言的顺序相同;
对应直线上的对应点顺序也相同.

平面的和空间的最普遍的合同定理如下:
\begin{theorem}\label{theorem:欧氏几何.定理28}
设\((A,B,C,\dotsc,L)\)
和\((A',B',C',\dotsc,L')\)
是两个合同的平面图形.
若\(P\)是第一个图形的平面上的一点,
则第二个图形的平面上恒有一点\(P'\)存在,
使得\((A,B,C,\dotsc,L,P)\)
和\((A',B',C',\dotsc,L',P')\)还是合同的图形.
若\((A,B,C,\dotsc,L)\)至少含有不在同一条直线上的三点,
则\(P'\)只有一个可能的作法.
\end{theorem}

\begin{theorem}\label{theorem:欧氏几何.定理29}
设\((A,B,C,\dotsc,L)\)
和\((A',B',C',\dotsc,L')\)
是两个合同的图形.
若\(P\)是任意一点,
则恒有一点\(P'\)存在,
使得\((A,B,C,\dotsc,L,P)\)
和\((A',B',C',\dotsc,L',P')\)还是合同的图形.
若\((A,B,C,\dotsc,L)\)至少含有不在同一平面上的四点,
则\(P'\)只有一个可能的作法.
\end{theorem}

\cref{theorem:欧氏几何.定理29} 说出了所有关于合同的空间事实,
因此,空间中运动的性质,都是上述的直线的和平面的五条合同公理(结合着第一组和第二组公理)的推论.
