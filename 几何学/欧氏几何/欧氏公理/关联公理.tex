\section{第一组公理:关联公理}
本组公理是在前面提到的点、直线和平面这三类几何元素之间建立联系,其条文如下.
\begin{axiom}[关联公理]\label{axiom:欧氏几何.关联公理}
点、直线和平面这三类几何元素存在如下的关系:
\begin{enumerate}
\item 对于两点\footnote{%
在本章中,当提到“两点”“两条直线”等时,都是指两个相异的几何元素.%
}\(A\)和\(B\),
恒有一直线\(l\),
它同\(A\)和\(B\)这两点的每一点都相关\footnote{%
同一种关系可能存在多种说法,
例如“直线\(l\)同\(A\)和\(B\)这两点的每一点都相关”
可以说成是“直线\(l\)通过点\(A\)、点\(B\)”
或“直线\(l\)连结点\(A\)和点\(B\)”,
而“点\(A\)与直线\(l\)相关”可以说成是“点\(A\)在直线\(l\)上”
“点\(A\)是直线\(l\)(上)的一点”或“直线\(l\)含有点\(A\)”.%
“点\(P\)既在直线\(a\)上,
又在直线\(b\)上”可以说成是“点\(P\)是直线\(a\)和直线\(b\)的\DefineConcept{交点}或\DefineConcept{公共点}”
或“直线\(a\)、\(b\)相交于点\(P\)”.
}.

\item 对于两点\(A\)和\(B\),
至多有一直线\footnote{%
除了可以用某个小写拉丁字母表示直线以外,与点\(A\)、\(B\)相关的直线还可以记作\(AB\).%
},它同\(A\)和\(B\)这两点的每一点都相关.

\item 一直线上恒至少有两点;至少有三点不在同一直线上.

\item 对于不在同一直线上的任意三点\(A\)、\(B\)和\(C\),恒有一平面\(\gamma\),它同\(A\)、\(B\)和\(C\)这三点的每一点相关;对于任一平面,恒有一点同这平面相关\footnote{%
“点\(A\)与平面\(\gamma\)相关”可以说成是“点\(A\)在\(\gamma\)上”或“点\(A\)是\(\gamma\)的点”.%
}.

\item 对于不在同一直线上的三点\(A\)、\(B\)和\(C\),至多有一平面\footnote{%
除了可以用某个小写希腊字母表示平面以外,由点\(A\)、\(B\)和\(C\)确定的平面还可以记作\(ABC\).%
},它同\(A\)、\(B\)和\(C\)这三点的每一点相关.

\item 若一直线\(l\)的两点\(A\)和\(B\)在一平面\(\gamma\)上,则\(l\)的每一点都在平面\(\gamma\)上\footnote{%
或者说“直线\(l\)在平面\(\gamma\)上”.%
}.

\item 若两平面\(\alpha\)和\(\beta\)有一个公共点\(A\),则它们至少还有一个(与\(A\)相异的)公共点\(B\)\footnote{%
这表明空间的维数不大于3.%
}.

\item 至少有四点不在同一平面上\footnote{%
这表明空间的维数不小于3.%
}.
\end{enumerate}
\end{axiom}
\cref{axiom:欧氏几何.关联公理}
的前3个命题可以统称为{\bf 平面公理},
后5个命题可以统称为{\bf 空间公理}.

依据\cref{axiom:欧氏几何.关联公理} 可以推证出以下两条定理.
\begin{theorem}\label{theorem:欧氏几何.定理1}
一平面上的两直线或有一公共点,或无公共点;
两平面或无公共点,或有一公共直线;
两平面无公共直线时无公共点;
一平面和不在其上的一直线或无公共点,或有一公共点.
\end{theorem}

\begin{theorem}\label{theorem:欧氏几何.定理2}
过一直线和不在这直线上的一点,或过有公共点的两条不同直线,恒有一个而且只有一个平面.
\end{theorem}
