\section{第四组公理:平行公理}
设\(\alpha\)是任一平面,
\(a\)是\(\alpha\)上的任一直线,
而且\(A\)是\(\alpha\)上的、但不在\(a\)上的一点,
在\(\alpha\)上作一直线\(c\),
通过\(A\)且和\(a\)相交,
再在\(\alpha\)上作一直线\(b\),通过\(A\),
且使得\(c\)交\(a\)和\(b\)于相等的同位角.
从\hyperref[theorem:欧氏几何.定理22]{外角定理},
易知\(a\)和\(b\)这两直线无公共点,这就是说,
在一平面\(\alpha\)上,而且通过一直线\(a\)外的一点\(A\),
恒有一直线不和\(a\)相交.

现在可将平行公理叙述如下:
\begin{axiom}[平行公理、欧几里得公理]\label{axiom:欧氏几何.平行公理}
设\(a\)是任一直线,\(A\)是\(a\)外的任一点.
在\(a\)和\(A\)所决定的平面上,
至多有一条直线通过\(A\),而且不和\(a\)相交.
\end{axiom}

根据上文和平行公理,我们知道:
在\(a\)和\(A\)所决定的平面上,恰有一直线,通过\(A\)且不和\(a\)相交,
我们把这条直线叫做“通过\(A\)的\(a\)的\DefineConcept{平行直线}”.

平行公理和下述的要求等价:
如果一平面上的\(a\)和\(b\)两直线都不和这平面上的第三条直线\(c\)相交,
那么\(a\)和\(b\)也不相交.

事实上,如果\(a\)和\(b\)有一公共点\(A\),那么在同一平面上,
就有了\(a\)和\(b\)这两条直线,都通过\(A\)而且不和\(c\)相交.
这和平行公理矛盾.
反之,从上述要求,也易推得平行公理.

平行公理是一条平面公理.
它的引入,使得几何的基础大大地简单化了,也使得几何的构造容易得多了.

例如,在合同公理之外,再加上平行公理,不难得到下列熟知的事实:
\begin{theorem}\label{theorem:欧氏几何.定理30}
若两平行直线被第三条直线所截,则同位角合同,内错角也合同;
反之,若同位角合同,或内错角合同,则前两直线平行.
\end{theorem}

\begin{theorem}\label{theorem:欧氏几何.定理31}
三角形的三个内角的和等于两个直角的和.
\end{theorem}
\begin{figure}[htb]
	\centering
	\begin{tikzpicture}
		\coordinate (A) at (0,0);
		\coordinate (B) at (4,0);
		\coordinate (C) at (3,2);
		\coordinate (P) at (2,2);
		\coordinate (Q) at (4,2);
		\draw (A)node[left]{\(A\)}
				-- (B)node[right]{\(B\)}
				-- (C)node[above]{\(C\)} -- (A);
		\draw (P)node[left]{\(P\)} -- (Q)node[right]{\(Q\)};
		\draw pic[draw=blue,angle radius=5mm]{angle=B--A--C}
				pic[draw=blue,angle radius=6mm]{angle=B--A--C}
				pic[draw=blue,angle radius=5mm]{angle=P--C--A}
				pic[draw=blue,angle radius=6mm]{angle=P--C--A}
				pic[draw=orange,angle radius=4mm]{angle=C--B--A}
				pic[draw=orange,angle radius=4mm]{angle=B--C--Q};
	\end{tikzpicture}
	\caption{}
	\label{figure:欧氏几何.三角形内角和等于平角}
\end{figure}

\begin{definition}
设\(M\)是一平面\(\alpha\)上的任一点.
考虑\(\alpha\)上的所有的那些点\(A\),
它们使线段\(MA\)都互相合同的.
这种点\(A\)的全体叫做一个\DefineConcept{圆},
点\(M\)叫做“这个圆的\DefineConcept{中心}”,简称\DefineConcept{圆心}.
\end{definition}

根据这个定义,我们容易从第三组和第四组公理,推证关于圆的若干熟知的定理.
特别是下述的定理:
通过不在同一条直线上的三点,能作一圆;
关于同一条弦上的圆周角合同的定理;
关于内接于圆的一个四边形的角的定理.
