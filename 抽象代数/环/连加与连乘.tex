\section{连加与连乘}
% 这个小节都是我自己定义的概念,暂未找到其他书上有类似说法
现在我们来尝试重新定义连加号与连乘号.

\begin{definition}
设\(X\)是非空集合,
\(I\in\omega\),
序列\(f\colon I \to X\),
序列\(g\colon I \to X\),
\(+\)是定义在\(X\)上的一个二元代数运算.
如果对于\(\forall i\in I\)有\begin{equation*}
	g(i) = \left\{ \begin{array}{lc}
		f(0), & i=0, \\
		g(i-1) + f(i), & 0<i<I,
	\end{array} \right.
\end{equation*}
则称“\(g\)是\(f\)从\(0\)到\(I\)对\(+\)运算的\DefineConcept{累积}”,
记\begin{equation*}
	\Accumulate_{0 \leq i < I}^+ f(i)
	\defeq
	g(I-1).
\end{equation*}
\end{definition}

我们可以进一步扩展上述定义:
\begin{definition}
设\(X\)是非空集合,
\(I,J\in\omega\)且\(J<I\),
序列\(f\colon I \to X\),
序列\(g\colon I \to X\),
\(+\)是定义在\(X\)上的一个二元代数运算,
\(o\)是\(\opair{X,+}\)的单位元.
如果对于\(\forall i\in I\)有\begin{equation*}
	g(i) = \left\{ \begin{array}{lc}
		o, & i<J, \\
		f(J), & i=J, \\
		g(i-1) + f(i), & J<i<I,
	\end{array} \right.
\end{equation*}
则称“\(g\)是\(f\)从\(J\)到\(I\)对\(+\)运算的\DefineConcept{累积}”,
记\begin{equation*}
	\Accumulate_{J \leq i < I}^+ f(i)
	\defeq
	g(I-1).
\end{equation*}
\end{definition}

\begin{definition}
设\(X\)是数域,
\(\{x_n\}\)是在\(X\)内的一个数列.
定义:\begin{equation*}
	\sum_{i=s}^t x_i
	\defeq
	\Accumulate_{s \leq i < t+1}^+ x_i,
\end{equation*}\begin{equation*}
	\prod_{i=s}^t x_i
	\defeq
	\Accumulate_{s \leq i < t+1}^\times x_i.
\end{equation*}
\end{definition}
