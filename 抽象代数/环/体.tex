\section{体}
\begin{definition}
设\(\opair{R,+,\times}\)是有单位元的环,
\(e\)是它的单位元,\(o\)是它的零元,
\(e \neq o\).
如果\(R\)中每个非零元都是可逆元,即\[
	(\forall a \in R-\{o\})(\exists b \in R)[a \times b = b \times a = e],
\]
那么称“\(\opair{R,+,\times}\)是一个\DefineConcept{除环}(division ring)
或\DefineConcept{体}(skew field)”.
\end{definition}

\(\opair{\mathbb{Z},+,\times}\)不是除环.

%TODO proof 证明:四元数\(\opair{\mathbb{Z},+,\times}\)成环.
\begin{example}
%@see: 《离散数学》(邓辉文) P151 例5-22
证明:四元数环\(\opair{\mathbb{H},+,\times}\)是除环.
\begin{proof}
对于任意\(a+b\iu+c\ju+d\ku\in\mathbb{H}-0\),
其乘法逆元为\[
	(a+b\iu+c\ju+d\ku)^{-1}
	= \frac1{a^2+b^2+c^2+d^2} (a-b\iu-c\ju-d\ku),
\]
因此\(\opair{\mathbb{H},+,\times}\)是除环.
\end{proof}
\end{example}
%TODO \(\opair{\mathbb{H},\times}\)不是交换群.

\begin{table}[hbt]
	\centering
	\begin{tabular}{r|*3r}
		\(\times\) & \(\iu\) & \(\ju\) & \(\ku\) \\ \hline
		\(\iu\) & \(-1\) & \(\ku\) & \(-\ju\) \\
		\(\ju\) & \(-\ku\) & \(-1\) & \(\iu\) \\
		\(\ku\) & \(\ju\) & \(-\iu\) & \(-1\) \\
	\end{tabular}
	\caption{四元数的虚数单位的乘法运算}
\end{table}

\begin{definition}
%@see: 《高等代数(第三版 下册)》(丘维声) P69 定义1
设\(\opair{F,+,\times}\)是一个除环,
如果\(\times\)满足交换律,
则称“\(\opair{F,+,\times}\)是一个\DefineConcept{域}(field)”.
% 如果\(F\)是一个有单位元\(1(\neq0)\)的交换环,
% 并且\(F\)中每个非零元都是可逆元,
% 那么称\(F\)是一个域.
\end{definition}

有理数环\(\mathbb{Q}\)、实数环\(\mathbb{R}\)和复数环\(\mathbb{C}\)都是域.

有理复数集\(\Set{ a+b\iu \given a,b\in\mathbb{Q} }\)也成为域,
我们特别称其为\DefineConcept{高斯数域}.

只含有限多个元素的域,称为\DefineConcept{有限域};
否则,称为\DefineConcept{无限域}.
