\section{除环}
\begin{definition}
设\(\opair{R,+,\times}\)是有单位元的环,
\(e\)是它的单位元,\(o\)是它的零元,
\(e \neq o\).
如果\(R\)中每个非零元都是可逆元,即\begin{equation*}
	(\forall a \in R-\{o\})(\exists b \in R)[a \times b = b \times a = e],
\end{equation*}
那么称“\(\opair{R,+,\times}\)是一个\DefineConcept{除环}(division ring)
或\DefineConcept{体}(skew field)”.
\end{definition}

\begin{theorem}
%@see: 《高等几何(第四版)》(梅向明,刘增贤,王汇淳,王智秋) P111
设\(R\)是一个非空集合,
\(+,\times\)是定义在\(R\)上的两个二元代数运算.
如果\begin{itemize}
	\item \(R\)对\(+\)成为一个交换群,
	\item \(o\)是\(\opair{R,+}\)的单位元,
	\item \(R-\{o\}\)对\(\times\)成为一个群,
	\item \(\times\)满足对\(+\)的左、右分配律,即\begin{equation*}
		(x + y) \times z = x \times z + y \times z,
		\qquad
		x \times (y + z) = x \times y + x \times z,
	\end{equation*}
\end{itemize}
则\(\opair{R,+,\times}\)是一个体.
\end{theorem}

\begin{example}
\(\opair{\mathbb{Z},+,\times}\)不是除环.
\end{example}

%TODO proof 证明:四元数\(\opair{\mathbb{Z},+,\times}\)成环.
\begin{example}
%@see: 《离散数学》(邓辉文) P151 例5-22
证明:四元数环\(\opair{\mathbb{H},+,\times}\)是除环.
\begin{table}[hbt]
%@see: 《离散数学》(邓辉文) P151 表5-7
	\centering
	\begin{tabular}{r|*3r}
		\(\times\) & \(\iu\) & \(\ju\) & \(\ku\) \\ \hline
		\(\iu\) & \(-1\) & \(\ku\) & \(-\ju\) \\
		\(\ju\) & \(-\ku\) & \(-1\) & \(\iu\) \\
		\(\ku\) & \(\ju\) & \(-\iu\) & \(-1\) \\
	\end{tabular}
	\caption{四元数的虚数单位的乘法运算}
\end{table}
\begin{proof}
对于任意\(a+b\iu+c\ju+d\ku\in\mathbb{H}-0\),
其乘法逆元为\begin{equation*}
	(a+b\iu+c\ju+d\ku)^{-1}
	= \frac1{a^2+b^2+c^2+d^2} (a-b\iu-c\ju-d\ku),
\end{equation*}
因此\(\opair{\mathbb{H},+,\times}\)是除环.
\end{proof}
\end{example}

%TODO \(\opair{\mathbb{H},\times}\)不是交换群.
