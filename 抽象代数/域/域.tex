\section{域}
\subsection{域}
\begin{definition}
%@see: 《高等代数(第三版 下册)》(丘维声) P69 定义1
设\(\opair{F,+,\times}\)是一个除环,
如果\(\times\)满足交换律,
则称“\(\opair{F,+,\times}\)是一个\DefineConcept{域}(field)”.
\end{definition}

\begin{theorem}
\(F\)是一个域,
当且仅当
\(F\)是一个有单位元\(1(\neq0)\)的交换环,
且\(F\)中每个非零元都是可逆元.
%TODO proof
\end{theorem}

\begin{example}
整数环\(\mathbb{Z}\)不是域.
\end{example}

\begin{example}
有理数环\(\mathbb{Q}\)是域.
\end{example}

\begin{example}
实数环\(\mathbb{R}\)是域.
\end{example}

\begin{example}
复数环\(\mathbb{C}\)是域.
\end{example}

\begin{example}
集合\(
	\mathbb{Q}(\sqrt{2})
	\defeq
	\Set{ a + b \sqrt{2} \given a,b \in \mathbb{Q} }
\)
对于实数的加法、乘法成为一个域.
\end{example}

\begin{example}
集合\(
	\mathbb{Q}(\iu)
	\defeq
	\Set{ a + b \iu \given a,b \in \mathbb{Q} }
\)
对于复数的加法、乘法成为一个域.
我们特别称其为\DefineConcept{高斯数域}.
\end{example}

\subsection{有限域}
\begin{definition}
设\(F\)是一个域.
\begin{itemize}
	\item 如果\(F\)是有限集,则称“\(F\)是一个\DefineConcept{有限域}”;
	\item 如果\(F\)是无限集,则称“\(F\)是一个\DefineConcept{无限域}”.
\end{itemize}
\end{definition}

%\begin{theorem}
%有理数域是最小的数域,即任意数域都包含有理数域.
%\begin{proof}
%设\(K\)是一个数域,则\(0,1 \in K\),从而
%\begin{equation*}
%    2 = 1 + 1 \in K,
%    3 = 2 + 1 \in K,
%    \dotsc,
%    n = (n-1) + 1 \in K.
%\end{equation*}
%这就是说,任一正整数\(n \in K\).
%又由于\(-n = 0 - n \in K\),
%因此任一负整数\(-n \in K\).
%由上可知,\(\mathbb{Z} \subseteq K\).
%于是,任一分数\begin{equation*}
%\frac{a}{b} \in K \quad(a,b\in\mathbb{Z} \land b\neq0).
%\end{equation*}
%也就是说,\(\mathbb{Q} \subseteq K\).
%\end{proof}
%\end{theorem}
%
%\begin{theorem}
%复数域是最大的数域,即任意数域都包含于复数域.
%\end{theorem}

\begin{example}
%@see: 《高等代数(第三版 下册)》(丘维声) P71 习题7.11 1.
证明:域\(F\)中没有非平凡的零因子,域一定是整环.
%TODO proof
\end{example}

\subsection{有序域}
\begin{definition}
%@see: 《复变函数》(史济怀、刘太顺) P2 定义1.1.1
设\(F\)是一个域,\(<\)是\(F\)上的一个关系.
如果\begin{enumerate}
	\item 关系\(<\)服从三一律;
	\item 关系\(<\)具有传递性;
	\item 对于\(F\)中任意两个元素\(a,b\),只要\(a<b\),那么对于\(F\)中任意一个元素\(c\),有\(a+c<b+c\);
	\item 对于\(F\)中任意三个元素\(a,b,c\),只要\(a<b\)且\(c>0\),那么\(ac<bc\),
\end{enumerate}
则称“\(<\)是\(F\)上的一个\DefineConcept{序关系}”.
\end{definition}

\begin{definition}
%@see: 《复变函数》(史济怀、刘太顺) P2 定义1.1.1
设\(F\)是一个域.
如果存在\(F\)上的一个关系\(<\),
使得\(<\)是\(F\)上的一个序关系,
则称“\(F\)是一个\DefineConcept{有序域}”;
否则称“\(F\)是一个\DefineConcept{无序域}”.
\end{definition}

\begin{theorem}
有理数域\(\mathbb{Q}\)是有序域.
\end{theorem}

\begin{theorem}
实数域\(\mathbb{R}\)是有序域.
\end{theorem}

\begin{theorem}
%@see: 《复变函数》(史济怀、刘太顺) P2 定理1.1.2
复数域\(\mathbb{C}\)是无序域.
\begin{proof}
用反证法.
假设\(\mathbb{C}\)是有序域,
那么因为\(\iu\neq0\),
所以由三一律可知,
要么成立\(\iu>0\),要么成立\(\iu<0\).

假设\(\iu>0\),
那么\(\iu\cdot\iu>\iu\cdot0\),
即\(-1>0\),
从而\(-1+1>0+1\),
即\(0>1\);
另一方面,
由\(-1>0\)
可得\((-1)\cdot(-1)>0\cdot(-1)\),
即\(1>0\),
这与刚刚得到的\(0>1\)矛盾!

同理可知,由\(\iu<0\)也可推出矛盾.

因此\(\mathbb{C}\)是无序域.
\end{proof}
\end{theorem}

\subsection{子域,扩域}
\begin{definition}
设\(\opair{F,+,\times}\)是域,
\(S\)是\(F\)的一个非空子集.
如果\(\opair{S,+,\times}\)成域,
那么称“\(S\)是\(F\)的一个\DefineConcept{子域}(subfield)”
“\(F\)是\(S\)的一个\DefineConcept{扩域}(extension field)”,
记作\(\opair{S,+,\times}\AlgebraSubstructure\opair{F,+,\times}\).
%@see: https://mathworld.wolfram.com/Subfield.html
%@see: https://mathworld.wolfram.com/ExtensionField.html
\end{definition}

\begin{definition}
设\(F\)是一个域.
把集合\begin{equation*}
	F(\sqrt{1+\lambda^2})
	\defeq
	\Set{
		\sqrt{1+\lambda^2}
		\given
		\lambda \in F
	}
\end{equation*}
称为“域\(F\)的\DefineConcept{毕达哥拉斯扩张}(Pythagorean Extension)”.
%@see: https://mathworld.wolfram.com/PythagoreanExtension.html
\end{definition}

\begin{definition}
设\(F\)是一个域.
如果\(F\)的毕达哥拉斯扩张恰好就是\(F\)本身,
则称“\(F\)是一个\DefineConcept{毕达哥拉斯域}(Pythagorean field)”.
%@see: https://mathworld.wolfram.com/PythagoreanField.html
\end{definition}

\begin{theorem}
%@see: 《高等几何(第四版)》(梅向明,刘增贤,王汇淳,王智秋) P112
设\(F\)是一个域.
如果\begin{equation*}
	(\forall x,y \in F)
	(\exists h \in F)
	[
		h \times h
		= x \times x + y \times y
	],
\end{equation*}
则\(F\)是一个毕达哥拉斯域.
\end{theorem}

\begin{definition}
%@see: 《高等几何(第四版)》(梅向明,刘增贤,王汇淳,王智秋) P112
设\(F\)是一个有序的毕达哥拉斯域,
记\(
	F^+
	\defeq
	\Set{
		x \in F
		\given
		x > 0
	}
\).
如果\begin{equation*}
	(\forall x \in F^+)
	(\exists y \in F)
	[
		y \times y = x
	],
\end{equation*}
则称“\(F\)是一个\DefineConcept{二次根域}”.
\end{definition}
