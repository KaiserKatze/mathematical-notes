\section{子群}
\subsection{子群}
在介绍子群的概念之前,让我们首先回顾\hyperref[definition:集合论.二元代数运算]{二元代数运算的定义}:
任意一个定义在集合\(A\)上的运算,本质上是一个从\(A \times A\)到\(A\)的映射.
现在我们来讨论更一般的情形,
假设我们有一个映射\(f\colon A \times A \to B\),其中\(B \supseteq A\),
如果\[
	(\forall x,y \in A)[f(x,y) \in A],
\]
那么实际上这个映射的值域为\(\ran f = A\);
在这种情况下,我们就说“集合\(A\)对运算\(f\) \DefineConcept{封闭}%
(set \(A\) is \emph{closed} under operation \(f\))”.
回过头来,我们可以看出,二元代数运算的定义保证了它的封闭性.
但是,假如我们从\(A\)中取出一个非空子集\(C\),
我们能否断言\(f\)在直积\(C \times C\)上的限制具有封闭性呢?

\begin{lemma}\label{theorem:抽象代数.群的运算在子集上的限制的不变性}
设\(\times\)是定义在\(G\)上的一个二元代数运算,
\(\emptyset \neq H \subseteq G\),
\(\otimes = (\times \setrestrict(H \times H))\),
则\[
	(\forall a,b \in H)[a \otimes b = a \times b].
\]
\begin{proof}
根据限制的定义可知\[
	(\forall a,b,c)[
		\opair{\opair{a,b},c} \in \otimes
		\iff
		\opair{\opair{a,b},c} \in \times
		\land
		\opair{a,b} \in H \times H
	],
\]
于是\((\forall a,b \in H)[a \otimes b = a \times b]\).
\end{proof}
\end{lemma}
从\cref{theorem:抽象代数.群的运算在子集上的限制的不变性} 我们可以看出,
运算\(\times\)的限制\(\otimes\)在新的定义域\(H \times H\)上仍然具有和原本的运算\(\times\)一样的性质.
如果\(\times\)在\(G \times G\)上服从结合律,那么\(\otimes\)在\(H \times H\)上也服从结合律.
如果\(\times\)在\(G \times G\)上服从交换律,那么\(\otimes\)在\(H \times H\)上也服从交换律.

但是应该注意到,映射的限制只是限定了它的定义域,并没有限定它的值域,
因此我们可能会遇到值\(c\)落在\(H\)之外的情形,
也就是说\(a \otimes b \in(G - H)\)是可能的,
\(a \otimes b \in H\)不一定成立.

\begin{definition}\label{definition:抽象代数.子群的定义}
%@see: 《近世代数》(丘维声,2015) P40 定义1
%@see: https://mathworld.wolfram.com/Subgroup.html
设\(\opair{G,\times}\)是群,
\(H\)是\(G\)的一个非空子集.
又设映射\(\otimes\)是\(\times\)在直积\(H \times H\)上的限制,
即\(\otimes = (\times \setrestrict(H \times H))\).
如果\(H\)对于运算\(\otimes\)也成为一个群\(\opair{H,\otimes}\),
那么称“\(\opair{H,\otimes}\)是\(\opair{G,\times}\)的一个\DefineConcept{子群}(subgroup)”,
记作\(\opair{H,\otimes} \subseteq \opair{G,\times}\).
\end{definition}

鉴于\cref{theorem:抽象代数.群的运算在子集上的限制的不变性}
指出对运算的定义域的限制不会改变运算的性质,
于是,当我们不用特别强调\(\otimes\)是\(\times\)在\(H\times H\)上的限制时,
我们可以把\cref{definition:抽象代数.子群的定义} 中
子群\(\opair{H,\otimes}\)的符号改写为\(\opair{H,\times}\).

现在我们来研究一个子集成为子群的充分必要条件.

由于根据\hyperref[definition:抽象代数.群的定义]{群的定义},
任意一个群一定有单位元,
所以任给一个群的子集,只要它不含单位元,它就一定不对这个群的运算成群.

一方面,仅由群\(\opair{G,\times}\)的单位元\(e\)%
组成的集合\(E=\{e\}\)对于运算\[
	\odot=(\times\setrestrict(E\times E))
	= \Set{\opair{\opair{e,e},e}}
\]
也成为一个群\(\opair{E,\odot}\),
那么它就是\(\opair{G,\times}\)的一个子群.
在这个群中,虽然只有一个元素\(e\),但是它却满足\[
	(\forall a \in E)[a \times e = e \times a = e],
\]
于是\(e\)是\(\opair{E,\odot}\)的单位元.
因此,我们可以说群\(\opair{G,\times}\)和子群\(\opair{E,\odot}\)的单位元相同.

另一方面,\(\opair{G,\times}\)本身也是\(\opair{G,\times}\)的一个子群,
即\(\opair{G,\times}\subseteq\opair{G,\times}\);
同样地,群\(\opair{G,\times}\)和子群\(\opair{G,\times}\)的单位元相同.

我们不禁好奇,任意给定一个群,再从中任取一个子群,
是不是这个群的单位元与它的子群的单位元总是相同的?

\begin{proposition}\label{theorem:抽象代数.子群.群的单位元与其子群的单位元相同}
设\(\opair{G,\times}\)是群,
\(\opair{H,\otimes}\subseteq\opair{G,\times}\).
那么\(\opair{G,\times}\)的单位元与\(\opair{H,\otimes}\)的单位元相同.
\begin{proof}
设\(e'\)是\(\opair{H,\otimes}\)的单位元,
则\[
	e' \otimes e' = e'.
\]
根据子群的定义,\(\otimes = (\times \setrestrict(H \times H))\);
再根据\cref{theorem:抽象代数.群的运算在子集上的限制的不变性},
\((\forall a,b \in H)[a \otimes b = a \times b]\);
于是\[
	e' \times e' = e'.
\]
设\(e\)是\(\opair{G,\times}\)的单位元.
假设\(e'\)在\(G\)中的逆元是\((e')^{-1}\),
即\(e' \times (e')^{-1} = e\),
那么在上式等号两边右乘\((e')^{-1}\),得\[
	(e' \times e') \times (e')^{-1} = (e') \times (e')^{-1};
\]
利用\(\times\)结合律可得\(e' \times e = e\);
又由于\(e' \times e = e'\),因此\(e = e'\);
这就说明,\(e\)是\(\opair{H,\otimes}\)的单位元,
\(\opair{G,\times}\)的单位元与\(\opair{H,\otimes}\)的单位元相同.
\end{proof}
\end{proposition}

\begin{definition}
设\(\opair{G,\times}\)是群,
\(e\)是\(\opair{G,\times}\)的单位元.
我们将\(\opair{\{e\},\times}\)和\(\opair{G,\times}\)
统称为“\(\opair{G,\times}\)的\DefineConcept{平凡子群}(trivial subgroup)”.
\end{definition}

\begin{definition}
设\(\opair{H,\times}\)是群\(\opair{G,\times}\)的子群.
如果\(H \subset G\),
那么称“\(\opair{H,\times}\)是\(\opair{G,\times}\)的\DefineConcept{真子群}(proper subgroup)”,
记作\(\opair{H,\times}\subset\opair{G,\times}\).
\end{definition}

从\cref{definition:抽象代数.子群的定义} 看出,
如果\(\opair{H,\otimes}\)是群\(\opair{G,\times}\)的一个子群,
那么根据群的定义以及二元代数运算的定义可知,
\(H\)对\(\otimes\)封闭,即\[
	(\forall a,b \in H)[a \otimes b \in H].
\]
由\cref{theorem:抽象代数.子群.群的单位元与其子群的单位元相同} 可知,
\(\opair{G,\times}\)的单位元与\(\opair{H,\times}\)的单位元相同,
不妨设它们的单位元为\(e\).
任给\(b \in H\),假设\(b\)在\(H\)中的逆元为\(c\),
则\[
	b \times c = c \times b = e;
\]
又假设\(b\)在\(G\)中的逆元为\(b^{-1}\),
则\[
	b \times b^{-1} = b^{-1} \times b = e;
\]
那么\[
	c = c \times e
	= c \times (b \times b^{-1})
	= (c \times b) \times b^{-1}
	= e \times b^{-1}
	= b^{-1};
\]
因此\(b^{-1} \in H\).
综上所述,如果\(\opair{H,\otimes}\)是群\(\opair{G,\times}\)的一个子群,
那么\[
	a,b \in H
	\implies
	a \times b^{-1} \in H.
\]
下面我们来证明,\([a,b \in H \implies a \times b^{-1} \in H]\)
是“\(\opair{H,\otimes}\)是\(\opair{G,\times}\)的子群”的充分条件.
\begin{theorem}
%@see: 《近世代数》(熊全淹) P41 命题1
设\(\opair{G,\times}\)是群,
\(H\)是\(G\)的一个非空子集,
\(\otimes = (\times \setrestrict(H \times H))\).
那么“\(\opair{H,\otimes}\)是\(\opair{G,\times}\)的子群”的充分必要条件是:\[
	a,b \in H \implies a \times b^{-1} \in H,
\]
其中\(b^{-1}\)是\(b\)在\(G\)中的逆元.
\begin{proof}
我们已经证得必要性,接下来只需证明充分性.
由于\(H\)不是空集,因此存在\(c \in H\).
假设\(e\)是\(\opair{G,\times}\)的单位元,
\(c^{-1}\)是\(c\)在\(G\)中的逆元,
由已知条件有\[
	c \times c^{-1} \in H;
\]
又因为\(c \times c^{-1} = e\),
于是有\(e \in H\).
再次利用已知条件可得\(e \times c^{-1} = c^{-1} \in H\),
这就是说\(H\)中任一元在\(G\)中的逆元实际上也在\(H\)中.

任给\(a,b \in H\),
由上面已证明的结论可知,\(b^{-1} \in H\);
再由已知条件可知,\[
	a \times (b^{-1})^{-1} \in H;
\]
考虑到\((b^{-1})^{-1} = b\),
于是\[
	a \times b = a \otimes b \in H,
\]
这就是说\(H\)对\(\otimes\)封闭,
或者说\(\otimes\)可以看成是定义在\(H\)上的一个二元代数运算.
由于运算\(\times\)服从结合律,
那么作为\(\times\)的限制,
根据\cref{theorem:抽象代数.群的运算在子集上的限制的不变性},
运算\(\otimes\)也服从结合律.
上面已证\(\opair{G,\times}\)的单位元\(e \in H\),
再次利用\cref{theorem:抽象代数.群的运算在子集上的限制的不变性},
就有\[
	(\forall a \in H)[
		a \otimes e = a \times e = a = e \times a = e \otimes a
	],
\]
这就说明\(e\)是\(\opair{H,\otimes}\)的单位元.
由上面已证明的结论可知,\(H\)中任一元\(b\)有逆元\(b^{-1}\).
综上所述,\(H\)是一个群,从而\(H\)是\(G\)的子群.
\end{proof}
\end{theorem}

%\subsection{循环群}
%\begin{definition}
%设\(\opair{G,\times}\)是群,\(H \subseteq G\).
%如果\[
%	(\forall \gamma \in G)
%	(\exists I \subseteq G)
%	[
%		(\gamma=\textstyle\prod_{a \in I} a)
%		\land
%		(\forall a \in I)[a \in H \lor a^{-1} \in H]
%	],
%\]
%那么称“\(H\)生成\(G\)
%(\(H\) generates \(G\))”;
%把\(\opair{G,\times}\)称为\DefineConcept{循环群}(cyclic group);
%把\(H\)的元素称为\(G\)的\DefineConcept{生成元}(generator).
%\end{definition}
%
%
%例如,给定元素\(a\),
%构造集合\(G = \Set{
%	\opair{a,n}
%	\given
%	n\in\mathbb{Z}
%}\),
%定义运算:\[
%	\otimes\colon G \times G \to G,
%	\opair{a,m}\otimes\opair{a,n}\mapsto\opair{a,m+n},
%\]
%并规定\(\opair{a,-n}=\opair{\opair{a,-1},n}\).
%我们已经熟知了整数集\(\mathbb{Z}\)上的加法运算,
%这里可以看出,运算\(\otimes\)与整数的加法一样,也服从结合律;
%\(\opair{a,0}\)是\(G\)的单位元;
%最后对于任意整数\(n\),\(\opair{a,n}\)与\(\opair{a,-n}\)互为逆元;
%由此可知\(G\)对\(\otimes\)成群.
%于是\(G\)是由\(a\)生成的循环群,\(a\)是\(G\)的生成元.
%
%再例如,整数加法\(\opair{\mathbb{Z},+}\)本身就是一个循环群,
%同时它也是唯一一个“无限”循环群.
