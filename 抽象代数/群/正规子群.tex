\section{正规子群}
\subsection{群子集乘积}
我们有时候会研究这样的问题:
从一个群内的不同子集各取一个元素,
观察这两个元素的运算结果具备一些什么性质.

我们以整数乘群\(\opair{\mathbb{Z},\times}\)为例.
取\(H\)为正整数集,\(K\)为负整数集,即\[
	H = \Set{ h \in \mathbb{Z} \given h > 0 }, \qquad
	K = \Set{ k \in \mathbb{Z} \given k < 0 },
\]
由于\[
	(\forall h \in H)(\forall k \in K)[h \times k < 0],
\]
所以我们从\(H\)和\(K\)中各取一个元素,算得的乘积组成的集合恰好就是\(K\),
这就是我们常说的“正负得负”.

\begin{definition}\label{definition:抽象代数.群子集乘积.群子集乘积的定义}
设\(\opair{G,*}\)是群,
\(H \subseteq G,
K \subseteq G\).
我们把集合\[
	\Set{ h * k \given h \in H \land k \in K }
\]
称为“\(H\)与\(K\)的\DefineConcept{乘积}”,
记作\(H * K\).
\end{definition}
必须要说明的是,这里我们没有像之前一样用\(\opair{G,\times}\)表示群,
而是用\(\opair{G,*}\)表示群,
是为了避免把\(H\)与\(K\)的乘积表示成\(H \times K\),
因此这会与集合的笛卡尔积产生符号冲突,而我们总是希望能够尽可能规避符号冲突.

\begin{proposition}
设\(\opair{G,*}\)是群,
\(H \subseteq G,
K \subseteq G\).
那么\(H * K \subseteq G\).
\begin{proof}
根据\hyperref[definition:抽象代数.群子集乘积.群子集乘积的定义]{群子集乘积的定义},有\[
	(\forall h)
	(\forall k)
	[h \in H \land k \in K \iff h * k \in H * K].
	\eqno(1)
\]
再根据\hyperref[definition:集合论.子集的定义]{子集的定义},有\[
	(\forall h)
	[H \subseteq G \iff (h \in H \implies h \in G)],
	\qquad
	(\forall k)
	[K \subseteq G \iff (k \in K \implies k \in G)],
\]
所以\[
	(\forall h)
	(\forall k)
	[h \in H \land k \in K \implies h,k \in G \implies h * k \in G].
	\eqno(2)
\]
于是由(1)(2)两式可知\(a \in h * k \implies a \in G\),\(H * K \subseteq G\).
\end{proof}
\end{proposition}

\begin{proposition}\label{theorem:抽象代数.群子集乘积.群子集乘积满足结合律}
群中子集的乘积满足结合律.
\begin{proof}
设\(\opair{G,*}\)是群,
\(H \subseteq G,
K \subseteq G,
L \subseteq G\).
因为\[
	(\forall h \in H)
	(\forall k \in K)
	(\forall l \in L)
	[h * (k * l) = (h * k) * l],
\]
所以\(H * (K * L) = (H * K) * L\).
\end{proof}
\end{proposition}

\begin{proposition}\label{theorem:抽象代数.群子集乘积.子群与自身的乘积等于自身}
设\(\opair{G,*}\)是群,
\(\opair{H,*} \AlgebraSubstructure \opair{G,*}\).
那么\[
	H * H = H.
\]
\begin{proof}
由于\(\opair{H,*}\AlgebraSubstructure\opair{G,*}\),
所以\(H\)对\(*\)封闭,即\[
	(\forall h,k \in H)
	[h * k \in H];
\]
于是\(H * H \subseteq H\).

设\(e\)是\(\opair{G,*}\)的单位元.
因为\[
	(\forall h \in H)
	[h * e = h],
\]
所以\(H = H * \{e\} \subseteq H * H\).

综上所述,\(H * H = H\).
\end{proof}
\end{proposition}

应该注意到,\cref{theorem:抽象代数.群子集乘积.子群与自身的乘积等于自身} 的逆命题并不成立,
即当\(H * H = H\)时,\(H\)不一定对\(*\)成群.
譬如,在整数加群\(\opair{\mathbb{Z},+}\)中,正整数集\(\mathbb{Z}^+\)对加法\(+\)并不成群.

对于有限子群的乘积我们有如下定理.
\begin{theorem}
%@see: 《近世代数》(熊全淹) P35 定理1
设\(H,K\)是群\(\opair{G,*}\)的有限子群.
那么\[
	\card(H * K)
	= \frac{\card H \cdot \card K}{\card(H \cap K)}.
\]
\begin{proof}
取定\(h_1,h_2 \in H\),\(k_1,k_2 \in K\),
使得\(h_1 * k_1 = h_2 * k_2\),
在等号两边同时左乘\(h_2^{-1}\)并右乘\(k_1^{-1}\),
便有\[
	h_2^{-1} * h_1 = k_2 * k_1^{-1};
\]
令\(d = h_2^{-1} * h_1\),
那么\(d \in H \cap K\).
因此\[
	h_2 = h_1 * d^{-1}, \qquad
	k_2 = d * k_1.
\]

反过来,对于给定的\(h_1 \in H\)和\(k_1 \in K\),
任取\(H \cap K\)中的一个元素\(d\),
设\(h_2 = h_1 * d^{-1},
k_2 = d * k_1\),
就得到\(h_1 * k_1 = h_2 * k_2\),
这就是说,对于任意\(h_1,k_1\),
在\(H * K\)中有\(\card(H \cap K)\)个与\(h_1 * k_1\)相等的元素.
\end{proof}
\end{theorem}

特别地,当\(H \cap K = \{e\}\),即当\(\card(H \cap K) = 1\)时,
\(\card(H * K) = \card H \cdot \card K\).

假设\(H,K\)都是\(\opair{G,*}\)的子群,
\(H\)和\(K\)的乘积\(H * K\)通常不一定成群.
于是我们想要知道,在什么条件下,\(H * K\)也能成群?

\begin{theorem}\label{theorem:群的子集的乘积成群的充分必要条件}
%@see: 《近世代数》(熊全淹) P36 定理2
群\(\opair{G,*}\)的子群\(H,K\)的乘积\(H * K\)成群的充分必要条件是:
\(H\)与\(K\)可交换.
\begin{proof}
首先我们假设\(H * K\)成群,
任取\(h \in H\),任取\(k \in K\),
因为\(K * H\)中的元素\(k * h\)
是\(H * K\)中元素\(h^{-1} * k^{-1}\)的逆元,
所以\(k * h \in H * K\),
因此\(K * H \subseteq H * K\).
哟因为\((h\ otimes k)^{-1} \in H * K\),
于是\(h * k \in K * H\),
所以\(H * K \subseteq K * H\).
综上,\(H * K = K * H\);
也就是说,假如\(H * K\)成群,那么\(H\)与\(K\)可交换.

反过来,假如\(H * K = K * H\),
那么\(H * K\)中任一元\(h * k\)的逆元\(k^{-1} * h^{-1}\)
在\(K * H = H * K\)中,
又因为\[
	H * K * H * K
	= H * H * K * K
	= H * K,
\]
所以\(H * K\)中任意两元的乘积仍然在\(H * K\)中,
于是\(H * K\)成群.
\end{proof}
\end{theorem}

应该注意到,\cref{theorem:群的子集的乘积成群的充分必要条件} 所说的“\(H\)与\(K\)可交换”
指的是\(H * K\)和\(K * H\)这两个集合相等,
即\[
	H * K = K * H,
\]
但是这并不意味着元素的乘积也同样相等,
即\[
	(\exists h \in H)(\exists k \in K)[h * k \neq k * h].
\]

\begin{corollary}
交换群的任意两个子群的乘积仍然是一个子群.
\end{corollary}

我们知道,
\(H * K\)包含在由\(H,K\)生成的群\(\opair{H,K}\)中.
当\(H * K\)成群时,
由\(H,K\)生成的群就是\(H * K\),
即\(\opair{H,K} = H * K\),
也就是说,\(H * K\)是\(\opair{G,*}\)中包含\(H,K\)的最小子群.

假如\(\opair{G,*}\)是加群,
只要\(H,K\)是子群,
自然\(H * K\)也是子群,
这时仍然把\(H * K\)写成\((H,K)\),
叫做“\(H\)和\(K\)的和”.

\subsection{陪集,拉格朗日定理}
\begingroup
\def\RQuotient{G/\kern-2pt\sim}
\def\LQuotient{G/\kern-2pt\backsim}

利用群\(\opair{G,*}\)的子群\(\opair{H,*}\)可以研究群\(\opair{G,*}\)的结构,
这时因为利用子群\(\opair{H,*}\)可以给出集合\(G\)的一个划分.
而为了给出\(G\)的一个划分,需要在\(G\)上建立一个等价关系,于是我们定义:\[
	a \sim b
	\defiff
	a * b^{-1} \in H.
\]
可以验证\(\sim\)是等价关系:
首先,由于\(a * a^{-1} = e \in H\),因此\(a \sim a\);
这就说明\(\sim\)具有自反性.
然后,由\(a \sim b\)可得\(a * b^{-1} \in H\);
那么由\(H \subseteq G\)可知\((a * b^{-1})^{-1} \in H\);
于是\(b * a^{-1} \in H\),从而\(b \sim a\);
这就说明\(\sim\)具有对称性.
最后,由\(a \sim b\)和\(b \sim c\)得\(a * b^{-1} \in H\)和\(b * c^{-1} \in H\);
由于\(H \subseteq G\),因此\((a * b^{-1}) * (b * c^{-1}) \in H\);
于是\(a * c^{-1} \in H\),从而\(a \sim c\);
这就说明\(\sim\)具有传递性.
综上所述,\(\sim\)兼具自反性、对称性、传递性,
所以\(\sim\)是\(G\)上的一个等价关系.

任给\(a \in G\),
它在关系\(\sim\)下的等价类为\begin{align*}
	[a] &= \Set{ x \in G \given x \sim a }
	= \Set{ x \in G \given x * a^{-1} \in H }
	= \Set{ x \in G \given x * a^{-1} = h \land h \in H } \\
	&= \Set{ x \in G \given x = h * a \land h \in H }
	= \Set{ h * a \given h \in H }.
\end{align*}
可以注意到,这里的等价类\([a]\)等于\(H * \{a\}\).
我们称\(H * \{a\}\)为“\(H\)的一个\DefineConcept{右陪集}”;
称\(a\)为它的\DefineConcept{陪集代表}.
于是\(H\)的全体右陪集组成的集合也就是“\(G\)关于子群\(H\)的\DefineConcept{右商集}”,
记为\(\RQuotient\),
它是\(G\)的一个划分.

类似地,我们还可以重新定义:\[
	a \backsim b
	\defiff
	b^{-1} * a \in H.
\]
同理可证\(\backsim\)也是\(G\)上的一个等价关系.
任给\(a \in G\),它在关系\(\backsim\)下的等价类为\begin{align*}
	[a] &= \Set{ x \in G \given x \backsim a }
	= \Set{ x \in G \given a^{-1} * x \in H }
	= \Set{ x \in G \given a^{-1} * x = h \land h \in H } \\
	&= \Set{ x \in G \given x = a * h \land h \in H }
	= \Set{ a * h \given h \in H }.
\end{align*}
可以注意到,这里的等价类\([a]\)等于\(\{a\} * H\).
我们称\(\{a\} * H\)为“\(H\)的一个\DefineConcept{左陪集}”;
称\(a\)为它的\DefineConcept{陪集代表}.
于是\(H\)的全体左陪集组成的集合也就是“\(G\)关于子群\(H\)的\DefineConcept{左商集}”,
记为\(\LQuotient\),
它也是\(G\)的一个划分.

现在定义映射\[
	\sigma\colon (\LQuotient) \to (\RQuotient),
	(\{a\} * H) \mapsto (H * \{a^{-1}\}).
\]
由于\begin{align*}
	\{a\} * H = \{c\} * H
	&\iff
	c^{-1} \times a \in H \\
	&\iff
	c^{-1} \times (a^{-1})^{-1} \in H \\
	&\iff
	H * \{c^{-1}\} = H * \{a^{-1}\},
\end{align*}
因此\(\sigma\)是单射.
任给\(H * \{b\} \in \RQuotient\),
有\[
	\sigma(b^{-1} * H)
	= (H * b^{-1})^{-1}
	= H * \{b\},
\]
因此\(\sigma\)又是满射,
从而\(\sigma\)是双射.
于是左陪集与右陪集的基数相同,即\[
	\card(\LQuotient) = \card(\RQuotient).
\]
由此我们引出下述概念.
\begin{definition}
%@see: 《近世代数》(丘维声,2015) P42 定义2
%@see: 《近世代数》(熊全淹) P38 定义2
设\(H\)是群\(G\)的一个子群,
把\(G\)关于\(H\)的左商集\(\LQuotient\)的基数\(\card(\LQuotient)\)
称为“\(H\)在\(G\)中的\DefineConcept{指数}”,
记作\([G:H]\).
\end{definition}

假设群\(\opair{G,*}\)的子群\(H\)在\(G\)中的指数为\([G:H]=r\),则有
\begin{equation}\label{equation:抽象代数.关于子群的左陪集分解式}
%@see: 《近世代数》(丘维声,2015) P43 公式(1)
	G = H \cup (\{a_1\} * H)
	\cup (\{a_2\} * H)
	\cup \dotsb
	\cup (\{a_{r-1}\} * H),
\end{equation}
其中\(H,(\{a_1\} * H),(\{a_2\} * H),\dotsc,(\{a_{r-1}\} * H)\)两两互斥.

我们把\cref{equation:抽象代数.关于子群的左陪集分解式}
称为“群\(G\)关于子群\(H\)的\DefineConcept{左陪集分解式}”;
把\(\Set{e,\AutoTuple{a}{r-1}}\)称为“\(H\)在\(G\)中的\DefineConcept{左陪集代表系}”.

\begin{lemma}
%@see: 《近世代数》(丘维声,2015) P43
设\(\opair{H,*}\)是群\(\opair{G,*}\)的子群,
那么\begin{enumerate}
	\item 从\(H\)到它的任一左陪集的映射\[
		\tau\colon H \to (\{a\} * H), h \mapsto a * h
	\]是双射.

	\item 子群\(H\)与它的任一左陪集的基数相同,
	即\(\card H = \card(\{a\} * H)\).
\end{enumerate}
\end{lemma}

\begin{theorem}[拉格朗日定理]
%@see: 《近世代数》(丘维声,2015) P43 定理1
设\(\opair{G,*}\)是有限群,\(H\)是\(G\)的任一子群,则\[
	\card G = [G:H] \cdot \card H.
\]
\begin{proof}
设\([G:H]=r\),
则由\cref{equation:抽象代数.关于子群的左陪集分解式} 得\begin{align*}
	\card G
	&= \card H + \card(\{a_1\} * H) + \dotsb + \card(\{a_{r-1}\} * H) \\
	&= \card H + \card H + \dotsb + \card H \\
	&= r \card H.
	\qedhere
\end{align*}
\end{proof}
\end{theorem}
拉格朗日定理说明:
\(G\)的任一子群\(H\)的阶是\(G\)的阶的因数.

\begin{corollary}
设\(G\)是有限群,则\(G\)的任一元素\(a\)的阶是\(G\)的阶的因数,从而\(a^{\card G}=e\).
\end{corollary}


\endgroup%end subsection{陪集,拉格朗日定理}

\subsection{正规子群}
一般地,\(H\)的左陪集不一定是它的右陪集.
假如\(H\)的一个左陪集同时又是它的右陪集,
也就是说对于任一元\(a\),
我们有\(a\)与\(H\)可交换,即\(\{a\} * H = H * \{a\}\).
\begin{definition}
%@see: 《近世代数》(丘维声,2015) P53 定义3
%@see: 《近世代数》(熊全淹) P40 定义3
假如群\(\opair{G,*}\)的子群\(H\)满足:
\(H\)的任一左陪集同时也是它的右陪集,即\[
	(\forall a\in G)[\{a\} * H = H * \{a\}],
\]
那么称“\(H\)是\(G\)的\DefineConcept{正规子群}”,
记作\(H \triangleleft G\).
\end{definition}

设\(e\)是\(G\)的单位元,
那么\(\{e\}\)和\(G\)都是\(G\)的正规子群,
我们称它们为\DefineConcept{平凡正规子群}.
