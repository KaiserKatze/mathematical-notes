\section{群}
\subsection{群的概念}
\begin{definition}
假设在非空集合\(G\)上定义了一个代数运算\(\times\),
且它满足结合律,即\[
	(\forall a,b,c \in G)
	[(a \times b) \times c = a \times (b \times c)].
\]
那么称“定义了\(\times\)运算的\(G\)集\(\opair{G,\times}\)是一个\DefineConcept{半群}(semigroup)”,
或称“集合\(G\)对于\(\times\)运算成半群”.
\end{definition}

\begin{definition}
设\(\opair{G,\times}\)是半群,
且\(G\)中存在“单位元”,即\[
	(\exists e \in G)(\forall a \in G)
	[e \times a = a \times e = a].
\]
那么称“定义了\(\times\)运算的\(G\)集\(\opair{G,\times}\)是一个\DefineConcept{幺半群}(monoid)”,
或称“集合\(G\)对于\(\times\)运算成幺半群”.

这里我们把满足\[
	e \in G
	\land
    (\forall a \in G)
    [e \times a = a \times e = a]
\]的元素\(e\)称为“\(\opair{G,\times}\)的\DefineConcept{单位元}(identity)”.
\end{definition}

\begin{definition}
设\(\opair{G,\times}\)是幺半群,
且运算\(\times\)可逆,即\[
	(\forall a \in G)(\exists b \in G)
	[b \times a = a \times b = e].
\]
那么称“定义了\(\times\)运算的\(G\)集\(\opair{G,\times}\)是一个\DefineConcept{群}(group)”,
或称“集合\(G\)对于\(\times\)运算成群”.

这里我们把满足\[
	b \in G
	\land
	(\forall a \in G)
    [b \times a = a \times b = e],
\]
的元素\(b\)称为“\(a\)的\DefineConcept{逆元}(inverse)”,记作\(a^{-1}\).
\end{definition}

由\cref{equation:集合论.自然数加法结合律,equation:集合论.自然数乘法结合律}
以及\cref{equation:集合论.整数加法结合律,equation:集合论.整数乘法结合律},
自然数集\(\omega\)、整数集\(\mathbb{Z}\)对于加法、乘法分别成半群,即\[
	\opair{\omega,+}, \qquad
	\opair{\omega,\times}, \qquad
	\opair{\mathbb{Z},+}, \qquad
	\opair{\mathbb{Z},\times}
\]都是半群.

由于\(0\)是自然数加法、整数加法的单位元,
而\(1\)是自然数乘法、整数乘法的单位元,
所以自然数集\(\omega\)、整数集\(\mathbb{Z}\)对于加法、乘法又分别成幺半群.

由\cref{equation:集合论.整数加法可逆} 可知,
整数集\(\mathbb{Z}\)对于加法成群.
但是,整数集\(\mathbb{Z}\)对于乘法不成群.
同样地,自然数集\(\omega\)对于加法、乘法不成群.

特别地,仅由一个自然数\(0\)组成的集合\(\Set{0}\)对于加法成群,但对乘法不成群;
而仅由一个自然数\(1\)组成的集合\(\Set{1}\)对于乘法成群,但对加法不成群.

一个群的单位元既与构成这个群的集合有关,又与定义在这个集合上的运算有关.
在同一个非空集合上定义的不同的代数运算对应的单位元有可能不同.
例如,对于\(\opair{\omega,+}\),自然数\(0\)是它的单位元;
然而,对于\(\opair{\omega,\times}\),自然数\(1\)是它的单位元.
又例如,对于\(\opair{\mathbb{Z},+}\)和\(\opair{\mathbb{Z},\times}\),
整数\(0\)、整数\(1\)分别是它们的单位元.

\begin{property}
设\(\opair{G,\times}\)是群.
那么\(\opair{G,\times}\)的单位元唯一,
\(G\)中每个元素\(a\)的逆元唯一,且\[
    (a^{-1})^{-1} = a.
\]
\end{property}

\begin{definition}
如果半群\(\opair{G,\times}\)的乘法还满足交换律,
那么称“\(\opair{G,\times}\)是\DefineConcept{交换半群}(commutative semigroup).
\end{definition}

\begin{definition}
如果群\(\opair{G,\times}\)的乘法还满足交换律,
那么称“\(\opair{G,\times}\)是\DefineConcept{交换群}(commutative group)%
或\DefineConcept{阿贝尔群}(Abelian group)”.
\end{definition}

由于自然数、整数的加法、乘法都满足交换律,所以\[
	\opair{\omega,+}, \qquad
	\opair{\omega,\times}, \qquad
	\opair{\mathbb{Z},+}, \qquad
	\opair{\mathbb{Z},\times}
\]都是交换群.

\subsection{群的性质}
\begin{theorem}
设\(\opair{G,\times}\)是群.
那么\(n\)个元素\(\v{a}{n}\)的运算结果\[
	a_1 \times a_2 \times \dotsb \times a_n
\]由它们自身以及它们的顺序唯一确定.
\end{theorem}

\begin{corollary}
设\(\opair{G,\times}\)是交换群.
那么\(n\)个元素\(\v{a}{n}\)的运算结果\[
	a_1 \times a_2 \times \dotsb \times a_n
\]由它们自身唯一确定.
\end{corollary}

\section{子群}
\begin{definition}
%@see: 《近世代数》(丘维声,2015) P40. 定义1
%@see: https://mathworld.wolfram.com/Subgroup.html
如果群\(G\)的一个非空子集\(H\)对于\(G\)的运算\(\times\)也成为一个群,
那么称“\(H\)是\(G\)的一个\DefineConcept{子群}”,记作\(H \subseteq G\).
\end{definition}

显而易见的是,仅由群\(G\)的单位元\(e\)组成的子集\(\{e\}\)是\(G\)的一个子群.
同时,\(G\)本身也是\(G\)的一个子群.
我们将\(\{e\}\)和\(G\)统称为“\(G\)的\DefineConcept{平凡子群}”.

另外,我们还可以看出,如果\(H\)是群\(G\)的一个子群,
那么任给\(a,b \in H\),有\(a \times b \in H\).
设\(e'\)是\(H\)的单位元,则\(e' \times e' = e'\);
再在两边右乘\(e'\)在\(G\)中的逆元\((e')^{-1}\)得\[
	e' \times e' \times (e')^{-1} = e' \times (e')^{-1};
\]
由此可得\(e' \times e = e\).
又由于\(e' \times e = e'\),因此\(e = e'\).
这就说明,群\(G\)的单位元\(e\)是\(G\)的子群\(H\)的单位元.
任给\(b \in H\),设\(b\)在\(H\)中的逆元为\(c\),
则\[
	b \times c = c \times b = e;
\]
从\(G\)中看上式得\(c = b^{-1}\),
因此\(b^{-1} \in H\).
综上所述,如果\(H\)是群\(G\)的一个子群,
那么从\(a,b \in H\)可推出\(a \times b^{-1} \in H\).
下面我们来证明,这个条件也是群\(G\)的非空子集\(H\)是\(G\)的子群的充分条件.
\begin{theorem}
群\(G\)的非空子集\(H\)是\(G\)的子群的充要条件是:\[
	a,b \in H \implies a \times b^{-1} \in H.
\]
\begin{proof}
我们已经证得必要性,接下来只需证明充分性.
由于\(H\)不是空集,因此存在\(c \in H\).
由已知条件\[
	e = c \times c^{-1} \in H;
\]
任给\(b \in H\),同样由已知条件得\(e \times b^{-1} \in H\),于是\(b^{-1} \in H\).

任给\(a,b \in H\),由已知条件和已证明的结论可知,\[
	a \times (b^{-1})^{-1} \in H;
\]于是\(a \times b \in H\).
因此群\(G\)的运算是\(H\)的运算.
由于群\(G\)的运算满足结合律,因此它在\(H\)中的限制也满足结合律.
上面已证\(G\)的单位元\(e \in H\),
从而\(H\)有单位元\(e\),
且任给\(b \in H\),有\(b^{-1} \in H\),因此\(b\)在\(H\)中有逆元\(b^{-1}\).

综上所述,\(H\)是一个群,从而\(H\)是\(G\)的子群.
\end{proof}
\end{theorem}

利用群\(G\)的子群\(H\)可以研究群\(G\)的结构,
这时因为利用子群\(H\)可以给出集合\(G\)的一个划分.
为了给出\(G\)的一个划分,需要在\(G\)上建立一个二元等价关系.
于是我们定义:\[
	(a \sim b)
	\defiff
	a \times b^{-1} \in H.
\]
首先,由于\(a \times a^{-1} = e \in H\),因此\(a \sim a\);
这就说明\(\sim\)具有自反性.
然后,由\(a \sim b\)可得\(a \times b^{-1} \in H\);
那么由\(H \subseteq G\)可知\((a \times b^{-1})^{-1} \in H\);
于是\(b \times a^{-1} \in H\),从而\(b \sim a\);
这就说明\(\sim\)具有对称性.
最后,由\(a \sim b\)和\(b \sim c\)得\(a \times b^{-1} \in H\)和\(b \times c^{-1} \in H\);
由于\(H \subseteq G\),因此\((a \times b^{-1}) \times (b \times c^{-1}) \in H\);
于是\(a \times c^{-1} \in H\),从而\(a \sim c\);
这就说明\(\sim\)具有传递性.
综上所述,\(\sim\)是\(G\)上的一个等价关系.

\section{群的同态}
\begin{definition}
设\(G,\tilde{G}\)都是群,若映射\(\sigma\colon G \to \tilde{G}\)满足\[
	(\forall a,b \in G)[\sigma(a \times b) = \sigma(a) \times \sigma(b)],
\]
则称“\(\sigma\)是\(G\)到\(\tilde{G}\)的一个\DefineConcept{同态}”.

若同态\(\sigma\)是单射,则称其为\DefineConcept{单同态};
若它是满射,则称其为\DefineConcept{满同态}.
\end{definition}

\begin{property}
设\(\sigma\)是\(G\)到\(\tilde{G}\)的同态.
\begin{enumerate}
	\item \(\sigma(e)=\tilde{e}\),其中\(e,\tilde{e}\)分别是\(G,\tilde{G}\)的单位元.
	\item \((\forall a \in G)[\sigma(a^{-1})=(\sigma(a))^{-1}]\).
	\item \(G\)的子群\(H\)在\(\sigma\)下的像\(\sigma(H)\)是\(\tilde{G}\)子群.
	特别地,\(\sigma(G)\)是\(\tilde{G}\)的子群.
	\item 若\(a \in G\)且\(a^n = e\),则\((\sigma(a))^n = \tilde{e}\).
	于是,若\(a\)是\(G\)的\(n\)阶元,则\(\sigma(a)\)的阶是\(n\)的一个因数.
\end{enumerate}
\end{property}

\begin{definition}
设\(\sigma\)是群\(G\)到\(\tilde{G}\)的一个同态.
我们把\(\tilde{G}\)的单位元\(\tilde{e}\)在\(\sigma\)下的原像集%
称为“\(\sigma\)的\DefineConcept{核}(kernel)”,
记作\(\ker\sigma\),即\[
	\ker\sigma \defeq \Set{ a \in G \given \sigma(a) = \tilde{e} }.
\]
\end{definition}

\begin{theorem}
设\(\sigma\)是群\(G\)到\(\tilde{G}\)的一个同态,
则\(\ker\sigma\)是\(G\)的一个子群.
\begin{proof}
由于\(\sigma(e)=\tilde{e}\),因此\(e\in\ker\sigma\).
任取\(a,b\in\ker\sigma\),则\(\sigma(a)=\sigma(b)=\tilde{e}\),从而\[
	\sigma(a \times b^{-1}) = \sigma(a) \times (\sigma(b))^{-1}
	= \tilde{e} \times \tilde{e}^{-1}
	= \tilde{e};
\]
于是\(a \times b^{-1} \in \ker\sigma\),
这就是说,\(\ker\sigma\)是\(G\)的一个子群.
\end{proof}
\end{theorem}

\begin{theorem}
设\(\sigma\)是群\(G\)到\(\tilde{G}\)的一个同态,
则\[
	\text{\(\sigma\)是单射}
	\iff
	\ker\sigma=\{e\}.
\]
\end{theorem}

\begin{theorem}
设\(\sigma\)是群\(G\)到\(\tilde{G}\)的一个同态,
则\[
	(\forall a \in G)[a(\ker\sigma)=(\ker\sigma)a].
\]
\end{theorem}
