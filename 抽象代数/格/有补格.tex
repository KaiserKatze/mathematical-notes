\section{有补格}
一般来说,偏序格\(L\)不一定存在最大元与最小元.
例如实数集\(\mathbb{R}\)关于小于等于关系\(\leq\)的偏序格\(\opair{R,\leq}\).

\begin{definition}
%@see: 《离散数学》(邓辉文) P159 定义5-24
设\(\opair{L,\leq}\)是偏序格.
若\(L\)存在最大元和最小元,
则称“\(\opair{L,\leq}\)是\DefineConcept{有界格}(bounded lattice)”.
\end{definition}

按偏序集中的约定:
有界格的最大元记为\(1\),最小元记为\(0\).

由定义可知,在有界格\(\opair{L,\leq}\)中,
对任意\(x \in L\),有\(0 \leq x \leq 1\).
进而有\begin{gather*}
	x + 1 = 1, \\
	x \cdot 1 = x, \\
	x + 0 = x, \\
	x \cdot 0 = 0.
\end{gather*}
于是,\(1\)是\(\opair{L,\cdot}\)的单位元,
\(0\)是\(\opair{L,+}\)的单位元.

\begin{example}
%@see: 《离散数学》(邓辉文) P159 例5-33
设\(X\)是非空集合.
证明:\(\opair{\Powerset X,\subseteq}\)是有界格.
%TODO proof
\end{example}

\begin{proposition}
%@see: 《离散数学》(邓辉文) P159
任意有限格是有界格.
%TODO proof
\end{proposition}
