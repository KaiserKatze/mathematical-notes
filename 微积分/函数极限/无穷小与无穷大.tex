\section{无穷小与无穷大}
\subsection{无穷小}
\begin{definition}
%@see: 《高等数学(第六版 上册)》 P39 定义1
%@see: 《数学分析(上册)》(陈纪修) P100 定义3.3.1
若\(\lim_{x \to x_0} f(x) = 0\),
则称“函数\(f\)是当\(x \to x_0\)时的\DefineConcept{无穷小}”.
\end{definition}
这里的极限过程\(x \to x_0\)可以扩充到\(x \to x_0^+\)、\(x \to x_0^-\)、\(x \to \infty\)、\(x \to +\infty\)、\(x \to -\infty\)等情况.

\begin{definition}
设\(f\in\mathbb{R}^X\),\(\mathcal{B}\)是\(X\)中的基.
若\(\lim_\mathcal{B} f(x) = 0\),
则称“函数\(f\)是在基\(\mathcal{B}\)上的\DefineConcept{无穷小}”.
\end{definition}

\begin{theorem}
%@see: 《高等数学(第六版 上册)》 P39 定理1
设\(f\in\mathbb{R}^X\),\(\mathcal{B}\)是\(X\)中的基.
\(\lim_\mathcal{B} f(x) = A \in \mathbb{R}\)的充分必要条件是:\[
	(\exists\alpha\in\mathbb{R}^X)
	\left[
		\lim_\mathcal{B} \alpha(x) = 0
		\land
		f(x) = A + \alpha(x)
	\right].
\]
\end{theorem}
