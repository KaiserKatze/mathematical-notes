\section{收敛准则}
\subsection{单调有界函数收敛定理}
相应于\hyperref[theorem:极限.数列的单调有界定理]{单调有界数列收敛定理},函数极限也有类似的收敛准则.
\begin{theorem}\label{theorem:极限.函数的单调有界定理}
设函数\(f(x)\)在点\(x_0\)的某个左邻域\((x_0-\delta,x_0)\)内单调并且有界,
则\(f(x)\)在\(x_0\)的左极限\(f(x_0^-)\)必定存在.
\end{theorem}

\subsection{柯西极限存在准则}
\begin{theorem}\label{theorem:极限.函数的柯西极限存在准则}
%@see: 《数学分析(上册)》(陈纪修) P85 定理3.1.6
设函数\(f\colon D\to\mathbb{R}\).
\begin{align*}
	\text{\(\lim_{x \to a} f(x)\)存在且有限}
	&\iff
	\begin{array}[t]{l}
		(\forall\epsilon>0)
		(\exists\delta>0)
		(\forall x_1,x_2\in D)\\ \relax
		[
			0 < \abs{x_1 - a} < \delta \land 0 < \abs{x_2 - a} < \delta
			\implies
			\abs{f(x_1) - f(x_2)} < \epsilon
		],
	\end{array} \\
	\text{\(\lim_{x \to a^+} f(x)\)存在且有限}
	&\iff
	\begin{array}[t]{l}
		(\forall\epsilon>0)
		(\exists\delta>0)
		(\forall x_1,x_2\in D)\\ \relax
		[
			0 < x_1 - a < \delta \land 0 < x_2 - a < \delta
			\implies
			\abs{f(x_1) - f(x_2)} < \epsilon
		],
	\end{array} \\
	\text{\(\lim_{x \to a^-} f(x)\)存在且有限}
	&\iff
	\begin{array}[t]{l}
		(\forall\epsilon>0)
		(\exists\delta>0)
		(\forall x_1,x_2\in D)\\ \relax
		[
			-\delta < x_1 - a < 0 \land -\delta < x_2 - a < 0
			\implies
			\abs{f(x_1) - f(x_2)} < \epsilon
		],
	\end{array} \\
	\text{\(\lim_{x\to\infty} f(x)\)存在且有限}
	&\iff
	\begin{array}[t]{l}
		(\forall\epsilon>0)
		(\exists X>0)
		(\forall x_1,x_2\in D)\\ \relax
		[
			\abs{x_1} > X \land \abs{x_2} > X
			\implies
			\abs{f(x_1) - f(x_2)} < \epsilon
		],
	\end{array} \\
	\text{\(\lim_{x\to+\infty} f(x)\)存在且有限}
	&\iff
	\begin{array}[t]{l}
		(\forall\epsilon>0)
		(\exists X>0)
		(\forall x_1,x_2\in D)\\ \relax
		[
			x_1 > X \land x_2 > X
			\implies
			\abs{f(x_1) - f(x_2)} < \epsilon
		],
	\end{array} \\
	\text{\(\lim_{x\to-\infty} f(x)\)存在且有限}
	&\iff
	\begin{array}[t]{l}
		(\forall\epsilon>0)
		(\exists X>0)
		(\forall x_1,x_2\in D)\\ \relax
		[
			x_1 < -X \land x_2 < -X
			\implies
			\abs{f(x_1) - f(x_2)} < \epsilon
		].
	\end{array}
\end{align*}
\end{theorem}
