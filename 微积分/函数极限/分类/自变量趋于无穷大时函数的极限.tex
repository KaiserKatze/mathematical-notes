\subsection{自变量趋于无穷大时函数的极限}
\begin{definition}\label{definition:极限.函数极限的定义4}
设函数\(f\colon D\to\mathbb{R}\)当\(\abs{x}\)大于某一正数时有定义,
即\begin{equation*}
	(\exists\rho>0)
	[(-\infty,-\rho)\cup(\rho,+\infty) \subseteq D].
\end{equation*}
如果存在常数\(A\in\mathbb{R}\),
对于任意给定的正数\(\epsilon\)(不论它多么小),
总存在正数\(X\),
使得当\(x\)满足不等式\(\abs{x} > X\)时,
对应的函数值\(f(x)\)都满足不等式\begin{equation*}
	\abs{f(x) - A} < \epsilon,
\end{equation*}
那么常数\(A\)就叫做“函数\(f\)当\(x \to \infty\)时的\DefineConcept{极限}”,
记作\begin{equation*}
	\lim_{x \to \infty} f(x) = A
	\quad\text{或}\quad
	f(\infty) = A
	\quad\text{或}\quad
	f(x) \to A\ (x \to \infty).
\end{equation*}
\end{definition}
\cref{definition:极限.函数极限的定义4} 可以简化为:\begin{equation*}
	\lim_{x\to\infty} f(x) = A
	\defiff
	(\forall\epsilon>0)
	(\exists X>0)
	[
		\abs{x} > X
		\implies
		\abs{f(x) - A} < \epsilon
	].
\end{equation*}

我们只要对\cref{definition:极限.函数极限的定义4} 中的条件稍加改变,
就可以得到以下两种不同的极限定义.
\begin{definition}\label{definition:极限.函数极限的定义5}
设函数\(f\colon D\to\mathbb{R}\)当\(x\)大于某一正数时有定义,
即\begin{equation*}
	(\exists\rho>0)
	[(\rho,+\infty) \subseteq D].
\end{equation*}
如果存在常数\(A\in\mathbb{R}\),
对于任意给定的正数\(\epsilon\),
总存在正数\(X\),
使得当\(x\)满足不等式\(x > X\)时,
对应的函数值\(f(x)\)都满足不等式\begin{equation*}
	\abs{f(x) - A} < \epsilon,
\end{equation*}
那么常数\(A\)就叫做“函数\(f\)当\(x \to +\infty\)时的\DefineConcept{极限}”,
记作\begin{equation*}
	\lim_{x \to +\infty} f(x) = A
	\quad\text{或}\quad
	f(+\infty) = A
	\quad\text{或}\quad
	f(x) \to A\ (x \to +\infty).
\end{equation*}
\end{definition}

\begin{definition}\label{definition:极限.函数极限的定义6}
设函数\(f\colon D\to\mathbb{R}\)当\(-x\)大于某一正数时有定义,
即\begin{equation*}
	(\exists\rho>0)
	[(-\infty,-\rho) \subseteq D].
\end{equation*}
如果存在常数\(A\),
对于任意给定的正数\(\epsilon\),
总存在正数\(X\),
使得当\(x\)满足不等式\(x < -X\)时,
对应的函数值\(f(x)\)都满足不等式\begin{equation*}
	\abs{f(x) - A} < \epsilon,
\end{equation*}
那么常数\(A\)就叫做“函数\(f\)当\(x \to -\infty\)时的\DefineConcept{极限}”,
记作\begin{equation*}
	\lim_{x \to -\infty} f(x) = A
	\quad\text{或}\quad
	f(-\infty) = A
	\quad\text{或}\quad
	f(x) \to A\ (x \to -\infty).
\end{equation*}
\end{definition}

\begin{example}
证明:\(\lim_{x\to\infty} \frac1x = 0\).
\begin{proof}
\(\forall\epsilon>0\),
要证\(\exists X > 0\),
当\(\abs{x}>X\)时,
不等式\begin{equation*}
	\abs{\frac1x-0}<\epsilon
\end{equation*}成立.
因这个不等式相当于\(\frac{1}{\abs{x}}<\epsilon\)或\(\abs{x}>\frac{1}{\epsilon}\).
由此可知,如果取\(X=\frac{1}{\epsilon}\),
那么当\(\abs{x}>X=\frac{1}{\epsilon}\)时,
不等式\(\abs{\frac1x-0}<\epsilon\)成立,
这就证明了\(\lim_{x\to\infty} \frac1x = 0\).
\end{proof}
\end{example}

与\cref{theorem:函数极限.极限与单侧极限的关系1} 类似,我们有如下命题.
\begin{proposition}\label{theorem:函数极限.极限与单侧极限的关系2}
%@see: 《高等数学(第六版 上册)》 P38 习题1-3 10.
设函数\(f\colon D\to\mathbb{R}\)当\(\abs{x}\)大于某一正数时有定义,
即\begin{equation*}
	(\exists X>0)
	[(-\infty,-X)\cup(X,+\infty) \subseteq \dom f],
\end{equation*}
则\begin{equation*}
	\lim_{x \to \infty} f(x) = A
	\iff
	\lim_{x \to -\infty} f(x) = \lim_{x \to +\infty} f(x) = A.
\end{equation*}
%TODO proof
\end{proposition}

\begin{table}[htp]
	\centering
	\begin{tblr}{c*4{|c}}
		\hline
		& \(f(x) \to A\)
			& \(f(x) \to \infty\)
			& \(f(x) \to +\infty\)
			& \(f(x) \to -\infty\) \\ \hline
		\(x \to \infty\)
		& \(\begin{aligned}[t]
			& (\forall\epsilon>0)
			(\exists X>0)\\
			& (\forall x \in D)\\
			& [\abs{x}>X\\
			& \implies \abs{f(x)-A}<\epsilon]\\
		\end{aligned}\)
		& \(\begin{aligned}[t]
			& (\forall G>0)
			(\exists X>0)\\
			& (\forall x \in D)\\
			& [\abs{x}>X\\
			& \implies \abs{f(x)}>G]\\
		\end{aligned}\)
		& \(\begin{aligned}[t]
			& (\forall G>0)
			(\exists X>0)\\
			& (\forall x \in D)\\
			& [\abs{x}>X\\
			& \implies f(x)>G]\\
		\end{aligned}\)
		& \(\begin{aligned}[t]
			& (\forall G>0)
			(\exists X>0)\\
			& (\forall x \in D)\\
			& [\abs{x}>X\\
			& \implies f(x)<-G]\\
		\end{aligned}\)
		\\ \hline
		\(x \to +\infty\)
		& \(\begin{aligned}[t]
			& (\forall\epsilon>0)
			(\exists X>0)\\
			& (\forall x \in D)\\
			& [x>X\\
			& \implies \abs{f(x)-A}<\epsilon]\\
		\end{aligned}\)
		& \(\begin{aligned}[t]
			& (\forall G>0)
			(\exists X>0)\\
			& (\forall x \in D)\\
			& [x>X\\
			& \implies \abs{f(x)}>G]\\
		\end{aligned}\)
		& \(\begin{aligned}[t]
			& (\forall G>0)
			(\exists X>0)\\
			& (\forall x \in D)\\
			& [x>X\\
			& \implies f(x)>G]\\
		\end{aligned}\)
		& \(\begin{aligned}[t]
			& (\forall G>0)
			(\exists X>0)\\
			& (\forall x \in D)\\
			& [x>X\\
			& \implies f(x)<-G]\\
		\end{aligned}\)
		\\ \hline
		\(x \to -\infty\)
		& \(\begin{aligned}[t]
			& (\forall\epsilon>0)
			(\exists X>0)\\
			& (\forall x \in D)\\
			& [x<-X\\
			& \implies \abs{f(x)-A}<\epsilon]\\
		\end{aligned}\)
		& \(\begin{aligned}[t]
			& (\forall G>0)
			(\exists X>0)\\
			& (\forall x \in D)\\
			& [x<-X\\
			& \implies \abs{f(x)}>G]\\
		\end{aligned}\)
		& \(\begin{aligned}[t]
			& (\forall G>0)
			(\exists X>0)\\
			& (\forall x \in D)\\
			& [x<-X\\
			& \implies f(x)>G]\\
		\end{aligned}\)
		& \(\begin{aligned}[t]
			& (\forall G>0)
			(\exists X>0)\\
			& (\forall x \in D)\\
			& [x<-X\\
			& \implies f(x)<-G]\\
		\end{aligned}\)
		\\ \hline
	\end{tblr}
	\caption{自变量趋于无穷大时函数的极限的定义}
\end{table}
