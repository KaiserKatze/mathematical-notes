\subsection{单侧极限}
\begin{definition}\label{definition:极限.函数极限的定义2}
%@see: 《数学分析(上册)》(陈纪修) P80 定义3.1.2
设函数\(f\colon D\to\mathbb{R}\)在点\(x_0\)的某个去心左邻域中有定义,
即\[
	(\exists\rho>0)
	[\mathring{U}_-(x_0,\rho) \subseteq D].
\]
如果存在常数\(A\in\mathbb{R}\),
对于任意给定的正数\(\epsilon\),
总存在正数\(\delta\),
使得当\(x\)满足不等式\(-\delta < x - x_0 < 0\)或\(x_0 - \delta < x < x_0\)时,
总有不等式\(\abs{f(x) - A} < \epsilon\)成立,
那么常数\(A\)就叫做“函数\(f\)当\(x \to x_0\)时的\DefineConcept{左极限}”,
记作\[
	\lim_{x \to x_0^-} f(x) = A,
	\quad\text{或}\quad
	f(x_0^-) = A.
\]
\end{definition}

\begin{definition}\label{definition:极限.函数极限的定义3}
%@see: 《数学分析(上册)》(陈纪修) P80 定义3.1.2
设函数\(f\colon D\to\mathbb{R}\)在点\(x_0\)的某个去心右邻域中有定义,
即\[
	(\exists\rho>0)
	[\mathring{U}_+(x_0,\rho) \subseteq D].
\]
如果存在常数\(A\in\mathbb{R}\),
对于任意给定的正数\(\epsilon\),
总存在正数\(\delta\),
使得当\(x\)满足不等式\(0 < x - x_0 < \delta\)或\(x_0 < x < x_0 + \delta\)时,
总有不等式\(\abs{f(x) - A} < \epsilon\)成立,
那么常数\(A\)就叫做“函数\(f(x)\)当\(x \to x_0\)时的\DefineConcept{右极限}”,
记作\[
	\lim_{x \to x_0^+} f(x) = A
	\quad\text{或}\quad
	f(x_0^+) = A.
\]
\end{definition}

左极限与右极限统称为\DefineConcept{单侧极限}.

\begin{remark}
我们还可以将上述两类极限的定义写成:\begin{gather*}
	\lim_{x \to x_0^-} f(x)
	\defeq
	\lim_{\substack{x \to x_0 \\ x < x_0}} f(x), \qquad
	\lim_{x \to x_0^+} f(x)
	\defeq
	\lim_{\substack{x \to x_0 \\ x > x_0}} f(x).
\end{gather*}
或者写成\begin{gather*}
	\lim_{x \to x_0^-} f(x) = A
	\defiff
	(\forall \epsilon > 0)
	(\exists \delta > 0)
	(\forall x \in D)
	[-\delta < x - x_0 < 0 \implies \abs{f(x) - A} < \epsilon], \\
	\lim_{x \to x_0^+} f(x) = A
	\defiff
	(\forall \epsilon > 0)
	(\exists \delta > 0)
	(\forall x \in D)
	[0 < x - x_0 < \delta \implies \abs{f(x) - A} < \epsilon].
\end{gather*}
\end{remark}

\begin{theorem}
设函数\(f\colon D\to\mathbb{R}\)在点\(x_0\)的某个去心邻域中有定义,
则\[
	\lim_{x \to x_0} f(x) = A
	\iff
	\lim_{x \to x_0^-} f(x) = \lim_{x \to x_0^+} f(x) = A.
\]
\end{theorem}

\begin{example}
证明:函数\[
	f(x) = \left\{ \begin{array}{lc}
		x-1, & x<0, \\
		0, & x=0, \\
		x+1, & x>0.
	\end{array} \right.
\]当\(x\to0\)时\(f(x)\)的极限不存在.
\begin{proof}
易证\[
	\lim_{x\to0^-} f(x) = \lim_{x\to0^-} x-1 = -1,
\]
而\[
	\lim_{x\to0^+} f(x) = \lim_{x\to0^+} x+1 = 1,
\]
因为左、右极限存在但不相等,所以\(\lim_{x\to0}\)不存在.
\end{proof}
\end{example}
