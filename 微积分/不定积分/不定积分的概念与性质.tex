\section{不定积分的概念与性质}
在\cref{chapter:导数}我们已经介绍了求导运算.
现在我们来探索求导运算的逆运算.

\subsection{原函数与不定积分的概念}
\begin{definition}\label{definition:不定积分.原函数}
%@see: 《高等数学(第六版 上册)》 P184 定义1
%@see: 《数学分析(第二版 上册)》(陈纪修) P241 定义6.1.1
如果在区间\(I\)上,
可导函数\(F\)的导函数为\(f\),
即对任一\(x \in I\),
都有\[
	F'(x)=f(x)
	\quad\text{或}\quad
	\dd{F(x)}=f(x) \dd{x},
\]
那么称
“\(F\)是\(f\)在区间\(I\)上的一个\DefineConcept{原函数}(primitive)”
“\(f\)在区间\(I\)上具有原函数\(F\)”.
\end{definition}
\begin{remark}
由\cref{definition:不定积分.原函数} 可知:
任一函数只要具有原函数,它的原函数一定可导.
反过来说,如果一个函数在某个区间上不可导,它一定不是定义在这个区间上的任意函数的原函数.
\end{remark}

关于原函数,我们首先要问:
一个函数具备什么条件,才能保证它的原函数一定存在?
这个问题将在下一章中讨论,这里先介绍一个结论.
\begin{theorem}[原函数存在定理]%提前叙述该定理.参见\cref{theorem:定积分.原函数存在定理}
如果函数\(f(x)\)在区间\(I\)上连续,
那么在区间\(I\)上存在可导函数\(F(x)\),使对任一\(x \in I\)都有\[
	F'(x)=f(x).
\]
换言之,连续函数一定有原函数.
\end{theorem}
下面还要说明两点.

第一,如果\(f(x)\)在区间\(I\)上有原函数,
即有一个函数\(F(x)\),
使对任一\(x \in I\),
都有\(F'(x) = f(x)\),
那么,对任何常数\(C\),显然也有\[
	[F(x) + C]' = f(x),
\]
即对任何常数\(C\),函数\(F(x) + C\)也是\(f(x)\)的原函数.
这说明,如果\(f(x)\)有一个原函数,那么\(f(x)\)就有无限多个原函数.

第二,如果在区间\(I\)上\(F(x)\)是\(f(x)\)的一个原函数,
那么\(f(x)\)的其他原函数与\(F(x)\)有什么关系?

设\(\Phi(x)\)是\(f(x)\)的另一个原函数,
即对任一\(x \in I\)有\[
	\Phi'(x) = f(x),
\]
于是\[
	[\Phi(x) - F(x)]' = \Phi'(x) - F'(x) = f(x) - f(x) = 0.
\]
在前面章节已经知道,
在一个区间上导数恒为零的函数必为常数,所以\[
	\Phi(x) - F(x) = C_0,
\]
其中\(C_0\)是某个常数.这就表明\(\Phi(x)\)与\(F(x)\)只差一个常数.
因此,当\(C\)为任意的常数时,
表达式\(F(x) + C\)就可表示\(f(x)\)的任意一个原函数.
也就是说,\(f(x)\)的全体原函数所组成的集合,就是函数族\[
	\Set{ F(x) + C \given C \in (-\infty,\infty) }.
\]

\begin{definition}
%@see: 《高等数学(第六版 上册)》 P185 定义2
%@see: 《数学分析(第二版 上册)》(陈纪修) P242 定义6.1.2
在区间\(I\)上,
函数\(f\)的全体原函数,
称为“\(f\)在区间\(I\)上的\DefineConcept{不定积分}(indefinite integral)”,
记作\[
	\int f(x) \dd{x},
\]
即\[
	\int f(x) \dd{x}
	\defeq
	\Set{ F \in \mathbb{R}^I \given (\forall x \in I)[F'(x) = f(x)] },
\]
其中记号\(\int\)称为\DefineConcept{积分号},
\(f(x)\)称为\DefineConcept{被积函数},
\(f(x) \dd{x}\)称为\DefineConcept{被积表达式},
\(x\)称为\DefineConcept{积分变量}.
\end{definition}
由此定义及前面的说明可知,
如果\(F(x)\)是\(f(x)\)在区间\(I\)上的一个原函数,
那么\(F(x) + C\)就是\(f(x)\)的不定积分,
即\[
	\int f(x) \dd{x} = F(x) + C.
\]
因而不定积分\(\int f(x) \dd{x}\)可以表示\(f(x)\)的任意一个原函数.

对初等函数来说,在其定义区间上,它的原函数一定存在,但原函数不一定都是初等函数,如\[
	\int e^{-x^2} \dd{x}, \qquad
	\int \frac{\sin x}{x} \dd{x}, \qquad
	\int \frac{\dd{x}}{\ln{x}}, \qquad
	\int \frac{\dd{x}}{\sqrt{1+x^4}}
\]
等等,它们的原函数就都不是初等函数.

\begin{example}
求\(\int \frac{\dd{x}}{x}\).
\begin{solution}
当\(x > 0\)时,由于\((\ln x)' = \frac{1}{x}\),
所以\(\ln x\)是\(\frac{1}{x}\)在区间\((0,+\infty)\)内的一个原函数.
因此,在\((0,+\infty)\)内,\[
	\int \frac{\dd{x}}{x} = \ln x + C_1.
\]

当\(x < 0\)时,
由于\([\ln(-x)]' = \frac{1}{-x} \cdot (-1) = \frac{1}{x}\),
所以\(\ln(-x)\)是\(\frac{1}{x}\)在\((-\infty,0)\)内的一个原函数.
因此,在\((-\infty,0)\)内,\[
	\int \frac{\dd{x}}{x} = \ln(-x) + C_2.
\]

把在\(x > 0\)及\(x < 0\)内的结果合起来,
可写作\begin{equation}
	\int \frac{\dd{x}}{x} = \left\{ \begin{array}{lc}
		\ln x + C_1, & x>0, \\
		\ln(-x) + C_2, & x<0.
	\end{array} \right.
\end{equation}
虽然常数\(C_1\)和\(C_2\)的取值可以是独立的,
但在不严谨的情况下,为方便记忆,上式也可写作\begin{equation}
	\int \frac{\dd{x}}{x} = \ln\abs{x} + C.
\end{equation}
\end{solution}
\end{example}

\begin{definition}
函数\(f(x)\)的原函数的图形称为\(f(x)\)的积分曲线.
\end{definition}

从不定积分的定义,即可知以下关系:
\begin{align*}
	\text{\(\int f(x) \dd{x}\)是\(f(x)\)的原函数}
	&\iff
	\dv{x}\relax\left[ \int f(x) \dd{x} \right] = f(x) \\
	&\iff
	\dd\relax\left[\int f(x) \dd{x}\right] = f(x) \dd{x}. \\
	\text{\(F(x)\)是\(F'(x)\)的原函数}
	&\iff
	\int F'(x) \dd{x} = F(x) + C \\
	&\iff
	\int \dd{F(x)} = F(x) + C.
\end{align*}

由此可见,微分运算(以记号\(\dd\relax\)表示)
与求不定积分的运算(简称积分运算,以记号\(\int\)表示)是互逆的;
当记号\(\int\)与\(\dd\relax\)连在一起时,或者抵消,或者抵消后差一个常数.

\subsection{基本积分表}
既然积分运算是微分运算的逆运算,
那么很自然地可以从导数公式得到相应的积分公式.

\subsection{不定积分的性质}
\begin{property}\label{theorem:不定积分.性质1}
%@see: 《高等数学(第六版 上册)》 P189 性质1
%@see: 《数学分析(第二版 上册)》(陈纪修) P243 定理6.1.1(线性性)
设函数\(f\)及\(g\)的原函数存在,
则\[
	\int [f(x) + g(x)] \dd{x}
	= \int f(x) \dd{x}
	+ \int g(x) \dd{x}.
\]
\end{property}
\begin{remark}
\cref{theorem:不定积分.性质1} 对于有限个函数都是成立的.
\end{remark}

\begin{property}\label{theorem:不定积分.性质2}
%@see: 《高等数学(第六版 上册)》 P190 性质2
%@see: 《数学分析(第二版 上册)》(陈纪修) P243 定理6.1.1(线性性)
设函数\(f\)的原函数存在,
\(k\)为非零常数,
则\[
	\int k f(x) \dd{x} = k \int f(x) \dd{x}.
\]
\end{property}
\begin{remark}
我们在\cref{theorem:不定积分.性质2} 中强调\(k\neq0\),
是因为如果\(k=0\),那么上式右边为\(0 \int f(x) \dd{x} = 0\),
而上式左边\(\int 0 f(x) \dd{x}\)却是全体常数函数的集合,
这就造成了定义上的混乱.
\end{remark}

\begin{example}
%@see: 《高等数学(第六版 上册)》 P191 例11
求\(\int 2^x e^x \dd{x}\).
\begin{solution}
应为\(2^x e^x = (2e)^x\),
所以由基本积分表有\[
	\int 2^x e^x \dd{x}
	= \int (2e)^x \dd{x}
	= \frac{(2e)^x}{\ln(2e)} + C
	= \frac{2^x e^x}{1+\ln2} + C.
\]
\end{solution}
\end{example}
\begin{example}
求\(\int \sin^2 x \dd{x}\)和\(\int \cos^2 x \dd{x}\).
\begin{solution}
记\[
	I_1 \defeq \int \sin^2 x \dd{x},
	\qquad
	I_2 \defeq \int \cos^2 x \dd{x}.
\]
那么\begin{gather*}
	I_1 + I_2
	= \int (\sin^2 x + \cos^2 x) \dd{x}
	= \int \dd{x}
	= x + C_1, \\
	I_2 - I_1
	= \int (\cos^2 x - \sin^2 x) \dd{x}
	= \int \cos 2x \dd{x}
	= \frac12 \sin 2x + C_2,
\end{gather*}
解得\begin{gather*}
	\int \sin^2 x \dd{x} = \frac{x}2 - \frac14 \sin 2x + C_3, \\
	\int \cos^2 x \dd{x} = \frac{x}2 + \frac14 \sin 2x + C_4.
\end{gather*}
\end{solution}
\end{example}
\begin{example}
%@see: 《高等数学(第六版 上册)》 P191 例12
求\(\int \tan^2 x \dd{x}\).
\begin{solution}
根据\hyperref[equation:三角函数.毕达哥拉斯三角恒等式2]{三角恒等式}
\(\tan^2 x + 1 = \sec^2 x\),
有\[
	\int \tan^2 x \dd{x}
	= \int (\sec^2 x - 1) \dd{x}
	= \int \sec^2 x \dd{x} - \int \dd{x}
	= \tan x - x + C.
\]
\end{solution}
\end{example}
\begin{example}
求\(\int \cot^2 x \dd{x}\).
\begin{solution}
根据\hyperref[equation:三角函数.毕达哥拉斯三角恒等式3]{三角恒等式}
\(1 + \cot^2 x = \csc^2 x\),
有\[
	\int \cot^2 x \dd{x}
	= \int (\csc^2 x - 1) \dd{x}
	= -\cot x - x + C.
\]
\end{solution}
\end{example}
