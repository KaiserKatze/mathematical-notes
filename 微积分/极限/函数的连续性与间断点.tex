设变量\(u\)从它的一个初值\(u_1\)变到终值\(u_2\),
终值与初值的差\(u_2 - u_1\)就叫做变量\(u\)的\DefineConcept{增量},
记作\(\increment u\),即\begin{equation*}
	\increment u = u_2 - u_1.
\end{equation*}

在实数域中,增量\(\increment u\)既可以是正的,也可以是负的.
当\(\increment u > 0\)时,变量\(u\)从初值变到终值时是增大的;
当\(\increment u < 0\)时,变量\(u\)从初值变到终值时是减小的.

应该注意到:
记号\(\increment u\)并不表示某个量\(\Delta\)与变量\(u\)的乘积,
而是一个不可分割的符号.

%\subsection{连续曲线}
%\begin{definition}
%设平面曲线\(L\)的参数方程为\begin{equation*}
%	\left\{ \begin{array}{l}
%		x = \phi(t) \\
%		y = \psi(t)
%	\end{array} \right.,
%	\quad
%	t \in [\alpha,\beta].
%\end{equation*}
%如果\(\phi(t)\)、\(\psi(t)\)在\([\alpha,\beta]\)上连续,
%则称曲线\(L\)为\DefineConcept{连续曲线}.
%点\(\opair{\phi(\alpha),\psi(\alpha)}\)称为曲线的\DefineConcept{起点},
%点\(\opair{\phi(\beta),\psi(\beta)}\)称为曲线的\DefineConcept{终点}.
%
%如果存在\(t_1\)、\(t_2\)满足\(\alpha \leq t_1 < t_2 \leq \beta\)
%且\((t_1-\alpha)^2+(t_2-\beta)^2 \neq 0\),使得对应的两点重合,
%即\(\opair{\phi(t_1),\psi(t_1)}=\opair{\phi(t_2),\psi(t_2)}\)成立,
%则称该点为曲线\(L\)的\DefineConcept{重点}.
%
%无重点的连续曲线称为\DefineConcept{若尔当曲线}或\DefineConcept{简单曲线}.
%
%仅起点和终点重合
%(即\(\opair{\phi(\alpha),\psi(\alpha)}
%=\opair{\phi(\beta),\psi(\beta)}\))
%的简单曲线称作\DefineConcept{若尔当闭曲线}或者\DefineConcept{简单闭曲线}.
%\end{definition}
