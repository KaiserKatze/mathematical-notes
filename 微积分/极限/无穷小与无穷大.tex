\section{无穷小与无穷大}



显然,当一个函数是无穷大时,必有该函数无界;但当一个函数无界时,却不一定有该函数是无穷大.
\begin{example}
证明:函数\(f(x) = \frac{1}{x} \sin\frac{1}{x}\)在区间\((0,1]\)上无界,
但该函数不是\(x\to0^+\)时的无穷大.
\begin{proof}
要证函数\(y = \frac{1}{x} \sin\frac{1}{x}\)在区间\((0,1]\)上无界,
只需证\[
	(\forall M > 0)
	(\exists x \in (0,1])
	[\abs{f(x)} > M].
\]

取数列\(u_n = \frac{\pi}{2} + n\pi\ (n=0,1,2,\dotsc)\),
那么恒有\(\abs{\sin u_n} = 1\)和\[
	1 < u_0 < u_1 < \dotsb < u_n < \dotsb,
\]\[
	0 < \dotsb < \frac{1}{u_n} < \dotsb < \frac{1}{u_1} < \frac{1}{u_0} < 1
\]成立.
易证\((\forall M > 0)[n > M/\pi \implies u_n > M]\),
也就是说数列\(\{u_n\}\)无界.

由于\[
	\abs{f(x)} = \abs{\frac{1}{x} \sin\frac{1}{x}}
	= \abs{\frac{1}{x}} \abs{\sin\frac{1}{x}}
	= \frac{1}{x} \abs{\sin\frac{1}{x}},
\]\[
	\abs{f(1/u_n)}
	= u_n \abs{\sin u_n} \equiv u_n,
\]
所以函数值数列\(\{\abs{f(1/u_n)}\}\)也无界,
自然地,函数\(f(x)\)也无界.

假设\(f(x)\)是\(x\to0^+\)时的无穷大,
那么\[
	(\forall M > 0)
	(\exists \delta > 0)
	[
		0 < x < \delta
		\implies
		\abs{f(x)} > M
	].
\]
取数列\(v_n = n\pi\ (n=1,2,\dotsc)\),
那么\(v_n > 1\)和\(\sin v_n = 0\)恒成立.
由函数值\[
	\abs{f(1/v_n)} = v_n \abs{\sin v_n} \equiv 0
\]构成的数列\(\{\abs{f(1/v_n)}\}\)恒小于任意正数,
与假设矛盾,说明\(f(x)\)不是\(x\to0^+\)时的无穷大.
\end{proof}
\end{example}
