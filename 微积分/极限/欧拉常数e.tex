\section{欧拉常数e}
可以证明,欧拉常数\(e\)是一个无理数,也是一个超越数,它的近似值约为2.718.

我们在前一节利用单调有界定理证明了数列极限\[
\lim_{n\to\infty} \left(1+\frac{1}{n}\right)^n = e,
\]
那么,当\(x\neq0\)时,有
\begin{align*}
	\lim_{n\to\infty} \left(1+\frac{x}{n}\right)^n
	&= \lim_{n\to\infty} \left(1+\frac{1}{n/x}\right)^{\frac{n}{x} \cdot x} \\
	&\xlongequal{t=n/x} \left[ \lim_{t\to\infty} \left(1+\frac{1}{t}\right)^t \right]^x
	= e^x;
\end{align*}
当\(x=0\)时,有\[
	\lim_{n\to\infty} \left(1+\frac{x}{n}\right)^n
	= \lim_{n\to\infty} (1+0)^n
	= \lim_{n\to\infty} 1
	= 1 \equiv e^0;
\]
因此我们可以定义“以\(e\)为底数的指数函数”为
\begin{equation}\label{equation:特殊函数.以e为底数的指数函数的极限定义}
	e^x
	\defeq
	\lim_{n\to\infty} \left(1+\frac{x}{n}\right)^n.
\end{equation}

我们还可以定义“以\(e\)为底数的对数函数”为
\begin{equation}\label{equation:特殊函数.以e为底数的对数函数的极限定义}
	\ln x
	\defeq
	\lim_{n\to\infty} n \left( \sqrt[n]{x} - 1 \right).
\end{equation}

我们可以得到一些关于自然对数的不等式.
\begin{gather}
	\frac{x}{1+x} < \ln(1+x) < x
	\quad(x>0), \\
	\frac{1}{n+1} < \ln(1+\frac{1}{n}) < \frac{1}{n}
	\quad(n\in\mathbb{N}^+), \\
	\ln(1+n) < \sum_{k=1}^n \frac{1}{k} < 1 + \ln n.
\end{gather}
