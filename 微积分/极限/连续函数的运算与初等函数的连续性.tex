\section{连续函数的运算与初等函数的连续性}
\subsection{连续函数的和、差、积、商的连续性}

\subsection{反函数与复合函数的连续性}

\subsection{初等函数的连续性}
\begin{theorem}\label{theorem:极限.连续函数的极限5}
基本初等函数在其定义域内都是连续的.
\end{theorem}

\begin{corollary}\label{theorem:极限.连续函数的极限6}
一切初等函数在其定义区间(即包含在定义域内的区间)内都是连续的.
\end{corollary}

\begin{example}
求:\(\lim_{x\to0}\frac{\log_a (1+x)}{x}\).
\begin{solution}
\(
\lim_{x\to0}\frac{\log_a (1+x)}{x}
= \lim_{x\to0}\log_a (1+x)^{\frac{1}{x}}
= \log_a \lim_{x\to0}(1+x)^{\frac{1}{x}}
= \log_a e
= \frac{1}{\ln a}.
\)
\end{solution}
\end{example}

\begin{example}
求:\(\lim_{x\to0}\frac{a^x - 1}{x}\).
\begin{solution}
\(
\lim_{x\to0}\frac{a^x - 1}{x}
\xlongequal{t=a^x-1} \lim_{t\to0}\frac{t}{\log_a (1+t)}
= \ln a.
\)
\end{solution}
\end{example}

\begin{proposition}
%@see: 《高等数学(第六版 上册)》 P69 习题1-9 2.
设函数\(f(x)\)和\(g(x)\)在点\(x_0\)连续,
则\[
	\phi(x)=\max\{f(x),g(x)\}, \qquad
	\psi(x)=\min\{f(x),g(x)\}
\]在点\(x_0\)也连续.
\begin{proof}
因为\[
	\phi(x)=\frac{f(x)+g(x)}{2}+\frac{\abs{f(x)-g(x)}}{2}, \qquad
	\psi(x)=\frac{f(x)+g(x)}{2}-\frac{\abs{f(x)-g(x)}}{2},
\]
所以\(\phi(x)\)和\(\psi(x)\)在点\(x_0\)也连续.
\end{proof}
\end{proposition}

\begin{theorem}\label{theorem:极限.连续函数的极限7}
对于幂指函数\(y = u(x)^{v(x)}\ (u(x) > 0\)且\(u(x) \not\equiv 1)\),如果\(\lim u(x) = a > 0\),\(\lim v(x) = b\),那么\[
\lim u(x)^{v(x)} = a^b.
\]
\end{theorem}

\begin{example}
\def\l{\lim_{n\to\infty}}
计算:\(\l \sin^2(\pi\sqrt{n^2+3n})\).
\begin{solution}
\def\a{\pi(\sqrt{n^2+3n}-\sqrt{n^2})}
显然有\begin{align*}
&\l \sin^2(\pi\sqrt{n^2+3n}) \\
&= \l \sin^2[n\pi+\a] \\
&= \l \left\{ \sin(n\pi) \cos\left[\a\right] + \cos(n\pi) \sin\left[\a\right] \right\}^2 \\
&= \l \sin^2\left[\a\right] \\
&= \l \sin^2\left( \pi \cdot \frac{3n}{\sqrt{n^2+3n}+n} \right) \\
&= \sin^2 \left( \pi \cdot \l \frac{3n}{\sqrt{n^2+3n}+n} \right) \\
&= \sin^2 \frac{3\pi}{2}
= 1.
\end{align*}
\end{solution}
\end{example}

\begin{example}
设\(x_1=10\),\(x_{n+1}=\sqrt{6+x_n}\ (n=1,2,\dotsc)\).
试证:数列\(\{x_n\}\)极限存在,并求此极限.
\begin{solution}
由\(x_1=10\),
\(x_2=\sqrt{6+x_1}=\sqrt{16}=4\),
可知\(x_1>x_2\).
因此我们猜想数列\(\{x_n\}\)是单调递减数列.
假设\(n=k\)时有\(x_k>x_{k+1}\)成立,
那么\[
	x_{k+1}=\sqrt{6+x_k}>\sqrt{6+x_{k+1}}=x_{k+2},
\]
所以利用数学归纳法便知\(x_n>x_{n+1}\ (n=1,2,\dotsc)\).

同理,由于\(x_{n+1}=\sqrt{6+x_n}\),
所以利用数学归纳法便知\(x_n>0\ (n=1,2,\dotsc)\).

根据\hyperref[theorem:极限.数列的单调有界定理]{单调有界定理},
数列\(\{x_n\}\)有极限.

设\(\lim_{n\to\infty}x_n=x\).
对\(x_{n+1}=\sqrt{6+x_n}\)两边取极限,
得\[
	\lim_{n\to\infty}x_{n+1}=x, \qquad
	\lim_{n\to\infty}\sqrt{6+x_n}
	=\sqrt{6+\lim_{n\to\infty}x_n}
	=\sqrt{6+x},
\]
于是\(x=\sqrt{6+x}\),
从而\(x^2-x-6=0\),
\((x-3)(x+2)=0\),
解得\(x=3\)或\(x=-2\)(舍去).
这就是说\(\lim_{n\to\infty}x_n=3\).
\end{solution}
\end{example}
