\section{极限的运算法则}\label{section:极限.极限的运算法则}
\subsection{无穷小的四则运算法则}
\begin{theorem}
%@see: 《高等数学(第六版 上册)》 P43 定理1
有限个无穷小的和也是无穷小.
\begin{proof}
考虑两个无穷小的和.设\(\alpha\)和\(\beta\)是当\(x \to x_0\)时的两个无穷小,而\(\gamma = \alpha+\beta\).
对于\(\forall\epsilon>0\),因为\(\alpha\)是当\(x \to x_0\)时的无穷小,对于\(\frac{\epsilon}{2}>0\),\(\exists \delta_1 > 0\),当\(0<\abs{x-x_0}<\delta_1\)时,不等式\(\abs{\alpha}<\frac{\epsilon}{2}\)成立.同样地,对于\(\frac{\epsilon}{2}>0\),\(\exists \delta_2 > 0\),当\(0<\abs{x-x_0}<\delta_2\)时,不等式\(\abs{\beta}<\frac{\epsilon}{2}\)成立.取\(\delta=\min\{\delta_1,\delta_2\}\),则当\(0<\abs{x-x_0}<\delta\)时,不等式\(\abs{\alpha}<\frac{\epsilon}{2}\)和\(\abs{\beta}<\frac{\epsilon}{2}\)同时成立,从而\(\abs{\gamma}=\abs{\alpha+\beta}\leq\abs{\alpha}+\abs{\beta}<\frac{\epsilon}{2}+\frac{\epsilon}{2}=\epsilon\).这就证明了\(\gamma\)也是当\(x \to x_0\)时的无穷小.
\end{proof}
\end{theorem}

\begin{theorem}
%@see: 《高等数学(第六版 上册)》 P43 定理2
有界函数与无穷小的乘积是无穷小.
\begin{proof}
设函数\(u\)在\(x_0\)的某一去心邻域\(\mathring{U}(x_0,\delta_1)\)内是有界的,
即对\(\forall x\in\mathring{U}(x_0,\delta_1)\),
\(\exists M>0\)使\(\abs{u} \leq M\)成立.
又设\(\alpha\)是当\(x \to x_0\)时的无穷小,
即\(\forall \epsilon > 0\),
\(\exists \delta_2 > 0\),
当\(x\in\mathring{U}(x_0,\delta_2)\)时,
有\(\abs{\alpha}<\frac{\epsilon}{M}\).
取\(\delta=\min\{\delta_1,\delta_2\}\),
则当\(x\in\mathring{U}(x_0,\delta)\)时,
不等式\(\abs{u} \leq M\)和\(\abs{\alpha} < \frac{\epsilon}{M}\)同时成立,
从而\(\abs{u \alpha} = \abs{u}\abs{\alpha} < M \cdot \frac{\epsilon}{M} = \epsilon\),
这就证明了\(u \alpha\)是当\(x \to x_0\)时的无穷小.
\end{proof}
\end{theorem}

\begin{corollary}
%@see: 《高等数学(第六版 上册)》 P44 推论1
常数与无穷小的乘积是无穷小.
\end{corollary}

\begin{corollary}
%@see: 《高等数学(第六版 上册)》 P44 推论2
有限个无穷小的乘积也是无穷小.
\end{corollary}
