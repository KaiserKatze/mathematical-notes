\section{极限的运算法则}\label{section:极限.极限的运算法则}
\subsection{无穷小的四则运算法则}
\begin{theorem}
%@see: 《高等数学(第六版 上册)》 P43 定理1
有限个无穷小的和也是无穷小.
\begin{proof}
考虑两个无穷小的和.设\(\alpha\)和\(\beta\)是当\(x \to x_0\)时的两个无穷小,而\(\gamma = \alpha+\beta\).
对于\(\forall\epsilon>0\),因为\(\alpha\)是当\(x \to x_0\)时的无穷小,对于\(\frac{\epsilon}{2}>0\),\(\exists \delta_1 > 0\),当\(0<\abs{x-x_0}<\delta_1\)时,不等式\(\abs{\alpha}<\frac{\epsilon}{2}\)成立.同样地,对于\(\frac{\epsilon}{2}>0\),\(\exists \delta_2 > 0\),当\(0<\abs{x-x_0}<\delta_2\)时,不等式\(\abs{\beta}<\frac{\epsilon}{2}\)成立.取\(\delta=\min\{\delta_1,\delta_2\}\),则当\(0<\abs{x-x_0}<\delta\)时,不等式\(\abs{\alpha}<\frac{\epsilon}{2}\)和\(\abs{\beta}<\frac{\epsilon}{2}\)同时成立,从而\(\abs{\gamma}=\abs{\alpha+\beta}\leq\abs{\alpha}+\abs{\beta}<\frac{\epsilon}{2}+\frac{\epsilon}{2}=\epsilon\).这就证明了\(\gamma\)也是当\(x \to x_0\)时的无穷小.
\end{proof}
\end{theorem}

\begin{theorem}
%@see: 《高等数学(第六版 上册)》 P43 定理2
有界函数与无穷小的乘积是无穷小.
\begin{proof}
设函数\(u\)在\(x_0\)的某一去心邻域\(\mathring{U}(x_0,\delta_1)\)内是有界的,
即对\(\forall x\in\mathring{U}(x_0,\delta_1)\),
\(\exists M>0\)使\(\abs{u} \leq M\)成立.
又设\(\alpha\)是当\(x \to x_0\)时的无穷小,
即\(\forall \epsilon > 0\),
\(\exists \delta_2 > 0\),
当\(x\in\mathring{U}(x_0,\delta_2)\)时,
有\(\abs{\alpha}<\frac{\epsilon}{M}\).
取\(\delta=\min\{\delta_1,\delta_2\}\),
则当\(x\in\mathring{U}(x_0,\delta)\)时,
不等式\(\abs{u} \leq M\)和\(\abs{\alpha} < \frac{\epsilon}{M}\)同时成立,
从而\(\abs{u \alpha} = \abs{u}\abs{\alpha} < M \cdot \frac{\epsilon}{M} = \epsilon\),
这就证明了\(u \alpha\)是当\(x \to x_0\)时的无穷小.
\end{proof}
\end{theorem}

\begin{corollary}
%@see: 《高等数学(第六版 上册)》 P44 推论1
常数与无穷小的乘积是无穷小.
\end{corollary}

\begin{corollary}
%@see: 《高等数学(第六版 上册)》 P44 推论2
有限个无穷小的乘积也是无穷小.
\end{corollary}

\subsection{函数极限的四则运算法则}

\subsection{复合函数的极限运算法则}
\begin{theorem}
设函数\(y=f[g(x)]\)是由函数\(u=g(x)\)与函数\(y=f(u)\)复合而成.
\begin{enumerate}
	\item 设\(f[g(x)]\)在点\(x_0\)的某去心邻域内有定义.
	\begin{enumerate}[label={\rm(\roman*)}]
		\item 若\(\lim_{x \to x_0} g(x) = u_0\),
		\(\lim_{u \to u_0} f(u) = A\),
		且存在\(\delta_0 > 0\),
		当\(x \in \mathring{U}(x_0,\,\delta_0)\)时,
		有\(g(x) \neq u_0\),
		则\[
			\lim_{x \to x_0} f[g(x)]
			= \lim_{u \to u_0} f(u) = A.
		\]
		\item 若\(\lim_{x \to x_0}g(x) = \infty\),
		\(\lim_{u \to \infty}f(u) = A\),
		则\[
			\lim_{x \to x_0} f[g(x)]
			= \lim_{u \to \infty} f(u) = A.
		\]
	\end{enumerate}

	\item 设\(f[g(x)]\)在\(\abs{x}\)大于某一正数时有定义.
	\begin{enumerate}[label={\rm(\roman*)}]
	\item 若\(\lim_{x \to \infty} g(x) = u_0\),
	\(\lim_{u \to u_0} f(u) = A\),
	且存在\(\delta_0 > 0\),
	当\(x \in \mathring{U}(x_0,\,\delta_0)\)时,
	有\(g(x) \neq u_0\),
	则\[
		\lim_{x \to \infty} f[g(x)]
		= \lim_{u \to u_0} f(u) = A.
	\]

	\item 若\(\lim_{x \to \infty}g(x) = \infty\),
	\(\lim_{u \to \infty}f(u) = A\),
	则\[
		\lim_{x \to \infty} f[g(x)]
		= \lim_{u \to \infty} f(u) = A.
	\]
	\end{enumerate}
\end{enumerate}
\begin{proof}
我们只证1-(a).
按函数极限的定义,
要证\[
	(\forall\epsilon>0)
	(\exists\delta>0)
	[
		0 < \abs{x - x_0} < \delta
		\implies
		\abs{ f[g(x)] - A } < \epsilon
	]
\]成立.

由于\(\lim_{u \to u_0} f(u) = A\),
那么\[
	(\forall\epsilon>0)
	(\exists\eta>0)
	[
		0 < \abs{u - u_0} < \eta
		\implies
		\abs{ f(u) - A } < \epsilon
	].
\]

又由于\(\lim_{x \to x_0} g(x) = u_0\),
对于上面得到的\(\eta > 0\),
\(\exists \delta_1 > 0\),
当\(0 < \abs{x - x_0} < \delta_1\)时,
\(\abs{ g(x) - u_0 } < \eta\)成立.

由假设,当\(x \in \mathring{U}(x_0,\delta_0)\)时,
\(g(x) \neq u_0\).
取\(\delta = \min\{\delta_0,\delta_1\}\),
则当\(0 < \abs{x - x_0} < \delta\)时,
\(\abs{ g(x) - u_0 } < \eta\)及\(\abs{ g(x) - u_0 } \neq 0\)同时成立,
即\(0 < \abs{ g(x) - u_0 } < \eta\)成立,
从而\[
	\abs{ f[g(x)] - A } = \abs{ f(u) - A } < \epsilon
\]成立.
\end{proof}
\end{theorem}
