\section{本章总结}

利用重积分定义计算极限(假设\(f\)可积):\begin{gather*}
	\lim_{n\to\infty} \frac1n \sum_{k=1}^n f\left( \frac{k}{n} \right)
	= \int_0^1 f(x) \dd{x}, \\
	\lim_{n\to\infty} \frac{b-a}{n} \sum_{k=1}^n f\left( \frac{(n-k)a+kb}{n} \right)
	= \int_a^b f(x) \dd{x}. \\
	\lim_{n\to\infty} \frac1{n^2} \sum_{i=1}^n \sum_{j=1}^n f\left( \frac{i}{n},\frac{j}{n} \right)
	= \int_0^1 \dd{x} \int_0^1 f(x,y) \dd{y}. \\
	\lim_{n\to\infty} \frac1{n^2} \sum_{i=1}^n \sum_{j=1}^i f\left( \frac{i}{n},\frac{j}{n} \right)
	= \int_0^1 \dd{x} \int_0^x f(x,y) \dd{y}.
\end{gather*}

物体质心坐标公式:\begin{equation*}
	\frac1M \iiint_\Omega \vb{r} \rho \dd{v},
\end{equation*}
其中\(\Omega\)是物体占有空间有界闭区域,
\(\rho\)是物体的质量体密度,
\(\dd{v}\)是体积元素,
\(M=\iiint_\Omega \rho \dd{v}\)是物体质量,
\(\vb{r}\)是物体内任意点的位矢.

\begin{table}[htp]
	\centering
	\begin{tblr}{*4{c|}c}
		\hline
		被积函数 & 偏导函数 & 上下限函数 & \SetCell[c=2]{c} 含参积分 \\ \hline
		\(f(x,y)\) & \(f'_x(x,y)\) & \(\alpha(x),\beta(x)\)
			& \(\phi(x)=\int_c^d f(x,y) \dd{y}\)
			& \(\Phi(x)=\int_{\alpha(x)}^{\beta(x)} f(x,y) \dd{y}\) \\
		\hline
		连续 & & & 连续 & \\ \hline
		连续 & 连续 & & 可微 & \\ \hline
		连续 & & 连续 & & 连续 \\ \hline
		连续 & 连续 & 可微 & & 可微 \\
		\hline
	\end{tblr}
\end{table}
