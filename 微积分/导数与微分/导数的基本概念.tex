\section{导数的基本概念}
\subsection{导数的定义}
\begin{definition}\label{definition:导数.函数在一点的可导性}
%@see: 《高等数学(第六版 上册)》 P79 定义
%@see: 《数学分析(第二版 上册)》(陈纪修) P122 定义4.1.2
设\(X\subseteq\mathbb{R}\),
函数\(f\in\mathbb{R}^X\)在点\(x_0\)的某个邻域\(U(x_0)\)内有定义,
\(x_0 + \increment x \in U(x_0)\).
如果函数增量\(\increment y = f(x_0 + \increment x) - f(x_0)\)
与\(\increment x\)之比\(\frac{\increment y}{\increment x}\)
当\(\increment x\to0\)时的极限\[
	\lim_{\increment x \to 0} \frac{\increment y}{\increment x}
\]存在,
则称“函数\(f\)在点\(x_0\)~\DefineConcept{可导}%
(\(f\) is \emph{differentiable} at \(x_0\))”
“\(f\)在点\(x_0\)具有导数”
“\(f\)在点\(x_0\)的导数存在”
“点\(x_0\)是\(f\)的\DefineConcept{可导点}”,
把这个极限称为“函数\(f\)在点\(x_0\)的\DefineConcept{导数}%
(the \emph{derivative} of \(f\) at \(x_0\))”,
记为\[
	f'(x_0), \qquad
	\eval{y'}_{x=x_0}, \qquad
	\eval{\dv{y}{x}}_{x=x_0}, \qquad
	\eval{\dv{f(x)}{x}}_{x=x_0},
	\quad\text{或}\quad
	\dv{f(x_0)}{x},
\]
即\begin{equation}
	f'(x_0)
	\defeq
	\lim_{\increment x\to0} \frac{f(x_0 + \increment x)-f(x_0)}{\increment x}.
\end{equation}

如果极限\[
	\lim_{\increment x \to 0} \frac{\increment y}{\increment x}
\]不存在,
则称“函数\(f\)在点\(x_0\)~\DefineConcept{不可导}”
“点\(x_0\)是\(f\)的\DefineConcept{不可导点}”.
%@see: https://mathworld.wolfram.com/Derivative.html
\end{definition}

\begin{example}\label{example:导数.导数定义式的变形}
%@see: 《数学分析(第二版 上册)》(陈纪修) P131 习题 1.
设函数\(f\)在点\(x_0\)可导.
计算极限\begin{itemize}
	\item \(\lim_{h\to0} \frac{f(x_0 - h) - f(x_0)}{h}\);
	\item \(\lim_{x \to x_0} \frac{f(x) - f(x_0)}{x - x_0}\);
	\item \(\lim_{h\to0} \frac{f(x_0+h) - f(x_0-h)}{h}\).
\end{itemize}
\begin{solution}
直接计算得\begin{gather*}
	\lim_{h\to0} \frac{f(x_0 - h) - f(x_0)}{h}
	= - \lim_{h\to0} \frac{f(x_0 + (-h)) - f(x_0)}{(-h)}
	= - f'(x_0), \\
	\lim_{x \to x_0} \frac{f(x) - f(x_0)}{x - x_0}
	\xlongequal{t = x - x_0}
	\lim_{t \to 0} \frac{f(x_0 + (x - x_0)) - f(x_0)}{t}
	= f'(x_0), \\
	\lim_{h\to0} \frac{f(x_0+h) - f(x_0-h)}{h}
	= \lim_{h\to0} \frac{f(x_0+h) - f(x_0)}{h}
	- \lim_{h\to0} \frac{f(x_0-h) - f(x_0)}{h}
	= 2 f'(x_0).
\end{gather*}
\end{solution}
\end{example}

\begin{proposition}
设\(f\colon\mathbb{R}\to\mathbb{R}\)在点\(x\)可导,
\(\lambda>0\)是常数,
则\[
	\lim_{h\to0} \frac{f(x+\lambda h)-f(x)}{h}
	= \lambda f'(x).
\]
\begin{proof}
直接计算得\begin{align*}
	\lim_{h\to0} \frac{f(x+\lambda h)-f(x)}{h}
	&= \lambda \lim_{h\to0} \frac{f(x+\lambda h)-f(x)}{\lambda h} \\
	&\xlongequal{t=\lambda h}
		\lambda \lim_{t\to0} \frac{f(x+t)-f(x)}{t}
	= \lambda f'(x).
	\qedhere
\end{align*}
\end{proof}
\end{proposition}

\subsection{单侧导数和狄尼导数的定义}
\begin{definition}
%@see: 《数学分析教程(第3版 上册)》(史济怀) P125 定义3.1.2
设\(X\subseteq\mathbb{R}\),
函数\(f\in\mathbb{R}^X\).
\begin{itemize}
	\item 如果\(f\)在点\(x_0\)的某个左邻域内有定义,且极限\[
		\lim_{h\to0^-} \frac{f(x_0+h)-f(x_0)}{h}
	\]存在且有限,
	那么把这个极限称为
	“函数\(f\)在点\(x_0\)的\DefineConcept{左导数}(left-sided derivative)”,
	记作\(f'_-(x_0)\).
	\item 如果\(f\)在点\(x_0\)的某个右邻域内有定义,且极限\[
		\lim_{h\to0^+} \frac{f(x_0+h)-f(x_0)}{h}
	\]存在且有限,
	那么把这个极限称为
	“函数\(f\)在点\(x_0\)的\DefineConcept{右导数}(right-sided derivative)”,
	记作\(f'_+(x_0)\).
\end{itemize}
左导数和右导数统称\DefineConcept{单侧导数}(one-sided derivative).
\end{definition}

\begin{definition}
设函数\(f\colon D\to\mathbb{R}\)在点\(x_0\)的某个去心邻域中有定义.
\begin{itemize}
	\item 把\[
		\lim_{\delta\to0^+} \sup_{0<x-x_0<\delta} f(x)
	\]称为“函数\(f\)在点\(x_0\)的\DefineConcept{右上导数}(upper right Dini derivative)”.
	\item 把\[
		\lim_{\delta\to0^+} \inf_{0<x-x_0<\delta} f(x)
	\]称为“函数\(f\)在点\(x_0\)的\DefineConcept{右下导数}(lower right Dini derivative)”.
	\item 把\[
		\lim_{\delta\to0^+} \sup_{-\delta<x-x_0<0} f(x)
	\]称为“函数\(f\)在点\(x_0\)的\DefineConcept{左上导数}(upper left Dini derivative)”.
	\item 把\[
		\lim_{\delta\to0^+} \inf_{-\delta<x-x_0<0} f(x)
	\]称为“函数\(f\)在点\(x_0\)的\DefineConcept{左下导数}(lower left Dini derivative)”.
\end{itemize}
右上导数、右下导数、左上导数和左下导数统称为\DefineConcept{狄尼导数}(Dini derivative).
%@see: https://mathworld.wolfram.com/DiniDerivative.html
%@see: https://people.math.sc.edu/schep/diffmonotone.pdf
\end{definition}

\begin{theorem}[导数存在的充分必要条件]\label{theorem:导数.函数在一点的可导性及其单侧可导性的关系}
%@see: 《数学分析教程(第3版 上册)》(史济怀) P125
函数\(f\)在点\(x_0\)可导的充分必要条件是:
其左导数\(f'_-(x_0)\)和右导数\(f'_+(x_0)\)都存在且相等.
%TODO proof
%\cref{theorem:函数极限.极限与单侧极限的关系1}
\end{theorem}

\begin{definition}\label{definition:导数.函数在开区间内可导}
%@see: 《数学分析教程(第3版 上册)》(史济怀) P127 定义3.1.3
如果函数\(f\colon(a,b)\to\mathbb{R}\)在开区间\((a,b)\)内的每一个点可导,
就称“函数\(f\)在开区间\((a,b)\)内可导”.
\end{definition}

\begin{definition}\label{definition:导数.函数在闭区间上可导}
%@see: 《数学分析教程(第3版 上册)》(史济怀) P127 定义3.1.3
如果函数\(f\colon[a,b]\to\mathbb{R}\)在开区间\((a,b)\)内可导,
且在点\(a\)的右导数\(f'_+(a)\)及在点\(b\)的左导数\(f'_-(b)\)都存在,
就说“函数\(f\)在闭区间\([a,b]\)上可导”.
\end{definition}

类似地,可以定义\(f\)在\([a,b)\)与\((a,b]\)上可导.

\begin{example}
%@see: 《2016年全国硕士研究生入学统一考试(数学一)》一选择题/第4题
设函数\[
	f(x) = \left\{ \begin{array}{cl}
		x, & x\leq0, \\
		1/n, & 1/(n+1)<x\leq1/n,\,n=1,2,\dotsc.
	\end{array} \right.
\]
考察函数\(f\)在点\(x=0\)的连续性和可导性.
\begin{solution}
显然\(f\)在点\(x=0\)左连续.
对于任意\(\epsilon>0\),存在正整数\(n_0\)满足\(\frac1{n_0}<\epsilon\),
取\(\delta=\frac1{n_0}\),
当\(0 \leq x < \delta\)时,
成立\[
	\abs{f(x) - f(0)}
	= f(x)
	\leq \frac1{n_0}
	< \epsilon,
\]
这就说明函数\(f\)在点\(x=0\)右连续.
由\cref{theorem:极限.函数在一点的连续性及其单侧连续性的关系} 可知\(f\)在点\(x=0\)连续.

显然\(f\)在点\(x=0\)的左导数存在:\[
	f'_-(0) = \lim_{x\to0^-} \frac{f(x) - f(0)}{x - 0}
	= \lim_{x\to0^-} \frac{x}{x}
	= 1.
\]
当\(\frac1{n+1} < x \leq \frac1n\)时,
有\[
	f(x) = \frac1n,
	\qquad
	n \leq \frac1x < n+1,
\]
从而有\[
	1 \leq \frac{f(x)}{x} < \frac{n+1}{n}.
\]
因为当\(n\to\infty\)时有\(x\to0^+\)和\(\frac{n+1}{n}\to1\),
所以由\cref{theorem:数列极限.夹逼准则} 可得\[
	\frac{f(x)}{x} \to 1
	\quad(n\to\infty).
\]
于是\(f\)在点\(x=0\)的右导数为\[
	f'_+(0) = \lim_{x\to0^+} \frac{f(x) - f(0)}{x - 0}
	= \lim_{x\to0^+} \frac{f(x)}{x}
	= 1.
\]
由\cref{theorem:导数.函数在一点的可导性及其单侧可导性的关系} 可知\(f\)在点\(x=0\)可导.
\end{solution}
%@Mathematica: Plot[Piecewise[{{1/Floor[1/x], 1/(Floor[1/x] + 1) < x <= 1/Floor[1/x] && x > 0}, {x, x <= 0}}], {x, -.5, .5}]
\end{example}

\subsection{导函数的定义}
设函数\(f\colon I\to\mathbb{R}\)在区间\(I\)内的每一个点可导,
那么,对于\(\forall x_0 \in I\),
都对应着\(f(x_0)\)的一个确定的导数值\(f'(x_0)\).
这样就构成一个新的函数.

\begin{definition}
设函数\(f\colon I\to\mathbb{R}\)在区间\(I\)内的每一个点可导,
则\(I\)中每一点\(x_0\)与其相应的\(f\)在点\(x_0\)的导数\(f'(x_0)\)的关系\[
	\Set{ (x_0,f'(x_0)) \given x_0 \in I }
\]称为“函数\(f\)的\DefineConcept{导函数}(derivative function)”,
简称\DefineConcept{导数},
记作\(f'\)或\(\dv{f}{x}\)或\(\dv{x} f\).
\end{definition}

\begin{definition}\label{definition:函数族.可导函数族}
由区间\(I\)上全部的可导函数组成的集合,称作\DefineConcept{可导函数族},
记作\(D(I)\)\footnote{当\(I=(a,b)\)时,可将\(D(I)\)改写为\(D(a,b)\).
当\(I=[a,b]\)时,可将\(D(I)\)改写为\(D[a,b]\).
以此类推.},
即\[
	D(I)
	\defeq
	\Set*{
		f\in\mathbb{R}^I
		\given
		(\forall x \in I)
		[\text{\(f\)在点\(x\)可导}]
	}.
\]
\end{definition}

\begin{example}%\label{example:导数.常数函数的导数}
%@see: 《高等数学(第六版 上册)》 P81 例1
求函数\(f(x) = C\)的导数,其中\(C\)为常数.
\begin{solution}
\(f'(x)
= \lim_{h\to0} \frac{f(x+h)-f(x)}{h}
= \lim_{h\to0} \frac{C-C}{h}
= 0\).
\end{solution}
\end{example}

\begin{example}%\label{example:导数.幂函数的导数}
%@see: 《高等数学(第六版 上册)》 P81 例2
求函数\(f(x) = x^n\)在\(x=a\)处的导数,
其中\(a\in\mathbb{R},
n\in\mathbb{N}^+\).
\begin{solution}
\(f'(a)
= \lim_{x \to a} \frac{x^n-a^n}{x-a}
= \lim_{x \to a} (x^{n-1}+ax^{n-2}+\dotsb+a^{n-1})
= na^{n-1}\).
\end{solution}
\end{example}

更一般地,对于幂函数\(y=x^{\mu}\ (\mu\in\mathbb{R})\),
有\begin{equation}
	(x^{\mu})' = \mu x^{\mu-1}.
\end{equation}

\begin{example}%\label{example:导数.正弦函数的导数}
%@see: 《高等数学(第六版 上册)》 P81 例3
求函数\(f(x) = \sin x\)的导数.
\begin{solution}
\(f'(x) = \lim_{h\to0} \frac{\sin(x+h)-\sin x}{h}
= \lim_{h\to0} \cos(x+\frac{h}{2}) \frac{\sin(h/2)}{h/2}
= \cos x\).
\end{solution}
\end{example}

\begin{example}%\label{example:导数.余弦函数的导数}
求函数\(f(x) = \cos x\)的导数.
\begin{solution}
\(f'(x) = \lim_{h\to0} \frac{\cos(x+h)-\cos x}{h}
= - \lim_{h\to0} \sin\left(x+\frac{h}2\right) \frac{\sin(h/2)}{h/2}
= - \sin x\).
\end{solution}
\end{example}

\begin{example}%\label{example:导数.指数函数的导数}
%@see: 《高等数学(第六版 上册)》 P82 例4
求函数\(f(x) = a^x\ (a>0,a\neq1)\)的导数.
\begin{solution}
\(f'(x)
= \lim_{h\to0}\frac{a^{x+h}-a^x}{h}
= a^x \lim_{h\to0}\frac{a^h-1}{h}
= a^x \ln a\).
\end{solution}
\end{example}
\begin{remark}
特别地,当\(a=e\)时,因\(\ln e = 1\),故有\[
	(e^x)' = e^x.
\]
\end{remark}

\begin{example}%\label{example:导数.对数函数的导数}
%@see: 《高等数学(第六版 上册)》 P82 例5
求函数\(f(x) = \log_a x\ (a>0,a\neq1)\)的导数.
\begin{solution}
\(f'(x)
= \lim_{h\to0}\frac{\log_a(x+h)-\log_a x}{h}
= \lim_{h\to0}{\frac{1}{h} \log_a\frac{x+h}{x}}
= \frac{1}{x} \lim_{h\to0}\frac{\log_a(1+h/x)}{h/x}
= \frac{1}{x \ln a}\).
\end{solution}
\end{example}

\begin{example}
%@see: 《高等数学(第六版 上册)》 P82 例6
求函数\(f(x) = \abs{x}\)在\(x=0\)处的导数.
\begin{solution}
\(\frac{f(0+h)-f(0)}{h} = \frac{\abs{h}-0}{h} = \frac{\abs{h}}{h}\).

当\(h < 0\)时,\(\frac{\abs{h}}{h} = -1\),
故\(\lim_{h\to0^-}\frac{f(0+h)-f(0)}{h}
= \lim_{h\to0^-}\frac{\abs{h}}{h} = -1\).

当\(h > 0\)时,\(\frac{\abs{h}}{h} = 1\),
故\(\lim_{h\to0^+}\frac{f(0+h)-f(0)}{h}
= \lim_{h\to0^+}\frac{\abs{h}}{h} = 1\).

综上,\(\lim_{h\to0}\frac{f(0+h)-f(0)}{h}\)不存在,即函数\(f(x) = \abs{x}\)在\(x = 0\)处不可导.
\end{solution}
\end{example}

\subsection{导数的几何意义}
曲线\(y=f(x)\)在点\(M(x_0,y_0)\)处的\DefineConcept{切线方程}为\[
	y-y_0=f'(x_0)(x-x_0).
\]

过切点\(M(x_0,y_0)\)且与切线垂直的直线叫做曲线\(y=f(x)\)在点\(M\)处的\DefineConcept{法线}.
如果\(f'(x_0) \neq 0\),则法线的斜率为\(-\frac{1}{f'(x_0)}\),从而法线方程为\[
	y-y_0=-\frac{1}{f'(x_0)}(x-x_0);
\]
而如果\(f'(x_0) = 0\),则法线方程为\(x = x_0\).

\begin{proposition}\label{theorem:导数与微分.导函数的奇偶性}
设\(R>0\),\(f'\)是函数\(f\colon(-R,R)\to\mathbb{R}\)的导函数.
\begin{itemize}
	\item 如果\(f\)是奇函数,那么\(f'\)是偶函数.
	\item 如果\(f\)是偶函数,那么\(f'\)是奇函数.
\end{itemize}
\begin{proof}
由导数的定义有\begin{equation*}
	f'(-x)
	= \lim_{h\to0} \frac{f(-x+h)-f(-x)}{h}.
	\eqno(1)
\end{equation*}
由\cref{example:导数.导数定义式的变形} 有\begin{equation*}
	-f'(x)
	= \lim_{h\to0} \frac{f(x-h)-f(x)}{h}.
	\eqno(2)
\end{equation*}

假设\(f\)是奇函数,
即\begin{equation*}
	(\forall x)
	[
		-R < x < R
		\implies
		f(-x) = -f(x)
	],
\end{equation*}
那么\(f(-x+h) = -f(x-h)\),
再由(1)(2)两式可知\begin{equation*}
	f'(-x)
	= \lim_{h\to0} \frac{-f(x-h)+f(x)}{h}
	= -\lim_{h\to0} \frac{f(x-h)-f(x)}{h}
	= f'(x).
\end{equation*}
这就说明\(f'\)是偶函数.

假设\(f\)是偶函数,
即\begin{equation*}
	(\forall x)
	[
		-R < x < R
		\implies
		f(-x) = f(x)
	],
\end{equation*}
那么\(f(-x+h) = f(x-h)\),
再由(1)(2)两式可知\[
	f'(-x)
	= \lim_{h\to0} \frac{f(x-h)-f(x)}{h}
	= -f'(x).
\]
这就说明\(f'\)是奇函数.
\end{proof}
\end{proposition}
\begin{proposition}\label{theorem:导数与微分.导函数的周期性}%周期函数的导数也是周期函数且周期相同
以\(T\)为周期的可导函数\(f\colon\mathbb{R}\to\mathbb{R}\)的导函数也是以\(T\)为周期的函数.
\begin{proof}
%@see: https://www.bilibili.com/video/BV1sg4y1i7Ak/
由给定条件有\[
	f(x+T) = f(x),
\]
那么对于\(\forall h\in\mathbb{R}\)成立\[
	f(x+T+h) = f(x+h),
\]
于是\begin{equation*}
	f'(x+T)
	= \lim_{h\to0} \frac{f(x+T+h) - f(x+T)}{h}
	= \lim_{h\to0} \frac{f(x+h) - f(x)}{h}
	= f'(x).
	\qedhere
\end{equation*}
\end{proof}
%\cref{theorem:定积分.周期函数的积分}
\end{proposition}

\subsection{函数可导性与连续性的关系}
\begin{theorem}\label{theorem:导数与微分.函数可导性与连续性的关系}
%@see: 《数学分析教程(第3版 上册)》(史济怀) P125 定理3.1.1
如果函数\(f\)在点\(x_0\)可导,
则\(f\)必定在点\(x_0\)连续.
\begin{proof}
记\(f\)在点\(x_0\)的导数为\(f'(x_0)\),
于是由\begin{align*}
	\lim_{x \to x_0} (f(x) - f(x_0))
	&= \lim_{x \to x_0} \frac{f(x) - f(x_0)}{x - x_0} \cdot (x - x_0) \\
	&= \lim_{x \to x_0} \frac{f(x) - f(x_0)}{x - x_0} \cdot \lim_{x \to x_0} (x - x_0) \\
	&= f'(x_0) \cdot 0
	= 0,
\end{align*}
可知\(\lim_{x \to x_0} f(x) = f(x_0)\),
这就说明\(f\)在点\(x_0\)连续.
\end{proof}
\end{theorem}
借用\cref{definition:函数族.连续函数族} 和\cref{definition:函数族.可导函数族} 的记号,
可以将\cref{theorem:导数与微分.函数可导性与连续性的关系} 描述为:\begin{equation*}
	% f \in D(I) \implies f \in C(I)
	D(I) \subseteq C(I).
\end{equation*}

\begin{example}
设函数\(f\)在点\(x=0\)可导,
且\(f\left( \frac1n \right) = \frac2n\ (n=1,2,\dotsc)\).
求\(f'(0)\).
\begin{solution}
由\cref{theorem:导数与微分.函数可导性与连续性的关系} 可知\(f\)在点\(x=0\)连续,
而\(\lim_{x\to0^+} f(x)
= \lim_{n\to\infty} f\left( \frac1n \right)
= 0\),
%\cref{definition:极限.函数在一点的连续性}
所以\(f(0) = 0\),
%\cref{definition:导数.函数在一点的可导性}
%\cref{theorem:极限.海涅定理}
那么\begin{equation*}
	f'(0) = \lim_{x\to0} \frac{f(x) - f(0)}{x - 0}
	= \lim_{n\to\infty} \frac{f(1/n)}{1/n}
	= \lim_{n\to\infty} \frac{2/n}{1/n}
	= 2.
\end{equation*}
\end{solution}
\end{example}

\begin{example}\label{example:导数.函数在一点的导数是无穷大}
函数\(y=f(x)=\sqrt[3]x\)
在区间\((-\infty,+\infty)\)内连续,
但是在点\(x=0\)处不可导.
这是因为在点\(x=0\)处有\[
	\frac{f(0+h)-f(0)}{h}
	=\frac{\sqrt[3]{h}-0}{h}
	=\frac{1}{h^{2/3}}>0,
\]
因而,
\(\lim_{h\to0} \frac{f(0+h)-f(0)}{h}
=\lim_{h\to0} \frac{1}{h^{2/3}}
=\infty\),
即导数为无穷大(导数不存在).
这事实在图形中表现为:
曲线\(y=\sqrt[3]x\)在原点具有垂直于\(x\)轴的切线\(x=0\).
\end{example}

\begin{example}\label{example:导数.函数在一点的左右导数不相等}
函数\(y=\sqrt{x^2}\)(即\(y=\abs{x}\))
在\((-\infty,+\infty)\)内连续,
但是在\(x=0\)处不可导,
曲线\(y=\sqrt{x^2}\)在原点没有切线.
这是因为\[
	\lim_{x\to0^+} \frac{f(x)-f(0)}{x-0}
	= \lim_{x\to0^+} \frac{x-0}{x-0}
	= 1,
\]
而\[
	\lim_{x\to0^-} \frac{f(x)-f(0)}{x-0}
	= \lim_{x\to0^-} \frac{(-x)-0}{x-0}
	= -1.
\]
\end{example}

\begin{theorem}
“函数\(f\)在点\(x_0\)连续”是“\(f\)在点\(x_0\)可导”的必要不充分条件.
\begin{proof}
由\cref{theorem:导数与微分.函数可导性与连续性的关系} 已知:
如果函数\(f\)在点\(x_0\)可导,则\(f\)必定在点\(x_0\)连续.
由\cref{example:导数.函数在一点的导数是无穷大,example:导数.函数在一点的左右导数不相等} 可知:
纵使函数\(f\)在点\(x_0\)连续,\(f\)在点\(x_0\)也可能不可导.
\end{proof}
\end{theorem}

\begin{example}\label{example:连续函数.狄利克雷函数改2只在一点可导}
%@see: https://www.bilibili.com/video/BV1Rr421M7T8/
证明:函数\(f(x) = x^2 D(x)\)在点\(x=0\)可导,在其他各点既不可导也不连续.
\begin{proof}
由定义有\[
	f'(0) = \lim_{x\to0} \frac{f(x) - f(0)}{x - 0}
	= \lim_{x\to0} \frac{x^2 D(x)}{x}
	= \lim_{x\to0} x D(x)
	= 0,%\cref{example:连续函数.狄利克雷函数改1只在一点连续}
\]
函数\(f\)在点\(x=0\)可导.

当\(x\neq0\)时,有\(D(x) = f(x) / x^2\).
用反证法.
假设\(f\)在点\(x_0\neq0\)连续,那么\[
	\lim_{x \to x_0} D(x)
	= \lim_{x \to x_0} \frac{f(x)}{x^2}
	% 连续性、极限的四则运算
	= \frac{f(x_0)}{x_0^2}
	= D(x_0),
\]
即\(D\)在点\(x_0\)连续,
这与\cref{example:连续函数.狄利克雷函数处处不连续} 的结论矛盾,
从而说明\(f\)在任意一点\(x_0\neq0\)不连续,自然也不可导.
\end{proof}
\end{example}
\begin{remark}
\cref{example:连续函数.狄利克雷函数改2只在一点可导} 说明:
单靠“函数\(f\)在点\(x_0\)可导”这个条件,
既不能推出“\(f\)在点\(x_0\)的邻域可导”,
也不能推出“\(f\)在点\(x_0\)的邻域连续”.
%@credit: {de3029b8-10a6-4ae5-8f64-108dae1c10a9}
由于函数\(f\)在非零点处处不可导,
所以\(f\)不具有二阶导数.
同理可知,对于任意整数\(n\geq3\),
函数\(x \mapsto x^n D(x)\)
只有一阶导数,没有二阶以上导数.
因此得出结论:
即便有\(f(x) = o(x^n)\ (x\to0)\)成立,
也不能推出“\(f\)在点\(x=0\)具有\(n\)阶导数”.
\end{remark}

\begin{example}
\DefineConcept{魏尔斯特拉斯函数}\[
	W(x) = \lim_{n\to\infty} \sum_{k=0}^n a^k \cos(b^k \pi x),
\]在定义域上处处连续而又处处不可导,
其中,参数\(a\)和\(b\)满足\[
	0<a<1,
	\qquad
	b\in\Set{ p \given p = 2q+1, q\in\mathbb{N} },
	\qquad
	ab > 1+\frac{3}{2}\pi.
\]
%TODO proof
\end{example}

\begin{example}
举例说明:可导函数\(f\)满足\(\lim_{x\to+\infty} f(x) < \infty\)和\(\lim_{x\to+\infty} f'(x) = 0\).
\begin{solution}
取\(f(x) = \frac{\sin x}{x}\),
则\(\lim_{x\to+\infty} f(x) = \lim_{x\to+\infty} f'(x) = 0\).
\end{solution}
\end{example}
\begin{example}
%@see: https://www.bilibili.com/video/BV1My411q7aE/
举例说明:可导函数\(f\)满足\(\lim_{x\to+\infty} f(x) < \infty\),但是不满足\(\lim_{x\to+\infty} f'(x) = 0\).
\begin{solution}
取\(f(x) = \frac{\sin x^2}{x}\),
则\(\lim_{x\to+\infty} f(x) = 0\),
而\[
	f'(x) = 2 \cos x^2 - \frac{\sin x^2}{x^2}.
\]
当\(x\to+\infty\)时,虽然\(\frac{\sin x^2}{x^2} \to 0\),
但是\(\cos x^2\)的极限不存在,
也就是说\(\lim_{x\to+\infty} f'(x)\)不存在.
\end{solution}
\end{example}

% 考研数学经常考察抽象函数在点\(x=0\)的连续性、可导性
\begin{example}
设函数\(f\)在点\(x=0\)的某个邻域内连续,
函数\(g\)是当\(x\to0\)时的无穷小,
且\(f\)是\(g\)的同阶无穷小或高阶无穷小.
要使\(f\)在点\(x=0\)可导,
\(g\)应该满足什么条件?
\begin{solution}
由于\(f\)是无穷小,
% 即\(\lim_{x\to0} f(x) = 0\),
且\(f\)在点\(x=0\)连续,
% 即\(f(0) = \lim_{x\to0} f(x)\),
所以\(f(0) = 0\).
根据导数的定义,有\[
	f'(0)
	= \lim_{x\to0} \frac{f(x) - f(0)}{x-0} % 函数\(f\)在点\(x=0\)的导数的定义
	= \lim_{x\to0} \frac{f(x)}{x} % 代入\(f\)在点\(x=0\)的函数值\(f(0) = 0\)
	= \lim_{x\to0} \frac{f(x)}{g(x)} \cdot \frac{g(x)}{x}. % 极限的四则运算法则
\]
由此可见,要使\(f\)在点\(x=0\)可导,
只需要极限\(\lim_{x\to0} \frac{g(x)}{x}\)存在且有限,
即\(g\)是\(x\)的同阶无穷小或高阶无穷小.
\end{solution}
\end{example}
\begin{example}
%@see: 《2020年全国硕士研究生入学统一考试(数学一)》一选择题/第2题
设函数\(f\)在点\(x=0\)可导,
且\(\lim_{x\to0} f(x) = 0\).
证明:\(\lim_{x\to0} \frac{f(x)}{\sqrt{\abs{x}}}\)存在且有限.
\begin{proof}
因为函数\(f\)在点\(x=0\)可导,
所以\(f\)在点\(x=0\)连续,
从而\(f(0) = \lim_{x\to0} f(x) = 0\),
于是函数\(f\)在点\(x=0\)的导数为\begin{equation*}
	f'(0)
	= \lim_{x\to0} \frac{f(x) - f(0)}{x - 0}
	= \lim_{x\to0} \frac{f(x)}{x},
\end{equation*}
所以\begin{gather*}
	\lim_{x\to0^+} \frac{f(x)}{\sqrt{\abs{x}}}
	% 当\(x>0\)时,\(\abs{x} = x\)
	= \lim_{x\to0^+} \frac{f(x)}{\sqrt{x}}
	% 当\(x>0\)时,\(\sqrt{x}^2 = x\)
	= \lim_{x\to0^+} \frac{f(x)}{x} \cdot \sqrt{x}
	%\cref{theorem:极限.极限的四则运算法则}
	= \lim_{x\to0^+} \frac{f(x)}{x} \cdot \lim_{x\to0^+} \sqrt{x}
	= 0, \\
	\lim_{x\to0^-} \frac{f(x)}{\sqrt{\abs{x}}}
	% 当\(x<0\)时,\(\abs{x} = -x\)
	= \lim_{x\to0^-} \frac{f(x)}{\sqrt{-x}}
	% 当\(x<0\)时,\(\sqrt{-x}^2 = -x\)
	= \lim_{x\to0^-} \frac{f(x)}{-x} \cdot \sqrt{-x}
	%\cref{theorem:极限.极限的四则运算法则}
	= \lim_{x\to0^-} \frac{f(x)}{-x} \cdot \lim_{x\to0^-} \sqrt{-x}
	= 0,
\end{gather*}
于是由\cref{theorem:函数极限.极限与单侧极限的关系1} 可知
\(\lim_{x\to0} \frac{f(x)}{\sqrt{\abs{x}}} = 0\).
\end{proof}
\end{example}

\subsection{导函数的连续性与间断点}
\begin{definition}
设函数\(f\)在点\(x_0\)的某一邻域内可导.
如果\(f\)的导函数\(f'\)在点\(x_0\)连续,
则称“函数\(f\)在点\(x_0\)~\DefineConcept{连续可导}(continuously differentiable)”
%@see: https://mathworld.wolfram.com/ContinuouslyDifferentiableFunction.html
“函数\(f\)在点\(x_0\)具有\DefineConcept{连续导数}(continuous derivative)”.
\end{definition}

%@see: https://zhuanlan.zhihu.com/p/666265696
让我们首先介绍几个例子,回顾导函数可能具有的间断点的类型.

\begin{example}
对函数\[
	f(x) = \left\{ \begin{array}{cl}
		x+1, & x>0, \\
		x-1, & x\leq0
	\end{array} \right.
\]求导得\[
	f'(x) = 1
	\quad(x\neq0).
\]
显然点\(x=0\)是导函数\(f'\)的可去间断点.
\end{example}

\begin{example}
对函数\[
	f(x) = \abs{x}
\]求导得\[
	f'(x) = \left\{ \begin{array}{cl}
		1, & x>0, \\
		-1, & x<0.
	\end{array} \right.
\]
显然点\(x=0\)是导函数\(f'\)的跳跃间断点.
\end{example}

\begin{example}
对函数\[
	f(x) = \frac1x
\]求导得\[
	f'(x) = -\frac1{x^2}.
\]
显然点\(x=0\)是导函数\(f'\)的无穷间断点.
\end{example}

\begin{example}
对函数\[
	f(x) = \left\{ \begin{array}{cl}
		x^2 \sin\frac1x, & x \neq 0, \\
		0, & x = 0
	\end{array} \right.
\]求导得\[
	f'(x) = \left\{ \begin{array}{cl}
		2x \sin\frac1x - \cos\frac1x, & x \neq 0, \\
		0, & x = 0.
	\end{array} \right.
\]
这就是说,函数\(f\)在\((-\infty,+\infty)\)内处处可导.
但是由于点\(x=0\)是函数\(x \mapsto \cos\frac1x\)的振荡间断点,
而\(\lim_{x\to0} x \sin\frac1x = 0\),
所以点\(x=0\)是导函数\(f'\)的振荡间断点,
自然\(f'\)在点\(x=0\)的极限不存在.
%@Mathematica: Plot[Piecewise[{{x^2 Sin[1/x], x != 0}, {0, x == 0}}], {x, -.5, .5}, PlotRange -> {-.1, .1}]
%@Mathematica: Plot[Piecewise[{{2 x Sin[1/x] - Cos[1/x], x != 0}, {0, x == 0}}], {x, -.5, .5}]
\end{example}

以上四个例子说明:函数的不可导点可能是它的导函数的任意一种类型的间断点.
