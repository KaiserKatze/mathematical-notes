\section{高阶导数}
\subsection{函数的\texorpdfstring{\(n\)}{n}阶导数}
\begin{definition}
一般地,函数\(y=f(x)\)的导数\(y'=f'(x)\)仍然是\(x\)的函数.
我们把\(y'=f'(x)\)的导数叫做函数\(y=f(x)\)的\DefineConcept{二阶导数},
记作\[
	y''
	\quad\text{或}\quad
	\dv[2]{y}{x},
	\]
即\[
	y''=(y')'
	\quad\text{或}\quad
	\dv[2]{y}{x}=\dv{x}(\dv{y}{x}).
\]

相应地,把\(y=f(x)\)的导数\(y'(x)\)叫做函数\(y=f(x)\)的\DefineConcept{一阶导数}.

类似地,二阶导数的导数叫做\DefineConcept{三阶导数},记作\[
	y'''
	\quad\text{或}\quad
	\dv[3]{y}{x};
\]
三阶导数的导数叫做\DefineConcept{四阶导数},记作\[
	y^{(4)}
	\quad\text{或}\quad
	\dv[4]{y}{x};
\]
以此类推,\((n-1)\)阶导数的导数叫做 \DefineConcept{\(n\)阶导数},记作\[
	y^{(n)}
	\quad\text{或}\quad
	\dv[n]{y}{x}.
\]
二阶及二阶以上的导数统称\DefineConcept{高阶导数}.

特别地,规定:\[
	f^{(0)}(x) = f(x).
\]

函数\(y=f(x)\)具有\(n\)阶导数,也常说成“函数\(f(x)\)为\(n\)阶可导”.
\end{definition}

\begin{theorem}
设\(u=u(x)\)和\(v=v(x)\)都\(n\)阶可导,则
\begin{equation}
	(u \pm v)^{(n)} = u^{(n)} \pm v^{(n)}.
\end{equation}
\end{theorem}

\begin{theorem}
设\(u=u(x)\)和\(v=v(x)\)都\(n\)阶可导,则
\begin{equation}\label{equation:导数与微分.莱布尼茨公式}
	(u v)^{(n)}
	= \sum_{k=0}^n C_n^k u^{(n-k)} v^{(k)}.
\end{equation}
\end{theorem}
\cref{equation:导数与微分.莱布尼茨公式} 称为\DefineConcept{莱布尼茨公式}.

\begin{definition}\label{definition:函数族.n阶可导函数族}
由区间\(I\)上全部的\(n\)阶可导函数组成的集合,
称作\(n\)阶\DefineConcept{可导函数族},
记作\(D^n(I)\),即\[
	D^n(I)
	\defeq
	\Set*{
		f\in\mathbb{R}^I
		\given
		(\forall i\in[0,n-1]\cap\mathbb{Z})
		[f^{(i)} \in D(I)]
	}.
\]
\end{definition}

\begin{definition}\label{definition:函数族.n阶连续可导函数族}
由区间\(I\)上全部的\(n\)阶连续可导函数组成的集合,
称作\(n\)阶\DefineConcept{连续可导函数族},
记作\(C^n(I)\),即\[
	C^n(I)
	\defeq
	\Set*{
		f\in\mathbb{R}^I
		\given
		[f\in D^n(I)]
		\land
		[f^{(n)}\in C(I)]
	}.
\]
\end{definition}

\begin{theorem}
如果函数\(f(x)\)在点\(x\)处具有\(n\)阶导数,
那么\(f(x)\)在点\(x\)的某一邻域内必定具有一切低于\(n\)阶的导数.
\end{theorem}
换句话说,\(
C(I) \supseteq
D(I) = D^1(I) \supseteq
C^1(I) \supseteq
D^2(I) \supseteq
C^2(I) \supseteq
D^3(I) \supseteq
\dotsb
\).

\subsection{光滑函数}
\begin{definition}\label{definition:函数族.光滑函数族}
定义:\[
	D^\infty (I) \defeq \bigcap_{n\geq1} D^n(I),
\]\[
	C^\infty (I) \defeq \bigcap_{n\geq1} C^n(I).
\]

称函数\(f \in C^\infty (I)\)为\(I\)上的\DefineConcept{光滑函数}(smooth function).
\end{definition}

\begin{property}\label{theorem:函数族.光滑函数族的性质1}
\(D^\infty (I) = C^\infty (I)\).
\end{property}
