\section{本章总结}
\subsection*{显函数的导数公式}
\begin{table}[ht]
	\centering
	\begin{tblr}{*2{p{8cm}}}
		幂函数 \\ \hline
		%\cref{example:导数.常数函数的导数}
		\(C' = 0\ (\text{$C$是常数})\)
		%\cref{example:导数.幂函数的导数}
		& \((x^\mu)'=\mu x^{\mu-1}\) \\
		\((\sqrt{x})' = \frac1{2\sqrt{x}}\)
		& \((\sqrt[n]{x})' = \frac{\sqrt[n]{x}}{n x}\) \\
		\(\left(\frac1x\right)' = -\frac1{x^2}\)
		& \(\left(\frac1{x^n}\right)' = -\frac{n}{x^{n+1}}\) \\
		\SetCell[c=2]{l}
		\((x^\mu)^{(n)}
		= \mu(\mu-1)(\mu-2)\dotsm(\mu-n+1) \cdot x^{\mu-n}
		= x^{\mu-n}\prod_{k=0}^{n-1} {(\mu - k)}\) \\
		\((x^n)^{(n)} = n!\)
		& \((x^n)^{(n+1)} = 0\) \\
	\end{tblr}
\end{table}

\begin{table}[ht]
	\centering
	\begin{tblr}{*2{p{8cm}}}
		指数函数 \\ \hline
		%\cref{example:导数.指数函数的导数}
		\((e^x)' = e^x\)
		& \((a^x)' = a^x \ln a\) \\
		\((e^x)^{(n)} = e^x\)
		& \((a^x)^{(n)} = a^x \ln^n a\) \\
	\end{tblr}
\end{table}

\begin{table}[ht]
	\centering
	\begin{tblr}{*2{p{8cm}}}
		对数函数 \\ \hline
		%\cref{example:导数.对数函数的导数}
		\((\ln x)' = \frac1x\ (x>0)\)
		& \((\ln\abs{x})' = \frac1x\ (x\neq0)\) \\
		\((\log_a x)' = \frac1{x \ln a}\ (x>0)\) \\
		\([\ln(1+x)]^{(n)} = (-1)^{n-1} \frac{(n-1)!}{(1+x)^n}\) \\
	\end{tblr}
\end{table}

\begin{table}[ht]
	\centering
	\begin{tblr}{*2{p{8cm}}}
		三角函数 \\ \hline
		%\cref{example:导数.正弦函数的导数}
		\((\sin x)' = \cos x\)
		%\cref{example:导数.余弦函数的导数}
		& \((\cos x)' = - \sin x\) \\
		%\cref{example:导数.正切函数的导数}
		\((\tan x)' = \sec^2 x\)
		%\cref{example:导数.余切函数的导数}
		& \((\cot x)' = - \csc^2 x\) \\
		%\cref{example:导数.正割函数的导数}
		\((\sec x)' = \sec x \tan x\)
		%\cref{example:导数.余割函数的导数}
		& \((\csc x)' = - \csc x \cot x\) \\
		\((\sin x)^{(n)} = \sin\left(x+\frac{n\pi}2\right)\)
		& \((\cos x)^{(n)} = \cos\left(x+\frac{n\pi}2\right)\) \\
	\end{tblr}
\end{table}

\begin{table}[ht]
	\centering
	\begin{tblr}{*2{p{8cm}}}
		反三角函数 \\ \hline
		\((\arcsin x)' = \frac1{\sqrt{1 - x^2}} \quad (-1<x<1)\)
		& \((\arccos x)' = - \frac1{\sqrt{1 - x^2}} \quad (-1<x<1)\) \\
		\((\arctan x)' = \frac1{1 + x^2}\)
		& \((\arccot x)' = - \frac1{1 + x^2}\) \\
		\((\arcsec x)' = \frac1{\abs{x} \sqrt{x^2-1}}\)
		& \((\arccsc x)' = -\frac1{\abs{x} \sqrt{x^2-1}}\) \\
	\end{tblr}
\end{table}

\begin{table}[ht]
	\centering
	\begin{tblr}{*2{p{8cm}}}
		双曲函数 \\ \hline
		%\cref{example:导数.双曲正弦函数的导数}
		\((\sinh x)' = \cosh x\)
		%\cref{example:导数.双曲余弦函数的导数}
		& \((\cosh x)' = \sinh x\) \\
		%\cref{example:导数.双曲正切函数的导数}
		\((\tanh x)' = \sech^2 x\)
		%\cref{example:导数.双曲余切函数的导数}
		& \((\coth x)' = -\csch^2 x\) \\
		%\cref{example:导数.双曲正割函数的导数}
		\((\sech x)' = -\sech x \tanh x\)
		%\cref{example:导数.双曲余割函数的导数}
		& \((\csch x)' = -\csch x \coth x\) \\
	\end{tblr}
\end{table}

\begin{table}[ht]
	\centering
	\begin{tblr}{*2{p{8cm}}}
		反双曲函数 \\ \hline
		\((\arsinh x)' = \frac1{\sqrt{1 + x^2}}\)
		& \((\arcosh x)' = \frac1{\sqrt{x^2 - 1}} \quad (x > 1)\) \\
		\((\artanh x)' = \frac1{1 - x^2} \quad(\abs{x}<1)\)
		& \((\arcoth x)' = \frac1{1-x^2} \quad(\abs{x}>1)\) \\
	\end{tblr}
\end{table}

\clearpage
\subsection*{隐函数的导数}
幂指函数\[
	y = u^v
	\quad(u > 0),
\]的导数是\[
	\dv{x} e^{v \ln u}
	= u^v \left( v' \ln u + \frac{u'v}{u} \right).
\]

对于参数方程\[
	\left\{ \begin{array}{l}
		x = \phi(t), \\
		y = \psi(t),
	\end{array} \right.
\]
若函数\(x = \phi(t)\)具有单调连续反函数\(t=\phi^{-1}(x)\),
且\(x = \phi(t)\)和\(y = \psi(t)\)都可导,
且\(\phi'(t) \neq 0\),
则\[
	\dv{y}{x}
	= \frac{\psi'(t)}{\phi'(t)};
\]
若\(x = \phi(t)\)和\(y = \psi(t)\)都二阶可导,
则\[
	\dv[2]{y}{x}
	= \frac{\psi''(t)~\phi'(t) - \psi'(t)~\phi''(t)}{(\phi'(t))^3}.
\]
