\section{隐函数及由参数方程所确定的函数的导数}\label{section:导数与微分.隐函数及由参数方程所确定的函数的导数}
\subsection{隐函数的导数}
\begin{definition}
形如\(y=f(x)\)的函数,在等号左端是因变量的符号,而右端是含有自变量的式子,当自变量取定义域内任一值时,由这个式子能确定对应的函数.用这种方式表达的函数叫做\DefineConcept{显函数}.
相对地,形如\(g(x,y)=0\)的函数,称为\DefineConcept{隐函数}.
把一个隐函数化为显函数,叫做\DefineConcept{隐函数的显化}.
\end{definition}

\begin{example}
例如从方程\(x+y^3-1=0\)解出\(y=\sqrt[3]{1-x}\),就把隐函数化成了显函数.
\end{example}

\begin{example}
%@see: 《高等数学(第六版 上册)》 P104 例1
求由方程\(e^y + xy - e = 0\)所确定的隐函数的导数\(\displaystyle\dv{y}{x}\).
\begin{solution}
把方程两边分别对\(x\)求导得\begin{equation*}
	\dv{x}(e^y+xy-e) = e^y \dv{y}{x} + y + x \dv{y}{x} = 0,
\end{equation*}
解得\begin{equation*}
	\dv{y}{x} = -\frac{y}{x+e^y},
	\quad x+e^y \neq 0.
\end{equation*}
在这个结果中,分式中的\(y=y(x)\)是由方程\(e^y + xy - e = 0\)所确定的隐函数.
\end{solution}
%@Mathematica: eqn = Exp[y[x]] + x y[x] - E == 0
%@Mathematica: Solve[D[eqn, x], y'[x]]
\end{example}

%@see: 《数学分析(第二版 上册)》(陈纪修) P126 例4.2.1
\begin{example}
%@see: 《高等数学(第六版 上册)》 P105 例3
%@see: 《数学分析(第二版 上册)》(陈纪修) P128 例4.2.2
求椭圆\(\frac{x^2}{16}+\frac{y^2}{9}=1\)在点\(\left(2,\frac32\sqrt3\right)\)处的切线方程.
\begin{solution}
由导数的几何意义可知,所求切线的斜率为\begin{equation*}
	k = \eval{y'}_{x=2}.
\end{equation*}

椭圆方程的两边分别对\(x\)求导,得\begin{equation*}
	\frac{x}{8}+\frac{2}{9}y\dv{y}{x}=0,
\end{equation*}\begin{equation*}
	\dv{y}{x}=-\frac{9x}{16y}.
	\eqno(1)
\end{equation*}

当\(x=2\)时,\(y=\frac{3}{2}\sqrt3\),代入(1)式可得\begin{equation*}
	\eval{\dv{y}{x}}_{x=2}=-\frac{\sqrt3}{4}.
\end{equation*}
于是所求的切线方程为\begin{equation*}
	y-\frac{3}{2}\sqrt3 = -\frac{\sqrt3}{4}(x-2),
\end{equation*}
即\begin{equation*}
	\sqrt3 x + 4 y - 8\sqrt3 = 0.
\end{equation*}
\end{solution}
\end{example}
\begin{remark}
一般地,椭圆\begin{equation*}
	\frac{(x-x_0)^2}{a^2} + \frac{(y-y_0)^2}{b^2} = 1
\end{equation*}
在点\((x_1,y_1)\)的切线的方程为\begin{equation*}
	\frac{(x-x_1)(x_0-x_1)}{a^2} + \frac{(y-y_1)(y_0-y_1)}{b^2} = 0.
\end{equation*}
%@Mathematica: eqn = (x - x0)^2/a^2 + (y[x] - y0)^2/b^2 == 1
%@Mathematica: sol = Solve[D[eqn, x], y'[x]]
%@Mathematica: dydx = y'[x] /. sol /. {x -> x1, y -> y1}
\end{remark}

% \begin{lemma}
% 设函数\(y=\ln\abs{x}\).
% 当\(x\neq0\)时,\(y'=1/x\).
% \end{lemma}

\begin{example}
已知双曲线\(C: y^2 = 2px\ (p>0)\),
过它的焦点弦\(PQ\)的两个端点作它的两条切线,求这两条切线的交点的坐标.
\begin{figure}[htb]
	\centering
	\pgfmathsetmacro{\p}{1}%参数p
	\pgfmathsetmacro{\f}{\p/2}%焦点横坐标
	\pgfmathsetmacro{\k}{3}%焦点弦斜率k
	\pgfmathsetmacro{\a}{\p/(2*\k^2)}
	\pgfmathsetmacro{\b}{\p/\k}
	\pgfmathsetmacro{\c}{sqrt(\k^2+1)}
	\pgfmathsetmacro{\pa}{\a*(\c+1)^2}% x_1
	\pgfmathsetmacro{\pb}{\b*(\c+1)}% y_1
	\pgfmathsetmacro{\qa}{\a*(\c-1)^2}% x_2
	\pgfmathsetmacro{\qb}{-\b*(\c-1)}% y_2
	\pgfmathsetmacro{\sa}{-\p/2}% x_0
	\pgfmathsetmacro{\sb}{\p/\k}% y_0
	\begin{tikzpicture}
		\begin{axis}[
			xmin=-2,xmax=2,
			axis lines=middle,
			xlabel=$x$,
			ylabel=$y$,
		]
			\coordinate(F)at(\f,0);
			\coordinate(X)at(10,0);
			\coordinate(P)at(\pa,\pb);
			\coordinate(Q)at(\qa,\qb);
			\coordinate(S)at(\sa,\sb);
			\draw[purple](-\f,-4)--(-\f,4);%准线
			\draw(2,-3)node[above right]{$C$};
			\begin{scope}[color=orange,samples=100,smooth]
				\addplot[domain=0:10]{sqrt(2*\p*x)};
				\addplot[domain=0:10]{-sqrt(2*\p*x)};
			\end{scope}
			\draw[blue](P)--(F)--(Q);
			\draw(P)--(S)--(Q);
			\draw pic["$\theta$",draw=gray,-,angle eccentricity=1.7,angle radius=5mm]{angle=X--F--P};
			\fill(P)circle(2pt)node[below]{$P$}
				(F)circle(2pt)node[below]{$F$}
				(Q)circle(2pt)node[right]{$Q$}
				(S)circle(2pt)node[left]{$S$};
		\end{axis}
	\end{tikzpicture}
	\caption{}
	\label{figure:导数与微分.抛物线[y^2=2px]的焦点弦与切线}
\end{figure}
\begin{solution}
如\cref{figure:导数与微分.抛物线[y^2=2px]的焦点弦与切线},
设直线\(l: y=k\left(x-\frac{p}{2}\right)\)(不妨设\(k>0\))
与双曲线\(C\)相交于\(P(x_1,y_1)\)和\(Q(x_2,y_2)\)两点.
那么\begin{equation*}
	x_1 = \frac{p}{2 k^2} (\sqrt{k^2+1} + 1)^2,
	\qquad
	y_1 = \frac{p}{k} (1+\sqrt{k^2+1}),
\end{equation*}\begin{equation*}
	x_2 = \frac{p}{2 k^2} (\sqrt{k^2+1} - 1)^2,
	\qquad
	y_2 = \frac{p}{k} (1-\sqrt{k^2+1}).
\end{equation*}

隐函数\(y^2 = 2px\)对\(x\)求导,得\begin{equation*}
	2 y \dv{y}{x} = 2 p,
\end{equation*}
整理得\begin{equation*}
	\dv{y}{x} = \frac{p}{y},
\end{equation*}
所以\begin{equation*}
	k_1 = \eval{\dv{y}{x}}_P
	= \frac{k}{1+\sqrt{k^2+1}},
	\qquad
	k_2 = \eval{\dv{y}{x}}_Q
	= \frac{k}{1-\sqrt{k^2+1}}.
\end{equation*}

那么抛物线过点\(P\)和点\(Q\)的切线的方程为\begin{equation*}
	l_1: y-y_1 = k_1 (x-x_1),
\end{equation*}\begin{equation*}
	l_2: y-y_2 = k_2 (x-x_2).
\end{equation*}
联立切线方程,消去\(y\),得\begin{equation*}
	y_1 + k_1(x - x_1) = y_2 + k_2(x - x_2),
\end{equation*}
移项得\begin{equation*}
	(k_1 - k_2) x = y_2 - y_1 + k_1 x_1 - k_2 x_2,
\end{equation*}
解得\begin{equation*}
	x_0 = \frac{y_2 - y_1 + k_1 x_1 - k_2 x_2}{k_1 - k_2}
	= -\frac{p}{2},
\end{equation*}
代入得\begin{equation*}
	y_0 = y_1 + k_1(x_0 - x_1)
	= \frac{p}{k}.
\end{equation*}
综上所述,所求的交点坐标为\begin{equation*}
	\left(-\frac{p}{2},\frac{p}{k}\right).
\end{equation*}
\end{solution}
\end{example}

\subsection{对数求导法}
在某些场合,利用所谓\emph{对数求导法}求导数比用通常的方法简便些,
它适用于求解函数的积、商,根式形式、幂形式、指数形式或幂指形式的函数.
这种方法是先在\(y=f(x)\)的两边取对数,然后再求出\(y\)的导数\(y'\).
但要注意,因为对数运算要求其真数必须大于零(即\(\log_a x \implies x>0\)),
所以在使用对数求导法时,一定要注意\(f(x)\)的定义域.

\begin{example}
%@see: 《高等数学(第六版 上册)》 P106 例5
求\(y=x^{\sin x}\ (x>0)\)的导数.
\begin{solution}
因为\(x > 0\),
所以\(y=x^{\sin x} > 0\).
对函数等式两边取对数得\begin{equation*}
	\ln y = \sin x \ln x,
\end{equation*}
对上式两边求导得\begin{equation*}
	\frac{y'}{y} = \cos x \ln x + \frac{\sin x}{x},
\end{equation*}
于是\begin{equation*}
	y' = x^{\sin x} \left( \cos x \ln x + \frac{\sin x}{x} \right).
\end{equation*}
\end{solution}
\end{example}

\begin{theorem}
%@see: 《高等数学(第六版 上册)》 P106
对于一般形式的幂指函数\begin{equation*}
	y = u^v
	\quad(u > 0),
\end{equation*}
如果\(u=u(x),v=v(x)\)都可导,
则\begin{equation}
	y' = \dv{x} e^{v \ln u}
	= u^v \left( v' \ln u + \frac{u'v}{u} \right).
\end{equation}
\end{theorem}
这里一定要注意\(u>0\)这个条件,它是函数\(\ln u\)对定义域的要求.

\begin{example}
求\(y=\sqrt{\frac{(x-1)(x-2)}{(x-3)(x-4)}}\)的导数.
\begin{solution}
由\(\frac{(x-1)(x-2)}{(x-3)(x-4)}\geq0\)
得\(x \in (-\infty,1]\cup[2,3)\cup(4,+\infty)\).
这就是函数\(y=y(x)\)的定义域.

当\(x>4\)时,对函数式两边取对数,得\begin{equation*}
	\ln y = \frac{1}{2} \bigl[
		\ln(x-1)+\ln(x-2)-\ln(x-3)-\ln(x-4)
	\bigr],
\end{equation*}
上式两边对\(x\)求导,得\begin{equation*}
	\frac{y'}{y} = \frac{1}{2} \left(
		\frac{1}{x-1} + \frac{1}{x-2} - \frac{1}{x-3} - \frac{1}{x-4}
	\right),
\end{equation*}
于是\begin{equation*}
	y' = \frac{y}{2} \left(
		\frac{1}{x-1} + \frac{1}{x-2} - \frac{1}{x-3} - \frac{1}{x-4}
	\right).
\end{equation*}

当\(x<1\)时,对原函数变形得\(y=\sqrt{\frac{(1-x)(2-x)}{(3-x)(4-x)}}\);
当\(2<x<3\)时,对原函数变形得\(y=\sqrt{\frac{(x-1)(x-2)}{(3-x)(4-x)}}\);
用同样的方法可得与上面相同的结果.
\end{solution}
\end{example}

\subsection{由参数方程所确定的函数的导数}
对于参数方程\begin{equation*}
%@see: 《高等数学(第六版 上册)》 P107 (3)
	\left\{ \begin{array}{l}
		x = \phi(t), \\
		y = \psi(t),
	\end{array} \right.
\end{equation*}
若函数\(x = \phi(t)\)具有单调连续反函数\(t=\phi^{-1}(x)\),
且该反函数可与函数\(y = \psi(t)\)构成复合函数,
那么由上述参数方程所确定的函数可以看成是
由函数\(y=\psi(t)\)、\(t=\phi^{-1}(x)\)复合而成的函数\(y=\psi(\phi^{-1}(x))\).
现在,要计算这个复合函数的导数.
为此再假定函数\(x = \phi(t)\)和\(y = \psi(t)\)都可导,且\(\phi'(t) \neq 0\).
于是根据复合函数的求导法则与反函数的求导法则,就有\begin{equation*}
	\dv{y}{x}
	= \dv{y}{t} \cdot \dv{t}{x},
\end{equation*}
即\begin{equation}\label{equation:导数.参数方程确定的函数的一阶导数}
%@see: 《高等数学(第六版 上册)》 P108 (4)
	\dv{y}{x}
	= \frac{\psi'(t)}{\phi'(t)}.
\end{equation}
这就是上述参数方程所确定的\(x\)的函数的导数公式.

如果函数\(x = \phi(t)\)和\(y = \psi(t)\)还都是二阶可导的,
那么有二阶导数公式\begin{equation*}
	\dv[2]{y}{x}
	= \dv{x}(\dv{y}{x})
	= \dv{t}(\dv{y}{x}) \cdot \dv{t}{x},
\end{equation*}
即\begin{equation}\label{equation:导数.参数方程确定的函数的二阶导数}
%@see: 《高等数学(第六版 上册)》 P108 (5)
	\dv[2]{y}{x}
	= \frac{\psi''(t) \phi'(t) - \psi'(t) \phi''(t)}{(\phi'(t))^3}.
\end{equation}

\begin{example}
%@see: 《数学分析(第二版 上册)》(陈纪修) P149 例4.4.8
计算由摆线的参数方程\begin{equation*}
	\left\{ \begin{array}{l}
		x=a(t-\sin t), \\
		y=a(1-\cos t)
	\end{array} \right.
\end{equation*}所确定的函数\(y=y(x)\)的二阶导数.
\begin{solution}
求导得\begin{equation*}
	\dv{x}{t}=a(1-\cos t), \qquad
	\dv{y}{t}=a\sin t.
\end{equation*}
令\(\dv{x}{t}\neq0\),
得\(a(1-\cos t)\neq0\),
\(\cos t\neq1\),
即\begin{equation*}
	t\neq 2n\pi
	\quad(n\in\mathbb{Z}).
\end{equation*}
因此\begin{equation*}
	\dv{y}{x}
	=\frac{y'(t)}{x'(t)}
	=\frac{a\sin t}{a(1-\cos t)}
	=\frac{\sin t}{1-\cos t}
	=\cot\frac{t}{2}
	\quad(t\neq 2n\pi,n\in\mathbb{Z}).
\end{equation*}
再求导,得\begin{align*}
	\dv[2]{y}{x}
	&=\dv{t}(\cot\frac{t}{2}) \cdot \dv{t}{x}
	=\dv{t}(\cot\frac{t}{2}) \cdot \frac{1}{x'(t)}
	=\frac{-1}{2\sin^2\frac{t}{2}}
	\cdot\frac{1}{a(1-\cos t)} \\
	&=-\frac{1}{a(1-\cos t)^2}
	\quad(t\neq 2n\pi,n\in\mathbb{Z}).
\end{align*}
\end{solution}
\end{example}

\begin{example}
%@see: 《2009年全国硕士研究生入学统一考试(数学二)》二填空题/第9题
求曲线\begin{equation*}
	\left\{ \begin{array}{l}
		x = \int_0^{1-t} e^{-u^2} \dd{u}, \\
		y = t^2 \ln(2-t^2)
	\end{array} \right.
\end{equation*}在点\((0,0)\)处的切线方程.
\begin{solution}
令\(x = 0, y = 0\)得\(t = 1\).
求导得\begin{equation*}
	\dv{x}{t}
	= -e^{-(1-t)^2},
	\qquad
	\dv{y}{t}
	= 2t \ln(2-t^2) - 2\frac{t^3}{2-t^2},
\end{equation*}
于是\begin{equation*}
	\eval{\dv{y}{x}}_{(0,0)}
	= \eval{\frac{y'_t}{x'_t}}_{t=1}
	= \frac{-2}{-1}
	= 2,
\end{equation*}
那么所求切线方程为\(y = 2 x\).
\end{solution}
\end{example}

\subsection{相关变化率}
假设\(x=x(t)\)和\(y=y(t)\)都是可导函数,
而变量\(x\)与\(y\)之间存在某种关系,
从而变化率\(\dv{x}{t}\)与\(\dv{y}{t}\)之间也存在一定关系.
这两个相互依赖的变化率称为\DefineConcept{相关变化率}.
相关变化率问题就是研究这两个变化率之间的关系,
以便从其中一个变化率求出另一个变化率.

\begin{example}
%@see: 《高等数学(第六版 上册)》 P111 例10
一个气球从离开观察员 $\qty{500}{\meter}$ 处离地铅直上升,
当气球高度为 $\qty{500}{\meter}$ 时,
其速率为 $\qty[per-mode=symbol]{140}{\meter\per\minute}$.
求此时观察员视线的仰角增加的速率是多少?
\begin{solution}
设气球上升\(t\)~\unit{\second} 后,其高度为\(h\),观察员的仰角为\(\alpha\),
则\begin{equation*}
	\tan\alpha=\frac{h}{500},
\end{equation*}
其中\(\alpha\)及\(h\)都与\(t\)存在可导的函数关系.
上式两边对\(t\)求导,得\begin{equation*}
	\sec^2\alpha \cdot \dv{\alpha}{t}
	= \frac1{500} \cdot \dv{h}{t}.
\end{equation*}

由已知条件,在某个时刻(\(t=t_0\)时),
有\(\eval{h}_{t=t_0}=\qty{500}{\meter}\),
\(\eval{\dv{h}{t}}=\qty[per-mode=symbol]{140}{\meter\per\minute}\).
可以计算出\(\eval{\tan\alpha}_{t=t_0}=1\),
从而\(\eval{\sec^2\alpha}_{t=t_0}=2\).
代入上式得\begin{equation*}
	2 \eval{\dv{\alpha}{t}}_{t=t_0} = \frac{140}{500},
\end{equation*}
所以\begin{equation*}
	\eval{\dv{\alpha}{t}}_{t=t_0} = \frac{70}{500}
	= \qty[per-mode=symbol]{0.14}{\radian\per\minute},
\end{equation*}
此时观察员视线的仰角增加的速率是 \qty[per-mode=symbol]{0.14}{\radian\per\minute}.
\end{solution}
\end{example}
