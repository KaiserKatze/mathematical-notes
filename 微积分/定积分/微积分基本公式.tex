\section{微积分基本公式}
\subsection{积分上限的函数及其导数}
\begin{theorem}\label{theorem:定积分.变限积分定理}
%@see: 《高等数学(第六版 上册)》 P237 定理1
%@see: 《数学分析(第二版 上册)》(陈纪修) P296 定理7.3.1
设函数\(f\colon[a,b]\to\mathbb{R}\),
令\(\Phi(x)
= \int_a^x f(t) \dd{t}
\ (a \leq x \leq b)\).
\begin{itemize}
	\item 如果\(f\)在\([a,b]\)上黎曼可积,
	则\(\Phi\)在\([a,b]\)上连续.

	\item 如果\(f\)在\([a,b]\)上连续,
	则\(\Phi\)在\([a,b]\)上可导,
	并且它的导数为\begin{equation*}
		\Phi'(x)
		= \dv{x} \int_a^x f(t) \dd{t}
		= f(x),
		\quad a \leq x \leq b.
	\end{equation*}
\end{itemize}
\begin{proof}
易知\(\Phi\)在整个闭区间\([a,b]\)上有定义.
我们按\(x\)的取值,分为\(x\in(a,b)\)、\(x=a\)、\(x=b\)三种情况讨论.

若\(x\in(a,b)\),
取\(x + \increment x \in (a,b)\),
则\begin{equation*}
	\Phi(x + \increment x) = \int_a^{x+\increment x} f(t) \dd{t},
\end{equation*}
于是\begin{align*}
	\increment\Phi
	&= \Phi(x + \increment x) - \Phi(x) \\
	&= \int_a^{x+\increment x} f(t) \dd{t} - \int_a^x f(t) \dd{t} \\
	&= \int_a^x f(t) \dd{t} + \int_x^{x+\increment x} f(t) \dd{t} - \int_a^x f(t) \dd{t}
		\tag{\cref{theorem:定积分.定积分性质3}} \\
	&= \int_x^{x+\increment x} f(t) \dd{t}.
\end{align*}
记\(m \defeq \min_{a \leq x \leq b} f(x),
M \defeq \max_{a \leq x \leq b} f(x),
\alpha \defeq \min\{x,x+\increment x\},
\beta \defeq \max\{x,x+\increment x\}\).
由\hyperref[theorem:定积分.积分中值定理1]{定积分第一中值定理}可知:
如果\(f\)在\([a,b]\)上黎曼可积,
则\(\exists\mu\in[m,M]\),
使得\(\increment\Phi = \mu \cdot \increment x\);
如果\(f\)在\([a,b]\)上连续,
则\(\exists\xi\in(\alpha,\beta)\),
使得\(\increment\Phi = f(\xi) \cdot \increment x\).
显然,不管在哪种情况下,当\(\increment x\to0\)时,
都有\(\increment\Phi\to0\),
即\(\Phi\)在\([a,b]\)上连续.

如果\(f\)在\([a,b]\)上连续,
当\(\increment x\to0\)时\(\xi \to x\),
因而\(f(\xi) \to f(x)\),
于是当\(x\in(a,b)\)时,成立\begin{equation*}
	\Phi'(x)
	= \lim_{\increment x\to0} \frac{\increment\Phi}{\increment x}
	= \lim_{\increment x\to0} f(\xi)
	= \lim_{\xi \to x} f(\xi)
	= f(x).
\end{equation*}
当\(x = a\)时,取\(\increment x > 0\),则同理可证\begin{equation*}
	\Phi'_+(a) = f(a).
\end{equation*}
当\(x = b\)时,取\(\increment x < 0\),则同理可证\begin{equation*}
	\Phi'_-(b) = f(b).
	\qedhere
\end{equation*}
\end{proof}
\end{theorem}
\cref{theorem:定积分.变限积分定理} 指出了一个重要结论:
连续函数\(f\)取变上限\(x\)的定积分然后求导,其结果还原为\(f\)本身.
联想到原函数的定义,就可以从该定理推知\(\Phi\)是连续函数\(f\)的一个原函数.
因此,我们引出如下的原函数的存在定理:
\begin{theorem}[原函数存在定理]\label{theorem:定积分.原函数存在定理}
%@see: 《高等数学(第六版 上册)》 P238 定理2
设函数\(f\)在区间\([a,b]\)上连续,
则函数\begin{equation*}
	\Phi(x) = \int_a^x f(t) \dd{t}
\end{equation*}就是\(f\)在\([a,b]\)上的一个原函数.
%TODO proof
\end{theorem}
\begin{remark}
\cref{theorem:定积分.原函数存在定理} 为判定一个函数的原函数是否存在,提供了一个充分不必要条件.
例如,虽然函数\(f(x) = \frac1x\)在点\(x=0\)不连续,
但是函数\(g(x) = \ln\abs{x}\)确实是它的一个原函数.
\end{remark}
\begin{remark}
\cref{theorem:定积分.原函数存在定理} 的重要意义是:
一方面肯定了连续函数的原函数是存在的,
另一方面初步地揭示了积分学中的定积分与原函数之间的联系.
因此,我们就有可能通过原函数来计算定积分.
\end{remark}

\begin{example}%野题
设\begin{equation*}
%@Mathematica: f[x_] := Piecewise[{{Sin[x], 0 <= x < \[Pi]}, {2, \[Pi] <= x <= 2 \[Pi]}}]
%@Mathematica: F[x_] := Integrate[f[t], {t, 0, x}, Assumptions -> {x \[Element] Reals}]
%@Mathematica: Plot[{f[x], F[x]}, {x, 0, 2 \[Pi]}, PlotLegends -> "Expressions"]
	f(x) = \left\{ \begin{array}{cl}
		\sin x, & 0 \leq x < \pi, \\
		2, & \pi \leq 2 \leq 2\pi,
	\end{array} \right.
	\qquad
	F(x) = \int_0^x f(t) \dd{t}.
\end{equation*}
考察函数\(F\)在点\(x=\pi\)的连续性和可导性.
\begin{solution}
当\(0 \leq x < \pi\)时,
有\begin{equation*}
	F(x) = \int_0^x f(t) \dd{t}
	= \int_0^x \sin t \dd{t}
	= \eval{\cos t}_x^0
	= 1 - \cos x,
\end{equation*}
%@Mathematica: Limit[F[x], x -> \[Pi], Direction -> "FromBelow"]
故\(\lim_{x\to\pi^-} F(x) = 2\).
当\(\pi \leq x \leq 2\pi\)时,
有\begin{equation*}
	F(x) = \int_0^\pi f(t) \dd{t} + \int_\pi^x f(t) \dd{t}
	= 2(1 + x - \pi),
\end{equation*}
%@Mathematica: Limit[F[x], x -> \[Pi], Direction -> "FromAbove"]
故\(\lim_{x\to\pi^+} F(x) = 2\).
又因为\(F(\pi) = 2\),
所以函数\(F\)在点\(x=\pi\)连续.

因为\begin{align*}
%@Mathematica: Limit[(F[x] - F[\[Pi]])/(x - \[Pi]), x -> \[Pi], Direction -> "FromBelow"]
	F'_-(\pi)
	&= \lim_{x\to\pi^-} \frac{F(x) - F(\pi)}{x - \pi}
	= \lim_{x\to\pi^-} \frac{1 - \cos x - 2}{x - \pi}
	% 洛必达法则
	= 0, \\
%@Mathematica: Limit[(F[x] - F[\[Pi]])/(x - \[Pi]), x -> \[Pi], Direction -> "FromAbove"]
	F'_+(\pi)
	&= \lim_{x\to\pi^+} \frac{F(x) - F(\pi)}{x - \pi}
	= \lim_{x\to\pi^+} \frac{2(1 + x - \pi) - 2}{x - \pi}
	= 2,
\end{align*}
左导数与右导数不相等,
所以函数\(F\)在点\(x=\pi\)不可导.
\end{solution}
\end{example}

\begin{example}
%@credit: {ce603838-a24d-4616-9395-d7b223e8cb72}
举例说明:函数可积但它的原函数不存在.
\begin{solution}
取\begin{equation*}
	f(x) = \sgn x
	\quad(-1 \leq x \leq 1).
\end{equation*}
显然\(f\)在\([-1,1]\)上可积,
且\begin{equation*}
	\int_{-1}^x f(t) \dd{t}
	= \abs{x} - 1.
\end{equation*}
假设\(F\colon[-1,1]\to\mathbb{R}\)是\(f\)的一个原函数.
由定义,对\(\forall x\in[-1,1]\),有\begin{equation*}
	F'(x) = f(x).
\end{equation*}
特别地,有\(F'(0) = f(0) = 0\).
%@credit: {c6cf8893-93b7-42d9-8607-d381c488d453}
现在来检验函数\(F\)在点\(x=0\)的右导数:\begin{equation*}
	F'_+(0)
	= \lim_{h\to0^+} \frac{F(0+h)-F(0)}{h}
	= \lim_{h\to0^+} F'(h)
	= \lim_{h\to0^+} f(h)
	= 1
	\neq F'(0),
\end{equation*}
矛盾!
因此\(f\)的原函数不存在.
\end{solution}
\end{example}

\begin{example}
%@credit: {ce603838-a24d-4616-9395-d7b223e8cb72}
举例说明:函数不可积但它的原函数存在.
\begin{solution}
取\begin{equation*}
	f(x) = 2 x \sin\frac1{x^2} - \frac2x \cos\frac1{x^2}
	\quad(x\neq0).
\end{equation*}
显然\(g(x) = x^2 \sin\frac1{x^2}\)是\(f\)的一个原函数.
但是\(f\)在点\(x=0\)的邻域内无界振荡,不可积.
\end{solution}
\end{example}

\begin{example}
设函数\(f\)在区间\([a,b]\)上连续,函数\(\phi\)和\(\psi\)可导.
证明:\begin{equation*}
	F(x) = \int_{\psi(x)}^{\phi(x)} f(t) \dd{t}
\end{equation*}在\([a,b]\)上可导,
且其导数为\begin{equation}\label{equation:定积分.含参变量积分的导数1}
	F'(x) = \dv{x} \int_{\psi(x)}^{\phi(x)} f(t) \dd{t}
	= f[\phi(x)] \phi'(x) - f[\psi(x)] \psi'(x).
\end{equation}
\begin{proof}
令\(F_1(x) = \int_{\xi}^{\phi(x)} f(t) \dd{t}\),\(u = \phi(x)\),
设函数\(G\)是被积函数\(f\)的一个原函数,
那么\begin{equation*}
	F_1(x) = G[\phi(x)] - G(\xi),
\end{equation*}
从而\begin{align*}
	F'_1(x) = \dv{x} F_1(x)
	&= \dv{x} G[\phi(x)] - \dv{x} G(\xi) = \dv{G}{u} \dv{u}{x} - 0 \\
	&= f(u) \phi'(x) = f[\phi(x)] \phi'(x).
\end{align*}
同样地,令\(F_2(x) = \int_{\psi(x)}^{\xi} f(t) \dd{t}\),\(v = \psi(x)\),
则有\begin{equation*}
	F'_2(x) = \dv{x} F_2(x) = -\dv{x} \int_{\xi}^{\psi(x)} f(x) \dd{t}
	= -f[\psi(x)] \psi'(x).
\end{equation*}
因为\(F(x) = F_1(x) + F_2(x)\),所以\begin{equation*}
	F'(x) = F'_1(x) + F'_2(x)
	= f[\phi(x)] \phi'(x) - f[\psi(x)] \psi'(x).
	\qedhere
\end{equation*}
\end{proof}
\end{example}

\begin{example}
计算极限\begin{equation*}
	\lim_{x \to a} \frac{x}{x-a} \int_a^x f(t) \dd{t}
	\quad(a\neq0),
\end{equation*}
其中\(f\)在点\(a\)的邻域内连续.
\begin{solution}
显然\begin{equation*}
	\frac{x}{x-a} \int_a^x f(t) \dd{t}
	= \frac1{1 - a/x} \int_a^x f(t) \dd{t}.
\end{equation*}

根据\cref{equation:定积分.上下限相等的定积分为零} 有\begin{equation*}
	\lim_{x \to a} \int_a^x f(t) \dd{t} = 0, \qquad
	\lim_{x \to a} \left( 1 - \frac{a}{x} \right) = 0,
\end{equation*}

根据\cref{theorem:定积分.变限积分定理},有\begin{equation*}
	\dv{x} \int_a^x f(t) \dd{t} = f(x);
\end{equation*}
又有\begin{equation*}
	\dv{x}(1 - \frac{a}{x}) = \frac{a}{x^2};
\end{equation*}
那么根据\cref{theorem:微分中值定理.洛必达法则1} 有\begin{equation*}
	\lim_{x \to a} \frac{x}{x-a} \int_a^x f(t) \dd{t}
	= \lim_{x \to a} \frac{f(x)}{a/x^2}
	= a f(a).
\end{equation*}
\end{solution}
\end{example}

\begin{example}
%@see: 《高等数学(第六版 上册)》 P244 习题5-2 10.
设\begin{equation*}
	f(x) = \left\{ \begin{array}{cl}
		x^2, & 0 \leq x < 1, \\
		x, & 1 \leq x \leq 2.
	\end{array} \right.
\end{equation*}
求\(\Phi(x) = \int_0^x f(t) \dd{t}\)
在\([0,2]\)上的表达式,
并讨论\(\Phi\)在\((0,2)\)内的连续性.
\begin{solution}
当\(0 \leq x < 1\)时,有\begin{equation*}
	\Phi(x) = \int_0^x f(t) \dd{t}
	= \int_0^x t^2 \dd{t}
	= \frac13 x^3.
\end{equation*}

当\(1 \leq x \leq 2\)时,有\begin{equation*}
	\Phi(x) = \int_0^x f(t) \dd{t}
	= \int_0^1 t^2 \dd{t}
	+ \int_1^x t \dd{t}
	= \frac13 + \frac12 (x^2-1).
\end{equation*}

因此\begin{equation*}
	\Phi(x) = \left\{ \def\arraystretch{1.5} \begin{array}{cl}
		\frac13 x^3, & 0 \leq x < 1, \\
		\frac13 + \frac12 (x^2-1), & 1 \leq x \leq 2.
	\end{array} \right.
\end{equation*}
易见\begin{equation*}
	\lim_{x\to1^-} \Phi(x)
	= \lim_{x\to1^+} \Phi(x)
	= \Phi(1)
	= \frac13,
\end{equation*}
所以\(\Phi\)在\((0,2)\)内连续.
\end{solution}
\end{example}

\begin{example}
%@see: 《高等数学(第六版 上册)》 P244 习题5-2 12.
设函数\(f \in C[a,b] \cap D(a,b)\),
而\begin{equation*}
	F(x) = \frac1{x-a} \int_a^x f(t) \dd{t}.
\end{equation*}
证明:若在\((a,b)\)内有\(f'(x) \leq 0\),
则在\((a,b)\)内有\(F'(x) \leq 0\).
\begin{proof}
求导得\begin{equation*}
	F'(x) = \frac1{(x-a)^2} \left[ (x-a)~f(x) - \int_a^x f(t) \dd{t} \right].
\end{equation*}
那么\(F'(x) \leq 0\)当且仅当\begin{equation*}
	(x-a)~f(x) - \int_a^x f(t) \dd{t} \leq 0.
\end{equation*}
记\(\phi(x) = (x-a)~f(x) - \int_a^x f(t) \dd{t}\).
显然\(\phi(a) = 0\).
%@credit: {cde036ea-2111-4ff3-b743-6d81e96b49cf},{957ff224-9b08-4ea6-97e8-f592d5fc610b}
求导得\begin{equation*}
	\phi'(x) = f(x) + (x-a)~f'(x) - f(x)
	= (x-a)~f'(x).
\end{equation*}
因为在\((a,b)\)内有\(f'(x) \leq 0\),
所以在\((a,b)\)内有\(\phi'(x) \leq 0\),
即\(\phi\)在\((a,b)\)上严格单调减少,
又因为\(\phi(a) = 0\),
所以在\((a,b)\)内有\(\phi(x) \leq 0\),
从而有\(F'(x) \leq 0\)成立.
\end{proof}
\end{example}
\begin{example}
%@see: 《高等数学(第六版 上册)》 P244 习题5-2 13.
设\(F(x) = \int_0^x \frac{\sin t}{t} \dd{t}\).
求\(F'(0)\).
\begin{solution}
显然\(F(0) = 0\).
注意到点\(x=0\)是\(f(x) = \frac{\sin x}{x}\)的间断点,
不能直接对\(F\)求导,
于是根据导数的定义可知\begin{equation*}
	F'(0) = \lim_{x\to0} \frac{F(x)-F(0)}{x-0}
	= \lim_{x\to0} \frac1x \int_0^x \frac{\sin t}{t} \dd{t}
	= \lim_{x\to0} \frac{\sin x}{x}
	= 1.
\end{equation*}
\end{solution}
\end{example}

\subsection{牛顿--莱布尼茨公式}
现在我们根据\hyperref[theorem:定积分.原函数存在定理]{原函数存在定理}%
来证明一个重要定理,
它给出了用原函数计算定积分的公式.
\begin{theorem}
%@see: 《高等数学(第六版 上册)》 P239 定理3
%@see: 《数学分析(第二版 上册)》(陈纪修) P298 定理7.3.2(微积分基本定理)
设函数\(F\)是连续函数\(f\)在区间\([a,b]\)上的一个原函数,
则\begin{equation}\label{equation:定积分.牛顿--莱布尼茨公式}
	\int_a^b f(x) \dd{x} = F(b) - F(a).
\end{equation}
\begin{proof}
已知函数\(F\)是连续函数\(f\)的一个原函数,
又根据\hyperref[theorem:定积分.原函数存在定理]{原函数存在定理}知道,
积分上限的函数\begin{equation*}
	\Phi(x) = \int_a^x f(t) \dd{t}
\end{equation*}也是\(f\)的一个原函数.
于是这两个原函数之差\(F(x) - \Phi(x)\)在\([a,b]\)上必定是某一个常数\(C\),
即\begin{equation*}
	F(x) - \Phi(x) = C. \eqno(1)
\end{equation*}

在上式中令\(x=a\),得\(F(a) - \Phi(a) = C\).
又由\(\Phi(x)\)的定义式及定积分的补充规定可知\(\Phi(a) = 0\),因此,\(C = F(a)\).
以\(F(a)\)代入(1)式中的\(C\),
以\(\int_a^x f(t) \dd{t}\)代入(1)式中的\(\Phi(x)\),
可得\begin{equation*}
	\int_a^x f(t) \dd{t} = F(x) - F(a).
\end{equation*}
在上式中令\(x=b\),
就得到所要证明的公式,
并称之为\DefineConcept{牛顿--莱布尼茨公式},
或\DefineConcept{微积分基本公式}.
\end{proof}
\end{theorem}
为求简便,可将\(F(b) - F(a)\)记成\(\eval{F(x)}_a^b\),
于是又有\begin{equation*}
	\int_a^b f(x) \dd{x} = \eval{F(x)}_a^b.
\end{equation*}

牛顿--莱布尼茨公式进一步揭示了定积分与被积函数的原函数或不定积分之间的联系.
它表明:一个连续函数在区间\([a,b]\)上的定积分等于
它的任一个原函数在区间\([a,b]\)上的增量.
这就给定积分提供了一个有效而简便的计算方法,大大简化了定积分的计算手续.

\begin{example}
%@see: 《高等数学(第六版 上册)》 P240 例1
计算\(\int_0^1 x^2 \dd{x}\).
\begin{solution}
由于\(F(x) = x^3/3\)是\(x^2\)的一个原函数,所以按牛顿--莱布尼茨公式,有\begin{equation*}
	\int_0^1 x^2 \dd{x} = \left[\frac{x^3}{3}\right]_0^1
	= \frac{1^3}{3} - \frac{0^3}{3} = \frac1{3} - 0 = \frac1{3}.
\end{equation*}
\end{solution}
\end{example}

\begin{example}
%@see: 《高等数学(第六版 上册)》 P240 例2
计算\(\int_{-2}^{-1} \frac{\dd{x}}{x}\).
\begin{solution}
当\(x<0\)时,\(1/x\)的一个原函数是\(\ln\abs{x}\),
现在积分区间是\([-2,-1]\),
这个原函数的形式可以化为\(\ln(-x)\),
所以按牛顿--莱布尼茨公式,有\begin{equation*}
	\int_{-2}^{-1} \frac{\dd{x}}{x}
	= [ \ln\abs{x} ]_{-2}^{-1}
	= \ln1 - \ln2
	= -\ln2.
\end{equation*}
\end{solution}
\end{example}

\begin{proposition}\label{theorem:定积分.积分中值定理1推论2改进}
%@see: 《高等数学(第六版 上册)》 P241 例6
设函数\(f\)在\([a,b]\)上连续,
则\(\exists\xi\in(a,b)\),
使\begin{equation*}
	\int_a^b f(x) \dd{x} = f(\xi) (b-a).
\end{equation*}
\begin{proof}
因\(f\)连续,故它的原函数存在,
设\(F\)为\(f\)的一个原函数,
即在\([a,b]\)上\(F'(x) = f(x)\).
根据\hyperref[equation:定积分.牛顿--莱布尼茨公式]{牛顿--莱布尼茨公式},
有\begin{equation*}
	\int_a^b f(x) \dd{x}
	= F(b) - F(a).
\end{equation*}

由\cref{theorem:定积分.变限积分定理} 可知,
函数\(F\)在区间\([a,b]\)上可导,
满足\hyperref[theorem:微分中值定理.拉格朗日中值定理]{拉格朗日中值定理}的条件,
因此存在实数\(\xi\in(a,b)\),使\begin{equation*}
	F(b) - F(a) = F'(\xi) (b-a),
\end{equation*}
也即\begin{equation*}
	\int_a^b f(x) \dd{x} = f(\xi) (b-a).
	\qedhere
\end{equation*}
\end{proof}
\end{proposition}
\begin{remark}
我们也把\cref{theorem:定积分.积分中值定理1推论2改进} 称为“积分中值定理”.
它正是对\cref{theorem:定积分.积分中值定理1推论2} 的改进.
从证明过程中不难看出积分中值定理
与\hyperref[theorem:微分中值定理.拉格朗日中值定理]{微分中值定理}的联系.
\end{remark}

\begin{example}
%@see: https://www.bilibili.com/video/BV19M411j7uq/
设函数\(f\)在\([0,1]\)上二阶连续可导,
\(f(x)\geq0,
f''(x)<0\),
令\begin{equation*}
	I_n = \int_0^1 f(x^n) \dd{x}
	\quad(n\in\mathbb{N}).
\end{equation*}
\begin{itemize}
	\item 证明:\(I_n \leq f\left( \frac1{n+1} \right)\).
	\item 设\(f(0) = 0\),求极限\(\lim_{n\to\infty} I_n\).
\end{itemize}
\begin{solution}
写出\(f\)的泰勒公式:\begin{equation*}
	f(x) = f(x_0) + f'(x_0) (x-x_0) + \frac1{2!} f''(\xi) (x-x_0)^2
	\quad(\text{$\xi$在$x$与$x_0$之间}).
\end{equation*}
因为\(f''(\xi)<0\),
所以\begin{equation*}
	f(x) \leq f(x_0) + f'(x_0) (x-x_0).
\end{equation*}
用\(x^n\)代\(x\),用\(\frac1{n+1}\)代\(x_0\),得\begin{equation*}
	f(x^n)
	\leq
	f\left(\frac1{n+1}\right) + f'\left(\frac1{n+1}\right) \left(x^n-\frac1{n+1}\right).
\end{equation*}
积分得\begin{equation*}
	I_n = \int_0^1 f(x^n) \dd{x}
	\leq f\left(\frac1{n+1}\right)
	+ f'\left(\frac1{n+1}\right) \int_0^1 \left(x^n-\frac1{n+1}\right) \dd{x}.
\end{equation*}
这里\begin{equation*}
	\int_0^1 \left(x^n-\frac1{n+1}\right) \dd{x} = 0,
\end{equation*}
故\begin{equation*}
	I_n \leq f\left(\frac1{n+1}\right).
\end{equation*}

接下来对不等式\(0 < I_n \leq f\left(\frac1{n+1}\right)\)
令\(n\to\infty\)得\begin{equation*}
	0 \leq \lim_{n\to\infty} I_n \leq f(0) = 0,
\end{equation*}
于是\(\lim_{n\to\infty} I_n = 0\).
\end{solution}
\end{example}

\begin{example}
%@see: 《高等数学(第六版 上册)》 P241 例7
%@see: 《数学分析(第二版 上册)》(陈纪修) P310 习题 3.
\def\fu{\displaystyle \int_0^x t f(t) \dd{t}}
\def\fv{\displaystyle \int_0^x f(t) \dd{t}}
\def\fvv{\left[ \fv \right]^2}
\def\fw{\displaystyle \int_0^x (x-t) f(t) \dd{t}}
设\(f\)在\([0,+\infty)\)内连续且\(f(x) > 0\).
证明:函数\begin{equation*}
	F(x) = \frac{\fu}{\fv}
\end{equation*}在\([0,+\infty)\)内为单调增加函数.
\begin{proof}
因为\begin{equation*}
	\dv{x} \fu = x f(x),
	\qquad
	\dv{x} \fv = f(x),
\end{equation*}
所以\begin{equation*}
	F'(x) = \frac{x f(x) \fv - f(x) \fu}{\fvv}
	= \frac{f(x) \fw}{\fvv}.
\end{equation*}
按假设,当\(0 < t < x\)时,\(f(t) > 0\),\((x-t) f(t) > 0\),
根据\hyperref[theorem:定积分.积分中值定理1推论2改进]{改进后的积分中值定理}可知\begin{equation*}
	\fv > 0, \qquad \fw > 0,
\end{equation*}
所以\(F'(x) > 0\ (x > 0)\),
从而\(F\)在\((0,+\infty)\)内为单调增加函数.
\end{proof}
\end{example}

\begin{example}
%@see: 《高等数学(第六版 上册)》 P241 例8
求\(\lim_{x\to0} \frac1{x^2} \int_{\cos x}^1 \exp(-t^2) \dd{t}\).
\begin{solution}
已知这是一个\(0/0\)型未定式,我们利用洛必达法则来计算.
分子可写成\begin{equation*}
	- \int_1^{\cos x} \exp(-t^2) \dd{t},
\end{equation*}
它是以\(\cos x\)为上限的积分,
作为\(x\)的函数可看成是以\(u = \cos x\)为中间变量的复合函数,
故\begin{align*}
	\dv{x} \int_{\cos x}^1 \exp(-t^2) \dd{t}
	&= -\dv{x} \int_1^{\cos x} \exp(-t^2) \dd{t} \\
	&= -\left[ \dv{u}\int_1^u \exp(-t^2) \dd{t} \right]_{u=\cos x} \cdot (\cos x)' \\
	&= -\exp(-\cos^2 x) \cdot (-\sin x) \\
	&= \sin x \exp(-\cos^2 x).
\end{align*}
因此\begin{equation*}
	\lim_{x\to0} \frac1{x^2} \int_{\cos x}^1 \exp(-t^2) \dd{t}
	= \lim_{x\to0} \frac{\sin x \exp(-\cos^2 x)}{2x}
	= \frac1{2e}.
\end{equation*}
\end{solution}
\end{example}

\begin{example}
%@see: 《数学分析(第二版 上册)》(陈纪修) P299 例7.3.5
计算\(\lim_{n\to\infty} \left(\frac1{n+1}+\frac1{n+2}+\dotsb+\frac1{2n}\right)\).
\begin{solution}
将和式改写为\begin{equation*}
	\frac1{n+1}+\frac1{n+2}+\dotsb+\frac1{2n}
	= \frac1n \left(\frac1{1+1/n}+\frac1{1+2/n}+\dotsb+\frac1{1+n/n}\right),
\end{equation*}
这相当于在\([0,1]\)中对函数\(f(x) = \frac1{1+x}\)作\(\increment x_i = \frac1n\)的等距分割后,
在小区间\([x_{i-1},x_i]\)上将\(\xi_i\)取为\(x_i\ (i=1,2,\dotsc,n)\)的积分和\begin{equation*}
	\sum_{i=1}^n f(\xi_i) \increment x_i.
\end{equation*}
于是\begin{align*}
	\lim_{n\to\infty} \left(\frac1{n+1}+\frac1{n+2}+\dotsb+\frac1{2n}\right)
	&= \lim_{\lambda\to0} \sum_{i=1}^n \frac1{1+\xi_i} \increment x_i \\
	&= \int_0^1 \frac{\dd{x}}{1+x}
	= \eval{\ln(1+x)}_0^1
	= \ln2.
\end{align*}
\end{solution}
\end{example}

\begin{example}
%@see: 《数学分析(第二版 上册)》(陈纪修) P313 习题 21.
设函数\(f\)的导数\(f'\)在\([a,b]\)上连续.
证明:\begin{equation*}
	\max_{a \leq x \leq b} \abs{f(x)}
	\leq \abs{\frac1{b-a} \int_a^b f(x) \dd{x}}
	+ \int_a^b \abs{f'(x)} \dd{x}.
\end{equation*}
\begin{proof}
因为\(f\)在\([a,b]\)上连续,
所以由\hyperref[theorem:极限.最值定理]{最值定理}可知,
\(f\)必能取得它的最大值和最小值,
不妨设\begin{gather*}
	\abs{f(\xi)} = \max_{a \leq x \leq b} \abs{f(x)},
	\quad\xi\in[a,b], \\
	\abs{f(\eta)} = \min_{a \leq x \leq b} \abs{f(x)},
	\quad\xi\in[a,b].
\end{gather*}
于是\begin{align*}
	\max_{a \leq x \leq b} \abs{f(x)} - \min_{a \leq x \leq b} \abs{f(x)}
	&= \abs{f(\xi)} - \abs{f(\eta)} \\
	&\leq \abs{f(\xi) - f(\eta)}
		\tag{\hyperref[theorem:不等式.三角不等式2]{三角不等式}} \\
	&= \abs{\int_\eta^\xi f'(x) \dd{x}}
		\tag{\hyperref[equation:定积分.牛顿--莱布尼茨公式]{牛顿--莱布尼茨公式}} \\
	&\leq \int_\eta^\xi \abs{f'(x)} \dd{x}
		\tag{\hyperref[theorem:定积分.定积分性质5推论2]{绝对可积性}} \\
	&\leq \int_a^b \abs{f'(x)} \dd{x}.
		\tag{\cref{theorem:定积分.定积分性质5推论3}}
\end{align*}

由\hyperref[theorem:定积分.积分中值定理1推论2]{积分中值定理}可知,
存在\(\zeta\in[a,b]\),使得\begin{equation*}
	f(\zeta) = \frac1{b-a} \int_a^b f(x) \dd{x}.
\end{equation*}
于是\begin{equation*}
	\min_{a \leq x \leq b} \abs{f(x)}
	\leq \abs{f(\zeta)}
	= \abs{\frac1{b-a} \int_a^b f(x) \dd{x}}.
\end{equation*}

综上所述,我们有\begin{align*}
	\max_{a \leq x \leq b} \abs{f(x)}
	&= \max_{a \leq x \leq b} \abs{f(x)} - \min_{a \leq x \leq b} \abs{f(x)}
		+ \min_{a \leq x \leq b} \abs{f(x)} \\
	&\leq \int_a^b \abs{f'(x)} \dd{x}
		+ \abs{\frac1{b-a} \int_a^b f(x) \dd{x}}.
	\qedhere
\end{align*}
\end{proof}
\end{example}
