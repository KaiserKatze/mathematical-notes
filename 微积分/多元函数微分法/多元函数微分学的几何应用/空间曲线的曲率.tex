\subsection{空间曲线的曲率}
我们在\cref{section:微分中值定理.曲率}中研究平面曲线时引入了“曲率”这样一个概念,
现在我们将它推广到空间曲线.

设空间曲线\(\Gamma: \vb{r} = \vb{r}(t)\ (\alpha \leq t \leq \beta)\)是一段光滑曲线.
当参数\(t = t_0\)时,曲线\(\Gamma\)上动点\(P\)的坐标为\(\vb{r}\),
而点\(P\)处的切向量为\(\vb{r}'(t_0)\).
记向量\(\vb{r}'(t_0)\)与\(\vb{r}'(t_0+\increment t)\)之间的夹角为\(\increment\theta\),
记\(\vb{r}(t_0)\)与\(\vb{r}(t_0+\increment t)\)之间的弧长为\(\increment s\).
如果极限\begin{equation*}
	\lim_{\increment s \to 0} \abs{ \frac{\increment\theta}{\increment s} }
\end{equation*}存在,
就称此极限为曲线\(\Gamma\)在点\(P\)处的\DefineConcept{曲率},
记为\(K(t_0)\),
即\begin{equation*}
	K(t_0)
	= \lim_{\increment s \to 0} \abs{ \frac{\increment\theta}{\increment s} }.
\end{equation*}
因此\begin{align*}
	\abs{ \dv{\vb{\tau}}{s} }
	&= \abs{
			\lim_{\increment s \to 0}
				\frac{\vb{\tau}(s+\increment s) - \vb{\tau}(s)}{\increment s}
		} \\
	&= \lim_{\increment s \to 0}
		\frac{\abs{\vb{\tau}(s+\increment s) - \vb{\tau}(s)}}{\abs{\increment s}} \\
	&= \lim_{\increment s \to 0}
		\abs{ \frac{2 \sin(\increment\theta / 2)}{\increment s} } \\
	&= \lim_{\increment s \to 0}
		\left(
			\abs{ \frac{2 \sin(\increment\theta / 2)}{\increment\theta/2} }
			\cdot \abs{ \frac{\increment\theta}{\increment s} }
		\right) \\
	&= \lim_{\increment s \to 0}
		\abs{ \frac{\increment\theta}{\increment s} }.
\end{align*}
又因为\(\vb{\tau} = \dv{\vb{r}}{s}\),
所以\(\abs{ \dv{\vb{\tau}}{s} }
= \abs{ \dv{s}( \dv{\vb{r}}{s} ) }
= \abs{ \dv[2]{\vb{r}}{s} }\),
那么上式又可写作\begin{equation}
	\abs{ \dv[2]{\vb{r}}{s} }
	= \lim_{\increment s \to 0} \abs{ \frac{\increment\theta}{\increment s} }.
\end{equation}

假设空间曲线\(\Gamma\)由参数方程\(\vb{r} = \vb{r}(t)\)给出.
由于位矢\(\vb{r}\)对弧长\(s\)的一阶、二阶导数分别为\begin{equation*}
	\dv{\vb{r}}{s}
	= \dv{\vb{r}}{t} \dv{t}{s},
	\qquad
	\dv[2]{\vb{r}}{s}
	= \dv[2]{\vb{r}}{t} \left(\dv{t}{s}\right)^2 + \dv{\vb{r}}{t} \dv[2]{t}{s}.
\end{equation*}
将以上两式对应的两边分别作向量积,
得\begin{equation*}
	\VectorOuterProduct{\dv{\vb{r}}{s}}{\dv[2]{\vb{r}}{s}}
	= \left(
			\VectorOuterProduct{\dv{\vb{r}}{t}}{\dv[2]{\vb{r}}{t}}
		\right)
		\left(\dv{t}{s}\right)^3.
\end{equation*}
由于\(\dv{\vb{r}}{s}\)是单位向量,
所以\(\dv{\vb{r}}{s} \cdot \dv{\vb{r}}{s} = 1\),
再对\(s\)求导,
得\begin{equation*}
	\dv{s}( \dv{\vb{r}}{s} \cdot \dv{\vb{r}}{s} )
	= \dv[2]{\vb{r}}{s} \cdot \dv{\vb{r}}{s} + \dv{\vb{r}}{s} \cdot \dv[2]{\vb{r}}{s}
	= 2 \left( \dv{\vb{r}}{s} \cdot \dv[2]{\vb{r}}{s} \right)
	= 0,
\end{equation*}
也就是说向量\(\dv{\vb{r}}{s}\)与\(\dv[2]{\vb{r}}{s}\)互相正交,
所以曲线\(\Gamma\)的曲率为\begin{equation*}
	K(t)
	= \abs{ \dv[2]{\vb{r}}{s} }
	= \abs{ \VectorOuterProduct{\dv{\vb{r}}{s}}{\dv[2]{\vb{r}}{s}} }
	= \abs{
			\VectorOuterProduct{\dv{\vb{r}}{t}}{\dv[2]{\vb{r}}{t}}
		}
		\abs{ \dv{t}{s} }^3.
\end{equation*}
但因\(\abs{ \dd{\vb{r}} } = \abs{ \dd{s} }\),
\(\abs{ \dv{s}{t} } = \abs{ \dv{\vb{r}}{t} }\),
所以\begin{equation*}
	\abs{ \dv{t}{s} }^3 = \abs{ \dv{\vb{r}}{t} }^{-3}.
\end{equation*}
由此得出曲率为\begin{equation}
	K(t) = \frac{
			\abs{
				\VectorOuterProduct{\vb{r}'_t}{\vb{r}''_t}
			}
		}{
			\abs{ \vb{r}'_t }^3
		}.
\end{equation}
