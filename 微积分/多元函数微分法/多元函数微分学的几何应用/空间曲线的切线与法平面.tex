\subsection{空间曲线的切线与法平面}
假设空间曲线\(\Gamma\)满足参数方程 \labelcref{equation:解析几何.曲线的参数方程}
\begin{equation*}
	\vb{r}
	= \vb{f}(t)
	= (x(t),y(t),z(t)),
	\quad \alpha \leq t \leq \beta.
\end{equation*}
其中\(\vb{f}\)的分量函数\(x(t),y(t),z(t)\)在\([\alpha,\beta]\)上都可导,
且它们的导数\(x'(t),y'(t),z'(t)\)不同时为零.

设点\(M(x_0,y_0,z_0)\)在曲线\(\Gamma\)上,对应的参数为\(t_0\).
由向量值函数的导向量的几何意义知,
向量\begin{equation}
	\vb{T}
	= \vb{f}'(t_0)
	= (x'(t_0),y'(t_0),z'(t_0)).
\end{equation}
就是曲线\(\Gamma\)在点\(M\)处的一个切向量,
从而曲线\(\Gamma\)在点\(M\)处的切线方程为
\begin{equation}
	\begin{bmatrix}
		x \\
		y \\
		z
	\end{bmatrix}
	= \lambda \begin{bmatrix}
		x'(t_0) \\
		y'(t_0) \\
		z'(t_0)
	\end{bmatrix}
	+ \begin{bmatrix}
		x_0 \\
		y_0 \\
		z_0
	\end{bmatrix},
	\quad \lambda\in\mathbb{R},
\end{equation}
或
\begin{equation}\label{equation:多元函数微分学的几何应用.曲线的切线方程}
	\frac{x-x_0}{x'(t_0)}
	=\frac{y-y_0}{y'(t_0)}
	=\frac{z-z_0}{z'(t_0)}.
\end{equation}

我们知道,通过点\(M\)且与切线垂直的平面,就是曲线\(\Gamma\)在点\(M\)处的法平面.
它是通过点\(M(x_0,y_0,z_0)\)且以\(\vb{T}=\vb{f}'(t_0)\)为法向量的平面,
因此法平面方程为\begin{equation}
	\VectorInnerProductDot{\vb{T}}{(x-x_0,y-y_0,z-z_0)}=0
\end{equation}
或\begin{equation}
	x'(t_0) \cdot (x-x_0) + y'(t_0) \cdot (y-y_0) + z'(t_0) \cdot (z-z_0) = 0.
\end{equation}

若空间曲线\(\Gamma\)的方程以\begin{equation*}
	\left\{ \begin{array}{l}
		y = \phi(x), \\
		z = \psi(x)
	\end{array} \right.
\end{equation*}的形式给出,
取\(x\)为参数,
它就可以表为参数方程的形式\begin{equation*}
	\left\{ \begin{array}{l}
		x = x, \\
		y = \phi(x), \\
		z = \psi(x).
	\end{array} \right.
\end{equation*}
若\(\phi(x),\psi(x)\)都在\(x=x_0\)可导,
那么根据上面的讨论可知,\begin{equation*}
	\vb{T} = (1,\phi'(x_0),\psi'(x_0)),
\end{equation*}
因此曲线\(\Gamma\)在点\(M(x_0,y_0,z_0)\)处的切线方程为
\begin{equation}\label{equation:多元函数微分学的几何应用.曲线的切线方程.变式1}
	\frac{x-x_0}{1}
	=\frac{y-y_0}{\phi'(x_0)}
	=\frac{z-z_0}{\psi'(x_0)}.
\end{equation}
曲线\(\Gamma\)在点\(M(x_0,y_0,z_0)\)处的法平面方程为
\begin{equation}\label{equation:多元函数微分学的几何应用.曲线的法平面方程.变式1}
	(x-x_0) + \phi'(x_0) \cdot (y-y_0) + \psi'(x_0) \cdot (z-z_0) = 0.
\end{equation}

假设空间曲线\(\Gamma\)满足一般方程 \labelcref{equation:解析几何.曲线的一般方程}
\begin{equation*}
	\left\{ \begin{array}{l}
		F(x,y,z) = 0, \\
		G(x,y,z) = 0,
	\end{array} \right.
\end{equation*}
\(M(x_0,y_0,z_0)\)是曲线\(\Gamma\)上的一个点.
又设\(F\)、\(G\)有对各个变量的连续偏导数,
且\begin{equation*}
	\eval{\jacobi{F,G}{y,z}}_{(x_0,y_0,z_0)} \neq 0.
\end{equation*}
这时方程组 \labelcref{equation:解析几何.曲线的一般方程}
在点\(M(x_0,y_0,z_0)\)的某一邻域内确定了一组函数\(y=\phi(x)\)和\(z=\psi(x)\).
要求曲线\(\Gamma\)在点\(M\)处的切线方程和法平面方程,
只要求出\(\phi'(x_0)\)、\(\psi'(x_0)\),
然后代入\cref{equation:多元函数微分学的几何应用.曲线的切线方程.变式1,equation:多元函数微分学的几何应用.曲线的法平面方程.变式1} 就行了.
为此,我们在恒等式\begin{equation*}
	F[x,\phi(x),\psi(x)] = 0,
	\quad\text{和}\quad
	G[x,\phi(x),\psi(x)] = 0
\end{equation*}两边分别对\(x\)求全导数,
得\begin{equation*}
	\left\{ \def\arraystretch{1.5} \begin{array}{l}
		\pdv{F}{x} + \pdv{F}{y} \dv{y}{x} + \pdv{F}{z} \dv{z}{x} = 0, \\
		\pdv{G}{x} + \pdv{G}{y} \dv{y}{x} + \pdv{G}{z} \dv{z}{x} = 0.
	\end{array} \right.
\end{equation*}
由假设可知,
在点\(M\)的某个邻域内
雅克比式\(J=J(x,y,z)\)恒不等于零,
即\begin{equation*}
	J = \jacobi{F,G}{y,z} \neq 0,
\end{equation*}
故可解得\begin{equation*}
	\dv{y}{x} = \frac{1}{J} \jacobi{F,G}{z,x},
	\qquad
	\dv{z}{x} = \frac{1}{J} \jacobi{F,G}{x,y}.
\end{equation*}

于是\(\vb{T} = (1,\phi'(x_0),\psi'(x_0))\)是曲线\(\Gamma\)在点\(M\)处的一个切向量,
这里\begin{equation*}
	\phi'(x_0)
	= \frac{
			\begin{vmatrix}
				F'_z & F'_x \\
				G'_z & G'_x
			\end{vmatrix}_M
		}{
			\begin{vmatrix}
				F'_y & F'_z \\
				G'_y & G'_z
			\end{vmatrix}_M
		},
	\qquad
	\psi'(x_0)
	= \frac{
			\begin{vmatrix}
				F'_x & F'_y \\
				G'_x & G'_y
			\end{vmatrix}_M
		}{
			\begin{vmatrix}
				F'_y & F'_z \\
				G'_y & G'_z
			\end{vmatrix}_M
		},
\end{equation*}
分子分母中带下标\(M\)的行列式表达行列式在点\(M(x_0,y_0,z_0)\)的值.
把上面的切向量\(\vb{T}\)乘以\(J\),得\begin{align*}
	\vb{T}_1
	&= \left(
		\begin{vmatrix}
			F'_y & F'_z \\
			G'_y & G'_z \\
		\end{vmatrix}_M,
		\begin{vmatrix}
			F'_z & F'_x \\
			G'_z & G'_x \\
		\end{vmatrix}_M,
		\begin{vmatrix}
			F'_x & F'_y \\
			G'_x & G'_y \\
		\end{vmatrix}_M
	\right) \\
	&= \begin{vmatrix}
		\vb{i} & \vb{j} & \vb{k} \\
		F'_x & F'_y & F'_z \\
		G'_x & G'_y & G'_z
	\end{vmatrix},
\end{align*}
这也是曲线\(\Gamma\)在点\(M\)处的一个切向量.
由此可以写出曲线\(\Gamma\)在点\(M(x_0,y_0,z_0)\)处的切线方程为
\begin{equation}%\label{equation:多元函数微分学的几何应用.曲线的切线方程.变式2}
	\begin{bmatrix}
		x \\ y \\ z
	\end{bmatrix}
	= \lambda \begin{vmatrix}
		\vb{i} & \vb{j} & \vb{k} \\
		F'_x & F'_y & F'_z \\
		G'_x & G'_y & G'_z
	\end{vmatrix}
	+ \begin{bmatrix}
		x_0 \\ y_0 \\ z_0
	\end{bmatrix},
\end{equation}
或\begin{equation}%\label{equation:多元函数微分学的几何应用.曲线的切线方程.变式3}
	\frac{x-x_0}{\begin{vmatrix}
		F'_y & F'_z \\
		G'_y & G'_z \\
	\end{vmatrix}_M}
	=\frac{y-y_0}{\begin{vmatrix}
		F'_z & F'_x \\
		G'_z & G'_x \\
	\end{vmatrix}_M}
	=\frac{z-z_0}{\begin{vmatrix}
		F'_x & F'_y \\
		G'_x & G'_y \\
	\end{vmatrix}_M}.
\end{equation}
曲线\(\Gamma\)在点\(M(x_0,y_0,z_0)\)处的法平面方程为
\begin{equation}\label{equation:多元函数微分学的几何应用.曲线的法平面方程.变式2}
	\begin{vmatrix}
		F'_y & F'_z \\
		G'_y & G'_z \\
	\end{vmatrix}_M
	(x-x_0)
	+ \begin{vmatrix}
		F'_z & F'_x \\
		G'_z & G'_x \\
	\end{vmatrix}_M
	(y-y_0)
	+ \begin{vmatrix}
		F'_x & F'_y \\
		G'_x & G'_y \\
	\end{vmatrix}_M
	(z-z_0)
	= 0.
\end{equation}
如果\(\eval{\jacobi{F,G}{y,z}}_M = 0\),
而\(\eval{\jacobi{F,G}{z,x}}_M\)和\(\eval{\jacobi{F,G}{x,y}}_M\)中至少有一个不等于零,
我们可得同样的结果.
