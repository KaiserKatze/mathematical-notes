\section{隐函数的求导公式}
%@see: https://mathworld.wolfram.com/ImplicitFunctionTheorem.html
\subsection{一个方程的情形}
在\cref{section:导数与微分.隐函数及由参数方程所确定的函数的导数}中,
我们已经提出了隐函数的概念,并且指出了不经过显化直接由方程\begin{equation*}
%@see: 《高等数学(第六版 下册)》 P83 (1)
	F(x,y) = 0
\end{equation*}求它所确定的隐函数的导数的方法.
现在介绍隐函数存在定理,并根据多元复合函数的求导法推导出隐函数的导数公式.

要想由方程\(F(x,y)=0\)求出\(\dv{y}{x}\),
我们可以假设这个方程确定的函数为\(y=f(x)\),并将函数代入方程中,得恒等式\begin{equation*}
	F(x,f(x))=0,
\end{equation*}
其左端可以看作是\(x\)的一个复合函数.
要想求这个函数的全导数,由于恒等式两端求导后仍然恒等,即得\begin{equation*}
	\pdv{F}{x}+\pdv{F}{y}\dv{y}{x}=0.
\end{equation*}
如果\(F'_y\)连续,且\(F'_y(x_0,y_0)\neq0\),
那么存在\((x_0,y_0)\)的一个邻域,在其中\(F'_y\neq0\),
于是得\begin{equation*}
	\dv{y}{x}=-\frac{F'_x}{F'_y}.
\end{equation*}

我们依照这个思路给出下面的定理,并给出严格的证明.

\begin{theorem}[隐函数存在定理1]\label{theorem:多元函数微分法.隐函数存在定理1}
%@see: 《高等数学(第六版 下册)》 P83 隐函数存在定理1
设函数\(F(x,y)\)在点\(P(x_0,y_0)\)的某一邻域内具有连续偏导数,
且\(F(x_0,y_0)=0\),\(F'_y(x_0,y_0) \neq 0\),
则方程\(F(x,y)=0\)在点\((x_0,y_0)\)的某一邻域内%
恒能唯一确定一个连续且具有连续导数的函数\(y=f(x)\),
它满足条件\(y_0=f(x_0)\),并有
\begin{equation}\label{equation:多元函数微分法.隐函数存在定理1.一阶导数}
%@see: 《高等数学(第六版 下册)》 P84 (2)
	\dv{y}{x} = -\frac{F'_x}{F'_y}.
\end{equation}
\begin{proof}
假设\(F(x,y)\)在邻域
\(G \subseteq K_0=\Set{\abs{x-x_0}\leq a}\times\Set{\abs{y-y_0}\leq b}\)内具有连续偏导数;
又假设\(F'_y(x_0,y_0)>0\),
那么\(\exists a_1\in(0,a),
\exists b_1\in(0,b)\),
使得\(F'_y(x,y)>0\)在闭矩形区域\begin{equation*}
	K_1 = \Set{\abs{x-x_0}\leq a_1}\times\Set{\abs{y-y_0}\leq b_1}
	\subset K_0
\end{equation*}上恒成立.
记\(\alpha(y) = F(x_0,y)\ (\abs{y-y_0}\leq b_1)\),
则\(\alpha(y)\)在定义域上严格单调递增,且满足\begin{equation*}
	\alpha(y_0-b_1) < 0 = F(x_0,y_0) = \alpha(y_0) < \alpha(y_0+b_1).
\end{equation*}
又因为\(F(x,y)\)在\(K_1\)上连续,所以\(\exists a_2\in(0,a_1]\),使得
\begin{equation*}
	\begin{aligned}
		\forall(x,y)\in(x_0-a_2,x_0+a_2)\times\Set{y_0-b_1} \bigl[ F(x,y) < 0 \bigr], \\
		\forall(x,y)\in(x_0-a_2,x_0+a_2)\times\Set{y_0+b_1} \bigl[ F(x,y) > 0 \bigr].
	\end{aligned}
	\eqno(1)
\end{equation*}

对\(\forall\xi\in(x_0-a_2,x_0+a_2)\),
关于\(y\)的一元函数\(F(\xi,y)\)在\([y_0-b_1,y_0+b_1]\)上连续,严格单调递增,且在端点处异号;
那么由\hyperref[theorem:极限.零点定理]{零点定理}可知,
\(\exists!\eta\in(y_0-b_1,y_0+b_1)\),使得\(F(\xi,\eta)=0\)成立.
这样就得到如下函数\begin{equation*}
	f\colon (x_0-a_2,x_0+a_2)\to(y_0-b_1,y_0+b_1), x \mapsto y = f(x),
	\eqno(2)
\end{equation*}
它满足恒等式\(F(x,f(x))=0\)且\(y_0=f(x_0)\).

既然对\(\forall\xi\in(x_0-a_2,x_0+a_2)\),有\(F(\xi,\eta)=0\),其中\(\eta = f(\xi)\);
那么,由(1)式可知,\begin{equation*}
	(\forall\epsilon\in(0,b_1])
	[F(\xi,\eta-\epsilon) < 0 < F(\xi,\eta+\epsilon)].
\end{equation*}
利用函数\(F(x,y)\)在\(K_1\)上的连续性可知,\begin{equation*}
	(\forall\epsilon\in(0,b_1])
	(\exists\delta>0)
	(\forall x\in(\xi-\delta,\xi+\delta))
	[F(x,\eta-\epsilon) < 0 < F(x,\eta+\epsilon)].
\end{equation*}
再次根据零点定理和 关于\(y\)的一元函数\(F(x,y)\)的严格递增性,
可知\(\exists!y\in(\eta-\epsilon,\eta+\epsilon)\)
满足\(F(x,y) = 0\)且\(y = f(x)\),即\begin{equation*}
	\abs{x-\xi}<\delta
	\implies
	\abs{f(x)-f(\xi)}<\epsilon.
	\eqno(3)
\end{equation*}
这就说明,函数\(f(x)\)在点\(\xi\)连续.

取充分小的\(\increment x\),使之满足\(\xi+\increment x\in(x_0-a_2,x_0+a_2)\).
记\(\increment y = f(\xi+\increment x) - \eta\),那么\begin{equation*}
	F(\xi,\eta) = 0 = F(\xi+\increment x,\eta+\increment y).
\end{equation*}
根据微分中值定理 ,\(\exists\theta\in(0,1)\)满足\begin{equation*}\begin{aligned}
	0 &= F(\xi+\increment x,\eta+\increment y) - F(\xi,\eta) \\
	&= F'_x(\xi+\theta\increment x,\eta+\theta\increment y) \increment x
	+ F'_y(\xi+\theta\increment x,\eta+\theta\increment y) \increment y.
\end{aligned}\end{equation*}
因为在\(K_1\)上偏导数\(F'_y(x,y)>0\),因此\begin{equation*}
	\frac{\increment y}{\increment x}
	= - \frac{F'_x(\xi+\theta\increment x,\eta+\theta\increment y)}
	{F'_y(\xi+\theta\increment x,\eta+\theta\increment y)},
\end{equation*}
于是\begin{equation*}
	f'(\xi) = \lim_{\increment x\to0} \frac{\increment y}{\increment x}
	= - \frac{F'_x(\xi,\eta)}{F'_y(\xi,\eta)}.
	\eqno(4)
\end{equation*}
从上述等式可以看出,函数\(f(x)\)在点\(\xi\)处具有连续导数.
\end{proof}
\end{theorem}

在\cref{theorem:多元函数微分法.隐函数存在定理1} 的条件下,
如果\(F(x,y)\)的二阶偏导数也都连续,
我们可以把\cref{equation:多元函数微分法.隐函数存在定理1.一阶导数} 两端
看作\(x\)的复合函数而再一次求导,即得\begin{equation*}
	\dv[2]{y}{x}
	= \pdv{x}\left(\dv{y}{x}\right)
	+ \pdv{y}\left(\dv{y}{x}\right) \cdot \dv{y}{x},
\end{equation*}
代入\(\dv{y}{x}=-\frac{F'_x}{F'_y}\)得\begin{align}
	\dv[2]{y}{x}
	&= \pdv{x}\left(-\frac{F'_x}{F'_y}\right)
	+ \pdv{y}\left(-\frac{F'_x}{F'_y}\right) \cdot \left(-\frac{F'_x}{F'_y}\right) \notag\\
	&= -\frac{F''_{xx}F'_y-F''_{yx}F'_x}{(F'_y)^2}
	- \frac{F''_{xy}F'_y-F''_{yy}F'_x}{(F'_y)^2}
	\left(-\frac{F'_x}{F'_y}\right) \notag\\
	&= -\frac{F''_{xx}(F'_y)^2-2F''_{xy}F'_xF'_y+F''_{yy}(F'_x)^2}{(F'_y)^3}.
\end{align}%\label{equation:多元函数微分法.隐函数存在定理1.二阶导数}

隐函数存在定理还可以推广到多元函数.
既然一个二元函数可以确定一个一元隐函数,那么一个三元方程\begin{equation*}
	F(x,y,z) = 0
\end{equation*}就有可能确定一个二元隐函数.

与\cref{theorem:多元函数微分法.隐函数存在定理1} 一样,
我们同样可以由三元函数\(F(x,y,z)\)的性质来断定
由方程\(F(x,y,z) = 0\)所确定的二元函数\(z = f(x,y)\)的存在以及这个函数的性质.

我们将显函数代入方程中,得恒等式\begin{equation*}
	F(x,y,f(x,y))=0,
\end{equation*}
将上式两端分别对\(x\)和\(y\)求导,
应用复合函数求导法则,得\begin{equation*}
	\pdv{F}{x} + \pdv{F}{z} \pdv{z}{x} = 0, \qquad
	\pdv{F}{y} + \pdv{F}{z} \pdv{z}{y} = 0.
\end{equation*}
如果\(F'_z\)连续,且\(F'_z(x_0,y_0,z_0)\neq0\),
所以存在点\((x_0,y_0,z_0)\)的一个邻域,
在这个邻域内\(F'_z\neq0\),
于是得\begin{equation*}
	\pdv{z}{x}=-\frac{F'_x}{F'_z}, \qquad
	\pdv{z}{y}=-\frac{F'_y}{F'_z}.
\end{equation*}

我们依照这个思路给出下面的定理.
\begin{theorem}[隐函数存在定理2]\label{theorem:多元函数微分法.隐函数存在定理2}
%@see: 《高等数学(第六版 下册)》 P85 隐函数存在定理2
设函数\(F(x,y,z)\)在点\(P(x_0,y_0,z_0)\)的某一邻域内具有连续偏导数,
且\(F(x_0,y_0,z_0)=0\),\(F'_z(x_0,y_0,z_0) \neq 0\),
则方程\(F(x,y,z)=0\)在点\((x_0,y_0,z_0)\)的某一邻域内
恒能唯一确定一个连续且具有连续偏导数的函数\(z=f(x,y)\),
它满足条件\(z_0=f(x_0,y_0)\),并有
\begin{equation}\label{equation:多元函数微分法.隐函数存在定理2.一阶偏导数}
	\pdv{z}{x}=-\frac{F'_x}{F'_z},
	\qquad
	\pdv{z}{y}=-\frac{F'_y}{F'_z}.
\end{equation}
\end{theorem}
在\cref{theorem:多元函数微分法.隐函数存在定理2} 的条件下,
如果\(F(x,y,z)\)的二阶偏导数也都连续,
我们可以把\cref{equation:多元函数微分法.隐函数存在定理2.一阶偏导数} 两端
看作复合函数而再一次求导,即得\begin{gather*}
	\pdv[2]{z}{x}
	= \pdv{x}\left(\pdv{z}{x}\right)
	+ \pdv{z}\left(\pdv{z}{x}\right) \cdot \pdv{z}{x}, \\
	\pdv[2]{z}{x}{y}
	= \pdv{y}\left(\pdv{z}{x}\right)
	+ \pdv{z}\left(\pdv{z}{x}\right) \cdot \pdv{z}{y}, \\
	\pdv[2]{z}{y}
	= \pdv{y}\left(\pdv{z}{y}\right)
	+ \pdv{z}\left(\pdv{z}{y}\right) \cdot \pdv{z}{y},
\end{gather*}
代入\(\pdv{z}{x}=-\frac{F'_x}{F'_z},
\pdv{z}{y}=-\frac{F'_y}{F'_z}\)得\begin{gather*}
	\pdv[2]{z}{x}
	= \frac{2 F'_x F'_z F''_{xz} - (F'_x)^2 F''_{zz} - (F'_z)^2 F''_{xx}}{(F'_z)^3}, \\
	\pdv[2]{z}{y}
	= \frac{2 F'_y F'_z F''_{yz} - (F'_y)^2 F''_{zz} - (F'_z)^2 F''_{yy}}{(F'_z)^3}, \\
	\pdv[2]{z}{x}{y}
	= \frac{F'_x F'_z F''_{yz} - (F'_z)^2 F''_{xy} + F'_y F'_z F''_{xz} - F'_x F'_y F''_{zz}}{(F'_z)^3}.
\end{gather*}

\begin{example}
设\(f(x,y,z) = e^x + y^2 z\),
其中\(z=z(x,y)\)是由方程\(x+y+z+xyz=0\)所确定的隐函数,
求\(f'_x(0,1,-1)\).
\begin{solution}
记\(F(x,y,z) = x+y+z+xyz\).
那么\begin{equation*}
	\pdv{F}{x} = 1+yz, \qquad
	\pdv{F}{z} = 1+xy,
\end{equation*}
从而\begin{equation*}
	\pdv{z}{x} = -\frac{F'_x}{F'_z}
	= -\frac{1+yz}{1+xy}.
\end{equation*}
那么\begin{equation*}
	\pdv{f}{x}
	= e^x + y^2 \pdv{z}{x}
	= e^x - y^2 \frac{1+yz}{1+xy},
\end{equation*}
代入得\begin{equation*}
	f'_x(0,1,-1) = 1.
\end{equation*}
\end{solution}
\end{example}

\begin{example}
%@see: 《高等数学(第六版 下册)》 P85 例2
设\(x^2+y^2+z^2-4z=0\),求\(\pdv[2]{z}{x}\).
\begin{solution}
设\(F(x,y,z) = x^2+y^2+z^2-4z\),则\(F'_x = 2x\),\(F'_z = 2z-4\).
当\(z\neq2\)时,有\begin{equation*}
	\pdv{z}{x} = -\frac{2x}{2z-4} = \frac{x}{2-z}.
\end{equation*}
再一次对\(x\)求偏导数,得\begin{equation*}
	\pdv[2]{z}{x}
	= \frac{1}{(2-z)^2} \left[ (2-z)+x\pdv{z}{x} \right]
	= \frac{(2-z)^2+x^2}{(2-z)^3}.
\end{equation*}
\end{solution}
\end{example}

\begin{example}
设可导函数\(y = y(x)\)由方程\begin{equation*}
	\int_0^{x+y} e^{-t^2} \dd{t}
	= \int_0^x x \sin t^2 \dd{t}
\end{equation*}确定,
求\(\eval{\dv{y}{x}}_{x=0}\).
\begin{solution}
方程对\(x\)求导得\begin{equation*}
	e^{-(x+y)^2} \cdot (1+y')
	= \int_0^x \sin t^2 \dd{t} + x \sin x^2,
\end{equation*}
其中\(y' = \dv{y}{x}\),又令\(x=0\),得\begin{equation*}
	e^{-y^2} (1+y') = 0.
\end{equation*}

在原积分方程中令\(x=0\),得\begin{equation*}
	\int_0^y e^{-t^2} \dd{t} = 0,
\end{equation*}
因为被积函数\(e^{-t^2}\)在\((-\infty,+\infty)\)恒大于零,故必有\(y = 0\).

因此\(e^0 (1+y') = 0 \implies y'=-1\).
\end{solution}
\end{example}

\subsection{方程组的情形}
下面我们将隐函数存在定理作另一方面的推广.
我们不仅增加方程中变量的个数,而且增加方程的个数.

例如,考虑方程组\begin{equation*}
	\left\{ \begin{array}{c}
		F(x,y,u,v)=0, \\
		G(x,y,u,v)=0.
	\end{array} \right.
\end{equation*}
这时,在四个变量中,一般只能有两个变量独立自由变化,
因此这个方程组就有可能确定两个二元函数.
在这种情况下,我们可以由函数\(F,G\)的性质来断定
由这个方程组所确定的两个二元函数
\(u=u(x,y)\)和\(v=v(x,y)\)的存在,以及它们的性质.

我们将显函数代入方程组中,得恒等式\begin{equation*}
	F(x,y,u(x,y),v(x,y))=0, \qquad
	G(x,y,u(x,y),v(x,y))=0,
\end{equation*}
将恒等式两端分别对\(x\)求导,应用复合函数求导法则,得\begin{equation*}
	\left\{ \def\arraystretch{1.6} \begin{array}{l}
		\pdv{F}{x}+\pdv{F}{u}\pdv{u}{x}+\pdv{F}{v}\pdv{v}{x}=0, \\
		\pdv{G}{x}+\pdv{G}{u}\pdv{u}{x}+\pdv{G}{v}\pdv{v}{x}=0.
	\end{array} \right.
\end{equation*}
这是关于\(\pdv{u}{x},\pdv{v}{x}\)的线性方程组.
如果在点\(P(x_0,y_0,u_0,v_0)\)的某一邻域内,
系数行列式\begin{equation*}
	J = \jacobi{F,G}{u,v}
	= \begin{vmatrix}
		\pdv{F}{u} & \pdv{G}{u} \\
		\pdv{F}{v} & \pdv{G}{v}
	\end{vmatrix}
	\neq0,
\end{equation*}
那么这个线性方程组有唯一解\begin{equation*}
	\pdv{u}{x} = -\frac{1}{J} \jacobi{F,G}{x,v}, \qquad
	\pdv{v}{x} = -\frac{1}{J} \jacobi{F,G}{u,x}.
\end{equation*}
同理可得\begin{equation*}
	\pdv{u}{y} = -\frac{1}{J} \jacobi{F,G}{y,v}, \qquad
	\pdv{v}{y} = -\frac{1}{J} \jacobi{F,G}{u,y}.
\end{equation*}

我们依照这个思路给出下面的定理.
\begin{theorem}[隐函数存在定理3]\label{theorem:多元函数微分法.隐函数存在定理3}
设\(F(x,y,u,v),G(x,y,u,v)\)
在点\(P(x_0,y_0,u_0,v_0)\)的某一邻域内具有对各个变量的连续偏导数;
又\begin{equation*}
	F(x_0,y_0,u_0,v_0)=0, \qquad
	G(x_0,y_0,u_0,v_0)=0;
\end{equation*}
且雅克比式\(\jacobi{F,G}{u,v}\)在点\(P(x_0,y_0,u_0,v_0)\)不等于零,即\begin{equation*}
	\jacobi{F,G}{u,v}\eval_P \neq0;
\end{equation*}
则方程组\begin{equation*}
	\left\{ \begin{array}{c}
		F(x,y,u,v)=0, \\
		G(x,y,u,v)=0,
	\end{array} \right.
\end{equation*}
在点\((x_0,y_0,u_0,v_0)\)的某一邻域内恒能唯一确定一组连续且具有连续偏导数的函数
\begin{equation*}
	u=u(x,y), \qquad
	v=v(x,y),
\end{equation*}
它们满足条件\(u_0=u(x_0,y_0)\),\(v_0=v(x_0,y_0)\),并有
\begin{equation}\label{equation:多元函数微分法.隐函数存在定理3.一阶偏导数}
	\begin{split}
		\pdv{u}{x}
		= -\frac{1}{J} \pdv{(F,G)}{(x,v)},
		\qquad
		\pdv{v}{x}
		= -\frac{1}{J} \pdv{(F,G)}{(u,x)},
		\\
		\pdv{u}{y}
		= -\frac{1}{J} \pdv{(F,G)}{(y,v)},
		\qquad
		\pdv{v}{y}
		= -\frac{1}{J} \pdv{(F,G)}{(u,y)}.
	\end{split}
\end{equation}
\end{theorem}
