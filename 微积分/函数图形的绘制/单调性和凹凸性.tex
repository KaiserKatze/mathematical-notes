\section{函数的单调性与曲线的凹凸性}
\subsection{函数单调性的判定法}
\begin{theorem}[函数的单调性]\label{theorem:图形绘制.单调性.利用导数判定函数单调性}
%@see: 《数学分析(第二版 上册)》(陈纪修) P172 定理5.1.5(一阶导数与单调性的关系)
%@see: 《高等数学(第六版 上册)》 P146 定理1
设函数\(f\colon[a,b]\to\mathbb{R}\)在\([a,b]\)上连续,在\((a,b)\)内可导.
\begin{itemize}
	\item 如果在\((a,b)\)内\(f'(x)>0\),那么函数\(f\)在\([a,b]\)上单调增加.
	\item 如果在\((a,b)\)内\(f'(x)<0\),那么函数\(f\)在\([a,b]\)上单调减少.
\end{itemize}
%TODO proof
\end{theorem}
\begin{remark}
如果将\cref{theorem:图形绘制.单调性.利用导数判定函数单调性} 中的闭区间
换成其他各种区间(包括无穷区间),结论也成立.
\end{remark}
\begin{example}\label{example:图形绘制.单调性.函数在定义域上不单调}
%@see: 《高等数学(第六版 上册)》 P146 例2
讨论函数\(f(x) = e^x - x - 1\)的单调性.
\begin{solution}
函数\(f\)的定义域为\(\dom = (-\infty,+\infty)\).
因为在\((-\infty,0)\)内\(f'(x)<0\),
所以\(f\)在\((-\infty,0)\)上单调减少.
因为在\((0,+\infty)\)内\(f'(x)>0\),
所以\(f\)在\((0,+\infty)\)上单调增加.
\end{solution}
\end{example}
\begin{example}\label{example:图形绘制.单调性.函数在个别点不可导}
%@see: 《高等数学(第六版 上册)》 P147 例3
讨论函数\(f(x) = \sqrt[3]{x^2}\)的单调性.
\begin{solution}
函数\(f\)的定义域为\(\dom f = (-\infty,+\infty)\).
当\(x\neq0\)时,它的导数为\[
	f'(x) = \frac23 x^{-\frac13}.
\]
当\(x=0\)时,它的导数不存在.
在\((-\infty,0)\)内,\(f'(x)<0\),
因此\(f\)在\((-\infty,0)\)上单调减少.
在\((0,+\infty)\)内,\(f'(x)>0\),
因此\(f\)在\((0,+\infty)\)上单调增加.
\end{solution}
\end{example}
\begin{remark}
%@see: 《高等数学(第六版 上册)》 P147
从\cref{example:图形绘制.单调性.函数在定义域上不单调} 看出,
有些函数在它的定义区间上不是单调的.
从\cref{example:图形绘制.单调性.函数在个别点不可导} 看出,
有些函数在有限个点不可导.
因此,我们可以得出如下结论:
如果函数在定义区间上连续,
除去有限个导数不存在的点外导数存在且连续,
那么只要用方程\(f'(x) = 0\)的根
及\(f'(x)\)不存在的点
来划分函数\(f\)的定义区间,
就能保证\(f'\)在各个部分区间内保持固定符号,
因而函数\(f\)在每个部分区间上单调.
\end{remark}
\begin{example}\label{example:图形绘制.单调性.函数在个别点导数为零}
%@see: 《高等数学(第六版 上册)》 P148 例5
讨论函数\(f(x) = x^3\)的单调性.
\begin{solution}
函数\(f\)的定义域为\(\dom f = (-\infty,+\infty)\).
它的导数为\(f'(x) = 3x^2\).
显然,除了点\(x=0\)使得\(y'(x) = 0\)以外,在其余各点均有\(f'(x)>0\).
因此,函数\(f\)在\((-\infty,0]\)及\([0,\infty)\)上都是单调增加的,
从而它在整个定义域\((-\infty,+\infty)\)内是单调增加的.
\end{solution}
\end{example}
\begin{remark}
%@see: 《高等数学(第六版 上册)》 P148
从\cref{example:图形绘制.单调性.函数在个别点导数为零} 可以看出:
如果\(f'\)在某区间内的有限个点处为零,
在其余各点处均为正(或负)时,
那么\(f\)在该区间上仍旧是单调增加(或单调减少)的.
\end{remark}

\begin{example}
%@see: 《高等数学(第六版 上册)》 P148 例6
证明:当\(x > 1\)时,\(2 \sqrt{x} > 3 - \frac{1}{x}\).
\begin{proof}
令\(f(x) = 2 \sqrt{x} - \left(3 - \frac{1}{x}\right)\),
则\[
	f'(x) = \frac{1}{\sqrt{x}} - \frac{1}{x^2}
	= \frac{1}{x^2} (x \sqrt{x} - 1).
\]

\(f\)在\([1,+\infty)\)上连续,在\((1,+\infty)\)内\(f'(x) > 0\),
因此在\([1,+\infty)\)上\(f\)单调增加,
从而当\(x > 1\)时,\(f(x) > f(1) = 0\),
即\(2 \sqrt{x} - \left(3 - \frac{1}{x}\right) > 0\),
\(2 \sqrt{x} > 3 - \frac{1}{x}\).
\end{proof}
\end{example}

\begin{example}
证明\DefineConcept{若尔当不等式}:
当\(0<x<\frac\pi2\)时,
有\begin{equation}\label{equation:微分中值定理.若尔当不等式}
	\frac2\pi < \frac{\sin x}{x} < 1.
\end{equation}
\begin{proof}
设\(f(x) = \frac{\sin x}{x}\ (0<x<\frac\pi2)\),
那么\(f'(x) = \frac{x\cos x - \sin x}{x^2}\).
又设\(g(x) = x \cos x - \sin x\),
那么\[
	g'(x) = \cos x - x \sin x - \cos x = -x \sin x < 0\ (0<x<\frac\pi2),
\]
说明\(g\)是\((0,\frac\pi2)\)上的单调减少函数.
又因为\(g(0) = 0\),
从而\(g(x) < 0\ (0<x<\frac\pi2)\),
所以\(f'(x) < 0\),
\(f\)也是\((0,\frac\pi2)\)上的单调减少函数.
应用洛必达法则,得\[
	\lim_{x\to0^+} f(x)
	= \lim_{x\to0^+} \cos x
	= 1.
\]
再因为\[
	\lim_{x\to\frac\pi2^-} f(x)
	= \frac{\sin(\pi/2)}{\pi/2}
	= \frac2\pi,
\]
所以\[
	\frac2\pi < \frac{\sin x}{x} < 1.
	\qedhere
\]
\end{proof}
\end{example}

\begin{example}
证明:
当\(0<x<\frac\pi2\)时,
有\begin{equation}\label{equation:单调性.正切不等式}
	\tan x > x.
\end{equation}
\begin{proof}
设\(f(x) = \tan x - x\ (0<x<\frac\pi2)\).
求导得\(f'(x) = \sec^2 x - 1 = \tan^2 x\).
显然\(f'(x) > 0\ (0<x<\frac\pi2)\),
\(f\)是\((0,\frac\pi2)\)上的严格单调增加函数.
又因为\(f(0) = \tan0 - 0 = 0\),
所以\(f(x) = \tan x - x > 0\ (0<x<\frac\pi2)\),
即\(\tan x > x\ (0<x<\frac\pi2)\).
\end{proof}
\end{example}

\begin{example}
%@see: https://www.bilibili.com/video/BV19YpbepEGF
设数列\(\{x_n\}\)有递推公式\(x_{n+1} = f(x_n)\ (n=1,2,\dotsc)\).
证明:\begin{itemize}
	\item 如果\(f'(x) > 0\),则\(\{x_n\}\)是单调的.
	\item 如果\(f'(x) < 0\),则\(\{x_n\}\)不是单调的.
\end{itemize}
\begin{proof}
假设\(f'(x) > 0\)且\(x_2 \geq x_1\),
那么\(x_3 = f(x_2) \geq f(x_1) = x_2\),
利用数学归纳法可证\(x_{n+1} \geq x_n\)对\(n=1,2,\dotsc\)成立.
因此,当\(f'(x) > 0\)时,\(\{x_n\}\)是单调的.

假设\(f'(x) < 0\)且\(x_2 \geq x_1\),
那么\(x_3 = f(x_2) \leq f(x_1) = x_2\),
显然,当\(f'(x) < 0\)时,\(\{x_n\}\)不是单调的.
\end{proof}
\end{example}

\subsection{曲线的凹凸性与拐点}
\begin{definition}
%@see: 《数学分析(第7版 第一卷)》(卓里奇) P202 定义1
%@see: 《数学分析(第7版 第一卷)》(卓里奇) P203 定义2
%@see: 《数学分析(第二版 上册)》(陈纪修) P173 定义5.1.2
设函数\(f\colon D\to\mathbb{R}\),其中\(D=(a,b)\subset\mathbb{R}\).

若\[
	(\forall x_1,x_2 \in D)
	(\forall \lambda \in [0,1])
	[
		f(\lambda x_1 + (1-\lambda) x_2)
		\leq
		\lambda f(x_1) + (1-\lambda) f(x_2)
	],
\]
则称“\(f(x)\)是\(D\)上的\DefineConcept{凹函数}(convex function)”.

若\[
	(\forall x_1,x_2 \in D)
	(\forall \lambda \in [0,1])
	[
		f(\lambda x_1 + (1-\lambda) x_2)
		\geq
		\lambda f(x_1) + (1-\lambda) f(x_2)
	],
\]
则称“\(f(x)\)是\(D\)上的\DefineConcept{凸函数}(concave function)
\footnote{有的书把凹函数称为\DefineConcept{下凸函数},
把凸函数称为\DefineConcept{上凸函数}.
% 例如:
% 《数学分析(第二版 上册)》(陈纪修)
有的书反过来把凹函数称为凸函数,
把凸函数称为凹函数.
% 例如:
% 《数学分析(第7版 第一卷)》(卓里奇)
}”.

若\[
	(\forall x_1,x_2 \in D)
	[
		x_1 \neq x_2
		\implies
		(\forall \lambda \in (0,1))
		[
			f(\lambda x_1 + (1-\lambda) x_2)
			<
			\lambda f(x_1) + (1-\lambda) f(x_2)
		]
	],
\]
则称“\(f(x)\)是\(D\)上的\DefineConcept{严格凹函数}
(strictly convex function)”.

若\[
	(\forall x_1,x_2 \in D)
	[
		x_1 \neq x_2
		\implies
		(\forall \lambda \in (0,1))
		[
			f(\lambda x_1 + (1-\lambda) x_2)
			>
			\lambda f(x_1) + (1-\lambda) f(x_2)
		]
	],
\]
则称“\(f(x)\)是\(D\)上的\DefineConcept{严格凸函数}
(strictly concave function)”.
%@see: https://mathworld.wolfram.com/ConvexFunction.html
%@see: https://mathworld.wolfram.com/ConcaveFunction.html
%@see: https://doi.org/10.1007/978-3-030-41804-5
\end{definition}

\begin{definition}
设函数\(f\colon D\to\mathbb{R}\).

若\[
	(\exists\alpha>0)
	[\text{函数\(g(x)=f(x)-\alpha\abs{x}^2\)是凹函数}],
\]
则称“\(f(x)\)是\(D\)上的\DefineConcept{强凹函数}
(strongly convex function)”.

若\[
	(\exists\alpha>0)
	[\text{函数\(g(x)=f(x)-\alpha\abs{x}^2\)是凸函数}],
\]
则称“\(f(x)\)是\(D\)上的\DefineConcept{强凸函数}
(strongly concave function)”.
%@see: https://www.princeton.edu/~aaa/Public/Teaching/ORF523/S16/ORF523_S16_Lec7_gh.pdf
\end{definition}

\begin{proposition}
强凹(凸)函数必定严格凹(凸).
\end{proposition}

\begin{proposition}[延森不等式]
%@see: 《数学分析(第7版 第一卷)》(卓里奇) P207 命题7(延森不等式)
%@see: 《数学分析(第二版 上册)》(陈纪修) P176 定理5.1.8
%@see: 《数学分析(第二版 上册)》(陈纪修) P184 习题 24.(Jensen不等式)
设函数\(f\colon(a,b)\to\mathbb{R}\),
\(\AutoTuple{x}{n}\in(a,b)\),
\(\AutoTuple{k}{n}\in(0,1)\)
且\(\AutoTuple{k}{n}[+]=1\).
\begin{itemize}
	\item 如果\(f\)是凸函数,
	则\[
		%@see: 《数学分析(第7版 第一卷)》(卓里奇) P207 (15)
		f(k_1 x_1 + \dotsb + k_n x_n)
		\geq
		k_1 f(x_1) + \dotsb + k_n f(x_n).
	\]
	\item 如果\(f\)是凹函数,
	则\[
		%@see: 《数学分析(第7版 第一卷)》(卓里奇) P207 (14)
		f(k_1 x_1 + \dotsb + k_n x_n)
		\leq
		k_1 f(x_1) + \dotsb + k_n f(x_n).
	\]
\end{itemize}
如果\(\AutoTuple{k}{n}\)均不为零,
那么上述两条不等式的取等条件就是\(x_1=x_2=\dotsb=x_n\).
\end{proposition}

\begin{theorem}[曲线凹凸的判定]\label{theorem:微分中值定理.曲线凹凸的判定}
%@see: 《数学分析(第二版 上册)》(陈纪修) P173 定理5.1.6(二阶导数与凹凸性的关系)
设\(f\colon[a,b]\to\mathbb{R}\)在\([a,b]\)上连续,在\((a,b)\)内具有一阶和二阶导数.
\begin{itemize}
	\item 若在\((a,b)\)内\(f''(x)>0\),
	则\(f\)是\([a,b]\)上的凹函数.
	\item 若在\((a,b)\)内\(f''(x)<0\),
	则\(f\)是\([a,b]\)上的凸函数.
\end{itemize}
\begin{proof}
在情形1,设\(x_1\)和\(x_2\)为\([a,b]\)内任意两点,且\(x_1 < x_2\),
记\(\frac{x_1 + x_2}{2} = x_0\),
并记\(x_2 - x_0 = x_0 - x_1 = h\),
则\(x_1 = x_0 - h\),\(x_2 = x_0 + h\),
由拉格朗日中值公式,
得\begin{gather*}
	f(x_0 + h) - f(x_0) = f'(x_0 + \theta_1 h) h, \\
	f(x_0) - f(x_0 - h) = f'(x_0 - \theta_2 h) h,
\end{gather*}
其中\(0 < \theta_1 < 1\),\(0 < \theta_2 < 1\).
两式相减,即得\[
	f(x_0 + h) + f(x_0 - h) - 2 f(x_0)
	= [ f'(x_0 + \theta_1 h) - f'(x_0 - \theta_2 h) ] h.
\]
对\(f'(x)\)在区间\([x_0 - \theta_2 h,x_0 + \theta_1 h]\)上再利用拉格朗日中值公式,
得\[
	[ f'(x_0 + \theta_1 h) - f'(x_0 - \theta_2 h) ] h
	= f''(\xi) (\theta_1 + \theta_2) h^2,
\]
其中\(x_0 - \theta_2 h < \xi < x_0 + \theta_1 h\).
按情形1的假设,\(f''(\xi) > 0\),
故有\[
	f(x_0 + h) + f(x_0 - h) - 2 f(x_0) > 0,
\]
即\[
	\frac{f(x_0 + h) + f(x_0 - h)}{2} > f(x_0),
\]
亦即\[
	\frac{f(x_1) + f(x_2)}{2} > f\left(\frac{x_1 + x_2}{2}\right),
\]
所以\(f(x)\)在\([a,b]\)上的图形是凹的.

类似地可证情形2.
\end{proof}
\end{theorem}

\begin{definition}
一般的,设\(y=f(x)\)在区间\(I\)上连续,
\(x_0\)是\(I\)的内点\footnote{区间\(I\)的内点是指除端点外的\(I\)内的点}.
如果曲线\(y=f(x)\)在经过点\((x_0,f(x_0))\)时,
曲线的凹凸性改变了,
那么就称“点\((x_0,f(x_0))\)是曲线\(y = f(x)\)的\DefineConcept{拐点}”.
\end{definition}

\begin{example}
函数\(f(x) = x^3\)以\(x=0\)为拐点.
函数\(g(x) = \abs{x(x-1)}\)以\(x=0\)和\(x=1\)为拐点.
\end{example}

由\cref{theorem:微分中值定理.曲线凹凸的判定} 可知,由\(f''(x)\)的符号可以判定曲线的凹凸性.
因此,如果\(f''(x)\)在\(x_0\)的左右两侧邻近异号,那么点\((x_0,f(x_0))\)就是曲线的一个拐点,
所以,要寻找拐点,只要找出\(f''(x)\)符号发生变化的分界点即可.
如果\(f(x)\)在区间\((a,b)\)内具有二阶连续导数,那么在这样的分界点处必然有\(f''(x)=0\);
除此以外,\(f(x)\)的二阶导数不存在的点,也有可能是\(f''(x)\)的符号发生变化的分界点.

{\color{red}
综合以上分析,我们可以按下列步骤来判定区间\(I\)上的连续曲线\(y=f(x)\)的拐点:
\begin{enumerate}
	\item 求\(f''(x)\);

	\item 令\(f''(x) = 0\),解出这个方程在区间\(I\)内的实根,
	并求出在区间\(I\)内\(f''(x)\)不存在的点;

	\item 对于上一步中求出的每一个实根或二阶导数不存在的点\(x_0\),
	检查\(f''(x)\)在\(x_0\)左右两侧邻近的符号,
	那么当两侧的符号相反时,点\((x_0,f(x_0))\)就是拐点;
	当两侧的符号相同时,点\((x_0,f(x_0))\)不是拐点.
\end{enumerate}}

\begin{example}
证明:当\(x>0\),\(y>0\),\(x \neq y\),且\(n>1\)时,
有\begin{equation}\label{equation:微分中值定理.平均数的比较1}
	\frac{1}{2} (x^n+y^n) > \left(\frac{x+y}{2}\right)^n
\end{equation}成立.
\begin{proof}
设\(f(x) = x^n\),那么\(f'(x) = n x^{n-1}\),\(f''(x) = n(n-1) x^{n-2}\).
由指数函数的性质可知,
当\(x > 0\)时,\(\forall \mu \in \mathbb{R}\)都有\(x^{\mu} > 0\)成立,
故当\(n > 1\)时,\(n(n-1)>0\),\(f''(x) = n(n-1) x^{n-1} > 0\)成立,
即\(f(x)\)在\((0,+\infty)\)上是凹函数,
那么\(\forall x,y>0\),
只要\(x \neq y\),
就有\[
	\frac{f(x)+f(y)}{2} = \frac{x^n+y^n}{2}
	> \left(\frac{x+y}{2}\right)^n = f\left(\frac{x+y}{2}\right)
\]成立.
\end{proof}
\end{example}

\begin{example}
证明:当\(x \neq y\)时,有\begin{equation}
	\frac{e^x + e^y}{2} > e^{\frac{x+y}{2}}.
\end{equation}
\begin{proof}
设\(f(x) = e^x\),那么\(f'(x) = f''(x) = e^x > 0\ (x\in\mathbb{R})\),可知\(f(x)\)是凹函数.
因此\[
	f\left(\frac{x+y}{2}\right) = e^{\frac{x+y}{2}}
	< \frac{e^x+e^y}{2} = \frac{f(x)+f(y)}{2}.
	\qedhere
\]
\end{proof}
\end{example}
