\section{数列极限的四则运算}
\begin{theorem}\label{theorem:极限.数列极限的四则运算法则}
%@see: 《高等数学(第六版 上册)》 P45 定理4
%@see: 《数学分析(上册)》(陈纪修) P42 定理2.2.5
设数列\(\{x_n\}\)和\(\{y_n\}\)
满足\(\lim_{n\to\infty} x_n = A\)
和\(\lim_{n\to\infty} y_n = B\),
那么\begin{enumerate}
	\item \(\lim_{n\to\infty} (x_n \pm y_n) = A \pm B\);
	\item \(\lim_{n\to\infty} (x_n \cdot y_n) = A \cdot B\);
	\item 当\(y_n \neq 0\ (n=1,2,\dotsc,)\)且\(B \neq 0\)时,
	\(\lim_{n\to\infty}{\frac{x_n}{y_n}}=\frac{A}{B}\).
\end{enumerate}
\end{theorem}

\begin{corollary}
%@see: 《数学分析(上册)》(陈纪修) P42 定理2.2.5
设\(\lim_{n\to\infty} x_n = A,
\lim_{n\to\infty} y_n = B\),
\(a,b\)是常数,
则\(\lim_{n\to\infty} (a x_n + b y_n) = a A + b B\).
\end{corollary}

\begin{example}\label{example:极限.常数的方根的极限2}
%@see: 《数学分析(上册)》(陈纪修) P43 例2.2.10
设\(a>0\),证明:\(\lim_{n\to\infty} \sqrt[n]{a} = 1\).
\begin{proof}
在\cref{example:极限.常数的方根的极限1} 中我们已经证明
当\(a>1\)时\(\lim_{n\to\infty} \sqrt[n]{a} = 1\).
当\(a=1\)时,\(\sqrt[n]{a}\)恒等于\(1\),从而也有\(\lim_{n\to\infty} \sqrt[n]{a} = 1\).
现在考虑\(0<a<1\),这时\(\frac1a>1\),从而\(\lim_{n\to\infty} \sqrt[n]{\frac1a} = 1\),
那么利用\hyperref[theorem:极限.数列极限的四则运算法则]{极限的四则运算法则}可得
\(\lim_{n\to\infty} \sqrt[n]{a}
= \lim_{n\to\infty} \frac1{\sqrt[n]{1/a}} = 1\).
\end{proof}
\end{example}

\begin{example}
%@see: 《数学分析(上册)》(陈纪修) P43 例2.2.11
求极限\(\lim_{n\to\infty} n(\sqrt{n^2+1}-\sqrt{n^2-1})\).
\begin{solution}
直接计算得\begin{align*}
	\lim_{n\to\infty} n(\sqrt{n^2+1}-\sqrt{n^2-1})
	&= \lim_{n\to\infty} \frac{2n}{\sqrt{n^2+1}+\sqrt{n^2-1}} \\
	&= \lim_{n\to\infty} \frac2{\sqrt{1+(1/n)^2}+\sqrt{1-(1/n)^2}}
	= 1.
\end{align*}
\end{solution}
\end{example}

\begin{corollary}
%@see: 《数学分析(上册)》(陈纪修) P43
设\(\lim_{n\to\infty} x_{in} = A_i\ (i=1,2,\dotsc,m)\),
则\[
	\lim_{n\to\infty} \sum_{i=1}^m x_{in} = \sum_{i=1}^m A_i,
\]\[
	\lim_{n\to\infty} \prod_{i=1}^m x_{in} = \prod_{i=1}^m A_i.
\]
\end{corollary}

\begin{remark}
数列极限的四则运算法则只能推广到有限个数列相加或相乘的情况,
不能随意推广到无限个数列上去.
例如,若将\cref{theorem:极限.数列极限的四则运算法则} 随意推广,
可能会得出极限\[
	\lim_{n\to\infty} \left(
		\frac1{\sqrt{n^2+1}}
		+ \frac1{\sqrt{n^2+2}}
		+ \dotsb + \frac1{\sqrt{n^2+n}}
	\right)
\]为\(0\)的错误结论;
但是,由于\[
	\frac{n}{\sqrt{n^2+n}}
	< \frac1{\sqrt{n^2+1}}
	+ \frac1{\sqrt{n^2+2}}
	+ \dotsb + \frac1{\sqrt{n^2+n}}
	< \frac{n}{\sqrt{n^2+1}},
\]
利用\hyperref[theorem:极限.夹逼准则]{夹逼准则}可知该极限其实为\(1\).
\end{remark}

\begin{example}\label{example:极限.收敛数列前n项积的n次方根}
%@see: 《数学分析(上册)》(陈纪修) P44 例2.2.12
设\(a_n>0\),\(\lim_{n\to\infty} a_n = a\),证明:
\(\lim_{n\to\infty} \sqrt[n]{a_1 a_2 \dotsm a_n} = a\).
\begin{proof}
应用\hyperref[theorem:不等式.均值不等式]{均值不等式}可得\[
	\frac{a_1+a_2+\dotsb+a_n}n
	\geq \sqrt[n]{a_1 a_2 \dotsm a_n}
	\geq n\left(\frac1{a_1}+\frac1{a_2}+\dotsb+\frac1{a_n}\right)^{-1}
	> 0.
\]
由\cref{example:极限.数列的算术平均的极限} 我们已经知道
\(\lim_{n\to\infty} \frac{a_1+a_2+\dotsb+a_n}{n} = a\).
要想利用\hyperref[theorem:极限.夹逼准则]{夹逼准则}证明上述问题,
就必须证明上面不等式中\(\sqrt[n]{a_1 a_2 \dotsm a_n}\)右边的数列极限也等于\(a\).

下面按\(a\)的不同取值,分两种情况讨论:
\begin{itemize}
	\item 当\(a=0\)时,
	有\(\sqrt[n]{a_1 a_2 \dotsm a_n} > 0 = a\).

	\item 当\(a>0\)时,
	由\hyperref[theorem:极限.数列极限的四则运算法则]{数列极限的四则运算法则}可知
	\(\lim_{n\to\infty} \frac1{a_n} = \frac1a\),
	故\[
		\lim_{n\to\infty} \frac1n \left(\frac1{a_1}+\frac1{a_2}+\dotsb+\frac1{a_n}\right) = \frac1a,
	\]
	从而
	\(\lim_{n\to\infty} n\left(\frac1{a_1}+\frac1{a_2}+\dotsb+\frac1{a_n}\right)^{-1} = a\).
\end{itemize}
在上述两种情况下,\(\sqrt[n]{a_1 a_2 \dotsm a_n}\)右边的数列极限都等于\(a\),
所以\(\lim_{n\to\infty} \sqrt[n]{a_1 a_2 \dotsm a_n} = a\).
\end{proof}
\end{example}

\begin{example}
%@see: 《数学分析(上册)》(陈纪修) P45 习题 8.(1)
求极限\(\lim_{n\to\infty} \sqrt[n]{1+\frac12+\dotsb+\frac1n}\).
\begin{solution}
首先我们有\(1 \leq 1+\frac12+\dotsb+\frac1n \leq n\),
因为\(\lim_{n\to\infty} \sqrt[n]{1} = \lim_{n\to\infty} \sqrt[n]{n} = 1\),
所以\[
	\lim_{n\to\infty} \sqrt[n]{1+\frac12+\dotsb+\frac1n} = 1.
\]
\end{solution}
\end{example}

\begin{example}
%@see: 《数学分析(上册)》(陈纪修) P45 习题 8.(2)
求极限\(\lim_{n\to\infty} \left(\frac1{n+\sqrt1}+\frac1{n+\sqrt2}+\dotsb+\frac1{n+\sqrt{n}}\right)\).
\begin{solution}
首先有\(\frac{n}{n+\sqrt{n}}
\leq \frac1{n+\sqrt1}+\frac1{n+\sqrt2}+\dotsb+\frac1{n+\sqrt{n}}
\leq \frac{n}{n+1}\).
因为\[
	\lim_{n\to\infty} \frac{n}{n+\sqrt{n}}
	= \lim_{n\to\infty} \frac{n}{n+1} = 1,
\]
所以\(\lim_{n\to\infty} \left(\frac1{n+\sqrt1}+\frac1{n+\sqrt2}+\dotsb+\frac1{n+\sqrt{n}}\right) = 1\).
\end{solution}
\end{example}

\begin{example}
%@see: 《数学分析(上册)》(陈纪修) P45 习题 9.(7)
求极限\(\lim_{n\to\infty} \sqrt[n]{\frac1{n!}}\).
\begin{solution}\let\qed\relax
\begin{proof}[解法一]
因为\(\lim_{n\to\infty} \frac1n = 0\),
所以由\cref{example:极限.收敛数列前n项积的n次方根} 可知
\(\lim_{n\to\infty} \sqrt[n]{\frac1{n!}}
= \lim_{n\to\infty} \sqrt[n]{\frac11 \cdot \frac12 \dotsm \frac1n}
= 0\).
\end{proof}
\begin{proof}[解法二]
注意到\[
	\left(\frac1{n!}\right)^2
	= \prod_{k=1}^n \frac1{(n-k+1)k},%首尾相乘
\]
而\(\frac1{(n-k+1)k} \leq \frac1n\),
\(\prod_{k=1}^n \frac1{(n-k+1)k} \leq \frac1{n^n}\),
\(\frac1{n!} \leq \sqrt{\frac1{n^n}}\),
\(0 < \sqrt[n]{\frac1{n!}} \leq \frac1{\sqrt{n}}\),
加之\(\lim_{n\to\infty} \frac1{\sqrt{n}} = 0\),
因此根据\hyperref[theorem:极限.夹逼准则]{夹逼准则}可得
\(\lim_{n\to\infty} \sqrt[n]{\frac1{n!}} = 0\).
\end{proof}
\end{solution}
\end{example}

\begin{example}
%@see: 《数学分析(上册)》(陈纪修) P45 习题 9.(8)
求极限\(\lim_{n\to\infty} \left(1-\frac1{2^2}\right) \left(1-\frac1{3^2}\right) \dotsm \left(1-\frac1{n^2}\right)\).
\begin{solution}
直接计算得\begin{align*}
	&\hspace{-20pt}
	\lim_{n\to\infty} \left(1-\frac1{2^2}\right) \left(1-\frac1{3^2}\right) \dotsm \left(1-\frac1{n^2}\right) \\
	&= \lim_{n\to\infty} \frac{1\cdot3}{2^2} \cdot \frac{2\cdot4}{3^2} \cdot \frac{3\cdot5}{4^2} \dotsm \frac{(n-2)n}{(n-1)^2} \cdot \frac{(n-1)(n+1)}{n^2} \\
	&= \lim_{n\to\infty} \frac{2n(n+1)}{2^2 n^2}
	= \lim_{n\to\infty} \frac{n+1}{2n}
	= \frac12.
\end{align*}
\end{solution}
\end{example}

\begin{example}
%@see: 《数学分析(上册)》(陈纪修) P45 习题 9.(9)
求极限\(\lim_{n\to\infty} \sqrt[n]{n \ln n}\).
\begin{solution}
因为\(n < n \ln n < n^2\ (n\geq3)\),
\(\lim_{n\to\infty} \sqrt[n]{n}
= \lim_{n\to\infty} \sqrt[n]{n^2} = 1\),
所以\(\lim_{n\to\infty} \sqrt[n]{n \ln n} = 1\).
\end{solution}
\end{example}
