\section{收敛准则}
\subsection{单调有界数列收敛定理}
由\cref{theorem:极限.收敛数列的有界性} 可知,收敛的数列一定有界.
但是,我们也知道,有界的数列不一定收敛,例如\[
	\{ x_n = \sin n \}, \qquad
	\{ y_n = (-1)^n \}.
\]
于是我们有这样两个问题:
对有界数列加上什么条件,就可以保证它必定收敛?
不对有界数列加任何条件,我们可以得到怎样的(比收敛稍弱一些的)结论?

我们先来回答第一个问题:
如果数列不仅有界,而且是单调的,那么这数列一定收敛,
其极限就是它的值域的上确界或下确界.

\begin{theorem}\label{theorem:极限.数列的单调有界定理}
%@see: 《高等数学(第六版 上册)》 P52 准则II
%@see: 《数学分析(上册)》(陈纪修) P52 定理2.4.1
单调有界数列必有极限.
\begin{proof}
不妨设数列\(\{a_n\}\)是单调增加的,
即\[
	a_n \leq a_{n+1},
	\quad n=1,2,\dotsc;
	\eqno(1)
\]
又设\(\{a_n\}\)有界,
且\[
	\abs{a_n} < c,
	\quad n=1,2,\dotsc.
	\eqno(2)
\]

现在我们把连续统分成两个集合\(A\)和\(B\),
把大于所有\(a_n\)的任何实数(例如数\(c\))放入集合\(B\),
而把其余的所有实数放入\(A\),即取\[
	B = \Set{ x \in \mathbb{R} \given x > a_n\ (n=1,2,\dotsc) },
	\eqno(3)
\]\[
	A = \mathbb{R} - B.
	\eqno(4)
\]
显然\(\Set{A,B}\)是\(\mathbb{R}\)的一个分割.
设\(\alpha\)是这个分割的界限,
那么必有\[
	a_n \leq \alpha,
	\quad n=1,2,\dotsc;
	\eqno(5)
\]
这是因为假设这个数列的某一项\(a_m\)满足\(a_m > \alpha\),
依照界限的定义会有\(a_m \in B\),而这与\(B\)的定义式(3)矛盾.

假设“\(\alpha\)不是\(\{a_n\}\)的极限”,
根据数列发散的定义,\[
	(\exists\epsilon>0)
	(\forall n\in\mathbb{N}^+)
	(\exists n_0>n)
	[\abs{a_{n_0} - \alpha} > \epsilon]
\]成立;
由(5)可知,\(\abs{a_{n_0} - \alpha} = \alpha - a_{n_0}\);
又因为\(\{a_n\}\)是单调增加的,
所以\(a_{n_0} \geq a_n\),
\(-a_{n_0} \leq -a_n\),
\(\alpha - a_{n_0} \leq \alpha - a_n\).
因此,\(\exists\epsilon>0\),对\(\forall n\in\mathbb{N}^+\),都有\[
	\alpha - a_n > \epsilon
	\quad\text{或}\quad
	a_n < \alpha - \epsilon.
	\eqno(6)
\]
结合集合\(B\)的定义(3),
由(6)便得\(\alpha - \epsilon \in B\);
但由\(\alpha - \epsilon < \alpha\)可知,应该有\((\alpha - \epsilon) \in A\);
矛盾!因此假设不成立,\(\alpha\)就是数列\(\{a_n\}\)的极限,
即\(\lim_{n\to\infty} a_n = \alpha\).
\end{proof}
\end{theorem}

按极限的定义证明一个数列收敛必须提前知道它的极限是什么.
这个要求对于许多实际情况来说并不现实,
我们往往无法事先得知收敛数列的极限.
\cref{theorem:极限.数列的单调有界定理} 使得我们可以从数列本身出发去研究其敛散性,
在确定数列数列收敛以后,再利用极限运算的性质去求出相应的极限.

\begin{example}
%@see: 《数学分析(上册)》(陈纪修) P53 例2.4.1
设数列\(\{x_n\}\)满足\(x_1>0\),且有递推公式\(x_{n+1}=1+\frac{x_n}{1+x_n}\ (n=1,2,\dotsc)\).
证明:\(\{x_n\}\)收敛,并求它的极限.
\begin{solution}
首先,应用数学归纳法可以直接得到\(1<x_n<2\ (n\geq2)\).
然后由递推公式可得\[
	x_{n+1}-x_n = \frac{x_n-x_{n-1}}{(1+x_n)(1+x_{n-1})}.
\]
这说明,对于\(\forall n\geq2\),\(x_{n+1}-x_n\)具有相同的符号,从而\(\{x_n\}\)是单调数列.
由\cref{theorem:极限.数列的单调有界定理} 可知,\(\{x_n\}\)收敛.

设\(\{x_n\}\)的极限是\(a\).
在递推公式等号两边同时求极限,
得到方程\[
	a = 1 + \frac{a}{1+a},
\]
解得\(a = \frac{1\pm\sqrt5}2\).
由\(x_n>0\),舍去负值,即有\(\lim_{n\to\infty} x_n = \frac{1+\sqrt5}2\).
\end{solution}
\end{example}

\begin{example}
%@see: 《数学分析(上册)》(陈纪修) P53 例2.4.2
设数列\(\{x_n\}\)满足\(0<x_1<1\),且有递推公式\(x_{n+1}=x_n(1-x_n)\ (n=1,2,\dotsc)\).
证明:\(\{x_n\}\)收敛,并求它的极限.
\begin{solution}
应用数学归纳法,可以得到\(0<x_n<1\ (n=1,2,\dotsc)\).
然后由递推公式可得\[
	x_{n+1}-x_n = -x_n^2 < 0.
\]
这说明\(\{x_n\}\)单调减少有下界.
由\cref{theorem:极限.数列的单调有界定理} 可知,\(\{x_n\}\)收敛.

设\(\{x_n\}\)的极限是\(a\).
在递推公式等号两边同时求极限,
得到方程\[
	a=a(1-a),
\]
解得\(a=0\).
于是\(\lim_{n\to\infty} x_n = 0\).
\end{solution}
\end{example}

\begin{example}
%@see: 《数学分析(上册)》(陈纪修) P54 例2.4.3
设数列\(\{x_n\}\)满足\(x_1=\sqrt2\),且有递推公式\(x_{n+1}=\sqrt{3+2x_n}\ (n=1,2,\dotsc)\).
证明:\(\{x_n\}\)收敛,并求它的极限.
\begin{solution}
对\(0<x_1=\sqrt2<3\)应用数学归纳法可得\(0<x_n<3\ (n=1,2,\dotsc)\).
然后由递推公式可得\[
	x_{n+1}-x_n = \sqrt{3+2x_n} - x_n
	= \frac{(3-x_n)(1+x_n)}{\sqrt{3+2x_n}+x_n}
	> 0,
\]
这说明\(\{x_n\}\)单调增加且有上界.
由\cref{theorem:极限.数列的单调有界定理} 可知,\(\{x_n\}\)收敛.
设\(\{x_n\}\)的极限是\(a\).
在递推公式等号两边同时求极限,
得到方程\[
	a=\sqrt{3+2a},
\]
解得\(a\in\{3,-1\}\).
由\(x_n>0\),舍去负值,即有\(\lim_{n\to\infty} x_n = 3\).
\end{solution}
\end{example}

\subsection{无理数\texorpdfstring{$\pi$}{\textpi}}
设单位圆内接正\(n\)边形的半周长为\(L_n\),
则\(L_n = n \sin\frac{180^\circ}{n}\).
%@see: 《数学分析(上册)》(陈纪修) P56 例2.4.5
令\(t=\frac{180^\circ}{n(n+1)}\).
当\(n\geq3\)时,\(nt\leq45^\circ\).
于是\[
	\tan nt
	= \frac{\tan(n-1)t + \tan t}{1 - \tan(n-1)t \tan t}
	\geq \tan(n-1)t + \tan t
	\geq \dotsb \geq n \tan t,
\]
从而\[
	\sin(n+1)t = \sin nt \cos t + \cos nt \sin t
	= \sin nt \cos t \left(1 + \frac{\tan t}{\tan nt}\right)
	\leq \frac{n+1}n \sin nt.
\]
所以,当\(n\geq3\)时,
\(L_n = n \sin\frac{180^\circ}{n}
\leq (n+1) \sin\frac{180^\circ}{n+1} = L_{n+1}\).

另一方面,单位圆内接正\(n\)边形的面积\[
	S_n = n \sin\frac{180^\circ}{n} \cos\frac{180^\circ}{n} < 4.
\]
因此,当\(n\geq3\)时,
\(L_n = n \sin\frac{180^\circ}{n}
< 4 \left(\cos\frac{180^\circ}{n}\right)^{-1}
\leq \frac4{\cos60^\circ}
= 8\).

综上所述,数列\(\{L_n\}\)单调增加且有上界,因此收敛.
我们把\(\lim_{n\to\infty} L_n\)称为圆周率,记作\(\pi\),
即\[
	\pi \defeq \lim_{n\to\infty} n \sin\frac{180^\circ}{n}.
\]

\subsection{无理数\texorpdfstring{$e$}{e}}
设\(x_n=\left(1+\frac1n\right)^n,
y_n=\left(1+\frac1n\right)^{n+1}\).
我们接下来证明数列\(\{x_n\}\)单调增加,
数列\(\{y_n\}\)单调减少,
且两者收敛于同一个极限.

由\cref{theorem:不等式.基本不等式n几何平均数与算术平均数} 可得\[
	x_n = \left(1+\frac1n\right)^n \cdot 1
	\leq \left[\frac{n\left(1+\frac1n\right)+1}{n+1}\right]^{n+1} = x_{n+1},
\]
和\[
	\frac1{y_n} = \left(\frac{n}{n+1}\right)^{n+1} \cdot 1
	\leq \left[\frac{(n+1)\frac{n}{n+1}+1}{n+2}\right]^{n+2} = \frac1{y_{n+1}}.
\]
可见\(\{x_n\}\)单调增加,\(\{y_n\}\)单调减少.
又因为\[
	2 = x_1 \leq x_n < y_n \leq y_1 = 4,
\]
所以\(\{x_n\},\{y_n\}\)都收敛.
因为\(y_n=x_n\left(1+\frac1n\right)\),
所以\(\{x_n\},\{y_n\}\)具有相同的极限.
我们习惯上用字母\(e\)表记这个极限,
即\[
	e \defeq \lim_{n\to\infty} \left(1+\frac1n\right)^n.
\]

\subsection{闭区间套定理}
\begin{definition}\label{definition:极限.闭区间套的定义}
闭区间序列\(\{[a_n,b_n]\}\)如果满足
\begin{enumerate}
	\item \([a_{n+1},b_{n+1}] \subseteq [a_n,b_n]\ (n=1,2,\dotsc)\);
	\item \(\lim_{n\to\infty} (b_n - a_n) = 0\),
\end{enumerate}
就称该序列为\DefineConcept{闭区间套}(nested intervals).
\end{definition}

\begin{theorem}\label{definition:极限.闭区间套定理}
如果序列\(\{[a_n,b_n]\}\)是闭区间套,
那么\[
	(\exists!\xi\in\mathbb{R})
	\left[
		\xi \in \bigcap_{i=1}^\infty [a_i,b_i]
		\land
		\lim_{n\to\infty} a_n = \lim_{n\to\infty} b_n = \xi
	\right].
\]
\begin{proof}
根据闭区间套的定义,我们有\[
	a_1 \leq a_2 \leq \dotsb \leq a_{n-1} \leq a_n
	< b_n \leq b_{n-1} \leq \dotsb \leq b_2 \leq b_1.
\]
显然数列\(\{a_n\}\)单调增加且有上界\(b_1\),
数列\(\{b_n\}\)单调减少而有下界\(\{a_1\}\).
根据\hyperref[theorem:极限.数列的单调有界定理]{数列的单调有界定理},
\(\{a_n\}\)与\(\{b_n\}\)都收敛.

设\(\lim_{n\to\infty} a_n = \xi\),
则\[
	\lim_{n\to\infty} b_n
	= \lim_{n\to\infty} [(b_n - a_n) + a_n]
	= \lim_{n\to\infty} (b_n - a_n) + \lim_{n\to\infty} a_n
	= \xi.
\]
由于\(\xi\)是\(\{a_n\}\)的值域的上确界,
也是\(\{b_n\}\)的值域的下确界,
于是有\(a_n \leq \xi \leq b_n\ (n=1,2,\dotsc)\),
即\(\xi\)属于所有的闭区间\([a_n,b_n]\).
假设另有实数\(\eta\)属于所有的闭区间\([a_n,b_n]\),
则也有\(a_n \leq \eta \leq b_n\ (n=1,2,\dotsc)\).
令\(n\to\infty\),
由\hyperref[theorem:极限.夹逼准则]{夹逼准则}得到\(\eta=\lim_{n\to\infty} a_n=\xi\).
这就说明满足结论的实数\(\xi\)是唯一的.
\end{proof}
\end{theorem}

需要指出,如果将\cref{definition:极限.闭区间套定理} 条件中的闭区间套改为开区间套,
则数列\(\{a_n\},\{b_n\}\)依然收敛于同一个极限\(\xi\),
但是这个\(\xi\)可能不属于任何一个开区间\((a_n,b_n)\).
例如,令\(a_n = 0, b_n = \frac1n\),
易见\(\lim_{n\to\infty} a_n = \lim_{n\to\infty} b_n = 0\),
但是\(0\)不属于任何一个开区间\((a_n,b_n)\).
