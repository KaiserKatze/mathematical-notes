\begin{theorem}
%@see: 《数学分析(第二版 上册)》(陈纪修) P47 定理2.3.1
设\(x_n\neq0\),
则\(\{x_n\}\)是无穷大的充分必要条件是\(\{1/x_n\}\)是无穷小.
\begin{proof}
设\(\{x_n\}\)是无穷大.
对\(\forall\epsilon>0\),
取\(G = \frac1\epsilon\),
那么\begin{equation*}
	\text{\(\{x_n\}\)是无穷大}
	\implies
	(\exists N\in\mathbb{N})
	(\forall n\in\mathbb{N})
	\left[
		n>N
		\implies
		\abs{x_n} > G = \frac1\epsilon
		\implies
		\abs{\frac1{x_n}} < \epsilon
	\right],
\end{equation*}
即\(\{1/x_n\}\)是无穷小.

反过来,设\(\{1/x_n\}\)是无穷小.
对\(\forall G>0\),
取\(\epsilon = \frac1G\),
那么\begin{equation*}
	\text{\(\{1/x_n\}\)是无穷小}
	\implies
	(\exists N\in\mathbb{N})
	(\forall n\in\mathbb{N})
	\left[
		n>N
		\implies
		\abs{\frac1{x_n}} < \epsilon = \frac1G
		\implies
		\abs{x_n} > G
	\right],
\end{equation*}
即\(\{x_n\}\)是无穷大.
\end{proof}
\end{theorem}

关于无穷大的运算,如下的性质是显然的:
同号无穷大之和仍然是该符号的无穷大,
而异号无穷大之差是与被减无穷大的符号相同的无穷大,
无穷大与有界量之和或之差仍然是无穷大,
同号无穷大之积是正无穷大,
异号无穷大之积是负无穷大,
\begin{gather*}
	(+\infty) + (+\infty) = +\infty, \\
	(-\infty) + (-\infty) = -\infty, \\
	(+\infty) - (-\infty) = +\infty, \\
	(-\infty) - (+\infty) = -\infty, \\
	(+\infty) \cdot (+\infty) = +\infty, \\
	(-\infty) \cdot (-\infty) = +\infty, \\
	(+\infty) \cdot (-\infty) = -\infty, \\
	(-\infty) \cdot (+\infty) = -\infty.
\end{gather*}

\begin{theorem}%无穷大与无界量之积还是无穷大
%@see: 《数学分析(第二版 上册)》(陈纪修) P47 定理2.3.2
设数列\(\{x_n\}\)是无穷大,若当\(n>N\)时\(\abs{y_n}\geq\delta>0\)成立,
则\(\{x_n y_n\}\)是无穷大.
\end{theorem}

\begin{corollary}%无穷大与收敛于非零数的数列之积还是无穷大
%@see: 《数学分析(第二版 上册)》(陈纪修) P48 推论
设\(\{x_n\}\)是无穷大,
\(\lim_{n\to\infty} y_n = b \neq 0\),
则\(\{x_n y_n\}\)与\(\{x_n/y_n\}\)都是无穷大.
\end{corollary}
