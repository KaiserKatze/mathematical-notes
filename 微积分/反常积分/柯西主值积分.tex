\section{柯西主值积分}
\subsection{第一类柯西主值积分}
%@see: https://mathworld.wolfram.com/CauchyPrincipalValue.html
\begin{definition}
%@see: 《数学分析(第二版 上册)》(陈纪修) P366 定义8.1.3
设函数\(f\colon\mathbb{R}\to\mathbb{R}\).
如果极限\begin{equation*}
	\lim_{R\to+\infty} \int_{-R}^R f(x) \dd{x}
\end{equation*}存在且有限,
则称“反常积分\(\int_{-\infty}^{+\infty} f(x) \dd{x}\)~\DefineConcept{在柯西主值意义下收敛}”,
并把这个极限称为
“反常积分\(\int_{-\infty}^{+\infty} f(x) \dd{x}\)
的\DefineConcept{柯西主值}(Cauchy principal value)”
或“函数\(f\)在区间\((-\infty,+\infty)\)上的积分的柯西主值
(the Cauchy principal value of the integral of \(f\) over \((-\infty,+\infty)\))”,
记为\begin{equation*}
	\pvint_{-\infty}^{+\infty} f(x) \dd{x},
\end{equation*}
即\begin{equation*}
	\pvint_{-\infty}^{+\infty} f(x) \dd{x}
	\defeq \lim_{R\to+\infty} \int_{-R}^R f(x) \dd{x}.
\end{equation*}
\end{definition}

若\(\int_{-\infty}^{+\infty} f(x) \dd{x}\)收敛,
则它在柯西主值意义下也收敛,
且\(\int_{-\infty}^{+\infty} f(x) \dd{x}\)的值
与它的柯西主值\(\pvint_{-\infty}^{+\infty} f(x) \dd{x}\)相等.
可是,当柯西主值\(\pvint_{-\infty}^{+\infty} f(x) \dd{x}\)收敛时,
\(\int_{-\infty}^{+\infty} f(x) \dd{x}\)本身未必一定收敛.
例如,\(\pvint_{-\infty}^{+\infty} x \dd{x} = 0\),
而\(\int_{-\infty}^{+\infty} x \dd{x}\)发散.
但在一般给出的问题中,要么只需要求柯西主值,
要么不难预先看出\(\int_{-\infty}^{+\infty} f(x) \dd{x}\)收敛,
因此,只要求出柯西主值\(\pvint_{-\infty}^{+\infty} f(x) \dd{x}\)的值,
也就求出了\(\int_{-\infty}^{+\infty} f(x) \dd{x}\)的值.

\subsection{第二类柯西主值积分}
\begin{definition}
设函数\(f\colon[a,b]\to\mathbb{R}\),点\(c\in(a,b)\)是一个瑕点.
如果极限\begin{equation*}
	\lim_{\epsilon\to0^+} \left[
		\int_a^{c-\epsilon} f(x) \dd{x}
		+ \int_{c+\epsilon}^b f(x) \dd{x}
	\right]
\end{equation*}存在且有限,
则称“反常积分\(\int_a^b f(x) \dd{x}\)~\DefineConcept{在柯西主值意义下收敛}”,
并把这个极限称为
“反常积分\(\int_a^b f(x) \dd{x}\)
的\DefineConcept{柯西主值}(Cauchy principal value)”,
记为\begin{equation*}
	\pvint_a^b f(x) \dd{x},
\end{equation*}
即\begin{equation*}
	\pvint_a^b f(x) \dd{x}
	\defeq \lim_{\epsilon\to0^+} \left[
		\int_a^{c-\epsilon} f(x) \dd{x}
		+ \int_{c+\epsilon}^b f(x) \dd{x}
	\right].
\end{equation*}
\end{definition}

%@Mathematica: Integrate[1/x, {x, -1, 2}, PrincipalValue -> True]
