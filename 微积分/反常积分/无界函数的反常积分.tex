\section{无界函数的反常积分}
本节我们讨论第二类反常积分 --- “无界函数的反常积分”.

先看一个具体的例子.
表达式\[
	\int_0^1 \frac{\dd{x}}{\sqrt{x}}
\]在黎曼积分的意义下是没有意义的,
因为被积函数\(\frac{1}{\sqrt{x}}\)在\(0\)的右邻域无界,
于是我们把点\(x=0\)称为这个积分的“瑕点”.
但是,对于\(\forall\epsilon\in(0,1)\),积分\[
	\int_\epsilon^1 \frac{\dd{x}}{\sqrt{x}}
\]是有意义的,
它是一个带有变动下限\(\epsilon\)的积分.
由于极限\[
	\lim_{\epsilon\to0^+} \int_\epsilon^1 \frac{\dd{x}}{\sqrt{x}}
	= \lim_{\epsilon\to0^+} 2\sqrt{x}\eval_\epsilon^1
	= 2 \lim_{\epsilon\to0^+} (1-\sqrt\epsilon)
	= 2
\]存在且有限,
我们就定义\[
	\int_0^1 \frac{\dd{x}}{\sqrt{x}}=2.
\]

\subsection{无界函数的反常积分的概念}
\begin{definition}\label{definition:定积分.无界函数的反常积分的定义1}
设函数\(f\colon(a,b]\to\mathbb{R}\)满足\[
	\lim_{x \to a^+} f(x) = \infty;
\]
但是对于\(\forall\epsilon\in(0,b-a)\),
函数\(f\)在\([a+\epsilon,b]\)上可积.

如果极限\[
	\lim_{\epsilon\to0^+} \int_{a+\epsilon}^b f(x) \dd{x}
\]存在且有限,
那么把这个极限称为“函数\(f(x)\)在\((a,b]\)上的\DefineConcept{广义积分}”,
记作\(\int_a^b f(x) \dd{x}\);
我们还称“反常积分\(\int_a^b f(x) \dd{x}\) \DefineConcept{收敛}”;
另外,我们把点\(a\)称为“反常积分\(\int_a^b f(x) \dd{x}\)的\DefineConcept{瑕点}”.

否则,称“反常积分\(\int_a^b f(x) \dd{x}\) \DefineConcept{发散}”.
\end{definition}

类似地,当\(b\)是瑕点时,
我们也可以定义反常积分\[
	\int_a^b f(x) \dd{x}
	\defeq
	\lim_{\epsilon\to0^+} \int_a^{b-\epsilon} f(x) \dd{x}.
\]
当\(a\)和\(b\)都是瑕点时,
那么任取一点\(c\in(a,b)\),我们定义\begin{align*}
	\int_a^b f(x) \dd{x}
	&\defeq
	\int_a^c f(x) \dd{x}
	+ \int_c^b f(x) \dd{x} \\
	&= \lim_{\epsilon\to0^+} \int_{a+\epsilon}^c f(x) \dd{x}
	+ \lim_{\epsilon\to0^+} \int_c^{b-\epsilon} f(x) \dd{x}.
\end{align*}
当开区间\((a,b)\)内一点\(d\)是瑕点时,
我们定义\begin{align*}
	\int_a^b f(x) \dd{x}
	&\defeq \int_a^d f(x) \dd{x}
		+ \int_d^b f(x) \dd{x} \\
	&= \lim_{\epsilon\to0^+} \int_a^{d-\epsilon} f(x) \dd{x}
		+ \lim_{\epsilon\to0^+} \int_{d+\epsilon}^b f(x) \dd{x}.
\end{align*}

\subsection{无界函数的反常积分的计算法}
计算无界函数的反常积分,也可借助于牛顿--莱布尼茨公式.
\begin{theorem}\label{theorem:定积分.利用牛顿莱布尼茨公式计算无界函数的反常积分1}
设函数\(f \in C(a,b]\),点\(a\)是\(f\)的瑕点,
函数\(F\)是\(f\)在区间\((a,b]\)上的一个原函数.
若极限\(F(a^+) \equiv \lim_{x \to a^+} F(x)\)存在,
则有
\begin{equation}\label{equation:定积分.利用牛顿莱布尼茨公式计算无界函数的反常积分1'}
\int_a^b f(x) \dd{x}
= F(b) - F(a^+);
\end{equation}
若极限\(F(a^+)\)不存在,则反常积分\(\int_a^b f(x) \dd{x}\)发散.
\end{theorem}
我们仍用记号\([F(x)]_a^b\)来表示\(F(b) - F(a^+)\),
这样就可以简化\cref{equation:定积分.利用牛顿莱布尼茨公式计算无界函数的反常积分1'},
从而形式上仍有
\begin{equation}\label{equation:定积分.利用牛顿莱布尼茨公式计算无界函数的反常积分1}
\int_a^b f(x) \dd{x} = [F(x)]_a^b.
\end{equation}

对于\(f\)在\([a,b)\)上连续、\(b\)为瑕点的反常积分,也有类似的计算公式.
这里不再详述.

\begin{theorem}\label{theorem:定积分.利用牛顿莱布尼茨公式计算无界函数的反常积分2}
设函数\(f \in C[a,b]\),点\(c\in(a,b)\)是\(f\)的瑕点,
函数\(F\)是\(f\)在区间\((a,b]\)上的一个原函数.
若极限\(F(c^+),F(c^-)\)都存在,
则有
\begin{equation}\label{equation:定积分.利用牛顿莱布尼茨公式计算无界函数的反常积分2}
\int_a^b f(x) \dd{x}
= [F(b) - F(c^+)] + [F(c^-) - F(a)];
\end{equation}
否则,反常积分\(\int_a^b f(x) \dd{x}\)发散.
\end{theorem}
在\cref{theorem:定积分.利用牛顿莱布尼茨公式计算无界函数的反常积分2} 中,
由于\(F(c^+)\)与\(F(c^-)\)不一定相等(例如,点\(c\)可能是函数\(F\)的跳跃间断点),
所以\[
\int_a^b f(x) \dd{x}
\neq [F(x)]_a^b = F(b) - F(a).
\]

\begin{example}
计算反常积分\[
\int_0^a \frac{\dd{x}}{\sqrt{a^2-x^2}}\quad(a>0).
\]
\begin{solution}
因为\[
\lim_{x \to a^-} \frac{1}{\sqrt{a^2-x^2}} = +\infty,
\]所以点\(a\)是瑕点,于是\[
\int_0^a \frac{\dd{x}}{\sqrt{a^2-x^2}}
= \left[ \arcsin\frac{x}{a} \right]_0^a
= \lim_{x \to a^-} \arcsin\frac{x}{a} - 0 = \frac{\pi}{2}.
\]
\end{solution}
\end{example}

\begin{example}
讨论反常积分\(\int_{-1}^1 \frac{\dd{x}}{x^2}\)的收敛性.
\begin{solution}
被积函数\(f(x) = \frac{1}{x^2}\)在积分区间\([-1,1]\)上除\(x=0\)外连续,且\[
\lim_{x\to0} \frac{1}{x^2} = +\infty.
\]

由于\[
\int_{-1}^0 \frac{\dd{x}}{x^2}
= \left[-\frac{1}{x}\right]_{-1}^0
= \lim_{x\to0^-} \left(-\frac{1}{x}\right) - 1
= +\infty,
\]即反常积分\(\int_{-1}^0 \frac{\dd{x}}{x^2}\)发散,所以反常积分\(\int_{-1}^1 \frac{\dd{x}}{x^2}\)发散.
\end{solution}

需要注意的是,如果疏忽了\(x=0\)是被积函数的瑕点,就可能得到以下错误结果:\[
\int_{-1}^1 \frac{\dd{x}}{x^2}
= \left[ -\frac{1}{x} \right]_{-1}^1
= -1 - 1 = -2.
\]
\end{example}

\subsection{\texorpdfstring{\(q\)}{q}积分}
\begin{proposition}[\(q\)积分]\label{example:定积分.q积分}
反常积分\[
	\int_a^b \frac{\dd{x}}{(x-a)^q}
	\quad(a \neq b)
\]
当\(0 < q < 1\)时收敛;
当\(q \geq 1\)时发散.
\begin{proof}
当\(q=1\)时,\[
	\int_a^b \frac{\dd{x}}{(x-a)^q}
	= \int_a^b \frac{\dd{x}}{x-a}
	= \eval{\ln(x-a)}_a^b
	= \ln(b-a) - \lim_{x \to a^+} \ln(x-a)
	= +\infty.
\]

当\(q\neq1\)时,\[
	\frac{\dd{x}}{(x-a)^q}
	= \frac{1}{1-q} \dd((x-a)^{1-q}),
\]
于是,当\(0<q<1\)时,
\(\int_a^b \frac{\dd{x}}{(x-a)^q}
= \frac{(b-a)^{1-q}}{1-q}\);
当\(q>1\)时,
\(\int_a^b \frac{\dd{x}}{(x-a)^q}
= +\infty\).

综上,当\(0<q<1\)时,这反常积分收敛于\(\frac{(b-a)^{1-q}}{1-q}\);
当\(q\geq1\)时,这反常积分发散.
\end{proof}
\end{proposition}

如果收敛的反常积分\(\int_a^b f(x) \dd{x}\)的%
被积函数\(f \in C(a,b)\)(\(a\)可以是\(-\infty\),
\(b\)可以是\(+\infty\),\(a\)、\(b\)也可以是\(f(x)\)的瑕点),
且应用的换元函数在\((a,b)\)内单调增加(或减少),
那么可以像定积分一样换元.

\begin{example}
求反常积分\(\int_0^{+\infty} \frac{\dd{x}}{\sqrt{x(x+1)^3}}\).
\begin{solution}
这里,积分上限为\(+\infty\),且下限\(x=0\)为被积函数的瑕点.

令\(\sqrt{x} = t\),
则\(x = t^2\),
\(x\to0^+\)时\(t\to0\),
\(x\to+\infty\)时\(t\to+\infty\).
于是\[
	\int_0^{+\infty} \frac{\dd{x}}{\sqrt{x(x+1)^3}}
	= \int_0^{+\infty} \frac{2t\dd{t}}{t(t^2+1)^{\frac32}}
	= 2 \int_0^{+\infty} \frac{\dd{t}}{(t^2+1)^{\frac32}}.
\]
再令\(t = \tan u\),
则\(u = \arctan t\),
\(t=0\)时\(u=0\),
\(t\to+\infty\)时,
\(u\to\frac\pi2\).
于是\[
	\int_0^{+\infty} \frac{\dd{x}}{\sqrt{x(x+1)^3}}
	= 2 \int_0^{\frac\pi2} \frac{\sec^2 u \dd{u}}{\sec^3 u}
	= 2 \int_0^{\frac\pi2} \cos u \dd{u}
	= 2.
\]
\end{solution}
本例如用变换\(t = \frac{1}{x}\)或\(t = \frac{1}{x+1}\),计算会更简单些.
\end{example}
