\section{本章总结}

我们在本章学习了无穷限的反常积分和无界函数的反常积分这两类反常积分的基本概念%
(\cref{definition:定积分.无穷限的反常积分的定义1,%
definition:定积分.无穷限的反常积分的定义3%
},以及\cref{definition:定积分.无界函数的反常积分的定义1}).
它们是对常义积分的定义的扩展.

我们可以利用莱布尼茨公式
\labelcref{equation:定积分.利用牛顿莱布尼茨公式计算无穷限的反常积分1,%
equation:定积分.利用牛顿莱布尼茨公式计算无穷限的反常积分2,%
equation:定积分.利用牛顿莱布尼茨公式计算无穷限的反常积分3,%
equation:定积分.利用牛顿莱布尼茨公式计算无界函数的反常积分1,%
equation:定积分.利用牛顿莱布尼茨公式计算无界函数的反常积分2%
}
计算反常积分.

\begin{table}[hb]
	\centering
	\begin{tblr}{*2cl}
		\hline
		名称 & 表达式 & 敛散条件 \\
		\hline
		{\hyperref[example:定积分.p积分]{p 积分}}
			& \(\int_a^{+\infty} \frac{\dd{x}}{x^p}\ (a>0)\)
			& 当\(p > 1\)时收敛于\(\frac{1}{p-1} a^{1-p}\),当\(p \leq 1\)时发散 \\
		{\hyperref[example:定积分.q积分]{q 积分}}
			& \(\int_a^b \frac{\dd{x}}{(x-a)^q}\ (a<b)\)
			& 当\(0 < q < 1\)时收敛于\(\frac{1}{1-q} (b-a)^{1-q}\),当\(q \geq 1\)时发散 \\
		& \(\int_0^{+\infty} e^{-ax} \dd{x}\) %\cref{example:反常积分.重要反常积分公式1}
			& 当\(a>0\)时收敛于\(\frac1a\),当\(a\leq0\)时发散 \\
		%@see: https://www.bilibili.com/video/BV1pgsteTEBB
		%@see-comment: 这个视频有点问题!反常积分\(\int_0^{+\infty} \frac{\dd{x}}{1-x^n}\)发散,除非明说是柯西主值积分,否则没有意义!
		%TODO 反常积分\(\int_0^{+\infty} \frac{\dd{x}}{1+x^n}\)的收敛性还没有讨论,以后有空再补完吧.
		%@Mathematica: Integrate[1/(1 + x^n), {x, 0, +Infinity}, Assumptions -> {n > 1}]
		& \(\int_0^{+\infty} \frac{\dd{x}}{1+x^n}\)
			& \(n>1\)时收敛于\(\frac\pi{n} \csc\frac\pi{n}\) \\
		\hline
	\end{tblr}
	\caption{重要反常积分及其敛散条件}
\end{table}

\begin{gather*}
	%\cref{example:反常积分.指数函数与三角函数之积的反常积分}
	\int_0^{+\infty} e^{-ax} \cos bx \dd{x} = \frac{a}{a^2+b^2}. \\
	\int_0^{+\infty} e^{-ax} \sin bx \dd{x} = \frac{b}{a^2+b^2}.
\end{gather*}

在本章我们还学习了一个特殊函数:
\hyperref[equation:特殊函数.伽马函数的积分定义]{\(\Gamma\)函数}\[
%\cref{equation:特殊函数.伽马函数的积分定义}
	\Gamma(s)
	\defeq
	\int_0^{+\infty} t^{s-1} e^{-t} \dd{t}
	\quad(s>0).
\]
它具有几个重要性质:\begin{itemize}
	%\cref{equation:伽马函数.递推公式}
	\item 对于任意\(s > 0\),总有\(\Gamma(s+1) = s~\Gamma(s)\).
	\item \(\Gamma(1) = 1\).
	%\cref{equation:定积分.伽马函数与阶乘的联系}
	\item \(\Gamma(n+1) = n!\ (n\in\mathbb{N})\).
	%\cref{equation:定积分.余元公式}
	\item \(\Gamma(s) \cdot \Gamma(1-s) = \frac{\pi}{\sin{\pi s}} \quad (0 < s < 1)\).
	\item \(\Gamma\left(1/2\right) = \sqrt{\pi}\).
	%\cref{equation:定积分.伽马函数与双阶乘的联系1}
	\item \((2n)!! = 2^n \cdot \Gamma(n+1)\).
	%\cref{equation:定积分.伽马函数与双阶乘的联系2}
	\item \((2n-1)!! = \frac{\Gamma(2n)}{2^{n-1} \cdot \Gamma(n)}\).
\end{itemize}
