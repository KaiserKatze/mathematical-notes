\chapter{偏微分方程}
从本章起,我们开始学习如何求解未知函数是多变量函数的微分方程,
也就是\DefineConcept{偏微分方程}和\DefineConcept{全微分方程}.

\section{拉普拉斯方程}
\begin{definition}
形如\begin{equation*}
	\sum_{k=1}^n \pdv[2]{y}{x_k} = 0
\end{equation*}的微分方程称为\DefineConcept{拉普拉斯方程}(Laplace's equation).
%@see: https://mathworld.wolfram.com/LaplacesEquation.html
\end{definition}

\begin{example}
函数\(z=\ln\sqrt{x^2+y^2}\)是方程\begin{equation*}
	\pdv[2]{z}{x} + \pdv[2]{z}{y} = 0
\end{equation*}的一个解.
\end{example}

\begin{example}
函数\(u=\frac{1}{r}\)是方程\begin{equation*}
	\pdv[2]{u}{x} + \pdv[2]{u}{y} + \pdv[2]{u}{z} = 0
\end{equation*}的一个解,
其中\(r=\sqrt{x^2+y^2+z^2}\).
\end{example}

\section{全微分方程}
\subsection{全微分方程的概念}
\begin{definition}
形如\begin{equation*}
	P(x,y)\dd{x} + Q(x,y)\dd{y} = 0
\end{equation*}的微分方程,
如果恰好有\begin{equation*}
	\dd{u(x,y)} = P(x,y)\dd{x} + Q(x,y)\dd{y},
\end{equation*}
则称其为\DefineConcept{全微分方程}(total differential equation)%
或\DefineConcept{恰当方程}(exact differential equation).
\end{definition}

\subsection{全微分方程的隐式通解}
设全微分方程的隐式通解为\begin{equation*}
	u(x,y) = C,
\end{equation*}
其中\(C\)为常数.

如果\(P,Q\)在单连通区域\(G\)内具有一阶连续偏导数,
则曲线积分\(\int_L P\dd{x}+Q\dd{y}\)在\(G\)内与路径无关,
进而有全微分方程在\(G\)内的显式通解为\begin{equation*}
	u(x,y) \equiv \int_{(x_0,y_0)}^{(x,y)} P\dd{x}+Q\dd{y} = C.
\end{equation*}

已知\begin{equation*}
	\pdv{u}{x} = P(x,y),
	\qquad
	\pdv{u}{y} = Q(x,y)
\end{equation*}的情况下,
首先对\(x\)积分,得\begin{equation*}
	u = \int P \dd{x} + \phi(y);
\end{equation*}
再对\(y\)求偏导,将所得结果与\(\pdv{u}{y} = Q\)比较,可得\(\phi(y)\).

\subsection{积分因子法}
对于微分方程\begin{equation*}
	P(x,y)\dd{x} + Q(x,y)\dd{y} = 0,
\end{equation*}
如果存在连续可微函数\(\mu=\mu(x,y)\),
使得\(\mu P \dd{x} + \mu Q \dd{y} = 0\)成为全微分方程,
即\(\mu P \dd{x} + \mu Q \dd{y} = \dd{u}\),
则称函数\(\mu\)为该微分方程的\DefineConcept{积分因子}.

如果微分方程\begin{equation*}
	P(x,y)\dd{x} + Q(x,y)\dd{y} = 0,
\end{equation*}
满足\(\frac{P}{Q}=-f\left(\frac{y}{x}\right)\),
则积分因子为\begin{equation*}
	\mu = \frac{1}{xP+yQ}.
\end{equation*}
