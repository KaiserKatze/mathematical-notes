\section{两类曲线积分之间的联系}
\begingroup
\def\innerProductTau{(\phi'(t))^2+(\psi'(t))^2}%法向量内积
\def\lenTau{\sqrt{\innerProductTau}}%法向量的模长
\def\fTau#1{\frac{#1}{\lenTau}}%法向量的方向余弦
\def\funcParam{(\phi(t),\psi(t))}%

设有向曲线弧\(L\)的起点为\(A\),终点为\(B\).
曲线弧\(L\)由参数方程\[
	\begin{cases}
		x = \phi(t), \\
		y = \psi(t)
	\end{cases}
\]给出,
起点\(A\)、终点\(B\)分别对应参数\(\alpha\)、\(\beta\),
且\(\alpha < \beta\).
又设函数\(\phi(t)\)、\(\psi(t)\)在闭区间\([\alpha,\beta]\)上具有一阶连续导数,
且\(\innerProductTau\neq0\),
又函数\(P(x,y)\)、\(Q(x,y)\)在\(L\)上连续.
那么有向曲线弧\(L\)的在点\(M(\phi(t),\psi(t))\)处的切向量为\[
	\vb{\tau} = \phi'(t) \vb{i} + \psi'(t) \vb{j},
\]
它的方向余弦为\[
	\cos\alpha
	=\fTau{\phi'(t)},
	\qquad
	\cos\beta
	=\fTau{\psi'(t)}.
\]

由对坐标的曲线积分计算公式和对弧长的曲线积分的计算公式有
\begin{align*}
	&\hspace{-20pt}
	\int_L (P(x,y),Q(x,y))\cdot\frac{\vb{\tau}}{\abs{\vb{\tau}}}\dd{s} \\
	&=\int_L [P(x,y)\cos\alpha+Q(x,y)\cos\beta]\dd{s} \\
	&=\int_\alpha^\beta
		\left[
			\begin{array}{l}
				P\funcParam\fTau{\phi'(t)} \\
				+Q\funcParam\fTau{\psi'(t)}
			\end{array}
		\right]
		\lenTau \dd{t} \\
	&=\int_\alpha^\beta \left\{
			P\funcParam\phi'(t) + Q\funcParam\psi'(t)
		\right\} \dd{t} \\
	&=\int_L{P(x,y)\dd{x}+Q(x,y)\dd{y}}.
\end{align*}
\endgroup

由此可见,平面曲线\(L\)上的两类曲线积分之间有如下联系:
\begin{equation}\label{equation:线积分与面积分.平面曲线上两类曲线积分之间的联系}
	\int_L P\dd{x}+Q\dd{y}
	=\int_L (P\cos\alpha+Q\cos\beta)\dd{s},
\end{equation}
其中\(\alpha(x,y)\)、\(\beta(x,y)\)为有向曲线弧\(L\)在点\((x,y)\)处的切向量的方向角.

类似地可知,空间曲线\(\Gamma\)上的两类曲线积分之间有如下联系:
\begin{equation}\label{equation:线积分与面积分.空间曲线上两类曲线积分之间的联系}
	\int_\Gamma P\dd{x}+Q\dd{y}+R\dd{z}
	=\int_\Gamma (P\cos\alpha+Q\cos\beta+R\cos\gamma)\dd{s},
\end{equation}
其中\(\alpha\)、\(\beta\)、\(\gamma\)为有向曲线弧\(\Gamma\)在点\((x,y,z)\)处的切向量的方向角.

两类曲线积分之间的联系也可用向量的形式表达.
例如,空间曲线\(\Gamma\)上的两类曲线积分之间的联系可写成如下形式:
\begin{equation}\label{equation:线积分与面积分.空间曲线上两类曲线积分之间的联系的向量形式}
	\int_\Gamma \VectorInnerProductDot{\vb{A}}{\dd{\vb{r}}}
	= \int_\Gamma \VectorInnerProductDot{\vb{A}}{\vb{\tau}} \dd{s},
\end{equation}
其中\(\vb{A}=(P,Q,R)\),
\(\vb{\tau}=(\cos\alpha,\cos\beta,\cos\gamma)\)为有向曲线弧\(\Gamma\)在点\((x,y,z)\)处的单位切向量,
\(\dd{\vb{r}}
=\vb{\tau}\dd{s}
=(\dd{x},\dd{y},\dd{z})\)称为\DefineConcept{有向曲线元}.
