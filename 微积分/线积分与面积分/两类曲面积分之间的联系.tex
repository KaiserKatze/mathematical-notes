\section{两类曲面积分之间的联系}
设有向曲面\(\Sigma\)由方程\(z = z(x,y)\)给出,
\(\Sigma\)在\(xOy\)面上的投影区域为\(D_{xy}\),
函数\(z = z(x,y)\)在\(D_{xy}\)上具有一阶连续偏导数,
\(R(x,y,z)\)在\(\Sigma\)上连续.
如果\(\Sigma\)取上侧,
则由对坐标的曲面积分计算公式 \labelcref{equation:线积分与面积分.第二类曲面积分的计算式1} 有\[
	\iint_\Sigma R(x,y,z) \dd{x}\dd{y} = \iint_{D_{xy}} R[x,y,z(x,y)] \dd{x}\dd{y}.
\]

另一方面,因上述有向曲面\(\Sigma\)的法向量的方向余弦为\[
	\begin{split}
		\cos\alpha=\frac{-z'_x}{\sqrt{1+(z'_x)^2+(z'_y)^2}}, \\
		\cos\beta=\frac{-z'_y}{\sqrt{1+(z'_x)^2+(z'_y)^2}}, \\
		\cos\gamma=\frac{1}{\sqrt{1+(z'_x)^2+(z'_y)^2}},
	\end{split}
\]
故由对面积的曲面积分计算公式 \labelcref{equation:线积分与面积分.第一类曲面积分的计算式1} 有\[
	\iint_\Sigma R(x,y,z) \cos\gamma \dd{S}
	= \iint_{D_{xy}} R[x,y,z(x,y)] \dd{x}\dd{y}.
\]
由此可见,有\begin{equation}\label{equation:两类曲面积分之间的联系.关系式1}
	\iint_\Sigma R(x,y,z) \dd{x}\dd{y}
	= \iint_\Sigma R(x,y,z) \cos\gamma \dd{S}.
\end{equation}
注意到等号两边的积分曲面都是\(\Sigma\),
被积函数都是\(R(x,y,z)\),
于是在形式上有\(\dd{x}\dd{y} = \cos\gamma \dd{S}\).

如果\(\Sigma\)取下侧,则\[
	\iint_\Sigma R(x,y,z) \cos\gamma \dd{S}
	= -\iint_{D_{xy}} R[x,y,z(x,y)] \dd{x}\dd{y}.
\]
但这时\(\cos\gamma=\frac{-1}{\sqrt{1+(z'_x)^2+(z'_y)^2}}\),
因此\cref{equation:两类曲面积分之间的联系.关系式1} 仍成立.

类似地可推得\begin{gather}
	\iint_\Sigma P(x,y,z) \dd{y}\dd{z}
	= \iint_\Sigma P(x,y,z) \cos\alpha \dd{S},
	\label{equation:两类曲面积分之间的联系.关系式2} \\
	\iint_\Sigma Q(x,y,z) \dd{z}\dd{x}
	= \iint_\Sigma Q(x,y,z) \cos\beta \dd{S}.
	\label{equation:两类曲面积分之间的联系.关系式3}
\end{gather}
于是在形式上有\(\dd{y}\dd{z} = \cos\alpha \dd{S}\)
和\(\dd{z}\dd{x} = \cos\beta \dd{S}\).

合并\cref{equation:两类曲面积分之间的联系.关系式1,equation:两类曲面积分之间的联系.关系式2,equation:两类曲面积分之间的联系.关系式3},
得两类曲面积分之间的如下联系:
\begin{equation}\label{equation:线积分与面积分.两类曲面积分之间的联系}
	\iint_\Sigma P \dd{y}\dd{z} + Q \dd{z}\dd{x} + R \dd{x}\dd{y}
	=\iint_\Sigma (P\cos\alpha+Q\cos\beta+R\cos\gamma) \dd{S},
\end{equation}
其中\(\cos\alpha\)、\(\cos\beta\)、\(\cos\gamma\)是
有向曲面\(\Sigma\)在点\((x,y,z)\)处的法向量的方向余弦.

两类曲面积分之间的联系也可写成如下的向量形式:
\begin{equation}\label{equation:线积分与面积分.两类曲面积分之间的联系的向量形式}
	\iint_\Sigma \vb{A} \cdot \dd{\vb{S}}
	=\iint_\Sigma \vb{A} \cdot \vb{n} \dd{S},
\end{equation}
其中\(\vb{A}=(P,Q,R)\),
\(\vb{n}=(\cos\alpha,\cos\beta,\cos\gamma)\)为有向曲面\(\Sigma\)在点\((x,y,z)\)处的单位法向量,
\(\dd{\vb{S}}
= \vb{n}\dd{S}
= (\dd{y}\dd{z},\dd{z}\dd{x},\dd{x}\dd{y})\)称为\DefineConcept{有向曲面元}.

另一方面,
由\cref{equation:两类曲面积分之间的联系.关系式1,equation:两类曲面积分之间的联系.关系式2,equation:两类曲面积分之间的联系.关系式3}
有\[
	\dd{S}
	= \frac{\dd{y}\dd{z}}{\cos\alpha}
	= \frac{\dd{z}\dd{x}}{\cos\beta}
	= \frac{\dd{x}\dd{y}}{\cos\gamma},
\]
\begingroup
\def\NamedAngle#1{%
	\ifnum1=0#1\relax\alpha\fi%
	\ifnum2=0#1\relax\beta\fi%
	\ifnum3=0#1\relax\gamma\fi%
}%
\def\NamedFraction#1#2{\frac{\cos\NamedAngle#1}{\cos\NamedAngle#2}}%
\def\NamedAxis#1{%
	\ifnum1=0#1\relax x \fi%
	\ifnum2=0#1\relax y \fi%
	\ifnum3=0#1\relax z \fi%
}%
\def\NamedDerivative#1#2{%
	\NamedAxis#2'_\NamedAxis#1%
}%
于是\begin{align*}
	\iint_\Sigma P \dd{y}\dd{z} + Q \dd{z}\dd{x} + R \dd{x}\dd{y}
	&= \iint_\Sigma \left(P \cdot \NamedFraction13 + Q \cdot \NamedFraction23 + R\right)\dd{x}\dd{y} \\
	&= \iint_\Sigma \left(P \cdot \NamedFraction12 + Q + R \cdot \NamedFraction32\right)\dd{z}\dd{x} \\
	&= \iint_\Sigma \left(P + Q \cdot \NamedFraction21 + R \cdot \NamedFraction31\right)\dd{y}\dd{z}.
\end{align*}
再由\begin{gather*}
	\NamedFraction13
	= -\NamedDerivative13,
	\qquad
	\NamedFraction23
	= -\NamedDerivative23, \\
	\NamedFraction12
	= -\NamedDerivative12,
	\qquad
	\NamedFraction32
	= -\NamedDerivative32, \\
	\NamedFraction21
	= -\NamedDerivative21,
	\qquad
	\NamedFraction31
	= -\NamedDerivative31
\end{gather*}
可得\begin{equation}\label{equation:两类曲面积分之间的联系.变换公式}
	\begin{split}
		\iint_\Sigma P\dd{y}\dd{z}+Q\dd{z}\dd{x}+R\dd{x}\dd{y}
		&= \iint_\Sigma \left(R - P \cdot \NamedDerivative13 - Q \cdot \NamedDerivative23\right)\dd{x}\dd{y} \\
		&= \iint_\Sigma \left(Q - P \cdot \NamedDerivative12 - R \cdot \NamedDerivative32\right)\dd{z}\dd{x} \\
		&= \iint_\Sigma \left(P - Q \cdot \NamedDerivative21 - R \cdot \NamedDerivative31\right)\dd{y}\dd{z}.
	\end{split}
\end{equation}
\endgroup

\begin{example}
%@see: 《高等数学(第六版 下册)》 P228 例3
计算曲面积分\(I=\iint_\Sigma (z^2+x) \dd{y}\dd{z} - z \dd{x}\dd{y}\),
其中\(\Sigma\)是旋转抛物面\(z = \frac{1}{2}(x^2+y^2)\)介于平面\(z=0\)和\(z=2\)之间的部分的下侧.
\begin{solution}
由两类曲面积分之间的联系 \labelcref{equation:线积分与面积分.两类曲面积分之间的联系},可得\[
	\iint_\Sigma (z^2+x) \dd{y}\dd{z}
	= \iint_\Sigma (z^2+x) \frac{\cos\alpha}{\cos\gamma} \dd{x}\dd{y}.
\]
在曲面\(\Sigma\)上,
有\(z'_x = x, z'_y = y\),
从而\[
	\cos\alpha
	= \frac{x}{\sqrt{1+x^2+y^2}},
	\qquad
	\cos\gamma
	= \frac{-1}{\sqrt{1+x^2+y^2}}.
\]
故\[
	I = \iint_\Sigma [(z^2+x)(-x) - z] \dd{x}\dd{y}.
\]
再按对坐标的曲面积分的计算法,
代入\(z = \frac{1}{2}(x^2+y^2)\),
便得\[
	I = - \iint_{D_{xy}} \left\{
		\left[
			\frac{1}{4} (x^2+y^2)^2
			+ x
		\right] \cdot (-x)
		- \frac{1}{2} (x^2+y^2)
	\right\} \dd{x}\dd{y}.
\]
因为\(\iint_{D_{xy}} \frac{1}{4} x(x^2+y^2)^2 \dd{x}\dd{y} = 0\),
所以\begin{align*}
	I
	&= \iint_{D_{xy}} \left[x^2+\frac{1}{2}(x^2+y^2)\right] \dd{x}\dd{y} \\
	&= \int_0^{2\pi} \dd{\theta}
		\int_0^2 \left(\rho^2 \cos^2\theta + \frac{1}{2} \rho^2\right) \rho \dd{\rho}
	= 8\pi.
\end{align*}
%@Mathematica: Plot3D[1/2 (x^2 + y^2), {x, -2, 2}, {y, -2, 2}, PlotRange -> {0, 2}]
%@Mathematica: Integrate[Integrate[(r^2 Cos[t]^2 + 1/2 r^2) r, {r, 0, 2}], {t, 0, 2 Pi}]
\end{solution}
\end{example}
