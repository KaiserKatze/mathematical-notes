\section{对面积的曲面积分}
\subsection{对面积的曲面积分的概念}
\begin{definition}
%@see: 《数学分析(第二版 下册)》(陈纪修) P298
设曲面\(\Sigma\)的方程为\[
	\left\{ \begin{array}{l}
		x = x(u,v), \\
		y = y(u,v), \\
		z = z(u,v) \\
	\end{array} \right.
	\quad
	(u,v) \in D,
\]
其中\(D\)是\(uOv\)平面上具有光滑边界或分段光滑边界的有界闭区域.

如果这个映射是双射,则称“\(\Sigma\)是\DefineConcept{简单曲面}”.

如果进一步,\(x,y,z\)对\(u\)和\(v\)具有连续偏导数,
相应的雅克比矩阵\[
	\def\arraystretch{1.5}
	\vb{J} = \begin{bmatrix}
		\pdv{x}{u} & \pdv{x}{v} \\
		\pdv{y}{u} & \pdv{y}{v} \\
		\pdv{z}{u} & \pdv{z}{v}
	\end{bmatrix}
\]满秩,
则称“\(\Sigma\)是\DefineConcept{光滑曲面}”.
\end{definition}

\begin{definition}
%@see: 《高等数学(第六版 下册)》 P215 定义
设曲面\footnote{以后都假定曲面的边界曲线是分段光滑的闭曲线,且曲面有界.}%
\(\Sigma\)是光滑的\footnote{如果曲面上各点处都具有切平面,则称该曲面是光滑的.
如果曲面是由有限个光滑曲面所组成的,则称该曲面是分片光滑的.}.
函数\(f(x,y,z)\)在\(\Sigma\)上有界.
把\(\Sigma\)任意分成\(n\)小块\(\increment S_i\)
(\(\increment S_i\)同时也代表第\(i\)小块曲面的面积),
设\((\xi_i,\eta_i,\zeta_i)\)是\(\increment S_i\)上任取的一点,
作乘积\(f(\xi_i,\eta_i,\zeta_i) \increment S_i\ (i=1,2,\dotsc,n)\),
并作和\(\sum_{i=1}^n f(\xi_i,\eta_i,\zeta_i) \increment S_i\),
如果当各小块曲面的直径\footnote{%
曲面的直径是指曲面上任意两点间距离的最大者.}的最大值\(\lambda\to0\)时,
这和的极限总存在,
则称此极限为
“函数\(f(x,y,z)\)在曲面\(\Sigma\)上\DefineConcept{对面积的曲面积分}”
或“函数\(f(x,y,z)\)在曲面\(\Sigma\)上的\DefineConcept{第一类曲面积分}%
(type I surface integral)”,
记作\(\iint_\Sigma f(x,y,z) \dd{S}\),
即\[
	\iint_\Sigma f(x,y,z)\dd{S}
	\defeq
	\lim_{\lambda\to0} \sum_{i=1}^n f(\xi_i,\eta_i,\zeta_i) \increment S_i,
\]
其中\(f(x,y,z)\)叫做\DefineConcept{被积函数},
\(\Sigma\)叫做\DefineConcept{积分曲面}.

如果曲面\(\Sigma\)是分片光滑的,
我们规定:
函数在\(\Sigma\)上对面积的曲面积分,
等于函数在光滑的各片曲面上对面积的曲面积分之和.
\end{definition}
我们指出,当被积函数\(f(x,y,z)\)在光滑曲面\(\Sigma\)上连续时,对面积的曲面积分是存在的.

\subsection{对面积的曲面积分的性质}
由对面积的曲面积分的定义可知,它具有与对弧长的曲线积分相类似的性质,这里不再赘述.

\subsection{对面积的曲面积分的计算法}
\begin{theorem}
设积分曲面\(\Sigma\)由方程\(z=z(x,y)\)给出,
\(\Sigma\)在\(xOy\)面上的投影区域为\(D_{xy}\),
函数\(z=z(x,y)\)在\(D_{xy}\)上具有连续偏导数,
被积函数\(f(x,y,z)\)在\(\Sigma\)上连续.
那么有\begin{equation}\label{equation:线积分与面积分.第一类曲面积分的计算式1}
	\iint_\Sigma f(x,y,z) \dd{S}
	= \iint_{D_{xy}} f[x,y,z(x,y)] \sqrt{1+[z'_x(x,y)]^2+[z'_y(x,y)]^2} \dd{x}\dd{y}.
\end{equation}
\end{theorem}
这就是把对面积的曲面积分化为二重积分的公式.
在计算时,只要把变量\(z\)换为\(z(x,y)\),\(\dd{S}\)换为
\(\sqrt{1+(z'_x)^2+(z'_y)^2} \dd{x}\dd{y}\),
再确定\(\Sigma\)在\(xOy\)面上的投影区域\(D_{xy}\),
这样就把对面积的曲面积分化为二重积分了.

如果积分曲面\(\Sigma\)由方程\(x=x(y,z)\)或\(y=y(z,x)\)给出,
也可类似地把对面积的曲面积分化为相应的二重积分.

可以将上述对面积的曲面积分的计算法总结为如下更一般的方法.
\begin{theorem}
设积分曲面\(\Sigma\)由二元向量值参数方程\[
	\vb{r}(u,v) = x(u,v) \vb{i} + y(u,v) \vb{j} + z(u,v) \vb{k},
	\quad (u,v) \in D
\]给出,
这里\(D\)是\(uv\)平面上具有分段光滑边界的有界闭区域,
函数\(\vb{r}(u,v)\)在\(D\)上具有连续偏导数,
被积函数\(f(x,y,z)\)在\(\Sigma\)上连续,
那么有\[
	\iint_\Sigma f(x,y,z) \dd{S}
	= \iint_D f(\vb{r}(u,v))
				\abs{
					\VectorOuterProduct{\vb{r}'_u}{\vb{r}'_v}
				}
			\dd{\sigma}.
\]
\end{theorem}
这里,参数\(u\)、\(v\)可以代换为\(x\)、\(y\)、\(z\)中任意两个,
区域\(D\)可以代换为\(D_{xy}\)、\(D_{yz}\)或\(D_{zx}\).

\begin{example}
%@see: 《高等数学(第六版 下册)》 P217 例1
计算曲面积分\(\iint_\Sigma \frac{\dd{S}}{z}\),
其中\(\Sigma\)是球面\(x^2+y^2+z^2=a^2\)被平面\(z = h\ (0<h<a)\)截出的顶部.
\begin{solution}
\(\Sigma\)的方程为\(z = \sqrt{a^2-x^2-y^2}\).
\(\Sigma\)在\(xOy\)面上的投影区域\(D_{xy}\)为圆形闭区域\[
	\Set{(x,y) \given x^2+y^2 \leq a^2-h^2}.
\]
又\[
	\sqrt{1+(z'_x)^2+(z'_y)^2} = \frac{a}{\sqrt{a^2-x^2-y^2}},
\]
所以\[
	\iint_\Sigma \frac{\dd{S}}{z}
	= \iint_{D_{xy}} \frac{a\dd{x}\dd{y}}{a^2-x^2-y^2}.
\]
利用极坐标,得\[
	\iint_\Sigma \frac{\dd{S}}{z}
	= \iint_{D_{xy}} \frac{a\rho\dd{\rho}\dd{\theta}}{a^2-\rho^2}
	= a \int_0^{2\pi} \dd{\theta} \int_0^{\sqrt{a^2-h^2}} \frac{\rho\dd{\rho}}{a^2-\rho^2}
	= 2\pi a \ln\frac{a}{h}.
\]
\end{solution}
\end{example}

\begin{example}
%@see: 《高等数学(第六版 下册)》 P218 例2
计算\(\oiint_\Sigma xyz \dd{S}\),
其中\(\Sigma\)是由平面\(x=0\)、\(y=0\)、\(z=0\)及\(x+y+z=1\)所围成的四面体的整个边界曲面.
\begin{solution}
整个边界曲面\(\Sigma\)在平面\(x=0\)、\(y=0\)、\(z=0\)及\(x+y+z=1\)上的部分
依次记为\(\AutoTuple{\Sigma}{4}\),于是\[
	\oiint_\Sigma xyz \dd{S}
	= \oiint_{\Sigma_1} xyz \dd{S}
	+ \oiint_{\Sigma_2} xyz \dd{S}
	+ \oiint_{\Sigma_3} xyz \dd{S}
	+ \oiint_{\Sigma_4} xyz \dd{S}.
\]
由于在\(\AutoTuple{\Sigma}{3}\)上,被积函数\(f(x,y,z)=xyz\)均为零,所以\[
	\oiint_{\Sigma_1} xyz \dd{S}
	= \oiint_{\Sigma_2} xyz \dd{S}
	= \oiint_{\Sigma_3} xyz \dd{S}
	= 0.
\]

在\(\Sigma_4\)上,\(z=1-x-y\),\(z'_x = z'_y = -1\),所以\[
	\sqrt{1+(z'_x)^2+(z'_y)^2}
	= \sqrt{1+(-1)^2+(-1)^2}
	= \sqrt{3},
\]
从而\[
	\oiint_\Sigma xyz \dd{S}
	= \oiint_{\Sigma_4} xyz \dd{S}
	= \iint_{D_{xy}} \sqrt{3} xy (1-x-y) \dd{x} \dd{y},
\]
其中\(D_{xy}\)是\(\Sigma_4\)在\(xOy\)面上的投影区域,
即由直线\(x=0\)、\(y=0\)及\(x+y=1\)所围成的闭区域.
因此\[
	\oiint_\Sigma xyz \dd{S}
	= \sqrt{3} \int_0^1 x \dd{x} \int_0^{1-x} y (1-x-y) \dd{y}
	= \frac{\sqrt{3}}{120}.
\]
\end{solution}
\end{example}

\begin{example}
设点\(P\)是椭球面\(S: x^2 + y^2 + z^2 - yz = 1\)上的动点,
若\(S\)在点\(P\)处的切平面与\(xOy\)面垂直,求点\(P\)的轨迹\(C\),并计算曲面积分\[
	I = \iint_\Sigma \frac{(x+\sqrt{3}) \abs{y-2z}}{\sqrt{4+y^2+z^2-4yz}} \dd{S},
\]
其中\(\Sigma\)是椭球面\(S\)位于曲线\(C\)上方的部分.
\begin{solution}
设\(F(x,y,z) = x^2 + y^2 + z^2 - yz - 1\).
求导得\[
	F'_x = 2x, \qquad
	F'_y = 2y - z, \qquad
	F'_z = 2z - y,
\]
那么\(S\)在点\(P(x,y,z)\)处的切向量为\(\vb{n} = \opair{2x,2y-z,2z-y}\).
因为点\(P\)处的切平面与\(xOy\)面垂直,
所以法向量\(\vb{n}\)与\(xOy\)面的法向量\(\opair{0,0,1}\)垂直,\[
	2x\cdot0 + (2y-z)\cdot0+(2z-y)\cdot1=0
	\quad\text{或}\quad
	2z-y=0.
\]
因此,轨迹\(C\)的方程是\[
	\begin{cases}
		x^2 + y^2 + z^2 - yz = 1, \\
		2z-y=0.
	\end{cases}
\]
消去轨迹\(C\)方程中的\(z\),得到\(C\)在\(xOy\)面上的投影为\[
	C_{xy}: x^2+\frac{3}{4}y^2=1.
\]
又因为\[
	z'_x = - \frac{F'_x}{F'_z} = \frac{2x}{y-2z}, \qquad
	z'_y = - \frac{F'_y}{F'_z} = \frac{2y-z}{y-2z},
\]
\begin{align*}
	\dd{S} &= \sqrt{1+(z'_x)^2+(z'_y)^2} \dd{x}\dd{y} \\
	&= \sqrt{1+\left(\frac{2x}{y-2z}\right)^2+\left(\frac{2y-z}{y-2z}\right)^2} \dd{x}\dd{y} \\
	&= \frac{\sqrt{4x^2+5y^2+5z^2-8yz}}{\abs{y-2z}} \dd{x}\dd{y} \\
	&= \frac{\sqrt{4+y^2+z^2-4yz}}{\abs{y-2z}} \dd{x}\dd{y},
\end{align*}
所以\begin{align*}
	&\hspace{-20pt}
	\iint_\Sigma \frac{(x+\sqrt{3}) \abs{y-2z}}{\sqrt{4+y^2+z^2-4yz}} \dd{S} \\
	&= \iint_{x^2+\frac{3}{4}y^2\leq1} \frac{(x+\sqrt{3}) \abs{y-2z}}{\sqrt{4+y^2+z^2-4yz}}
		\cdot \frac{\sqrt{4+y^2+z^2-4yz}}{\abs{y-2z}} \dd{x}\dd{y} \\
	&= \iint_{x^2+\frac{3}{4}y^2\leq1} (x+\sqrt{3}) \dd{x}\dd{y} \\
	&= \sqrt{3} \iint_{x^2+\frac{3}{4}y^2\leq1} \dd{x}\dd{y}
	= 2\pi.
\end{align*}
\end{solution}
\end{example}

\begin{example}
%@see: 《数学分析(第二版 下册)》(陈纪修) P307 通讯卫星的电波覆盖的地球面积
将通讯卫星发射到赤道的上空,使它在赤道面内自西向东地绕地球飞行,
那么它始终在地球某一个位置的上空,相对地面静止.
这样的卫星称为\emph{对地静止卫星}(geostationary satellite),
是一种特殊的\emph{地球同步卫星}(geosynchronous satellite).
现在来计算这类卫星的电波所能覆盖的地球的表面积.
为了简化问题,把地球看成一个半径为\(R\)球体,
不考虑其他天体对卫星的影响,把卫星轨道看成一个的圆,
假设地球自转的角速度是\(\omega\).
我们首先确定卫星离地面的高度\(h\).
要使卫星在轨道上稳定飞行,卫星所受的地球引力必须与它绕地球飞行所受的离心力相等,
即\[
	\frac{GMm}{(R+h)^2}
	= m \omega^2 (R+h),
\]
其中\(M\)是地球的质量,\(m\)是卫星的质量,\(G\)是引力常量.
由于地球表面的重力加速度\(g\)满足\[
	\frac{GMm}{R^2}
	= mg,
\]
所以\[
	(R+h)^3 = \frac{GM}{\omega^2}
	= g\frac{R^2}{\omega^2},
\]
于是\[
	h = \sqrt[3]{\frac{GM}{\omega^2}} - R.
\]
代入\(R \approx \qty{6371000}{\meter}\),
\(\omega \approx \frac{2\pi}{24 \times 3600}\)~\unit[per-mode=symbol]{\radian\per\second},
\(g \approx \qty[per-mode=symbol]{9.8}{\meter\per\second\squared}\),
就得到卫星离地面的高度为\[
	h \approx \qty{36000000}{\meter}.
\]
\begin{figure}[htb]
	\centering
	\begin{tikzpicture}
		\pgfmathsetmacro{\a}{70}
		\pgfmathsetmacro{\R}{1}
		\fill[cyan](0,0)coordinate(O)circle(\R);
		\draw(O)--({\R*sin(\a)},{\R*cos(\a)})coordinate(P)
			--(0,{\R/cos(\a)})coordinate(Q);
		\draw pic["$\alpha$",draw=orange,-,angle eccentricity=2,angle radius=3mm]{angle=P--O--Q};
		\draw(O)node[below left]{地球};
		\fill(Q)circle(2pt)node[right]{卫星};
		\begin{scope}[->,>=Stealth]
			\draw(-1.5,0)--(1.5,0)node[below]{$x$};
			\draw(0,-1.5)--(Q)--++(0,.5)node[right]{$z$};
		\end{scope}
	\end{tikzpicture}
	\caption{}
	\label{figure:线积分与面积分.对地静止卫星}
\end{figure}
如\cref{figure:线积分与面积分.对地静止卫星} 所示,
取地心为坐标原点,取过地心与卫星中心、方向从地心到卫星中心的有向直线为\(z\)轴,
可以看出卫星电波可以覆盖的最大纬度\(\alpha\)满足\(\cos\alpha=\frac{R}{R+h}\),
解得\[
	\alpha=\arccos\frac{R}{R+h} \approx \qty{81.3521}{\degree},
\]
也就是说卫星的电波的覆盖范围是南纬\(\qty{81}{\degree}\)到北纬\(\qty{81}{\degree}\)之间的球面区域,
对应的覆盖面积为\[
	S = \iint_\Sigma \dd{S},
\]
其中\(\Sigma\)是上半球面\(x^2+y^2+z^2=R^2,z\geq0\)上满足\(z \geq R\cos\alpha\)的部分,
即\[
	\Sigma: z=\sqrt{R^2-x^2-y^2}, \quad x^2+y^2 \leq R^2\sin^2\alpha.
\]
利用第一类曲面积分的计算公式得\[
	S = \iint_D \sqrt{1+\left(\pdv{z}{x}\right)^2+\left(\pdv{z}{y}\right)^2} \dd{x}\dd{y}
	= \iint_D \frac{R}{\sqrt{R^2-x^2-y^2}} \dd{x}\dd{y},
\]
这里\(D\)是\(xOy\)平面上的区域
\(\Set{ (x,y) \given x^2+y^2 \leq R^2\sin^2\alpha }\).
利用极坐标变换,得\[
	S = \int_0^{2\pi} \dd{\theta}
	\int_0^{R\sin\alpha} \frac{R}{\sqrt{R^2-r^2}} r\dd{r}
	= 2\pi R \eval{(-\sqrt{R^2-r^2})}_0^{R\sin\alpha}
	= 2\pi R^2(1-\cos\alpha).
\]
因为\(\cos\alpha=\frac{R}{R+h}\),
所以\[
	S = 2\pi R^2\frac{h}{R+h}
	\approx \qty{2.16575e8}{\kilo\meter\squared}.
\]
由于地球的表面积是\(4\pi R^2 \approx \qty{5.10064e8}{\kilo\meter\squared}\),
所以一颗卫星的电波可以覆盖的地球面积占比为\[
	\frac{h}{2(R+h)}
	\approx \num{0.4248}.
\]
因此,卫星的电波覆盖了地球表面三分之一以上的面积,
从理论上说,只要在赤道上空使用三颗角度相间\(2\pi/3\)的通讯卫星,
它们的电波就可以覆盖几乎整个地球表面.
我们可以算出它们的电波的实际覆盖面积为\begin{align*}
	S' &= 2\int_0^{R\sin\alpha} 2\pi y \sqrt{1+(y')^2} \dd{x}
	= 4\pi \int_0^{R\sin\alpha} \sqrt{R^2-x^2} \sqrt{1+\left(\frac{-x}{\sqrt{R^2-x^2}}\right)^2} \dd{x} \\
	&= 4\pi R^2\sin\alpha
	= 4\pi R^2 \frac{\sqrt{(R+h)^2-R^2}}{R+h}
	\approx \qty{5.04266e8}{\kilo\meter\squared}.
\end{align*}
没有覆盖到的球面区域的面积约为\(\qty{0.05798e8}{\kilo\meter\squared}\),
仅占地球表面积的\(\qty{1.14}{\percent}\).
\end{example}

\subsection{利用对称性简化第一类曲面积分的计算}
%@see: https://www.bilibili.com/video/BV1Dn4y1R7gW/
%奇偶对称性
设积分曲面\(\Sigma\)关于\(xOy\)面(平面\(z=0\))对称.
\begin{itemize}
	\item 若被积函数\(f\)在曲面\(\Sigma\)上是关于\(z\)的奇函数,
	即\[
		f(x,y,-z) = -f(x,y,z),
	\]
	则\[
		\iint_\Sigma f(x,y,z) \dd{S} = 0.
	\]

	\item 若被积函数\(f\)在曲面\(\Sigma\)上是关于\(z\)的偶函数,
	即\[
		f(x,y,-z) = f(x,y,z),
	\]
	则\[
		\iint_\Sigma f(x,y,z) \dd{S}
		= 2 \iint_{\Sigma_1} f(x,y,z) \dd{S},
		% = 2 \iint_{D_{xy}} f[x,y,z(x,y)] \sqrt{1+(z'_x)^2+(z'_y)^2} \dd{x}\dd{y}.
	\]
	其中\(\Sigma_1\)是曲面\(\Sigma\)在坐标面\(xOy\)上侧的部分.
\end{itemize}

曲面\(\Sigma\)关于坐标面\(yOz\)(平面\(x=0\))对称,
被积函数\(f\)在\(\Sigma\)上关于变量\(x\)具有奇偶性时,
也有类似的结论.

曲面\(\Sigma\)关于坐标面\(zOx\)(平面\(y=0\))对称,
被积函数\(f\)在\(\Sigma\)上关于变量\(y\)具有奇偶性时,
也有类似的结论.

设曲面\(\Sigma\)关于三个坐标面都对称,
被积函数\(f\)关于三个变量\(x,y,z\)都是偶函数,
则\[
	\iiint_\Sigma f(x,y,z) \dd{S}
	= 8 \iiint_{\Sigma_1} f(x,y,z) \dd{S},
\]
其中\(\Sigma_1\)是\(\Sigma\)在第一卦限内的部分.

\begin{example}
计算曲面积分\(\iint_\Sigma (x^2 z \sin y + x e^{y^2 z} + 4) \dd{S}\),
其中\(\Sigma: x^2+y^2+z^2=9, z\geq0\).
\begin{solution}
因为\(\Sigma\)关于坐标面\(zOx\)对称,
函数\(x^2 z \sin y\)是关于变量\(y\)的奇函数,
所以\(\iint_\Sigma x^2 z \sin y \dd{S} = 0\).

因为\(\Sigma\)关于坐标面\(yOz\)对称,
函数\(x e^{y^2 z}\)是关于变量\(x\)的奇函数,
所以\(\iint_\Sigma x e^{y^2 z} \dd{S} = 0\).

于是原积分等于\(\iint_\Sigma 4 \dd{S} = 4\cdot\frac12\cdot4\pi\cdot3^3 = 72\pi\).
\end{solution}
\end{example}

%@see: https://www.bilibili.com/video/BV13t421K7yY/
%轮换对称性
设曲面\(\Sigma'\)与曲面\(\Sigma\)关于平面\(y=x\)对称,
即\[
	\Sigma' = \Set{ (x,y,z) \given (y,x,z) \in \Sigma },
\]
则\[
	\iint_\Sigma f(x,y,z) \dd{S}
	= \iint_{\Sigma'} f(y,x,z) \dd{S}.
\]

设曲面\(\Sigma\)关于平面\(y=z\)对称,
则\[
	\iint_\Sigma f(x,y,z) \dd{S}
	= \iint_\Sigma f(y,x,z) \dd{S}
	= \frac12 \iint_\Sigma (f(x,y,z) + f(y,x,z)) \dd{S}.
\]

曲面\(\Sigma\)关于平面\(y=z\)或\(z=x\)对称时,也有类似的结论.

\begin{example}
设\(\Sigma: x^2+y^2+4z^2=1\).
判断曲面积分\[
	\iiint_\Sigma x^2 \dd{S}
	\quad\text{和}\quad
	\iiint_\Sigma y^2 \dd{S}
\]是否相等.
\begin{solution}
将\(\Sigma\)的表达式\(x^2+y^2+4z^2=1\)中的变量\(x\)与变量\(y\)互换,
得到的\(y^2+x^2+4z^2=1\)与原式等价,
所以\(\Sigma\)关于平面\(y=x\)对称,
因此\[
	\iiint_\Sigma x^2 \dd{S}
	= \iiint_\Sigma y^2 \dd{S}.
\]
\end{solution}
\end{example}

\begin{example}
设\(\Sigma: x^2+y^2+4z^2=1\).
判断曲面积分\[
	\iiint_\Sigma x^2 \dd{S}
	\quad\text{和}\quad
	\iiint_\Sigma z^2 \dd{S}
\]是否相等.
\begin{solution}
将\(\Sigma\)的表达式\(x^2+y^2+4z^2=1\)中的变量\(x\)与变量\(z\)互换,
得到的\(z^2+y^2+4x^2=1\)与原式不等价,
所以\(\Sigma\)不关于平面\(z=x\)对称,
于是\[
	\iiint_\Sigma x^2 \dd{S}
	\neq \iiint_\Sigma z^2 \dd{S}.
\]
\end{solution}
\end{example}

\begin{example}
计算\(\oiint_\Sigma x^2 \dd{S}\),
其中\(\Sigma: x^2+y^2+z^2=a^2\).
\begin{solution}
利用对称性可得\begin{align*}
	\oiint_\Sigma x^2 \dd{S}
	&= \frac{1}{3} \oiint_\Sigma (x^2+y^2+z^2) \dd{S} \\
	&= \frac{1}{3} \oiint_\Sigma a^2 \dd{S}
	= \frac{a^2}{3} \oiint_\Sigma \dd{S} \\
	&= \frac{a^2}{3} \cdot 4\pi a^2
	= \frac{4}{3} \pi a^4.
\end{align*}
\end{solution}
\end{example}

\begin{example}
%@see: 《2007年全国硕士研究生入学统一考试(数学一)》二填空题/第14题
设曲面\(\Sigma: \abs{x}+\abs{y}+\abs{z}=1\).
计算曲面积分\(\oiint_\Sigma (x+\abs{y}) \dd{S}\).
\begin{solution}
因为曲面关于三个坐标面都对称,
而函数\(f_1(x,y,z) = x\)在\(\Sigma\)上是关于\(x\)的奇函数,
所以\[
	\oiint_\Sigma x \dd{S} = 0,
\]
从而有\[
	\oiint_\Sigma (x+\abs{y}) \dd{S}
	= \oiint_\Sigma \abs{y} \dd{S}.
\]
又由于曲面方程具有轮换对称性,
所以\begin{align*}
	\oiint_\Sigma \abs{y} \dd{S}
	&= \frac13 \oiint_\Sigma (\abs{x}+\abs{y}+\abs{z}) \dd{S}
	= \frac13 \oiint_\Sigma \dd{S} \\
	&= \frac13 \cdot 8 \cdot \sqrt{\frac{3\sqrt2}2 \left( \frac{3\sqrt2}2 - \sqrt2 \right)^3}
	= \frac{4\sqrt3}3.
\end{align*}
\end{solution}
\end{example}
