\section{微分方程的解的存在性和唯一性}
\subsection{微分方程的解的存在性}
\begin{definition}\label{definition:微分方程.函数系的一致有界性}
设函数系\(S\)是由定义在某个区间\(D = [a,b]\)上的一些函数组成的集合.
如果\[
	(\exists M>0)
	(\forall x \in D)
	(\forall F \in S)
	[\abs{F(x)} < M],
\]
则称“函数系\(S\)在区间\(D\)上是\DefineConcept{一致有界的}(uniformly bounded)”.
%@see: \cref{definition:微分方程.函数系的一致有界性}
\end{definition}

\begin{definition}\label{definition:微分方程.函数系的等度连续性}
设函数系\(S\)是由定义在某个区间\(D = [a,b]\)上的一些函数组成的集合.
如果\[
	(\forall\epsilon>0)
	(\exists\delta>0)
	(\forall F \in S)
	(\forall x_1,x_2 \in D)
	[\abs{x_1-x_2} < \delta \implies \abs{F(x_1)-F(x_2)} < \epsilon],
\]
%@see: https://mathworld.wolfram.com/Equicontinuous.html
则称“函数系\(S\)在区间\(D\)上是\DefineConcept{等度连续的}(equicontinuous)%
\footnote{尽管有很多相似之处,但是一定注意“等度连续”与“一致连续”的区别:
“等度连续”描述的对象是函数系,
“一致连续”描述的对象是函数、函数列或函数项级数
(见\cref{definition:极限.函数的一致连续性} 和
\cref{definition:无穷级数.函数项级数的一致收敛性}).}”;
称\[
	\sup_{F_1,F_2 \in S} \abs{F_1(x)-F_2(x)}
\]为“函数系\(S\)在区间\([a,b]\)上的\DefineConcept{宽度}%
\footnote{注意与\hyperref[definition:极限.函数在集合上的振幅]{函数在集合上的振幅}相区别.}”.
\end{definition}

\begin{lemma}[阿斯科拉--阿尔泽拉引理]\label{theorem:微分方程概论.阿斯科拉--阿尔泽拉引理}
任何在区间\([a,b]\)上一致有界且等度连续的函数系\(S\)都包含在此区间上一致收敛的函数列.
\end{lemma}

\begin{theorem}
设函数\(f(x,y)\)在\(Oxy\)平面的某个有界闭区域\(D\)上连续,
那么对于\(D\)中任意一个内点\((x_0,y_0)\),
总存在着函数\(y = \phi(x)\),在点\(x_0\)的某个邻域内满足微分方程\[
	\dv{y}{x} = f(x,y),
\]且同时有\(y_0 = \phi(x_0)\).
\end{theorem}

\subsection{微分方程的解的唯一性}
\begin{theorem}
设函数\(f(x,y)\)在区域\(D\)上连续,
且满足“利普希茨条件”,即\[
	(\exists k>0)
	[
		\abs{f(x,y_1) - f(x,y_2)}
		\leq
		k \abs{y_1 - y_2}
	],
\]
那么对于\(D\)中任意一个内点\((x_0,y_0)\),
存在唯一的函数\(\phi\)满足微分方程\[
	\dv{y}{x} = f(x,y),
\]
且同时有\(y_0 = \phi(x_0)\).
\end{theorem}
