\section{可降阶的高阶微分方程}
从这一节起我们将讨论二阶及二阶以上的微分方程,即所谓的“高阶微分方程”.

对于有些高阶微分方程,我们可以通过代换将它化成较低阶的方程来求解.
以二阶微分方程\begin{equation}\label[differential-equation]{equation:微分方程.二阶微分方程.1111型}
	y'' = f(x,y,y')
\end{equation}而论,
如果我们能设法作代换把它从二阶降为一阶,
那么就有可能应用前面几节中所讲的方法来求出它的解了.

下面介绍三种容易降阶的高阶微分方程的求解方法.

\subsection{\texorpdfstring{形如\(y^{(n)} = f(x)\)}{由自变量确定n阶导数}的微分方程}
微分方程\begin{equation}\label[differential-equation]{equation:微分方程.简单高阶微分方程}
	y^{(n)} = f(x)
\end{equation}的右端仅含有自变量\(x\).
容易看出,只要把\(y^{(n-1)}\)作为新的未知函数,
那么\cref{equation:微分方程.简单高阶微分方程} 就是新未知函数的一阶微分方程.
两边积分,就得到一个\(n-1\)阶的微分方程\[
	y^{(n-1)} = \int f(x) \dd{x} \dd{x} + C_1.
\]
同理可得\[
	y^{(n-2)} = \int \left[ \int f(x) \dd{x} + C_1 \right] \dd{x} + C_2.
\]
依此法继续进行,接连积分\(n\)次,
便得\cref{equation:微分方程.简单高阶微分方程} 的含有\(n\)个任意常数的通解.

\begin{example}
求微分方程\[
	y''' = e^{2x} - \cos x
\]的通解.
\begin{solution}
对所给方程接连积分三次,得\begin{gather*}
	y'' = \frac{1}{2} e^{2x} - \sin x + C, \\
	y' = \frac{1}{4} e^{2x} + \cos x + C x + C_2, \\
	y = \frac{1}{8} e^{2x} + \sin x + C_1 x^2 + C_2 x + C_3.
\end{gather*}
这就是所求的通解.
\end{solution}
\end{example}

\subsection{\texorpdfstring{形如\(y'' = f(x,y')\)}{由自变量与一阶导数确定二阶导数}的微分方程}
微分方程\begin{equation}\label[differential-equation]{equation:微分方程.二阶微分方程.1011型}
	y'' = f(x, y')
\end{equation}的右端不显含未知函数\(y\).
如果设\(y' = p\),那么\[
	y'' = \dv[2]{y}{x} = \dv{p}{x} = p',
\]
而\cref{equation:微分方程.二阶微分方程.1011型} 就成为\[
	p' = f(x, p),
\]
这是一个关于变量\(x\)、\(p\)的一阶微分方程.
设其通解为\[
	p = \phi(x, C_1),
\]
但是\(p = \dv{y}{x}\),因此又得到一个一阶微分方程\[
	\dv{y}{x} = \phi(x, C_1).
\]
对它进行积分,便得\cref{equation:微分方程.二阶微分方程.1011型} 的
通解为\begin{equation}\label{equation:微分方程.二阶微分方程.1011型.通解}
	y = \int \phi(x, C_1) \dd{x} + C_2.
\end{equation}

\begin{example}
求微分方程\[
	(1+x^2) y'' = 2xy'
\]满足初始条件\[
	\eval{y}_{x=0} = 1,
	\qquad
	\eval{y'}_{x=0} = 3
\]的特解.
\begin{solution}
所给方程是\(y'' = f(x, y')\)型的.
设\(y' = p\),代入方程并分离变量后,有\[
	\frac{\dd{p}}{p} = \frac{2x}{1+x^2} \dd{x}.
\]
两端积分,得\[
	\ln\abs{p} = \ln(1+x^2) + C,
\]
即\(p = y' = C_1(1+x^2)\ (C_1 = \pm e^C)\).
由条件\(\eval{y'}_{x=0}=3\),得\(C_1 = 3\),所以\[
	y' = 3(1+x^2).
\]
两端再积分,得\[
	y = x^3 + 3x + C_2.
\]
又由条件\(\eval{y}_{x=0}=1\),得\(C_2=1\),于是所求的特解为\[
	y = x^3 + 3x + 1.
\]
\end{solution}
\end{example}

\begin{example}
设有一均匀、柔软的绳索,两端固定,绳索仅受重力的作用而下垂.
试问该绳索在平衡状态时是怎样的曲线?
\begin{solution}
设绳索的最低点为\(A\).
取\(y\)轴通过点\(A\)铅直向上,并取\(x\)轴水平向右,且\(\abs{OA}\)等于某个定值.
设绳索曲线的方程为\(y = \phi(x)\).
考察绳索上点\(A\)到另一点\(M(x,y)\)间的一段弧\(\Arc{AM}\),设其长为\(s\).
假定绳索的线密度为\(\rho\),则弧\(\Arc{AM}\)所受重力为\(\rho gs\).
由于绳索是柔软的,因而在点\(A\)处的张力沿水平的切线方向,其大小设为\(H\);
在点\(M\)处的张力沿该点处的切线方向,设其水平倾角为\(\theta\),其大小为\(T\).
因作用于弧段\(\Arc{AM}\)的外力相互平衡,
把作用于弧\(\Arc{AM}\)上的力沿铅直、水平两方向分解,得\[
	T \sin\theta = \rho gs,
	\qquad
	T \cos\theta = H.
\]
将此两式相除,得\[
	\tan\theta = \frac{1}{a} s
	\quad(a = \frac{H}{\rho g}).
\]
由于\(\tan\theta = y'\),\(s = \int_0^x \sqrt{1+(y')^2} \dd{x}\),代入上式即得\[
	y' = \frac{1}{a} \int_0^x \sqrt{1+(y')^2} \dd{x}.
\]
将上式两端对\(x\)求导,便得\(y = \phi(x)\)满足的微分方程\[
	y'' = \frac{1}{a} \sqrt{1+(y')^2}.
\]

取原点\(O\)到点\(A\)的距离为定值\(a\),即\(\abs{OA}=a\),那么初始条件为\[
	\eval{y}_{x=0} = a, \qquad \eval{y'}_{x=0} = 0.
\]

设\(y' = p\),则\(y'' = \dv{p}{x}\),代回微分方程,并分离变量,得\[
	\frac{\dd{p}}{\sqrt{1+p^2}} = \frac{\dd{x}}{a}.
\]
两端积分,得\[
	\ln(p+\sqrt{1+p^2}) = \frac{x}{a} + C_1.
\]
把条件\(\eval{y'}_{x=0} = \eval{p}_{x=0} = 0\)代入上式,得\(C_1 = 0\),于是上式成为\[
	\ln(p+\sqrt{1+p^2}) = \frac{x}{a},
\]
解得\[
	y' = p = \frac{1}{2} \left( e^{x/a} - e^{-x/a} \right).
\]
再积分,便得\[
	y = \frac{a}{2} \left( e^{x/a} + e^{-x/a} \right) + C_2.
\]
将条件\(\eval{y}_{x=0} = a\)代入上式,得\(C_2 = 0\).
于是该绳索的形状可由曲线方程
\begin{equation}\label{equation:微分方程.悬链线}
	y = \frac{a}{2} \left( e^{x/a} + e^{-x/a} \right)
	= a \cosh(\frac{x}{a}).
\end{equation}来表示.
这曲线叫做\DefineConcept{悬链线}.
\end{solution}
\end{example}

\subsection{\texorpdfstring{形如\(y'' = f(y,y')\)}{由因变量与一阶导数确定二阶导数}的微分方程}
方程\begin{equation}\label[differential-equation]{equation:微分方程.二阶微分方程.0111型}
	y'' = f(y, y')
\end{equation}
中不显含自变量\(x\).
为了求出它的解,我们令\(y'=p\),并利用复合函数的求导法则把\(y''\)化为对\(y\)的导数,即\[
	y'' = \dv{p}{x} = \dv{p}{y} \cdot \dv{y}{x} = p~\dv{p}{y}.
\]
这样,\cref{equation:微分方程.二阶微分方程.0111型} 就成为\[
	p~\dv{p}{y} = f(y, p).
\]
这是一个关于变量\(y\)、\(p\)的一阶微分方程.
设它的通解为\[
	y' = p = \phi(y, C_1),
\]
分离变量并积分,便得\cref{equation:微分方程.二阶微分方程.0111型} 的
通解为\begin{equation}\label{equation:微分方程.二阶微分方程.0111型.通解}
	\int \frac{\dd{y}}{\phi(y,C_1)} = x + C_2.
\end{equation}

\begin{example}
设有一质量为\(m\)的物体,在空中由静止开始下落,
如果空气阻力为\(R = cv\)(其中\(c\)是常数,\(v\)是物体运动的速度),
试求物体下落的距离\(s\)与时间\(t\)的函数关系.
\begin{solution}
物体的加速度\(\dv{v}{t}\)与速度\(v\)满足\[
	m \dv{v}{t} = mg - cv.
\]
当\(mg>cv\)时,有\[
	\frac{m}{mg-cv} \dd{v} = \dd{t},
\]
即\[
	-\frac{m}{c} \frac{1}{mg-cv} \dd(mg-cv) = \dd{t},
\]
两端积分,得\[
	-\frac{m}{c} \ln(mg-cv) = t + C_1
	\quad\text{或}\quad
	v = \frac{mg}{c} - \frac{1}{c} C_2 e^{-\frac{c}{m} t}.
\]
代入初始条件\(\eval{v}_{t=0} = 0\),则有\(C_2 = mg\),那么上式即为\[
	v = \frac{mg}{c} \left( 1 - e^{-\frac{c}{m} t} \right).
\]
再积分,得\[
	s(t) = \int_0^t v \dd{t}
	= \int_0^t \frac{mg}{c} \left( 1 - e^{-\frac{c}{m} t} \right) \dd{t} \\
	= \frac{mg}{c} \left( t + \frac{m}{c} e^{-\frac{c}{m} t} \right) + C_3.
\]
代入初始条件\(\eval{s}_{t=0} = 0\),
则有\(C_3 = -\frac{m^2}{c^2} g\),
那么上式即为\[
	s(t) = \frac{mg}{c} \left( t + \frac{m}{c} e^{-\frac{c}{m} t} - \frac{m}{c} \right).
\]
\end{solution}
\end{example}

\begin{example}
设函数\(y(x)\)具有二阶导数,且曲线\(l: y=y(x)\)与直线\(y=x\)相切于原点.
记\(\alpha\)为曲线\(l\)在点\((x,y)\)处的倾角,
且\(\dv{\alpha}{x}=\dv{y}{x}\),
求\(y(x)\)的表达式.
\begin{solution}
由导数的几何意义可知,\(\tan\alpha=y'\),
那么\(\sec^2\alpha \dv{\alpha}{x} = y''\),
即\[
	\dv{\alpha}{x} = \frac{y''}{\sec^2\alpha}
	= \frac{y''}{1+\tan^2\alpha}
	= \frac{y''}{1+(y')^2}.
\]
由题意有,\(\frac{y''}{1+(y')^2} = y'\),
令\(y' = p,
y'' = p\dv{p}{y}\),
得\begin{gather*}
	p\dv{p}{y} = p(1+p^2), \\
	\dv{p}{y} = 1+p^2, \\
	\dd{y} = \frac{\dd{p}}{1+p^2}, \\
	\arctan p = y + C_1, \\
	p = \tan(y+C_1) = y',
\end{gather*}
因为曲线\(l: y=y(x)\)与直线\(y=x\)相切于原点,
\(y(0) = 0\),\(y'(0) = 1\),
代入得\(\tan C_1 = 1\),
那么可取\(C_1 = \pi/4\),
于是\(y' = \tan(y+\pi/4)\).
因此\begin{gather*}
	\cot(y+\frac{\pi}{4}) \dd{y} = \dd{x}, \\
	\ln\abs{\sin(y+\frac{\pi}{4})} = x + C_2, \\
	\sin(y+\frac{\pi}{4}) = C_3 e^x,
\end{gather*}
又因为\(y(0) = 0\),
代入得\(\sqrt{2}/2 = C_3\),
于是\[
	y = \arcsin(\frac{\sqrt{2}}{2} e^x) - \frac{\pi}{4}.
\]
\end{solution}
\end{example}

\begin{example}
%@see: 《2025年全国硕士研究生入学统一考试(数学一)》三解答题/第18题
已知函数\(f(u)\)在区间\((0,+\infty)\)内具有二阶导数,
函数\(g(x,y) = f(x/y)\)
满足\(x^2 \pdv[2]{g}{x} + xy \pdv[2]{g}{x}{y} + y^2 \pdv[2]{g}{y} = 1\),
且\(g(x,x) = 1,
\eval{\pdv{g}{x}}_{(x,x)} = \frac2x\).
求\(f(u)\).
\begin{solution}
对\(g\)求导得\begin{align*}
	\pdv{g}{x} &= f'\left( \frac{x}{y} \right) \cdot \frac1y, \\
	\pdv{g}{y} &= -\frac{x}{y^2} f'\left( \frac{x}{y} \right), \\
	\pdv[2]{g}{x} &= f''\left( \frac{x}{y} \right) \cdot \frac1{y^2}, \\
	\pdv[2]{g}{x}{y} &= -\frac{x}{y^3} f''\left( \frac{x}{y} \right) - \frac1{y^2} \cdot f'\left( \frac{x}{y} \right), \\
	\pdv[2]{g}{y} &= \frac{x^2}{y^4} f''\left( \frac{x}{y} \right) + \frac{2x}{y^3} \cdot f'\left( \frac{x}{y} \right),
\end{align*}
代入\(x^2 \pdv[2]{g}{x} + xy \pdv[2]{g}{x}{y} + y^2 \pdv[2]{g}{y} = 1\),得\begin{equation*}
	\frac{x^2}{y^2} f''(\frac{x}{y}) + \frac{x}{y} f'\left( \frac{x}{y} \right) = 1.
\end{equation*}
令\(u=x/y\),得\(u^2 f''(u) + u f'(u) = 1\).
令\(p=f'(u)\),得\(p' + \frac1u p = \frac1{u^2}\),
这是一个一阶线性微分方程,
解得\(p = \frac1u (\ln u + C_1)\),
其中\(C_1\)是待定常数.

由\(\eval{\pdv{g}{x}}_{(x,x)} = \frac2x\)可得\(f'(1)=2\).
代入\(f'(u) = \frac1u (\ln u + C_1)\),得\(C_1=2\),
于是\(f'(u) = \frac1u (\ln u + 2)\).
积分得\(f(u) = \frac12 \ln^2 u + 2 \ln u + C_2\),
其中\(C_2\)是待定常数.

由\(g(x,x) = 1\)可得\(f(1)=1\).
代入\(f(u) = \frac12 \ln^2 u + 2 \ln u + C_2\),得\(C_2=1\),
于是\(f(u) = \frac12 \ln^2 u + 2 \ln u + 1\).
\end{solution}
\end{example}
