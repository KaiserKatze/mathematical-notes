% \begin{landscape}
\section{本章总结}
%@see: https://www.bilibili.com/video/BV11PqhYwEZv/
在解微分方程时,我们可以按以下步骤解题:
首先要找出最高阶导数,
确定它是一阶方程、二阶方程,还是三阶以上方程.
然后利用代数方法将最高阶导数的系数化为\(1\).
接下来分情况研究方程,
当它是一阶方程时,
观察它是不是可分离变量,是不是齐次方程,是不是可以化为齐次方程,
是不是线性方程,是不是伯努利方程;
当它是二阶方程时,
观察它是不是\(y'' = f(x,y')\)(缺\(y\))
或\(y'' = f(y,y')\)(缺\(x\))这两类方程,
是不是常系数线性方程;
当它是三阶以上方程时,
观察它是不是欧拉方程.

\begin{table}[!htp]
	\centering
	\begin{tblr}{l|l|l}
		\hline%
		大类 & 小类 & 解法(通解) \\ \hline%
		\begin{tblr}{l}
			一阶微分方程 \\
			\(y'=f(x,y)\) \\
		\end{tblr}
			& \begin{tblr}{l}
				可分离变量的微分方程\\
				\(g(y) \dd{y} = f(x) \dd{x}\) \\
			\end{tblr}
			& \begin{tblr}{l}
				\(\int g(y) \dd{y} = \int f(x) \dd{x}\) \\
			\end{tblr} \\ \cline{2-3}%
			& \begin{tblr}{l}
				齐次方程 \\
				\(\dv{y}{x} = \phi\left(\frac{y}{x}\right)\) \\
			\end{tblr}
			& \begin{tblr}{l}
				计算\(\int \frac{\dd{u}}{\phi(u) - u} = \int \frac{\dd{x}}{x}\), \\
				再换元\(u=\frac{y}{x}\) \\
			\end{tblr} \\ \cline{2-3}%
			& \begin{tblr}{l}
				可化齐次方程 \\
				\(\dv{y}{x} = \frac{ax+by+c}{a_1x+b_1y+c_1}\) \\
			\end{tblr}
			& \begin{tblr}{l}
				化成齐次方程 \\
				%\cref{equation:微分方程.可化为齐次的方程.换元得到的齐次方程1}
				\(\dv{Y}{X} = \frac{aX+bY}{a_1 X+b_1 Y}\); \\
				%\cref{equation:微分方程.可化为齐次的方程.换元得到的齐次方程2}
				或化为\(\dv{y}{x} = \frac{(ax+by)+c}{\lambda(ax+by)+c_1}\) \\
				(再用\(v = a x + b y\)换元) \\
			\end{tblr} \\ \cline{2-3}%
			& \begin{tblr}{l}
				一阶线性微分方程 \\
				\(\dv{y}{x} + P(x) y = Q(x)\) \\ %\cref{equation:微分方程.一阶线性非齐次微分方程}
			\end{tblr}
			& \begin{tblr}{l}
				利用常数变易法 \\
				%\cref{equation:微分方程.一阶线性非齐次微分方程的通解}
				\(y = e^{ -\int P(x) \dd{x} } \left( \int Q(x) e^{ \int P(x) \dd{x} } \dd{x} + C \right)\) \\
			\end{tblr} \\ \cline{2-3}%
			& \begin{tblr}{l}
				伯努利方程 \\
				\(\dv{y}{x} + P(x) y = Q(x) y^n\) \\
			\end{tblr}
			& \begin{tblr}{l}
				用\(z = y^{1-n}\)换元, \\
				解方程\(\dv{z}{x} + (1-n) P(x) z = (1-n) Q(x)\) \\
			\end{tblr}
		\\ \hline
		\SetCell[r=2]{c}
		\begin{tblr}{l}
			二阶微分方程 \\
			\(y'' = f(x,y,y')\) \\ %\cref{equation:微分方程.二阶微分方程.1111型}
		\end{tblr}
			& \begin{tblr}{l}
				\(y'' = f(x,y')\) \\ %\cref{equation:微分方程.二阶微分方程.1011型}
			\end{tblr}
			& \begin{tblr}{l}
				\(y = \int \phi(x, C_1) \dd{x} + C_2\) \\
			\end{tblr} \\ \cline{2-3} %\cref{equation:微分方程.二阶微分方程.1011型.通解}
			& \begin{tblr}{l}
				\(y'' = f(y, y')\) \\ %\cref{equation:微分方程.二阶微分方程.0111型}
			\end{tblr}
			& \begin{tblr}{l}
				\(\int \frac{\dd{y}}{\phi(y,C_1)} = x + C_2\) \\ %\cref{equation:微分方程.二阶微分方程.0111型.通解}
			\end{tblr}
		\\ \hline
		\SetCell[c=2]{l}
		\begin{tblr}{l}
			欧拉方程 \\ %\cref{equation:微分方程.欧拉方程的一般形式}
			\(x^n y^{(n)} + p_1 x^{n-1} y^{(n-1)}\) \\
			\(+ \dotsb + p_{n-1} x y' + p_n y = f(x)\) \\
		\end{tblr}
			&& \begin{tblr}{l}
				依次用\(x=e^t\)和\(x=-e^t\)换元 \\
			\end{tblr}
		\\ \hline
	\end{tblr}
\end{table}
% \end{landscape}
