\chapter{积分变换}
\section{傅里叶变换}
\subsection{对连续函数的傅里叶变换}
\begin{definition}
设函数\(f(x)\)是可积函数,则广义积分\begin{equation*}
	\int_{-\infty}^{+\infty}
	f(x) e^{-2 \pi i x \xi}
	\dd{x},
	\quad \xi \in \mathbb{R}
\end{equation*}
称为“对函数\(f(x)\)的\DefineConcept{傅里叶变换}”,
通常记为\(\hat{f}(\xi)\),
即\begin{equation*}
	\hat{f}(\xi)
	= \int_{-\infty}^{+\infty} f(x) e^{-2 \pi i x \xi} \dd{x},
	\quad \xi \in \mathbb{R}.
\end{equation*}
\end{definition}

\begin{theorem}
设函数\(\hat{f}(x)\)是对函数\(f(x)\)的傅里叶变换,
则有\begin{equation*}
	f(x)
	= \int_{-\infty}^{+\infty}
	\hat{f}(\xi) e^{2 \pi i \xi x} \dd{\xi},
	\quad x \in \mathbb{R}.
\end{equation*}
\end{theorem}

\section{拉普拉斯变换}
\subsection{对周期函数的拉普拉斯变换}
\begin{definition}
设\(f\colon[0,+\infty)\to\mathbb{C}\),
\(D\subseteq\mathbb{C}\).
若对\(\forall p \in D\),
都有含参反常积分\(F(p)=\int_0^{+\infty} f(t) e^{-pt} \dd{t}\)收敛,
则把\(F(p)\)称为“\(f(t)\)的\DefineConcept{拉普拉斯变换}”,
记作\(\lt f(p)\);
同时,相对地,把\(f(t)\)称为“\(F(p)\)的\DefineConcept{拉普拉斯逆变换}”,
记作\(\lt^{-1} F(t)\).
%@see: https://ins.sjtu.edu.cn/people/songtingli/resources/slides/MathPhysics/Lecture11.pdf
%@see: https://www.lamda.nju.edu.cn/yehj/dsp2021/09.pdf
\end{definition}

\begin{theorem}
设函数\(f(x)\)是周期为\(T\)的连续函数,
\(s>0\),
则\begin{equation*}
	\int_0^{+\infty} f(x) e^{-sx} \dd{x}
	= \frac1{1-e^{-sT}}
	\int_0^T f(x) e^{-sx} \dd{x}.
\end{equation*}
%@see: https://www.bilibili.com/video/BV1ja4y1Q7eF
\end{theorem}
