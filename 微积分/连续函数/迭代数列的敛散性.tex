\section{迭代数列的敛散性}
\begin{definition}\label{definition:迭代数列.不动点}
%@see: 《数学分析习题课讲义(第2版 上册)》(谢惠民、恽自求、易法槐、钱定边) P49 命题2.6.1(第一律)
设函数\(f\colon D\to\mathbb{R}\).
若点\(x_0 \in D\)满足\begin{equation*}
	f(x_0) = x_0,
\end{equation*}
则称“\(x_0\)是\(f\)的\DefineConcept{不动点}”.
\end{definition}
\begin{proposition}\label{theorem:数列极限.迭代数列.第一律}
%@see: 《数学分析习题课讲义(第2版 上册)》(谢惠民、恽自求、易法槐、钱定边) P49 命题2.6.1(第一律)
设数列\(\{x_n\}_{n\geq1}\)满足递推公式\(x_{n+1} = f(x_n)\ (n\in\mathbb{N}^+)\).
若有\begin{equation*}%equation:数列极限.迭代数列.第一律.条件1
	\lim_{n\to\infty} x_n = x_0,
\end{equation*}和\begin{equation}\label{equation:数列极限.迭代数列.第一律.条件2}
	\lim_{n\to\infty} f(x_n) = f(x_0),
\end{equation}
则\(x_0\)是\(f\)的不动点.
\begin{proof}
在递推公式等号两边同时求极限,令\(n\to\infty\),
得\begin{equation*}
	x_0 = \lim_{n\to\infty} x_{n+1} = \lim_{n\to\infty} f(x_n) = f(x_0).
	\qedhere
\end{equation*}
\end{proof}
\end{proposition}
\begin{remark}
\cref{equation:数列极限.迭代数列.第一律.条件2} 可以加强为
“\(f\)在点\(x_0\)~\emph{连续}”
(参考\cref{theorem:连续函数.函数连续点与海涅定理的关系,theorem:连续函数.单调迭代数列收敛定理}).
\cref{theorem:数列极限.迭代数列.第一律} 用处在于:
即便我们不知道数列是否收敛,也可以先去求解方程\(f(x) = x\).
而求出方程的根对于判定数列\(\{x_n\}_{n\geq1}\)的收敛性往往是有帮助的.
例如,如果方程\(f(x) = x\)在实数范围内无解,
就可以直接断定:满足\(x_{n+1} = f(x_n)\)的数列\(\{x_n\}_{n\geq1}\)一定是发散的.
\end{remark}
\begin{proposition}\label{theorem:数列极限.迭代数列.第二律}
%@see: 《数学分析习题课讲义(第2版 上册)》(谢惠民、恽自求、易法槐、钱定边) P49 命题2.6.2(第二律)
设数列\(\{x_n\}_{n\geq1}\)满足\(x_{n+1} = f(x_n)\ (n\in\mathbb{N}^+)\),
其中\(f\colon D\to\mathbb{R}\)是单调函数,
且\(\{x_n\}_{n\geq1}\)的值域为\(D\).
\begin{itemize}
	\item 当\(f\)单调增加时,\(\{x_n\}_{n\geq1}\)是单调数列.
	\item 当\(f\)单调减少时,\(\{x_n\}_{n\geq1}\)不是单调数列,
	但它的奇子列\(\{x_{2k-1}\}_{k\geq1}\)和偶子列\(\{x_{2k}\}_{k\geq1}\)都是单调数列,
	且具有相反的单调性.
\end{itemize}
\begin{proof}
假设\(f\)单调增加.
由给定条件有\(x_n \in D\ (n\geq1)\).
如果\(x_2 \geq x_1\),
那么\begin{equation*}
	x_3 = f(x_2) \geq f(x_1) = x_2,
\end{equation*}
利用数学归纳法可证\(x_{n+1} \geq x_n\)对\(n\geq1\)成立,
数列\(\{x_n\}_{n\geq1}\)单调增加.
同理,如果\(x_2 \leq x_1\),
那么数列\(\{x_n\}_{n\geq1}\)单调减少.

假设\(f\)单调减少.
由\cref{theorem:函数.两个严格单调减少函数的复合严格单调增加} 可知,
复合函数\(f \circ f\)单调增加,
这就是说,只要有\begin{equation*}
	a,b,f(a),f(b) \in D,
	\quad\text{且}\quad
	a<b,
\end{equation*}
就有\begin{equation*}
	f(f(a)) \leq f(f(b)).
\end{equation*}
如果\(x_1 = x_3\),则奇子列\(\{x_{2k-1}\}_{k\geq1}\)是常数列.
如果\(x_1 < x_3\),
由\(f\)单调减少可知\begin{equation*}
	x_2 = f(x_1) \geq f(x_3) = x_4,
\end{equation*}
然后推出\begin{equation*}
	x_3 = f(x_2) \leq f(x_4) = x_5.
\end{equation*}
用数学归纳法可证奇子列\(\{x_{2k-1}\}_{k\geq1}\)单调增加,
偶子列\(\{x_{2k}\}_{k\geq1}\)单调减少.
对于\(x_1>x_3\)的讨论完全类似,从略.
\end{proof}
\end{proposition}
\begin{remark}
从\cref{theorem:数列极限.迭代数列.第二律} 的证明过程中不难看出,
当\(f\)单调增加时,若\(\{x_n\}_{n\geq1}\)单调增加,
只有两种可能情形:
要么\(\{x_n\}_{n\geq1}\)从某项起成为常数列(每一项都是\(f\)的不动点),
要么\(\{x_n\}_{n\geq1}\)是严格单调增加数列.
\(\{x_n\}_{n\geq1}\)单调减少的情况类似.

另外,我们还可以得出结论:
如果数列\(\{x_n\}_{n\geq1}\)的值域包含于函数\(f\)的单调区间\(D\),
即\(x_n \in D\ (n=1,2,\dotsc)\),
且\(\{x_n\}_{n\geq1}\)有界(\(D\)可以是无界的),
那么当\(f\)单调增加时,数列\(\{x_n\}_{n\geq1}\)必定收敛;
当\(f\)单调减少时,数列\(\{x_n\}_{n\geq1}\)可能收敛,也可能发散,
但它的两个子列\(\{x_{2k-1}\}_{k\geq1}\)和\(\{x_{2k}\}_{k\geq1}\)必定收敛,
于是\(\{x_n\}_{n\geq1}\)是否收敛取决于
\(\{x_{2k-1}\}_{k\geq1}\)和\(\{x_{2k}\}_{k\geq1}\)的极限是否相等.
对于数列\(\{x_n\}_{n\geq1}\)无界的情况,可以作出类似的讨论.
%TODO 类似的讨论是什么意思?
\end{remark}
\begin{figure}[htb]
%@see: 《数学分析习题课讲义(第2版 上册)》(谢惠民、恽自求、易法槐、钱定边) P50 图2.4
	\centering
	\pgfplotsset{ticks=none}
	\begin{tikzpicture}
		\begin{axis}[
			xmin=0,xmax=1.5,
			ymin=0,ymax=1.5,
			axis lines=middle,
			axis equal=true,
			xlabel=$x$,
			ylabel=$y$,
			enlarge x limits=0.1,
			enlarge y limits=0.1,
			x label style={at={(ticklabel* cs:1.00)}, inner sep=5pt, anchor=south},
			y label style={at={(ticklabel* cs:1.00)}, inner sep=2pt, anchor=west},
		]
			\addplot[color=black!30,dashed,samples=2,smooth,domain=0:1.5]{x};
			\addplot[color=blue,samples=50,smooth,domain=0:1.5]{ln(x+2)};
			\begin{scope}
				\def\DrawArrow#1#2#3{
					\draw[black!30,dashed](#1,#2)--(#1,0);
					\draw[>=Stealth,->](#1,#2)--(#1,{ln(#1+2)});
					\draw[>=Stealth,->](#1,{ln(#1+2)})--({ln(#1+2)},{ln(#1+2)});
					\draw(#1,0)node[below]{#3};
				}
				\DrawArrow{.1}{0}{$x_1$}
				\DrawArrow{.74}{.74}{$x_2$}
				\DrawArrow{1}{1}{$x_3$}
			\end{scope}
		\end{axis}
	\end{tikzpicture}~\begin{tikzpicture}
		\begin{axis}[
			xmin=0,xmax=1.5,
			ymin=0,ymax=1.5,
			axis lines=middle,
			axis equal=true,
			xlabel=$x$,
			ylabel=$y$,
			enlarge x limits=0.1,
			enlarge y limits=0.1,
			x label style={at={(ticklabel* cs:1.00)}, inner sep=5pt, anchor=south},
			y label style={at={(ticklabel* cs:1.00)}, inner sep=2pt, anchor=west},
		]
			\addplot[color=black!30,dashed,samples=2,smooth,domain=0:1.5]{x};
			\addplot[color=blue,samples=50,smooth,domain=0:1.5]{exp(-x)};
			\begin{scope}
				\def\DrawArrow#1#2#3{
					\draw[black!30,dashed](#1,#2)--(#1,0);
					\draw[>=Stealth,->](#1,#2)--(#1,{exp(-#1)});
					\draw[>=Stealth,->](#1,{exp(-#1)})--({exp(-#1)},{exp(-#1)});
					\draw(#1,0)node[below]{#3};
				}
				\DrawArrow{.1}{0}{$x_1$}
				\DrawArrow{.9048}{.9048}{$x_2$}
				\DrawArrow{.4046}{.4046}{$x_3$}
				\DrawArrow{.6672}{.6672}{$x_4$}
			\end{scope}
		\end{axis}
	\end{tikzpicture}
	\caption{}
\end{figure}
\begin{proposition}\label{theorem:连续函数.单调迭代数列收敛定理}
%@see: 《数学分析习题课讲义(第2版 上册)》(谢惠民、恽自求、易法槐、钱定边) P51 命题2.6.3
%@see: 《数学分析:原理与方法》(胡适耕) P35 4.1.6命题
设\(a\)是函数\(f\)的不动点,
\(f\)在区间\((a-r,a+r)\)内严格单调增加且连续,
并且在区间\((a-r,a)\)内有\(f(x)>x\),
而在区间\((a,a+r)\)内有\(f(x)<x\),
那么只要数列\(\{x_n\}_{n\geq1}\)满足\begin{equation*}
	x_1 \in (a-r,a)\cup(a,a+r),
	\qquad
	x_{n+1} = f(x_n)\ (n=1,2,\dotsc),
\end{equation*}
则\begin{itemize}
	% 数列\(\{x_n\}_{n\geq1}\)以后不会越出区间\((a-r,a+r)\),
	\item \(x_n \in (a-r,a+r)\ (n=1,2,\dotsc)\),
	\item \(x_n \to a\ (n\to\infty)\),
	\item \(\{x_n\}_{n\geq1}\)是严格单调数列.
\end{itemize}
\begin{proof}
从条件可知,\(f\)在点\(a\)的两侧均有\(f(x) \neq x\),
因此\(f\)在区间\((a-r,a+r)\)内只可能有唯一的不动点\(a\).

不妨设\(x_1 \in (a-r,a)\).
从\(f\)的严格单调性和\(a_1 < a\)
得到\(a_2 = f(a_1) < f(a) = a\).
又因为在区间\((a-r,a)\)上满足条件\(f(x) > x\),
就有\(a_2 = f(a_1) > a_1\).
合起来就有\(a_1 < a_2 < a\).

用数学归纳法可以证明数列\(\{x_n\}_{n\geq1}\)完全落在区间\((a-r,a)\)内,
且严格单调增加,
即\begin{equation*}
	x_n < x_{n+1} < a
	\quad(n=1,2,\dotsc).
\end{equation*}
由于它以\(a\)为上界,
因此由\hyperref[theorem:极限.数列的单调有界定理]{单调有界定理}可知收敛.
不妨设\(\lim_{n\to\infty} x_n = b\).
由\cref{theorem:极限.收敛数列的保序性2} 可知\begin{equation*}
	b \in [a_1,a] \subseteq (a-r,a].
\end{equation*}
由\cref{theorem:连续函数.函数连续点与海涅定理的关系} 可知\begin{equation*}
	\lim_{n\to\infty} f(x_n) = f(b).
\end{equation*}
于是\begin{equation*}
	b = \lim_{n\to\infty} x_{n+1}
	= \lim_{n\to\infty} f(x_n)
	= f(b),
\end{equation*}
也就是说\(b\)是\(f\)的不动点.
因为\(a\)是唯一的不动点,
所以\(b = a\),
即\begin{equation*}
	\lim_{n\to\infty} x_n = a.
\end{equation*}

同理可证当\(x_1 \in (a,a+r)\)时,
数列\(\{x_n\}_{n\geq1}\)是以\(a\)为极限的严格单调减少数列.
\end{proof}
\end{proposition}
\begin{remark}
\cref{theorem:连续函数.单调迭代数列收敛定理} 的
条件“\(f\)在区间\((a-r,a+r)\)内连续”
不能弱化为“\(f\)在点\(a\)连续”,
否则无法保证在区间\((a-r,a+r)\)内只有一个不动点.
%@credit: 反例是由 {5f4d2f8a-fc8b-4798-85d6-98670f6761e7} 给出的
取\begin{equation*}
	f(x) = \left\{ \begin{array}{cc}
		\left( x + \frac12 \right)^3 - \frac12, & -1 < x < -\frac12, \\
		x^3, & -\frac12 \leq x < 1.
	\end{array} \right.
\end{equation*}
%@Mathematica: f[x_] := Piecewise[{{(x + 1/2)^3 - 1/2, -1 < x < -1/2}, {x^3, -1/2 <= x < 1}}]
%@Mathematica: Plot[{x, f[x]}, {x, -1, 1}]
%@Mathematica: Limit[f[x], x -> -1/2, Direction -> "FromBelow"]
%@Mathematica: Plot[f[x] - x, {x, -1, 1}]
%@Mathematica: Plot[f'[x] - 1, {x, -1, 1}]
点\(a=0\)是函数\(f\)的不动点,
\(f\)在点\(a\)连续,在\((-1,1)\)上严格单调增加,
并且在区间\((-1,0)\)上有\(f(x)>x\),
在区间\(0,1\)上有\(f(x)<x\).
可以看出点\(a\)是唯一的不动点.
但是由于\(f\)在点\(x=-1/2\)的左极限是\(-1/2\),
也具有类似不动点的性质:
当\(x_1 \in (-1,-1/2)\)时,
由\(x_{n+1} = f(x_n)\)迭代生成的
数列\(\{x_n\}_{n\geq1}\)的极限
就是\(-\frac12\),而不是\(a\).
\end{remark}

\begin{example}
%@see: https://www.bilibili.com/video/BV1hhtHegEnf
设数列\(\{x_n\}\)满足\(x_1=1\),且有递推公式\(x_n=x_{n+1}+2\ln(1+x_{n+1})\ (n=1,2,\dotsc)\).
\begin{itemize}
	\item 证明\(\lim_{n\to\infty} x_n\)存在,并计算它的值.
	\item 证明\(\lim_{n\to\infty} 2^n x_n\)存在,并计算它的值.
\end{itemize}
\begin{solution}
首先利用数学归纳法证明\(\{x_n\}\)有下界.
注意到\(x_1=1>0\).
假设\(x_k>0\ (k=1,2,\dotsc)\).
用反证法,假设\(x_{k+1}\leq0\),则\(\ln(1+x_{k+1})\leq0\),
从而有\(x_k=x_{k+1}+2\ln(1+x_{k+1})\leq0\),矛盾,于是\(x_{k+1}>0\).
由上可知,\(0\)是数列\(\{x_n\}\)的一个下界.
又因为\begin{equation*}
	x_n-x_{n+1}=2\ln(1+x_{n+1})>0
	\quad(n=1,2,\dotsc),
\end{equation*}
数列\(\{x_n\}\)严格单调减少,
所以由单调有界定理可知数列\(\{x_n\}\)收敛.
不妨设\(\lim_{n\to\infty} x_n = x\),
再对递推公式令\(n\to\infty\)
得\(x=x+2\ln(1+x)\),
解得\(x = \lim_{n\to\infty} x_n = 0\).

因为\begin{equation*}
	\frac{2^{n+1} x_{n+1}}{2^n x_n}
	= 2~\frac{x_{n+1}}{x_n} % 代入递推公式
	= 2~\frac{x_{n+1}}{x_{n+1}+2\ln(1+x_{n+1})} % \(\{x_n\}\)是无穷小,可以进行等价替换
	\to \frac23 < 1
	\quad(n\to\infty),
\end{equation*}
所以由\cref{theorem:无穷级数.正项级数的比值审敛法}
可知正项级数\(\sum_{n=1}^\infty 2^n x_n\)收敛,
再由\cref{theorem:无穷级数.级数收敛的必要条件}
可知\(\lim_{n\to\infty} 2^n x_n\)收敛.
\end{solution}
\end{example}

\begin{example}
%@see: https://www.bilibili.com/video/BV1cg2rY3E6L/
设数列\(\{x_n\}\)满足\(\ln x_n + \frac1{x_{n+1}} \leq 1\).
证明:\(\lim_{n\to\infty} x_n\)存在,并计算它的值.
\begin{solution}
记\(f(x) = \ln x + \frac1x\ (x>0)\),
求导得\(f'(x) = \frac1x - \frac1{x^2} = \frac{x-1}{x^2}\),
可见\(f\)在\((0,1)\)上单调减少,在\([1,+\infty)\)上单调增加,
在\(x=1\)取得最小值\(f(1) = 1\),
故对\(\forall x>0\)成立\begin{equation*}
	f(x) \geq 1.
	\eqno(1)
\end{equation*}
在(1)式中用\(x_n\)代\(x\),并由题设可知\begin{equation*}
	\ln x_n + \frac1{x_{n+1}}
	\leq 1
	\leq f(x_n)
	= \ln x_n + \frac1{x_n},
\end{equation*}
在不等号两边消去\(\ln x_n\)得\begin{equation*}
	\frac1{x_{n+1}} \leq \frac1{x_n},
	\quad\text{即}\quad
	x_n \leq x_{n+1},
\end{equation*}
数列\(\{x_n\}\)单调增加.
又因为\begin{equation*}
	\ln x_n
	\leq \ln x_n + \frac1{x_{n+1}}
	\leq 1,
\end{equation*}
所以\(x_n \leq e\),
数列\(\{x_n\}\)有上界.
由单调有界定理可知数列\(\{x_n\}\)收敛.
不妨设\(\lim_{n\to\infty} x_n = A\).
对不等式\(\ln x_n + \frac1{x_{n+1}} \leq 1\)
令\(n\to\infty\)得\begin{equation*}
	1 \leq f(A) = \ln A + \frac1A \leq 1,
\end{equation*}
于是\(f(A) = 1\),而\(A\)只能等于\(1\),
即\(\lim_{n\to\infty} x_n = 1\).
\end{solution}
\end{example}
