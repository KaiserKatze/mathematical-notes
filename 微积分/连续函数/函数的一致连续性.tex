\section{函数的一致连续性}
\subsection{一致连续性的概念}
在\cref{section:连续函数.函数的连续性与间断点}中,
我们已经指出,函数\(f\)在某个区间\(X\)上连续,
是指它在区间\(X\)上的每一点连续(对区间端点是指左连续与右连续).
我们只要观察函数在一点的连续性的定义\[
	(\forall\epsilon>0)
	(\exists\delta>0)
	(\forall x)
	[
		\abs{x - x_0} < \delta
		\implies
		\abs{f(x) - f(x_0)} < \epsilon
	],
\]
就可以发现,这里的\(\delta\)与两个因素有关:
它既依赖于\(\epsilon\),也依赖于\(x_0\).
也就是说,\(\delta\)是由\(x_0\)和\(\epsilon\)两个变量决定的函数\(\delta(x_0,\epsilon)\).

这样就产生一个问题:
对于任意给定的\(\epsilon>0\),
能否找到一个只与\(\epsilon\)有关,
而对区间\(X\)上一切点都适用的\(\delta=\delta(\epsilon)\).

这个问题的答案是不一定的.
它不仅与所讨论的函数\(f\)有关,也与所讨论的区间\(X\)有关.

\begin{definition}\label{definition:极限.函数的一致连续性}
%@see: 《数学分析(第二版 上册)》(陈纪修) P113 定理3.4.1
%@see: 《高等数学(第六版 上册)》 P73 定义
设函数\(f\)在区间\(I\)上有定义.
如果对于任意给定的正数\(\epsilon\),
总存在着正数\(\delta\),
使得对于区间\(I\)上的任意两点\(x_1\)、\(x_2\),
当\(\abs{x_1 - x_2} < \delta\)时,就有\[
	\abs{f(x_1) - f(x_2)} < \epsilon,
\]
那么称“函数\(f\)在区间\(I\)上是\DefineConcept{一致连续的}(uniformly continuous)”.
\end{definition}
函数的一致连续性表示:
不论在区间\(I\)的任何部分,
只要自变量的两个数值接近到一定程度,
就可使对应的两个函数值达到所指定的接近程度.

需要注意的是,讲述函数的一致连续性时,一定要讲明它是在哪个区间上一致连续.
一个函数虽说在区间\(I_1\)上不是一致连续的,但可以在不同的区间\(I_2\)上一致连续.

\begin{example}
%@see: 《数学分析(第二版 上册)》(陈纪修) P113 例3.4.3
证明:函数\(f(x)=\sin x\)在区间\((-\infty,+\infty)\)上是一致连续的.
\begin{proof}
因为\begin{align*}
	\abs{f(x_1)-f(x_2)}
	&= \abs{\sin x_1 - \sin x_2}
	= 2\abs{\cos\frac{x_1+x_2}{2} \sin\frac{x_1-x_2}{2}} \\
	&\leq 2\abs{\sin\frac{x_1-x_2}{2}},
\end{align*}
而当\(\abs{x_1-x_2} \leq \pi\)时有\[
	\abs{\sin\frac{x_1-x_2}{2}} \leq \abs{\frac{x_1-x_2}{2}}
\]成立,
所以\(\forall\epsilon>0\)(设\(\epsilon<\pi\)),
总可取\(\delta=\epsilon\),
对于\(\forall x_1,x_2\in\mathbb{R}\),
当\(0<\abs{x_1-x_2}<\delta=\epsilon\)时,
能使不等式\[
	\abs{f(x_1)-f(x_2)} \leq \abs{x_1-x_2} < \epsilon
\]成立,
也就是说\(f(x)\)是一致连续的.
\end{proof}
\end{example}

\begin{example}\label{example:极限.无界函数可以是一致连续的}
证明:函数\(f(x)=x\)在区间\((-\infty,+\infty)\)上是一致连续的.
\begin{proof}
因为\(\abs{f(x_1)-f(x_2)}=\abs{x_1-x_2}\),
所以\(\forall\epsilon>0\),
总可取\(\delta=\epsilon\),
对于\(\forall x_1,x_2\in\mathbb{R}\),
当\(0<\abs{x_1-x_2}<\delta=\epsilon\)时,
能使不等式\[
	\abs{f(x_1)-f(x_2)} = \abs{x_1-x_2} < \epsilon
\]成立,
也就是说\(f\)是一致连续的.
\end{proof}
\end{example}
\begin{remark}
\cref{example:极限.无界函数可以是一致连续的} 说明:
无界函数可以是一致连续的.
\end{remark}

\begin{example}\label{example:极限.在半开区间连续的函数不一定在该区间上一致连续}
%@see: 《高等数学(第六版 上册)》 P73 例2
%@see: 《数学分析(第二版 上册)》(陈纪修) P114 例3.4.4
试证:函数\(f(x) = \frac{1}{x}\)在区间\((0,1]\)上是连续的,但不是一致连续的.
\begin{proof}
因为函数\(f(x) = 1/x\)是初等函数,
它在区间\((0,1]\)上有定义,
所以在\((0,1]\)上是连续的.
假定\(f(x) = 1/x\)是\((0,1]\)上一致连续,
那么\(\forall \epsilon \in (0,1)\),
\(\exists \delta > 0\),
使得对于\(\forall x_1,x_2 \in (0,1]\),
当\(\abs{x_1 - x_2} < \delta\)时,
就有\(\abs{f(x_1) - f(x_2)} < \epsilon\).

现在取原点附近的两点\[
	x_1 = \frac{1}{n}, \quad
	x_2 = \frac{1}{n+1},
\]
其中\(n\in\mathbb{N}^+\),
显然\(x_1,x_2 \in (0,1]\)上.
因\[
	\abs{x_1 - x_2} = \abs{\frac{1}{n} - \frac{1}{n+1}}
	= \frac{1}{n(n+1)},
\]
故只要\(n\)取得足够大,
总能使\(\abs{x_1 - x_2} < \delta\).
但这时有\[
	\abs{f(x_1) - f(x_2)}
	= \abs{\frac{1}{1/n} - \frac{1}{1/(n+1)}}
	= \abs{n - (n+1)}
	= 1 > \epsilon,
\]
不符合一致连续的定义,
所以\(f(x) = \frac{1}{x}\)在\((0,1]\)上不是一致连续的.
\end{proof}
\end{example}
\begin{remark}
\cref{example:极限.在半开区间连续的函数不一定在该区间上一致连续} 说明:
在半开区间连续的函数不一定在该区间上一致连续.
\end{remark}

\begin{example}\label{example:极限.在开区间上有界且连续的函数不一定在该区间上一致连续}
证明:函数\(\sin\frac{1}{x}\)在\((0,1)\)上是不一致连续的.
\begin{proof}
取\[
	s_n = \frac{1}{2n\pi+\pi/2},
	\qquad
	t_n = \frac{1}{2n\pi},
\]
故\[
	s_n,t_n\in(0,1),
	\quad n\in\mathbb{N}^*.
\]我们有\[
	0 < t_n - s_n = \frac{\pi/2}{2n\pi(2n\pi+\pi/2)} < \frac{1}{2n\pi} < \frac{1}{n},
\]
但是\[
	\abs{\sin\frac{1}{t_n} - \sin\frac{1}{s_n}}
	= \abs{\sin(2n\pi) - \sin(2n\pi+\frac{\pi}{2})}
	= 1.
\]
这就证明了函数\(\sin\frac{1}{x}\)在\((0,1)\)上不是一致连续的.
\end{proof}
\end{example}
\begin{remark}
\cref{example:极限.在开区间上有界且连续的函数不一定在该区间上一致连续} 说明:
在开区间上有界且连续的函数不一定在该区间上一致连续.
\end{remark}

\begin{example}\label{example:极限.两个一致连续函数的乘积不一定一致连续}
证明:函数\(h(x) = x \sin x\)在\((-\infty,+\infty)\)上不是一致连续的.
\end{example}
\begin{remark}
\cref{example:极限.两个一致连续函数的乘积不一定一致连续} 说明:
两个一致连续函数的乘积不一定一致连续.
\end{remark}

\subsection{一致连续性的判别法}
\begin{theorem}
%@see: 《数学分析(第二版 上册)》(陈纪修) P114 定理3.4.5
设函数\(f\)在区间\(X\)上有定义,
则\(f\)在\(X\)上一致连续的充分必要条件是:
对任何点列\(\{a_n\}\ (a_n \in X)\)和\(\{b_n\}\ (b_n \in X)\),
只要满足\(\lim_{n\to\infty} (a_n-b_n) = 0\),
就成立\(\lim_{n\to\infty} (f(a_n)-f(b_n)) = 0\).
\end{theorem}

\begin{example}
%@see: 《数学分析(第二版 上册)》(陈纪修) P115 例3.4.5
证明:函数\(f(x) = x^2\)在\([0,+\infty)\)上不是一致连续的,
但是在\([0,A]\)上一致连续(\(A\)是任意有限正数).
\begin{proof}
取\(a_n = \sqrt{n+1},
b_n = \sqrt{n}
\ (n=1,2,\dotsc)\),
于是\[
	\lim_{n\to\infty} (a_n-b_n)
	= \lim_{n\to\infty} (\sqrt{n+1}-\sqrt{n})
	= \lim_{n\to\infty} \frac1{\sqrt{n+1}+\sqrt{n}}
	= 0,
\]
但是\(\lim_{n\to\infty} (f(a_n)-f(b_n)) = 1\),
所以\(f\)在\([0,+\infty)\)上不是一致连续的.

当区间限制在\([0,A]\)时,
有\[
	\abs{x_1^2-x_2^2}
	= \abs{(x_1+x_2)(x_1-x_2)}
	\leq 2A \abs{x_1-x_2}.
\]
对于任意给定\(\epsilon>0\),
可以取\(\delta=\frac\epsilon{2A}\),
对任意\(x_1,x_2\in[0,A]\),
只要\(\abs{x_1-x_2}<\delta\),
就成立\(\abs{x_1^2-x_2^2}<\epsilon\),
即\(f\)在\([0,A]\)上一致连续.
\end{proof}
\end{example}

\subsection{一致连续函数族}
\begin{definition}\label{definition:函数族.一致连续函数族}
定义:区间\(I\)上一致连续函数族\[
	C_U(I)
	\defeq
	\Set{
		f\in\mathbb{R}^I
		\given
		\text{\(f\)在区间\(I\)上是一致连续的}
	}.
\]
\end{definition}

已知函数\(f\colon I\to\mathbb{R}\).
上述对函数一致连续性的定义可以简化为:\[
	f \in C_U(I)
	\iff
	(\forall\epsilon>0)
	(\exists\delta>0)
	(\forall x_1,x_2 \in I)
	[
		\abs{x_1 - x_2} < \delta
		\implies
		\abs{f(x_1) - f(x_2)} < \epsilon
	];
\]
相反地,有\[
	f \notin C_U(I)
	\iff
	(\exists\epsilon_0>0)
	(\forall\delta>0)
	(\exists x_1,x_2 \in I)
	[
		\abs{x_1-x_2}<\delta
		\land
		\abs{f(x_1)-f(x_2)}\geq\epsilon_0
	].
\]

\begin{theorem}\label{theorem:极限.闭区间上连续函数的性质.一致连续函数一定连续}
如果函数\(f\)在区间\(I\)上一致连续,那么\(f\)在区间\(I\)上连续.
\begin{proof}
因为函数\(f\)在区间\(I\)上一致连续,
只要任意取定一点\(x_0 \in I\),
就有\[
	(\forall \epsilon > 0)
	(\exists \delta > 0)
	(\forall x,x_0 \in I)
	[
		\abs{x-x_0} < \delta
		\implies
		\abs{f(x)-f(x_0)} < \epsilon
	],
\]
这就是说,函数\(f\)在区间\(I\)内任意一点\(x_0\)连续,
因此\(f\)在区间\(I\)上连续.
\end{proof}
\end{theorem}

\begin{theorem}[一致连续函数的四则运算法则]\label{theorem:极限.闭区间上连续函数的性质.一致连续函数的四则运算法则}
设函数\(f,g \in C_U(I)\),则
\begin{enumerate}
	\item 两个一致连续函数的线性组合也是一致连续的,
	即\[
		(\forall\alpha,\beta\in\mathbb{R})
		[\alpha f + \beta g \in C_U(I)].
	\]

	\item 如果\(f,g\)在\(I\)上有界,
	则\(f \cdot g \in C_U(I)\).

	\item 如果\(f\)在\(I\)上有界,
	且\((\exists\epsilon_0)
	[x \in I \implies g \geq \epsilon_0]\),
	则\[
		\frac{f}{g} \in C_U(I).
	\]
\end{enumerate}
\end{theorem}

\begin{theorem}[一致连续性定理]\label{theorem:极限.一致连续性定理}
如果函数\(f\)在闭区间\([a,b]\)上连续,
那么它在该区间上一致连续,
即\[
	f \in C[a,b]
	\implies
	f \in C_U[a,b].
\]
\begin{proof}
用反证法.
假设当\(f \in C[a,b]\)时\(f \notin C_U[a,b]\),那么\[
	(\exists\delta_0>0)
	(\exists\{x_n\},\{y_n\}\subseteq[a,b])
	[
		x_n-y_n\to0
		\land
		\abs{f(x_n)-f(y_n)}\geq\epsilon_0
	].
\]

由于\(\{y_n\}\)有界,可以找到收敛子列\(\{y_{n_k}\}\)满足\[
	y_{n_k} \to y_0\in[a,b].
\]

取\(x_{n_k} = y_{n_k} + (x_{n_k} - y_{n_k})
\to y_0 + 0 = y_0\),
得到\[
	0 < \epsilon_0 \leq \abs{f(x_{n_k})-f(y_{n_k})}
	\to \abs{f(y_0)-f(y_0)} = 0.
\]矛盾,
说明函数\(f\)在\([a,b]\)上定是一致连续的.
\end{proof}
\end{theorem}
我们也把\hyperref[theorem:极限.一致连续性定理]{一致连续性定理}称为\DefineConcept{康托--海涅定理}.

\begin{example}
设函数\(f\)对于\(\forall x,y \in [a,b]\),
恒有\(\abs{f(x) - f(y)} \leq L \abs{x - y}\),
其中常数\(L > 0\),且\(f(a) \cdot f(b) < 0\).
证明:至少有一点\(\xi \in (a,b)\),使得\(f(\xi) = 0\).
\begin{proof}
因为\(\forall x,y \in [a,b] \implies \abs{f(x) - f(y)} \leq L \abs{x - y}\),
对于\(\forall \epsilon > 0\),
为了使当\(\abs{x - y} < \delta\)时不等式\[
\abs{f(x) - f(y)} \leq L \abs{x - y} < \epsilon
\]成立,
只要\(L \delta = \epsilon\)或\(\delta = \epsilon / L\)即可.
这说明\(f\)在区间\([a,b]\)上一致连续.
再由零点定理可知命题成立.
\end{proof}
\end{example}

\begin{example}
试证:若\(f\)在\([a,b]\)上连续,
\(a < x_1 < x_2 < \dotsb < x_n < b \ (n \geq 3)\),
则在\((x_1,x_n)\)内至少有一点\(\xi\),使\[
f(\xi) = \frac{1}{n} \bigl[
	f(x_1) + f(x_2) + \dotsb + f(x_n)
\bigr].
\]
\begin{proof}
根据有界性与最大值最小值定理,因为\(f\)在\([a,b]\)上连续,所以\[
\forall x \in [a,b] :
	\alpha \leq f(x) \leq \beta,
\]其中\(\alpha = \min f(x)\),\(\beta = \max f(x)\);那么\[
n \alpha \leq f(x_1) + f(x_2) + \dotsb + f(x_n) \leq n \beta,
\]\[
\alpha \leq \frac{1}{n} \bigl[f(x_1) + f(x_2) + \dotsb + f(x_n)\bigr] \leq \beta.
\]根据介值定理,在\([a,b]\)上必存在一点\(\xi\)使得\[
f(\xi) = \frac{1}{n} \bigl[ f(x_1) + f(x_2) + \dotsb + f(x_n) \bigr]
\]成立.
\end{proof}
\end{example}

\begin{theorem}\label{theorem:极限.闭区间上连续函数的性质.开区间上的连续函数一致连续的充分必要条件1}
设\(f \in C(a,b)\),
则\(f \in C_U(a,b)\)的充分必要条件是:
极限\(f(a^+)\)和\(f(b^-)\)都存在.
\end{theorem}

\begin{theorem}\label{theorem:极限.闭区间上连续函数的性质.开区间上的连续函数一致连续的充分必要条件2}
设\(f \in C(-\infty,+\infty)\),
则\(f \in C_U(-\infty,+\infty)\)的充分必要条件是:
极限\(f(-\infty)\)和\(f(+\infty)\)都存在.
\end{theorem}

\begin{definition}
对于任意一个函数\(f\colon X\to\mathbb{R}\),
集合\(E \subseteq X\),
我们把\[
	\sup_{\substack{
		\abs{x_1-x_2}<\delta \\
		x_1,x_2 \in E
	}}\abs{f(x_1)-f(x_2)},
\]
称为“函数\(f\)的\DefineConcept{连续模}”,
记为\(\amp(f;E;\delta)\).
\end{definition}

我们可以用连续模来度量一个函数的一致连续性.
