\section{连续函数}\label{section:连续函数.函数的连续性与间断点}
\subsection{连续点}
\begin{definition}\label{definition:极限.函数在一点的连续性}
%@see: 《数学分析(第二版 上册)》(陈纪修) P88 定义3.2.1
%@see: 《高等数学(第六版 上册)》 P61 定义
设函数\(f\)在点\(x_0\)的某一邻域内有定义.
如果\begin{equation*}
	\lim_{x \to x_0} f(x) = f(x_0),
\end{equation*}
那么就称“函数\(f\)在点\(x_0\)~\DefineConcept{连续}
(\(f\) is continuous at \(x_0\))”,
称“点\(x_0\)是函数\(f\)的\DefineConcept{连续点}(point of continuity)”.
\end{definition}

上述对函数连续的定义可以简化为:\begin{equation*}
	\text{\(f\)在点\(x_0\)连续}
	\defiff
	(\forall\epsilon>0)
	(\exists\delta>0)
	(\forall x)
	[
		\abs{x - x_0} < \delta
		\implies
		\abs{f(x) - f(x_0)} < \epsilon
	].
\end{equation*}

\begin{proposition}%函数在一点连续的等价定义
设函数\(f\)在点\(x_0\)的某一邻域内有定义.
如果\begin{equation*}
	\lim_{h\to0} (f(x_0+h) - f(x_0)) = 0,
\end{equation*}
则函数\(f\)在点\(x_0\)连续.
%TODO proof
\end{proposition}
\begin{proposition}\label{theorem:连续函数.函数连续点与海涅定理的关系}
%@see: 《数学分析习题课讲义(第2版 上册)》(谢惠民、恽自求、易法槐、钱定边) P49
函数\(f\)在点\(x_0\)连续的充分必要条件是:
对于每个收敛于\(x_0\)的数列\(\{x_n\}_{n\geq1}\),
% 与海涅定理有一点差别,这里不再要求数列\(\{x_n\}_{n\geq1}\)在\(x_0\)的去心邻域内
成立\begin{equation*}
	\lim_{n\to\infty} f(x_n) = f(x_0).
\end{equation*}
\begin{proof}
由\hyperref[theorem:极限.海涅定理]{海涅定理}立即可得.
\end{proof}
\end{proposition}

\begin{definition}\label{definition:极限.函数在一点的左连续性}
%@see: 《数学分析(第二版 上册)》(陈纪修) P89 定义3.2.3
如果函数\(f\)在点\(x_0\)的某一左邻域内有定义,
极限\(f(x_0^-) \defeq \lim_{x \to x_0^-} f(x)\)存在,
且\begin{equation*}
	f(x_0^-) = f(x_0),
\end{equation*}
则称“函数\(f\)在点\(x_0\)~\DefineConcept{左连续}
(\(f\) is left-continuous at \(x_0\))”.
\end{definition}
上述对函数左连续的定义可以简化为:\begin{equation*}
%@see: 《数学分析(第二版 上册)》(陈纪修) P89
	\text{\(f\)在点\(x_0\)左连续}
	\defiff
	(\forall\epsilon>0)
	(\exists\delta>0)
	(\forall x)
	[
		-\delta < x - x_0 \leq 0
		\implies
		\abs{f(x) - f(x_0)} < \epsilon
	].
\end{equation*}

\begin{definition}\label{definition:极限.函数在一点的右连续性}
%@see: 《数学分析(第二版 上册)》(陈纪修) P89 定义3.2.3
如果函数\(f\)在点\(x_0\)的某一右邻域内有定义,
极限\(f(x_0^+) \defeq \lim_{x \to x_0^+} f(x)\)存在,
且\begin{equation*}
	f(x_0^+) = f(x_0),
\end{equation*}
则称“函数\(f\)在点\(x_0\)~\DefineConcept{右连续}
(\(f\) is right-continuous at \(x_0\))”.
\end{definition}
上述对函数右连续的定义可以简化为:\begin{equation*}
%@see: 《数学分析(第二版 上册)》(陈纪修) P89
	\text{\(f\)在点\(x_0\)右连续}
	\defiff
	(\forall\epsilon>0)
	(\exists\delta>0)
	(\forall x)
	[
		0 \leq x - x_0 < \delta
		\implies
		\abs{f(x) - f(x_0)} < \epsilon
	].
\end{equation*}

我们把左连续和右连续这两个概念统称为\DefineConcept{单侧连续}.

\begin{proposition}\label{theorem:极限.函数在一点的连续性及其单侧连续性的关系}
%@see: 《数学分析(第二版 上册)》(陈纪修) P92
函数\(f\)在点\(x_0\)连续的充分必要条件是\begin{equation*}
	f(x_0^-) = f(x_0^+) = f(x_0).
\end{equation*}
\end{proposition}

\subsection{连续区间}
\begin{definition}
%@see: 《数学分析(第二版 上册)》(陈纪修) P89 定义3.2.2
如果函数\(f\)满足\begin{equation*}
	(\forall x_0\in(a,b))
	[\text{\(f\)在点\(x_0\)连续}],
\end{equation*}
那么称“函数\(f\)在开区间\((a,b)\)内连续”.
\end{definition}

\begin{example}
设函数\(f\colon(a,b)\to\mathbb{R}\)在开区间\((a,b)\)内连续,
举例说明:\(f(a^+)\)和\(f(b^-)\)均不存在.
\begin{solution}
%@credit: {61d1026b-642e-438a-9506-08e3e7865f96}
取\(f(x) = \tan x\),
它在\(\left( -\frac\pi2,\frac\pi2 \right)\)内连续,
但\begin{equation*}
	\lim_{x\to-\frac\pi2^-} f(x) = -\infty,
	\qquad
	\lim_{x\to\frac\pi2^+} f(x) = +\infty.
\end{equation*}
\end{solution}
\end{example}
\begin{example}
设函数\(f\)在点\(x_0\)的某个去心邻域内连续,
举例说明:\(f\)在点\(x_0\)的极限不存在.
\begin{solution}
取\(f(x) = \frac1x\),
它在点\(x=0\)的任意去心邻域内连续,
但是\begin{equation*}
	\lim_{x\to0} f(x) = \infty.
\end{equation*}
也可取\(g(x) = \sgn x\),
它也在点\(x=0\)的任意去心邻域内连续,
但是\begin{equation*}
	\lim_{x\to0^-} f(x) = -1,
	\neq
	\lim_{x\to0^+} f(x) = 1,
\end{equation*}
\(g\)在点\(x=0\)的极限不存在.
\end{solution}
\end{example}

\begin{example}\label{example:连续函数.狄利克雷函数处处不连续}
%@see: 《数学分析教程(第3版 上册)》(史济怀) P91 例4(1)
证明:狄利克雷函数\begin{equation*}
	D(x) = \left\{ \begin{array}{ll}
		1, & x \in \mathbb{Q}, \\
		0, & x \in \mathbb{R}-\mathbb{Q}
	\end{array} \right.
\end{equation*}在\((-\infty,\infty)\)上的每一个点都不连续.
\begin{proof}
由\cref{example:海涅定理.狄利克雷函数在任意一点的极限都不存在}
可知狄利克雷函数在任意一点的极限都不存在,
那么它当然在任意一点不连续.
\end{proof}
\end{example}
\begin{example}\label{example:连续函数.狄利克雷函数改1只在一点连续}
%@see: 《数学分析教程(第3版 上册)》(史济怀) P91 例4(2)
证明:函数\(f(x) = x~D(x)\)除了在点\(x=0\)连续之外,在其他各点均不连续.
\begin{proof}
当\(x\to0\)时,\(D(x)\)是一个有界量,
于是由\cref{theorem:函数极限.无穷小.有界函数与无穷小的乘积是无穷小} 可知
\(\lim_{x\to0} f(x) = f(0) = 0\).

取\(x_0\neq0\),有\begin{equation*}
	\lim_{\substack{x \to x_0 \\ x \in \mathbb{Q}}} f(x)
	= x_0
	\neq
	\lim_{\substack{x \to x_0 \\ x \in \mathbb{R}-\mathbb{Q}}} f(x)
	= 0,
\end{equation*}
所以由\hyperref[theorem:极限.海涅定理]{海涅定理}可知\(\lim_{x \to x_0} f(x)\)不存在.
\end{proof}
\end{example}
\begin{remark}
%@see: https://www.bilibili.com/video/BV1Rr421M7T8/
\cref{example:连续函数.狄利克雷函数处处不连续} 说明:
狄利克雷函数处处不连续.
\cref{example:连续函数.狄利克雷函数改1只在一点连续} 说明:
函数在某一点连续是函数在该点的邻域内连续的必要不充分条件.
\end{remark}

\begin{definition}
%@see: 《数学分析(第二版 上册)》(陈纪修) P90 定义3.2.4
如果函数\(f\)不仅在开区间\((a,b)\)内连续,
还在点\(a\)处右连续,且在\(b\)处左连续,
那么称“函数\(f\)在闭区间\([a,b]\)上连续”.
\end{definition}

\begin{remark}
%@see: 《数学分析(第二版 上册)》(陈纪修) P90 注
上述定义可以统一地表示为如下形式:
设函数\(f\)在某区间\(X\)上有定义.
如果\begin{equation*}
	(\forall x_0\in X)
	(\forall\epsilon>0)
	(\exists\delta>0)
	(\forall x\in X)
	[
		\abs{x-x_0}<\delta
		\implies
		\abs{f(x)-f(x_0)}<\epsilon
	],
\end{equation*}
则称“函数\(f\)在区间\(X\)上连续”.
\end{remark}

\begin{example}
根式函数\(\sqrt{x}\)在\([0,+\infty)\)上连续.
%\cref{example:极限.根式函数在某一点的极限}
\end{example}

\begin{example}
有理整函数\begin{equation*}
	P_n(x) = a_0 x^n + a_1 x^{n-1} + \dotsb + a_n
\end{equation*}在\((-\infty,+\infty)\)上连续.
%\cref{equation:函数极限.重要极限3}
\end{example}

\begin{example}
有理分式函数\begin{equation*}
	F(x) = \frac{P_n(x)}{P_m(x)}
\end{equation*}在其定义域\(\Set{ x\in\mathbb{R} \given P_m(x)\neq0 }\)上连续.
\end{example}

\begin{example}\label{example:极限.正弦函数在实数域上连续}
%@see: 《数学分析(第二版 上册)》(陈纪修) P90 例3.2.3
证明:函数\(f(x) = \sin x\)在\((-\infty,+\infty)\)上连续.
\begin{proof}
任取\(x_0\in(-\infty,+\infty)\).
由\hyperref[equation:函数.三角函数.和积互化公式12]{和积互化公式}有\begin{equation*}
	\abs{\sin x - \sin x_0}
	= 2 \abs{\cos\frac{x+x_0}2 \sin\frac{x-x_0}2}
	= 2 \abs{\cos\frac{x+x_0}2} \abs{\sin\frac{x-x_0}2}.
\end{equation*}
因为\((\forall\alpha\in\mathbb{R})[\abs{\cos\alpha}\leq1]\),
所以\begin{equation*}
	\abs{\sin x - \sin x_0} \leq 2 \abs{\sin\frac{x-x_0}2}.
\end{equation*}
又因为当\(\alpha=0\)时有\(0=\sin\alpha=\alpha\),
而当\(\alpha\neq0\)时有\(0\leq\abs{\sin\alpha}<\abs{\alpha}\),
所以\((\forall\alpha\in\mathbb{R})[\abs{\sin\alpha}\leq\abs{\alpha}]\),
于是\begin{equation*}
	\abs{\sin x - \sin x_0}
	\leq 2 \abs{\frac{x-x_0}2}
	= \abs{x-x_0}.
\end{equation*}
对于\(\forall\epsilon>0\),
取\(\delta=\epsilon\),
当\(\abs{x-x_0}<\delta\)时,
就有\(\abs{\sin x-\sin x_0}<\epsilon\),
所以\(\sin x\)在\((-\infty,+\infty)\)上连续.
\end{proof}
\end{example}

类似地可以证明,函数\(f(x) = \cos x\)在区间\((-\infty,+\infty)\)内是连续的.

% \begin{example}
% %@see: 《数学分析(第二版 上册)》(陈纪修) P91 例3.2.4
% 证明:函数\(f(x) = a^x\ (a>0,a\neq1)\)在\((-\infty,+\infty)\)上连续.
% \begin{proof}
% 首先有\begin{equation*}
% 	(\forall x_0\in\mathbb{R})
% 	[a^x-a^{x_0} = a^{x_0}(a^{x-x_0}-1)].
% \end{equation*}
% 因此,证\(\lim_{x\to x_0} a^x = a^{x_0}\)就归结为证\(\lim_{t\to0} a^t = 1\).
% \end{proof}
% \end{example}

\subsection{连续函数族}
\begin{definition}\label{definition:函数族.连续函数族}
由区间\(I\)上全部的连续函数组成的集合,称作\DefineConcept{连续函数族},
记作\(C(I)\),
即\begin{equation*}
	C(I)
	\defeq
	\Set*{
		f\in\mathbb{R}^I
		\given
		(\forall x \in I)
		[\text{\(f\)在点\(x\)连续}]
	}.
\end{equation*}
%@see: https://mathworld.wolfram.com/ContinuousFunction.html
\end{definition}

\subsection{间断点}
\begin{definition}
%@see: 《数学分析(第二版 上册)》(陈纪修) P92
设函数\(f\)在点\(x_0\)的某去心邻域内有定义.
如果函数\(f\)有下列三种情形之一:
\begin{itemize}
	\item 在\(x=x_0\)没有定义;
	\item 虽在\(x=x_0\)有定义,
	但\(\lim_{x \to x_0} f(x)\)不存在;
	\item 虽在\(x=x_0\)有定义,
	且\(\lim_{x \to x_0} f(x)\)存在,
	但\(\lim_{x \to x_0} f(x) \neq f(x_0)\),
\end{itemize}
则称“函数\(f\)在点\(x_0\)不连续”
“点\(x_0\)是函数\(f\)的\DefineConcept{不连续点}”
或“点\(x_0\)是函数\(f\)的\DefineConcept{间断点}(discontinuity)”.
\end{definition}

%@see: 《数学分析教程(第3版 上册)》(史济怀) P94 定义2.7.4
如果\(\lim_{x \to x_0} f(x) = \infty\),
则称点\(x_0\)为“函数\(f\)的\DefineConcept{无穷间断点}(infinite discontinuity)”.
例如,点\(x=0\)是函数\(y=\frac{1}{x}\)的无穷间断点.
%@see: https://mathworld.wolfram.com/InfiniteDiscontinuity.html

如果\(f\)在点\(x_0\)的某一邻域是有界的,
但其左、右极限均不存在,
则称点\(x_0\)为“函数\(f\)的\DefineConcept{振荡间断点}(oscillating discontinuity)”.
% 在 wolfram 网站上没有找到与“振荡间断点”对应的概念,这里的英文单词是我翻译的
例如,点\(x=0\)是函数\(y=\sin\frac{1}{x}\)的振荡间断点.

如果\(\lim_{x \to x_0} f(x) = A < \infty\),
但是\(f\)在点\(x_0\)没有定义,或者\(f(x_0) \neq A\),
% 有的教科书在定义可去间断点时,暗示\(f\)在点\(x_0\)有定义,例如:
% 《数学分析教程(第3版 上册)》(史济怀)
则称点\(x_0\)为“函数\(f\)的\DefineConcept{可去间断点}(removable discontinuity)”.
例如,点\(x=0\)是函数\(y=\frac{\sin x}{x}\)的可去间断点.
%@see: https://mathworld.wolfram.com/RemovableDiscontinuity.html

如果\(f\)在点\(x_0\)的左、右极限均存在且有限但不相等,
即\(\lim_{x \to x_0^-} f(x) \neq \lim_{x \to x_0^+} f(x)\),
则称点\(x_0\)为“函数\(f\)的\DefineConcept{跳跃间断点}(jump discontinuity)”,
把\(\abs{f(x_0^+)-f(x_0^-)}\)称为
“函数\(f\)在点\(x_0\)的\DefineConcept{跳跃}”.
例如,点\(x=0\)是函数\(y=\sgn x\)的跳跃间断点.
%@see: https://mathworld.wolfram.com/JumpDiscontinuity.html

如果\(x_0\)是函数\(f\)的间断点,
但左极限\(f(x_0^-)\)及右极限\(f(x_0^+)\)都存在,
那么\(x_0\)称为“函数\(f\)的\DefineConcept{第一类间断点}(discontinuity of the first kind)”.
不是第一类间断点的间断点,称为\DefineConcept{第二类间断点}(discontinuity of the second kind).
% 有的教科书在定义第二类间断点时,说“如果\(f(x_0^+)\)与\(f(x_0^-)\)二者中至少有一个不存在或者不是有限的数,那么\(x_0\)叫作\(f\)的第二类间断点”,例如:
% 《数学分析教程(第3版 上册)》(史济怀)

显然,可去间断点、跳跃间断点是第一类间断点,
无穷间断点、振荡间断点是第二类间断点.

\begin{example}
%@see: 《数学分析(第二版 上册)》(陈纪修) P93 例3.2.7
\DefineConcept{黎曼函数}\begin{equation*}
	R(x) = \left\{ \begin{array}{cl}
		\frac1p, & \text{$x=\frac{q}{p}$是既约分数}, \\
		1, & x=0, \\
		0, & \text{$x$是无理数}
	\end{array} \right.
\end{equation*}是以\(1\)为周期的周期函数,
它在任意一点\(x_0\)的极限存在且等于\(0\),
一切无理点是\(R\)的连续点,
一切有理点是\(R\)的可去间断点.
%TODO proof
\end{example}

\begin{theorem}
%@see: 《数学分析教程(第3版 上册)》(史济怀) P95 定理2.7.5
%@see: 《数学分析(第二版 上册)》(陈纪修) P94 例3.2.8
设\(f\)是开区间\((a,b)\)上的单调函数,
则\(f\)的间断点一定是跳跃间断点,它的跳跃间断点集至多是可数的.
%TODO proof
\end{theorem}

\begin{example}
已知函数\(f(x) = \frac{x-x^3}{\sin \pi x}\),求该函数的可去间断点的个数.
\begin{solution}
当\(\sin \pi x = 0\)或\(x \in \mathbb{Z}\)时,函数\(f\)无定义;
也就是说,点\(x\in\mathbb{Z}\)都是\(f\)的间断点.
要使点\(x\)成为函数\(f\)的可去间断点,必有\(x-x^3=0\),解得\(x\in\{-1,0,1\}\).
又因为\begin{gather*}
	\lim_{x\to0} \frac{x-x^3}{\sin \pi x}
	= \lim_{x\to0} \frac{x(1-x^2)}{\pi x}
	= \frac1\pi, \\
	\lim_{x\to1} \frac{x-x^3}{\sin \pi x}
	= \lim_{x\to1} \frac{1-3x^2}{\pi \cos \pi x}
	= \frac2\pi, \\
	\lim_{x\to-1} \frac{x-x^3}{\sin \pi x}
	= \lim_{x\to-1} \frac{1-3x^2}{\pi \cos \pi x}
	= \frac2\pi,
\end{gather*}
综上,函数\(f\)共有3个可去间断点.
\end{solution}
\end{example}

\begin{example}
设函数\(f(x) = \frac{x^2-x}{x^2-1}\sqrt{1+\frac{1}{x^2}}\).
试计算\(f\)的间断点种类及其个数.
\begin{solution}
因为\begin{equation*}
	f(x) = \frac{x(x-1)}{(x-1)(x+1)} \frac{\sqrt{x^2+1}}{\abs{x}},
\end{equation*}
所以\(f\)的间断点为\(x\in\{-1,0,1\}\).

又因为\begin{align*}
	&\lim_{x\to1} f(x)
	= \lim_{x\to1} \frac{\sqrt{x^2+1}}{x+1}
	= \frac{\sqrt{2}}{2}, \\
	&\lim_{x\to0^+} f(x)
	= \lim_{x\to0^+} \frac{\sqrt{x^2+1}}{x+1}
	= 1, \\
	&\lim_{x\to0^-} f(x)
	= \lim_{x\to0^-} -\frac{\sqrt{x^2+1}}{x+1}
	= -1, \\
	&\lim_{x\to-1} f(x)
	= \lim_{x\to-1} \frac{\sqrt{x^2+1}}{x+1}
	= \infty,
\end{align*}
所以点\(x=1\)是可去间断点,
点\(x=0\)是跳跃间断点,
点\(x=-1\)是无穷间断点.
那么可去间断点、跳跃间断点和无穷间断点的个数均为1个.
\end{solution}
\end{example}

% \subsection{半连续性}
% \begin{definition}
% 设\(f\colon D\to\mathbb{R}\).
% 规定:\begin{equation*}
% 	\begin{split}
% 		\text{\(f\)在点\(x_0\) \DefineConcept{上半连续}}
% 		\defiff
% 			&\text{\(f\)在\(D\)内有上界} \\
% 			&\land
% 			(\forall\epsilon>0)
% 			(\exists\delta>0)
% 			(\forall x \in D)
% 			[
% 				\abs{x-x_0}<\delta
% 				\implies
% 				f(x)<f(x_0)+\epsilon
% 			]; \\
% 		\text{\(f\)在点\(x_0\) \DefineConcept{下半连续}}
% 		\defiff
% 			&\text{\(f\)在\(D\)内有下界} \\
% 			&\land
% 			(\forall\epsilon>0)
% 			(\exists\delta>0)
% 			(\forall x \in D)
% 			[
% 				\abs{x-x_0}<\delta
% 				\implies
% 				f(x)>f(x_0)-\epsilon
% 			].
% 	\end{split}
% \end{equation*}
% %@see: https://healy.econ.ohio-state.edu/kcb/Ec181/Lecture13.pdf
% \end{definition}
