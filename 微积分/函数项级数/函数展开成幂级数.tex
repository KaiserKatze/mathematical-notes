\section{函数展开成幂级数}
前面讨论了幂级数的收敛域及其和函数的性质.
但在许多应用中,我们遇到的却是相反的问题:
给定函数\(f(x)\),
要考虑它是否能在某个区间内“展开成幂级数”,
就是说,是否能找到这样一个幂级数,它在某区间内收敛,且其和恰好就是给定的函数\(f(x)\).
如果能找到这样的幂级数,我们就说“函数\(f(x)\)在该区间内能展开成幂级数”,
也说“这个幂级数在该区间内表达了函数\(f(x)\)”.

\subsection{泰勒展式}
\begin{definition}
设函数\(f\colon D\to\mathbb{R}\).
如果在区间\(I \subseteq D\)内,
存在收敛的幂级数\(\sum_{n=0}^\infty a_n x^n\)使得\begin{equation*}
	f(x) = \sum_{n=0}^\infty a_n x^n
	\quad(x \in I)
\end{equation*}成立,
则称“函数\(f(x)\)在区间\(I\)内能展开成幂级数”.
\end{definition}

\begin{definition}
级数\begin{equation*}
	\sum_{n=0}^\infty \frac{1}{n!} f^{(n)}(x_0) (x-x_0)^n
\end{equation*}
为“函数\(f(x)\)在点\(x_0\)处的\DefineConcept{泰勒级数}(Taylor series)”.
%@see: https://mathworld.wolfram.com/TaylorSeries.html

等式\begin{equation*}
	f(x) = \sum_{n=0}^\infty {\frac{1}{n!} f^{(n)}(x_0) (x-x_0)^n},
	\quad x \in U(x_0).
\end{equation*}
称为“函数\(f(x)\)在点\(x_0\)处的\DefineConcept{泰勒展开式}”.
\end{definition}

可以看出,函数\(f(x)\)在\(U(x_0)\)内能展开成幂级数的充分必要条件为:
泰勒级数\begin{equation*}
	\sum_{n=0}^\infty {\frac{1}{n!} f^{(n)}(x_0) (x-x_0)^n}
\end{equation*}在\(U(x_0)\)内收敛,
且收敛到\(f(x)\).
但是“泰勒级数收敛到函数\(f(x)\)”这一叙述非常模糊,所以我们有以下定理:
\begin{theorem}
设函数\(f(x)\)在点\(x_0\)的某一邻域\(U(x_0)\)内具有各阶导数,
则\(f(x)\)在该邻域内能展开成泰勒级数的充分必要条件是:
在该邻域内\(f(x)\)的泰勒公式中的余项\(R_n(x)\)满足\begin{equation*}
	\lim_{n\to\infty} R_n(x) = 0,
	\quad x \in U(x_0).
\end{equation*}
\end{theorem}

\begin{definition}
级数\begin{equation*}
	\sum_{n=0}^\infty \frac{1}{n!} f^{(n)}(0) x^n
\end{equation*}
称为函数\(f(x)\)的\DefineConcept{麦克劳林级数}.

等式\begin{equation*}
	f(x) = \sum_{n=0}^\infty {\frac{1}{n!} f^{(n)}(0) x^n}
	\quad(\abs{x} < r)
\end{equation*}
称为函数\(f(x)\)的\DefineConcept{麦克劳林展开式}.
\end{definition}

\begin{theorem}
若函数\(f(x)\)在区间\(I\)内能展开成幂级数,
则它的幂级数展开式是唯一的.
\end{theorem}

要把函数\(f(x)\)展开成麦克劳林级数,可以按照下列步骤进行:
\begin{algorithm}[麦克劳林展开]\label{algorithm:函数项级数.麦克劳林展开}
\hfill
\begin{enumerate}
	\item 从一阶导数\(f'(x)\)开始,
	由低阶导数到高阶导数,
	逐个计算函数\(f(x)\)在点\(x=0\)的各阶导数\begin{equation*}
		f'(0),f''(0),\dotsc,f^{(n)}(0),\dotsc.
	\end{equation*}
	在上述计算过程中,
	\begin{itemize}
		\item 如果在\(x=0\)处某阶导数不存在,就停止展开,因为这就说明函数\(f(x)\)不能展开为麦克劳林级数;

		\item 如果各阶倒数在点\(x=0\)都存在,就继续展开.
		\begin{enumerate}
			\item 写出幂级数\begin{equation*}
				f(0) + f'(0) x + \frac{f''(0)}{2!} x^2 + \dotsb + \frac{f^{(n)}(0)}{n!} x^n + \dotsb,
			\end{equation*}
			并求出收敛半径\(R\).

			\item 利用余项\(R_n(x)\)的表达式\begin{equation*}
				R_n(x) = \frac{1}{(n+1)!} f^{(n+1)}(\theta x) x^{n+1}
				\quad(0 < \theta < 1),
			\end{equation*}
			考察当\(x\)在区间\((-R,R)\)内时余项的极限\(\lim_{n\to\infty} R_n(x)\)是否为零.
			\begin{itemize}
				\item 如果\(\lim_{n\to\infty} R_n(x) \neq 0\),则函数\(f(x)\)不能展开为麦克劳林级数.

				\item 如果\(\lim_{n\to\infty} R_n(x) = 0\),
				则函数\(f(x)\)在区间\((-R,R)\)内的麦克劳林展开式为\begin{equation*}
					f(x) = \sum_{n=0}^\infty \frac{1}{n!} f^{(n)}(0) x^n
					\quad(-R < x < R).
				\end{equation*}
			\end{itemize}
		\end{enumerate}
	\end{itemize}
\end{enumerate}
\end{algorithm}

需要注意的是,不要错误地认为“如果一个函数的泰勒级数在点\(x_0\)处收敛,那么该级数就一定收敛到这个函数”.
\begin{example}\label{example:无穷级数.函数的泰勒级数不一定收敛到函数}
最常见的例子是:\begin{equation*}
	f(x) = \left\{ \begin{array}{ll}
		e^{-1/x^2}, & x\neq0, \\
		0, & x=0.
	\end{array} \right.
\end{equation*}
根据导数的定义,以及\(\forall k\in\mathbb{R}: x^k e^{-1/x^2} \to 0\ (x\to0)\),
可知\(f^{(n)}(0) = 0\ (n=0,1,2,\dotsc)\).
于是,函数\(f\)在点\(x=0\)处的泰勒级数的每一项都是零,其和恒等于零;
只不过,当\(x\neq0\)时,\(f(x)\neq0\).
\end{example}

\begin{example}
将函数\(f(x) = e^x\)展开成\(x\)的幂级数.
\begin{solution}
所给函数的各阶导数为\begin{equation*}
	f^{(n)}(x) = e^x
	\quad(n=1,2,\dotsc),
\end{equation*}
因此\begin{equation*}
	f^{(n)}(x) = 1
	\quad(n=0,1,2,\dotsc).
\end{equation*}
于是得级数\begin{equation*}
	1+x+\frac{x^2}{2!}+\dotsb+\frac{x^n}{n!}+\dotsb,
\end{equation*}它的收敛半径\(R = +\infty\).

对于任何有限的数\(x\)和\(\xi\)(\(\xi\)在\(0\)与\(x\)之间),
余项的绝对值为\begin{equation*}
	\abs{R_n(x)} = \abs{\frac{e^{\xi}}{(n+1)!} x^{n+1}}
	< e^{\abs{x}} \cdot \frac{\abs{x}^{n+1}}{(n+1)!}.
\end{equation*}
因为\(e^{\abs{x}}\)有限,
而\(\frac{\abs{x}^{n+1}}{(n+1)!}\)是
收敛级数\(\sum_{n=0}^\infty \frac{\abs{x}^{n+1}}{(n+1)!}\)的一般项,
所以当\(n\to\infty\)时,\(e^{\abs{x}} \cdot \frac{\abs{x}^{n+1}}{(n+1)!}\to0\),
即\begin{equation*}
	\lim_{n\to\infty} \abs{R_n(x)} = 0.
\end{equation*}
于是得展开式\begin{equation*}
	e^x = 1+x+\frac{x^2}{2!}+\dotsb+\frac{x^n}{n!}+\dotsb
	\quad(-\infty<x<+\infty).
\end{equation*}
\end{solution}
\end{example}

\begin{example}
将函数\(f(x) = \sin x\)展开成\(x\)的幂级数.
\begin{solution}
所给函数的各阶导数为\begin{equation*}
	f^{(n)}(x) = \sin\left(x + n\cdot\frac{\pi}{2}\right)
	\quad(n=1,2,\dotsc),
\end{equation*}
易见\(f^{(n)}(0)\)依顺序循环地取\(0,1,0,-1,\dotsc\),于是得级数\begin{equation*}
	x-\frac{x^3}{3!}+\frac{x^5}{5!}-\dotsb+(-1)^k \frac{x^{2k+1}}{(2k+1)!}+\dotsb,
\end{equation*}
它的收敛半径\(R=+\infty\).

对于任何有限的数\(x\)和\(\xi\)(\(\xi\)在\(0\)与\(x\)之间),余项的绝对值为\begin{equation*}
	\abs{R_n(x)}
	= \abs{ \frac{x^{n+1}}{(n+1)!} \sin\left[\xi+\frac{n+1}{2}\pi\right] }
	\leq \frac{\abs{x}^{n+1}}{(n+1)!}.
\end{equation*}
而\begin{equation*}
	\lim_{n\to\infty} \frac{\abs{x}^{n+1}}{(n+1)!} = 0.
\end{equation*}
因而得展开式\begin{equation*}
	\sin x = x-\frac{x^3}{3!}+\frac{x^5}{5!}-\dotsb+(-1)^k \frac{x^{2k+1}}{(2k+1)!}+\dotsb
	\quad(-\infty<x<+\infty).
\end{equation*}
\end{solution}
\end{example}

以上将函数展开成幂级数的例子,是直接按公式\begin{equation*}
a_n = \frac{1}{n!} f^{(n)}(0)
\end{equation*}计算幂级数的系数,最后考察余项\(R_n(x)\)是否趋于零.
这种直接展开的方法计算量较大,而且研究余项即使在初等函数中也不是一件容易的事.
下面介绍间接展开的方法,这就是利用一些已知的函数展开式,通过幂级数的运算(如四则运算、逐项求导、逐项积分)以及变量代换等,将所给函数展开成幂级数.
这样做不但计算简单,而且可以避免研究余项.

前面我们已经求得的幂级数展开式有\begin{gather}
	e^x = \sum_{n=0}^\infty \frac{x^n}{n!}
		\quad(-\infty<x<+\infty), \label{equation:无穷级数.幂级数展开式1} \\
	\sin x = \sum_{k=0}^\infty \frac{(-1)^k}{(2k+1)!} x^{2k+1}
		\quad(-\infty<x<+\infty), \label{equation:无穷级数.幂级数展开式2} \\
	\frac{1}{1+x} = \sum_{n=0}^\infty (-1)^n x^n
		\quad(-1<x<1). \label{equation:无穷级数.幂级数展开式3}
\end{gather}
利用这三个展开式,可以求得许多函数的幂级数展开式.
例如对\cref{equation:无穷级数.幂级数展开式3} 两边从\(0\)到\(x\)积分,可得
\begin{equation}\label{equation:无穷级数.幂级数展开式4}
	\ln(1+x) = \sum_{n=0}^\infty \frac{(-1)^n}{n+1} x^{n+1}
	= \sum_{n=1}^\infty \frac{(-1)^{n-1}}{n} x^n
	\quad(-1<x\leq1);
\end{equation}

对\cref{equation:无穷级数.幂级数展开式2} 两边求导,即得
\begin{equation}\label{equation:无穷级数.幂级数展开式5}
	\cos x = \sum_{k=0}^\infty \frac{(-1)^k}{(2k)!} x^{2k}
	\quad(-\infty<x<+\infty);
\end{equation}

把\cref{equation:无穷级数.幂级数展开式1} 中的\(x\)换成\(x \ln a\),可得
\begin{equation}\label{equation:无穷级数.幂级数展开式6}
	a^x = e^{x \ln a} = \sum_{n=0}^\infty \frac{\ln^n a}{n!} x^n
	\quad(-\infty<x<+\infty);
\end{equation}

把\cref{equation:无穷级数.幂级数展开式3} 中的\(x\)换成\(x^2\),可得
\begin{equation}\label{equation:无穷级数.幂级数展开式7}
	\frac{1}{1+x^2} = \sum_{n=0}^\infty (-1)^n x^{2n}
	\quad(-1<x<1);
\end{equation}

对\cref{equation:无穷级数.幂级数展开式7} 从\(0\)到\(x\)积分,可得
\begin{equation}
	\arctan x = \sum_{n=0}^\infty \frac{(-1)^n}{2n+1} x^{2n+1}
	\quad(-1 \leq x \leq 1).
\end{equation}

下面再举几个用间接法把函数展开成幂级数的例子.

\begin{example}
把函数\(f(x) = (1-x) \ln(1+x)\)展开成\(x\)的幂级数.
\begin{solution}
由\cref{equation:无穷级数.幂级数展开式4}
得\begin{align*}
	f(x) &= (1-x) \sum_{n=1}^\infty \frac{(-1)^{n-1}}{n} x^n \\
	&= \sum_{n=1}^\infty \frac{(-1)^{n-1}}{n} x^n
		- \sum_{n=1}^\infty \frac{(-1)^{n-1}}{n} x^{n+1} \\
	&= \sum_{n=1}^\infty \frac{(-1)^{n-1}}{n} x^n
		- \sum_{n=2}^\infty \frac{(-1)^n}{n-1} x^n \\
	&= x + \sum_{n=2}^\infty \frac{(-1)^{n-1} (2n-1)}{n(n-1)} x^n
	\quad(-1 < x \leq 1).
\end{align*}
\end{solution}
\end{example}

\begin{example}
将函数\(\sin x\)展开成\(\left(x-\frac{\pi}{4}\right)\)的幂级数.
\begin{solution}
因为\begin{align*}
	\sin x &= \sin\left[\frac{\pi}{4}+\left(x-\frac{\pi}{4}\right)\right] \\
	&= \sin\frac{\pi}{4} \cos\left(x-\frac{\pi}{4}\right) + \cos\frac{\pi}{4} \sin\left(x-\frac{\pi}{4}\right) \\
	&= \frac{1}{\sqrt{2}} \left[\cos\left(x-\frac{\pi}{4}\right) + \sin\left(x-\frac{\pi}{4}\right)\right],
\end{align*}
又由\begin{gather*}
	\cos\left(x-\frac{\pi}{4}\right)
	= 1 - \frac{1}{2!} \left(x-\frac{\pi}{4}\right)^2 + \frac{1}{4!} \left(x-\frac{\pi}{4}\right)^4 - \dotsb
	\quad(-\infty < x < +\infty), \\
	\sin\left(x-\frac{\pi}{4}\right)
	= \left(x-\frac{\pi}{4}\right) - \frac{1}{3!} \left(x-\frac{\pi}{4}\right)^3 + \frac{1}{5!} \left(x-\frac{\pi}{4}\right)^5 - \dotsb
	\quad(-\infty < x < +\infty),
\end{gather*}
所以\begin{equation*}
	\sin x = \frac{1}{\sqrt{2}} \left[
	1 + \left(x-\frac{\pi}{4}\right)
	- \frac{1}{2!} \left(x-\frac{\pi}{4}\right)^2
	- \frac{1}{3!} \left(x-\frac{\pi}{4}\right)^3
	+ \dotsb
	\right]
	\quad(-\infty < x < +\infty).
\end{equation*}
\end{solution}
\end{example}

\begin{example}
将函数\(f(x) = \frac{1}{x^2+4x+3}\)展开成\((x-1)\)的幂级数.
\begin{solution}
因为\begin{align*}
	f(x) &= \frac{1}{x^2+4x+3}
	= \frac{1}{(x+1)(x+3)} \\
	&= \frac{1}{2(1+x)} - \frac{1}{2(3+x)} \\
	&= \frac{1}{4\left(1+\frac{x-1}{2}\right)}
	- \frac{1}{8\left(1+\frac{x-1}{4}\right)},
\end{align*}
而\begin{gather*}
	\frac{1}{4\left(1+\frac{x-1}{2}\right)}
	= \frac{1}{4} \sum_{n=0}^\infty \frac{(-1)^n}{2^n} (x-1)^n
	\quad(-1<x<3), \\
	\frac{1}{8\left(1+\frac{x-1}{4}\right)}
	= \frac{1}{8} \sum_{n=0}^\infty \frac{(-1)^n}{4^n} (x-1)^n
	\quad(-3<x<5),
\end{gather*}
所以\begin{equation*}
	f(x) = \frac{1}{x^2+4x+3}
	= \sum_{n=0}^\infty (-1)^n \left(\frac{1}{2^{n+2}}-\frac{1}{2^{2n+3}}\right) (x-1)^n
	\quad(-1<x<3).
\end{equation*}
\end{solution}
\end{example}

\begin{example}
将函数\(\sinh x = \frac{e^x - e^{-x}}{2}\)展开成\(x\)的幂级数,并求展开式成立的区间.
\begin{solution}
由\cref{equation:无穷级数.幂级数展开式1} 可知,\begin{equation*}
	\sinh x
	= \frac{e^x - e^{-x}}{2}
	= \frac{1}{2} \left[
	\sum_{n=0}^\infty \frac{1}{n!} x^n
	- \sum_{n=0}^\infty \frac{1}{n!} (-x)^n
	\right].
\end{equation*}
因为\begin{equation*}
	(-x)^n = \left\{ \begin{array}{cl}
		x^n, & n=2k, \\
		-x^n, & n=2k+1,
	\end{array} \right.
	\quad(k\in\mathbb{N}),
\end{equation*}
所以\begin{equation*}
	\sinh x
	= \sum_{k=0}^\infty \frac{x^{2k+1}}{(2k+1)!}
	\quad(-\infty<x<+\infty).
\end{equation*}
\end{solution}
\end{example}

最后,再举一个用直接法展开的例子.

\begin{example}
将函数\(f(x) = (1+x)^m\)展开成\(x\)的幂级数,其中\(m\)为任意实数.
\begin{solution}
\(f(x)\)的各阶导数为\begin{align*}
	&f'(x) = m (1+x)^{m-1},
	f''(x) = m(m-1) (1+x)^{m-2},
	\dotsc, \\
	&f^{(n)}(x) = m(m-1)(m-2)\dotsm(m-n+1) (1+x)^{m-n},
	\dotsc,
\end{align*}
所以\begin{align*}
	&f(0) = 1,
	f'(0) = m,
	f''(0) = m(m-1),
	\dotsc, \\
	&f^{(n)}(0) = m(m-1)\dotsm(m-n+1),\dotsc,
\end{align*}
于是得级数\begin{equation*}
	1+mx+\frac{m(m-1)}{2!}x^2+\dotsb+\frac{m(m-1)\dotsm(m-n+1)}{n!}x^n+\dotsb.
\end{equation*}
这级数相邻的系数之比的绝对值\begin{align*}
	\abs{\frac{a_{n+1}}{a_n}}
	&= \abs{ \frac{m(m-1)\dotsm(m-n)}{(n+1)!} \Bigg/ \frac{m(m-1)\dotsm(m-n+1)}{n!} } \\
	&= \abs{ \frac{m-n}{n+1} }
	\to 1 \quad(n\to\infty),
\end{align*}
因此,对于任何实数\(m\)这级数在开区间\((-1,1)\)内收敛.

为了避免直接研究余项,设这级数在开区间\((-1,1)\)内收敛到函数\(F(x)\):\begin{align*}
	F(x)
	&= 1+mx+\frac{m(m-1)}{2!}x^2+\dotsb \\
	&\hspace{20pt}+\frac{m(m-1)\dotsm(m-n+1)}{n!}x^n+\dotsb
	\quad(-1<x<1),
\end{align*}
下面证明\(F(x) = (1+x)^m\ (-1<x<1)\).

逐项求导,得\begin{equation*}
	F'(x) = m \left[
		1+\frac{m-1}{1}x+\dotsb+\frac{(m-1)\dotsm(m-n+1)}{(n-1)!}x^{n-1}+\dotsb
	\right],
\end{equation*}
两边各乘以\((1+x)\),并把含有\(x^n\ (n=1,2,\dotsc)\)的两项合并起来.
根据\cref{theorem:组合数性质2}
\begin{align*}
	&\frac{(m-1)\dotsm(m-n+1)}{(n-1)!}
		+ \frac{(m-1)\dotsm(m-n)}{n!} \\
	&= \frac{m(m-1)\dotsm(m-n+1)}{n!}
		\quad(n=1,2,\dotsc),
\end{align*}
可得\begin{align*}
	(1+x) F'(x)
	&= m \Biggl[
	1+mx+\frac{m(m-1)}{2!}x^2+\dotsb \\
	&\hspace{30pt} +\frac{m(m-1)\dotsm(m-n+1)}{n!}x^n+\dotsb
	\Biggr] \\
	&= m F(x)
	\quad(-1<x<1).
\end{align*}

现在令\(\phi(x) = \frac{F(x)}{(1+x)^m}\),
于是\(\phi(0) = F(0) = 1\),
且\begin{align*}
	\phi'(x)
	&= \frac{(1+x)^m F'(x) - m(1+x)^{m-1} F(x)}{(1+x)^{2m}} \\
	&= \frac{(1+x)^{m-1} [(1+x) F'(x) - m F(x)]}{(1+x)^{2m}}
	= 0,
\end{align*}
所以\(\phi(x)\)是常数函数.
又因为\(\phi(0) = 1\),
所以\(\phi(x) = 1\),
即\begin{equation*}
	F(x) = (1+x)^m.
\end{equation*}
因此原级数在区间\((-1,1)\)内有展开式
\begin{equation}\label{equation:无穷级数.二项展开式}
\begin{split}
	(1+x)^m
	&= 1+mx+\frac{m(m-1)}{2!}x^2+\dotsb \\
	&\hspace{20pt} +\frac{m(m-1)\dotsm(m-n+1)}{n!}x^n+\dotsb
	\quad(-1<x<1).
\end{split}
\end{equation}
在区间的端点,展开式是否成立要看\(m\)的数值而定.
\end{solution}
\end{example}

特殊地,当\(m\)为正整数时,级数为\(x\)的\(m\)次多项式,这就是代数学中的二项式定理.

对应于\(m=\pm1/2\)的二项展开式分别为\begin{gather*}
	\sqrt{1+x}
	= 1+\frac{1}{2}x-\frac{1}{2\cdot4}x^2+\frac{1\cdot3}{2\cdot4\cdot6}x^3-\frac{1\cdot3\cdot5}{2\cdot4\cdot6\cdot8}x^4+\dotsb
	\quad(-1 \leq x \leq 1), \\
	\frac{1}{\sqrt{1+x}}
	= 1-\frac{1}{2}x+\frac{1\cdot3}{2\cdot4}x^2-\frac{1\cdot3\cdot5}{2\cdot4\cdot6}x^3+\frac{1\cdot3\cdot5\cdot7}{2\cdot4\cdot6\cdot8}x^4-\dotsb
	\quad(-1 < x \leq 1).
\end{gather*}

现在我们可以利用幂级数展开式计算定积分.
\begin{example}
计算\begin{equation*}
	\int_0^1 \frac{\sin\ln x}{\ln x} \dd{x}.
\end{equation*}
\begin{solution}
由\cref{equation:无穷级数.幂级数展开式2} \begin{equation*}
	\sin x = \sum_{k=0}^\infty \frac{(-1)^k}{(2k+1)!} x^{2k+1}
	\quad(-\infty<x<+\infty),
\end{equation*}
有\begin{equation*}
	\frac{\sin\ln x}{\ln x} = \sum_{k=0}^\infty \frac{(-1)^k}{(2k+1)!} (\ln x)^{2k},
\end{equation*}
那么\begin{align*}
	\int_0^1 \frac{\sin\ln x}{\ln x} \dd{x}
	&= \int_0^1 \sum_{k=0}^\infty \frac{(-1)^k}{(2k+1)!} (\ln x)^{2k} \dd{x} \\
	&= \sum_{k=0}^\infty \frac{(-1)^k}{(2k+1)!} \int_0^1 (\ln x)^{2k} \dd{x} \\
	&\xlongequal{\ln x = -t}
		\sum_{k=0}^\infty \frac{(-1)^k}{(2k+1)!}
		\int_{+\infty}^0 (-t)^{2k} \cdot (-e^{-t}) \dd{t} \\
	&= \sum_{k=0}^\infty \frac{(-1)^k}{(2k+1)!}
		\int^{+\infty}_0 e^{-t} t^{2k} \dd{t} \\
	&= \sum_{k=0}^\infty \frac{(-1)^k}{(2k+1)!} \Gamma(2k+1) \\
	&= \sum_{k=0}^\infty \frac{(-1)^k}{(2k+1)!} (2k)! \\
	&= \sum_{k=0}^\infty (-1)^k \frac{1}{2k+1} \\
	&= \sum_{k=0}^\infty (-1)^k \int_0^1 u^{2k} \dd{u} \\
	&= \int_0^1 \sum_{k=0}^\infty (-u^2)^k \dd{u} \\
	&= \int_0^1 \frac{1}{1-(-u^2)} \dd{u} \\
	&= \left(\arctan u\right)_0^1
	= \frac{\pi}{4}.
\end{align*}
\end{solution}
\end{example}

\begin{example}
求幂级数\(\sum_{n=1}^\infty \frac{(-1)^{n-1}}{2n-1} x^{2n}\)的收敛域及和函数.
%TODO
\end{example}
