\section{查普曼--柯尔莫哥洛夫方程}
%@see: 《应用随机过程:概率模型导论(第11版)》(Sheldon M. Ross,龚光鲁译) P153
%@see: https://encyclopediaofmath.org/wiki/Kolmogorov-Chapman_equation
现在,我们用\(P^n_{ij}\)表示“处于状态\(i\)的过程,在经过\(n\)次转移后,处于状态\(j\)”的概率,
即\begin{equation*}
	P^n_{ij}
	\defeq
	P(
		X_{n+k} = j
		\vert
		X_k = i
	),
	\quad(n,i,j=0,1,2,\dotsc).
\end{equation*}
显然\(P^1_{ij} = P_{ij}\).

容易看出:\begin{align}
	P^{n+m}_{ij}
	&= P(X_{n+m} = j \vert X_0 = i) \notag\\
	&= \sum_k P(X_{n+m} = j, X_n = k \vert X_0 = i) \notag\\
	&= \sum_k P(X_{n+m} = j \vert X_n = k, X_0 = i) P(X_n = k \vert X_0 = i) \notag\\
	%@see: 《应用随机过程:概率模型导论(第11版)》(Sheldon M. Ross,龚光鲁译) P153 (4.2)
	&= \sum_k P^m_{kj} P^n_{ik}.
		\label{equation:马尔可夫链.CK方程}
\end{align}
我们把方程组 \labelcref{equation:马尔可夫链.CK方程}
称为\DefineConcept{查普曼--柯尔莫哥洛夫方程}(Chapman-Kolmogorov equation).

我们把矩阵\begin{equation*}
	\vb{P}^{(n)}
	\defeq \begin{bmatrix}
		P^n_{00} & P^n_{01} & \dots \\
		P^n_{10} & P^n_{11} & \dots \\
		\vdots & \vdots & \\
	\end{bmatrix}
\end{equation*}
称为“马尔可夫链\(\{X_n\}_{n \in T}\)的 \DefineConcept{\(n\)步转移概率矩阵}(probability transition matrix,stochastic matrix)”
或“马尔可夫链\(\{X_n\}_{n \in T}\)的 \DefineConcept{\(n\)步马尔可夫矩阵}(Markov matrix)”.
于是\hyperref[equation:马尔可夫链.CK方程]{查普曼--柯尔莫哥洛夫方程}表明:\begin{equation}
	\vb{P}^{(n+m)}
	= \vb{P}^{(n)} \vb{P}^{(m)}.
\end{equation}
因此,\(
	\vb{P}^{(2)}
	= \vb{P}^{(1+1)}
	= \vb{P} \vb{P}
	= (\vb{P})^2
\),
利用数学归纳法可得\begin{equation}
	\vb{P}^{(n)}
	= (\vb{P})^n,
\end{equation}
也就是说,\(n\)步转移概率矩阵就是矩阵\(\vb{P}\)的\(n\)次幂.

\begin{example}
%@see: 《应用随机过程:概率模型导论(第11版)》(Sheldon M. Ross,龚光鲁译) P155 例4.10
在一个瓮中放入两颗红球.
每隔一分钟,从瓮中取出一颗球,同时放回一颗新球
(新球的颜色有\(0.8\)的概率与刚刚取出的球同色,有\(0.2\)的概率与之反色).
求“第五次取到的球是红色”的概率.
\begin{solution}
首先定义一个合适的马尔可夫链,
注意到取到红球的概率是由抽取时瓮中的球的组成决定,
于是我们将\(X_n\)定义为“经过\(n\)次抽取、放回后瓮中红球的个数”,
那么\(\{X_n\}_{n\geq0}\)是一个以\(\{0,1,2\}\)为状态的马尔可夫链,
它的转移矩阵为\begin{equation*}
	\vb{P} \defeq \begin{bmatrix}
		0.8 & 0.2 & 0.0 \\
		0.1 & 0.8 & 0.1 \\
		0.0 & 0.2 & 0.8
	\end{bmatrix}.
\end{equation*}
% 可以这样理解上述转移矩阵:考虑“瓮中红球从\(1\)个变为\(0\)个”这种情形,
% 这表明已经取出的球必定是红球(它发生的概率是\(0.5\)),
% 同时必须放回一个颜色相反的球(它发生的概率是\(0.2\)),
% 即\(P_{10} = (0.5)(0.2) = 0.1\).
因为\begin{equation*}
	\vb{P}^4
	\approx \begin{bmatrix}
		0.4872 & 0.4352 & 0.0776 \\
		0.2176 & 0.5648 & 0.2176 \\
		0.0776 & 0.4352 & 0.4872 \\
	\end{bmatrix}.
\end{equation*}
%@Mathematica: P = {{0.8, 0.2, 0.0}, {0.1, 0.8, 0.1}, {0.0, 0.2, 0.8}};
%@Mathematica: P4 = MatrixPower[P, 4];
%@Mathematica: (0.5) * P4[[3, 2]] + (1.0) * P4[[3, 3]]
于是\begin{align*}
	&\hspace{-20pt}
	P(\text{第五次取到的是红球}) \\
	&= \sum_{i=0}^2 P(\text{第五次取到的是红球} \vert X_4 = i) P(X_4 = i \vert X_0 = 2) \\
	&= (0.0) P_{20}^4 + (0.5) P_{21}^4 + (1.0) P_{22}^4 \\
	&\approx (0.5)(0.4352) + (1.0)(0.4872)
	= 0.7048.
\end{align*}
\end{solution}
\end{example}

\begin{example}
%@see: 《应用随机过程:概率模型导论(第11版)》(Sheldon M. Ross,龚光鲁译) P156 例4.11
假定以等可能把球逐个分配到8个空瓮中.
求“在分配9次后,恰有3个瓮非空”的概率.
\begin{solution}
用\(X_n\)表示“第\(n\)次分配后,非空瓮的数目”,
那么\(\{X_n\}_{n\geq0}\)是一个以\(\{0,1,2,\dotsc,8\}\)为状态的马尔可夫链.
由题意可知,一开始8个瓮都是空的,即\(X_0 = 0\).

在第一次分配后,不论把球放到哪个瓮中,必然有一个瓮非空,
因此\begin{equation*}
	P_{01} = 1,
	\qquad
	P_{0k} = 0
	\ (k\neq1).
\end{equation*}

鉴于非空瓮的数目不能减小,也不可能一次分配增加超过1个球,
因此\begin{equation*}
	P_{ij} = 0
	\quad(j<i \lor j-i>1).
\end{equation*}

接下来,从第二次分配开始,每次放球有两种可能,
一种可能是把球放进已有球的瓮中,
一种可能是把球放进空瓮中,
因此\begin{equation*}
	P_{ii} + P_{i,i+1} = 1
	\quad(0 \leq i \leq 8).
\end{equation*}

%@credit: {Gemini}
转移概率\(P_{ii}\)表示“当非空瓮数目为\(i\)时,经过一次分配,非空瓮数目不变”的概率.
要想保持非空瓮数目不变,就必须把球放入\(i\)个非空瓮中,
于是可能的分配方式种数为\(i\),
同时总的分配方式种数为\(8\),
因此\begin{equation*}
	P_{ii} = i/8
	\quad(i=1,2,\dotsc).
\end{equation*}

%@Mathematica: P[i_, j_] := Piecewise[{{1, i == 0 \[And] j == 1}, {0, j < i \[Or] j - i > 1}, {i/8, j == i}, {1 - P[i, i], j == i + 1}}];
%@Mathematica: MatP = Table[P[i, j], {i, 0, 8}, {j, 0, 8}];
综上所述,转移矩阵为\begin{equation*}
	\vb{P} \defeq \begin{bmatrix}
		\begin{tblr}{*9c}
			0 & 1 & 0 & 0 & 0 & 0 & 0 & 0 & 0 \\
			0 & \frac{1}{8} & \frac{7}{8} & 0 &
			0 & 0 & 0 & 0 & 0 \\
			0 & 0 & \frac{1}{4} & \frac{3}{4} &
			0 & 0 & 0 & 0 & 0 \\
			0 & 0 & 0 & \frac{3}{8} &
			\frac{5}{8} & 0 & 0 & 0 & 0 \\
			0 & 0 & 0 & 0 & \frac{1}{2} &
			\frac{1}{2} & 0 & 0 & 0 \\
			0 & 0 & 0 & 0 & 0 & \frac{5}{8} &
			\frac{3}{8} & 0 & 0 \\
			0 & 0 & 0 & 0 & 0 & 0 & \frac{3}{4}
			& \frac{1}{4} & 0 \\
			0 & 0 & 0 & 0 & 0 & 0 & 0 &
			\frac{7}{8} & \frac{1}{8} \\
			0 & 0 & 0 & 0 & 0 & 0 & 0 & 0 & 1 \\
		\end{tblr}
	\end{bmatrix}.
\end{equation*}

%@Mathematica: (MatrixPower[MatP, 8] // N)[[2, 4]]
于是,“在分配9次后,恰有3个瓮非空”的概率是\begin{equation*}
	P_{03}^9
	= \sum_k P_{0k} P_{k3}^8
	= P_{01} P_{13}^8
	= P_{13}^8
	\approx 0.007~573.
\end{equation*}
\end{solution}
\end{example}
