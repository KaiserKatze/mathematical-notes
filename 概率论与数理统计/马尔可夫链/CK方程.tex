\section{查普曼--柯尔莫哥洛夫方程}
%@see: 《应用随机过程:概率模型导论(第11版)》(Sheldon M. Ross,龚光鲁译) P153
%@see: https://encyclopediaofmath.org/wiki/Kolmogorov-Chapman_equation
现在,我们用\(P^n_{ij}\)表示“处于状态\(i\)的过程,在经过\(n\)次转移后,处于状态\(j\)”的概率,
即\begin{equation*}
	P^n_{ij}
	\defeq
	P(
		X_{n+k} = j
		\vert
		X_k = i
	),
	\quad(n,i,j=0,1,2,\dotsc).
\end{equation*}
显然\(P^1_{ij} = P_{ij}\).

容易看出:\begin{align}
	P^{n+m}_{ij}
	&= P(X_{n+m} = j \vert X_0 = i) \notag\\
	&= \sum_k P(X_{n+m} = j, X_n = k \vert X_0 = i) \notag\\
	&= \sum_k P(X_{n+m} = j \vert X_n = k, X_0 = i) P(X_n = k \vert X_0 = i) \notag\\
	%@see: 《应用随机过程:概率模型导论(第11版)》(Sheldon M. Ross,龚光鲁译) P153 (4.2)
	&= \sum_k P^m_{kj} P^n_{ik}.
		\label{equation:马尔可夫链.CK方程}
\end{align}
我们把方程组 \labelcref{equation:马尔可夫链.CK方程}
称为\DefineConcept{查普曼--柯尔莫哥洛夫方程}(Chapman-Kolmogorov equation).

我们把矩阵\begin{equation*}
	\vb{P}^{(n)}
	\defeq \begin{bmatrix}
		P^n_{00} & P^n_{01} & \dots \\
		P^n_{10} & P^n_{11} & \dots \\
		\vdots & \vdots & \\
	\end{bmatrix}
\end{equation*}
称为“马尔可夫链\(\{X_n\}_{n \in T}\)的 \DefineConcept{\(n\)步转移概率矩阵}(probability transition matrix,stochastic matrix)”
或“马尔可夫链\(\{X_n\}_{n \in T}\)的 \DefineConcept{\(n\)步马尔可夫矩阵}(Markov matrix)”.
于是\hyperref[equation:马尔可夫链.CK方程]{查普曼--柯尔莫哥洛夫方程}表明:\begin{equation}
	\vb{P}^{(n+m)}
	= \vb{P}^{(n)} \vb{P}^{(m)}.
\end{equation}
因此,\(
	\vb{P}^{(2)}
	= \vb{P}^{(1+1)}
	= \vb{P} \vb{P}
	= (\vb{P})^2
\),
利用数学归纳法可得\begin{equation}
	\vb{P}^{(n)}
	= (\vb{P})^n,
\end{equation}
也就是说,\(n\)步转移概率矩阵就是矩阵\(\vb{P}\)的\(n\)次幂.
