\section{马尔可夫链}
\begin{definition}
%@see: 《应用随机过程:概率模型导论(第11版)》(Sheldon M. Ross,龚光鲁译) P150
设\(\{X_n\}_{n \in T}\)是一个随机过程,
\(\{X_n\}_{n \in T}\)的指标集与值域是\(\mathbb{N}\)中的某两个区间,
记\begin{equation*}
	P_{ij}
	\defeq
	P(
		X_{n+1} = j
		\vert
		X_n = i,
		X_{n-1} = k_{n-1},
		X_{n-2} = k_{n-2},
		\dotsc,
		X_1 = k_1,
		X_0 = k_0
	).
\end{equation*}
如果\begin{equation*}
	P_{ij}
	= P(
		X_{n+1} = j
		\vert
		X_n = i
	),
\end{equation*}
那么称“\(\{X_n\}_{n \in T}\)是一个\DefineConcept{马尔可夫链}(Markov chain)”.
%@see: https://mathworld.wolfram.com/MarkovChain.html
\end{definition}

显然\(P_{ij}\)表示“过程处于状态\(i\)时,下一次转移到状态\(j\)”的概率.
由于概率都是非负的,
又由于过程必须转移到某个状态,
所以有\begin{equation*}
	P_{ij} \geq 0\ (i,j=0,1,2,\dotsc),
	\qquad
	\sum_j P_{ij} = 1\ (i=0,1,2,\dotsc).
\end{equation*}

我们把矩阵\begin{equation*}
	\vb{P}
	\defeq \begin{bmatrix}
		P_{00} & P_{01} & \dots \\
		P_{10} & P_{11} & \dots \\
		\vdots & \vdots & \\
	\end{bmatrix}
\end{equation*}
称为“马尔可夫链\(\{X_n\}_{n \in T}\)的\DefineConcept{转移概率矩阵}(probability transition matrix,stochastic matrix)”
或“马尔可夫链\(\{X_n\}_{n \in T}\)的\DefineConcept{马尔可夫矩阵}(Markov matrix)”.
%@see: https://mathworld.wolfram.com/StochasticMatrix.html
