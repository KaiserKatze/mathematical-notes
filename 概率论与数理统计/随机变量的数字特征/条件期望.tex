\section{条件期望}
%@see: 《概率论(上册)》(李增沪、张梅、何辉) P120
考虑概率空间\((\Omega,\mathcal{F},P)\),
设事件\(B \in \mathcal{F}\)满足\(P(B) > 0\),
我们知道,映射\begin{equation*}
%@see: 《概率论(上册)》(李增沪、张梅、何辉) P120 (4.4.1)
	P_B\colon \mathcal{F} \to \mathbb{R},
	A \mapsto P(A \vert B) \defeq \frac{P(AB)}{P(B)}.
\end{equation*}
也是可测空间\((\Omega,\mathcal{F})\)上的一个概率测度,
即\((\Omega,\mathcal{F},P_B)\)也是一个概率空间,
并且概率空间\((\Omega,\mathcal{F},P)\)上的任意一个随机变量\(X\)
也可以视为概率空间\((\Omega,\mathcal{F},P_B)\)上的一个随机变量.

\begin{definition}
%@see: 《概率论(上册)》(李增沪、张梅、何辉) P120 定义4.4.1
随机变量\(X\)在概率空间\((\Omega,\mathcal{F},P_B)\)上的数学期望,
称为“随机变量\(X\)的(给定事件\(B\)的)\DefineConcept{条件期望}”,
记作\(E(X \vert B)\).
\end{definition}

用\(F_X(x \vert B)\)表示\(X\)作为\((\Omega,\mathcal{F},P_B)\)上的随机变量的分布函数,
给定\(\mathbb{R}\)上的波莱尔可测函数,
% 根据定理4.2.7有
\begin{equation}
%@see: 《概率论(上册)》(李增沪、张梅、何辉) P120 (4.4.2)
	E(g(X) \vert B)
	= \int_{-\infty}^{+\infty} g(x) \dd{F_X(x \vert B)}.
\end{equation}
特别地,如果\(X\)是取值于可数集\(\{x_1,x_2,\dotsc\} \subseteq \mathbb{R}\)的离散型随机变量,
% 应用命题4.2.6 得
\begin{equation}
%@see: 《概率论(上册)》(李增沪、张梅、何辉) P120 (4.4.3)
	E(g(X) \vert B)
	= \sum_{i=1}^\infty g(x_i) P(X=x_i \vert B).
\end{equation}

% \begin{definition}
% %@see: 《概率论与数理统计》(茆诗松、周纪芗、张日权) P165 定义3.4.1
% 条件概率分布的数学期望称为\DefineConcept{条件期望}.
% \end{definition}
% \begin{equation}
% %@see: 《概率论与数理统计》(茆诗松、周纪芗、张日权) P165 (3.4.10)
% 	E(X \vert y)
% 	= \left\{ \begin{array}{cl}
% 		\sum_i x_i P(X=x_i \vert Y = y),
% 		& \text{$(X,Y)$是二维离散型随机变量}, \\
% 		\int_{-\infty}^{+\infty} x p_{X \vert Y}(x \vert y) \dd{x},
% 		& \text{$(X,Y)$是二维连续型随机变量},
% 	\end{array} \right.
% \end{equation}
% 其中\(P(X=x_i \vert Y=y)\)是\(Y=y\)条件下\(X\)的条件概率分布,
% \(p(x \vert y)\)是\(Y=y\)条件下\(X\)的条件密度函数.

与条件期望相对,把数学期望称为\DefineConcept{无条件期望}.

条件期望是条件概率分布的数学期望,故它具有数学期望的一切性质.
\begin{itemize}
	\item 如果\(c\)是常数,则\begin{equation*}
		E(c \vert y) = c;
	\end{equation*}

	%@see: 《概率论与数理统计》(茆诗松、周纪芗、张日权) P167 (3.4.12)
	\item \(E(a_1 X_1 + a_2 X_2 \vert y)
	= a_1 E(X_1 \vert y) + a_2 E(X_2 \vert y)\).
	%@see: 《概率论与数理统计》(茆诗松、周纪芗、张日权) P167 (3.4.13)
	对任一函数\(g(X)\),有\begin{equation*}
		E(g(X) \vert y)
		= \left\{ \begin{array}{cl}
			\sum_i g(x_i) P(X=x_i \vert Y = y),
			& \text{在离散场合}, \\
			\int_{-\infty}^{+\infty} g(x) p(x \vert y) \dd{x},
			& \text{在连续场合}.
		\end{array} \right.
	\end{equation*}
\end{itemize}

\begin{theorem}\label{theorem:条件期望.条件期望与期望的关系}
%@see: 《概率论与数理统计》(茆诗松、周纪芗、张日权) P167 定理3.4.1
条件期望的期望就是无条件期望,
即\begin{equation}
	%@see: 《概率论与数理统计》(茆诗松、周纪芗、张日权) P167 (3.4.14)
	E(E(X \vert Y)) = E(X).
\end{equation}
%TODO proof
\end{theorem}

\begin{example}
%@see: 《概率论与数理统计》(茆诗松、周纪芗、张日权) P168 例3.4.9
一矿工被困在有三个门的矿井里.
第一扇门通向一个坑道,沿此坑道走3小时可使他到达安全地点;
第二扇门可使他走5小时后又回到原处;
第三扇门可使他走7小时后也回到原地.
假设该矿工在任何时刻都等可能地选定其中一扇门,
试问他到达安全地点平均要用多长时间?
\begin{solution}
设\(X\)为该矿工到达安全地点所需时间(单位:小时),
\(Y\)为他所选的门,则\begin{equation*}
	E(X)
	= E(X \vert Y=1) P(Y=1)
	+ E(X \vert Y=2) P(Y=2)
	+ E(X \vert Y=3) P(Y=3),
\end{equation*}
其中\(P(Y=1) = P(Y=2) = P(Y=3) = 1/3\),
\(P(X \vert Y=1) = 3\),
而\(E(X \vert Y=2)\)为矿工从第二扇门出去,要到达安全地点所需平均时间.
而他沿此坑道走5小时又转回原地,而一旦返回原地,问题就与当初他还没有进第二扇门之前一样,
因此他要到达安全地点平均还需用\(E(X)\)小时,
故\begin{equation*}
	E(X \vert Y=2)
	= 5 + E(X).
\end{equation*}
同理可知\begin{equation*}
	E(X \vert Y=3)
	= 7 + E(X).
\end{equation*}
代回原式,可得\begin{equation*}
	E(X) = \frac13[ 3 + (5 + E(X)) + (7 + E(X)) ],
\end{equation*}
解得\(E(X) = 15\)小时,
即该矿工到达安全地点平均需要15小时.
\end{solution}
\end{example}
