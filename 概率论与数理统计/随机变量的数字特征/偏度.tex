\section{偏度}
\begin{definition}
%@see: 《概率论与数理统计》(茆诗松、周纪芗、张日权) P107 定义2.5.3
设随机变量\(X\)的前三阶矩存在.
把\[
	\frac{\mu_3}{\mu_2^{3/2}}
	=\frac{E[X-E(X)]^3}{[D(X)]^{3/2}}
\]称为“\(X\)的\DefineConcept{偏度}”,
记作\(\beta_s\).

当\(\beta_s>0\)时,称这个分布是\DefineConcept{正偏}或\DefineConcept{右偏};
当\(\beta_s<0\)时,称这个分布是\DefineConcept{负偏}或\DefineConcept{左偏}.
当一个分布的偏度不等于零时,我们称这个分布为\DefineConcept{偏态分布}.
\end{definition}

偏度\(\beta_s\)是描述分布偏离对称性程度的一个特征数.

当分布密度函数\(f(x)\)的图形关于它的数学期望\(E(X)\)定出的直线\(x=E(X)\)对称,
即\(f(E(X)-x)=f(E(X)+x)\)时,其三阶中心矩\(\mu_3\)必为零,从而它的偏度为零.
作为特例,正态分布\(N(\mu,\sigma^2)\)关于\(E(X)=\mu\)是对称的,故正态分布的偏度都是零.
