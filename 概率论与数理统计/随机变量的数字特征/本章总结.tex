\section{本章总结}
\begin{table}[!htb]
	\centering
	\begin{tblr}{c|c|c}
		\hline
		& 离散型 & 连续型 \\
		\hline
		\begin{tblr}{c}
			数学期望 \\
			\(E(X)\) \\
		\end{tblr}
		%\cref{equation:随机变量的数字特征.离散型数学期望的定义式}
		& \(\sum_{k=1}^\infty x_k p_k\)
		%\cref{equation:随机变量的数字特征.连续型数学期望的定义式}
		& \(\int_{-\infty}^{+\infty} x f(x) \dd{x}\) \\
		\hline
		\SetCell[r=3]{c}
		\begin{tblr}{c}
			方差 \\
			\(D(X)\) \\
		\end{tblr}
		%\cref{equation:随机变量的数字特征.方差的定义式}
		& \SetCell[c=2]{c} \(E[X-E(X)]^2\) \\ \cline{2-3}
		%\cref{equation:随机变量的数字特征.离散型方差的计算式}
		& \(\sum_{k=1}^\infty [x_k - E(X)]^2 p_k\)
		%\cref{equation:随机变量的数字特征.连续型方差的计算式}
		& \(\int_{-\infty}^{+\infty} [x - E(X)]^2 f(x) \dd{x}\) \\ \cline{2-3}
		%\cref{theorem:随机变量的数字特征.常用的方差的计算式}
		& \SetCell[c=2]{c} \(E(X^2) - E^2(X)\) \\ \hline
		\SetCell[r=2]{c}
		\begin{tblr}{c}
			协方差 \\
			\(\Cov(X,Y)\) \\
		\end{tblr}
		%\cref{equation:随机变量的数字特征.协方差的定义式}
		& \SetCell[c=2]{c} \(E((X-E(X))(Y-E(Y)))\) \\ \cline{2-3}
		%\cref{equation:随机变量的数字特征.协方差的计算式1}
		& \SetCell[c=2]{c} \(E(XY) - E(X) E(Y)\) \\ \hline
		\SetCell[r=2]{c}
		\begin{tblr}{c}
			相关系数 \\
			\(R(X,Y)\) \\
		\end{tblr}
		%\cref{definition:随机变量的数字特征.相关系数}
		& \SetCell[c=2]{c} \(\Cov(X^*,Y^*)\) \\ \cline{2-3}
		%\cref{theorem:随机变量的数字特征.相关系数的性质2}
		& \SetCell[c=2]{c} \(\frac{\Cov(X,Y)}{\sqrt{D(X)} \sqrt{D(Y)}}\)
		\\ \hline
	\end{tblr}
	\caption{}
\end{table}

%\cref{theorem:随机变量的数字特征.数学期望的性质1}
%\cref{theorem:随机变量的数字特征.数学期望的性质2}
设\(\AutoTuple{X}{n}\)都是随机变量,
而\(C_1,C_2,\dotsc,C_n,b\)都是常数,
则有\begin{equation*}
	E\left(\sum_{i=1}^n C_i X_i + b\right)
	= \sum_{i=1}^n C_i E(X_i) + b.
\end{equation*}

%\cref{theorem:随机变量的数字特征.数学期望的性质3}
%\cref{theorem:随机变量的数字特征.数学期望的性质4}
若随机变量\(\AutoTuple{X}{n}\)~{\color{red}相互独立},
则\begin{equation*}
	E\left( \bigcap_{i=1}^n X_i \right)
	= \prod_{i=1}^n E(X_i).
\end{equation*}

%\cref{theorem:随机变量的数字特征.柯西--施瓦茨不等式}
设\(X,Y\)都是随机变量,
则\begin{math}
	E(XY)^2 \leq E(X^2) E(Y^2).
\end{math}

%\cref{theorem:随机变量的数字特征.方差的性质1}
%\cref{theorem:随机变量的数字特征.方差的性质2}
%\cref{theorem:随机变量的数字特征.方差的性质3}
若随机变量\(\AutoTuple{X}{n}\)~{\color{red}相互独立},
且它们的方差都存在,
而\(C_1,C_2,\dotsc,C_n\)都是常数,
则\begin{equation*}
	D\left( \sum_{i=1}^n C_i X_i \right)
	= \sum_{i=1}^n C_i^2 \cdot D(X_i).
\end{equation*}

%\cref{theorem:随机变量的数字特征.协方差的性质3}
设\(X\)、\(Y\)、\(Z\)以及\(X_i\)、\(Y_j\)均为随机变量,
则\begin{itemize}
	%@see: 《概率论与数理统计》(茆诗松、周纪芗、张日权) P149
    \item \(\Cov(X,X) = D(X)\);
	%@see: 《概率论与数理统计》(茆诗松、周纪芗、张日权) P150 定理3.3.5(1)
    \item \(\Cov(X,Y)=\Cov(Y,X)\);
	%@see: 《概率论与数理统计》(茆诗松、周纪芗、张日权) P150 定理3.3.5(4)
    \item \(\Cov(X,a)=0\ (\text{\(a\)是常数})\);
	%@see: 《概率论与数理统计》(茆诗松、周纪芗、张日权) P150 定理3.3.5(2)
    \item \(\Cov(aX,bY)=ab\Cov(X,Y)\ (\text{\(a,b\)是常数})\);
    \item \(\Cov(X+Y,Z)=\Cov(X,Z)+\Cov(Y,Z)\);
    \item \(\Cov\left(\sum_{i=1}^n{a_i X_i},\sum_{j=1}^m{b_j Y_j}\right)
    = \sum_{i=1}^n \sum_{j=1}^m {a_i b_j \Cov(X_i,Y_j)}\ (\text{\(a_i,b_j\)是常数})\);
	%@see: 《概率论与数理统计》(茆诗松、周纪芗、张日权) P150 定理3.3.5(4)
    \item 若\(X\)与\(Y\)相互独立,则\(\Cov(X,Y)=0\).
\end{itemize}

%\cref{equation:随机变量的数字特征.协方差不等式1}
设随机变量\(X,Y\)的方差都存在,
则\begin{equation*}
    \Cov^2(X,Y) \leq D(X) D(Y).
\end{equation*}

%\cref{theorem:随机变量的数字特征.方差与协方差的联系1}
设随机变量\(X\)、\(Y\)的方差\(D(X)\)、\(D(Y)\)都存在,
且它们的协方差\(\Cov(X,Y)\)也存在,
那么\begin{equation*}
    D(X \pm Y) = D(X) + D(Y) \pm 2 \Cov(X,Y).
\end{equation*}

%\cref{theorem:随机变量的数字特征.方差与协方差的联系2}
设随机变量\(\AutoTuple{X}{n}\)的方差\(D(X_i)\ (i=1,2,\dotsc,n)\)、
协方差\(\Cov(X_i,X_j)\ (i,j=1,2,\dotsc,n)\)都存在,
那么\begin{equation*}
    D\left( \sum_{i=1}^n X_i \right)
    = \sum_{i=1}^n D(X_i)
    + 2 \sum_{i<j} \Cov(X_i,X_j).
\end{equation*}
