\section{相关系数}
\subsection{标准化随机变量}
\begin{definition}
设随机变量\(X\)的期望、方差都存在,
且\(D(X) > 0\),
则将随机变量\[
    \frac{X-E(X)}{\sqrt{D(X)}}
\]
称为“随机变量\(X\)的\DefineConcept{标准化随机变量}”.
\end{definition}

\begin{property}\label{theorem:随机变量的数字特征.标准化随机变量的数字特征}
任意随机变量\(X\)的标准化随机变量\(X^*\)具有以下性质:
\begin{enumerate}
    \item \(E(X^*)=0\);
    \item \(D(X^*)=1\).
\end{enumerate}
\begin{proof}
设\(X\)的标准差为\(\sqrt{D(X)}=\sigma\).
那么\begin{align*}
    E(X^*)
	&= E\left(\frac{X-E(X)}{\sigma}\right) \\
	&= \frac1\sigma E(X-E(X))
		\tag{\cref{theorem:随机变量的数字特征.数学期望的性质1}} \\
	&= \frac1\sigma (E(X) - E(X))
	= 0, \\
    D(X^*)
	&= E((X^*)^2)
		\tag{\cref{theorem:随机变量的数字特征.常用的方差的计算式}} \\
	&= E\left(\frac{(X-E(X))^2}{\sigma^2}\right) \\
	&= \frac{1}{\sigma^2} E(X-E(X))^2
	= 1.
    \qedhere
\end{align*}
\end{proof}
\end{property}

\subsection{相关系数}
\begin{definition}\label{definition:随机变量的数字特征.相关系数}
%@see: 《概率论与数理统计》(茆诗松、周纪芗、张日权) P153 定义3.3.2
设随机变量\(X\)、\(Y\)的标准化随机变量分别为\(X^*\)、\(Y^*\).
称“\(X^*\)与\(Y^*\)的协方差\(\Cov(X^*,Y^*)\)”为
“\(X\)与\(Y\)的\DefineConcept{(线性)相关系数}”,
记作\(R(X,Y)\)或\(\operatorname{Corr}(X,Y)\),
即\[
    R(X,Y) \defeq \Cov(X^*,Y^*).
\]
\end{definition}

\begin{definition}
\(X,Y\)都是随机变量.
当满足\(R(X,Y) > 0\)时,
称“\(X\)与\(Y\) \DefineConcept{正(线性)相关}”;
当满足\(R(X,Y) < 0\)时,
称“\(X\)与\(Y\) \DefineConcept{负(线性)相关}”.
当\(R(X,Y) = 0\)时,
称“\(X\)与\(Y\) \DefineConcept{不相关}”,即“\(X\)与\(Y\)没有线性相关关系”.
当满足\(R(X,Y) = 1\)时,
称“\(X\)与\(Y\) \DefineConcept{完全正线性相关}”;
当满足\(R(X,Y) = -1\)时,
称“\(X\)与\(Y\) \DefineConcept{完全负线性相关}”.
\end{definition}

\begin{theorem}\label{theorem:随机变量的数字特征.相关系数的性质1}
\(R(X,Y) = E(X^* Y^*)\).
\begin{proof}
根据\cref{theorem:随机变量的数字特征.协方差的性质1,%
theorem:随机变量的数字特征.标准化随机变量的数字特征} 立即可得.
\end{proof}
\end{theorem}

\begin{theorem}\label{theorem:随机变量的数字特征.相关系数的性质2}
\(R(X,Y) = \frac{\Cov(X,Y)}{\sqrt{D(X)} \sqrt{D(Y)}}\).
\end{theorem}

\begin{property}
\(R(X,Y)=R(Y,X)\).
\end{property}

\begin{property}
%@see: 《概率论与数理统计》(茆诗松、周纪芗、张日权) P155 定理3.3.8
\(\abs{R(X,Y)} \leq 1\).
\begin{proof}
由\cref{equation:随机变量的数字特征.协方差不等式1} 立即可得.
\end{proof}
\end{property}

\begin{theorem}
%@see: 《概率论与数理统计》(茆诗松、周纪芗、张日权) P155 定理3.3.9
\(R(X,Y) = \pm1\)的充分必要条件是:
在\(X\)与\(Y\)之间几乎处处有线性关系.
\begin{proof}
设\(D(X) = \sigma_X,
D(Y) = \sigma_Y\).

充分性.
若\(Y=aX+b\),
则\(\sigma_Y^2=a^2\sigma_X^2\),
从而\[
	\Cov(X,Y)
	= \Cov(X,aX+b)
	= a\Cov(X,X)
	= a\sigma_X^2,
\]
于是\[
	R(X,Y)
	= \frac{\Cov(X,Y)}{\sigma_X \sigma_Y}
	= \frac{a \sigma_X^2}{\abs{a} \sigma_X^2}
	= \left\{ \begin{array}{rl}
		1, & a>0, \\
		-1, & a<0.
	\end{array} \right.
\]

必要性.
由于\begin{align*}
	D\left(\frac{X}{\sigma_X} \pm \frac{Y}{\sigma_Y}\right)
	&= \frac1{\sigma_X^2} D(X) + \frac1{\sigma_Y^2} D(Y)
		\pm \frac2{\sigma_X \sigma_Y} \Cov(X,Y) \\
	&= 2[1 \pm R(X,Y)],
\end{align*}
所以,
当\(R(X,Y)=1\)时,
有\(D\left(\frac{X}{\sigma_X} - \frac{Y}{\sigma_Y}\right) = 0\),
故\(P\left(\frac{X}{\sigma_X} - \frac{Y}{\sigma_Y} = c\right) = 1\);
当\(R(X,Y)=-1\)时,
有\(D\left(\frac{X}{\sigma_X} + \frac{Y}{\sigma_Y}\right) = 0\),
故\(P\left(\frac{X}{\sigma_X} + \frac{Y}{\sigma_Y} = c\right) = 1\).
综上所述,当\(R(X,Y)=\pm1\)时,
在\(X\)与\(Y\)之间几乎处处有线性关系.
\end{proof}
\end{theorem}

\begin{theorem}\label{theorem:相关系数.两个独立随机变量的相关系数等于零}
%@see: 《概率论与数理统计》(茆诗松、周纪芗、张日权) P156 定理3.3.10
若\(X\)与\(Y\)是相互独立的随机变量,
则\(R(X,Y)=0\);
反之不然.
\begin{proof}
若\(X\)与\(Y\)相互独立,
则\(\Cov(X,Y)=0\),
从而\(R(X,Y)=0\).
这就说明,当\(X\)与\(Y\)相互独立时,必有\(X\)与\(Y\)不相关.

%FIXME 后学的内容(正态分布)在前面章节(随机变量的数字特征)引用了,需要调整顺序
但是,当\(X\)与\(Y\)不相关时,却不必然有\(X\)与\(Y\)相互独立.
设\(X \sim N(0,1)\),\(Y=X^2\),
于是有\[
    E(X) = 0,
    \qquad
    E(Y) = E(X^2) = 1,
    \qquad
    E(XY) = E(X^3) = 0,
\]
可见\(\Cov(X,Y) = E(XY) - E(X) E(Y) = 0\),从而\(R(X,Y) = 0\).
但\(X\)与\(Y\)之间存在函数关系\(Y=X^2\),不能说\(X\)与\(Y\)独立.
\end{proof}
\end{theorem}
