\section{相关系数}
\subsection{标准化随机变量}
\begin{definition}
设随机变量\(X\)的期望、方差都存在,
且\(D(X) > 0\),
则将随机变量\[
    \frac{X-E(X)}{\sqrt{D(X)}}
\]
称为“随机变量\(X\)的\DefineConcept{标准化随机变量}”.
\end{definition}

\begin{property}\label{theorem:随机变量的数字特征.标准化随机变量的数字特征}
%@see: 《概率论与数理统计》(茆诗松、周纪芗、张日权) P101 例2.4.9
任意随机变量\(X\)的标准化随机变量\(X^*\)具有以下性质:\begin{itemize}
    \item \(E(X^*)=0\);
    \item \(D(X^*)=1\).
\end{itemize}
\begin{proof}
设\(X\)的标准差为\(\sqrt{D(X)}=\sigma\).
那么\begin{align*}
    E(X^*)
	&= E\left(\frac{X-E(X)}{\sigma}\right) \\
	&= \frac1\sigma E(X-E(X))
		\tag{\cref{theorem:随机变量的数字特征.数学期望的性质1}} \\
	&= \frac1\sigma (E(X) - E(X))
	= 0, \\
    D(X^*)
	&= E((X^*)^2)
		\tag{\cref{theorem:随机变量的数字特征.常用的方差的计算式}} \\
	&= E\left(\frac{(X-E(X))^2}{\sigma^2}\right) \\
	&= \frac{1}{\sigma^2} E(X-E(X))^2
	= 1.
    \qedhere
\end{align*}
\end{proof}
\end{property}

\subsection{相关系数}
\begin{definition}\label{definition:随机变量的数字特征.相关系数}
%@see: 《概率论与数理统计》(茆诗松、周纪芗、张日权) P153 定义3.3.2
设随机变量\(X\)、\(Y\)的标准化随机变量分别为\(X^*\)、\(Y^*\).
称“\(X^*\)与\(Y^*\)的协方差\(\Cov(X^*,Y^*)\)”为
“\(X\)与\(Y\)的\DefineConcept{线性相关系数}”,
记作\(R(X,Y)\)或\(\operatorname{Corr}(X,Y)\),
即\[
    R(X,Y) \defeq \Cov(X^*,Y^*).
\]
\end{definition}

\begin{definition}
\(X,Y\)都是随机变量.
当满足\(R(X,Y) > 0\)时,
称“\(X\)与\(Y\)~\DefineConcept{正线性相关}”;
当满足\(R(X,Y) < 0\)时,
称“\(X\)与\(Y\)~\DefineConcept{负线性相关}”.
当\(R(X,Y) = 0\)时,
称“\(X\)与\(Y\)~\DefineConcept{不相关}”,即“\(X\)与\(Y\)没有线性相关关系”.
当满足\(R(X,Y) = 1\)时,
称“\(X\)与\(Y\)~\DefineConcept{完全正线性相关}”;
当满足\(R(X,Y) = -1\)时,
称“\(X\)与\(Y\)~\DefineConcept{完全负线性相关}”.
\end{definition}

\begin{theorem}\label{theorem:随机变量的数字特征.相关系数的性质1}
\(R(X,Y) = E(X^* Y^*)\).
\begin{proof}
根据\cref{theorem:随机变量的数字特征.协方差的性质1,%
theorem:随机变量的数字特征.标准化随机变量的数字特征} 立即可得.
\end{proof}
\end{theorem}

\begin{theorem}\label{theorem:随机变量的数字特征.相关系数的性质2}
%@see: 《概率论与数理统计》(茆诗松、周纪芗、张日权) P153 (3.3.13)
\(R(X,Y) = \frac{\Cov(X,Y)}{\sqrt{D(X)} \sqrt{D(Y)}}\).
\end{theorem}

\begin{property}
\(R(X,Y)=R(Y,X)\).
\end{property}

\begin{property}
%@see: 《概率论与数理统计》(茆诗松、周纪芗、张日权) P155 定理3.3.8
%@see: 《概率论与数理统计》(茆诗松、周纪芗、张日权) P155 (3.3.14)
\(\abs{R(X,Y)} \leq 1\).
\begin{proof}
由\cref{equation:随机变量的数字特征.协方差不等式1} 立即可得.
\end{proof}
\end{property}

\begin{theorem}
%@see: 《概率论与数理统计》(茆诗松、周纪芗、张日权) P155 定理3.3.9
\(R(X,Y) = \pm1\)的充分必要条件是:
在\(X\)与\(Y\)之间几乎处处有线性关系.
\begin{proof}
设\(D(X) = \sigma_X,
D(Y) = \sigma_Y\).

充分性.
若\(Y=aX+b\),
则\(\sigma_Y^2=a^2\sigma_X^2\),
从而\[
	\Cov(X,Y)
	= \Cov(X,aX+b)
	= a\Cov(X,X)
	= a\sigma_X^2,
\]
于是\[
	R(X,Y)
	= \frac{\Cov(X,Y)}{\sigma_X \sigma_Y}
	= \frac{a \sigma_X^2}{\abs{a} \sigma_X^2}
	= \left\{ \begin{array}{rl}
		1, & a>0, \\
		-1, & a<0.
	\end{array} \right.
\]

必要性.
由于\begin{align*}
	D\left(\frac{X}{\sigma_X} \pm \frac{Y}{\sigma_Y}\right)
	&= \frac1{\sigma_X^2} D(X) + \frac1{\sigma_Y^2} D(Y)
		\pm \frac2{\sigma_X \sigma_Y} \Cov(X,Y) \\
	&= 2[1 \pm R(X,Y)],
\end{align*}
所以,
当\(R(X,Y)=1\)时,
有\(D\left(\frac{X}{\sigma_X} - \frac{Y}{\sigma_Y}\right) = 0\),
故\(P\left(\frac{X}{\sigma_X} - \frac{Y}{\sigma_Y} = c\right) = 1\);
当\(R(X,Y)=-1\)时,
有\(D\left(\frac{X}{\sigma_X} + \frac{Y}{\sigma_Y}\right) = 0\),
故\(P\left(\frac{X}{\sigma_X} + \frac{Y}{\sigma_Y} = c\right) = 1\).
综上所述,当\(R(X,Y)=\pm1\)时,
在\(X\)与\(Y\)之间几乎处处有线性关系.
\end{proof}
\end{theorem}

\begin{theorem}\label{theorem:相关系数.两个独立随机变量的相关系数等于零}
%@see: 《概率论与数理统计》(茆诗松、周纪芗、张日权) P156 定理3.3.10
若\(X\)与\(Y\)是相互独立的随机变量,
则\(R(X,Y)=0\);
反之不然.
\begin{proof}
若\(X\)与\(Y\)相互独立,
则\(\Cov(X,Y)=0\),
从而\(R(X,Y)=0\).
这就说明,当\(X\)与\(Y\)相互独立时,必有\(X\)与\(Y\)不相关.

%FIXME 后学的内容(正态分布)在前面章节(随机变量的数字特征)引用了,需要调整顺序
但是,当\(X\)与\(Y\)不相关时,却不必然有\(X\)与\(Y\)相互独立.
设\(X \sim N(0,1)\),\(Y=X^2\),
于是有\[
    E(X) = 0,
    \qquad
    E(Y) = E(X^2) = 1,
    \qquad
    E(XY) = E(X^3) = 0,
\]
可见\(\Cov(X,Y) = E(XY) - E(X) E(Y) = 0\),从而\(R(X,Y) = 0\).
但\(X\)与\(Y\)之间存在函数关系\(Y=X^2\),不能说\(X\)与\(Y\)独立.
\end{proof}
\end{theorem}


\begin{theorem}\label{theorem:正态分布与自然指数分布族.性质2}
%@see: 《概率论与数理统计》(陈鸿建、赵永红、翁洋) P146
若\((X,Y) \sim N(\mu_1,\mu_2;\sigma_1^2,\sigma_2^2;r)\),
则\[
	R(X,Y) = r.
\]
\begin{proof}
根据\cref{theorem:随机变量的数字特征.相关系数的性质1},
有\begin{align*}
	R(X,Y)
	&= E(X^* Y^*) \\
	&= \int_{-\infty}^{+\infty} \int_{-\infty}^{+\infty}
		\frac{(x-\mu_1)(y-\mu_2)}{\sigma_1 \sigma_2}
		\cdot
		\frac{1}{2\pi \sigma_1 \sigma_2 \sqrt{1-r^2}}
		e^{-u(x,y)}
		\dd{x} \dd{y}.
\end{align*}
{%define \u and \v
\def\u{u}%
\def\v{v}%
\def\intx{\int_{-\infty}^{+\infty}}%
令\[
	\u = \frac{x-\mu_1}{\sigma_1},
	\qquad
	\v = \frac{y-\mu_2}{\sigma_2},
\]从而\[
	R(X,Y)
	= \intx
		\frac{1}{\sqrt{2\pi}}
		\u e^{-\frac{\u^2}{2}}
		\left[
			\intx
			\frac{\v}{\sqrt{2\pi} \sqrt{1-r^2}}
			e^{-\frac{(\v-r\u)^2}{2(1-r^2)}}
			\dd{\v}
		\right]
		\dd{\u}.
\]
由数学期望的定义可知,上式括号中的部分是正态分布\(N(r\u,1-r^2)\)的数学期望,等于\(r\u\),于是得到\[
	R(X,Y)
	= r \intx \frac{1}{\sqrt{2\pi}} \u^2 e^{-\frac{\u^2}{2}} \dd{\u}.
\]
由于上式中的积分是标准正态分布\(N(0,1)\)的方差,等于\(1\),因此\(R(X,Y) = r\).
}%undefine \u and \v
\end{proof}
\end{theorem}
\begin{corollary}\label{theorem:正态分布与自然指数分布族.性质3}
%@see: 《概率论与数理统计》(陈鸿建、赵永红、翁洋) P146 定理5.7
设\((X,Y) \sim N(\mu_1,\mu_2;\sigma_1^2,\sigma_2^2;r)\),
则\(X\)与\(Y\)相互独立的充分必要条件为\(r=0\)(\(X\)与\(Y\)不相关).
\begin{proof}
当\(X\)与\(Y\)相互独立时,
由\cref{theorem:相关系数.两个独立随机变量的相关系数等于零}
可知\(R(X,Y)=0\).

当\(R(X,Y)=0\)时,
由二维正态分布的定义可知,
\begin{align*}
	f(x,y)
	&= \frac1{2\pi\sigma_1\sigma_2} \exp\left\{
		-\frac12 \left[
			\frac{(x-\mu_1)^2}{\sigma_1^2}
			+\frac{(y-\mu_2)^2}{\sigma_2^2}
		\right]
	\right\} \\
	&= \frac1{\sqrt{2\pi}\sigma_1} \exp[-\frac{(x-\mu_1)^2}{2\sigma_1^2}]
		\cdot \frac1{\sqrt{2\pi}\sigma_2} \exp[-\frac{(y-\mu_2)^2}{2\sigma_2^2}] \\
	&= f_X(x) \cdot f_Y(y),
\end{align*}
\(X\)与\(Y\)相互独立.
\end{proof}
\end{corollary}


\begin{example}
%@see: 《2016年全国硕士研究生入学统一考试(数学一)》一选择题/第8题
随机试验\(E\)有三种两两不相容的结果\(A_1,A_2,A_3\),
且三种结果发生的概率均为\(\frac13\),
将试验\(E\)独立重复做2次,
\(X\)表示2次试验中结果\(A_1\)发生的次数,
\(Y\)表示2次试验中结果\(A_2\)发生的次数,
求\(X\)与\(Y\)的相关系数.
\begin{solution}
由题意有\(X,Y\)都服从于\(B(2,\frac13)\).
由于\begin{gather*}
	E(X) = E(Y) = \frac23, \qquad
	D(X) = D(Y) = \frac43, \\
	E(XY) = 0 \cdot P(XY=0) + 1 \cdot P(XY=1)
	= P(X=1,Y=1)
	= C_2^1 \cdot \frac13 \cdot \frac13
	= \frac29,
\end{gather*}
所以\[
	R(X,Y)
	= \frac{E(XY) - E(X) E(Y)}{\sqrt{D(X)} \sqrt{D(Y)}}
	= -\frac12.
\]
\end{solution}
\end{example}

\begin{example}
%@see: 《2021年全国硕士研究生入学统一考试(数学一)》一选择题/第16题
甲、乙两个盒子中各装有2个红球和2个白球,
先从甲盒中任取一球,观察颜色后放入乙盒,再从乙盒中任取一球,
令\(X,Y\)分别表示甲盒和乙盒中取到的红球的个数,
计算\(X\)与\(Y\)的相关系数.
\begin{solution}
由题意有,在取第一次球时,
甲盒中有2个红球和2个白球,
由等可能概型可知取到红球的概率为\(\frac24 = \frac12\),
取到白球的概率也是\(\frac12\).
假如第一次取到的是红球,
那么在把球放入乙盒以后,
乙盒里有3个红球和2个白球,
此时取到红球的概率为\(\frac35\),取到白球的概率为\(\frac25\).
假如第一次取到的是白球,
那么在把球放入乙盒以后,
乙盒里有2个红球和3个白球,
此时取到红球的概率为\(\frac25\),取到白球的概率为\(\frac35\).

根据以上分析可知,
\(X\)的可能取值为\(\{0,1\}\),
且\begin{equation*}
	P(X=0) = \frac12,
	\qquad
	P(X=1) = \frac12;
\end{equation*}
\(Y\)的可能取值为\(\{0,1\}\),
且\begin{gather*}
	P(Y=1) = P(Y=1 \vert X=1) P(X=1) + P(Y=1 \vert X=0) P(X=0)
	= \frac35 \cdot \frac12 + \frac25 \cdot \frac12
	= \frac12, \\
	P(Y=0) = 1 - P(Y=1) = \frac12;
\end{gather*}
\(XY\)的可能取值为\(\{0,1\}\),
且\begin{gather*}
	P(XY=1)
	= P(X=1,Y=1)
	= P(Y=1 \vert X=1) P(X=1)
	= \frac35 \cdot \frac12
	= \frac3{10}, \\
	P(XY=0)
	= 1 - P(XY=1)
	= \frac7{10}.
\end{gather*}
于是\begin{align*}
	E(X) &= 0 \cdot \frac12 + 1 \cdot \frac12 = \frac12, \\
	E(X^2) &= 0^2 \cdot \frac12 + 1^2 \cdot \frac12 = \frac12, \\
	D(X) &= E(X^2) - E^2(X) = \frac12 - \left( \frac12 \right)^2 =  \frac14, \\
	E(XY) &= 0 \cdot \frac7{10} + 1 \cdot \frac3{10} = \frac3{10}, \\
	R(X,Y) &= \frac{E(XY) - E(X) E(Y)}{\sqrt{D(X)} \sqrt{D(Y)}}
	= \frac{\frac3{10} - \frac12 \cdot \frac12}{\frac12 \cdot \frac12}
	= \frac15.
\end{align*}
\end{solution}
\end{example}
\begin{example}
%@see: 《2022年全国硕士研究生入学统一考试(数学一)》一选择题/第10题
设\(X \sim N(0,1)\),
在\(X=x\)条件下\(Y \sim N(x,1)\).
求\(X\)与\(Y\)的相关系数.
\begin{solution}\let\qed\relax
\begin{proof}[解法一]
由题意有\begin{gather*}
	f_X(x) = \frac1{\sqrt{2\pi}} \exp(-\frac{x^2}2), \\
	f_{Y \vert X}(y \vert x)
	= \frac{f(x,y)}{f_X(x)}
	= \frac1{\sqrt{2\pi}} \exp[-\frac{(y-x)^2}2],
\end{gather*}
那么\(X\)与\(Y\)的联合密度函数为\[
	f(x,y) = f_X(x) \cdot f_{Y \vert X}(y \vert x)
	= \frac1{2\pi} \exp[-\frac12(2x^2-2xy+y^2)],
\]
\(Y\)的边缘密度函数为\begin{align*}
	f_Y(y) &= \int_{-\infty}^{+\infty} f(x,y) \dd{x} \\
	&= \frac1{2\pi} \int_{-\infty}^{+\infty} \exp[-\frac12(2x^2-2xy+y^2)] \dd{x} \\
	&= \frac1{2\pi} \exp(-\frac{y^2}4)
	\int_{-\infty}^{+\infty} \exp[-\left( x-\frac{y}2 \right)^2] \dd{x} \\
	&= \frac1{2\sqrt\pi} \exp(-\frac{y^2}4).
\end{align*}
于是\begin{align*}
	E(X) &= \int_{-\infty}^{+\infty} x f_X(x) \dd{x} = 0, \\
	E(Y) &= \int_{-\infty}^{+\infty} y f_Y(y) \dd{y} = 0, \\
	E(XY) &= \int_{-\infty}^{+\infty} x y f(x,y) \dd{x}\dd{y} = 1, \\
	E(X^2) &= \int_{-\infty}^{+\infty} x^2 f_X(x) \dd{x} = 1, \\
	E(Y^2) &= \int_{-\infty}^{+\infty} y^2 f_Y(x) \dd{y} = 2.
\end{align*}
因此\[
	D(X) = 1, \qquad
	D(Y) = 2, \qquad
	\Cov(X,Y) = 1,
\]
而\[
	R(X,Y) = \frac{\Cov(X,Y)}{\sqrt{D(X)} \sqrt{D(Y)}}
	= \frac1{\sqrt2}.
\]
\end{proof}
\begin{proof}[解法二]
由\cref{theorem:随机变量的数字特征.协方差的性质1,theorem:随机变量的数字特征.相关系数的性质2}
可知\[
	R(X,Y)
	= \frac{E(XY)-E(X)~E(Y)}{\sqrt{D(X)~D(Y)}}.
	\eqno(1)
\]
由\cref{theorem:随机变量的数字特征.正态分布的数字特征}
可知\[
	E(X) = 0,
	\qquad
	D(X) = 1.
\]
由\cref{theorem:条件期望.条件期望与期望的关系} 可知\begin{gather*}
	E(Y) = E(E(Y \vert X))
	= E(X)
	= 0, \\
	E(XY) = E(E(XY \vert X))
	= E(X^2)
	= 1.
\end{gather*}
由{双期望定理}可得\[%TODO cref
	D(Y)
	= E(D(Y \vert X)) + D(E(Y \vert X))
	= E(1) + D(X)
	= 2.
\]
代入(1)式得\(R(X,Y) = \frac1{\sqrt2}\).
\end{proof}
\end{solution}
\end{example}
