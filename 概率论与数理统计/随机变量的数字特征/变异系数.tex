\section{变异系数}
\begin{definition}
%@see: 《概率论与数理统计》(陈鸿建、赵永红、翁洋) P119 定义4.4
%@see: 《概率论与数理统计》(茆诗松、周纪芗、张日权) P107 定义2.5.2
设随机变量\(X\)的期望、方差均存在,
且\(E(X) \neq 0\).
把\begin{equation*}
	\frac{\sqrt{D(X)}}{\abs{E(X)}}
\end{equation*}
称为“\(X\)的\DefineConcept{变异系数}(variation coefficient)”,
记作\(C_v\).
%@see: https://mathworld.wolfram.com/VariationCoefficient.html
%@see: https://en.wikipedia.org/wiki/Coefficient_of_variation
\end{definition}
变异系数\(C_v\)是一个无单位的量.
变异系数\(C_v\)衡量了\(X\)取值在\(E(X)\)周围的相对集中程度.

\begin{example}
%@see: 《概率论与数理统计》(陈鸿建、赵永红、翁洋) P120 例4.15
设随机变量\(X \sim \Gamma(\alpha,\beta)\),
求\(X\)的变异系数.
\begin{solution}
因为\(E(X)=\frac\alpha\beta,
D(X)=\frac{\alpha}{\beta^2}\),
所以变异系数为\begin{equation*}
	C_v = \frac{\sqrt{\alpha/\beta^2}}{\alpha/\beta}
	= \frac{\sqrt{\alpha}}{\alpha}.
\end{equation*}
\end{solution}
\end{example}
