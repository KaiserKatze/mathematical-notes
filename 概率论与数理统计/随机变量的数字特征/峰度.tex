\section{峰度}
\begin{definition}
%@see: 《概率论与数理统计》(茆诗松、周纪芗、张日权) P109 定义2.5.4
设随机变量\(X\)的前四阶矩存在.
把\[
	\frac{\mu_4}{\mu_2^2}-3
	=\frac{E[X-E(X)]^4}{[D(X)]^2}-3
\]称为“\(X\)的\DefineConcept{峰度}”,
记作\(\beta_k\).
\end{definition}

峰度是描述分布尖峭程度、尾部粗细的一个特征数.

\begin{example}
正态分布\(N(\mu,\sigma^2)\)的
\(\mu_2=\sigma^2,
\mu_4=3\sigma^4\),
故按定义,它的峰度为\(\beta_k=0\).
可见这里谈论的“峰度”不是指一般密度函数的峰值高低,
因为正态分布\(N(\mu,\sigma^2)\)的峰值是\(\frac1{\sqrt{2\pi}\sigma}\),
它与标准差\(\sigma\)成反比;
\(\sigma\)越小,
正态分布的峰值越高,
可正态分布的峰度与\(\sigma\)无关.
\end{example}
