\section{协方差}
\subsection{协方差}
给定二维随机变量\((X,Y)\),一般\(X\)与\(Y\)之间存在一定的关系.
我们希望有一个数字特征可以用来描述\(X\)与\(Y\)的关系.
我们知道,当\(X\)与\(Y\)独立时,
有\(E(XY) = E(X) E(Y)\)成立,
从而\begin{align*}
    &E((X-E(X))(Y-E(Y))) \\
    &= E(XY - Y E(X) - X E(Y) + E(X) E(Y)) \\
    &= E(XY) - E(Y) E(X) - E(X) E(Y) + E(X) E(Y) \\
    &= E(XY) - E(X) E(Y) = 0.
\end{align*}
也就是说,当\(E((X-E(X))(Y-E(Y))) \neq 0\)时,\(X\)与\(Y\)就不独立.
这就说明,\(E((X-E(X))(Y-E(Y)))\)能在一定程度上刻画\(X\)与\(Y\)的关系.

\begin{definition}
%@see: 《概率论与数理统计》(茆诗松、周纪芗、张日权) P149 定义3.3.1
设二维随机变量\((X,Y)\)的两个方差都存在,
把\(E((X-E(X))(Y-E(Y)))\)
称为“\(X\)与\(Y\)的\DefineConcept{协方差}(covariance)”,
%@see: https://mathworld.wolfram.com/Covariance.html
记为\(\Cov(X,Y)\),
即\begin{equation}\label{equation:随机变量的数字特征.协方差的定义式}
%@see: 《概率论与数理统计》(茆诗松、周纪芗、张日权) P149 (3.3.8)
    \Cov(X,Y) \defeq E((X-E(X))(Y-E(Y))).
\end{equation}
\end{definition}

根据本节开篇的推导,不难得到如下的性质:
\begin{property}\label{theorem:随机变量的数字特征.协方差的性质1}
%@see: 《概率论与数理统计》(茆诗松、周纪芗、张日权) P150 定理3.3.5(3)
对于二维随机变量\((X,Y)\),
总有\begin{equation}\label{equation:随机变量的数字特征.协方差的计算式1}
%@see: 《概率论与数理统计》(茆诗松、周纪芗、张日权) P150 (3.3.9)
    \Cov(X,Y) = E(XY) - E(X)~E(Y).
\end{equation}
\end{property}

\begin{property}\label{theorem:随机变量的数字特征.协方差的性质3}
设\(X\)、\(Y\)、\(Z\)以及\(X_i\)、\(Y_j\)均为随机变量,
则\begin{itemize}
	%@see: 《概率论与数理统计》(茆诗松、周纪芗、张日权) P149
    \item \(\Cov(X,X) = D(X)\);
	%@see: 《概率论与数理统计》(茆诗松、周纪芗、张日权) P150 定理3.3.5(1)
    \item \(\Cov(X,Y)=\Cov(Y,X)\);
	%@see: 《概率论与数理统计》(茆诗松、周纪芗、张日权) P150 定理3.3.5(4)
    \item \(\Cov(X,a)=0\ (\text{\(a\)是常数})\);
	%@see: 《概率论与数理统计》(茆诗松、周纪芗、张日权) P150 定理3.3.5(2)
    \item \(\Cov(aX,bY)=ab\Cov(X,Y)\ (\text{\(a,b\)是常数})\);
    \item \(\Cov(X+Y,Z)=\Cov(X,Z)+\Cov(Y,Z)\);
    \item \(\Cov\left(\sum_{i=1}^n{a_i X_i},\sum_{j=1}^m{b_j Y_j}\right)
    = \sum_{i=1}^n \sum_{j=1}^m {a_i b_j \Cov(X_i,Y_j)}\ (\text{\(a_i,b_j\)是常数})\);
	%@see: 《概率论与数理统计》(茆诗松、周纪芗、张日权) P150 定理3.3.5(4)
    \item 若\(X\)与\(Y\)相互独立,则\(\Cov(X,Y)=0\).
\end{itemize}
\end{property}

\begin{theorem}
%@see: 《概率论与数理统计》(茆诗松、周纪芗、张日权) P151 定理3.3.6
设随机变量\(X,Y\)的方差都存在,
则\begin{equation}\label{equation:随机变量的数字特征.协方差不等式1}
%@see: 《概率论与数理统计》(茆诗松、周纪芗、张日权) P151 (3.3.10)
    \Cov^2(X,Y) \leq D(X) D(Y).
\end{equation}
\begin{proof}
记\(D(X) = \sigma_1^2,
D(Y) = \sigma_2^2\).

如果\(\sigma_1^2=0\),
那么由\cref{theorem:大数律.方差为零的随机变量的性质} 可知,\(X\)几乎处处是常数;
而由\cref{theorem:随机变量的数字特征.协方差的性质3} 可知,常数与\(Y\)的协方差必为零;
于是\cref{equation:随机变量的数字特征.协方差不等式1} 两端都是零,
因此\cref{equation:随机变量的数字特征.协方差不等式1} 成立.

如果\(\sigma_1^2>0\),
因为\begin{equation*}
	E\{t[X-E(X)]+[Y-E(Y)]\}^2
	= t^2 \sigma_1^2
	+ 2t \Cov(X,Y)
	+ \sigma_2^2
	\geq 0,
\end{equation*}
所以这个关于\(t\)的二次三项式的判别式非正,即\begin{equation*}
	[2\Cov(X,Y)]^2 - 4 \sigma_1^2 \sigma_2^2 \leq 0,
\end{equation*}
移项便得\cref{equation:随机变量的数字特征.协方差不等式1}.
\end{proof}
\end{theorem}

现在我们对\cref{theorem:随机变量的数字特征.方差的性质2} 做一番推广.
\begin{theorem}\label{theorem:随机变量的数字特征.方差与协方差的联系1}
%@see: 《概率论与数理统计》(茆诗松、周纪芗、张日权) P151 定理3.3.7
设随机变量\(X\)、\(Y\)的方差\(D(X)\)、\(D(Y)\)都存在,
且它们的协方差\(\Cov(X,Y)\)也存在,
那么\begin{equation}
%@see: 《概率论与数理统计》(茆诗松、周纪芗、张日权) P151 (3.3.11)
    D(X \pm Y) = D(X) + D(Y) \pm 2 \Cov(X,Y).
\end{equation}
\begin{proof}
由方差的定义可知
\begin{align*}
    D(X \pm Y)
    &= E[(X \pm Y) - E(X \pm Y)]^2 \\
    &= E\bigl\{[X - E(X)] \pm [Y - E(Y)]\bigr\}^2 \\
    &= E\bigl\{
    [X - E(X)]^2 + [Y - E(Y)]^2 - 2 [X - E(X)][Y - E(Y)]
    \bigr\} \\
    &= E[X - E(X)]^2 + E[Y - E(Y)]^2 - 2 E[X - E(X)][Y - E(Y)] \\
    &= D(X) + D(Y) \pm 2 \Cov(X,Y).
    \qedhere
\end{align*}
\end{proof}
\end{theorem}

我们可以进一步将上述定理推广到任意有限个随机变量相加的情形:
\begin{corollary}\label{theorem:随机变量的数字特征.方差与协方差的联系2}
设随机变量\(\AutoTuple{X}{n}\)的方差\(D(X_i)\ (i=1,2,\dotsc,n)\)、
协方差\(\Cov(X_i,X_j)\ (i,j=1,2,\dotsc,n)\)都存在,
那么\begin{equation}
%@see: 《概率论与数理统计》(茆诗松、周纪芗、张日权) P151 (3.3.12)
    D\left( \sum_{i=1}^n X_i \right)
    = \sum_{i=1}^n D(X_i)
    + 2 \sum_{i<j} \Cov(X_i,X_j).
\end{equation}
\end{corollary}

\begin{example}
设二维随机向量\((X,Y)\)的联合密度函数为\begin{equation*}
    p(x,y) = \left\{ \begin{array}{cl}
    3x, & 0<y<x<1, \\
    0, & \text{其他}.
    \end{array} \right.
\end{equation*}
求协方差\(\Cov(X,Y)\).
\begin{solution}
根据\cref{equation:随机变量的数字特征.协方差的计算式1} 可以直接计算得:
\begin{align*}
    E(XY)
    &= \int_0^1 \dd{x} \int_0^x xy \cdot 3x \dd{y}
    = \frac{3}{10}, \\
    E(X)
    &= \int_0^1 \dd{x} \int_0^x x \cdot 3x \dd{y}
    = \frac{3}{4}, \\
    E(Y)
    &= \int_0^1 \dd{x} \int_0^x y \cdot 3x \dd{y}
    = \frac{3}{8}, \\
    \Cov(X,Y)
    &= E(XY) - E(X) E(Y)
    = \frac{3}{160}.
\end{align*}
\end{solution}
\end{example}

\begin{example}[配对问题]
%@see: 《概率论与数理统计》(茆诗松、周纪芗、张日权) P152 例3.3.6
有\(n\)个人,每人将自己的礼品扔入同一箱中,把礼品充分混合后,每人再随机从中选取一个.
试求选中自己礼品的人数\(X\)的数学期望与方差.
\begin{solution}
设随机变量\begin{equation*}
    X_i = \left\{ \begin{array}{ll}
    1, & \text{当第\(i\)个人取出自己的礼品}, \\
    0, & \text{当第\(i\)个人取出别人的礼品},
    \end{array} \right.
    \quad(i=1,2,\dotsc,n)
\end{equation*}同分布于\(\begin{bmatrix}
	0 & 1 \\
	\frac{1}{n} & 1-\frac{1}{n}
\end{bmatrix}\),
即\begin{equation*}
    P(X_i=1) = \frac{1}{n}, \qquad
    P(X_i=0) = 1-\frac{1}{n}
    \quad(i=1,2,\dotsc,n).
\end{equation*}
那么其数学期望与方差分别为\begin{equation*}
    E(X_i) = \frac{1}{n},
    \qquad
    D(X_i) = \frac{1}{n}\left(1-\frac{1}{n}\right)
    \quad(i=1,2,\dotsc,n).
\end{equation*}

在上述假设下,\(n\)个人中选中自己礼品的人数恰好为\begin{equation*}
    X = X_1+X_2+\dotsb+X_n.
\end{equation*}因此\begin{equation*}
    E(X) = E(X_1)+E(X_2)+\dotsb+E(X_n)
    = n \cdot \frac{1}{n}
    = 1.
\end{equation*}

由于\(X_i\ (i=1,2,\dotsc,n)\)不是相互独立的,所以\(X\)的方差为\begin{equation*}
    D(X) = \sum_{i=1}^n D(X_i) + 2 \sum_{i<j} \Cov(X_i,X_j).
\end{equation*}
为了计算\(\Cov(X_i,X_j)\ (i \neq j)\),我们来考察\(X_i X_j\)的含义:\begin{equation*}
    X_i X_j = \left\{ \begin{array}{ll}
    1, & \text{当第\(i\)个人和第\(j\)个人都恰好取出各自的礼品}, \\
    0, & \text{其他},
    \end{array} \right.
\end{equation*}于是\begin{equation*}
    E(X_i X_j) = 1 \cdot P(X_i=1,X_j=1)
    = P(X_i=1) \cdot P(X_j=1 \vert X_i=1)
    = \frac{1}{n} \cdot \frac{1}{n-1},
\end{equation*}
因此\begin{equation*}
    \Cov(X_i,X_j) = \frac{1}{n(n-1)} - \left(\frac{1}{n}\right)^2
    = \frac{1}{n^2(n-1)},
\end{equation*}\begin{equation*}
    D(X) = n \cdot \frac{n-1}{n^2} + 2 \cdot \frac{n(n-1)}{2} \cdot \frac{1}{n^2(n-1)}
    = 1.
\end{equation*}
\end{solution}
由此可见,在配对问题,成对个数的均值与方差都是1,与人数\(n\)无关.
\end{example}

\begin{example}
%@see: 《2020年全国硕士研究生入学统一考试(数学一)》二填空题/第14题
设\(X\)服从区间\(\left( -\frac\pi2,\frac\pi2 \right)\)上的均匀分布,
\(Y = \sin X\).
求协方差\(\Cov(X,Y)\).
\begin{solution}
记\begin{equation*}
	f(x) = \left\{ \def\arraystretch{1.5} \begin{array}{cl}
		\frac1\pi, & -\frac\pi2 < x < \frac\pi2, \\
		0, & \text{其他}.
	\end{array} \right.
\end{equation*}
因为\begin{gather*}
	E(X)
	= \int_{-\infty}^{+\infty} x \cdot f(x) \dd{x}
	= \int_{-\frac\pi2}^{\frac\pi2} x \cdot \frac1\pi \dd{x}
	= 0, \\
	E(XY)
	= \int_{-\infty}^{+\infty} x \sin x \cdot f(x) \dd{x}
	= \int_{-\frac\pi2}^{\frac\pi2} x \sin x \cdot \frac1\pi \dd{x}
	= \frac2\pi,
\end{gather*}
所以\begin{gather*}
	\Cov(X,Y)
	= E(XY) - E(X) E(Y)
	= \frac2\pi - 0
	= \frac2\pi.
\end{gather*}
\end{solution}
\end{example}
\begin{example}
%@see: 《2024年全国硕士研究生入学统一考试(数学一)》一选择题/第9题
设随机变量\(X\)的概率密度为\(f(x) = \left\{ \begin{array}{cl}
	2(1-x), & 0<x<1, \\
	0, & \text{其他}.
\end{array} \right.\)
在\(X=x\ (0<x<1)\)的条件下,
随机变量\(Y\)服从区间\((x,1)\)上的均匀分布.
求\(\Cov(X,Y)\).
\begin{solution}
由题意有\(f_{Y \vert X}(y \vert x) = \frac1{1-x}\ (x<y<1)\).
记\(D = \Set{ (x,y) \given x<y<1,0<x<1 }\).
那么当\((x,y) \in D\)时,
\((X,Y)\)的联合密度函数为\begin{equation*}
	\phi(x,y) = f_{Y \vert X}(y \vert x) \cdot f(x)
	= \frac1{1-x} \cdot 2(1-x)
	= 2,
\end{equation*}
从而\begin{equation*}
	\iint_D \phi(x,y) \dd{x}\dd{y}
	= 2 \iint_D \dd{x}\dd{y}
	= 1,
\end{equation*}
那么必定成立:在\(D\)以外的平面区域上\(\phi(x,y)\)恒等于零.
另外,\(Y\)的边缘密度函数为\begin{equation*}
	g(y) = \int_{-\infty}^{+\infty} \phi(x,y) \dd{x}
	= \int_0^y 2 \dd{x}
	= 2y
	\quad(0<y<1).
\end{equation*}
因此\begin{gather*}
	E(X) = \int_{-\infty}^{+\infty} x f(x) \dd{x}
	= \frac13, \\
	E(Y) = \int_{-\infty}^{+\infty} y g(y) \dd{y}
	= \frac23, \\
	E(XY) = \int_{-\infty}^{+\infty} x y \phi(x,y) \dd{x}\dd{y}
	= \frac14, \\
	Cov(X,Y) = E(XY) - E(X) E(Y)
	= \frac1{36}.
\end{gather*}
\end{solution}
% 相似题目:《2022年全国硕士研究生入学统一考试(数学一)》一选择题/第10题
\end{example}

\subsection{均值向量\ 协方差阵}
\begin{definition}
对于二维随机变量\((X,Y)\),称向量\begin{equation*}
    (E(X),E(Y))
\end{equation*}
为“\((X,Y)\)的\DefineConcept{数学期望}或\DefineConcept{均值向量}”.
称矩阵\begin{equation*}
    \vb{V} = \begin{bmatrix}
    D(X) & \Cov(X,Y) \\
    \Cov(Y,X) & D(Y)
    \end{bmatrix}
\end{equation*}为“\((X,Y)\)的\DefineConcept{协方差阵}”.

一般地,对\(n\)维随机变量\((\AutoTuple{X}{n})\),称向量\begin{equation*}
    (E(X_1),E(X_2),\dotsc,E(X_n))
\end{equation*}
为“\((\AutoTuple{X}{n})\)的\DefineConcept{数学期望}或\DefineConcept{均值向量}”.
记\begin{equation*}
    \sigma_{ij} = \Cov(X_i,Y_j),
    \quad i,j=1,2,\dotsc,n;
\end{equation*}
称矩阵\begin{equation*}
    \vb{V} = \begin{bmatrix}
    \sigma_{11} & \sigma_{12} & \dots & \sigma_{1n} \\
    \sigma_{21} & \sigma_{22} & \dots & \sigma_{2n} \\
    \vdots & \vdots & & \vdots \\
    \sigma_{n1} & \sigma_{n2} & \dots & \sigma_{nn}
    \end{bmatrix}
\end{equation*}
为“\((\AutoTuple{X}{n})\)的\DefineConcept{协方差阵}”.
\end{definition}

易见协方差阵\(\vb{V}\)总是对称阵.
