\section{矩}
\begin{definition}
若对非负整数\(k\)以及某点\(a\),随机变量\((X-a)^k\)存在期望,
则把\(E\abs{X-a}^k\)
称为“随机变量\(X\)的关于点\(a\)的\(k\)阶\DefineConcept{绝对矩}(absolute moment)”.
%@see: https://mathworld.wolfram.com/AbsoluteMoment.html
\end{definition}

\begin{definition}
若对非负整数\(k\),随机变量\(X^k\)存在期望,
则把\(m_k \defeq E(X^k)\)
称为“随机变量\(X\)的\(k\)阶\DefineConcept{原点矩}(raw moment)”.
%@see: https://mathworld.wolfram.com/RawMoment.html
\end{definition}

\begin{definition}
若对非负整数\(k\),随机变量\([X-E(X)]^k\)存在期望,
则把\(\mu_k \defeq E[X-E(X)]^k\)
称为“随机变量\(X\)的\(k\)阶\DefineConcept{中心矩}(absolute moment)”.
%@see: https://mathworld.wolfram.com/CentralMoment.html
\end{definition}

显然有\(E(X) = m_1\),\(D(X) = \mu_2\).
而\(m_0 = \mu_0 = 1\).

原点矩和中心矩可以相互表示:
\begin{align*}
	\mu_k &= E[X-E(X)]^k
	= E\left[ \sum_{r=0}^k{C_k^r X^r (-m_1)^{k-r}} \right] \\
	&= \sum_{r=0}^k{C_k^r E(X^r) (-m_1)^{k-r}}
	= \sum_{r=0}^k{C_k^r m_r (-m_1)^{k-r}}.
\end{align*}
