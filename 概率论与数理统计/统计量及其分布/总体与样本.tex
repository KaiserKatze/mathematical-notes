\section{总体与样本}
\begin{definition}
在数理统计中,我们把研究对象的全体称为\DefineConcept{总体}(population),
把总体中的每个成员称为\DefineConcept{个体}(individual).
\end{definition}

如果总体中个体数是有限的,那么称这个总体为\DefineConcept{有限总体};
反之,称这个总体为\DefineConcept{无限总体}.

在数理统计中,我们研究的不是总体的全部属性,而是总体的某项数量指标.
这项数量指标可以用一个随机变量或其对应的一个分布来刻画.
因此约定以后提到总体时,将其表记为“总体\(X\)”或“总体\(F(x)\)”.

在抽样试验前,\(n\)个个体的特征\(\{X_n\}\)是与总体同分布的随机变量;
而在抽样试验后,是\(n\)个数据\(\AutoTuple{x}{n}\).
而一般在总体中抽取的这一小部分个体要对总体有充分的代表性,需要满足以下两个条件:
\begin{enumerate}
	\item {\bf 随机性},
	即总体中每个个体都有同等机会被抽取到,
	通常可用编号抽签的方法或用随机数表来实现;
	\item {\bf 独立性},
	各次抽取必须是相互独立的,即各个个体是否被抽取到彼此独立的.
\end{enumerate}

\begin{definition}
设随机变量\(\AutoTuple{X}{n}\)与总体\(X\)独立同分布,
那么我们把\(\AutoTuple{X}{n}\)称为%
“一个来自总体\(X\)的\DefineConcept{简单随机样本}(simple random sample)”,
简称\DefineConcept{样本}(sample);
还称“这个样本的\DefineConcept{样本容量}(sample size)是\(n\)”.
而\(\AutoTuple{X}{n}\)的取值\(\AutoTuple{x}{n}\)叫做\DefineConcept{样本观测值}.
\end{definition}

从总体中进行有放回地抽样,得到的显然是简单随机样本.
从有限总体中进行不放回地抽样,虽然不是简单随机样本,但当总体个体数\(N\)很大而样本容量\(n\)较小(通常要求\(n/N \leq 0.1\)),则可以近似地看作是有放回抽样,因而近似看作是简单随机抽样,即近似地得到简单随机样本.

若将样本看作一个\(n\)维随机变量\((\AutoTuple{X}{n})\),则可求出其概率分布或密度.

当总体\(X\)是离散型随机变量,
且有分布律\(P(X = x) = p(x)\),
则样本\((\AutoTuple{X}{n})\)的分布律为\[
	P(X_1=x_1,X_2=x_2,\dotsc,X_n=x_n)
	= \prod_{i=1}^n P(X_i=x_i).
\]

当总体\(X\)是连续型随机变量,
且有密度函数\(f(x)\),
则样本\((\AutoTuple{X}{n})\)的密度函数为\[
	f(\AutoTuple{x}{n}) = \prod_{i=1}^n f(x_i).
\]
