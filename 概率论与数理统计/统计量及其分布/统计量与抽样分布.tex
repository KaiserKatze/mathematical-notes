\section{统计量与抽样分布}
\subsection{统计量}
样本来自总体,样本的观测值就含有总体各方面的信息,但这些信息较为分散.
为了使这些分散在样本中有关总体的信息集中起来,反映总体的各种特征,对总体的有关问题进行推断,
我们首先需要对样本进行加工,一种有效的方法是构造样本的函数,用不同的样本函数反映总体的不同特征.

\begin{definition}
%@see: 《概率论与数理统计》(茆诗松、周纪芗、张日权) P199 定义4.2.1
\def\g#1{g(\AutoTuple{#1}{n})}
样本\(\AutoTuple{X}{n}\)的一个连续函数\(\g{X}\)称为\DefineConcept{样本函数}.
若\(\g{X}\)不含任何未知参数,则称\(\g{X}\)为一个\DefineConcept{统计量}.
而代入样本观测值后\(\g{x}\)叫做\DefineConcept{统计量的观测值}.
统计量的分布称为\DefineConcept{抽样分布}.
\end{definition}

例如,对于总体\(X \sim N(\mu,\sigma^2)\),\(\sigma^2\)已知,\(\mu\)未知,
\(\AutoTuple{X}{n}\)是来自总体\(X\)的样本,
那么\begin{equation*}
	\sum_{i=1}^n X_i,
	\qquad
	\sum_{i=1}^n \frac{X_i^2}{\sigma^2}
\end{equation*}是统计量;
而\begin{equation*}
	\sum_{i=1}^n \frac{(X_i-\mu)^2}{\sigma^2}
\end{equation*}是含有未知参数的样本函数,不是统计量.

\begin{table}[htb]
%@see: 《概率论与数理统计》(茆诗松、周纪芗、张日权) P199 定义4.2.2
%@see: 《概率论与数理统计》(茆诗松、周纪芗、张日权) P202 定义4.2.3
%@see: 《概率论与数理统计》(茆诗松、周纪芗、张日权) P207 定义4.2.4
	\centering
	\begin{tabular}{*3c}
		\hline
		名称 & 统计量 & 观测值 \\ \hline
		样本均值 & \(\overline{X} = \frac1n \sum_{i=1}^n X_i\)
			& \(\overline{x} = \frac1n \sum_{i=1}^n x_i\) \\[.7cm]
		样本方差 & \(S^2 = \frac{1}{n-1} \sum_{i=1}^n (X_i-\overline{X})^2\)
			& \(s^2 = \frac{1}{n-1} \sum_{i=1}^n (x_i-\overline{x})^2\) \\[.5cm]
		样本标准差 & \(S=\sqrt{S^2}\)
			& \(s=\sqrt{s^2}\) \\[.2cm]
		样本\(k\)阶原点矩 & \(A_k=\frac1n \sum_{i=1}^n X_i^k\)
			& \(a_k=\frac1n \sum_{i=1}^n x_i^k\) \\[.5cm]
		样本\(k\)阶中心矩 & \(B_k=\frac1n \sum_{i=1}^n (X_i-\overline{X})^k\)
			& \(b_k=\frac1n \sum_{i=1}^n (x_i-\overline{x})^k\) \\[.5cm]
		\hline
	\end{tabular}
	\caption{常用的统计量}
\end{table}

与总体矩一样,样本\(k\)阶中心矩也可由各阶样本原点矩表示.
例如,因为\begin{align*}
	\sum_{i=1}^n (X_i-\overline{X})^2
	&= \sum_{i=1}^n (X_i^2 - 2 X_i \overline{X} + \overline{X}^2) \\
	&= \sum_{i=1}^n X_i^2
		- 2 \overline{X} \sum_{i=1}^n X_i
		+ n \overline{X}^2 \\
	&= \sum_{i=1}^n X_i^2
		- 2 \overline{X} \cdot n \overline{X}
		+ n \overline{X}^2 \\
	&= \sum_{i=1}^n X_i^2
			- n \overline{X}^2,
\end{align*}
又\begin{equation*}
	B_2 = \frac1n \sum_{i=1}^n (X_i-\overline{X})^2,
	\qquad
	A_2 = \frac1n \sum_{i=1}^n X_i^2,
\end{equation*}
所以我们有\begin{equation}\label{equation:统计量.2阶中心矩-2阶原点矩-均值的关系}
	B_2
	= A_2 - \overline{X}^2.
\end{equation}

由上可知,样本均值\(\overline{X}\)是一阶样本原点矩\(A_1\),
即\begin{equation}\label{equation:统计量.均值-1阶原点矩的关系}
	\overline{X} = A_1;
\end{equation}
但是,样本方差\(S^2\)不是二阶样本中心矩\(B_2\),
只能说\begin{equation}\label{equation:统计量.方差-2阶中心矩的关系}
	(n-1) S^2 = n B_2.
\end{equation}

另外,我们还应该注意到,统计量是随机变量,其观测值是一个实数.

此外,还有一些不常用到的统计量,也罗列于此.
\begin{definition}
%@see: 《概率论与数理统计》(茆诗松、周纪芗、张日权) P207 定义4.2.5
设\(\AutoTuple{X}{n}\)是来自某总体的一个样本,
把\begin{equation}
	SK \defeq \frac{B_3}{(B_2)^{3/2}}
\end{equation}
称为\DefineConcept{样本偏度}(skewness).
\end{definition}
样本偏度反映了总体分布密度曲线的对称性信息.
当\(SK > 0\)时,分布的形状是右尾长,称其为“正偏的”;
当\(SK < 0\)时,分布的形状是左尾长,称其为“负偏的”.

\begin{definition}
%@see: 《概率论与数理统计》(茆诗松、周纪芗、张日权) P208 定义4.2.6
设\(\AutoTuple{X}{n}\)是来自某总体的一个样本,
把\begin{equation}
	KU \defeq \frac{B_4}{(B_2)^2} - 3
\end{equation}
称为\DefineConcept{样本峰度}(kurtosis).
\end{definition}
样本峰度反映了总体分布密度曲线在其峰值附近的陡峭程度的信息.
当\(KU > 0\)时,分布密度曲线在其峰附近比正态分布来得更陡峭;
当\(KU < 0\)时,比正态分布来得更平坦.

\subsection{抽样分布定理}
从理论上说,当知道总体分布时,
统计量与样本函数的分布都可以确定,
但事实上一般确定统计量与样本函数的分布却十分困难.
而当总体服从正态分布时,
一些常用统计量与样本函数的分布则是容易确定的.
我们把常用的统计量与样本函数的分布的结果叫做\DefineConcept{抽样分布定理}.

\begin{example}
样本\(\AutoTuple{X}{n}\)来自总体\(X\),
其中\(X\)服从指数分布\(e(\lambda)\),
求样本均值\(\overline{X}\)的分布.
\begin{solution}
记\(Y = \frac1n X\),则\(Y\)的值域为\(R_Y = (0,+\infty)\).
对于\(\forall y>0\),
\(Y\)有分布函数\begin{equation*}
	F_Y(y) = P(Y \leq y)
	= P\left(\frac1n X \leq y\right)
	= P(X \leq ny)
	= \int_0^{ny} \lambda e^{-\lambda x} \dd{x}
	= 1 - e^{-n\lambda y},
\end{equation*}
密度函数\begin{equation*}
	f_Y(y) = F'_Y(y) = \left\{ \begin{array}{lc}
		n\lambda e^{-n\lambda y}, & y>0, \\
		0, & y \leq 0,
	\end{array} \right.
\end{equation*}
即\(Y=\frac1nX \sim e(n\lambda)\),
也即\(Y \sim \Gamma(1,n\lambda)\).

注意到\begin{equation*}
	\overline{X} = \frac1n X_1 + \frac1n X_2 + \dotsb + \frac1n X_n,
\end{equation*}
而\(\frac1n X_i\ (i=1,2,\dotsc,n)\)独立同服从于\(\Gamma(1,n\lambda)\)分布,
那么根据~\hyperref[theorem:多维随机变量及其分布.伽马分布的可加性1]{\(\Gamma\)分布的可加性},
可知\begin{equation*}
	\overline{X} \sim \Gamma(n,n\lambda).
\end{equation*}
\end{solution}
\end{example}

\subsubsection{一个正态总体下的抽样分布定理}
\begin{theorem}
%@see: 《概率论与数理统计》(陈鸿建、赵永红、翁洋) P179 定理7.3
%@see: 《应用随机过程:概率模型导论(第11版)》(Sheldon M. Ross,龚光鲁译) P56 命题2.5
样本\(\AutoTuple{X}{n}\)来自正态总体\(N(\mu,\sigma^2)\),则\begin{gather}
	\label{equation:抽样分布定理.一个正态总体的抽样分布1}
	\overline{X} \sim N\left(\mu,\frac{\sigma^2}{n}\right), \\
	\label{equation:抽样分布定理.一个正态总体的抽样分布2}
	\frac{\overline{X}-\mu}{\sigma / \sqrt{n}} \sim N(0,1).
\end{gather}
%TODO proof
% \begin{proof}
% 因为\begin{equation*}
% 	E(\overline{X})
% 	= E\left(\frac1n \sum_{i=1}^n X_i\right)
% 	= \frac1n \sum_{i=1}^n E(X_i)
% 	= \mu,
% \end{equation*}\begin{equation*}
% 	D(\overline{X})
% 	= D\left(\frac1n \sum_{i=1}^n X_i\right)
% 	= \frac{1}{n^2} \sum_{i=1}^n D(X_i)
% 	= \frac{\sigma^2}{n}.
% \end{equation*}

% 又由\hyperref[theorem:正态分布与自然指数分布族.正态分布的可加性2]{正态分布可加性}可得\begin{equation*}
% 	\overline{X} = \frac1n \sum_{i=1}^n X_i
% 	\sim N\left(\mu,\frac{\sigma^2}{n}\right).
% 	\qedhere
% \end{equation*}
% \end{proof}
\end{theorem}

% \begin{corollary}
% %@see: 《概率论与数理统计》(陈鸿建、赵永红、翁洋) P180 推论
% 样本\(\AutoTuple{X}{n}\)来自任何总体,
% 都有\begin{gather}
% 	E(\overline{X}) = E(X), \\
% 	D(\overline{X}) = \frac{D(X)}{n}.
% \end{gather}
% \end{corollary}

\begin{example}
样本\(\AutoTuple{X}{n}\)来自任何总体.
试证:\begin{equation}
	E(S^2) = \sigma^2.
\end{equation}
\begin{proof}
显然\begin{align*}
	E(S^2)
	&= E\left[\frac{1}{n-1} \sum_{i=1}^n (X_i-\overline{X})^2\right]
	= \frac{1}{n-1} \sum_{i=1}^n E(X_i-\overline{X})^2 \\
	&= \frac{1}{n-1} \sum_{i=1}^n \left[ E(X_i^2) + E(\overline{X}^2) - 2 E(\overline{X} X_i) \right],
\end{align*}
其中\begin{gather*}
	E(X_i^2) = E(X^2) = D(X) + E^2(X) = \sigma^2 + \mu^2, \\
	E(\overline{X}^2)
	= D(\overline{X}) + [E(\overline{X})]^2
	= \frac{\sigma^2}{n} + \mu^2, \\
	E(\overline{X} X_i)
	= E\left(\frac1n \sum_{j=1}^n X_j X_i\right)
	= \frac1n \left[ E(X_i^2) + E\left(\sum_{\substack{1 \leq j \leq n \\ j \neq i}} X_i X_j\right) \right], \\
	E\left(\sum_{\substack{1 \leq j \leq n \\ j \neq i}} X_i X_j\right)
	= \sum_{\substack{1 \leq j \leq n \\ j \neq i}} E(X_i) E(X_j)
	= (n-1) \mu^2,
\end{gather*}
因此\begin{align*}
	E(S^2) &= \frac{1}{n-1} \sum_{i=1}^n \left\{
			\sigma^2 + \mu^2
			+ \frac1n \sigma^2 + \mu^2
			- 2 \frac1n \left[ \sigma^2 + \mu^2 + (n-1)\mu^2 \right]
		\right\} \\
	&= \frac{1}{n-1} n \cdot \frac{n-1}{n} \sigma^2
	= \sigma^2.
	\qedhere
\end{align*}
\end{proof}
\end{example}
上例也就说明了为什么样本方差的定义式是\begin{equation*}
	S^2 = \frac{1}{n-1} \sum_{i=1}^n (X_i-\overline{X})^2,
\end{equation*}
而不是2阶中心矩\begin{equation*}
	B_2 = \frac1n \sum_{i=1}^n (X_i-\overline{X})^2.
\end{equation*}

不过,由于\(B_2 = \frac{n-1}{n} S^2\),
其数学期望\begin{equation*}
	E(B_2)
	= E\left(\frac{n-1}{n} S^2\right)
	= \frac{n-1}{n} E(S^2)
	= \left(1-\frac1n\right) \sigma^2
	\to \sigma^2
	\quad(n\to\infty),
\end{equation*}
所以在工程上,当\(n\)足够大时,也可以将\(B_2\)作为总体方差的估计量.

虽然一般情况下样本方差的抽样分布不易精确得出,
但是总体为\(N(\mu,\sigma^2)\)的样本方差的抽样分布可以精确求出.
\begin{theorem}\label{theorem:数理统计的基础知识.正态分布总体下样本方差的抽样分布}
%@see: 《概率论与数理统计》(陈鸿建、赵永红、翁洋) P180 定理7.4
%@see: 《概率论与数理统计》(茆诗松、周纪芗、张日权) P206 定理4.2.2
%@see: 《应用随机过程:概率模型导论(第11版)》(Sheldon M. Ross,龚光鲁译) P56 命题2.5
样本\(\AutoTuple{X}{n}\)来自正态总体\(N(\mu,\sigma^2)\),
则\begin{equation}\label{equation:抽样分布定理.一个正态总体的抽样分布3}
	\frac{(n-1)S^2}{\sigma^2} \sim \chi^2(n-1),
\end{equation}
且\(\overline{X}\)与\(S^2\)相互独立.
\begin{proof}
对样本\((\AutoTuple{X}{n})\)作线性变换,
令\begin{equation*}
	\left\{ \def\arraystretch{1.5} \begin{array}{l}
		Z_1 = \frac{1}{\sqrt{2}} X_1 - \frac{1}{\sqrt{2}} X_2, \\
		Z_2 = \frac{1}{\sqrt{2\cdot3}} (X_1+X_2) - \frac{2}{\sqrt{2\cdot3}} X_3, \\
		Z_3 = \frac{1}{\sqrt{3\cdot4}} (X_1+X_2+X_3) - \frac{3}{\sqrt{3\cdot4}} X_4, \\
		\hdotsfor{1} \\
		Z_{n-1} = \frac{1}{\sqrt{(n-1)n}} (X_1+X_2+\dotsb+X_{n-1}) - \frac{n-1}{\sqrt{(n-1)n}} X_n, \\
		Z_n = \frac{1}{\sqrt{n}} (X_1+X_2+\dotsb+X_n) = \sqrt{n} \cdot \overline{X}.
	\end{array} \right.
\end{equation*}
由于\(\AutoTuple{X}{n}\)独立同分布于\(N(\mu,\sigma^2)\),
且\begin{equation*}
	\sum_{i=1}^{n-1} \left[ \frac{1}{\sqrt{(n-1)n}} \right]^2
	+ \left[ \frac{n-1}{\sqrt{(n-1)n}} \right]^2
	= \frac{n-1}{(n-1)n}
	+ \frac{(n-1)^2}{(n-1)n}
	= 1,
\end{equation*}
所以,根据\cref{theorem:正态分布与自然指数分布族.正态分布的可加性2} 有\begin{equation*}
	Z_1,Z_2,\dotsc,Z_{n-1} \sim N(0,\sigma^2), \qquad
	Z_n \sim N(\sqrt{n} \mu,\sigma^2),
\end{equation*}
且\(\Cov(Z_i,Z_j) = 0\ (i \neq j)\).
这就说明,\(\AutoTuple{Z}{n}\)相互独立.

由于\begin{equation*}
	\frac{1}{\sigma^2} \sum_{i=1}^n (X_i-\overline{X})^2
	= \frac{1}{\sigma^2} \left( \sum_{i=1}^n X_i^2 - n \overline{X}^2 \right)
	= \frac{1}{\sigma^2} \left( \sum_{i=1}^n Z_i^2 - Z_n^2 \right)
	= \sum_{i=1}^{n-1} \left( \frac{Z_i}{\sigma} \right)^2,
\end{equation*}
且\(\AutoTuple{Z}{n-1}\)相互独立,
且均服从于\(N(0,\sigma^2)\),
所以\(\frac{Z_1}{\sigma},\frac{Z_2}{\sigma},\dotsc,\frac{Z_{n-1}}{\sigma}\)仍相互独立,
且均服从于\(N(0,1)\).
那么由~\hyperref[definition:数理统计的基础知识.卡方分布的定义]{\(\chi^2\)分布的定义}可知\begin{equation*}
	\left( \frac{Z_1}{\sigma} \right)^2
	+ \left( \frac{Z_2}{\sigma} \right)^2
	+ \dotsb
	+ \left( \frac{Z_{n-1}}{\sigma} \right)^2
	\sim \chi^2(n-1),
\end{equation*}即\begin{equation*}
	\frac{1}{\sigma^2} \sum_{i=1}^n (X_i-\overline{X})^2 \sim \chi^2(n-1).
\end{equation*}
又因为\(\AutoTuple{Z}{n}\)相互独立,
且\begin{equation*}
	\frac{1}{\sigma^2} \sum_{i=1}^n (X_i-\overline{X})^2
	= \sum_{i=1}^{n-1} \left( \frac{Z_i}{\sigma} \right)^2,
	\qquad
	\overline{X} = \frac{1}{\sqrt{n}} Z_n,
\end{equation*}
所以\(\frac{1}{\sigma^2} \sum_{i=1}^n (X_i-\overline{X})^2\)与\(\overline{X}\)独立.
\end{proof}
\end{theorem}
\begin{remark}
注意\(\frac{(n-1) S^2}{\sigma^2}\)有一个等价的表达式\(\frac{n B_2}{\sigma^2}\),
即\begin{equation*}
	\frac{(n-1) S^2}{\sigma^2}
	= \frac{1}{\sigma^2} \sum_{i=1}^n (X_i - \overline{X})^2
	= \frac{n B_2}{\sigma^2}.
\end{equation*}
\end{remark}

\begin{example}
%@see: 《2023年全国硕士研究生入学统一考试(数学一)》一选择题/第9题
设\(\AutoTuple{X}{n}\)是来自总体\(N(\mu_1,\sigma_1^2)\)的简单随机样本,
\(\AutoTuple{Y}{m}\)是来自总体\(N(\mu_2,\sigma_2^2)\)的简单随机样本,
且两样本相互独立.
记\(\overline{X} = \frac1n \sum_{i=1}^n X_i,
\overline{Y} = \frac1m \sum_{i=1}^m Y_i,
S_1^2 = \frac1{n-1} \sum_{i=1}^n (X_i - \overline{X})^2,
S_2^2 = \frac1{m-1} \sum_{i=1}^m (Y_i - \overline{Y})^2\).
证明:\(\frac{S_1^2/\sigma_1^2}{S_2^2/\sigma_2^2} \sim F(n-1,m-1)\).
\begin{proof}
由\hyperref[theorem:数理统计的基础知识.正态分布总体下样本方差的抽样分布]{抽样分布定理}可知\begin{equation*}
	\frac{(n-1) S_1^2}{\sigma_1^2} \sim \chi^2(n-1),
	\qquad
	\frac{(m-1) S_2^2}{\sigma_2^2} \sim \chi^2(m-1),
\end{equation*}
所以\begin{equation*}
	\frac{S_1^2/\sigma_1^2}{S_2^2/\sigma_2^2}
	= \frac{
		\frac1{n-1} \cdot \frac{(n-1) S_1^2}{\sigma_1^2}
	}{
		\frac1{m-1} \cdot \frac{(m-1) S_2^2}{\sigma_2^2}
	}
	\sim F(n-1,m-1).
	\qedhere
\end{equation*}
\end{proof}
\end{example}

\begin{theorem}
%@see: 《概率论与数理统计》(陈鸿建、赵永红、翁洋) P180 定理7.5
样本\(\AutoTuple{X}{n}\)来自正态总体\(N(\mu,\sigma^2)\),
则\begin{equation}\label{equation:抽样分布定理.一个正态总体的抽样分布4}
	t = \frac{\overline{X}-\mu}{S / \sqrt{n}} \sim t(n-1).
\end{equation}
\begin{proof}
由于\begin{equation*}
	U = \frac{\overline{X}-\mu}{\sigma/\sqrt{n}} \sim N(0,1),
	\qquad
	V = \frac{(n-1)S^2}{\sigma^2} \sim \chi^2(n-1),
\end{equation*}
且\(\overline{X}\)与\(S^2\)相互独立,
从而\(U\)与\(V\)相互独立.
于是由\(t\)分布定义可得\begin{equation*}
	\frac{U}{\sqrt{V/(n-1)}}
	= \frac{\overline{X}-\mu}{S/\sqrt{n}}
	= t \sim t(n-1).
	\qedhere
\end{equation*}
\end{proof}
\end{theorem}
% 由于t分布的极限分布是标准正态分布\(N(0,1)\),
% 故在上述定理条件下,\(n\)充分大时,
% 样本函数\(t = \frac{\overline{X}-\mu}{S / \sqrt{n}}\)近似地服从标准正态分布.
% 这个结论还可以推广到非正态总体的情形.
%@see: 《概率论与数理统计》(陈鸿建、赵永红、翁洋) P181 定理7.6

\subsubsection{两个正态总体下的抽样分布定理}
\begin{theorem}
%@see: 《概率论与数理统计》(陈鸿建、赵永红、翁洋) P181 定理7.7
若两个总体\(X \sim N(\mu_1,\sigma_1^2)\),\(Y \sim N(\mu_2,\sigma_2^2)\),
则统计量\begin{equation}
	\overline{X}-\overline{Y}
	\sim
	N\left(\mu_1-\mu_2,\frac{\sigma_1^2}{n_1}+\frac{\sigma_2^2}{n_2}\right),
\end{equation}
从而\begin{equation}
	U = \frac{
		(\overline{X}-\overline{Y})-(\mu_1-\mu_2)
	}{
		\sqrt{\frac{\sigma_1^2}{n_1}+\frac{\sigma_2^2}{n_2}}
	}
	\sim
	N(0,1).
\end{equation}
%TODO proof
\end{theorem}

\begin{theorem}
%@see: 《概率论与数理统计》(陈鸿建、赵永红、翁洋) P181 定理7.8
若两个总体\(X \sim N(\mu_1,\sigma^2)\),\(Y \sim N(\mu_2,\sigma^2)\),
则\begin{equation}
	T = \frac{
			(\overline{X}-\overline{Y})-(\mu_1-\mu_2)
		}{
			S_w \sqrt{\frac{1}{n_1}+\frac{1}{n_2}}
		}
	\sim
	t(n_1+n_2-2),
\end{equation}
其中\begin{equation*}
	S_w^2 = \frac{(n_1-1)S_1^2+(n_2-1)S_2^2}{n_1+n_2-2}.
\end{equation*}
%TODO proof
\end{theorem}

\begin{theorem}
%@see: 《概率论与数理统计》(陈鸿建、赵永红、翁洋) P182 定理7.9
若两个总体\(X \sim N(\mu_1,\sigma_1^2)\),\(Y \sim N(\mu_2,\sigma_2^2)\),
则\begin{equation}
	F = \frac{S_1^2 / \sigma_1^2}{S_2^2 / \sigma_2^2} \sim F(n_1-1,n_2-1).
\end{equation}
%TODO proof
\end{theorem}

\subsubsection{一个任何总体下的抽样分布定理}
\begin{theorem}
%@see: 《概率论与数理统计》(茆诗松、周纪芗、张日权) P202 定理4.2.1
设样本\(\AutoTuple{X}{n}\)来自任何总体%
\footnote{所称“任何总体”是指该总体的分布未知,
也就是说,它可能是离散的,也可能连续的,可能是均匀分布,也可能是偏态分布.},
该总体的均值、方差分别为\(\mu\)、\(\sigma^2\in(0,+\infty)\),
则当样本量\(n\)充分大时,
样本均值\(\overline{X}\)近似服从正态分布,
其均值仍为\(\mu\),其方差为\(\sigma^2/n\),
即\begin{equation}
	%@see: 《概率论与数理统计》(茆诗松、周纪芗、张日权) P202 (4.2.5)
	\overline{X}
	\dotsim
	N\left(\mu,\frac{\sigma^2}{n}\right).
\end{equation}
\begin{proof}
由\hyperref[theorem:极限定理.林德伯格--列维中心极限定理]{林德伯格--列维中心极限定理}可知\begin{equation*}
	\frac{\sum_{i=1}^n X_i - n\mu}{\sqrt{n} \sigma} \dotsim N(0,1),
\end{equation*}
由此可知\begin{equation*}
	X_1+X_2+\dotsb+X_n \dotsim N(n\mu,n\sigma^2),
\end{equation*}\begin{equation*}
	\overline{X} \dotsim N\left(\mu,\frac{\sigma^2}{n}\right).
	\qedhere
\end{equation*}
\end{proof}
\end{theorem}
这一定理表明,无论总体分布是什么,
只要样本容量\(n\)充分大,
则样本均值\(\overline{X}\)总可近似看作正态分布.

\begin{theorem}
%@see: 《概率论与数理统计》(陈鸿建、赵永红、翁洋) P181 定理7.6
对任何总体\(X\),\(E(X)=\mu\),\(D(X)=\sigma^2>0\),
\(\AutoTuple{X}{n}\)为来自总体\(X\)的样本,
则当\(n\)充分大时,近似地有\begin{gather}
	\frac{\overline{X}-\mu}{\sigma/\sqrt{n}} \sim N(0,1), \\
	\frac{\overline{X}-\mu}{S/\sqrt{n}} \sim N(0,1).
\end{gather}
%TODO proof
\end{theorem}
