\section{本章总结}
本章学习的重心是样本函数、统计量以及抽样分布.

我们首先学习了卡方分布、\(t\)分布和\(F\)分布.
\begin{table}[htb]
	\centering
	\begin{tblr}{*2c|p{4cm}p{4cm}}
		\hline
		\SetCell[c=2]{c} && \(E(X)\) & \(D(X)\) \\ \hline
		\hyperref[equation:离散型分布.几何分布的分布律]{几何分布}
			& \(X \sim G(p)\)
			& \hyperref[theorem:随机变量的数字特征.几何分布的数学期望]{\(\frac1p\)}
			& \hyperref[theorem:随机变量的数字特征.几何分布的方差]{\(\frac{1-p}{p^2}\)}
			\\ \hline
		\hyperref[equation:离散型分布.超几何分布的分布律]{超几何分布}
			& \(X \sim H(n,m,N)\)
			& \hyperref[theorem:随机变量的数字特征.超几何分布的数学期望]{\(\frac{nm}{N}\)}
			& \hyperref[theorem:随机变量的数字特征.超几何分布的方差]{\(\frac{n m}{N} \left( 1 - \frac{M}{N} \right) \frac{N - n}{N - 1}\)}
			\\ \hline
		\hyperref[equation:离散型分布.二项分布的分布律]{二项分布}
			& \(X \sim B(n,p)\)
			& \hyperref[theorem:随机变量的数字特征.二项分布的数学期望]{\(np\)}
			& \hyperref[theorem:随机变量的数字特征.二项分布的方差]{\(np(1-p)\)}
			\\ \hline
		\hyperref[equation:离散型分布.泊松分布的分布律]{泊松分布}
			& \(X \sim P(\lambda)\)
			& \hyperref[theorem:随机变量的数字特征.泊松分布的数学期望]{\(\lambda\)}
			& \hyperref[theorem:随机变量的数字特征.泊松分布的方差]{\(\lambda\)}
			\\ \hline
		\hyperref[equation:离散型分布.负二项分布的分布律]{负二项分布}
			& \(X \sim NB(r,p)\)
			& \(\frac{r}{p}\)
			& \(\frac{r(1-p)}{p^2}\)
			\\ \hline
		\hyperref[equation:连续型分布.均匀分布的密度函数]{均匀分布}
			& \(X \sim U(a,b)\)
			& \hyperref[theorem:随机变量的数字特征.均匀分布的方差]{\(\frac{a+b}2\)}
			& \hyperref[theorem:随机变量的数字特征.均匀分布的方差]{\(\frac{(b-a)^2}{12}\)}
			\\ \hline
		\hyperref[equation:连续型分布.指数分布的密度函数]{指数分布}
			& \(X \sim e(\lambda)\)
			& \hyperref[theorem:随机变量的数字特征.指数分布的数学期望]{\(\frac1\lambda\)}
			& \hyperref[theorem:随机变量的数字特征.指数分布的方差]{\(\frac1{\lambda^2}\)}
			\\ \hline
		\hyperref[equation:连续型分布.伽马分布的密度函数]{伽马分布}
			& \(X \sim \Gamma(\alpha,\beta)\)
			& \hyperref[theorem:随机变量的数字特征.伽马分布的期望]{\(\frac\alpha\beta\)}
			& \hyperref[theorem:随机变量的数字特征.伽马分布的方差]{\(\frac\alpha{\beta^2}\)}
			\\ \hline
		\hyperref[equation:连续型分布.正态分布的密度函数]{正态分布}
			& \(X \sim N(\mu,\sigma^2)\)
			& \hyperref[theorem:随机变量的数字特征.正态分布的数字特征]{\(\mu\)}
			& \hyperref[theorem:随机变量的数字特征.正态分布的数字特征]{\(\sigma^2\)}
			\\ \hline
		\hyperref[theorem:数理统计的基础知识.卡方分布的密度函数]{卡方分布}
			& \(X \sim \x(n)\)
			& \hyperref[theorem:数理统计的基础知识.卡方分布的数字特征]{\(n\)}
			& \hyperref[theorem:数理统计的基础知识.卡方分布的数字特征]{\(2n\)}
			\\ \hline
		\hyperref[theorem:数理统计的基础知识.学生氏分布的密度函数]{学生氏分布}
			& \(X \sim t(n)\)
			& \(0\ (n>1)\)
			&
			\\ \hline
	\end{tblr}
	\caption{常见分布的数字特征}
\end{table}

常见的统计量包括:\begin{itemize}
	\item 样本均值\(\overline{X} = \frac1n \sum_{i=1}^n X_i\).
	\item 样本方差\(S^2 = \frac{1}{n-1} \sum_{i=1}^n (X_i-\overline{X})^2\).
	\item 样本标准差\(S=\sqrt{S^2}\).
	\item 样本\(k\)阶原点矩\(A_k=\frac1n \sum_{i=1}^n X_i^k\).
	\item 样本\(k\)阶中心矩\(B_k=\frac1n \sum_{i=1}^n (X_i-\overline{X})^k\).
\end{itemize}

\begin{table}[htb]
	\centering
	\begin{tblr}{*5c}
		\hline
		样本来源
			& \(\overline{X}\)
			& \(\frac{\overline{X}-\mu}{\sigma/\sqrt{n}}\)
			& \(\frac{(n-1)S^2}{\sigma^2}\)
			& \(\frac{\overline{X}-\mu}{S/\sqrt{n}}\)
			\\
		\hline
		\(X \sim N(\mu,\sigma^2)\)
			& \(N\left(\mu,\frac{\sigma^2}{n}\right)\)
			& \(N(0,1)\)
			& \(\x(n-1)\)
			& \(t(n-1)\)
			\\
		\(X \sim e(\lambda)\)
			& \(\Gamma(n,n\lambda)\)
			\\
		任何总体(\(n\)充分大)
			&
			& \(N(0,1)\)
			&
			& \(N(0,1)\)
			\\
		\hline
	\end{tblr}
	\caption{一个总体下的抽样分布}
\end{table}

\begin{table}[htb]
	\centering
	\begin{tblr}{*5c}
		\hline
		样本来源
			& \(\overline{X}-\overline{Y}\)
			& \(\frac{(\overline{X}-\overline{Y})-(\mu_1-\mu_2)}{\sqrt{(\sigma_1^2/n_1)+(\sigma_2^2/n_2)}}\)
			& \(\frac{S_1^2/\sigma_1^2}{S_2^2/\sigma_2^2}\)
			\\
		\hline
		\(\begin{array}{l}
			X \sim N(\mu_1,\sigma_1^2) \\
			Y \sim N(\mu_2,\sigma_2^2)
		\end{array}\)
			& \(N\left(\mu_1-\mu_2,\frac{\sigma_1^2}{n_1}+\frac{\sigma_2^2}{n_2}\right)\)
			& \(N(0,1)\)
			& \(F(n_1-1,n_2-1)\)
			\\
		\hline
	\end{tblr}
	\caption{两个总体下的抽样分布}
\end{table}
