\section{充分统计量}
%@see: 《数理统计教程》(王兆军,邹长亮) P29
统计量的引入,是为了简化繁杂的样本.
但是在利用统计量进行统计推断时,
一个自然的问题是:
既然我们对样本的某些属性感兴趣,
那么我们采用的统计量是否囊括了所有与这些属性有关的信息?
1922年,费舍尔与爱丁顿两人,
就利用样本估计正态总体\(N(\mu,\sigma^2)\)的标准差的问题,展开了激烈争辩,
后者主张用\begin{equation}\label{equation:充分统计量.爱丁顿主张的统计量}
	\sqrt{\frac\pi2}
	\frac1n
	\sum_{i=1}^n \abs{X_i - \overline{X}}
\end{equation}
进行估计,
前者则主张用\begin{equation}\label{equation:充分统计量.费舍尔主张的统计量}
	\sqrt{
		\frac{1}{n-1} \sum_{i=1}^n (X_i-\overline{X})^2
	}
\end{equation}
进行估计.
费舍尔认为“\cref{equation:充分统计量.费舍尔主张的统计量}
包含了样本中有关\(\sigma\)的全部信息,
而\cref{equation:充分统计量.爱丁顿主张的统计量} 则没有”.
这就是“充分统计量”的思想源泉.

\begin{definition}
%@see: 《数理统计教程》(王兆军,邹长亮) P31 定义1.4.1
设\(F_\theta\)是总体\(X\)的分布函数,
其中\(\theta\)是未知参数.
设\(\AutoTuple{X}{n}\)是取自总体\(X\)的一个样本,
其观测值\(\AutoTuple{x}{n}\)未知.
如果当统计量\(T = T(\AutoTuple{X}{n})\)的观测值
\(T(\AutoTuple{x}{n})\)等于\(t\)时
样本\((\AutoTuple{X}{n})\)的条件概率分布
与总体分布\(F_\theta\)或参数\(\theta\)无关,
即条件概率分布的表达式中不含\(\theta\),
则称“统计量\(T\)是分布族\(\Set{ F_\theta \given \theta \in \Theta }\)的\DefineConcept{充分统计量}(sufficient statistic)”
或“统计量\(T\)是参数\(\theta\)的\DefineConcept{充分统计量}”.
\end{definition}
