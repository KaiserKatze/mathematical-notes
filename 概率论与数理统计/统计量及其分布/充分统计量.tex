\section{充分统计量}
%@see: 《数理统计教程》(王兆军,邹长亮) P29
统计量的引入,是为了简化繁杂的样本.
但是在利用统计量进行统计推断时,
一个自然的问题是:
既然我们对样本的某些属性感兴趣,
那么我们采用的统计量是否囊括了所有与这些属性有关的信息?
1922年,费舍尔与爱丁顿两人,
就利用样本估计正态总体\(N(\mu,\sigma^2)\)的标准差的问题,展开了激烈争辩,
后者主张用\begin{equation}\label{equation:充分统计量.爱丁顿主张的统计量}
	\sqrt{\frac\pi2}
	\frac1n
	\sum_{i=1}^n \abs{X_i - \overline{X}}
\end{equation}
进行估计,
前者则主张用\begin{equation}\label{equation:充分统计量.费舍尔主张的统计量}
	\sqrt{
		\frac{1}{n-1} \sum_{i=1}^n (X_i-\overline{X})^2
	}
\end{equation}
进行估计.
费舍尔认为“\cref{equation:充分统计量.费舍尔主张的统计量}
包含了样本中有关\(\sigma\)的全部信息,
而\cref{equation:充分统计量.爱丁顿主张的统计量} 则没有”.
这就是“充分统计量”的思想源泉.

\begin{definition}
%@see: 《数理统计教程》(王兆军,邹长亮) P31 定义1.4.1
设\(F_\theta\)是总体\(X\)的分布函数,
其中\(\theta\)是未知参数.
设\(\AutoTuple{X}{n}\)是取自总体\(X\)的一个样本,
其观测值\(\AutoTuple{x}{n}\)未知.
如果当统计量\(T = T(\AutoTuple{X}{n})\)的观测值
\(T(\AutoTuple{x}{n})\)等于\(t\)时
样本\((\AutoTuple{X}{n})\)的条件概率分布
与总体分布\(F_\theta\)或参数\(\theta\)无关,
即条件概率分布的表达式中不含\(\theta\),
则称“统计量\(T\)是分布族\(\Set{ F_\theta \given \theta \in \Theta }\)的\DefineConcept{充分统计量}(sufficient statistic)”
或“统计量\(T\)是参数\(\theta\)的\DefineConcept{充分统计量}”.
\end{definition}

\begin{example}
%@see: 《数理统计教程》(王兆军,邹长亮) P30 例1.4.2
设\(X_1,\dotsc,X_n\)独立同分布于\(B(1,p)\),
其中\(0<p<1,n>2\).
试问\(
	T_1 \defeq \sum_{i=1}^n X_i,
	T_2 \defeq X_1 + X_2
\)是不是参数\(p\)的充分统计量.
\begin{solution}
由题意有,样本分布为\begin{equation*}
	P(X_1=x_1,\dotsc,X_n=x_n)
	= p^{x_1+\dotsb+x_n}
	(1-p)^{n-(x_1+\dotsb+x_n)},
\end{equation*}
且\(T_1 \sim B(n,p)\),
故当\(T_1 = t\)时样本的条件分布为\begin{align*}
	&P(X_1=x_1,\dotsc,X_n=x_n \vert T_1 = t) \\
	&= \frac{
		P\left(
			X_1=x_1,
			X_2=x_2,
			X_3=x_3,
			\dotsc,
			X_{n-1}=x_{n-1},
			X_n=t-(x_1+\dotsb+x_{n-1})
		\right)
	}{
		P(T_1=t)
	} \\
	&= \frac{p^t (1-p)^{n-t}}{C_n^t p^t (1-p)^{n-t}}
	= (C_n^t)^{-1}.
\end{align*}
显然\(P(X_1=x_1,\dotsc,X_n=x_n \vert T_1 = t) = (C_n^t)^{-1}\)与参数\(p\)无关,
该条件分布中已经没有关于\(p\)的任何信息了,
或者说,样本中关于\(p\)的所有信息都包含在统计量\(T_1\)中了.
因此\(T_1\)是\(p\)的充分统计量.
但是对于统计量\(T_2\)而言,
由于\(T_2 \sim B(2,p)\),
所以\begin{align*}
	&P(X_1=x_1,\dotsc,X_n=x_n \vert T_2 = t) \\
	&= \frac{
		P\left(
			X_1=x_1,
			X_2=t-x_1,
			X_3=x_3,
			\dotsc,
			X_{n-1}=x_{n-1},
			X_n=x_n
		\right)
	}{
		P(T_2=t)
	} \\
	&= \frac{
		p^{t+(x_3+\dotsb+x_n)}
		(1-p)^{n-t-(x_3+\dotsb+x_n)}
	}{
		C_2^t p^t (1-p)^{2-t}
	}
	= (C_2^t)^{-1} p^{x_3+\dotsb+x_n} (1-p)^{n-2-(x_3+\dotsb+x_n)}.
\end{align*}
显然\(P(X_1=x_1,\dotsc,X_n=x_n \vert T_2 = t)\)与\(p\)有关,
说明\(T_2\)没有把样本中关于\(p\)的所有信息都包含进来,
\(T_2\)不是\(p\)的充分统计量.
\end{solution}
\end{example}
