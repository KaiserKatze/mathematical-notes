\section{随机过程}
\begin{definition}
%@see: 《应用随机过程:概率模型导论(第11版)》(Sheldon M. Ross,龚光鲁译) P65
如果有标集族\(\{X_t\}_{t \in T}\)中的每一个元素\(X_t\)都是一个随机变量,
那么把\(\{X_t\}_{t \in T}\)称为一个\DefineConcept{随机过程}(random process,stochastic process),
称\(X_t\)是过程在时间\(t\)的\DefineConcept{状态}(state),
称\(T\)是过程的\DefineConcept{指标集}(index set).
\end{definition}

\begin{definition}
设\(\{X_t\}_{t \in T}\)是一个随机过程.
把有标集族\(\{X_t\}_{t \in T}\)的值域\(
	\Set{
		X_t
		\given
		t \in T
	}
\)称为“随机过程\(\{X_t\}_{t \in T}\)的\DefineConcept{状态空间}(state space)”.
\end{definition}
