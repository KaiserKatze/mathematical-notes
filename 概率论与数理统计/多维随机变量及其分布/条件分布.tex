\section{条件分布与条件密度}
当二维随机变量\((X,Y)\)中\(X\)与\(Y\)不独立时,
随机变量\(X\)与\(Y\)应有一定的相互影响的关系,
即当\(P(Y = y) > 0\)时,
通常有\(P(X \leq x \vert Y = y) \neq P(X \leq x)\).
可以看出,条件概率\(P(X \leq x \vert Y = y)\)一般受\(y\)的影响.
于是我们把\[
	F_{X \vert Y}(x \vert y)
	\defeq
	P(X \leq x \vert Y = y)
	\quad(x\in\mathbb{R})
\]称为“\(Y=y\)条件下\(X\)的\DefineConcept{条件分布函数}”.

\subsection{离散型随机变量的条件分布}
设二维离散型随机变量\((X,Y)\)有二维概率分布\[
	p_{ij} = P(X=x_i,Y=y_j),
	\quad i,j=1,2,\dotsc.
\]
从而\(X\)及\(Y\)有边缘分布\begin{align*}
	p_{i*}
	&= P(X=x_i)
	= \sum_j p_{ij},
	\quad i=1,2,\dotsc; \\
	p_{*j}
	&= P(Y=y_j)
	= \sum_i p_{ij},
	\quad j=1,2,\dotsc.
\end{align*}
那么,对于任意给定\(y_j\),
若\(P(Y=y_j) = p_{*j} > 0\),
称\[
	P(X=x_i \vert Y=y_j) = \frac{p_{ij}}{p_{*j}}, \quad i=1,2,\dotsc
\]为\(Y=y_j\)条件下\(X\)的\DefineConcept{条件概率分布}.
同理,对于任意给定\(x_i\),
若\(P(X=x_i) = p_{i*} > 0\),
称\[
	P(Y=y_j \vert X=x_i) = \frac{p_{ij}}{p_{i*}}, \quad j=1,2,\dotsc
\]为\(X=x_i\)条件下\(Y\)的\DefineConcept{条件概率分布}.

\begin{property}
离散型随机变量的条件概率分布有如下性质:
\begin{enumerate}
	\item \(P(X=x_i \vert Y=y_j) \geq 0, \quad i=1,2,\dotsc;\)
	\item \(\sum_{i}{P(X=x_i \vert Y=y_j)} = \sum_{i}{\frac{p_{ij}}{p_{*j}}} = 1.\)
\end{enumerate}
可见,条件概率分布也是离散型概率分布.
\end{property}

\begin{theorem}
对任意\(x\)、\(y\),由条件分布可得条件分布函数的表示:
\begin{align*}
	F_{X \vert Y}(x \vert y_j)
	= P(X \leq x \vert Y=y_j)
	= \sum_{x_i \leq x}{\frac{p_{ij}}{p_{*j}}}, \\
	F_{Y \vert X}(y \vert x_i)
	= P(Y \leq y \vert X=x_i)
	= \sum_{y_j \leq y}{\frac{p_{ij}}{p_{i*}}}.
\end{align*}
\end{theorem}

\subsection{连续型随机变量的条件密度函数}
\begin{definition}
设\((X,Y)\)为二维连续型随机变量.若对任意\(\epsilon > 0\),有\[
	P(y - \epsilon < Y \leq y + \epsilon) > 0,
\]
且对\(x\in\mathbb{R}\),
极限\[
	\lim_{\epsilon\to0^+} P(X \leq x \vert y - \epsilon < Y \leq y + \epsilon)
\]存在,
则称该极限为“连续型随机变量\(Y=y\)条件下\(X\)的\DefineConcept{条件分布函数}”,
记为\[
	F_{X \vert Y}(x \vert y)
	\quad\text{或}\quad
	P(X \leq x \vert Y = y).
\]

类似地,可以定义“连续型随机变量\(X=x\)条件下\(Y\)的\DefineConcept{条件分布函数}”,
记为\[
	F_{Y \vert X}(y \vert x)
	\quad\text{或}\quad
	P(Y \leq y \vert X = x).
\]
\end{definition}

\begin{theorem}
设二维连续型随机变量\((X,Y)\)有二维密度\(f(x,y)\),
从而\(X\)及\(Y\)有边缘密度\(f_X(x)\)、\(f_Y(y)\),则
\begin{align*}
	F_{X \vert Y}(x \vert y)
	= \int_{-\infty}^x \frac{f(u,y)}{f_Y(y)}\dd{u}, \quad x \in \mathbb{R}; \\
	F_{Y \vert X}(y \vert x)
	= \int_{-\infty}^y \frac{f(x,v)}{f_X(x)}\dd{v}, \quad y \in \mathbb{R}.
\end{align*}

那么,相应的密度函数
\begin{gather}
	f_{X \vert Y}(x \vert y)
	= \frac{f(x,y)}{f_Y(y)},
		\label{equation:多维随机变量及其分布.条件密度、联合密度、边缘密度的关系1} \\
	f_{Y \vert X}(y \vert x)
	= \frac{f(x,y)}{f_X(x)},
		\label{equation:多维随机变量及其分布.条件密度、联合密度、边缘密度的关系2}
\end{gather}
分别称为“\(X\)关于\(Y\)的\DefineConcept{条件密度函数}”%
和“\(Y\)关于\(X\)的\DefineConcept{条件密度函数}”.
\begin{proof}
不妨设\(f(x,y)\)连续,\(f_Y(y)\)连续且\(f_Y(y)>0\),
\def\l{\lim_{\epsilon\to0^+}}%
那么\begin{align*}
	F_{X \vert Y}(x \vert y)
	&= \lim_{\epsilon\to0^+}
		P(X \leq x \vert y - \epsilon < Y \leq y + \epsilon) \\
	&= \lim_{\epsilon\to0^+}
		\frac{
			P(X \leq x, y - \epsilon < Y \leq y + \epsilon)
		}{
			P(y - \epsilon < Y \leq y + \epsilon)
		} \\
	&= \lim_{\epsilon\to0^+}
		\frac{
			F(x,y+\epsilon) - F(x,y-\epsilon)
		}{
			F_Y(y+\epsilon) - F_Y(y-\epsilon)
		} \\
	&= \lim_{\epsilon\to0^+}
		\frac{
			[F(x,y+\epsilon) - F(x,y-\epsilon)] \frac{1}{2 \epsilon}
		}{
			[F_Y(y+\epsilon) - F_Y(y-\epsilon)] \frac{1}{2 \epsilon}
		} \\
	&= \pdv{F(x,y)}{y} \bigg/ F_Y'(y) \\
	&= \int_{-\infty}^x \frac{f(u,y)}{f_Y(y)} \dd{u}.
	\qedhere
\end{align*}
\end{proof}
\end{theorem}

可以证明,条件分布函数也是分布函数.

\begin{corollary}
已知边缘密度函数和条件密度函数,可以求出二维密度,即\[
	f(x,y) = f_Y(y) \cdot f_{X \vert Y}(x \vert y)
	= f_X(x) \cdot f_{Y \vert X}(y \vert x).
\]
\end{corollary}

\begin{example}
设\(X \sim U(0,1)\);
对\(\forall x\in(0,1)\),
当\(X=x\)时,\(Y \sim U(x^2,1)\).
求\(P(X > Y)\).
\begin{solution}
\(X\)的密度函数为\[
	f_X(x) = \left\{ \begin{array}{cl}
		1, & 0<x<1, \\
		0, & \text{其他}.
	\end{array} \right.
\]当\(X=x\in(0,1)\)时,\(Y\)有条件密度\[
	f_{Y \vert X}(y \vert x)
	= \left\{ \begin{array}{cl}
		\frac{1}{1-x^2}, & x^2<y<1, \\
		0, & \text{其他}.
	\end{array} \right.
\]因此\[
	f(x,y) = f_X(x) \cdot f_{Y \vert X}(y \vert x)
	= \left\{ \begin{array}{cl}
		\frac{1}{1-x^2}, & 0<x<1 \land x^2<y<1, \\
		0, & \text{其他}.
	\end{array} \right.
\]\[
	P(X > Y)
	= \int_0^1 \dd{x} \int_{x^2}^x \frac{1}{1-x^2} \dd{y}
	= 1 - \ln2.
\]
\end{solution}
\end{example}

\begin{example}
%@see: 《2020年全国硕士研究生入学统一考试(数学一)》三解答题/第22题
设随机变量\(X_1,X_2,X_3\)相互独立,
其中\(X_1\)与\(X_2\)均服从标准正态分布,
\(X_3\)的概率分布为\begin{equation*}
	P(X_3=0)
	= P(X_3=1)
	= \frac12.
\end{equation*}
\(Y = X_3 X_1 + (1 - X_3) X_2\).
求二维随机变量\((X_1,Y)\)的分布函数,和\(Y\)的边缘分布函数.
\begin{solution}
设\((X_1,Y)\)的分布函数为\(F(x,y)\),
则\begin{align*}
	F(x,y) &= P(X_1 \leq x,Y \leq y)
	= P(X_1 \leq x,X_3 X_1 + (1 - X_3) X_2 \leq y) \\
	% 全概率公式
	&= P(X_1 \leq x,X_3 X_1 + (1 - X_3) X_2 \leq y \vert X_3 = 0) \cdot P(X_3 = 0) \\
	&\hspace{20pt}+ P(X_1 \leq x,X_3 X_1 + (1 - X_3) X_2 \leq y \vert X_3 = 1) \cdot P(X_3 = 1) \\
	% 把条件事件代入
	&= P(X_1 \leq x,X_2 \leq y \vert X_3 = 0) \cdot P(X_3 = 0)
	+ P(X_1 \leq x,X_1 \leq y \vert X_3 = 1) \cdot P(X_3 = 1) \\
	&= P(X_1 \leq x,X_2 \leq y,X_3 = 0)
	+ P(X_1 \leq x,X_1 \leq y,X_3 = 1) \\
	&= P(X_1 \leq x) P(X_2 \leq y) P(X_3 = 0)
	+ P(X_1 \leq x,X_1 \leq y) P(X_3 = 1) \\
	&= \frac12 \Phi(x) \Phi(y) + \frac12 P(X_1 \leq \min\{x,y\}).
\end{align*}
当\(x \leq y\)时,有\begin{equation*}
	P(X_1 \leq \min\{x,y\})
	= P(X_1 \leq x) = \Phi(x).
\end{equation*}
当\(x > y\)时,有\begin{equation*}
	P(X_1 \leq \min\{x,y\})
	= P(X_1 \leq y) = \Phi(y).
\end{equation*}
因此\begin{equation*}
	F(x,y)
	= \left\{ \def\arraystretch{1.5} \begin{array}{cl}
		\frac12 \Phi(x) (\Phi(y) + 1), & x \leq y, \\
		\frac12 \Phi(y) (\Phi(x) + 1), & x > y.
	\end{array} \right.
\end{equation*}
于是\(Y\)的边缘分布为\begin{align*}
	F_Y(y) = F(+\infty,y)
	= \lim_{x\to+\infty} \frac12 \Phi(y) (\Phi(x) + 1)
	= \frac12 \Phi(y) (\Phi(+\infty) + 1)
	% \(\Phi(+\infty) = 1\)
	= \Phi(y).
\end{align*}
\end{solution}
\end{example}
