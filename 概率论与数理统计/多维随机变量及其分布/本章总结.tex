\section{本章总结}
\subsection*{边缘分布,随机变量的独立性}
%\cref{theorem:多维随机变量及其分布.联合密度、边缘密度的关系}
设\(F(x,y)\)为二维随机变量\((X,Y)\)的二维分布函数,
则\(X\)与\(Y\)的边缘分布函数分别为\begin{gather*}
	F_X(x) = F(x,+\infty)
	\quad(-\infty < x < +\infty), \\
	F_Y(y) = F(+\infty,y)
	\quad(-\infty < x < +\infty).
\end{gather*}

%\cref{definition:多维随机变量及其分布.随机变量的独立性}
设\(\AutoTuple{X}{n}\)是\(n\)维随机变量.
若对任意\(n\)个实数\(\AutoTuple{x}{n}\),
\(n\)个事件\((X_1 \leq x_1),\allowbreak\dotsc,\allowbreak(X_n \leq x_n)\)相互独立,
即有\[
	P(X_1 \leq x_1,\dotsc,X_n \leq x_n)
	= \prod_{i=1}^n P(X_i \leq x_i)
	= P(X_1 \leq x_1) \dotsm P(X_n \leq x_n)
\]
或\[
	F(x_1,\dotsc,x_n)
	= \prod_{i=1}^n F_i(x_i)
	= F_1(x_1) \dotsm F_n(x_n),
\]
其中\(F\)是\(n\)维随机变量\(\AutoTuple{X}{n}\)的联合分布函数,
而\(F_1,\dotsc,F_n\)分别是\(X_1,\dotsc,X_n\)的边缘分布函数,
则称“\(n\)个随机变量\(\AutoTuple{X}{n}\)相互独立”;
否则称“\(n\)个随机变量\(\AutoTuple{X}{n}\)不相互独立”
或“\(n\)个随机变量\(\AutoTuple{X}{n}\)相依”.

\(n\)个事件两两独立是它们相互独立的必要不充分条件.

%\cref{theorem:多维随机变量及其分布.两个离散型随机变量相互独立的充分必要条件}
设\((X,Y)\)是二维离散型随机变量,有二维概率分布\[
	p_{ij} = P(X=x_i,Y=y_j), \quad i,j=1,2,\dotsc,
\]
和边缘分布\begin{gather*}
	p_{i*} = \sum_j p_{ij},
	\quad i=1,2,\dotsc; \\
	p_{*j} = \sum_i p_{ij},
	\quad j=1,2,\dotsc,
\end{gather*}
则随机变量\(X\)与\(Y\)相互独立的充分必要条件是:\[
	p_{ij} = p_{i*} p_{*j}, \quad i,j=1,2,\dotsc.
\]

%\cref{theorem:多维随机变量及其分布.两个连续型随机变量相互独立的充分必要条件}
设二维连续型随机变量\((X,Y)\)的二维密度为\(f(x,y)\),
\(X\)与\(Y\)的边缘密度分别为\(f_X(x)\)和\(f_Y(y)\),
则\begin{align*}
	f_X(x) = \int_{-\infty}^{+\infty} f(x,y) \dd{y}, \\
	f_Y(y) = \int_{-\infty}^{+\infty} f(x,y) \dd{x}.
\end{align*}
而\(X\)与\(Y\)相互独立的充分必要条件是:\[
	f(x,y) = f_X(x) f_Y(y).
\]在三个密度函数的公共连续点上成立.

\subsection*{联合分布、边缘分布与条件分布的联系}
设二维连续型随机变量\((X,Y)\)有二维密度\(f(x,y)\),
从而\(X\)及\(Y\)有边缘密度\(f_X(x),f_Y(y)\),
则\begin{gather*}
	F_{X \vert Y}(x \vert y)
	= \int_{-\infty}^x \frac{f(u,y)}{f_Y(y)}\dd{u}
	\quad(-\infty < x < +\infty), \\
	F_{Y \vert X}(y \vert x)
	= \int_{-\infty}^y \frac{f(x,v)}{f_X(x)}\dd{v}
	(\quad y \in \mathbb{R}).
\end{gather*}
\(X\)关于\(Y\)的条件密度函数为\begin{equation*}
	%\cref{equation:多维随机变量及其分布.条件密度、联合密度、边缘密度的关系1}
	f_{X \vert Y}(x \vert y)
	= \frac{f(x,y)}{f_Y(y)}.
\end{equation*}
\(Y\)关于\(X\)的条件密度函数为\begin{equation*}
	%\cref{equation:多维随机变量及其分布.条件密度、联合密度、边缘密度的关系2}
	f_{Y \vert X}(y \vert x)
	= \frac{f(x,y)}{f_X(x)}.
\end{equation*}
反过来,可以利用边缘密度函数和条件密度函数计算联合密度函数:\begin{equation*}
	f(x,y) = f_Y(y) \cdot f_{X \vert Y}(x \vert y)
	= f_X(x) \cdot f_{Y \vert X}(y \vert x).
\end{equation*}

\subsection*{分布的可加性}
%\cref{theorem:多维随机变量及其分布.二项分布的可加性1}
设\(X \sim B(n,p)\),
\(Y \sim B(m,p)\),
且\(X\)与\(Y\)相互独立,
则\[
	X+Y \sim B(n+m,p).
\]

%\cref{theorem:多维随机变量及其分布.泊松分布的可加性1}
设\(X \sim P(\lambda_1)\),
\(Y \sim P(\lambda_2)\),
且\(X\)与\(Y\)相互独立,
则\[
	X+Y \sim P(\lambda_1 + \lambda_2).
\]

%\cref{theorem:正态分布与自然指数分布族.正态分布的可加性1}
设\(X \sim N(\mu_1,\sigma_1^2)\),
\(Y \sim N(\mu_2,\sigma_2^2)\),
且\(X\)与\(Y\)相互独立,
则\begin{equation*}
	X+Y \sim N(\mu_1+\mu_2,\sigma_1^2+\sigma_2^2).
\end{equation*}

%\cref{theorem:正态分布与自然指数分布族.正态分布的可加性2}
设随机变量\(\AutoTuple{X}{n}\)相互独立,
且\[
	X_i \sim N(\mu_i,\sigma_i^2),
	\quad i=1,2,\dotsc,n,
\]
且\(C_1,C_2,\dotsc,C_n\)为常数,
则\begin{equation*}
	\sum_{i=1}^n {C_i X_i}
	\sim N\left(
	\sum_{i=1}^n {C_i \mu_i},
	\sum_{i=1}^n {C_i^2 \sigma_i^2}
	\right).
\end{equation*}
