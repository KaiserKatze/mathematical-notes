\section{分布的可加性}
\begin{definition}
当\(\AutoTuple{X}{n}\)相互独立且具有同一类型分布时,
若\(X_1+X_2+\dotsb+X_n\)也服从这一类型的分布,
就称这种类型的分布具有\DefineConcept{可加性}.
\end{definition}

\subsection{二项分布的可加性}
\begin{theorem}\label{theorem:多维随机变量及其分布.二项分布的可加性1}
%@see: 《概率论与数理统计》(陈鸿建、赵永红、翁洋) P91 定理3.11
设\(X \sim B(n,p)\),
\(Y \sim B(m,p)\),
且\(X\)与\(Y\)相互独立,
则\[
	X+Y \sim B(n+m,p).
\]
\begin{proof}
记\(Z = X+Y\).
\(Z\)的取值为\([0,n+m]\cap\mathbb{N}\).
事件\((Z=k)\)可以表示为\[
	(Z=k)
	= \bigcup_{r=0}^k (X=r,Y=k-r)
	\quad(k=0,1,\dotsc,n+m).
\]
注意上式右端为\(k+1\)个两两互斥事件之并,
再注意到\(X\)与\(Y\)独立,
则\begin{align*}
	P(Z=k)
	&= \sum_{r=0}^k P(X=r,Y=k-r) \\
	&= \sum_{r=0}^k P(X=r) P(Y=k-r) \\
	&= \sum_{r=0}^k C_n^r p^r (1-p)^{n-r} \cdot C_m^{k-r} p^{k-r} (1-p)^{m-k+r} \\
	&= p^k (1-p)^{n+m-k} \sum_{r=0}^k C_n^r C_m^{k-r} \\
	&= C_{n+m}^k p^k (1-p)^{n+m-k},
	\quad k=0,1,\dotsc,n+m.
\end{align*}
于是\(Z \sim B(n+m,p)\).
\end{proof}
\end{theorem}

\begin{corollary}\label{theorem:多维随机变量及其分布.二项分布的可加性2}
%@see: 《概率论与数理统计》(陈鸿建、赵永红、翁洋) P92 推论1
设\(X_i \sim B(n_i,p)\ (i=1,2,\dotsc,n)\),
且\(\AutoTuple{X}{n}\)相互独立,
则\[
	X_1+X_2+\dotsb+X_n \sim B\left(\sum_{i=1}^n n_i,p\right).
\]
\end{corollary}

\begin{corollary}\label{theorem:多维随机变量及其分布.二项分布的可加性3}
%@see: 《概率论与数理统计》(陈鸿建、赵永红、翁洋) P92 推论2
设\(X_i\ (i=1,2,\dotsc,n)\)独立同分布于\(0-1\)分布\(B(1,p)\),则\[
	X_1+X_2+\dotsb+X_n \sim B(n,p).
\]
\end{corollary}

\begin{example}
%@see: 《2025年全国硕士研究生入学统一考试(数学一)》一选择题/第9题
设\(X_1,X_2,\dotsc,X_{20}\)是来自总体\(B(1,0.1)\)的简单随机样本,
令\(T=\sum_{i=1}^{20} X_i\).
利用泊松分布近似表示二项分布的方法,计算\(P(T\leq1)\).
\begin{solution}
%\cref{theorem:多维随机变量及其分布.二项分布的可加性2}
由题意有\(T \sim B(20,0.1)\),
记\(n=20,p=0.1\),
从而泊松分布的参数为\(\lambda = n p = 2\),
那么\begin{equation*}
	P(T\leq1)
	= P(T=0) + P(T=1)
	= \left( \frac{\lambda^0}{0!} + \frac{\lambda^1}{1!} \right) e^{-\lambda}
	= (1 + \lambda) e^{-\lambda}
	= \frac3{e^2}.
\end{equation*}
\end{solution}
\end{example}

\subsection{泊松分布的可加性}
\begin{theorem}\label{theorem:多维随机变量及其分布.泊松分布的可加性1}
%@see: 《概率论与数理统计》(陈鸿建、赵永红、翁洋) P92 定理3.12
设\(X \sim P(\lambda_1)\),
\(Y \sim P(\lambda_2)\),
且\(X\)与\(Y\)相互独立,
则\[
	X+Y \sim P(\lambda_1 + \lambda_2).
\]
\end{theorem}

\begin{corollary}\label{theorem:多维随机变量及其分布.泊松分布的可加性2}
%@see: 《概率论与数理统计》(陈鸿建、赵永红、翁洋) P92 推论
设\(X_i \sim P(\lambda_i)\ (i=1,2,\dotsc,n)\),
且\(\AutoTuple{X}{n}\)相互独立,
则\[
	X_1+X_2+\dotsb+X_n \sim P\left(\sum_{i=1}^n \lambda_i\right).
\]
\end{corollary}

\subsection{\texorpdfstring{\(\Gamma\)分布的可加性}{伽马分布的可加性}}
\begin{theorem}\label{theorem:多维随机变量及其分布.伽马分布的可加性1}
%@see: 《概率论与数理统计》(陈鸿建、赵永红、翁洋) P93 定理3.14
设随机变量\(X_i \sim \Gamma(\alpha_i,\beta)\ (i=1,2,\dotsc,n)\),
且\(\AutoTuple{X}{n}\)相互独立,
则\[
	X_1+X_2+\dotsb+X_n
	\sim
	\Gamma\left(\sum_{i=1}^n \alpha_i,\beta\right).
\]
\end{theorem}

\subsection{正态分布的可加性}
\begin{theorem}\label{theorem:正态分布与自然指数分布族.正态分布的可加性1}
设\(X \sim N(\mu_1,\sigma_1^2)\),
\(Y \sim N(\mu_2,\sigma_2^2)\),
且\(X\)与\(Y\)相互独立,
则\begin{equation}
	X+Y \sim N(\mu_1+\mu_2,\sigma_1^2+\sigma_2^2).
\end{equation}
\end{theorem}

\begin{corollary}\label{theorem:正态分布与自然指数分布族.正态分布的可加性2}
设随机变量\(\AutoTuple{X}{n}\)相互独立,
且\[
	X_i \sim N(\mu_i,\sigma_i^2),
	\quad i=1,2,\dotsc,n,
\]
且\(C_1,C_2,\dotsc,C_n\)为常数,
则\begin{equation}
	\sum_{i=1}^n {C_i X_i}
	\sim N\left(
	\sum_{i=1}^n {C_i \mu_i},
	\sum_{i=1}^n {C_i^2 \sigma_i^2}
	\right).
\end{equation}
\end{corollary}

\begin{example}
设\(\AutoTuple{X}{n}\)独立同分布于\(N(\mu,\sigma^2)\),试计算其算术平均数\[
	\overline{X} = \frac{1}{n} (X_1+X_2+\dotsb+X_n)
\]的分布.
\begin{solution}
由\hyperref[theorem:正态分布与自然指数分布族.正态分布的可加性2]{正态分布的卷积公式}可知\[
	X_1+X_2+\dotsb+X_n \sim N(n\mu,n\sigma^2).
\]又由\hyperref[theorem:正态分布与自然指数分布族.正态分布的线性性质]{正态分布的线性性}可知\[
	\overline{X} = \frac{1}{n} (X_1+X_2+\dotsb+X_n) \sim N\left(\mu,\frac{\sigma^2}{n}\right).
\]
\end{solution}
由此可见,\(n\)个独立同分布于正态分布\(N(\mu,\sigma^2)\)
的随机变量的算术平均数\(\overline{X}\)仍服从正态分布,
其均值与原分布的均值\(\mu\)相同;
但其方差缩小了\(n\)倍,变为\(\sigma^2/n\);
其标准差缩小了\(\sqrt{n}\)倍,
变为\(\sigma/\sqrt{n}\).
这表明\(\overline{X}\)的分布更加集中,
这也是为什么在测量物体的尺寸时我们应该多次读数并取算术平均值.
\end{example}
