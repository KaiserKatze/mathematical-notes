\section{二维随机变量及其分布函数}
\subsection{二维随机变量及其分布函数}
\begin{definition}
%@see: 《概率论与数理统计》(陈鸿建、赵永红、翁洋) P64 定义3.1
设\(X\)与\(Y\)是定义在同一样本空间\(\Omega\)上的两个随机变量,
则称“\((X,Y)\)是\DefineConcept{二维随机变量}”
或“\((X,Y)\)是\DefineConcept{二维随机向量}”.
\end{definition}

随机变量\(X\)与\(Y\)从不同角度刻画同一随机试验,
因此需要作为一个整体研究二维随机变量\((X,Y)\)的统计规律性,
这就是\((X,Y)\)的分布函数.

\begin{definition}
%@see: 《概率论与数理统计》(陈鸿建、赵永红、翁洋) P64 定义3.2
设\((X,Y)\)是二维随机变量,
对任意实数\(x,y\),
把\begin{equation}\label{equation:多维随机变量及其分布.二维分布函数的定义式}
%@see: 《概率论与数理统计》(陈鸿建、赵永红、翁洋) P64 (3.1.1)
	F(x,y) = P(X \leq x, Y \leq y)
\end{equation}
称为“二维随机变量\((X,Y)\)的\DefineConcept{二维分布函数}”
或“\(X\)与\(Y\)的\DefineConcept{联合分布函数}(joint distribution function)”.
%@see: https://mathworld.wolfram.com/JointDistributionFunction.html
\end{definition}
二维分布函数\(F(x,y)\)表示事件\((X \leq x)\)与\((Y \leq y)\)同时发生的概率.
如果将\((X,Y)\)看为平面上的随机点坐标,
则\(F(x,y)\)为\((X,Y)\)
落在广义矩形区域\(G=\Set{(u,v) \given u \leq x, v \leq y}\)上的概率.

特别地,\((X,Y)\)落在矩形区域\((x_1,x_2]\times(y_1,y_2]\)上的概率为
\begin{equation}
%@see: 《概率论与数理统计》(陈鸿建、赵永红、翁洋) P64 (3.1.2)
	P(x_1 < X \leq x_2, y_1 < Y \leq y_2)
	= F(x_2,y_2) - F(x_2,y_1) - F(x_1,y_2) + F(x_1,y_1).
\end{equation}

与一维随机变量的分布函数类似,二维分布函数\(F(x,y)\)具有如下性质:
\begin{property}
%@see: 《概率论与数理统计》(陈鸿建、赵永红、翁洋) P64 定理3.1
设\(F(x,y)\)为随机变量\((X,Y)\)的分布函数,则
\begin{enumerate}
	\item \(F(x,y)\)分别关于\(x\)及\(y\)单调不减,即
	当\(x_1 < x_2\)时,有\(F(x_1,y) \leq F(x_2,y)\);
	当\(y_1 < y_2\)时,有\(F(x,y_1) \leq F(x,y_2)\).

	\item \(F(-\infty,-\infty)=F(-\infty,y)=F(x,-\infty)=0\),
	\(F(+\infty,+\infty)=1\).

	\item \(F(x,y)\)关于\(x\)及\(y\)都右连续,
	即对任意实数\(x\)、\(y\)有
	\(F(x^+,y)=F(x,y)\)和\(F(x,y^+)=F(x,y)\)成立.

	\item 对任意\(x_1 < x_2\)、\(y_1 < y_2\)有\begin{equation*}
		P(x_1 < X \leq x_2, y_1 < Y \leq y_2)
		= F(x_2,y_2) - F(x_2,y_1) - F(x_1,y_2) + F(x_1,y_1)
		\geq 0.
	\end{equation*}
\end{enumerate}
\end{property}

需要指出,上述四点也是二维分布函数的特征,也就是说,
任何一个二元函数只要满足这四点就是某二维随机变量的分布函数.

\begin{example}
%@see: 《概率论与数理统计》(陈鸿建、赵永红、翁洋) P65 例3.1
考虑二元函数\begin{equation*}
	G(x,y) = \left\{ \begin{array}{cl}
		1, & x+y\geq0, \\
		0, & x+y<0.
	\end{array} \right.
\end{equation*}
由于\begin{equation*}
	G(1,1)-G(1,-0.5)-G(-0.5,1)+G(-0.5,-0.5)=-1,
\end{equation*}
所以\(G\)不是二维分布函数.
\end{example}

\subsection{二维离散型随机变量及其概率分布的概念与性质}
\begin{definition}
%@see: 《概率论与数理统计》(陈鸿建、赵永红、翁洋) P65
如果二维随机变量\((X,Y)\)只取有限个或可数无穷个点对\((x_i,y_i)\ (i,j=1,2,\dotsc)\),
则称\((X,Y)\)为\DefineConcept{二维离散型随机变量}.
\end{definition}

\begin{definition}
%@see: 《概率论与数理统计》(陈鸿建、赵永红、翁洋) P65 定义3.3
设二维离散型随机变量\((X,Y)\)所有可能取值为\((x_i,y_i)\ (i,j=1,2,\dotsc)\).
把\begin{equation*}
	p_{ij} = P(X = x_i, Y = y_j)
	\quad(i,j = 1,2,\dotsc)
\end{equation*}称为“\((X,Y)\)的\DefineConcept{二维概率分布}”,
“\((X,Y)\)的\DefineConcept{二维分布律}”,
或“\(X\)与\(Y\)的\DefineConcept{联合概率分布}”.
\end{definition}

二维概率分布可以用\cref{table:多维随机变量及其分布.二维概率分布} 表示.

\begin{table}[htb]
	\centering
	\begin{tblr}{c|*5c}
		\diagbox{$X$}{$Y$}
			& \(y_1\) & \(y_2\) & \(\dots\) & \(y_j\) & \(\dots\) \\ \hline
		\(x_1\) & \(p_{11}\) & \(p_{12}\) & \(\dots\) & \(p_{1j}\) & \(\dotsc\) \\
		\(x_2\) & \(p_{21}\) & \(p_{22}\) & \(\dots\) & \(p_{2j}\) & \(\dotsc\) \\
		\(\vdots\) & \(\vdots\) & \(\vdots\) & & \(\vdots\) \\
		\(x_i\) & \(p_{i1}\) & \(p_{i2}\) & \(\dots\) & \(p_{ij}\) & \(\dotsc\) \\
		\(\vdots\) & \(\vdots\) & \(\vdots\) & & \(\vdots\) \\
	\end{tblr}
	\caption{\((X,Y)\)的二维概率分布}
	\label{table:多维随机变量及其分布.二维概率分布}
\end{table}

\begin{property}
%@see: 《概率论与数理统计》(陈鸿建、赵永红、翁洋) P66
二维离散型随机变量的概率分布有如下的性质:
\begin{enumerate}
	\item {\rm\bf 非负性}:
	\(p_{ij} \geq 0\ (i,j=1,2,\dotsc)\);

	\item {\rm\bf 规范性}:
	\(\sum_{i,j} p_{ij} = 1\).
\end{enumerate}
\end{property}

\begin{theorem}
对于任意一个二维点集\(G\),
对任意二维离散型随机变量\((X,Y)\)可以求事件\(((X,Y) \in G)\)的概率,
即\begin{equation*}
%@see: 《概率论与数理统计》(陈鸿建、赵永红、翁洋) P66 (3.1.4)
	P\left[(X,Y) \in G\right] = \sum_{(x_i,y_j) \in G} p_{ij}.
\end{equation*}

特别地,二维离散型随机变量\((X,Y)\)的二维分布函数可用概率分布求出,即\begin{equation*}
%@see: 《概率论与数理统计》(陈鸿建、赵永红、翁洋) P66 (3.1.5)
	F(x,y) = \sum_{x_i \leq x}\sum_{y_j \leq y} p_{ij},
\end{equation*}且有\begin{equation*}
%@see: 《概率论与数理统计》(陈鸿建、赵永红、翁洋) P66 (3.1.6)
	p_{ij} = F(x_i,y_j) - F(x_i,y_{j-1}) - F(x_{i-1},y_j) + F(x_{i-1},y_{j-1}),
	\quad i,j = 1,2,\dotsc,
\end{equation*}
其中,规定\(x_0 = y_0 = -\infty\).
\end{theorem}

\subsection{常见的二维离散型分布}
\subsubsection{三项分布}
\begin{definition}
%@see: 《概率论与数理统计》(陈鸿建、赵永红、翁洋) P67
在\(n\)重独立试验中,若每次试验只有\(A_1\)、\(A_2\)、\(A_3\)三个可能结果,
且\(0 < p_i = P(A_i) < 1\ (i=1,2,3)\),则\(p_1 + p_2 + p_3 = 1\).
令随机变量\(X\)及\(Y\)分别表示\(n\)次试验中\(A_1\)与\(A_2\)发生的次数,
则\(X\)与\(Y\)的联合概率分布为\begin{equation*}
%@see: 《概率论与数理统计》(陈鸿建、赵永红、翁洋) P67 (3.1.7)
	P(X=k_1,Y=k_2)
	= \frac{n!}{k_1! k_2! (n-k_1-k_2)!} p_1^{k_1} p_2^{k_2} p_3^{n-k_1-k_2},
\end{equation*}
其中\(k_1+k_2 = 0,1,\dotsc,n\),\(k_1 \geq 0\),\(k_2 \geq 0\),
并称“\((X,Y)\)服从参数为\(p _1,p_2,n\)的\DefineConcept{三项分布}”,
记为\((X,Y) \sim T(n;p_1,p_2)\).
\end{definition}

\subsection{二维连续型随机变量及其密度函数的概念与性质}
\begin{definition}
%@see: 《概率论与数理统计》(陈鸿建、赵永红、翁洋) P68 定义3.4
设二维随机变量\((X,Y)\)有分布函数\(F(x,y)\),
如果存在二元非负函数\(f(x,y)\),
使得对任意实数\(x,y\)有\begin{equation*}
%@see: 《概率论与数理统计》(陈鸿建、赵永红、翁洋) P68 (3.1.8)
	F(x,y) = \int_{-\infty}^x \int_{-\infty}^y f(u,v) \dd{u} \dd{v},
\end{equation*}
则称“\((X,Y)\)是\DefineConcept{二维连续型随机变量}”,
称“\(f(x,y)\)是\((X,Y)\)的\DefineConcept{二维概率密度函数}”,
或“\(f(x,y)\)是\(X\)与\(Y\)的\DefineConcept{联合密度函数}”.
\end{definition}

\begin{property}
%@see: 《概率论与数理统计》(陈鸿建、赵永红、翁洋) P68
二维连续型随机变量的密度函数有如下的性质:
\begin{enumerate}
	\item {\rm\bf 非负性}:
	\((\forall (x,y)\in\mathbb{R}^2)[f(x,y) \geq 0]\);

	\item {\rm\bf 规范性}:
	\(F(+\infty,+\infty)
	= \int_{-\infty}^{+\infty} \int_{-\infty}^{+\infty} f(x,y) \dd{x} \dd{y} = 1\).
\end{enumerate}
\end{property}

\begin{theorem}
%@see: 《概率论与数理统计》(陈鸿建、赵永红、翁洋) P68 定理3.2
设二维连续型随机变量\((X,Y)\)有密度函数\(f(x,y)\),则
\begin{enumerate}
	\item \(F(x,y)\)是连续函数且在\(f(x,y)\)的连续点\((x,y)\),
	有\begin{equation*}
	%@see: 《概率论与数理统计》(陈鸿建、赵永红、翁洋) P68 (3.1.10)
		f(x,y) = \pdv{F(x,y)}{x}{y};
	\end{equation*}

	\item 对平面上任意区域\(G \subseteq \mathbb{R}^2\),
	若\(f(x,y)\)在\(G\)上可积,
	有\begin{equation*}
	%@see: 《概率论与数理统计》(陈鸿建、赵永红、翁洋) P68 (3.1.11)
		P\left[(X,Y) \in G\right] = \iint_G{f(x,y) \dd{x}\dd{y}};
	\end{equation*}

	\item 对平面上任一条曲线\(L\),有\begin{equation*}
		P\left[(X,Y) \in L\right] = 0.
	\end{equation*}
\end{enumerate}
\end{theorem}

\subsection{常见的二维连续型分布}
\subsubsection{均匀分布}
\begin{definition}
令\(G\)是平面上一个有界区域,若二维随机变量\((X,Y)\)有密度函数\begin{equation*}
%@see: 《概率论与数理统计》(陈鸿建、赵永红、翁洋) P68 (3.1.12)
	f(x,y) = \left\{ \begin{array}{ll}
		\frac{1}{m(G)}, & (x,y) \in G, \\
		0, & \text{其他}, \\
	\end{array} \right.
\end{equation*}
其中\(m(G)\)为\(G\)的面积,
则称“\((X,Y)\)服从在\(G\)上的\DefineConcept{均匀分布}”,
记为\((X,Y) \sim U(G)\).
\end{definition}

\subsubsection{二维正态分布}
\begin{definition}
%@see: 《概率论与数理统计》(陈鸿建、赵永红、翁洋) P145 定义5.3
设二维随机变量\((X,Y)\)有二维密度函数
\begin{equation}
	f(x,y) = \frac{1}{2\pi\sigma_1\sigma_2\sqrt{1-r^2}}
		\exp\left[- u\left(
			\frac{x-\mu_1}{\sigma_1},
			\frac{y-\mu_2}{\sigma_2}
		\right)\right]
	\quad(x,y)\in\mathbb{R}^2,
\end{equation}
其中\begin{equation*}
	u(x,y)
	= \frac{1}{2(1-r^2)}
	\begin{bmatrix}
		x & y
	\end{bmatrix}
	\begin{bmatrix}
		1 & -r \\
		-r & 1
	\end{bmatrix}
	\begin{bmatrix}
		x \\ y
	\end{bmatrix},
\end{equation*}
\(\mu_1\in\mathbb{R},
\mu_2\in\mathbb{R},
\sigma_1\in\mathbb{R}^+,
\sigma_2\in\mathbb{R}^+,
r\in(-1,1)\)是参数,
则称“\((X,Y)\)服从\DefineConcept{二维正态分布}”,
记为\((X,Y) \sim N(\mu_1,\mu_2;\sigma_1^2,\sigma_2^2;r)\).
\end{definition}

\begin{theorem}\label{theorem:正态分布与自然指数分布族.性质1}
%@see: 《概率论与数理统计》(陈鸿建、赵永红、翁洋) P145 定理5.6
若\((X,Y) \sim N(\mu_1,\mu_2;\sigma_1^2,\sigma_2^2;r)\),
则对应的边缘分布均为正态分布,且\begin{equation*}
	X \sim N(\mu_1,\sigma_1^2),
	\qquad
	Y \sim N(\mu_2,\sigma_2^2).
\end{equation*}
\begin{proof}
首先有\begin{align*}
	f_X(x) = \int_{-\infty}^{+\infty} f(x,y) \dd{y}
	= \frac{1}{2\pi\sigma_1\sigma_2\sqrt{1-r^2}}
		\int_{-\infty}^{+\infty} e^{-u(x,y)} \dd{y},
\end{align*}
其中\begin{align*}
	u(x,y)
	&= \frac{1}{2(1-r^2)} \left[
			\frac{(x-\mu_1)^2}{\sigma_1^2}
			-2r\frac{(x-\mu_1)(y-\mu_2)}{\sigma_1\sigma_2}
			+\frac{(y-\mu_2)^2}{\sigma_2^2}
		\right] \\
	&= \frac{1}{2 \sigma_1^2} (x-\mu_1)^2
		+ \frac{1}{2(1-r^2)} \left[
			\frac{y-\mu_2}{\sigma_2}
			- \frac{r(x-\mu_1)^2}{\sigma_1}
		\right]^2.
\end{align*}
令\begin{equation*}
	t = \frac{1}{\sqrt{1-r^2}} \left[
		\frac{y-\mu_2}{\sigma_2}
		- \frac{r(x-\mu_1)}{\sigma_1}
	\right],
\end{equation*}
则有\begin{equation*}
	f_X(x)
	= \frac{1}{\sqrt{2\pi} \sigma_1} e^{-\frac{(x-\mu_1)^2}{2\sigma_1^2}} \int_{-\infty}^{+\infty} \frac{1}{\sqrt{2\pi}} e^{-\frac{t^2}{2}} \dd{t} \\
	= \frac{1}{\sqrt{2\pi} \sigma_1} e^{-\frac{(x-\mu_1)^2}{2\sigma_1^2}}
	\quad(x\in\mathbb{R}).
\end{equation*}
同理可得\begin{equation*}
	f_Y(y)
	= \frac{1}{\sqrt{2\pi} \sigma_2} e^{-\frac{(y-\mu_2)^2}{2\sigma_2^2}}
	\quad(y\in\mathbb{R}).
	\qedhere
\end{equation*}
\end{proof}
\end{theorem}

\begin{example}\label{theorem:正态分布与自然指数分布族.性质4}
%@see: 《概率论与数理统计》(陈鸿建、赵永红、翁洋) P147 例5.7
设二维随机变量\((X,Y) \sim N(\mu_1,\mu_2;\sigma_1^2,\sigma_2^2;r)\).
求条件密度\(f_{X \vert Y}(x \vert y)\).
\begin{solution}
\def\A{\frac{1}{\sqrt{2\pi}\sigma_1\sqrt{1-r^2}}}%
\def\B{\frac{1}{2(1-r^2)}}%
直接计算得\begin{align*}
	&f_{X \vert Y}(x \vert y) = \frac{f(x,y)}{f_Y(y)} \\
	&= \A
		\exp\Biggl\{
			- \B
			\left[
				\frac{(x-\mu_1)^2}{\sigma_1^2}
				- 2r\frac{(x-\mu_1)(y-\mu_2)}{\sigma_1\sigma_2}
				+ r^2\frac{(y-\mu_2)^2}{\sigma_2^2}
			\right]
		\Biggr\} \\
	&= \A
		\exp\left\{
			- \B
			\left[
				\frac{x-\mu_1}{\sigma_1}
				- r\frac{y-\mu_2}{\sigma_2}
			\right]^2
		\right\} \\
	&= \A
		\exp\left\{
			- \B
			\frac{1}{\sigma_1^2}
			\left[
				x - \mu_1
				- r\frac{\sigma_1}{\sigma_2}(y-\mu_2)
			\right]^2
		\right\}.
\end{align*}
这个条件密度函数恰好是期望为\(\mu_1+r\frac{\sigma_1}{\sigma_2}(y-\mu_2)\),
方差为\(\sigma_1^2(1-r^2)\)的正态分布的密度函数.
\end{solution}
\end{example}
\begin{remark}
\cref{theorem:正态分布与自然指数分布族.性质4} 说明:
二维正态分布的条件分布仍然是正态分布.
\end{remark}

\begin{theorem}\label{theorem:正态分布与自然指数分布族.二维随机变量服从二维正态分布的充分必要条件}
%@see: 《概率论与数理统计》(陈鸿建、赵永红、翁洋) P147 定理5.8
二维随机变量\((X,Y)\)服从二维正态分布的充分必要条件是:
\(X\)与\(Y\)的任意非零线性组合\(Z = a X + b Y\)服从一维正态分布\(N(E(Z),D(Z))\).
%TODO proof
\end{theorem}

\begin{example}
%@see: 《概率论与数理统计》(陈鸿建、赵永红、翁洋) P147 例5.8
设\((X,Y) \sim N\left(2,3;4,9;\frac{1}{2}\right)\),
\(Z = \frac{1}{2} X - \frac{1}{3} Y\),
求\(E(\abs{Z})\).
\begin{solution}
由\cref{theorem:正态分布与自然指数分布族.性质1} 有\begin{equation*}
	X \sim N(2,4), \qquad
	Y \sim N(3,9),
\end{equation*}且相关系数\(R(X,Y) = 1/2\),
于是\begin{equation*}
	\Cov(X,Y) = R(X,Y) \sqrt{D(X)} \sqrt{D(Y)} = 3.
\end{equation*}\begin{equation*}
	E(Z) = \frac{1}{2} E(X) - \frac{1}{3} E(Y) = 0.
\end{equation*}\begin{equation*}
	D(Z) = \frac{1}{4} D(X) + \frac{1}{9} D(Y)
		- 2 \cdot \frac{1}{2} \cdot \frac{1}{3} \Cov(X,Y)
	= 1.
\end{equation*}
由\cref{theorem:正态分布与自然指数分布族.二维随机变量服从二维正态分布的充分必要条件}
有\begin{align*}
	E(\abs{Z})
	&= \int_{-\infty}^{+\infty} \abs{z} \frac{1}{\sqrt{2\pi}} e^{-\frac{z^2}{2}} \dd{z}
	= \frac{2}{\sqrt{2\pi}} \int_0^{+\infty} z e^{-\frac{z^2}{2}} \dd{z} \\
	&= \frac{\sqrt{2}}{\sqrt{\pi}} \int_0^{+\infty} \dd(-e^{-\frac{z^2}{2}})
	= \sqrt{\frac{2}{\pi}} \left(-e^{-\frac{z^2}{2}}\right)_0^{+\infty}
	= \sqrt{\frac{2}{\pi}}.
\end{align*}
\end{solution}
\end{example}

\begin{example}
设随机变量\(X\)、\(Y\)相互独立,且\(X \sim U(0,1)\),\(Y \sim e(1/2)\).
求关于\(a\)的一元二次方程\(a^2 + 2aX + Y = 0\)有实根的概率.
\begin{solution}
根据均匀分布和指数分布的定义,\begin{equation*}
	f_X(x) = \left\{ \begin{array}{cl}
		1, & x\in(0,1), \\
		0, & \text{其他};
	\end{array} \right.
	\qquad
	f_Y(y) = \left\{ \begin{array}{cl}
		\frac{1}{2} e^{-\frac{1}{2} y}, & y>0, \\
		0, & y \leq 0.
	\end{array} \right.
\end{equation*}
因为随机变量\(X\)、\(Y\)相互独立,
所以\(X\)与\(Y\)的联合密度函数为\begin{equation*}
	f(x,y) = f_X(x) \cdot f_Y(y)
	= \left\{ \begin{array}{cl}
		\frac{1}{2} e^{-\frac{1}{2} y}, & 0<x<1 \land y>0, \\
		0, & \text{其他}.
	\end{array} \right.
\end{equation*}

一元二次方程有实根的概率为\begin{align*}
	P[(2X)^2 - 4Y \geq 0]
	&= P(X^2 \geq Y)
	= \int_0^1 \dd{x} \int_0^{x^2} \frac{1}{2} e^{-\frac{1}{2} y} \dd{y} \\
	&= 1 - \sqrt{2\pi} \left[ \Phi(1) - \frac{1}{2} \right]
	\approx 0.144~376.
\end{align*}
\end{solution}
\end{example}

\begin{example}
设二维随机变量\((X,Y)\)的概率密度为\begin{equation*}
	f(x,y) = A e^{-2x^2+2xy-y^2}, \quad x,y\in\mathbb{R},
\end{equation*}
求常数\(A\)及条件概率密度\(f_{Y \vert X}(y \vert x)\).
\begin{solution}
先求\(X\)的边缘密度函数,有\begin{align*}
	f_X(x) &= \int_{-\infty}^{+\infty} f(x,y) \dd{y} \\
	&= \int_{-\infty}^{+\infty} A e^{-2x^2+2xy-y^2} \dd{y} \\
	&= A e^{-x^2} \int_{-\infty}^{+\infty} e^{-(y-x)^2} \dd{y} \\
	&= A e^{-x^2} \sqrt{\pi} \int_{-\infty}^{+\infty}
		\frac{1}{\sqrt{2\pi} \sqrt{\frac{1}{2}}}
		e^{\frac{-(y-x)^2}{2 \cdot \frac{1}{2}}} \dd{y} \\
	&= A \sqrt{\pi} e^{-x^2}.
\end{align*}
由\hyperref[theorem:随机变量及其分布.连续型随机变量的密度函数的性质]{规范性}%
和重要积分公式 \labelcref{equation:重积分.常用积分2} 可知\begin{equation*}
	\int_{-\infty}^{+\infty} f_X(x) \dd{x}
	= A \sqrt{\pi} \int_{-\infty}^{+\infty} e^{-x^2} \dd{x}
	= A \pi = 1,
\end{equation*}
因此,\(A = \frac{1}{\pi}\).
那么根据\cref{equation:多维随机变量及其分布.条件密度、联合密度、边缘密度的关系2} 有\begin{equation*}
	f_{Y \vert X}(y \vert x)
	= \frac{f(x,y)}{f_X(x)}
	= \frac{\frac{1}{\pi} e^{-2x^2+2xy-y^2}}{\frac{1}{\pi} \sqrt{\pi} e^{-x^2}}
	= \frac{1}{\sqrt{\pi}} e^{-(x-y)^2},
	\quad y\in\mathbb{R}.
\end{equation*}
\end{solution}
\end{example}
