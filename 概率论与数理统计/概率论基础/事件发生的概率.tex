\section{事件发生的概率}
\subsection{频率的概念及性质}
对于一个事件,除去必然事件与不可能事件外,
它在一次试验中有可能发生,
也有可能不发生.
为了揭示这些事件内在的统计规律性,
往往需要知道这些事件在一次试验中发生的可能性的大小,
以便更好地认识客观事物.
比如医学工作者在研制一种新药的过程中,
需要做临床试验测试其是否有效,
可否投入临床使用.

为了刻画事件在一次试验中发生的可能性,我们首先引入频率的概念.

\begin{definition}
在\(n\)次重复试验中,
若事件\(A\)发生了\(k\)次,
则称\(k\)为事件\(A\)发生的频数,
称\(\frac{k}{n}\)为事件\(A\)发生的频率,
记作\(f_n(A)\),
即\begin{equation*}
	f_n(A) = \frac{k}{n}.
\end{equation*}
\end{definition}

\begin{property}
由定义可知,频率有如下性质:
\begin{enumerate}
	\item \(0 \leq f_n(A) \leq 1\);

	\item \(f_n(\Omega) = 1\),\(f_n(\emptyset) = 0\);

	\item 若\(A_1,A_2,\dotsc,A_r\)为\(r\)个两两互斥的事件,
	则\begin{equation*}
		f_n\left( \bigcup_{i=1}^r A_i \right)
		= \sum_{i=1}^r f_n(A_i).
	\end{equation*}
\end{enumerate}
\end{property}

由于事件\(A\)在\(n\)次试验中发生的频率是
\(A\)发生的频数与试验次数\(n\)之比,
频率大小表示了\(A\)发生的频繁程度.
频率越大,事件\(A\)在\(n\)次试验中发生得越频繁,
就意味着\(A\)在一次试验中发生的可能性越大.
因此,频率在一定程度上可以反映事件发生可能性的大小.

但是,另一方面频率具有不客观性.我们来看下面列出的数据:
\begin{example}
历史上,许多著名的统计学家做过“抛硬币”试验,得到如下数据:
\begin{center}
	\begin{tblr}{c|c|c|c}
	\hline
	试验者 & 抛硬币次数\(n\) & 正面朝上次数\(n_A\) & 频率\(f_n(A)\) \\ \hline
	Buffon & 4 040 & 2 048 & 0.506 9 \\
	Fisher & 10 000 & 4 979 & 0.497 9 \\
	Pearson & 12 000 & 6 019 & 0.501 6 \\
	Pearson & 24 000 & 12 012 & 0.500 5 \\ \hline
	\end{tblr}
\end{center}
\end{example}
可以看出,频率具有波动性.
当试验次数\(n\)不同时,
频率不相同(事实上,即便试验次数\(n\)相同,
不同的实验者得到的频率也未必相同).
进一步仔细观察这两组数据可以发现,
当\(n\)较小时,频率波动较大;
而当\(n\)较大时,频率波动越来越小,
且频率总稳定在一个客观数量附近,
例如“抛硬币”试验中频率的稳定值是\(0.5\).

\subsection{概率的公理化定义及性质}
在实践中,我们通常不可能,
也无必要对每个事件做大量的试验来获取频率的稳定值.
历史上,数学家是在不同的概率模型下给出事件概率的计算公式,
再抽象地公理化地定义概率.

\begin{definition}
设\(\Omega\)为一个试验的样本空间.
对\(\Omega\)中任意一个事件\(A\),
对应一个实数\(P(A)\).
若这个集合函数\(P\)满足以下三个条件,
则称“\(P(A)\)是事件\(A\)发生的\DefineConcept{概率}(probability)”:
\begin{enumerate}
	\item 非负性:
	\begin{equation}
	P(A) \geq 0;
	\end{equation}

	\item 规范性:
	\begin{equation}
	P(\Omega) = 1;
	\end{equation}

	\item 可列可加性:
	若\(A_1,A_2,\dotsc,A_n,\dotsc\)可列个两两互斥的事件,
	则\begin{equation}
		P\left(\bigcup_{i=1}^\infty A_i\right)
		= \sum_{i=1}^\infty P(A_i).
	\end{equation}
\end{enumerate}
\end{definition}
这个概率的公理化定义是苏联科学家柯尔莫哥洛夫在1933年给出的.

由概率的定义,可得概率有如下性质:
\begin{property}
\begin{equation}
	P(\emptyset) = 0.
\end{equation}
\end{property}

\begin{property}[有限可加性]
若\(n\)个事件\(A_1,A_2,\dotsc,A_n\)两两互斥,
则\begin{equation}
	P\left(\bigcup_{i=1}^n A_i\right)
	= \sum_{i=1}^n P(A_i).
\end{equation}
\end{property}

\begin{property}
\begin{equation}
	P(\overline{A}) = 1 - P(A).
\end{equation}
\end{property}

\begin{property}[概率的减法]
\begin{equation}
	P(A - B) = P(A) - P(AB).
\end{equation}

特别地,若\(B \subseteq A\),有
\begin{equation}
	P(A - B) = P(A) - P(B),
\end{equation}
且
\begin{equation}
	P(A) \geq P(B).
\end{equation}
\begin{proof}
由事件\(A\)满足:\begin{equation*}
	A = A \Omega
	= A(B+\overline{B})
	= AB+A\overline{B},
\end{equation*}
故\begin{equation*}
	P(A) = P(AB)+P(A\overline{B}),
\end{equation*}
进而有\begin{equation*}
	P(A-B) = P(A\overline{B}) = P(A) - P(AB).
\end{equation*}

当\(B \subseteq A\)时,
\(B = BB \subseteq AB\);
又由\(AB \subseteq B\),
所以\(AB = B\),
进而有\begin{equation*}
	P(A-B) = P(A) - P(B).
	\qedhere
\end{equation*}
\end{proof}
\end{property}

\begin{property}
对任意事件\(A\),
有\begin{equation}
	P(A) \leq 1.
\end{equation}
\end{property}

\begin{theorem}[概率的加法]
对任意两事件\(A,B\),
有\begin{equation}
	P(A \cup B) = P(A) + P(B) - P(AB).
\end{equation}
\end{theorem}

\begin{corollary}
对任意三事件\(A,B,C\),
有\begin{equation}
	P(A \cup B \cup C)
	= P(A) + P(B) + P(C)
	- P(AB) - P(AC) - P(BC)
	+ P(ABC).
\end{equation}
\end{corollary}

\begin{theorem}
对任意两事件\(A,B\),
有\begin{equation}
	P(A \cup B) \leq P(A) + P(B).
\end{equation}
\end{theorem}

\begin{corollary}[布尔不等式]
对任意多个事件\(A_i\ (i=1,2,\dotsc)\),
不等式\begin{equation}\label{equation:概率论基础.布尔不等式}
	P\left(\bigcup_i A_i\right)
	\leq
	\sum_i P(A_i)
\end{equation}
成立,
当且仅当“\(A_1,A_2,\dotsc\)两两互斥”时取“\(=\)”.
\end{corollary}
