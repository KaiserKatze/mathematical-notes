\section{条件概率}
\subsection{条件概率的概念}
\begin{definition}\label{definition:条件概率.条件概率的概念}
设\(A,B\)是样本空间\(\Omega\)中的两个事件,
且\(P(B) > 0\),
我们把\begin{equation*}
	\frac{P(AB)}{P(B)}
\end{equation*}称为“在事件\(B\)已发生的条件下,事件\(A\)发生的概率”,
或简称“\(A\)对\(B\)的\DefineConcept{条件概率}”,
记作\(P(A \vert B)\),
即\begin{equation}
	P(A \vert B)
	\defeq
	\frac{P(AB)}{P(B)}.
\end{equation}
\end{definition}

\begin{property}
条件概率有如下性质:
\begin{enumerate}
	\item \(P(A \vert B) \geq 0\);

	\item \(P(\Omega \vert B) = 1\);

	\item 若事件\(A_1,A_2,\dotsc,A_n,\dotsc\)两两互斥,
	则\begin{equation*}
		P\left(\bigcup_{i=1}^\infty A_i \Bigg\vert B\right)
		= \sum_{i=1}^\infty {P(A_i \vert B)}.
	\end{equation*}
\end{enumerate}
\begin{proof}
性质3推导如下:
当\(A_1,A_2,\dotsc,A_n,\dotsc\)两两互斥的,
则\(A_1 B,A_2 B,\dotsc,A_n B,\dotsc\)也两两互斥.
从而由条件概率定义及概率的可列可加性,
有\begin{align*}
	P\left(\bigcup_{i=1}^\infty A_i \Bigg\vert B\right)
	&= \frac{1}{P(B)} P\left[\left(\displaystyle\bigcup_{i=1}^\infty A_i\right) B\right]
	= \frac{1}{P(B)} P\left(\displaystyle\bigcup_{i=1}^\infty {A_i B}\right) \\
	&= \sum_{i=1}^\infty {\frac{P(A_i B)}{P(B)}}
	= \sum_{i=1}^\infty {P(A_i \vert B)}.
	\qedhere
\end{align*}
\end{proof}
\end{property}

当事件\(B\)给定时,
\(P(A \vert B)\)是事件\(A\)的集合函数.
而上述三个性质正好是概率定义中三个公理化条件,
于是条件概率也是概率.
这样,条件概率满足概率的一切性质,如:

\begin{property}
\(P(\SetComplementaryL{A} \vert B)
= 1 - P(A \vert B)\).
\end{property}

\begin{property}
\(P(A \cup B \vert C)
= P(A \vert C) + P(B \vert C) - P(AB \vert C)\).
\end{property}

\begin{example}
历史资料表明,
某地区从某次特大洪水发生以后在30年内发生特大洪水的概率为80\%,
在40年内发生特大洪水的概念为85\%.
问现在已30年无特大洪水的该地区,
在未来10年内将发生特大洪水的概率是多少?
\begin{solution}
设\(A\)表示“该地区从某次特大洪水后30年内无特大洪水”,
\(B\)表示“该地区从某次特大洪水后40年内无特大洪水”,
则\(P(A) = 0.2\),\(P(B) = 0.15\).
因为\(B \subseteq A\)使得\(AB = B\),
那么所求概率为\begin{equation*}
	P(\SetComplementaryL{B} \vert A)
	= 1 - P(B \vert A)
	= 1 - \frac{P(AB)}{P(A)}
	= 1 - \frac{P(B)}{P(A)}
	= 1 - \frac{0.15}{0.2}
	= 0.25.
\end{equation*}
\end{solution}
\end{example}

\subsection{乘法公式}
\begin{theorem}
设\(A,B\)是\(2\)个事件,
且\(P(A) > 0\),
则有\begin{equation}\label{equation:概率论.条件概率.概率的乘法公式1}
	P(AB)
	= P(A) P(B \vert A)
	= P(B) P(A \vert B).
\end{equation}
\end{theorem}
我们把\cref{equation:概率论.条件概率.概率的乘法公式1}
称为\DefineConcept{乘法公式}.

\begin{corollary}
设\(A,B\)是\(2\)个事件,
且\(P(A) > 0\),\(P(B) > 0\),
则有\begin{equation}
	P(A) P(B \vert A)
	= P(B) P(A \vert B).
\end{equation}
\end{corollary}

\begin{corollary}
设\(A,B,C\)是\(3\)个事件,
且\(P(B C) > 0\),
则有\begin{equation}%\label{equation:概率论.条件概率.概率的乘法公式2}
	P(A B \vert C)
	= P(A \vert B C) P(B \vert C).
\end{equation}
\end{corollary}

\begin{corollary}
设\(A_1,A_2,\dotsc,A_n\)是\(n\)个事件,
且\(P(A_1 A_2 \dotsm A_{n-1}) > 0\),
则有\begin{equation}%\label{equation:概率论.条件概率.概率的乘法公式3}
	P(A_1 A_2 \dotsm A_n)
	= P(A_1) P(A_2 \vert A_1) P(A_3 \vert A_1 A_2) \dotsm P(A_n \vert A_1 A_2 \dotsm A_{n-1}).
\end{equation}
\end{corollary}

\begin{example}
一个小组有10名同学,其中4名女同学.
每周依次有一名同学收作业.
求第一、二周是女同学,
第三、四周是男同学收作业的概率.
\begin{solution}
设\(A_i = \Set{\text{第\(i\)周由女同学收作业}}\ (i=1,2,3,4)\),
则有\begin{align*}
	P(A_1 A_2 \SetComplementaryL{A}_3 \SetComplementaryL{A}_4)
	&= P(A_1) P(A_2 \vert A_1)
	P(\SetComplementaryL{A}_3 \vert A_1 A_2)
	P(\SetComplementaryL{A}_4 \vert A_1 A_2 \SetComplementaryL{A}_3) \\
	&= \frac{4}{10} \times \frac{3}{9} \times \frac{6}{8} \times \frac{5}{7} = \frac{1}{14}.
\end{align*}
\end{solution}
\end{example}

\subsection{全概率公式与贝叶斯公式}
\begin{theorem}
设\(A_1,A_2,\dotsc,A_n\)为样本空间\(\Omega\)的一个完备事件组,
且\begin{equation*}
	P(A_i) > 0 \quad(i=1,2,\dotsc,n).
\end{equation*}
设\(B\)是任一事件,
则有\begin{itemize}
	\item {\rm\bf 全概率公式}:\begin{equation}\label{equation:条件概率.全概率公式}
		P(B) = \sum_{i=1}^n P(A_i) P(B \vert A_i);
	\end{equation}

	\item {\rm\bf 贝叶斯公式}:
	若\(P(B) > 0\),
	还有\begin{equation}\label{equation:条件概率.贝叶斯公式}
		P(A_i \vert B) = \frac{P(A_i) P(B \vert A_i)}{P(B)}
		= \frac{P(A_i) P(B \vert A_i)}{\sum_{j=1}^n P(A_j) P(B \vert A_j)},
		\quad i = 1,2,\dotsc,n.
	\end{equation}
\end{itemize}
\begin{proof}
全概率公式和贝叶斯公式的成立是显然的.
因为\(B = B \Omega
= B(A_1 \cup A_2 \cup \dotsb \cup A_n)
= A_1 B \cup A_2 B \cup \dotsb A_n B\),
所以\begin{equation*}
	P(B) = P\left(\bigcup_{i=1}^n A_i B\right)
	= \sum_{i=1}^n P(A_i B)
	= \sum_{i=1}^n P(A_i) P(B \vert A_i).
\end{equation*}

又因为\(P(A_i \vert B)
= \frac{P(A_i B)}{P(B)}\),
而\(P(B \vert A_i)
= \frac{P(A_i B)}{P(A_i)}\),
所以\begin{equation*}
	P(A_i \vert B) = \frac{P(A_i) P(B \vert A_i)}{P(B)}.
	\qedhere
\end{equation*}
\end{proof}
\end{theorem}

构成完备事件组的每个事件\(A_i\)的发生,
都有可能引起事件\(B\)的发生,
故可视\(A_i\)为引起事件\(B\)发生的“原因事件”,
\(B\)视为“结果事件”.
只要知道各“原因事件”发生的概率,
且知道“原因事件”\(A_i\)发生后引起“结果事件”\(B\)发生的条件概率,
则可求出\(B\)发生的概率.

如果我们\(P(A_i)\)称为“事件\(A_i\)的\DefineConcept{先验概率}(prior probability)”,
\(P(A_i \vert B)\)称为“事件\(A_i\)的\DefineConcept{后验概率}(posterior probability)”.
这就是当我们已知“结果事件”\(B\)发生后,
追究是由哪一个“原因事件”引发的概率,
从而由后验概率做出贝叶斯决策.
%@see: https://en.wikipedia.org/wiki/Prior_probability
%@see: https://en.wikipedia.org/wiki/Posterior_probability

在应用全概率公式与贝叶斯公式时,
选择完备事件组较为常见的情形有两种:
\begin{enumerate}
	\item 将某过程的第一个步骤的所有情况作为完备事件组;

	\item 将某先决事件\(A\)与\(\SetComplementaryL{A}\)作为完备事件组.
\end{enumerate}

\begin{example}
一盒中装有12个球,其中8个是新球.
第一次比赛从盒中任取2个球,
使用后放入盒中.
第二次比赛时,
再从盒中任取两球,
求\begin{enumerate}
	\item 第二次取出2个新球的概率;
	\item 已知第二次取出2个新球,而第一次仅取出一个新球的概率.
\end{enumerate}
\begin{solution}
设\(B\)表示“第二次取出2个新球”,
\(A_i\)表示“第一次取出2个球中有\(i\)个新球”,
\(i=0,1,2\),
则\(A_0,A_1,A_2\)是第一次取球的完备事件组,
且\begin{equation*}
	P(A_0) = \frac{C_4^2}{C_{12}^2} = \frac{1}{11},
	\quad
	P(A_1) = \frac{C_4^1 C_8^1}{C_{12}^2} = \frac{16}{33},
	\quad
	P(A_2) = \frac{C_8^2}{C_{12}^2} = \frac{14}{33},
	\end{equation*}\begin{equation*}
	P(B \vert A_0) = \frac{C_8^2}{C_{12}^2} = \frac{14}{33},
	\quad
	P(B \vert A_1) = \frac{C_7^2}{C_{12}^2} = \frac{7}{22},
	\quad
	P(B \vert A_2) = \frac{C_6^2}{C_{12}^2} = \frac{5}{22}.
\end{equation*}

由全概率公式,
有\begin{align*}
	P(B) &= P(A_0) P(B \vert A_0)
		+ P(A_1) P(B \vert A_1)
		+ P(A_2) P(B \vert A_2) \\
	&= \frac{1}{11} \times \frac{14}{33}
		+ \frac{16}{33} \times \frac{7}{22}
		+ \frac{14}{33} \times \frac{5}{22}
	= 0.289~3.
\end{align*}

由贝叶斯公式,有\begin{equation*}
	P(A_1 \vert B) = \frac{P(A_1) P(B \vert A_1)}{P(B)}
	= \frac{\frac{16}{33} \times \frac{7}{22}}{0.289~3}
	= 0.533~3.
\end{equation*}
\end{solution}
\end{example}
