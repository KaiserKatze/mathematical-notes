\section{事件的独立性及伯努利概型}
\subsection{事件的独立性}
从上一节可以发现,一般有\(P(A \vert B) \neq P(A)\),
即事件\(B\)发生后会影响事件\(A\)发生的概率,
但也有例外的情形.

例如一袋中有3个白球7个红球,有放回地取两次球,每次一球.
\(A_i\)表示“第\(i\)次取红球”,\(i=1,2\).
此时显然有\(P(A_2) = P(A_2 \vert A_1) = \frac{7}{10}\).
这表明\(A_1\)是否发生不影响\(A_2\)的发生.

一般地,如果\(P(B) > 0\),且\(P(A \vert B) = P(A)\)时,
则事件\(B\)的发生不影响事件\(A\)的发生.
若还有\(P(A) > 0\),
则必有\(P(B \vert A) = P(B)\).
这是因为\(P(A \vert B) = \frac{P(AB)}{P(B)} = P(A)\),
则有\(P(AB) = P(A) P(B)\),
从而\(P(B \vert A) = \frac{P(AB)}{P(A)} = P(B)\).

可见事件彼此间的影响是相互的.
事件\(A\)与\(B\)的某一个发生不影响另一个发生的情形
称为事件\(A\)与\(B\)相互独立,
这等价于\(P(AB) = P(A) P(B)\).
于是我们定义:

\begin{definition}
对同一试验的任意两事件\(A\)、\(B\),
若\begin{equation*}
	P(AB) = P(A) P(B),
\end{equation*}
则称“事件\(A\)与\(B\) \DefineConcept{相互独立}”.
\end{definition}

注意,“\(A\)与\(B\)相互独立”
和“\(A\)与\(B\)互斥”是两个不同的概念.
事实上,只要\(P(A) > 0\),\(P(B) > 0\),
“\(A\)与\(B\)相互独立”和“\(A\)与\(B\)互斥”这两种情况绝不会同时出现.
这是因为当“\(A\)与\(B\)相互独立”时,
\(P(AB) = P(A) P(B) > 0\),
这就与“\(A\)与\(B\)互斥”时
\(AB = \emptyset\),
\(P(AB) = 0\)相矛盾.

设试验的样本空间为\(\Omega\),
显然\(P(\Omega) = 1,
P(\emptyset) = 0\).
因为对于任意事件\(A\),
\(A \Omega = A\),
\(A \emptyset = \emptyset\),
那么\begin{equation*}
	P(A \Omega) = P(A) = P(A) P(\Omega), \qquad
	P(A \emptyset) = P(\emptyset) = P(A) P(\emptyset),
\end{equation*}
所以\(A\)与\(\Omega\)、\(A\)与\(\emptyset\)都相互独立.

\begin{theorem}
若事件\(A\)与\(B\)相互独立,
则\(A\)与\(\overline{B}\),
\(\overline{A}\)与\(B\),
\(\overline{A}\)与\(\overline{B}\)也相互独立.
\begin{proof}
因为\(A\overline{B}=A(\Omega-B)=A-AB\),
\(P(AB)=P(A)P(B)\),
所以\begin{equation*}
	P(A\overline{B})
	= P(A) - P(AB)
	= P(A) - P(A) P(B)
	= P(A) [1 - P(B)]
	= P(A) P(\overline{B}),
\end{equation*}
从而\(A\)与\(\overline{B}\)相互独立.
根据对称性易证\(\overline{A}\)与\(B\),
\(\overline{A}\)与\(\overline{B}\)也都相互独立.
\end{proof}
\end{theorem}

相互独立的概念可推广到多个事件的情形.
\begin{definition}
设\(A_1,A_2,\dotsc,A_n\ (n \geq 2)\)是\(n\)个事件,
若任取两事件\(A_i,A_j\ (i \neq j)\),
有\begin{equation*}
	P(A_i A_j) = P(A_i) P(A_j),
\end{equation*}
则称“这\(n\)个事件\DefineConcept{两两独立}”.
\end{definition}

\begin{definition}
设\(A_1,A_2,\dotsc,A_n\ (n \geq 2)\)是\(n\)个事件,
若对其中任意\(k\ (2 \leq k \leq n)\)个事件\(A_{i_1},A_{i_2},\dotsc,A_{i_k}\)
有\begin{equation*}
	P(A_{i_1} A_{i_2} \dotsm A_{i_k})
	= P(A_{i_1}) P(A_{i_2}) \dotsm P(A_{i_k}),
\end{equation*}
则称“这\(n\)个事件\DefineConcept{相互独立}”.
\end{definition}

\begin{theorem}
如果\(n\)个事件相互独立,
则必有这\(n\)个事件两两独立;反之不然.
\end{theorem}

由定义判断\(n\)个事件的相互独立性,
需要验证\(C_n^2+C_n^3+\dotsb+C_n^n=2^n-n-1\)个等式.
因此,在应用中,通常由实际意义判断事件的相互独立性.

概率论中,
通常把概率小于0.05的事件叫做\DefineConcept{小概率事件}.
小概率事件有两个特点:
第一,小概率事件在一次试验中几乎不可能发生;
第二,小概率事件在大量重复试验中几乎必定会至少发生一次.

\begin{example}
设\(0<P(B)<1\).
证明:\(A\)与\(B\)相互独立的充分必要条件为
\(P(A \vert B) = P(A \vert \overline{B})\).
\begin{proof}
必要性.
当\(A\)与\(B\)独立时,
有\(P(A \vert B) = P(A)\).
而此时,\(A\)与\(\overline{B}\)也独立,
有\(P(A \vert \overline{B}) = P(A)\).
因此\(P(A \vert B) = P(A \vert \overline{B})\).

充分性.
若\(P(A \vert B) = P(A \vert \overline{B})\),
有\begin{equation*}
	\frac{P(AB)}{P(B)}
	= \frac{P(A\overline{B})}{P(\overline{B})}
	= \frac{P(A)-P(AB)}{1-P(B)},
\end{equation*}\begin{equation*}
	P(AB) [1-P(B)] = P(B) [P(A)-P(AB)],
\end{equation*}\begin{equation*}
	P(AB) = P(A) P(B),
\end{equation*}
即\(A\)与\(B\)相互独立.
\end{proof}
\end{example}

\subsection{伯努利概型}
\begin{definition}
将随机试验重复进行\(n\)次,若各次试验的结果互不影响,
即每次试验各可能结果出现的概率都不依赖其他各次试验的结果,
这样的试验称为\(n\)重\DefineConcept{独立试验}.

特别地,\(n\)重\DefineConcept{独立试验}中,若每次试验结果只有两个,
即\(A\)与\(\overline{A}\),且\(0 < P(A) < 1\),
则这样的试验称为\(n\)重\DefineConcept{伯努利试验},
相应的数学模型叫做\DefineConcept{伯努利概型}.
\end{definition}

对于伯努利概型,
我们需要计算\(A\)在\(n\)次试验中恰好发生\(k\)次的概率.

\begin{theorem}[二项概率]
在\(n\)重伯努利试验中,
设\(A\)在各次试验中发生的概率为\(p = P(A)\ (0 < p < 1)\),
则在\(n\)次试验中\(A\)恰好发生\(k\)次的概率为\begin{equation}
	P_n(k) = C_n^k p^k (1-p)^{n-k}, \quad k=0,1,\dotsc,n.
\end{equation}
\begin{proof}
设事件\(A_i\)表示“\(A\)在第\(i\)次试验发生”,
则\(P(A_i)=p\),
\(P(\overline{A_i})=1-p\),
\(i=1,2,\dotsc,n\),
且各\(A_i\)相互独立.
由此可知,
\(A\)在\(n\)次试验中某指定\(k\)次
(如前\(k\)次)
试验中发生而在其余\(n-k\)次试验中不发生的概率为\begin{equation*}
	P(A_1 \dotsm A_k \overline{A_{k+1}} \dotsm \overline{A_n})
	= P(A_1) \dotsm P(A_k) P(\overline{A_{k+1}}) \dotsm P(\overline{A_n})
	= p^k (1-p)^{n-k}.
\end{equation*}
由于\(A\)在\(n\)次试验中恰好发生\(k\)次
共有\(C_n^k\)种指定\(k\)次试验\(A\)发生的方式,
且这\(C_n^k\)种指定的方式是两两互斥的,
从而有概率的可加性,
知\begin{equation*}
	P_n(k) = C_n^k p^k (1-p)^{n-k},
	\quad k=0,1,\dotsc,n.
	\qedhere
\end{equation*}
\end{proof}
\end{theorem}
由于\(P_n(k)\)正好是二项式\([p+(1-p)]^n\)展开式中的第\(k+1\)项,
所以通常称\(P_n(k)\)为\DefineConcept{二项概率}.
同时也可看出,
二项概率\(P_n(k)\)满足\begin{equation}
	\sum_{k=0}^n P_n(k)
	= \sum_{k=0}^n C_n^k p^k (1-p)^{n-k} = 1.
\end{equation}

在\(n\)重独立试验中,
若每次试验有\(k\)个结果\(A_1,A_2,\dotsc,A_k\),
我们来求\(n\)次试验中\(A_i\)各发生\(r_i\)次概率.

\begin{theorem}[多项概率]
在\(n\)重独立试验中,
每次试验可能的结果是\(A_1,A_2,\dotsc,A_k\),
且\(0 < p_i = P(A_i) < 1\ (i=1,2,\dotsc,k)\),
且\(p_1+p_2+\dotsb+p_k=1\),
则\(A_1,A_2,\dotsc,A_k\)在\(n\)次试验中
各发生\(r_1,r_2,\dotsc,r_k\)次的概率为
\begin{equation}\label{equation:概率论基础.多项概率公式}
	\frac{n!}{r_1! r_2! \dotsm r_k!}
	p_1^{r_1} p_2^{r_2} \dotsm p_k^{r_k},
\end{equation}
其中\(r_1+r_2+\dotsb+r_k=n\).
\end{theorem}
\cref{equation:概率论基础.多项概率公式}
叫做\DefineConcept{多项概率公式}.
