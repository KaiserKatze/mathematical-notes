\chapter{概率论的公理化}
\section{可测空间,概率空间}
\begin{definition}
%@see: 《应用随机过程》(林元烈) P1 定义1.1.1
设\(\Omega\)是一个非空集合,
\(\mathcal{F}\)是\(\Omega\)的一个子集族,且\(\mathcal{F}\neq\emptyset\).
如果\begin{itemize}
	\item \(\Omega \in \mathcal{F}\);
	\item \(A \in \mathcal{F}\)蕴含\(\SetComplementaryL{A} \in \mathcal{F}\),其中\(\SetComplementaryL{A} \defeq \Omega - A\);
	\item \(A_n \in \mathcal{F}\ (n \in \mathbb{N})\)蕴含\(\bigcup_{n=1}^\infty A_n \in \mathcal{F}\),
\end{itemize}
则称“\(\mathcal{F}\)是一个 \DefineConcept{\(\sigma\)代数}(\(\sigma\)-algebra)”
或“\(\mathcal{F}\)是一个 \DefineConcept{\(\sigma\)域}”,
同时称“\((\Omega,\mathcal{F})\)是一个\DefineConcept{可测空间}(measurable space)”.
%@see: https://mathworld.wolfram.com/Sigma-Algebra.html
%@see: https://mathworld.wolfram.com/MeasurableSpace.html
\end{definition}

\begin{example}
%@see: 《应用随机过程》(林元烈) P1
设\(\Omega\)是一个非空集合,
则\begin{itemize}
	\item 集类\(\mathcal{F}_0 \defeq \{\emptyset,\Omega\}\)是一个\(\sigma\)代数;
	\item 集类\(\mathcal{F}_1 \defeq \{\emptyset,A,\Omega-A,\Omega\}\)是一个\(\sigma\)代数(其中\(A\)是\(\Omega\)的一个非空子集);
	\item 集类\(\mathcal{F}_2 \defeq \Set{ A \given A \subseteq \Omega}\)也是一个\(\sigma\)代数;
	\item 集类\(\{\emptyset,A,\Omega\}\)不是\(\sigma\)代数(其中\(A\)是\(\Omega\)的一个非空子集).
\end{itemize}
\end{example}

\begin{property}
%@see: 《应用随机过程》(林元烈) P1
设\(\mathcal{F}\)是一个\(\sigma\)代数,
则\(\mathcal{F}\)对可列交封闭.
\end{property}

\begin{property}
%@see: 《应用随机过程》(林元烈) P1
设\(\mathcal{F}\)是一个\(\sigma\)代数,
则\(\mathcal{F}\)对可列并封闭.
\end{property}

\begin{property}
%@see: 《应用随机过程》(林元烈) P1
设\(\mathcal{F}\)是一个\(\sigma\)代数,
则\(\mathcal{F}\)对可列差封闭.
\end{property}

\begin{definition}
%@see: 《应用随机过程》(林元烈) P1
设\(\Omega\)是一个非空集合,
\(\mathcal{A}\)是\(\Omega\)的一个子集族,且\(\mathcal{A}\neq\emptyset\).
把集类\begin{equation*}
	\bigcap\Set{
		\mathcal{F}
		\given
		\text{$\mathcal{F}$是一个$\sigma$代数},
		\mathcal{F} \supseteq \mathcal{A}
	}
\end{equation*}
称为“由\(\mathcal{A}\)生成的\(\sigma\)域”
或“包含\(\mathcal{A}\)的最小\(\sigma\)域”,
记为\(\sigma(\mathcal{A})\).
\end{definition}

\begin{example}
%@see: 《应用随机过程》(林元烈) P1
设\(\Omega\)是一个非空集合,
\(A\)是\(\Omega\)的一个非空子集,
记\(\mathcal{A} \defeq \{\emptyset,A,\Omega\}\),
则\(\sigma(\mathcal{A}) = \{\emptyset,A,\Omega-A,\Omega\}\).
\end{example}

\begin{definition}
%@see: 《应用随机过程》(林元烈) P1
%@see: 《随机数学引论》(林元烈) P55
把实数域\(\Omega \defeq \mathbb{R}\)的所有形如\((-\infty,a]\)的区间组成的集合\begin{equation*}
	\mathcal{A}
	\defeq
	\Set{
		A \subseteq \mathbb{R}
		\given
		A = (-\infty,a],
		a \in (-\infty,+\infty)
	}
\end{equation*}
的最小\(\sigma\)域\(\sigma(\mathcal{A})\)
称为\DefineConcept{一维波莱尔\(\sigma\)域}(one-dimensional Borel \(\sigma\)-algebra),
把其中的每一个元素称为一个\DefineConcept{一维波莱尔集}(one-dimensional Borel set).
%@see: https://mathworld.wolfram.com/BorelSigma-Algebra.html
%@see: https://mathworld.wolfram.com/BorelSet.html
\end{definition}

\begin{definition}
%@see: 《随机数学引论》(林元烈) P56
把\(n\)维实向量空间\(\Omega \defeq \mathbb{R}^n\)的子集族\begin{equation*}
	\mathcal{A}
	\defeq
	\Set*{
		A \subseteq \mathbb{R}^n
		\given
		A = \BigTimes_{i=1}^n (-\infty,a_i],
		-\infty < a_i < +\infty,
		i = 1,2,\dotsc,n
	}
\end{equation*}
的最小\(\sigma\)域\(\sigma(\mathcal{A})\)
称为 \DefineConcept{\(n\)维波莱尔\(\sigma\)域},
把其中的每一个元素称为一个 \DefineConcept{\(n\)维波莱尔集}.
\end{definition}

\section{概率测度,概率空间,随机事件及其概率}
\begin{definition}
%@see: 《应用随机过程》(林元烈) P1 定义1.1.2
设\((\Omega,\mathcal{F})\)是一个可测空间.
如果映射\(P\colon \mathcal{F} \to \mathbb{R}\)满足\begin{itemize}
	\item {\rm\bf 非负性},即\begin{equation*}
		(\forall A \in \mathcal{F})
		[
			P(A) \geq 0
		];
	\end{equation*}

	\item {\rm\bf 规范性},即\begin{equation*}
		P(\Omega) = 1;
	\end{equation*}

	\item {\rm\bf 可列可加性},即\begin{equation*}
		(\forall A_1,A_2,\dotsc \in \mathcal{F})
		\left[
			(\forall A_i)(\forall A_j)
			[
				A_i A_j = \emptyset
			]
			\implies
			P\left( \bigcup_{i=1}^\infty A_i \right)
			= \sum_{i=1}^\infty P(A_i)
		\right],
	\end{equation*}
\end{itemize}
则称“映射\(P\)是可测空间\((\Omega,\mathcal{F})\)上的一个\DefineConcept{概率测度}(probability measure)”,
在不致混淆的情况下简称为\DefineConcept{概率}(probability);
同时把\((\Omega,\mathcal{F},P)\)称为一个\DefineConcept{概率空间}(probability space),
把\(\mathcal{F}\)称为一个\DefineConcept{事件域}.
如果\(A \in \mathcal{F}\),
则称“\(A\)是(概率空间\((\Omega,\mathcal{F},P)\)中的)一个\DefineConcept{随机事件}(random event)”,
简称\DefineConcept{事件}(event),
同时把\(P(A)\)称为“事件\(A\)的\DefineConcept{概率}(probability)”.
%@see: https://mathworld.wolfram.com/ProbabilityMeasure.html
%@see: https://mathworld.wolfram.com/ProbabilitySpace.html
\end{definition}

公理化的事件概率,与我们之前建立的朴素认知,具有相同的基本性质.

\section{随机变量,分布函数}
\begin{definition}
%@see: 《应用随机过程》(林元烈) P5 定义1.2.1
设\((\Omega,\mathcal{F},P)\)是一个概率空间.
如果映射\(X\colon \Omega \to \mathbb{R}\)满足\begin{equation*}
	(\forall a \in \mathbb{R})
	[
		\Set{
			\omega \in \Omega
			\given
			X(\omega) \leq a
		}
		\in \mathcal{F}
	],
\end{equation*}
则称“映射\(X\)是(概率空间\((\Omega,\mathcal{F},P)\)中的)一个\DefineConcept{随机变量}(random variable)”.
\end{definition}
\begin{remark}
%@see: 《应用随机过程》(林元烈) P6
随机变量的定义要求
所有满足\(X(\omega) \leq a\)的样本点\(\omega\)的集合\(
	\Set{
		\omega \in \Omega
		\given
		X(\omega) \leq a
	}
\)必须是概率空间\((\Omega,\mathcal{F},P)\)中的某个事件,
如此方可定义它的概率.
\end{remark}
\begin{remark}
%@see: 《应用随机过程》(林元烈) P6
为了书写方便,我们常把集合\(
	\Set{
		\omega \in \Omega
		\given
		X(\omega) \leq a
	}
\)简写为\((X \leq a)\).
\end{remark}

\begin{theorem}
%@see: 《应用随机过程》(林元烈) P6
设\(X\)是概率空间\((\Omega,\mathcal{F},P)\)中的一个随机变量.
任意取定\(a,b \in \overline{\mathbb{R}}\),
则集合\begin{gather*}
	(X > a)
	\defeq
	\Set{
		\omega \in \Omega
		\given
		X(\omega) > a
	}, \\
	(X < a)
	\defeq
	\Set{
		\omega \in \Omega
		\given
		X(\omega) < a
	}, \\
	(X = a)
	\defeq
	\Set{
		\omega \in \Omega
		\given
		X(\omega) = a
	}, \\
	(a < X \leq b)
	\defeq
	\Set{
		\omega \in \Omega
		\given
		a < X(\omega) \leq b
	}, \\
	(a \leq X < b)
	\defeq
	\Set{
		\omega \in \Omega
		\given
		a \leq X(\omega) < b
	}, \\
	(a < X < b)
	\defeq
	\Set{
		\omega \in \Omega
		\given
		a < X(\omega) < b
	}, \\
	(a \leq X \leq b)
	\defeq
	\Set{
		\omega \in \Omega
		\given
		a \leq X(\omega) \leq b
	}
\end{gather*}
都是概率空间\((\Omega,\mathcal{F},P)\)中的随机事件.
\end{theorem}

\begin{example}
%@see: 《概率论基础及其应用(第三版)》(王梓坤) P48 注意1
设\((\Omega,\mathcal{F},P)\)是一个概率空间,
其中\(\mathcal{F} = \Powerset\Omega\),
那么\(\Omega\)的任意一个子集
都是\((\Omega,\mathcal{F},P)\)中的一个随机事件,
并且映射空间\(\mathbb{R}^\Omega\)中的任意一个映射
都是\((\Omega,\mathcal{F},P)\)中的一个随机变量.
\end{example}

\begin{example}
%@see: 《概率论基础及其应用(第三版)》(王梓坤) P48
设\(A\)是概率空间\((\Omega,\mathcal{F},P)\)中的一个随机事件,
\(\chi_A\)是定义在\(\Omega\)上的\(A\)的示性函数,
则\(\chi_A\)是概率空间\((\Omega,\mathcal{F},P)\)中的一个随机变量,
这是因为对于\(
	F(x)
	\defeq
	\Set{
		\omega \in \Omega
		\given
		\chi_A(\omega) \leq x
	}
\)而言,
有\begin{equation*}
	F(x) = \begin{cases}
		\Omega, & x \geq 1, \\
		\Omega-A, & 0 \leq x < 1, \\
		\emptyset, & x < 0,
	\end{cases}
\end{equation*}
也就是说,对于任意实数\(x\),
总有\(
	(\chi_A \leq x)
	\in \mathcal{F}
\).
\end{example}

\begin{example}
%@see: 《应用随机过程》(林元烈) P6 例1
设\(\Omega\)是一个非空集合,
\(A\)和\(B\)是\(\Omega\)的两个非空子集,
\(\mathcal{F} \defeq \{\emptyset,A,\Omega-A,\Omega\}\).
如果\(B \notin \mathcal{F}\),
那么定义在\(\Omega\)上的\(B\)的示性函数\(\chi_B\)
满足\((\chi_B \leq 1/2) \notin F\),
也就是说\(\chi_B\)不满足随机变量的定义.
\end{example}

\begin{example}
%@see: 《应用随机过程》(林元烈) P6 例2
设\((\Omega,\mathcal{F})\)是一个可测空间,
序列\(\{B_k\}_{k\geq0}\)是\(\Omega\)的一个划分,
且\(B_k \in \mathcal{F}\ (k\geq0)\),
而\(\chi_{B_k}\)是定义在\(\Omega\)上的\(B_k\)的示性函数,
则\(
	X(\omega) \defeq \sum_{k=1}^\infty x_k \chi_{B_k}(\omega)
\)是一个随机变量.
\end{example}
