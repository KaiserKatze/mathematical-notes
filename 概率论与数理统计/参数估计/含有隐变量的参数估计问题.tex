\section{含有隐变量的参数估计问题}
\subsection{EM算法}
%@see: https://www.columbia.edu/~mh2078/MachineLearningORFE/EM_Algorithm.pdf
给定两枚非均质硬币\(A\)和\(B\)(即它们正面朝上的概率\(P_A\)和\(P_B\)都不等于\(1/2\)),
且\(P_A \neq P_B\).
把这两枚硬币放进一个不透光的袋子中,
每次从袋中随机摸出一枚硬币抛掷,
记录抛掷的结果,
最终正面向上的次数为\(n_1\),
反面向上的次数为\(n_2\).
我们不禁要问:
可否仅凭上述信息估计两枚硬币各自正面向上的概率呢?
容易看出,
如果采用极大似然估计法,
那么只靠上述信息是无法估计\(P_A,P_B\)的,
这是因为上述模型不仅含有可以直接观测到的变量(即每次抛掷硬币是正面朝上还是反面朝上),
还含有隐藏的、未知的变量(即每次抛掷的硬币到底是硬币\(A\)还是硬币\(B\)).

我们把可以直接观测到的随机变量\(X\)称为\DefineConcept{可观测变量}(observable variable),
把不可直接观测到的随机变量\(Y\)称为\DefineConcept{隐变量}(latent variable, hidden variable).
将可观测变量与隐变量合在一起,把得到的随机变量\(Z \defeq (X,Y)\)称为\DefineConcept{完全变量}(complete variable);
相对地,把可观测变量\(X\)称为\DefineConcept{不完全变量}(incomplete variable).

针对上述含有隐变量的混合模型,我们不能直接使用极大似然估计法.
这种情况下就需要用到迭代法(又称\DefineConcept{EM算法}、\DefineConcept{极大期望估计法}).
\begin{algorithm}[极大期望估计法]
\hfill
\begin{enumerate}
	\item 通过试验,取得随机变量\(X\)的观测值\(x\);
	\item 对未知参数\(\theta\)作出合理的初始估计值\(\theta_0\);
	\item (E步)根据随机变量\(X\)的观测值\(x\)和未知参数\(\theta\)的估计值\(\theta_k\ (k=0,1,2,\dotsc)\),
	计算隐变量\(Y\)的对数似然函数\(y \mapsto \ln p(x,y \vert \theta)\)的条件数学期望\begin{equation*}
	%@see: 《数理统计教程》(王兆军,邹长亮) P54 (2.3.10)
		Q(\theta',\theta_k)
		\defeq
		E(\ln p(x,Y \vert \theta=\theta') \vert X=x, \theta=\theta_k);
	\end{equation*}
	\item (M步)计算极大值点\begin{equation*}
	%@see: 《数理统计教程》(王兆军,邹长亮) P54 (2.3.11)
		\theta_{k+1}
		\defeq \argmax_{\theta'} Q(\theta',\theta_k),
	\end{equation*}
	并把它作为\(\theta\)的新的估计值;
	\item 重复第3步和第4步,直到\(\theta\)的估计值收敛(即\(\theta_{k+1}\)与\(\theta_k\)之间的差值足够小).
\end{enumerate}
\end{algorithm}

下面我们来证明EM算法的收敛性.

%@see: 《数理统计教程》(王兆军,邹长亮) P55 定理2.3.1
设\(X,Y\)都是随机变量,
\(X\)是可观测的,
\(Y\)是隐藏的,
\(X,Y\)相互独立,
总体\(Z \defeq (X,Y)\)只含一个未知参数\(\theta\),
那么已知\((X,\theta) = (x,\theta')\)条件下\(Y\)的条件密度函数为\begin{equation*}
	p_{Y \vert X}(y \vert X=x, \theta=\theta')
	= \frac{p(x,y \vert \theta=\theta')}{p_X(x \vert \theta=\theta')},
\end{equation*}
其中\(p(x,y \vert \theta=\theta')\)是已知\(\theta = \theta'\)条件下\((X,Y)\)的联合密度函数,
\(p_X(x \vert \theta=\theta')\)是已知\(\theta = \theta'\)条件下\(X\)的边缘密度函数,
从而有\begin{equation*}
	p_X(x \vert \theta=\theta')
	= \frac{p(x,y \vert \theta=\theta')}{p_{Y \vert X}(y \vert X=x, \theta=\theta')}.
\end{equation*}
对上式取对数,得\begin{equation*}%\label{equation:含有隐变量的参数估计问题.对数似然函数等式1}
	\ln p_X(x \vert \theta=\theta')
	= \ln p(x,y \vert \theta=\theta')
	- \ln p_{Y \vert X}(y \vert X=x, \theta=\theta').
\end{equation*}
在上式等号两边同时求\(Y\)在\((X,\theta) = (x,\theta_k)\)条件下的期望,
得到\begin{align*}
	&\hspace{-20pt}
	E(\ln p_X(x \vert \theta=\theta') \vert X=x, \theta=\theta_k) \\
	&= Q(\theta',\theta_k)
	- E(\ln p_{Y \vert X}(Y \vert X=x, \theta=\theta') \vert X=x, \theta=\theta_k),
\end{align*}
即\begin{align*}
%@credit: {5f4d2f8a-fc8b-4798-85d6-98670f6761e7} 指出我在本节中有几处符号使用有误,现已修改
	&\hspace{-20pt}
	\int p_{Y \vert X}(y \vert X=x, \theta=\theta_k) \cdot \ln p_X(x \vert \theta=\theta') \dd{y} \\
	&= Q(\theta',\theta_k)
	- \int p_{Y \vert X}(y \vert X=x, \theta=\theta_k) \cdot \ln p_{Y \vert X}(y \vert X=x, \theta=\theta') \dd{y}.
\end{align*}
因为\(X,Y\)相互独立,
所以\(\ln p_X(x \vert \theta=\theta')\)与\(y\)无关;
又因为\(\int p_{Y \vert X}(y \vert X=x, \theta=\theta_k) \dd{y} = 1\),
所以\begin{align*}
	&\hspace{-20pt}
	\int p_{Y \vert X}(y \vert X=x, \theta=\theta_k) \cdot \ln p_X(x \vert \theta=\theta') \dd{y} \\
	&= \ln p_X(x \vert \theta=\theta') \cdot \int p_{Y \vert X}(y \vert X=x, \theta=\theta_k) \dd{y} \\
	&= \ln p_X(x \vert \theta=\theta') \cdot 1
	= \ln p_X(x \vert \theta=\theta'),
\end{align*}
从而有\begin{align*}
%@see: 《数理统计教程》(王兆军,邹长亮) P55 (*)
	\ln p_X(x \vert \theta=\theta')
	= Q(\theta',\theta_k)
	- \int p_{Y \vert X}(y \vert X=x, \theta=\theta_k) \cdot \ln p_{Y \vert X}(y \vert X=x, \theta=\theta') \dd{y}.
\end{align*}
上式依次取\(\theta' \defeq \theta_k\)和\(\theta' \defeq \theta_{k+1}\),
便得\begin{align*}
%@see: 《数理统计教程》(王兆军,邹长亮) P55 (*1)
%@see: 《数理统计教程》(王兆军,邹长亮) P55 (*2)
	\ln p_X(x \vert \theta=\theta_k)
	&=  Q(\theta_k,\theta_k)
	- \int p_{Y \vert X}(y \vert X=x, \theta=\theta_k) \cdot \ln p_{Y \vert X}(y \vert X=x, \theta=\theta_k) \dd{y}, \\
	\ln p_X(x \vert \theta=\theta_{k+1})
	&=  Q(\theta_{k+1},\theta_k)
	- \int p_{Y \vert X}(y \vert X=x, \theta=\theta_k) \cdot \ln p_{Y \vert X}(y \vert X=x, \theta=\theta_{k+1}) \dd{y}.
\end{align*}
两式相减得\begin{align*}
	&\hspace{-20pt}
	\ln p_X(x \vert \theta=\theta_{k+1})
	- \ln p_X(x \vert \theta=\theta_k) \\
	&= Q(\theta_{k+1},\theta_k) - Q(\theta_k,\theta_k)
	- \int p_{Y \vert X}(y \vert X=x, \theta=\theta_k)
		\cdot \ln \frac{
			p_{Y \vert X}(y \vert X=x, \theta=\theta_{k+1})
		}{
			p_{Y \vert X}(y \vert X=x, \theta=\theta_k)
		}
		\dd{y}.
\end{align*}
因为M步确保了\(
	\theta_{k+1}
	\defeq \argmax_{\theta'} Q(\theta',\theta_k)
\),
所以\(
	Q(\theta_{k+1},\theta_k)
	\geq Q(\theta_k,\theta_k)
\).
再由\hyperref[theorem:随机变量的数字特征.延森不等式]{延森不等式}可得\begin{align*}
	&\hspace{-20pt}
	\int p_{Y \vert X}(y \vert X=x, \theta=\theta_k)
		\cdot \ln \frac{
			p_{Y \vert X}(y \vert X=x, \theta=\theta_{k+1})
		}{
			p_{Y \vert X}(y \vert X=x, \theta=\theta_k)
		}
		\dd{y} \\
	&\leq
	\ln \int p_{Y \vert X}(y \vert X=x, \theta=\theta_k)
		\cdot \frac{
			p_{Y \vert X}(y \vert X=x, \theta=\theta_{k+1})
		}{
			p_{Y \vert X}(y \vert X=x, \theta=\theta_k)
		}
		\dd{y} \\
	&\leq
	\ln \int p_{Y \vert X}(y \vert X=x, \theta=\theta_{k+1}) \dd{y}
	= \ln1
	= 0.
\end{align*}
因此\begin{equation*}
	\ln p_X(x \vert \theta=\theta_{k+1})
	\geq \ln p_X(x \vert \theta=\theta_k).
\end{equation*}
这就说明,通过EM算法得到的序列\(\{\theta_k\}_{k\geq0}\)
会使可观测变量\(X\)的对数似然函数\begin{equation*}
	l(\theta') \defeq \ln p_X(x \vert \theta=\theta')
\end{equation*}
单调增加,
即\begin{equation*}
%@see: 《数理统计教程》(王兆军,邹长亮) P55 (2.3.12)
	l(\theta_{k+1})
	\geq l(\theta_k)
	\geq 0.
\end{equation*}


利用EM算法求得的估计量\(\hat{\theta}\)的渐进方差
近似等于可观测变量的费舍尔信息量的倒数\begin{equation*}
%@see: 《数理统计教程》(王兆军,邹长亮) P56 (2.3.13)
	\left( - \eval{ \pdv[2]{ \ln p_X(x,\theta) }{ \theta } }_{\hat{\theta}} \right)^{-1}.
\end{equation*}


\subsection{GEM算法}
%@see: 《数理统计教程》(王兆军,邹长亮) P58
虽然EM算法已经得到了广泛的应用,但是在某些情况下,其M步的最大值是难以计算的,
于是我们常常采用一个简单的方法确定\(\theta^{k+1}\)
%TODO 所谓“简单的方法”具体是指什么?
使得\begin{equation*}
%@see: 《数理统计教程》(王兆军,邹长亮) P58 (2.3.14)
	Q(\theta_{k+1},\theta_k)
	> Q(\theta_k,\theta_k).
\end{equation*}
这种求取\(\theta\)的极大似然估计量的方法
称为\DefineConcept{广义EM算法}或\DefineConcept{GEM算法}.

\subsection{蒙特卡洛EM算法}
GEM算法仅考虑了M步的计算难度,
但是在某些情况下,E步的计算也很困难,
它的本质是计算完全变量的对数似然函数
在已知\((X,\theta) = (x,\theta_k)\)条件下的期望,
既然期望可以用样本平均值予以估计,
那么我们可以把E步拆成以下两步:\begin{enumerate}
	\item (E1步)依据\(Y\)的条件分布\(p_{Y \vert X}(y \vert X=x, \theta=\theta_k)\)抽取\(m\)个随机数\(z_1,\dotsc,z_m\);
	\item (E2步)令\begin{equation*}
		Q(\theta',\theta_k)
		\defeq
		\frac1m \sum_{i=1}^m \ln p_{Y \vert X}(\theta' \vert X=x, \theta=z_i).
	\end{equation*}
\end{enumerate}
