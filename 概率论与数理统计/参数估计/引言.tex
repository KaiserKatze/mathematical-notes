在实际应用中,
一个总体\(X\)的分布函数往往含有未知参数\(\theta\)或未知参数向量\(\vb\theta\),
从而可记总体分布函数为\(F(x,\theta)\)或\(F(x,\vb\theta)\).
解决实际问题时需要了解未知参数或未知参数向量,
因此可以利用样本提供的信息,
对未知参数或未知参数向量有一个基本的估计.
这就是参数的估计问题.

参数的估计问题分为点估计和区间估计两大类.

在点估计中,我们要构造一个统计量
\(\hat{\theta}=\hat{\theta}(\AutoTuple{X}{n})\),
作为未知参数\(\theta\)的\DefineConcept{点估计量}(point estimator).
%@see: https://mathworld.wolfram.com/Estimate.html
然后把它的样本观测值\(\hat{\theta}(\AutoTuple{x}{n})\)
作为未知参数\(\theta\)的\DefineConcept{点估计值}.
在点估计问题中,我们常用以下两类方法:矩估计法和极大似然估计法.

在区间估计中,我们要构造两个统计量\(\hat{\theta}_1\)和\(\hat{\theta}_2\),
且\(\hat{\theta}_1<\hat{\theta}_2\),
然后以区间\([\hat{\theta}_1,\hat{\theta}_2]\)的形式给出对未知参数\(\theta\)的估计.
