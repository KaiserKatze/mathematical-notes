\section{本章总结}

极大似然估计法:\begin{enumerate}
	\item 抽取\(n\)个样本\(\AutoTuple{X}{n}\),记录相应的观测值\(\AutoTuple{x}{n}\).

	\item 依据总体\(X\)的分布律或密度函数\(p(x,\theta)\),
	构造未知参数\(\theta\)的似然函数\begin{equation*}
		L(\theta) = \prod_{i=1}^n p(x_i,\theta).
	\end{equation*}
	然后取对数得到对数似然函数\begin{equation*}
		\ln L(\theta) = \sum_{i=1}^n \ln p(x_i,\theta).
	\end{equation*}

	\item 利用方程\(\dv{\theta} \ln L(\theta) = 0\)
	求出使得\(L(\theta)\)取得最大值的极大似然估计值\(\hat{\theta}\).
\end{enumerate}

\begin{table}[htb]
	\centering
	\begin{tblr}{*5{|c}|}
		\hline
		总体分布
			& 已知量
			& 未知量
			& 矩估计量
			& 极大似然估计量
		\\ \hline
		任何分布
			&
			& \(\mu,\sigma^2\)
			& \(\begin{aligned}
					\hat{\mu} &= \overline{X}, \\
					\hat{\sigma^2} &= B_2
				\end{aligned}\)
			&
		\\ \hline
		正态分布\(N(\mu,\sigma^2)\)
			&
			& \(\mu,\sigma^2\)
			& 同上
			& 同左
		\\ \hline
		二项分布\(B(N,p)\)
			& \(N\)
			& \(p\)
			& \(\begin{aligned}
					\hat{p} &= \frac{\overline{X}}{N}, \\
					\hat{\sigma^2} &= \overline{X} - \frac{\overline{X}^2}{N}
				\end{aligned}\)
			& \(\hat{p} = \frac{\overline{X}}{N}\)
		\\ \hline
		泊松分布\(P(\lambda)\)
			&
			& \(\lambda\)
			& \(\hat{\lambda} = \overline{X}\)
			& 同左
		\\ \hline
		几何分布\(G(p)\)
			&
			& \(p\)
			& \(\hat{p} = \frac{1}{\overline{X}}\)
			& \(\begin{aligned}
					\hat{p} &= \frac{1}{\overline{X}}, \\
					\hat{\sigma^2} &= \overline{X}^2-\overline{X}
				\end{aligned}\)
		\\ \hline
		指数分布\(e(\lambda)\)
			&
			& \(\lambda\)
			& \(\hat{\lambda} = \frac{1}{\overline{X}}\)
			& 同左
		\\ \hline
		均匀分布\(U(\theta_1,\theta_2)\)
			&
			& \(\theta_1,\theta_2\)
			& \(\begin{aligned}
					\hat{\theta_1} &= \overline{X} - \sqrt{3 B_2}, \\
					\hat{\theta_2} &= \overline{X} + \sqrt{3 B_2}
				\end{aligned}\)
			& \(\begin{aligned}
					\hat{\theta_1} &= \min_{1\leq i\leq n} X_i, \\
					\hat{\theta_2} &= \max_{1\leq i\leq n} X_i
				\end{aligned}\)
		\\ \hline
	\end{tblr}
	\caption{常见分布的参数估计}
\end{table}
