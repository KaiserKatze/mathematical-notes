\chapter{线性回归分析与方差分析}
\section{线性回归分析}
在实际工作中,遇到的两个变量通常有相互制约、相互关联的关系.这些关系又可分为两大类:一类是确定性关系,如理想状态下匀速直线运动中时间\(t\)与距离\(s\)的关系;另一类是相关关系,如中老年健康调查中年龄\(x\)与血压\(Y\)的关系,水稻的单位面积施肥量\(x\)与单位面积产量\(Y\)的关系.

当\(x\)与\(Y\)存在相关关系时,最常见、最普通的这种相关关系是线性相关关系,即\(x\)与\(Y\)的相关关系有线性趋势.围绕线性相关关系的线性趋势展开的统计分析叫做\DefineConcept{线性回归分析}.

\subsection{线性回归模型}
\begin{definition}
设\(x\)是普通变量,\(Y\)是随机变量,且\[
Y = \alpha + \beta x + \epsilon,
\quad \epsilon \sim N(0,\sigma^2),
\]其中\(\alpha\),\(\beta\),\(\sigma^2\)是不依赖于\(x\)的未知参数.称此模型为\DefineConcept{一元线性回归模型}.

易见,此时\[
	Y \sim N(\alpha + \beta x,\sigma^2).
\]
称\(Y\)的期望\[
	\tilde{Y} = E(Y) = \alpha + \beta x
\]为\(Y\)关于\(x\)的\DefineConcept{线性回归函数}.
\(\alpha\)和\(\beta\)称为\DefineConcept{回归系数},
\(x\)称为\DefineConcept{回归变量}.

在一元线性回归模型下,\(x\)与\(Y\)就存在线性相关关系,也叫做\DefineConcept{线性回归关系}.

取得样本观测值\((x_1,y_1),(x_2,y_2),\dotsc,(x_n,y_n)\)后,若能得\(\alpha\)、\(\beta\)的点估计\(\hat{\alpha}\)和\(\hat{\beta}\),称\[
\hat{Y} = \hat{\alpha} + \hat{\beta} x
\]为\DefineConcept{线性回归方程},它所代表的直线叫做\DefineConcept{线性回归直线}.
\end{definition}
线性回归直线描述了\(x\)与\(Y\)线性相关关系中线性趋势.

\subsection{线性回归模型参数的极大似然估计及性质}
取得样本观测值\((x_1,y_1),(x_2,y_2),\dotsc,(x_n,y_n)\)后,似然函数为\[
L(\alpha,\beta,\sigma^2)
= \prod_{i=1}^n{ \frac{1}{\sqrt{2\pi}\sigma} e^{ -\frac{[y_i - (\alpha+\beta x_i)]^2}{2\sigma^2} } }
= (2\pi)^{-\frac{n}{2}} (\sigma^2)^{-\frac{n}{2}} e^{ -\frac{1}{2\sigma^2} \sum_{i=1}^n[y_i - (\alpha+\beta x_i)]^2 },
\]从而有\[
\ln L(\alpha,\beta,\sigma^2)
= -\frac{n}{2} \ln(2\pi) - \frac{n}{2} \ln(\sigma^2) - \frac{1}{2\sigma^2} \sum_{i=1}^n[y_i-(\alpha+\beta x_i)]^2.
\]故可得对数似然方程组\[
\left\{ \begin{array}{l}
\pdv{\ln L(\alpha,\beta,\sigma^2)}{\alpha}
= \frac{1}{\sigma^2} \sum_{i=1}^n{y_i-(\alpha+\beta x_i)} = 0, \\
\pdv{\ln L(\alpha,\beta,\sigma^2)}{\beta}
= \frac{1}{\sigma^2} \sum_{i=1}^n{x_i[y_i-(\alpha+\beta x_i)]} = 0, \\
\pdv{\ln L(\alpha,\beta,\sigma^2)}{\sigma^2}
= -\frac{n}{2\sigma^2} + \frac{1}{2\sigma^4} \sum_{i=1}^n[y_i-(\alpha+\beta x_i)]^2 = 0.
\end{array} \right.
\]整理得\[
\left\{ \begin{array}{l}
n \alpha + \beta \sum_{i=1}^n{x_i} = \sum_{i=1}^n{y_i}, \\
\alpha \sum_{i=1}^n{x_i} + \beta \sum_{i=1}^n{x_i^2} = \sum_{i=1}^n{x_i y_i}, \\
\sigma^2 = \frac{1}{n} \sum_{i=1}^n[y_i - (\alpha+\beta x_i)]^2.
\end{array} \right.
\]解得\[
\left\{ \begin{array}{l}
\hat{\alpha} = \overline{y} - \hat{\beta} \overline{x}, \\
\hat{\beta} = \frac{s_{xy}}{s_{xx}}, \\
\hat{\sigma}^2 = \frac{1}{n} \sum_{i=1}^n[y_i - (\hat{\alpha}+\hat{\beta}x_i)]^2.
\end{array} \right.
\]其中,\[
s_{xy} = \sum_{i=1}^n(x_i-\overline{x})(y_i-\overline{y})
= \sum_{i=1}^n{x_i y_i - \overline{x}y_i - \overline{y}x_i + \overline{x}\overline{y}}
= \sum_{i=1}^n{x_i y_i - n \overline{x} \overline{y}},
\]\[
s_{xx} = \sum_{i=1}^n(x_i - \overline{x})^2
= \sum_{i=1}^n{x_i^2 - n \overline{x}^2}.
\]
在上述结果中可以用\(Y_i\)代替观测值\(y_i\).

注意,\(\alpha\)和\(\beta\)的极大似然估计值是使似然函数中\(e\)的指数部分\[
Q(\alpha,\beta) = \sum_{i=1}^n[y_i-(\alpha+\beta x_i)]^2
\]达到最小的估计,所以\(\alpha\)和\(\beta\)的极大似然估计\(\hat{\alpha}\)和\(\hat{\beta}\)又叫做最小二乘估计.

事实上,\(Q(\hat{\alpha},\hat{\beta})\)是数据\((x_i,y_i)\)对估计\(\hat{y}_i=\hat{\alpha}+\beta x_i\)的偏差平方和(又称残差平方和).最小二乘估计使数据点沿着与\(y\)轴平行的方向到回归直线的距离的平方和达到最小.

得到\(\hat{\alpha}\)和\(\hat{\beta}\)后,可得回归方程\[
\hat{y} = \hat{\alpha} + \hat{\beta} x
= \overline{y} + \hat{\beta} (x - \overline{x}),
\]可见回归直线过点\((\overline{x},\overline{y})\).

\begin{theorem}
\begin{enumerate}
\item \(\overline{Y}\)、\(\hat{\beta}\)和\(\hat{\sigma}^2\)相互独立;
\item \((\hat{\alpha},\hat{\beta}) \sim N\left(
	\alpha, \beta;
	\sigma^2 \left( \frac{1}{n} + \frac{\overline{x}^2}{s_{xx}} \right),
	\frac{\sigma^2}{s_{xx}},
	\frac{-\overline{x}}{
		\sqrt{s_{xx}} \sqrt{\frac{1}{n} + \frac{\overline{x}^2}{s_{xx}}}
	}
\right)\);
\item \(\frac{n \hat{\sigma}^2}{\sigma^2} \sim \x(n-2)\).
\end{enumerate}
\end{theorem}

\subsection{线性回归方程的显著性检验}
当\(x\)与\(Y\)不存在线性相关关系时,虽然也可以计算出\(\hat{\alpha}\)和\(\hat{\beta}\),从而也可以在形式上得到一条“回归直线”.但这条直线并不代表\(x\)与\(Y\)的相关关系趋势.因此在实际情况中,\(x\)与\(Y\)是否存在线性相关(线性回归)关系,即模型中\(E(Y) = \widetilde{Y} = \alpha + \beta x\)的假定是否符合实际情况,是需要判断的.

首先可以根据有关的专业知识和实践来判断,也可以由统计理论和方法来判断.直观和初步的统计判断方法可根据散点图来进行.

散点图是将各样本观测值\((x_i,Y_i)\ (i=1,2,\dotsc,n)\)描绘在一个直角坐标平面上得到的图.若这些点大致散布在某条直线周围,则可初步认定\(x\)与\(Y\)之间存在线性相关关系或线性回归关系.

对\(x\)与\(Y\)是否存在线性相关关系(线性回归关系)的细致分析应是对回归模型的线性性进行显著性检验.检验假设为\[
H_0: \beta = 0, \quad H_1: \beta \neq 0.
\]当\(H_0\)成立(即\(\beta = 0\))时,相当于\(x\)与\(Y\)不存在线性回归关系,且有\[
\hat{\beta} \sim N\left(0, \frac{\sigma^2}{s_{xx}}\right),
\]\[
\frac{n \hat{\sigma}^2}{\sigma^2} \sim \x(n-2),
\]且\(\hat{\beta}\)和\(\hat{\sigma}^2\)独立.于是\[
t = \frac{\hat{\beta} / \frac{\sigma}{\sqrt{s_{xx}}}}{\sqrt{\frac{n\hat{\sigma}^2}{\sigma^2(n-2)}}}
= \frac{\hat{\beta} \sqrt{s_{xx}}}{\sqrt{\frac{n}{n-2} \hat{\sigma}^2}}
\sim t(n-2).
\]而当\(H_1\)成立时,检验统计量\(t\)有\(\abs{t}\)取值偏大,得到检验的拒绝域为\[
W = \{ \abs{t} > t_{1-\frac{\alpha}{2}}(n-2) \},
\]其中\(\alpha\)为显著性水平.

关于检验统计量\(t\)中\(\hat{\sigma}^2\)的计算,记\(\hat{Y}_i = \hat{\alpha}+\hat{\beta} x_i\),因为\[
\hat{\sigma}^2 = \frac{1}{n} \sum_{i=1}^n[Y_i-\hat{Y}_i]^2,
\]\begin{align*}
\sum_{i=1}^n(Y_i-\hat{Y}_i)^2
&= \sum_{i=1}^n[Y_i-\overline{Y}-\hat{\beta}(x_i-\overline{x})]^2 \\
&= \sum_{i=1}^n(Y_i-\overline{Y})^2
	-2\hat{\beta} \sum_{i=1}^n(x_i-\overline{x})(Y_i-\overline{Y})
	+\hat{\beta}^2 \sum_{i=1}^n(x_i-\overline{x})^2 \\
&= s_{YY} - 2 \hat{\beta} s_{xY} + \hat{\beta}^2 s_{xx} \\
&= s_{YY} - \hat{\beta} s_{xY},
\end{align*}那么有\[
\hat{\sigma}^2 = \frac{1}{n} (s_{YY} - \hat{\beta} s_{xY}),
\]其中\[
s_{YY} = \sum_{i=1}^n(Y_i - \overline{Y})^2
= \sum_{i=1}^n{Y_i^2 - n\overline{Y}^2}.
\]

\subsection{预测}
经检验后,若认为\(x\)与\(Y\)存在线性回归关系后,回归方程重要的用途是预测和控制,即在\(x\)的指定值\(x_0\)处,对于对应的\(Y_0\)的平均取值和\(Y_0\)的\(1-\alpha\)取值区间进行预测,就是\(Y_0\)的点预测和区间预测.而控制问题则相反,当\(Y\)的取值限定在区间\((Y_1,Y_2)\)间时,要求\(x\)应控制在什么范围内.
