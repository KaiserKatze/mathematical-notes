\section{仿射空间}
本节利用公理法描述仿射几何中的几何元素.

\subsection{仿射空间}
\begin{definition}
%@see: 《基础代数(第二卷)》(席南华) P161 定义4.1
%@see: 《代数学讲义(下册)》(李文威)
设\(A\)是一个非空集合,
\(V\)是域\(F\)上的一个线性空间.
若映射\(f\colon A \times V \to A\)满足\begin{gather*}
	(\forall p \in A)
	(\forall x,y \in V)
	[
		f(f(p,x),y)
		= f(p,x+y)
	], \\
	(\forall p \in A)
	[
		0 \in V
		\implies
		f(p,0) = p
	], \\
	(\forall p,q \in A)
	(\exists! x \in V)
	[
		f(p,x) = q
	],
\end{gather*}
则称“\((A,f)\)是与线性空间\(V\)关联的\DefineConcept{仿射空间}”.
把\(f\)称为“仿射空间\(A\)的\DefineConcept{加法}”,
在不致混淆的情况下把\(f(p,x)\)记为\(p + x\).
把\(A\)中的每一个元素称为“仿射空间\(A\)中的一个\DefineConcept{点}”.
对于\(A\)中任意两点\(p,q\),把满足\(f(p,x) = q\)的向量\(x \in V\)
称为“连接点\(p\)和点\(q\)的\DefineConcept{向量}”,记作\(\vec{pq}\)或\(q - p\).
把线性空间\(V\)的维数\(\dim V\)称为“仿射空间\(A\)的\DefineConcept{维数}”,记为\(\dim A\).
\end{definition}

\begin{property}%\label{theorem:仿射空间.仿射空间的减法1}
设\((A,f)\)是与线性空间\(V\)关联的仿射空间,
则\(x = \vec{pq}\)当且仅当\(f(p,x) = q\).
\end{property}

\begin{property}%\label{theorem:仿射空间.仿射空间的减法2}
设\((A,f)\)是与线性空间\(V\)关联的仿射空间,
则\(f(p,v_1) = f(p,v_2)\)当且仅当\(v_1 = v_2\).
\end{property}

\begin{proposition}%\label{theorem:仿射空间.仿射空间与线性空间等势}
设\(A\)是与线性空间\(V\)关联的仿射空间,
则\(A\)与\(V\)等势.
\begin{proof}
由于仿射空间\(A\)是非空的,我们可以从中任意取定一个点\(o\).
然后构造一个从\(A\)到\(V\)的映射\(\sigma\colon A \to V, p \mapsto \vec{op}\).
根据仿射空间的定义,
对于\(A\)中任意一点\(p\),
满足\(f(o,v) = p\)的向量\(v \in V\)存在且唯一,
由此保证\(\sigma\)是单值的,且\(\ran\sigma \subseteq V\).
由\begin{equation*}
	(\forall p_1,p_2 \in A)
	[
		\sigma(p_1) = \sigma(p_2) = v
		\iff
		\vec{op_1} = \vec{op_2} = v
		\iff
		f(o,v) = p_1 = p_2
	]
\end{equation*}
可知\(\sigma\)是单射.
因为\begin{equation*}
	\sigma(q) = v
	\iff  % 映射\(\sigma\)的定义
	\vec{oq} = v
	\iff  % “连接两点的向量”的定义,\cref{theorem:仿射空间.仿射空间的减法1}
	f(o,v) = q,
\end{equation*}
% 又因为\(f(o,v) = f(o,v)\),
所以\(
	(\forall v \in V)
	[
		\sigma(f(o,v)) = v
	]
\),
可知\(\sigma\)是满射.
因此\(\sigma\)是从\(A\)到\(V\)的双射,
\(A\)与\(V\)等势.
\end{proof}
\end{proposition}

\begin{example}
%@see: 《基础代数(第二卷)》(席南华) P162 例4.2
设\(V\)是域\(F\)上的一个线性空间.
证明:\(V\)是一个仿射空间.
\begin{proof}
任意取定\(p,q \in V\).
对于任意\(x,y \in V\),
利用向量加法的结合律可得\begin{equation*}
	(p + x) + y
	= p + (x + y),
	\qquad
	p + 0
	= p.
\end{equation*}
同时,我们还有\begin{equation*}
	p + x = q
	\iff
	x = q - p \in V,
\end{equation*}
这就说明\(
	(\exists! x \in V)
	[
		f(p,x) = q
	]
\).
综上所述,\((V,+)\)是与\(V\)关联的仿射空间.
\end{proof}
\end{example}

\begin{example}\label{example:仿射空间.仿射空间与仿射几何的联系1}
%@see: 《基础代数(第二卷)》(席南华) P162 例4.3
设\(V\)是域\(F\)上的一个线性空间,\(W\)是\(V\)的一个子空间,
\(S\)是以向量\(\alpha \in V\)为代表、子空间\(W\)的一个陪集.
证明:\((S,+)\)是与\(W\)关联的仿射空间.
\begin{proof}
由题意有\(S = \alpha + W\).
任意取定\(p,q \in S\),
则存在\(w_1,w_2 \in W\)
使得\(p = \alpha + w_1, q = \alpha + w_2\).
由向量加法的封闭性可知\(p,q \in V\).
对于任意\(x,y \in W\),
利用向量加法的结合律可得\begin{equation*}
	(p + x) + y
	= p + (x + y),
	\qquad
	p + 0
	= p.
\end{equation*}
同时,我们还有\begin{equation*}
	p + x = q
	\iff
	x = q - p
	\iff
	x = (\alpha + w_2) - (\alpha + w_1)
	\iff
	x = w_2 - w_1 \in W,
\end{equation*}
这就说明\(
	(\exists! x \in W)
	[
		f(p,x) = q
	]
\).
综上所述,\((S,+)\)是与\(W\)关联的仿射空间.
\end{proof}
\end{example}

\subsection{仿射空间中的仿射映射与仿射同构}
\begin{definition}%\label{definition:仿射空间.仿射映射}
%@see: 《基础代数(第二卷)》(席南华) P163 定义4.4
设\(V,V'\)都是域\(F\)上的线性空间,
\(A,A'\)分别是与线性空间\(V,V'\)关联的仿射空间.
如果存在线性映射\(g \in \Hom(V,V')\)
使得映射\(f\colon A \to A'\)满足\begin{equation*}
%@see: 《基础代数(第二卷)》(席南华) P163 (4.1.1)
	(\forall p \in A)
	(\forall v \in V)
	[
		f(p + v)
		= f(p) + g(v)
	],
\end{equation*}
则称“\(f\)是从\(A\)到\(A'\)的一个\DefineConcept{仿射映射}”;
把\(g\)称为“仿射映射\(f\)的\DefineConcept{线性部分}”
或“仿射映射\(f\)的\DefineConcept{微分}”,
记作\(\dd{f}\).
\end{definition}

\begin{definition}
设\(V,V'\)都是域\(F\)上的线性空间,
\(A,A'\)分别是与线性空间\(V,V'\)关联的仿射空间.
如果映射\(f\colon A \to A\)是从\(A\)到\(A\)的一个仿射映射,
则称“\(f\)是\(A\)上的一个\DefineConcept{仿射变换}”.
\end{definition}

\begin{definition}%\label{definition:仿射空间.平移变换}
%@see: 《基础代数(第二卷)》(席南华) P162
设\(A\)是与线性空间\(V\)关联的仿射空间,
向量\(\alpha \in V\),
把映射\begin{equation*}
	\tau_\alpha\colon A \to A,
	p \mapsto p + \alpha
\end{equation*}
称为“(仿射空间\(A\)上)由向量\(\alpha\)决定的\DefineConcept{平移变换}”.
\end{definition}

\begin{theorem}%\label{theorem:仿射空间.平移变换对乘法成群}
%@see: 《基础代数(第二卷)》(席南华) P162
设\(V\)是域\(F\)上的一个线性空间,
\(A\)是与线性空间\(V\)关联的仿射空间,
\(\alpha,\beta \in V\),
\(\tau_\alpha,\tau_\beta\)分别是\(A\)上由\(\alpha,\beta\)决定的平移变换,
\(I\)是\(A\)上的恒等映射,
则\begin{gather*}
	\tau_\alpha \tau_\beta = \tau_{\alpha + \beta}, \\
	\tau_{-\alpha} \tau_\alpha = I.
\end{gather*}
\end{theorem}
\begin{remark}
%@see: 《基础代数(第二卷)》(席南华) P162
上述定理说明:
两个平移的复合还是平移,
平移的逆映射也是平移.
全体平移对复合成群,它同构于线性空间\(V\)的加法群.
不难验证:
对于任意\(x,y \in F\),
如果令\begin{equation*}
	x \tau_\alpha + y \tau_\beta
	\defeq \tau_{x \alpha + y \beta},
\end{equation*}
则\(A\)的全体平移就成为一个线性空间(记作\(A^\sharp\)),它与\(V\)同构.
\end{remark}

\begin{definition}%\label{definition:仿射空间.伸缩变换}
%@see: 《基础代数(第二卷)》(席南华) P163 例4.6
设\(V\)是域\(F\)上的一个线性空间,
\(A\)是与线性空间\(V\)关联的仿射空间,
\(o\)是\(A\)中的一个点,
数\(\lambda \in F\),
把映射\begin{equation*}
	\delta_\lambda\colon A \to A,
	p \mapsto o + \lambda\vec{op}
\end{equation*}
称为“(仿射空间\(A\)上)以点\(o\)为中心、\(\lambda\)为伸缩率的\DefineConcept{伸缩变换}(dilatation)”.
\end{definition}

\begin{proposition}\label{theorem:仿射空间.仿射映射的微分的唯一性}
%@see: 《基础代数(第二卷)》(席南华) P163
设\(A,A'\)都是域\(F\)上的仿射空间,
\(p,q\)是\(A\)中两点,
\(f\)是从\(A\)到\(A'\)的一个仿射映射,
记\(p' \defeq f(p), q' \defeq f(q)\),
则\(
%@see: 《基础代数(第二卷)》(席南华) P163 (4.1.2)
	\vec{p'q'}
	= \dd{f}(\vec{pq})
\).
\end{proposition}
\begin{remark}
由\cref{theorem:仿射空间.仿射映射的微分的唯一性} 可以看出,
仿射映射的微分存在且唯一.
\end{remark}

\begin{definition}%\label{definition:仿射空间.常值变换}
%@see: 《基础代数(第二卷)》(席南华) P163 例4.5(1)
设\(A\)是一个仿射空间,
点\(a \in A\).
把映射\(f\colon A \to A, x \mapsto a\)
称为“仿射空间\(A\)上的\DefineConcept{常值变换}”.
\end{definition}

\begin{proposition}%\label{theorem:仿射空间.常值变换的微分是零映射}
%@see: 《基础代数(第二卷)》(席南华) P163 例4.5(1)
仿射空间\(A\)上的常值变换的微分是零映射.
\end{proposition}

\begin{definition}%\label{definition:仿射空间.仿射同构}
%@see: 《基础代数(第二卷)》(席南华) P163 定义4.4
设\(V,V'\)都是域\(F\)上的线性空间,
\(A,A'\)分别是与线性空间\(V,V'\)关联的仿射空间.
如果从\(A\)到\(A'\)的一个仿射映射\(f\)是双射,
则称\(f\)是“从\(A\)到\(A'\)的一个\DefineConcept{仿射同构}”.
如果存在从\(A\)到\(A'\)的一个仿射同构,
则称“仿射空间\(A\)与\(A'\) \DefineConcept{同构}”.
\end{definition}

\begin{definition}%\label{definition:仿射空间.仿射自同构}
%@see: 《基础代数(第二卷)》(席南华) P163 定义4.4
设\(V\)都是域\(F\)上的线性空间,
\(A\)分别是与线性空间\(V\)关联的仿射空间,
\(f\)是从\(A\)到\(A\)的一个仿射映射.
如果\(f\)是双射,
则称\(f\)是“\(A\)上的一个\DefineConcept{仿射自同构}”.
\end{definition}

\begin{proposition}%\label{theorem:仿射空间.仿射映射是双射当且仅当它的微分是双射}
%@see: 《基础代数(第二卷)》(席南华) P164 命题4.7
仿射映射\(f\)是双射,当且仅当\(f\)的微分\(\dd{f}\)是双射.
%TODO proof
\end{proposition}

\begin{proposition}%\label{theorem:仿射空间.两个仿射空间同构当且仅当它们的维数相同}
%@see: 《基础代数(第二卷)》(席南华) P164 定理4.8
设\(A\)和\(A'\)是域\(F\)上的两个仿射空间,
则\(A\)与\(A'\)同构的充分必要条件是\(\dim A = \dim A'\).
\end{proposition}
