\section{射影几何}
本节利用构造法描述射影几何中的几何元素.

\subsection{射影几何}
\begin{definition}
%@see: 《高等代数与解析几何(第三版 下册)》(孟道骥) P452 定义10.5.1
设\(V\)是域\(F\)上的一个线性空间.
把\(V\)的全体子空间\begin{equation*}
	\Set{
		W
		\given
		W \AlgebraSubstructure V
	}
\end{equation*}
称为“域\(F\)上线性空间\(V\)上的\DefineConcept{射影几何}(projective geometry)”,
记为\(\mathcal{P}(V)\).
%@see: https://mathworld.wolfram.com/ProjectiveGeometry.html
\end{definition}

\begin{definition}
%@see: 《高等代数与解析几何(第三版 下册)》(孟道骥) P452 定义10.5.2
设\(W\)是线性空间\(V\)的一个子空间.
把\((\dim W - 1)\)称为“\(W\)的\DefineConcept{射影维数}”,
记为\(\pdim W\),
即\begin{equation*}
	\pdim W \defeq \dim W - 1.
\end{equation*}

将\(\mathcal{P}(V)\)中\(0\)维元素称为\DefineConcept{射影点}.

将\(\mathcal{P}(V)\)中\(1\)维元素称为\DefineConcept{射影直线}.

将\(\mathcal{P}(V)\)中\(2\)维元素称为\DefineConcept{射影平面}.

将\(\mathcal{P}(V)\)中\((\pdim V-1)\)维元素称为\DefineConcept{射影超平面}.
\end{definition}

\begin{definition}
%@see: 《高等代数与解析几何(第三版 下册)》(孟道骥) P452 定义10.5.2
设\(W\)是线性空间\(V\)的一个子空间.
如果\(W\)是射影点,
则把\(W\)中的每一个非零向量称为“射影点\(W\)的一个\DefineConcept{齐性向量}”.
\end{definition}

\begin{definition}
%@see: 《高等代数与解析几何(第三版 下册)》(孟道骥) P452 定义10.5.2
设\(V\)是域\(F\)上的一个线性空间.
把射影几何\(\mathcal{P}(V)\)中全体射影点\begin{equation*}
	\Set{
		W \AlgebraSubstructure V
		\given
		\pdim W = 0
	}
\end{equation*}
称为“由射影几何\(\mathcal{P}(V)\)决定的\DefineConcept{射影空间}”.
\end{definition}

\begin{property}
%@see: 《高等代数与解析几何(第三版 下册)》(孟道骥) P452
线性空间\(V\)的零子空间\(0\)不在由\(\mathcal{P}(V)\)决定的射影空间中.
%TODO proof
\end{property}

\begin{definition}
%@see: 《高等代数与解析几何(第三版 下册)》(孟道骥) P452
把线性空间\(V\)的零子空间\(0\)
称为“(由\(\mathcal{P}(V)\)决定的)射影空间的\DefineConcept{空子集}”.
\end{definition}

\begin{proposition}
%@see: 《高等代数与解析几何(第三版 下册)》(孟道骥) P452
设\(M,N \in \mathcal{P}(V)\),
则\(M \cap N = 0\)
当且仅当\(\pdim(M \cap N) = -1\).
\end{proposition}

\begin{definition}
%@see: 《高等代数与解析几何(第三版 下册)》(孟道骥) P452
设\(M,N \in \mathcal{P}(V)\).
如果\(M \cap N = 0\),
则称“\(M\)与\(N\)是\DefineConcept{交错的}”.
\end{definition}

%@see: 《高等代数与解析几何(第三版 下册)》(孟道骥) P452
在2维射影几何中,
两个不同点的联是一条直线,
两条不同直线的交是一个点.
在3维射影几何中,
两个不同点的联是一条直线,
两个不同平面的交是一条直线,
两条不同的相交直线的联是一个平面,
两条不同的共面直线的交是一个点,
一个点与不包含该点的一条直线的联是一个平面,
一个平面与不在该平面中的一条直线的交是一个点.

\begin{theorem}
%@see: 《高等代数与解析几何(第三版 下册)》(孟道骥) P453
设\(U\)和\(W\)是线性空间\(V\)的两个子空间,
则\begin{equation*}
	\pdim(U + W)
	+ \pdim (U \cap W)
	= \pdim U
	+ \pdim W.
\end{equation*}
\end{theorem}

\subsection{射影几何与仿射几何的联系}
%@see: 《高等代数与解析几何(第三版 下册)》(孟道骥) P453
假设我们从同一个线性空间\(V\)出发,定义了仿射几何\(\mathcal{A}(V)\)和射影几何\(\mathcal{P}(V)\).
容易看出\(\mathcal{P}(V)\)是\(\mathcal{A}(V)\)的一个子集,
后者由线性空间\(V\)中的所有陪集组成,
前者却只含有那些包含零子空间(或者说经过原点)的陪集.
这种看法是自然而平凡的.
更有趣、更深刻的思想是将仿射几何作为射影几何的一部分,
换句话说,是给仿射几何添上一些东西使其成为射影几何.

\begin{theorem}\label{theorem:射影几何.嵌入定理}
%@see: 《高等代数与解析几何(第三版 下册)》(孟道骥) P453 定理10.5.1(嵌入定理)
设\(V\)是域\(F\)上的一个线性空间,
\(H\)是射影空间\(\mathcal{P}(V)\)中一个超平面,
\(\alpha \in V - H\),
则映射\(
	\phi\colon \mathcal{A}(\alpha + H) \to \mathcal{P}(V),
	S \mapsto \Span\{S\}
\)具有以下性质:\begin{itemize}
	\item \(\phi\)是双射;

	\item \(\mathcal{A}(\alpha + H)\)在\(\phi\)下的像是
	\(\mathcal{P}(H)\)在\(\mathcal{P}(V)\)中的补集,
	即\(\phi(\mathcal{A}(\alpha + H)) = \mathcal{P}(V) - \mathcal{P}(H)\);

	\item 对于任意\(S,T \in \mathcal{A}(\alpha + H)\),
	\(S \subseteq T\)当且仅当\(\phi(S) \subseteq \phi(T)\);

	\item 如果\(\bigcap_i S_i \neq \emptyset\),
	则\(\phi\left( \bigcap_i S_i \right) = \bigcap_i \phi(S_i)\);

	\item \(\phi\left( \bigvee_i S_i \right) = \sum_i S_i\);

	\item 对于任意\(S \in \mathcal{A}(\alpha + H)\),
	有\(\dim S = \pdim \phi(S)\);

	\item 对于任意\(S,T \in \mathcal{A}(\alpha + H)\),
	\(S \parallel T\)当且仅当\begin{equation*}
		\phi(S) \cap H \subseteq \phi(T) \cap H
		\quad\text{或}\quad
		\phi(T) \cap H \subseteq \phi(S) \cap H.
	\end{equation*}
\end{itemize}
%TODO proof
\end{theorem}
\begin{remark}
%@see: 《高等代数与解析几何(第三版 下册)》(孟道骥) P455
下面我们用三维空间给\cref{theorem:射影几何.嵌入定理} 一个形象的几何解释.
假设\(V\)是三维实向量空间,
\(H\)是\(V\)的一个二维子空间.
由于\(\alpha\)不在\(H\)中,
所以\(\alpha + H\)是一个平行于\(H\)但不经过原点\(O\)的平面.
假设\(B\)是\(\alpha + H\)上的一个点,
则\(\phi(B)\)是经过原点\(O\)和点\(B\)的一条直线,
同时\(\phi(B)\)也是\(\mathcal{P}(V)\)中的一个射影点.
假设\(\beta + M\)是\(\alpha + H\)中的一条直线(\(\dim M = 1\)),
则\(\phi(\beta + M)\)是经过原点\(O\)和直线\(\beta + M\)的一个平面,
同时\(\phi(\beta + M)\)是\(\mathcal{P}(V)\)中的一条射影直线.
假设\(\beta + M, \gamma + M\)是\(\alpha + H\)中两条不同的直线,
显然它们是平行的,它们的交是空集,
但是它们在\(\phi\)下的像的交是\(\mathcal{P}(V)\)中的一个射影点\(M\).
\end{remark}
\begin{remark}
%@see: 《高等代数与解析几何(第三版 下册)》(孟道骥) P456
从\cref{theorem:射影几何.嵌入定理} 我们可以看到,
只要把\(\mathcal{A}(\alpha + H)\)与它在\(\phi\)下的像\(A \defeq \phi(\mathcal{A}(\alpha + H))\)等同起来,
或者说把\(V\)的不在\(H\)中的子空间的集合等同起来,
则\(\mathcal{A}(\alpha + H)\)中的点、直线、平面等几何元素
就分别对应\(\mathcal{P}(V)\)中的射影点、射影直线、射影平面等几何元素.
在这种等同下,\(\mathcal{A}(\alpha + H)\)中元素的联、交等运算结果
就成为了\(A\)中对应元素的联、交等运算结果.
像这样,我们就可以将\(\mathcal{A}(\alpha + H)\)视为\(A\)的一部分或者\(\mathcal{P}(V)\)的一部分,
或者说,我们将仿射几何嵌入射影几何中了,
又或者说,我们给仿射几何\(\mathcal{A}(\alpha + H)\)添上一些元素(即添上\(H\))就能将其扩充为射影几何了.
\end{remark}

\subsection{无穷远点}
%@see: 《高等代数与解析几何(第三版 下册)》(孟道骥) P456
在射影几何\(\mathcal{P}(V)\)中没有平行的概念,
而在仿射几何\(A \defeq \mathcal{A}(\alpha + H)\)中有平行的概念.
我们可以想象\(\mathcal{P}(H)\)(或\(H\))是在\(A\)的无穷远处.

对于任意\(M \in A\),
如果\(\dim(M \cap H) = \dim M - 1\),
则称“\(M \cap H\)是\(\mathcal{P}(M)\)中\DefineConcept{无穷远处的超平面}”.
这样,\(H\)就是\(\mathcal{P}(V)\)的唯一的无穷远处的超平面.

假设\(M\)是\(A\)中一条射影直线,
则称“\(M \cap H\)是射影直线\(M\)上的\DefineConcept{无穷远点}”.

假设\(N\)是\(A\)中一个射影平面,
则称“\(N \cap H\)是射影平面\(N\) \DefineConcept{在无穷远处的直线}”.

\cref{theorem:射影几何.嵌入定理} 说明:
对于任意\(S,T \in A\),
\(S \parallel T\)
当且仅当\(S\)在无穷远处的超平面包含或包含于\(T\)在无穷远处的超平面.
特别地,\(A\)中两条直线相互平行,
当且仅当它们作为\(\mathcal{P}(V)\)中的射影直线交于无穷远点;
\(A\)中一条直线与一个平面平行,
当且仅当它们作为\(\mathcal{P}(V)\)的射影直线与射影平面时,
直线的无穷远点落在平面的无穷远处的直线上.

总结以上论述,人们可以认为射影几何是比仿射几何稍大一点的几何,
或者说,射影几何是仿射几何添上无穷远超平面的几何.

% \subsection{射影的构形}
%@see: 《高等代数与解析几何(第三版 下册)》(孟道骥) P456
% 射影几何与仿射几何的联系,使得我们可以画出射影的构形图.

% 设\(\mathcal{C}\)是射影平面\(\mathcal{P}(V)\)中的一个构形.
% %TODO 什么是构形?
% 任意取定任意一条射影直线\(H\),
% 取向量\(\alpha \in V - H\),
% 取\(\alpha + H\)与\(\mathcal{C}\)的截线(即\((\alpha + H) \cap \mathcal{C}\)的图形).
% %TODO 什么是截线?
% 截线产生仿射构形图,即在\(\phi^{-1}\)下的像.
% 不同截面\(\alpha + H\)可以产生不同的仿射构形图,
% 但是它们总是代表相同的射影构形图.
% 自然地,我们总是选取使\(\mathcal{C}\)失去的东西最少的仿射构形图.
% %TODO 构形图是什么?有什么用途?

\subsection{射影几何基本定理}
\begin{theorem}\label{theorem:射影几何基本定理.德萨格定理}
%@see: 《高等代数与解析几何(第三版 下册)》(孟道骥) P458 定理10.6.1(Desargues及逆)
两个三角形有透射中心,当且仅当它们有透射轴.
%TODO proof
\end{theorem}

\begin{theorem}\label{theorem:射影几何基本定理.帕普斯六边形定理}
%@see: 《高等代数与解析几何(第三版 下册)》(孟道骥) P458 定理10.6.2(Pappus)
设\(A,B,C\)与\(A',B',C'\)分别在两条共面直线上,
记\begin{equation*}
	L \defeq AB' \cap A'B,
	\qquad
	M \defeq BC' \cap B'C,
	\qquad
	N \defeq CA' \cap C'A,
\end{equation*}
则\(L,M,N\)共线.
%TODO proof
\end{theorem}

\begin{theorem}\label{theorem:射影几何基本定理.帕普斯调和定理}
%@see: 《高等代数与解析几何(第三版 下册)》(孟道骥) P458 定理10.6.3(调和结构)
设\(A,B\)是射影平面\(M\)上不同两点,点\(G\)在\(AB\)上,点\(C\)在\(M\)上但不在\(AB\)上,
\(D\)在\(CG\)上但不同于\(C,G\),
记\begin{equation*}
	E \defeq AD \cap BC,
	\qquad
	F \defeq BD \cap CA,
	\qquad
	H \defeq EF \cap AB,
\end{equation*}
则\(H\)与\(C,D\)的选择无关.
%TODO proof
\end{theorem}

\subsection{射影几何中的射影同构}
\begin{definition}
%@see: 《高等代数与解析几何(第三版 下册)》(孟道骥) P460 定义10.7.1
设\(P,P'\)都是射影几何,\(\sigma\)是从\(P\)到\(P'\)的一个双射.
如果\begin{equation*}
	\sigma(M_1) \subseteq \sigma(M_2)
	\iff
	M_1 \subseteq M_2,
\end{equation*}
那么称
“\(\sigma\)是从\(P\)到\(P'\)的一个\DefineConcept{同构}(isomorphism)”
“射影几何\(P\)与\(P'\)同构(\(P\) is \emph{isomorphic} to \(P'\))”,
记作\(P \Isomorphism P'\).
\end{definition}

%@see: 《高等代数与解析几何(第三版 下册)》(孟道骥) P460 性质1
%@see: 《高等代数与解析几何(第三版 下册)》(孟道骥) P460 性质2
%@see: 《高等代数与解析几何(第三版 下册)》(孟道骥) P460 性质3
显然,射影几何的同构关系是一个等价关系.

\begin{property}
%@see: 《高等代数与解析几何(第三版 下册)》(孟道骥) P461 性质4
设\(P,P'\)都是射影几何,\(\sigma\)是从\(P\)到\(P'\)的一个同构,
则对任意\(M,N \in P\),有\begin{gather*}
	\sigma(M \cap N) = \sigma(M) \cap \sigma(N), \\
	\sigma(M + N) = \sigma(M) + \sigma(N).
\end{gather*}
%TODO proof
\end{property}

\begin{theorem}%\label{theorem:射影几何.射影同构.射影几何同构的充分必要条件}
%@see: 《高等代数与解析几何(第三版 下册)》(孟道骥) P461 性质5
设\(P,P'\)都是射影几何,
则\(P \Isomorphism P'\)的充分必要条件是
\(\dim P = \dim P'\).
%TODO proof
\end{theorem}
%TODO 射影同构难道不要求它们的域相同吗?仿射同构(\cref{theorem:仿射几何.仿射同构.仿射几何同构的充分必要条件})就有这个要求!

\subsection{射影变换}
\begin{definition}
%@see: 《高等代数与解析几何(第三版 下册)》(孟道骥) P461 定义10.7.2
设\(V,V'\)都是域\(F\)上的线性空间,
\(P,P'\)分别是\(V,V'\)上的射影几何,
\(f\)是从\(V\)到\(V'\)的一个线性同构,
把映射\begin{equation*}
	\hat{f}\colon P \to P',
	M \mapsto f(M)
\end{equation*}
称为“从射影几何\(P\)到射影几何\(P'\)的一个\DefineConcept{射影变换}”,
记为\(\mathcal{P}(f)\).
\end{definition}

\begin{definition}
%@see: 《高等代数与解析几何(第三版 下册)》(孟道骥) P461 定义10.7.2
设\(V\)是域\(F\)上的一个线性空间,
\(P\)是\(V\)上的射影几何,
\(f\)是\(V\)上一个线性自同构,
把映射\begin{equation*}
	\hat{f}\colon P \to P,
	M \mapsto f(M)
\end{equation*}
称为“射影几何\(P\)上的一个\DefineConcept{直射变换}”.
\end{definition}

\begin{property}%\label{theorem:射影几何.射影变换.射影变换的乘法}
%@see: 《高等代数与解析几何(第三版 下册)》(孟道骥) P461
设\(f\colon V \to V'\)和\(g\colon V' \to V''\)都是线性同构,
则\begin{gather*}
	\mathcal{P}(gf) = \mathcal{P}(g) \mathcal{P}(f), \\
	\mathcal{P}(f^{-1}) = (\mathcal{P}(f))^{-1}.
\end{gather*}
%TODO proof
\end{property}

\begin{theorem}%\label{theorem:射影几何.射影变换.两个射影变换相等的充分必要条件}
%@see: 《高等代数与解析几何(第三版 下册)》(孟道骥) P461 定理10.7.1
设\(V,V'\)都是域\(F\)上的线性空间,
\(f,g\)都是从\(V\)到\(V'\)的线性同构,
则\(\mathcal{P}(f) = \mathcal{P}(g)\)
当且仅当对于任意\(k \in F - \{0\}\),
有\(g = k f\).
%TODO proof
\end{theorem}
