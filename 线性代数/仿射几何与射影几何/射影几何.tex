\section{射影几何}
本节利用构造法描述射影几何中的几何元素.

\subsection{射影几何}
\begin{definition}
%@see: 《高等代数与解析几何(第三版 下册)》(孟道骥) P452 定义10.5.1
设\(V\)是域\(F\)上的一个线性空间.
把\(V\)的全体子空间\begin{equation*}
	\Set{
		W
		\given
		W \AlgebraSubstructure V
	}
\end{equation*}
称为“域\(F\)上线性空间\(V\)上的\DefineConcept{射影几何}(projective geometry)”,
记为\(\mathcal{P}(V)\).
%@see: https://mathworld.wolfram.com/ProjectiveGeometry.html
\end{definition}

\begin{definition}
%@see: 《高等代数与解析几何(第三版 下册)》(孟道骥) P452 定义10.5.2
设\(W\)是线性空间\(V\)的一个子空间.
把\((\dim W - 1)\)称为“\(W\)的\DefineConcept{射影维数}”,
记为\(\pdim W\),
即\begin{equation*}
	\pdim W \defeq \dim W - 1.
\end{equation*}

将\(\mathcal{P}(V)\)中\(0\)维元素称为\DefineConcept{射影点}.

将\(\mathcal{P}(V)\)中\(1\)维元素称为\DefineConcept{射影直线}.

将\(\mathcal{P}(V)\)中\(2\)维元素称为\DefineConcept{射影平面}.

将\(\mathcal{P}(V)\)中\((\pdim V-1)\)维元素称为\DefineConcept{射影超平面}.
\end{definition}

\begin{definition}
%@see: 《高等代数与解析几何(第三版 下册)》(孟道骥) P452 定义10.5.2
设\(W\)是线性空间\(V\)的一个子空间.
如果\(W\)是射影点,
则把\(W\)中的每一个非零向量称为“射影点\(W\)的一个\DefineConcept{齐性向量}”.
\end{definition}

\begin{definition}
%@see: 《高等代数与解析几何(第三版 下册)》(孟道骥) P452 定义10.5.2
设\(V\)是域\(F\)上的一个线性空间.
把射影几何\(\mathcal{P}(V)\)中全体射影点\begin{equation*}
	\Set{
		W \AlgebraSubstructure V
		\given
		\pdim W = 0
	}
\end{equation*}
称为“由射影几何\(\mathcal{P}(V)\)决定的\DefineConcept{射影空间}”.
\end{definition}

\begin{property}
%@see: 《高等代数与解析几何(第三版 下册)》(孟道骥) P452
线性空间\(V\)的零子空间\(0\)不在由\(\mathcal{P}(V)\)决定的射影空间中.
%TODO proof
\end{property}

\begin{definition}
%@see: 《高等代数与解析几何(第三版 下册)》(孟道骥) P452
把线性空间\(V\)的零子空间\(0\)
称为“(由\(\mathcal{P}(V)\)决定的)射影空间的\DefineConcept{空子集}”.
\end{definition}

\begin{proposition}
%@see: 《高等代数与解析几何(第三版 下册)》(孟道骥) P452
设\(M,N \in \mathcal{P}(V)\),
则\(M \cap N = 0\)
当且仅当\(\pdim(M \cap N) = -1\).
\end{proposition}

\begin{definition}
%@see: 《高等代数与解析几何(第三版 下册)》(孟道骥) P452
设\(M,N \in \mathcal{P}(V)\).
如果\(M \cap N = 0\),
则称“\(M\)与\(N\)是\DefineConcept{交错的}”.
\end{definition}

%@see: 《高等代数与解析几何(第三版 下册)》(孟道骥) P452
在2维射影几何中,
两个不同点的联是一条直线,
两条不同直线的交是一个点.
在3维射影几何中,
两个不同点的联是一条直线,
两个不同平面的交是一条直线,
两条不同的相交直线的联是一个平面,
两条不同的共面直线的交是一个点,
一个点与不包含该点的一条直线的联是一个平面,
一个平面与不在该平面中的一条直线的交是一个点.

\begin{theorem}
%@see: 《高等代数与解析几何(第三版 下册)》(孟道骥) P453
设\(U\)和\(W\)是线性空间\(V\)的两个子空间,
则\begin{equation*}
	\pdim(U + W)
	+ \pdim (U \cap W)
	= \pdim U
	+ \pdim W.
\end{equation*}
\end{theorem}

\subsection{射影几何与仿射几何的联系}
%@see: 《高等代数与解析几何(第三版 下册)》(孟道骥) P453
假设我们从同一个线性空间\(V\)出发,定义了仿射几何\(\mathcal{A}(V)\)和射影几何\(\mathcal{P}(V)\).
容易看出\(\mathcal{P}(V)\)是\(\mathcal{A}(V)\)的一个子集,
后者由线性空间\(V\)中的所有陪集组成,
前者却只含有那些包含零子空间(或者说经过原点)的陪集.
这种看法是自然而平凡的.
更有趣、更深刻的思想是将仿射几何作为射影几何的一部分,
换句话说,是给仿射几何添上一些东西使其成为射影几何.

\begin{theorem}
%@see: 《高等代数与解析几何(第三版 下册)》(孟道骥) P453 定理10.5.1(嵌入定理)
设\(V\)是域\(F\)上的一个线性空间,
\(H\)是射影空间\(\mathcal{P}(V)\)中一个超平面,
\(\alpha \in V - H\),
则映射\(
	\phi\colon \mathcal{A}(\alpha + H) \to \mathcal{P}(V),
	S \mapsto \Span\{S\}
\)具有以下性质:\begin{itemize}
	\item \(\phi\)是双射;

	\item \(\mathcal{A}(\alpha + H)\)在\(\phi\)下的像是
	\(\mathcal{P}(H)\)在\(\mathcal{P}(V)\)中的补集,
	即\(\phi(\mathcal{A}(\alpha + H)) = \mathcal{P}(V) - \mathcal{P}(H)\);

	\item 对于任意\(S,T \in \mathcal{A}(\alpha + H)\),
	\(S \subseteq T\)当且仅当\(\phi(S) \subseteq \phi(T)\);

	\item 如果\(\bigcap_i S_i \neq \emptyset\),
	则\(\phi\left( \bigcap_i S_i \right) = \bigcap_i \phi(S_i)\);

	\item \(\phi\left( \bigvee_i S_i \right) = \sum_i S_i\);

	\item 对于任意\(S \in \mathcal{A}(\alpha + H)\),
	有\(\dim S = \pdim \phi(S)\);

	\item 对于任意\(S,T \in \mathcal{A}(\alpha + H)\),
	\(S \parallel T\)当且仅当
	\(\phi(S) \cap H \subseteq \phi(T) \cap H\)
	或\(\phi(T) \cap H \subseteq \phi(S) \cap H\).
\end{itemize}
%TODO proof
\end{theorem}
