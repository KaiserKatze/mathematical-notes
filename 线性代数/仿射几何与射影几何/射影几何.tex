\section{射影几何}
本节利用构造法描述射影几何中的几何元素.

\subsection{射影几何}
\begin{definition}
%@see: 《高等代数与解析几何(第三版 下册)》(孟道骥) P452 定义10.5.1
设\(V\)是域\(F\)上的一个线性空间.
把\(V\)的全体子空间\begin{equation*}
	\Set{
		W
		\given
		W \AlgebraSubstructure V
	}
\end{equation*}
称为“域\(F\)上线性空间\(V\)上的\DefineConcept{射影几何}(projective geometry)”,
记为\(\mathcal{P}(V)\).
%@see: https://mathworld.wolfram.com/ProjectiveGeometry.html
\end{definition}

\begin{definition}
%@see: 《高等代数与解析几何(第三版 下册)》(孟道骥) P452 定义10.5.2
设\(W\)是线性空间\(V\)的一个子空间.
把\((\dim W - 1)\)称为“\(W\)的\DefineConcept{射影维数}”,
记为\(\pdim W\),
即\begin{equation*}
	\pdim W \defeq \dim W - 1.
\end{equation*}

将\(\mathcal{P}(V)\)中\(0\)维元素称为\DefineConcept{射影点}.

将\(\mathcal{P}(V)\)中\(1\)维元素称为\DefineConcept{射影直线}.

将\(\mathcal{P}(V)\)中\(2\)维元素称为\DefineConcept{射影平面}.

将\(\mathcal{P}(V)\)中\((\pdim V-1)\)维元素称为\DefineConcept{射影超平面}.
\end{definition}

\begin{definition}
%@see: 《高等代数与解析几何(第三版 下册)》(孟道骥) P452 定义10.5.2
设\(W\)是线性空间\(V\)的一个子空间.
如果\(W\)是射影点,
则把\(W\)中的每一个非零向量称为“射影点\(W\)的一个\DefineConcept{齐性向量}”.
\end{definition}

\begin{definition}
%@see: 《高等代数与解析几何(第三版 下册)》(孟道骥) P452 定义10.5.2
设\(V\)是域\(F\)上的一个线性空间.
把射影几何\(\mathcal{P}(V)\)中全体射影点\begin{equation*}
	\Set{
		W \AlgebraSubstructure V
		\given
		\pdim W = 0
	}
\end{equation*}
称为“由射影几何\(\mathcal{P}(V)\)决定的\DefineConcept{射影空间}”.
\end{definition}

\begin{property}
%@see: 《高等代数与解析几何(第三版 下册)》(孟道骥) P452
线性空间\(V\)的零子空间\(0\)不在由\(\mathcal{P}(V)\)决定的射影空间中.
%TODO proof
\end{property}

\begin{definition}
%@see: 《高等代数与解析几何(第三版 下册)》(孟道骥) P452
把线性空间\(V\)的零子空间\(0\)
称为“(由\(\mathcal{P}(V)\)决定的)射影空间的\DefineConcept{空子集}”.
\end{definition}

\begin{proposition}
%@see: 《高等代数与解析几何(第三版 下册)》(孟道骥) P452
设\(M,N \in \mathcal{P}(V)\),
则\(M \cap N = 0\)
当且仅当\(\pdim(M \cap N) = -1\).
\end{proposition}

\begin{definition}
%@see: 《高等代数与解析几何(第三版 下册)》(孟道骥) P452
设\(M,N \in \mathcal{P}(V)\).
如果\(M \cap N = 0\),
则称“\(M\)与\(N\)是\DefineConcept{交错的}”.
\end{definition}
