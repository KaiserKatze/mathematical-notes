\section{\texorpdfstring{$\lambda$}{\textlambda} - 矩阵}
\subsection{\texorpdfstring{$\lambda$}{\textlambda} - 矩阵的概念}
\begin{definition}% lambda矩阵的定义
%@see: 《高等代数》(丁南庆、刘公祥、纪庆忠、郭学军) P265 定义6.1.1
%@see: 《矩阵分析》(史荣昌、魏丰) P55 定义2.1.1
设\(F[\lambda]\)是域\(F\)上的一元多项式环,
对于\(F[\lambda]\)中任意取定的\(mn\)个多项式\begin{equation*}
	a_{ij}(\lambda)
	\quad(i=1,2,\dotsc,m;j=1,2,\dotsc,n),
\end{equation*}
矩阵\begin{equation*}
	A(\lambda)
	\defeq
	\begin{bmatrix}
		a_{11}(\lambda) & a_{12}(\lambda) & \dotsb & a_{1n}(\lambda) \\
		a_{21}(\lambda) & a_{22}(\lambda) & \dotsb & a_{2n}(\lambda) \\
		\vdots & \vdots & & \vdots \\
		a_{m1}(\lambda) & a_{m2}(\lambda) & \dotsb & a_{mn}(\lambda) \\
	\end{bmatrix}
\end{equation*}
称为“域\(F\)上的一个\(m \times n\) \DefineConcept{$\lambda$ - 矩阵}(\(\lambda\) - matrix)”.
把\(A(\lambda)\)的多项式元素的最高次数\begin{equation*}
	\max_{i,j} \deg a_{ij}(\lambda)
	= \max_{1 \leq i \leq m} \max_{1 \leq j \leq n} \deg a_{ij}(\lambda)
\end{equation*}
称为“\(A(\lambda)\)的\DefineConcept{次数}”,
记作\(\deg A(\lambda)\).
特别地,\(n \times n\) \(\lambda\) - 矩阵
称为\(n\)阶\(\lambda\) - 矩阵.
\end{definition}

为了与域\(F\)上的\(\lambda\)矩阵相区别,
以后我们把域\(F\)上的矩阵称为\DefineConcept{数字矩阵}.
当我们把数字矩阵的元素看作零次多项式时,
数字矩阵就成为了一种特殊的\(\lambda\) - 矩阵.

当我们把某个数字矩阵\(A\)的特征矩阵\(\lambda E - A\)中的\(\lambda\)看作不定元时,
\(\lambda E - A\)就成为了一种特殊的\(\lambda\) - 矩阵.

为了简化描述,我们定义:\begin{align*}
	M_{m \times n}(F[\lambda])
	&\defeq
	\Set{
		A(\lambda)
		\given
		\text{\(A(\lambda)\)是一元多项式环\(F[\lambda]\)上的一个\(m \times n\)矩阵}
	}
	\\
	&\hphantom{:}= \Set{
		A(\lambda)
		\given
		\text{\(A(\lambda)\)是域\(F\)上的一个\(m \times n\) \(\lambda\) - 矩阵}
	}, \\
	M_n(F[\lambda])
	&\defeq
	M_{n \times n}(F[\lambda]).
\end{align*}

%@see: 《高等代数》(丁南庆、刘公祥、纪庆忠、郭学军) P267
域\(F\)上的\(m \times n\)数字矩阵的加法、
域\(F\)中的数与域\(F\)上的\(m \times n\)数字矩阵的纯量乘法、
域\(F\)上的\(m \times n\)数字矩阵与域\(F\)上的\(n \times s\)数字矩阵的乘法,
以及域\(F\)上的\(n\)阶数字矩阵的行列式,
都是通过域\(F\)中的加法、减法、乘法定义的,
因此域\(F\)上的\(m \times n\) \(\lambda\) - 矩阵的\DefineConcept{加法}、
\(F[\lambda]\)中的多项式与域\(F\)上的\(m \times n\) \(\lambda\) - 矩阵的\DefineConcept{纯量乘法}、
域\(F\)上的\(m \times n\) \(\lambda\) - 矩阵与域\(F\)上的\(n \times s\) \(\lambda\) - 矩阵的\DefineConcept{乘法},
以及\(n\)阶\(\lambda\) - 矩阵的\DefineConcept{行列式},
也可以类似地通过\(F[\lambda]\)中的多项式的加法、减法、乘法定义.
同理,我们可以定义域\(F\)上的\(m \times n\) \(\lambda\) - 矩阵的\DefineConcept{转置}.

%@see: 《高等代数》(丁南庆、刘公祥、纪庆忠、郭学军) P267
我们将\(\lambda\) - 矩阵的\DefineConcept{秩}定义为
矩阵中非零子式的最大阶数,记为\(\rank A(\lambda)\).
显然,域\(F\)上的一个\(n\)阶\(\lambda\) - 矩阵\(A(\lambda)\)的秩为\(n\)
当且仅当\(\DeterminantA{A(\lambda)} \neq 0\).

我们将\(\lambda\) - 矩阵的\DefineConcept{迹}定义为
矩阵的主对角线上的元素之和,记为\(\tr A(\lambda)\).

%@see: 《高等代数》(丁南庆、刘公祥、纪庆忠、郭学军) P267
域\(F\)上的数字矩阵的那些只涉及域\(F\)的加、减、乘三种运算的性质
在\(\lambda\) - 矩阵中仍然成立,
譬如,域\(F\)上的一个\(n\)阶\(\lambda\) - 矩阵\(A(\lambda)\)的伴随矩阵\(A(\lambda)^*\)
仍然是域\(F\)上的一个\(n\)阶\(\lambda\) - 矩阵;%
\hyperref[theorem:行列式.拉普拉斯定理]{拉普拉斯定理}%
和\hyperref[equation:线性方程组.柯西比内公式]{柯西--比内公式}%
在\(\lambda\) - 矩阵中仍然成立.

\subsection{\texorpdfstring{$\lambda$}{\textlambda} - 矩阵的逆}
%@see: 《高等代数》(丁南庆、刘公祥、纪庆忠、郭学军) P267
%@see: 《矩阵分析》(史荣昌、魏丰) P56 定义2.1.3
可逆\(\lambda\) - 矩阵与可逆数字矩阵可以类似定义.
设\(A(\lambda) \in M_n(F[\lambda])\),
若存在\(B(\lambda) \in M_n(F[\lambda])\)
使得\(
	A(\lambda) B(\lambda)
	B(\lambda) A(\lambda)
	= I_n
\),
其中\(I_n\)是\(n\)阶单位矩阵,
则称“\(A(\lambda)\)是\DefineConcept{可逆的}”
或“\(A(\lambda)\)是\DefineConcept{可逆矩阵}”;
并把\(B(\lambda)\)称为“\(A(\lambda)\)的\DefineConcept{逆}”,记为\(A(\lambda)^{-1}\).
可以证明任意\(\lambda\) - 矩阵\(A(\lambda)\)的逆\(B(\lambda)\)只要存在就必定唯一.
为了简化描述,我们定义:\begin{equation*}
	GL_n(F[\lambda])
	\defeq
	\Set{
		A(\lambda)
		\in M_n(F[\lambda])
		\given
		\text{$A(\lambda)$是可逆的}
	}.
\end{equation*}

\begin{definition}% 可逆多项式
设\(f(\lambda) \in F[\lambda]\).
如果存在\(g(\lambda) \in F[\lambda]\)
使得\(f(\lambda) g(\lambda) = 1\),
则称“\(f(\lambda)\)在\(F[\lambda]\)中\DefineConcept{可逆}”
或“\(f(\lambda)\)是\(F[\lambda]\)中的一个\DefineConcept{可逆多项式}”.
\end{definition}

\begin{theorem}% 多项式可逆的充分必要条件
设\(f(\lambda) \in F[\lambda]\),
则\(f(\lambda)\)是\(F[\lambda]\)中的一个可逆多项式,
当且仅当\(f(\lambda)\)是\(F[\lambda]\)中的一个零次多项式.
%TODO proof 根本原因是多项式只有非负次项,两个多项式相乘所得的新多项式的次数不小于原本的两个多项式
%TODO 是不是应该把这个定理和上面“可逆多项式”的定义放在{多项式}那一章里去?
\end{theorem}

\begin{theorem}
%@see: 《高等代数》(丁南庆、刘公祥、纪庆忠、郭学军) P267 定理6.1.2
%@see: 《矩阵分析》(史荣昌、魏丰) P56 定理2.1.1
设\(A(\lambda) \in M_n(F[\lambda])\),
则\begin{itemize}
	\item \(A(\lambda) \in GL_n(F[\lambda])\)
	当且仅当\(\DeterminantA{A(\lambda)}\)是一个零次多项式;
	\item 如果\(A(\lambda) \in GL_n(F[\lambda])\),
	则\(A(\lambda)^{-1} = \frac{1}{\DeterminantA{A(\lambda)}} A(\lambda)^*\),
	其中\(A(\lambda)^*\)是\(A(\lambda)\)的伴随矩阵.
\end{itemize}
%TODO proof
\end{theorem}

\subsection{\texorpdfstring{$\lambda$}{\textlambda} - 矩阵的初等变换与等价标准型}
下面将采用研究数字矩阵的等价标准型的思想和方法
来研究\(\lambda\) - 矩阵的等价标准型,
为此首先给出\(\lambda\) - 矩阵的初等变换.
\begin{definition}
%@see: 《高等代数》(丁南庆、刘公祥、纪庆忠、郭学军) P267 定义6.1.3
%@see: 《矩阵分析》(史荣昌、魏丰) P55 定义2.1.4
以下三种变换称为\(\lambda\) - 矩阵的\DefineConcept{初等行变换}:\begin{enumerate}
	\item 互换\(\lambda\) - 矩阵中两行的位置;
	\item 用\(F\)中一个非零数\(k\)乘\(\lambda\) - 矩阵的某一行;
	\item 将\(\lambda\) - 矩阵中某一行的\(f(\lambda)\)倍加到另一行,其中\(f(\lambda) \in F[\lambda]\).
\end{enumerate}
\end{definition}\begin{definition}
%@see: 《高等代数》(丁南庆、刘公祥、纪庆忠、郭学军) P267 定义6.1.3
%@see: 《矩阵分析》(史荣昌、魏丰) P55 定义2.1.4
以下三种变换称为\(\lambda\) - 矩阵的\DefineConcept{初等列变换}:\begin{enumerate}
	\item 互换\(\lambda\) - 矩阵中两列的位置;
	\item 用\(F\)中一个非零数\(k\)乘\(\lambda\) - 矩阵的某一列;
	\item 将\(\lambda\) - 矩阵中某一列的\(f(\lambda)\)倍加到另一列,其中\(f(\lambda) \in F[\lambda]\).
\end{enumerate}
\end{definition}

与数字矩阵一样,我们可以引入三类\DefineConcept{初等\(\lambda\) - 矩阵},
即对单位矩阵进行一次初等变换得到的矩阵:\begin{enumerate}
	\item 互换单位矩阵\(I_n\)的\(i\),\(j\)两行(列)所得的矩阵\begin{equation*}
		E(i,j)
		\defeq
		\begin{bmatrix}
			I_{i-1} & & & \\
			& 0 & & 1 & \\
			& & I_{j-i-1} & & \\
			& 1 & & 0 & \\
			& & & & I_{n-j}
		\end{bmatrix}_n;
	\end{equation*}

	\item 用非零数\(k\)乘以单位矩阵\(I_n\)的第\(i\)行(列)所得的矩阵\begin{equation*}
		E(i(k))
		\defeq
		\begin{bmatrix}
			I_{i-1} & & \\
			& k & \\
			& & I_{n-i}
		\end{bmatrix}_n;
	\end{equation*}

	\item 把单位矩阵\(I_n\)的第\(j\)行(第\(i\)列)的\(f(\lambda)\)倍加到第\(i\)行(第\(j\)列)所得的矩阵\begin{equation*}
		E(i,j(f(\lambda)))
		\defeq
		\begin{bmatrix}
			I_{i-1} & & & \\
			& 1 & & f(\lambda) & \\
			& & I_{j-i-1} & & \\
			& 0 & & 1 & \\
			& & & & I_{n-j}
		\end{bmatrix}_n
		\quad(i<j)
	\end{equation*}
	或\begin{equation*}
		E(i,j(f(\lambda)))
		\defeq
		\begin{bmatrix}
			I_{i-1} & & & \\
			& 1 & & 0 & \\
			& & I_{j-i-1} & & \\
			& f(\lambda) & & 1 & \\
			& & & & I_{n-j}
		\end{bmatrix}_n
		\quad(i>j).
	\end{equation*}
\end{enumerate}
%@see: 《高等代数》(丁南庆、刘公祥、纪庆忠、郭学军) P269
%@see: 《矩阵分析》(史荣昌、魏丰) P57
与初等数字矩阵类似,初等\(\lambda\) - 矩阵都是可逆的,
并且\begin{gather*}
	E(i(k))^{-1}
	= E(i(k^{-1})), \\
	E(i,j(f(\lambda)))^{-1}
	= E(i,j(-f(\lambda))), \\
	E(i,j)^{-1} = E(i,j).
\end{gather*}
%@see: 《高等代数》(丁南庆、刘公祥、纪庆忠、郭学军) P269
%@see: 《矩阵分析》(史荣昌、魏丰) P57 定理2.1.2
与数字矩阵的初等变换一样,
\(\lambda\) - 矩阵的初等变换也不改变矩阵的秩,
并且对\(\lambda\) - 矩阵\(A(\lambda)\)施行一次初等行变换相当于在它的左边乘相应的初等\(\lambda\) - 矩阵,
对\(A(\lambda)\)施行一次初等列变换相当于在它的右边乘相应的初等\(\lambda\) - 矩阵.

%@see: 《高等代数》(丁南庆、刘公祥、纪庆忠、郭学军) P269 定义6.1.5
%@see: 《矩阵分析》(史荣昌、魏丰) P57 定义2.1.5
设\(A(\lambda), B(\lambda) \in M_{m \times n}(F[\lambda])\).
若\(A(\lambda)\)可以经过有限次初等变换化为\(B(\lambda)\),
则称“\(A(\lambda)\)与\(B(\lambda)\) \DefineConcept{等价}”
或“\(A(\lambda)\)与\(B(\lambda)\) \DefineConcept{相抵}”,
记作\(A(\lambda) \cong B(\lambda)\).

%@see: 《高等代数》(丁南庆、刘公祥、纪庆忠、郭学军) P269
%@see: 《矩阵分析》(史荣昌、魏丰) P57 定理2.1.3
由\(\lambda\) - 矩阵的初等变换与初等\(\lambda\) - 矩阵的关系可知,
\(m \times n\) \(\lambda\) - 矩阵\(A(\lambda)\)与\(B(\lambda)\)等价的充分必要条件
是存在\(m\)阶初等\(\lambda\) - 矩阵\(P_1(\lambda),\dotsc,P_s(\lambda)\)
以及\(n\)阶初等\(\lambda\) - 矩阵\(Q_1(\lambda),\dotsc,Q_t(\lambda)\)
使得\begin{equation*}
%@see: 《高等代数》(丁南庆、刘公祥、纪庆忠、郭学军) P270 (6.1.1)
	B(\lambda)
	= P_1(\lambda) \dotsm P_s(\lambda)
	A(\lambda)
	Q_1(\lambda) \dotsm Q_t(\lambda).
\end{equation*}

%@see: 《高等代数》(丁南庆、刘公祥、纪庆忠、郭学军) P270 定理6.1.6
%@see: 《矩阵分析》(史荣昌、魏丰) P57
\(\lambda\) - 矩阵的等价关系满足自反性、对称性、传递性.

\begin{lemma}
%@see: 《高等代数》(丁南庆、刘公祥、纪庆忠、郭学军) P270 引理6.1.7
设\(A(\lambda) \in M_{m \times n}(F[\lambda]) - \{0\}\),
则存在\(B(\lambda) = (b_{ij}(\lambda))_{m \times n} \in M_{m \times n}(F[\lambda])\)
满足\begin{itemize}
	\item \(B(\lambda)\)与\(A(\lambda)\)等价,
	\item \(b_{11}(\lambda) \neq 0\),
	\item \(
		\ExactlyDividedBy{ b_{ij}(\lambda) }{ b_{11}(\lambda) }  % \(b_{11}\)整除\(b_{ij}\)
		\ (i=1,2,\dotsc,m;j=1,2,\dotsc,n)
	\).
\end{itemize}
%TODO proof
\end{lemma}

\begin{lemma}
%@see: 《矩阵分析》(史荣昌、魏丰) P58 引理2.1.1
设\(A(\lambda) = (a_{ij})_{m \times n} \in M_{m \times n}(F[\lambda])\)满足\begin{itemize}
	\item \(a_{11}(\lambda) \neq 0\),
	\item \(A(\lambda)\)中至少有一个元素不能被\(a_{11}(\lambda)\)整除,
\end{itemize}
那么存在\(B(\lambda) = (b_{ij}(\lambda))_{m \times n} \in M_{m \times n}(F[\lambda])\)
满足\begin{itemize}
	\item \(B(\lambda)\)与\(A(\lambda)\)等价,
	\item \(b_{11}(\lambda) \neq 0\),
	\item \(\deg b_{11}(\lambda) < \deg a_{11}(\lambda)\).
\end{itemize}
%TODO proof
\end{lemma}

\begin{theorem}\label{theorem:lambda矩阵.lambda矩阵的史密斯标准型的存在性}
%@see: 《高等代数》(丁南庆、刘公祥、纪庆忠、郭学军) P271 定理6.1.8
%@see: 《矩阵分析》(史荣昌、魏丰) P58 定理2.1.4
设\(A(\lambda) \in M_{m \times n}(F[\lambda])\),
且\(\rank A(\lambda) = r \geq 1\),
则\(A(\lambda)\)等价于矩阵\begin{equation*}
	S(\lambda)
	\defeq
	\begin{bmatrix}
		d_1(\lambda) \\
		& d_2(\lambda) \\
		&& \ddots \\
		&&& d_r(\lambda) \\
		&&&& 0_{(m-r) \times (n-r)} \\
	\end{bmatrix}_{m \times n},
\end{equation*}
其中\(d_1(\lambda),d_2(\lambda),\dotsc,d_r(\lambda)\)都是\(F\)上的首一多项式,
且\(
	\ExactlyDividedBy{ d_{i+1}(\lambda) }{ d_i(\lambda) }  % \(d_i\)整除\(d_{i+1}\)
	\ (i=1,2,\dotsc,r-1)
\).
%TODO proof
\end{theorem}

\begin{definition}
%@see: 《高等代数》(丁南庆、刘公祥、纪庆忠、郭学军) P272 定理6.1.8
%@see: 《高等代数》(丁南庆、刘公祥、纪庆忠、郭学军) P275 定义6.2.4
%@see: 《矩阵分析》(史荣昌、魏丰) P59 定义2.1.6
我们把\cref{theorem:lambda矩阵.lambda矩阵的史密斯标准型的存在性} 中的
矩阵\(S(\lambda)\)称为“矩阵\(A(\lambda)\)的\DefineConcept{史密斯标准型}(Smith normal form)”,
把非零多项式\(d_1(\lambda),\dotsc,d_r(\lambda)\)称为“矩阵\(A(\lambda)\)的\DefineConcept{不变因子}(invariant factor)”.
特别地,规定零矩阵的史密斯标准型是它本身.
\end{definition}

\begin{corollary}
%@see: 《高等代数》(丁南庆、刘公祥、纪庆忠、郭学军) P273 推论6.1.9
设\(A(\lambda) \in M_{m \times n}(F[\lambda])\),
则存在\(m\)阶初等\(\lambda\) - 矩阵\(P_1(\lambda),\dotsc,P_s(\lambda)\)
和\(n\)阶初等\(\lambda\) - 矩阵\(Q_1(\lambda),\dotsc,Q_t(\lambda)\)
使得\(
	P_1(\lambda) \dotsm P_s(\lambda)
	A(\lambda)
	Q_1(\lambda) \dotsm Q_t(\lambda)
\)是\(A(\lambda)\)的史密斯标准型.
%TODO proof
\end{corollary}

\begin{corollary}
%@see: 《高等代数》(丁南庆、刘公祥、纪庆忠、郭学军) P273 推论6.1.10
%@see: 《矩阵分析》(史荣昌、魏丰) P66 推论2.1.2
设\(A(\lambda) \in M_n(F[\lambda])\),
则\(A(\lambda)\)可逆
当且仅当\(A(\lambda)\)可以表示为有限个初等\(\lambda\) - 矩阵的乘积.
%TODO proof
\end{corollary}

\begin{corollary}
%@see: 《矩阵分析》(史荣昌、魏丰) P66 推论2.1.1
设\(A(\lambda) \in M_n(F[\lambda])\),
则\(A(\lambda)\)可逆
当且仅当\(A(\lambda)\)与单位矩阵等价.
%TODO proof
\end{corollary}

\begin{corollary}
%@see: 《高等代数》(丁南庆、刘公祥、纪庆忠、郭学军) P274 推论6.1.11
设\(A(\lambda), B(\lambda) \in M_{m \times n}(F[\lambda])\),
则\(A(\lambda)\)与\(B(\lambda)\)等价的充分必要条件是
存在\(P(\lambda) \in GL_m(F[\lambda])\)及\(Q(\lambda) \in GL_n(F[\lambda])\)
使得\(
	B(\lambda)
	= P(\lambda) A(\lambda) Q(\lambda)
\).
%TODO proof
\end{corollary}

\subsection{\texorpdfstring{$\lambda$}{\textlambda} - 矩阵的等价不变量}
\cref{theorem:lambda矩阵.lambda矩阵的史密斯标准型的存在性}
证明了任意一个\(\lambda\) - 矩阵的史密斯标准型的存在性,
接下来我们继续研究\(\lambda\) - 矩阵的等价不变量,
并证明\(\lambda\) - 矩阵的史密斯标准型的唯一性.

\begin{definition}
%@see: 《高等代数》(丁南庆、刘公祥、纪庆忠、郭学军) P274 定义6.2.1
%@see: 《矩阵分析》(史荣昌、魏丰) P64 定义2.1.7
设\(A(\lambda) \in M_{m \times n}(F[\lambda])\)且\(\rank A(\lambda) = r \geq 1\).
对于任意一个正整数\(k\ (1 \leq k \leq r)\),
% 为什么下面不需要强调\(k\)阶子式是非零子式呢?这是因为零多项式可以被任意一个多项式整除!
% 因为至少有一个\(k\)阶子式是非零多项式,所以\(k\)阶行列式因子一定是非零多项式!
\(A(\lambda)\)的所有\(k\)阶子式的首一最大公因式
称为“\(A(\lambda)\)的\(k\)阶\DefineConcept{行列式因子}(determinant divisor)”,
记为\(D_k(A(\lambda))\),
或在不致混淆的情况下简记为\(D_k(\lambda)\).
\end{definition}

显然,秩为\(r\ (\geq1)\)的\(\lambda\) - 矩阵\(A(\lambda)\)的行列式因子
\(D_1(\lambda),D_2(\lambda),\dotsc,D_r(\lambda)\)都是首一多项式.
由{行列式的展开定理}可知,
\(A(\lambda)\)的任意一个\(i+1\)阶子式都是
\(A(\lambda)\)的\(i\)阶子式的组合,
于是\(D_i(\lambda)\)整除\(A(\lambda)\)的每一个\(i+1\)阶子式,
归纳可得\(
	\ExactlyDividedBy{ D_{i+1}(\lambda) }{ D_i(\lambda) }
	\ (i=1,2,\dotsc,r-1)
\).

\begin{theorem}
%@see: 《高等代数》(丁南庆、刘公祥、纪庆忠、郭学军) P274 定理6.2.2
%@see: 《矩阵分析》(史荣昌、魏丰) P64 定理2.1.5
\(\lambda\) - 矩阵的初等变换不改变其行列式因子.
%TODO proof
\end{theorem}
\begin{remark}
上述定理说明:两个等价的\(\lambda\) - 矩阵
具有相同的各阶行列式因子,有相同的秩.
\end{remark}

\begin{theorem}
%@see: 《高等代数》(丁南庆、刘公祥、纪庆忠、郭学军) P274 定理6.2.3
%@see: 《矩阵分析》(史荣昌、魏丰) P65 定理2.1.6
\(\lambda\) - 矩阵的史密斯标准型是唯一的.
%TODO proof
\end{theorem}

\begin{theorem}\label{theorem:lambda矩阵.两个lambda矩阵等价的等价条件}
%@see: 《高等代数》(丁南庆、刘公祥、纪庆忠、郭学军) P275 定理6.2.5
%@see: 《矩阵分析》(史荣昌、魏丰) P65 定理2.1.7
%@see: 《矩阵分析》(史荣昌、魏丰) P65 定理2.1.8
设\(A(\lambda), B(\lambda) \in M_{m \times n}(F[\lambda])\)且\(A(\lambda) \neq 0, B(\lambda) \neq 0\),
则以下命题等价:\begin{itemize}
	\item \(A(\lambda)\)与\(B(\lambda)\)等价;
	\item \(A(\lambda)\)与\(B(\lambda)\)具有相同的不变因子;
	\item \(A(\lambda)\)与\(B(\lambda)\)具有相同的行列式因子.
\end{itemize}
%TODO proof
\end{theorem}
\begin{remark}
根据上述定理,求解一个\(\lambda\) - 矩阵的不变因子有两种方法:
一种是利用\(\lambda\) - 矩阵的元素具有的某些特性,先求行列式因子,再求不变因子;
一种是先应用初等变换将矩阵化为史密斯标准型,或者应用初等变换将矩阵化为对角\(\lambda\) - 矩阵并求出行列式因子,再写出不变因子.
\end{remark}

\subsection{\texorpdfstring{$\lambda$}{\textlambda} - 矩阵的初等因子}
%@see: 《高等代数》(丁南庆、刘公祥、纪庆忠、郭学军) P277 定义6.2.9
%@see: 《矩阵分析》(史荣昌、魏丰) P66 定义2.2.1
设\(A(\lambda) \in M_{m \times n}(F[\lambda])\)且\(\rank A(\lambda) = r \geq 1\).
将\(A(\lambda)\)的不变因子\(
	d_1(\lambda),
	d_2(\lambda),
	\dotsc,
	d_r(\lambda)
\)分解为互不相同的首一不可约因式的方幂的乘积\begin{equation*}
	\begin{array}{c}
		d_1(\lambda) = p_1(x)^{k_{11}} p_2(x)^{k_{12}} \dotsm p_t(x)^{k_{1t}}, \\
		d_2(\lambda) = p_1(x)^{k_{21}} p_2(x)^{k_{22}} \dotsm p_t(x)^{k_{2t}}, \\
		\hdotsfor{1} \\
		d_r(\lambda) = p_1(x)^{k_{r1}} p_2(x)^{k_{r2}} \dotsm p_t(x)^{k_{rt}},
	\end{array}
\end{equation*}
其中\(p_j(x)\ (j=1,2,\dotsc,t)\)都是域\(F\)上的首一不可约多项式,
\(k_{ij}\ (i=1,2,\dotsc,r;j=1,2,\dotsc,t)\)都是域\(F\)中的常数.
因为\(
	\ExactlyDividedBy{ d_{i+1}(\lambda) }{ d_i(\lambda) }  % \(d_i\)整除\(d_{i+1}\)
	\ (i=1,2,\dotsc,r-1)
\),
所以\(
	0 \leq k_{1j} \leq k_{2j} \leq \dotsb \leq k_{rj}
	\ (j=1,2,\dotsc,t)
\)
且\(
	k_{r1},
	k_{r2},
	\dotsc,
	k_{rt}
\)全不为零,
从而有“\(k_{ij} = 0\ (i=1,2,\dotsc,r-1;j=1,2,\dotsc,t)\)
蕴含\(k_{1j} = k_{2j} = \dotsb = k_{ij} = 0\)”.
我们把\begin{equation*}
%@see: 《矩阵分析》(史荣昌、魏丰) P67 (2.2.1)
	\begin{array}{c}
		p_1(x)^{k_{11}},
		p_2(x)^{k_{12}},
		\dotsc,
		p_t(x)^{k_{1t}}, \\
		p_1(x)^{k_{21}},
		p_2(x)^{k_{22}},
		\dotsc,
		p_t(x)^{k_{2t}}, \\
		\hdotsfor{1} \\
		p_1(x)^{k_{r1}},
		p_2(x)^{k_{r2}},
		\dotsc,
		p_t(x)^{k_{rt}}
	\end{array}
\end{equation*}
中不等于\(1\)的(即次数\(k_{ij} > 0\)的)首一不可约因式的方幂的汇集
称为“\(A(\lambda)\)的\DefineConcept{初等因子}(elementary divisor)”.
%@see: 《高等代数(第三版 下册)》(丘维声) P157 定义1

\begin{example}
%@see: 《高等代数》(丁南庆、刘公祥、纪庆忠、郭学军) P277 例6.2.10
如果\(\lambda\) - 矩阵\(A(\lambda)\)的不变因子是\begin{align*}
	d_1(\lambda) &\defeq \lambda(\lambda-1), \\
	d_2(\lambda) &\defeq \lambda(\lambda-1)(\lambda+1), \\
	d_3(\lambda) &\defeq \lambda(\lambda-1)^2(\lambda+1)^2,
\end{align*}
那么\(A(\lambda)\)的初等因子为\(
	\lambda,
	\lambda,
	\lambda,
	\lambda-1,
	\lambda-1,
	(\lambda-1)^2,
	\lambda+1,
	(\lambda+1)^2
\).
\end{example}

\begin{example}
%@see: 《矩阵分析》(史荣昌、魏丰) P67
如果\(\lambda\) - 矩阵\(A(\lambda)\)的不变因子是\begin{align*}
	d_1(\lambda) &\defeq 1, \\
	d_2(\lambda) &\defeq 1, \\
	d_3(\lambda) &\defeq (\lambda-2)^5(\lambda-3)^3, \\
	d_4(\lambda) &\defeq (\lambda-2)^5(\lambda-3)^4(\lambda+2),
\end{align*}
那么\(A(\lambda)\)的初等因子为\(
	(\lambda-2)^5,
	(\lambda-2)^5,
	(\lambda-3)^3,
	(\lambda-3)^4,
	\lambda+2
\).
\end{example}

\begin{theorem}
%@see: 《矩阵分析》(史荣昌、魏丰) P67
设\(A(\lambda), B(\lambda) \in M_{m \times n}(F[\lambda])\)且\(A(\lambda) \neq 0, B(\lambda) \neq 0\),
\(A(\lambda)\)与\(B(\lambda)\)等价,
则\(A(\lambda)\)与\(B(\lambda)\)的初等因子相同.
\begin{proof}
因为\(A(\lambda)\)与\(B(\lambda)\)等价,
所以由\cref{theorem:lambda矩阵.两个lambda矩阵等价的等价条件} 可知,
\(A(\lambda)\)与\(B(\lambda)\)的不变因子相同,
于是\(A(\lambda)\)与\(B(\lambda)\)的初等因子相同.
\end{proof}
\end{theorem}

\begin{example}
%@see: 《高等代数》(丁南庆、刘公祥、纪庆忠、郭学军) P278 注记6.2.12
%@see: 《矩阵分析》(史荣昌、魏丰) P67
举例说明:即便两个\(\lambda\) - 矩阵的初等因子相同,它们也可能不等价.
\begin{solution}
取\begin{equation*}
	A(\lambda)
	\defeq
	\begin{bmatrix}
		1 \\
		& \lambda-4 \\
		&& (\lambda-4)^2 \\
	\end{bmatrix}_3,
	\qquad
	B(\lambda)
	\defeq
	\begin{bmatrix}
		\lambda-4 \\
		& (\lambda-4)^2 \\
		&& 0 \\
	\end{bmatrix}_3,
\end{equation*}
则\(A(\lambda)\)与\(B(\lambda)\)的初等因子都是\(\lambda-4,(\lambda-4)^2\),
但是\(\rank A(\lambda) = 3\)而\(\rank B(\lambda) = 2\),它们的秩不相同,
从而\(A(\lambda)\)与\(B(\lambda)\)不等价.
\end{solution}
\end{example}

\begin{theorem}
%@see: 《高等代数》(丁南庆、刘公祥、纪庆忠、郭学军) P277 定理6.2.11
%@see: 《矩阵分析》(史荣昌、魏丰) P67 定理2.2.1
设\(A(\lambda), B(\lambda) \in M_{m \times n}(F[\lambda])\)且\(A(\lambda) \neq 0, B(\lambda) \neq 0\),
则\(A(\lambda)\)与\(B(\lambda)\)等价
当且仅当它们具有相同的秩和初等因子.
%TODO proof
\end{theorem}
\begin{remark}
从上述定理可以看出,\(\lambda\) - 矩阵的不变因子由它的秩和初等因子唯一确定.
\end{remark}

\begin{example}
%@see: 《矩阵分析》(史荣昌、魏丰) P68 例2.2.1
设\(A(\lambda)\)是复数域上的一个\(5\times6\) \(\lambda\) - 矩阵,
它的秩为\(4\),
它的初等因子为\begin{equation*}
	\lambda,
	\lambda,
	\lambda^2,
	\lambda-1,
	(\lambda-1)^2,
	(\lambda-1)^3,
	(\lambda+2\iu)^3,
	(\lambda-2\iu)^3.
\end{equation*}
求\(A(\lambda)\)的史密斯标准型.
\begin{solution}
首先写出\(A(\lambda)\)的不变因子\begin{align*}
	d_4(\lambda)
	&\defeq
	\lambda^2(\lambda-1)^3(\lambda+2\iu)^3(\lambda-2\iu)^3, \\
	d_3(\lambda)
	&\defeq
	\lambda(\lambda-1)^2, \\
	d_2(\lambda)
	&\defeq
	\lambda(\lambda-1), \\
	d_1(\lambda)
	&\defeq
	1,
\end{align*}
于是\(A(\lambda)\)的史密斯标准形为\begin{equation*}
	\begin{bmatrix}
		1 \\
		& \lambda (\lambda - 1) \\
		&& \lambda (\lambda - 1)^2 \\
		&&& \lambda^2 (\lambda-1)^3 (\lambda^2+4)^3 \\
		&&&& 0 & 0 \\
	\end{bmatrix}_{5\times6}.
\end{equation*}
\end{solution}
\end{example}

\begin{theorem}
%@see: 《高等代数》(丁南庆、刘公祥、纪庆忠、郭学军) P278 定理6.2.14
设\(A(\lambda) \in M_{m \times n}(F[\lambda])\)
经过有限次初等变换化为\begin{equation*}
%@see: 《高等代数》(丁南庆、刘公祥、纪庆忠、郭学军) P279 (6.2.2)
	B(\lambda) \defeq \begin{bmatrix}
		f_1(\lambda) \\
		& f_2(\lambda) \\
		&& \ddots \\
		&&& f_r(\lambda) \\
		&&&& 0_{(m-r)\times(n-r)} \\
	\end{bmatrix},
\end{equation*}
其中\(f_i(\lambda) \in F[\lambda]\)是首一多项式.
如果将\(f_1(\lambda),f_2(\lambda),\dotsc,f_r(\lambda)\)中
非零次多项式分解为互不相同的首一不可约因式的方幂的乘积,
那么这些首一不可约因式的方幂的汇集
就是\(A(\lambda)\)的全部初等因子.
\end{theorem}

\begin{theorem}
%@see: 《高等代数》(丁南庆、刘公祥、纪庆忠、郭学军) P279 定理6.2.15
%@see: 《矩阵分析》(史荣昌、魏丰) P68 定理2.2.2
%@see: 《矩阵分析》(史荣昌、魏丰) P71 定理2.2.3
%@see: 《矩阵分析》(史荣昌、魏丰) P71 定理2.2.4
设\(
	A_1(\lambda) \in M_{m \times n}(F[\lambda]),
	A_2(\lambda) \in M_{p \times q}(F[\lambda])
\),
令\begin{equation*}
	A(\lambda)
	\defeq
	\begin{bmatrix}
		A_1(\lambda) \\
		& A_2(\lambda)
	\end{bmatrix}
	\in M_{(m+p) \times (n+q)}(F[\lambda]),
\end{equation*}
则\(A_1(\lambda)\)的全部初等因子与\(A_2(\lambda)\)的全部初等因子的汇集
就是\(A(\lambda)\)的全部初等因子.
%TODO proof
\end{theorem}

%TODO
%@see: 《高等代数》(丁南庆、刘公祥、纪庆忠、郭学军) P276 例6.2.7
%@see: 《高等代数》(丁南庆、刘公祥、纪庆忠、郭学军) P278 例6.2.13
%@see: 《矩阵分析》(史荣昌、魏丰) P71 例2.2.2

\begin{example}
给定\(\lambda\) - 矩阵\begin{equation*}
%@see: 《高等代数(第三版 下册)》(丘维声) P138 (1)
	A(\lambda) = \begin{bmatrix}
		2 \lambda^3 + \lambda^2 + 1 & \lambda^2 - 3 \\
		\lambda^3 - 1 & 2 \lambda + 5
	\end{bmatrix},
\end{equation*}
我们可以按照整环上矩阵的加法、纯量乘法,
将它改写成\begin{align*}
%@see: 《高等代数(第三版 下册)》(丘维声) P138 (2)
	A(\lambda)
	&= \begin{bmatrix}
		2 \lambda^3 & 0 \\
		\lambda^3 & 0
	\end{bmatrix}
	+ \begin{bmatrix}
		\lambda^2 & \lambda^2 \\
		0 & 0
	\end{bmatrix}
	+ \begin{bmatrix}
		0 & 0 \\
		0 & 2 \lambda
	\end{bmatrix}
	+ \begin{bmatrix}
		1 & -3 \\
		-1 & 5
	\end{bmatrix} \\
	&= \lambda^3
	\begin{bmatrix}
		2 & 0 \\
		1 & 0
	\end{bmatrix}
	+ \lambda^2
	\begin{bmatrix}
		1 & 1 \\
		0 & 0
	\end{bmatrix}
	+ \lambda
	\begin{bmatrix}
		0 & 0 \\
		0 & 2
	\end{bmatrix}
	+ \begin{bmatrix}
		1 & -3 \\
		-1 & 5
	\end{bmatrix},
\end{align*}
其中\(\lambda^k\)的系数矩阵\begin{equation*}
	\begin{bmatrix}
		2 & 0 \\
		1 & 0
	\end{bmatrix},
	\qquad
	\begin{bmatrix}
		1 & 1 \\
		0 & 0
	\end{bmatrix},
	\qquad
	\begin{bmatrix}
		0 & 0 \\
		0 & 2
	\end{bmatrix},
	\qquad
	\begin{bmatrix}
		1 & -3 \\
		-1 & 5
	\end{bmatrix}
\end{equation*}都是域\(F\)上的矩阵.
\end{example}

假设我们把两个\(\lambda\) - 矩阵\(A(\lambda)\)和\(B(\lambda)\)
都展开成\begin{equation*}
	A(\lambda)
	= \sum_{i=0}^m \lambda^i \alpha_i,
	\qquad
	B(\lambda)
	= \sum_{i=0}^m \lambda^i \beta_i,
\end{equation*}
其中\(\alpha_i,\beta_i \in M_n(F)\),
那么根据
两个一元多项式相等的定义
以及两个\(\lambda\) - 矩阵相等的定义,
可以推出,
\(A(\lambda)\)与\(B(\lambda)\)相等,
当且仅当它们系数矩阵对应相等.
