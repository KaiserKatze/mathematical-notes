\section{\texorpdfstring{$\lambda$}{\textlambda} - 矩阵}
\begin{definition}% lambda矩阵的定义
%@see: 《高等代数》(丁南庆、刘公祥、纪庆忠、郭学军) P265 定义6.1.1
设\(F[\lambda]\)是域\(F\)上的一元多项式环,
对于\(F[\lambda]\)中任意取定的\(mn\)个多项式\begin{equation*}
	a_{ij}(\lambda)
	\quad(i=1,2,\dotsc,m;j=1,2,\dotsc,n),
\end{equation*}
矩阵\begin{equation*}
	A(\lambda)
	\defeq
	\begin{bmatrix}
		a_{11}(\lambda) & a_{12}(\lambda) & \dotsb & a_{1n}(\lambda) \\
		a_{21}(\lambda) & a_{22}(\lambda) & \dotsb & a_{2n}(\lambda) \\
		\vdots & \vdots & & \vdots \\
		a_{m1}(\lambda) & a_{m2}(\lambda) & \dotsb & a_{mn}(\lambda) \\
	\end{bmatrix}
\end{equation*}
称为“域\(F\)上的一个\(m \times n\) \DefineConcept{$\lambda$ - 矩阵}(\(\lambda\) - matrix)”.
把\(A(\lambda)\)的多项式元素的最高次数\begin{equation*}
	\max_{i,j} \deg a_{ij}(\lambda)
	= \max_{1 \leq i \leq m} \max_{1 \leq j \leq n} \deg a_{ij}(\lambda)
\end{equation*}
称为“\(A(\lambda)\)的\DefineConcept{次数}”,
记作\(\deg A(\lambda)\).
特别地,\(n \times n\) \(\lambda\) - 矩阵
称为\(n\)阶\(\lambda\) - 矩阵.
\end{definition}

为了与域\(F\)上的\(\lambda\)矩阵相区别,
以后我们把域\(F\)上的矩阵称为\DefineConcept{数字矩阵}.
当我们把数字矩阵的元素看作零次多项式时,
数字矩阵就成为了一种特殊的\(\lambda\) - 矩阵.

为了简化描述,我们定义:\begin{align*}
	M_{m \times n}(F[\lambda])
	&\defeq
	\Set{
		A(\lambda)
		\given
		\text{\(A(\lambda)\)是一元多项式环\(F[\lambda]\)上的一个\(m \times n\)矩阵}
	}
	\\
	&\hphantom{:}= \Set{
		A(\lambda)
		\given
		\text{\(A(\lambda)\)是域\(F\)上的一个\(m \times n\) \(\lambda\) - 矩阵}
	}.
\end{align*}


与域\(F\)上的矩阵类似,
我们也可以为\(M_{m \times n}(F[\lambda])\)
定义加法、纯量乘法、乘法、转置,
而且这四种运算满足与域\(F\)上矩阵一样的运算法则.
类似地,我们也可以定义整环\(R\)上\(n\)阶矩阵的行列式.
一般地,\(\lambda\) - 矩阵\(A(\lambda)\)的行列式是一个多项式.
而且我们在之前介绍的行列式的性质、行列式展开定理,
对于整环\(R\)上的\(n\)阶矩阵的行列式也成立.
对于整环\(R\)上的\(n\)阶矩阵\(A\),
有\begin{equation*}
	A A^* = A^* A = \abs{A} I,
\end{equation*}
其中\(A^*\)是\(A\)的伴随矩阵.

\begin{example}
给定\(\lambda\) - 矩阵\begin{equation*}
%@see: 《高等代数(第三版 下册)》(丘维声) P138 (1)
	A(\lambda) = \begin{bmatrix}
		2 \lambda^3 + \lambda^2 + 1 & \lambda^2 - 3 \\
		\lambda^3 - 1 & 2 \lambda + 5
	\end{bmatrix},
\end{equation*}
我们可以按照整环上矩阵的加法、纯量乘法,
将它改写成\begin{align*}
%@see: 《高等代数(第三版 下册)》(丘维声) P138 (2)
	A(\lambda)
	&= \begin{bmatrix}
		2 \lambda^3 & 0 \\
		\lambda^3 & 0
	\end{bmatrix}
	+ \begin{bmatrix}
		\lambda^2 & \lambda^2 \\
		0 & 0
	\end{bmatrix}
	+ \begin{bmatrix}
		0 & 0 \\
		0 & 2 \lambda
	\end{bmatrix}
	+ \begin{bmatrix}
		1 & -3 \\
		-1 & 5
	\end{bmatrix} \\
	&= \lambda^3
	\begin{bmatrix}
		2 & 0 \\
		1 & 0
	\end{bmatrix}
	+ \lambda^2
	\begin{bmatrix}
		1 & 1 \\
		0 & 0
	\end{bmatrix}
	+ \lambda
	\begin{bmatrix}
		0 & 0 \\
		0 & 2
	\end{bmatrix}
	+ \begin{bmatrix}
		1 & -3 \\
		-1 & 5
	\end{bmatrix},
\end{align*}
其中\(\lambda^k\)的系数矩阵\begin{equation*}
	\begin{bmatrix}
		2 & 0 \\
		1 & 0
	\end{bmatrix},
	\qquad
	\begin{bmatrix}
		1 & 1 \\
		0 & 0
	\end{bmatrix},
	\qquad
	\begin{bmatrix}
		0 & 0 \\
		0 & 2
	\end{bmatrix},
	\qquad
	\begin{bmatrix}
		1 & -3 \\
		-1 & 5
	\end{bmatrix}
\end{equation*}都是域\(F\)上的矩阵.
\end{example}

假设我们把两个\(\lambda\) - 矩阵\(A(\lambda)\)和\(B(\lambda)\)
都展开成\begin{equation*}
	A(\lambda)
	= \sum_{i=0}^m \lambda^i \alpha_i,
	\qquad
	B(\lambda)
	= \sum_{i=0}^m \lambda^i \beta_i,
\end{equation*}
其中\(\alpha_i,\beta_i \in M_n(F)\),
那么根据
两个一元多项式相等的定义
以及两个\(\lambda\) - 矩阵相等的定义,
可以推出,
\(A(\lambda)\)与\(B(\lambda)\)相等,
当且仅当它们系数矩阵对应相等.

\begin{example}
设\(A(\lambda)\)是一个\(5\times6\) \(\lambda\) - 矩阵,
它的秩为\(4\),
它的初等因子为\begin{equation*}
	\lambda,
	\lambda,
	\lambda^2,
	\lambda-1,
	(\lambda-1)^2,
	(\lambda-1)^3,
	(\lambda+2\iu)^3,
	(\lambda-2\iu)^3,
\end{equation*}
则\(A(\lambda)\)的史密斯标准形为\begin{equation*}
	\begin{bmatrix}
		1 \\
		& \lambda (\lambda - 1) \\
		&& \lambda (\lambda - 1)^2 \\
		&&& \lambda^2 (\lambda-1)^3 (\lambda^2+4)^3 \\
		&&&& 0 & 0 \\
	\end{bmatrix}.
\end{equation*}
\end{example}
