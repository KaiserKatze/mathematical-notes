
\section{\texorpdfstring{$\lambda$}{\textlambda} - 矩阵}
\begin{definition}
%@see: 《高等代数》(丁南庆、刘公祥、纪庆忠、郭学军) P265 定义6.1.1
设\(F[\lambda]\)是域\(F\)上的一元多项式环,
对于\(F[\lambda]\)中任意\(mn\)个多项式\(a_{ij}(\lambda)\ (i=1,2,\dotsc,m;j=1,2,\dotsc,n)\),
矩阵\begin{equation*}
	A(\lambda)
	\defeq
	\begin{bmatrix}
		a_{11}(\lambda) & a_{12}(\lambda) & \dotsb & a_{1n}(\lambda) \\
		a_{21}(\lambda) & a_{22}(\lambda) & \dotsb & a_{2n}(\lambda) \\
		\vdots & \vdots & & \vdots \\
		a_{n1}(\lambda) & a_{n2}(\lambda) & \dotsb & a_{nn}(\lambda) \\
	\end{bmatrix}
\end{equation*}
称为域\(F\)上的一个\(m \times n\) \DefineConcept{$\lambda$ - 矩阵}(\(\lambda\) - matrix).
\end{definition}


与域\(F\)上的矩阵类似,
我们也可以为
整环\(R\)上的矩阵
定义加法、纯量乘法、乘法,
而且这三种运算满足与域\(F\)上矩阵一样的运算法则.
类似地,我们也可以定义整环\(R\)上\(n\)阶矩阵的行列式.
而且我们在之前介绍的行列式的性质、行列式展开定理,
对于整环\(R\)上的\(n\)阶矩阵的行列式也成立.
对于整环\(R\)上的\(n\)阶矩阵\(A\),
有\begin{equation*}
	A A^* = A^* A = \abs{A} I,
\end{equation*}
其中\(A^*\)是\(A\)的伴随矩阵.

域\(F\)上的一元多项式环\(F[\lambda]\)是一个整环,
因此可以考虑环\(F[\lambda]\)上的\(n\)阶矩阵.
我们把环\(F[\lambda]\)上的\(n\)阶矩阵
称为 \DefineConcept{\(\lambda\) - 矩阵}.

\begin{example}
给定\(\lambda\) - 矩阵\begin{equation*}
%@see: 《高等代数(第三版 下册)》(丘维声) P138 (1)
	A(\lambda) = \begin{bmatrix}
		2 \lambda^3 + \lambda^2 + 1 & \lambda^2 - 3 \\
		\lambda^3 - 1 & 2 \lambda + 5
	\end{bmatrix},
\end{equation*}
我们可以按照整环上矩阵的加法、纯量乘法,
将它改写成\begin{align*}
%@see: 《高等代数(第三版 下册)》(丘维声) P138 (2)
	A(\lambda)
	&= \begin{bmatrix}
		2 \lambda^3 & 0 \\
		\lambda^3 & 0
	\end{bmatrix}
	+ \begin{bmatrix}
		\lambda^2 & \lambda^2 \\
		0 & 0
	\end{bmatrix}
	+ \begin{bmatrix}
		0 & 0 \\
		0 & 2 \lambda
	\end{bmatrix}
	+ \begin{bmatrix}
		1 & -3 \\
		-1 & 5
	\end{bmatrix} \\
	&= \lambda^3
	\begin{bmatrix}
		2 & 0 \\
		1 & 0
	\end{bmatrix}
	+ \lambda^2
	\begin{bmatrix}
		1 & 1 \\
		0 & 0
	\end{bmatrix}
	+ \lambda
	\begin{bmatrix}
		0 & 0 \\
		0 & 2
	\end{bmatrix}
	+ \begin{bmatrix}
		1 & -3 \\
		-1 & 5
	\end{bmatrix},
\end{align*}
其中\(\lambda^k\)的系数矩阵\begin{equation*}
	\begin{bmatrix}
		2 & 0 \\
		1 & 0
	\end{bmatrix},
	\qquad
	\begin{bmatrix}
		1 & 1 \\
		0 & 0
	\end{bmatrix},
	\qquad
	\begin{bmatrix}
		0 & 0 \\
		0 & 2
	\end{bmatrix},
	\qquad
	\begin{bmatrix}
		1 & -3 \\
		-1 & 5
	\end{bmatrix}
\end{equation*}都是域\(F\)上的矩阵.
\end{example}

假设我们把两个\(\lambda\) - 矩阵\(A(\lambda)\)和\(B(\lambda)\)
都展开成\begin{equation*}
	A(\lambda)
	= \sum_{i=0}^m \lambda^i \alpha_i,
	\qquad
	B(\lambda)
	= \sum_{i=0}^m \lambda^i \beta_i,
\end{equation*}
其中\(\alpha_i,\beta_i \in M_n(F)\),
那么根据
两个一元多项式相等的定义
以及两个\(\lambda\) - 矩阵相等的定义,
可以推出,
\(A(\lambda)\)与\(B(\lambda)\)相等,
当且仅当它们系数矩阵对应相等.
