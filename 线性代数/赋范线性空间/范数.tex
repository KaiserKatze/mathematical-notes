\section{范数}
尽管我们通常出于几何(特别是欧氏几何)的考量,
将向量\(\alpha\)的模(或范数)定义为\(\sqrt{\VectorInnerProductDot{\alpha}{\alpha}}\),
不过我们还可以定义其他形式的模(或范数).

\subsection{范数的定义}
\begin{definition}
%@see: 《矩阵分析与应用(第2版)》(张贤达) P23 定义1.3.7(范数和赋范向量空间)
设\(V\)是域\(F\)上的一个线性空间.
如果映射\(p\colon V \to \mathbb{R}\)
满足\begin{itemize}
	\item {\rm\bf 非负性}:\begin{equation*}
		(\forall \alpha \in V)
		[p(\alpha) \geq 0],
		\qquad
		(\forall \alpha \in V)
		[
			p(\alpha) = 0
			\iff
			\alpha = 0
		];
	\end{equation*}

	\item {\rm\bf 齐次性}:\begin{equation*}
		(\forall \alpha \in V)
		(\forall k \in F)
		[
			p(k \alpha) = \abs{k} p(\alpha)
		];
	\end{equation*}

	\item {\rm\bf 三角不等式}:\begin{equation*}
		(\forall \alpha,\beta \in V)
		[
			p(\alpha+\beta) \leq p(\alpha) + p(\beta)
		],
	\end{equation*}
\end{itemize}
则称“\(p\)是线性空间\(V\)的一个\DefineConcept{范数}(norm)”
“\((V,p)\)是域\(F\)上的一个\DefineConcept{赋范线性空间}”,
在不致混淆的情况下简称“\(V\)是一个\DefineConcept{赋范线性空间}”;
对于任意向量\(\alpha \in V\),
把\(\alpha\)在\(p\)下的像
称为“向量\(\alpha\)的\(p\) \DefineConcept{范数}”,
记作\(\VectorLengthN{\alpha}\),
即\begin{equation*}
	\VectorLengthN{\alpha}
	\defeq
	p(\alpha).
\end{equation*}
\end{definition}

\begin{example}
设\((V,p)\)是域\(F\)上的一个赋范线性空间,
其中\(\mathbb{R} \subseteq F \subseteq \mathbb{C}\).
给定\(\alpha \in V\)且\(\alpha \neq 0\).
证明:存在唯一的数\(k \in F\)和向量\(\epsilon \in V\),
使得\(\norm{\epsilon}_p = 1\)且\(\alpha = k \epsilon\).
\begin{proof}
%@credit: gemini
先证存在性.
因为\(\alpha \neq 0\),所以由范数的非负性可知\(p(\alpha) > 0\).
取\(k \defeq p(\alpha)\).
因为\(p(\alpha) > 0\),
所以\(k \in \mathbb{R}^+ \subseteq F\).
再取\(\epsilon \defeq k^{-1} \alpha\).
既然\(k \in \mathbb{R}^+\),
那么必有\(k^{-1} \in \mathbb{R}^+ \subseteq F\),
加之\(\alpha \in V\),
从而有\(\epsilon \in F\).
显然成立\(k \epsilon = k (k^{-1} \alpha) = (k k^{-1}) \alpha = \alpha\).
因为\(
	p(\epsilon)
	= p(k^{-1} \alpha)
	= \abs{k^{-1}} p(\alpha)  % 齐次性
	= k^{-1} k
	= 1
\),
所以\(\epsilon\)是单位向量.

再证唯一性.
假设存在另一组数\(k' \in \mathbb{R}^+\)和向量\(\epsilon' \in V\)
满足\(p(\epsilon') = 1\)且\(\alpha = k' \epsilon'\).
在等式\(\alpha = k' \epsilon'\)两边同时取范数\(p\),
得\(p(\alpha) = p(k' \epsilon')\).
由范数的齐次性可知\(p(\alpha) = \abs{k'} p(\epsilon')\).
考虑到\(k' \in \mathbb{R}^+\),便知\(\abs{k'} = k'\).
由假设\(p(\epsilon') = 1\)可知\(p(\alpha) = k' \cdot 1 = k'\).
因此,数\(k'\)必须等于\(p(\alpha)\),即\(k' = k\).
故\(k\)是唯一的.
既然\(k' = k\),那么有\(\alpha = k \epsilon = k \epsilon'\).
% 因为\(k \in \mathbb{R}^+\),所以\(k^{-1} \in \mathbb{R}^+ \subseteq F\).
在等式\(\alpha = k \epsilon'\)两边同时乘以\(k^{-1}\),
得\(k^{-1} \alpha = k^{-1} (k \epsilon') = (k^{-1} k) \epsilon' = \epsilon'\).
由于\(\epsilon = k^{-1} \alpha\),因此\(\epsilon' = \epsilon\).
故\(\epsilon\)也是唯一的.
\end{proof}
\end{example}

\subsection{半范数}
\begin{definition}
%@see: 《矩阵分析与应用(第2版)》(张贤达) P24 定义1.3.8
设\(V\)是域\(F\)上的一个线性空间.
如果映射\(p\colon V \to \mathbb{R}\)
满足\begin{itemize}
	\item {\rm\bf 非负性}:\begin{equation*}
		(\forall \alpha \in V)
		[p(\alpha) \geq 0];
	\end{equation*}

	\item {\rm\bf 齐次性}:\begin{equation*}
		(\forall \alpha \in V)
		(\forall k \in F)
		[
			p(k \alpha) = \abs{k} p(\alpha)
		];
	\end{equation*}

	\item {\rm\bf 三角不等式}:\begin{equation*}
		(\forall \alpha,\beta \in V)
		[
			p(\alpha+\beta) \leq p(\alpha) + p(\beta)
		],
	\end{equation*}
\end{itemize}
则称“\(f\)是线性空间\(V\)的一个\DefineConcept{半范数}(seminorm)”
或者“\(f\)是线性空间\(V\)的一个\DefineConcept{伪范数}(seminorm)”.
%@see: https://mathworld.wolfram.com/Seminorm.html
\end{definition}

半范数与范数的唯一区别是:
半范数不完全满足范数的非负性公理,
有可能当\(\alpha\neq0\)时成立\(p(\alpha) = 0\).
\begin{proposition}
设\(f\)是线性空间\(V\)的一个半范数.
如果\begin{equation*}
	(\forall \alpha \in V)
	[
		p(\alpha) = 0
		\iff
		\alpha = 0
	],
\end{equation*}
则\(f\)是线性空间\(V\)的一个范数.
\end{proposition}

\begin{example}
对于任意\(n\)维向量\(\alpha = (x_1,\dotsc,x_n)^T\),
只要满足\begin{equation*}
	x_1 + \dotsb + x_n = 0,
	% 即\(\alpha\)是“零均值向量”
\end{equation*}
则映射\begin{equation*}
	p(\alpha)
	\defeq
	x_1 + \dotsb + x_n
\end{equation*}
就是向量\(\alpha\)的一个半范数.
但是,显然\(p(\alpha) = 0\)并不意味着\(\alpha = 0\).
\end{example}

\subsection{拟范数}
\begin{definition}
%@see: 《矩阵分析与应用(第2版)》(张贤达) P24 定义1.3.9
设\(V\)是有序域\(F\)上的一个线性空间.
如果映射\(p\colon V \to \mathbb{R}\)
满足\begin{itemize}
	\item {\rm\bf 非负性}:\begin{equation*}
		(\forall \alpha \in V)
		[p(\alpha) \geq 0],
		\qquad
		(\forall \alpha \in V)
		[
			p(\alpha) = 0
			\iff
			\alpha = 0
		];
	\end{equation*}

	\item {\rm\bf 齐次性}:\begin{equation*}
		(\forall \alpha \in V)
		(\forall k \in F)
		[
			p(k \alpha) = \abs{k} p(\alpha)
		];
	\end{equation*}

	\item {\rm\bf 三角不等式}:\begin{equation*}
		(\forall \alpha,\beta \in V)
		(\exists c \in C)
		[
			p(\alpha+\beta) \leq c [p(\alpha) + p(\beta)]
		],
	\end{equation*}
	其中\(
		C
		\defeq
		\Set{
			x \in F
			\given
			x > 0,
			x \neq 1
		}
	\),
\end{itemize}
则称“\(f\)是线性空间\(V\)的一个\DefineConcept{拟范数}(quasinorm)”.
\end{definition}

拟范数与范数的唯一区别是:
拟范数不严格满足范数的三角不等式公理.

\begin{example}
%@see: 《矩阵分析与应用(第2版)》(张贤达) P24
同一个定义公式,有时给出拟范数,有时给出范数,取决于参数的变化.
例如,容易验证\begin{equation*}
%@see: 《矩阵分析与应用(第2版)》(张贤达) P24 (1.3.24)
	\norm{\alpha}_p
	\defeq
	\left( \sum_{i=1}^m x_i^p \right)^{1/p}
\end{equation*}
当\(0 < p < 1\)时给出的是拟范数,
当\(p \geq 1\)时给出的是范数.
\end{example}

\subsection{向量范数}
\begin{definition}
%@see: 《矩阵论》(詹兴致) P5
设\(\alpha \in \mathbb{C}^n\).
定义:\begin{equation}
	\norm{\alpha}_p
	\defeq
	\left( \sum_{i=1}^n \ComplexLengthA{a_i}^p \right)^{1/p},
\end{equation}
称之为“向量\(\alpha\)的 \DefineConcept{\(L_p\)范数}”,
其中\(\alpha = (\AutoTuple{a}{n})^T\).
\end{definition}

%@see: 《数值分析(第5版)》(李庆扬、王能超、易大义) P53
易见\begin{align}
	\norm{\alpha}_1
	&= \sum_{i=1}^n \abs{x_i}
	= \VectorLengthA{x_1} + \VectorLengthA{x_2} + \dotsb + \VectorLengthA{x_n}, \\
	\norm{\alpha}_2
	&= \left( \sum_{i=1}^n x_i^2 \right)^{\frac12}
	= \sqrt{x_1^2 + x_2^2 + \dotsb + x_n^2}, \\
	\norm{\alpha}_\infty
	&= \max_{1 \leq i \leq n} \abs{x_i}
	= \max\{\VectorLengthA{x_1},\VectorLengthA{x_2},\dotsc,\VectorLengthA{x_n}\}.
\end{align}

%@credit: {6c964576-9569-472e-969e-54699e35974b} 菌菌笔记
向量的\(L_1\)范数、\(L_2\)范数、\(L_\infty\)范数各有一个别名,
依次是\DefineConcept{曼哈顿范数}、\DefineConcept{欧氏范数}、\DefineConcept{最大范数}.

\subsection{函数范数}
\begin{definition}% 连续函数的范数
%@see: 《数值分析(第5版)》(李庆扬、王能超、易大义) P53
设\(f \in C[a,b]\).
定义:\begin{equation}
	\norm{f}_p
	\defeq
	\left( \int_a^b f^p(x) \dd{x} \right)^{\frac1p},
\end{equation}
称之为“函数\(f\)(在区间\([a,b]\)上)的 \DefineConcept{\(L_p\)范数}”.
\end{definition}

易见\begin{align}
	\norm{f}_\infty
	&\defeq
	\max_{a \leq x \leq b} \abs{f(x)},
		\label{equation:范数.连续函数的无穷范数} \\
	\norm{f}_1
	&\defeq
	\int_a^b \abs{f(x)} \dd{x}, \\
	\norm{f}_2
	&\defeq
	\left( \int_a^b f^2(x) \dd{x} \right)^{\frac12}.
		\label{equation:范数.连续函数的L2范数}
\end{align}

\begin{definition}\label{definition:范数.连续函数的带权范数}
设区间\(D \subseteq \mathbb{R}\).
由\hyperref[definition:欧几里得空间.连续函数的带权内积]{连续函数的带权内积}
\begin{equation*}
	(f,g)
	\defeq
	\int_D \rho(x) f(x) g(x) \dd{x}
\end{equation*}
导出的范数\begin{equation*}
%@see: 《数值分析(第5版)》(李庆扬、王能超、易大义) P55 (1.16)
	\sqrt{(f,f)}
	= \sqrt{
		\int_D \rho(x) f^2(x) \dd{x}
	}
\end{equation*}
称为“函数\(f\)的\DefineConcept{带权\(\rho\)的范数}”,
记作\(\norm{f}_2\).
\end{definition}

\subsection{矩阵范数}
由所有形状相同的矩阵组成的集合,对加法和标量乘法,成为一个线性空间.
于是我们可以在这个线性空间\(M_{m \times n}(K)\)上定义范数.

我们首先给出复数域上\(n\)阶方阵的矩阵范数的一般定义.
\begin{definition}
%@see: 《数值分析(第5版)》(李庆扬、王能超、易大义) P164 定义5(矩阵的范数)
设数域\(K \defeq \mathbb{C}\),
线性空间\(V \defeq M_n(K)\).
如果映射\(p\colon V \to \mathbb{R}\)
满足\begin{itemize}
	\item {\rm\bf 正定性}:\begin{equation*}
		(\forall \alpha \in V)
		[p(\alpha) \geq 0],
		\qquad
		(\forall \alpha \in V)
		[
			p(\alpha) = 0
			\iff
			\alpha = 0
		];
	\end{equation*}

	\item {\rm\bf 齐次性}:\begin{equation*}
		(\forall \alpha \in V)
		(\forall k \in K)
		[
			p(k \alpha) = \abs{k} p(\alpha)
		];
	\end{equation*}

	\item {\rm\bf 三角不等式}:\begin{equation*}
		(\forall \alpha,\beta \in V)
		[
			p(\alpha+\beta) \leq p(\alpha) + p(\beta)
		],
	\end{equation*}

	\item {\rm\bf 次可乘性}:\begin{equation*}
		(\forall \alpha,\beta \in V)
		[
			p(\alpha \beta) \leq p(\alpha) p(\beta)
		],
	\end{equation*}
\end{itemize}
则称“\(p\)是线性空间\(V\)的一个\DefineConcept{矩阵范数}(matrix norm)”.
\end{definition}

\begin{definition}
%@see: 《矩阵论》(詹兴致) P5
设\(A \in M_{m \times n}(\mathbb{C})\),
\(A^H\)是\(A\)的共轭转置.
定义:\begin{equation}
	\MatrixNorm{A}_F
	\defeq
	\sqrt{
		\tr(A^H A)
	},
\end{equation}
称之为“矩阵\(A\)的\DefineConcept{弗罗贝尼乌斯范数}”.
\end{definition}

\begin{property}
%@see: 《矩阵论》(詹兴致) P5
设\(A \in M_{m \times n}(\mathbb{C})\),
则\begin{equation*}
	\MatrixNorm{A}_F
	= \sqrt{
		\sum_{i=1}^m
		\sum_{j=1}^n
		\abs{a_{ij}}^2
	}.
\end{equation*}
%TODO proof
\end{property}

\begin{definition}
%@see: 《矩阵论》(詹兴致) P5
%@see: 《数值分析(第5版)》(李庆扬、王能超、易大义) P164 定义6(矩阵的算子范数)
设\(A \in M_{m \times n}(\mathbb{C})\),
\(f\)是\(\mathbb{C}^m\)上的一个范数,
\(g\)是\(\mathbb{C}^n\)上的一个范数.
定义:\begin{equation}
	\MatrixNorm{A}
	\defeq
	\max_{0 \neq x \in \mathbb{C}^n} \frac{f(Ax)}{g(x)},
\end{equation}
称之为“矩阵\(A\)由\(f\)和\(g\)诱导的\DefineConcept{算子范数}(operator norm)\footnote{
	%@credit: {cee35532-e299-4587-89a0-7b84e7454774} 小飞机说“算子范数”就是L2范数.
	在有的书上,矩阵的“算子范数”是指它的谱范数.
}”.
%@see: https://mathworld.wolfram.com/OperatorNorm.html
\end{definition}

\begin{definition}
%@see: 《矩阵论》(詹兴致) P5
%@see: 《数值分析(第5版)》(李庆扬、王能超、易大义) P164 定理16
如果由\(f\)和\(g\)诱导的算子范数
\(N\colon M_n(\mathbb{C}) \to \mathbb{R}, A \mapsto \MatrixNorm{A}\)
满足\begin{equation*}
	(\forall A,B \in M_n(\mathbb{C}))
	[
		\MatrixNorm{AB}
		\leq \MatrixNorm{A} \MatrixNorm{B}
	],
\end{equation*}
则称“由\(f\)和\(g\)诱导的算子范数\(N\)是\DefineConcept{次可乘的}(submultiplcative)”.
\end{definition}

\begin{definition}
%@credit: {8b6edada-f2fd-4ae5-9020-eb533149a54c},{6c964576-9569-472e-969e-54699e35974b}
设\(A \in M_{m \times n}(\mathbb{C})\).
\begin{itemize}
	\item 把矩阵\(A\)由\(L_q\)向量范数和\(L_p\)向量范数诱导的算子范数\begin{equation}
		\MatrixNorm{A}_{p \to q}
		\defeq
		\max_{0 \neq x \in \mathbb{C}^n} \frac{ \norm{Ax}_q }{ \norm{x}_p },
	\end{equation}
	称为“矩阵\(A\)的 \DefineConcept{\(L_p \to L_q\)范数}”.

	\item 把矩阵\(A\)的\(L_p \to L_p\)范数\begin{equation}
		\MatrixNorm{A}_p
		\defeq
		\MatrixNorm{A}_{p \to p}
		=
		\max_{0 \neq x \in \mathbb{C}^n} \frac{ \norm{Ax}_p }{ \norm{x}_p },
	\end{equation}
	称为“矩阵\(A\)的 \DefineConcept{\(L_p\)范数}”.
\end{itemize}
\end{definition}

\begin{theorem}% 列和范数
%@see: 《数值分析(第5版)》(李庆扬、王能超、易大义) P165 定理17(2)
%@credit: {6c964576-9569-472e-969e-54699e35974b} 菌菌笔记
设\(A \in M_{m \times n}(\mathbb{C})\),
则\begin{equation*}
	\MatrixNorm{A}_1
	= \max_{1 \leq j \leq n} \sum_{i=1}^m \abs{a_{ij}}.
\end{equation*}
\begin{proof}
根据矩阵\(L_p\)范数的定义有\(
	\MatrixNorm{A}_1
	= \max_{0 \neq x \in \mathbb{C}^n}
		\frac{ \norm{Ax}_1 }{ \norm{x}_1 }
\).
由于对于任意非零向量\(x \in \mathbb{C}^n\),
存在唯一的复数\(k\)和单位向量\(e\),
使得\(x = k e\),
所以\begin{equation*}
	\frac{ \norm{Ax}_1 }{ \norm{x}_1 }
	= \frac{ \norm{A(ke)}_1 }{ \norm{ke}_1 }
	= \frac{ k \norm{Ae}_1 }{ k \norm{e}_1 }
	= \frac{ \norm{Ae}_1 }{ \norm{e}_1 }
	= \norm{Ae}_1,
\end{equation*}
从而有\begin{align*}
	\MatrixNorm{A}_1
	&= \max_{\norm{x} = 1}
		\norm{Ax}_1 \\
	&= \max_{\norm{x} = 1}
		\sum_{i=1}^m
		\abs{
			\sum_{j=1}^n a_{ij} x_j
		}
		\tag{向量$L_p$范数的定义} \\
	&\leq \max_{\norm{x} = 1}
		\sum_{i=1}^m
		\sum_{j=1}^n
		\abs{
			a_{ij} x_j
		}
		\tag{\hyperref[theorem:不等式.三角不等式1.推论1]{三角不等式}} \\
	&= \max_{\norm{x} = 1}
		\sum_{i=1}^m
		\sum_{j=1}^n
		\abs{a_{ij}} \abs{x_j} \\
	&= \max_{\norm{x} = 1}
		\sum_{j=1}^n \abs{x_j}
		\sum_{i=1}^m \abs{a_{ij}}.
\end{align*}
记\(M \defeq \max_{1 \leq j \leq n} \sum_{i=1}^m \abs{a_{ij}}\),
则\(\sum_{i=1}^m \abs{a_{ij}} \leq M\ (j=1,2,\dotsc,n)\),
从而有\begin{align*}
	\max_{\norm{x} = 1} \sum_{j=1}^n \abs{x_j} \sum_{i=1}^m \abs{a_{ij}}
	\leq \max_{\norm{x} = 1} \sum_{j=1}^n \abs{x_j} M
	= M \max_{\norm{x} = 1} \sum_{j=1}^n \abs{x_j}
	= M \max_{\norm{x} = 1} \norm{x}
	= M,
\end{align*}
于是\(\MatrixNorm{A}_1 \leq M\).

不妨设矩阵\(A\)第1列的绝对值之和最大,
即\(M = \sum_{i=1}^m \abs{a_{i1}}\).
因为当\(x = (1,0,\dotsc,0)^T\)时,有\begin{equation*}
	\frac{ \norm{Ax}_1 }{ \norm{x}_1 }
	= \frac{ \norm{ (a_{11},a_{21},\dotsc,a_{m1})^T } }{ \norm{ (1,0,\dotsc,0)^T } }
	= \frac{ \abs{a_{11}} + \abs{a_{21}} + \dotsb + \abs{a_{m1}} }{ 1 }
	= M,
\end{equation*}
% 既然最大值不能超过\(M\),且能取到\(M\),那么最大值就是\(M\)。
所以\(
	\MatrixNorm{A}_1
	= \max_{0 \neq x \in \mathbb{C}^n}
		\frac{ \norm{Ax}_1 }{ \norm{x}_1 }
	= M
\).
\end{proof}
\end{theorem}

\begin{theorem}% 谱范数
%@credit: {6c964576-9569-472e-969e-54699e35974b} 菌菌笔记
设\(A \in M_{m \times n}(\mathbb{C})\),
则\begin{equation*}
	\MatrixNorm{A}_2
	= \max_{x\neq0} \frac{\MatrixNorm{Ax}_2}{\MatrixNorm{x}_2}.
\end{equation*}
%TODO
\end{theorem}

\begin{theorem}
%@see: 《数值分析(第5版)》(李庆扬、王能超、易大义) P165 定理17(3)
设\(A \in M_n(\mathbb{R})\),
则\(A\)的\(L_2\)范数等于\(A^T A\)的最大特征值的算术平方根,
即\begin{equation*}
	\MatrixNorm{A}_2
	= \sqrt{
		\max\Span(A^T A)
	}.
\end{equation*}
%TODO
\end{theorem}

\begin{theorem}% 行和范数
%@see: 《数值分析(第5版)》(李庆扬、王能超、易大义) P165 定理17(1)
%@credit: {6c964576-9569-472e-969e-54699e35974b} 菌菌笔记
设\(A \in M_{m \times n}(\mathbb{C})\),
则\begin{equation*}
	\MatrixNorm{A}_\infty
	= \max_{1 \leq i \leq m} \sum_{j=1}^n \abs{a_{ij}}.
\end{equation*}
%TODO
\end{theorem}

%@see: 《矩阵论》(詹兴致) P5
%@credit: {6c964576-9569-472e-969e-54699e35974b} 菌菌笔记
矩阵的\(L_1\)范数、\(L_2\)范数、\(L_\infty\)范数各有一个别名,
依次是\DefineConcept{列和范数}、\DefineConcept{谱范数}、\DefineConcept{行和范数}.

\begin{theorem}
%@see: 《数值分析(第5版)》(李庆扬、王能超、易大义) P166 定理18
%@see: 《数值分析(第5版)》(李庆扬、王能超、易大义) P166 (4.10)
设\(A \in M_n(\mathbb{R})\).
如果\(f\)是\(L_p\)范数或弗罗贝尼乌斯范数,
则\(A\)的谱半径不大于\(A\)的范数\(f(A)\).
%TODO proof
\end{theorem}

%TODO
%@see: 《数值分析(第5版)》(李庆扬、王能超、易大义) P166 (4.11)

\begin{theorem}
%@see: 《数值分析(第5版)》(李庆扬、王能超、易大义) P166 定理19
%@credit: {8b6edada-f2fd-4ae5-9020-eb533149a54c} 狗狗说矩阵的L2范数与它的谱半径相等.
设\(A \in M_n(\mathbb{C})\)是对称矩阵,
则\(A\)的谱范数\(\MatrixNorm{A}_2\)等于\(A\)的谱半径.
%TODO proof
\end{theorem}

\begin{theorem}
%@see: 《数值分析(第5版)》(李庆扬、王能超、易大义) P166 定理20
设\(B \in M_n(\mathbb{C})\)满足\(\MatrixNorm{B}_p < 1\),
则\(I \pm B\)是非奇异矩阵,
且\begin{equation*}
%@see: 《数值分析(第5版)》(李庆扬、王能超、易大义) P167 (4.12)
	\MatrixNorm{ (I \pm B)^{-1} }_p
	\leq ( 1 - \MatrixNorm{B}_p )^{-1}.
\end{equation*}
%TODO proof
\end{theorem}

\begin{property}
%@see: 《矩阵论》(詹兴致) P5
如果\(B\)是\(A \in M_{m \times n}(\mathbb{C})\)的一个子矩阵,
则\(\MatrixNorm{B}_2 \leq \MatrixNorm{A}_2\).
\end{property}

\begin{property}
%@see: 《矩阵论》(詹兴致) P5
设\(A \in M_{m \times n}(\mathbb{C})\),
\(X\)是一个\(m\)阶酉矩阵,
\(Y\)是一个\(n\)阶酉矩阵,
则\begin{equation}
	\MatrixNorm{A}_2
	= \MatrixNorm{AY}_2
	= \MatrixNorm{XA}_2.
\end{equation}
\end{property}

\begin{proposition}
%@credit: {8b6edada-f2fd-4ae5-9020-eb533149a54c} 狗狗说“算子范数”有专门针对实对称矩阵的定义.
定义:\begin{equation*}
	\MatrixNorm{A}
	\defeq
	\begin{cases}
		\text{$A$的谱半径}, & \text{$A$是实对称矩阵}, \\
		\sqrt{ \MatrixNorm{A^TA} }, & \text{其他},
	\end{cases}
\end{equation*}
则\(\MatrixNorm{A}\)等于\(A\)的谱范数\(\MatrixNorm{A}_2\).
\end{proposition}

\begin{gather*}
	\MatrixNorm{A x}_1
	\leq \MatrixNorm{A}_1 \cdot \norm{x}_1, \\
	\MatrixNorm{A x}_\infty
	\leq \MatrixNorm{A}_\infty \cdot \norm{x}_\infty, \\
	\MatrixNorm{A x}_2
	\leq \MatrixNorm{A}_2 \cdot \norm{x}_2, \\
	\MatrixNorm{A x}_2
	\leq \MatrixNorm{A}_F \cdot \norm{x}_2.
	%TODO proof
\end{gather*}
