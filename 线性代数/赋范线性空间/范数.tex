\section{范数}
尽管我们通常出于几何(特别是欧氏几何)的考量,
将向量\(\alpha\)的模(或范数)定义为\(\sqrt{\VectorInnerProductDot{\alpha}{\alpha}}\),
不过我们还可以定义其他形式的模(或范数).

\subsection{范数的定义}
\begin{definition}
%@see: 《矩阵分析与应用(第2版)》(张贤达) P23 定义1.3.7(范数和赋范向量空间)
设\(V\)是域\(F\)上的一个线性空间.
如果映射\(p\colon V \to \mathbb{R}\)
满足\begin{itemize}
	\item {\rm\bf 非负性}:\begin{equation*}
		(\forall \alpha \in V)
		[p(\alpha) \geq 0],
		\qquad
		(\forall \alpha \in V)
		[
			p(\alpha) = 0
			\iff
			\alpha = 0
		];
	\end{equation*}

	\item {\rm\bf 齐次性}:\begin{equation*}
		(\forall \alpha \in V)
		(\forall k \in F)
		[
			p(k \alpha) = \abs{k} p(\alpha)
		];
	\end{equation*}

	\item {\rm\bf 三角不等式}:\begin{equation*}
		(\forall \alpha,\beta \in V)
		[
			p(\alpha+\beta) \leq p(\alpha) + p(\beta)
		],
	\end{equation*}
\end{itemize}
则称“\(f\)是线性空间\(V\)的一个\DefineConcept{范数}(norm)”;
对于任意向量\(\alpha \in V\),
把\(\alpha\)在\(f\)下的像
称为“向量\(\alpha\)的\(f\) \DefineConcept{范数}”,
记作\(\VectorLengthN{\alpha}\),
即\begin{equation*}
	\VectorLengthN{\alpha}
	\defeq
	f(\alpha).
\end{equation*}
\end{definition}

\subsection{半范数}
\begin{definition}
%@see: 《矩阵分析与应用(第2版)》(张贤达) P24 定义1.3.8
设\(V\)是域\(F\)上的一个线性空间.
如果映射\(p\colon V \to \mathbb{R}\)
满足\begin{itemize}
	\item {\rm\bf 非负性}:\begin{equation*}
		(\forall \alpha \in V)
		[p(\alpha) \geq 0];
	\end{equation*}

	\item {\rm\bf 齐次性}:\begin{equation*}
		(\forall \alpha \in V)
		(\forall k \in F)
		[
			p(k \alpha) = \abs{k} p(\alpha)
		];
	\end{equation*}

	\item {\rm\bf 三角不等式}:\begin{equation*}
		(\forall \alpha,\beta \in V)
		[
			p(\alpha+\beta) \leq p(\alpha) + p(\beta)
		],
	\end{equation*}
\end{itemize}
则称“\(f\)是线性空间\(V\)的一个\DefineConcept{半范数}(seminorm)”
或“\(f\)是线性空间\(V\)的一个\DefineConcept{伪范数}(seminorm)”.
%@see: https://mathworld.wolfram.com/Seminorm.html
\end{definition}

半范数与范数的唯一区别是:
半范数不完全满足范数的非负性公理,
有可能当\(\alpha\neq0\)时成立\(p(\alpha) = 0\).
\begin{proposition}
设\(f\)是线性空间\(V\)的一个半范数.
如果\begin{equation*}
	(\forall \alpha \in V)
	[
		p(\alpha) = 0
		\iff
		\alpha = 0
	],
\end{equation*}
则\(f\)是线性空间\(V\)的一个范数.
\end{proposition}

\begin{example}
对于任意\(n\)维向量\(\alpha = (x_1,\dotsc,x_n)^T\),
只要满足\begin{equation*}
	x_1 + \dotsb + x_n = 0,
	% 即\(\alpha\)是“零均值向量”
\end{equation*}
则映射\begin{equation*}
	p(\alpha)
	\defeq
	x_1 + \dotsb + x_n
\end{equation*}
就是向量\(\alpha\)的一个半范数.
但是,显然\(p(\alpha) = 0\)并不意味着\(\alpha = 0\).
\end{example}

\subsection{拟范数}
\begin{definition}
%@see: 《矩阵分析与应用(第2版)》(张贤达) P24 定义1.3.9
设\(V\)是有序域\(F\)上的一个线性空间.
如果映射\(p\colon V \to \mathbb{R}\)
满足\begin{itemize}
	\item {\rm\bf 非负性}:\begin{equation*}
		(\forall \alpha \in V)
		[p(\alpha) \geq 0],
		\qquad
		(\forall \alpha \in V)
		[
			p(\alpha) = 0
			\iff
			\alpha = 0
		];
	\end{equation*}

	\item {\rm\bf 齐次性}:\begin{equation*}
		(\forall \alpha \in V)
		(\forall k \in F)
		[
			p(k \alpha) = \abs{k} p(\alpha)
		];
	\end{equation*}

	\item {\rm\bf 三角不等式}:\begin{equation*}
		(\forall \alpha,\beta \in V)
		(\exists c \in C)
		[
			p(\alpha+\beta) \leq c [p(\alpha) + p(\beta)]
		],
	\end{equation*}
	其中\(
		C
		\defeq
		\Set{
			x \in F
			\given
			x > 0,
			x \neq 1
		}
	\),
\end{itemize}
则称“\(f\)是线性空间\(V\)的一个\DefineConcept{拟范数}(quasinorm)”.
\end{definition}

拟范数与范数的唯一区别是:
拟范数不严格满足范数的三角不等式公理.

\begin{example}
%@see: 《矩阵分析与应用(第2版)》(张贤达) P24
同一个定义公式,有时给出拟范数,有时给出范数,取决于参数的变化.
例如,容易验证\begin{equation*}
%@see: 《矩阵分析与应用(第2版)》(张贤达) P24 (1.3.24)
	\norm{\alpha}_p
	\defeq
	\left( \sum_{i=1}^m x_i^p \right)^{1/p}
\end{equation*}
当\(0 < p < 1\)时给出的是拟范数,
当\(p \geq 1\)时给出的是范数.
\end{example}

\subsection{向量范数}
\begin{definition}
%@see: 《矩阵论》(詹兴致) P5
设\(\alpha \in \mathbb{C}^n\).
定义:\begin{equation}
	\MatrixNorm{A}_p
	\defeq
	\left( \sum_{i=1}^n \ComplexLengthA{a_i}^p \right)^{1/p},
\end{equation}
称之为“向量\(\alpha\)的 \DefineConcept{\(L_p\)范数}”,
其中\(\alpha = (\AutoTuple{a}{n})^T\).
\end{definition}

易见\begin{gather}
	\norm{\alpha}_1 = \VectorLengthA{x_1} + \VectorLengthA{x_2} + \dotsb + \VectorLengthA{x_n}, \\
	\norm{\alpha}_2 = \sqrt{x_1^2 + x_2^2 + \dotsb + x_n^2}, \\
	\norm{\alpha}_\infty = \max\{\VectorLengthA{x_1},\VectorLengthA{x_2},\dotsc,\VectorLengthA{x_n}\}.
\end{gather}

\subsection{矩阵范数}
由所有形状相同的矩阵组成的集合,对加法和标量乘法,成为一个线性空间.
于是我们可以在这个线性空间\(M_{m \times n}(K)\)上定义范数.

\begin{definition}
%@see: 《矩阵论》(詹兴致) P5
设\(A \in M_{m \times n}(\mathbb{C})\),
\(A^H\)是\(A\)的共轭转置.
定义:\begin{equation}
	\MatrixNorm{A}_F
	\defeq
	\sqrt{
		\tr(A^H A)
	},
\end{equation}
称之为“矩阵\(A\)的\DefineConcept{弗罗贝尼乌斯范数}”.
\end{definition}

\begin{definition}
%@see: 《矩阵论》(詹兴致) P5
设\(A \in M_{m \times n}(\mathbb{C})\).
定义:\begin{equation}
	\MatrixNorm{A}_r
	\defeq
	\max_{1 \leq i \leq m} \sum_{j=1}^n \abs{a_{ij}},
\end{equation}
称之为“矩阵\(A\)的\DefineConcept{行和范数}”.
\end{definition}

\begin{definition}
%@see: 《矩阵论》(詹兴致) P5
设\(A \in M_{m \times n}(\mathbb{C})\).
定义:\begin{equation}
	\MatrixNorm{A}_c
	\defeq
	\max_{1 \leq j \leq n} \sum_{i=1}^n \abs{a_{ij}},
\end{equation}
称之为“矩阵\(A\)的\DefineConcept{列和范数}”.
\end{definition}

\begin{definition}
%@see: 《矩阵论》(詹兴致) P5
设\(A \in M_{m \times n}(\mathbb{C})\),
\(f\)是\(\mathbb{C}^m\)上的一个范数,
\(g\)是\(\mathbb{C}^n\)上的一个范数.
定义:\begin{equation}
	\MatrixNorm{A}
	\defeq
	\max_{0 \neq x \in \mathbb{C}^n} \frac{f(Ax)}{g(x)},
\end{equation}
称之为“矩阵\(A\)由\(f\)和\(g\)诱导的\DefineConcept{算子范数}”.
\end{definition}

\begin{definition}
%@see: 《矩阵论》(詹兴致) P5
如果由\(f\)和\(g\)诱导的算子范数
\(N\colon M_n(\mathbb{C}) \to \mathbb{R}, A \mapsto \MatrixNorm{A}\)
满足\begin{equation*}
	(\forall A,B \in M_n(\mathbb{C}))
	[
		\MatrixNorm{AB}
		\leq \MatrixNorm{A} \MatrixNorm{B}
	],
\end{equation*}
则称“由\(f\)和\(g\)诱导的算子范数\(N\)是\DefineConcept{次可乘的}”.
\end{definition}

\begin{definition}
%@see: 《矩阵论》(詹兴致) P5
设\(A \in M_{m \times n}(\mathbb{C})\).
定义:\begin{equation}
	\MatrixNorm{A}_\infty
	\defeq
	\max_{x\neq0} \frac{\MatrixNorm{Ax}_2}{\MatrixNorm{x}_2},
\end{equation}
称之为“矩阵\(A\)的\DefineConcept{谱范数}”.
\end{definition}

\begin{property}
%@see: 《矩阵论》(詹兴致) P5
如果\(B\)是\(A \in M_{m \times n}(\mathbb{C})\)的一个子矩阵,
则\(\MatrixNorm{B}_\infty \leq \MatrixNorm{A}_\infty\).
\end{property}

\begin{property}
%@see: 《矩阵论》(詹兴致) P5
设\(A \in M_{m \times n}(\mathbb{C})\),
\(X\)是一个\(m\)阶酉矩阵,
\(Y\)是一个\(n\)阶酉矩阵,
则\begin{equation}
	\MatrixNorm{A}_\infty
	= \MatrixNorm{AY}_\infty
	= \MatrixNorm{XA}_\infty.
\end{equation}
\end{property}
