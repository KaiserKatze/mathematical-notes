\section{克拉默法则}
首先研究有两个二元一次方程的方程组\begin{equation}\label[equation-system]{equation:克拉默法则.两个二元一次方程}
%@see: 《高等代数(第三版 上册)》(丘维声) P18 (1)
	\left\{ \begin{array}{l}
		a_{11} x_1 + a_{12} x_2 = b_1, \\
		a_{21} x_1 + a_{22} x_2 = b_2,
	\end{array} \right.
\end{equation}
其中\(a_{11},a_{21}\)不全为零.
不妨设\(a_{11}\neq0\).
将这个方程组的增广矩阵经过初等行变换化成阶梯形矩阵:\[
	\begin{bmatrix}
		a_{11} & a_{12} & b_1 \\
		a_{21} & a_{22} & b_2
	\end{bmatrix}
	\to \begin{bmatrix}
		a_{11} & a_{12} & b_1 \\
		0 & a_{22}-\frac{a_{21}}{a_{11}} a_{12} & b_2-\frac{a_{21}}{a_{11}} b_1
	\end{bmatrix}.
\]
可以看出:\begin{itemize}
	\item 当\(a_{11} a_{22} - a_{12} a_{21} = 0\)时,
	\begin{itemize}
		\item 如果\(b_2-\frac{a_{21}}{a_{11}} b_1=0\),
		那么\cref{equation:克拉默法则.两个二元一次方程} 有无穷多解;
		\item 如果\(b_2-\frac{a_{21}}{a_{11}} b_1\neq0\),
		那么\cref{equation:克拉默法则.两个二元一次方程} 无解.
	\end{itemize}

	\item 当\(a_{11} a_{22} - a_{12} a_{21} \neq 0\)时,
	\cref{equation:克拉默法则.两个二元一次方程} 有唯一解:\[
	%@see: 《高等代数(第三版 上册)》(丘维声) P18 (2)
		\begin{bmatrix}
			\frac{b_1 a_{22} - b_2 a_{12}}{a_{11} a_{22} - a_{12} a_{21}},
			\frac{a_{11} b_2 - a_{21} b_1}{a_{11} a_{22} - a_{12} a_{21}}
		\end{bmatrix}^T.
	\]
\end{itemize}

我们想要知道,上述结论能否推广到
数域\(K\)上由\(n\)个\(n\)元线性方程组成的方程组:
\begin{equation}\label[equation-system]{equation:克拉默法则.线性方程组}
	\left\{ \begin{array}{l}
		a_{11}x_1 + a_{12}x_2 + \dotsb + a_{1n}x_n = b_1, \\
		a_{21}x_1 + a_{22}x_2 + \dotsb + a_{2n}x_n = b_2, \\
		\hdotsfor{1} \\
		a_{n1}x_1 + a_{n2}x_2 + \dotsb + a_{nn}x_n = b_n.
	\end{array} \right.
\end{equation}
我们把\cref{equation:克拉默法则.线性方程组} 的系数矩阵、增广矩阵分别记为\(\vb{A}\)和\(\widetilde{\vb{A}}\).
对增广矩阵\(\widetilde{\vb{A}}\)施行初等行变换,化成阶梯形矩阵\(\widetilde{\vb{J}}\),
则系数矩阵\(\vb{A}\)经过这些初等行变换也被化成阶梯形矩阵\(\vb{J}\).

假设\(\abs{\vb{A}}\neq0\),则\(\abs{\vb{J}}\neq0\),
于是\(\vb{J}\)没有零行,或者说\(\vb{J}\)的每一行都是非零行,因此\(\vb{J}\)有\(n\)个主元.
由于\(\vb{J}\)只有\(n\)列,因此\(\vb{J}\)的\(n\)个主元分别位于第\(1,2,\dotsc,n\)行,
即\[
	\vb{J} = \begin{bmatrix}
		c_{11} & c_{12} & \dots & c_{1n} \\
		0 & c_{22} & \dots & c_{2n} \\
		\vdots & \vdots & & \vdots \\
		0 & 0 & \dots & c_{nn}
	\end{bmatrix},
\]
其中\(c_{11} c_{22} \dotsm c_{nn} \neq 0\).
由于\(\widetilde{\vb{J}}\)比\(\vb{J}\)多一列,因此\[
	\widetilde{\vb{J}} = \begin{bmatrix}
		c_{11} & c_{12} & \dots & c_{1n} & d_1 \\
		0 & c_{22} & \dots & c_{2n} & d_2 \\
		\vdots & \vdots & & \vdots & \vdots \\
		0 & 0 & \dots & c_{nn} & d_n
	\end{bmatrix}.
\]
由此看出,原\cref{equation:克拉默法则.线性方程组} 有解,
由于\(\widetilde{\vb{J}}\)的非零行数目等于未知量数目,
所以原\cref{equation:克拉默法则.线性方程组} 有唯一解.

假设\(\abs{\vb{A}}=0\),则\(\abs{\vb{J}}=0\),于是我们可以断言\(\vb{J}\)必有零行,
那么\(\vb{J}\)的非零行数目\(r\)必定小于行数\(n\),
% 即\[
% 	\vb{J} = \begin{bmatrix}
% 		c_{11} & \dots & \dots & \dots
% 	\end{bmatrix}
% \]
%TODO

\begin{theorem}[克拉默法则]\label{theorem:线性方程组.克拉默法则}
%@see: 《高等代数(第三版 上册)》(丘维声) P46 定理1
%@see: 《高等代数(第三版 上册)》(丘维声) P48 定理3
%@see: https://mathworld.wolfram.com/CramersRule.html
给定一个未知量个数与方程个数相同的\cref{equation:克拉默法则.线性方程组}.
如果它的系数行列式满足\[
	d
	=\begin{vmatrix}
		a_{11} & a_{12} & \dots & a_{1n} \\
		a_{21} & a_{22} & \dots & a_{2n} \\
		\vdots & \vdots & & \vdots \\
		a_{n1} & a_{n2} & \dots & a_{nn}
	\end{vmatrix}
	\neq 0,
\]
则线性方程组 \labelcref{equation:克拉默法则.线性方程组} 存在唯一解:
\begin{equation}
	\vb{x}_0
	= \left( \frac{d_1}{d},\frac{d_2}{d},\dotsc,\frac{d_n}{d} \right)^T,
\end{equation}
其中\begin{equation}
	d_k
	= \begin{vmatrix}
		a_{11} & \dots & a_{1\ k-1} & b_1 & a_{1\ k+1} & \dots & a_{1n} \\
		a_{21} & \dots & a_{2\ k-1} & b_2 & a_{2\ k+1} & \dots & a_{2n} \\
		\vdots & & \vdots & \vdots & \vdots & & \vdots \\
		a_{n1} & \dots & a_{n\ k-1} & b_n & a_{n\ k+1} & \dots & a_{nn}
	\end{vmatrix},
	\quad k=1,2,\dotsc,n.
\end{equation}
%TODO 在不借助矩阵记号的情况下可能需要数学归纳法证明?
% \begin{proof}
% 上述原线性方程组可以改写为矩阵形式\(\vb{A} \vb{x} = \vb\beta\).
% 因为\(d = \abs{\vb{A}} \neq 0\),故\(\vb{A}\)可逆,那么线性方程组有唯一解\[
% 	\vb{x}_0 = \vb{A}^{-1} \vb\beta = \frac{1}{d} \vb{A}^* \vb\beta,
% \]其中\(\vb{A}^*\)是\(\vb{A}\)的伴随矩阵.
% 设\(A_{ij}\)表示\(\vb{A}\)中元素\(a_{ij}\ (i,j=1,2,\dotsc,n)\)的代数余子式,
% 则行列式\(d_k\)可按第\(k\)列展开,得\[
% 	d_k = b_1 A_{1k} + b_2 A_{2k} + \dotsb + b_n A_{nk},
% 	\quad k=1,2,\dotsc,n,
% \]
% 则\[
% 	\vb{x}_0 = \frac{1}{d} \vb{A}^* \vb\beta
% 	= \frac{1}{d} \begin{bmatrix}
% 		A_{11} & A_{21} & \dots & A_{n1} \\
% 		A_{12} & A_{22} & \dots & A_{n2} \\
% 		\vdots & \vdots & \ddots & \vdots \\
% 		A_{1n} & A_{2n} & \dots & A_{nn}
% 	\end{bmatrix} \begin{bmatrix} b_1 \\ b_2 \\ \vdots \\ b_n \end{bmatrix}
% 	= \frac{1}{d} \begin{bmatrix} d_1 \\ d_2 \\ \vdots \\ d_n \end{bmatrix}.
% 	\qedhere
% \]
% \end{proof}
\end{theorem}
