\section{高斯--若尔当消元法}
\begin{definition}
在数域\(K\)上,
设矩阵\(\vb{A} \in M_{s \times n}(K)\),
向量\(\vb{x} \in K^n\),
\(\vb\beta \in K^s\).
由\(n\)元线性方程组\begin{equation*}
	\vb{A} \vb{x} = \vb\beta
\end{equation*}的系数和常数项按原位置构成的\(s \times (n+1)\)矩阵\((\vb{A},\vb\beta)\),
称为该\(n\)元线性方程组的\DefineConcept{增广矩阵}(augmented matrix),
记作\(\widetilde{\vb{A}}\).
\end{definition}
%@Mathematica: AugmentMatrix[A_?MatrixQ, b_?VectorQ] := ArrayFlatten[{{A, Transpose[{b}]}}]
%@Mathematica: A = {{1, 1}, {1, 1}}
%@Mathematica: b = {1, 1}
%@Mathematica: AugmentMatrix[A, b]

\begin{theorem}
对线性方程组\(\vb{A} \vb{x} = \vb\beta\)的增广矩阵\(\widetilde{\vb{A}} = (\vb{A},\vb\beta)\)作初等行变换,
变为\(\widetilde{\vb{C}} = (\vb{C},\vb\gamma)\),
则相应的线性方程组\(\vb{C} \vb{x} = \vb\gamma\)与原线性方程组同解.
\begin{proof}
显然存在可逆矩阵\(\vb{P}\),
使得\(\widetilde{\vb{A}} \to \widetilde{\vb{C}} = \vb{P} \widetilde{\vb{A}} = (\vb{P}\vb{A},\vb{P}\vb\beta)\),
\(\vb{C} = \vb{P}\vb{A}\),\(\vb\gamma = \vb{P}\vb\beta\).
如果\(\vb{x}_0\)是原线性方程组的解,即\(\vb{A} \vb{x}_0 = \vb\beta\),
用\(\vb{P}\)左乘等式两端得\(\vb{C} \vb{x}_0 = \vb\gamma\);
反之,若\(\vb{x}_0\)满足\(\vb{C} \vb{x}_0 = \vb\gamma\),
用\(\vb{P}^{-1}\)左乘等式两端得\(\vb{A} \vb{x}_0 = \vb\beta\),
故两方程组同解.
\end{proof}
\end{theorem}
这就是消元法解线性方程组的理论根据.
具体化简\(\widetilde{\vb{A}}\)时,
可用一系列初等行变换将其变成一个较为简单的“阶梯形矩阵”(或更简单的“若尔当阶梯形矩阵”).

\begin{definition}
把如下形式的\(s \times n\)矩阵\begin{equation*}
	\vb{A} = \begin{bmatrix}
		0 & \dots & a_{1 j_1} & \dots & a_{1 j_2} & \dots & a_{1 j_r} & \dots & a_{1n} \\
		0 & \dots & 0 & \dots & a_{2 j_2} & \dots & a_{2 j_r} & \dots & a_{2n} \\
		\vdots & & \vdots & & \vdots & & \vdots & & \vdots \\
		0 & \dots & 0 & \dots & 0 & \dots & a_{r j_r} & \dots & a_{rn} \\
		0 & \dots & 0 & \dots & 0 & \dots & 0 & \dots & 0 \\
		\vdots & & \vdots & & \vdots & & \vdots & & \vdots \\
		0 & \dots & 0 & \dots & 0 & \dots & 0 & \dots & 0 \\
	\end{bmatrix}
\end{equation*}
称为\DefineConcept{行阶梯形矩阵}(row echelon form),
%@see: https://mathworld.wolfram.com/EchelonForm.html
其中\(a_{1 j_1},a_{2 j_2},\dotsc,a_{r j_r}\)均不为零,
\(j_1 < j_2 < \dotsb < j_r\),
\(\vb{A}\)的后\(s-r\)行全为零.

元素全为\(0\)的行,称为\DefineConcept{零行}.
元素不全为\(0\)的行,称为\DefineConcept{非零行}.

在非零行中,从左数起,第一个不为\(0\)的元素\(a_{i j_i}\ (i=1,2,\dotsc,r)\),
称为\DefineConcept{主元}(pivot)或\DefineConcept{非零首元}.

以主元为系数的未知量\(x_{j_1},x_{j_2},\dotsc,x_{j_r}\),
称为\DefineConcept{主变量};
不以主元为系数的未知量,称为\DefineConcept{自由未知量}.
\end{definition}

\begin{definition}
若阶梯形矩阵\(\vb{A}\)的非零行的非零首元全为\(1\),
它们所在列的其余元素全为零,
则把\(\vb{A}\)称为\DefineConcept{若尔当阶梯形矩阵}%
或\DefineConcept{行约化矩阵}(reduced row echelon form).
\end{definition}
% 在Mathematica中可以用RowReduce对矩阵进行初等行变换化为若尔当阶梯形矩阵.

\begin{lemma}
任何一个非零矩阵都可经初等行变换化为阶梯形矩阵.
\begin{proof}
设\(\vb{A}_{s \times n} \neq \vb0\),则\(\vb{A}\)经0次或1次交换两行的变换化为\(\vb{B}\),即\begin{equation*}
	\vb{A} \to \vb{B} = \begin{bmatrix}
		0 & \dots & 0 & b_{1 j_1} & \dots & b_{1n} \\
		0 & \dots & 0 & b_{2 j_1} & \dots & b_{2n} \\
		\vdots & & \vdots & \vdots & & \vdots \\
		0 & \dots & 0 & b_{s j_1} & \dots & b_{sn}
	\end{bmatrix},
\end{equation*}
其中\(b_{1 j_1} \neq 0\).
分别将\(\vb{B}\)的第一行的\(-b_{i j_1}/b_{1 j_1}\)倍加到第\(i\ (i=2,3,\dotsc,s)\)行,则\begin{equation*}
	\vb{B} \to \vb{C} = \begin{bmatrix}
		\vb0 & b_{1 j_1} & \vb{B}_1 \\
		\vb0 & \vb0 & \vb{C}_1
	\end{bmatrix},
\end{equation*}
其中\(\vb{C}_1\)是\((s-1)\times(n-j_1)\)矩阵,
再对\(\vb{C}\)的后面\(s-1\)行作类似的初等行变换化简.
因为矩阵行数有限,这样下去,最后总可化为阶梯形矩阵.
\end{proof}
\end{lemma}

\begin{corollary}\label{theorem:线性方程组.非零矩阵可经初等行变换化为若尔当阶梯形矩阵}
任何一个非零矩阵都可经初等行变换化为若尔当阶梯形矩阵.
\end{corollary}

\begin{theorem}
设\(\vb{A}\)为\(n\)阶方阵,
则齐次线性方程组\(\vb{A}\vb{x} = \vb0\)有非零解的充分必要条件是:
\(\abs{\vb{A}} = 0\).
\begin{proof}
必要性.
给定\(\vb{x}_0 \neq \vb0\)满足\(\vb{A} \vb{x}_0 = \vb0\).
假设\(\abs{\vb{A}} \neq 0\),
由克拉默法则可知方程组有唯一解\(\vb{x}_0 = \vb{A}^{-1} \vb0 = \vb0\),
矛盾,故\(\abs{\vb{A}} = 0\).

充分性.
用数学归纳法证明.
给定\(\abs{\vb{A}} = 0\).
当\(n=1\)时,\(\vb{A} = \vb0_{1 \times 1} = 0\),
\(0 \cdot x_1 = 0\)有非零解;
假设当\(n=k-1\geq1\)时,结论成立;
那么当\(n=k\)时,
设\(\vb{A}\)经初等行变换\(\vb{P}\)化为阶梯形矩阵\begin{equation*}
	\vb{B} = \begin{bmatrix}
		b & \vb\gamma \\
		\vb0 & \vb{C} \\
	\end{bmatrix} = \vb{P} \vb{A},
\end{equation*}
其中\(\vb{C}\)是\(n-1\)阶方阵,\(\vb{P}\)是\(n\)阶可逆矩阵.
取行列式得
\begin{equation*}
	\abs{\vb{B}} = \abs{\vb{P}} \abs{\vb{A}} = 0 = b \abs{\vb{C}}.
\end{equation*}
解同解方程组\(\vb{B} \vb{x} = \vb0\).
若\(b = 0\)则\((1,0,\dotsc,0)^T\)是一个非零解;
若\(b \neq 0\),则\(\abs{\vb{C}} = 0\),由归纳假设,齐次线性方程组\begin{equation*}
	\vb{C} \begin{bmatrix} x_2 \\ x_3 \\ \vdots \\ x_n \end{bmatrix} = \vb0
\end{equation*}有非零解\((k_2,k_3,\dotsc,k_n)^T\),
代入\(\vb{B} \vb{x} = \vb0\)的第一个方程,
因为\(x_1\)的系数\(b \neq 0\),可解出\(x_1\).
于是\((\AutoTuple{k}{n})^T\)是\(\vb{A} \vb{x} = \vb0\)的一个非零解.
\end{proof}
\end{theorem}

\begin{corollary}\label{theorem:线性方程组.方程个数少于未知量个数的齐次线性方程组必有非零解}
方程个数少于未知量个数的齐次线性方程组必有非零解.
\begin{proof}
设数域\(K\)上的线性方程组\(\vb{A} \vb{x} = \vb0\),
它的系数矩阵\(\vb{A} \in M_{s \times n}(K)\ (s<n)\).
在原方程组的后面添加\(n-s\)个\(0=0\)的方程,解不变,新方程组的系数矩阵为:
\begin{equation*}
	\vb{B}_n = \begin{bmatrix} \vb{A}_{s \times n} \\ \vb0_{(n-s) \times n} \end{bmatrix},
\end{equation*}
由于\(s < n\),有\(\abs{\vb{B}} = 0\),故\(\vb{B} \vb{x} = \vb0\)有非零解,从而\(\vb{A} \vb{x} = \vb0\)有非零解.
\end{proof}
%\cref{theorem:线性方程组.齐次线性方程组只有零解的充分必要条件}
%\cref{theorem:线性方程组.齐次线性方程组有非零解的充分必要条件}
\end{corollary}
\begin{remark}
\cref{theorem:线性方程组.方程个数少于未知量个数的齐次线性方程组必有非零解} 的逆命题不成立,
即方程组有非零解,不能得出未知量个数与方程个数的关系.
\end{remark}

\begin{example}
设\(
	\vb{A},\vb{B} \in M_{s \times n}(K),
	\vb\beta \in K^s
\).
举例说明:虽然有\(\rank\vb{A} = \rank\vb{B}\),
但\(\rank(\vb{A},\vb\beta) = \rank(\vb{B},\vb\beta)\)不成立.
\begin{solution}
取\(
	\vb{A}
	\defeq
	\begin{bmatrix}
		1 & 0 \\
		0 & 0
	\end{bmatrix},
	\vb{B}
	\defeq
	\begin{bmatrix}
		0 & 0 \\
		1 & 0
	\end{bmatrix},
	\vb\beta
	\defeq
	\begin{bmatrix}
		1 \\ 0
	\end{bmatrix}
\),
那么\(\rank\vb{A} = \rank\vb{B} = 1\),
但是\(
	\rank(\vb{A},\vb\beta)
	= 1
	\neq
	\rank(\vb{B},\vb\beta)
	= 2
\).
\end{solution}
%@Mathematica: A = {{1, 0}, {0, 0}};
%@Mathematica: B = {{0, 0}, {1, 0}};
%@Mathematica: b = {1, 0};
%@Mathematica: MatrixRank[AugmentedMatrix[A, b]]
%@Mathematica: MatrixRank[AugmentedMatrix[B, b]]
\end{example}
