\section{线性方程组}
\subsection{线性方程组的概念}
我们把含有\(n\)个\DefineConcept{未知量}(unknown)\(\AutoTuple{x}{n}\)的一次方程组
\begin{equation}\label[equation-system]{equation:线性方程组.线性方程组的代数形式}
	\left\{ \begin{array}{l}
		a_{11} x_1 + a_{12} x_2 + \dotsb + a_{1n} x_n = b_1, \\
		a_{21} x_1 + a_{22} x_2 + \dotsb + a_{2n} x_n = b_2, \\
		\hdotsfor{1} \\
		a_{s1} x_1 + a_{s2} x_2 + \dotsb + a_{sn} x_n = b_s
	\end{array} \right.
\end{equation}
称为“\(n\)元\DefineConcept{线性方程组}(\emph{linear system of equations} in \(n\) variables)”.
这里,\(s\)为方程的个数\footnote{%
在\cref{equation:线性方程组.线性方程组的代数形式} 中,
方程的数目\(s\)可以等于未知数的数目\(n\),
也可以不相等(即\(s<n\)或\(s>n\)).};
我们把数\begin{equation*}
	a_{ij}
	\quad(i=1,2,\dotsc,s; j=1,2,\dotsc,n)
\end{equation*}称为“第\(i\)个方程中\(x_j\)的\DefineConcept{系数}(coefficient)”;
把数\begin{equation*}
	b_i
	\quad(i=1,2,\dotsc,s)
\end{equation*}叫做“第\(i\)个方程的\DefineConcept{常数项}(constant term)”.

\begin{definition}
我们把常数项全为零的线性方程组
称为\DefineConcept{齐次线性方程组}(homogeneous linear systems of equations),
把常数项中有非零数的线性方程组
称为\DefineConcept{非齐次线性方程组}(nonhomogeneous linear systems of equations,
inhomogeneous linear systems of equations).
\end{definition}

\begin{definition}
如果存在\(n\)个数\(\AutoTuple{c}{n}\)满足\cref{equation:线性方程组.线性方程组的代数形式},即\begin{equation*}
	a_{i1} c_1 + a_{12} c_2 + \dotsb + a_{in} c_n \equiv b_i
	\quad(i=1,2,\dotsc,s),
\end{equation*}
则称“\cref{equation:线性方程组.线性方程组的代数形式} \DefineConcept{有解}”,
或“\cref{equation:线性方程组.线性方程组的代数形式} 是\DefineConcept{相容的}”;
否则,称“\cref{equation:线性方程组.线性方程组的代数形式} \DefineConcept{无解}”,
或“\cref{equation:线性方程组.线性方程组的代数形式} 是\DefineConcept{不相容的}”.

称这\(n\)个数\(\AutoTuple{c}{n}\)构成的列向量\((\AutoTuple{c}{n})^T\)为%
\cref{equation:线性方程组.线性方程组的代数形式} 的一个\DefineConcept{解}(solution)%
或\DefineConcept{解向量}(solution vector).
\cref{equation:线性方程组.线性方程组的代数形式} 的解的全体构成的集合,
称为“\cref{equation:线性方程组.线性方程组的代数形式} 的\DefineConcept{解集}”.
\end{definition}

\begin{definition}
元素全为零的解向量,称为\DefineConcept{零解}(zero solution).
\end{definition}
\begin{definition}
元素不全为零的解向量,称为\DefineConcept{非零解}(nonzero solution).
\end{definition}

\begin{theorem}
任意一个齐次线性方程组都有零解.
\end{theorem}

\begin{definition}
解集相等的两个线性方程组,称为\DefineConcept{同解方程组}.
\end{definition}

需要注意的是,线性方程组的解集与数域有关.

\begin{theorem}
\(n\)元线性方程组的解的情况有且只有三种可能:
无解、有唯一解、有无穷多个解.
\end{theorem}

\subsection{线性方程组的几何意义}
把集合\begin{equation*}
	\Set{
		(\AutoTuple{x}{n})
		\given
		a_{i1} x_1 + a_{i2} x_2 + \dotsb + a_{in} x_n = b_i
	}
	\quad(i=1,2,\dotsc,s)
\end{equation*}
绘制在\(\mathbb{R}^n\)上,得到的直线,
称为“\cref{equation:线性方程组.线性方程组的代数形式} 的
第\(i\)行的\DefineConcept{行图像}(row picture)”.

把向量\begin{equation*}
	(a_{1j},a_{2j},\dotsc,a_{sj})^T
	\quad(j=1,2,\dotsc,n)
\end{equation*}
绘制在\(\mathbb{R}^n\)上,得到的有向线段,
称为“\cref{equation:线性方程组.线性方程组的代数形式} 的
第\(j\)列的\DefineConcept{列图像}(column picture)”.

以关于\(x,y\)的2元线性方程组\begin{equation*}
	\left\{ \begin{array}{l}
		2x - y = 0, \\
		-x + 2y = 3
	\end{array} \right.
\end{equation*}
为例,\cref{figure:线性方程组.行图像1} 是该方程各行的行图像,
\cref{figure:线性方程组.列图像1} 是该方程各列的列图像.
\begin{figure}[hbt]
	\centering
	\def\subwidth{.4\linewidth}
	\begin{subfigure}[b]{\subwidth}
		\centering
		\begin{tikzpicture}
			\begin{axis}[
				xmin=-3,xmax=3,
				ymin=-3,ymax=3,
				restrict y to domain=-5:5,
				% grid=both,
				% width=\textwidth,height=\textwidth,
				axis lines=middle,
				axis equal=true,
				xlabel=$x$,
				ylabel=$y$,
				enlarge x limits=0.1,
				enlarge y limits=0.1,
			]
				\addplot[color=blue,samples=10,domain=-5:5]{2*x};
				\addplot[color=orange,samples=10,domain=-5:5]{(3+x)/2};
			\end{axis}
		\end{tikzpicture}
		\caption{}
		\label{figure:线性方程组.行图像1}
	\end{subfigure}\hspace{20pt}\begin{subfigure}[b]{\subwidth}
		\begin{tikzpicture}
			\begin{axis}[
				xmin=-3,xmax=3,
				ymin=-3,ymax=3,
				restrict y to domain=-3:3,
				% grid=both,
				% width=\textwidth,height=\textwidth,
				axis lines=middle,
				axis equal=true,
				xlabel=$x$,
				ylabel=$y$,
				enlarge x limits=0.1,
				enlarge y limits=0.1,
			]
				\begin{scope}[->,>=Stealth]
					\draw[red](0,0)--(2,-1);
					\draw[darkgreen](0,0)--(-1,2);
					\draw[orange](0,0)--(0,3);
					\begin{scope}[dashed]
						\draw(2,-1)--(1,1);
						\draw(1,1)--(0,3);
					\end{scope}
				\end{scope}
				\draw[dashed](-1,2)--(1,1);
		\end{axis}
		\end{tikzpicture}
		\caption{}
		\label{figure:线性方程组.列图像1}
	\end{subfigure}
	\caption{}
\end{figure}
