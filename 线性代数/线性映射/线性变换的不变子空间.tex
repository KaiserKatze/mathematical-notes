\section{线性变换的不变子空间}
在上一节,我们讨论了可对角化的线性变换的标准型.
本节讨论不可以对角化的线性变换的标准型.

我们首先注意到,
\(\vb{A}\)是可对角化的线性变换,
当且仅当空间\(V\)可以分解成\(\vb{A}\)的特征子空间的直和.
受此启发,我们在研究不可对角化的线性变换\(\vb{B}\)的结构时,
也可以从这里入手,研究如何将空间\(V\)分解成与\(\vb{B}\)有关的某种特殊类型的子空间的直和.

\subsection{线性变换的不变子空间}
\begin{definition}
%@see: 《高等代数(第三版 下册)》(丘维声) P131 定义1
设\(\vb{A}\)是域\(F\)上线性空间\(V\)上的线性变换,\(W\)是\(V\)的子空间.
如果\(W\)中的向量在\(\vb{A}\)下的象仍在\(W\)中,
即\((\forall\alpha \in W)[\vb{A}\alpha \in W]\),
则称“\(W\)是\(\vb{A}\)的一个\DefineConcept{不变子空间}”,
简称 \DefineConcept{\(\vb{A}\) - 子空间}.
\end{definition}

显然,对于\(V\)上每一个线性变换\(\vb{A}\)来说,
整个空间\(V\)和零子空间\(0\),
都是\(\vb{A}\)的不变子空间,
因此将它们称为“\(\vb{A}\)的\DefineConcept{平凡不变子空间}”.
如果\(W\)是\(\vb{A}\)的不变子空间,
且\(W\)不是\(\vb{A}\)的平凡不变子空间,
则称“\(W\)是\(\vb{A}\)的一个\DefineConcept{非平凡不变子空间}”.

\begin{proposition}%\label{theorem:线性映射.线性变换的不变子空间1}
%@see: 《高等代数(第三版 下册)》(丘维声) P131 命题1
\(V\)上线性变换\(\vb{A}\)的核\(\Ker\vb{A}\)与象\(\Im\vb{A}\),
以及\(\vb{A}\)的所有特征子空间,
都是\(\vb{A}\)的不变子空间.
\begin{proof}
任取\(\alpha \in \Ker\vb{A}\),
因为\(\vb{A}\alpha = 0 \in \Ker\vb{A}\),
所以\(\Ker\vb{A}\)是\(\vb{A}\)的不变子空间.

任取\(\alpha \in \Im\vb{A}\),
因为\(\vb{A}\alpha \in \Im\vb{A}\),
所以\(\Im\vb{A}\)也是\(\vb{A}\)的不变子空间.

任取\(\alpha \in V_\lambda\),
因为\(\vb{A}\alpha = \lambda \alpha \in V_\lambda\),
所以\(V_\lambda\)还是\(\vb{A}\)的不变子空间.
\end{proof}
\end{proposition}

\begin{proposition}%\label{theorem:线性映射.线性变换的不变子空间2}
%@see: 《高等代数(第三版 下册)》(丘维声) P131 命题2
如果线性变换\(\vb{A},\vb{B}\)可交换,即\(\vb{A}\vb{B} = \vb{B}\vb{A}\),
则\(\Ker\vb{B},
\Im\vb{B}\)
以及\(\vb{B}\)的特征子空间
都是\(\vb{A}\)的不变子空间.
\begin{proof}
任取\(\alpha \in \Ker\vb{B}\),
则\(\vb{B}\alpha = 0\).
于是\[
	\vb{B}(\vb{A}\alpha)
	= (\vb{B}\vb{A})\alpha
	= (\vb{A}\vb{B})\alpha
	= \vb{A}(\vb{B}\alpha)
	= \vb{A}0
	= 0.
\]
因此\(\vb{A}\alpha \in \Ker\vb{B}\),
从而\(\Ker\vb{B}\)是\(\vb{A}\)的不变子空间.

同理可证\(\Im\vb{B}\)也是\(\vb{A}\)的不变子空间.

在\(\vb{B}\)的特征子空间\(V_\lambda\)中,任取一个向量\(\alpha\),
则\(\vb{B}\alpha = \lambda \alpha\),
从而\[
	\vb{B}(\vb{A}\alpha)
	= \vb{A}(\vb{B}\alpha)
	= \vb{A}(\lambda\alpha)
	= \lambda(\vb{A}\alpha),
\]
因此\(\vb{A}\alpha \in V_\lambda\),
从而\(V_\lambda\)是\(A\)的不变子空间.
\end{proof}
\end{proposition}

\begin{corollary}\label{theorem:线性映射.线性变换的不变子空间3}
%@see: 《高等代数(第三版 下册)》(丘维声) P132 推论3
设\(\vb{A}\)是域\(F\)上线性空间\(V\)上的线性变换,
则对于域\(F\)上任意一个一元多项式\(f(x) \in F[x]\),
都有\(\Ker f(\vb{A}),
\Im f(\vb{A})\)
以及\(f(\vb{A})\)的特征子空间
都是\(\vb{A}\)的不变子空间.
\end{corollary}

\begin{proposition}%\label{theorem:线性映射.线性变换的不变子空间4}
%@see: 《高等代数(第三版 下册)》(丘维声) P132
设\(U_1,U_2\)是域\(F\)上线性空间\(V\)上的线性变换\(\vb{A}\)的两个不变子空间,
则\(U_1 \cap U_2\)和\(U_1 + U_2\)都是\(\vb{A}\)的不变子空间.
\end{proposition}

\begin{proposition}
%@see: 《高等代数(第三版 下册)》(丘维声) P132 命题4
设\(\vb{A}\)是域\(F\)上线性空间\(V\)上的线性变换,
\(W = \Span\{\AutoTuple{\alpha}{m}\}\)是\(V\)的一个子空间,
则\(W\)是\(\vb{A}\)的不变子空间,
当且仅当\(\vb{A}\alpha_i \in W\ (i=1,2,\dotsc,m)\).
\begin{proof}
由于\(W\)中任意一个向量\(\alpha\)均可表示成\(\AutoTuple{\alpha}{m}\)的一个线性组合,
不妨设\(\alpha = k_1 \alpha_1 + \dotsb + k_m \alpha_m\),
其中\(\AutoTuple{k}{m} \in F\),
那么\begin{align*}
	&\text{$W$是$\vb{A}$的不变子空间} \\
	&\iff \alpha \in W \implies \vb{A}\alpha \in W \\
	&\iff (\forall \AutoTuple{k}{m} \in F)[\vb{A}(k_1 \alpha_1 + \dotsb + k_m \alpha_m) \in W] \\
	&\iff \vb{A}\alpha_i \in W\ (i=1,2,\dotsc,m).
	\qedhere
\end{align*}
\end{proof}
\end{proposition}

\begin{example}
%@see: 《高等代数(第三版 下册)》(丘维声) P131 习题9.5 3.
设\(V\)是复数域上的\(n\)维线性空间,
\(\vb{A},\vb{B}\)都是\(V\)上的线性变换,
且\(\vb{A}\vb{B}=\vb{B}\vb{A}\).
证明:\(\vb{A}\)与\(\vb{B}\)至少有一个公共的特征向量.
%TODO proof
% 取\(\vb{A}\)的一个特征值\(\lambda\),则\(V_\lambda\)是\(\vb{B}\)的一个不变子空间
% 取\(\vb{B} \setrestrict V_\lambda\)的一个特征值\(\mu\),
% 则存在\(\xi \in V_\lambda\),
% 使得\(\vb{B} \xi = (\vb{B} \setrestrict V_\lambda) \xi = \mu \xi\)
\end{example}

\subsection{线性变换的非平凡不变子空间的存在性}
对于域\(F\)上有限维线性空间\(V\)上的线性变换\(\vb{A}\),
它有没有非平凡不变子空间,与它的矩阵表示的形式,有密切关系.
\begin{theorem}
%@see: 《高等代数(第三版 下册)》(丘维声) P132 定理5
设\(\vb{A}\)是域\(F\)上\(n\)维线性空间\(V\)上的线性变换,
则\(\vb{A}\)有非平凡不变子空间,
当且仅当\(V\)中存在一个基,
使得\(\vb{A}\)在这个基下的矩阵是一个分块上三角矩阵.
\begin{proof}
必要性.
设\(W\)是\(\vb{A}\)的非平凡不变子空间,
在\(W\)中取一个基\(\AutoTuple{\alpha}{r}\),
把它扩充成\(V\)的一个基:\[
	\AutoTuple{\alpha}{r},
	\AutoTuple{\alpha}[r+1]{n}.
\]
那么\begin{align*}
%@see: 《高等代数(第三版 下册)》(丘维声) P132 (3)
	&\vb{A} (\AutoTuple{\alpha}{r},\AutoTuple{\alpha}[r+1]{n}) \\
	&= (\AutoTuple{\alpha}{r},\AutoTuple{\alpha}[r+1]{n})
	\begin{bmatrix}
		a_{11} & \dots & a_{1r} & a_{1,r+1} & \dots & a_{1n} \\
		\vdots & & \vdots & \vdots & & \vdots \\
		a_{r1} & \dots & a_{rr} & a_{r,r+1} & \dots & a_{rn} \\
		0 & \dots & 0 & a_{r+1,r+1} & \dots & a_{r+1,n} \\
		\vdots & & \vdots & \vdots & & \vdots \\
		0 & \dots & 0 & a_{n,r+1} & \dots & a_{nn}
	\end{bmatrix}.
\end{align*}
因此\(\vb{A}\)在基\(\AutoTuple{\alpha}{r},\AutoTuple{\alpha}[r+1]{n}\)下的矩阵是
一个分块上三角矩阵\(\begin{bmatrix}
	A_1 & A_3 \\
	0 & A_2
\end{bmatrix}\),
其中\(A_1\)是\(A \setrestrict W\)在\(W\)的一个基\(\AutoTuple{\alpha}{r}\)下的矩阵.

充分性.
设\(\vb{A}\)在\(V\)的一个基\(\AutoTuple{\alpha}{n}\)下的矩阵是
一个分块上三角矩阵\(\begin{bmatrix}
	A_1 & A_3 \\
	0 & A_2
\end{bmatrix}\),
其中\(A_1\)是\(r\)阶矩阵,
且\(0 < r < n\).
%@see: 《高等代数(第三版 下册)》(丘维声) P133 (4)
令\(W = \Span\{\AutoTuple{\alpha}{r}\}\),
则\(\vb{A}\alpha_i \in W\ (i=1,2,\dotsc,r)\).
因此\(W\)是\(\vb{A}\)的不变子空间.
显然\(W\)是非平凡的.
此时\(A_1\)是\(A \setrestrict W\)在\(W\)的基\(\AutoTuple{\alpha}{r}\)下的矩阵.
\end{proof}
\end{theorem}

\begin{theorem}\label{theorem:线性映射.线性空间可以分解为线性变换的一些非平凡不变子空间的直和的充分必要条件}
%@see: 《高等代数(第三版 下册)》(丘维声) P133 定理6
设\(\vb{A}\)是域\(F\)上\(n\)维线性空间\(V\)上的线性变换,
则\(V\)能分解成\(\vb{A}\)的一些非平凡不变子空间的直和,
当且仅当\(V\)中存在一个基,
使得\(\vb{A}\)在这个基下的矩阵是一个分块对角矩阵.
\def\BasisV{\alpha_{11},\dotsc,\alpha_{1 r_1},\dotsc,\alpha_{s1},\dotsc,\alpha_{s r_s}}
\def\BasisWi{\alpha_{i1},\dotsc,\alpha_{ir_i}}
\begin{proof}
必要性.
设\(V\)是\(\vb{A}\)的一些非平凡不变子空间的直和:\[
%@see: 《高等代数(第三版 下册)》(丘维声) P133 (6)
	V = W_1 \DirectSum \dotsb \DirectSum W_s.
\]
在每个\(W_i\ (i=1,2,\dotsc,s)\)中取一个基\(\BasisWi\),
由上式可知\[
%@see: 《高等代数(第三版 下册)》(丘维声) P133 (7)
	\BasisV
\]是\(V\)的一个基.
由于\(W_i\)是\(\vb{A}\)的不变子空间,
因此\[
%@see: 《高等代数(第三版 下册)》(丘维声) P133 (8)
	\vb{A} (\BasisWi)
	= (\BasisWi) A_i,
	\quad i=1,2,\dotsc,s.
\]
从而\(\vb{A}\)在基\(\BasisV\)下的矩阵为\[
%@see: 《高等代数(第三版 下册)》(丘维声) P133 (9)
	\begin{bmatrix}
		A_1 \\
		& A_2 \\
		& & \ddots \\
		& & & A_s
	\end{bmatrix}.
\]

充分性.
设\(\vb{A}\)在\(V\)一个基\[
	\BasisV
\]下的矩阵\(\vb{A} = \diag\{\AutoTuple{A}{s}\}\),
其中\(A_i\)是\(r_i\)阶方阵,
而\(i=1,2,\dotsc,s\).
令\[
	W_i = \Span\{\BasisWi\}
	\quad(i=1,2,\dotsc,s).
\]
由于\[
%@see: 《高等代数(第三版 下册)》(丘维声) P133 (10)
	\vb{A} (\BasisWi) = (\BasisWi) A_i,
	\quad i=1,2,\dotsc,s,
\]
所以\(\vb{A}\alpha_{i1},\dotsc,\vb{A}\alpha_{i r_i} \in W_i\),
从而\(W_i\)是\(\vb{A}\)的不变子空间.
显然\(W_i\)是非平凡的.
由于\(W_i\)的一个基\(\BasisWi\ (i=1,2,\dotsc,s)\)合起来\(V\)的一个基,
所以\(V = W_1 \DirectSum \dotsb \DirectSum W_s\).
\end{proof}
\begin{remark}
从\cref{theorem:线性映射.线性空间可以分解为线性变换的一些非平凡不变子空间的直和的充分必要条件} 的证明过程中可以看出,
在线性变换\(\vb{A}\)的矩阵\[
	\begin{bmatrix}
		A_1 \\
		& A_2 \\
		& & \ddots \\
		& & & A_s
	\end{bmatrix}
\]中,
\(A_i\)就是\(\vb{A}\)在它的不变子空间\(W_i\)上的限制\(\vb{A} \setrestrict W_i\)
在\(W_i\)的一个基\(\BasisWi\)下的矩阵,
其中\(i=1,2,\dotsc,s\).
\end{remark}
\end{theorem}

\begin{example}
%@see: 《高等代数(第三版 下册)》(丘维声) P131 习题9.5 2.(1)
设\(W\)是线性空间\(V\)上可逆线性变换\(\vb{A}\)的有限维不变子空间.
证明:\(\vb{A} \setrestrict W\)是\(W\)上的可逆线性变换.
%TODO proof
% 先说明\(\vb{A} \setrestrict W\)是单射
% 再据此结论说明\(\vb{A} \setrestrict W\)是满射
\end{example}

\begin{example}
%@see: 《高等代数(第三版 下册)》(丘维声) P131 习题9.5 2.(2)
设\(W\)是线性空间\(V\)上可逆线性变换\(\vb{A}\)的有限维不变子空间.
证明:\(W\)是\(\vb{A}^{-1}\)的不变子空间,
且\((\vb{A} \setrestrict W)^{-1}
= \vb{A}^{-1} \setrestrict W\).
%TODO proof
% 利用【《高等代数(第三版 下册)》(丘维声) P131 习题9.5 2.(1)】的结论
% 根据定义去证\(W\)是\(\vb{A}^{-1}\)的不变子空间
% 任取\(\beta \in W\),
% 令\((\vb{A} \setrestrict W)^{-1} \beta = \gamma\)
% 则\(\gamma \in W\)
% 最后证明\(\gamma = (\vb{A}^{-1} \setrestrict W) \beta\)
\end{example}

\subsection{求解非平凡子空间的基本步骤}
虽然我们从\cref{theorem:线性映射.线性空间可以分解为线性变换的一些非平凡不变子空间的直和的充分必要条件} 可以看出,
如果\(n\)维线性空间\(V\)能分解成线性变换\(\vb{A}\)的一些非平凡不变子空间的直和,
那么\(V\)中存在一个基,
使得\(\vb{A}\)在这个基下的矩阵是一个分块对角矩阵,
但是,要如何找出\(\vb{A}\)的所有非平凡不不变子空间呢?
根据\cref{theorem:线性映射.线性变换的不变子空间3},
对于任意一个\(f(x) \in F[x]\)都有\(\Ker f(\vb{A})\)是\(\vb{A}\)的不变子空间.
受此启发,我们希望找到一些多项式\(f_1(x),\dotsc,f_s(x) \in F[x]\),
使得\[
%@see: 《高等代数(第三版 下册)》(丘维声) P134 (11)
	V = \Ker f_1(\vb{A}) \DirectSum \dotsb \DirectSum \Ker f_s(\vb{A}).
\]
为此,我们尚需研究:
对于不同的一元多项式\(f_1(x)\)和\(f_2(x)\),
不定元\(x\)用\(\vb{A}\)代入,
得到的\(f_1(\vb{A})\)与\(f_2(\vb{A})\)的核,
\(\Ker f_1(\vb{A})\)与\(\Ker f_2(\vb{A})\)之间,
有什么关系.
\begin{theorem}%\label{theorem:线性映射.线性映射多项式的核空间的直和分解式1}
%@see: 《高等代数(第三版 下册)》(丘维声) P134 定理7
设\(\vb{A}\)是域\(F\)上线性空间\(V\)上的线性变换,
而\(f(x),f_1(x),f_2(x)\)都是域\(F\)上的一元多项式.
如果\[
	f(x) = f_1(x) f_2(x)
	\quad\text{且}\quad
	(f_1(x),f_2(x)) = 1,
\]
则\[
%@see: 《高等代数(第三版 下册)》(丘维声) P134 (12)
	\Ker f(\vb{A})
	= \Ker f_1(\vb{A})
	\DirectSum \Ker f_2(\vb{A}).
\]
%TODO proof
\end{theorem}

用数学归纳法可以将上述定理推广.
\begin{corollary}\label{theorem:线性映射.线性映射多项式的核空间的直和分解式2}
%@see: 《高等代数(第三版 下册)》(丘维声) P135 推论8
设\(\vb{A}\)是域\(F\)上线性空间\(V\)上的线性变换,
\(f(x)\)以及\(f_1(x),\dotsc,f_s(x)\)都是域\(F\)上的一元多项式.
如果\[
	f(x) = f_1(x) \dotsm f_s(x)
	\quad\text{且}\quad
	\text{$f_1(x),\dotsc,f_s(x)$两两互素},
\]
则\[
%@see: 《高等代数(第三版 下册)》(丘维声) P135 (15)
	\Ker f(\vb{A})
	= \Ker f_1(\vb{A})
	\DirectSum
	\dotsb
	\DirectSum \Ker f_s(\vb{A}).
\]
\end{corollary}

\subsection{线性变换的零化多项式}
由于\(\Ker\vb0 = V\),
结合\cref{theorem:线性映射.线性映射多项式的核空间的直和分解式2} 给出的暗示,
不难想到,
如果能够找到一个多项式\(f(x)\),
使得\(f(\vb{A}) = \vb0\),
那么空间\(V\)就能分解成\[
%@see: 《高等代数(第三版 下册)》(丘维声) P135 (16)
	V = \Ker f_1(\vb{A})
	\DirectSum
	\dotsb
	\DirectSum \Ker f_s(\vb{A}),
\]
其中\(f(x) = f_1(x) \dotsm f_s(x)\),
且\(f_1(x),\dotsc,f_s(x)\)两两互素.

\begin{definition}
%@see: 《高等代数(第三版 下册)》(丘维声) P135 定义2
设\(\vb{A}\)是\(V\)上的线性变换.
如果域\(F\)上的一个一元多项式\(f(x)\)满足\(f(\vb{A}) = \vb0\),
则称“\(f(x)\)是\(\vb{A}\)的一个\DefineConcept{零化多项式}”.
\end{definition}

\begin{proposition}
%@see: 《高等代数(第三版 下册)》(丘维声) P135
有限维线性空间\(V\)上任意一个线性变换都有非零的零化多项式.
%TODO proof
\end{proposition}

\begin{definition}
%@see: 《高等代数(第三版 下册)》(丘维声) P135 定义3
设\(A\)是域\(F\)上的一个\(n\)阶矩阵.
如果域\(F\)上的一个一元多项式\(f(x)\)满足\(f(A) = 0\),
则称“\(f(x)\)是矩阵\(A\)的一个\DefineConcept{零化多项式}”.
\end{definition}

\begin{proposition}
%@see: 《高等代数(第三版 下册)》(丘维声) P135
设\(\vb{A}\)是域\(F\)上\(n\)维线性空间\(V\)上的线性变换,
\(A\)是\(\vb{A}\)在\(V\)的一个基下的矩阵,
则\[
	\text{$f(x)$是$\vb{A}$的零化多项式}
	\iff
	\text{$f(x)$是$A$的零化多项式}.
\]
%TODO proof
\end{proposition}
