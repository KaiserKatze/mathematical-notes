\section{线性变换的特征值与特征向量,线性变换可相似对角化的条件}
\subsection{线性变换的特征值与特征向量}
\cref{theorem:线性映射的矩阵表示.线性变换在不同基下的矩阵相似} 表明,
域\(F\)上\(n\)维线性空间\(V\)上的线性变换\(\vb{A}\)在\(V\)的不同基下的矩阵是相似的.
由于相似的矩阵有相同的行列式、秩、迹、特征多项式、特征值,
因此我们可以把线性变换\(\vb{A}\)在\(V\)的某一个基下的矩阵\(A\)的行列式、秩、迹、特征多项式、特征值,
分别叫做线性变换\(\vb{A}\)的行列式、秩、迹、特征多项式、特征值.

为了更好地理解线性变换的特征值的几何意义,以及对无限维线性空间上的线性变换也考虑它的特征值,
我们给出如下的定义:
\begin{definition}\label{definition:线性变换的特征值和特征向量.线性变换的特征值和特征向量}
%@see: 《高等代数(第三版 下册)》(丘维声) P127 定义1
%@see: 《Linear Algebra Done Right (Fourth Edition)》(Sheldon Axler) P134 5.5
%@see: 《Linear Algebra Done Right (Fourth Edition)》(Sheldon Axler) P135 5.8
设\(\vb{A}\)是域\(F\)上线性空间\(V\)上的一个线性变换.
如果\(V\)中存在一个非零向量\(\xi\),
使得\begin{equation*}
%@see: 《高等代数(第三版 下册)》(丘维声) P127 (1)
	\vb{A}\xi=\lambda_0\xi,
	\quad \lambda_0\in F,
\end{equation*}
则称“\(\lambda_0\)是\(\vb{A}\)的一个\DefineConcept{特征值}%
(\(\lambda_0\) is an \emph{eigenvalue} of \(\vb{A}\))”
“\(\xi\)是\(\vb{A}\)的属于特征值\(\lambda_0\)的一个\DefineConcept{特征向量}%
(\(\xi\) is an \emph{eigenvector} of \(\vb{A}\) corresponding to \(\lambda_0\))”.
\end{definition}
从\cref{definition:线性变换的特征值和特征向量.线性变换的特征值和特征向量} 看出,
线性变换\(\vb{A}\)的特征向量\(\xi\)有这样的“几何意义”:
\(\vb{A}\)对\(\xi\)的作用是把\(\xi\)“拉伸”或“压缩”\(\lambda_0\)倍.
这个倍数\(\lambda_0\)就是\(\vb{A}\)的一个特征值.

现在设\(V\)是域\(F\)上\(n\)维线性空间,
\(V\)中取定一个基\(\AutoTuple{\alpha}{n}\).
\(V\)上的一个线性变换\(\vb{A}\)在基\(\AutoTuple{\alpha}{n}\)下的矩阵是\(A\),
向量\(\xi\)在基\(\AutoTuple{\alpha}{n}\)下的坐标是\(X\),
\(\lambda_0\in F\).
于是\begin{equation}\label{equation:线性变换的特征值和特征向量.与矩阵的特征值和特征向量的联系}
%@see: 《高等代数(第三版 下册)》(丘维声) P127 (2)
	\vb{A}\xi=\lambda_0\xi
	\iff
	AX=\lambda_0X.
\end{equation}
由此得出\begin{align*}
%@see: 《高等代数(第三版 下册)》(丘维声) P127 (3)
	&\text{$\lambda_0$是$\vb{A}$的一个特征值} \\
	&\iff \text{$\lambda_0$是$A$的一个特征值} \\
%@see: 《高等代数(第三版 下册)》(丘维声) P127 (4)
	&\text{$\xi$是$\vb{A}$的属于特征值$\lambda_0$的一个特征向量} \\
	&\iff \text{$\xi$的坐标$X$是$A$的属于特征值$\lambda_0$的一个特征向量}.
\end{align*}
可以看出,对于有限维线性空间,
用线性变换的矩阵的特征值定义线性变换的特征值,与上述定义是一致的.
同时,我们还得到了求有限维线性空间上线性变换\(\vb{A}\)的全部特征值和特征向量的方法:
只要取求\(\vb{A}\)在\(V\)的一个基下的矩阵\(A\)的全部特征值和特征向量.
但是要注意:
矩阵\(A\)的特征向量\(X\)是线性变换\(\vb{A}\)的特征向量\(\xi\)在基\(\AutoTuple{\alpha}{n}\)下的坐标.

\begin{proposition}
%@see: 《Linear Algebra Done Right (Fourth Edition)》(Sheldon Axler) P135 5.7
设\(V\)是域\(F\)上的有限维线性空间,
\(\vb{A}\)是\(V\)上的一个线性变换,
\(\vb{I}\)是\(V\)上的恒等变换,
则下列命题互相等价:\begin{itemize}
	\item \(\lambda\)是\(\vb{A}\)的一个特征值;
	\item \(\lambda\vb{I}-\vb{A}\)不是单射;
	\item \(\lambda\vb{I}-\vb{A}\)不是满射;
	\item \(\lambda\vb{I}-\vb{A}\)不可逆.
\end{itemize}
\begin{proof}
根据定义,
\(\lambda\)是\(\vb{A}\)的一个特征值,
当且仅当关于\(\xi \in V\)的方程\(\vb{A}\xi=\lambda\xi\)或\((\lambda\vb{I}-\vb{A})\xi=\vb0\)有非零解.
根据\cref{theorem:线性映射.线性映射的单射性决定齐次线性方程组是否具有非零解},
\((\lambda\vb{I}-\vb{A})\xi=\vb0\)有非零解,
当且仅当\(\lambda\vb{I}-\vb{A}\)不是单射.
因为\hyperref[theorem:线性映射.有限维线性空间.单线性映射是满线性映射]{在有限维线性空间中单线性映射就是满线性映射},
所以\((\lambda\vb{I}-\vb{A})\xi=\vb0\)有非零解,
当且仅当\(\lambda\vb{I}-\vb{A}\)不是满射.
\end{proof}
\end{proposition}

\begin{example}
%@see: 《高等代数(第三版 下册)》(丘维声) P130 习题9.4 4.
%@see: 《Linear Algebra Done Right (Fourth Edition)》(Sheldon Axler) P136 5.11
设\(V\)是域\(F\)上任意一个线性空间,
\(\vb{A}\)是\(V\)上的一个线性变换.
证明:\(\vb{A}\)的属于不同特征值的特征向量是线性无关的.
\begin{proof}
设\(\lambda_1,\lambda_2\)是\(\vb{A}\)的两个不同特征值,
\(\xi_1\)是\(\vb{A}\)的属于\(\lambda_1\)的一个特征向量,
\(\xi_2\)是\(\vb{A}\)的属于\(\lambda_2\)的一个特征向量.
令\begin{equation*}
	k_1 \xi_1 + k_2 \xi_2 = 0,
	\quad k_1,k_2 \in F.
\end{equation*}
%\(\vb{A} \xi_1 = \lambda_1 \xi_1\)
%\(\vb{A} \xi_2 = \lambda_2 \xi_2\)
那么\begin{gather*}
	\vb{A}(k_1 \xi_1 + k_2 \xi_2)
	= k_1 \lambda_1 \xi_1 + k_2 \lambda_2 \xi_2
	= 0, \tag1 \\
	\lambda_1(k_1 \xi_1 + k_2 \xi_2)
	= k_1 \lambda_1 \xi_1 + k_2 \lambda_1 \xi_2
	= 0, \tag2
\end{gather*}
将(1)(2)两式相减,得\begin{equation*}
	k_2 (\lambda_2 - \lambda_1) \xi_2 = 0.
\end{equation*}
由于\(\xi_2\)是非零向量,
所以\(k_2 (\lambda_2 - \lambda_1) = 0\).
由于\(\lambda_1 \neq \lambda_2\),
所以\(k_2 = 0\).
同理可证\(k_1 = 0\).
因此向量组\(\xi_1,\xi_2\)线性无关.
\end{proof}
%\cref{theorem:矩阵相似对角化.不同特征值的特征向量线性无关}
\end{example}

\begin{example}
%@see: 《高等代数(第三版 下册)》(丘维声) P130 习题9.4 5.
设\(V\)是域\(F\)上任意一个线性空间,
\(\vb{A}\)是\(V\)上的一个线性变换,
\(\lambda_1,\lambda_2\)是\(\vb{A}\)的两个不同特征值,
\(\xi_1,\xi_2\)分别是\(\vb{A}\)的属于\(\lambda_1,\lambda_2\)的特征向量.
证明:\(\xi_1+\xi_2\)不是\(\vb{A}\)的特征向量.
\begin{proof}
由题意有\(\vb{A} \xi_1 = \lambda_1 \xi_1,
\vb{A} \xi_2 = \lambda_2 \xi_2\).
假设\(\xi_1+\xi_2\)是\(\vb{A}\)的属于特征值\(\lambda\)的特征向量,
即\begin{equation*}
	\vb{A} (\xi_1 + \xi_2) = \lambda (\xi_1 + \xi_2),
\end{equation*}
则\begin{gather*}
	\vb{A} \xi_1 + \vb{A} \xi_2
	= \lambda_1 \xi_1 + \lambda_2 \xi_2
	= \lambda \xi_1 + \lambda \xi_2, \\
	(\lambda - \lambda_1) \xi_1 + (\lambda - \lambda_2) \xi_2
	= 0.
\end{gather*}
由于\(\vb{A}\)的属于不同特征值的特征向量是线性无关的,
所以\(\lambda = \lambda_1 = \lambda_2\),矛盾!
这说明\(\xi_1+\xi_2\)不是\(\vb{A}\)的特征向量.
\end{proof}
%@see: 《线性代数》(张慎语、周厚隆) P95 例5
\end{example}

\begin{example}
%@see: 《高等代数(第三版 下册)》(丘维声) P130 习题9.4 6.
设\(V\)是域\(F\)上任意一个线性空间,
\(\vb{A}\)是\(V\)上的一个线性变换.
证明:如果\(V\)中每一个非零向量都是\(\vb{A}\)的特征向量,则\(\vb{A}\)是数乘变换.
\begin{proof}
设\(V\)中每一个非零向量都是\(\vb{A}\)的特征向量.
假设\(\lambda_1,\lambda_2\)是\(\vb{A}\)的两个不同特征值,
\(\xi_1\)是\(\vb{A}\)的属于\(\lambda_1\)的一个特征向量,
\(\xi_2\)是\(\vb{A}\)的属于\(\lambda_2\)的一个特征向量,
那么\(\xi_1+\xi_2\)不是\(\vb{A}\)的特征向量,矛盾!
这说明\(\vb{A}\)有且仅有1个特征值,
\(\vb{A}\)是数乘变换.
\end{proof}
\end{example}

\begin{example}
%@see: 《高等代数(第三版 下册)》(丘维声) P130 习题9.4 7.(1)
设\(V\)是域\(F\)上任意一个线性空间,
\(\vb{A}\)是\(V\)上的一个可逆线性变换.
证明:\(\vb{A}\)的特征值一定不等于\(0\).
\begin{proof}
\begin{proof}[证法一]
用反证法.
假设\(0\)是可逆线性变换\(\vb{A}\)的一个特征值,
那么存在一个非零向量\(\xi\)使得
\(\vb{A}\xi = 0\xi = 0\),
从而有\(\vb{A}(2\xi) = 0(2\xi) = 0\),
显然\(\xi \neq 2\xi\),
这就说明\(\vb{A}\)不是单射.
但是,由\(\vb{A}\)是可逆线性变换可知\(\vb{A}\)是双射,矛盾!
因此\(\vb{A}\)的特征值一定不等于\(0\).
\end{proof}
\begin{proof}[证法二]
设\(A\)是\(\vb{A}\)在\(V\)的某一个基下的矩阵.
既然\(\vb{A}\)是可逆线性变换,那么\(A\)是可逆矩阵.
由\cref{example:矩阵的特征值与特征向量.零不是非奇异矩阵的特征值} 可知,
可逆矩阵\(A\)的特征值一定不等于\(0\),
那么\(\vb{A}\)的特征值也一定不等于\(0\).
\end{proof}\let\qed\relax
\end{proof}
\end{example}

\begin{example}
%@see: 《高等代数(第三版 下册)》(丘维声) P130 习题9.4 7.(2)
设\(V\)是域\(F\)上任意一个线性空间,
\(\vb{A}\)是\(V\)上的一个可逆线性变换.
证明:如果\(\lambda\)是\(\vb{A}\)的特征值,则\(\lambda^{-1}\)是\(\vb{A}^{-1}\)的特征值.
\begin{proof}
\begin{proof}[证法一]
设\(\lambda\neq0\)是\(\vb{A}\)的一个特征值,
\(\xi\)是\(\vb{A}\)的属于\(\lambda\)的一个特征向量,
即\(\vb{A}\xi=\lambda\xi\),
那么\begin{equation*}
	\vb{A}^{-1}(\vb{A}\xi)
	= (\vb{A}^{-1}\vb{A})\xi
	= \vb{I}\xi
	= \xi,
	\qquad
	\vb{A}^{-1}(\lambda\xi)
	= \lambda(\vb{A}^{-1}\xi),
\end{equation*}
于是\(\lambda(\vb{A}^{-1}\xi)=\xi\),
即\(\vb{A}^{-1}\xi=\lambda^{-1}\xi\).
\end{proof}
\begin{proof}[证法二]
设\(A\)是\(\vb{A}\)在\(V\)的某一个基下的矩阵.
既然\(\vb{A}\)是可逆线性变换,那么\(A\)是可逆矩阵.
由\cref{example:矩阵的特征值与特征向量.矩阵的多项式的特征值3} 可知,
如果\(\lambda\)是\(A\)的一个特征值,
那么\(\lambda^{-1}\)就是\(A\)的逆矩阵\(A^{-1}\)的一个特征值,
于是\(\lambda^{-1}\)是\(\vb{A}^{-1}\)的特征值.
\end{proof}\let\qed\relax
\end{proof}
\end{example}

\begin{proposition}
%@see: 《Linear Algebra Done Right (Fourth Edition)》(Sheldon Axler) P136 5.11
设\(V\)是域\(F\)上任意一个线性空间,
\(\vb{A}\)是\(V\)上的一个可逆线性变换,
则\begin{equation*}
	\card\Set{
		\lambda
		\given
		\text{$\lambda$是$\vb{A}$的特征值}
	}
	\leq \dim V.
\end{equation*}
%TODO proof
\end{proposition}

\subsection{线性变换的特征子空间}
设\(\vb{A}\)是域\(F\)上线性空间\(V\)上的一个线性变换,
\(\lambda_0\)是\(\vb{A}\)的一个特征值.
令\begin{equation}\label{equation:线性变换的特征值和特征向量.特征子空间}
%@see: 《高等代数(第三版 下册)》(丘维声) P128 (5)
	V_{\lambda_0}
	\defeq
	\Set{ \alpha \in V \given \vb{A}\alpha=\lambda_0\alpha }.
\end{equation}
可以验证\(V_{\lambda_0}\)是\(V\)的一个子空间,
因此称“\(V_{\lambda_0}\)是\(\vb{A}\)的属于特征值\(\lambda_0\)的\DefineConcept{特征子空间}”.

\(V_{\lambda_0}-\{0\}\)中的全部向量就是\(\vb{A}\)的属于\(\lambda_0\)的全部特征向量.

易知\begin{equation*}
	V_{\lambda_0}
	= \Ker(\lambda_0 \vb{I} - \vb{A}).
\end{equation*}

\subsection{线性变换的根子空间}
%@see: 《高等代数(第三版 下册)》(丘维声) P139 习题9.6 1.
设\(\vb{A}\)是域\(F\)上线性空间\(V\)上的一个线性变换,
\(\lambda_0\)是\(\vb{A}\)的一个特征值.
定义:\begin{equation}
	R_{\lambda_0} \defeq \Set{
		\alpha \in V
		\given
		(\exists r>0)
		[
			(\lambda_0 \vb{I} - \vb{A})^r \alpha = 0
		]
	}.
\end{equation}
可以验证\(R_{\lambda_0}\)是一个线性空间,
因此称“\(R_{\lambda_0}\)是\(\vb{A}\)的属于特征值\(\lambda_0\)的\DefineConcept{根子空间}”.

\subsection{线性变换的几何重数、代数重数}
设\(V\)是域\(F\)上\(n\)维线性空间,
\(V\)上线性变换\(\vb{A}\)在\(V\)的一个基\(\AutoTuple{\alpha}{n}\)下的矩阵为\(A\),
\(\lambda_0\)是\(\vb{A}\)的一个特征值,
由\cref{equation:线性变换的特征值和特征向量.与矩阵的特征值和特征向量的联系,%
equation:线性变换的特征值和特征向量.特征子空间} 可得,
线性变换\(\vb{A}\)的属于\(\lambda_0\)的特征子空间\(V_{\lambda_0}\)的
所有向量的坐标组成的集合是矩阵\(A\)的属于\(\lambda_0\)的特征子空间,
后者就是齐次线性方程组\((\lambda_0 E - A) X = 0\)的解空间,
因此\begin{equation}
	\dim V_{\lambda_0}
	= n - \rank(\lambda_0 E - A).
\end{equation}
\(V_{\lambda_0}\)的维数称为
“\(\vb{A}\)的特征值\(\lambda_0\)的\DefineConcept{几何重数}”,
\(\lambda_0\)作为\(\vb{A}\)的特征多项式的根的重数称为
“\(\lambda_0\)的\DefineConcept{代数重数}”.

\subsection{线性变换可相似对角化的条件}
无论是在理论上,还是在实际应用上,
我们都希望能在\(V\)中找到一个适当的基,
使得预先给定的线性变换\(\vb{A}\)在这个基下的矩阵具有最简单的形式.
由于\(\vb{A}\)在\(V\)的不同基下的矩阵是相似的,
因此这个问题也就转化为,
求\(\vb{A}\)在\(V\)的一个基下的矩阵\(A\)的相似标准型.
之前我们曾经讨论过\(n\)阶矩阵的相似标准型问题,
给出了\(A\)的相似标准型为对角矩阵的充分必要条件.
但是,对于不可以对角化的矩阵\(A\),它的相似标准型是什么?
我们尚未讨论.
接下来,我们就开始研究如何寻找\(V\)的一个恰当的基,
使得线性变换\(\vb{A}\)在这个基下的矩阵具有最简形式.

如果\(V\)中存在一个基,使得线性变换\(\vb{A}\)在这个基下的矩阵是对角矩阵,
则称线性变换\(\vb{A}\)可对角化.

设线性变换\(\vb{A}\)在\(V\)的一个基\(\AutoTuple{\alpha}{n}\)下的矩阵为\(A\),
则\(\vb{A}\)可对角化,当且仅当\(A\)可对角化.
于是从\(n\)阶矩阵\(A\)可对角化的充分必要条件,
以及\(V\)与\(F^n\)同构的性质,
可以得出线性变换\(\vb{A}\)可对角化的充分必要条件:
\begin{theorem}
%@see: 《高等代数(第三版 下册)》(丘维声) P129 定理1
%@see: 《高等代数(第三版 下册)》(丘维声) P131 习题9.4 8.
设\(\vb{A}\)是域\(F\)上\(n\)维线性空间\(V\)上的一个线性变换,
则\begin{align*}
	&\text{$\vb{A}$可对角化} \\
	&\iff \text{$\vb{A}$有$n$个线性无关的特征向量} \\
	&\iff \text{$V$中存在由$\vb{A}$的特征向量组成的一个基} \\
	&\iff \text{$\vb{A}$的属于不同特征值的特征子空间的维数之和等于$n$} \\
	&\iff V = \BigDirectSum_{i=1}^s V_{\lambda_i},
\end{align*}
其中\(\AutoTuple{\lambda}{s}\)是\(\vb{A}\)的所有不同特征值.
\end{theorem}

如果\(\vb{A}\)可对角化,
则\(\vb{A}\)有\(n\)个线性无关的特征向量\(\AutoTuple{\xi}{n}\),
从而有\begin{equation*}
%@see: 《高等代数(第三版 下册)》(丘维声) P129 (7)
	\vb{A} (\AutoTuple{\xi}{n})
	= (\AutoTuple{\xi}{n})
	\diag(\AutoTuple{\lambda}{n}),
\end{equation*}
其中\(\vb{A} \xi_i = \lambda_i \xi_i\ (i=1,2,\dotsc,n)\).
从上式可以看出,等号右端的对角矩阵的主对角元,恰好是\(\vb{A}\)的全部特征值(重根按重数计算).
因此这个对角矩阵除了主对角线上元素的排列次序外,是由线性变换\(\vb{A}\)唯一决定的.
我们把这个对角矩阵称为线性变换\(\vb{A}\)的\DefineConcept{标准型}.

\begin{example}
%@see: 《高等代数(第三版 下册)》(丘维声) P131 习题9.4 9.
设\(\vb{A}\)是域\(F\)上\(n\)维线性空间\(V\)上的一个线性变换.
证明:\(\vb{A}\)可对角化,
当且仅当\(\vb{A}\)的特征多项式\(f(\lambda)\)在\(F[\lambda]\)中的标准分解式为\begin{equation*}
	f(\lambda)
	= (\lambda-\lambda_1)^{r_1} \dotsm (\lambda-\lambda_s)^{r_s},
\end{equation*}
并且\(\vb{A}\)的每一个特征值\(\lambda_i\)的几何重数等于它的代数重数.
%TODO proof
%\cref{theorem:矩阵可以相似对角化的充分必要条件.定理2}
\end{example}
