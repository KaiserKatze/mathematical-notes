\section{线性变换的特征值与特征向量,线性变换可相似对角化的条件}
\cref{theorem:线性映射的矩阵表示.线性变换在不同基下的矩阵相似} 表明,
域\(F\)上\(n\)维线性空间\(V\)上的线性变换\(\vb{A}\)在\(V\)的不同基下的矩阵是相似的.
由于相似的矩阵有相同的行列式、秩、迹、特征多项式、特征值,
因此我们可以把线性变换\(\vb{A}\)在\(V\)的某一个基下的矩阵\(A\)的行列式、秩、迹、特征多项式、特征值,
分别叫做线性变换\(\vb{A}\)的行列式、秩、迹、特征多项式、特征值.

为了更好地理解线性变换的特征值的几何意义,以及对无限维线性空间上的线性变换也考虑它的特征值,
我们给出如下的定义:
\begin{definition}\label{definition:线性变换的特征值和特征向量.线性变换的特征值和特征向量}
%@see: 《高等代数(第三版 下册)》(丘维声) P127 定义1
设\(\vb{A}\)是域\(F\)上维线性空间\(V\)上的一个线性变换.
如果\(V\)中存在一个非零向量\(\xi\),
使得\[
%@see: 《高等代数(第三版 下册)》(丘维声) P127 (1)
	\vb{A}\xi=\lambda_0\xi,
	\quad \lambda_0\in F,
\]
则称“\(\lambda_0\)是\(\vb{A}\)的一个\DefineConcept{特征值}”
“\(\xi\)是\(\vb{A}\)的属于特征值\(\lambda_0\)的一个\DefineConcept{特征向量}”.
\end{definition}
从\cref{definition:线性变换的特征值和特征向量.线性变换的特征值和特征向量} 看出,
线性变换\(\vb{A}\)的特征向量\(\xi\)有这样的“几何意义”:
\(\vb{A}\)对\(\xi\)的作用是把\(\xi\)“拉伸”或“压缩”\(\lambda_0\)倍.
这个倍数\(\lambda_0\)就是\(\vb{A}\)的一个特征值.

现在设\(V\)是域\(F\)上\(n\)维线性空间,
\(V\)中取定一个基\(\AutoTuple{\alpha}{n}\).
\(V\)上的一个线性变换\(\vb{A}\)在基\(\AutoTuple{\alpha}{n}\)下的矩阵是\(A\),
向量\(\xi\)在基\(\AutoTuple{\alpha}{n}\)下的坐标是\(X\),
\(\lambda_0\in F\).
于是\begin{equation}\label{equation:线性变换的特征值和特征向量.与矩阵的特征值和特征向量的联系}
%@see: 《高等代数(第三版 下册)》(丘维声) P127 (2)
	\vb{A}\xi=\lambda_0\xi
	\iff
	AX=\lambda_0X.
\end{equation}
由此得出\begin{align*}
%@see: 《高等代数(第三版 下册)》(丘维声) P127 (3)
	&\text{$\lambda_0$是$\vb{A}$的一个特征值} \\
	&\iff \text{$\lambda_0$是$A$的一个特征值} \\
%@see: 《高等代数(第三版 下册)》(丘维声) P127 (4)
	&\text{$\xi$是$\vb{A}$的属于特征值$\lambda_0$的一个特征向量} \\
	&\iff \text{$\xi$的坐标$X$是$A$的属于特征值$\lambda_0$的一个特征向量}.
\end{align*}
可以看出,对于有限维线性空间,
用线性变换的矩阵的特征值定义线性变换的特征值,与上述定义是一致的.
同时,我们还得到了求有限维线性空间上线性变换\(\vb{A}\)的全部特征值和特征向量的方法:
只要取求\(\vb{A}\)在\(V\)的一个基下的矩阵\(A\)的全部特征值和特征向量.
但是要注意:
矩阵\(A\)的特征向量\(X\)是线性变换\(\vb{A}\)的特征向量\(\xi\)在基\(\AutoTuple{\alpha}{n}\)下的坐标.

设\(\vb{A}\)是域\(F\)上线性空间\(V\)上的一个线性变换,
\(\lambda_0\)是\(\vb{A}\)的一个特征值.
令\begin{equation}\label{equation:线性变换的特征值和特征向量.特征子空间}
%@see: 《高等代数(第三版 下册)》(丘维声) P128 (5)
	V_{\lambda_0}
	\defeq
	\Set{ \alpha \in V \given \vb{A}\alpha=\lambda_0\alpha },
\end{equation}
可以验证\(V_{\lambda_0}\)是\(V\)的一个子空间,
因此称“\(V_{\lambda_0}\)是\(\vb{A}\)的属于特征值\(\lambda_0\)的\DefineConcept{特征子空间}”.
\(V_{\lambda_0}-\{0\}\)中的全部向量就是\(\vb{A}\)的属于\(\lambda_0\)的全部特征向量.

设\(V\)是域\(F\)上\(n\)维线性空间,
\(V\)上线性变换\(\vb{A}\)在\(V\)的一个基\(\AutoTuple{\alpha}{n}\)下的矩阵为\(A\),
\(\lambda_0\)是\(\vb{A}\)的一个特征值,
由\cref{equation:线性变换的特征值和特征向量.与矩阵的特征值和特征向量的联系,%
equation:线性变换的特征值和特征向量.特征子空间} 可得,
线性变换\(\vb{A}\)的属于\(\lambda_0\)的特征子空间\(V_{\lambda_0}\)的
所有向量的坐标组成的集合是矩阵\(A\)的属于\(\lambda_0\)的特征子空间,
后者就是齐次线性方程组\((\lambda_0 E - A) X = 0\)的解空间,
因此\begin{equation}
	\dim V_{\lambda_0}
	= n - \rank(\lambda_0 E - A).
\end{equation}
\(V_{\lambda_0}\)的维数称为
“\(\vb{A}\)的特征值\(\lambda_0\)的\DefineConcept{几何重数}”,
\(\lambda_0\)作为\(\vb{A}\)的特征多项式的根的重数称为
“\(\lambda_0\)的\DefineConcept{代数重数}”.
