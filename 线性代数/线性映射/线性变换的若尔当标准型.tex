\section{线性变换的若尔当标准型}
现在我们利用幂零变换的结构来研究最小多项式可以分解成一次因式乘积的线性变换的结构.
\begin{theorem}
%@see: 《高等代数(第三版 下册)》(丘维声) P153 定理1
设\(\vb{A}\)是域\(F\)上\(n\)维线性空间\(V\)上的线性变换.
如果\(\vb{A}\)的最小多项式\(m(\lambda)\)
在\(F[\lambda]\)中能分解成一次因式的乘积\begin{equation*}
	m(\lambda)
	= (\lambda-\lambda_1)^{l_1}
	(\lambda-\lambda_2)^{l_2}
	\dotsm
	(\lambda-\lambda_s)^{l_s},
\end{equation*}
则\(V\)中存在一个基,
使得\(\vb{A}\)在这个基下的矩阵\(A\)是若尔当形矩阵,其主对角元是\(\vb{A}\)的全部特征值,
主对角元为\(\lambda_j\)的若尔当块的总数为\begin{equation*}
	N_j = \dim V - \rank(\vb{A}-\lambda_j\vb{I}),
	\quad j=1,2,\dotsc,s,
\end{equation*}
且其中每个若尔当块的阶数不超过\(l_j\),
\(t\)阶若尔当块\(J_t(\lambda_j)\)的个数为\begin{equation*}
	N_j(t)
	= \rank(\vb{A}-\lambda_j\vb{I})^{t+1}
	+ \rank(\vb{A}-\lambda_j\vb{I})^{t-1}
	- 2 \rank(\vb{A}-\lambda_j\vb{I})^t.
\end{equation*}
\rm
这个若尔当形矩阵\(A\)称为\(\vb{A}\)的若尔当标准型.
除去若尔当块的排序次序外,\(\vb{A}\)的若尔当标准型是唯一的.
%TODO proof
\end{theorem}
