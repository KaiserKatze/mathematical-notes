\section{线性映射及其运算}
\subsection{线性映射的概念}
\begin{definition}
%@see: 《高等代数(第三版 下册)》(丘维声) P106 定义1
设\(V\)和\(V'\)都是域\(F\)上的线性空间,
\(\vb{A}\)是从\(V\)到\(V'\)的一个映射.
如果\begin{gather*}
	(\forall\alpha,\beta\in V)
	[\vb{A}(\alpha+\beta)=\vb{A}(\alpha)+\vb{A}(\beta)], \\
	(\forall\alpha\in V)
	(\forall k\in F)
	[\vb{A}(k\alpha)=k\vb{A}(\alpha)],
\end{gather*}
则称“\(\vb{A}\)是从\(V\)到\(V'\)的一个\DefineConcept{线性映射}(linear map)”.
%@see: https://mathworld.wolfram.com/LinearTransformation.html
\end{definition}

如果线性映射\(\vb{A}\)是单射,
则称“\(\vb{A}\)是\DefineConcept{单线性映射}”.

如果线性映射\(\vb{A}\)是满射,
则称“\(\vb{A}\)是\DefineConcept{满线性映射}”.

线性空间\(V\)到自身的线性映射称为
“\(V\)上的\DefineConcept{线性变换}”.

域\(F\)上的线性空间\(V\)到\(F\)的线性映射称为
“\(V\)上的\DefineConcept{线性函数}”.

\begin{example}
%@see: 《高等代数(第三版 下册)》(丘维声) P107 例1
设\(V\)和\(V'\)都是域\(F\)上的线性空间,
\(0'\)是\(V'\)的零元,
映射\(\vb{A}=V\times\{0'\}\).
我们把\(\vb{A}\)称为
“从\(V\)到\(V'\)的\DefineConcept{零映射}”,
记作\(\vb0\).
显然零映射\(\vb0\)是线性映射.
\end{example}

\begin{example}
%@see: 《高等代数(第三版 下册)》(丘维声) P107 例2
设\(V\)是域\(F\)上的线性空间,
映射\(\vb{A}\colon V\to V\)
满足\((\forall\alpha\in V)[\vb{A}(\alpha)=\alpha]\).
我们把\(\vb{A}\)称为
“\(V\)上的\DefineConcept{恒等变换}(the \emph{identity operator} on \(V\))”,
记作\(\vb1_V\)或\(\vb{I}\).
显然恒等变换\(\vb1_V\)是\(V\)上的一个线性变换.
\end{example}

\begin{example}
%@see: 《高等代数(第三版 下册)》(丘维声) P107 例3
给定\(k\in F\),
\(F\)上线性空间\(V\)到自身的一个映射\(\vb{k}(\alpha)=k\alpha\),
称为“\(V\)上由\(k\)决定的\DefineConcept{数乘变换}”,
它是\(V\)上的一个线性变换.
当\(k=0\)时,便得到零变换;
当\(k=1\)时,便得到恒等变换.
\end{example}

\begin{example}
%@see: 《高等代数(第三版 下册)》(丘维声) P107 例4
设\(\vb{A}\)是域\(F\)上的一个\(s \times n\)矩阵,
用\(\vb{A}\)左乘\(F^n\)中的向量时,
\(\vb{A}\)可以看成是\(F^n\)到\(F^s\)的一个线性映射.
\end{example}

\begin{example}
%@see: 《高等代数(第三版 下册)》(丘维声) P107 例5
区间\((a,b)\)上的\(1\)阶连续可导函数族\(C^1(a,b)\)
是实数域\(\mathbb{R}\)上的线性空间\(\mathbb{R}^{(a,b)}\)的一个子空间.
求导运算\(\vb{D}\)是\(C^1(a,b)\)到\(\mathbb{R}^{(a,b)}\)的一个线性映射.
\end{example}

\begin{example}
%@see: 《高等代数(第三版 下册)》(丘维声) P106
闭区间\([a,b]\)上全体连续函数\(C[a,b]\)对于函数的加法,以及数与函数的数量乘法,
成为实数域\(\mathbb{R}\)上的线性空间.
函数的定积分是从\(C[a,b]\)到\(\mathbb{R}\)的线性映射,
它具有下列性质:\begin{gather*}
	\int_a^b (f(x) + g(x)) \dd{x}
	= \int_a^b f(x) \dd{x} + \int_a^b f(x) \dd{x}, \\
	\int_a^b k f(x) \dd{x}
	= k \int_a^b f(x) \dd{x}.
\end{gather*}
这就说明函数的定积分保持加法、数量乘法两种运算.
函数的定积分是\(C[a,b]\)上的线性函数.
\end{example}

\begin{example}
%@see: 《高等代数(大学高等代数课程创新教材 第一版 下册)》(丘维声) P227 例8
%@see: 《高等代数(大学高等代数课程创新教材 第二版 下册)》(丘维声) P231 例8
设\(W\)是域\(F\)上线性空间\(V\)的一个子空间,
\(V/W\)是\(V\)对\(W\)的商空间.
从\(V\)到\(V/W\)的标准映射\[
	\pi\colon V \to V/W,
	\alpha \mapsto \alpha+W
\]是一个线性映射,
它具有下列性质:\begin{gather*}
	\pi(\alpha+\beta)
	= (\alpha+\beta)+W
	= (\alpha+W) + (\beta+W)
	= \pi(\alpha) + \pi(\beta), \\
	\pi(k\alpha)
	= (k\alpha)+W
	= k(\alpha+W)
	= k\pi(\alpha).
\end{gather*}
\end{example}

\begin{example}
%@see: 《高等代数(第三版 下册)》(丘维声) P112 习题9.1 1.(1)
判断\(K^3\)上的变换\[
	\vb{A}
	\begin{bmatrix}
		x_1 \\ x_2 \\ x_3
	\end{bmatrix}
	= \begin{bmatrix}
		x_1 - x_2 \\
		x_2 + x_3 \\
		x_3^2
	\end{bmatrix}
\]是不是线性变换.
\begin{solution}
取\(\alpha=(0,0,1),
k=2\),
则\[
	k\vb{A}(\alpha)
	= \begin{bmatrix}
		0 \\ 2 \\ 2
	\end{bmatrix}
	\neq
	\vb{A}(k\alpha)
	= \begin{bmatrix}
		0 \\ 2 \\ 4
	\end{bmatrix},
\]
\(\vb{A}\)不满足线性映射的定义,
于是\(\vb{A}\)不是线性变换.
\end{solution}
\end{example}

\begin{example}
%@see: 《高等代数(第三版 下册)》(丘维声) P112 习题9.1 2.(1)
设\(A \in M_n(K)\).
令\[
	\vb{A}(X) \defeq X A,
	\quad X \in M_n(K).
\]
判断\(\vb{A}\)是不是\(M_n(K)\)上的线性变换.
\begin{proof}
任取\(\alpha,\beta \in M_n(K),
k \in K\),
则\begin{gather*}
	\vb{A}(\alpha + \beta)
	= (\alpha + \beta) A
	= \alpha A + \beta A
	= \vb{A}(\alpha) + \vb{A}(\beta), \\
	\vb{A}(k \alpha)
	= k \alpha A
	= k \vb{A}(\alpha),
\end{gather*}
由此可见\(\vb{A}\)是\(M_n(K)\)上的线性变换.
\end{proof}
\end{example}

\begin{example}
%@see: 《高等代数(第三版 下册)》(丘维声) P112 习题9.1 2.(2)
设\(B,C \in M_n(K)\).
令\[
	\vb{A}(X) \defeq B X C,
	\quad X \in M_n(K).
\]
判断\(\vb{A}\)是不是\(M_n(K)\)上的线性变换.
\begin{proof}
任取\(\alpha,\beta \in M_n(K),
k \in K\),
则\begin{gather*}
	\vb{A}(\alpha+\beta)
	= (B(\alpha+\beta))C
	= (B\alpha+B\beta)C
	= B \alpha C + B \beta C
	= \vb{A}\alpha + \vb{A}\beta, \\
	\vb{A}(k \alpha)
	= (B(k\alpha))C
	= k (B \alpha C)
	= k \vb{A}\alpha,
\end{gather*}
由此可见\(\vb{A}\)是\(M_n(K)\)上的线性变换.
\end{proof}
\end{example}

\begin{example}
%@see: 《高等代数(第三版 下册)》(丘维声) P112 习题9.1 3.
设\(a \in K\).
判断\(K[x]\)上的变换\[
	\vb{A} \defeq \Set{
		(f(x),f(x+a))
		\given
		f(x) \in K[x]
	}
\]是不是线性变换.
\begin{solution}
任取\(u(x),v(x) \in K[x]\),
任取\(k \in K\),
则\begin{gather*}
	\vb{A}(u(x)+v(x))
	= u(x+a) + v(x+a)
	= \vb{A}u(x) + \vb{A}v(x), \\
	\vb{A}(k u(x))
	= k u(x+a)
	= k \vb{A}u(x),
\end{gather*}
所以\(\vb{A}\)是\(K[x]\)上的线性变换.
\end{solution}
\end{example}

\begin{example}
%@see: 《高等代数(第三版 下册)》(丘维声) P112 习题9.1 4.
在正实数集\(\mathbb{R}^+\)上定义加法、数量乘法:\begin{gather*}
	\oplus \defeq \Set{
		((a,b),ab)
		\given
		a,b \in \mathbb{R}^+
	}, \\
	\odot \defeq \Set{
		((k,a),a^k)
		\given
		a \in \mathbb{R}^+,
		k \in \mathbb{R}
	}.
\end{gather*}
设\(a>0\)且\(a\neq1\).
判断从\(\mathbb{R}^+\)到\(\mathbb{R}\)的映射\[
	\vb{A} \defeq \Set{
		(x,\log_a x)
		\given
		x \in \mathbb{R}^+
	}
\]是不是线性映射.
\begin{solution}
任取\(u,v\in\mathbb{R}^+\),
任取\(k\in\mathbb{R}\),
则\begin{gather*}
	\vb{A}(u \oplus v)
	= \log_a (u v)
	= \log_a u + \log_a v
	= \vb{A}u + \vb{A}v, \\
	\vb{A}(k \odot u)
	= \log_a u^k
	= k \log_a u
	= k \vb{A}u,
\end{gather*}
所以\(\vb{A}\)是从\((\mathbb{R}^+,\oplus,\odot)\)到\((\mathbb{R},+,\cdot)\)的线性映射.
\end{solution}
%@see: 《高等代数(第三版 下册)》(丘维声) P81 习题8.1 1.(2)
\end{example}

\begin{example}
%@see: 《高等代数(第三版 下册)》(丘维声) P113 习题9.1 5.
设\(V\)是\(K[x,y]\)中所有\(m\)次齐次多项式组成的集合,
它对于多项式的加法,以及数与多项式的乘法,成为数域\(K\)上的一个线性空间.
给定数域\(K\)上的一个2阶矩阵\begin{equation*}
	A = \begin{bmatrix}
		a_{11} & a_{12} \\
		a_{21} & a_{22}
	\end{bmatrix}.
\end{equation*}
定义:\begin{equation*}
	\vb{A} \defeq \Set{
		(f(x,y),f(a_{11}x+a_{21}y,a_{12}x+a_{22}y))
		\given
		f(x,y) \in V
	}.
\end{equation*}
判断\(\vb{A}\)是不是\(V\)上的一个线性变换.
\begin{solution}
显然\(\vb{A}\)是一个映射.
任取\(u(x,y),v(x,y) \in K[x,y]\),
任取\(k \in K\),
则\begin{align*}
	\vb{A}(u(x,y) + v(x,y))
	&= u(a_{11}x+a_{21}y,a_{12}x+a_{22}y) + v(a_{11}x+a_{21}y,a_{12}x+a_{22}y) \\
	&= \vb{A}u(x,y) + \vb{A}v(x,y), \\
	\vb{A}(k u(x,y))
	&= k u(a_{11}x+a_{21}y,a_{12}x+a_{22}y) \\
	&= k \vb{A}u(x,y),
\end{align*}
所以\(\vb{A}\)是\(V\)上的线性变换.
\end{solution}
\end{example}

\begin{example}
%@see: 《高等代数(第三版 下册)》(丘维声) P113 习题9.1 6.
把\(2^m\)个元素的有限域\(F_{2^m}\)看成\(F_2\)上的线性空间.
定义:\begin{equation*}
	\vb{A} \defeq \Set{
		(x,x^2)
		\given
		x \in F_{2m}
	}.
\end{equation*}
判断\(\vb{A}\)是不是\(F_{2^m}\)上的一个线性变换.
%TODO
\end{example}

\begin{example}
%@see: 《高等代数(第三版 下册)》(丘维声) P113 习题9.1 7.
定义:\begin{equation*}
	\vb{A} \defeq \Set{
		(f(x),x f(x))
		\given
		f(x) \in K[x]
	}.
\end{equation*}
证明:\begin{itemize}
	\item \(\vb{A}\)是\(K[x]\)上的一个线性变换;
	\item \(\vb{D}\vb{A}-\vb{A}\vb{D}=\vb{I}\),其中\(\vb{D}\)表示求导数.
\end{itemize}
%TODO
\end{example}

\subsection{线性映射的性质}
由于线性映射只比同构映射少了双射这一条件,
因此同构映射的性质中,
只要它的证明没有用到单射和满射的条件,
那么对于线性映射也成立.
\begin{property}
%@see: 《高等代数(第三版 下册)》(丘维声) P107
设\(\vb{A}\)是域\(F\)上线性空间\(V\)到\(V'\)的线性映射,
则\(\vb{A}\)有下述性质:
\begin{itemize}
	\item \(\vb{A}(0)=0'\),
	其中\(0\)和\(0'\)分别是\(V\)和\(V'\)的零元.

	\item \((\forall\alpha\in V)[\vb{A}(-\alpha)=-\vb{A}(\alpha)]\).

	\item \(\vb{A}(k_1\alpha_1+\dotsb+k_s\alpha_s)
	=k_1\vb{A}(\alpha_1)+\dotsb+k_s\vb{A}(\alpha_s)\).

	\item 如果\(\AutoTuple{\alpha}{s}\)是\(V\)的一个线性相关的向量组,
	则\(\vb{A}(\alpha_1),\dotsc,\vb{A}(\alpha_s)\)是\(V'\)的一个线性相关的向量组;
	但是反之不成立(线性映射可以把线性无关向量组变为线性相关向量组).

	\item 如果\(V\)是有限维的,
	且\(\AutoTuple{\alpha}{s}\)是\(V\)的一个基,
	则对于\(V\)中任一向量\(\alpha=k_1\alpha_1+\dotsb+k_s\alpha_s\),
	有\[
		\vb{A}(\alpha)
		=k_1\vb{A}(\alpha_1)+\dotsb+k_s\vb{A}(\alpha_s).
	\]
	这表明,只要知道了\(V\)的一个基\(\AutoTuple{k}{s}\)在\(\vb{A}\)下的象,
	那么\(V\)中任一向量在\(\vb{A}\)下的象就都确定了.
	或者说,\(n\)维线性空间\(V\)到\(V'\)的线性映射完全被它在\(V\)的一个基上的作用所决定.
\end{itemize}
\end{property}

\subsection{线性映射的存在性}
给了域\(F\)上任意两个线性空间\(V\)和\(V'\),
是否存在\(V\)到\(V'\)的一个线性映射?
如果\(V\)是有限维的,
那么回答是肯定的,
我们有下述结论.
\begin{theorem}\label{theorem:线性映射.线性映射的存在性}
%@see: 《高等代数(第三版 下册)》(丘维声) P108 定理1
设\(V\)和\(V'\)都是域\(F\)上的线性空间,
\(V\)的维数是\(n\),
\(V\)中取一个基\(\AutoTuple{\alpha}{n}\),
\(V'\)中任意取定\(n\)个向量\(\AutoTuple{\gamma}{n}\),
令\[
	\vb{A}\colon V\to V',
	\alpha=\sum_{i=1}^n k_i\alpha_i
	\mapsto
	\sum_{i=1}^n k_i\gamma_i,
\]
则\(\vb{A}\)是\(V\)到\(V'\)的一个线性映射,
且\(\vb{A}(\alpha_i)=\gamma_i\ (i=1,2,\dotsc,n)\).
\begin{proof}
由于\(\AutoTuple{\alpha}{n}\)是\(V\)的一个基,
因此\(\alpha\)表示成\(\AutoTuple{\alpha}{n}\)的线性组合的方式唯一,
从而\(\vb{A}\)是从\(V\)到\(V'\)的一个映射.
在\(V\)中任取两个向量\[
	\alpha = \sum_{i=1}^n a_i \alpha_i,
	\qquad
	\beta = \sum_{i=1}^n b_i \alpha_i,
\]
则\begin{align*}
	\vb{A}(\alpha + \beta)
	&= \vb{A}\left( \sum_{i=1}^n (a_i + b_i) \alpha_i \right) \\
	&= \sum_{i=1}^n (a_i + b_i) \gamma_i \\
	&= \sum_{i=1}^n a_i \gamma_i
		+ \sum_{i=1}^n b_i \gamma_i \\
	&= \vb{A}(\alpha) + \vb{A}(\beta), \\
	\vb{A}(k \alpha)
	&= \vb{A}\left( \sum_{i=1}^n (k a_i) \alpha_i \right) \\
	&= \sum_{i=1}^n (k a_i) \gamma_i \\
	&= k \sum_{i=1}^n a_i \gamma_i \\
	&= k \vb{A}(\alpha),
	\quad k \in F.
\end{align*}
因此\(\vb{A}\)是从\(V\)到\(V'\)的一个线性映射.
显然有\[
	\vb{A}(\alpha_i)
	= \vb{A}(0\alpha_1 + \dotsb + 0\alpha_{i-1}
		+ 1\alpha_i + 0\alpha_{i+1} + \dotsb + 0\alpha_n)
	= \gamma_i,
\]
其中\(i=1,2,\dotsc,n\).
\end{proof}
\end{theorem}

由于\(V\)到\(V'\)的线性映射完全被它在\(V\)上的一个基上的作用所决定,
因此上述定理中满足\[
	\vb{A}(\alpha_i)=\gamma_i
	\quad(i=1,2,\dotsc,n)
\]的线性映射是唯一的.

\subsection{投影的概念}
\begin{definition}\label{definition:线性映射.平行于某个子空间在另一个子空间的投影}
%@see: 《高等代数(第三版 下册)》(丘维声) P108 定理2
设\(V\)是域\(F\)上的一个线性空间,
\(U,W\)是\(V\)的两个子空间,
且\(V=U \DirectSum W\).
把\[
	\vb{P}_U
	\defeq
	\Set{
		(\alpha,\alpha_1)
		\given
		\alpha \in V,
		\alpha_1 \in U,
		(\exists \alpha_2 \in W)
		[\alpha=\alpha_1+\alpha_2]
	}
\]
称为“\(V\)平行于\(W\)在\(U\)上的\DefineConcept{投影}”.
\end{definition}
\begin{remark}
\cref{definition:线性映射.平行于某个子空间在另一个子空间的投影}
强调“平行于\(W\)”
是因为从\cref{example:线性空间.子空间.直和.例1}
可以知道\(\alpha_1\)的取值是由\(U,W\)以及\(\alpha\)共同决定的.
\end{remark}
\begin{remark}
类似地,可以定义\[
	\vb{P}_W
	\defeq
	\Set{
		(\alpha,\alpha_2)
		\given
		\alpha \in V,
		\alpha_2 \in W,
		(\exists \alpha_1 \in U)
		[\alpha=\alpha_1+\alpha_2]
	},
\]
并称之为“\(V\)平行于\(U\)在\(W\)上的投影”.
\end{remark}

\begin{theorem}\label{theorem:线性映射.投影是线性变换}
%@see: 《高等代数(第三版 下册)》(丘维声) P108 定理2
设\(V\)是域\(F\)上的一个线性空间,
\(U,W\)是\(V\)的两个子空间,
且\(V=U \DirectSum W\),
则\(V\)平行于\(W\)在\(U\)上的投影
\(\vb{P}_U\)是\(V\)上的一个线性变换,
且满足\[
%@see: 《高等代数(第三版 下册)》(丘维声) P108 (7)
	\vb{P}_U(\alpha)
	=\left\{ \begin{array}{ll}
		\alpha, & \alpha\in U, \\
		0, & \alpha\in W.
	\end{array} \right.
\]
\begin{proof}
由于\(V = U \DirectSum W\),
因此\(V\)中任意一个向量\(\alpha\)表示成\(U\)的一个向量与\(W\)的一个向量之和的方式唯一,
关系\(\vb{P}_U\)是单值的,
于是\(\vb{P}_U\)是从\(V\)到\(V\)的一个映射.
任取\(V\)中两个向量\[
	\alpha = \alpha_1 + \alpha_2,
	\qquad
	\beta = \beta_1 + \beta_2,
\]
其中\(\alpha_1,\beta_1 \in U,
\alpha_2,\beta_2 \in W\),
则\[
	\alpha_1 + \beta_1
	\in U,
	\qquad
	\alpha_2 + \beta_2
	\in W,
\]
从而\begin{align*}
	\vb{P}_U(\alpha + \beta)
	&= \vb{P}_U((\alpha_1 + \beta_1) + (\alpha_2 + \beta_2)) \\
	&= \alpha_1 + \beta_1 \\
	&= \vb{P}_U(\alpha) + \vb{P}_U(\beta), \\
	\vb{P}_U(k \alpha)
	&= \vb{P}_U(k \alpha_1 + k \alpha_2) \\
	&= k \alpha_1 \\
	&= k \vb{P}_U(\alpha),
	\quad k \in F,
\end{align*}
因此\(\vb{P}_U\)是\(V\)上的一个线性变换.

如果\(\alpha \in U\),
则\(\alpha = \alpha + 0\),
从而\(\vb{P}_U(\alpha) = \alpha\).

如果\(\alpha \in W\),
则\(\alpha = 0 + \alpha\),
从而\(\vb{P}_U(\alpha) = 0\).

设\(V\)上的线性变换\(\vb{A}\)也满足投影的定义,
任取\(\alpha \in V\),
设\[
	\alpha = \alpha_1 + \alpha_2,
	\quad
	\alpha_1 \in U,
	\alpha_2 \in W,
\]
则\[
	\vb{A}(\alpha)
	= \vb{A}(\alpha_1 + \alpha_2)
	= \vb{A}(\alpha_1) + \vb{A}(\alpha_2)
	= \alpha_1 + 0
	= \alpha_1
	= \vb{P}_U(\alpha),
\]
因此\(\vb{A} = \vb{P}_U\).
\end{proof}
\end{theorem}

\cref{theorem:线性映射.投影是线性变换} 告诉我们,
如果线性空间\(V\)可以分解成两个子空间的直和\(V = U \DirectSum W\),
那么\(V\)在子空间\(U\)上的投影\(\vb{P}_U\)
和\(V\)在子空间\(W\)上的投影\(\vb{P}_W\)
就都是\(V\)上的线性变换,
并且有\begin{equation*}
	\vb{P}_U(\alpha)
	=\left\{ \begin{array}{ll}
		\alpha, & \alpha\in U, \\
		0, & \alpha\in W,
	\end{array} \right.
	\qquad
	\vb{P}_W(\alpha)
	=\left\{ \begin{array}{ll}
		\alpha, & \alpha\in W, \\
		0, & \alpha\in U.
	\end{array} \right.
\end{equation*}
投影是非常重要的一类线性变换.

为求简便,对于任意一个线性映射\(\vb{A}\),任意一个向量\(\alpha\),
以后我们都用\(\vb{A}\alpha\)代替\(\vb{A}(\alpha)\).

\subsection{幂等变换,正交变换}
\begin{definition}
%@see: 《高等代数(第三版 下册)》(丘维声) P109
线性变换\(\vb{A}\)如果满足\(\vb{A}^2=\vb{A}\),
则称“\(\vb{A}\)是\DefineConcept{幂等变换}”.
\end{definition}

\begin{definition}
%@see: 《高等代数(第三版 下册)》(丘维声) P109
两个线性变换\(\vb{A},\vb{B}\)
如果满足\(\vb{A} \vb{B}=\vb{B} \vb{A}=\vb0\),
则称“\(\vb{A}\)与\(\vb{B}\)是\DefineConcept{正交的}”.
\end{definition}

\subsection{线性映射的加法、纯量乘法}
\begin{definition}
%@see: 《高等代数(第三版 下册)》(丘维声) P110 命题5
设\(V,V'\)都是域\(F\)上的线性空间.
对于\(\forall\vb{A},\vb{B}\in\Hom(V,V'),
\forall\alpha\in V,
\forall k\in F\),
定义:\begin{gather*}
	%@see: 《高等代数(第三版 下册)》(丘维声) P110 (9)
	(\vb{A}+\vb{B})\alpha
	\defeq
	\vb{A}\alpha+\vb{B}\alpha, \\
	%@see: 《高等代数(第三版 下册)》(丘维声) P110 (10)
	(k\vb{A})\alpha
	\defeq
	k(\vb{A}\alpha),
\end{gather*}
把\(\vb{A}+\vb{B}\)称为“\(\vb{A}\)与\(\vb{B}\)的\DefineConcept{和}”,
把\(k\vb{A}\)称为“\(k\)与\(\vb{A}\)的\DefineConcept{纯量乘积}”.
\end{definition}

\begin{proposition}
%@see: 《高等代数(第三版 下册)》(丘维声) P110 命题5
设\(\vb{A},\vb{B}\)都是域\(F\)上线性空间\(V\)到\(V'\)的线性映射,
\(k\in F\),
则\begin{itemize}
	\item \(\vb{A}\)与\(\vb{B}\)的和\(\vb{A}+\vb{B}\)是\(V\)到\(V'\)的线性映射,
	\item \(k\)与\(\vb{A}\)的纯量乘积\(k\vb{A}\)是\(V\)到\(V'\)的线性映射.
\end{itemize}
\begin{proof}
显然\(\vb{A}+\vb{B}\)是从\(V\)到\(V'\)的一个映射.
对于任意\(\alpha,\beta \in V,
l \in F\),
有\begin{align*}
	(\vb{A}+\vb{B})(\alpha+\beta)
	&= \vb{A}(\alpha+\beta) + \vb{B}(\alpha+\beta) \\
	&= \vb{A}\alpha + \vb{A}\beta + \vb{B}\alpha + \vb{B}\beta \\
	&= (\vb{A}+\vb{B})\alpha + (\vb{A}+\vb{B})\beta, \\
	(\vb{A}+\vb{B})(l\alpha)
	&= \vb{A}(l\alpha) + \vb{B}(l\alpha) \\
	&= l\vb{A}\alpha + l\vb{B}\alpha \\
	&= l(\vb{A}\alpha + \vb{B}\alpha) \\
	&= l(\vb{A}+\vb{B}) \alpha,
\end{align*}
因此\(\vb{A}+\vb{B}\)是从\(V\)到\(V'\)的线性映射.

同理可证\(k\vb{A}\)是从\(V\)到\(V'\)的线性映射.
\end{proof}
\end{proposition}

容易验证,线性映射的加法与纯量乘法满足
线性空间定义的8条运算法则,
因此域\(F\)上的线性空间\(V\)到\(V'\)的所有线性映射组成的集合
成为域\(F\)上的一个线性空间,
% 称为“域\(F\)上从线性空间\(V\)到\(V'\)的\DefineConcept{线性映射空间}”,
记作\(\Hom(V,V')\).

特别地,域\(F\)上线性空间\(V\)上的所有线性变换组成的集合
成为域\(F\)上的一个线性空间,
% 称为“域\(F\)上线性空间\(V\)上的\DefineConcept{线性变换空间}”,
记作\(\Hom(V,V)\)\footnote{
	% 《高等代数(第四版)》(谢启鸿 姚慕生)
	在有的书上,\(\Hom(V,V')\)和\(\Hom(V,V)\)
	分别记作\(\mathcal{L}(V,V')\)和\(\mathcal{L}(V)\).
}.

\subsection{线性映射的乘法}
\begin{definition}
设\(V,U,W\)都是域\(F\)上的线性空间.
对于\(\forall\vb{A}\in\Hom(V,U),
\forall\vb{B}\in\Hom(U,W),
\forall\alpha \in V\),
定义:\begin{equation*}
	(\vb{B}\vb{A})\alpha
	\defeq (\vb{B} \circ \vb{A})\alpha
	= \vb{B}(\vb{A}\alpha),
\end{equation*}
把\(\vb{B}\vb{A}\)称为“\(\vb{B}\)与\(\vb{A}\)的\DefineConcept{积}”.
\end{definition}

\begin{proposition}
%@see: 《高等代数(第三版 下册)》(丘维声) P109 命题3
设\(V,U,W\)都是域\(F\)上的线性空间,
\(\vb{A}\)是\(V\)到\(U\)的一个线性映射,
\(\vb{B}\)是\(U\)到\(W\)的一个线性映射,
则\(\vb{B}\vb{A}\)是\(V\)到\(W\)的一个线性映射.
\begin{proof}
显然\(\vb{B}\vb{A}\)是从\(V\)到\(W\)的一个映射.
任取\(\alpha,\beta \in V,
k \in F\),
有\begin{align*}
	(\vb{B}\vb{A})(\alpha+\beta)
	&= \vb{B}(\vb{A}(\alpha+\beta)) \\
	&= \vb{B}(\vb{A}\alpha+\vb{A}\beta), \\
	(\vb{B}\vb{A})(k\alpha)
	&= \vb{B}(\vb{A}(k \alpha)) \\
	&= \vb{B}(k \vb{A}\alpha) \\
	&= k(\vb{B}(\vb{A}\alpha)) \\
	&= k((\vb{B}\vb{A}) \alpha),
\end{align*}
因此\(\vb{B}\vb{A}\)是从\(V\)到\(W\)的一个线性映射.
\end{proof}
\end{proposition}

\begin{proposition}\label{theorem:线性映射.线性映射的乘法适合结合律但不适合交换律}
%@see: 《高等代数(第三版 下册)》(丘维声) P110
线性映射的乘法适合结合律,不适合交换律.
%TODO proof
% \begin{proof}
% 线性映射是映射.
% 由\cref{example:映射.映射的复合适合结合律} 可知,
% 映射的乘法适合结合律,不适合交换律.
% \end{proof}
\end{proposition}
\begin{definition}
设\(V\)是域\(F\)上的线性空间,
\(\vb{A},\vb{B}\)都是\(V\)上的线性变换.
如果\(\vb{A}\vb{B}=\vb{B}\vb{A}\),
则称“\(\vb{A}\)与\(\vb{B}\) \DefineConcept{可交换}”.
\end{definition}

由\cref{theorem:线性映射.线性映射的乘法适合结合律但不适合交换律} 可知,
线性变换的乘法满足结合律:\begin{equation*}
	(\forall\vb{A},\vb{B},\vb{C}\in\Hom(V,V))
	[
		(\vb{A}\vb{B})\vb{C}
		= \vb{A}(\vb{B}\vb{C})
	].
\end{equation*}
恒等变换\(\vb{I}\)满足\begin{equation*}
	(\forall\vb{A}\in\Hom(V,V))
	[\vb{I}\vb{A}=\vb{A}\vb{I}=\vb{A}].
\end{equation*}
容易验证,线性变换的乘法对于加法还满足左、右分配律:\begin{gather*}
	(\forall\vb{A},\vb{B},\vb{C}\in\Hom(V,V))
	[\vb{A}(\vb{B}+\vb{C})=\vb{A}\vb{B}+\vb{A}\vb{C}], \\
	(\forall\vb{A},\vb{B},\vb{C}\in\Hom(V,V))
	[(\vb{B}+\vb{C})\vb{A}=\vb{B}\vb{A}+\vb{C}\vb{A}].
\end{gather*}
综上所述,\(\Hom(V,V)\)的加法与乘法满足环定义的6条运算法则,
并且\(\vb{I}\)是\(\Hom(V,V)\)的乘法单位元,
因此\(\Hom(V,V)\)对于加法和乘法成为一个有单位元的环.
容易验证,
线性变换的乘法与纯量乘法满足\begin{equation*}
%@see: 《高等代数(第三版 下册)》(丘维声) P111 (11)
	(\forall k\in F)
	(\forall\vb{A},\vb{B}\in\Hom(V,V))
	[
		k(\vb{A}\vb{B})
		=(k\vb{A})\vb{B}
		=\vb{A}(k\vb{B})
	].
\end{equation*}

\begin{definition}
%@see: 《Linear Algebra Done Right (Fourth Eidition) P82 3.59
%@see: 《Linear Algebra Done Right (Fourth Eidition) P82 3.61
%@see: 《线性代数》(李炯生 查建国) P184
%@see: 《线性代数》(李炯生 查建国) P196
%@see: 《高等代数学(第四版)》(谢启鸿 姚慕生 吴泉水) P184
给定\(\vb{A}\in\Hom(V,W)\).
如果存在\(\vb{B}\in\Hom(W,V)\),
使得\(\vb{B}\vb{A}=\vb{I}_V\)且\(\vb{A}\vb{B}=\vb{I}_W\),
其中\(\vb{I}_V\)是\(V\)上的恒等变换,
\(\vb{I}_W\)是\(W\)上的恒等变换,
则称“\(\vb{A}\) \DefineConcept{可逆}(invertible)”,
称“\(\vb{B}\)是\(\vb{A}\)一个的\DefineConcept{逆}(\(\vb{B}\) is the \emph{inverse} of \(\vb{A}\))”,
记作\(\vb{A}^{-1}\).
\end{definition}
\begin{proposition}\label{theorem:线性映射.可逆线性映射有唯一逆}
%@see: 《Linear Algebra Done Right (Fourth Eidition) P82 3.60
可逆线性映射有唯一逆.
\begin{proof}
假设\(\vb{B}_1,\vb{B}_2\in\Hom(W,V)\)都是可逆线性映射\(\vb{A}\in\Hom(V,W)\)的逆,
那么\begin{equation*}
	\vb{B}_1 = \vb{B}_1 \vb{I}
	= \vb{B}_1 (\vb{A} \vb{B}_2)
	= (\vb{B}_1 \vb{A}) \vb{B}_2
	= \vb{I} \vb{B}_2
	= \vb{B}_2.
	\qedhere
\end{equation*}
\end{proof}
\end{proposition}

\begin{proposition}\label{theorem:线性映射.可逆线性映射是同构}
设\(V,V'\)都是域\(F\)上的线性空间,
映射\(\sigma\colon V \to V'\),
则\[
	\text{$\sigma$是可逆线性映射}
	\iff
	\text{$\sigma$是同构}.
\]
\begin{proof}
假设\(\sigma\)是可逆线性映射,
则存在线性映射\(\rho\colon V' \to V\)使得\(\rho\sigma = \vb{I}_V\).
由\cref{example:映射.可逆映射是双射} 可知
\(\sigma\)是双射,
于是\(\sigma\)是同构.

假设\(\sigma\)是同构,
则\(\sigma\)是双射,且有\begin{gather*}
	(\forall\alpha,\beta \in V)
	[\sigma(\alpha+\beta)=\sigma(\alpha)+\sigma(\beta)],
	\qquad
	(\forall\alpha \in V)
	(\forall k \in F)
	[\sigma(k\alpha)=k\sigma(\alpha)],
\end{gather*}
这就说明\(\sigma\)是线性映射.
记\(\rho \defeq \sigma^{-1}\).
那么由\hyperref[theorem:集合论.关系及其逆是映射的充分必要条件]{逆映射存在的充分必要条件}可知
\(\rho\)是映射.
任取\(\xi,\eta \in V'\),
任取\(k \in F\),
由\(\sigma\)是双射可知\begin{equation*}
	(\exists!\alpha \in V)[\xi=\sigma(\alpha),\rho(\xi)=\alpha],
	\qquad
	(\exists!\beta \in V)[\eta=\sigma(\beta),\rho(\eta)=\beta]
\end{equation*}
则\begin{gather*}
	\rho(\xi+\eta)
	= \rho(\sigma(\alpha)+\sigma(\beta))
	= \rho(\sigma(\alpha+\beta))
	= \alpha+\beta
	= \rho(\xi)+\rho(\eta), \\
	\rho(k\xi)
	= \rho(\sigma(k\alpha))
	= k\alpha
	= k\rho(\xi),
\end{gather*}
这就说明\(\rho\)是线性映射.
因此\(\sigma\)是可逆线性映射.
\end{proof}
\end{proposition}

\begin{proposition}
%@see: 《高等代数(第三版 下册)》(丘维声) P110
设\(V,V'\)都是域\(F\)上有限维线性空间.
\(V\)到\(V'\)的可逆线性映射存在的充分必要条件是
\(\dim V=\dim V'\).
\begin{proof}
由\cref{theorem:线性映射.可逆线性映射是同构} 可知,
\(V\)到\(V'\)的可逆线性映射存在,
当且仅当\(V\)与\(V'\)同构.
由\cref{theorem:线性空间的同构.线性空间同构的充分必要条件} 可知,
\(V\)与\(V'\)同构的充分必要条件是它们的维数相同.
\end{proof}
\end{proposition}

\begin{proposition}
%@see: 《高等代数(第三版 下册)》(丘维声) P110 命题4
%@see: 《高等代数(第三版 下册)》(丘维声) P107 例6
\(\vb{A}\)是线性空间\(V\)到\(V'\)的一个线性映射,
如果\(\vb{A}\)可逆,
则\(\vb{A}^{-1}\)是\(V'\)到\(V\)的一个线性映射.
\begin{proof}
直接有\begin{align*}
	&\text{$\vb{A}$是从$V$到$V'$的可逆线性映射} \\
	&\iff \text{$\vb{A}$是从$V$到$V'$的同构} \\
	&\implies \text{$\vb{A}^{-1}$是从$V'$到$V$的同构} \\
	&\implies \text{$\vb{A}^{-1}$是从$V'$到$V$的可逆线性映射}.
	\qedhere
\end{align*}
\end{proof}
%@see: https://mathworld.wolfram.com/InvertibleLinearMap.html
\end{proposition}

由于线性变换的乘法满足结合律,
因此可以定义线性变换\(\vb{A}\)的正整数指数幂:\begin{equation*}
%@see: 《高等代数(第三版 下册)》(丘维声) P111 (13)
	\vb{A}^m
	\defeq
	\underbrace{\vb{A}\cdot\vb{A}\cdot\dotsm\cdot\vb{A}}_{\text{$m$个}}.
\end{equation*}
还可以定义\(\vb{A}\)的零次幂:\begin{equation*}
%@see: 《高等代数(第三版 下册)》(丘维声) P111 (14)
	\vb{A}^0
	\defeq
	\vb{I}.
\end{equation*}
容易验证:
对于\(\forall m,n\in\mathbb{N}\),
有\begin{gather*}
%@see: 《高等代数(第三版 下册)》(丘维声) P111 (15)
	\vb{A}^m\cdot\vb{A}^n=\vb{A}^{m+n}, \\
	(\vb{A}^m)^n=\vb{A}^{mn}.
\end{gather*}
当\(\vb{A}\)可逆时,可以定义:\begin{equation*}
%@see: 《高等代数(第三版 下册)》(丘维声) P111 (16)
	\vb{A}^{-m}
	\defeq
	(\vb{A}^{-1})^m,
	\quad m\in\mathbb{N}.
\end{equation*}

设\(f(x)=a_0+a_1 x+\dotsb+a_m x^m\)是域\(F\)上的一元多项式,
\(x\)用\(V\)上的线性变换\(\vb{A}\)代入,
得\begin{equation*}
%@see: 《高等代数(第三版 下册)》(丘维声) P111 (17)
	f(\vb{A})=a_0\vb{I}+a_1\vb{A}+\dotsb+a_m\vb{A}^m.
\end{equation*}
显然,\(f(\vb{A})\)仍是\(V\)上的一个线性变换,
称“\(f(\vb{A})\)是\(\vb{A}\)的一个多项式”.
容易验证:线性变换\(\vb{A}\)的任意两个多项式
\(f(\vb{A})\)与\(g(\vb{B})\)是可交换的,
即\begin{equation*}
%@see: 《高等代数(第三版 下册)》(丘维声) P111 (18)
	f(\vb{A}) g(\vb{A})
	=g(\vb{A}) f(\vb{A}).
\end{equation*}
把\(V\)上线性变换\(\vb{A}\)的所有多项式组成的集合记作\(F[\vb{A}]\).
容易验证:
\begin{itemize}
	\item \(F[\vb{A}]\)对于线性变换的减法、乘法都封闭,
	从而\(F[\vb{A}]\)是环\(\Hom(V,V)\)的一个子环;

	\item \(F[\vb{A}]\)是交换环;

	\item \(\vb{I}\in F[\vb{A}]\).
\end{itemize}
\(F[\vb{A}]\)中所有数乘变换组成的集合是\(F[\vb{A}]\)的一个子环,
并且域\(F\)与这个子环同构,
从而\(F[\vb{A}]\)可看成是\(F\)的一个扩环.
于是根据一元多项式环\(F[x]\)的通用性质,
\(x\)可用\(F[\vb{A}]\)的任一元素代入,
从\(F[x]\)的有关加法和乘法的等式
得到\(F[\vb{A}]\)中有关加法和乘法的相应等式.

\subsection{代数,代数的维数}
\begin{definition}
%@see: 《高等代数(第三版 下册)》(丘维声) P111 定义2
%@see: 《高等代数学(第四版)》(谢启鸿 姚慕生 吴泉水) P189 定义4.2.2
设\(A\)是域\(F\)上的线性空间,
\(A\)对加法和乘法成为一个有单位元的环,
且\[
	% 乘法与数乘的相容性
	(\forall k\in F)
	(\forall\alpha,\beta\in A)
	[
		k(\alpha\beta)
		=(k\alpha)\beta
		=\alpha(k\beta)
	],
\]
则称“\(A\)是域\(F\)上的一个\DefineConcept{代数}”,
把\(A\)的乘法单位元称为“\(A\)的\DefineConcept{恒等元}”,
把\(A\)的维数\(\dim A\)称为“代数\(A\)的\DefineConcept{维数}”.
\end{definition}
\begin{remark}
我们需要注意区分两个概念:代数\(A\)的乘法单位元,域\(F\)的乘法单位元.
\end{remark}

\begin{example}
域\(F\)上线性空间\(V\)上的所有线性变换组成的集合\(\Hom(V,V)\),
对于线性变换的加法、乘法与纯量乘法,
成为域\(F\)上的一个代数.
\end{example}

\begin{example}
域\(F\)上所有\(n\)阶矩阵组成的集合\(M_n(F)\),
对于矩阵的加法、乘法与数量乘法,
成为域\(F\)上的一个代数.
\end{example}

\subsection{线性变换的性质}
利用线性变换的运算,
我们可以研究线性变换的性质.

\begin{proposition}
%@see: 《高等代数(第三版 下册)》(丘维声) P109
设\(V\)是域\(F\)上的一个线性空间,
\(U,W\)是\(V\)的两个子空间,
且\(V=U \DirectSum W\),
则\(V\)平行于\(W\)在\(U\)上的投影\(\vb{P}_U\)
和\(V\)平行于\(U\)在\(W\)上的投影\(\vb{P}_W\)
满足\begin{gather*}
	%@see: 《高等代数(第三版 下册)》(丘维声) P109 (8)
	\vb{P}_U^2
	=\vb{P}_U, \\
	\vb{P}_U \vb{P}_W
	=\vb0, \\
	\vb{P}_W \vb{P}_U
	=\vb0, \\
	\vb{P}_W^2
	=\vb{P}_W.
\end{gather*}
\begin{proof}
\(V\)在子空间\(U\)上的投影\(\vb{P}_U\)
和\(V\)在子空间\(W\)上的投影\(\vb{P}_W\)
都是从\(V\)到\(V\)的映射,
对于\(\forall\alpha_1\in U,
\forall\alpha_2\in W\),
记\(\alpha=\alpha_1+\alpha_2\),
有\begin{gather*}
	\vb{P}_U(\vb{P}_U(\alpha))
	=\vb{P}_U(\alpha_1)
	=\alpha_1
	=\vb{P}_U(\alpha), \\
	\vb{P}_U(\vb{P}_W(\alpha))
	=\vb{P}_U(\alpha_2)
	=0, \\
	\vb{P}_W(\vb{P}_U(\alpha))
	=\vb{P}_W(\alpha_1)
	=0, \\
	\vb{P}_W(\vb{P}_W(\alpha))
	=\vb{P}_W(\alpha_2)
	=\alpha_2
	=\vb{P}_W(\alpha).
	\qedhere
\end{gather*}
\end{proof}
\end{proposition}
\begin{remark}
%@see: 《高等代数(第三版 下册)》(丘维声) P109
这就说明:
% 前提
如果\(V = U \DirectSum W\),
% 结论1
那么\(\vb{P}_U,\vb{P}_W\)都是幂等变换,
% 结论2
而且\(\vb{P}_U\)与\(\vb{P}_W\)是正交的.
\end{remark}

\begin{proposition}
%@see: 《高等代数(第三版 下册)》(丘维声) P112
设\(V\)是域\(F\)上的一个线性空间,
\(U,W\)是\(V\)的两个子空间,
且\(V=U \DirectSum W\),
则\(V\)平行于\(W\)在\(U\)上的投影\(\vb{P}_U\)
和\(V\)平行于\(U\)在\(W\)上的投影\(\vb{P}_W\)
满足\[
%@see: 《高等代数(第三版 下册)》(丘维声) P111 (19)
	\vb{P}_U+\vb{P}_W=\vb{I}.
\]
\begin{proof}
对于\(\forall\alpha_1\in U,
\alpha_2\in W\),
令\(\alpha=\alpha_1+\alpha_2\),
则\[
	(\vb{P}_U+\vb{P}_W)\alpha
	=\vb{P}_U\alpha+\vb{P}_W\alpha
	=\alpha_1+\alpha_2
	=\vb{I}\alpha,
\]
因此\(\vb{P}_U+\vb{P}_W=\vb{I}\).
\end{proof}
\end{proposition}
\begin{remark}
这就说明:
如果\(V=U \DirectSum W\),
则投影\(\vb{P}_U\)与\(\vb{P}_W\)的和等于恒等变换\(\vb{I}\).
\end{remark}

\begin{example}
%@see: 《高等代数(第三版 下册)》(丘维声) P113 习题9.1 8.
设\(\AutoTuple{\alpha}{n}\)是线性空间\(V\)的一个基,
\(\vb{A}\)是\(V\)上的一个线性变换.
证明:\(\vb{A}\)可逆,
当且仅当\(\AutoTuple{\vb{A}\alpha}{n}\)是\(V\)的一个基.
%TODO proof
\end{example}

\begin{example}
%@see: 《高等代数(第三版 下册)》(丘维声) P113 习题9.1 9.
设\(\vb{A}\)是线性空间\(V\)上的一个线性变换,向量\(\alpha \in V\).
证明:如果存在正整数\(m\)使得\[
	\vb{A}^{m-1} \alpha \neq 0,
	\qquad
	\vb{A}^m \alpha = 0,
\]
则\(\alpha,\vb{A}\alpha,\vb{A}^2\alpha,\dotsc,\vb{A}^{m-1}\alpha\)线性无关.
%TODO proof
\end{example}

\begin{example}
%@see: 《高等代数(第三版 下册)》(丘维声) P113 习题9.1 10.
设\(\vb{A},\vb{B}\)是\(V\)上的线性变换,
且\(\vb{A}\vb{B}-\vb{B}\vb{A}=\vb{I}\).
证明:存在正整数\(k\),使得\begin{equation*}
	\vb{A}^k\vb{B}-\vb{B}\vb{A}^k = k\vb{A}^{k-1}.
\end{equation*}
%TODO proof
\end{example}

\begin{example}
%@see: 《高等代数(第三版 下册)》(丘维声) P113 习题9.1 11.
设\(V\)是域\(F\)上的一个线性空间,\(\FieldChar F \neq 2\),
\(\vb{A},\vb{B}\)是\(V\)上的幂等变换.
证明:\begin{itemize}
	\item \(\vb{A}+\vb{B}\)是幂等变换,当且仅当\(\vb{A}\vb{B}=\vb{B}\vb{A}=\vb0\);
	\item 如果\(\vb{A}\vb{B}=\vb{B}\vb{A}\),则\(\vb{A}+\vb{B}-\vb{A}\vb{B}\)也是幂等变换.
\end{itemize}
%TODO proof
\end{example}
