\section{线性映射的核与象}
\begin{definition}
%@see: 《高等代数(第三版 下册)》(丘维声) P113 定义1
设\(V\)和\(V'\)都是域\(F\)上的线性空间,
\(\vb{A}\)是\(V\)到\(V'\)的一个线性映射.
我们把\(V'\)中零向量\(0'\)在\(\vb{A}\)下的原象集
\(\Set{
	\alpha\in V
	\given
	\vb{A}\alpha=0'
}\)
称为“\(\vb{A}\)的\DefineConcept{核}(kernel)”,
记作\(\Ker\vb{A}\).
把映射\(\vb{A}\)的值域
\(\Set{
	\beta \in V'
	\given
	\beta = \vb{A}\alpha,
	\alpha \in V
}\)
称为“\(\vb{A}\)的\DefineConcept{象}(image)”,
记作\(\Im\vb{A}\)或\(\vb{A}V\).
\end{definition}
%“核”的概念在群同态中也有定义
%考虑零元是加法群的单位元,
%因此线性映射的核的定义只是群同态的特殊情况
\begin{remark}
线性映射的核有可能含有非零元.
例如,在几何空间\(V\)中,
\(V\)在\(xOy\)平面上的正投影\(\vb{P}_W\)
把\(z\)轴上的每一个向量\(\alpha=(0,0,z)\ (z\in\mathbb{R})\)
映射为零向量\(0=(0,0,0)\);
显然,当\(z\neq0\)时,\(\alpha\)就是一个非零元,
但它又是线性映射\(\vb{P}_W\)的核的一个元素.
\end{remark}

\begin{proposition}\label{theorem:线性映射.线性映射的核空间和像空间分别是定义域和陪域的子空间}
%@see: 《高等代数(第三版 下册)》(丘维声) P113 命题1
设\(\vb{A}\)是域\(F\)上线性空间\(V\)到\(V'\)的一个线性映射,
则\(\Ker\vb{A}\)是\(V\)的一个子空间,
\(\Im\vb{A}\)是\(V'\)的一个子空间.
\begin{proof}
由于\(\vb{A}(0)=0\),
因此\(0 \in \Ker\vb{A}\).
任取\(\alpha,\beta \in \Ker\vb{A},
k \in F\),
则有\begin{gather*}
	\vb{A}(\alpha+\beta)
	= \vb{A}\alpha + \vb{A}\beta
	= 0 + 0
	= 0, \\
	\vb{A}(k\alpha)
	= k \vb{A}\alpha
	= k0
	= 0,
\end{gather*}
因此\(\Ker\vb{A}\)是\(V\)的一个子空间.

显然\(\Im\vb{A}\)是\(V'\)的一个非空子集.
在\(\Im\vb{A}\)中任取两个元素\(\gamma_1,\gamma_2\),
则存在\(\alpha_1,\alpha_2 \in V\),
使得\begin{gather*}
	\gamma_1 = \vb{A}\alpha_1, \\
	\gamma_2 = \vb{A}\alpha_2,
\end{gather*}
从而\begin{gather*}
	\gamma_1 + \gamma_2
	= \vb{A}\alpha_1 + \vb{A}\alpha_2
	= \vb{A}(\alpha_1 + \alpha_2)
	\in \Im\vb{A}, \\
	k \gamma_1
	= k \vb{A}\alpha_1
	= \vb{A}(k \alpha_1)
	\in \Im\vb{A},
	\quad k \in F,
\end{gather*}
因此\(\Im\vb{A}\)是\(V'\)的一个子空间.
\end{proof}
\end{proposition}

\begin{proposition}
%@see: 《高等代数(第三版 下册)》(丘维声) P114 命题2
设\(\vb{A}\)是域\(F\)上线性空间\(V\)到\(V'\)的一个线性映射,
则\begin{gather*}
	\text{$\vb{A}$是单射}
	\iff
	\Ker\vb{A}=0, \\
	\text{$\vb{A}$是满射}
	\iff
	\Im\vb{A}=V'.
\end{gather*}
\begin{proof}
首先证明
\(\text{$\vb{A}$是单射}
\iff
\Ker\vb{A}=0\).
\begin{itemize}
	\item 设\(\vb{A}\)是单射.
	任取\(\alpha \in \Ker\vb{A}\),
	则\[
		% 第一个等号是由“核”的定义保证的
		\vb{A}\alpha = 0 = \vb{A}0,
	\]
	% 由于\(\vb{A}\)是单射
	从而有\(\alpha = 0\),
	因此\(\Ker\vb{A} = 0\).

	\item 设\(\Ker\vb{A} = 0\).
	设\(\alpha_1,\alpha_2 \in V\)满足\(\vb{A}\alpha_1 = \vb{A}\alpha_2\),
	则\[
		0 = \vb{A}\alpha_2 - \vb{A}\alpha_1 = \vb{A}(\alpha_2 - \alpha_1),
	\]
	从而\(\alpha_2 - \alpha_1 \in \Ker\vb{A}\).
	由\(\Ker\vb{A} = 0\)
	可知\(\alpha_2 - \alpha_1 = 0\),
	即\(\alpha_1 = \alpha_2\).
	这就说明\(\vb{A}\)是单射.
\end{itemize}

由满射的定义立即可得
\(\text{$\vb{A}$是满射}
\iff
\Im\vb{A}=V'\).
\end{proof}
\end{proposition}

\begin{definition}
设\(V\)和\(V'\)都是域\(F\)上的线性空间,
且\(V\)是有限维的,
\(\vb{A}\)是\(V\)到\(V'\)的一个线性映射.
我们把\(\vb{A}\)的核\(\Ker\vb{A}\)的维数\(\dim(\Ker\vb{A})\)
称为“\(\vb{A}\)的\DefineConcept{零度}(nullity)”,
%@see: https://mathworld.wolfram.com/Nullity.html
把\(\vb{A}\)的象\(\Im\vb{A}\)的维数\(\dim(\Im\vb{A})\)
称为“\(\vb{A}\)的\DefineConcept{秩}(rank)”.
\end{definition}

\begin{theorem}
%@see: 《高等代数(第三版 下册)》(丘维声) P114 定理3
设\(V\)和\(V'\)都是域\(F\)上的线性空间,
且\(V\)是有限维的,
\(\vb{A}\)是\(V\)到\(V'\)的一个线性映射,
则\(\Ker\vb{A}\)和\(\Im\vb{A}\)都是有限维的,
且\[
%@see: 《高等代数(第三版 下册)》(丘维声) P114 (2)
	\dim(\Ker\vb{A})
	+\dim(\Im\vb{A})
	=\dim V.
\]
\begin{proof}
因为\(V\)是有限维的,
%\cref{theorem:线性空间.子空间.有限维线性空间的子空间是有限维的}
%\cref{theorem:线性映射.线性映射的核空间和像空间分别是定义域和陪域的子空间}
所以它的子空间\(\Ker\vb{A}\)是有限维的.

设\(\dim(\Ker\vb{A}) = m,
\dim V = n\).
取\(\Ker\vb{A}\)的一个基\(\AutoTuple{\alpha}{m}\),
把它扩充成\(V\)的一个基\[
	\AutoTuple{\alpha}{m},
	\AutoTuple{\alpha}[m+1]{n}.
\]
由核的定义可知,
对于任意一个向量\(\alpha \in V\),
有\[
	% 核空间\(\Ker\vb{A}\)中的任意一个向量\(\alpha\)在线性映射\(\vb{A}\)下的像\(\vb{A}\alpha\)必定等于\(0\)
	\alpha \in \Ker\vb{A}
	\implies
	\vb{A}\alpha = 0,
\]
那么\[
	% 核空间\(\Ker\vb{A}\)的每一个基向量都是\(\Ker\vb{A}\)中的向量,自然它在线性映射\(\vb{A}\)下的像\(\vb{A}\alpha\)必定等于\(0\)
	\vb{A}\alpha_i = 0,
	\quad i=1,2,\dotsc,m,
\]
于是,
对于任意\(\AutoTuple{x}{m} \in F\),
有\[
	\vb{A} \sum_{i=1}^m x_i \alpha_i
	= \sum_{i=1}^m x_i \vb{A}\alpha_i
	= \sum_{i=1}^m x_i 0
	= 0.
\]
在\(\Im\vb{A}\)中任取一个向量\(\vb{A}\alpha\),
其中\(\alpha \in V\).
% \(\alpha\)作为\(V\)中的向量,自然可以由\(V\)的基\(\alpha_1,\dotsc,\alpha_n\)线性表出
设\(\alpha = \sum_{i=1}^n x_i \alpha_i\),
其中\(\AutoTuple{x}{m} \in F\),
则\[
%@see: 《高等代数(第三版 下册)》(丘维声) P114 (3)
	\vb{A}\alpha
	% 在\(\alpha = \sum_{i=1}^n x_i \alpha_i\)等号左右两边同时乘以\(\vb{A}\)便得
	= \sum_{i=1}^n x_i \vb{A} \alpha_i
	% 由上可知\(\sum_{i=1}^m x_i \vb{A} \alpha_i = 0\)
	= \sum_{i=m+1}^n x_i \vb{A} \alpha_i,
\]
因此\[
%@see: 《高等代数(第三版 下册)》(丘维声) P114 (4)
	\Im\vb{A} = \Span\{\AutoTuple{\vb{A}\alpha}[m+1]{n}\},
\]
从而\(\Im\vb{A}\)是有限维的.

接下来证明\(\AutoTuple{\vb{A}\alpha}[m+1]{n}\)线性无关.
令\[
	y_{m+1} \vb{A} \alpha_{m+1} + \dotsb + y_n \vb{A} \alpha_n = 0,
\]
则\[
	\vb{A} (y_{m+1} \alpha_{m+1} + \dotsb + y_n \alpha_n) = 0,
\]
于是\[
	\beta
	\defeq
	y_{m+1} \alpha_{m+1} + \dotsb + y_n \alpha_n \in \Ker\vb{A},
\]
% 既然\(\beta\)是核空间\(\Ker\vb{A}\)中的向量,
% 那么\(\beta\)肯定可以由\(\Ker\vb{A}\)的基\(\AutoTuple{\alpha}{m}\)线性表出
所以\[
	y_{m+1} \alpha_{m+1} + \dotsb + y_n \alpha_n
	= z_1 \alpha_1 + \dotsb + z_m \alpha_m,
\]
即\[
	-z_1 \alpha_1 - \dotsb - z_m \alpha_m + y_{m+1} \alpha_{m+1} + \dotsb + y_n \alpha_n = 0,
\]
% 既然\(\AutoTuple{\alpha}{n}\)是\(V\)的一个基,
% 那么\(\AutoTuple{\alpha}{n}\)一定是线性无关的向量组,
% 于是上面这个关于\(z_1,\dotsc,z_m,y_{m+1},\dotsc,y_n\)的线性方程只有零解
从而有\[
	z_1 = \dotsb = z_m = y_{m+1} = \dotsb = y_n = 0.
\]
这就说明\(\AutoTuple{\vb{A}\alpha}[m+1]{n}\)线性无关,
于是它是\(\Im\vb{A}\)的一个基.
因此\[
	\dim(\Im\vb{A})
	= \card\{\AutoTuple{\vb{A}\alpha}[m+1]{n}\}
	= n - m
	= \dim V - \dim(\Ker\vb{A}),
\]
移项得\(\dim V = \dim(\Ker\vb{A}) + \dim(\Im\vb{A})\).
\end{proof}
\end{theorem}

\begin{corollary}
%@see: 《高等代数(第三版 下册)》(丘维声) P115
设\(V\)和\(V'\)都是域\(F\)上的线性空间,
且\(V\)是有限维的,
\(\vb{A}\)是\(V\)到\(V'\)的一个线性映射.
若\(\AutoTuple{\alpha}{n}\)是\(V\)的一个基,
则\[
	\Im\vb{A}=\opair{\vb{A}\alpha_1,\dotsc,\vb{A}\alpha_n}.
\]
\end{corollary}

\begin{corollary}
%@see: 《高等代数(第三版 下册)》(丘维声) P115 推论4
设\(V\)和\(V'\)都是域\(F\)上的\(n\)维线性空间,
\(\vb{A}\)是\(V\)到\(V'\)的一个线性映射,
则\[
	\text{$\vb{A}$是单射}
	\iff
	\text{$\vb{A}$是满射}.
\]
\end{corollary}

\begin{corollary}
%@see: 《高等代数(第三版 下册)》(丘维声) P115 推论5
设\(\vb{A}\)是域\(F\)上的有限维线性空间\(V\)上的线性变换,
则\[
	\text{$\vb{A}$是单射}
	\iff
	\text{$\vb{A}$是满射}.
\]
\end{corollary}

\begin{remark}
对于有限维线性空间\(V\)上的线性变换\(\vb{A}\),
虽然子空间\(\Ker\vb{A}\)与\(\Im\vb{A}\)的维数之和等于\(\dim V\),
但是\(\Ker\vb{A}+\Im\vb{A}\)并不一定是整个空间\(V\).
例如,在线性空间\(K[x]_n\)中,
导数\(\vb{D}\)的象为
\(\Im\vb{D}=K[x]_{n-1}\),
它的核为\(\Ker\vb{D}=K\).
显然\(K+K[x]_{n-1}\neq K[x]_n\).
\end{remark}
