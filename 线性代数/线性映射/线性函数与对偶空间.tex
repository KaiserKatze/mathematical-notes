\section{线性函数与对偶空间}
%TODO 对偶空间与傅里叶级数、傅里叶变换有密切的联系,详见以下视频:
%@see: https://www.bilibili.com/video/BV1xxivYzEh8/
%@see: https://mathvideos.org/2021/richard-borcherds-rings-and-modules/

本节着重讨论一种特殊的线性映射 --- \hyperref[definition:线性映射.线性函数]{线性函数}.
\subsection{线性函数}
除了\cref{example:线性映射.给定区间上的定积分是线性函数} 中举出的
给定区间上的定积分是线性函数以外,
下面我们额外举几个例子.

\begin{example}
%@see: 《高等代数(第三版 下册)》(丘维声) P160
矩阵的迹,
是\(M_n(F)\)上的一个线性函数,
它把域\(F\)上每一个\(n\)阶矩阵,
对应到\(F\)中的一个元素,
并且保持加法与数量乘法.
\end{example}

\begin{example}
%@see: 《高等代数(第三版 下册)》(丘维声) P161 例1
设\(F\)是一个域,\(\AutoTuple{a}{n} \in F\),
令\begin{equation*}
	f\colon F^n \to F,
	(\AutoTuple{x}{n}) \mapsto a_1 x_1 + \dotsb + a_n x_n,
\end{equation*}
容易验证\(f\)是\(F^n\)上的一个线性函数.
\end{example}

\begin{example}
%@see: 《高等代数(第三版 下册)》(丘维声) P161 例2
\def\Z{\mathbb{Z}_2}
设\(\Z\)是模\(2\)剩余类域.
令\begin{equation*}
	f\colon \Z^2 \to \Z,
	(x_1,x_2) \mapsto x_1^2+x_2^2.
\end{equation*}
试判断\(f\)是不是\(\Z^2\)上的一个线性函数.
\begin{solution}
任取\((x_1,x_2),(y_1,y_2)\in\Z^2\),
有\begin{align*}
	f((x_1,x_2)+(y_1,y_2))
	&= f(x_1+y_1,x_2+y_2) \\  % 剩余类域上线性空间的加法
	&= (x_1+y_1)^2+(x_2+y_2)^2 \\  % 题设
	&= x_1^2+y_1^2+x_2^2+y_2^2
		\tag{\cref{example:域.域上的特征恒等式}} \\
	&= f(x_1,x_2)+f(y_1,y_2), \\  % 题设
	f(1\cdot(x_1,x_2))
	&= f(x_1,x_2)  % 剩余类域上线性空间的纯量乘法
	= 1\cdot f(x_1,x_2), \\  % 剩余类域上的乘法
	f(0\cdot(x_1,x_2))
	&= f(0,0)  % 剩余类域上线性空间的纯量乘法
	= 0 = 0\cdot f(x_1,x_2).
\end{align*}
因此\(f\)是\(\Z^2\)上的一个线性函数.
\end{solution}
\end{example}

\begin{example}
%@see: 《高等代数(第三版 下册)》(丘维声) P165 习题9.10 1.(1)
试判断\begin{equation*}
	T\colon C[a,b] \to \mathbb{R},
	f \mapsto \int_a^b f^2(x) \dd{x}
\end{equation*}
是不是线性空间\(C[a,b]\)上的线性函数.
\begin{solution}
任取\(k\in\mathbb{R}\),
任取\(f\in C[a,b]\),
因为\begin{equation*}
	T(k f)
	= \int_a^b (k f(x))^2 \dd{x}
	= k^2 \int_a^b f^2(x) \dd{x}
	\neq k T(f),
\end{equation*}
所以\(T\)不是线性空间\(C[a,b]\)上的线性函数.
\end{solution}
\end{example}
\begin{example}
%@see: 《高等代数(第三版 下册)》(丘维声) P165 习题9.10 1.(2)
给定函数\(g \in C[a,b]\).
试判断\begin{equation*}
	T\colon C[a,b] \to \mathbb{R},
	f \mapsto \int_a^b f(x) g(x) \dd{x}
\end{equation*}
是不是线性空间\(C[a,b]\)上的线性函数.
\begin{solution}
任取\(f_1,f_2\in C[a,b]\),
任取\(k\in\mathbb{R}\),
因为\begin{gather*}
	T(f_1+f_2)
	= \int_a^b (f_1(x)+f_2(x)) g(x) \dd{x}
	= \int_a^b f_1(x) g(x) \dd{x} + \int_a^b f_2(x) g(x) \dd{x}
	= T(f_1) + T(f_2), \\
	T(k f_1)
	= \int_a^b (k f_1(x)) g(x) \dd{x}
	= k \int_a^b f_1(x) g(x) \dd{x}
	= k T(f_1),
\end{gather*}
所以\(T\)是线性空间\(C[a,b]\)上的线性函数.
\end{solution}
\end{example}

\begin{example}
%@see: 《高等代数(第三版 下册)》(丘维声) P165 习题9.10 2.
设\(V\)是域\(F\)上的一个3维线性空间,
\(\AutoTuple{\alpha}{3}\)是\(V\)的一个基,
\(f\)是\(V\)上的一个线性函数.
已知\begin{equation*}
	f(\alpha_1+2\alpha_3) = 4,
	\qquad
	f(\alpha_2+3\alpha_3) = 0,
	\qquad
	f(4\alpha_1+2\alpha_2) = 5,
\end{equation*}
求\(f(x_1 \alpha_1 + x_2 \alpha_2 + x_3 \alpha_3)\).
\begin{solution}
记\(a_1 \defeq f(\alpha_1),
a_2 \defeq f(\alpha_2),
a_3 \defeq f(\alpha_3)\),
那么有\begin{equation*}
	\left\{ \begin{array}{l}
		f(\alpha_1+2\alpha_3)
		= a_1+2a_3
		= 4, \\
		f(\alpha_2+3\alpha_3)
		= a_2+3a_3
		= 0, \\
		f(4\alpha_1+2\alpha_2)
		= 4a_1+2a_2
		= 5,
	\end{array} \right.
	\quad\text{或}\quad
	\begin{bmatrix}
		1 & 0 & 2 \\
		0 & 1 & 3 \\
		4 & 1 & 0
	\end{bmatrix}
	\begin{bmatrix}
		a_1 \\ a_2 \\ a_3
	\end{bmatrix}
	= \begin{bmatrix}
		4 \\ 0 \\ 5
	\end{bmatrix},
\end{equation*}
解得\(a_1 = 2,
a_2 = -3,
a_3 = 1\),
于是\begin{equation*}
	f(x_1 \alpha_1 + x_2 \alpha_2 + x_3 \alpha_3)
	= x_1 f(\alpha_1) + x_2 f(\alpha_2) + x_3 f(\alpha_3)
	= 2 x_1 - 3 x_2 + x_3.
\end{equation*}
\end{solution}
\end{example}

\begin{example}
%@see: 《高等代数(第三版 下册)》(丘维声) P165 习题9.10 3.
设\(V\)是域\(F\)上的一个3维线性空间,
\(\AutoTuple{\alpha}{3}\)是\(V\)的一个基.
试找出一个\(V\)上的线性函数\(f\),
使得\begin{equation*}
	f(3\alpha_1+\alpha_2) = 2,
	\qquad
	f(\alpha_2-\alpha_3) = 1,
	\qquad
	f(2\alpha_1+\alpha_3) = 2.
\end{equation*}
\begin{solution}
易知\(f\)需要满足
\(f(\alpha_1) = -1,
f(\alpha_2) = 5,
f(\alpha_3) = 4\),
于是\begin{equation*}
	f(\alpha) = -x_1 + 5x_2 + 4x_3,
	\quad \alpha = x_1 \alpha_1 + x_2 \alpha_2 + x_3 \alpha_3.
\end{equation*}
\end{solution}
\end{example}

\subsection{对偶空间}
\begin{definition}
%@see: 《高等代数(第三版 下册)》(丘维声) P162
%@see: 《Linear Algebra Done Right (Fourth Eidition)》(Sheldon Axler) P105 3.110
设\(V\)是域\(F\)上的一个线性空间,
把\(\Hom(V,F)\)称为“\(V\)上的\DefineConcept{线性函数空间}”
或“\(V\)的\DefineConcept{对偶空间}(dual space)”,
简记为\(V^*\).
\end{definition}

\begin{proposition}\label{theorem:对偶空间.对偶空间的维数}
%@see: 《高等代数(第三版 下册)》(丘维声) P162
%@see: 《Linear Algebra Done Right (Fourth Eidition)》(Sheldon Axler) P105 3.111
设\(V\)是域\(F\)上的\(n\)维线性空间,
\(V^*\)是\(V\)的对偶空间,
%@see: 《高等代数(第三版 下册)》(丘维声) P162 (4)
则\(\dim V^* = n\),
%@see: 《高等代数(第三版 下册)》(丘维声) P162 (5)
且\(V^* \Isomorphism V\).
\begin{proof}
因为\(\dim F = 1\),
所以由\cref{theorem:线性映射.线性映射空间与矩阵空间同构1} 可知
\(\dim V^*
= (\dim V)(\dim F)
= n \cdot 1
= n\).
再由\cref{theorem:线性空间的同构.线性空间同构的充分必要条件} 可知
\(V^* \Isomorphism V\).
\end{proof}
\end{proposition}
\begin{remark}
我们可以在\(V\)与\(V^*\)之间构造一个同构:
在\(V\)中取一个基\(\AutoTuple{\alpha}{n}\),
假设它的对偶基是\(\AutoTuple{\phi}{n}\),
那么映射\begin{equation}\label{equation:对偶空间.线性空间及其对偶空间之间的一个同构}
%@see: 《高等代数(第三版 下册)》(丘维声) P164 (15)
	\sigma\colon V \to V^*,
	\alpha = \sum_{i=1}^n x_i \alpha_i \mapsto \sum_{i=1}^n x_i \phi_i
\end{equation}
就是\(\sigma\)是从\(V\)到\(V^*\)的一个同构.
\end{remark}

\subsection{对偶基}
%\cref{theorem:线性映射.线性映射的存在性}
\begin{definition}
%@see: 《高等代数(第三版 下册)》(丘维声) P162
%@see: 《Linear Algebra Done Right (Fourth Eidition)》(Sheldon Axler) P106 3.112
设\(V\)是域\(F\)上\(n\)维线性空间,
\(\AutoTuple{\alpha}{n}\)是\(V\)的一个基.
定义映射:\begin{gather}
%@see: 《高等代数(第三版 下册)》(丘维声) P162 (6)
	\phi_i(\alpha_j)
	\defeq \left\{ \begin{array}{cl}
		1, & j = i, \\
		0, & j \neq i,
	\end{array} \right.
	\quad i=1,2,\dotsc,n,
		\label{equation:对偶空间.对偶基1} \\
%@see: 《高等代数(第三版 下册)》(丘维声) P162 (7)
	\phi_i(\alpha_j+\alpha_k)
	\defeq \phi_i(\alpha_j) + \phi_i(\alpha_k),
	\quad i=1,2,\dotsc,n,
		\label{equation:对偶空间.对偶基2} \\
	\phi_i(\lambda \alpha_j)
	\defeq \lambda \phi_i(\alpha_j),
	\quad i=1,2,\dotsc,n.
		\label{equation:对偶空间.对偶基3}
\end{gather}
把\(\AutoTuple{\phi}{n}\)
称为“\(\AutoTuple{\alpha}{n}\)的\DefineConcept{对偶基}%
(the \emph{dual basis} of \(\AutoTuple{\alpha}{n}\))”.
\end{definition}

\begin{proposition}\label{theorem:对偶空间.对偶基的性质}
%@see: 《高等代数(第三版 下册)》(丘维声) P162
%@see: 《Linear Algebra Done Right (Fourth Eidition)》(Sheldon Axler) P106 3.112
%@see: 《Linear Algebra Done Right (Fourth Eidition)》(Sheldon Axler) P106 3.114
%@see: 《Linear Algebra Done Right (Fourth Eidition)》(Sheldon Axler) P107 3.116
设\(V\)是域\(F\)上\(n\)维线性空间,
\(V^*\)是\(V\)的对偶空间,
\(\AutoTuple{\alpha}{n}\)是\(V\)的一个基,
\(\AutoTuple{\phi}{n}\)是\(\AutoTuple{\alpha}{n}\)的对偶基,
则\begin{itemize}
	\item \(\AutoTuple{\phi}{n}\)都是\(V\)上的线性函数,
	从而对于\(\forall \AutoTuple{x}{n} \in F\),
	有\begin{equation*}
	%@see: 《高等代数(第三版 下册)》(丘维声) P162 (7)
		\phi_i\left( \sum_{j=1}^n x_j \alpha_j \right)
		= \sum_{j=1}^n x_j \phi_i(\alpha_j)
		= x_i,
		\quad i=1,2,\dotsc,n;
	\end{equation*}

	\item \(\AutoTuple{\phi}{n}\)是\(V^*\)的一个基;

	\item \(V\)中任意一个向量\(\alpha\)
	在基\(\AutoTuple{\alpha}{n}\)下的坐标的第\(i\)个分量,
	等于\(\AutoTuple{\alpha}{n}\)的对偶基的第\(i\)个线性函数\(\phi_i\)
	在\(\alpha\)的值\(\phi_i(\alpha)\),
	即\begin{equation*}
	%@see: 《高等代数(第三版 下册)》(丘维声) P163 (9)
		(\forall \alpha \in V)
		[\alpha = \phi_1(\alpha) \alpha_1 + \dotsb + \phi_n(\alpha) \alpha_n];
	\end{equation*}

	\item \(V^*\)中任意一个线性函数\(\phi\)
	在基\(\AutoTuple{\phi}{n}\)下的坐标的第\(i\)个分量,
	等于\(\phi\)在\(\alpha_i\)的值\(\phi(\alpha_i)\),
	即\begin{equation*}
	%@see: 《高等代数(第三版 下册)》(丘维声) P163 (11)
		(\forall \phi \in V^*)
		[\phi = \phi(\alpha_1) \phi_1 + \dotsb + \phi(\alpha_n) \phi_n].
	\end{equation*}
\end{itemize}
\begin{proof}
由\cref{equation:对偶空间.对偶基2} 可知\(\phi_i\)适合可加性,
由\cref{equation:对偶空间.对偶基3} 可知\(\phi_i\)适合齐次性,
因此\(\phi_i\)是\(V\)上的线性函数.
再由\cref{theorem:线性映射.线性映射的性质} 可知,
对于\(\forall \AutoTuple{x}{n} \in F\),
有\begin{equation*}
%@see: 《高等代数(第三版 下册)》(丘维声) P162 (7)
	\phi_i\left( \sum_{j=1}^n x_j \alpha_j \right)
	= \sum_{j=1}^n x_j \phi_i(\alpha_j)
	= x_i,
	\quad i=1,2,\dotsc,n;
\end{equation*}

令\begin{equation*}
	x_1 \phi_1 + \dotsb + x_n \phi_n = 0,
\end{equation*}
则\begin{eqnarray}
	(x_1 \phi_1 + \dotsb + x_n \phi_n) \alpha_j = 0,
	\quad j=1,2,\dotsc,n.
\end{eqnarray}
根据线性映射的加法、纯量乘法的定义得\begin{equation*}
	x_1 \phi_1(\alpha_j) + \dotsb + x_j \phi_j(\alpha_j) + \dotsb + x_n \phi_n(\alpha_j) = 0,
	\quad j=1,2,\dotsc,n.
\end{equation*}
于是\(x_j = 0\ (j=1,2,\dotsc,n)\),
这就说明\(\AutoTuple{\phi}{n}\)线性无关.
再由\cref{theorem:对偶空间.对偶空间的维数,theorem:线性空间.线性相关性3} 可知
\(\AutoTuple{\phi}{n}\)是\(V^*\)的一个基.

对于任意\(\alpha \in V\),
% 由\cref{theorem:线性空间.任一向量可由给定基唯一线性表出} 可知,
存在\(\AutoTuple{k}{n} \in F\),
使得\begin{equation*}
	\alpha = \sum_{j=1}^n k_j \alpha_j.
\end{equation*}
那么有\begin{equation*}
	\phi_i(\alpha)
	= \phi_i\left( \sum_{j=1}^n k_j \alpha_j \right)
	= k_i,
\end{equation*}
于是\begin{equation*}
	\alpha = \sum_{j=1}^n \phi_j(\alpha) \alpha_j.
\end{equation*}

对于任意\(\phi \in V^*\),
存在\(\AutoTuple{k}{n} \in F\),
使得\begin{equation*}
	\phi = \sum_{j=1}^n k_j \phi_j.
\end{equation*}
那么有\begin{equation*}
	\phi(\alpha_i)
	= \left( \sum_{j=1}^n k_j \phi_j \right)(\alpha_i)
	= \sum_{j=1}^n k_j \phi_j(\alpha_i)
	= k_i,
\end{equation*}
于是\begin{equation*}
	\phi = \sum_{j=1}^n \phi(\alpha_j) \phi_j.
	\qedhere
\end{equation*}
\end{proof}
\end{proposition}

\subsection{不同对偶基之间的过渡矩阵}
\begin{theorem}
%@see: 《高等代数(第三版 下册)》(丘维声) P163 定理1
设\(V\)是域\(F\)上一个\(n\)维线性空间,
在\(V\)中取两个基\(\AutoTuple{\alpha}{n}\)与\(\AutoTuple{\beta}{n}\),
它们的对偶基分别为\(\AutoTuple{\phi}{n}\)与\(\AutoTuple{\psi}{n}\).
如果基\(\AutoTuple{\alpha}{n}\)到基\(\AutoTuple{\beta}{n}\)的过渡矩阵是\(A\),
%@see: 《高等代数(第三版 下册)》(丘维声) P163 (12)
则基\(\AutoTuple{\phi}{n}\)到基\(\AutoTuple{\psi}{n}\)的过渡矩阵为\((A^{-1})^T\).
\begin{proof}
由\((\AutoTuple{\beta}{n}) = (\AutoTuple{\alpha}{n}) A\)得
%@see: 《高等代数(第三版 下册)》(丘维声) P163 (13)
\((\AutoTuple{\alpha}{n}) = (\AutoTuple{\beta}{n}) A^{-1}\),
这就说明\(\alpha_j\)在基\(\AutoTuple{\beta}{n}\)下的坐标是\(A^{-1}\)的第\(j\)列,
从而坐标的第\(i\)个分量为\(\MatrixEntry{A^{-1}}{i,j}\).
由于\(\AutoTuple{\beta}{n}\)的对偶基是\(\AutoTuple{\psi}{n}\),
可知\(\alpha_j\)在基\(\AutoTuple{\beta}{n}\)下的坐标的第\(i\)个分量等于\(\psi_i(\alpha_j)\).
因此\(\MatrixEntry{A^{-1}}{i,j} = \psi_i(\alpha_j)\).

由于\(\AutoTuple{\alpha}{n}\)的对偶基是\(\AutoTuple{\phi}{n}\),
可知\(\psi_i(\alpha_j)\)等于\(\psi_i\)在\(\AutoTuple{\phi}{n}\)下的坐标的第\(j\)个分量.
%@see: 《高等代数(第三版 下册)》(丘维声) P163 (14)
由已知条件可知\((\AutoTuple{\psi}{n}) = (\AutoTuple{\phi}{n}) B\),
这就说明\(\psi_i\)在基\(\AutoTuple{\phi}{n}\)下的坐标的第\(j\)个分量等于\(\MatrixEntry{B}{j,i}\).
因此\(\psi_i(\alpha_j) = \MatrixEntry{B}{j,i}\).

综上所述,有\(\MatrixEntry{A^{-1}}{i,j}
= \psi_i(\alpha_j)
= \MatrixEntry{B}{j,i}
= \MatrixEntry{B^T}{i,j}\),
其中\(i,j=1,2,\dotsc,n\).
因此\(A^{-1} = B^T\),
换言之\(B = (A^{-1})^T\).
\end{proof}
\end{theorem}

\subsection{双重对偶空间}
给定域\(F\)上的一个\(n\)维线性空间\(V\),
我们知道\(V\)的对偶空间\(V^*\)也是域\(F\)上的一个\(n\)维线性空间,
那么我们可以考虑\(V^*\)的对偶空间\((V^*)^*\).

\begin{definition}
%@see: 《高等代数(第三版 下册)》(丘维声) P164
设\(V\)是域\(F\)上的一个线性空间,
\(V^*\)是\(V\)的对偶空间.
把\(V^*\)的对偶空间\((V^*)^*\)
称为“\(V\)的\DefineConcept{双重对偶空间}”,
简记为\(V^{**}\).
\end{definition}

\begin{corollary}
%@see: 《高等代数(第三版 下册)》(丘维声) P164
设\(V\)是域\(F\)上的一个\(n\)维线性空间,
\(V^{**}\)是\(V\)的双重对偶空间,
则\(V \Isomorphism V^{**}\).
\begin{proof}
由\cref{theorem:对偶空间.对偶空间的维数}
可知\(V \Isomorphism V^*,
V^* \Isomorphism V^{**}\),
利用传递性可得\(V \Isomorphism V^{**}\).
\end{proof}
\end{corollary}

给定线性空间\(V\)及其对偶空间\(V^*\)之间的同构\(\sigma\)
(见\cref{equation:对偶空间.线性空间及其对偶空间之间的一个同构}),
对于\(\forall \beta = \sum_{i=1}^n y_i \alpha_i \in V\),
有\begin{equation*}
%@see: 《高等代数(第三版 下册)》(丘维声) P164 (16)
	\sigma(\alpha)(\beta)
	= \left( \sum_{i=1}^n x_i \phi_i \right)(\beta)
	= \sum_{i=1}^n x_i \phi_i(\beta)
	= \sum_{i=1}^n x_i y_i.
\end{equation*}
上式表明,\(\alpha\)在\(\sigma\)下的像\(\sigma(\alpha)\)
在\(\beta\)的值\(\sigma(\alpha)(\beta)\)
等于\(\alpha\)与\(\beta\)的坐标的对应分量乘积之和.

假设\(\rho\)是\(V^*\)及其对偶空间\(V^{**} = \Hom(V^*,F)\)之间的一个同构,
记\(\alpha^{**} \defeq \rho(\sigma(\alpha))\),
那么对于\(V^*\)中任意一个线性函数\(\phi\),
由上述讨论可知,
\(\alpha^{**}(\phi)\)
等于\(\sigma(\alpha)\)与\(\phi\)在基\(\AutoTuple{\phi}{n}\)下的坐标的对应分量乘积之和,
具体地说,鉴于\begin{equation*}
%@see: 《高等代数(第三版 下册)》(丘维声) P164 (18)
	\sigma(\alpha)
	= \sum_{i=1}^n x_i \phi_i,  %\cref{equation:对偶空间.线性空间及其对偶空间之间的一个同构}
	\qquad
	\phi = \sum_{i=1}^n \phi(\alpha_i) \phi_i,  %\cref{theorem:对偶空间.对偶基的性质}
\end{equation*}
便有\begin{equation*}
%@see: 《高等代数(第三版 下册)》(丘维声) P164 (19)
	\alpha^{**}(\phi)
	= \sum_{i=1}^n x_i \phi(\alpha_i)
	= \phi\left( \sum_{i=1}^n x_i \alpha_i \right)
	= \phi(\alpha).
\end{equation*}

记\(\tau \defeq \rho \circ \sigma\),
显然\(\tau\)是从\(V\)到\(V^{**}\)的一个同构.
由\begin{equation*}
%@see: 《高等代数(第三版 下册)》(丘维声) P164 (20)
%@see: 《高等代数(第三版 下册)》(丘维声) P164 (21)
	(\forall \alpha \in V)
	(\forall \phi \in V^*)
	[
		\tau(\alpha)(\phi)
		= \phi(\alpha)
	]
\end{equation*}
可以看出,\(\alpha\)在\(\tau\)下的像\(\tau(\alpha) = \alpha^{**}\)不依赖于\(V\)中基的选择.

\begin{definition}
%@see: 《高等代数(第三版 下册)》(丘维声) P164
设\(V\)是域\(F\)上的一个线性空间,
\(V^*\)是\(V\)的对偶空间,
\(V^{**}\)是\(V\)的双重对偶空间.
如果同构\(\tau\colon V \to V^{**}\)
满足\begin{equation*}
	(\forall \alpha \in V)
	(\forall \phi \in V^*)
	[
		\tau(\alpha)(\phi)
		= \phi(\alpha)
	],
\end{equation*}
则称“\(\tau\)是从\(V\)到\(V^{**}\)的\DefineConcept{标准同构}”
或“\(\tau\)是从\(V\)到\(V^{**}\)的\DefineConcept{自然同构}”.
\end{definition}

由于\(V\)与\(V^{**}\)之间存在自然同构,
因此可以把\(V\)视同\(V^{**}\),
从而把\(V\)看成是\(V^*\)的对偶空间,
这样\(V\)与\(V^*\)就互为对偶空间.
这就是我们把\(V^*\)称为\(V\)的对偶空间的原因.

\subsection{对偶映射}
\begin{definition}
%@see: 《Linear Algebra Done Right (Fourth Eidition)》(Sheldon Axler) P107 3.118
设\(V,W\)都是域\(F\)上线性空间,
\(V^*,W^*\)分别是\(V,W\)的对偶空间,
\(T\)是从\(V\)到\(W\)的一个线性映射,
\(T^*\)是从\(W^*\)到\(V^*\)的映射.
如果对于\(\forall \phi \in W^*\)
都有\begin{equation*}
	T^*(\phi) = \phi \circ T,
\end{equation*}
则称“\(T^*\)是\(T\)的\DefineConcept{对偶映射}(the \emph{dual map} of \(T\))”.
\end{definition}

\begin{proposition}
%@see: 《Linear Algebra Done Right (Fourth Eidition)》(Sheldon Axler) P107
设\(V,W\)都是域\(F\)上线性空间,
\(V^*,W^*\)分别是\(V,W\)的对偶空间,
\(T\)是从\(V\)到\(W\)的一个线性映射,
\(T^*\)是\(T\)的对偶映射,
则\(T^*\)是从\(W^*\)到\(V^*\)的一个线性映射.
\begin{proof}
对于\(\forall \phi,\psi \in W^*\)和\(\forall \lambda \in F\),
有\begin{gather*}
	T^*(\phi+\psi)
	= (\phi+\psi) \circ T
	= \phi \circ T + \psi \circ T
	= T^*(\phi) + T^*(\psi), \\
	T^*(\lambda\phi)
	= (\lambda\phi) \circ T
	= \lambda (\phi \circ T)
	= \lambda T^*(\phi).
	\qedhere
\end{gather*}
\end{proof}
\end{proposition}

\begin{property}
%@see: 《Linear Algebra Done Right (Fourth Eidition)》(Sheldon Axler) P108 3.120
设\(V,W\)都是域\(F\)上线性空间,
\(T\)是从\(V\)到\(W\)的一个线性映射,
则\begin{itemize}
	\item 对于从\(V\)到\(W\)的任意一个线性映射\(S\),有\begin{equation*}
		(S+T)^* = S^* + T^*;
	\end{equation*}
	\item 对于任意\(\lambda \in F\),有\begin{equation*}
		(\lambda T)^* = \lambda T^*;
	\end{equation*}
	\item 对于从\(W\)到\(U\)的任意一个线性映射\(S\),有\begin{equation*}
		(S T)^* = T^* S^*.
	\end{equation*}
\end{itemize}
%TODO proof
\end{property}
