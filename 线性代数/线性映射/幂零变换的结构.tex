\section{幂零变换的结构}
在上一节的最后,我们指出,研究不可对角化的线性变换\(\vb{A}\)的结构的问题,
只要\(\vb{A}\)的最小多项式\(m(\lambda)\)在\(F[\lambda]\)中可分解成一次因式的乘积,
那么该问题就可以归结为研究幂零变换的结构的问题,
也就是研究域\(F\)上\(r\)维线性空间\(W\)上的幂零变换\(\vb{B}\),
能否在\(W\)中找到一个基,使得\(\vb{B}\)在此基下的矩阵最简单.

\begin{proposition}\label{theorem:线性映射.由幂零变换衍生的线性无关向量组}
%@see: 《高等代数(第三版 下册)》(丘维声) P148 命题1
设\(\vb{B}\)是域\(F\)上\(r\)维线性空间\(W\)上的幂零变换,且幂零指数为\(l\),
则存在\(\xi \in W\),
使得向量组\[
	\vb{B}^{l-1}\xi,
	\vb{B}^{l-2}\xi,
	\dotsc,
	\vb{B}^2\xi,
	\vb{B}\xi,
	\xi
\]线性无关,
从而\(l \leq \dim W\).
%TODO proof
\end{proposition}

\def\BasisL{\vb{B}^{l-1}\xi,\vb{B}^{l-2}\xi,\dotsc,\vb{B}^2\xi,\vb{B}\xi,\xi}
首先讨论幂零指数\(l = \dim W\)的情形.
此时根据\cref{theorem:线性映射.由幂零变换衍生的线性无关向量组} 得,
存在\(\xi \in W\),使得向量组\[
	\BasisL
\]线性无关.
由于\(l = \dim W\),
因此它们就是\(W\)的一个基.
根据\cref{theorem:线性映射.可以JC分解的线性变换在某个基下的矩阵是若尔当块} 得,
\(\vb{B}\)在\(W\)的基\[
	\BasisL
\]下的矩阵为\(J_l(0)\).

接着讨论幂零指数\(l < \dim W\)的情形.
此时,仍由\cref{theorem:线性映射.由幂零变换衍生的线性无关向量组} 得,
存在\(\xi \in W\),使得向量组\[
	\BasisL
\]线性无关.
由于\(l < \dim W\),上述向量组不是\(W\)的一个基.
注意到它们生成的子空间\[
	U \defeq \Span\{\BasisL\}
\]是\(\vb{B}\)的不变子空间,
并且\(\vb{B}\)在这个子空间\(U\)上的限制\(\vb{B} \SetRestrict U\)
在基\[
	\BasisL
\]下的矩阵是一个若尔当块\(J_l(0)\).
为此,我们抽象出一个概念:
\begin{definition}\label{definition:线性映射.生成强循环子空间}
%@see: 《高等代数(第三版 下册)》(丘维声) P148 定义1
设\(\vb{B}\)是域\(F\)上\(r\)维线性空间\(W\)上的线性变换,
向量\(\xi \in W\).
记\[
	U \defeq \Span\{\BasisL\}.
\]
如果存在正整数\(t\),
使得\[
	\vb{B}^{t-1} \xi \neq 0,
	\qquad
	\vb{B}^t \xi = 0,
\]
则称“子空间\(U\)是由\(\xi\)生成的 \DefineConcept{\(\vb{B}\) - 强循环子空间}”,
记作\(Z_t(\xi;\vb{B})\).
\end{definition}

显然,\(Z_t(\xi;\vb{B})\)是\(\vb{B}\)的不变子空间.
由\cref{example:线性映射.强循环向量组} 可知,
向量组\begin{equation*}
	\BasisL
\end{equation*}线性无关,
从而它们是\(Z_t(\xi;\vb{B})\)的一个基,
于是\(l\)是\(Z_t(\xi;\vb{B})\)的维数.
由上述讨论可知,
\(\vb{B}\)在\(Z_t(\xi;\vb{B})\)上的限制
\(\vb{B} \SetRestrict Z_t(\xi;\vb{B})\)
在这个基下的矩阵是一个\(l\)阶若尔当块\(J_l(0)\).
于是,只要我们能把\(W\)分解成若干个\(\vb{B}\) - 强循环子空间的直和,
那么在每个\(\vb{B}\) - 强循环子空间上选择上述这样的基,
合起来成为\(W\)的一个基,
\(\vb{B}\)在\(W\)的这个基下的矩阵
就是由若干个主对角元为\(0\)的若尔当块组成的分块对角矩阵,即若尔当形矩阵.
我们来探索这个问题.

对于任意\(\alpha \in W\)且\(\alpha\neq0\),
由于\(\vb{B}^l=\vb0\),
因此\(\vb{B}^l\alpha=0\).
设正整数\(t\)使得\(\vb{B}^{t-1}\alpha\neq0\),而\(\vb{B}^t\alpha=0\),
则有一个\(\vb{B}\) - 强循环子空间\(\Span\{\BasisL\}\).
由于\begin{equation*}
	\vb{B}(\vb{B}^{t-1}\alpha)
	= \vb{B}^t\alpha
	= 0
	= 0(\vb{B}^{t-1}\alpha),
\end{equation*}
因此\(\vb{B}^{t-1}\alpha\)是\(\vb{B}\)的属于特征值\(0\)的一个特征向量.
这表明任意一个\(\vb{B}\) - 强循环子空间\(\Span\{\BasisL\}\)的
第一个基向量\(\vb{B}^{t-1}\alpha\)是\(\vb{B}\)的属于特征值\(0\)的一个特征向量.

\(\vb{B}\)是幂零变换,
它的特征值只有\(0\).
我们把\(\vb{B}\)的属于特征值\(0\)的特征子空间记作\(W_0\).
对于任意\(\eta \in W_0\),
由于\(\vb{B}\eta = 0\eta = 0\),
因此\(\Span\{\eta\}\)是一个\(\vb{B}\) - 强循环子空间.
根据上面一段描述,
有可能\(\eta\)是某一个\(\vb{B}\) - 强循环子空间的第一个基向量\(\vb{B}^{t-1}\alpha\).
当\(t=1\)时,这个\(\vb{B}\) - 强循环子空间就是\(\Span\{\eta\}\).

假如\(W\)能分解成若干个\(\vb{B}\) - 强循环子空间的直和:\begin{equation*}
	W = \Span\{\vb{B}^{t_1-1}\alpha_1,\dotsc,\alpha_1\}
		\DirectSum \dotsb
		\DirectSum \Span\{\vb{B}^{t_s-1}\alpha_s,\dotsc,\alpha_s\}
		\DirectSum \Span\{\eta_1\}
		\DirectSum \dotsb
		\DirectSum \Span\{\eta_q\}.
\end{equation*}
于是向量组\begin{equation*}
	\vb{B}^{t_1-1}\alpha_1,
	\vb{B}^{t_2-1}\alpha_2,
	\dotsc,
	\vb{B}^{t_s-1}\alpha_s,
	\eta_1,
	\dotsc,
	\eta_q
\end{equation*}线性无关.
根据上面两段论述可知,它们都属于\(W_0\).
又由于对于任意\(\eta \in W_0\),
\(\eta\)都属于某一个\(\vb{B}\) - 强循环子空间,
因此我们猜测上述向量组是\(W_0\)的一个基.
从而猜测有下述结论:
\begin{theorem}
%@see: 《高等代数(第三版 下册)》(丘维声) P149 定理2
设\(\vb{B}\)是域\(F\)上\(r\)维线性空间\(W\)上的幂零变换,其幂零指数为\(l\),
把\(\vb{B}\)的属于特征值\(0\)的特征子空间记为\(W_0\),
则\(W\)可以分解成\(\dim W_0\)个\(\vb{B}\) - 强循环子空间的直和.
%TODO proof
\end{theorem}

有了上述定理,我们就可以了解幂零变换的结构了.
\begin{theorem}\label{theorem:幂零变换的结构.幂零变换的若尔当标准型}
%@see: 《高等代数(第三版 下册)》(丘维声) P151 定理3
设\(\vb{B}\)是域\(F\)上\(r\)维线性空间\(W\)上的幂零变换,其幂零指数为\(l\),
则\(W\)中存在一个基,使得\(\vb{B}\)在这个基下的矩阵\(B\)为若尔当形矩阵,
其中每个若尔当块的主对角元都是\(0\),且阶数不超过\(l\),
若尔当块的总数等于\(r - \rank\vb{B}\),
\(t\)阶若尔当块的个数为\begin{equation}
%@see: 《高等代数(第三版 下册)》(丘维声) P151 (1)
	N(t) = \rank\vb{B}^{t+1} + \rank\vb{B}^{t-1} - 2 \rank\vb{B}^t.
\end{equation}
\rm
这个若尔当形矩阵\(B\)称为幂零变换\(\vb{B}\)的\DefineConcept{若尔当标准型}.
除去若尔当块的排序次序外,\(\vb{B}\)的若尔当标准型是唯一的.
%TODO proof
\end{theorem}

\begin{corollary}
%@see: 《高等代数(第三版 下册)》(丘维声) P152 推论4
设\(B\)是域\(F\)上的\(r\)阶幂零矩阵,其幂零指数为\(l\),
则\(B\)相似于一个若尔当形矩阵,
其中每个若尔当块的主对角元都是\(0\),且阶数不超过\(l\),
若尔当块的总数等于\(r - \rank B\),
\(t\)阶若尔当块的个数为\begin{equation}
%@see: 《高等代数(第三版 下册)》(丘维声) P152 (10)
	N(t) = \rank B^{t+1} + \rank B^{t-1} - 2 \rank B^t.
\end{equation}
\end{corollary}
