\section{哈密顿--凯莱定理}
设数域\(K\)上的2阶矩阵\(A\)为\begin{equation*}
	A = \begin{bmatrix}
		1 & 2 \\
		0 & -1
	\end{bmatrix},
\end{equation*}
则\begin{equation*}
	A^2
	= \begin{bmatrix}
		1 & 0 \\
		0 & 1
	\end{bmatrix}
	= I,
\end{equation*}
可见\(A^2 - I = 0\),
因此\(f(\lambda) = \lambda^2 - 1\)就是\(A\)的一个零化多项式.
由于\begin{equation*}
	\abs{\lambda I - A}
	= \begin{vmatrix}
		\lambda-1 & -2 \\
		0 & \lambda+1
	\end{vmatrix}
	= (\lambda-1)(\lambda+1)
	= \lambda^2-1,
\end{equation*}
因此\(A\)的特征多项式\(f(\lambda) = \lambda^2-1\)就是\(A\)的一个零化多项式.
我们不禁好奇,是不是域\(F\)上的任意一个\(n\)阶矩阵的特征多项式都是它的零化多项式.
为了讨论这个问题,我们需要首先把域上的矩阵的概念推广维整环上的矩阵.

\subsection{哈密顿--凯莱定理}
有了上述准备知识以后,
我们就可以着手证明下面的哈密顿--凯莱定理了.
\begin{theorem}\label{theorem:线性映射.哈密顿--凯莱定理1}
%@see: 《高等代数(第三版 下册)》(丘维声) P138 Hamilton-Cayley定理
设\(A\)是域\(F\)上的\(n\)阶矩阵,
则\(A\)的特征多项式\(f(\lambda)\)是\(A\)的一个零化多项式.
%TODO proof
\end{theorem}

\begin{corollary}\label{theorem:线性映射.哈密顿--凯莱定理2}
%@see: 《高等代数(第三版 下册)》(丘维声) P139 Hamilton-Cayley定理
设\(\vb{A}\)是域\(F\)上\(n\)维线性空间\(V\)上的一个线性变换,
则\(\vb{A}\)的特征多项式\(f(\lambda)\)是\(\vb{A}\)的一个零化多项式.
\end{corollary}

\begin{example}
%@see: 《高等代数(第三版 下册)》(丘维声) P139 习题9.6 1.
设\(\vb{A}\)是域\(F\)上\(n\)维线性空间\(V\)上的线性变换.
证明:如果\(\vb{A}\)的特征多项式\(f(\lambda)\)在\(F[\lambda]\)中可以分解成\begin{equation*}
	f(\lambda) = (\lambda-\lambda_1)^{r_1} \dotsm (\lambda-\lambda_s)^{r_s},
\end{equation*}
则\begin{equation*}
	V = \Ker(\lambda_1 \vb{I} - \vb{A})^{r_1} \DirectSum \dotsb \DirectSum \Ker(\lambda_s \vb{I} - \vb{A})^{r_s}.
\end{equation*}
%TODO proof
\end{example}

\begin{example}
%@see: 《高等代数(第三版 下册)》(丘维声) P139 习题9.6 2.
设\(A\)是域\(F\)上\(n\)阶可逆矩阵,
\(I\)是域\(F\)上\(n\)阶单位矩阵.
证明:存在\(F\)中元素\(k_0,k_1,\dotsc,k_{n-1}\),
使得\begin{equation*}
	A^{-1}
	= k_{n-1} A^{n-1} + k_{n-2} A^{n-2}
	+ \dotsb + k_2 A^2 + k_1 A + k_0 I.
\end{equation*}
%TODO proof
\end{example}

\begin{example}
%@see: 《高等代数(第三版 下册)》(丘维声) P139 习题9.6 3.
设\(A\)是域\(F\)上的\(n\)阶矩阵,
\(B\)是域\(F\)上的\(m\)阶矩阵.
证明:矩阵方程\(AX-XB=0\)只有零解的充分必要条件是
\(A\)与\(B\)没有公共特征值.
%TODO proof
\end{example}
