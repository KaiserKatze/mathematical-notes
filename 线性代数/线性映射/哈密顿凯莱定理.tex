\section{哈密顿--凯莱定理}
设数域\(K\)上的2阶矩阵\(A\)为\begin{equation*}
	A = \begin{bmatrix}
		1 & 2 \\
		0 & -1
	\end{bmatrix},
\end{equation*}
则\begin{equation*}
	A^2
	= \begin{bmatrix}
		1 & 0 \\
		0 & 1
	\end{bmatrix}
	= I,
\end{equation*}
可见\(A^2 - I = 0\),
因此\(f(\lambda) = \lambda^2 - 1\)就是\(A\)的一个零化多项式.
由于\begin{equation*}
	\abs{\lambda I - A}
	= \begin{vmatrix}
		\lambda-1 & -2 \\
		0 & \lambda+1
	\end{vmatrix}
	= (\lambda-1)(\lambda+1)
	= \lambda^2-1,
\end{equation*}
因此\(A\)的特征多项式\(f(\lambda) = \lambda^2-1\)就是\(A\)的一个零化多项式.
我们不禁好奇,是不是域\(F\)上的任意一个\(n\)阶矩阵的特征多项式都是它的零化多项式.
为了讨论这个问题,我们需要首先把域上的矩阵的概念推广维整环上的矩阵.

\subsection{\texorpdfstring{$\lambda$}{\textlambda} - 矩阵}
与域\(F\)上的矩阵类似,
我们也可以为
整环\(R\)上的矩阵
定义加法、纯量乘法、乘法,
而且这三种运算满足与域\(F\)上矩阵一样的运算法则.
类似地,我们也可以定义整环\(R\)上\(n\)阶矩阵的行列式.
而且我们在之前介绍的行列式的性质、行列式展开定理,
对于整环\(R\)上的\(n\)阶矩阵的行列式也成立.
对于整环\(R\)上的\(n\)阶矩阵\(A\),
有\begin{equation*}
	A A^* = A^* A = \abs{A} I,
\end{equation*}
其中\(A^*\)是\(A\)的伴随矩阵.

域\(F\)上的一元多项式环\(F[\lambda]\)是一个整环,
因此可以考虑环\(F[\lambda]\)上的\(n\)阶矩阵.
我们把环\(F[\lambda]\)上的\(n\)阶矩阵
称为 \DefineConcept{\(\lambda\) - 矩阵}.

\begin{example}
给定\(\lambda\) - 矩阵\begin{equation*}
%@see: 《高等代数(第三版 下册)》(丘维声) P138 (1)
	A(\lambda) = \begin{bmatrix}
		2 \lambda^3 + \lambda^2 + 1 & \lambda^2 - 3 \\
		\lambda^3 - 1 & 2 \lambda + 5
	\end{bmatrix},
\end{equation*}
我们可以按照整环上矩阵的加法、纯量乘法,
将它改写成\begin{align*}
%@see: 《高等代数(第三版 下册)》(丘维声) P138 (2)
	A(\lambda)
	&= \begin{bmatrix}
		2 \lambda^3 & 0 \\
		\lambda^3 & 0
	\end{bmatrix}
	+ \begin{bmatrix}
		\lambda^2 & \lambda^2 \\
		0 & 0
	\end{bmatrix}
	+ \begin{bmatrix}
		0 & 0 \\
		0 & 2 \lambda
	\end{bmatrix}
	+ \begin{bmatrix}
		1 & -3 \\
		-1 & 5
	\end{bmatrix} \\
	&= \lambda^3
	\begin{bmatrix}
		2 & 0 \\
		1 & 0
	\end{bmatrix}
	+ \lambda^2
	\begin{bmatrix}
		1 & 1 \\
		0 & 0
	\end{bmatrix}
	+ \lambda
	\begin{bmatrix}
		0 & 0 \\
		0 & 2
	\end{bmatrix}
	+ \begin{bmatrix}
		1 & -3 \\
		-1 & 5
	\end{bmatrix},
\end{align*}
其中\(\lambda^k\)的系数矩阵\begin{equation*}
	\begin{bmatrix}
		2 & 0 \\
		1 & 0
	\end{bmatrix},
	\qquad
	\begin{bmatrix}
		1 & 1 \\
		0 & 0
	\end{bmatrix},
	\qquad
	\begin{bmatrix}
		0 & 0 \\
		0 & 2
	\end{bmatrix},
	\qquad
	\begin{bmatrix}
		1 & -3 \\
		-1 & 5
	\end{bmatrix}
\end{equation*}都是域\(F\)上的矩阵.
\end{example}

假设我们把两个\(\lambda\) - 矩阵\(A(\lambda)\)和\(B(\lambda)\)
都展开成\begin{equation*}
	A(\lambda)
	= \sum_{i=0}^m \lambda^i \alpha_i,
	\qquad
	B(\lambda)
	= \sum_{i=0}^m \lambda^i \beta_i,
\end{equation*}
其中\(\alpha_i,\beta_i \in M_n(F)\),
那么根据
两个一元多项式相等的定义
以及两个\(\lambda\) - 矩阵相等的定义,
可以推出,
\(A(\lambda)\)与\(B(\lambda)\)相等,
当且仅当它们系数矩阵对应相等.

\subsection{哈密顿--凯莱定理}
有了上述准备知识以后,
我们就可以着手证明下面的哈密顿--凯莱定理了.
\begin{theorem}\label{theorem:线性映射.哈密顿--凯莱定理1}
%@see: 《高等代数(第三版 下册)》(丘维声) P138 Hamilton-Cayley定理
设\(A\)是域\(F\)上的\(n\)阶矩阵,
则\(A\)的特征多项式\(f(\lambda)\)是\(A\)的一个零化多项式.
%TODO proof
\end{theorem}

\begin{corollary}\label{theorem:线性映射.哈密顿--凯莱定理2}
%@see: 《高等代数(第三版 下册)》(丘维声) P139 Hamilton-Cayley定理
设\(\vb{A}\)是域\(F\)上\(n\)维线性空间\(V\)上的一个线性变换,
则\(\vb{A}\)的特征多项式\(f(\lambda)\)是\(\vb{A}\)的一个零化多项式.
\end{corollary}

\begin{example}
%@see: 《高等代数(第三版 下册)》(丘维声) P139 习题9.6 1.
设\(\vb{A}\)是域\(F\)上\(n\)维线性空间\(V\)上的线性变换.
证明:如果\(\vb{A}\)的特征多项式\(f(\lambda)\)在\(F[\lambda]\)中可以分解成\begin{equation*}
	f(\lambda) = (\lambda-\lambda_1)^{r_1} \dotsm (\lambda-\lambda_s)^{r_s},
\end{equation*}
则\begin{equation*}
	V = \Ker(\lambda_1 \vb{I} - \vb{A})^{r_1} \DirectSum \dotsb \DirectSum \Ker(\lambda_s \vb{I} - \vb{A})^{r_s}.
\end{equation*}
%TODO proof
\end{example}

\begin{example}
%@see: 《高等代数(第三版 下册)》(丘维声) P139 习题9.6 2.
设\(A\)是域\(F\)上\(n\)阶可逆矩阵,
\(I\)是域\(F\)上\(n\)阶单位矩阵.
证明:存在\(F\)中元素\(k_0,k_1,\dotsc,k_{n-1}\),
使得\begin{equation*}
	A^{-1}
	= k_{n-1} A^{n-1} + k_{n-2} A^{n-2}
	+ \dotsb + k_2 A^2 + k_1 A + k_0 I.
\end{equation*}
%TODO proof
\end{example}

\begin{example}
%@see: 《高等代数(第三版 下册)》(丘维声) P139 习题9.6 3.
设\(A\)是域\(F\)上的\(n\)阶矩阵,
\(B\)是域\(F\)上的\(m\)阶矩阵.
证明:矩阵方程\(AX-XB=0\)只有零解的充分必要条件是
\(A\)与\(B\)没有公共特征值.
%TODO proof
\end{example}
