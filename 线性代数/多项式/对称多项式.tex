\section{对称多项式}
观察下述三元多项式\(f(x_1,x_2,x_3)\)有什么特点?
\[
	f(x_1,x_2,x_3)
	=x_1^3+x_2^3+x_3^3
	+x_1^2x_2
	+x_1^2x_3
	+x_2^2x_3
	+x_1x_2^2
	+x_1x_3^2
	+x_2x_3^2.
\]
直观上看,
\(x_1,x_2,x_3\)在\(f(x_1,x_2,x_3)\)中的地位是对称的,
即同时有\(x_1^3,x_2^3,x_3^3\)这三项,
且同时有\(x_1^2x_2,
x_1^2x_3,
x_2^2x_3,
x_1x_2^2,
x_1x_3^2,
x_2x_3^2\)这六项.
由此受到启发,
我们来研究具有这种性质的\(n\)元多项式\(f(x_1,\dotsc,x_n)\):
若\(f(x_1,\dotsc,x_n)\)含有一项\(a x_1^{i_1} \dotsm x_n^{i_n}\),
则它也含有一项\(a x_{j_1}^{i_1} \dotsm x_{j_n}^{i_n}\),
其中\(j_1 \dotso j_n\)是任意一个\(n\)元排列.

于是我们抽象出下述概念.
\begin{definition}
%@see: 《高等代数(第三版 下册)》(丘维声) P57 定义1
设\(f(x_1,\dotsc,x_n)\)是数域\(K\)上的一个\(n\)元多项式.
如果对于任意一个\(n\)元排列\(j_1 \dotso j_n\)都有\[
	f(x_{j_1},\dotsc,x_{j_n})
	=f(x_1,\dotsc,x_n),
\]
则称“\(f(x_1,\dotsc,x_n)\)是数域\(K\)上的一个\(n\)元\DefineConcept{对称多项式}”.
\end{definition}

定义表明,
在数域\(K\)上的\(n\)元多项式环\(K[x_1,\dotsc,x_n]\)中,
对于\(f(x_1,\dotsc,x_n)\),
如果任给一个\(n\)元排列\(j_1 \dotso j_n\),
不定元\(x_1,\dotsc,x_n\)用\(x_{j_1},\dotsc,x_{j_n}\)代入,
都有\(f(x_{j_1},\dotsc,x_{j_n})=f(x_1,\dotsc,x_n)\),
那么\(n\)元多项式\(f(x_1,\dotsc,x_n)\)是一个对称多项式.

容易看出,零多项式和零次多项式都是对称多项式.

在\(K[x_1,\dotsc,x_n]\)中,
我们来构造含有项\(x_1\)且项数最少的对称多项式.
由定义可知,
\(x_1+\dotsb+x_n\)就是\(n\)元对称多项式,
把它记作\(\sigma_1(x_1,\dotsc,x_n)\),
即\[
	\sigma_1(x_1,\dotsc,x_n)
	=x_1+\dotsb+x_n.
\]
我们来构造含有项\(x_1x_2\)且项数最少的对称多项式.
令\begin{align*}
	\sigma_2(x_1,\dotsc,x_n)
	&=\begin{array}[t]{l}
		x_1x_2+x_1x_3+\dotsb+x_1x_n \\
		+x_2x_3+\dotsb+x_2x_n
		+\dotsb
		+x_{n-1}x_n
	\end{array} \\
	&=\sum_{1\leq i<j\leq n} x_i x_j,
\end{align*}
则\(\sigma_2(x_1,\dotsc,x_n)\)是\(n\)元对称多项式.
同理,对于\(\forall k\in\{2,\dotsc,n-1\}\),
我们来构造含有项\(x_1 \dotsm x_k\),
且项数最少得对称多项式.
令\[
	\sigma_k(x_1,\dotsc,x_n)
	=\sum_{1\leq j_1<\dotsb<j_k\leq n}
	x_{j_1} \dotsm x_{j_k},
\]
则\(\sigma_k(x_1,\dotsc,x_n)\)是\(n\)元对称多项式.
最后,根据定义有,\[
	\sigma_n(x_1,\dotsc,x_n)
	=x_1 \dotsm x_n
\]是\(n\)元对称多项式.

我们把上述\(n\)个\(n\)元对称多项式
\(\sigma_i(x_1,\dotsc,x_n)\ (i=1,\dotsc,n)\)
统称为\(n\)元\DefineConcept{初等对称多项式}.

下面我们把数域\(K\)上所有\(n\)元对称多项式组成的集合记为\(W\).
我们想要知道\(W\)的结构是怎样的.

\begin{proposition}
%@see: 《高等代数(第三版 下册)》(丘维声) P58 命题1
\(W\)是\(K[x_1,\dotsc,x_n]\)的一个子环.
\begin{proof}
显然\(W\)非空集.
任取\(f(x_1,\dotsc,x_n),g(x_1,\dotsc,x_n) \in W\),
设\begin{gather*}
	h(x_1,\dotsc,x_n)
	=f(x_1,\dotsc,x_n)
	-g(x_1,\dotsc,x_n), \\
	p(x_1,\dotsc,x_n)
	=f(x_1,\dotsc,x_n)
	g(x_1,\dotsc,x_n).
\end{gather*}
任给一个\(n\)元排列\(j_1 j_2 \dotso j_n\),
\(x_1,\dotsc,x_n\)用\(x_{j_1},\dotsc,x_{j_n}\)代入,
从以上两式分别得到\begin{align*}
	h(x_{j_1},\dotsc,x_{j_n})
	&=f(x_{j_1},\dotsc,x_{j_n})
	-g(x_{j_1},\dotsc,x_{j_n}) \\
	&=f(x_1,\dotsc,x_n)
	-g(x_1,\dotsc,x_n) \\
	&=h(x_1,\dotsc,x_n), \\
	p(x_{j_1},\dotsc,x_{j_n})
	&=f(x_{j_1},\dotsc,x_{j_n})
	g(x_{j_1},\dotsc,x_{j_n}) \\
	&=f(x_1,\dotsc,x_n)
	g(x_1,\dotsc,x_n) \\
	&=p(x_1,\dotsc,x_n).
\end{align*}
因此\(h(x_1,\dotsc,x_n),p(x_1,\dotsc,x_n) \in W\).
这就说明\(W\)是\(K[x_1,\dotsc,x_n]\)的一个子环.
\end{proof}
\end{proposition}

\begin{proposition}
%@see: 《高等代数(第三版 下册)》(丘维声) P59 命题2
设\(f_1,\dotsc,f_n \in W\),
则对\(K[x_1,\dotsc,x_n]\)中任意一个多项式\[
	g(x_1,\dotsc,x_n)
	=\sum_{i_1,\dotsc,i_n}
	b_{i_1 \dotso i_n}
	x_1^{i_1} \dotsm x_n^{i_n},
\]
有\[
	g(f_1,\dotsc,f_n)
	=\sum_{i_1,\dotsc,i_n}
	b_{i_1 \dotso i_n}
	f_1^{i_1} \dotsm f_n^{i_n}
	\in W.
\]
\end{proposition}

\begin{theorem}[对称多项式基本定理]
%@see: 《高等代数(第三版 下册)》(丘维声) P59 定理3
对于\(K[x_1,\dotsc,x_n]\)中任意一个对称多项式\(f(x_1,\dotsc,x_n)\),
都存在\(K[x_1,\dotsc,x_n]\)中唯一的一个多项式\(g(x_1,\dotsc,x_n)\),
使得\(f(x_1,\dotsc,x_n)=g(\sigma_1,\dotsc,\sigma_n)\).
\begin{proof}
存在性.
采取首项消去法.
设对称多项式\(f(x_1,\dotsc,x_n)\)的首项是\(a x_1^{l_1} \dotsm x_n^{l_n}\),
其中\(a\neq0\),且\(l_1 \geq \dotsb \geq l_n\).
为了消去\(f(x_1,\dotsc,x_n)\)的首项,
同时又要出现\(\sigma_1,\dotsc,\sigma_n\),
我们作多项式\[
	\phi_1(x_1,\dotsc,x_n)
	= a \sigma_1^{l_1-l_2} \sigma_2^{l_2-l_3}
	\dotsm \sigma_{n-1}^{l_{n-1}-l_n} \sigma_n^{l_n},
\]
因为\(K[x_1,\dotsc,x_n]\)中对称多项式的乘积还是对称多项式,
所以\(\phi_1(x_1,\dotsc,x_n)\)是对称多项式.
又由于多项式的乘积的首项等于它们的首项的乘积,
因此\(\phi_1(x_1,\dotsc,x_n)\)的首项是\begin{align*}
	&a x_1^{l_1-l_2} (x_1 x_2)^{l_2-l_3}
	\dotsm (x_1 x_2 \dotsm x_{n-1})^{l_{n-1}-l_n}
	(x_1 x_2 \dotsm x_{n-1} x_n)^{l_n} \\
	&= a x_1^{l_1} x_2^{l_2} \dotsm x_{n-1}^{l_{n-1}} x_n^{l_n},
\end{align*}
它等于\(f(x_1,\dotsc,x_n)\)的首项.
令\[
	f_1(x_1,\dotsc,x_n)
	=f(x_1,\dotsc,x_n)
	-\phi_1(x_1,\dotsc,x_n),
\]
则\(f\)的首项的幂指数组先于\(f_1\)的首项的幂指数组,
并且由于对称多项式的差仍是对称多项式,
所以\(f_1(x_1,\dotsc,x_n)\)是\(K[x_1,\dotsc,x_n]\)中的对称多项式.

对\(f_1(x_1,\dotsc,x_n)\)重复上述做法,
我们又得到\(K[x_1,\dotsc,x_n]\)中的一个对称多项式\[
	f_2(x_1,\dotsc,x_n)
	=f_1(x_1,\dotsc,x_n)
	-\phi_2(x_1,\dotsc,x_n),
\]
其中\(\phi_2(x_1,\dotsc,x_n)\)是\(\sigma_1,\dotsc,\sigma_n\)的适当方幂的乘积,
并且其系数等于\(f_1\)的首项系数,
\(f_1\)的首项的幂指数组先于\(f_2\)的首项的幂指数组.

如此继续下去,我们得到\(K[x_1,\dotsc,x_n]\)中一系列的对称多项式\[
	f,
	f_1=f-\phi_1,
	f_2=f_1-\phi_2,
	\dotsc,
	f_i=f_{i-1}-\phi_i,
	\dotsc,
\]
它们的首项的幂指数组一个比一个小,
其中\(\phi_i\)是\(\sigma_1,\dotsc,\sigma_n\)的适当方幂的乘积,
并且其系数等于\(f_{i-1}\)的首项系数.
设\(f_i\)的首项的幂指数组为\((p_1,\dotsc,p_n)\),
则\((l_1,\dotsc,l_n)>(p_1,\dotsc,p_n)\),
于是\(l_1 \geq p_1\).
又因为\(f_i\)是对称多项式,
所以\(p_1 \geq \dotsb \geq p_n\).
因此\(l_1 \geq p_1 \geq p_2 \geq \dotsb p_n\).
满足这个条件的非负整数组\((p_1,\dotsc,p_n)\)只有有限多个,
因此上述对称多项式序列中只能有有限多个\(f_i\)不为零,
换言之,存在正整数\(s\),使得\(f_s=0\).
于是\[
	f_1=f-\phi_1,
	f_2=f_1-\phi_2,
	\dotsc,
	f_{s-1}=f_{s-2}-\phi_{s-1},
	f_s=f_{s-1}-\phi_s=0,
\]
从而得到\[
	f=\phi_1+\dotsb+\phi_s.
\]

设\(\phi_i(x_1,\dotsc,x_n)
=a_i \sigma_1^{t_{i1}} \dotsm \sigma_n^{t_{in}}\),
令\[
	g(x_1,\dotsc,x_n)
	=\sum_{i=1}^s a_i x_1^{t_{i1}} \dotsm x_n^{t_{in}}
	\in K[x_1,\dotsc,x_n],
\]
则\begin{align*}
	g(\sigma_1,\dotsc,\sigma_n)
	&=\sum_{i=1}^s a_i \sigma_1^{t_{i1}} \dotsm \sigma_n^{t_{in}} \\
	&=\sum_{i=1}^s \phi_i(x_1,\dotsc,x_n) \\
	&=f(x_1,\dotsc,x_n).
\end{align*}
存在性成立.

唯一性.
如果\(K[x_1,\dotsc,x_n]\)中有两个不同的多项式
\(g_1(x_1,\dotsc,x_n)\)和\(g_2(x_1,\dotsc,x_n)\),
使得\begin{align*}
	f(x_1,\dotsc,x_n)
	&=g_1(\sigma_1,\dotsc,\sigma_n) \\
	&=g_2(\sigma_1,\dotsc,\sigma_n),
\end{align*}
则\(g_1(\sigma_1,\dotsc,\sigma_n)-g_2(\sigma_1,\dotsc,\sigma_n)=0\).
令\[
	g(x_1,\dotsc,x_n)
	=g_1(x_1,\dotsc,x_n)-g_2(x_1,\dotsc,x_n),
\]
则\(g(\sigma_1,\dotsc,\sigma_n)=0\).
由假设可知\(g(x_1,\dotsc,x_n)\neq0\),
于是由\cref{theorem:多项式.多元多项式环.引理1} 可知,
存在\(b_1,\dotsc,b_n \in K\),
使得\(g(b_1,\dotsc,b_n)\neq0\).
令\[
	\phi(x)=x^n-b_1x^{n-1}+\dotsb+(-1)^kb_kx^{n-k}+\dotsb+(-1)^nb_n,
\]
设\(\phi(x)\)的\(n\)个复根是\(\alpha_1,\dotsc,\alpha_n\),
则从韦达公式推出\[
	b_k=\sigma_k(\alpha_1,\dotsc,\alpha_n),
	\quad
	k=1,2,\dotsc,n.
\]
\(x_1,\dotsc,x_n\)用\(\alpha_1,\dotsc,\alpha_n\)代入,
于是\[
	g(\sigma_1(\alpha_1,\dotsc,\alpha_n),\dotsc,\sigma_n(\alpha_1,\dotsc,\alpha_n))=0,
\]
即\(g(b_1,\dotsc,b_n)=0\),矛盾!
唯一性成立.
\end{proof}
\end{theorem}
