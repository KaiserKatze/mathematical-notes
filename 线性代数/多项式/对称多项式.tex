\section{对称多项式}
\subsection{对称多项式}
观察下述三元多项式\(f(x_1,x_2,x_3)\)有什么特点?
\begin{equation*}
	f(x_1,x_2,x_3)
	=x_1^3+x_2^3+x_3^3
	+x_1^2x_2
	+x_1^2x_3
	+x_2^2x_3
	+x_1x_2^2
	+x_1x_3^2
	+x_2x_3^2.
\end{equation*}
直观上看,
\(x_1,x_2,x_3\)在\(f(x_1,x_2,x_3)\)中的地位是对称的,
即同时有\(x_1^3,x_2^3,x_3^3\)这三项,
且同时有\(x_1^2x_2,
x_1^2x_3,
x_2^2x_3,
x_1x_2^2,
x_1x_3^2,
x_2x_3^2\)这六项.
由此受到启发,
我们来研究具有这种性质的\(n\)元多项式\(f(x_1,\dotsc,x_n)\):
若\(f(x_1,\dotsc,x_n)\)含有一项\(a x_1^{i_1} \dotsm x_n^{i_n}\),
则它也含有一项\(a x_{j_1}^{i_1} \dotsm x_{j_n}^{i_n}\),
其中\(j_1 \dotso j_n\)是任意一个\(n\)元排列.

于是我们抽象出下述概念.
\begin{definition}
%@see: 《高等代数(第三版 下册)》(丘维声) P57 定义1
设\(f(x_1,\dotsc,x_n)\)是数域\(K\)上的一个\(n\)元多项式.
如果对于任意一个\(n\)元排列\(j_1 \dotso j_n\)都有\begin{equation*}
	f(x_{j_1},\dotsc,x_{j_n})
	=f(x_1,\dotsc,x_n),
\end{equation*}
则称“\(f(x_1,\dotsc,x_n)\)是数域\(K\)上的一个\(n\)元\DefineConcept{对称多项式}”.
\end{definition}

定义表明,
在数域\(K\)上的\(n\)元多项式环\(K[x_1,\dotsc,x_n]\)中,
对于\(f(x_1,\dotsc,x_n)\),
如果任给一个\(n\)元排列\(j_1 \dotso j_n\),
不定元\(x_1,\dotsc,x_n\)用\(x_{j_1},\dotsc,x_{j_n}\)代入,
都有\(f(x_{j_1},\dotsc,x_{j_n})=f(x_1,\dotsc,x_n)\),
那么\(n\)元多项式\(f(x_1,\dotsc,x_n)\)是一个对称多项式.

容易看出,零多项式和零次多项式都是对称多项式.

\subsection{初等对称多项式}
在\(K[x_1,\dotsc,x_n]\)中,
我们来构造含有项\(x_1\)且项数最少的对称多项式.
由定义可知,
\(x_1+\dotsb+x_n\)就是\(n\)元对称多项式,
把它记作\(\sigma_1(x_1,\dotsc,x_n)\),
即\begin{equation*}
	\sigma_1(x_1,\dotsc,x_n)
	=x_1+\dotsb+x_n.
\end{equation*}
我们来构造含有项\(x_1x_2\)且项数最少的对称多项式.
令\begin{align*}
	\sigma_2(x_1,\dotsc,x_n)
	&=\begin{array}[t]{l}
		x_1x_2+x_1x_3+\dotsb+x_1x_n \\
		+x_2x_3+\dotsb+x_2x_n
		+\dotsb
		+x_{n-1}x_n
	\end{array} \\
	&=\sum_{1\leq i<j\leq n} x_i x_j,
\end{align*}
则\(\sigma_2(x_1,\dotsc,x_n)\)是\(n\)元对称多项式.
同理,对于\(\forall k\in\{2,\dotsc,n-1\}\),
我们来构造含有项\(x_1 \dotsm x_k\),
且项数最少得对称多项式.
令\begin{equation*}
	\sigma_k(x_1,\dotsc,x_n)
	=\sum_{1\leq j_1<\dotsb<j_k\leq n}
	x_{j_1} \dotsm x_{j_k},
\end{equation*}
则\(\sigma_k(x_1,\dotsc,x_n)\)是\(n\)元对称多项式.
最后,根据定义有,\begin{equation*}
	\sigma_n(x_1,\dotsc,x_n)
	=x_1 \dotsm x_n
\end{equation*}是\(n\)元对称多项式.

我们把上述\(n\)个\(n\)元对称多项式
\(\sigma_i(x_1,\dotsc,x_n)\ (i=1,\dotsc,n)\)
统称为\(n\)元\DefineConcept{初等对称多项式}.

\subsection{对称多项式环}
下面我们把数域\(K\)上所有\(n\)元对称多项式组成的集合记为\(W\).
我们想要知道\(W\)的结构是怎样的.

\begin{proposition}
%@see: 《高等代数(第三版 下册)》(丘维声) P58 命题1
\(W\)是\(K[x_1,\dotsc,x_n]\)的一个子环.
\begin{proof}
显然\(W\)非空集.
任取\(f(x_1,\dotsc,x_n),g(x_1,\dotsc,x_n) \in W\),
设\begin{gather*}
	h(x_1,\dotsc,x_n)
	=f(x_1,\dotsc,x_n)
	-g(x_1,\dotsc,x_n), \\
	p(x_1,\dotsc,x_n)
	=f(x_1,\dotsc,x_n)
	g(x_1,\dotsc,x_n).
\end{gather*}
任给一个\(n\)元排列\(j_1 j_2 \dotso j_n\),
\(x_1,\dotsc,x_n\)用\(x_{j_1},\dotsc,x_{j_n}\)代入,
从以上两式分别得到\begin{align*}
	h(x_{j_1},\dotsc,x_{j_n})
	&=f(x_{j_1},\dotsc,x_{j_n})
	-g(x_{j_1},\dotsc,x_{j_n}) \\
	&=f(x_1,\dotsc,x_n)
	-g(x_1,\dotsc,x_n) \\
	&=h(x_1,\dotsc,x_n), \\
	p(x_{j_1},\dotsc,x_{j_n})
	&=f(x_{j_1},\dotsc,x_{j_n})
	g(x_{j_1},\dotsc,x_{j_n}) \\
	&=f(x_1,\dotsc,x_n)
	g(x_1,\dotsc,x_n) \\
	&=p(x_1,\dotsc,x_n).
\end{align*}
因此\(h(x_1,\dotsc,x_n),p(x_1,\dotsc,x_n) \in W\).
这就说明\(W\)是\(K[x_1,\dotsc,x_n]\)的一个子环.
\end{proof}
\end{proposition}

\begin{proposition}
%@see: 《高等代数(第三版 下册)》(丘维声) P59 命题2
设\(f_1,\dotsc,f_n \in W\),
则对\(K[x_1,\dotsc,x_n]\)中任意一个多项式\begin{equation*}
	g(x_1,\dotsc,x_n)
	=\sum_{i_1,\dotsc,i_n}
	b_{i_1 \dotso i_n}
	x_1^{i_1} \dotsm x_n^{i_n},
\end{equation*}
有\begin{equation*}
	g(f_1,\dotsc,f_n)
	=\sum_{i_1,\dotsc,i_n}
	b_{i_1 \dotso i_n}
	f_1^{i_1} \dotsm f_n^{i_n}
	\in W.
\end{equation*}
\end{proposition}

\begin{theorem}[对称多项式基本定理]
%@see: 《高等代数(第三版 下册)》(丘维声) P59 定理3
对于\(K[x_1,\dotsc,x_n]\)中任意一个对称多项式\(f(x_1,\dotsc,x_n)\),
都存在\(K[x_1,\dotsc,x_n]\)中唯一的一个多项式\(g(x_1,\dotsc,x_n)\),
使得\(f(x_1,\dotsc,x_n)=g(\sigma_1,\dotsc,\sigma_n)\).
\begin{proof}
存在性.
采取首项消去法.
设对称多项式\(f(x_1,\dotsc,x_n)\)的首项是\(a x_1^{l_1} \dotsm x_n^{l_n}\),
其中\(a\neq0\),且\(l_1 \geq \dotsb \geq l_n\).
为了消去\(f(x_1,\dotsc,x_n)\)的首项,
同时又要出现\(\sigma_1,\dotsc,\sigma_n\),
我们作多项式\begin{equation*}
	\phi_1(x_1,\dotsc,x_n)
	= a_1 \sigma_1^{l_1-l_2} \sigma_2^{l_2-l_3}
	\dotsm \sigma_{n-1}^{l_{n-1}-l_n} \sigma_n^{l_n},
\end{equation*}
其中\(a_1=a\).
因为\(K[x_1,\dotsc,x_n]\)中对称多项式的乘积还是对称多项式,
所以\(\phi_1(x_1,\dotsc,x_n)\)是对称多项式.
又由于多项式的乘积的首项等于它们的首项的乘积,
因此\(\phi_1(x_1,\dotsc,x_n)\)的首项是\begin{align*}
	&a_1 x_1^{l_1-l_2} (x_1 x_2)^{l_2-l_3}
	\dotsm (x_1 x_2 \dotsm x_{n-1})^{l_{n-1}-l_n}
	(x_1 x_2 \dotsm x_{n-1} x_n)^{l_n} \\
	&= a_1 x_1^{l_1} x_2^{l_2} \dotsm x_{n-1}^{l_{n-1}} x_n^{l_n},
\end{align*}
它等于\(f(x_1,\dotsc,x_n)\)的首项.
令\begin{equation*}
	f_1(x_1,\dotsc,x_n)
	=f(x_1,\dotsc,x_n)
	-\phi_1(x_1,\dotsc,x_n),
\end{equation*}
则\(f\)的首项的幂指数组\((l_1,\dotsc,l_n)\)
先于\(f_1\)的首项的幂指数组\((p_{11},\dotsc,p_{1n})\),
并且由于对称多项式的差仍是对称多项式,
所以\(f_1(x_1,\dotsc,x_n)\)是\(K[x_1,\dotsc,x_n]\)中的对称多项式.

对\(f_1(x_1,\dotsc,x_n)\)重复上述做法,
我们又得到\(K[x_1,\dotsc,x_n]\)中的一个对称多项式\begin{equation*}
	f_2(x_1,\dotsc,x_n)
	=f_1(x_1,\dotsc,x_n)
	-\phi_2(x_1,\dotsc,x_n),
\end{equation*}
其中\begin{equation*}
	\phi_2(x_1,\dotsc,x_n)
	=a_2 \sigma_1^{p_{11}-p_{12}} \sigma_2^{p_{12}-p_{13}}
	\dotsm \sigma_{n-1}^{p_{1,n-1}-p_{1n}} \sigma_n^{p_{1n}},
\end{equation*}
%\(\phi_2(x_1,\dotsc,x_n)\)是\(\sigma_1,\dotsc,\sigma_n\)的适当方幂的乘积,
并且其系数\(a_2\)等于\(f_1\)的首项系数,
\(f_1\)的首项的幂指数组\((p_{11},\dotsc,p_{1n})\)
先于\(f_2\)的首项的幂指数组\((p_{21},\dotsc,p_{2n})\).

如此继续下去,我们得到\(K[x_1,\dotsc,x_n]\)中一系列的对称多项式\begin{equation*}
	f,
	f_1=f-\phi_1,
	f_2=f_1-\phi_2,
	\dotsc,
	f_i=f_{i-1}-\phi_i,
	\dotsc,
\end{equation*}
其中\begin{equation*}
	\phi_i(x_1,\dotsc,x_n)
	=a_i \sigma_1^{p_{i1}-p_{i2}} \sigma_2^{p_{i2}-p_{i3}}
	\dotsm \sigma_{n-1}^{p_{i,n-1}-p_{in}} \sigma_n^{p_{in}},
\end{equation*}
%\(\phi_i\)是\(\sigma_1,\dotsc,\sigma_n\)的适当方幂的乘积,
并且其系数\(a_i\)等于\(f_{i-1}\)的首项系数.
容易看出,在上述多项式序列中,它们首项的幂指数组一个比一个小,即\begin{equation*}
	(l_1,\dotsc,l_2)
	>(p_{11},\dotsc,p_{1n})
	>\dotsb
	>(p_{k1},\dotsc,p_{kn})
	>\dotsb,
\end{equation*}
于是\(l_1 \geq p_{11}\).
又因为\(f_i\)是对称多项式,
所以\(p_{11} \geq \dotsb \geq p_n\).
因此\(l_1 \geq p_{11} \geq p_{12} \geq \dotsb p_{1n}\).
满足这个条件的非负整数组\((p_{11},\dotsc,p_{1n})\)只有有限多个,
因此上述对称多项式序列中只能有有限多个\(f_i\)不为零,
换言之,存在正整数\(s\),使得\(f_s=0\).
于是\begin{equation*}
	f_1=f-\phi_1,
	f_2=f_1-\phi_2,
	\dotsc,
	f_{s-1}=f_{s-2}-\phi_{s-1},
	f_s=f_{s-1}-\phi_s=0,
\end{equation*}
从而得到\begin{equation*}
	f=\phi_1+\dotsb+\phi_s.
\end{equation*}

设\(\phi_i(x_1,\dotsc,x_n)
=a_i \sigma_1^{t_{i1}} \dotsm \sigma_n^{t_{in}}\),
令\begin{equation*}
	g(x_1,\dotsc,x_n)
	=\sum_{i=1}^s a_i x_1^{t_{i1}} \dotsm x_n^{t_{in}}
	\in K[x_1,\dotsc,x_n],
\end{equation*}
则\begin{align*}
	g(\sigma_1,\dotsc,\sigma_n)
	&=\sum_{i=1}^s a_i \sigma_1^{t_{i1}} \dotsm \sigma_n^{t_{in}} \\
	&=\sum_{i=1}^s \phi_i(x_1,\dotsc,x_n) \\
	&=f(x_1,\dotsc,x_n).
\end{align*}
存在性成立.

唯一性.
如果\(K[x_1,\dotsc,x_n]\)中有两个不同的多项式
\(g_1(x_1,\dotsc,x_n)\)和\(g_2(x_1,\dotsc,x_n)\),
使得\begin{align*}
	f(x_1,\dotsc,x_n)
	&=g_1(\sigma_1,\dotsc,\sigma_n) \\
	&=g_2(\sigma_1,\dotsc,\sigma_n),
\end{align*}
则\(g_1(\sigma_1,\dotsc,\sigma_n)-g_2(\sigma_1,\dotsc,\sigma_n)=0\).
令\begin{equation*}
	g(x_1,\dotsc,x_n)
	=g_1(x_1,\dotsc,x_n)-g_2(x_1,\dotsc,x_n),
\end{equation*}
则\(g(\sigma_1,\dotsc,\sigma_n)=0\).
由假设可知\(g(x_1,\dotsc,x_n)\neq0\),
于是由\cref{theorem:多项式.多元多项式环.引理1} 可知,
存在\(b_1,\dotsc,b_n \in K\),
使得\(g(b_1,\dotsc,b_n)\neq0\).
令\begin{equation*}
	\phi(x)=x^n-b_1x^{n-1}+\dotsb+(-1)^kb_kx^{n-k}+\dotsb+(-1)^nb_n,
\end{equation*}
设\(\phi(x)\)的\(n\)个复根是\(\alpha_1,\dotsc,\alpha_n\),
则从韦达公式推出\begin{equation*}
	b_k=\sigma_k(\alpha_1,\dotsc,\alpha_n),
	\quad
	k=1,2,\dotsc,n.
\end{equation*}
\(x_1,\dotsc,x_n\)用\(\alpha_1,\dotsc,\alpha_n\)代入,
于是\begin{equation*}
	g(\sigma_1(\alpha_1,\dotsc,\alpha_n),\dotsc,\sigma_n(\alpha_1,\dotsc,\alpha_n))=0,
\end{equation*}
即\(g(b_1,\dotsc,b_n)=0\),矛盾!
唯一性成立.
\end{proof}
\end{theorem}

对称多项式基本定理完全解决了\(K[x_1,\dotsc,x_n]\)中所有对称多项式组成的子环\(W\)的结构问题.
定理中存在性的证明是构造性的,
可以实际地利用它去求多项式\(g(x_1,\dotsc,x_n)\),
使得\begin{equation*}
	f(x_1,\dotsc,x_n)
	=g(\sigma_1,\dotsc,\sigma_n).
\end{equation*}

\begin{example}
%@see: 《高等代数(第三版 下册)》(丘维声) P61 例1
在\(K[x_1,x_2,x_3]\)中,
用初等对称多项式表示出对称多项式\begin{equation*}
	f(x_1,x_2,x_3)
	=x_1^2 x_2^2
	+x_1^2 x_3^2
	+x_2^2 x_3^2.
\end{equation*}
\begin{solution}
\(f(x_1,x_2,x_3)\)的首项是\(x_1^2 x_2^2\),
它的幂指数组为\((2,2,0)\).
作多项式\begin{equation*}
	\phi_1(x_1,x_2,x_3)
	=\sigma_1^{2-2} \sigma_2^{2-0} \sigma_3^0
	=\sigma_2^2,
\end{equation*}
令\begin{align*}
	f_1(x_1,x_2,x_3)
	&= f(x_1,x_2,x_3)
	- \phi_1(x_1,x_2,x_3) \\
	&= -2 \sigma_1 \sigma_3,
\end{align*}
于是\begin{equation*}
	f(x_1,x_2,x_3)
	=\phi_1+f_1
	=\sigma_2^2 - 2 \sigma_1 \sigma_3.
\end{equation*}
\end{solution}
\end{example}

对于较复杂的\(n\)元对称多项式\(f(x_1,\dotsc,x_n)\),
求一个多项式\(g(x_1,\dotsc,x_n)\),
使得\begin{equation*}
	f(x_1,\dotsc,x_n)
	=g(\sigma_1,\dotsc,\sigma_n),
\end{equation*}
采用待定系数法更为简便.
我们举一个例子来说明这种方法.

\begin{example}
%@see: 《高等代数(第三版 下册)》(丘维声) P62 例2
在\(K[x_1,\dotsc,x_3]\)中,
用初等对称多项式表出
含有项\(x_1^2 x_2^2\)的项数最少的
对称多项式\(f(x_1,\dotsc,x_n)\).
%TODO
% \begin{solution}
% \(f\)的首项\(x_1^2\)的幂指数组为\((2,2,0,\dotsc,0)\).
% 所以
% 对称多项式基本定理指出,
% \end{solution}
\end{example}

如果给定的对称多项式不是齐次的,
那么可以把它表示成它的齐次成分的和.
将其中每一个齐次成分看作一个对称多项式,
按照上述做法计算,
最后把所得结果相加即可.

\subsection{复数域上的多项式的重根的存在性}
对称多项式基本定理的一个重要应用是,
研究数域\(K\)上的一个多项式
在复数域中有没有重根.

设数域\(K\)上首项系数为1的多项式\begin{equation*}
	f(x)=x^n+a_{n-1} x^{n-1}+\dotsb+a_1 x+a_0
\end{equation*}在复数域中的\(n\)个根为\(c_1,\dotsc,c_n\).
记\begin{equation*}
	D(c_1,\dotsc,c_n)
	\defeq
	\prod_{1\leq j<i\leq n} (c_i-c_j)^2,
\end{equation*}
容易看出\begin{align*}
	&\text{$f(x)$在复数域中有重根} \\
	&\iff
	D(c_1,\dotsc,c_n)=0.
\end{align*}

\(f(x)\)的\(n\)个复根\(c_1,\dotsc,c_n\)是未知的,
于是我们想用\(f(x)\)的系数来表示\(D(c_1,\dotsc,c_n)\).
根据韦达公式有\begin{equation*}
	\left\{ \begin{array}{l}
		-a_{n-1}
		=c_1+\dotsb+c_n
		=\sigma_1(c_1,\dotsc,c_n), \\
		a_{n-1}
		=\sum_{1\leq i<j\leq n} c_i c_j
		=\sigma_2(c_1,\dotsc,c_n), \\
		\hdotsfor1 \\
		(-1)^n a_0
		=c_1 \dotsm c_n
		=\sigma_n(c_1,\dotsc,c_n).
	\end{array} \right.
\end{equation*}
受此启发,
如果\(D(c_1,\dotsc,c_n)\)能够用\(\sigma_1(c_1,\dotsc,c_n),
\dotsc,
\sigma_n(c_1,\dotsc,c_n)\)表示出来,
那么它就能够用\(f(x)\)的系数\(a_{n-1},a_{n-2},\dotsc,a_0\)表示出来.
注意到\(D(c_1,\dotsc,c_n)\)是关于\(c_1,\dotsc,c_n\)对称的表达式,
因此自然会想到运用对称多项式基本定理.

根据对称多项式基本定理,
数域\(K\)上\(n\)元对称多项式\begin{equation*}
	D(x_1,\dotsc,x_n)
	=\prod_{1\leq j<i\leq n} (x_i-x_j)^2
\end{equation*}
存在\(K[x_1,\dotsc,x_n]\)中唯一的一个多项式\(g(x_1,\dotsc,x_n)\),
使得\begin{equation*}
	D(x_1,\dotsc,x_n)
	=g(\sigma_1,\dotsc,\sigma_n).
\end{equation*}
不定元\(x_1,\dotsc,x_n\)分别用\(c_1,\dotsc,c_n\)代入,
于是有\begin{equation*}
	D(c_1,\dotsc,c_n)
	=g(-a_{n-1},a_{n-2},\dotsc,(-1)^n a_0).
\end{equation*}
因此我们有以下结论.

\begin{proposition}
%@see: 《高等代数(第三版 下册)》(丘维声) P63 命题4
数域\(K\)上首项系数为\(1\)的\(n\)次多项式\begin{equation*}
	f(x)=x^n+a_{n-1} x^{n-1}+\dotsb+a_1 x+a_0
\end{equation*}
在复数域中有重根的充分必要条件为\begin{equation*}
	g(-a_{n-1},a_{n-2},\dotsc,(-1)^n a_0)=0.
\end{equation*}
\end{proposition}

我们把\(f(x)\)的系数\(a_{n-1},a_{n-2},\dotsc,a_0\)的多项式\begin{equation*}
	g(-a_{n-1},a_{n-2},\dotsc,(-1)^n a_0)
\end{equation*}称为“\(f(x)\)的\DefineConcept{判别式}”,
记作\(D(f)\).

现在我们来求\(f(x)\)的判别式\(D(f)\).
\begin{align*}
	D(f)
	&= g(-a_{n-1},a_{n-2},\dotsc,(-1)^n a_0) \\
	&= D(c_1,\dotsc,c_n) \\
	&= \prod_{1\leq j<i\leq n} (c_i-c_j)^2.
\end{align*}
表达式\(\prod_{1\leq j<i\leq n} (c_i-c_j)^2\)使人联想起范德蒙德行列式\begin{equation*}
	\begin{vmatrix}
		1 & 1 & 1 & \dots & 1 \\
		x_1 & x_2 & x_3 & \dots & x_n \\
		x_1^2 & x_2^2 & x_3^2 & \dots & x_n^2 \\
		\vdots & \vdots & \vdots& & \vdots \\
		x_1^{n-1} & x_2^{n-1} & x_3^{n-1} & \dots & x_n^{n-1}
	\end{vmatrix}
	= \prod_{1 \leq j < i \leq n}(x_i-x_j).
\end{equation*}
若记\begin{equation*}
	\vb{V}(x_1,\dotsc,x_n) = \begin{bmatrix}
		1 & 1 & 1 & \dots & 1 \\
		x_1 & x_2 & x_3 & \dots & x_n \\
		x_1^2 & x_2^2 & x_3^2 & \dots & x_n^2 \\
		\vdots & \vdots & \vdots& & \vdots \\
		x_1^{n-1} & x_2^{n-1} & x_3^{n-1} & \dots & x_n^{n-1}
	\end{bmatrix},
\end{equation*}
考虑到\(\abs{\vb{V}(x_1,\dotsc,x_n)}=\abs{\vb{V}^T(x_1,\dotsc,x_n)}\),
于是有\begin{align*}
%@see: 《高等代数(第三版 下册)》(丘维声) P64 公式(26)
	D(f)
	&= \prod_{1\leq j<i\leq n} (c_i-c_j)^2 \\
	&= \abs{\vb{V}(c_1,\dotsc,c_n)} \abs{\vb{V}^T(c_1,\dotsc,c_n)} \\
	&= \abs{\vb{V}(c_1,\dotsc,c_n) \vb{V}^T(c_1,\dotsc,c_n)}.
\end{align*}
于是\begin{align}
%@see: 《高等代数(第三版 下册)》(丘维声) P64 公式(27)
	D(f)
	&= \abs{
		\begin{bmatrix}
			1 & 1 & \dots & 1 \\
			c_1 & c_2 & \dots & c_n \\
			\vdots & \vdots && \vdots \\
			c_1^{n-1} & c_2^{n-1} & \dots & c_n^{n-1}
		\end{bmatrix}
		\begin{bmatrix}
			1 & c_1 & \dots & c_1^{n-1} \\
			1 & c_2 & \dots & c_2^{n-1} \\
			\vdots & \vdots && \vdots \\
			1 & c_n & \dots & c_n^{n-1}
		\end{bmatrix}
	} \notag \\
	&= \begin{vmatrix}
		n & \sum_{i=1}^n c_i & \dots & \sum_{i=1}^n c_i^{n-1} \\
		\sum_{i=1}^n c_i & \sum_{i=1}^n c_i^2 & \dots & \sum_{i=1}^n c_i^n \\
		\vdots & \vdots && \vdots \\
		\sum_{i=1}^n c_i^{n-1} & \sum_{i=1}^n c_i^n & \dots & \sum_{i=1}^n c_i^{2n-2}
	\end{vmatrix}.
	\label{equation:多项式.对称多项式.范德蒙德}
\end{align}
上式表明,
为了求出\(D(f)\),
就需要计算\begin{equation*}
%@see: 《高等代数(第三版 下册)》(丘维声) P64 公式(28)
	\sum_{i=1}^n c_i^k,
	\quad k=0,1,\dotsc,2n-2.
\end{equation*}
由于\(f(x)\)的\(n\)个复根\(c_1,\dotsc,c_n\)是未知的,
因此必须想办法通过\(f(x)\)的系数来计算\(\sum_{i=1}^n c_i^k\).
由于\(\sum_{i=1}^n c_i^k\)是对称多项式,
因此仍然想到运用对称多项式基本定理.
为此我们考虑下列\(n\)元对称多项式\begin{equation*}
%@see: 《高等代数(第三版 下册)》(丘维声) P65 公式(29)
	s_k(x_1,\dotsc,x_n)
	=x_1^k+\dotsb+x_n^k,
	\quad k=0,1,2,\dotsc.
\end{equation*}
这些\(n\)元对称多项式称为\DefineConcept{幂和}.

根据对称多项式基本定理,
幂和\(s_k\)能表示成初等对称多项式的多项式.
具体的表示方法可以用递推公式求出.

当\(1\leq k\leq n\)时,
\begin{equation}\label{equation:多项式.对称多项式.牛顿公式1}
%@see: 《高等代数(第三版 下册)》(丘维声) P65 公式(30)
	s_k
	- \sigma_1 s_{k-1}
	+ \sigma_2 s_{k-2}
	+ \dotsb
	+ (-1)^{k-1} \sigma_{k-1} s_1
	+ (-1)^k k \sigma_k
	=0;
\end{equation}
当\(k>n\)时,
\begin{equation}\label{equation:多项式.对称多项式.牛顿公式2}
%@see: 《高等代数(第三版 下册)》(丘维声) P65 公式(31)
	s_k
	- \sigma_1 s_{k-1}
	+ \sigma_2 s_{k-2}
	+ \dotsb
	+ (-1)^{n-1} \sigma_{n-1} s_{k-n+1}
	+ (-1)^n \sigma_n s_{k-n}
	=0.
\end{equation}
我们把\cref{equation:多项式.对称多项式.牛顿公式1,equation:多项式.对称多项式.牛顿公式2}
并称为\DefineConcept{牛顿公式}.

利用牛顿公式,
可以从\(s_{k-1}(c_1,\dotsc,c_n),\dotsc,s_1(c_1,\dotsc,c_n)\)
以及\(\sigma_1(c_1,\dotsc,c_n),\dotsc,\sigma_n(c_1,\dotsc,c_n)\)
计算出\(s_k(c_1,\dotsc,c_n)\).
再从\cref{equation:多项式.对称多项式.范德蒙德}
就可以求出\(f(x)\)的判别式\(D(f)\).

多项式\(f(x)\)的判别式\(D(f)\)也称为方程\(f(x)=0\)的判别式.

\begin{remark}
在上述讨论中,\(f(x)\)的首项系数是1.
如果\(f(x)\)的首项系数是\(a_n\),
则可以先对\(a_n^{-1} f(x)\)运用上述方法,
求出它的判别式\(D(a_n^{-1} f)\),
然后规定\(f(x)\)的判别式为
\begin{equation}
%@see: 《高等代数(第三版 下册)》(丘维声) P65 公式(32)
	D(f)
	\defeq
	a_n^{2n-2} D(a_n^{-1} f).
\end{equation}
\end{remark}

\begin{example}
%@see: 《高等代数(第三版 下册)》(丘维声) P65 例3
求数域\(K\)上二次方程\(f(x)=x^2+bx+c=0\)的判别式.
\begin{solution}
设\(f(x)\)的复根是\(c_1,c_2\),
则\begin{equation*}
	\sigma_1(c_1,c_2)=-b, \qquad
	\sigma_2(c_1,c_2)=c.
\end{equation*}
根据牛顿公式有\begin{equation*}
	s_1=\sigma_1, \qquad
	s_2=\sigma_1 s_1 - 2\sigma_2
	=\sigma_1^2-2\sigma_2,
\end{equation*}
于是\begin{equation*}
	s_1(c_1,c_2)=-b, \qquad
	s_2(c_1,c_2)=b^2-2c,
\end{equation*}
因此\begin{equation*}
%@see: 《高等代数(第三版 下册)》(丘维声) P65 公式(33)
	D(f)
	=\begin{vmatrix}
		2 & -b \\
		-b & b^2-2c
	\end{vmatrix}
	=b^2-4c.
\end{equation*}
\end{solution}
\end{example}

\begin{example}
%@see: 《高等代数(第三版 下册)》(丘维声) P65 例4
求数域\(K\)上不完全三次方程\(f(x)=x^3+ax+b=0\)的判别式.
\begin{solution}
设\(f(x)\)的复根是\(c_1,c_2,c_3\),
则\begin{equation*}
	\sigma_1(c_1,c_2,c_3)=0, \qquad
	\sigma_2(c_1,c_2,c_3)=a, \qquad
	\sigma_3(c_1,c_2,c_3)=-b.
\end{equation*}
根据牛顿公式有\begin{gather*}
	s_1(c_1,c_2,c_3)=0, \qquad
	s_2(c_1,c_2,c_3)=-2a, \\
	s_3(c_1,c_2,c_3)=-3b, \qquad
	s_4(c_1,c_2,c_3)=2a^2,
\end{gather*}
所以\begin{equation*}
%@see: 《高等代数(第三版 下册)》(丘维声) P66 公式(35)
	D(f)
	=\begin{vmatrix}
		3 & 0 & -2a \\
		0 & -2a & -3b \\
		-2a & -3b & 2a^2
	\end{vmatrix}
	=-4a^3-27b^2.
\end{equation*}
\end{solution}
\end{example}
