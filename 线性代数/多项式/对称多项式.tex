\section{对称多项式}
观察下述三元多项式\(f(x_1,x_2,x_3)\)有什么特点?
\[
	f(x_1,x_2,x_3)
	=x_1^3+x_2^3+x_3^3
	+x_1^2x_2
	+x_1^2x_3
	+x_2^2x_3
	+x_1x_2^2
	+x_1x_3^2
	+x_2x_3^2.
\]
直观上看,
\(x_1,x_2,x_3\)在\(f(x_1,x_2,x_3)\)中的地位是对称的,
即同时有\(x_1^3,x_2^3,x_3^3\)这三项,
且同时有\(x_1^2x_2,
x_1^2x_3,
x_2^2x_3,
x_1x_2^2,
x_1x_3^2,
x_2x_3^2\)这六项.
由此受到启发,
我们来研究具有这种性质的\(n\)元多项式\(f(x_1,\dotsc,x_n)\):
若\(f(x_1,\dotsc,x_n)\)含有一项\(a x_1^{i_1} \dotsm x_n^{i_n}\),
则它也含有一项\(a x_{j_1}^{i_1} \dotsm x_{j_n}^{i_n}\),
其中\(j_1 \dotso j_n\)是任意一个\(n\)元排列.

于是我们抽象出下述概念.
\begin{definition}
%@see: 《高等代数(第三版 下册)》(丘维声) P57 定义1
设\(f(x_1,\dotsc,x_n)\)是数域\(K\)上的一个\(n\)元多项式.
如果对于任意一个\(n\)元排列\(j_1 \dotso j_n\)都有\[
	f(x_{j_1},\dotsc,x_{j_n})
	=f(x_1,\dotsc,x_n),
\]
则称“\(f(x_1,\dotsc,x_n)\)是数域\(K\)上的一个\(n\)元\DefineConcept{对称多项式}”.
\end{definition}
