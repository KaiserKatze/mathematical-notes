\section{模m剩余类环}
\begin{definition}
%@see: 《高等代数(第三版 下册)》(丘维声) P67
设\(a,b,m\)都是整数.
如果\(m\mid(a-b)\),
则称“\(a\)与\(b\) \DefineConcept{模\(m\)同余}”,
记作\(a\equiv b\pmod m\).
\end{definition}
\begin{definition}
%@see: 《高等代数(第三版 下册)》(丘维声) P67
对于给定正整数\(m\ (m>1)\),
在整数集\(\mathbb{Z}\)上定义一个二元关系\begin{equation*}
	a \sim b
	\defiff
	a\equiv b\pmod m,
\end{equation*}
称之为\DefineConcept{模\(m\)同余关系}.
\end{definition}
\begin{theorem}
%@see: 《高等代数(第三版 下册)》(丘维声) P67
模\(m\)同余关系\(\sim\),具有反身性、对称性、传递性,是\(\mathbb{Z}\)上的一个等价关系.
\end{theorem}
\begin{definition}
%@see: 《高等代数(第三版 下册)》(丘维声) P67
对于每一个整数\(x\),
把\(x\)在模\(m\)同余关系\(\sim\)下的等价类,
称为一个\DefineConcept{模\(m\)剩余类}.
\end{definition}
\begin{proposition}
%@see: 《高等代数(第三版 下册)》(丘维声) P67
给定整数\(m\),
则模\(m\)剩余类一共有\(m\)个,
包括\begin{equation*}
	\overline{0}
	= \Set{ x\in\mathbb{Z} \given x\equiv0\pmod m },
	\dotsc,
	\overline{m-1}
	= \Set{ x\in\mathbb{Z} \given x\equiv m-1\pmod m }.
\end{equation*}
\end{proposition}
\begin{definition}
%@see: 《高等代数(第三版 下册)》(丘维声) P67
\(\mathbb{Z}\)对模\(m\)同余关系\(\sim\)的商集,
记作\(\mathbb{Z}_m\).
\end{definition}

\begin{proposition}
%@see: 《高等代数(第三版 下册)》(丘维声) P67 命题1
在\(\mathbb{Z}\)中,
若\(a\equiv b\pmod m,
c\equiv d\pmod m\),
则\[
	a+c\equiv b+d\pmod m, \qquad
	ac\equiv bd\pmod m.
\]
\begin{proof}
由已知条件,
\(m\mid(a-b),
m\mid(c-d)\).
从而\(m\mid[(a-b)+(c-d)]\),
即\(m\mid[(a+c)-(b+d)]\).
因此\(a+c\equiv b+d\pmod m\).

由于\(ac-bd
=ac-bc+bc-bd
=(a-b)c+b(c-d)\),
又有\(m\mid[(a-b)c+b(c-d)]\),
因此\(m\mid(ac-bd)\),
从而\(ac\equiv bd\pmod m\).
\end{proof}
\end{proposition}

\begin{theorem}
%@see: 《高等代数(第三版 下册)》(丘维声) P69 定理2
若\(p\)是素数,
则模\(p\)剩余类环\(\mathbb{Z}_p\)是一个域.
\begin{proof}
已知\(\mathbb{Z}_p\)是一个有单位元\(\overline1\)的交换环.
任取\(\mathbb{Z}_p\)的一个非零元\(\overline{a}\),
其中\(0<a<p\).
于是\(p \nmid a\).
又由于\(p\)是素数,
因此\((p,a)=1\).
于是存在\(u,v\in\mathbb{Z}\),
使得\(up+va=1\).
因此\[
	\overline1
	=\overline{up+va}
	=\overline{up}
	+\overline{va}
	=\overline{u}~\overline{p}
	+\overline{v}~\overline{a}
	=\overline{v}~\overline{a}.
\]
可见\(\overline{a}\)是可逆元.
所以\(\mathbb{Z}_p\)是一个域.
\end{proof}
\end{theorem}

\begin{theorem}
%@see: 《高等代数(第三版 下册)》(丘维声) P71 习题7.11 2.
若\(p\)是合数,
则模\(p\)剩余类环\(\mathbb{Z}_p\)不是域.
%TODO proof
\end{theorem}

给定素数\(p\),
我们把\(\mathbb{Z}_p\)称为\DefineConcept{模\(p\)剩余类域}.

模\(p\)剩余类域\(\mathbb{Z}_p\)与数域\(K\)有以下两个不同点:
\begin{enumerate}
	\item 数域\(K\)是无限域,
	而模\(p\)剩余类域\(\mathbb{Z}_p\)是有限域.

	\item 在\(\mathbb{Z}_p\)中,
	\(p\overline1
	=\overline{p}
	=\overline0\),
	\(l\overline1
	=\overline{l}
	\neq\overline0\ (0<l<p)\).
	在数域\(K\)中,
	有\((\forall n\in\mathbb{N}^*)[n1=n\neq0]\).
\end{enumerate}

\begin{theorem}
%@see: 《高等代数(第三版 下册)》(丘维声) P70 定理3
设\(F\)是一个域,
它的单位元为\(e\),
则要么\((\forall n\in\mathbb{N}^*)[ne\neq0]\);
要么存在一个素数\(p\),使得\(pe=0\),
且当\(0<l<p\)时,有\(le\neq0\).
\begin{proof}
设\(n\)是使得\(ne=0\)成立的最小正整数.
假设\(n\)不是素数,
则\[
	n=n_1 n_2,
	\qquad
	0<n_1 \leq n_2<n.
\]
于是%根据习题7.1 11
\[
	(n_1 e)(n_2 e)
	=n_1[e(n_2 e)]
	=n_1[n_2(ee)]
	=n_1(n_2 e)
	=(n_1 n_2)e
	=ne=0.
\]
由于正整数\(n_1,n_2\)都小于\(n\),
因此\(n_1 e\neq0,
n_2 e\neq0\).
由于\(F\)是域,
所以\(n_1 e\)是可逆元.
于是\[
	n_2 e
	=[(n_1 e)^{-1} (n_1 e)](n_2 e)
	=(n_1 e)^{-1}
	[(n_1 e)(n_2 e)]
	=(n_1 e)^{-1} 0,
\]
矛盾!
因此\(n\)是素数.
\end{proof}
\end{theorem}

\begin{definition}
%@see: 《高等代数(第三版 下册)》(丘维声) P70 定义2
设\(F\)是一个域,
它的单位元为\(e\).
如果对于任一正整数\(n\)都有\(ne\neq0\),
那么称“域\(F\)的\DefineConcept{特征}为0”,
记作\(\FieldChar F=0\);
如果存在一个素数\(p\)使得\(pe=0\),
而当\(0<l<p\)时\(le\neq0\),
那么称“域\(F\)的\DefineConcept{特征}为\(p\)”,
记作\(\FieldChar F=p\).
\end{definition}

据此定义,域\(F\)的特征要么是零,要么是一个素数.
具体来说,模\(p\)剩余类域\(\mathbb{Z}_p\)的特征是\(p\),
任一数域的特征是零.

\begin{corollary}
%@see: 《高等代数(第三版 下册)》(丘维声) P70 推论4
如果域\(F\)的特征是素数\(p\),
则\[
	ne=0
	\iff
	p \mid n.
\]
\begin{proof}
充分性.
设\(p \mid n\),
则\(n=lp\).
于是\(ne
=(lp)e
=0\).

必要性.
设\(ne=0\)
且\(n=hp+r,0\leq r<p\),
则\[
	0=ne
	=(hp+r)e
	=hpe+re
	=re.
\]
由于\(\FieldChar F=p\),
且\(r<p\),
所以由上式得\(r=0\).
从而\(n=hp\),
即\(p \mid n\).
\end{proof}
\end{corollary}

\begin{corollary}
%@see: 《高等代数(第三版 下册)》(丘维声) P70 推论5
设域\(F\)的特征是素数\(p\),
任取\(a \in F-\{0\}\),
则\[
	na=0
	\iff
	p \mid n.
\]
\begin{proof}
\(na=0
\iff
n(ea)=0
\iff
(ne)a=0
\iff
ne=0
\iff
p \mid n\).
\end{proof}
\end{corollary}

类似于数域\(K\)上的多项式,
我们可以定义任一域\(F\)上的多项式,
并且得出域\(F\)上的一元多项式环\(F[x]\)
和多元多项式环\(F[x_1,\dotsc,x_n]\).
不难看出,
有关数域\(K\)上一元多项式环\(K[x]\)的结论,
只要在它的证明中没有用到这个域含有无穷多个元素,
那么它对于任一域\(F\)上的一元多项式环\(F[x]\)也成立.
还需要注意,
如果域\(F\)的特征是素数\(p\),
则\(F\)的任一元素的\(p\)倍等于零.

例如,对于数域\(K\)上的两个一元多项式\(f(x)\)与\(g(x)\),
如果它们不相等,
那么由它们分别确定的多项式函数\(f\)与\(g\)也不相等.
这个结论的证明需要用到数域\(K\)有无穷多个元素.
因此这个结论对于有限域上的多项式就不成立.
譬如,在\(\mathbb{Z}_3[x]\)中,
设\(f(x)=x^3-x,
g(x)=0\).
显然\(f(x) \neq g(x)\).
但是由\(f(x)\)确定的多项式函数\(f\)满足\[
	f(\overline0)=\overline0, \qquad
	f(\overline1)=\overline{1^3}-\overline1=\overline0, \qquad
	f(\overline2)=\overline{2^3}-\overline2=\overline0,
\]
因此\(f\)是零函数.
而\(g\)也是零函数.

在任一域\(F\)上的一元多项式环\(F[x]\)中,
也有不可约多项式的概念和唯一因式分解定理,
也有根与一次因式的关系,等等.
