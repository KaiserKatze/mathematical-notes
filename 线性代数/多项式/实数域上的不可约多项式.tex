\section{实数域上的不可约多项式}
这一节我们要找出实数域上的所有不可约多项式.
由于每一个复数都可以表示成\(a+b\iu\)的形式,
其中\(a,b\)都是实数,
因此我们可以利用复数域上多项式的信息来研究实数域上的不可约多项式.

\begin{theorem}\label{theorem:实数域上的不可约多项式.多项式的复根的共轭也是根}
%@see: 《高等代数(第三版 下册)》(丘维声) P40 定理1
设\(f(x)\)是实系数多项式,
如果\(c\)是\(f(x)\)的一个复根,
则\(c\)的共轭复数\(\ComplexConjugate{c}\)也是\(f(x)\)的一个复根.
\begin{proof}
设\(f(x)=a_n x^n+a_{n-1} x^{n-1}+\dotsb+a_1 x+a_0\),
其中\(a_i\in\mathbb{R}\ (i=0,1,\dotsc,n)\).
因为\(c\)是\(f(x)\)的复根,
所以\begin{equation*}
	f(c)=a_n c^n+a_{n-1} c^{n-1}+\dotsb+a_1 c+a_0=0.
\end{equation*}
在上式两边取共轭,得\begin{equation*}
	a_n \ComplexConjugate{c}^n+a_{n-1} \ComplexConjugate{c}^{n-1}+\dotsb+a_1 \ComplexConjugate{c}+a_0=0,
\end{equation*}
因此\(\ComplexConjugate{c}\)是\(f(x)\)的一个复根.
\end{proof}
\end{theorem}

\begin{theorem}\label{theorem:实数域上的不可约多项式.实数域上的不可约多项式}
%@see: 《高等代数(第三版 下册)》(丘维声) P40 定理2
实数域上的不可约多项式都是一次多项式或判别式小于零的二次多项式.
\begin{proof}
设\(f(x)\in\mathbb{R}[x]\)是不可约的.
把\(f(x)\)看成复系数多项式,
根据代数基本定理,
\(f(x)\)有一个复根\(c\).

如果\(c\)是实数,
则\(f(x)\)是\(\mathbb{R}[x]\)中有一次因式\(x-c\).
因为\(f(x)\)不可约,
所以一定有\(f(x) \sim (x-c)\),
从而有\(f(x)=a(x-c)\),
其中\(a\)是非零实数,
因此\(f(x)\)是一次多项式.

如果\(c\)是虚数,
根据\cref{theorem:实数域上的不可约多项式.多项式的复根的共轭也是根},
\(\ComplexConjugate{c}\)也是\(f(x)\)的一个复根.
由于\(c\neq\ComplexConjugate{c}\),
所以\((x-c,x-\ComplexConjugate{c})=1\).
在\(\mathbb{C}[x]\)中,
\((x-c) \mid f(x),
(x-\ComplexConjugate{c}) \mid f(x)\),
从而根据\cref{theorem:多项式.互素.性质2} 得\begin{equation*}
	(x-c)(x-\ComplexConjugate{c}) \mid f(x).
\end{equation*}
于是\begin{equation*}
	(x-c)(x-\ComplexConjugate{c})
	=x^2-(c+\ComplexConjugate{c})x+c\ComplexConjugate{c},
\end{equation*}
而\(c+\ComplexConjugate{c}\)和\(c\ComplexConjugate{c}\)都是实数.
既然在\(\mathbb{C}[x]\)中
有\([x^2-(c+\ComplexConjugate{c})x+c\ComplexConjugate{c}] \mid f(x)\),
所以在\(\mathbb{R}[x]\)中
也有\([x^2-(c+\ComplexConjugate{c})x+c\ComplexConjugate{c}] \mid f(x)\).
由于\(f(x)\)在\(\mathbb{R}[x]\)中不可约,
因此\(f(x)=a[x^2-(c+\ComplexConjugate{c})x+c\ComplexConjugate{c}]\),
其中\(a\)是非零实数;
又因为\begin{align*}
	&\text{实系数二次多项式$f(x)=ax^2+bx+c$不可约} \\
	&\iff \text{$f(x)$在$\mathbb{R}[x]$中没有一次因式} \\
	&\iff \text{$f(x)$没有实根} \\
	&\iff b^2-4ac<0,
\end{align*}
因此\(f(x)\)是判别式小于零的二次多项式.
\end{proof}
\end{theorem}

\begin{theorem}[实系数多项式唯一因式分解定理]
%@see: 《高等代数(第三版 下册)》(丘维声) P41 定理3
每个次数大于零的实系数多项式
在实数域上都可以唯一地分解成
一次因式与判别式小于零的二次因式的乘积.
\begin{proof}
由\cref{theorem:实数域上的不可约多项式.实数域上的不可约多项式,theorem:多项式.唯一因式分解定理}
立即可得.
\end{proof}
\end{theorem}
