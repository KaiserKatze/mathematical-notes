\section{最小公倍式}
\begin{definition}
在\(K[x]\)中,如果\(c(x)\)既是\(f(x)\)的倍式,又是\(g(x)\)的倍式,
则称“\(c(x)\)是\(f(x)\)与\(g(x)\)的一个\DefineConcept{公倍式}”.
\end{definition}

\begin{definition}
%@see: 《高等代数(第三版 下册)》(丘维声) P22 习题7.3 10.
设\(f(x),g(x) \in K[x]\),
\(m(x)\)是\(f(x)\)与\(g(x)\)的一个公倍式.
如果\(f(x)\)与\(g(x)\)的任一公倍式都是\(m(x)\)的倍式,
则称“\(m(x)\)是\(f(x)\)与\(g(x)\)的一个\DefineConcept{最小公倍式}”.
\end{definition}

我们约定,用\[
	[f(x), g(x)]
\]表示首项系数是\(1\)的那个最小公倍式.

\begin{example}
%@see: 《高等代数(第三版 下册)》(丘维声) P22 习题7.3 10.(1)
证明:\(K[x]\)中任意两个多项式都有最小公倍式,
并且在相伴的意义下是唯一的.
%TODO proof
\end{example}

\begin{example}
%@see: 《高等代数(第三版 下册)》(丘维声) P22 习题7.3 10.(2)
证明:如果\(f(x),g(x)\)的首项系数都是\(1\),
则\[
	[f(x),g(x)]
	= \frac{f(x) g(x)}{(f(x),g(x))}.
\]
%TODO proof
\end{example}
