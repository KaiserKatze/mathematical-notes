\section{多项式}
\begin{definition}\label{definition:多项式.多项式的定义}
设\(K\)是一个数域,
\(x\)是一个不属于\(K\)的符号.
任意给定一个非负整数\(n\),
在\(K\)中任意取定\(\AutoTuple{a}[0]{n}\),
如果表达式\begin{equation}\label[polynomial]{equation:多项式.多项式}
	a_n x^n + a_{n-1} x^{n-1} + \dotsb + a_1 x + a_0
\end{equation}
满足\begin{enumerate}
	\item 两个这种形式的表达式相等当且仅当它们除了系数为零的项以外含有完全相同的项,
	即\begin{align*}
		&a_n x^n + a_{n-1} x^{n-1} + \dotsb + a_1 x + a_0
		= b_n y^n + b_{n-1} y^{n-1} + \dotsb + b_1 y + b_0 \\
		&\iff
		(\forall i\in\Set{0,1,\dotsc,n})[a_i = b_i = 0 \lor a_i x^i = b_i y^i].
	\end{align*}
	\item 允许从表达式中删去系数为零的项,也允许向表达式中添进系数为零的项,
\end{enumerate}
那么称之为“数域\(K\)上的一个一元\DefineConcept{多项式}(polynomial)”,
把\(x\)称为\DefineConcept{不定元}.
%@see: 《Linear Algebra Done Right (Fourth Eidition)》(Sheldon Axler) P30 2.10
\end{definition}

系数全为零的多项式称为\DefineConcept{零多项式}.
在\cref{equation:多项式.多项式} 中,
把\(a_i x^i\)称为“\(i\)次\DefineConcept{项}”,
把\(a_0\)称为\DefineConcept{零次项}或\DefineConcept{常数项}.

从\cref{definition:多项式.多项式的定义} 知道,
数域\(K\)上两个一元多项式相等当且仅当它们的同次项的系数都相等.

%@see: 《Linear Algebra Done Right (Fourth Eidition)》(Sheldon Axler) P31 2.11
设\(f(x)\)表示\cref{equation:多项式.多项式}.
如果\(a_n\neq0\),
则称“\(a_n x^n\)是多项式\(f(x)\)的\DefineConcept{首项}”;
并称“\(f(x)\)的\DefineConcept{次数}是\(n\)”
或“\(f(x)\)是\(n\)次多项式”,
记作\(\deg f\).

零多项式的次数定义为\(-\infty\),并规定:\begin{gather*}
	(-\infty)+(-\infty)=-\infty, \\
	(-\infty)+n=-\infty, \\
	-\infty<n,
\end{gather*}
其中\(n\)是任意非负整数.

零次多项式总是一个常数\(a\),它满足\(a \in K \land a \neq 0\).

我们把数域\(K\)上的所有一元多项式组成的集合记作\(K[x]\).
我们可以在\(K[x]\)中规定“加法”与“乘法”运算.
设\begin{equation*}
	f(x) = \sum_{i=0}^n a_i x^i, \qquad
	g(x) = \sum_{i=0}^n b_i x^i,
\end{equation*}
如果\(n \ge m\),那么\begin{gather}
	f(x) + g(x) \defeq \sum_{i=0}^n (a_i+b_i) x^i, \\
	f(x) \cdot g(x) \defeq \sum_{s=0}^{n+m} \left( \sum_{i+j=s} a_i b_j \right) x^s,
\end{gather}
我们把\(f(x)+g(x)\)称为“\(f(x)\)与\(g(x)\)的\DefineConcept{和}”,
把\(f(x) \cdot g(x)\)称为“\(f(x)\)与\(g(x)\)的\DefineConcept{积}”.

容易验证上面所定义的多项式的加法与乘法满足下列运算法则:
\begin{enumerate}
	\item 加法交换律,即\begin{equation*}
		(\forall f,g \in K[x])[f+g=g+f].
	\end{equation*}

	\item 加法结合律,即\begin{equation*}
		(\forall f,g,h \in K[x])[(f+g)+h=f+(g+h)].
	\end{equation*}

	\item 零多项式\(0\)是加法单位元,即\begin{equation*}
		(\forall f \in K[x])[0+f=f+0=f].
	\end{equation*}

	\item \(K[x]\)具有负元.

	设\(f(x)=\sum_{i=0}^n a_i x^i\),
	定义\(-f(x)=\sum_{i=0}^n (-a_i) x^i\),则\begin{equation*}
		f+(-f)=0.
	\end{equation*}
	称\(-f\)为\(f\)的\DefineConcept{负元}.

	\item 乘法交换律,即\begin{equation*}
		(\forall f,g \in K[x])[fg=gf].
	\end{equation*}

	\item 乘法结合律,即\begin{equation*}
		(\forall f,g,h \in K[x])[(fg)h=f(gh)].
	\end{equation*}

	\item 零次多项式\(1\)是乘法单位元,即\begin{equation*}
		1f=f1=f.
	\end{equation*}

	\item 乘法对加法的分配律,即\begin{equation*}
		(\forall f,g,h \in K[x])[f(g+h)=fg+fh],
	\end{equation*}\begin{equation*}
		(\forall f,g,h \in K[x])[(g+h)f=gf+hf].
	\end{equation*}

	\item 乘法消去律,即\begin{equation*}
		fg=fh \land f\neq0 \implies g=h.
	\end{equation*}
\end{enumerate}

多项式的减法定义如下:\begin{equation}
	f-g \defeq f+(-g).
\end{equation}

%@see: 《Linear Algebra Done Right (Fourth Eidition)》(Sheldon Axler) P31 2.12
我们把数域\(K\)上的所有次数不超过\(m\ (m\geq0)\)的一元多项式组成的集合记作\(K[x]_m\).

\begin{proposition}
%@see: 《高等代数(第三版 下册)》(丘维声) P3 命题1
设\(f,g \in K[x]\),则\begin{gather}
	\deg(f \pm g) \leq \max\{\deg f, \deg g\},
	\label{equation:多项式.和的次数} \\
	\deg(fg) = \deg f + \deg g.
	\label{equation:多项式.积的次数}
\end{gather}
\begin{proof}
如果\(f=0\)或\(g=0\),
则上述两式显然成立.
下面假设\begin{equation*}
	f(x)
	= \sum_{i=0}^n a_i x^i
	\neq0, \qquad
	g(x)
	= \sum_{i=0}^m b_i x^i
	\neq0,
\end{equation*}
其中\(a_n\neq0,b_m\neq0\).
于是\(\deg f=n,
\deg g=m\).
不妨设\(n \geq m\).
根据定义\begin{equation*}
	f(x) \pm g(x)
	= \sum_{i=0}^n (a_i \pm b_i) x^i,
\end{equation*}
因此\begin{equation*}
	\deg(f \pm g)
	\leq n
	= \max\{
		\deg f,
		\deg g
	\}.
\end{equation*}

又因为\(a_n b_m \neq 0\),
因此\(a_n b_m x^{n+m}\)是\(f(x) g(x)\)的首项,
从而\begin{equation*}
	\deg(fg) = n+m = \deg f + \deg g.
	\qedhere
\end{equation*}
\end{proof}
\end{proposition}

在\(K[x]\)中,我们根据多项式的乘法的定义,以及多项式的加法和乘法满足的运算法则,有
\begin{equation}\label{equation:多项式.示例公式1}
	(2x+3)(x+5)
	=2x^2+10x+3x+15
	=2x^2+13x+15.
\end{equation}

设\(\vb{A} \in M_n(K)\),在\(K[\vb{A}]\)中,根据矩阵乘法的定义及其分配律等运算法则,有
\begin{equation}\label{equation:多项式.示例公式2}
	(2\vb{A}+3\vb{E})(\vb{A}+5\vb{E})
	=2\vb{A}^2+10\vb{A}\vb{E}+3\vb{E}\vb{A}+(3\vb{E})(5\vb{E})
	=2\vb{A}^2+13\vb{A}+15\vb{E}.
\end{equation}

以上两式的计算过程类似.
这促使我们设想:
能不能不必进行\cref{equation:多项式.示例公式2} 的计算过程,
而从\cref{equation:多项式.示例公式1}
的计算结果直接得出\cref{equation:多项式.示例公式2} 呢?

\(K[x]\)中所有零次多项式添上零多项式组成的集合\(S\),
对于多项式的减法与乘法封闭,
因此\(S\)是\(K[x]\)的一个子环.
显然\(K[x]\)的单位元\(1\)属于\(S\),
从而\(1\)也是\(S\)的单位元.
我们可以建立数域\(K\)到\(S\)的一个映射\(\sigma\):
让非零数\(a\)对应于零次多项式\(a\),
让数\(0\)对应到零次多项式,
可以证明\(\sigma\)是一个同构映射.

给定\(\vb{A}\in M_n(K)\),
形如\begin{equation*}
	a_m \vb{A}^m + a_{m-1} \vb{A}^{m-1} + \dotsb + a_1 \vb{A} + a_0 \vb{E}
\end{equation*}的表达式称为“数域\(K\)上矩阵\(\vb{A}\)的多项式”,
其中\(m\)是非负整数,
\(\vb{E}\)是\(n\)阶单位矩阵,
\(a_i \in K\ (i=0,1,\dotsc,m)\).
把数域\(K\)上矩阵\(\vb{A}\)的所有多项式组成的集合记作\(K[\vb{A}]\),即\begin{equation*}
	K[\vb{A}]
	\defeq
	\Set{
		a_m \vb{A}^m + a_{m-1} \vb{A}^{m-1} + \dotsb + a_1 \vb{A} + a_0 \vb{E}
		\given
		m \in \mathbb{N}
		\land
		a_i \in K\ (i=0,1,\dotsc,m)
	}.
\end{equation*}

设\(f(\vb{A})=\sum_{i=0}^m a_i \vb{A}^i\),
\(g(\vb{A})=\sum_{i=0}^n b_i \vb{A}^i\),
\(m \geq n\).
从矩阵的运算法则可以得到
\begin{gather}
	f(\vb{A})-g(\vb{A})
	= \sum_{i=0}^m (a_i - b_i) \vb{A}^i, \\
	f(\vb{A}) g(\vb{A})
	= \sum_{s=0}^{m+n} \left( \sum_{i+j=s} a_i b_j \right) \vb{A}^s.
	\label{equation:多项式.矩阵多项式.乘法}
\end{gather}
因此\(K[\vb{A}]\)是\(M_n(K)\)的一个子环.
从\cref{equation:多项式.矩阵多项式.乘法} 容易看出\begin{equation*}
	f(\vb{A}) g(\vb{A}) = g(\vb{A}) f(\vb{A});
\end{equation*}
又由于\(\vb{E} \in K[\vb{A}]\),
所以\(K[\vb{A}]\)是有单位元的交换环.

\(K[\vb{A}]\)中所有数量矩阵组成的集合\(W\),
对于矩阵的减法与乘法封闭,
因此\(W\)是\(K[\vb{A}]\)的一个子环.
显然\(\vb{E} \in W\).
我们可以建立数域\(K\)到\(W\)的一个映射\(\tau\colon a \mapsto a \vb{E}\).
显然\(\tau\)是同构映射.

现在我们就可以回答前面提出的问题了.
因为\(K\)与\(W\)同构,所以我们可以用矩阵\(\vb{A}\)代入\(x\),
再把每一项的系数换成它在\(K\)到\(W\)的环同构映射\(\tau\)下的像,
就直接得到了\cref{equation:多项式.示例公式2}.

\begin{theorem}\label{theorem:多项式.多项式环的同构映射}
%@see: 《高等代数(第三版 下册)》(丘维声) P7 定理4
设\(K\)是一个数域,\(R\)是一个有单位元的交换环,
并且\(K\)到\(R\)的一个子环\(R'\)有一个环同构映射\(\tau\).
对于任意给定\(t \in R\),令\begin{equation*}
	\sigma_t\colon
	K[x] \to R,
	f(x)=\sum_{i=0}^n a_i x^i \mapsto \sum_{i=0}^n \tau(a_i) t^i \defeq f(t),
\end{equation*}
则\(\sigma_t\)是\(K[x]\)到\(R\)的映射;
并且\(\sigma_t\)保持加法与乘法运算,即\begin{equation*}
	f(x)+g(x)=h(x) \land f(x) g(x) = p(x)
	\implies
	f(t)+g(t)=h(t) \land f(t) g(t) = p(t);
\end{equation*}
此外,\(\sigma_t(x) = t\).
我们把映射\(\sigma_t\)叫做“\(x\)用\(t\)代入”.
\end{theorem}

我们把\cref{theorem:多项式.多项式环的同构映射}
称为“一元多项式环\(K[x]\)的通用性质”.

\cref{theorem:多项式.多项式环的同构映射} 告诉我们,
如果\(R\)是有单位元的交换环,
且\(R\)有一个子环\(R_1\)满足\cref{theorem:多项式.多项式环的同构映射} 的条件
(这时我们称“\(R\)可看成是\(K\)的一个\DefineConcept{扩环}”),
那么一元多项式环\(K[x]\)中所有通过加法与乘法表示的关系,
在不定元\(x\)用\(R\)的任一元素\(t\)代入后仍然保持.
因此我们只要把一元多项式环\(K[x]\)中有关加法与乘法的等式研究清楚了,
通过不定元\(x\)用环\(R\)中任一元素\(t\)代入,就可以得到环\(R\)中有关加法与乘法的等式.
这就是一元多项式环\(K[x]\)的通用性质的含义.

从前面的讨论知道,
\(K[x],K[\vb{A}]\)都可以作为\cref{theorem:多项式.多项式环的同构映射} 中的环\(R\).
因此,不定元\(x\)可以用\(x\)的任一多项式代入,
也可以用矩阵\(\vb{A}\)的任一多项式代入,
从\(K[x]\)中已知的有关加法和乘法的等式,
得到\(K[x]\)中另一些有关加法和乘法的等式,
或者得到\(K[\vb{A}]\)中一些有关加法和乘法的等式.

\begin{example}
%@see: 《高等代数(第三版 下册)》(丘维声) P8 例1
设\(\vb{B}\)是数域\(K\)上的\(n\)阶幂零矩阵,
其幂零指数为\(l\).
令\(\vb{A}=\vb{E}+k\vb{B}\ (k \in K)\).
证明:\(\vb{A}\)可逆,并且求\(\vb{A}^{-1}\).
\begin{proof}
在\(K[x]\)中直接计算可得\begin{equation*}
	(1-x)(1+x+x^2+\dotsb+x^{l-1})
	=1-x^l.
\end{equation*}
不定元\(x\)用\(-k\vb{B}\)代入,就可以从上式得到\begin{equation*}
	(\vb{E}+k\vb{B})[\vb{E}+(-k\vb{B})+(-k\vb{B})^2+\dotsb+(-k\vb{B})^{l-1}]
	=\vb{E}-(-k\vb{B})^l.
\end{equation*}
由于\(\vb{B}^l=\vb0\),
因此从上式得\begin{equation*}
	(\vb{E}+k\vb{B})[\vb{E}-k\vb{B}+k^2\vb{B}^2+\dotsb+(-k)^{l-1}\vb{B}^{l-1}]
	=\vb{E}.
\end{equation*}
上式表明\(\vb{E}+k\vb{B}\)可逆,
并且\begin{equation*}
	(\vb{E}+k\vb{B})^{-1}
	= \vb{E}-k\vb{B}+k^2\vb{B}^2+\dotsb+(-k)^{l-1}\vb{B}^{l-1}.
	\qedhere
\end{equation*}
\end{proof}
\end{example}
