\section{多元多项式环}
\subsection{多元多项式}
\begin{definition}
%@see: 《高等代数(第三版 下册)》(丘维声) P50 定义1
设\(K\)是一个数域,
用不属于\(K\)的\(n\)个符号\(\AutoTuple{x}{n}\)作表达式\begin{equation*}
	\sum_{\AutoTuple{i}{n}}
	a_{i_1 \dotsm i_n}
	x_1^{i_1} \dotsm x_n^{i_n},
\end{equation*}
其中\(a_{i_1 \dotsm i_n} \in K\),
\(\AutoTuple{i}{n}\)是非负整数,
上式中的每一项称为一个\DefineConcept{单项式},
上式称为\DefineConcept{数域\(K\)上的\(n\)元多项式}.
如果它具有下述性质:
只有有限多个单项式的系数不为零,
并且两个这种形式的表达式相等当且仅当它们除去系数为零的单项式外含有完全相同的单项式,
而系数为零的单项式允许任意删去或添入.
这时,符号\(\AutoTuple{x}{n}\)称为\(n\)个\DefineConcept{无关不定元}.

在数域\(K\)上的\(n\)元多项式中,
如果两个单项式的幂指数都对应相等,
则称这两个单项式为\DefineConcept{同类项}.
我们约定\(n\)元多项式中的单项式都是不同类的,
即要把同类项合并成一项.

如果数域\(K\)上一个\(n\)元多项式的所有系数全为零,
则称它为\DefineConcept{零多项式},记为\(0\).

我们把\(i_1+\dotsb+i_n\)
称为“单项式\(a_{i_1 \dotsm i_n}
x_1^{i_1} \dotsm x_n^{i_n}\)的\DefineConcept{次数}”.

一个\(n\)元多项式\(f(\AutoTuple{x}{n})\)的系数不为零的单项式的次数的最大值,
称为“\(f(\AutoTuple{x}{n})\)的\DefineConcept{次数}”.

零多项式的全次数规定为\(-\infty\).
\end{definition}

\begin{example}
\(5x_1^4+3x_1^3x_2+2x_1x_2x_3^2+x_2^3+x_2x_3\)
是3元4次多项式,
其中单项式\(5x_1^4,3x_1^3x_2,2x_1x_2x_3^2\)的次数都是4.
\end{example}

数域\(K\)上所有\(n\)元多项式组成的集合,
记作\(K[\AutoTuple{x}{n}]\).

在\(K[\AutoTuple{x}{n}]\)中定义加法与乘法如下:
\begin{gather}
	\begin{split}
		&\hspace{-20pt}
		\sum_{\AutoTuple{i}{n}}
		a_{i_1 \dotsm i_n}
		x_1^{i_1} \dotsm x_n^{i_n}
		+
		\sum_{\AutoTuple{i}{n}}
		b_{i_1 \dotsm i_n}
		x_1^{i_1} \dotsm x_n^{i_n} \\
		&\defeq
		\sum_{\AutoTuple{i}{n}}
		(a_{i_1 \dotsm i_n} + b_{i_1 \dotsm i_n})
		x_1^{i_1} \dotsm x_n^{i_n},
	\end{split} \\
	\begin{split}
		&\hspace{-20pt}
		\left(
		\sum_{\AutoTuple{i}{n}}
		a_{i_1 \dotsm i_n}
		x_1^{i_1} \dotsm x_n^{i_n}
		\right) \left(
		\sum_{\AutoTuple{j}{n}}
		b_{j_1 \dotsm j_n}
		x_1^{j_1} \dotsm x_n^{j_n}
		\right) \\
		&\defeq
		\sum_{\AutoTuple{s}{n}}
		c_{s_1 \dotsm s_n}
		x_1^{s_1} \dotsm x_n^{s_n},
	\end{split}
\end{gather}
其中\begin{equation}
	c_{s_1 \dotsm s_n}
	= \sum_{i_1+j_1=s_1}
	\sum_{i_2+j_2=s_2}
	\dotso
	\sum_{i_n+j_n=s_n}
	a_{i_1 \dotsm i_n}
	b_{j_1 \dotsm j_n}.
\end{equation}
不难验证\(K[\AutoTuple{x}{n}]\)对于如上定义的加法与乘法成为一个环.
它的零元是零多项式.
它有单位元,即零次多项式\(1\).
它是交换环.
我们把这个环称为\DefineConcept{数域\(K\)上的\(n\)元多项式环}.

显然有\begin{equation}
	\deg(f+g)
	\leq
	\max\{\deg f,\deg g\}.
\end{equation}

先来对\(n\)元多项式\(f(\AutoTuple{x}{n})\)的各项规定一个排列顺序,
从而给出首项的概念.

每一类单项式\(a_{i_1 \dotsm i_n} x_1^{i_1} \dotsm x_n^{i_n}\)
对应一个\(n\)元有序非负整数组\((\AutoTuple{i}{n})\),
这个对应是双射.
为了给出各类单项式之间的一个排列顺序的方法,
就只需要对\(n\)元有序非负整数组定义一个先后顺序.

对于两个\(n\)元有序非负整数组
\((\AutoTuple{i}{n})\)
和\((j_1,\dotsc,j_n)\),
如果\begin{equation*}
	i_1=j_1,
	i_2=j_2,
	\dotsc,
	i_{s-1}=j_{s-1},
	i_s>j_s
	\quad(1\leq s\leq n)
\end{equation*}
则称“\((\AutoTuple{i}{n})\)~\DefineConcept{先于}~\((j_1,\dotsc,j_n)\)”,
记作\((\AutoTuple{i}{n})>(j_1,\dotsc,j_n)\).

由上述定义立即看出,
对于任意两个\(n\)元有序非负整数组
\((\AutoTuple{i}{n})\)
和\((j_1,\dotsc,j_n)\),
关系\begin{gather*}
	(\AutoTuple{i}{n})>(j_1,\dotsc,j_n), \\
	(\AutoTuple{i}{n})=(j_1,\dotsc,j_n), \\
	(j_1,\dotsc,j_n)>(\AutoTuple{i}{n})
\end{gather*}
中有且仅有一个成立.

这里关系“\(>\)”具有传递性,
即,
如果\((\AutoTuple{i}{n})>(j_1,\dotsc,j_n)\)
且\((j_1,\dotsc,j_n)>(k_1,\dotsc,k_n)\),
那么\((\AutoTuple{i}{n})>(k_1,\dotsc,k_n)\).
这是因为\(i_l-k_l=(i_l-j_l)+(j_l-k_l)\).

\begin{example}
由\((4,2,3,3)>(4,2,2,4)\)和\((4,2,2,4)>(4,1,4,3)\)
可得\((4,2,3,3)>(4,1,4,3)\).
\end{example}

这样我们的确给出了\(n\)元有序非负整数组之间的一个顺序.
相应地,\(n\)元各类单项式之间也有了一个先后顺序.
这种排列顺序的方法是模仿字典中单词的排列原则给出的,
因而称之为\DefineConcept{字典排列法}.

\begin{example}
多项式\(2x_1^4x_2x_3+x_1x_2^5x_3+6x_1^3\)
按字典排列法写出来就是
\(2x_1^4x_2x_3+6x_1^3+x_1x_2^5x_3\).
\end{example}

按字典排列法写出来的第一个系数不为零的单项式称为\(n\)元多项式的\DefineConcept{首项}.

\begin{example}
多项式\(2x_1^4x_2x_3+x_1x_2^5x_3+6x_1^3\)的首项
是\(2x_1^4x_2x_3\).
要注意,首项不一定具有最大的次数.
多项式\(2x_1^4x_2x_3+x_1x_2^5x_3+6x_1^3\)的次数是7,
而它的首项的次数是6.
\end{example}

\begin{theorem}\label{theorem:多项式.多元多项式环.两个非零多项式的乘积的首项等于它们的首项的乘积}
%@see: 《高等代数(第三版 下册)》(丘维声) P52 定理1
在\(K[\AutoTuple{x}{n}]\)中两个非零多项式的乘积的首项等于它们的首项的乘积.
\begin{proof}
设\(f(\AutoTuple{x}{n}),g(\AutoTuple{x}{n})\)是\(K[\AutoTuple{x}{n}]\)中两个非零多项式.
设\(f(\AutoTuple{x}{n})\)的首项是\(a x_1^{p_1} x_2^{p_2} \dotsm x_n^{p_n}\ (a\neq0)\),
\(g(\AutoTuple{x}{n})\)的首项是\(b x_1^{q_1} x_2^{q_2} \dotsm x_n^{q_n}\ (b\neq0)\).
为了证明\(fg\)的首项是
\(ab x_1^{p_1+q_1} x_2^{p_2+q_2} \dotsm x_n^{p_n+q_n}\),
只要证明\begin{equation*}
	(p_1+q_1,p_2+q_2,\dotsc,p_n+q_n)
\end{equation*}先于\(fg\)中其他单项式的幂指数组就行了.
\(fg\)的其他单项式的幂指数组只有三种可能情形:\begin{equation*}
	(p_1+j_1,p_2+j_2,\dotsc,p_n+j_n)
\end{equation*}或\begin{equation*}
	(i_1+q_1,i_2+q_2,\dotsc,i_n+q_n)
\end{equation*}或\begin{equation*}
	(i_1+j_1,i_2+j_2,\dotsc,i_n+j_n),
\end{equation*}
其中\begin{equation*}
	(p_1,p_2,\dotsc,p_n)
	>
	(i_1,i_2,\dotsc,i_n), \qquad
	(q_1,q_2,\dotsc,q_n)
	>
	(j_1,j_2,\dotsc,j_n).
\end{equation*}
显然有\begin{gather*}
	(p_1+q_1,p_2+q_2,\dotsc,p_n+q_n)
	>
	(i_1+q_1,i_2+q_2,\dotsc,i_n+q_n), \\
	(i_1+q_1,i_2+q_2,\dotsc,i_n+q_n)
	>
	(i_1+j_1,i_2+j_2,\dotsc,i_n+j_n),
\end{gather*}
于是由传递性得\begin{equation*}
	(p_1+q_1,p_2+q_2,\dotsc,p_n+q_n)
	>
	(i_1+j_1,i_2+j_2,\dotsc,i_n+j_n).
\end{equation*}
这就证明了
\(ab x_1^{p_1+q_1} x_2^{p_2+q_2} \dotsm x_n^{p_n+q_n}\)
不可能与\(fg\)中其他的单项式相消,
而且它先于\(fg\)中其他的单项式,
它就是\(fg\)的首项.
\end{proof}
\end{theorem}

从\cref{theorem:多项式.多元多项式环.两个非零多项式的乘积的首项等于它们的首项的乘积}
可以推得以下三个命题.

\begin{proposition}
%@see: 《高等代数(第三版 下册)》(丘维声) P52 定理1
在\(K[\AutoTuple{x}{n}]\)中两个非零多项式的乘积仍是非零多项式.
\end{proposition}

\begin{proposition}
%@see: 《高等代数(第三版 下册)》(丘维声) P52 定理1
\(K[\AutoTuple{x}{n}]\)是无零因子环.
\end{proposition}

\begin{proposition}
%@see: 《高等代数(第三版 下册)》(丘维声) P52 定理1
在\(K[\AutoTuple{x}{n}]\)中,消去律成立.
\end{proposition}

\begin{corollary}
%@see: 《高等代数(第三版 下册)》(丘维声) P52 推论2
在\(K[\AutoTuple{x}{n}]\)中,
如果\(f_i\neq0\ (i=1,2,\dotsc,m)\),
则\(f_1 f_2 \dotsm f_m\)的首项等于它们的首项的乘积.
\begin{proof}
对\cref{theorem:多项式.多元多项式环.两个非零多项式的乘积的首项等于它们的首项的乘积}
运用数学归纳法可以证得.
\end{proof}
\end{corollary}

\subsection{齐次多项式}
\begin{definition}
%@see: 《高等代数(第三版 下册)》(丘维声) P53 定义2
设\(g(\AutoTuple{x}{n})\)是数域\(K\)上的\(n\)元多项式.
如果\(g(\AutoTuple{x}{n})\)的每个系数不为零的单项式都是\(m\)次的,
则称其为~\DefineConcept{\(m\)次齐次多项式}.
\end{definition}

\begin{example}
\(2x_1^4+3x_1^2x_2x_3+x_1x_2x_3^2\)是一个4次齐次多项式.
\end{example}

\begin{proposition}
\(K[\AutoTuple{x}{n}]\)中任意两个齐次多项式的乘积仍是齐次多项式,
它的次数等于这两个多项式的次数的和.
\end{proposition}

对于任何一个\(n\)元多项式\(f(\AutoTuple{x}{n})\),
如果把\(f\)中所有次数相同的单项式并在一起,
则\(f\)可以唯一地表示成\begin{equation*}
	f(\AutoTuple{x}{n})
	=\sum_{i=0}^m
	f_i(\AutoTuple{x}{n}),
\end{equation*}
其中\(m=\deg f\),
\(f_i(\AutoTuple{x}{n})\)是\(i\)次齐次多项式,
它称为“\(f(\AutoTuple{x}{n})\)的~\DefineConcept{\(i\)次齐次成分}”.

\begin{theorem}
%@see: 《高等代数(第三版 下册)》(丘维声) P53 定理3
设\(f(\AutoTuple{x}{n}),g(\AutoTuple{x}{n}) \in K[\AutoTuple{x}{n}]\),
则\begin{equation}
	\deg(fg)=\deg f+\deg g.
\end{equation}
\begin{proof}
若\(f,g\)中有一个是零多项式,
则\(\deg(fg)=\deg f+\deg g\)成立.

现在设\(f\neq0,g\neq0,\deg f=m,\deg g=s\),
则\begin{equation*}
	f=f_0+f_1+\dotsb+f_m, \qquad
	g=g_0+g_1+\dotsb+g_s,
\end{equation*}
其中\(f_i\)是\(f\)的\(i\)次齐次成分,
\(g_j\)是\(g\)的\(j\)次齐次成分,
\(f_m\neq0\),
\(g_s\neq0\).
我们有\begin{equation*}
	fg
	=(f_0 g_0+\dotsb+f_0 g_s)
	+(f_1 g_0+\dotsb+f_1 g_s)
	+\dotsb
	+(f_m g_0+\dotsb+f_m g_s),
\end{equation*}
其中\(f_i g_j\)是\(i+j\)次齐次多项式.
因为\(f_m\neq0,g_s\neq0\),
所以\(f_m g_s\neq0\).
于是\(f_m g_s\)是\(m+s\)次齐次多项式.
从而有\(\deg(fg)=m+s=\deg f+\deg g\).
\end{proof}
\end{theorem}

\begin{remark}
当\(n>1\)时,
\(K[\AutoTuple{x}{n}]\)中没有带余除法,
但是唯一因式分解定理仍然成立.
\end{remark}

和数域\(K\)上的一元多项式环\(K[x]\)具有通用性质一样,
数域\(K\)上的\(n\)元多项式环\(K[\AutoTuple{x}{n}]\)也具有通用性质:
\begin{theorem}
%@see: 《高等代数(第三版 下册)》(丘维声) P53 定理4
设\(K\)是一个数域,
\(R\)是一个带有单位元\(e\)的交换环,
并且\(R\)有一个子环\(R_1\)(含有\(e\)),
\(\tau\)是\(K\)到\(R_1\)的一个环同构映射,
\(\AutoTuple{t}{n}\)是\(R\)的元素,
令\begin{equation*}
	\sigma_{\AutoTuple{t}{n}}
	\colon
	K[\AutoTuple{x}{n}] \to R,
	f(\AutoTuple{x}{n}) \mapsto f(\AutoTuple{t}{n}),
\end{equation*}
其中\begin{equation*}
	f(\AutoTuple{x}{n})
	= \sum_{\AutoTuple{i}{n}}
	a_{i_1 \dotsm i_n}
	x_1^{i_1} \dotsm x_n^{i_n},
\end{equation*}
而\begin{equation*}
	f(\AutoTuple{t}{n})
	= \sum_{\AutoTuple{i}{n}}
	\tau(a_{i_1 \dotsm i_n})
	t_1^{i_1} \dotsm t_n^{i_n},
\end{equation*}
则\(\sigma_{\AutoTuple{t}{n}}\)是\(K[\AutoTuple{x}{n}]\)到\(R\)的一个映射,
它满足\(\sigma_{\AutoTuple{t}{n}}(x_i)=t_i\ (i=1,2,\dotsc,n)\),
且它保持加法、乘法运算,
即\begin{gather*}
	f(\AutoTuple{x}{n})
	+g(\AutoTuple{x}{n})
	=h(\AutoTuple{x}{n}), \\
	f(\AutoTuple{x}{n})
	g(\AutoTuple{x}{n})
	=p(\AutoTuple{x}{n}),
\end{gather*}
那么\begin{gather*}
	f(\AutoTuple{t}{n})
	+g(\AutoTuple{t}{n})
	=h(\AutoTuple{t}{n}), \\
	f(\AutoTuple{t}{n})
	g(\AutoTuple{t}{n})
	=p(\AutoTuple{t}{n}).
\end{gather*}
\rm
我们把映射\(\sigma_{\AutoTuple{t}{n}}\)称为
“\(\AutoTuple{x}{n}\)用\(\AutoTuple{t}{n}\)代入”.
\begin{proof}
证明过程参考\cref{theorem:多项式.多项式环的同构映射}.
\end{proof}
\end{theorem}

\(K[\AutoTuple{x}{n}]\)中所有零次多项式
添上零多项式组成的子集是
\(K[\AutoTuple{x}{n}]\)的一个子环,
它与\(K\)是环同构的,
因此\(\AutoTuple{x}{n}\)
可以用\(K[\AutoTuple{x}{n}]\)中任意\(n\)个元素代入,
这种代入是保持加法与乘法运算.

特别重要的一种情形是:
\(\AutoTuple{x}{n}\)
用\(K\)中任意\(n\)个元素
\(\AutoTuple{c}{n}\)代入.
由此我们可引进多元多项式函数的概念.

设\(f(\AutoTuple{x}{n})\)是数域\(K\)上的一个\(n\)元多项式,
对于\(K\)中任意\(n\)个元素\(\AutoTuple{c}{n}\),
将\(\AutoTuple{x}{n}\)用\(\AutoTuple{c}{n}\)代入,
得\(f(\AutoTuple{c}{n}) \in K\).
于是\(n\)元多项式\(f(\AutoTuple{x}{n})\)确定了
从\(K^n\)到\(K\)的一个映射
(即\(K\)上的\(n\)元函数),
仍用\(f\)表示,即\begin{equation*}
	f\colon
	K^n \to K,
	(\AutoTuple{c}{n})
	\mapsto
	f(\AutoTuple{c}{n}).
\end{equation*}
这种由数域\(K\)上的\(n\)元多项式确定的\(K\)上的\(n\)元函数
称为\DefineConcept{数域\(K\)上的\(n\)元多项式函数}.

对于数域\(K\)上的两个\(n\)元多项式
\(f(\AutoTuple{x}{n})\)和\(g(\AutoTuple{x}{n})\),
如果它们相等,
则它们确定的\(n\)元多项式函数\(f\)与\(g\)也相等;
反之亦然.

\begin{lemma}\label{theorem:多项式.多元多项式环.引理1}
%@see: 《高等代数(第三版 下册)》(丘维声) P54 引理1
设\(h(\AutoTuple{x}{n})\)是数域\(K\)上的一个\(n\)元多项式,
如果\(h(\AutoTuple{x}{n})\neq0\),
则\(h\)不是零函数.
\begin{proof}
对不定元的数目\(n\)运用数学归纳法.
当\(n=1\)时,
由于数域\(K\)上非零的一元多项式\(h(x)\)给出的函数不是零函数,
因此存在\(c \in K\)使得\(h(c)\neq0\).

假设命题对\(K[\AutoTuple{x}{n-1}]\)中的多项式成立,
现在看\(K[\AutoTuple{x}{n}]\)中的多项式\(h(\AutoTuple{x}{n})\).
把\(h(\AutoTuple{x}{n})\)写成\begin{equation*}
	h(\AutoTuple{x}{n})
	=u_0(\AutoTuple{x}{n-1})
	+u_1(\AutoTuple{x}{n-1})
	+\dotsb
	+u_s(\AutoTuple{x}{n-1}) x_n^s,
\end{equation*}
其中\(u_i(\AutoTuple{x}{n-1}) \in K[\AutoTuple{x}{n-1}]\ (i=0,1,\dotsc,s)\),
且\(u_s(\AutoTuple{x}{n-1})\neq0\).
根据归纳假设,\(u_s\)不是零函数,
因此存在\(\AutoTuple{c}{n-1} \in K\)
使得\(u_s(\AutoTuple{c}{n-1})\neq0\).
于是一元多项式环\(K[x_n]\)中的多项式\begin{equation*}
	h(\AutoTuple{c}{n-1},x_n)
	=u_0(\AutoTuple{c}{n-1})
	+u_1(\AutoTuple{c}{n-1}) x_n
	+\dotsb
	+u_s(\AutoTuple{c}{n-1}) x_n^s
\end{equation*}是非零多项式,
因此存在\(c_n \in K\),
使得\begin{equation*}
	h(\AutoTuple{c}{n})
	=u_0(\AutoTuple{c}{n-1})
	+u_1(\AutoTuple{c}{n-1}) c_n
	+\dotsb
	+u_s(\AutoTuple{c}{n-1}) c_n^s
	\neq0.
	\qedhere
\end{equation*}
\end{proof}
\end{lemma}

\begin{theorem}
%@see: 《高等代数(第三版 下册)》(丘维声) P55 定理5
设\(f(\AutoTuple{x}{n}),g(\AutoTuple{x}{n}) \in K[\AutoTuple{x}{n}]\).
如果多项式\(f\)与\(g\)不相等,
则由它们确定的\(n\)元多项式函数\(f\)与\(g\)也不相等.
\begin{proof}
考虑多项式\(
	h(\AutoTuple{x}{n})
	=f(\AutoTuple{x}{n})
	-g(\AutoTuple{x}{n})
\).
如果多项式\(f\)与\(g\)不相等,
则\(h(\AutoTuple{x}{n})\neq0\).
根据\cref{theorem:多项式.多元多项式环.引理1},
\(h\)不是零函数,
于是存在\(\AutoTuple{c}{n} \in K\)
使得\(h(\AutoTuple{c}{n})\neq0\).
\(\AutoTuple{x}{n}\)用\(\AutoTuple{c}{n}\)代入,
用上述式子可以推出\(f(\AutoTuple{c}{n}) \neq g(\AutoTuple{c}{n})\),
所以映射\(f\)与\(g\)不相等.
\end{proof}
\end{theorem}

我们把数域\(K\)所有\(n\)元多项式函数组成的集合记作\(K_{npol}\),
在这个集合中规定加法与乘法如下:
对于\(\forall(\AutoTuple{c}{n}) \in K^n\)有\begin{gather*}
	(f+g)(\AutoTuple{c}{n})
	\defeq
	f(\AutoTuple{c}{n})+g(\AutoTuple{c}{n}), \\
	(fg)(\AutoTuple{c}{n})
	\defeq
	f(\AutoTuple{c}{n}) g(\AutoTuple{c}{n}).
\end{gather*}
容易验证\(K_{npol}\)是一个环,
我们把它称为\DefineConcept{数域\(K\)上的\(n\)元多项式函数环}.
容易证明:
数域\(K\)上的\(n\)元多项式环\(K[\AutoTuple{x}{n}]\)
与\(K\)上的\(n\)元多项式函数环\(K_{npol}\)是同构的.
因此我们可以把数域\(K\)上的\(n\)元多项式
与\(K\)上的\(n\)元多项式函数等同看待.

设\(f(\AutoTuple{x}{n}) \in K[\AutoTuple{x}{n}]\),
对于\(\AutoTuple{c}{n} \in K\),
如果\(f(\AutoTuple{c}{n})=0\),
则称“\((\AutoTuple{c}{n})\)是\(f(\AutoTuple{x}{n})\)的一个\DefineConcept{零点}”.
当\(K\)取实数域时,
若\(n=2\),
则\(f(x,y)\)的零点组成的集合就是平面上的一条\DefineConcept{代数曲线};
若\(n=3\),
则\(f(x,y,z)\)的零点组成的集合就是空间中的一个\DefineConcept{代数曲面}.
研究数域\(K\)上一组\(n\)元多项式的公共零点组成的集合,
就是代数几何的一个基本内容.

利用不定式\(\AutoTuple{x}{n}\)用\(K[\AutoTuple{x}{n}]\)的\(n\)个元素代入是保持运算的,
我们可以得出齐次多项式的一个特征性质.
\begin{theorem}
%@see: 《高等代数(第三版 下册)》(丘维声) P56 定理6
设\(f(\AutoTuple{x}{n}) \in K[\AutoTuple{x}{n}]\)且\(f\neq0\),
\(m\)是一个非负整数,
则\(f(\AutoTuple{x}{n})\)是\(m\)次齐次多项式的充分必要条件是:
对于\(\forall t \in K\),有\begin{equation*}
	f(t x_1,\dotsc,t x_n)
	= t^m f(\AutoTuple{x}{n}).
\end{equation*}
\begin{proof}
必要性.
设\(f(\AutoTuple{x}{n})\)是\(m\)次齐次多项式,
即\begin{equation*}
	f(\AutoTuple{x}{n})
	=\sum_{\AutoTuple{i}{n}}
	a_{i_1 \dotsm i_n}
	x_1^{i_1} \dotsm x_n^{i_n},
	\qquad
	i_1+\dotsb+i_n=m.
\end{equation*}
任取\(t \in K\),
不定元\(\AutoTuple{x}{n}\)用\(t x_1,\dotsc,t x_n\)代入,
从上式得\begin{align*}
	f(t x_1,\dotsc,t x_n)
	&=\sum_{\AutoTuple{i}{n}}
	a_{i_1 \dotsm i_n}
	(t x_1)^{i_1} \dotsm (t x_n)^{i_n} \\
	&=t^{i_1+\dotsb+i_n}
	\sum_{\AutoTuple{i}{n}}
	a_{i_1 \dotsm i_n}
	x_1^{i_1} \dotsm x_n^{i_n} \\
	&=t^m f(\AutoTuple{x}{n}).
\end{align*}

充分性.
设对\(\forall t \in K\),
有\(f(t x_1,\dotsc,t x_n)
= t^m f(\AutoTuple{x}{n})\).
设\(\deg f=s\),
则可将\(f(\AutoTuple{x}{n})\)写成\begin{equation*}
	f=f_0+\dotsb+f_s,
\end{equation*}
其中\(f_i\)是\(f\)的\(i\)次齐次成分.
任取\(t \in K\),
\(\AutoTuple{x}{n}\)用\(t x_1,\dotsc,t x_n\)代入,
从上式得出\begin{equation*}
	f(t x_1,\dotsc,t x_n)
	=f_0(t x_1,\dotsc,t x_n)
	+f_1(t x_1,\dotsc,t x_n)
	+\dotsb
	+f_s(t x_1,\dotsc,t x_n),
\end{equation*}
那么由必要性的结论以及充分性的假设得\begin{equation*}
	t^m f(\AutoTuple{x}{n})
	=f_0(\AutoTuple{x}{n})
	+t f_1(\AutoTuple{x}{n})
	+\dotsb
	+t^s f_s(\AutoTuple{x}{n}).
\end{equation*}
于是根据\(K[\AutoTuple{x}{n}]\)中两个多项式相等的规定,
我们得到\begin{equation*}
	t^m f_i(\AutoTuple{x}{n})
	= t^i f_i(\AutoTuple{x}{n}),
	\quad
	i=0,1,\dotsc,s.
\end{equation*}
对于\(\forall i \in \{0,1,\dotsc,s\}-\{m\}\),
如果\(f_i\neq0\),
则从上式两边消去\(f_i\)得\((\forall t \in K)[t^m=t^i]\).
由\cref{theorem:多项式.多项式函数是否相等取决于多项式是否相等}
得\(x^m=x^i\),
这与\(i \neq m\)矛盾!
因此当\(i \neq m\)时必有\(f_i=0\).
所以\(f=f_m\).
已知\(f\neq0\),
因此\(f\)是\(m\)次齐次多项式.
\end{proof}
\end{theorem}

我们指出,设\(R\)是\(K\)的一个交换扩环,
任取\(R\)中\(n\)个元素\(\AutoTuple{t}{n}\),
对于\(f(\AutoTuple{x}{n}) \in K[\AutoTuple{x}{n}]\),
把\(\AutoTuple{x}{n}\)用\(\AutoTuple{t}{n}\)代入,
得到的\(f(\AutoTuple{t}{n})\)称为“\(\AutoTuple{t}{n}\)在\(K\)上的一个多项式”.
尽管从形式上看,
\(\AutoTuple{t}{n}\)的多项式\(f(\AutoTuple{t}{n})\)
与\(n\)元多项式\(f(\AutoTuple{x}{n})\)类似,
但是它们有本质不同:
\(\AutoTuple{t}{n}\)的一个多项式的表法可能不唯一,
而一个\(n\)元多项式的表法唯一.
