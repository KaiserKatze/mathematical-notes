\section{舒尔定理}
\begin{theorem}\label{theorem:逆矩阵.舒尔定理}
设\(\vb{A}\)是\(m\)阶可逆矩阵,
\(\vb{B},\vb{C},\vb{D}\)分别是\(m \times p, n \times m, n \times p\)矩阵,
则有\begin{gather}
	\begin{bmatrix}
		\vb{E}_m & \vb0 \\
		-\vb{C}\vb{A}^{-1} & \vb{E}_n
	\end{bmatrix}
	\begin{bmatrix}
		\vb{A} & \vb{B} \\
		\vb{C} & \vb{D}
	\end{bmatrix}
	= \begin{bmatrix}
		\vb{A} & \vb{B} \\
		\vb0 & \vb{D} - \vb{C} \vb{A}^{-1} \vb{B}
	\end{bmatrix},
	\\
	\begin{bmatrix}
		\vb{A} & \vb{B} \\
		\vb{C} & \vb{D}
	\end{bmatrix}
	\begin{bmatrix}
		\vb{E}_m & -\vb{A}^{-1} \vb{B} \\
		\vb0 & \vb{E}_p
	\end{bmatrix}
	= \begin{bmatrix}
		\vb{A} & \vb0 \\
		\vb{C} & \vb{D} - \vb{C} \vb{A}^{-1} \vb{B}
	\end{bmatrix},
	\\
	\begin{bmatrix}
		\vb{E}_m & \vb0 \\
		-\vb{C} \vb{A}^{-1} & \vb{E}_n
	\end{bmatrix}
	\begin{bmatrix}
		\vb{A} & \vb{B} \\
		\vb{C} & \vb{D}
	\end{bmatrix}
	\begin{bmatrix}
		\vb{E}_m & -\vb{A}^{-1} \vb{B} \\
		\vb0 & \vb{E}_p
	\end{bmatrix}
	= \begin{bmatrix}
		\vb{A} & \vb0 \\
		\vb0 & \vb{D} - \vb{C} \vb{A}^{-1} \vb{B}
	\end{bmatrix},
\end{gather}
\rm
其中\(\vb{D} - \vb{C} \vb{A}^{-1} \vb{B}\)
称为“矩阵\(\begin{bmatrix}
	\vb{A} & \vb{B} \\
	\vb{C} & \vb{D}
\end{bmatrix}\)
关于\(\vb{A}\)的\DefineConcept{舒尔补}(Schur complement)”.
%TODO proof
\end{theorem}
我们把\cref{theorem:逆矩阵.舒尔定理} 称为“舒尔定理”.

可以看到只要\(\vb{A}\)可逆,
就能通过初等分块矩阵直接对分块矩阵进行分块相似对角化,
而在操作过程中就把一些原本不为0的分块矩阵变成了零矩阵,
这个过程可以形象地称为“矩阵打洞”,
即让矩阵出现尽可能多的0.

利用\hyperref[theorem:逆矩阵.舒尔定理]{舒尔定理}可以证明下述行列式降阶定理:
\begin{theorem}[行列式降阶定理]\label{theorem:逆矩阵.行列式降阶定理}
设\(\vb{M} = \begin{bmatrix}
	\vb{A} & \vb{B} \\
	\vb{C} & \vb{D}
\end{bmatrix}\)是方阵.
\begin{itemize}
	\item 若\(\vb{A}\)可逆,则\begin{equation}\label{equation:逆矩阵.行列式降阶公式1}
		\abs{\vb{M}} = \abs{\vb{A}} \abs{\vb{D} - \vb{C} \vb{A}^{-1} \vb{B}}.
	\end{equation}

	\item 若\(\vb{D}\)可逆,则\begin{equation}\label{equation:逆矩阵.行列式降阶公式2}
		\abs{\vb{M}} = \abs{\vb{D}} \abs{\vb{A} - \vb{B} \vb{D}^{-1} \vb{C}}.
	\end{equation}

	\item 若\(\vb{A},\vb{D}\)均可逆,则\begin{equation}
		\abs{\vb{A}} \abs{\vb{D} - \vb{C} \vb{A}^{-1} \vb{B}}
		= \abs{\vb{D}} \abs{\vb{A} - \vb{B} \vb{D}^{-1} \vb{C}}.
	\end{equation}
\end{itemize}
\begin{proof}
由\cref{theorem:逆矩阵.舒尔定理,equation:行列式.广义三角阵的行列式1} 立即可得.
\end{proof}
\end{theorem}
\begin{remark}
这里给出一个有利于记忆\hyperref[theorem:逆矩阵.行列式降阶定理]{行列式降阶定理}的口诀:
对于\cref{equation:逆矩阵.行列式降阶公式1} 中等号右边的第二个行列式,
我们从\(\vb{D}\)出发,先以顺时针方向依次写出分块矩阵\(\vb{D},\vb{C},\vb{A},\vb{B}\),
再在第一个和第二个分块阵之间插入一个减号,最后取第三个分块阵的逆矩阵.
\cref{equation:逆矩阵.行列式降阶公式2} 也可以相同的记忆方法.
\end{remark}

\begin{example}\label{example:逆矩阵.行列式降阶定理的重要应用1}
设\(\vb{A} \in M_{s \times n}(K),
\vb{B} \in M_{n \times s}(K)\).
证明:\[
	\begin{vmatrix}
		\vb{E}_n & \vb{B} \\
		\vb{A} & \vb{E}_s
	\end{vmatrix}
	= \abs{\vb{E}_s - \vb{A} \vb{B}}
	= \abs{\vb{E}_n - \vb{B} \vb{A}}.
\]
\begin{proof}
由于\(\vb{E}_n,\vb{E}_s\)都是单位矩阵,必可逆,
那么由\hyperref[equation:逆矩阵.行列式降阶公式1]{行列式降阶定理}有\begin{equation*}
	\begin{vmatrix}
		\vb{E}_n & \vb{B} \\
		\vb{A} & \vb{E}_s
	\end{vmatrix}
	= \abs{\vb{E}_n} \abs{\vb{E}_s - \vb{A}(\vb{E}_n)^{-1}\vb{B}}
	= \abs{\vb{E}_s - \vb{A} \vb{B}}.
\end{equation*}
同理,由\hyperref[equation:逆矩阵.行列式降阶公式2]{行列式降阶定理}有\begin{equation*}
	\begin{vmatrix}
		\vb{E}_n & \vb{B} \\
		\vb{A} & \vb{E}_s
	\end{vmatrix}
	= \abs{\vb{E}_n - \vb{B} \vb{A}}.
\end{equation*}
因此\(\abs{\vb{E}_s - \vb{A} \vb{B}} = \abs{\vb{E}_n - \vb{B} \vb{A}}\).
\end{proof}
%\cref{example:单位矩阵与两矩阵乘积之差.单位矩阵与两矩阵乘积之差的秩}
%\cref{example:单位矩阵与两矩阵乘积之差.单位矩阵与两矩阵乘积之差的行列式}
\end{example}

\begin{example}
%@see: https://www.bilibili.com/video/BV1ki4y1a7dL/
设\(\vb{A},\vb{B},\vb{C},\vb{D} \in M_n(K)\),
且\(\vb{A} \vb{C} = \vb{C} \vb{A}\),
则\begin{equation*}
	\begin{vmatrix}
		\vb{A} & \vb{B} \\
		\vb{C} & \vb{D}
	\end{vmatrix}
	= \abs{\vb{A} \vb{D} - \vb{C} \vb{B}}.
\end{equation*}
\begin{proof}
当\(\vb{A}\)可逆时,由\hyperref[equation:逆矩阵.行列式降阶公式1]{行列式降阶定理}有\begin{align*}
	\begin{vmatrix}
		\vb{A} & \vb{B} \\
		\vb{C} & \vb{D}
	\end{vmatrix}
	&= \abs{\vb{A}} \abs{\vb{D} - \vb{C} \vb{A}^{-1} \vb{B}} \\
	&= \abs{\vb{A} (\vb{D} - \vb{C} \vb{A}^{-1} \vb{B})} \\
	&= \abs{\vb{A} \vb{D} - \vb{A} \vb{C} \vb{A}^{-1} \vb{B}} \\
	&= \abs{\vb{A} \vb{D} - \vb{C} \vb{A} \vb{A}^{-1} \vb{B}} \\
	&= \abs{\vb{A} \vb{D} - \vb{C} \vb{B}}.
	\qedhere
\end{align*}
%TODO 还没有证明“\(\vb{A}\)不可逆”的情形
%@credit: {1ef6baa6-d8ce-4b2c-97d3-f6156721c52f} 说可以用摄动法
\end{proof}
\end{example}

\begin{example}
\def\M{\vb{M}}
求解行列式\(\det \M\),其中\[
	\M = \begin{bmatrix}
		1+a_1 b_1 & a_1 b_2 & \dots & a_1 b_n \\
		a_2 b_1 & 1+a_2 b_2 & \dots & a_2 b_n \\
		\vdots & \vdots & & \vdots \\
		a_n b_1 & a_n b_2 & \dots & 1+a_n b_n
	\end{bmatrix}.
\]
\begin{solution}
记\(\vb\alpha = (\AutoTuple{a}{n})^T,
\vb\beta = (\AutoTuple{b}{n})\),
则\(\M = \vb{E}_n + \vb\alpha\vb\beta\).

注意到\(\M = \vb{E}_n + \vb\alpha\vb\beta\)形似某个矩阵的舒尔补,因此考虑下面的矩阵:\[
	\vb{A} = \begin{bmatrix}
		1 & -\vb\beta \\
		\vb\alpha & \vb{E}_n
	\end{bmatrix}.
\]由于\[
	\begin{bmatrix}
		1 & \vb0 \\
		-\vb\alpha & \vb{E}_n
	\end{bmatrix} \vb{A}
	= \begin{bmatrix}
		1 & \vb0 \\
		-\vb\alpha & \vb{E}_n
	\end{bmatrix}
	\begin{bmatrix}
		1 & -\vb\beta \\
		\vb\alpha & \vb{E}_n
	\end{bmatrix}
	= \begin{bmatrix}
		1 & -\vb\beta \\
		\vb0 & \M
	\end{bmatrix},
\]
且\[
	\begin{bmatrix}
		1 & \vb\beta \\
		\vb0 & \vb{E}_n
	\end{bmatrix} \vb{A}
	= \begin{bmatrix}
		1 & \vb\beta \\
		\vb0 & \vb{E}_n
	\end{bmatrix}
	\begin{bmatrix}
		1 & -\vb\beta \\
		\vb\alpha & \vb{E}_n
	\end{bmatrix}
	= \begin{bmatrix}
		1+\vb\beta\vb\alpha & \vb0 \\
		\vb\alpha & \vb{E}_n
	\end{bmatrix},
\]
所以\[
	\abs{\vb{A}}
	= \abs{\M}
	= \abs{1+\vb\beta\vb\alpha}
	= 1+\vb\beta\vb\alpha
	= 1 + \sum_{k=1}^n a_k b_k.
\]
\end{solution}
\end{example}
