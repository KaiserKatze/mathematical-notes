\section{舒尔定理}
\begin{theorem}\label{theorem:逆矩阵.舒尔定理}
设\(\A\)是\(m\)阶可逆矩阵,
\(\B,\C,\D\)分别是\(m \times p, n \times m, n \times p\)矩阵,
则有\begin{gather}
	\begin{bmatrix}
		\E_m & \z \\
		-\C\A^{-1} & \E_n
	\end{bmatrix}
	\begin{bmatrix}
		\A & \B \\
		\C & \D
	\end{bmatrix}
	= \begin{bmatrix}
		\A & \B \\
		\z & \D - \C \A^{-1} \B
	\end{bmatrix},
	\\
	\begin{bmatrix}
		\A & \B \\
		\C & \D
	\end{bmatrix}
	\begin{bmatrix}
		\E_m & -\A^{-1} \B \\
		\z & \E_p
	\end{bmatrix}
	= \begin{bmatrix}
		\A & \z \\
		\C & \D - \C \A^{-1} \B
	\end{bmatrix},
	\\
	\begin{bmatrix}
		\E_m & \z \\
		-\C \A^{-1} & \E_n
	\end{bmatrix}
	\begin{bmatrix}
		\A & \B \\
		\C & \D
	\end{bmatrix}
	\begin{bmatrix}
		\E_m & -\A^{-1} \B \\
		\z & \E_p
	\end{bmatrix}
	= \begin{bmatrix}
		\A & \z \\
		\z & \D - \C \A^{-1} \B
	\end{bmatrix},
\end{gather}
\rm
其中\(\D - \C \A^{-1} \B\)
称为“矩阵\(\begin{bmatrix}
	\A & \B \\
	\C & \D
\end{bmatrix}\)
关于\(\A\)的\DefineConcept{舒尔补}(Schur complement)”.
%TODO proof
\end{theorem}
我们把\cref{theorem:逆矩阵.舒尔定理} 称为“舒尔定理”.

可以看到只要\(\A\)可逆,
就能通过初等分块矩阵直接对分块矩阵进行分块相似对角化,
而在操作过程中就把一些原本不为0的分块矩阵变成了零矩阵,
这个过程可以形象地称为“矩阵打洞”,
即让矩阵出现尽可能多的0.

利用\hyperref[theorem:逆矩阵.舒尔定理]{舒尔定理}可以证明下述行列式降阶定理:
\begin{theorem}[行列式降阶定理]\label{theorem:逆矩阵.行列式降阶定理}
\def\M{\vb{M}}
设\(\M = \begin{bmatrix}
	\A & \B \\
	\C & \D
\end{bmatrix}\)是方阵.
\begin{itemize}
	\item 若\(\A\)可逆,则\begin{equation}\label{equation:逆矩阵.行列式降阶公式1}
		\abs{\M} = \abs{\A} \abs{\D - \C \A^{-1} \B}.
	\end{equation}

	\item 若\(\D\)可逆,则\begin{equation}\label{equation:逆矩阵.行列式降阶公式2}
		\abs{\M} = \abs{\D} \abs{\A - \B \D^{-1} \C}.
	\end{equation}

	\item 若\(\A,\D\)均可逆,则\begin{equation}
		\abs{\A} \abs{\D - \C \A^{-1} \B}
		= \abs{\D} \abs{\A - \B \D^{-1} \C}.
	\end{equation}
\end{itemize}
\begin{proof}
由\cref{theorem:逆矩阵.舒尔定理,equation:行列式.广义三角阵的行列式1} 立即可得.
\end{proof}
\end{theorem}
\begin{remark}
这里给出一个有利于记忆\hyperref[theorem:逆矩阵.行列式降阶定理]{行列式降阶定理}的口诀:
对于\cref{equation:逆矩阵.行列式降阶公式1} 中等号右边的第二个行列式,
我们从\(\D\)出发,先以顺时针方向依次写出分块矩阵\(\D,\C,\A,\B\),
再在第一个和第二个分块阵之间插入一个减号,最后取第三个分块阵的逆矩阵.
\cref{equation:逆矩阵.行列式降阶公式2} 也可以相同的记忆方法.
\end{remark}

\begin{example}\label{example:逆矩阵.行列式降阶定理的重要应用1}
设\(\A \in M_{s \times n}(K),
\B \in M_{n \times s}(K)\).
证明:\[
	\begin{vmatrix}
		\E_n & \B \\
		\A & \E_s
	\end{vmatrix}
	= \abs{\E_s - \A \B}
	= \abs{\E_n - \B \A}.
\]
\begin{proof}
由于\(\E_n,\E_s\)都是单位矩阵,必可逆,
那么由\hyperref[equation:逆矩阵.行列式降阶公式1]{行列式降阶定理}有\begin{equation*}
	\begin{vmatrix}
		\E_n & \B \\
		\A & \E_s
	\end{vmatrix}
	= \abs{\E_n} \abs{\E_s - \A(\E_n)^{-1}\B}
	= \abs{\E_s - \A \B}.
\end{equation*}
同理,由\hyperref[equation:逆矩阵.行列式降阶公式2]{行列式降阶定理}有\begin{equation*}
	\begin{vmatrix}
		\E_n & \B \\
		\A & \E_s
	\end{vmatrix}
	= \abs{\E_n - \B \A}.
\end{equation*}
因此\(\abs{\E_s - \A \B} = \abs{\E_n - \B \A}\).
\end{proof}
%\cref{example:单位矩阵与两矩阵乘积之差.单位矩阵与两矩阵乘积之差的秩}
%\cref{example:单位矩阵与两矩阵乘积之差.单位矩阵与两矩阵乘积之差的行列式}
\end{example}

\begin{example}
%@see: https://www.bilibili.com/video/BV1ki4y1a7dL/
设\(\A,\B,\C,\D \in M_n(K)\),
且\(\A \C = \C \A\),
则\begin{equation*}
	\begin{vmatrix}
		\A & \B \\
		\C & \D
	\end{vmatrix}
	= \abs{\A \D - \C \B}.
\end{equation*}
\begin{proof}
当\(\A\)可逆时,由\hyperref[equation:逆矩阵.行列式降阶公式1]{行列式降阶定理}有\begin{align*}
	\begin{vmatrix}
		\A & \B \\
		\C & \D
	\end{vmatrix}
	&= \abs{\A} \abs{\D - \C \A^{-1} \B} \\
	&= \abs{\A (\D - \C \A^{-1} \B)} \\
	&= \abs{\A \D - \A \C \A^{-1} \B} \\
	&= \abs{\A \D - \C \A \A^{-1} \B} \\
	&= \abs{\A \D - \C \B}.
	\qedhere
\end{align*}
%TODO 还没有证明“\(\A\)不可逆”的情形
%@credit: {1ef6baa6-d8ce-4b2c-97d3-f6156721c52f} 说可以用摄动法
\end{proof}
\end{example}

\begin{example}
\def\M{\vb{M}}
求解行列式\(\det \M\),其中\[
	\M = \begin{bmatrix}
		1+a_1 b_1 & a_1 b_2 & \dots & a_1 b_n \\
		a_2 b_1 & 1+a_2 b_2 & \dots & a_2 b_n \\
		\vdots & \vdots & & \vdots \\
		a_n b_1 & a_n b_2 & \dots & 1+a_n b_n
	\end{bmatrix}.
\]
\begin{solution}
记\(\a = (\AutoTuple{a}{n})^T,
\b = (\AutoTuple{b}{n})\),
则\(\M = \E_n + \a\b\).

注意到\(\M = \E_n + \a\b\)形似某个矩阵的舒尔补,因此考虑下面的矩阵:\[
	\A = \begin{bmatrix}
		1 & -\b \\
		\a & \E_n
	\end{bmatrix}.
\]由于\[
	\begin{bmatrix}
		1 & \z \\
		-\a & \E_n
	\end{bmatrix} \A
	= \begin{bmatrix}
		1 & \z \\
		-\a & \E_n
	\end{bmatrix}
	\begin{bmatrix}
		1 & -\b \\
		\a & \E_n
	\end{bmatrix}
	= \begin{bmatrix}
		1 & -\b \\
		\z & \M
	\end{bmatrix},
\]
且\[
	\begin{bmatrix}
		1 & \b \\
		\z & \E_n
	\end{bmatrix} \A
	= \begin{bmatrix}
		1 & \b \\
		\z & \E_n
	\end{bmatrix}
	\begin{bmatrix}
		1 & -\b \\
		\a & \E_n
	\end{bmatrix}
	= \begin{bmatrix}
		1+\b\a & \z \\
		\a & \E_n
	\end{bmatrix},
\]
所以\[
	\abs{\A}
	= \abs{\M}
	= \abs{1+\b\a}
	= 1+\b\a
	= 1 + \sum_{k=1}^n a_k b_k.
\]
\end{solution}
\end{example}
