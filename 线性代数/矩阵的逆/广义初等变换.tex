\section{广义初等变换}
\subsection{广义初等变换的概念}
\begin{definition}
\def\originalmatrix{%
	\begin{bmatrix}%
	\A & \B \\
	\C & \D
	\end{bmatrix}%
}%
分块矩阵有以下三种\DefineConcept{广义初等行变换}:
\def\originalmatrixTail{%
	\originalmatrix \begin{matrix} m \\ n \end{matrix}%
}%
\begin{enumerate}
\item 交换两行,\[
\originalmatrixTail \to \begin{bmatrix}
\C & \D \\
\A & \B
\end{bmatrix} = {\color{red} \begin{bmatrix}
\z & \E_n \\
\E_m & \z
\end{bmatrix}} \originalmatrix
\]

\item 用一个可逆矩阵\(\P_m\)左乘某一行,\[
\originalmatrixTail \to \begin{bmatrix}
\P\A & \P\B \\
\C & \D
\end{bmatrix} = {\color{red} \begin{bmatrix}
\P & \z \\
\z & \E_n
\end{bmatrix}} \originalmatrix
\]

\item 用一个矩阵\(\Q_{n \times m}\)左乘某一行后加到另一行,\[
\originalmatrixTail \to \begin{bmatrix}
\A & \B \\
\C+\Q\A & \D+\Q\B
\end{bmatrix} = {\color{red} \begin{bmatrix}
\E_m & \z \\
\Q & \E_n
\end{bmatrix}} \originalmatrix
\]
\end{enumerate}

类似地,有\DefineConcept{广义初等列变换}:
\def\originalmatrixHead{%
	\overset{\begin{matrix} s & t \end{matrix}}{ \originalmatrix }%
}%
\begin{enumerate}
\item 交换两列,\[
\originalmatrixHead \to \begin{bmatrix}
\B & \A \\
\D & \C
\end{bmatrix} = \originalmatrix {\color{red} \begin{bmatrix}
\z & \E_s \\
\E_t & \z
\end{bmatrix}}
\]

\item 用一个可逆矩阵\(\P_t\)右乘某一列,\[
\originalmatrixHead \to \begin{bmatrix}
\A & \B \P \\
\C & \D \P
\end{bmatrix} = \originalmatrix {\color{red} \begin{bmatrix}
\E_s & \z \\
\z & \P
\end{bmatrix}}
\]

\item 用一个矩阵\(\Q_{t \times s}\)右乘某一列后加到另一列,\[
\originalmatrixHead \to \begin{bmatrix}
\A + \B \Q & \B \\
\C + \D \Q & \D
\end{bmatrix} = \originalmatrix {\color{red} \begin{bmatrix}
\E_s & \z \\
\Q & \E_t
\end{bmatrix}}
\]
\end{enumerate}

广义初等行变换与广义初等列变换统称为\DefineConcept{广义初等变换}.
\end{definition}

类比于初等矩阵,我们定义分块初等矩阵如下:
\begin{definition}
将\(n\)阶单位矩阵\(\E\)分为\(m\)块后,
进行\emph{一次}广义初等变换所得的矩阵称为\DefineConcept{分块初等矩阵}.
\end{definition}

\begin{property}
分块初等矩阵都是可逆矩阵.
\end{property}
