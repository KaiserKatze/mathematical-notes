\section{广义初等变换}
设\(\A \in M_{m \times s}(K),
\B \in M_{m \times t}(K),
\C \in M_{n \times s}(K),
\D \in M_{n \times t}(K)\).

\def\OriginalMatrix{
	\begin{bmatrix}
		\A & \B \\
		\C & \D
	\end{bmatrix}
}
分块矩阵有以下三种\DefineConcept{广义初等行变换}:
\begin{enumerate}
	\item 交换两行,\[
		\OriginalMatrix
		\mapsto \begin{bmatrix}
			\C & \D \\
			\A & \B
		\end{bmatrix}
		= {\color{red} \begin{bmatrix}
			\z & \E_n \\
			\E_m & \z
		\end{bmatrix}}
		\OriginalMatrix.
	\]

	\item 用一个可逆矩阵\(\P_m\)左乘某一行,\[
		\OriginalMatrix
		\mapsto \begin{bmatrix}
			\P\A & \P\B \\
			\C & \D
		\end{bmatrix}
		= {\color{red} \begin{bmatrix}
			\P & \z \\
			\z & \E_n
		\end{bmatrix}}
		\OriginalMatrix.
	\]

	\item 用一个矩阵\(\Q_{n \times m}\)左乘某一行后加到另一行,\[
		\OriginalMatrix
		\mapsto \begin{bmatrix}
		\A & \B \\
		\C+\Q\A & \D+\Q\B
		\end{bmatrix}
		= {\color{red} \begin{bmatrix}
		\E_m & \z \\
		\Q & \E_n
		\end{bmatrix}}
		\OriginalMatrix.
	\]
\end{enumerate}

类似地,有\DefineConcept{广义初等列变换}:
\begin{enumerate}
	\item 交换两列,\[
		\OriginalMatrix
		\mapsto \begin{bmatrix}
			\B & \A \\
			\D & \C
		\end{bmatrix}
		= \OriginalMatrix {\color{red} \begin{bmatrix}
			\z & \E_s \\
			\E_t & \z
		\end{bmatrix}}.
	\]

	\item 用一个可逆矩阵\(\P_t\)右乘某一列,\[
		\OriginalMatrix
		\mapsto \begin{bmatrix}
			\A & \B\P \\
			\C & \D\P
		\end{bmatrix}
		= \OriginalMatrix {\color{red} \begin{bmatrix}
			\E_s & \z \\
			\z & \P
		\end{bmatrix}}.
	\]

	\item 用一个矩阵\(\Q_{t \times s}\)右乘某一列后加到另一列,\[
		\OriginalMatrix
		\mapsto \begin{bmatrix}
			\A + \B\Q & \B \\
			\C + \D\Q & \D
		\end{bmatrix}
		= \OriginalMatrix {\color{red} \begin{bmatrix}
			\E_s & \z \\
			\Q & \E_t
		\end{bmatrix}}.
	\]
\end{enumerate}

广义初等行变换与广义初等列变换统称为\DefineConcept{广义初等变换}.

类比于初等矩阵,我们定义分块初等矩阵如下:
将\(n\)阶单位矩阵\(\E\)分为\(m\)块后,
进行\emph{一次}广义初等变换所得的矩阵称为\DefineConcept{分块初等矩阵}.

\begin{property}
分块初等矩阵都是可逆矩阵.
\end{property}
