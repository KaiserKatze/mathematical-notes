\section{广义逆矩阵}
\subsection{广义逆的概念及其存在性}
对于线性方程组\(\vb{A} \vb{x} = \vb\beta\),如果\(\vb{A}\)可逆,
%@see: 《高等代数(第三版 上册)》(丘维声) P164 (1)
那么它有\(\vb{x} = \vb{A}^{-1} \vb\beta\).
如果\(\vb{A}\)不可逆,但\(\vb{A} \vb{x} = \vb\beta\)有解,那么它的解是否也可表达为类似的简洁公式呢?
我们接下来带着这个问题,开始对\(\vb{A}^{-1}\)的性质的分析.

如果\(\vb{A}\)可逆,
%@see: 《高等代数(第三版 上册)》(丘维声) P165 (2)
那么\(\vb{A} \vb{A}^{-1} = \vb{E}\).
显然,只要在等式两端同时右乘\(\vb{A}\),
%@see: 《高等代数(第三版 上册)》(丘维声) P165 (3)
便得\(\vb{A} \vb{A}^{-1} \vb{A} = \vb{A}\).
这就表明:当\(\vb{A}\)可逆时,\(\vb{A}^{-1}\)是矩阵方程\(\vb{A} \vb{X} \vb{A} = \vb{A}\)的一个解.
受此启发,当\(\vb{A}\)不可逆时,为了找到\(\vb{A}^{-1}\)的替代物,应当去找矩阵方程\(\vb{A} \vb{X} \vb{A} = \vb{A}\)的解.
问题是:矩阵方程\(\vb{A} \vb{X} \vb{A} = \vb{A}\)一定有解吗?

\begin{theorem}[广义逆存在定理]\label{theorem:线性方程组.广义逆1}
%@see: 《高等代数(第三版 上册)》(丘维声) P165 定理1
%@see: 《高等代数(大学高等代数课程创新教材 第二版 上册)》(丘维声) P250 定理1
设\(\vb{A}\)是数域\(K\)上的\(s \times n\)非零矩阵,
则矩阵方程\begin{equation}\label{equation:线性方程组.广义逆1矩阵方程}
%@see: 《高等代数(第三版 上册)》(丘维声) P165 (4)
	\vb{A} \vb{X} \vb{A} = \vb{A}
\end{equation}一定有解.
如果\(\rank\vb{A}=r\),并且\begin{equation*}
%@see: 《高等代数(第三版 上册)》(丘维声) P165 (5)
	\vb{A}
	= \vb{P}
	\begin{bmatrix}
		\vb{E}_r & \vb0 \\
		\vb0 & \vb0
	\end{bmatrix}
	\vb{Q},
\end{equation*}
其中\(\vb{P},\vb{Q}\)分别是\(K\)上\(s\)阶、\(n\)阶可逆矩阵,
那么矩阵方程 \labelcref{equation:线性方程组.广义逆1矩阵方程} 的通解为
\begin{equation}\label{equation:线性方程组.广义逆1矩阵方程的通解}
%@see: 《高等代数(第三版 上册)》(丘维声) P165 (6)
	\vb{X}
	= \vb{Q}^{-1}
	\begin{bmatrix}
		\vb{E}_r & \vb{B} \\
		\vb{C} & \vb{D}
	\end{bmatrix}
	\vb{P}^{-1},
\end{equation}
其中\(\vb{B}\in M_{r\times(s-r)}(K),
\vb{C}\in M_{(n-r)\times r}(K),
\vb{D}\in M_{(n-r)\times(s-r)}(K)\).
\begin{proof}
首先解方程\(\vb{A} \vb{X} \vb{A} = \vb{A}\).
把\(
	\vb{A}
	= \vb{P}
	\begin{bmatrix}
		\vb{E}_r & \vb0 \\
		\vb0 & \vb0
	\end{bmatrix}
	\vb{Q}
\)
%@see: 《高等代数(第三版 上册)》(丘维声) P165 (7)
代入\(\vb{A} \vb{X} \vb{A} = \vb{A}\)
得\begin{equation*}
	\vb{P}
	\begin{bmatrix}
		\vb{E}_r & \vb0 \\
		\vb0 & \vb0
	\end{bmatrix}
	\vb{Q} \vb{X} \vb{P}
	\begin{bmatrix}
		\vb{E}_r & \vb0 \\
		\vb0 & \vb0
	\end{bmatrix}
	\vb{Q}
	= \vb{P}
	\begin{bmatrix}
		\vb{E}_r & \vb0 \\
		\vb0 & \vb0
	\end{bmatrix}
	\vb{Q},
\end{equation*}
上式两边同时左乘\(\vb{P}^{-1}\)并右乘\(\vb{Q}^{-1}\)
得\begin{equation*}
%@see: 《高等代数(第三版 上册)》(丘维声) P165 (8)
	\begin{bmatrix}
		\vb{E}_r & \vb0 \\
		\vb0 & \vb0
	\end{bmatrix}
	\vb{Q} \vb{X} \vb{P}
	\begin{bmatrix}
		\vb{E}_r & \vb0 \\
		\vb0 & \vb0
	\end{bmatrix}
	=
	\begin{bmatrix}
		\vb{E}_r & \vb0 \\
		\vb0 & \vb0
	\end{bmatrix}.
\end{equation*}
把\(\vb{Q} \vb{X} \vb{P}\)写成分块矩阵的形式:\begin{equation*}
%@see: 《高等代数(第三版 上册)》(丘维声) P165 (9)
	\vb{Q} \vb{X} \vb{P}
	= \begin{bmatrix}
		\vb{H} & \vb{B} \\
		\vb{C} & \vb{D}
	\end{bmatrix},
\end{equation*}
其中\(
	\vb{H} \in M_r(K),
	\vb{B} \in M_{r \times (s-r)}(K),
	\vb{C} \in M_{(n-r) \times r}(K),
	\vb{D} \in M_{(n-r) \times (s-r)}(K)
\).
于是\begin{equation*}
	\begin{bmatrix}
		\vb{E}_r & \vb0 \\
		\vb0 & \vb0
	\end{bmatrix}
	\begin{bmatrix}
		\vb{H} & \vb{B} \\
		\vb{C} & \vb{D}
	\end{bmatrix}
	\begin{bmatrix}
		\vb{E}_r & \vb0 \\
		\vb0 & \vb0
	\end{bmatrix}
	%@see: 《高等代数(第三版 上册)》(丘维声) P166 (10)
	= \begin{bmatrix}
		\vb{H} & \vb0 \\
		\vb0 & \vb0
	\end{bmatrix}
	= \begin{bmatrix}
		\vb{E}_r & \vb0 \\
		\vb0 & \vb0
	\end{bmatrix},
\end{equation*}
从而有\(\vb{H} = \vb{E}_r\),
因此\begin{equation*}
%@see: 《高等代数(第三版 上册)》(丘维声) P166 (11)
	\vb{X}
	= \vb{Q}^{-1}
	\begin{bmatrix}
		\vb{E}_r & \vb{B} \\
		\vb{C} & \vb{D}
	\end{bmatrix}
	\vb{P}^{-1}.
\end{equation*}

可以验证:对于任意矩阵\(
	\vb{B} \in M_{r \times (s-r)}(K),
	\vb{C} \in M_{(n-r) \times r}(K),
	\vb{D} \in M_{(n-r) \times (s-r)}(K)
\),
矩阵\(
	\vb{Q}^{-1}
	\begin{bmatrix}
		\vb{E}_r & \vb{B} \\
		\vb{C} & \vb{D}
	\end{bmatrix}
	\vb{P}^{-1}
\)总是矩阵方程\(\vb{A} \vb{X} \vb{A} = \vb{A}\)的一个解.

综上所述,矩阵方程 \labelcref{equation:线性方程组.广义逆1矩阵方程} 一定有解,
且\cref{equation:线性方程组.广义逆1矩阵方程的通解} 是它的通解.
\end{proof}
\end{theorem}

\begin{definition}
%@see: 《高等代数(第三版 上册)》(丘维声) P166 定义1
%@see: 《高等代数(大学高等代数课程创新教材 第二版 上册)》(丘维声) P251 定义1
设\(\vb{A}\)是数域\(K\)上的\(s \times n\)矩阵,
矩阵方程\(\vb{A} \vb{X} \vb{A} = \vb{A}\)的每一个解都称为
“\(\vb{A}\)的一个\DefineConcept{广义逆矩阵}(generalized inverse)”,
简称“\(\vb{A}\)的\DefineConcept{广义逆}”,
记作\(\vb{A}^-\).
\end{definition}

\begin{corollary}
%@see: 《矩阵论简明教程(第三版)》(徐仲、张凯院、陆全、冷国伟) P146 推论
设\(\vb{A}\)是数域\(K\)上的\(s \times n\)非零矩阵,
则矩阵方程 \labelcref{equation:线性方程组.广义逆1矩阵方程} 有唯一解的
充分必要条件是\(\rank\vb{A} = s = n\).
\end{corollary}

\subsection{广义逆的性质}
\begin{property}\label{theorem:线性方程组.广义逆的性质1}
设\(\vb{A} \in M_{s \times n}(K)\),
\(\vb{A}^-\)是\(\vb{A}\)的一个广义逆,
%@see: 《高等代数(第三版 上册)》(丘维声) P166 (12)
则\(\vb{A} \vb{A}^- \vb{A} = \vb{A}\).
\begin{proof}
根据广义逆的定义,显然成立\(\vb{A} \vb{A}^- \vb{A} = \vb{A}\).
\end{proof}
\end{property}

\begin{property}\label{theorem:线性方程组.广义逆的性质2}
%@see: 《高等代数(第三版 上册)》(丘维声) P166
任意一个\(n \times s\)矩阵都是\(\vb0_{s \times n}\)的广义逆.
\begin{proof}
设\(\vb{A} \in M_{n \times s}(K)\),
则\(\vb0_{s \times n} \vb{A} \vb0_{s \times n} = \vb0_{s \times n}\).
\end{proof}
\end{property}

\begin{theorem}
%@see: 《矩阵论简明教程(第三版)》(徐仲、张凯院、陆全、冷国伟) P146 定理6.4(1)
设\(\vb{A} \in M_{s \times n}(K)\),
\(\vb{A}^-\)是\(\vb{A}\)的一个广义逆,
则\((\vb{A}^-)^T\)是\(\vb{A}^T\)的一个广义逆.
\begin{proof}
因为\(
	\vb{A}^T (\vb{A}^-)^T \vb{A}^T
	= (\vb{A} \vb{A}^- \vb{A})^T
	= \vb{A}^T
\),
所以\((\vb{A}^-)^T\)是\(\vb{A}^T\)的一个广义逆.
\end{proof}
\end{theorem}

\begin{theorem}
%@see: 《矩阵论简明教程(第三版)》(徐仲、张凯院、陆全、冷国伟) P146 定理6.4(1)
设\(\vb{A} \in M_{s \times n}(\mathbb{C})\),
\(\vb{A}^-\)是\(\vb{A}\)的一个广义逆,
则\((\vb{A}^-)^H\)是\(\vb{A}^H\)的一个广义逆.
\begin{proof}
因为\(
	\vb{A}^H (\vb{A}^-)^H \vb{A}^H
	= (\vb{A} \vb{A}^- \vb{A})^H
	= \vb{A}^H
\),
所以\((\vb{A}^-)^H\)是\(\vb{A}^H\)的一个广义逆.
\end{proof}
\end{theorem}

\begin{theorem}
%@see: 《矩阵论简明教程(第三版)》(徐仲、张凯院、陆全、冷国伟) P146 定理6.4(2)
设\(\vb{A} \in M_{s \times n}(K)\),
\(\vb{A}^-\)是\(\vb{A}\)的一个广义逆,
\(\lambda \in K\),
则\(\lambda^- \vb{A}^-\)是\(\lambda \vb{A}\)的一个广义逆,
其中\begin{equation*}
%@see: 《矩阵论简明教程(第三版)》(徐仲、张凯院、陆全、冷国伟) P146 (6.3)
	\lambda^-
	\defeq
	\begin{cases}[cl]
		\lambda^{-1}, & \lambda\neq0, \\
		0, & \lambda=0.
	\end{cases}
\end{equation*}
\begin{proof}
当\(\lambda\neq0\)时,
有\begin{equation*}
	(\lambda \vb{A})
	(\lambda^- \vb{A}^-)
	(\lambda \vb{A})
	= (\lambda \lambda^{-1} \lambda) (\vb{A} \vb{A}^- \vb{A})
	= \lambda \vb{A}.
\end{equation*}
当\(\lambda=0\)时,
有\begin{equation*}
	(\lambda \vb{A})
	(\lambda^- \vb{A}^-)
	(\lambda \vb{A})
	= \vb0
	= \lambda \vb{A}.
\end{equation*}
综上所述,\(\lambda^- \vb{A}^-\)是\(\lambda \vb{A}\)的一个广义逆.
\end{proof}
\end{theorem}

\begin{property}
%@see: 《矩阵论简明教程(第三版)》(徐仲、张凯院、陆全、冷国伟) P146 定理6.4(4)
设\(\vb{A} \in M_{s \times n}(K)\),
\(\vb{A}^-\)是\(\vb{A}\)的一个广义逆,
则\(\rank\vb{A}^- \geq \rank\vb{A}\).
\begin{proof}
由\cref{theorem:线性方程组.矩阵乘积的秩} 可知\(
	\rank\vb{A}
	= \rank(\vb{A} \vb{A}^- \vb{A})
	\leq \rank\vb{A}^-
\).
\end{proof}
\end{property}

\begin{property}
%@see: 《矩阵论简明教程(第三版)》(徐仲、张凯院、陆全、冷国伟) P146 定理6.4(5)
%@credit: {ce603838-a24d-4616-9395-d7b223e8cb72}
设\(\vb{A} \in M_{s \times n}(K)\),
\(\vb{A}^-\)是\(\vb{A}\)的一个广义逆,
则\(
	\rank(\vb{A} \vb{A}^-)
	= \rank(\vb{A}^- \vb{A})
	= \rank\vb{A}
\).
\begin{proof}
因为\(
	\rank\vb{A}
	= \rank(\vb{A} \vb{A}^- \vb{A})
	\leq \min\{
		\rank(\vb{A} \vb{A}^-),
		\rank(\vb{A}^- \vb{A})
	\}
	\leq \rank\vb{A}
\),
所以\(
	\rank(\vb{A} \vb{A}^-)
	\allowbreak
	= \rank(\vb{A}^- \vb{A})
	= \rank\vb{A}
\).
\end{proof}
\end{property}

\begin{example}
%@see: 《高等代数(大学高等代数课程创新教材 第二版 上册)》(丘维声) P254 例1
设\(\vb{A} \in M_{s \times n}(K)\).
证明:如果\(\vb{A}\)是行满秩矩阵,
那么对于\(\vb{A}\)的任意一个广义逆,
有\(\vb{A} \vb{A}^- = \vb{E}_s\),
其中\(\vb{E}_s\)是\(s\)阶单位矩阵.
\begin{proof}
由于\(\rank\vb{A} = s\),
因此存在\(s\)阶可逆矩阵\(\vb{P}\)和\(n\)阶可逆矩阵\(\vb{Q}\),
使得\begin{equation*}
	\vb{A} = \vb{P} (\vb{E}_s,\vb0) \vb{Q},
\end{equation*}
从而\(
	\vb{A}^-
	= \vb{Q}^{-1}
	\begin{bmatrix}
		\vb{E}_s \\
		\vb{C}
	\end{bmatrix}
	\vb{P}^{-1}
\),
于是\(
	\vb{A} \vb{A}^-
	= \vb{P}
	(\vb{E}_s,\vb0)
	\vb{Q}
	\vb{Q}^{-1}
	\begin{bmatrix}
		\vb{E}_s \\
		\vb{C}
	\end{bmatrix}
	\vb{P}^{-1}
	= \vb{P} \vb{E}_s \vb{P}^{-1}
	= \vb{E}_s
\).
\end{proof}
\end{example}

\begin{proposition}
%@see: 《矩阵论简明教程(第三版)》(徐仲、张凯院、陆全、冷国伟) P146 定理6.4(6)
%@see: 《矩阵论简明教程(第三版)》(徐仲、张凯院、陆全、冷国伟) P146 定理6.4(7)
设\(\vb{A} \in M_{s \times n}(K)\),
\(\vb{A}^-\)是\(\vb{A}\)的一个广义逆,
\(\vb{E}_s\)是\(s\)阶单位矩阵,
\(\vb{E}_n\)是\(n\)阶单位矩阵,
则\begin{gather*}
	\vb{A} \vb{A}^- = \vb{E}_s
	\iff
	\rank\vb{A} = s, \\
	\vb{A}^- \vb{A} = \vb{E}_n
	\iff
	\rank\vb{A} = n.
\end{gather*}
\end{proposition}

\subsection{用广义逆表示线性方程组的解}
\begin{theorem}[非齐次线性方程组的相容性定理]\label{theorem:线性方程组.非齐次线性方程组的相容性定理}
%@see: 《高等代数(第三版 上册)》(丘维声) P166 定理2(非齐次线性方程组的相容性定理)
%@see: 《高等代数(大学高等代数课程创新教材 第二版 上册)》(丘维声) P251 定理2(非齐次线性方程组的相容性定理)
非齐次线性方程组\(\vb{A} \vb{X} = \vb\beta\)有解的充分必要条件是
%@see: 《高等代数(第三版 上册)》(丘维声) P166 (14)
\(\vb\beta = \vb{A} \vb{A}^- \vb\beta\).
\begin{proof}
必要性.
设\(\vb\alpha\)是\(\vb{A} \vb{X} = \vb\beta\)的一个解,
则\(
	\vb\beta
	= \vb{A} \vb\alpha
	= (\vb{A} \vb{A}^- \vb{A}) \vb\alpha
	= \vb{A} \vb{A}^- \vb\beta
\).

充分性.
设\(\vb\beta = \vb{A} \vb{A}^- \vb\beta\),
则\(\vb{A}^- \vb\beta\)是\(\vb{A} \vb{X} = \vb\beta\)的解.
\end{proof}
\end{theorem}

\begin{theorem}[非齐次线性方程组的解的结构定理]\label{theorem:线性方程组.非齐次线性方程组的解的结构定理}
%@see: 《高等代数(第三版 上册)》(丘维声) P167 定理3(非齐次线性方程组的解的结构定理)
%@see: 《高等代数(大学高等代数课程创新教材 第二版 上册)》(丘维声) P252 定理3(非齐次线性方程组的解的结构定理)
非齐次线性方程组\(\vb{A} \vb{X} = \vb\beta\)有解时,
%@see: 《高等代数(第三版 上册)》(丘维声) P167 (15)
%\label{equation:线性方程组.非齐次线性方程组的通解1}
它的通解为\(
	\vb{X} = \vb{A}^- \vb\beta
\),
其中\(\vb{A}^-\)是\(\vb{A}\)的任意一个广义逆.
\begin{proof}
设\(
	\vb{A} \in M_{s \times n}(K),
	\vb\beta \in K^s
\),
且\(\vb\gamma\)是\(\vb{A} \vb{X} = \vb\beta\)的一个解,即\(\vb{A} \vb\gamma = \vb\beta\).
再设\(\rank\vb{A} = r\),而\(\vb{E}_r\)是\(r\)阶单位矩阵.
下面我们首先证明:存在\(\vb{A}\)的一个广义逆\(\vb{A}^-\),使得\(\vb{A}^- \vb\beta = \vb\gamma\).
要找到这样的\(\vb{A}^-\),
关键是在\(\vb{A}^-\)的表达式\(
%@see: 《高等代数(第三版 上册)》(丘维声) P167 (13)
	\vb{A}^-
	= \vb{Q}^{-1}
	\begin{bmatrix}
		\vb{E}_r & \vb{B} \\
		\vb{C} & \vb{D}
	\end{bmatrix}
	\vb{P}^{-1}
\)中,
选取合适的矩阵\(\vb{B},\vb{C},\vb{D}\).
设\(
%@see: 《高等代数(第三版 上册)》(丘维声) P167 (16)
	\vb{A}
	= \vb{P}
	\begin{bmatrix}
		\vb{E}_r & \vb0 \\
		\vb0 & \vb0
	\end{bmatrix}
	\vb{Q}
\),
那么代入\(\vb{A} \vb\gamma = \vb\beta\)
得\(
	\left(
		\vb{P}
		\begin{bmatrix}
			\vb{E}_r & \vb0 \\
			\vb0 & \vb0
		\end{bmatrix}
		\vb{Q}
	\right)
	\vb\gamma
	= \vb\beta
\),
从而\(
%@see: 《高等代数(第三版 上册)》(丘维声) P167 (17)
	\begin{bmatrix}
		\vb{E}_r & \vb0 \\
		\vb0 & \vb0
	\end{bmatrix}
	\vb{Q} \vb\gamma
	= \vb{P}^{-1} \vb\beta.
\)
为了求出\(\vb\gamma\)的表达式,
先来求\(\vb{Q} \vb\gamma\)的表达式.
把\(\vb{Q} \vb\gamma\)和\(\vb{P}^{-1} \vb\beta\)分别写成分块矩阵的形式:\begin{equation*}
%@see: 《高等代数(第三版 上册)》(丘维声) P167 (18)
	\vb{Q} \vb\gamma
	= \begin{bmatrix}
		\vb{Y}_1 \\ \vb{Y}_2
	\end{bmatrix},
	\qquad
	\vb{P}^{-1} \vb\beta
	= \begin{bmatrix}
		\vb{Z}_1 \\ \vb{Z}_2
	\end{bmatrix},
\end{equation*}
其中\(
	\vb{Y}_1 \in K^r,
	\vb{Y}_2 \in K^{n-r},
	\vb{Z}_1 \in K^r,
	\vb{Z}_2 \in K^{s-r}
\).
于是\begin{equation*}
%@see: 《高等代数(第三版 上册)》(丘维声) P167 (19)
	\begin{bmatrix}
		\vb{E}_r & \vb0 \\
		\vb0 & \vb0
	\end{bmatrix}
	\begin{bmatrix}
		\vb{Y}_1 \\ \vb{Y}_2
	\end{bmatrix}
	= \begin{bmatrix}
		\vb{Y}_1 \\
		\vb0
	\end{bmatrix}
	= \begin{bmatrix}
		\vb{Z}_1 \\ \vb{Z}_2
	\end{bmatrix},
\end{equation*}
从而有\(
	\vb{Z}_1 = \vb{Y}_1,
	\vb{Z}_2 = \vb0
\).
接下来求\(\vb{Y}_2\)的表达式.
由于\(\vb\beta \neq \vb0\),
因此\(\vb{P}^{-1} \vb\beta \neq \vb0\),
从而\(\vb{Z}_1 \neq \vb0\).
设\(\vb{Z}_1 = (\AutoTuple{k}{r})^T\),
其中\(k_i \neq 0\).
在\(\vb{A}^-\)的表达式中
取\(
%@see: 《高等代数(第三版 上册)》(丘维声) P167 (20)
	\vb{C} = (
		\vb0_{(n-r) \times (i-1)},
		% \underbrace{\vb0,\dotsc,\vb0}_{\text{$i-1$个}},
		k_i^{-1} \vb{Y}_2,
		\vb0_{(n-r) \times (r-i)}
		% \underbrace{\vb0,\dotsc,\vb0}_{\text{$r-i$个}}
	)
\),
%@see: 《高等代数(第三版 上册)》(丘维声) P167 (21)
便得\(\vb{C} \vb{Z}_1 = \vb{Y}_2\),
于是\(
	\vb{Q} \vb\gamma
	= \begin{bmatrix}
		\vb{Y}_1 \\
		\vb{Y}_2
	\end{bmatrix}
	= \begin{bmatrix}
		\vb{Z}_1 \\
		\vb{C} \vb{Z}_1
	\end{bmatrix}
	= \begin{bmatrix}
		\vb{E}_r & \vb0 \\
		\vb{C} & \vb0
	\end{bmatrix}
	\begin{bmatrix}
		\vb{Z}_1 \\
		\vb0
	\end{bmatrix}
\).
由此得出\begin{equation*}
	\vb\gamma
	= \vb{Q}^{-1}
	\begin{bmatrix}
		\vb{E}_r & \vb0 \\
		\vb{C} & \vb0
	\end{bmatrix}
	\begin{bmatrix}
		\vb{Z}_1 \\
		\vb0
	\end{bmatrix}
	= \left(
		\vb{Q}^{-1}
		\begin{bmatrix}
			\vb{E}_r & \vb0 \\
			\vb{C} & \vb0
		\end{bmatrix}
		\vb{P}^{-1}
	\right)
	\vb\beta.
\end{equation*}
这就是说,
要让\(\vb{A}^- \vb\beta\)成为\(\vb{A} \vb{X} = \vb\beta\)的一个解,
\(\vb{A}^-\)就应该取为\(
	\vb{Q}^{-1}
	\begin{bmatrix}
		\vb{E}_r & \vb0 \\
		\vb{C} & \vb0
	\end{bmatrix}
	\vb{P}^{-1}
\).

反过来,对于\(\vb{A}\)的任意一个广义逆\(\vb{A}^-\),
因为\(\vb{A} \vb{X} = \vb\beta\)有解,
所以由\cref{theorem:线性方程组.非齐次线性方程组的相容性定理}
可知\(\vb\beta = \vb{A} \vb{A}^- \vb\beta\),
因此\(\vb{A}^- \vb\beta\)是\(\vb{A} \vb{X} = \vb\beta\)的一个解.

综上所述,\(\vb{A} \vb{X} = \vb\beta\)有解时,
它的通解是\(\vb{X} = \vb{A}^- \vb\beta\),
其中\(\vb{A}^-\)是\(\vb{A}\)的任意一个广义逆.
\end{proof}
\end{theorem}
%@see: 《高等代数(第三版 上册)》(丘维声) P168
从\cref{theorem:线性方程组.非齐次线性方程组的解的结构定理} 看出,
任意非齐次线性方程组\(\vb{A} \vb{X} = \vb\beta\)有解时,
它的通解有简洁漂亮的形式\(\vb{X} = \vb{A}^- \vb\beta\).
这与当\(\vb{A}\)可逆时,\(\vb{A} \vb{X} = \vb\beta\)的唯一解
\(\vb{X} = \vb{A}^{-1} \vb\beta\)相媲美.

\begin{theorem}[齐次线性方程组的解的结构定理]\label{theorem:线性方程组.齐次线性方程组的解的结构定理}
%@see: 《高等代数(大学高等代数课程创新教材 第二版 上册)》(丘维声) P253 定理4(齐次线性方程组的解的结构定理)
数域\(K\)上\(n\)元齐次线性方程组\(\vb{A} \vb{X} = \vb0\)的通解为\begin{equation}\label{equation:线性方程组.齐次线性方程组的通解}
	\vb{X}=(\vb{E}_n - \vb{A}^- \vb{A}) \vb{Z},
\end{equation}
其中\(\vb{A}^-\)是\(\vb{A}\)的任意给定的一个广义逆,
\(\vb{Z}\)取遍\(K^n\)中任意列向量.
\begin{proof}
任取\(\vb{Z} \in K^n\),
有\(
	\vb{A} \vb{X}
	= \vb{A}
	(
		(\vb{E}_n - \vb{A}^- \vb{A})
		\vb{Z}
	)
	= (\vb{A} \vb{E}_n - \vb{A} \vb{A}^- \vb{A}) \vb{Z}
	= (\vb{A} - \vb{A}) \vb{Z}
	= \vb0
\),
因此\(\vb{X} = (\vb{E}_n - \vb{A}^- \vb{A}) \vb{Z}\)是齐次线性方程组\(\vb{A} \vb{X} = \vb0\)的一个解.

反之,设\(\vb\eta\)是\(\vb{A} \vb{X} = \vb0\)的一个解,
则\(
	(\vb{E}_n - \vb{A}^- \vb{A}) \vb\eta
	= \vb{E}_n \vb\eta - \vb{A}^- \vb{A} \vb\eta
	= \vb\eta - \vb0
	= \vb\eta
\).

综上所述,\(\vb{X} = (\vb{E}_n - \vb{A}^- \vb{A}) \vb{Z}\)是齐次线性方程组\(\vb{A} \vb{X} = \vb0\)的通解.
\end{proof}
\end{theorem}

\begin{corollary}\label{theorem:线性方程组.齐次线性方程组的解的结构定理.推论1}
%@see: 《高等代数(大学高等代数课程创新教材 第二版 上册)》(丘维声) P253 推论1
设数域\(K\)上\(n\)元非齐次线性方程组\(\vb{A} \vb{X} = \vb\beta\)有解,
则它的通解为\begin{equation}\label{equation:线性方程组.非齐次线性方程组的通解2}
%@see: 《高等代数(大学高等代数课程创新教材 第二版 上册)》(丘维声) P253 (21)
	\vb{X} = \vb{A}^- \vb\beta + (\vb{E}_n - \vb{A}^- \vb{A}) \vb{Z},
\end{equation}
其中\(\vb{A}^-\)是\(\vb{A}\)的任意给定的一个广义逆,\(\vb{Z}\)取遍\(K^n\)中任意列向量.
\begin{proof}
%\cref{theorem:线性方程组.非齐次线性方程组的解的结构定理}
由于\(\vb{A}^- \vb\beta\)是\(\vb{A} \vb{X} = \vb\beta\)的一个解,
%\cref{theorem:线性方程组.齐次线性方程组的解的结构定理}
且\((\vb{E}_n - \vb{A}^- \vb{A}) \vb{Z}\)是导出组\(\vb{A} \vb{X} = \vb0\)的通解,
所以\(\vb{X} = \vb{A}^- \vb\beta + (\vb{E}_n - \vb{A}^- \vb{A}) \vb{Z}\)
是\(\vb{A} \vb{X} = \vb\beta\)的通解.
\end{proof}
\end{corollary}

\begin{example}
%@see: 《高等代数(大学高等代数课程创新教材 第二版 上册)》(丘维声) P254 例2
设\(\vb{A} \in M_{s \times n}(K), \vb{B} \in M_{s \times m}(K)\).
证明:\begin{itemize}
	\item 矩阵方程\(\vb{A} \vb{X} = \vb{B}\)有解的
	充分必要条件是\(\vb{B} = \vb{A} \vb{A}^- \vb{B}\);
	\item 当矩阵方程\(\vb{A} \vb{X} = \vb{B}\)有解时,
	它的通解为\(\vb{X} = \vb{A}^- \vb{B} + (\vb{E}_n - \vb{A}^- \vb{A}) \vb{W}\),
	其中\(\vb{W}\)是任意\(n \times m\)矩阵,
	\(\vb{A}^-\)是\(\vb{A}\)的任意取定的一个广义逆.
\end{itemize}
%TODO proof
\end{example}

\subsection{穆尔--彭罗斯广义逆}
%@see: 《高等代数(第三版 上册)》(丘维声) P168
一般情况下,矩阵方程\(\vb{A} \vb{X} \vb{A} = \vb{A}\)的解不唯一,从而\(\vb{A}\)的广义逆不唯一.
但是我们有时候希望\(\vb{A}\)的满足特殊条件的广义逆是唯一的,这就引出以下概念:
\begin{definition}
%@see: 《高等代数(第三版 上册)》(丘维声) P168 定义2
设\(\vb{A} \in M_{s \times n}(\mathbb{C})\).
关于矩阵\(\vb{X}\)的矩阵方程组\begin{equation}\label{equation:线性方程组.彭罗斯方程组}
%@see: 《高等代数(大学高等代数课程创新教材 第二版 上册)》(丘维声) P253 (22)
	\begin{cases}
		\vb{A} \vb{X} \vb{A} = \vb{A}, \\
		\vb{X} \vb{A} \vb{X} = \vb{X}, \\
		(\vb{A} \vb{X})^H = \vb{A} \vb{X}, \\
		(\vb{X} \vb{A})^H = \vb{X} \vb{A}
	\end{cases}
\end{equation}
称为“\(\vb{A}\)的\DefineConcept{彭罗斯方程组}”,
它的解称为“\(\vb{A}\)的\DefineConcept{穆尔--彭罗斯广义逆}”,
记作\(\vb{A}^+\).
%@see: https://mathworld.wolfram.com/Moore-PenroseMatrixInverse.html
\end{definition}

\begin{theorem}[穆尔--彭罗斯广义逆的唯一性]\label{theorem:线性方程组.穆尔--彭罗斯广义逆的唯一性}
%@see: 《高等代数(第三版 上册)》(丘维声) P168
%@see: 《高等代数(大学高等代数课程创新教材 第二版 上册)》(丘维声) P253 定理5
如果\(\vb{A} \in M_{s \times n}(\mathbb{C})\),
则\(\vb{A}\)的彭罗斯方程组 \labelcref{equation:线性方程组.彭罗斯方程组} 总是有解,
并且它的解唯一.

设\(\vb{A} = \vb{B} \vb{C}\),
其中\(\vb{B}\)、\(\vb{C}\)分别是列满秩矩阵、行满秩矩阵,
则方程组 \labelcref{equation:线性方程组.彭罗斯方程组} 的唯一解是
\begin{equation}\label{equation:线性方程组.彭罗斯方程组的唯一解}
%@see: 《高等代数(大学高等代数课程创新教材 第二版 上册)》(丘维声) P253 (23)
	\vb{X} = \vb{C}^H (\vb{C} \vb{C}^H)^{-1} (\vb{B}^H \vb{B})^{-1} \vb{B}^H.
\end{equation}
\begin{proof}
首先考虑\(\vb{A}\neq\vb0\).
把\cref{equation:线性方程组.彭罗斯方程组的唯一解}
代入彭罗斯方程组 \labelcref{equation:线性方程组.彭罗斯方程组} 的每一个方程,
不难验证每一个方程都将变成恒等式
\footnote{
	我们知道 \hyperref[example:齐次线性方程组的解集的结构.矩阵及其转置的乘积可逆的充分必要条件]{
		\(\vb{A}^T \vb{A}\)可逆当且仅当\(\vb{A}\)是列满秩矩阵;
		\(\vb{A} \vb{A}^T\)可逆当且仅当\(\vb{A}\)是行满秩矩阵
	}.
	我们还知道 \hyperref[equation:矩阵乘积的秩.复矩阵及其共轭转置矩阵的乘积的秩]{
		\(
			\rank\vb{A}
			=\allowbreak
			\rank(\vb{A} \vb{A}^H)
			=\allowbreak
			\rank(\vb{A}^H \vb{A})
		\)
	}.
	因为
		\(\vb{B}\)是列满秩矩阵,
		\(\vb{C}\)是行满秩矩阵,
	所以\(\vb{B}^H \vb{B}\)和\(\vb{C} \vb{C}^H\)都是\(r\)阶可逆矩阵.
},
由此可知\cref{equation:线性方程组.彭罗斯方程组的唯一解} 的确是
彭罗斯方程组 \labelcref{equation:线性方程组.彭罗斯方程组} 的解.

要证明这种广义逆的唯一性,
先设\(\vb{X}_1\)和\(\vb{X}_2\)都是
彭罗斯方程组 \labelcref{equation:线性方程组.彭罗斯方程组} 的解,
则\begin{align*}
	\vb{X}_1
	&= \vb{X}_1\vb{A} \vb{X}_1
	= \vb{X}_1(\vb{A} \vb{X}_2\vb{A})\vb{X}_1
	= \vb{X}_1(\vb{A} \vb{X}_2)(\vb{A} \vb{X}_1)
	= \vb{X}_1(\vb{A} \vb{X}_2)^H(\vb{A} \vb{X}_1)^H \\
	&= \vb{X}_1(\vb{A} \vb{X}_1\vb{A} \vb{X}_2)^H
	= \vb{X}_1(\vb{A} \vb{X}_2)^H
	= \vb{X}_1\vb{A} \vb{X}_2
	= \vb{X}_1(\vb{A} \vb{X}_2\vb{A})\vb{X}_2 \\
	&= (\vb{X}_1\vb{A})(\vb{X}_2\vb{A})\vb{X}_2
	= (\vb{X}_1\vb{A})^H(\vb{X}_2\vb{A})^H\vb{X}_2
	= (\vb{X}_2\vb{A} \vb{X}_1\vb{A})^H \vb{X}_2 \\
	&= (\vb{X}_2\vb{A})^H \vb{X}_2
	= \vb{X}_2\vb{A} \vb{X}_2
	= \vb{X}_2.
\end{align*}
这就证明了彭罗斯方程组 \labelcref{equation:线性方程组.彭罗斯方程组} 的解的唯一性.
\end{proof}
\end{theorem}
\begin{remark}
现在考虑\(\vb{A} = \vb0\).
设\(\vb{X}_0\)是零矩阵\(\vb0\)的穆尔--彭罗斯广义逆,
则\begin{equation*}
	\vb{X}_0 = \vb{X}_0 \vb0 \vb{X}_0 = \vb0.
\end{equation*}
显然\(\vb0\)是零矩阵的彭罗斯方程组的解,因此零矩阵的穆尔--彭罗斯广义逆是零矩阵本身.
综上所述,对任意复矩阵\(\vb{A}\),它的穆尔--彭罗斯广义逆存在且唯一.
\end{remark}

\begin{example}
%@see: 《高等代数(大学高等代数课程创新教材 第二版 上册)》(丘维声) P255 例3
设\(\vb{A} \in M_{s \times n}(\mathbb{C})\).
证明:\((\vb{A}^+)^+ = \vb{A}\).
\begin{proof}
因为\(\vb{A}^+\)是\(\vb{A}\)的彭罗斯方程组的解,
所以\(\vb{A}\)是\(\vb{A}^+\)的彭罗斯方程组的解.
由于彭罗斯方程组的解唯一,
因此\(\vb{A} = (\vb{A}^+)^+\).
\end{proof}
\end{example}

\begin{example}
%@see: 《高等代数(大学高等代数课程创新教材 第二版 上册)》(丘维声) P255 例4
设\(\vb{B} \in M_{s \times r}(\mathbb{C})\)是列满秩矩阵,
\(\vb{C} \in M_{r \times n}(\mathbb{C})\)是行满秩矩阵,
则\begin{equation*}
	(\vb{B} \vb{C})^+
	= \vb{C}^+ \vb{B}^+.
\end{equation*}
\begin{proof}
令\(\vb{A} \defeq \vb{B} \vb{C}\).
由\cref{equation:线性方程组.彭罗斯方程组的唯一解} 有\begin{equation*}
	\vb{A}^+ = \vb{C}^H (\vb{C} \vb{C}^H)^{-1} (\vb{B}^H \vb{B})^{-1} \vb{B}^H.
\end{equation*}
由于\(\vb{B} = \vb{B} \vb{E}_r\)且\(\vb{C} = \vb{E}_r \vb{C}\),
其中\(\vb{E}_r\)是\(r\)阶单位矩阵,
因此\begin{gather*}
	\vb{B}^+
	= \vb{E}_r^H (\vb{E}_r \vb{E}_r^H)^{-1} (\vb{B}^H \vb{B})^{-1} \vb{B}^H
	= (\vb{B}^H \vb{B})^{-1} \vb{B}^H, \\
	\vb{C}^+
	= \vb{C}^H (\vb{C} \vb{C}^H)^{-1} (\vb{E}_r^H \vb{E}_r)^{-1} \vb{E}_r^H
	= \vb{C}^H (\vb{C} \vb{C}^H)^{-1}.
\end{gather*}
于是\(
	\vb{C}^+ \vb{B}^+
	= \vb{C}^H (\vb{C} \vb{C}^H)^{-1}
		(\vb{B}^H \vb{B})^{-1} \vb{B}^H
	= \vb{A}^+
	= (\vb{B} \vb{C})^+
\).
\end{proof}
\end{example}
\begin{remark}
一般地,\((\vb{A} \vb{B})^+ \neq \vb{B}^+ \vb{A}^+\).
\end{remark}
