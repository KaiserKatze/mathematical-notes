\section{应用初等变换求解逆矩阵}
\begin{property}\label{theorem:逆矩阵.初等矩阵的性质3}
初等矩阵可逆.
\end{property}

\begin{theorem}\label{theorem:逆矩阵.可逆矩阵与初等矩阵的关系}
设\(\vb{A}=(a_{ij})_n\),则\(\vb{A}\)可逆的充分必要条件是:
\(\vb{A}\)可经一系列初等行变换化为单位矩阵\(\vb{E}_n\),
即\(\vb{A} \cong \vb{E}_n\).
\begin{proof}
\def\Ps{\vb{P}_t \vb{P}_{t-1} \dotsm \vb{P}_2 \vb{P}_1}
存在与\(t\)次初等行变换对应的\(t\)个初等矩阵\(\vb{P}_t,\vb{P}_{t-1},\dotsc,\vb{P}_2,\vb{P}_1\),使\begin{equation*}
	\vb{A} \to \vb{E}_n = \Ps \vb{A},
\end{equation*}
则\(\vb{A}\)可逆且\(\vb{A}^{-1} = \Ps\).

对矩阵\((\vb{A},\vb{E}_n)\)作以上初等行变换,则\begin{align*}
	(\vb{A},\vb{E}_n) \to &\Ps(\vb{A},\vb{E}_n) = \vb{A}^{-1}(\vb{A},\vb{E}_n) \\
	&= (\vb{A}^{-1}\vb{A},\vb{A}^{-1}\vb{E}_n) = (\vb{E}_n,\vb{A}^{-1}).
	\qedhere
\end{align*}
\end{proof}
\end{theorem}

\begin{corollary}\label{theorem:逆矩阵.计算逆矩阵的方法}
如果方阵\(\vb{A}\)经\(t\)次初等行变换为\(\vb{E}_n\),
那么同样的初等行变换会将\(\vb{E}_n\)变为\(\vb{A}^{-1}\).
\end{corollary}

\begin{corollary}
可逆矩阵\(\vb{A}\)可以表示成若干个初等矩阵的乘积.
\end{corollary}

\begin{corollary}
\(n\)阶方阵\(\vb{A}\)可逆的充分必要条件是:
\(\vb{A}\)可经过一系列初等列变换变为\(\vb{E}_n\),
且同样的初等列变换将\(\begin{bmatrix}\vb{A}\\\vb{E}_n\end{bmatrix}\)变为
\(\begin{bmatrix}\vb{E}_n\\\vb{A}^{-1}\end{bmatrix}\).
\end{corollary}

当\(\vb{A}\)可逆时,我们可以利用初等行变换解矩阵方程\(\vb{A} \vb{X} = \vb{B}\):\begin{equation*}
	\vb{A}^{-1} (\vb{A},\vb{B})
	= (\vb{A}^{-1} \vb{A},\vb{A}^{-1} \vb{B})
	= (\vb{E},\vb{A}^{-1} \vb{B}),
\end{equation*}
其中\(\vb{X} = \vb{A}^{-1} \vb{B}\)就是原方程的解.
\begin{example}
%@see: https://www.bilibili.com/video/BV1qT421275n/
设\begin{equation*}
	\vb{A} = \begin{bmatrix}
		1 & 1 & 1 \\
		2 & 1 & 0 \\
		1 & -1 & 0
	\end{bmatrix},
	\qquad
	\vb{B} = \begin{bmatrix}
		0 & 1 \\
		1 & 2 \\
		-1 & 1
	\end{bmatrix},
\end{equation*}
解矩阵方程\(\vb{A} \vb{X} = \vb{B}\).
\begin{solution}
对\((\vb{A},\vb{B})\)作初等行变换得\begin{equation*}
	\begin{bmatrix}
		1 & 1 & 1 & 0 & 1 \\
		2 & 1 & 0 & 1 & 2 \\
		1 & -1 & 0 & -1 & 1
	\end{bmatrix}
	\to \begin{bmatrix}
		1 & & & 0 & 1 \\
		& 1 & & 1 & 0 \\
		& & 1 & -1 & 0
	\end{bmatrix},
\end{equation*}
于是\(\vb{X} = \begin{bmatrix}
	0 & 1 \\
	1 & 0 \\
	-1 & 0
\end{bmatrix}\).
%@Mathematica: RowReduce[{{1, 1, 1, 0, 1}, {2, 1, 0, 1, 2}, {1, -1, 0, -1, 1}}]
\end{solution}
\end{example}

\begin{theorem}
设\(\vb{A}\)与\(\vb{B}\)都是\(s \times n\)矩阵,
则\(\vb{A}\)与\(\vb{B}\)等价的充分必要条件是:
存在\(s\)阶可逆矩阵\(\vb{P}\)与\(n\)阶可逆矩阵\(\vb{Q}\),使得\(\vb{B}=\vb{P}\vb{A}\vb{Q}\).
\end{theorem}

\begin{example}
初等矩阵的逆:\begin{gather*}
	[\vb{P}(i,j)]^{-1}=\vb{P}(i,j), \\
	[\vb{P}(i(c))]^{-1}=\vb{P}(i(c^{-1})), \\
	[\vb{P}(i,j(k))]^{-1}=\vb{P}(i,j(-k)).
\end{gather*}
\end{example}
