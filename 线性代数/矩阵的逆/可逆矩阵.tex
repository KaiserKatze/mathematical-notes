\section{可逆矩阵}
% \begin{lemma}
% 设\(\vb{A},\vb{B}\)是数域\(K\)上的\(n\)阶矩阵,
% \(\vb{E}\)是数域\(K\)上的\(n\)阶单位矩阵.
% 如果\(\vb{A}\vb{B}=\vb{E}\),则\(\vb{B}\vb{A}=\vb{E}\).
% \begin{proof}
% 只要在等式\(\vb{A}\vb{B}=\vb{E}\)等号两边同时左乘\(\vb{B}\)并右乘\(\vb{A}\),
% 就有\begin{equation*}
% 	(\vb{B}\vb{A})(\vb{B}\vb{A})
% 	= \vb{B}(\vb{A}\vb{B})\vb{A}
% 	= \vb{B}\vb{E}\vb{A}
% 	= \vb{B}\vb{A}.
% \end{equation*}
% 记\(\vb{X}\defeq\vb{B}\vb{A}\),
% 则\(\vb{X}^2=\vb{X}\),
% 整理得\(\vb{X}(\vb{X}-\vb{E})=\vb0\),
% 解得\(\vb{X}=\vb0\)或\(\vb{X}=\vb{E}\).
% 假设\(\vb{B}\vb{A}=\vb0\),
% 于是\(\abs{\vb{B}\vb{A}}=\abs{\vb{B}}\abs{\vb{A}}=0\),
% 这与\(\abs{\vb{A}\vb{B}}=\abs{\vb{A}}\abs{\vb{B}}=1\)矛盾,
% 因此\(\vb{B}\vb{A}=\vb{E}\).
% \end{proof}
% \end{lemma}
%DELETE: 这个引理是错误的!由\(\vb{X}^2=\vb{X}\)只能得到\(\vb{X}\)是幂等矩阵,不能说明它是零矩阵或单位矩阵!

\begin{definition}\label{definition:可逆矩阵.可逆矩阵的定义}
%@see: 《线性代数》(张慎语、周厚隆) P43 定义1
%@see: 《高等代数(第三版 上册)》(丘维声) P128 定义1
设\(\vb{E}\)是数域\(K\)上的\(n\)阶单位矩阵.
对于数域\(K\)上的矩阵\(\vb{A}\),
如果存在数域\(K\)上的矩阵\(\vb{B}\),使得\begin{equation*}
	\vb{A}\vb{B}=\vb{B}\vb{A}=\vb{E},
\end{equation*}
则称“\(\vb{A}\)是一个\DefineConcept{可逆矩阵}(\(\vb{A}\) is an invertible matrix)”,
%@see: https://mathworld.wolfram.com/InvertibleMatrix.html
或称“\(\vb{A}\) \DefineConcept{可逆}(\(\vb{A}\) is invertible)”;
并称“\(\vb{B}\)是\(\vb{A}\)的\DefineConcept{逆矩阵}(inverse matrix)”,
记作\(\vb{A}^{-1}\),
即\begin{equation*}
	(\forall \vb{A},\vb{B}\in M_n(K))
	[\vb{A}^{-1} = \vb{B} \defiff \vb{A}\vb{B}=\vb{B}\vb{A}=\vb{E}].
\end{equation*}
\end{definition}

从定义可知,如果矩阵\(\vb{A},\vb{B}\)满足\(\vb{A}\vb{B}=\vb{B}\vb{A}=\vb{E}\),
那么这两个矩阵都是可逆矩阵,且两者互为逆矩阵.

\begin{definition}
设\(\vb{A} \in M_n(K)\).
\begin{itemize}
	\item 若\(\abs{\vb{A}}=0\),
	则称“\(\vb{A}\)是\DefineConcept{奇异矩阵}(singular matrix)”.
	%@see: https://mathworld.wolfram.com/SingularMatrix.html
	\item 若\(\abs{\vb{A}} \neq 0\),
	则称“\(\vb{A}\)是\DefineConcept{非奇异矩阵}(nonsingular matrix)”.
	%@see: https://mathworld.wolfram.com/NonsingularMatrix.html
\end{itemize}
\end{definition}

\begin{definition}
若\(\vb{A}\)是可逆矩阵,那么规定:
对于正整数\(k\),有
\begin{equation}
	\vb{A}^{-k} = (\vb{A}^{-1})^k
	= \underbrace{\vb{A}^{-1}\vb{A}^{-1}\dotsm\vb{A}^{-1}}_{\text{$k$个}}.
\end{equation}
\end{definition}

\begin{theorem}\label{theorem:逆矩阵.矩阵可逆的充分必要条件1}
%@see: 《线性代数》(张慎语、周厚隆) P43 定理1
设\(\vb{A}\)是\(n\)阶方阵,则“\(\vb{A}\)可逆”的充分必要条件是“\(\vb{A}\)是非奇异矩阵”.
\begin{proof}
必要性.
假设矩阵\(\vb{A}\)可逆,那么存在\(n\)阶方阵\(\vb{B}\),使得\(\vb{A}\vb{B}=\vb{E}\),于是\(\abs{\vb{A}\vb{B}}=\abs{\vb{E}}\);
而根据\cref{theorem:行列式.矩阵乘积的行列式},
\(\abs{\vb{A}\vb{B}}=\abs{\vb{A}}\abs{\vb{B}}=1\),\(\abs{\vb{A}}\neq0\).

充分性.
设\(\abs{\vb{A}}\neq0\),\(\vb{A}^*\)是\(\vb{A}\)的伴随矩阵.
根据\cref{equation:行列式.伴随矩阵.恒等式1},
若令\begin{equation*}
	\vb{B}=\frac{1}{\abs{\vb{A}}} \vb{A}^*,
\end{equation*}
则有\(\vb{A}\vb{B} = \vb{B}\vb{A} = \vb{E}\),
故由可逆矩阵的定义可知,矩阵\(\vb{A}\)可逆,
且有\(\vb{A}^{-1} = \vb{B}\).
\end{proof}
\end{theorem}

\begin{property}\label{theorem:逆矩阵.逆矩阵的唯一性}
设\(\vb{A}\)是可逆矩阵,则它的逆矩阵存在且唯一,
且有\begin{equation}
	\vb{A}^{-1} = \abs{\vb{A}}^{-1} \vb{A}^*.
\end{equation}
\begin{proof}
存在性.
在\cref{theorem:逆矩阵.矩阵可逆的充分必要条件1} 的证明过程中,
我们看到矩阵\(\vb{B}=\abs{\vb{A}}^{-1} \vb{A}^*\)是可逆矩阵\(\vb{A}\)的一个逆矩阵,即\(\vb{A}\vb{B}=\vb{E}\).

唯一性.
设矩阵\(\vb{C}\)也是\(\vb{A}\)的逆矩阵,即\(\vb{C}\vb{A}=\vb{E}\),于是\begin{equation*}
	\vb{C}=\vb{C}\vb{E}=\vb{C}(\vb{A}\vb{B})=(\vb{C}\vb{A})\vb{B}=\vb{E}\vb{B}=\vb{B}.
	\qedhere
\end{equation*}
\end{proof}
\end{property}

\begin{property}\label{theorem:逆矩阵.单位矩阵可逆}
单位矩阵\(\vb{E}\)可逆,且\(\vb{E}^{-1}=\vb{E}\).
\end{property}

\begin{property}\label{theorem:逆矩阵.逆矩阵的行列式}
%@see: 《线性代数》(张慎语、周厚隆) P44 性质3
设\(\vb{A}\)可逆,则\(\abs{\vb{A}^{-1}} = \abs{\vb{A}}^{-1}\).
\end{property}

\begin{property}\label{theorem:逆矩阵.逆矩阵的逆}
%@see: 《线性代数》(张慎语、周厚隆) P44 性质4
设\(\vb{A}\)可逆,则\(\vb{A}^{-1}\)可逆,且\((\vb{A}^{-1})^{-1} = \vb{A}\).
\end{property}

\begin{property}\label{theorem:逆矩阵.矩阵乘积的逆1}
%@see: 《线性代数》(张慎语、周厚隆) P44 性质5
设\(\vb{A}\)、\(\vb{B}\)都是\(n\)阶可逆矩阵,则\(\vb{A}\vb{B}\)可逆,且\begin{equation}
	(\vb{A} \vb{B})^{-1} = \vb{B}^{-1} \vb{A}^{-1}.
\end{equation}
\begin{proof}
因为\(\vb{A},\vb{B}\)都可逆,可设它们的逆矩阵分别为\(\vb{A}^{-1},\vb{B}^{-1}\),
于是\begin{equation*}
	(\vb{A}\vb{B})(\vb{B}^{-1}\vb{A}^{-1})
	= \vb{A}(\vb{B}\vb{B}^{-1})\vb{A}^{-1}
	= \vb{A}\vb{E}\vb{A}^{-1}
	= \vb{A}\vb{A}^{-1}
	= \vb{E}.
	\qedhere
\end{equation*}
\end{proof}
\end{property}

\cref{theorem:逆矩阵.矩阵乘积的逆1} 可以推广到有限个\(n\)阶可逆矩阵乘积的情形.
\begin{property}\label{theorem:逆矩阵.矩阵乘积的逆2}
设\(\AutoTuple{\vb{A}}{n}\)都是\(n\)阶可逆矩阵,
则\(\vb{A}_1 \vb{A}_2 \dotsm \vb{A}_{n-1} \vb{A}_n\)可逆,且\begin{equation}
	(\vb{A}_1 \vb{A}_2 \dotsm \vb{A}_{n-1} \vb{A}_n)^{-1}
	= \vb{A}_n^{-1} \vb{A}_{n-1}^{-1} \dotsm \vb{A}_2^{-1} \vb{A}_1^{-1}.
\end{equation}
\end{property}

\begin{property}\label{theorem:逆矩阵.数与矩阵乘积的逆}
%@see: 《线性代数》(张慎语、周厚隆) P44 性质6
设数域\(K\)上的\(n\)阶矩阵\(\vb{A}\)可逆,
\(k \in K-\{0\}\),则\(k\vb{A}\)可逆,且
\begin{equation}
	(k \vb{A})^{-1} = k^{-1} \vb{A}^{-1}.
\end{equation}
\begin{proof}
由\cref{theorem:行列式.性质2.推论2},
\(\abs{k\vb{A}} = k^n\abs{\vb{A}}\).
因为\(\vb{A}\)可逆,所以\(\abs{\vb{A}}\neq0\);
又因为\(k\neq0\),所以\(\abs{k\vb{A}}\neq0\),即\(k\vb{A}\)可逆.
因此\begin{equation*}
	(k^{-1}\vb{A}^{-1})(k\vb{A})
	= (k^{-1} \cdot k)(\vb{A}^{-1}\vb{A})
	= 1 \vb{E} = \vb{E},
\end{equation*}
也就是说\(k^{-1}\vb{A}^{-1}\)是\(k\vb{A}\)的逆矩阵.
\end{proof}
%\cref{equation:行列式.伴随矩阵.数与矩阵乘积的伴随}
\end{property}

\begin{property}\label{theorem:逆矩阵.转置矩阵的逆与逆矩阵的转置}
%@see: 《线性代数》(张慎语、周厚隆) P44 性质7
设\(\vb{A}\)可逆,则\(\vb{A}^T\)可逆,且\begin{equation}
	(\vb{A}^T)^{-1} = (\vb{A}^{-1})^T.
\end{equation}
\begin{proof}
由\cref{theorem:行列式.性质1},
\(\abs{\vb{A}^T}=\abs{\vb{A}}\neq0\),
于是\(\vb{A}^T\)可逆.
由\cref{theorem:矩阵.矩阵乘积的转置},
\((\vb{A} \vb{A}^{-1})^T = (\vb{A}^{-1})^T \vb{A}^T\).
既然\(\vb{A} \vb{A}^{-1} = \vb{E}, \vb{E}^T = \vb{E}\),
于是\((\vb{A}^{-1})^T \vb{A}^T = \vb{E}\),
那么由逆矩阵的定义可知,
\((\vb{A}^T)^{-1}=(\vb{A}^{-1})^T\).
\end{proof}
\end{property}

\begin{example}
设矩阵\(\vb{A}\)、\(\vb{B}\)可交换,\(\vb{A}\)可逆.
证明:\(\vb{A}^{-1}\)与\(\vb{B}\)可交换.
\begin{proof}
因为\(\vb{A}\vb{B} = \vb{B}\vb{A}\),在等式两边同时左乘\(\vb{A}^{-1}\),得\begin{equation*}
	\vb{B} = (\vb{A}^{-1}\vb{A})\vb{B} = \vb{A}^{-1}(\vb{A}\vb{B}) = \vb{A}^{-1}(\vb{B}\vb{A});
\end{equation*}
再在等式两边右乘\(\vb{A}^{-1}\),得\begin{equation*}
	\vb{B}\vb{A}^{-1} = (\vb{A}^{-1}\vb{B}\vb{A})\vb{A}^{-1} = \vb{A}^{-1}\vb{B}(\vb{A}\vb{A}^{-1}) = \vb{A}^{-1}\vb{B}.
	\qedhere
\end{equation*}
\end{proof}
\end{example}

\begin{example}
下面看一些常见矩阵的逆矩阵:\begin{gather*}
	\begin{bmatrix}
		a_{11} & a_{12} \\
		a_{21} & a_{22}
	\end{bmatrix}
	= \begin{vmatrix}
		a_{11} & a_{12} \\
		a_{21} & a_{22}
	\end{vmatrix}^{-1}
	\begin{bmatrix}
		a_{22} & -a_{12} \\
		-a_{21} & a_{11}
	\end{bmatrix}, \\
	\begin{bmatrix}
		\lambda_1 \\
		& \lambda_2 \\
		&& \ddots \\
		&&& \lambda_n
	\end{bmatrix}^{-1}
	= \begin{bmatrix}
		\lambda_1^{-1} \\
		& \lambda_2^{-1} \\
		&& \ddots \\
		&&& \lambda_n^{-1}
	\end{bmatrix}, \\
	\begin{bmatrix}
		& & & & \lambda_1 \\
		& & & \lambda_2 \\
		& & \iddots \\
		& \lambda_{n-1} \\
		\lambda_n
	\end{bmatrix}^{-1}
	= \begin{bmatrix}
		& & & & \lambda_n^{-1} \\
		& & & \lambda_{n-1}^{-1} \\
		& & \iddots \\
		& \lambda_2^{-1} \\
		\lambda_1^{-1}
	\end{bmatrix}.
\end{gather*}
\end{example}

\begin{example}\label{theorem:逆矩阵.伴随矩阵的逆与逆矩阵的伴随}
%@see: 《线性代数》(张慎语、周厚隆) P46 例4
设\(\vb{A}\)可逆.
证明:\(\vb{A}\)的伴随矩阵\(\vb{A}^*\)可逆,
且\begin{equation}
	(\vb{A}^*)^{-1}
	= \abs{\vb{A}}^{-1} \vb{A}
	= (\vb{A}^{-1})^*.
\end{equation}
\begin{proof}
因为\begin{align*}
	(\vb{A}^*)^{-1}
	&= \left( \abs{\vb{A}} \vb{A}^{-1} \right)^{-1}
		\tag{\cref{theorem:逆矩阵.逆矩阵的唯一性}} \\
	&= \abs{\vb{A}}^{-1} (\vb{A}^{-1})^{-1}
		\tag{\cref{theorem:逆矩阵.数与矩阵乘积的逆}} \\
	&= \abs{\vb{A}}^{-1} \vb{A}
		\tag{\cref{theorem:逆矩阵.逆矩阵的逆}} \\
	&= \abs{\vb{A}^{-1}} (\vb{A}^{-1})^{-1}
		\tag{\cref{theorem:逆矩阵.逆矩阵的行列式}} \\
	&= (\vb{A}^{-1})^*,
		\tag{\cref{theorem:逆矩阵.逆矩阵的唯一性}}
\end{align*}
所以\((\vb{A}^*)^{-1}
= \abs{\vb{A}}^{-1} \vb{A}
= (\vb{A}^{-1})^*\).
\end{proof}
\end{example}
\begin{example}
%@see: 《高等代数学习指导书(第三版)》(姚慕生、谢启鸿) P62 例2.23
设\(\vb{A} \in M_n(K)\)满足\(\vb{A}^m = \vb{E}\),
其中\(\vb{E}\)是数域\(K\)上的\(n\)阶单位矩阵.
证明:\((\vb{A}^*)^m = \vb{E}\).
\begin{proof}
由\(\vb{A}^m = \vb{E}\)得\(\abs{\vb{A}}^m = 1\),\(\vb{A}\)可逆,
那么\(\vb{A}^* = \abs{\vb{A}} \vb{A}^{-1}\),
于是\begin{equation*}
	(\vb{A}^*)^m
	= (\abs{\vb{A}} \vb{A}^{-1})^m
	= \abs{\vb{A}}^m (\vb{A}^{-1})^m
	= (\vb{A}^m)^{-1}
	= \vb{E}.
	\qedhere
\end{equation*}
\end{proof}
\end{example}

\begin{example}\label{example:对合矩阵.对合矩阵的逆矩阵}
设\(\vb{A}\)是数域\(K\)上的\(n\)阶对合矩阵.
证明\(\vb{A}^{-1} = \vb{A}\).
\begin{proof}
假设\(\vb{A}^2=\vb{E}\),
其中\(\vb{E}\)是数域\(K\)上的\(n\)阶单位矩阵,
那么由\cref{theorem:行列式.矩阵乘积的行列式} 可知\begin{equation*}
	\abs{\vb{A}^2}
	= \abs{\vb{A}}^2
	= 1,
\end{equation*}
从而\(\abs{\vb{A}}\neq0\),
\(\vb{A}\)可逆,
因此\begin{equation*}
	\vb{A}^{-1}
	= \vb{A}^{-1}\vb{E}
	= \vb{A}^{-1}(\vb{A}^2)
	= (\vb{A}^{-1}\vb{A})\vb{A}
	= \vb{E}\vb{A}
	= \vb{A}.
	\qedhere
\end{equation*}
\end{proof}
\end{example}

\begin{example}\label{example:可逆矩阵.分块上三角矩阵的逆}
%@see: 《线性代数》(张慎语、周厚隆) P46 例5
设\(\vb{A} \in M_s(K),
\vb{B} \in M_n(K),
\vb{C} \in M_{s \times n}(K)\),
\(\vb{A}\)和\(\vb{B}\)都可逆.
证明:矩阵\begin{equation*}
	\vb{M} = \begin{bmatrix}
		\vb{A} & \vb{C} \\
		\vb0 & \vb{B}
	\end{bmatrix}
\end{equation*}可逆,且\begin{equation*}
	\vb{M}^{-1} = \begin{bmatrix}
		\vb{A}^{-1} & -\vb{A}^{-1} \vb{C} \vb{B}^{-1} \\
		\vb0 & \vb{B}^{-1}
	\end{bmatrix}.
\end{equation*}
\begin{proof}
因为\(\vb{A}\)、\(\vb{B}\)为可逆矩阵,\(\abs{\vb{A}} \neq 0\),\(\abs{\vb{B}} \neq 0\).
所以\(\abs{\vb{M}}=\abs{\vb{A}}\abs{\vb{B}} \neq 0\),即\(\vb{M}\)可逆.

令\(\vb{M}\vb{x}=\vb{E}\),即\begin{equation*}
	\begin{bmatrix}
		\vb{A} & \vb{C} \\
		\vb0 & \vb{B}
	\end{bmatrix}
	\begin{bmatrix}
		\vb{x}_1 & \vb{x}_2 \\
		\vb{x}_3 & \vb{x}_4
	\end{bmatrix}
	= \begin{bmatrix}
		\vb{E} & \vb0 \\
		\vb0 & \vb{E}
	\end{bmatrix}
\end{equation*}则\begin{equation*}
	\begin{bmatrix}
		\vb{A}\vb{x}_1+\vb{C}\vb{x}_3 & \vb{A}\vb{x}_2+\vb{C}\vb{x}_4 \\
		\vb{B}\vb{x}_3 & \vb{B}\vb{x}_4
	\end{bmatrix}
	= \begin{bmatrix}
		\vb{E} & \vb0 \\
		\vb0 & \vb{E}
	\end{bmatrix}
\end{equation*}
进而有\begin{equation*}
	\left\{ \begin{array}{l}
		\vb{A}\vb{x}_1+\vb{C}\vb{x}_3 = \vb{E} \\
		\vb{A}\vb{x}_2+\vb{C}\vb{x}_4 = \vb0 \\
		\vb{B}\vb{x}_3 = \vb0 \\
		\vb{B}\vb{x}_4 = \vb{E}
	\end{array} \right.
\end{equation*}
由第4式可得\(\vb{x}_4 = \vb{B}^{-1}\).
代入第2式得\(\vb{A}\vb{x}_2=-\vb{C}\vb{B}^{-1}\),
\(\vb{x}_2=-\vb{A}^{-1}\vb{C}\vb{B}^{-1}\).
用\(\vb{B}^{-1}\)左乘第3式左右两端,\(\vb{B}^{-1}\vb{B}\vb{x}_3=\vb{x}_3=\vb0\).
则第1式化为\(\vb{A}\vb{x}_1=\vb{E}\),显然\(\vb{x}_1=\vb{A}^{-1}\),所以\begin{equation*}
	\vb{M}^{-1} = \vb{x} = \begin{bmatrix}
		\vb{A}^{-1} & -\vb{A}^{-1}\vb{C}\vb{B}^{-1} \\
		\vb0 & \vb{B}^{-1}
	\end{bmatrix}.
	\qedhere
\end{equation*}
\end{proof}
\end{example}

\begin{remark}
从\cref{example:可逆矩阵.分块上三角矩阵的逆} 的结论\begin{equation*}
	\begin{bmatrix}
		\vb{A} & \vb{C} \\
		\vb0 & \vb{B}
	\end{bmatrix}^{-1}
	= \begin{bmatrix}
		\vb{A}^{-1} & -\vb{A}^{-1} \vb{C} \vb{B}^{-1} \\
		\vb0 & \vb{B}^{-1}
	\end{bmatrix}
\end{equation*}出发,
任取\(\vb{D} \in M_{n \times s}(K)\),
我们还可以得到\begin{equation*}
	\begin{bmatrix}
		\vb{A} & \vb0 \\
		\vb{D} & \vb{B}
	\end{bmatrix}^{-1}
	= \begin{bmatrix}
		\vb{A}^{-1} & \vb0 \\
		-\vb{B}^{-1} \vb{D} \vb{A}^{-1} & \vb{B}^{-1}
	\end{bmatrix},
\end{equation*}\begin{equation*}
	\begin{bmatrix}
		\vb{C} & \vb{A} \\
		\vb{B} & \vb0
	\end{bmatrix}^{-1}
	= \begin{bmatrix}
		\vb0 & \vb{B}^{-1} \\
		\vb{A}^{-1} & -\vb{A}^{-1}\vb{C}\vb{B}^{-1}
	\end{bmatrix},
\end{equation*}\begin{equation*}
	\begin{bmatrix}
		\vb0 & \vb{A} \\
		\vb{B} & \vb{D}
	\end{bmatrix}^{-1}
	= \begin{bmatrix}
		-\vb{B}^{-1}\vb{D}\vb{A}^{-1} & \vb{B}^{-1} \\
		\vb{A}^{-1} & \vb0
	\end{bmatrix}.
\end{equation*}

我们还可以进一步利用\cref{theorem:逆矩阵.逆矩阵的唯一性}
以及\cref{equation:行列式.广义三角阵的行列式1,equation:行列式.广义三角阵的行列式2},
得到\begin{equation*}
	\begin{bmatrix}
		\vb{A} & \vb{C} \\
		\vb0 & \vb{B}
	\end{bmatrix}^*
	= \begin{bmatrix}
		\abs{\vb{B}} \vb{A}^* & -\vb{A}^*\vb{C}\vb{B}^* \\
		\vb0 & \abs{\vb{A}} \vb{B}^*
	\end{bmatrix},
\end{equation*}\begin{equation*}
	\begin{bmatrix}
		\vb{A} & \vb0 \\
		\vb{D} & \vb{B}
	\end{bmatrix}^*
	= \begin{bmatrix}
		\abs{\vb{B}} \vb{A}^* & \vb0 \\
		-\vb{B}^* \vb{D} \vb{A}^* & \abs{\vb{A}} \vb{B}^*
	\end{bmatrix},
\end{equation*}\begin{equation*}
	\begin{bmatrix}
		\vb{C} & \vb{A} \\
		\vb{B} & \vb0
	\end{bmatrix}^*
	= (-1)^{sn} \begin{bmatrix}
		\vb0 & \abs{\vb{A}} \vb{B}^* \\
		\abs{\vb{B}} \vb{A}^* & -\vb{A}^*\vb{C}\vb{B}^*
	\end{bmatrix},
\end{equation*}\begin{equation*}
	\begin{bmatrix}
		\vb0 & \vb{A} \\
		\vb{B} & \vb{D}
	\end{bmatrix}^*
	= (-1)^{sn} \begin{bmatrix}
		-\vb{B}^*\vb{D}\vb{A}^* & \abs{\vb{A}} \vb{B}^* \\
		\abs{\vb{B}} \vb{A}^* & \vb0
	\end{bmatrix}.
\end{equation*}
\end{remark}
