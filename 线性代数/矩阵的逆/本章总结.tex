\section{本章总结}
\subsection*{矩阵可逆的等价条件}
%@see: https://mathworld.wolfram.com/InvertibleMatrixTheorem.html
矩阵\(\vb{A} \in M_n(K)\)可逆的充分必要条件:\begin{itemize}
	\item 矩阵\(\vb{A}\)等价于数域\(K\)上的\(n\)阶单位矩阵.
	\item 存在数域\(K\)上的\(n\)阶矩阵\(\vb{B}\),使得\(\vb{B}\vb{A}\)或\(\vb{A}\vb{B}\)等于数域\(K\)上的\(n\)阶单位矩阵.
	\item 方程\(\vb{A}\vb{x}=\vb0\)只有零解\(\vb{x}=\vb0\).
	\item 矩阵\(\vb{A}\)的列向量线性无关.
	\item 线性变换\(\vb{x} \mapsto \vb{A}\vb{x}\)是双射.
	\item 线性变换\(\vb{x} \mapsto \vb{A}\vb{x}\)是满射.
	\item 对于任意向量\(\vb\beta \in K^n\),方程\(\vb{A}\vb{x}=\vb\beta\)有唯一解.
	\item 矩阵\(\vb{A}\)的行空间是\(K^n\).
	\item 矩阵\(\vb{A}\)的列空间是\(K^n\).
	\item 矩阵\(\vb{A}\)的列向量组是向量空间\(K^n\)的一组基.
	\item 矩阵\(\vb{A}\)的列向量组可以张成向量空间\(K^n\).
	\item 矩阵\(\vb{A}\)的转置矩阵\(\vb{A}^T\)可逆.
	\item 矩阵\(\vb{A}\)的列空间的维数等于\(n\).
	\item 矩阵\(\vb{A}\)的秩等于\(n\).
	\item 矩阵\(\vb{A}\)的零空间是\(\{\vb0\}\).
	\item 矩阵\(\vb{A}\)的零空间的维数等于\(0\).
	\item \(0\)不是矩阵\(\vb{A}\)的特征值.
	\item 矩阵\(\vb{A}\)的行列式不等于零.
	\item 矩阵\(\vb{A}\)是非奇异矩阵.
	% \item 矩阵\(\vb{A}\)的列空间的正交补是\(\{\vb0\}\).
	% \item 矩阵\(\vb{A}\)的零空间的正交补是向量空间\(K^n\).
\end{itemize}

\subsection*{重要公式}
假设\(\vb{A},\vb{B}\)可逆,则\begin{gather*}
	\vb{A}^{-1} = \abs{\vb{A}}^{-1} \vb{A}^*, \\ %\cref{theorem:逆矩阵.逆矩阵的唯一性}
	\abs{\vb{A}^{-1}} = \abs{\vb{A}}^{-1}, \\ %\cref{theorem:逆矩阵.逆矩阵的行列式}
	(\vb{A}^{-1})^{-1} = \vb{A}, \\ %\cref{theorem:逆矩阵.逆矩阵的逆}
	(\vb{A} \vb{B})^{-1} = \vb{B}^{-1} \vb{A}^{-1}, \\ %\cref{theorem:逆矩阵.矩阵乘积的逆1,theorem:逆矩阵.矩阵乘积的逆2}
	(k \vb{A})^{-1} = k^{-1} \vb{A}^{-1}, \\ %\cref{theorem:逆矩阵.数与矩阵乘积的逆}
	(\vb{A}^T)^{-1} = (\vb{A}^{-1})^T, \\ %\cref{theorem:逆矩阵.转置矩阵的逆与逆矩阵的转置}
	\diag(\AutoTuple{\lambda}{n}) = \diag(\AutoTuple{\lambda^{-1}}{n}), \\
	(\vb{A}^*)^{-1}
	= \abs{\vb{A}}^{-1} \vb{A}
	= (\vb{A}^{-1})^*, \\ %\cref{theorem:逆矩阵.伴随矩阵的逆与逆矩阵的伴随}
	%\cref{example:可逆矩阵.分块上三角矩阵的逆}
	\begin{bmatrix}
		\vb{A} & \vb{C} \\
		\vb0 & \vb{B}
	\end{bmatrix}^{-1}
	= \begin{bmatrix}
		\vb{A}^{-1} & -\vb{A}^{-1} \vb{C} \vb{B}^{-1} \\
		\vb0 & \vb{B}^{-1}
	\end{bmatrix}, \\
	\begin{bmatrix}
		\vb{A} & \vb0 \\
		\vb{D} & \vb{B}
	\end{bmatrix}^{-1}
	= \begin{bmatrix}
		\vb{A}^{-1} & \vb0 \\
		-\vb{B}^{-1} \vb{D} \vb{A}^{-1} & \vb{B}^{-1}
	\end{bmatrix}, \\
	\begin{bmatrix}
		\vb{C} & \vb{A} \\
		\vb{B} & \vb0
	\end{bmatrix}^{-1}
	= \begin{bmatrix}
		\vb0 & \vb{B}^{-1} \\
		\vb{A}^{-1} & -\vb{A}^{-1}\vb{C}\vb{B}^{-1}
	\end{bmatrix}, \\
	\begin{bmatrix}
		\vb0 & \vb{A} \\
		\vb{B} & \vb{D}
	\end{bmatrix}^{-1}
	= \begin{bmatrix}
		-\vb{B}^{-1}\vb{D}\vb{A}^{-1} & \vb{B}^{-1} \\
		\vb{A}^{-1} & \vb0
	\end{bmatrix}, \\
	\begin{bmatrix}
		\vb{A} & \vb{C} \\
		\vb0 & \vb{B}
	\end{bmatrix}^*
	= \begin{bmatrix}
		\abs{\vb{B}} \vb{A}^* & -\vb{A}^*\vb{C}\vb{B}^* \\
		\vb0 & \abs{\vb{A}} \vb{B}^*
	\end{bmatrix}, \\
	\begin{bmatrix}
		\vb{A} & \vb0 \\
		\vb{D} & \vb{B}
	\end{bmatrix}^*
	= \begin{bmatrix}
		\abs{\vb{B}} \vb{A}^* & \vb0 \\
		-\vb{B}^* \vb{D} \vb{A}^* & \abs{\vb{A}} \vb{B}^*
	\end{bmatrix}, \\
	\vb{A} \in M_s(K),
	\vb{B} \in M_n(K)
	\implies
	\begin{bmatrix}
		\vb{C} & \vb{A} \\
		\vb{B} & \vb0
	\end{bmatrix}^*
	= (-1)^{sn} \begin{bmatrix}
		\vb0 & \abs{\vb{A}} \vb{B}^* \\
		\abs{\vb{B}} \vb{A}^* & -\vb{A}^*\vb{C}\vb{B}^*
	\end{bmatrix}, \\
	\vb{A} \in M_s(K),
	\vb{B} \in M_n(K)
	\implies
	\begin{bmatrix}
		\vb0 & \vb{A} \\
		\vb{B} & \vb{D}
	\end{bmatrix}^*
	= (-1)^{sn} \begin{bmatrix}
		-\vb{B}^*\vb{D}\vb{A}^* & \abs{\vb{A}} \vb{B}^* \\
		\abs{\vb{B}} \vb{A}^* & \vb0
	\end{bmatrix}. \\
	%\cref{equation:逆矩阵.行列式降阶公式1}
	\text{$\vb{A}$可逆}
	\implies
	\begin{vmatrix}
		\vb{A} & \vb{B} \\
		\vb{C} & \vb{D}
	\end{vmatrix}
	= \abs{\vb{A}} \abs{\vb{D} - \vb{C} \vb{A}^{-1} \vb{B}}, \\
	%\cref{equation:逆矩阵.行列式降阶公式2}
	\text{$\vb{D}$可逆}
	\implies
	\begin{vmatrix}
		\vb{A} & \vb{B} \\
		\vb{C} & \vb{D}
	\end{vmatrix}
	= \abs{\vb{D}} \abs{\vb{A} - \vb{B} \vb{D}^{-1} \vb{C}}. \\
	%\cref{example:逆矩阵.行列式降阶定理的重要应用1}
	\vb{A} \in M_{s \times n}(K),
	\vb{B} \in M_{n \times s}(K)
	\implies
	\begin{vmatrix}
		\vb{E}_n & \vb{B} \\
		\vb{A} & \vb{E}_s
	\end{vmatrix}
	= \abs{\vb{E}_s - \vb{A}\vb{B}}.
\end{gather*}
