\section{初等矩阵}
\subsection{初等变换}
\begin{definition}
对矩阵施行以下变换,称为矩阵的\DefineConcept{初等行变换}(elementary row operation):
\begin{enumerate}
	\item 互换两行的位置;
	\item 用一非零数\(c\)乘以某行;
	\item 将某行的\(k\)倍加到另一行.
\end{enumerate}
类似地,可以定义矩阵的\DefineConcept{初等列变换}(elementary column operation):
\begin{enumerate}
	\item 互换两列的位置;
	\item 用一非零数\(c\)乘以某列;
	\item 将某列的\(k\)倍加到另一列.
\end{enumerate}
矩阵的初等行变换、初等列变换统称为矩阵的\DefineConcept{初等变换}(elementary operation).
\end{definition}

%我们约定:
%矩阵\(\vb{A}\)经过一次初等行变换\(\sigma_1\)化为矩阵\(\vb{B}\)的过程
%可以表示为在连接矩阵\(\vb{A}\)和\(\vb{B}\)的箭头上方标记\(\sigma_1\),即\begin{equation*}
%	\vb{A} \xlongrightarrow{\sigma_1} \vb{B};
%\end{equation*}而矩阵\(\vb{A}\)经过一次初等列变换\(\sigma_2\)化为矩阵\(\vb{B}\)的过程
%可以表示为在连接矩阵\(\vb{A}\)和\(\vb{B}\)的箭头下方标记\(\sigma_2\),即\begin{equation*}
%	\vb{A} \xlongrightarrow[\sigma_2]{} \vb{B}.
%\end{equation*}

\subsection{初等矩阵的概念}
\begin{definition}\label{definition:逆矩阵.矩阵等价}
若矩阵\(\vb{A}\)可以经过一系列初等变换化为矩阵\(\vb{B}\),
则称“\(\vb{A}\)与\(\vb{B}\)~\DefineConcept{等价}(equivalent)”,
或“\(\vb{A}\)与\(\vb{B}\)~\DefineConcept{相抵}”,
记作\(\vb{A}\cong\vb{B}\).
\end{definition}

\begin{definition}
由\(n\)阶单位矩阵\(\vb{E}\)经过\emph{一次}初等变换所得矩阵
称为\(n\)阶\DefineConcept{初等矩阵}(elementary matrix).
\end{definition}

对应于矩阵的三类初等变换,有三种类型的初等矩阵:
\begin{enumerate}
	\item 互换\(\vb{E}\)的\(i\),\(j\)两行(列)所得的矩阵\begin{equation*}
		\vb{P}(i,j) = \begin{bmatrix}
			\vb{E}_{i-1} & & & \\
			& 0 & & 1 & \\
			& & \vb{E}_{j-i-1} & & \\
			& 1 & & 0 & \\
			& & & & \vb{E}_{n-j}
		\end{bmatrix}_n;
	\end{equation*}
	\item 用非零数\(c\)乘以\(\vb{E}\)的第\(i\)行(列)所得的矩阵\begin{equation*}
		\vb{P}(i(c)) = \begin{bmatrix}
			\vb{E}_{i-1} & & \\
			& c & \\
			& & \vb{E}_{n-i}
		\end{bmatrix}_n;
	\end{equation*}
	\item 把\(\vb{E}\)的第\(j\)行(第\(i\)列)的\(k\)倍加到第\(i\)行(第\(j\)列)所得的矩阵\begin{equation*}
		\vb{P}(i,j(k)) = \begin{bmatrix}
			\vb{E}_{i-1} & & & \\
			& 1 & & k & \\
			& & \vb{E}_{j-i-1} & & \\
			& 0 & & 1 & \\
			& & & & \vb{E}_{n-j}
		\end{bmatrix}_n.
	\end{equation*}
\end{enumerate}

\subsection{初等矩阵的性质}
\begin{property}\label{theorem:逆矩阵.初等矩阵的性质1}
初等矩阵具有以下性质:\begin{gather}
	\abs{\vb{P}(i,j)} = -1, \\
	\abs{\vb{P}(i(c))} = c, \\
	\abs{\vb{P}(i,j(k))} = 1, \\
	\vb{P}(i,j)^T = \vb{P}(i,j), \\
	\vb{P}(i(c))^T = \vb{P}(i(c)), \\
	\vb{P}(i,j(k))^T = \vb{P}(j,i(k)), \\
	\vb{P}(i,j)^{-1} = \vb{P}(i,j), \\
	\vb{P}(i(c))^{-1} = \vb{P}(i(c^{-1})), \\
	\vb{P}(i,j(k))^{-1} = \vb{P}(i,j(-k)).
\end{gather}
\end{property}

\begin{property}\label{theorem:逆矩阵.初等矩阵的性质2}
对\(n \times t\)矩阵\(\vb{A}\)施行一次初等行变换,相当于用一个相应的\(n\)阶初等矩阵左乘\(\vb{A}\);
对\(\vb{A}\)施行一次初等列变换,相当于用一个相应的\(t\)阶初等矩阵右乘\(\vb{A}\).
\begin{proof}
用\(n\)阶矩阵\(\vb{P}(i,j)\)左乘\(\vb{A}\),将矩阵\(\vb{A}\)作相应分块,有\begin{equation*}
	\vb{P}(i,j) \vb{A} = \begin{bmatrix}
		\vb{E}_{i-1} \\
		& 0 & & 1 \\
		& & \vb{E}_{j-i-1} \\
		& 1 & & 0 \\
		& & & & \vb{E}_{n-j}
	\end{bmatrix}
	\begin{bmatrix}
		\vb{A}_1 \\ \vb\alpha_i \\ \vb{A}_2 \\ \vb\alpha_j \\ \vb{A}_3
	\end{bmatrix}
	= \begin{bmatrix}
		\vb{A}_1 \\ \vb\alpha_j \\ \vb{A}_2 \\ \vb\alpha_i \\ \vb{A}_3
	\end{bmatrix},
\end{equation*}
即\(\vb{A}\)交换\(i\)、\(j\)两行.

用\(n\)阶矩阵\(\vb{P}(i(c))\)左乘\(\vb{A}\),将矩阵\(\vb{A}\)作相应分块,有\begin{equation*}
	\vb{P}(i(c)) \vb{A} = \begin{bmatrix}
		\vb{E}_{i-1} \\
		& c \\
		& & \vb{E}_{n-i}
	\end{bmatrix}
	\begin{bmatrix}
		\vb{A}_1 \\ \vb\alpha_i \\ \vb{A}_2
	\end{bmatrix}
	= \begin{bmatrix}
		\vb{A}_1 \\ c \vb\alpha_i \\ a_2
	\end{bmatrix},
\end{equation*}
即用一非零数\(c\)乘以第\(i\)行.

用\(n\)阶矩阵\(\vb{P}(i,j(k))\ (i < j)\)左乘\(\vb{A}\),将矩阵\(\vb{A}\)作相应分块,有\begin{equation*}
	\vb{P}(i,j(k)) \vb{A} = \begin{bmatrix}
		\vb{E}_{i-1} \\
		& 1 & & k \\
		& & \vb{E}_{j-i-1} \\
		& 0 & & 1 \\
		& & & & \vb{E}_{n-j}
	\end{bmatrix}
	\begin{bmatrix}
		\vb{A}_1 \\ \vb\alpha_i \\ \vb{A}_2 \\ \vb\alpha_j \\ \vb{A}_3
	\end{bmatrix}
	= \begin{bmatrix}
		\vb{A}_1 \\ \vb\alpha_i + k \vb\alpha_j \\ \vb{A}_2 \\ \vb\alpha_j \\ \vb{A}_3
	\end{bmatrix},
\end{equation*}
即把\(\vb{A}\)第\(j\)行的\(k\)倍加到第\(i\)行.
\end{proof}
\end{property}
