\section{矩阵与向量的概念}
\subsection{矩阵与向量的基本概念}
%@see: https://math.stackexchange.com/a/1162585/591741
\begin{definition}
设\(s,n\)都是正整数,\(K\)是数域.
我们把从\(\{1,\dotsc,s\}\times\{1,\dotsc,n\}\)到\(K\)的映射,
称为“数域\(K\)上的\(s \times n\)~\DefineConcept{矩阵}(matrix)”.
%@see: https://mathworld.wolfram.com/Matrix.html
\end{definition}

为了简化描述,我们定义:\begin{equation*}
	M_{s \times n}(K)
	\defeq
	\Set{
		\vb{A} \given
		\text{\(\vb{A}\)是数域\(K\)上的\(s \times n\)矩阵}
	}.
\end{equation*}

设\(\vb{A} \in M_{s \times n}(K)\).
我们把\(s\)称为“\(\vb{A}\)的\DefineConcept{行数}”,
把\(n\)称为“\(\vb{A}\)的\DefineConcept{列数}”.

当我们把矩阵\(\vb{A} \in M_{s \times n}(K)\)
看作从\(\{1,\dotsc,s\}\times\{1,\dotsc,n\}\)到\(K\)的一个映射时,
把它在\((i,j)\)的值\begin{equation*}
	\vb{A}(i,j)
	\quad(i=1,2,\dotsc,s;j=1,2,\dotsc,n)
\end{equation*}
称为“\(\vb{A}\)的\((i,j)\)元素”
或“\(\vb{A}\)的第\(i\)行第\(j\)列\DefineConcept{元素}(element)”,
记作\(\MatrixEntry{\vb{A}}{i,j}\).
%@see: https://mathworld.wolfram.com/MatrixElement.html

对于任意给定的两个矩阵\(\vb{A}\)和\(\vb{B}\),
如果它们行数、列数都相同,
则称“\(\vb{A}\)与\(\vb{B}\) \DefineConcept{同型}”.

如果矩阵\(\vb{A}\)的行数\(s\)恰好等于它的列数\(n\),
我们就把它称为“\(n\)阶矩阵”或“\(n\)阶\DefineConcept{方阵}(square matrix)”.
%@see: https://mathworld.wolfram.com/SquareMatrix.html
定义:\begin{equation*}
	M_n(K)
	\defeq
	M_{n \times n}(K).
\end{equation*}

设\(\vb{A} \in M_{s \times n}(K)\).
我们把\begin{equation*}
	\MatrixEntry{\vb{A}}{k,k}
	\quad(k=1,2,\dotsc,\min\{s,n\})
\end{equation*}
称为“\(\vb{A}\)的\DefineConcept{主对角线}(main diagonal)上的元素”
或“\(\vb{A}\)的\DefineConcept{主对角元}”.
%@see: https://mathworld.wolfram.com/Diagonal.html
%@see: https://mathworld.wolfram.com/SkewDiagonal.html

我们把行数为\(1\)、列数为\(n\)的矩阵称为“\(n\)维\DefineConcept{行向量}(row vector)”;
%@see: https://mathworld.wolfram.com/RowVector.html
把列数为\(1\)、行数为\(n\)的矩阵称为“\(n\)维\DefineConcept{列向量}(column vector)”.
%@see: https://mathworld.wolfram.com/ColumnVector.html
\(n\)维行向量和\(n\)维列向量统称为“\(n\)维\DefineConcept{向量}(vector)”.
通常用一个黑体小写希腊字母(如\(\vb\alpha,\vb\beta,\vb\gamma\)等)表示向量.

设\(\vb\alpha\in M_{s \times 1}(K)\).
我们把\(s\)称为“\(\vb\alpha\)的\DefineConcept{维数}”.
定义:\begin{equation*}
	\vb\alpha(i) \defeq \vb\alpha(i,1).
\end{equation*}
把\(\vb\alpha(i)\ (i=1,2,\dotsc,s)\)称为“\(\vb\alpha\)的第\(i\)个\DefineConcept{分量}”.

设\(\vb\beta\in M_{1 \times n}(K)\).
把\(n\)称为“\(\vb\beta\)的\DefineConcept{维数}”.
定义:\begin{equation*}
	\vb\beta(i) \defeq \vb\beta(1,i).
\end{equation*}
把\(\vb\beta(i)\ (i=1,2,\dotsc,n)\)称为“\(\vb\beta\)的第\(i\)个\DefineConcept{分量}”.

有时候为了强调矩阵的行数和列数,
我们会在表示矩阵的字母的右下角标注矩阵的形状,例如,
如果\(\vb{A} \in M_{s \times n}(K)\),那么我们可以写出\(\vb{A}_{s \times n}\);
如果\(\vb{B}\in M_n(K)\),那么我们可以写出\(\vb{B}_n\).

“矩阵”之所以得名,是因为我们可以把矩阵\(\vb{A}\)写成\(s\)行\(n\)列的矩形表,再加上括号:\begin{equation*}
	\begin{bmatrix}
		a_{11} & a_{12} & \dots & a_{1n} \\
		a_{21} & a_{22} & \dots & a_{2n} \\
		\vdots & \vdots & & \vdots \\
		a_{s1} & a_{s2} & \dots & a_{sn}
	\end{bmatrix},
\end{equation*}
其中\(
	a_{ij}
	= \MatrixEntry{\vb{A}}{i,j}
	\ (i=1,2,\dotsc,s;j=1,2,\dotsc,n)
\).
有时候为了强调矩阵的元素,
我们会把\(\vb{A}_{s \times n}\)记作\(\vb{A}=(a_{ij})_{s \times n}\).

类似地,我们把行向量表示为\begin{equation*}
	\begin{bmatrix}
		a_1 & a_2 & \dots & a_n
	\end{bmatrix};
\end{equation*}
把列向量表示为\begin{equation*}
	\begin{bmatrix}
		a_1 \\ a_2 \\ \vdots \\ a_n
	\end{bmatrix}.
\end{equation*}

\subsection{子矩阵}
\begin{definition}
在矩阵\(\vb{A}=(a_{ij})_{s \times n}\)中,
任取\(k\)行\(l\)列,
位于这些行与列交叉处的\(kl\)个元素,按原顺序排成的\(k \times l\)矩阵\begin{equation*}
	\begin{bmatrix}
		a_{i_1,j_1} & a_{i_1,j_2} & \dots & a_{i_1,j_l} \\
		a_{i_2,j_1} & a_{i_2,j_2} & \dots & a_{i_2,j_l} \\
		\vdots & \vdots & & \vdots \\
		a_{i_k,j_1} & a_{i_k,j_2} & \dots & a_{i_k,j_l}
	\end{bmatrix},
	\quad
	\begin{array}{c}
		1 \leq i_1 < i_2 < \dotsb < i_k \leq s; \\
		1 \leq j_1 < j_2 < \dotsb < j_l \leq n
	\end{array}
\end{equation*}称为“\(\vb{A}\)的一个\(k\)阶\DefineConcept{子矩阵}(submatrix)”.
%@see: https://mathworld.wolfram.com/Submatrix.html
\end{definition}

\subsection{矩阵的分块}
\begin{definition}
设\(\vb{A} \in M_{s \times n}(K)\);
给定介于\(1\)和\(s\)之间的两个正整数\(a,b\),
以及介于\(1\)和\(n\)之间的两个正整数\(c,d\).
令\begin{equation*}
	X=\Set{ x\in\mathbb{Z}^+ \given 1 \leq a \leq x \leq b \leq s },
	\qquad
	Y=\Set{ y\in\mathbb{Z}^+ \given 1 \leq c \leq y \leq d \leq n }.
\end{equation*}
我们把映射\(\vb{A}\)在\(X \times Y\)上的限制\begin{equation*}
	\vb{A}\SetRestrict(X \times Y)
\end{equation*}称为“\(\vb{A}\)的\DefineConcept{子块}”.
以子块为元素的矩阵称为\DefineConcept{分块矩阵}.
\end{definition}

我们可以把矩阵\(\vb{A} \in M_{s \times n}(K)\)分解成如下形式:
\begin{equation*}
	\vb{A}
	= \begin{matrix}
		& \begin{matrix} n_1 & n_2 & \dots & n_r \end{matrix} \\
			\begin{matrix} s_1 \\ s_2 \\ \vdots \\ s_t \end{matrix} & \begin{bmatrix}
			\vb{A}_{11} & \vb{A}_{12} & \dots & \vb{A}_{1r} \\
			\vb{A}_{21} & \vb{A}_{22} & \dots & \vb{A}_{2r} \\
			\vdots & \vdots & & \vdots \\
			\vb{A}_{t1} & \vb{A}_{t2} & \dots & \vb{A}_{tr}
		\end{bmatrix},
	\end{matrix}
\end{equation*}
其中\(\vb{A}_{ij} \in M_{s_i \times n_j}(K)\),
\(\vb{A}_{ij}\)
且\(
	\sum_{i=1}^t s_i = s,
	\sum_{j=1}^r n_j = n
\).

\subsection{矩阵的行向量组、列向量组}
设\(\vb{A} \in M_{s \times n}(K)\).
现在我们以\(\vb{A}\)为对象,研究两种特殊的分块方式.

如果我们只把第\(i\ (i=1,2,\dotsc,s)\)行上的元素取出来,
构成一个行向量\begin{equation*}
	\vb\alpha=\begin{bmatrix}
		a_{i1} & a_{i2} & \dots & a_{in}
	\end{bmatrix},
\end{equation*}
或者说,我们取\begin{equation*}
	\vb\alpha=\vb{A}\SetRestrict(\{i\}\times\{1,\dotsc,n\}),
\end{equation*}
那么称“\(\vb\alpha\)是\(\vb{A}\)的第\(i\) \DefineConcept{行向量}”,
记作\(\MatrixEntry{\vb{A}}{i,*}\).

如果我们只把第\(j\ (j=1,2,\dotsc,n)\)列上的元素取出来,
构成一个列向量\begin{equation*}
	\vb\beta=\begin{bmatrix}
		a_{1j} \\ a_{2j} \\ \vdots \\ a_{sj}
	\end{bmatrix},
\end{equation*}
或者说,我们取\begin{equation*}
	\vb\beta=\vb{A}\SetRestrict(\{1,\dotsc,s\}\times\{j\}),
\end{equation*}
那么称“\(\vb\beta\)是\(\vb{A}\)的第\(j\) \DefineConcept{列向量}”,
记作\(\MatrixEntry{\vb{A}}{*,j}\).

把\(\vb{A}\)的全部行向量\begin{equation*}
	\Set{
		\vb\alpha
		\given
		\vb\alpha = \vb{A} \SetRestrict (\{i\}\times\{1,\dotsc,n\}),
		i=1,2,\dotsc,s
	}
\end{equation*}
或者说\begin{equation*}
	\Set{
		\MatrixEntry{\vb{A}}{1,*},
		\MatrixEntry{\vb{A}}{2,*},
		\dotsc,
		\MatrixEntry{\vb{A}}{s,*}
	}
\end{equation*}
称为“\(\vb{A}\)的\DefineConcept{行向量组}”.

把\(\vb{A}\)的全部列向量\begin{equation*}
	\Set{
		\vb\beta
		\given
		\vb\beta = \vb{A} \SetRestrict (\{1,\dotsc,s\}\times\{j\}),
		j=1,2,\dotsc,n
	}
\end{equation*}
或者说\begin{equation*}
	\Set{
		\MatrixEntry{\vb{A}}{*,1},
		\MatrixEntry{\vb{A}}{*,2},
		\dotsc,
		\MatrixEntry{\vb{A}}{*,n}
	}
\end{equation*}
称为“\(\vb{A}\)的\DefineConcept{列向量组}”.

\subsection{矩阵的转置}
\begin{definition}
设\(\vb{A} \in M_{s \times n}(K),
\vb{B}\in M_{n \times s}(K)\).
如果\begin{equation*}
	\MatrixEntry{\vb{B}}{i,j}
	= \MatrixEntry{\vb{A}}{j,i},
	\quad i=1,2,\dotsc,s;j=1,2,\dotsc,n,
\end{equation*}
那么把\(\vb{B}\)称为“\(\vb{A}\)的\DefineConcept{转置矩阵}(transpose)”,
记作\(\vb{A}^T\).
\end{definition}
\begin{remark}
矩阵的转置运算可以看作一个从\(M_{s \times n}(K)\)到\(M_{n \times s}(K)\)上的映射.
\end{remark}

\begin{property}\label{theorem:矩阵的转置.性质1}
设\(\vb{A} \in M_n(K)\),
则\begin{equation}
	(\vb{A}^T)^T = \vb{A}.
\end{equation}
\end{property}

\begin{property}\label{theorem:矩阵的转置.性质2}
设\(\vb{A},\vb{B} \in M_n(K)\),
则\begin{equation}
	(\vb{A}+\vb{B})^T = \vb{A}^T + \vb{B}^T.
\end{equation}
\end{property}

\begin{property}\label{theorem:矩阵的转置.性质3}
设\(\vb{A} \in M_n(K)\),\(k \in K\),
则\begin{equation}
	(k \vb{A})^T = k \vb{A}^T.
\end{equation}
\end{property}

\begin{definition}
设矩阵\(\vb{A} \in M_{s \times n}(\mathbb{C})\).
把对\(\vb{A}\)的各元素取共轭得到的矩阵
称为“矩阵\(\vb{A}\)的\DefineConcept{共轭矩阵}(conjugate)”,
%@see: https://mathworld.wolfram.com/ConjugateMatrix.html
记作\(\ComplexConjugate{\vb{A}}\).
\end{definition}

\begin{definition}
设矩阵\(\vb{A} \in M_{s \times n}(\mathbb{C})\).
将\(\vb{A}\)转置后,再对各元素取共轭,
把这样的矩阵
称为“矩阵\(\vb{A}\)的\DefineConcept{共轭转置矩阵}(conjugate transpose)”,
%@see: https://mathworld.wolfram.com/ConjugateTranspose.html
记作\(\vb{A}^H\),即\begin{equation*}
    \vb{A}^H \defeq \ComplexConjugate{\vb{A}^T} \equiv \ComplexConjugate{\vb{A}}^T.
\end{equation*}
\end{definition}

\begin{property}
设\(\vb{A} \in M_{s \times n}(\mathbb{R})\),
则\(\ComplexConjugate{\vb{A}} = \vb{A}\).
\end{property}

\begin{property}
设\(\vb{A} \in M_{s \times n}(\mathbb{R})\),
则\(\vb{A}^H = \vb{A}^T\).
\end{property}

\section{向量的线性运算}
\begin{definition}
对于\(n\)维向量\(\vb\alpha = (\AutoTuple{a}{n})\)和\(\vb\beta = (\AutoTuple{b}{n})\).
\begin{enumerate}
	\item {\rm\bf 加法}:
	把向量\((a_1+b_1,a_2+b_2,\dotsc,a_n+b_n)\)
	称为“\(\vb\alpha\)与\(\vb\beta\)的\DefineConcept{和}”,记作\begin{equation*}
		\vb\alpha+\vb\beta=(a_1+b_1,a_2+b_2,\dotsc,a_n+b_n)
	\end{equation*}
	\item {\rm\bf 数量乘法}:
	设\(k\)为数,把向量\((k a_1, k a_2, \dotsc, k a_n)\)
	称为“\(k\)与\(\vb\alpha\)的\DefineConcept{数乘}”,记作\begin{equation*}
		k\vb\alpha = (k a_1, k a_2, \dotsc, k a_n)
	\end{equation*}
	\item 分量全为零的向量\((0,0,\dotsc,0)\)称为\DefineConcept{零向量},记作\(\vb0\).
	\item 称\((-a_1,-a_2,\dotsc,-a_n)\)为\(\vb\alpha\)的\DefineConcept{负向量},记作\(-\vb\alpha\).
\end{enumerate}

向量的加法、数乘统称为向量的\DefineConcept{线性运算}.
\end{definition}

\begin{theorem}
由上述定义可知,向量的线性运算满足下面八条运算规律:
\begin{enumerate}
	\item \(\vb\alpha + \vb\beta = \vb\beta + \vb\alpha\)
	\item \((\vb\alpha + \vb\beta) + \vb\gamma = \vb\alpha + (\vb\beta + \vb\gamma)\)
	\item \(\vb\alpha + \vb0 = \vb\alpha\)
	\item \(\vb\alpha + (-\vb\alpha) = \vb0\)
	\item \(1 \vb\alpha = \vb\alpha\)
	\item \(k(l \vb\alpha) = (kl) \vb\alpha\)
	\item \(k(\vb\alpha + \vb\beta) = k\vb\alpha + k\vb\beta\)
	\item \((k+l)\vb\alpha = k\vb\alpha + l\vb\alpha\)
\end{enumerate}
\end{theorem}

\begin{property}
向量的运算还满足以下性质:
\begin{enumerate}
	\item \(0 \vb\alpha = \vb0\)
	\item \((-1) \vb\alpha = -\vb\alpha\)
	\item \(k \vb0 = \vb0\)
	\item \(k \vb\alpha = \vb0 \implies k = 0 \lor \vb\alpha = \vb0\)
\end{enumerate}
\end{property}

\begin{definition}
对于\(n\)维向量\(\vb\alpha = (\AutoTuple{a}{n})\)和\(\vb\beta = (\AutoTuple{b}{n})\).
称向量\(\vb\alpha\)与向量\(\vb\beta\)的负向量\(-\vb\beta\)的和为向量\(\vb\alpha\)与向量\(\vb\beta\)的\DefineConcept{差},即\begin{equation*}
	\vb\alpha - \vb\beta = \vb\alpha + (-\vb\beta).
\end{equation*}
\end{definition}

\section{矩阵的线性运算}
\begin{definition}
设\(\vb{A},\vb{B}\in M_{s\times n}(K)\),
\(\vb{A}=(a_{ij})_{s \times n}\),
\(\vb{B}=(b_{ij})_{s \times n}\).
\begin{enumerate}
	\item 称矩阵\begin{equation*}
		(a_{ij} + b_{ij})_{s \times n} = \begin{bmatrix}
			a_{11}+b_{11} & a_{12}+b_{12} & \dots & a_{1n}+b_{1n} \\
			a_{21}+b_{21} & a_{22}+b_{22} & \dots & a_{2n}+b_{2n} \\
			\vdots & \vdots & & \vdots \\
			a_{s1}+b_{s1} & a_{s2}+b_{s2} & \dots & a_{sn}+b_{sn}
		\end{bmatrix}
	\end{equation*}为“\(\vb{A}\)与\(\vb{B}\)的\DefineConcept{和}(sum)”,
	记作\(\vb{A}+\vb{B}\).

	\item 任取\(k\in K\),称矩阵\begin{equation*}
		(ka_{ij})_{s \times n} = \begin{bmatrix}
			ka_{11} & ka_{12} & \dots & ka_{1n} \\
			ka_{21} & ka_{22} & \dots & ka_{2n} \\
			\vdots & \vdots & & \vdots \\
			ka_{s1} & ka_{s2} & \dots & ka_{sn}
		\end{bmatrix}
	\end{equation*}为“\(k\)与矩阵\(\vb{A}\)的\DefineConcept{数乘}”,
	记作\(k\vb{A}\).

	\item 称元素全为零的矩阵为\DefineConcept{零矩阵}(zero matrix),记作\(\vb0\).

	\item 称矩阵\begin{equation*}
		(-a_{ij})_{s \times n}=\begin{bmatrix}
			-a_{11} & -a_{12} & \dots & -a_{1n} \\
			-a_{21} & -a_{22} & \dots & -a_{2n} \\
			\vdots & \vdots & & \vdots \\
			-a_{s1} & -a_{s2} & \dots & -a_{sn}
		\end{bmatrix}
	\end{equation*}为\(\vb{A}\)的\DefineConcept{负矩阵},记作\(-\vb{A}\).
\end{enumerate}
\end{definition}

定义\DefineConcept{非零矩阵}:\begin{equation*}
	M_{s \times n}^*(K) \defeq M_{s \times n}(K)-\{\vb0\},
	\qquad
	M_n^*(K) \defeq M_n(K)-\{\vb0\}.
\end{equation*}

\begin{theorem}
矩阵的线性运算满足以下运算规律:
\begin{gather}
	(\forall\vb{A},\vb{B}\in M_{s\times n}(K))[\vb{A} + \vb{B} = \vb{B} + \vb{A}], \\
	(\forall\vb{A},\vb{B},\vb{C}\in M_{s\times n}(K))[(\vb{A} + \vb{B}) + \vb{C} = \vb{A} + (\vb{B} + \vb{C})], \\
	(\forall\vb{A} \in M_{s\times n}(K))[\vb{A} + \vb0 = \vb{A}], \\
	(\forall\vb{A} \in M_{s\times n}(K))[\vb{A} + (-\vb{A}) = \vb0], \\
	(\forall\vb{A} \in M_{s\times n}(K))[1 \vb{A} = \vb{A}], \\
	(\forall\vb{A} \in M_{s\times n}(K))(\forall k,l\in K)[k(l \vb{A}) = (kl) \vb{A}], \\
	(\forall\vb{A},\vb{B}\in M_{s\times n}(K))(\forall k\in K)[k(\vb{A} + \vb{B}) = k\vb{A} + k\vb{B}], \\
	(\forall\vb{A} \in M_{s\times n}(K))(\forall k,l\in K)[(k+l)\vb{A} = k\vb{A} + l\vb{A}], \\
	(\forall\vb{A} \in M_{s\times n}(K))[0\vb{A} = \vb0], \\
	(\forall\vb{A} \in M_{s\times n}(K))[(-1)\vb{A} = -\vb{A}], \\
	(\forall k\in K)[k\vb0 = \vb0], \\
	k\vb{A} = \vb0 \implies k = 0 \lor \vb{A} = \vb0.
\end{gather}
\end{theorem}
