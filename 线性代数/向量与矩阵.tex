\chapter{向量、矩阵及其基本运算}
\section{矩阵与向量的概念}
\subsection{矩阵与向量的基本概念}
%@see: https://math.stackexchange.com/a/1162585/591741
\begin{definition}
设\(s,n\)都是正整数,\(K\)是数域.
我们把从\(\{1,\dotsc,s\}\times\{1,\dotsc,n\}\)到\(K\)的映射,
称为“数域\(K\)上的\(s \times n\)~\DefineConcept{矩阵}(matrix)”.
%@see: https://mathworld.wolfram.com/Matrix.html
\end{definition}

为了简化描述,我们定义:\[
	M_{s \times n}(K)
	\defeq
	\Set{
		\A \given
		\text{\(\A\)是数域\(K\)上的\(s \times n\)矩阵}
	}.
\]

设\(\A\in M_{s \times n}(K)\).
我们把\(s\)称为“\(\A\)的\DefineConcept{行数}”,
把\(n\)称为“\(\A\)的\DefineConcept{列数}”.
把\[
	\A(i,j)
	\quad(i=1,2,\dotsc,s;j=1,2,\dotsc,n)
\]称为“\(\A\)的\((i,j)\)元素”
或“\(\A\)的第\(i\)行第\(j\)列\DefineConcept{元素}(element)”.
%@see: https://mathworld.wolfram.com/MatrixElement.html

对于任意给定的两个矩阵\(\A\)和\(\B\),
如果它们行数、列数都相同,
则称“\(\A\)与\(\B\) \DefineConcept{同型}”.

如果矩阵\(\A\)的行数\(s\)恰好等于它的列数\(n\),
我们就把它称为“\(n\)阶矩阵”或“\(n\)阶\DefineConcept{方阵}(square matrix)”.
%@see: https://mathworld.wolfram.com/SquareMatrix.html
定义:\[
	M_n(K)
	\defeq
	M_{n \times n}(K).
\]

设\(\A\in M_{s \times n}(K)\).
我们把\[
	\A(k,k)
	\quad(k=1,2,\dotsc,\min\{s,n\})
\]称为“\(\A\)的\DefineConcept{主对角线}(main diagonal)上的元素”.
%@see: https://mathworld.wolfram.com/Diagonal.html
%@see: https://mathworld.wolfram.com/SkewDiagonal.html

我们把行数为\(1\)、列数为\(n\)的矩阵称为“\(n\)维\DefineConcept{行向量}(row vector)”;
%@see: https://mathworld.wolfram.com/RowVector.html
把列数为\(1\)、行数为\(n\)的矩阵称为“\(n\)维\DefineConcept{列向量}(column vector)”.
%@see: https://mathworld.wolfram.com/ColumnVector.html
\(n\)维行向量和\(n\)维列向量统称为“\(n\)维\DefineConcept{向量}(vector)”.
通常用一个黑体小写希腊字母(如\(\a,\b,\g\)等)表示向量.

设\(\a\in M_{s \times 1}(K)\).
我们把\(s\)称为“\(\a\)的\DefineConcept{维数}”.
定义:\[
	\a(i) \defeq \a(i,1).
\]
把\(\a(i)\ (i=1,2,\dotsc,s)\)称为“\(\a\)的第\(i\)个\DefineConcept{分量}”.

设\(\b\in M_{1 \times n}(K)\).
把\(n\)称为“\(\b\)的\DefineConcept{维数}”.
定义:\[
	\b(i) \defeq \b(1,i).
\]
把\(\b(i)\ (i=1,2,\dotsc,n)\)称为“\(\b\)的第\(i\)个\DefineConcept{分量}”.

有时候为了强调矩阵的行数和列数,
我们会在表示矩阵的字母的右下角标注矩阵的形状,例如,
如果\(\A\in M_{s \times n}(K)\),那么我们可以写出\(\A_{s \times n}\);
如果\(\B\in M_n(K)\),那么我们可以写出\(\B_n\).

“矩阵”之所以得名,是因为我们可以把矩阵\(\A\)写成\(s\)行\(n\)列的矩形表,再加上括号:\[
	\begin{bmatrix}
		a_{11} & a_{12} & \dots & a_{1n} \\
		a_{21} & a_{22} & \dots & a_{2n} \\
		\vdots & \vdots & & \vdots \\
		a_{s1} & a_{s2} & \dots & a_{sn}
	\end{bmatrix},
\]
其中\(a_{ij}=\A(i,j)\ (i=1,2,\dotsc,s;j=1,2,\dotsc,n)\).
有时候为了强调矩阵的元素,
我们会把\(\A_{s \times n}\)记作\(\A=(a_{ij})_{s \times n}\).

类似地,我们把行向量表示为\[
	\begin{bmatrix}
		a_1 & a_2 & \dots & a_n
	\end{bmatrix};
\]
把列向量表示为\[
	\begin{bmatrix}
		a_1 \\ a_2 \\ \vdots \\ a_n
	\end{bmatrix}.
\]

\subsection{子矩阵}
\begin{definition}
在矩阵\(\A=(a_{ij})_{s \times n}\)中,
任取\(k\)行\(l\)列,
位于这些行与列交叉处的\(kl\)个元素,按原顺序排成的\(k \times l\)矩阵\[
	\begin{vmatrix}
		a_{i_1,j_1} & a_{i_1,j_2} & \dots & a_{i_1,j_l} \\
		a_{i_2,j_1} & a_{i_2,j_2} & \dots & a_{i_2,j_l} \\
		\vdots & \vdots & & \vdots \\
		a_{i_k,j_1} & a_{i_k,j_2} & \dots & a_{i_k,j_l}
	\end{vmatrix},
	\quad
	\begin{array}{c}
		1 \leq i_1 < i_2 < \dotsb < i_k \leq s; \\
		1 \leq j_1 < j_2 < \dotsb < j_l \leq n
	\end{array}
\]称为“\(\A\)的一个\(k\)阶\DefineConcept{子矩阵}(submatrix)”.
%@see: https://mathworld.wolfram.com/Submatrix.html
\end{definition}

\subsection{矩阵的分块}
\begin{definition}
设\(\A\in M_{s \times n}(K)\);
给定介于\(1\)和\(s\)之间的两个正整数\(a,b\),
以及介于\(1\)和\(n\)之间的两个正整数\(c,d\).
令\[
	X=\Set{ x\in\mathbb{Z}^+ \given 1 \leq a \leq x \leq b \leq s },
	\qquad
	Y=\Set{ y\in\mathbb{Z}^+ \given 1 \leq c \leq y \leq d \leq n }.
\]
我们把映射\(\A\)在\(X \times Y\)上的限制\[
	\A\setrestrict(X \times Y)
\]称为“\(\A\)的\DefineConcept{子块}”.
以子块为元素的矩阵称为\DefineConcept{分块矩阵}.
\end{definition}

我们可以把矩阵\(\A\)分解成如下形式:
\[
	\begin{matrix}
		& \begin{matrix} n_1 & n_2 & \dots & n_r \end{matrix} \\
			\begin{matrix} s_1 \\ s_2 \\ \vdots \\ s_t \end{matrix} & \begin{bmatrix}
			\A_{11} & \A_{12} & \dots & \A_{1r} \\
			\A_{21} & \A_{22} & \dots & \A_{2r} \\
			\vdots & \vdots & \ddots & \vdots \\
			\A_{t1} & \A_{t2} & \dots & \A_{tr}
		\end{bmatrix}.
	\end{matrix}
\]

\subsection{矩阵的行向量组、列向量组}
设\(\A\in M_{s \times n}(K)\).
现在我们以\(\A\)为对象,研究两种特殊的分块方式.

如果我们只把第\(i\ (i=1,2,\dotsc,s)\)行上的元素取出来,
构成一个行向量\[
	\a=\begin{bmatrix}
		a_{i1} & a_{i2} & \dots & a_{in}
	\end{bmatrix},
\]
或者说,我们取\[
	\a=\A\setrestrict(\{i\}\times\{1,\dotsc,n\}),
\]
那么称“\(\a\)是\(\A\)的第\(i\) \DefineConcept{行向量}”,
记作\(\MatrixEntry\A{i,*}\).

如果我们只把第\(j\ (j=1,2,\dotsc,n)\)列上的元素取出来,
构成一个列向量\[
	\b=\begin{bmatrix}
		a_{1j} \\ a_{2j} \\ \vdots \\ a_{sj}
	\end{bmatrix},
\]
或者说,我们取\[
	\b=\A\setrestrict(\{1,\dotsc,s\}\times\{j\}),
\]
那么称“\(\b\)是\(\A\)的第\(j\) \DefineConcept{列向量}”,
记作\(\MatrixEntry\A{*,j}\).

把\(\A\)的全部行向量\[
	\Set{ \a \given \a=\A\setrestrict(\{i\}\times\{1,\dotsc,n\}), i=1,2,\dotsc,s }
\]或者说\[
	\Set{\MatrixEntry\A{1,*},\MatrixEntry\A{2,*},\dotsc,\MatrixEntry\A{s,*}}
\]
称为“\(\A\)的\DefineConcept{行向量组}”.

把\(\A\)的全部列向量\[
	\Set{ \b \given \b=\A\setrestrict(\{1,\dotsc,s\}\times\{j\}), j=1,2,\dotsc,n }
\]或者说\[
	\Set{\MatrixEntry\A{*,1},\MatrixEntry\A{*,2},\dotsc,\MatrixEntry\A{*,n}}
\]
称为“\(\A\)的\DefineConcept{列向量组}”.

\subsection{矩阵的转置}
\begin{definition}
设\(\A\in M_{s \times n}(K),
\B\in M_{n \times s}(K)\).
如果\[
	\B(i,j)=\A(j,i),
	\quad i=1,2,\dotsc,s;j=1,2,\dotsc,n,
\]
那么把\(\B\)称为“\(\A\)的\DefineConcept{转置矩阵}(transpose)”,
记作\(\A^T\).
\end{definition}
\begin{remark}
矩阵的转置运算可以看作一个从\(M_{s \times n}(K)\)到\(M_{n \times s}(K)\)上的映射.
\end{remark}

\begin{property}\label{theorem:矩阵的转置.性质1}
设\(\A \in M_n(K)\),
则\begin{equation}
	(\A^T)^T = \A.
\end{equation}
\end{property}

\begin{property}\label{theorem:矩阵的转置.性质2}
设\(\A,\B \in M_n(K)\),
则\begin{equation}
	(\A+\B)^T = \A^T + \B^T.
\end{equation}
\end{property}

\begin{property}\label{theorem:矩阵的转置.性质3}
设\(\A \in M_n(K)\),\(k \in K\),
则\begin{equation}
	(k \A)^T = k \A^T.
\end{equation}
\end{property}

\begin{definition}
设矩阵\(\A \in M_{s \times n}(\mathbb{C})\).
把对\(\A\)的各元素取共轭得到的矩阵
称为“矩阵\(\A\)的\DefineConcept{共轭矩阵}(conjugate)”,
%@see: https://mathworld.wolfram.com/ConjugateMatrix.html
记作\(\overline{\A}\).
\end{definition}

\begin{definition}
设矩阵\(\A \in M_{s \times n}(\mathbb{C})\).
将\(\A\)转置后,再对各元素取共轭,
把这样的矩阵
称为“矩阵\(\A\)的\DefineConcept{共轭转置矩阵}(conjugate transpose)”,
%@see: https://mathworld.wolfram.com/ConjugateTranspose.html
记作\(\A^H\),即\[
    \A^H \defeq \overline{\A^T} \equiv \overline{\A}^T.
\]
\end{definition}

\begin{property}
设\(\A \in M_{s \times n}(\mathbb{R})\),
则\(\overline{\A} = \A\).
\end{property}

\begin{property}
设\(\A \in M_{s \times n}(\mathbb{R})\),
则\(\A^H = \A^T\).
\end{property}

\section{向量的线性运算}
\begin{definition}
对于\(n\)维向量\(\a = (\AutoTuple{a}{n})\)和\(\b = (\AutoTuple{b}{n})\).
\begin{enumerate}
	\item {\rm\bf 加法}:
	把向量\((a_1+b_1,a_2+b_2,\dotsc,a_n+b_n)\)
	称为“\(\a\)与\(\b\)的\DefineConcept{和}”,记作\[
		\a+\b=(a_1+b_1,a_2+b_2,\dotsc,a_n+b_n)
	\]
	\item {\rm\bf 数量乘法}:
	设\(k\)为数,把向量\((k a_1, k a_2, \dotsc, k a_n)\)
	称为“\(k\)与\(\a\)的\DefineConcept{数乘}”,记作\[
		k\a = (k a_1, k a_2, \dotsc, k a_n)
	\]
	\item 分量全为零的向量\((0,0,\dotsc,0)\)称为\DefineConcept{零向量},记作\(\z\).
	\item 称\((-a_1,-a_2,\dotsc,-a_n)\)为\(\a\)的\DefineConcept{负向量},记作\(-\a\).
\end{enumerate}

向量的加法、数乘统称为向量的\DefineConcept{线性运算}.
\end{definition}

\begin{theorem}
由上述定义可知,向量的线性运算满足下面八条运算规律:
\begin{enumerate}
	\item \(\a + \b = \b + \a\)
	\item \((\a + \b) + \g = \a + (\b + \g)\)
	\item \(\a + \z = \a\)
	\item \(\a + (-\a) = \z\)
	\item \(1 \a = \a\)
	\item \(k(l \a) = (kl) \a\)
	\item \(k(\a + \b) = k\a + k\b\)
	\item \((k+l)\a = k\a + l\a\)
\end{enumerate}
\end{theorem}

\begin{property}
向量的运算还满足以下性质:
\begin{enumerate}
	\item \(0 \a = \z\)
	\item \((-1) \a = -\a\)
	\item \(k \z = \z\)
	\item \(k \a = \z \implies k = 0 \lor \a = \z\)
\end{enumerate}
\end{property}

\begin{definition}
对于\(n\)维向量\(\a = (\AutoTuple{a}{n})\)和\(\b = (\AutoTuple{b}{n})\).
称向量\(\a\)与向量\(\b\)的负向量\(-\b\)的和为向量\(\a\)与向量\(\b\)的\DefineConcept{差},即\[
	\a - \b = \a + (-\b).
\]
\end{definition}

\section{矩阵的线性运算}
\begin{definition}
设\(\A,\B\in M_{s\times n}(K)\),
\(\A=(a_{ij})_{s \times n}\),
\(\B=(b_{ij})_{s \times n}\).
\begin{enumerate}
	\item 称矩阵\[
		(a_{ij} + b_{ij})_{s \times n} = \begin{bmatrix}
			a_{11}+b_{11} & a_{12}+b_{12} & \dots & a_{1n}+b_{1n} \\
			a_{21}+b_{21} & a_{22}+b_{22} & \dots & a_{2n}+b_{2n} \\
			\vdots & \vdots & & \vdots \\
			a_{s1}+b_{s1} & a_{s2}+b_{s2} & \dots & a_{sn}+b_{sn}
		\end{bmatrix}
	\]为“\(\A\)与\(\B\)的\DefineConcept{和}(sum)”,
	记作\(\A+\B\).

	\item 任取\(k\in K\),称矩阵\[
		(ka_{ij})_{s \times n} = \begin{bmatrix}
			ka_{11} & ka_{12} & \dots & ka_{1n} \\
			ka_{21} & ka_{22} & \dots & ka_{2n} \\
			\vdots & \vdots & & \vdots \\
			ka_{s1} & ka_{s2} & \dots & ka_{sn}
		\end{bmatrix}
	\]为“\(k\)与矩阵\(\A\)的\DefineConcept{数乘}”,
	记作\(k\A\).

	\item 称元素全为零的矩阵为\DefineConcept{零矩阵}(zero matrix),记作\(\z\).

	\item 称矩阵\[
		(-a_{ij})_{s \times n}=\begin{bmatrix}
			-a_{11} & -a_{12} & \dots & -a_{1n} \\
			-a_{21} & -a_{22} & \dots & -a_{2n} \\
			\vdots & \vdots & & \vdots \\
			-a_{s1} & -a_{s2} & \dots & -a_{sn}
		\end{bmatrix}
	\]为\(\A\)的\DefineConcept{负矩阵},记作\(-\A\).
\end{enumerate}
\end{definition}

定义\DefineConcept{非零矩阵}:\[
	M_{s \times n}^*(K) \defeq M_{s \times n}(K)-\{\z\},
	\qquad
	M_n^*(K) \defeq M_n(K)-\{\z\}.
\]

\begin{theorem}
矩阵的线性运算满足以下运算规律:
\begin{gather}
	(\forall\A,\B\in M_{s\times n}(K))[\A + \B = \B + \A], \\
	(\forall\A,\B,\C\in M_{s\times n}(K))[(\A + \B) + \C = \A + (\B + \C)], \\
	(\forall\A\in M_{s\times n}(K))[\A + \z = \A], \\
	(\forall\A\in M_{s\times n}(K))[\A + (-\A) = \z], \\
	(\forall\A\in M_{s\times n}(K))[1 \A = \A], \\
	(\forall\A\in M_{s\times n}(K))(\forall k,l\in K)[k(l \A) = (kl) \A], \\
	(\forall\A,\B\in M_{s\times n}(K))(\forall k\in K)[k(\A + \B) = k\A + k\B], \\
	(\forall\A\in M_{s\times n}(K))(\forall k,l\in K)[(k+l)\A = k\A + l\A], \\
	(\forall\A\in M_{s\times n}(K))[0\A = \z], \\
	(\forall\A\in M_{s\times n}(K))[(-1)\A = -\A], \\
	(\forall k\in K)[k\z = \z], \\
	k\A = \z \implies k = 0 \lor \A = \z.
\end{gather}
\end{theorem}
