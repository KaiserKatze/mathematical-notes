\section{实二次型的分类}
\subsection{实二次型的分类标准}
\begin{definition}\label{definition:实二次型的分类.实二次型的分类}
%@see: 《线性代数》(张慎语、周厚隆) P129 定义5
%@see: 《线性代数》(张慎语、周厚隆) P131 定义7
给定\(n\)元实二次型\(f(\vb{x}) = \vb{x}^T\vb{A}\vb{x}\).
\begin{enumerate}
	\item 如果\begin{equation*}
		(\forall\vb{x}\in\mathbb{R}^n-\{\vb0\})
		[f(\vb{x}) > 0],
	\end{equation*}
	则称“\(f\)是\DefineConcept{正定的}(positive definite)”,
	“\(\vb{A}\)是一个\DefineConcept{正定矩阵}(positive definite matrix)”,
	记为\(\vb{A}\succ\vb0\).

	\item 如果\begin{equation*}
		(\forall\vb{x}\in\mathbb{R}^n-\{\vb0\})
		[f(\vb{x}) \geq 0],
	\end{equation*}
	则称“\(f\)是\DefineConcept{半正定的}(positive semi-definite)”,
	“\(\vb{A}\)是一个\DefineConcept{半正定矩阵}(positive semi-definite matrix)”,
	记为\(\vb{A}\succeq\vb0\).

	\item 如果\begin{equation*}
		(\forall\vb{x}\in\mathbb{R}^n-\{\vb0\})
		[f(\vb{x}) < 0],
	\end{equation*}
	则称“\(f\)是\DefineConcept{负定的}(negative definite)”,
	“\(\vb{A}\)是一个\DefineConcept{负定矩阵}(negative definite matrix)”,
	记为\(\vb{A}\prec\vb0\).

	\item 如果\begin{equation*}
		(\forall\vb{x}\in\mathbb{R}^n-\{\vb0\})
		[f(\vb{x}) \leq 0],
	\end{equation*}
	则称“\(f\)是\DefineConcept{半负定的}(negative semi-definite)”,
	“\(\vb{A}\)是一个\DefineConcept{半负定矩阵}(negative semi-definite matrix)”,
	记为\(\vb{A}\preceq\vb0\).

	\item 否则,称\(f\)是\DefineConcept{不定的}(indefinite).
\end{enumerate}
\end{definition}

\begin{example}
%@see: 《线性代数》(张慎语、周厚隆) P129 例1
\(f(x_1,x_2,x_3) \defeq 3 x_1^2 + x_2^2 + 5 x_3^2\)是正定的.
\end{example}

\begin{example}
%@see: 《线性代数》(张慎语、周厚隆) P129 例1
\(f(x_1,x_2,x_3) \defeq -x_1^2 - x_3^2\)是半负定的.
\end{example}

\begin{example}
%@see: 《线性代数》(张慎语、周厚隆) P129 例1
\(f(x_1,x_2,x_3) \defeq x_1 x_2 + x_2^2 + 5 x_3^2\)是不定的.
\end{example}

\subsection{惯性定理}
\begin{theorem}\label{theorem:二次型.惯性定理}
%@see: 《线性代数》(张慎语、周厚隆) P129 惯性定理(inertial theorem)
\(n\)元实二次型\(f(\vb{x}) = \vb{x}^T\vb{A}\vb{x}\)经过任意满秩线性变换化为标准型,
所得的标准型的正平方项的项数\(p\)及负平方项的项数\(q\)都是唯一确定的.
\begin{proof}
设实二次型的秩为\(r\).
假设\(f\)经过两个不同的可逆线性替换\(\vb{x}=\vb{C}\vb{y},\vb{x}=\vb{D}\vb{z}\)分别化为标准型\begin{equation*}
	f \xlongequal{\vb{x}=\vb{C}\vb{y}}
	c_1 y_1^2 + c_2 y_2^2 + \dotsb + c_p y_p^2 - c_{p+1} y_{p+1}^2 - \dotsb - c_r y_r^2,
	\eqno(1)
\end{equation*}\begin{equation*}
	f \xlongequal{\vb{x}=\vb{D}\vb{z}}
	d_1 z_1^2 + d_2 z_2^2 + \dotsb + d_q z_q^2 - d_{q+1} z_{q+1}^2 - \dotsb - d_r z_r^2,
	\eqno(2)
\end{equation*}
其中\(c_i,d_i>0\ (i=1,2,\dotsc,r)\).

用反证法.
设\(p > q\),由\(\vb{x} = \vb{C}\vb{y} = \vb{D}\vb{z}\),\(\vb{D}\)可逆,得\(\vb{z} = \vb{D}^{-1} \vb{C} \vb{y}\).
\def\zexpr#1{h_{#1 1} y_1 + h_{#1 2} y_2 + \dotsb + h_{#1 n} y_n}%
记\(\vb{H} = (h_{ij})_n = \vb{B}^{-1} \vb{C}\),
则\(\vb{z} = \vb{H}\vb{y}\),
即\begin{equation*}
	z_i = \zexpr{i}
	\quad(i=1,2,\dotsc,n).
\end{equation*}
于是\begin{equation*}
	\begin{aligned}
		&\hspace{-40pt}
		c_1 y_1^2 + c_2 y_2^2 + \dotsb
			+ c_p y_p^2 - c_{p+1} y_{p+1}^2 - \dotsb - c_r y_r^2 \\
		&\hspace{-20pt}= d_1 (\zexpr{1})^2 + d_2 (\zexpr{2})^2 \\
		&+ \dotsb + d_q (\zexpr{q})^2 \\
		&- d_{q+1} (\zexpr{q+1})^2 - \dotsb \\
		&- d_r (\zexpr{r})^2.
	\end{aligned}
	\eqno(3)
\end{equation*}
由此可以构造齐次线性方程组\begin{equation*}
	\begin{cases}
		\zexpr{1} = 0, \\
		\hdotsfor{1} \\
		\zexpr{q} = 0, \\
		y_{p+1} = 0, \\
		\hdotsfor{1} \\
		y_n = 0.
	\end{cases}
	\eqno(4)
\end{equation*}
这个方程组中有\(n\)个未知量,\(q+n-p < n\)个方程,
于是它有非零解\((\AutoTuple{y}{p},0,\dotsc,0)^T\),
代入(3)式两端,得到左边大于零,右边小于等于零,矛盾,因此\(p \leq q\).
同理又有\(q \leq p\),于是\(p = q\).
\end{proof}
%@see: https://mathworld.wolfram.com/SylvestersInertiaLaw.html
\end{theorem}
我们把\cref{theorem:二次型.惯性定理} 称为“惯性定理(Inertial Theorem)”.

\begin{corollary}
%@see: 《线性代数》(张慎语、周厚隆) P130 推论
任意\(n\)元实二次型\(f(\vb{x}) = \vb{x}^T\vb{A}\vb{x}\),
总可经过满秩线性变换,化为以下形式:\begin{equation*}
	y_1^2+y_2^2+ \dotsb +y_p^2
	-y_{p+1}^2-\dotsb-y_r^2.
\end{equation*}
我们将其称为“\(f(\vb{x})\)的\DefineConcept{规范型}({\rm normal form})”,
且规范型是唯一的.
\begin{proof}
根据惯性定理,\(f\)经过可逆线性替换化为标准型:\begin{equation*}
	f \xlongequal{\vb{x}=\vb{D}\vb{z}}
	d_1 z_1^2 + d_2 z_2^2 + \dotsb + d_q z_q^2 - d_{q+1} z_{q+1}^2 - \dotsb - d_r z_r^2,
\end{equation*}
其中\(d_i>0\ (i=1,2,\dotsc,r)\).
令\begin{equation*}
	\vb{C} = \vb{D} \diag(d_1^{-1/2},\dotsc,d_r^{-1/2},1,\dotsc,1),
\end{equation*}
则\(\vb{x} = \vb{D}\vb{z} = \vb{C}\vb{y}\)是可逆线性替换,
使得\begin{equation*}
	f \xlongequal{\vb{x}=\vb{D}\vb{z}} y_1^2 + y_2^2 + \dotsb + y_q^2 - y_{q+1}^2 - \dotsb - y_r^2.
\end{equation*}
二次型的规范型的唯一性可以由惯性定理得到.
\end{proof}
\end{corollary}

\begin{definition}\label{definition:二次型.惯性系数的定义}
%@see: 《线性代数》(张慎语、周厚隆) P131 定义6
在秩为\(r\)的实二次型\(f(\vb{x})\)所化成的标准型(或规范型)中,
\begin{itemize}
	\item 正平方项的项数\(p\)
	称为“\(f\)的\DefineConcept{正惯性指数}(positive index of inertia)”;
	\item 负平方项的项数\(q=r-p\)
	称为“\(f\)的\DefineConcept{负惯性指数}(minus index of inertia)”;
	\item 正、负惯性指数之差\(d=p-q\)
	称为“\(f\)的\DefineConcept{符号差}(signature)”.
\end{itemize}
% 下面是一个数值算法,它的原理是:先计算矩阵的特征值,再分别统计正特征值和负特征值的个数.
%@Mathematica: InertiaIndices[A_?SymmetricMatrixQ] :=
%				Module[{eigvals},
%				eigvals = Eigenvalues[A];        (* 计算所有特征值 *)
%				{Count[eigvals, _?Positive],     (* 正惯性指数 *)
%				Count[eigvals, _?Negative]       (* 负惯性指数 *)}];
\end{definition}
由\hyperref[theorem:二次型.惯性定理]{惯性定理}%
和\cref{definition:二次型.惯性系数的定义} 可知:
可逆线性替换不改变二次型的正、负惯性指数.
因此我们可以根据二次型的正、负惯性指数确定二次型的类型.

\begin{corollary}
设\(\vb{A}\)是\(n\)阶实对称矩阵,
\(\vb{R}\)是\(\vb{A}\)的行最简形,
\(\vb{A}\)的正惯性指数、负惯性指数分别是\(p_{\vb{A}}\)、\(q_{\vb{A}}\),
\(\vb{R}\)的正主元个数、负主元个数分别是\(p_{\vb{R}}\)、\(q_{\vb{R}}\),
则\(
	p_{\vb{A}} = p_{\vb{R}},
	q_{\vb{A}} = q_{\vb{R}}
\).
\end{corollary}

\begin{theorem}
设\(\vb{A}\)和\(\vb{B}\)是同阶实对称矩阵.
这两个矩阵合同的充分必要条件是两者的秩、正负惯性指数均相等,
即\begin{equation*}
	\vb{A}\simeq\vb{B}
	\iff
	\rank\vb{A}=\rank\vb{B},
	p_{\vb{A}}=p_{\vb{B}},
	q_{\vb{A}}=q_{\vb{B}}.
\end{equation*}
\end{theorem}

\subsection{正定矩阵的等价条件}
\begin{theorem}
%@see: 《线性代数》(张慎语、周厚隆) P131 定理5
设\(\vb{A}\)为\(n\)阶实对称矩阵,
\(f(\AutoTuple{x}{n}) = \vb{x}^T\vb{A}\vb{x}\),
则下列命题相互等价:\begin{itemize}
	\item \(\vb{A}\)为正定矩阵;
	\item \(\vb{A}\)的特征值全是正实数;
	\item \(f(\AutoTuple{x}{n})\)的正惯性指数\(p=n\);
	\item \(\vb{A} \cong \vb{E}\);
	\item 存在可逆实阵\(\vb{P}\),使得\(\vb{A}=\vb{P}^T\vb{P}\).
\end{itemize}
%TODO proof
\end{theorem}

\begin{corollary}
%@see: 《线性代数》(张慎语、周厚隆) P132 推论1
正定矩阵的行列式大于零.
\begin{proof}
因为\(\vb{A}\)正定,
所以存在可逆实阵\(\vb{P}\),使得\(\vb{A}=\vb{P}^T\vb{P}\),
则\begin{equation*}
	\abs{\vb{A}}
	=\abs{\vb{P}^T\vb{P}}
	=\abs{\vb{P}^T} \abs{\vb{P}\vphantom{\vb{P}^T}}
	=\abs{\vb{P}}^2>0.
	\qedhere
\end{equation*}
\end{proof}
\end{corollary}

\begin{theorem}
%@see: 《线性代数》(张慎语、周厚隆) P132 定理6
\(n\)元实二次型\(f(\AutoTuple{x}{n}) = \vb{x}^T\vb{A}\vb{x}\ (\vb{A}=\vb{A}^T)\)正定的充分必要条件是:
矩阵\(\vb{A}\)的各阶顺序主子式\(\det\vb{A}_k\)均大于零,
即\begin{equation*}
	\det\vb{A}_k > 0
	\quad(k=1,2,\dotsc,n).
\end{equation*}
\begin{proof}
必要性.
对于任意不全为零的\(n\)个实数\(c_1,c_2,\dotsc,c_k,0,\dotsc,0\),
总有\begin{equation*}
	f(c_1,c_2,\dotsc,c_k,0,\dotsc,0)
	= \sum_{i=1}^k \sum_{j=1}^k c_i c_j a_{ij} > 0,
\end{equation*}
从而\(k\)元实二次型\(f_k(x_1,x_2,\dotsc,x_k)
=\sum_{i=1}^k
\sum_{j=1}^k
x_i x_j a_{ij}\)正定,
而\(f_k\)的矩阵为\(\vb{A}_k = (a_{ij})_k\),
那么\(\abs{\vb{A}_k} > 0\ (k=1,2,\dotsc,n)\).

充分性.当\(n=1\)时,\(a_{11} > 0\),\(f_1(x_1) = a_{11} x_1^2\)正定.
设\(n=k-1\)时结论成立,
当\(n=k\)时,
将\(\vb{A}\)分块得\(\vb{A} = \begin{bmatrix}
	\vb{A}_{k-1} & \vb\alpha \\
	\vb\alpha^T & a_{nn}
\end{bmatrix}\),
其中\(\vb{A}_{k-1}\)为各阶顺序主子式都大于零的\(k-1\)阶实对称矩阵.
由归纳假设,\(\vb{A}_{k-1}\)正定,故存在\(k-1\)阶可逆矩阵\(\vb{Q}\),
使得\(\vb{A}_{k-1} = \vb{Q}^T \vb{Q}\),\(\vb{A}_{k-1}\)可逆,
\(\vb{A}_{k-1}^{-1} = \vb{Q}^{-1}(\vb{Q}^{-1})^T\)是对称矩阵.
令\(\vb{P} = \begin{bmatrix}
	\vb{Q}^{-1} & -\vb{A}_{k-1}^{-1} \vb\alpha \\
	\vb0 & 1
\end{bmatrix}\),
则\(\vb{P}\)可逆,
于是\begin{align*}
	\vb{P}^T \vb{A} \vb{P} &= \begin{bmatrix}
		(\vb{Q}^{-1})^T & \vb0 \\
		-\vb\alpha^T \vb{A}_{k-1}^{-1} & 1
	\end{bmatrix}
	\begin{bmatrix}
		\vb{Q}^T \vb{Q} & \vb\alpha \\
		\vb\alpha^T & a_{nn}
	\end{bmatrix}
	\begin{bmatrix}
		\vb{Q}^{-1} & -\vb{A}_{k-1}^{-1} \vb\alpha \\
		\vb0 & 1
	\end{bmatrix} \\
	&= \begin{bmatrix}
		\vb{Q} & (\vb{Q}^{-1})^T \vb\alpha \\
		\vb0 & b
	\end{bmatrix}
	\begin{bmatrix}
		\vb{Q}^{-1} & -\vb{A}_{k-1}^{-1} \vb\alpha \\
		\vb0 & 1
	\end{bmatrix}
	= \begin{bmatrix}
		\vb{E}_{k-1} & \vb0 \\
		\vb0 & b
	\end{bmatrix} = \vb{B},
\end{align*}
其中\(b=a_{nn}-\vb\alpha^T \vb{A}_{k-1}^{-1} \vb\alpha\).
由于\(\vb{A}\)与\(\vb{B}\)合同,\(\abs{\vb{A}} > 0\),
得\(\abs{\vb{B}} = b > 0\),
作可逆线性替换\(\vb{x} = \vb{P}\vb{y}\),则\begin{equation*}
	f \xlongequal{\vb{x}=\vb{Q}\vb{y}} y_1^2 + y_2^2 + \dotsb + y_{n-1}^2 + b y_n^2,
\end{equation*}
故\(f\)的正惯性指数为\(n\),\(f\)正定.
\end{proof}
\end{theorem}

\begin{corollary}
%@see: 《线性代数》(张慎语、周厚隆) P133 推论2
\(n\)元实二次型\(f(\AutoTuple{x}{n}) = \vb{x}^T\vb{A}\vb{x}\ (\vb{A}=\vb{A}^T)\)负定的充分必要条件是:
矩阵\(\vb{A}\)的奇数阶顺序主子式为负,偶数阶顺序主子式为正,
即\begin{equation*}
	(-1)^k \det\vb{A}_k > 0
	\quad(k=1,2,\dotsc,n).
\end{equation*}
\end{corollary}

\begin{proposition}
设\(\vb{A} \in M_{s \times n}(\mathbb{R})\),
则\begin{equation*}
	\text{\(\vb{A}^T\vb{A}\)正定}
	\iff
	\rank\vb{A} = n.
\end{equation*}
\begin{proof}
显然矩阵\(\vb{A}^T\vb{A}\)是\(n\)阶实对称矩阵,
于是\begin{align*}
	&\text{矩阵\(\vb{A}^T\vb{A}\)正定} \\
	&\iff (\forall\vb{x}\in\mathbb{R}^n-\{\vb0\})[\vb{x}^T (\vb{A}^T \vb{A}) \vb{x} > 0]
		\tag{\hyperref[definition:实二次型的分类.实二次型的分类]{正定矩阵的定义}} \\
	&\iff (\forall\vb{x}\in\mathbb{R}^n-\{\vb0\})[\norm{\vb{A}\vb{x}} > 0] \\
	&\iff (\forall\vb{x}\in\mathbb{R}^n-\{\vb0\})[\vb{A}\vb{x}\neq\vb0] \\
	&\iff \text{\(\vb{A}\vb{x}=\vb0\)只有零解} \\
	&\iff \rank\vb{A} = n,
		\tag{\cref{theorem:向量空间.有解的非齐次线性方程组的解的个数定理}}
\end{align*}
也就是说,“矩阵\(\vb{A}^T\vb{A}\)正定”的充分必要条件是“\(\vb{A}\)是列满秩矩阵”.
\end{proof}
\end{proposition}
\begin{corollary}
设\(\vb{A} \in M_{s \times n}(\mathbb{R})\),
则\(\vb{A}^T \vb{A}\)半正定.
\end{corollary}

\begin{example}
设\(\vb{A} \in M_s(\mathbb{R}),
\vb{B} \in M_n(\mathbb{R})\).
证明:
若矩阵\(\begin{bmatrix}
	\vb{A} & \vb0 \\
	\vb0 & \vb{B}
\end{bmatrix}\)是正定矩阵,
则\(\vb{A}\)和\(\vb{B}\)都是正定矩阵.
\begin{proof}
由题意有,
对于\(\forall\vb{x}\in\mathbb{R}^s,
\forall\vb{y}\in\mathbb{R}^n\),
只要\((\vb{x},\vb{y})\neq\vb0\),
就有\begin{equation*}
	\begin{bmatrix}
		\vb{x}^T & \vb{y}^T
	\end{bmatrix}
	\begin{bmatrix}
		\vb{A} & \vb0 \\
		\vb0 & \vb{B}
	\end{bmatrix}
	\begin{bmatrix}
		\vb{x} \\ \vb{y}
	\end{bmatrix}
	= \vb{x}^T \vb{A} \vb{x} + \vb{y}^T \vb{B} \vb{y}
	> 0.
	\eqno(1)
\end{equation*}
当\(\vb{x}\neq\vb0,\vb{y}=\vb0\)时,
由(1)式有\(\vb{x}^T\vb{A}\vb{x}>0\);
这就是说,矩阵\(\vb{A}\)正定.
同理,当\(\vb{x}=\vb0,\vb{y}\neq\vb0\)时,
由(1)式有\(\vb{y}^T\vb{B}\vb{y}>0\);
这就是说,矩阵\(\vb{B}\)正定.
\end{proof}
\end{example}

\begin{example}
%@see: 《线性代数》(张慎语、周厚隆) P134 例4
设\(\vb{A}\)为实对称矩阵.
证明:当实数\(t\)充分大时,\(t\vb{E}+\vb{A}\)是正定矩阵.
\begin{proof}
因为\(\vb{A}\)为实对称矩阵,所以存在正交矩阵\(\vb{Q}\),使得\begin{equation*}
	\vb{Q}^T\vb{A}\vb{Q} = \diag(\AutoTuple{\lambda}{n}).
\end{equation*}
又因为\begin{equation*}
	\vb{Q}^T(t\vb{E}+\vb{A})\vb{Q}
	= \diag(t+\lambda_1,t+\lambda_2,\dotsc,t+\lambda_n),
\end{equation*}
所以当\(t+\lambda_1,t+\lambda_2,\dotsc,t+\lambda_n\)都大于零时,\(t\vb{E}+\vb{A}\)正定.
\end{proof}
\end{example}

\begin{example}
%@see: 《线性代数》(张慎语、周厚隆) P134 习题6.3 3.(1)
%@see: 《线性代数》(张慎语、周厚隆) P134 习题6.3 3.(2)
设\(\vb{A}\)、\(\vb{B}\)是同阶正定矩阵.
证明:\(\vb{A}+\vb{B}\)、\(\vb{A}^{-1}\)、\(\vb{A}^*\)是正定矩阵.
\begin{proof}
根据正定矩阵的定义,因为\(\vb{A}\)是正定矩阵,任取非零列向量\(\vb{x}\),都有\begin{equation*}
	\vb{x}^T\vb{A}\vb{x} > 0;
\end{equation*}
同样地,有\(\vb{x}^T\vb{B}\vb{x} > 0\).

又根据矩阵的乘法分配律,成立\begin{equation*}
	\vb{x}^T(\vb{A}+\vb{B})\vb{x} = \vb{x}^T\vb{A}\vb{x} + \vb{x}^T\vb{B}\vb{x} > 0,
\end{equation*}
即\(\vb{A}+\vb{B}\)是正定矩阵.

因为\(\vb{A}\)是正定矩阵,存在可逆实阵\(\vb{P}\)使得\(\vb{P}^T\vb{A}\vb{P}=\vb{E}\),
所以\begin{equation*}
	\vb{A} = (\vb{P}^T)^{-1}\vb{E}\vb{P}^{-1} = (\vb{P}^T)^{-1}\vb{P}^{-1}
	\implies
	\vb{A}^{-1} = \vb{P}\vb{P}^T,
\end{equation*}
说明\(\vb{A}^{-1}\)是正定矩阵.

由逆矩阵的定义,\(\vb{A}^{-1}=\frac{1}{\abs{\vb{A}}}\vb{A}^*\),
那么\(\vb{A}^*=\abs{\vb{A}}\vb{A}^{-1}\),\(\abs{\vb{A}}>0\),
显然\(\vb{A}^*\)也是正定矩阵.
\end{proof}
\end{example}

\begin{example}
%@see: 《线性代数》(张慎语、周厚隆) P134 习题6.3 3.(3)
设\(\vb{A}\)是正定矩阵.
试证:存在正定矩阵\(\vb{B}\),使得\(\vb{A}=\vb{B}^2\).
\begin{proof}
设\(\vb{A}\)是\(n\)阶正定矩阵,那么存在正交矩阵\(\vb{P}\)满足\begin{equation*}
	\vb{P}^T\vb{A}\vb{P} = \diag(\AutoTuple{\lambda}{n}) = \vb{\Lambda},
\end{equation*}
其中\(\AutoTuple{\lambda}{n} \in \mathbb{R}^+\).
又设矩阵\(\vb{B}\)满足\(\vb{B}^2=\vb{A}\),那么\begin{equation*}
	\vb{P}^T\vb{A}\vb{P} = \vb{P}^T\vb{B}^2\vb{P} = \vb{\Lambda}
	\iff
	\vb{B}^2 = \vb{P}\vb{\Lambda}\vb{P}^T,
\end{equation*}
只需要令\(\vb{B} = \vb{P} \diag(\sqrt{\lambda_1},\sqrt{\lambda_2},\dotsc,\sqrt{\lambda_n}) \vb{P}^T\)即可.
\end{proof}
\end{example}

\begin{theorem}
设\(\vb{A}\)为\(n\)阶实对称矩阵,
\(f(\AutoTuple{x}{n}) = \vb{x}^T\vb{A}\vb{x}\),
则下列命题相互等价:\begin{itemize}
	\item \(\vb{A}\)是半正定矩阵.
	\item \(\vb{A}\)的特征值全是非负实数.
	\item \(f(\AutoTuple{x}{n})\)的正惯性指数\(p\)等于\(\vb{A}\)的秩\(\rank\vb{A}\).
	\item \(\vb{A} \cong \diag(\vb{E}_r,\vb0)\),其中\(\vb{E}_r\)是\(r\)阶单位矩阵.
\end{itemize}
%TODO proof
\end{theorem}
