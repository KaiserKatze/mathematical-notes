\section{矩阵合同的概念}
\begin{definition}
%@see: 《线性代数》(张慎语、周厚隆) P118 定义2
%@see: 《高等代数(第三版 上册)》(丘维声) P193 定义3
设\(\A\)和\(\B\)是数域\(K\)上的两个\(n\)阶矩阵.
若存在数域\(K\)上的一个可逆矩阵\(\C\),
使得\[
	\B=\C^T\A\C,
\]
则称“\(\A\)与\(\B\)~\DefineConcept{合同}(congruent)”
“\(\B\)是\(\A\)的\DefineConcept{合同矩阵}”,
记为\(\A\simeq\B\).
\end{definition}

\begin{property}
%@see: 《线性代数》(张慎语、周厚隆) P118
%@see: 《高等代数(第三版 上册)》(丘维声) P193
矩阵的合同关系是等价关系,
即具备下列三条性质:\begin{enumerate}
	\item {\rm\bf 反身性}:\(\A\simeq\A\);
	\item {\rm\bf 对称性}:\(\A\simeq\B \implies \B\simeq\A\);
	\item {\rm\bf 传递性}:\(\A\simeq\B \land \B\simeq\C \implies \A\simeq\C\).
\end{enumerate}
\begin{proof}
合同是集合\(M_n(K)\)上的一个二元关系.

由于\[
	\A = \E^T \A \E,
\]
所以\(\A \simeq \A\).

假设\(\A \simeq \B\),
那么存在可逆矩阵\(\C\)使得\[
	\B=\C^T\A\C,
\]
于是\[
	\A=(\C^T)^{-1}\B\C^{-1}
	=(\C^{-1})^T\B\C^{-1},
\]
即\(\B \simeq \A\).

假设\(\A \simeq \B\)且\(\B \simeq \C\),
那么存在可逆矩阵\(\D_1,\D_2\)
使得\[
	\B = \D_1^T \A \D_1, \qquad
	\C = \D_2^T \B \D_2,
\]
从而有\[
	\C = \D_2^T (\D_1^T \A \D_1) \D_2
	= (\D_1 \D_2)^T \A (\D_1 \D_2),
\]
因此\(\A \simeq \C\).
\end{proof}
\end{property}

\begin{definition}
%@see: 《高等代数(第三版 上册)》(丘维声) P193
把矩阵\(\A \in M_n(K)\)在合同关系下的等价类\[
	\Set{ \B \in M_n(K) \given \A \simeq \B }
\]称为“矩阵\(\A\)的\DefineConcept{合同类}”.
\end{definition}

\begin{property}
合同矩阵与原矩阵等价,即\(\A\simeq\B \implies \A\cong\B\).
\begin{proof}
由\hyperref[definition:逆矩阵.矩阵等价]{矩阵等价的定义}显然有.
\end{proof}
\end{property}

\begin{property}
%@see: 《线性代数》(张慎语、周厚隆) P119 习题6.1 2(2)选项(A)
合同矩阵的秩与原矩阵相等,即\(\A\simeq\B \implies \rank\A=\rank\B\).
\begin{proof}
由\cref{theorem:线性方程组.初等变换不变秩} 立得.
\end{proof}
\end{property}

\begin{proposition}\label{theorem:矩阵的合同.对称矩阵的合同矩阵也是对称的}
对称矩阵的合同矩阵也是对称的,
即\[
	\A^T = \A
	\implies
	[\A\simeq\B \implies \B^T = \B].
\]
\begin{proof}
设可逆矩阵\(\C\)满足\(\C^T\A\C=\B\),
那么\[
	\B^T = (\C^T\A\C)^T = \C^T\A^T\C = \C^T\A\C = \B.
	\qedhere
\]
\end{proof}
\end{proposition}

\begin{proposition}\label{theorem:矩阵的合同.反对称矩阵的合同矩阵也是反对称的}
反对称矩阵的合同矩阵也是反对称的,
即\[
	\A^T = -\A
	\implies
	[\A\simeq\B \implies \B^T = -\B].
\]
\begin{proof}
设可逆矩阵\(\C\)满足\(\C^T\A\C=\B\),
那么\[
	\B^T = (\C^T\A\C)^T = \C^T\A^T\C = -\C^T\A\C = -\B.
	\qedhere
\]
\end{proof}
\end{proposition}

\begin{remark}
从\cref{theorem:矩阵的合同.对称矩阵的合同矩阵也是对称的,theorem:矩阵的合同.反对称矩阵的合同矩阵也是反对称的}
可以看出:
对称矩阵不可能与反对称矩阵合同,
它们都不可能与非对称矩阵合同.
\end{remark}

\begin{proposition}\label{theorem:矩阵合同.合同矩阵的行列式的关系}
设\(n\)阶矩阵\(\A,\B,\C\)满足\(\B=\C^T\A\C^T\),
则\[
	\abs{\A}\abs{\C}^2
	=\abs{\B}.
\]
\begin{proof}
根据\cref{theorem:行列式.性质1,theorem:行列式.矩阵乘积的行列式},
必有\[
	\abs{\B}
	=\abs{\C^T\A\C}
	=\abs{\C^T}\abs{\A}\abs{\C}
	=\abs{\A}\abs{\C}^2.
	\qedhere
\]
\end{proof}
\end{proposition}

\begin{proposition}
%@see: 《线性代数》(张慎语、周厚隆) P119 习题6.1 2(2)选项(B)
合同矩阵的行列式同号,即\(\A\simeq\B\implies\sgn\abs{\A}=\sgn\abs{\B}\).
\begin{proof}
设\(n\)阶矩阵\(\A,\B,\C\)满足\(\B=\C^T\A\C^T\),
由\cref{theorem:矩阵合同.合同矩阵的行列式的关系} 有\[
	\abs{\A}\abs{\B}
	=\abs{\A}(\abs{\A}\abs{\C}^2)
	=\abs{\A}^2\abs{\C}^2 \geq 0.
	\qedhere
\]
\end{proof}
\end{proposition}

\begin{example}
相似的两个矩阵不一定合同.
例如,矩阵\(\A=\begin{bmatrix}
	1 & 1 \\
	0 & 2
\end{bmatrix}\)
和\(\B=\begin{bmatrix}
	0 & -1 \\
	2 & 3
\end{bmatrix}\)相似,
这是因为可逆矩阵\(\P=\begin{bmatrix}
	1 & 0 \\
	-1 & -1
\end{bmatrix}\)满足\(\P^{-1}\A\P=\B\).
%@Mathematica: Inverse[{{1, 0}, {-1, -1}}].{{1, 1}, {0, 2}}.{{1, 0}, {-1, -1}}
现在假设存在矩阵\(\Q=\begin{bmatrix}
	a & b \\
	c & d
\end{bmatrix}\)使得\(\Q^T\A\Q=\B\).
那么有\[
	\Q^T\A\Q-\B
	=\begin{bmatrix}
		a^2+ac+2c^2 & 1+ab+ad+2cd \\
		-2+ab+bc+2cd & -3+b^2+bd+2d^2
	\end{bmatrix}
	=\vb0,
\]
%@Mathematica: Expand[{{a, c}, {b, d}}.{{1, 1}, {0, 2}}.{{a, b}, {c, d}} - {{0, -1}, {2, 3}}]
得到关于\(a,b,c,d\)的方程组
\begin{align*}
	\left\{ \begin{array}{l}
		a^2+ac+2c^2 = 0, \\
		1+ab+ad+2cd = 0, \\
		-2+ab+bc+2cd = 0, \\
		-3+b^2+bd+2d^2 = 0,
	\end{array} \right.
	\tag*{$\begin{matrix}(1)\\(2)\\(3)\\(4)\end{matrix}$}
\end{align*}
(2)式减去(3)式得\[
	3+ad-bc=0,
\]
这就是说\(\abs{\Q}=-3\).
但是根据\cref{theorem:矩阵合同.合同矩阵的行列式的关系},
因为\(\abs{\A}=\abs{\B}=2\),所以\(\abs{\Q}^2=1\).
矛盾!因此\(\A\)与\(\B\)不合同!
\end{example}

\begin{example}\label{example:矩阵合同.合同矩阵不一定相似}
合同的两个矩阵不一定相似.
例如,矩阵\(\A=\begin{bmatrix}
	1 & 0 \\
	0 & 1
\end{bmatrix}\)与\(\B=\begin{bmatrix}
	4 & 0 \\
	0 & 1
\end{bmatrix}\)合同,
这是因为可逆矩阵\(\Q=\begin{bmatrix}
	2 & 0 \\
	0 & 1
\end{bmatrix}\)满足\(\Q^T\A\Q=\B\).
%@Mathematica: {{2, 0}, {0, 1}}.{{1, 0}, {0, 1}}.{{2, 0}, {0, 1}}
但是\(\tr\A=2 \neq \tr\B = 5\),
根据\cref{theorem:特征值与特征向量.矩阵相似的必要条件4},
\(\A\)与\(\B\)必不相似!
\end{example}

\begin{proposition}\label{theorem:二次型.实对称矩阵相似必合同}
设\(\A,\B\)都是实对称矩阵,
则“\(\A\)与\(\B\)相似”是“\(\A\)与\(\B\)合同”的充分不必要条件.
\begin{proof}
因为\(\A\)、\(\B\)都是实对称矩阵,
\(\A^T=\A\),
\(\B^T=\B\),
且存在正交矩阵\(\Q_1,\Q_2\)使得\[
	\Q_1^{-1} \A \Q_1 = \Q_1^T \A \Q_1 = \V_1,
	\qquad
	\Q_2^{-1} \B \Q_2 = \Q_2^T \B \Q_2 = \V_2,
\]
其中\(\V_1,\V_2\)是对角阵.
又因为\(\A\sim\B\),
所以\(\A\)与\(\B\)有相同的特征多项式、特征值,
即\(\V_1=\V_2\),
或\[
	\Q_1^{-1} \A \Q_1 = \Q_2^{-1} \B \Q_2,
	\qquad
	(\Q_2 \Q_1^{-1}) \A (\Q_1 \Q_2^{-1}) = \B.
\]
令\(\P = \Q_1 \Q_2^{-1}\),
\(\P^T = (\Q_1 \Q_2^{-1})^T
= (\Q_2^{-1})^T \Q_1^T
= \Q_2 \Q_1^{-1}
= (\Q_1 \Q_2^{-1})^{-1}
= \P^{-1}\),
那么\[
	\P^T \A \P = \B,
\]
也就是说\(\A\)与\(\B\)合同.

再根据\cref{example:矩阵合同.合同矩阵不一定相似},合同的两个实对称矩阵未必相似.
因此,对于实对称矩阵\(\A,\B\)而言,
“\(\A\)与\(\B\)相似”是“\(\A\)与\(\B\)合同”的充分不必要条件.
\end{proof}
\end{proposition}

\begin{proposition}
设\(\A,\B \in M_n(K)\).
记\begin{gather*}
	\A_1 = \frac12 (\A+\A^T), \qquad
	\A_2 = \frac12 (\A-\A^T), \\
	\B_1 = \frac12 (\B+\B^T), \qquad
	\B_2 = \frac12 (\B-\B^T),
\end{gather*}
其中\(\A^T,\B^T\)分别为\(\A,\B\)的转置.
那么\[
	\A \simeq \B \implies \A_1 \simeq \B_1 \land \A_2 \simeq \B_2.
\]
\begin{proof}
当\(\A \simeq \B\)时,
必存在可逆矩阵\(\C\),使得\[
	\B
	=\C^T\A\C;
	\eqno(1)
\]
于是\[
	\B^T
	=(\C^T\A\C)^T
	=\C^T\A^T\C,
	\eqno(2)
\]
也就是说\(\B^T \simeq \A^T\).
由(1)(2)两式不难得到\[
	\B\pm\B^T
	=\C^T(\A\pm\A^T)\C,
\]
于是\(\A_1\simeq\B_1,
\A_2\simeq\B_2\).
\end{proof}
\end{proposition}

\begin{example}
设\(\AutoTuple{i}{n}\)是\(1,2,\dotsc,n\)的一个排列,
而\[
	\A=\diag(\AutoTuple{d}{n}),
	\qquad
	\B=\diag(d_{i_1},d_{i_2},\dotsc,d_{i_n}).
\]
证明:矩阵\(\A,\B\)合同且相似.
\begin{proof}
显然\(\diag(d_{i_1},d_{i_2},\dotsc,d_{i_n})\)
可由\(\diag(\AutoTuple{d}{n})\)同时左乘和右乘若干个初等矩阵\[
	\P(i,j) \quad(1 \leq i < j \leq n)
\]得到,
又因为\(\P(i,j)^T = \P(i,j)^{-1} = \P(i,j)\),
所以只要令这些初等矩阵的乘积为\(\P\),
就有\[
	\P^{-1} \A \P = \B,
	\qquad
	\P^T \A \P = \B.
\]
也就是说\(\A\simeq\B\land\A\sim\B\).
\end{proof}
\end{example}

根据\cref{theorem:特征值与特征向量.实对称矩阵3},
任一实对称矩阵都与某个对角阵合同且相似,
实际上,我们可以把这个结论推广到复数域上的对称矩阵.
\begin{theorem}
任一对称矩阵都合同于某个对角矩阵.
\end{theorem}
