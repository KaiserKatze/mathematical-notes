\section{本章总结}

\begin{table}[htb]
	\centering
	\begin{tblr}{*4{|c}|}
		\hline
		& 等价\(\vb{A}\cong\vb{B}\) & 相似\(\vb{A}\sim\vb{B}\) & 合同\(\vb{A}\simeq\vb{B}\) \\ \hline
		秩 & \(\rank\vb{A}=\rank\vb{B}\) & 同左 & 同左 \\ \hline
		行列式 & & \(\DeterminantA{\vb{A}}=\DeterminantA{\vb{B}}\) & \(\sgn\DeterminantA{\vb{A}}=\sgn\DeterminantA{\vb{B}}\) \\ \hline
		特征多项式 & & \(\DeterminantA{\lambda\vb{E}-\vb{A}}=\DeterminantA{\lambda\vb{E}-\vb{B}}\) \\ \hline
		迹 & & \(\tr\vb{A}=\tr\vb{B}\) \\ \hline
		\SetCell[r=4]{c} 三者的关系 & \SetCell[c=3]{c} 相似一定等价,但不一定合同. \\
				& \SetCell[c=3]{c} 合同一定等价,但不一定相似. \\
				& \SetCell[c=3]{c} 等价既不一定相似,也不一定合同. \\
				& \SetCell[c=3]{c} 相似实对称阵必定合同. \\
		\hline
	\end{tblr}
	\caption{}
\end{table}

任给一个二次型\(f(\AutoTuple{x}{n}) = \vb{x}^T\vb{A}\vb{x}\),
我们想要求出正交变换\(\vb{x}=\vb{Q}\vb{y}\),把这个二次型化为它的标准型\(g\),
可以按以下步骤操作:
\begin{enumerate}
	\item 首先写出二次型\(f\)的矩阵\(\vb{A}\).

	\item 写出特征多项式\(\DeterminantA{\lambda\vb{E}-\vb{A}}\),
	令\(\DeterminantA{\lambda\vb{E}-\vb{A}}=0\),
	求出特征值\(\AutoTuple{\lambda}{n}\).

	\item 建立线性方程\((\lambda\vb{E}-\vb{A})\vb{x}=\vb0\),求出特征向量\(\vb{x}\).

	\item 利用施密特方法,将\(\vb{A}\)的特征向量正交化.

	注意到\(\vb{A}\)作为实对称矩阵,
	它的任意一对不同特征值\(\lambda_1,\lambda_2\)分别对应的特征向量\(\vb{x}_1,\vb{x}_2\)总是正交的,
	所以我们只需要针对它的每一组对应相同特征值的特征向量进行正交化,
	就可以取得\(\vb{A}\)的正交化特征向量组.

	不过,我们也可以将第3步、第4步合并成一个步骤,
	具体来说,就是在解方程\((\lambda\vb{E}-\vb{A})\vb{x}=\vb0\)时,
	只要解得一个解向量\(\vb{x}_0\),就从这个解向量出发,
	考察它的各个坐标,
	构造出一组既可以满足方程\((\lambda\vb{E}-\vb{A})\vb{x}=\vb0\),
	也可以与\(\vb{x}_0\)正交的向量.

	\item 规范化\(\vb{A}\)的全部特征向量,
	得到\(\vb{A}\)的正交规范化特征向量组\(\{\AutoTuple{\vb{\xi}}{n}\}\).

	\item 写出正交矩阵\(\vb{Q}=(\AutoTuple{\vb{\xi}}{n})\),
	还可以写出二次型\(f\)对应的标准型\begin{equation*}
		g(\AutoTuple{y}{n}) = \lambda_1 y_1^2 + \dotsb + \lambda_n y_n^2.
	\end{equation*}
\end{enumerate}
