\section{本章总结}
\subsection*{行列式的计算方法}
我们可以采用以下方法计算行列式:
\begin{enumerate}
	\item 利用行列式的定义计算.
	\item 利用初等变换,将行列式化为上三角形,如此行列式就等于主对角线上元素之积.
	\item 拆成若干个行列式的和.
	\item 按行(或列)展开.
	\item 数学归纳法.
	\item 递推关系法.
	\item 升阶法.
	\item \hyperref[theorem:逆矩阵.行列式降阶定理]{降阶法}.
	\item 利用矩阵乘法,例如\begin{gather*}
		\abs{\vb{A}}^2 = \abs{\vb{A}\vphantom{\vb{A}^T}} \abs{\vb{A}^T}, \\
		\abs{\vb{A}} = \frac{\abs{\vb{A} \vb{B}}}{\abs{\vb{B}}} \quad(\abs{\vb{B}}\neq0).
	\end{gather*}
	\item 利用\hyperref[equation:行列式.范德蒙德行列式]{范德蒙德行列式}等特殊行列式计算.
\end{enumerate}

\subsection*{重要公式}
设\(A_{ij}\)是\(\vb{A}\)中\(a_{ij}\)的代数余子式,
则\begin{gather*}
	\sum_{j=1}^n a_{ij} A_{kj}
	= \left\{ \begin{array}{cl}
		\abs{\vb{A}}, & k = i, \\
		0, & k \neq i,
	\end{array} \right.
	\quad i=1,2,\dotsc,n, \\
	\sum_{i=1}^n a_{ij} A_{ik}
	= \left\{ \begin{array}{cl}
		\abs{\vb{A}}, & k = j, \\
		0, & k \neq j,
	\end{array} \right.
	\quad j=1,2,\dotsc,n.
\end{gather*}

与伴随矩阵有关的等式:
\begin{gather*}
	\vb{A} \vb{A}^* = \vb{A}^* \vb{A} = \abs{\vb{A}} \vb{E}, \\ %\cref{equation:行列式.伴随矩阵.恒等式1}
	(\vb{A}^*)^T = (\vb{A}^T)^*, \\ %\cref{equation:行列式.伴随矩阵.恒等式2}
	(\vb{A} \vb{B})^* = \vb{B}^* \vb{A}^*, \\ %\cref{equation:行列式.伴随矩阵.恒等式3}
	(k \vb{A})^* = k^{n-1} \vb{A}^*. %\cref{equation:行列式.伴随矩阵.数与矩阵乘积的伴随}
\end{gather*}

\subsection*{重要行列式}
\begin{gather*}
	\begin{vmatrix}
		a_{11} & a_{12} & \dots & a_{1n} \\
		& a_{22} & \dots & a_{2n} \\
		& & \ddots & \vdots \\
		& & & a_{nn}
	\end{vmatrix}
	= a_{11} a_{22} \dotsm a_{nn}. \\%
	\begin{vmatrix}
		& & & & a_{1n} \\
		& & & a_{2,n-1} & a_{2n} \\
		& & & \vdots & \vdots \\
		& a_{n-1,2} & \dots & a_{n-1,n-1} & a_{n-1,n} \\
		a_{n1} & a_{n2} & \dots & a_{n,n-1} & a_{nn}
	\end{vmatrix}
	=(-1)^{\frac{1}{2}n(n-1)} a_{1n} a_{2,n-1} \dotsm a_{n-1,2} a_{n1}. \\
	\begin{vmatrix}
		k & \lambda & \lambda & \dots & \lambda \\
		\lambda & k & \lambda & \dots & \lambda \\
		\lambda & \lambda & k & \dots & \lambda \\
		\vdots & \vdots & \vdots & & \vdots \\
		\lambda & \lambda & \lambda & \dots & k
	\end{vmatrix}_n
	= [k+(n-1)\lambda] (k-\lambda)^{n-1},
	\quad k\neq\lambda,n=1,2,\dotsc. \\
	%\cref{equation:行列式.范德蒙德行列式}
	\begin{vmatrix}
		1 & 1 & 1 & \dots & 1 \\
		x_1 & x_2 & x_3 & \dots & x_n \\
		x_1^2 & x_2^2 & x_3^2 & \dots & x_n^2 \\
		\vdots & \vdots & \vdots& & \vdots \\
		x_1^{n-1} & x_2^{n-1} & x_3^{n-1} & \dots & x_n^{n-1}
	\end{vmatrix}
	= \prod_{1 \leq j < i \leq n}(x_i-x_j). \\
	%\cref{example:行列式的展开.三对角行列式}
	\begin{vmatrix}
		a+b & ab & \\
		1 & a+b & ab & \\
		& 1 & a + b & \ddots & \\
		& & \ddots & \ddots & ab \\
		& & & 1 & a+b \\
	\end{vmatrix}_n
	= a^n + a^{n-1} b + a^{n-2} b^2 + \dotsb + a b^{n-1} + b^n. \\
	\begin{vmatrix}
		1 & 1 & \\
		1 & 1 & 1 & \\
		& 1 & 1 & \ddots & \\
		& & \ddots & \ddots & 1 \\
		& & & 1 & 1 \\
	\end{vmatrix}_n
	= \begin{vmatrix}
		1 & 1 & \\
		1 & 1 & 1 & \\
		& 1 & 1 & \ddots & \\
		& & \ddots & \ddots & 1 \\
		& & & 1 & 1 \\
	\end{vmatrix}_{n-1}
	+ \begin{vmatrix}
		1 & 1 & \\
		1 & 1 & 1 & \\
		& 1 & 1 & \ddots & \\
		& & \ddots & \ddots & 1 \\
		& & & 1 & 1 \\
	\end{vmatrix}_{n-2}. \\
	\begin{vmatrix}
		a & b & c & d \\
		-b & a & d & -c \\
		-c & -d & a & b \\
		-d & c & -b & a
	\end{vmatrix}
	= (a^2+b^2+c^2+d^2)^2.
\end{gather*}
