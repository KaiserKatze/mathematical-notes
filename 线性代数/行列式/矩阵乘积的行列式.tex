\section{矩阵乘积的行列式}
\begin{lemma}
%@see: 《线性代数》(张慎语、周厚隆) P38 引理
%@see: 《高等代数(第三版 上册)》(丘维声) P52 例1
设\(\vb{A}\in M_n^*(K),
\vb{B}\in M_m^*(K),
\vb{C}\in M_{m\times n}(K),
\vb{D}\in M_{n\times m}(K)\),
则\begin{align}
	\begin{vmatrix}
		\vb{A} & \vb0 \\
		\vb{C} & \vb{B}
	\end{vmatrix}
	&= \begin{vmatrix}
		\vb{A} & \vb{D} \\
		\vb0 & \vb{B}
	\end{vmatrix}
	= \abs{\vb{A}} \abs{\vb{B}}, \label{equation:行列式.广义三角阵的行列式1} \\
	\begin{vmatrix}
		\vb0 & \vb{A} \\
		\vb{B} & \vb{C}
	\end{vmatrix}
	&= \begin{vmatrix}
		\vb{D} & \vb{A} \\
		\vb{B} & \vb0
	\end{vmatrix}
	= (-1)^{mn} \abs{\vb{A}} \abs{\vb{B}}. \label{equation:行列式.广义三角阵的行列式2}
\end{align}
%TODO proof
\end{lemma}

\begin{theorem}[矩阵乘积的行列式定理]\label{theorem:行列式.矩阵乘积的行列式}
%@see: 《线性代数》(张慎语、周厚隆) P39 定理5(矩阵乘积的行列式定理)
%@see: 《高等代数(第三版 上册)》(丘维声) P123 定理3
设\(\vb{A},\vb{B}\in M_n(K)\),
则\begin{equation}
	\abs{\vb{A} \vb{B}} = \abs{\vb{A}} \abs{\vb{B}}.
\end{equation}
%TODO proof
\end{theorem}

\begin{example}\label{example:幂零矩阵.幂零矩阵的行列式}
证明:幂零矩阵的行列式等于\(0\).
\begin{proof}
设\(\vb{A}\)是数域\(K\)上的\(n\)阶幂零矩阵,\(\vb{A}^m = \vb0\ (m\in\mathbb{N}^+)\).
于是由\cref{theorem:行列式.矩阵乘积的行列式} 可知\[
	\abs{\vb{A}}^m
	= \abs{\vb{A}^m}
	= \abs{\vb0}
	= 0,
\]
解得\(\abs{\vb{A}} = 0\).
\end{proof}
\end{example}

\begin{example}\label{example:正交矩阵.行列式小于零的正交矩阵与单位矩阵之和的行列式等于零}
%@see: 《线性代数》(张慎语、周厚隆) P39 例1
%@see: 《1995年全国硕士研究生入学统一考试(数学一)》九解答题
设矩阵\(\vb{A} \in M_n(K)\),
\(\vb{E}\)是数域\(K\)上的\(n\)阶单位矩阵,
且\(\vb{A} \vb{A}^T = \vb{E}\),\(\abs{\vb{A}}<0\).
证明:\[
	\abs{\vb{E}+\vb{A}}=0.
\]
\begin{proof}
等式\(\vb{A} \vb{A}^T=\vb{E}\)的两端取行列式,得\[
	\abs{\vb{A}} \abs{\vb{A}^T}
	= \abs{\vb{E}}.
\]
由\(\abs{\vb{A}} = \abs{\vb{A}^T}\)可知\[
	\abs{\vb{A}} \abs{\vb{A}^T}
	= \abs{\vb{A}}^2,
\]
从而有\[
	\abs{\vb{A}}^2 = \abs{\vb{E}}.
\]
由于\(\abs{\vb{A}} < 0\),
所以\[
	\abs{\vb{A}} = -1.
\]
于是\begin{align*}
	\abs{\vb{E}+\vb{A}}
	&= \abs{(\vb{A} \vb{A}^T)+(\vb{A} \vb{E})}
	= \abs{\vb{A}(\vb{A}^T+\vb{E})}
	= \abs{\vb{A}} \abs{\vb{A}^T+\vb{E}} \\
	&= -1 \cdot \abs{(\vb{A}^T+\vb{E})^T}
	= -\abs{\vb{A}+\vb{E}},
\end{align*}
因此\(\abs{\vb{E}+\vb{A}}=0\).
\end{proof}
\end{example}

\begin{example}
设矩阵\(\vb{A},\vb{B}\)满足\(\abs{\vb{A}}=3,\abs{\vb{B}}=2,\abs{\vb{A}^{-1}+\vb{B}}=2\),
试求\(\abs{\vb{A}+\vb{B}^{-1}}\).
\begin{solution}
由于\(\abs{\vb{E}+\vb{A}\vb{B}} = \abs{\vb{A}(\vb{A}^{-1}+\vb{B})} = \abs{\vb{A}} \abs{\vb{A}^{-1}+\vb{B}} = 6\),
所以\[
	\abs{\vb{A}+\vb{B}^{-1}}
	= \frac{\abs{(\vb{A}+\vb{B}^{-1})\vb{B}}}{\abs{\vb{B}}}
	= \frac{\abs{\vb{A}\vb{B}+\vb{E}}}{\abs{\vb{B}}}
	= 3.
\]
\end{solution}
\end{example}

\begin{example}
用\(\abs{\vb{A}}^2 = \abs{\vb{A}} \abs{\vb{A}^T}\)的方法计算行列式\[
	\abs{\vb{A}} = \begin{vmatrix}
		a & b & c & d \\
		-b & a & d & -c \\
		-c & -d & a & b \\
		-d & c & -b & a
	\end{vmatrix}.
\]
\begin{solution}
因为\begin{align*}
	\abs{\vb{A}}^2 &= \abs{\vb{A}} \abs{\vb{A}^T} = \abs{\vb{A} \vb{A}^T} \\
	&= \abs{\begin{bmatrix}
		a & b & c & d \\
		-b & a & d & -c \\
		-c & -d & a & b \\
		-d & c & -b & a
	\end{bmatrix}
	\begin{bmatrix}
		a & -b & -c & -d \\
		b & a & -d & c \\
		c & d & a & -b \\
		d & -c & b & a
	\end{bmatrix}} \\
	&= \abs{(a^2+b^2+c^2+d^2) \vb{E}}_4
	= (a^2+b^2+c^2+d^2)^4,
\end{align*}
所以\(\abs{\vb{A}} = \pm(a^2+b^2+c^2+d^2)^2\),
再由\(\abs{\vb{A}}\)中含有项\(a^4\),
得\[
	\abs{\vb{A}} = (a^2+b^2+c^2+d^2)^2.
\]
\end{solution}
%@Mathematica: Det[({{a, b, c, d},{-b, a, d, -c},{-c, -d, a, b},{-d, c, -b, a}})] // Factor
\end{example}

\begin{example}
计算:\[
	D = \begin{vmatrix}
		a & a & a & a \\
		a & a & -a & -a \\
		a & -a & a & -a \\
		a & -a & -a & a
	\end{vmatrix}.
\]
\begin{solution}
因为\[
	\begin{bmatrix}
		1 & 0 & 0 & 0 \\
		0 & 0 & 1 & 0 \\
		0 & 1 & 0 & 0 \\
		0 & 0 & 0 & 1
	\end{bmatrix} \begin{bmatrix}
		a & a & a & a \\
		a & a & -a & -a \\
		a & -a & a & -a \\
		a & -a & -a & a
	\end{bmatrix}
	= \begin{bmatrix}
		1 & 0 & 0 & 0 \\
		1 & 1 & 0 & 0 \\
		1 & 0 & 1 & 0 \\
		1 & 1 & 1 & 1
	\end{bmatrix} \begin{bmatrix}
		a & a & a & a \\
		0 & -2 a & 0 & -2 a \\
		0 & 0 & -2 a & -2 a \\
		0 & 0 & 0 & 4 a
	\end{bmatrix},
\]而\[
	\begin{vmatrix}
		1 & 0 & 0 & 0 \\
		0 & 0 & 1 & 0 \\
		0 & 1 & 0 & 0 \\
		0 & 0 & 0 & 1
	\end{vmatrix} = -1,
	\qquad
	\begin{vmatrix}
		1 & 0 & 0 & 0 \\
		1 & 1 & 0 & 0 \\
		1 & 0 & 1 & 0 \\
		1 & 1 & 1 & 1
	\end{vmatrix} = 1,
\]\[
	\begin{vmatrix}
		a & a & a & a \\
		0 & -2 a & 0 & -2 a \\
		0 & 0 & -2 a & -2 a \\
		0 & 0 & 0 & 4 a
	\end{vmatrix}
	= a\cdot(-2a)\cdot(-2a)\cdot(4a) = 16a^2,
\]
所以\[
	\begin{vmatrix}
		a & a & a & a \\
		a & a & -a & -a \\
		a & -a & a & -a \\
		a & -a & -a & a
	\end{vmatrix}
	= -16a^2.
\]
\end{solution}
\end{example}

\begin{example}
设\(s_k = a_1^k + a_2^k + a_3^k + a_4^k\ (k=1,2,3,4,5,6)\).
计算:\[
	D = \begin{vmatrix}
		4 & s_1 & s_2 & s_3 \\
		s_1 & s_2 & s_3 & s_4 \\
		s_2 & s_3 & s_4 & s_5 \\
		s_3 & s_4 & s_5 & s_6
	\end{vmatrix}.
\]
\begin{solution}
令矩阵\[
	\vb{A} = \begin{bmatrix}
		1 & 1 & 1 & 1 \\
		a_1 & a_2 & a_3 & a_4 \\
		a_1^2 & a_2^2 & a_3^2 & a_4^2 \\
		a_1^3 & a_2^3 & a_3^3 & a_4^3
	\end{bmatrix},
\]
显然\[
	D = \det(\vb{A} \vb{A}^T) = \abs{\vb{A}}^2.
\]
而根据\cref{equation:行列式.范德蒙德行列式},
\(\abs{\vb{A}}
= \prod_{1 \leq j < i \leq n} (a_i - a_j)\),
故\(D = \prod_{1 \leq j < i \leq n} (a_i - a_j)^2\).
\end{solution}
\end{example}
