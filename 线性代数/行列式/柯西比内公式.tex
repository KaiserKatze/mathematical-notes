\section{柯西--比内公式}
\begin{theorem}
%@see: 《高等代数(第三版 上册)》(丘维声) P141 定理1
已知数域\(K\).
设矩阵\(\vb{A} \in M_{m \times n}(K),
\vb{B} \in M_{n \times m}(K)\).
如果\(m < n\),
那么\begin{equation}\label{equation:线性方程组.柯西比内公式}
	\abs{\vb{A}\vb{B}}
	= \sum_{1 \leq i_1 < i_2 < \dotsb < i_m \leq n}
	\MatrixMinor{\vb{A}}{
		1,2,\dotsc,m \\
		i_1,i_2,\dotsc,i_m
	}
	\MatrixMinor{\vb{B}}{
		i_1,i_2,\dotsc,i_m \\
		1,2,\dotsc,m
	}.
\end{equation}
\begin{proof}
考虑\(m+n\)阶分块矩阵\begin{equation*}
	\begin{bmatrix}
		\vb{E}_n & \vb{B} \\
		\vb0 & \vb{A}\vb{B}
	\end{bmatrix},
\end{equation*}
其中\(\vb{E}_n\)是数域\(K\)上的\(n\)阶单位矩阵.
由于\begin{equation*}
	\begin{vmatrix}
		\vb{E}_n & \vb{B} \\
		\vb0 & \vb{A}\vb{B}
	\end{vmatrix}
	= \abs{\vb{E}_n} \abs{\vb{A}\vb{B}}
	= \abs{\vb{A}\vb{B}},
\end{equation*}
所以\begin{equation*}
	\begin{bmatrix}
		\vb{E}_n & \vb{B} \\
		\vb0 & \vb{A}\vb{B}
	\end{bmatrix}
	\to
	\begin{bmatrix}
		\vb{E}_n & \vb{B} \\
		-\vb{A} & \vb0
	\end{bmatrix}
	= \begin{bmatrix}
		\vb{E}_n & \vb0 \\
		-\vb{A} & \vb{E}_m
	\end{bmatrix} \begin{bmatrix}
		\vb{E}_n & \vb{B} \\
		\vb0 & \vb{A}\vb{B}
	\end{bmatrix},
\end{equation*}\begin{equation*}
	\begin{vmatrix}
		\vb{E}_n & \vb{B} \\
		-\vb{A} & \vb0
	\end{vmatrix}
	= \begin{vmatrix}
		\vb{E}_n & \vb0 \\
		-\vb{A} & \vb{E}_m
	\end{vmatrix} \begin{vmatrix}
		\vb{E}_n & \vb{B} \\
		\vb0 & \vb{A}\vb{B}
	\end{vmatrix}
	= \begin{vmatrix}
		\vb{E}_n & \vb{B} \\
		\vb0 & \vb{A}\vb{B}
	\end{vmatrix},
\end{equation*}
其中\(\vb{E}_m\)是数域\(K\)上的\(m\)阶单位矩阵.
利用\hyperref[theorem:行列式.拉普拉斯定理]{拉普拉斯定理}把上式最左端行列式按后\(m\)行展开得\begin{equation*}
	\begin{vmatrix}
		\vb{E}_n & \vb{B} \\
		-\vb{A} & \vb0
	\end{vmatrix}
	= \sum_{1 \leq i_1 < \dotsb < i_m \leq n}
	\MatrixMinor{(-\vb{A})}{
		1,2,\dotsc,m \\
		i_1,i_2,\dotsc,i_m
	}
	(-1)^{[(n+1)+\dotsb+(n+m)]+(i_1+\dotsb+i_m)}
	\abs{(\vb\epsilon_{\mu_1},\dotsc,\vb\epsilon_{\mu_{n-m}},\vb{B})},
\end{equation*}
其中\(\Set{\mu_1,\dotsc,\mu_{n-m}}
= \Set{1,\dotsc,n}-\Set{i_1,\dotsc,i_s}\),
且\(\mu_1<\dotsb<\mu_{n-m}\).

把\(\abs{(\vb\epsilon_{\mu_1},\dotsc,\vb\epsilon_{\mu_{n-m}},\vb{B})}\)
按前\(n-m\)行展开得\begin{equation*}
	\abs{(\vb\epsilon_{\mu_1},\dotsc,\vb\epsilon_{\mu_{n-m}},\vb{B})}
	= \abs{\vb{E}_{n-m}}
	(-1)^{(\mu_1+\dotsb+\mu_{n-m})+[1+\dotsb+(n-m)]}
	\MatrixMinor{\vb{B}}{
		i_1,i_2,\dotsc,i_m \\
		1,2,\dotsc,m
	}.
\end{equation*}
因此\begin{align*}
	\begin{vmatrix}
		\vb{E}_n & \vb{B} \\
		-\vb{A} & \vb0
	\end{vmatrix}
	&= \sum_{1 \leq i_1 < \dotsb < i_m \leq n}
	(-1)^{m+m^2+n+n^2}
	\MatrixMinor{\vb{A}}{
		1,2,\dotsc,m \\
		i_1,i_2,\dotsc,i_m
	}
	\MatrixMinor{\vb{B}}{
		i_1,i_2,\dotsc,i_m \\
		1,2,\dotsc,m
	} \\
	&= \sum_{1 \leq i_1 < \dotsb < i_m \leq n}
	\MatrixMinor{\vb{A}}{
		1,2,\dotsc,m \\
		i_1,i_2,\dotsc,i_m
	}
	\MatrixMinor{\vb{B}}{
		i_1,i_2,\dotsc,i_m \\
		1,2,\dotsc,m
	}.
\end{align*}
综上所述,\begin{equation*}
	\abs{\vb{A}\vb{B}}
	= \sum_{1 \leq i_1 \leq i_2 \leq \dotsb \leq i_m \leq n}
	\MatrixMinor{\vb{A}}{
		1,2,\dotsc,m \\
		i_1,i_2,\dotsc,i_m
	}
	\MatrixMinor{\vb{B}}{
		i_1,i_2,\dotsc,i_m \\
		1,2,\dotsc,m
	}.
	\qedhere
\end{equation*}
\end{proof}
%\cref{theorem:矩阵乘积的秩.多行少列矩阵与少行多列矩阵的乘积的行列式}
\end{theorem}
\cref{equation:线性方程组.柯西比内公式} 称为\DefineConcept{柯西--比内公式}.


\begin{example}
设\(\vb{A} = (\vb{B},\vb{C}) \in M_{n \times m}(\mathbb{R})\),
其中\(\vb{B} \in M_{n \times s}(\mathbb{R})\),
\(\vb{C} \in M_{n \times (m-s)}(\mathbb{R})\).
证明:\begin{equation}
\abs{\vb{A}^T \vb{A}} \leq \abs{\vb{B}^T \vb{B}} \abs{\vb{C}^T \vb{C}}.
\end{equation}
%TODO
\end{example}

\begin{example}
设\(\vb{A} = (a_{ij})_n \in M_n(\mathbb{R})\).
证明:\begin{equation}\label{equation:线性方程组.Hadamard不等式}
	\abs{\vb{A}}^2 \leq \prod_{j=1}^n \sum_{i=1}^n a_{ij}^2.
\end{equation}
%TODO
\end{example}

\begin{example}
设\(\vb{A} = (a_{ij})_n \in M_n(\mathbb{R})\),
且\(\abs{a_{ij}} < M\ (i,j=1,2,\dotsc,n)\).
证明:\begin{equation}
	\abs{\det\vb{A}} \leq M^n n^{n/2}.
\end{equation}
%TODO
\end{example}
