\section{排列与逆序数}
\begin{definition}
\(n\)个不同的自然数按一定顺序排列组成的一个有序数组\begin{equation*}
	(\AutoTuple{k}{n})
\end{equation*}称为一个\(n\)阶\DefineConcept{排列}.

当\(1 \leq i<j \leq n\)时,
如果\(k_i>k_j\),则称“\(k_i,k_j\)构成一个\DefineConcept{逆序}”;
如果\(k_i<k_j\),则称“\(k_i,k_j\)构成一个\DefineConcept{顺序}”.
此排列中的逆序的总数叫它的\DefineConcept{逆序数},记作\(\tau(\AutoTuple{k}{n})\).

逆序数为偶数的排列叫\DefineConcept{偶排列},逆序数为奇数的排列叫\DefineConcept{奇排列}.
\end{definition}

计算一个排列的逆序数时,排列中的逆序不能重复计算,也不能漏掉.
可按公式\begin{equation*}
	\tau(\AutoTuple{k}{n})=m_1+m_2+\dotsb+m_n
\end{equation*}计算,其中\(m_i\)为排列中排在\(k_i\)后面比它小的数的个数.
\begin{example}
在4阶排列2341中,2与3形成的数对\((2,3)\),小的数在前,大的数在后,这一对数构成一个顺序;
而2与1形成的数对\((2,1)\),大的数在前,小的数在后,这一对数构成一个逆序.

\begin{figure}[htb]
	\centering
	\begin{tikzpicture}
		\foreach \j in {0,...,3} {
			\fill[ballblue](\j cm+1pt,1pt)rectangle(\j cm+1cm-1pt,1cm-1pt);
		}
		\foreach \i in {0,...,3} {
			\fill[orangepeel](-1pt,-\i cm-1pt)rectangle(-1cm+1pt,-\i cm-1cm+1pt);
		}
		\tiny
		\draw[black](-.5,5pt)node{\(k_i\)}
				(-5pt,.5)node{\(k_j\)};
		\normalsize
		\draw(-.5,-.5)node{2}
			(-.5,-1.5)node{3}
			(-.5,-2.5)node{4}
			(-.5,-3.5)node{1};
		\draw(.5,.5)node{2}
			(1.5,.5)node{3}
			(2.5,.5)node{4}
			(3.5,.5)node{1};
		\draw[applegreen](1.5,-.5)node{(2,3)}
			(2.5,-.5)node{(2,4)}
			(2.5,-1.5)node{(3,4)};
		\draw[tangelo](3.5,-.5)node{(2,1)}
			(3.5,-1.5)node{(3,1)}
			(3.5,-2.5)node{(4,1)};
		\draw[black!30,dashed](0,0)--(4,-4);
	\end{tikzpicture}
	\caption{}
	\label{figure:行列式.4阶排列2341的所有数对}
\end{figure}

如\cref{figure:行列式.4阶排列2341的所有数对},
构成逆序的数对有\((2,1),(3,1),(4,1)\),
构成顺序的数对有\((2,3),(2,4),(3,4)\),
这个4阶排列的逆序数是3,即\(\tau(2341)=3\),
它是一个奇排列.
\end{example}

\begin{property}
\(n\)阶排列共有\(n!\)个.
\end{property}

\begin{definition}
排列\(1,2,\dotsc,n\)由小到大按自然顺序排列,叫做\(n\)阶\DefineConcept{自然排列}.
\end{definition}

\begin{property}
自然排列中没有逆序,即\begin{equation}
	\tau(1,2,\dotsc,n)=0.
\end{equation}
\end{property}

\begin{example}
证明:\begin{equation}
	\tau(n,n-1,\dotsc,1)=\frac{n(n-1)}{2}.
\end{equation}
\begin{proof}
由于对于排列中的每一个数来说,其后的所有数都比它小,所以\begin{equation*}
	\tau(n,n-1,\dotsc,1)
	= (n-1) + (n-2) + \dotsb + 1 + 0
	= \frac{(n-1)n}{2}.
	\qedhere
\end{equation*}
\end{proof}
\end{example}

\begin{definition}
把排列中的两个数的位置互换,其余数字不动,得到另一个排列;
像这样的变换称为\DefineConcept{对换}.
\end{definition}

\begin{theorem}
排列经一次对换后奇偶性改变.
\begin{proof}
我们首先讨论对换的两个数在\(n\)阶排列中相邻的情形:
排列\(k_1,\dotsc,k_i,k_{i+1},\dotsc,k_n\)
对换\(k_i\)与\(k_{i+1}\)这两个数会得到\(k_1,\dotsc,k_{i+1},k_i,\dotsc,k_n\);
除\(k_i,k_{i+1}\)以外的数构成的数对是顺序还是逆序,
在变换前与变换后是一样的;
\(k_i\)和\(k_{i+1}\)以外的数与\(k_i\)或\(k_{i+1}\)构成的数对是顺序还是逆序,
在变换前后也是一样的.
只有数对\((k_i,k_{i+1})\),
如果它在变换前是顺序,那么它在变换后是逆序,
这时变换后排列的逆序数比变换前排列的逆序数多1,即\begin{equation*}
	\tau(k_1,\dotsc,k_{i+1},k_i,\dotsc,k_n)
	= \tau(k_1,\dotsc,k_i,k_{i+1},\dotsc,k_n) + 1;
\end{equation*}
如果它在变换前是逆序,那么它在变换后是顺序,
这时变换后排列的逆序数比变换前排列的逆序数少1,即\begin{equation*}
	\tau(k_1,\dotsc,k_{i+1},k_i,\dotsc,k_n)
	= \tau(k_1,\dotsc,k_i,k_{i+1},\dotsc,k_n) - 1.
\end{equation*}
因此,在对换的两个数在\(n\)阶排列中相邻的情形下,变换前后排列的奇偶性相反.

再讨论一般情形:
排列\(k_1,\dotsc,k_{i-1},k_i,k_{i+1},\dotsc,k_{j-1},k_j,k_{j+1},\dotsc,k_n\)
对换\(k_i,k_j\)这两个数会得到
\(k_1,\dotsc,k_{i-1},k_j,k_{i+1},\dotsc,k_{j-1},k_i,k_{j+1},\dotsc,k_n\);
由于这次对换可以视作一系列相邻两数的对换,
即“对换\(k_i\)与\(k_{i+1}\)”“对换\(k_{i+1}\)与\(k_{i+2}\)”%
......%
“对换\(k_{j-2}\)与\(k_{j-1}\)”“对换\(k_{j-1}\)与\(k_j\)”,
而这就是作了\(s+1+s=2s+1\)次相邻两数的对换,
像这样奇数次相邻两数的对换回改变排列的奇偶性,也就是说,变换前后排列的奇偶性相反.
\end{proof}
\end{theorem}

\begin{corollary}
排列经奇数次对换后奇偶性改变,经偶数次对换后奇偶性不变.
\end{corollary}
我们可以把这个推论表述为如下形式:
设对换次数为\(s\),变换前后的排列分别为\begin{equation*}
	\AutoTuple{\mu}{n}
	\quad\text{和}\quad
	\AutoTuple{\nu}{n},
\end{equation*}
则\begin{equation*}
	(-1)^{\tau(\AutoTuple{\nu}{n})} = (-1)^s (-1)^{\tau(\AutoTuple{\mu}{n})}.
\end{equation*}

\begin{theorem}\label{theorem:行列式.任意排列可化为自然序}
任意一个\(n\)阶排列\(\AutoTuple{k}{n}\)都可经一系列对换变成自然顺序排列,
且对换的次数\(s\)与\(\tau(\AutoTuple{k}{n})\)同奇偶,即\begin{equation*}
	(-1)^s = (-1)^{\tau(\AutoTuple{k}{n})}.
\end{equation*}
\begin{proof}
设\(n\)阶排列\(\AutoTuple{k}{n}\)经过\(s\)次对换变成\(1,2,\dotsc,n\).
考虑到\(1,2,\dotsc,n\)是偶排列,
因此,如果\(\AutoTuple{k}{n}\)是奇排列,则\(s\)必为奇数,才能把奇排列变成偶排列;
如果\(\AutoTuple{k}{n}\)是偶排列,则\(s\)必为偶数,才能保持排列的奇偶性不变.
显然,如果\(n\)阶排列\(\AutoTuple{k}{n}\)经过\(s\)次对换变成自然排列\(1,2,\dotsc,n\),
那么\(1,2,\dotsc,n\)经过上述\(n\)次对换(次序相反)就变成排列\(\AutoTuple{k}{n}\).
\end{proof}
\end{theorem}

\begin{example}
证明:在全部\(n\)阶排列中,奇偶排列各占一半.
%TODO
\end{example}

\begin{example}
试证:\(\tau(\AutoTuple{i}{n})+\tau(i_n,i_{n-1},\dotsc,i_1)=\frac{n(n-1)}{2}\).
\begin{proof}
记\(I=\Set{i_1,i_2,\dotsc,i_{n-1},i_n}\).
对于\(\forall p,q \in I\),
根据排列的定义必有\(p \neq q\),即有\(p<q\)或\(p>q\)成立.
因此,对于数对\((i_k,i_{k+1})\)和\((i_{k+1},i_k)\),
有且仅有以下两种情况:\begin{itemize}
	\item \(i_k,i_{k+1}\)构成一个顺序,\(i_{k+1},i_k\)构成一个逆序;
	\item \(i_k,i_{k+1}\)构成一个逆序,\(i_{k+1},i_k\)构成一个顺序.
\end{itemize}
不论是哪种情况,都有\(\tau(i_k,i_{k+1})+\tau(i_{k+1},i_k)=1\).

由上可知,排列\((i_1,i_2,\dotsc,i_{n-1},i_n)\)
与\((i_n,i_{n-1},\dotsc,i_2,i_1)\)的逆序数之和
相当于是从集合\(I\)中任取两个数构成一个逆序的取法,
那么\begin{equation*}
	\tau(i_1,i_2,\dotsc,i_{n-1},i_n)+\tau(i_n,i_{n-1},\dotsc,i_2,i_1)
	= C_n^2
	= \frac{n(n-1)}{2}.
	\qedhere
\end{equation*}
\end{proof}
\end{example}
