\section{行列式}
\subsection{行列式的概念}
\begin{definition}
设\begin{equation*}
	\vb{A} = \begin{bmatrix}
		a_{11} & a_{12} & \dots & a_{1n} \\
		a_{21} & a_{22} & \dots & a_{2n} \\
		\vdots & \vdots & & \vdots \\
		a_{n1} & a_{n2} & \dots & a_{nn}
	\end{bmatrix}
\end{equation*}是数域\(K\)上的一个\(n\)阶方阵.
从矩阵\(\vb{A}\)中取出不同行又不同列的\(n\)个元素作乘积
\begin{equation}\label{equation:行列式.行列式的项1}
	(-1)^{\tau(\AutoTuple{j}{n})}
	a_{1 j_1} a_{2 j_2} \dotsm a_{n j_n},
\end{equation}
构成一项;%
我们可以像这样构造\(n!\)项,
并且称这\(n!\)项之和\begin{equation*}
	\sum_{\AutoTuple{j}{n}}
	(-1)^{\tau(\AutoTuple{j}{n})}
	a_{1 j_1} a_{2 j_2} \dotsm a_{n j_n}
\end{equation*}为“矩阵\(\vb{A}\)的\DefineConcept{行列式}(determinant)”,
%@see: https://mathworld.wolfram.com/Determinant.html
记作\begin{equation*}
	\begin{vmatrix}
		a_{11} & a_{12} & \dots & a_{1n} \\
		a_{21} & a_{22} & \dots & a_{2n} \\
		\vdots & \vdots & & \vdots \\
		a_{n1} & a_{n2} & \dots & a_{nn}
	\end{vmatrix},
\end{equation*}或\(\det\vb{A}\),或\(\abs{\vb{A}}\);
即
\begin{equation}\label{equation:行列式.行列式的定义式}
	\begin{vmatrix}
		a_{11} & a_{12} & \dots & a_{1n} \\
		a_{21} & a_{22} & \dots & a_{2n} \\
		\vdots & \vdots & & \vdots \\
		a_{n1} & a_{n2} & \dots & a_{nn}
	\end{vmatrix}
	\defeq
	\sum_{\AutoTuple{j}{n}}
	(-1)^{\tau(\AutoTuple{j}{n})}
	a_{1 j_1} a_{2 j_2} \dotsm a_{n j_n}.
\end{equation}
这里,求和指标\(\AutoTuple{j}{n}\)表示遍取所有\(n\)阶排列.

我们称\cref{equation:行列式.行列式的定义式}
为“行列式\(\abs{A}\)的\DefineConcept{完全展开式}”.
\end{definition}

特别地,
一阶行列式为
\begin{equation}
	\begin{vmatrix} a \end{vmatrix} = a.
\end{equation}

二阶行列式为
\begin{equation}
	\begin{vmatrix}
		a_{11} & a_{12} \\
		a_{21} & a_{22}
	\end{vmatrix}
	= a_{11} a_{22} - a_{12} a_{21}.
\end{equation}

三阶行列式为
\begin{equation}
	\begin{vmatrix}
		a_{11} & a_{12} & a_{13} \\
		a_{21} & a_{22} & a_{23} \\
		a_{31} & a_{32} & a_{33}
	\end{vmatrix}
	= \begin{array}[t]{l}
		(a_{11} a_{22} a_{33} + a_{12} a_{23} a_{31} + a_{13} a_{21} a_{32} \\
		\hspace{20pt}
		- a_{13} a_{22} a_{31} - a_{12} a_{21} a_{33} - a_{11} a_{23} a_{32})
	\end{array}.
\end{equation}

我们还可以用数学归纳法证明以下两条公式:
\begin{gather}
	\begin{vmatrix}
		a_{11} & a_{12} & \dots & a_{1n} \\
		& a_{22} & \dots & a_{2n} \\
		& & \ddots & \vdots \\
		& & & a_{nn}
	\end{vmatrix}
	= a_{11} a_{22} \dotsm a_{nn}, \\%
	\begin{vmatrix}
		& & & & a_{1n} \\
		& & & a_{2,n-1} & a_{2n} \\
		& & & \vdots & \vdots \\
		& a_{n-1,2} & \dots & a_{n-1,n-1} & a_{n-1,n} \\
		a_{n1} & a_{n2} & \dots & a_{n,n-1} & a_{nn}
	\end{vmatrix}
	=(-1)^{\frac{1}{2}n(n-1)} a_{1n} a_{2,n-1} \dotsm a_{n-1,2} a_{n1}.
\end{gather}

\begin{lemma}
设\(\vb{A}=(a_{ij})_n\),而\(\AutoTuple{i}{n}\)与\(\AutoTuple{j}{n}\)是两个\(n\)阶排列,则
\begin{equation}\label{equation:行列式.行列式的项2}
	(-1)^{\tau(\AutoTuple{i}{n})+\tau(\AutoTuple{j}{n})}
	a_{i_1j_1} a_{i_2j_2} \dotsm a_{i_nj_n}
\end{equation}
是\(\abs{\vb{A}}\)的项.
\begin{proof}
由乘法交换律,\cref{equation:行列式.行列式的项2} 可以经过\(s\)次互换两个因子的次序写成\begin{equation*}
(-1)^{\tau(\AutoTuple{i}{n})+\tau(\AutoTuple{j}{n})}
	a_{1 l_1} a_{2 l_2} \dotsm a_{n l_n},
\end{equation*}其中,\(\AutoTuple{l}{n}\)是一个\(n\)阶排列.

同时,行标排列\(\AutoTuple{i}{n}\)与列标排列\(\AutoTuple{j}{n}\)
分别经过\(s\)次对换变到\(1,2,\dotsc,n\)与\(\AutoTuple{l}{n}\),
而它们的奇偶性都分别改变了\(s\)次,总共改变了\(2s\)次(偶数次),故\begin{equation*}
	(-1)^{\tau(\AutoTuple{i}{n})+\tau(\AutoTuple{j}{n})}
	= (-1)^{\tau(1,2,\dotsc,n)+\tau(\AutoTuple{l}{n})}
	= (-1)^{\tau(\AutoTuple{l}{n})},
\end{equation*}这说明\cref{equation:行列式.行列式的项2} 是行列式\(\abs{\vb{A}}\)的项.
\end{proof}
\end{lemma}

\begin{corollary}
给定行指标的一个排列\(\AutoTuple{i}{n}\),则\(n\)阶矩阵\(\vb{A}\)的行列式为
\begin{equation}\label{equation:行列式.给定行指标排列下的行列式的完全展开式}
\abs{\vb{A}}
= \sum_{\AutoTuple{j}{n}}
(-1)^{\tau(\AutoTuple{i}{n})+\tau(\AutoTuple{j}{n})}
a_{i_1 j_1} a_{i_2 j_2} \dotsm a_{i_n j_n};
\end{equation}
或者给定列指标的一个排列\(\AutoTuple{j}{n}\),则\(n\)阶矩阵\(\vb{A}\)的行列式为
\begin{equation}\label{equation:行列式.给定列指标排列下的行列式的完全展开式}
	\abs{\vb{A}}
	= \sum_{\AutoTuple{i}{n}}
	(-1)^{\tau(\AutoTuple{i}{n})+\tau(\AutoTuple{j}{n})}
	a_{i_1 j_1} a_{i_2 j_2} \dotsm a_{i_n j_n}.
\end{equation}

特别地,\(n\)阶行列式\(\abs{\vb{A}}\)的每一项可以按列指标成自然序排好位置,
这时用行指标所成排列的奇偶性来决定该项前面所带的符号,即
\begin{equation}\label{equation:行列式.给定列指标为自然序下行列式的完全展开式}
	\abs{\vb{A}} =
	\sum_{\AutoTuple{i}{n}}
	(-1)^{\tau(\AutoTuple{i}{n})}
	a_{i_1 1} a_{i_2 2} \dotsm a_{i_n n}.
\end{equation}
\end{corollary}

\begin{example}
若\(n\)阶行列式\(\det\vb{A}\)中为零的元多于\(n^2-n\)个,证明:\(\det\vb{A}=0\).
%TODO
\end{example}

\begin{example}
证明:如果\(n\ (n\geq2)\)阶矩阵\(\vb{A}\)的元素为\(1\)或\(-1\),则\(\abs{\vb{A}}\)必为偶数.
%TODO
\end{example}

\subsection{行列式的性质}
\begin{property}\label{theorem:行列式.性质1}
设\(\vb{A} \in M_n(K)\),则\(\det\vb{A} = \det\vb{A}^T\).
\begin{proof}
由\cref{equation:行列式.行列式的定义式,equation:行列式.给定列指标为自然序下行列式的完全展开式}
立即可得.
\end{proof}
\end{property}
这就说明,行列互换,行列式的值不变.

\begin{property}\label{theorem:行列式.性质2}
设\(\AutoTuple{\vb\alpha}{n} \in K^n\),\(k \in K\).
那么\begin{equation*}
	\det(\vb\alpha_1,\dotsc,k\vb\alpha_j,\dotsc,\vb\alpha_n)
	= k \cdot \det(\vb\alpha_1,\dotsc,\vb\alpha_j,\dotsc,\vb\alpha_n).
\end{equation*}
\end{property}
这就说明,行列式某一列(或某一行)各元素的公因子可以提到行列式外.

\begin{corollary}\label{theorem:行列式.性质2.推论1}
设\(\AutoTuple{\vb\alpha}{n} \in K^n\),
\(\vb0\)是\(K^n\)的零向量.
那么\begin{equation*}
	\det(\vb\alpha_1,\dotsc,\vb0,\dotsc,\vb\alpha_n) = 0.
\end{equation*}
\end{corollary}
也就是说,如果行列式中某一列(或某一行)元素全为零,则行列式等于零.

\begin{corollary}\label{theorem:行列式.性质2.推论2}
设\(k \in K\),\(\vb{A} \in M_n(K)\).
那么\(\det(k\vb{A}) = k^n \det \vb{A}\).
\end{corollary}

应该注意到,一般说来,\(\det(k\vb{A}) \neq k \det\vb{A}\).

\begin{property}\label{theorem:行列式.性质3}
%@see: 《高等代数(第三版 上册)》(丘维声) P28 性质3
设\(\AutoTuple{\vb\alpha}{n} \in K^n\),且\(\vb\beta,\vb\gamma \in K^n\).
那么\begin{equation*}
	\det(\vb\alpha_1,\dotsc,\vb\beta + \vb\gamma,\dotsc,\vb\alpha_n)
	= \det(\vb\alpha_1,\dotsc,\vb\beta,\dotsc,\vb\alpha_n)
	+ \det(\vb\alpha_1,\dotsc,\vb\gamma,\dotsc,\vb\alpha_n).
\end{equation*}
\begin{proof}
直接计算得
\begin{align*}
	\det(\vb\alpha_1,\dotsc,\vb\beta + \vb\gamma,\dotsc,\vb\alpha_n)
	&= \sum_{\AutoTuple{i}{n}}{
		(-1)^{\tau(\AutoTuple{i}{n})}
		a_{i_1 1} \dotsm (b_{i_j} + c_{i_j}) \dotsm a_{i_n n}
	} \\
	&= \sum_{\AutoTuple{i}{n}}{
		(-1)^{\tau(\AutoTuple{i}{n})}
		a_{i_1 1} \dotsm b_{i_j} \dotsm a_{i_n n}
	} \\
	&\hspace{20pt}+ \sum_{\AutoTuple{i}{n}}{
		(-1)^{\tau(\AutoTuple{i}{n})}
		a_{i_1 1} \dotsm c_{i_j} \dotsm a_{i_n n}
	} \\
	&= \det(\vb\alpha_1,\dotsc,\vb\beta,\dotsc,\vb\alpha_n)
		+ \det(\vb\alpha_1,\dotsc,\vb\gamma,\dotsc,\vb\alpha_n).
	\qedhere
\end{align*}
\end{proof}
\end{property}
注:一般地,\(\det(\vb\alpha_1+\vb\beta_1,\vb\alpha_2+\vb\beta_2,\dotsc,\vb\alpha_n+\vb\beta_n)\)可以拆成\(2^n\)个行列式之和.

\begin{property}\label{theorem:行列式.性质4}
设\(\AutoTuple{\vb\alpha}{n} \in K^n\).
那么\begin{equation*}
	\det(\vb\alpha_1,\dotsc,\vb\alpha_s,\dotsc,\vb\alpha_t,\dotsc,\vb\alpha_n)
	= -\det(\vb\alpha_1,\dotsc,\vb\alpha_t,\dotsc,\vb\alpha_s,\dotsc,\vb\alpha_n).
\end{equation*}
\end{property}
也就是说,交换两列(行),行列式变号.

\begin{property}\label{theorem:行列式.性质5}
设\(\AutoTuple{\vb\alpha}{n} \in K^n\),且\(\vb\alpha \in K^n\),\(k,l \in K\).
那么\begin{equation*}
	\det(\vb\alpha_1,\dotsc,k\vb\alpha,\dotsc,l\vb\alpha,\dotsc,\vb\alpha_n) = 0.
\end{equation*}
\end{property}
这就说明,行列式中若有两列(行)成比例,则行列式等于零.

\begin{property}\label{theorem:行列式.性质6}
设\(\AutoTuple{\vb\alpha}{n} \in K^n\),且\(k \in K\).
那么\begin{equation*}
	\det(\vb\alpha_1,\dotsc,\vb\alpha_s,\dotsc,\vb\alpha_t,\dotsc,\vb\alpha_n)
	= \det(\vb\alpha_1,\dotsc,\vb\alpha_s,\dotsc,\vb\alpha_t + k\vb\alpha_s,\dotsc,\vb\alpha_n).
\end{equation*}
\end{property}
这说明,将一列的\(k\)倍加到另一列,行列式的值不变.

\begin{example}
设\(\vb{A}\)为奇数阶反对称矩阵,即\(\vb{A}^T = -\vb{A}\),则\(\det\vb{A}=0\).
\begin{proof}
假设\(\vb{A} \in M_n(K)\),其中\(n\)是奇数.
因为\(\vb{A}^T = -\vb{A}\),根据行列式的性质,有\begin{equation*}
	\det\vb{A}
	= \det\vb{A}^T
	= \det(-\vb{A})
	= (-1)^n \det\vb{A}
	= -\det\vb{A},
\end{equation*}
于是\(\det\vb{A} = 0\).
\end{proof}
\end{example}

\begin{example}
计算\(n\)阶行列式\begin{equation*}
	D_n = \begin{vmatrix}
		k & \lambda & \lambda & \dots & \lambda \\
		\lambda & k & \lambda & \dots & \lambda \\
		\lambda & \lambda & k & \dots & \lambda \\
		\vdots & \vdots & \vdots & & \vdots \\
		\lambda & \lambda & \lambda & \dots & k
	\end{vmatrix},
	\quad k\neq\lambda.
\end{equation*}
\begin{solution}
当\(n>1\)时,有\begin{align*}
	D_n &= \begin{vmatrix}
		k+(n-1)\lambda & \lambda & \lambda & \dots & \lambda \\
		k+(n-1)\lambda & k & \lambda & \dots & \lambda \\
		k+(n-1)\lambda & \lambda & k & \dots & \lambda \\
		\vdots & \vdots & \vdots & & \vdots \\
		k+(n-1)\lambda & \lambda & \lambda & \dots & k
	\end{vmatrix} \\
	&= [k+(n-1)\lambda] \begin{vmatrix}
		1 & \lambda & \lambda & \dots & \lambda \\
		1 & k & \lambda & \dots & \lambda \\
		1 & \lambda & k & \dots & \lambda \\
		\vdots & \vdots & \vdots & & \vdots \\
		1 & \lambda & \lambda & \dots & k
	\end{vmatrix} \\
	&= [k+(n-1)\lambda] \begin{vmatrix}
		1 & \lambda & \lambda & \dots & \lambda \\
		0 & k-\lambda & 0 & \dots & 0 \\
		0 & 0 & k-\lambda & \dots & 0 \\
		\vdots & \vdots & \vdots & & \vdots \\
		0 & 0 & 0 & \dots & k-\lambda
	\end{vmatrix} \\
	&= [k+(n-1)\lambda] (k-\lambda)^{n-1}.
	\tag1
\end{align*}

当\(n=1\)时,\(D_1 = k\)符合(1)式.
\end{solution}
\end{example}

\begin{example}\label{example:行列式.两个向量的乘积矩阵的行列式}
设\(\vb\alpha=(\AutoTuple{a}{n})^T,
\vb\beta=(\AutoTuple{b}{n})^T\)是\(n\)维列向量.
求:\(\abs{\vb\alpha\vb\beta^T}\).
\begin{solution}
根据\cref{theorem:行列式.性质2},
有\begin{align*}
	\abs{\vb\alpha\vb\beta^T} = \begin{vmatrix}
		a_1 b_1 & a_1 b_2 & \dots & a_1 b_n \\
		a_2 b_1 & a_2 b_2 & \dots & a_2 b_n \\
		\vdots & \vdots & & \vdots \\
		a_n b_1 & a_n b_2 & \dots & a_n b_n
	\end{vmatrix}
	= a_1 a_2 \dotsm a_n \cdot \begin{vmatrix}
		b_1 & b_2 & \dots & b_n \\
		b_1 & b_2 & \dots & b_n \\
		\vdots & \vdots & & \vdots \\
		b_1 & b_2 & \dots & b_n
	\end{vmatrix}.
\end{align*}
而\begin{equation*}
\begin{vmatrix}
	b_1 & b_2 & \dots & b_n \\
	b_1 & b_2 & \dots & b_n \\
	\vdots & \vdots & & \vdots \\
	b_1 & b_2 & \dots & b_n
\end{vmatrix}
\end{equation*}各行成比例,
根据\cref{theorem:行列式.性质5},那么该行列式等于0,可知\(\abs{\vb\alpha\vb\beta^T} = 0\).
\end{solution}
%\cref{theorem:矩阵乘积的秩.多行少列矩阵与少行多列矩阵的乘积的行列式}
\end{example}
