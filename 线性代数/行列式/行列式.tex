\section{行列式}
\subsection{行列式的概念}
\begin{definition}
设\[
	\A = \begin{bmatrix}
		a_{11} & a_{12} & \dots & a_{1n} \\
		a_{21} & a_{22} & \dots & a_{2n} \\
		\vdots & \vdots & & \vdots \\
		a_{n1} & a_{n2} & \dots & a_{nn}
	\end{bmatrix}
\]是数域\(K\)上的一个\(n\)阶方阵.
从矩阵\(\A\)中取出不同行又不同列的\(n\)个元素作乘积
\begin{equation}\label{equation:行列式.行列式的项1}
	(-1)^{\tau(\AutoTuple{j}{n})}
	a_{1 j_1} a_{2 j_2} \dotsm a_{n j_n},
\end{equation}
构成一项;%
我们可以像这样构造\(n!\)项,
并且称这\(n!\)项之和\[
	\sum_{\AutoTuple{j}{n}}
	(-1)^{\tau(\AutoTuple{j}{n})}
	a_{1 j_1} a_{2 j_2} \dotsm a_{n j_n}
\]为“矩阵\(\A\)的\DefineConcept{行列式}(determinant)”,
%@see: https://mathworld.wolfram.com/Determinant.html
记作\[
	\begin{vmatrix}
		a_{11} & a_{12} & \dots & a_{1n} \\
		a_{21} & a_{22} & \dots & a_{2n} \\
		\vdots & \vdots & & \vdots \\
		a_{n1} & a_{n2} & \dots & a_{nn}
	\end{vmatrix},
\]或\(\det\A\),或\(\abs{\A}\);
即
\begin{equation}\label{equation:行列式.行列式的定义式}
	\begin{vmatrix}
		a_{11} & a_{12} & \dots & a_{1n} \\
		a_{21} & a_{22} & \dots & a_{2n} \\
		\vdots & \vdots & & \vdots \\
		a_{n1} & a_{n2} & \dots & a_{nn}
	\end{vmatrix}
	\defeq
	\sum_{\AutoTuple{j}{n}}
	(-1)^{\tau(\AutoTuple{j}{n})}
	a_{1 j_1} a_{2 j_2} \dotsm a_{n j_n}.
\end{equation}
这里,求和指标\(\AutoTuple{j}{n}\)表示遍取所有\(n\)阶排列.

我们称\cref{equation:行列式.行列式的定义式}
为“行列式\(\abs{A}\)的\DefineConcept{完全展开式}”.
\end{definition}

特别地,
一阶行列式为
\begin{equation}
	\begin{vmatrix} a \end{vmatrix} = a.
\end{equation}

二阶行列式为
\begin{equation}
	\begin{vmatrix}
		a_{11} & a_{12} \\
		a_{21} & a_{22}
	\end{vmatrix}
	= a_{11} a_{22} - a_{12} a_{21}.
\end{equation}

三阶行列式为
\begin{equation}
	\begin{vmatrix}
		a_{11} & a_{12} & a_{13} \\
		a_{21} & a_{22} & a_{23} \\
		a_{31} & a_{32} & a_{33}
	\end{vmatrix}
	= \begin{array}[t]{l}
		(a_{11} a_{22} a_{33} + a_{12} a_{23} a_{31} + a_{13} a_{21} a_{32} \\
		\hspace{20pt}
		- a_{13} a_{22} a_{31} - a_{12} a_{21} a_{33} - a_{11} a_{23} a_{32})
	\end{array}.
\end{equation}

我们还可以用数学归纳法证明以下两条公式:
\begin{gather}
	\begin{vmatrix}
		a_{11} & a_{12} & \dots & a_{1n} \\
		& a_{22} & \dots & a_{2n} \\
		& & \ddots & \vdots \\
		& & & a_{nn}
	\end{vmatrix}
	= a_{11} a_{22} \dotsm a_{nn}, \\%
	\begin{vmatrix}
		& & & & a_{1n} \\
		& & & a_{2,n-1} & a_{2n} \\
		& & & \vdots & \vdots \\
		& a_{n-1,2} & \dots & a_{n-1,n-1} & a_{n-1,n} \\
		a_{n1} & a_{n2} & \dots & a_{n,n-1} & a_{nn}
	\end{vmatrix}
	=(-1)^{\frac{1}{2}n(n-1)} a_{1n} a_{2,n-1} \dotsm a_{n-1,2} a_{n1}.
\end{gather}

\begin{lemma}
设\(\A=(a_{ij})_n\),而\(\AutoTuple{i}{n}\)与\(\AutoTuple{j}{n}\)是两个\(n\)阶排列,则
\begin{equation}\label{equation:行列式.行列式的项2}
	(-1)^{\tau(\AutoTuple{i}{n})+\tau(\AutoTuple{j}{n})}
	a_{i_1j_1} a_{i_2j_2} \dotsm a_{i_nj_n}
\end{equation}
是\(\abs{\A}\)的项.
\begin{proof}
由乘法交换律,\cref{equation:行列式.行列式的项2} 可以经过\(s\)次互换两个因子的次序写成\[
(-1)^{\tau(\AutoTuple{i}{n})+\tau(\AutoTuple{j}{n})}
	a_{1 l_1} a_{2 l_2} \dotsm a_{n l_n},
\]其中,\(\AutoTuple{l}{n}\)是一个\(n\)阶排列.

同时,行标排列\(\AutoTuple{i}{n}\)与列标排列\(\AutoTuple{j}{n}\)
分别经过\(s\)次对换变到\(1,2,\dotsc,n\)与\(\AutoTuple{l}{n}\),
而它们的奇偶性都分别改变了\(s\)次,总共改变了\(2s\)次(偶数次),故\[
	(-1)^{\tau(\AutoTuple{i}{n})+\tau(\AutoTuple{j}{n})}
	= (-1)^{\tau(1,2,\dotsc,n)+\tau(\AutoTuple{l}{n})}
	= (-1)^{\tau(\AutoTuple{l}{n})},
\]这说明\cref{equation:行列式.行列式的项2} 是行列式\(\abs{\A}\)的项.
\end{proof}
\end{lemma}

\begin{corollary}
给定行指标的一个排列\(\AutoTuple{i}{n}\),则\(n\)阶矩阵\(\A\)的行列式为
\begin{equation}\label{equation:行列式.给定行指标排列下的行列式的完全展开式}
\abs{\A}
= \sum_{\AutoTuple{j}{n}}
(-1)^{\tau(\AutoTuple{i}{n})+\tau(\AutoTuple{j}{n})}
a_{i_1 j_1} a_{i_2 j_2} \dotsm a_{i_n j_n};
\end{equation}
或者给定列指标的一个排列\(\AutoTuple{j}{n}\),则\(n\)阶矩阵\(\A\)的行列式为
\begin{equation}\label{equation:行列式.给定列指标排列下的行列式的完全展开式}
	\abs{\A}
	= \sum_{\AutoTuple{i}{n}}
	(-1)^{\tau(\AutoTuple{i}{n})+\tau(\AutoTuple{j}{n})}
	a_{i_1 j_1} a_{i_2 j_2} \dotsm a_{i_n j_n}.
\end{equation}

特别地,\(n\)阶行列式\(\abs{\A}\)的每一项可以按列指标成自然序排好位置,
这时用行指标所成排列的奇偶性来决定该项前面所带的符号,即
\begin{equation}\label{equation:行列式.给定列指标为自然序下行列式的完全展开式}
	\abs{\A} =
	\sum_{\AutoTuple{i}{n}}
	(-1)^{\tau(\AutoTuple{i}{n})}
	a_{i_1 1} a_{i_2 2} \dotsm a_{i_n n}.
\end{equation}
\end{corollary}

\begin{example}
若\(n\)阶行列式\(\det\A\)中为零的元多于\(n^2-n\)个,证明:\(\det\A=0\).
%TODO
\end{example}

\begin{example}
证明:如果\(n\ (n\geq2)\)阶矩阵\(\A\)的元素为\(1\)或\(-1\),则\(\abs{\A}\)必为偶数.
%TODO
\end{example}

\subsection{行列式的性质}
\begin{property}\label{theorem:行列式.性质1}
设\(\A \in M_n(K)\),则\(\det\A = \det\A^T\).
\begin{proof}
由\cref{equation:行列式.行列式的定义式,equation:行列式.给定列指标为自然序下行列式的完全展开式}
立即可得.
\end{proof}
\end{property}
这就说明,行列互换,行列式的值不变.

\begin{property}\label{theorem:行列式.性质2}
设\(\AutoTuple{\a}{n} \in K^n\),\(k \in K\).
那么\[
	\det(\a_1,\dotsc,k\a_j,\dotsc,\a_n)
	= k \cdot \det(\a_1,\dotsc,\a_j,\dotsc,\a_n).
\]
\end{property}
这就说明,行列式某一列(或某一行)各元素的公因子可以提到行列式外.

\begin{corollary}\label{theorem:行列式.性质2.推论1}
设\(\AutoTuple{\a}{n} \in K^n\),
\(\z\)是\(K^n\)的零向量.
那么\[
	\det(\a_1,\dotsc,\z,\dotsc,\a_n) = 0.
\]
\end{corollary}
也就是说,如果行列式中某一列(或某一行)元素全为零,则行列式等于零.

\begin{corollary}\label{theorem:行列式.性质2.推论2}
设\(k \in K\),\(\A \in M_n(K)\).
那么\(\det(k\A) = k^n \det \A\).
\end{corollary}

应该注意到,一般说来,\(\det(k\A) \neq k \det\A\).

\begin{property}\label{theorem:行列式.性质3}
%@see: 《高等代数(第三版 上册)》(丘维声) P28 性质3
设\(\AutoTuple{\a}{n} \in K^n\),且\(\b,\g \in K^n\).
那么\[
	\det(\a_1,\dotsc,\b + \g,\dotsc,\a_n)
	= \det(\a_1,\dotsc,\b,\dotsc,\a_n)
	+ \det(\a_1,\dotsc,\g,\dotsc,\a_n).
\]
\begin{proof}
直接计算得
\begin{align*}
	\det(\a_1,\dotsc,\b + \g,\dotsc,\a_n)
	&= \sum_{\AutoTuple{i}{n}}{
		(-1)^{\tau(\AutoTuple{i}{n})}
		a_{i_1 1} \dotsm (b_{i_j} + c_{i_j}) \dotsm a_{i_n n}
	} \\
	&= \sum_{\AutoTuple{i}{n}}{
		(-1)^{\tau(\AutoTuple{i}{n})}
		a_{i_1 1} \dotsm b_{i_j} \dotsm a_{i_n n}
	} \\
	&\hspace{20pt}+ \sum_{\AutoTuple{i}{n}}{
		(-1)^{\tau(\AutoTuple{i}{n})}
		a_{i_1 1} \dotsm c_{i_j} \dotsm a_{i_n n}
	} \\
	&= \det(\a_1,\dotsc,\b,\dotsc,\a_n)
		+ \det(\a_1,\dotsc,\g,\dotsc,\a_n).
	\qedhere
\end{align*}
\end{proof}
\end{property}
注:一般地,\(\det(\a_1+\b_1,\a_2+\b_2,\dotsc,\a_n+\b_n)\)可以拆成\(2^n\)个行列式之和.

\begin{property}\label{theorem:行列式.性质4}
设\(\AutoTuple{\a}{n} \in K^n\).
那么\[
	\det(\a_1,\dotsc,\a_s,\dotsc,\a_t,\dotsc,\a_n)
	= -\det(\a_1,\dotsc,\a_t,\dotsc,\a_s,\dotsc,\a_n).
\]
\end{property}
也就是说,交换两列(行),行列式变号.

\begin{property}\label{theorem:行列式.性质5}
设\(\AutoTuple{\a}{n} \in K^n\),且\(\a \in K^n\),\(k,l \in K\).
那么\[
	\det(\a_1,\dotsc,k\a,\dotsc,l\a,\dotsc,\a_n) = 0.
\]
\end{property}
这就说明,行列式中若有两列(行)成比例,则行列式等于零.

\begin{property}\label{theorem:行列式.性质6}
设\(\AutoTuple{\a}{n} \in K^n\),且\(k \in K\).
那么\[
	\det(\a_1,\dotsc,\a_s,\dotsc,\a_t,\dotsc,\a_n)
	= \det(\a_1,\dotsc,\a_s,\dotsc,\a_t + k\a_s,\dotsc,\a_n).
\]
\end{property}
这说明,将一列的\(k\)倍加到另一列,行列式的值不变.

\begin{example}
设\(\A\)为奇数阶反对称矩阵,即\(\A^T = -\A\),则\(\det\A=0\).
\begin{proof}
假设\(\A \in M_n(K)\),其中\(n\)是奇数.
因为\(\A^T = -\A\),根据行列式的性质,有\[
	\det\A
	= \det\A^T
	= \det(-\A)
	= (-1)^n \det\A
	= -\det\A,
\]
于是\(\det\A = 0\).
\end{proof}
\end{example}

\begin{example}
计算\(n\)阶行列式\[
	D_n = \begin{vmatrix}
		k & \lambda & \lambda & \dots & \lambda \\
		\lambda & k & \lambda & \dots & \lambda \\
		\lambda & \lambda & k & \dots & \lambda \\
		\vdots & \vdots & \vdots & & \vdots \\
		\lambda & \lambda & \lambda & \dots & k
	\end{vmatrix},
	\quad k\neq\lambda.
\]
\begin{solution}
当\(n>1\)时,有\begin{align*}
	D_n &= \begin{vmatrix}
		k+(n-1)\lambda & \lambda & \lambda & \dots & \lambda \\
		k+(n-1)\lambda & k & \lambda & \dots & \lambda \\
		k+(n-1)\lambda & \lambda & k & \dots & \lambda \\
		\vdots & \vdots & \vdots & & \vdots \\
		k+(n-1)\lambda & \lambda & \lambda & \dots & k
	\end{vmatrix} \\
	&= [k+(n-1)\lambda] \begin{vmatrix}
		1 & \lambda & \lambda & \dots & \lambda \\
		1 & k & \lambda & \dots & \lambda \\
		1 & \lambda & k & \dots & \lambda \\
		\vdots & \vdots & \vdots & & \vdots \\
		1 & \lambda & \lambda & \dots & k
	\end{vmatrix} \\
	&= [k+(n-1)\lambda] \begin{vmatrix}
		1 & \lambda & \lambda & \dots & \lambda \\
		0 & k-\lambda & 0 & \dots & 0 \\
		0 & 0 & k-\lambda & \dots & 0 \\
		\vdots & \vdots & \vdots & & \vdots \\
		0 & 0 & 0 & \dots & k-\lambda
	\end{vmatrix} \\
	&= [k+(n-1)\lambda] (k-\lambda)^{n-1}.
	\tag1
\end{align*}

当\(n=1\)时,\(D_1 = k\)符合(1)式.
\end{solution}
\end{example}

\begin{example}\label{example:行列式.两个向量的乘积矩阵的行列式}
设\(\a=(\AutoTuple{a}{n})^T,\b=(\AutoTuple{b}{n})^T\)是\(n\)维列向量.
求:\(\abs{\a\b^T}\).
\begin{solution}
根据\cref{theorem:行列式.性质2},
有\begin{align*}
	\abs{\a\b^T} = \begin{vmatrix}
		a_1 b_1 & a_1 b_2 & \dots & a_1 b_n \\
		a_2 b_1 & a_2 b_2 & \dots & a_2 b_n \\
		\vdots & \vdots & & \vdots \\
		a_n b_1 & a_n b_2 & \dots & a_n b_n
	\end{vmatrix}
	= a_1 a_2 \dotsm a_n \cdot \begin{vmatrix}
		b_1 & b_2 & \dots & b_n \\
		b_1 & b_2 & \dots & b_n \\
		\vdots & \vdots & & \vdots \\
		b_1 & b_2 & \dots & b_n
	\end{vmatrix}.
\end{align*}
而\[
\begin{vmatrix}
	b_1 & b_2 & \dots & b_n \\
	b_1 & b_2 & \dots & b_n \\
	\vdots & \vdots & & \vdots \\
	b_1 & b_2 & \dots & b_n
\end{vmatrix}
\]各行成比例,
根据\cref{theorem:行列式.性质5},那么该行列式等于0,可知\(\abs{\a\b^T} = 0\).
\end{solution}
%\cref{theorem:矩阵乘积的秩.多行少列矩阵与少行多列矩阵的乘积的行列式}
\end{example}
