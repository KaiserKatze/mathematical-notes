\section{线性空间的结构}
\subsection{线性空间的概念与性质}
\begin{definition}\label{definition:线性空间.线性空间的结构.线性空间的定义}
%@see: 《高等代数(第三版 下册)》(丘维声) P72 定义1
%@see: 《Linear Algebra and Its Applications (Second Edition)》(Peter D. Lax) P1
设\(V\)是一个非空集合,
\(F\)是一个域,
映射\(g\colon V \times V \to V\),
映射\(h\colon F \times V \to V\).
%@see: 《Linear Algebra Done Right (Fourth Edition)》(Sheldon Axler) P12 1.20
%@see: 《Linear Algebra Done Right (Fourth Edition)》(Sheldon Axler) P15 1.28
\begingroup
\let\labelitemi\relax
\begin{itemize}
		\item
		\vspace{-1.2cm}
		\begin{axiom}[加法交换律]\label{definition:线性空间.运算法则1}
		%@see: 《Linear Algebra and Its Applications (Second Edition)》(Peter D. Lax) P2 (2)
			\begin{equation*}
				(\forall\alpha,\beta\in V)
				[
					g(\alpha,\beta)
					= g(\beta,\alpha)
				].
			\end{equation*}
		\end{axiom}
		\begin{axiom}[加法结合律]\label{definition:线性空间.运算法则2}
		%@see: 《Linear Algebra and Its Applications (Second Edition)》(Peter D. Lax) P2 (3)
			\begin{equation*}
				(\forall\alpha,\beta,\gamma\in V)
				[
					g(g(\alpha,\beta),\gamma)
					= g(\alpha,g(\beta,\gamma))
				].
			\end{equation*}
		\end{axiom}
		\begin{axiom}[零元的存在性]\label{definition:线性空间.运算法则3}
		%@see: 《Linear Algebra and Its Applications (Second Edition)》(Peter D. Lax) P2 (4)
			\begin{equation*}
				(\forall \alpha \in V)
				(\exists \beta \in V)
				[
					g(\alpha,\beta)
					= \alpha
				].
			\end{equation*}

			如果\(\beta \in V\)满足\begin{equation*}
				(\forall \alpha \in V)
				[
					g(\alpha,\beta)
					= \alpha
				],
			\end{equation*}
			则称“\(\beta\)是\(V\)的一个\DefineConcept{零元}(additive identity)”,
			记为\(0\).
		\end{axiom}
		\begin{axiom}[负元的存在性]\label{definition:线性空间.运算法则4}
		%@see: 《Linear Algebra and Its Applications (Second Edition)》(Peter D. Lax) P2 (5)
			\begin{equation*}
				(\forall \alpha \in V)
				(\exists \beta \in V)
				[
					g(\alpha,\beta)
					= 0
				].
			\end{equation*}

			如果\(\alpha,\beta \in V\)满足\begin{equation*}
				g(\alpha,\beta)
				= 0,
			\end{equation*}
			则称“\(\beta\)是\(\alpha\)的一个\DefineConcept{负元}(additive inverse)”,
			记作\((-\alpha)\).
		\end{axiom}
		\begin{axiom}[域的单位元]\label{definition:线性空间.运算法则5}
		%@see: 《Linear Algebra and Its Applications (Second Edition)》(Peter D. Lax) P2 (9)
			设\(1\)是\(F\)的单位元,
			则\begin{equation*}
				(\forall \alpha \in V)
				[
					h(1,\alpha)
					= \alpha
				].
			\end{equation*}
		\end{axiom}
		\begin{axiom}[纯量乘法结合律]\label{definition:线性空间.运算法则6}
		%@see: 《Linear Algebra and Its Applications (Second Edition)》(Peter D. Lax) P2 (6)
			\begin{equation*}
				(\forall \alpha \in V)
				(\forall k,l \in F)
				[
					h(k,h(l,\alpha))
					= h(k l,\alpha)
				].
			\end{equation*}
		\end{axiom}
		\begin{axiom}[纯量乘法对纯量加法的分配律]\label{definition:线性空间.运算法则7}
		%@see: 《Linear Algebra and Its Applications (Second Edition)》(Peter D. Lax) P2 (8)
			\begin{equation*}
				(\forall \alpha \in V)
				(\forall k,l \in F)
				[
					h(k+l,\alpha)
					= g(
						h(k,\alpha)
						+ h(l,\alpha)
					)
				].
			\end{equation*}
		\end{axiom}
		\begin{axiom}[纯量乘法对向量加法的分配律]\label{definition:线性空间.运算法则8}
		%@see: 《Linear Algebra and Its Applications (Second Edition)》(Peter D. Lax) P2 (7)
			\begin{equation*}
				(\forall \alpha,\beta \in V)
				(\forall k\in F)
				[
					h(k,g(\alpha,\beta))
					= g(
						h(k,\alpha),
						h(k,\beta)
					)
				].
			\end{equation*}
		\end{axiom}
\end{itemize}
\endgroup
如果映射\(g\)和映射\(h\)满足上述八条公理,
则称“\((V,F,g,h)\)是一个\DefineConcept{线性空间}%
(\((V,F,g,h)\) is a \emph{linear space})”
或“\(V\)对\(f\)、\(g\)成为域\(F\)上的一个线性空间”,
在不致混淆的情况下简称“\(V\)是域\(F\)上的一个\DefineConcept{线性空间}%
(\(V\) is a \emph{linear space} over field \(F\))”,
或者进一步简称“\(V\)是一个\DefineConcept{线性空间}%
(\(V\) is a \emph{linear space})”;
%@see: 《Linear Algebra Done Right (Fourth Edition)》(Sheldon Axler) P12 1.21
把\(V\)中的每一个元素称为一个\DefineConcept{向量}(vector)或一个\DefineConcept{点}(point),
把\(F\)中的每一个元素称为一个\DefineConcept{标量}(scalar),
%@see: 《Linear Algebra Done Right (Fourth Edition)》(Sheldon Axler) P12 1.19
把映射\(g\)称为\DefineConcept{加法}(addition),
同时把任意两个向量\(\alpha,\beta\)在映射\(g\)下的像\(g(\alpha,\beta)\)记为\(\alpha + \beta\);
把映射\(h\)成为\DefineConcept{纯量乘法}(scalar multiplication)
\footnote{
	当域\(F\)是一个数域(例如\(\mathbb{Q},\mathbb{R},\mathbb{C}\))时,
	纯量乘法又称为\DefineConcept{数量乘法}.
},
同时把任意一个标量\(k\)和任意一个向量\(\alpha\)在映射\(h\)下的像\(h(k,\alpha)\)记为\(k \alpha\);
把加法与纯量乘法统称为\DefineConcept{线性运算}(linear operation).
\end{definition}
\begin{remark}
与我们在初等代数中学习的自然数、整数、有理数、实数的加法、乘法不同,
线性空间的加法、纯量乘法是抽象的,
线性空间的加法可以是映射空间\(V^{V \times V}\)中的任意一个映射,
线性空间的纯量乘法可以是映射空间\(V^{F \times V}\)中的任意一个映射.
另外,向量空间的加法、纯量乘法和域\(F\)的加法、乘法毫无关系.
\end{remark}
\begin{remark}
线性空间\(V\)对加法成群.
\end{remark}

\begin{definition}
设\(V\)是域\(F\)上的一个线性空间.
把映射\begin{equation*}
	t\colon V \times V \to V,
	(\alpha,\beta) \mapsto \alpha + (-\beta)
\end{equation*}
称为\DefineConcept{减法},
同时把任意两个向量\(\alpha,\beta\)在映射\(t\)下的像\(t(\alpha,\beta)\)记为\(\alpha - \beta\).
\end{definition}

\begin{definition}
%@see: 《Linear Algebra Done Right (Fourth Edition)》(Sheldon Axler) P13 1.22
把实数域\(\mathbb{R}\)上的每一个线性空间称为一个\DefineConcept{实线性空间}(real vector space).
\end{definition}

\begin{definition}
把复数域\(\mathbb{C}\)上的每一个线性空间称为一个\DefineConcept{复线性空间}(complex vector space).
\end{definition}

实线性空间与复线性空间,是代数结构完全不同的两个线性空间.

\begin{example}
下面列举一些常见的线性空间.
\begin{itemize}
	\item 只含零元\(0 \in V\)的线性空间\(\{0\}\),
	称为\DefineConcept{零空间}.

	%@see: 《高等代数(第三版 下册)》(丘维声) P73 例4
	\item 复数域\(\mathbb{C}\)
	可以看成是实数域\(\mathbb{R}\)上的一个线性空间,
	其加法是复数的加法,
	其数量乘法是实数与复数的乘法.

	%@see: 《高等代数(第三版 下册)》(丘维声) P73 例5
	\item 任一数域\(K\)都可以看成是自身上的线性空间,
	其加法就是数域\(K\)中的加法,
	其数量乘法就是数域\(K\)中的乘法.

	\item 集合\(\mathbb{R}^{n \times 1}\)关于向量的加法、实数与向量的纯量乘法,构成实线性空间.

	\item 集合\(\mathbb{R}^{s \times n}\)关于矩阵的加法、实数与矩阵的纯量乘法,构成实线性空间.

	\item 域\(F\)上全体\(n\)阶对称矩阵
	关于矩阵的加法、数与矩阵的纯量乘法,
	成为域\(F\)上的一个线性空间.

	\item 域\(F\)上全体\(n\)阶上三角矩阵
	关于矩阵的加法、数与矩阵的纯量乘法,
	成为域\(F\)上的一个线性空间.

	%@see: 《高等代数(第三版 下册)》(丘维声) P73 例2
	%@see: 《Linear Algebra Done Right (Fourth Edition)》(Sheldon Axler) P13 1.24
	\item 设\(F\)是一个域,\(X\)是一个非空集合,
	则映射空间\(F^X\)
	对函数的加法\begin{equation*}
		(f+g)(x) \defeq f(x) + g(x),
		\quad f,g \in F^X, x \in X,
	\end{equation*}
	以及实数与函数的数量乘法\begin{equation*}
		(k f)(x) \defeq k f(x),
		\quad f \in F^X, k \in F, x \in X,
	\end{equation*}
	成为\(F\)上的一个线性空间.
	\(F^X\)的零元是零函数\begin{equation*}
		0(x) = 0,
		\quad x \in X.
	\end{equation*}
	%上式等号左边的0表示零函数,等号右边的0表示域\(F\)的零元

	%@see: 《高等代数(第三版 下册)》(丘维声) P73 例3
	\item 数域\(K\)上的一元多项式环\(K[x]\)
	对多项式的加法,以及数与多项式的乘法,
	成为\(K\)上的一个线性空间.

	\item 数域\(K\)上所有次数小于\(n\)的一元多项式组成的集合\(K[x]_n\)
	对多项式的加法,以及数与多项式的乘法,
	成为\(K\)上的一个线性空间.
	% 这个线性空间的标准基是\(1,x,x^2,\dotsc,x^{n-1}\).
	% 这个线性空间的维数等于\(n\).

	\item 数域\(K\)上所有次数不大于\(n\)的一元多项式组成的集合
	对多项式的加法,以及数与多项式的乘法,
	成为\(K\)上的一个线性空间.
	% 这个线性空间的标准基是\(1,x,x^2,\dotsc,x^{n-1},x^n\).
	% 这个线性空间的维数等于\(n+1\).
\end{itemize}
\end{example}

\begin{example}
数域\(K\)上所有次数等于\(n\)的一元多项式组成的集合
不是线性空间,
因为零多项式不属于这个集合,
它对线性运算不封闭.
\end{example}

上述例子表明,线性空间这一数学模型适用性很广.
从现在开始,我们将从线性空间的定义出发,
作逻辑推理,深入揭示线性空间的性质和结构,
它们对于所有的具体的线性空间都成立.

\begin{property}%\label{theorem:线性空间.线性空间的结构.线性空间的性质1}
%@see: 《高等代数(第三版 下册)》(丘维声) P74
%@see: 《Linear Algebra Done Right (Fourth Edition)》(Sheldon Axler) P14 1.26
线性空间的零元是唯一的.
\begin{proof}
设\(V\)是域\(F\)上的一个线性空间.
假设\(\beta_1,\beta_2\)都是\(V\)的零元,
那么由\cref{definition:线性空间.运算法则1,definition:线性空间.运算法则3}
可得\begin{equation*}
	\beta_1
	% \cref{definition:线性空间.运算法则3}
	= \beta_1 + \beta_2
	% \cref{definition:线性空间.运算法则1}
	= \beta_2 + \beta_1
	% \cref{definition:线性空间.运算法则3}
	= \beta_2.
\end{equation*}
因此\(V\)的零元是唯一的.
\end{proof}
\end{property}

\begin{property}\label{theorem:线性空间.线性空间的结构.线性空间的性质2}
%@see: 《高等代数(第三版 下册)》(丘维声) P74
%@see: 《Linear Algebra Done Right (Fourth Edition)》(Sheldon Axler) P15 1.27
线性空间中每一个向量的负元是唯一的.
\begin{proof}
设\(V\)是域\(F\)上的一个线性空间.
假设\(\beta_1,\beta_2\)都是\(\alpha\)的负元,
那么由\cref{definition:线性空间.运算法则2,definition:线性空间.运算法则3,definition:线性空间.运算法则4}
可得\begin{equation*}
	\beta_1
	% \cref{definition:线性空间.运算法则3}
	= \beta_1 + 0
	% \cref{definition:线性空间.运算法则4}
	= \beta_1 + (\alpha + \beta_2)
	% \cref{definition:线性空间.运算法则2}
	= (\beta_1 + \alpha) + \beta_2
	% \cref{definition:线性空间.运算法则4}
	= 0 + \beta_2
	% \cref{definition:线性空间.运算法则3}
	= \beta_2.
\end{equation*}
因此\(V\)中每个元素的负元是唯一的.
\end{proof}
\end{property}

\begin{property}\label{theorem:线性空间.线性空间的结构.线性空间的性质3}
%@see: 《高等代数(第三版 下册)》(丘维声) P74
%@see: 《Linear Algebra Done Right (Fourth Edition)》(Sheldon Axler) P15 1.30
%@see: 《Linear Algebra and Its Applications (Second Edition)》(Peter D. Lax) P2 (10)
零与任一向量数乘得零向量.
\begin{proof}
设\(V\)是域\(F\)上的一个线性空间.
由\cref{definition:线性空间.运算法则5,definition:线性空间.运算法则7} 可得\begin{equation*}
	0\alpha + \alpha
	% \cref{definition:线性空间.运算法则5}
	= 0\alpha + 1\alpha
	% \cref{definition:线性空间.运算法则7}
	= (0+1)\alpha
	% 域的加法
	= 1\alpha
	% \cref{definition:线性空间.运算法则5}
	= \alpha.
\end{equation*}
上式两边同时加上\((-\alpha)\)得\(
	(0\alpha + \alpha) + (-\alpha)
	= \alpha + (-\alpha)
\).
由\cref{definition:线性空间.运算法则4} 可知\(
	\alpha + (-\alpha)
	= 0
\).
于是\begin{equation*}
	(0\alpha + \alpha) + (-\alpha)
	= 0.
	\eqno(1)
\end{equation*}
由\cref{definition:线性空间.运算法则2,definition:线性空间.运算法则3,definition:线性空间.运算法则4} 可得\begin{equation*}
	(0\alpha + \alpha) + (-\alpha)
	% \cref{definition:线性空间.运算法则2}
	= 0\alpha + (\alpha + (-\alpha))
	% \cref{definition:线性空间.运算法则4}
	= 0\alpha + 0
	% \cref{definition:线性空间.运算法则3}
	= 0\alpha.
	\eqno(2)
\end{equation*}
由(1)(2)两式可得\(0\alpha = 0\).
因此\((\forall\alpha\in V)[0\alpha=0]\).
\end{proof}
\end{property}

\begin{property}\label{theorem:线性空间.线性空间的结构.线性空间的性质4}
%@see: 《高等代数(第三版 下册)》(丘维声) P74
%@see: 《Linear Algebra Done Right (Fourth Edition)》(Sheldon Axler) P16 1.31
任一标量与零向量数乘得零向量.
\begin{proof}
设\(V\)是域\(F\)上的一个线性空间.
由\cref{definition:线性空间.运算法则3,definition:线性空间.运算法则8}
可得\begin{equation*}
	k0 + k0
	% \cref{definition:线性空间.运算法则8}
	= k(0+0)
	% \cref{definition:线性空间.运算法则3}
	= k0.
\end{equation*}
上式两边同时加上\((-k0)\),得\(
	(k0 + k0) + (-k0)
	= k0 + (-k0)
\).
由\cref{definition:线性空间.运算法则4} 可知\(
	k0 + (-k0)
	= 0
\).
于是\begin{equation*}
	(k0 + k0) + (-k0)
	= 0.
	\eqno(1)
\end{equation*}
由\cref{definition:线性空间.运算法则2,definition:线性空间.运算法则3,definition:线性空间.运算法则4}
可得\begin{equation*}
	(k0 + k0) + (-k0)
	% \cref{definition:线性空间.运算法则2}
	= k0 + (k0 + (-k0))
	% \cref{definition:线性空间.运算法则4}
	= k0 + 0
	% \cref{definition:线性空间.运算法则3}
	= k0.
	\eqno(2)
\end{equation*}
由(1)(2)两式可得\begin{equation*}
	k0 = 0.
\end{equation*}
因此\((\forall k\in F)[k0=0]\).
\end{proof}
\end{property}

\begin{property}\label{theorem:线性空间.线性空间的结构.线性空间的性质5}
%@see: 《高等代数(第三版 下册)》(丘维声) P74
%@see: 《Linear Algebra Done Right (Fourth Edition)》(Sheldon Axler) P16 Exercise 1B 2
设\(V\)是域\(F\)上的一个线性空间,
标量\(k \in F\),向量\(\alpha \in V\),
则\begin{equation*}
	k\alpha=0
	\implies
	k=0 \lor \alpha=0.
\end{equation*}
\begin{proof}
由\cref{theorem:线性空间.线性空间的结构.线性空间的性质3}
可知\(k = 0 \implies k\alpha = 0\).
假设\(k\neq0\)且\(k\alpha = 0\),
则由\cref{definition:线性空间.运算法则5,definition:线性空间.运算法则6}
以及\cref{theorem:线性空间.线性空间的结构.线性空间的性质4}
可得\begin{equation*}
	\alpha
	% \cref{definition:线性空间.运算法则5}
	= 1\alpha
	% \(k\neq0\),域的定义(域中每一个非零元都是可逆元)
	= (k^{-1} k) \alpha
	% \cref{definition:线性空间.运算法则6}
	= k^{-1} (k \alpha)
	% \(k\alpha = 0\)
	= k^{-1} 0
	% \cref{theorem:线性空间.线性空间的结构.线性空间的性质4}
	= 0.
	\qedhere
\end{equation*}
\end{proof}
\end{property}

\begin{property}%\label{theorem:线性空间.线性空间的结构.线性空间的性质6}
%@see: 《高等代数(第三版 下册)》(丘维声) P74
%@see: 《Linear Algebra Done Right (Fourth Edition)》(Sheldon Axler) P16 1.32
设\(V\)是域\(F\)上的一个线性空间,
则\((\forall\alpha\in V)[(-1)\alpha=-\alpha]\).
\begin{proof}
由\cref{definition:线性空间.运算法则5,definition:线性空间.运算法则7}
以及\cref{theorem:线性空间.线性空间的结构.线性空间的性质3}
可得\begin{equation*}
	\alpha + (-1)\alpha
	% \cref{definition:线性空间.运算法则5}
	= 1\alpha + (-1)\alpha
	% \cref{definition:线性空间.运算法则7}
	= (1+(-1))\alpha
	% 域的加法
	= 0\alpha
	% \cref{theorem:线性空间.线性空间的结构.线性空间的性质3}
	= 0.
\end{equation*}
再由\hyperref[theorem:线性空间.线性空间的结构.线性空间的性质2]{负元的唯一性}可知\((-1)\alpha = -\alpha\).
\end{proof}
\end{property}

\begin{property}
%@see: 《Linear Algebra Done Right (Fourth Edition)》(Sheldon Axler) P16 Exercise 1B 1
设\(V\)是域\(F\)上的一个线性空间,
\(\alpha \in V\),
则\(-(-\alpha) = \alpha\).
\begin{proof}
由\cref{definition:线性空间.运算法则4}
可知\(\alpha\)是\((-\alpha)\)的负元.
再由\hyperref[theorem:线性空间.线性空间的结构.线性空间的性质2]{负元的唯一性}可知\(\alpha = -(\alpha)\).
\end{proof}
\end{property}

\begin{example}
假设把\cref{definition:线性空间.运算法则5} 修改为\begin{equation*}
	(\exists e \in F)
	(\forall \alpha \in V)
	[
		e \alpha = \alpha
	].
\end{equation*}
试讨论:满足\(
	(\forall \alpha \in V)
	[
		e \alpha = \alpha
	]
\)的\(e \in F\)与\(F\)的单位元\(1\)是否相等.
\begin{solution}
%@credit: {gemini},{8b6edada-f2fd-4ae5-9020-eb533149a54c},{855486ab-2fcf-40c1-b774-09956dfb4012}
假设\(e \in F\)满足
对于任意\(\alpha \in V\)成立\(e \alpha = \alpha\),
那么\(1 (e \alpha) = 1 \alpha\),
再由\cref{definition:线性空间.运算法则6} 可知\begin{equation*}
	(1 e) \alpha = 1 (e \alpha) = 1 \alpha.
\end{equation*}
根据域的单位元的定义,有\(1 e = e 1 = e\),
那么对于任意\(\alpha \in V\),成立\begin{equation*}
	(1 e) \alpha = e \alpha = \alpha.
\end{equation*}
于是对于任意\(\alpha \in V\),成立\begin{equation*}
	1 \alpha = \alpha = e \alpha,
	\eqno(1)
\end{equation*}
从而有\((1-e) \alpha = 0\).

\cref{theorem:线性空间.线性空间的结构.线性空间的性质5} 的证明过程
用到了修改之前的\cref{definition:线性空间.运算法则5},
所以我们不能直接由它得出以下结论:\begin{equation*}
	(\forall k \in F)
	(\forall \alpha \in V)
	[
		k\alpha=0 \implies k=0 \lor \alpha=0
	].
\end{equation*}

如果\(V\)只含零向量,
那么由\cref{theorem:线性空间.线性空间的结构.线性空间的性质4} 可知,
不论\((1-e)\)是否为零,
\((1-e)\alpha\)恒等于零.
这就说明,当\(V = \{0\}\)时,
\(e\)不一定是\(F\)的单位元,
事实上,\(e\)可以是\(F\)中的任意一个元素.

如果\(V\)含有非零向量,
那么,为了证明\(e=1\),只需证\begin{equation*}
	\alpha\neq0, k\alpha = 0 \implies k=0.
\end{equation*}
首先假设\(\alpha\neq0, k\alpha = 0\).
接着用反证法,假设\(k\neq0\).
鉴于域\(F\)中每个非零元\(k\)都有逆元\(k^{-1}\),
在\(k\alpha = 0\)两边同时乘以\(k^{-1}\)得\begin{equation*}
	k^{-1}(k\alpha) = k^{-1}0.
	\eqno(2)
\end{equation*}
由\cref{theorem:线性空间.线性空间的结构.线性空间的性质4} 可知\begin{equation*}
	k^{-1}0 = 0.
	\eqno(3)
\end{equation*}
由\cref{definition:线性空间.运算法则6} 可知\begin{equation*}
	k^{-1}(k\alpha)
	= (k^{-1} k)\alpha,
\end{equation*}
因为\(k^{-1} k = 1\),
所以\begin{equation*}
	(k^{-1} k)\alpha
	= 1\alpha,
\end{equation*}
于是\begin{equation*}
	k^{-1}(k\alpha)
	= 1\alpha,
\end{equation*}
再由(1)式可知\(1\alpha = \alpha\),从而有\begin{equation*}
	k^{-1}(k\alpha)
	= \alpha.
	\eqno(4)
\end{equation*}
由(2)(3)(4)可得\begin{equation*}
	\alpha = 0.
\end{equation*}
这与前提条件“\(\alpha\neq0\)”相矛盾,
因此一定有\(k=0\).
于是由\((1-e)\alpha = 0\)得\(1-e=0\),即\(e=1\).

综上所述,
在将\cref{definition:线性空间.运算法则5} 修改为\begin{equation*}
	(\exists e \in F)
	(\forall \alpha \in V)
	[
		e \alpha = \alpha
	]
\end{equation*}
以后,
要么\(e\)就是域\(F\)的单位元(即\(e = 1\)),
要么\(V\)是只含零向量的线性空间(即\(V = \{0\}\)).
\end{solution}
\end{example}

\begin{example}
%@see: 《高等代数(第三版 下册)》(丘维声) P81 习题8.1 1.(2)
在正实数集\(\mathbb{R}^+\)上定义加法、数量乘法:\begin{gather*}
	\oplus \defeq \Set{
		((a,b),ab)
		\given
		a,b \in \mathbb{R}^+
	}, \\
	\odot \defeq \Set{
		((k,a),a^k)
		\given
		a \in \mathbb{R}^+,
		k \in \mathbb{R}
	}.
\end{gather*}
试判断\((\mathbb{R}^+,\oplus,\odot)\)是不是实数域\(\mathbb{R}\)上的线性空间.
\begin{solution}
显然\(\oplus,\odot\)都是映射.
由于实数的乘法运算适合交换律、结合律,
所以\(\oplus\)也适合交换律、结合律.
正实数\(1\)是\(\mathbb{R}^+\)的零元,
这是因为对于任意\(a \in \mathbb{R}^+\)总有\(1 \oplus a = 1a = a\).
任意一个正实数\(a\)的倒数\(1/a\)就是\(a\)的负元.
实数\(1\)还是\(\mathbb{R}^+\)的单位元,
这是因为\(1 \odot a = a^1 = a\).
对于任意正实数\(a,b\)和任意实数\(k,l\),
显然有\begin{gather*}
	k \odot (l \odot a)
	= (a^l)^k
	= a^{k l}
	= (kl) \odot a, \\
	(k+l) \odot a
	= a^{k+l}
	= a^k a^l
	= (k \odot a) \oplus (l \odot a), \\
	k \odot (a \oplus b)
	= (ab)^k
	= a^k b^k
	= (k \odot a) \oplus (k \odot b).
\end{gather*}
综上所述,\((\mathbb{R}^+,\oplus,\odot)\)确实是实数域\(\mathbb{R}\)上的一个线性空间.
\end{solution}
\end{example}

\begin{example}
设数域为\(\mathbb{R}\),
集合为\(V \defeq \Set{ \alpha \given \alpha = (\xi_1,\xi_2), \xi_i \in \mathbb{R} }\).
对于\(\alpha = (\xi_1,\xi_2)\)、\(\beta = (\eta_1,\eta_2)\)及\(k \in \mathbb{R}\),
指定运算如下:\begin{center}
	\begin{tblr}{*2c}
		加法运算: & \(\alpha \oplus \beta \defeq (\xi_1 + \eta_1,\xi_2 + \eta_2 + \xi_1 \eta_1)\) \\
		数乘运算: & \(k \odot \alpha \defeq (k \xi_1,k \xi_2 + \frac12 k (k-1) \xi_1^2)\) \\
	\end{tblr}
\end{center}
判断\(V\)是否构成\(\mathbb{R}\)上的线性空间.
\begin{solution}
因为实数域对加法、乘法封闭,
所以对于任意实数\(\xi_1,\xi_2,\eta_1,\eta_2,k\),
总有\begin{equation*}
	\xi_1 + \eta_1,
	\xi_2 + \eta_2 + \xi_1 \eta_1,
	k \xi_1,
	k \xi_2 + \frac12 k(k-1) \xi_1^2
	\in \mathbb{R},
\end{equation*}
于是对于任意向量\(\alpha = (\xi_1,\xi_2),\beta = (\eta_1,\eta_2) \in V = \mathbb{R}^2\)和任意实数\(k\),
总有\(
	\alpha \oplus \beta,
	k \odot \alpha
	\in V
\),
也就是说\(V\)对加法、数乘运算都封闭.

因为对于任意向量\(\alpha = (\xi_1,\xi_2),\beta = (\eta_1,\eta_2) \in V\)
总有\begin{equation*}
	\beta \oplus \alpha
	= (\eta_1 + \xi_1,\eta_2 + \xi_2 + \eta_1 \xi_1)
	= (\xi_1 + \eta_1,\xi_2 + \eta_2 + \xi_1 \eta_1)
	= \alpha \oplus \beta,
\end{equation*}
所以\(\oplus\)满足交换律.

因为对于任意向量\(\alpha = (\xi_1,\xi_2),\beta = (\eta_1,\eta_2),\gamma = (\zeta_1,\zeta_2) \in V\)
总有\begin{align*}
	&\hspace{-20pt}
	\alpha \oplus (\beta \oplus \gamma) \\
	&= (\xi_1,\xi_2) \oplus (\eta_1 + \zeta_1,\eta_2 + \zeta_2 + \eta_1 \zeta_1) \\
	&= (\xi_1 + \eta_1 + \zeta_1,\xi_2 + \eta_2 + \zeta_2 + \eta_1 \zeta_1 + \xi_1 (\eta_1 + \zeta_1)) \\
	&= (\xi_1 + \eta_1 + \zeta_1,\xi_2 + \eta_2 + \zeta_2 + \xi_1 \eta_1 + (\xi_1 + \eta_1) \zeta_1) \\
	&= (\xi_1 + \eta_1,\xi_2 + \eta_2 + \xi_1 \eta_1) \oplus (\zeta_1,\zeta_2) \\
	&= (\alpha \oplus \beta) \oplus \zeta,
\end{align*}
所以\(\oplus\)满足结合律.

假设对于任意向量\(\alpha = (\xi_1,\xi_2) \in V\),
向量\(\omega = (\eta_1,\eta_2) \in V\)
总满足\(\alpha \oplus \omega = \alpha\),
那么\begin{equation*}
	(\xi_1 + \eta_1,\xi_2 + \eta_2 + \xi_1 \eta_1)
	= (\xi_1,\xi_2),
\end{equation*}
建立关于\(\eta_1,\eta_2\)的方程组\begin{equation*}
	\begin{cases}
		\xi_1 + \eta_1 = \xi_1, \\
		\xi_2 + \eta_2 + \xi_1 \eta_1 = \xi_2
	\end{cases}
\end{equation*}
解得\(\eta_1 = \eta_2 = 0\),
即\(\omega = (0,0)\).
因此\(\omega\)是\(\oplus\)的单位元,亦即是\(V\)的零向量.

假设对于任意向量\(\alpha = (\xi_1,\xi_2) \in V\),
向量\(\beta = (\eta_1,\eta_2) \in V\)
总满足\(\alpha \oplus \beta = \omega\),
那么\begin{equation*}
	(\xi_1 + \eta_1,\xi_2 + \eta_2 + \xi_1 \eta_1)
	= (0,0),
\end{equation*}
建立关于\(\eta_1,\eta_2\)的方程组\begin{equation*}
	\begin{cases}
		\xi_1 + \eta_1 = 0, \\
		\xi_2 + \eta_2 + \xi_1 \eta_1 = 0,
	\end{cases}
\end{equation*}
解得\(\eta_1 = -\xi_1,\eta_2 = \xi_1^2 - \xi_2\),
即\(\beta = (-\xi_1,\xi_1^2 - \xi_2)\).
因此\(V\)中任意一个向量都存在负向量.

\(1\)是实数域的单位元,
对于任意向量\(\alpha = (\xi_1,\xi_2) \in V\)
总有\begin{equation*}
	1 \odot \alpha
	= (1 \xi_1,1 \xi_2 + \frac12 1 (1-1) \xi_1^2)
	= (\xi_1,\xi_2)
	= \alpha.
\end{equation*}

对于任意实数\(k,l\)和任意向量\(\alpha = (\xi_1,\xi_2) \in V\),
总有\begin{align*}
	&\hspace{-20pt}
	k \odot (l \odot \alpha) \\
	&= k \odot (l \xi_1,l \xi_2 + \frac12 l (l-1) \xi_1^2) \\
	&= (k l \xi_1,k(l \xi_2 + \frac12 l (l-1) \xi_1^2) + \frac12 k (k-1) (l \xi_1)^2) \\
	&= (kl \xi_1,kl \xi_2 + \frac12 kl (l-1) \xi_1^2 + \frac12 k(k-1) l^2 \xi_1^2) \\
	&= (kl \xi_1,kl \xi_2 + \frac12 kl (kl-1) \xi_1^2) \\
	&= (k l) \odot \alpha,
\end{align*}
所以\(\odot\)满足结合律.

因为对于任意实数\(k,l\)和任意向量\(\alpha = (\xi_1,\xi_2) \in V\),
总有\begin{align*}
	&\hspace{-20pt}
	(k + l) \odot \alpha \\
	&= ((k+l) \xi_1,(k+l) \xi_2 + \frac12 (k+l) ((k+l)-1) \xi_1^2) \\
	&= (
		k \xi_1 + l \xi_1,
		k \xi_2 + \frac12 k (k-1) \xi_1^2
		+ l \xi_2 + \frac12 l (l-1) \xi_1^2
		+ (k \xi_1) (l \xi_1)
	) \\
	&= (k \xi_1,k \xi_2 + \frac12 k (k-1) \xi_1^2)
		\oplus (l \xi_1,l \xi_2 + \frac12 l (l-1) \xi_1^2) \\
	&= (k \odot \alpha) \oplus (l \odot \alpha),
\end{align*}
所以\(\odot\)满足对\(+\)的分配律.

因为对于任意实数\(k\)和任意向量\(\alpha = (\xi_1,\xi_2),\beta = (\eta_1,\eta_2) \in V\),
总有\begin{align*}
	&\hspace{-20pt}
	k \odot (\alpha \oplus \beta) \\
	&= k \odot (\xi_1 + \eta_1,\xi_2 + \eta_2 + \xi_1 \eta_1) \\
	&= (
		k(\xi_1 + \eta_1),
		k(\xi_2 + \eta_2 + \xi_1 \eta_1)
		+ \frac12 k (k-1) (\xi_1 + \eta_1)^2
	) \\
	&= (
		k \xi_1 + k \eta_1,
		(k \xi_2 + \frac12 k (k-1) \xi_1^2)
		+ (k \eta_2 + \frac12 k (k-1) \eta_1^2)
		+ (k \xi_1) (k \eta_1)
	) \\
	&= (k \xi_1,k \xi_2 + \frac12 k (k-1) \xi_1^2)
		\oplus (k \eta_1,k \eta_2 + \frac12 k (k-1) \eta_1^2) \\
	&= (k \odot \alpha) \oplus (k \odot \beta),
\end{align*}
所以\(\odot\)满足对\(\oplus\)的分配律.

综上所述,\((V,\mathbb{R},\oplus,\odot)\)是一个线性空间.
\end{solution}
\end{example}

\subsection{线性空间的线性关系}
域\(F\)上的线性空间\(V\)的有限子集,称为“\(V\)中的一个\DefineConcept{向量组}”.

向量组\(A\)的子集,称为“\(A\)的一个\DefineConcept{部分组}”.

%@see: 《高等代数(第三版 下册)》(丘维声) P75
%@see: 《Linear Algebra and Its Applications (Second Edition)》(Peter D. Lax) P4 Definition
%@see: 《Linear Algebra Done Right (Fourth Edition)》(Sheldon Axler) P28 2.2
设\(\AutoTuple{\alpha}{s}\)是\(V\)中一个向量组,
任给\(F\)中一组元素\(\AutoTuple{k}{s}\),
向量\(k_1\alpha_1+\dotsb+k_s\alpha_s\)
称为“\(\AutoTuple{\alpha}{s}\)的一个\DefineConcept{线性组合}(linear combination)”,
称\(\AutoTuple{k}{s}\)为\DefineConcept{系数}.

%@see: 《高等代数(第三版 下册)》(丘维声) P75
对于\(\beta\in V\),
如果有\(F\)中一组元素\(\AutoTuple{c}{s}\),
使得\(\beta=c_1\alpha_1+\dotsb+c_s\alpha_s\),
则称“\(\beta\)可以由\(\AutoTuple{\alpha}{s}\)~\DefineConcept{线性表出}%
(\(\beta\) can be expressed as a linear combination of \(\AutoTuple{\alpha}{s}\))”.

\begin{definition}
%@see: 《高等代数(第三版 下册)》(丘维声) P75 定义2
%@see: 《Linear Algebra and Its Applications (Second Edition)》(Peter D. Lax) P4 Definition
%@see: 《Linear Algebra and Its Applications (Second Edition)》(Peter D. Lax) P5 Definition
%@see: 《Linear Algebra Done Right (Fourth Edition)》(Sheldon Axler) P32 2.15
%@see: 《Linear Algebra Done Right (Fourth Edition)》(Sheldon Axler) P33 2.17
设\(\AutoTuple{\alpha}{s}\ (s\geq1)\)是\(V\)中一个向量组.
如果有\(F\)中不全为零的元素\(\AutoTuple{k}{s}\),
使得\(k_1\alpha_1+\dotsb+k_s\alpha_s=0\),
则称“\(\AutoTuple{\alpha}{s}\)是\DefineConcept{线性相关的}%
(\(\AutoTuple{\alpha}{s}\) are \emph{linearly dependent})”;
否则称“\(\AutoTuple{\alpha}{s}\)是\DefineConcept{线性无关的}%
(\(\AutoTuple{\alpha}{s}\) are \emph{linearly independent})”.
\end{definition}

%@see: 《Linear Algebra Done Right (Fourth Edition)》(Sheldon Axler) P32 2.15
空向量组\(\emptyset\)是线性无关的.

\begin{definition}
%@see: 《高等代数(第三版 下册)》(丘维声) P75 定义3
设\(W\)是\(V\)的任一无限子集.
如果\(W\)有一个有限子集是线性相关的,
则称“\(W\)是\DefineConcept{线性相关的}%
(\(W\) is \emph{linearly dependent})”;
如果\(W\)的任何有限子集都是线性无关的,
则称“\(W\)是\DefineConcept{线性无关的}%
(\(W\) is \emph{linearly independent})”.
\end{definition}

可以证明,
数域\(K\)上的线性方程组的理论,
和数域\(K\)上的矩阵、行列式理论,
在把数域\(K\)换成任意域\(F\)以后,
仍然成立.
\begin{property}\label{theorem:线性空间.线性相关性1}
%@see: 《高等代数(第三版 下册)》(丘维声) P75 例6
%@see: 《高等代数(第三版 下册)》(丘维声) P75 例7
%@see: 《高等代数(第三版 下册)》(丘维声) P75 命题1
%@see: 《高等代数(第三版 下册)》(丘维声) P75 命题2
%@see: 《Linear Algebra Done Right (Fourth Edition)》(Sheldon Axler) P33 2.19
设\(V\)是域\(F\)上的一个线性空间.
\begin{itemize}
	\item \(\text{$\alpha$线性相关}\iff\alpha=0\).
	\item 包含零向量的向量组一定线性相关.
	\item 基数大于或等于\(2\)的向量组\(W\)线性相关
	当且仅当\(W\)中至少有一个向量可以由其余向量中的有限多个线性表出.
	\item 向量\(\beta\)可以由线性无关向量组\(\AutoTuple{\alpha}{s}\)线性表出的充分必要条件是
	\(\AutoTuple{\alpha}{s},\beta\)线性相关.
\end{itemize}
\end{property}

\begin{definition}
%@see: 《高等代数(第三版 下册)》(丘维声) P76 定义4
设\(W_1,W_2\)都是\(V\)的非空子集,
如果\(W_1\)中每一个向量都可以由\(W_2\)中有限多个向量线性表出,
则称“\(W_1\)可以由\(W_2\)~\DefineConcept{线性表出}”.
如果\(W_1\)与\(W_2\)可以互相线性表出,
则称“\(W_1\)与\(W_2\)是\DefineConcept{等价的}”.
\end{definition}

容易证明,“线性表出”具有传递性,
从而“等价”也具有传递性.
显然,向量组的“等价”具有反身性与对称性.

\begin{property}\label{theorem:线性空间.线性相关性2}
%@see: 《高等代数(第三版 下册)》(丘维声) P76 引理1
%@see: 《高等代数(第三版 下册)》(丘维声) P76 推论3
%@see: 《高等代数(第三版 下册)》(丘维声) P76 推论4
设\(V\)是域\(F\)上的一个线性空间.
\begin{itemize}
	\item 设向量组\(\AutoTuple{\beta}{r}\)
	可以由向量组\(\AutoTuple{\alpha}{s}\)线性表出,则\begin{gather*}
		r>s
		\implies
		\text{$\AutoTuple{\beta}{r}$线性相关}, \\
		\text{$\AutoTuple{\beta}{r}$线性无关}
		\implies
		r\leq s.
	\end{gather*}

	\item 等价的线性无关的向量组所含向量的个数相等.
\end{itemize}
\end{property}

\subsection{向量组的秩}
\begin{definition}
%@see: 《高等代数(第三版 下册)》(丘维声) P76 定义5
设\(V\)是域\(F\)上的一个线性空间,
\(A\)是\(V\)的一个子集,
\(a\)是\(A\)的有限子集.
如果\(a\)是线性无关的,
但是\begin{equation*}
	(\forall\beta \in A-a)
	[\text{$a \cup \{\beta\}$是线性相关的}],
\end{equation*}
则称“\(a\)是\(A\)的一个\DefineConcept{极大线性无关组}”.
\end{definition}

\begin{property}
%@see: 《高等代数(第三版 下册)》(丘维声) P76 推论5
%@see: 《高等代数(第三版 下册)》(丘维声) P76 推论6
设\(V\)是域\(F\)上的一个线性空间.
\begin{itemize}
	\item 向量组与它的极大线性无关组等价.
	\item 向量组的任意两个极大线性无关组的基数相等.
\end{itemize}
\end{property}

\begin{definition}
%@see: 《高等代数(第三版 下册)》(丘维声) P76 定义6
向量组\(A=\{\AutoTuple{\alpha}{s}\}\)的一个极大线性无关组的基数,
称为“向量组\(A\)的\DefineConcept{秩}(rank)”,
记为\(\rank A\)或\(\rank\{\AutoTuple{\alpha}{s}\}\).
\end{definition}

\begin{property}\label{theorem:线性空间.向量组的秩的性质}
%@see: 《高等代数(第三版 下册)》(丘维声) P76 命题8
%@see: 《高等代数(第三版 下册)》(丘维声) P76 命题9
%@see: 《高等代数(第三版 下册)》(丘维声) P76 推论9
设\(V\)是域\(F\)上的一个线性空间.
\begin{itemize}
	\item 全由零向量组成的向量组的秩为零.

	\item 向量组线性无关的充分必要条件是
	它的秩等于它的基数.

	\item 设\(A,B\)都是向量组.
	如果\(A\)可以由\(B\)线性表出,
	则\(\rank A \leq \rank B\).

	\item 等价的向量组有相同的秩.
\end{itemize}
\end{property}

\subsection{线性空间的基}
\begin{definition}\label{definition:线性空间.线性空间的基}
%@see: 《高等代数(第三版 下册)》(丘维声) P76 定义7
%@see: 《Linear Algebra Done Right (Fourth Edition)》(Sheldon Axler) P39 2.26
设\(V\)是域\(F\)上的一个线性空间,\(S \subseteq V\).
如果\begin{itemize}
	\item \(S\)线性无关,
	\item \(V\)中每一个向量都可以由\(S\)中有限多个向量线性表出,
\end{itemize}
则称“\(S\)是\(V\)的一个\DefineConcept{基}%
(\(S\) is a \emph{basis} for \(V\))”.
\end{definition}
\begin{remark}
%@credit: {647826c9-7e2a-49d1-b176-cd39b299b349} 说:除了 Hamel 基以外,还有 Schauder 基等其他定义
%@credit: {85841724-e8e0-4a39-88bf-973ade1b5e13} 说:参考《代数学(一)》(李方、邓少强、冯荣权、刘东文) P101 定义5.1.2
在\hyperref[definition:线性空间.线性空间的基]{基的定义}中,
必须要注意第二个条件中“有限多个”这个限定,
它说明这里定义的基是\emph{哈莫基}(Hamel basis).
%@see: https://zh.wikipedia.org/wiki/%E5%9F%BA_(%E7%B7%9A%E6%80%A7%E4%BB%A3%E6%95%B8)
%@see: https://en.wikipedia.org/wiki/Basis_(linear_algebra)
\end{remark}

\begin{property}\label{theorem:线性空间的结构.零空间的基是空集}
%@see: 《高等代数(第三版 下册)》(丘维声) P77
零空间的基是空集.
%TODO 无法确定这个究竟是定义还是性质
\end{property}

\begin{property}
%@see: 《高等代数(第三版 下册)》(丘维声) P77
%@see: 《Linear Algebra Done Right (Fourth Edition)》(Sheldon Axler) P41 2.31
域\(F\)上的任一线性空间\(V\)都有基.
%TODO proof
\end{property}

\begin{example}
%@see: 《高等代数(第三版 下册)》(丘维声) P77 例8
在数域\(K\)上全体\(s \times n\)矩阵形成的线性空间\(M_{s \times n}(K)\)中,
所有基本矩阵组成的子集\begin{equation*}
	\Set{
		E_{11},E_{12},\dotsc,E_{1n},
		\dotsc,
		E_{s1},E_{s2},\dotsc,E_{sn}
	}
\end{equation*}是\(M_{s \times n}(K)\)的一个基.
\begin{proof}
每个\(s \times n\)矩阵\(A = (a_{ij})_{s \times n}\)都可以表示成\begin{equation*}
	A = \sum_{i=1}^s \sum_{j=1}^n a_{ij} E_{ij}.
\end{equation*}

假设\begin{equation*}
	\sum_{i=1}^s \sum_{j=1}^n a_{ij} E_{ij} = 0,
\end{equation*}
则矩阵\(A = (a_{ij})_{s \times n}\)是零矩阵,
从而\(a_{ij} = 0\ (i=1,2,\dotsc,s;j=1,2,\dotsc,n)\).
因此\begin{equation*}
	\Set{
		E_{11},E_{12},\dotsc,E_{1n},
		\dotsc,
		E_{s1},E_{s2},\dotsc,E_{sn}
	}
\end{equation*}线性无关,
从而说明它是\(M_{s \times n}(K)\)的一个基.
\end{proof}
\end{example}

\begin{example}
%@see: 《高等代数(第三版 下册)》(丘维声) P77 例9
数域\(K\)上所有一元多项式形成的线性空间\(K[x]\)中,
子集\begin{equation*}
	\{1,x,x^2,\dotsc,x^n,\dotsc\}
\end{equation*}是\(K[x]\)的一个基.
\begin{proof}
\(K[x]\)上每一个一元多项式\(f(x)\)
可以写成\(f(x)=a_0+a_1 x+a_2 x^2+\dotsb+a_n x^n\).
任取\(S\)的一个有限子集\(\{x^{i_1},\dotsc,x^{i_m}\}\).
设\(k_1 x^{i_1}+\dotsb+k_m x^{i_m}=0\),
则由一元多项式的定义得
\(k_1=\dotsb=k_m=0\),
从而这个子集线性无关,
因此\(S\)线性无关,
于是\(S\)是\(K[x]\)的一个基.
\end{proof}
\end{example}

\subsection{有限维线性空间,无限维线性空间}
\begin{definition}
%@see: 《高等代数(第三版 下册)》(丘维声) P77 定义8
%@see: 《Linear Algebra and Its Applications (Second Edition)》(Peter D. Lax) P5 Definition
%@see: 《Linear Algebra Done Right (Fourth Edition)》(Sheldon Axler) P31 2.13
设\(V\)是域\(F\)上的一个线性空间,
\(S\)是\(V\)的一个基.
如果\(S\)是有限集,
则称“\(V\)是\DefineConcept{有限维的}(finite-dimensional)”;
否则称“\(V\)是\DefineConcept{无限维的}(infinite-dimensional)”.
\end{definition}

\begin{example}
数域\(K\)上全体\(s \times n\)矩阵\(M_{s \times n}(K)\)是有限维的.
\end{example}

\begin{example}
%@see: 《Linear Algebra Done Right (Fourth Edition)》(Sheldon Axler) P31 2.14
数域\(K\)上全体一元多项式\(K[x]\)是无限维的.
\end{example}

\subsection{有限维线性空间的维数}
\begin{theorem}\label{theorem:线性空间.同一个线性空间的任意两个基的基数相等}
%@see: 《高等代数(第三版 下册)》(丘维声) P77 定理10
%@see: 《Linear Algebra and Its Applications (Second Edition)》(Peter D. Lax) P5 Theorem 3.
%@see: 《Linear Algebra Done Right (Fourth Edition)》(Sheldon Axler) P44 2.34
% 原话是: Any two bases of a finite-dimensional vector space have the same length.
设\(V\)是域\(F\)上的一个线性空间.
如果\(V\)是有限维的,
则\(V\)的任意两个基的基数相等.
\begin{proof}
不妨设\(V\)有一个基包含有限多个向量\(\AutoTuple{\alpha}{n}\).
设\(S\)是\(V\)的另一个基.

假如\(\card S>n\),
则\(S\)中可取出\(n+1\)个向量\(\AutoTuple{\beta}{n+1}\),
它们可以由\(\AutoTuple{\alpha}{n}\)线性表出.
由\cref{theorem:线性空间.线性相关性2},%引理1
可知\(\AutoTuple{\beta}{n+1}\)线性相关.
这与\(S\)线性无关矛盾,
因此\(\card S\leq n\).

设\(S=\{\AutoTuple{\beta}{m}\}\ (m\leq n)\),
由\cref{theorem:线性空间.线性相关性2},%推论4
又可知\(m=n\).
\end{proof}
\end{theorem}

\begin{definition}
%@see: 《高等代数(第三版 下册)》(丘维声) P78 定义9
%@see: 《Linear Algebra Done Right (Fourth Edition)》(Sheldon Axler) P44 2.35
设\(V\)是域\(F\)上的一个有限维线性空间,
则\(V\)的一个基的基数
称为“线性空间\(V\)的\DefineConcept{维数}(the \emph{dimension} of \(V\))”,
记作\(\dim_F V\),
简记为\(\dim V\).
\end{definition}

\begin{property}
零空间的维数为\(0\).
\begin{proof}
由\cref{theorem:线性空间的结构.零空间的基是空集} 立即可得.
\end{proof}
\end{property}

\begin{example}
\(\dim M_{s \times n}(K)=sn\).
\end{example}

\begin{example}
\(\dim M_n(K) = n^2\).
\end{example}

维数对于研究有限维线性空间的结构起着重要的作用.

\begin{property}\label{theorem:线性空间.线性相关性3}
%@see: 《高等代数(第三版 下册)》(丘维声) P78 命题11
%@see: 《高等代数(第三版 下册)》(丘维声) P78 命题12
%@see: 《Linear Algebra and Its Applications (Second Edition)》(Peter D. Lax) P5 Lemma 1.
%@see: 《Linear Algebra Done Right (Fourth Edition)》(Sheldon Axler) P35 2.22
%@see: 《Linear Algebra Done Right (Fourth Edition)》(Sheldon Axler) P45 2.38
设\(V\)是域\(F\)上的一个线性空间.
\begin{itemize}
	\item 如果\(\dim V=n\),
	则\(V\)中任意\(n+1\)个向量都线性无关.

	\item 如果\(\dim V=n\),
	则\(V\)中任意\(n\)个线性无关的向量都是\(V\)的一个基.

	\item 如果\(V\)的有限子集\(\AutoTuple{\alpha}{s}\)是线性无关的,
	则\(s \leq n\).
\end{itemize}
%TODO proof
\end{property}

基对于研究线性空间的结构起着重要的作用.

\begin{property}\label{theorem:线性空间.任一向量可由给定基唯一线性表出}
%@see: 《高等代数(第三版 下册)》(丘维声) P78 命题13
%@see: 《Linear Algebra Done Right (Fourth Edition)》(Sheldon Axler) P39 2.28
设\(V\)是域\(F\)上的一个线性空间,
\(\AutoTuple{\alpha}{n}\)是\(V\)的一个基,
则\(V\)中每一个向量\(\alpha\)
可以唯一地表成\(\AutoTuple{\alpha}{n}\)的线性组合.
\begin{proof}
从基的定义知道,任意一个向量\(\alpha\),均可由\(\AutoTuple{\alpha}{n}\)线性表出.
假设有如下两种表出方式:\begin{gather*}
	\alpha = x_1 \alpha_1 + x_2 \alpha_2 + \dotsb + x_n \alpha_n, \\
	\alpha = y_1 \alpha_1 + y_2 \alpha_2 + \dotsb + y_n \alpha_n.
\end{gather*}
相减得\begin{equation*}
	0 = (x_1 - y_1) \alpha_1 + (x_2 - y_2) \alpha_2 + \dotsb + (x_n - y_n) \alpha_n.
\end{equation*}
由于\(\AutoTuple{\alpha}{n}\)线性无关,
所以\begin{equation*}
	x_1 - y_1
	= x_2 - y_2
	= \dotsb
	= x_n - y_n
	= 0.
\end{equation*}
由此可见表出方式唯一.
\end{proof}
\end{property}

\begin{example}\label{example:线性空间.生成子空间等于线性空间的向量组就是基}
%@see: 《高等代数(第三版 下册)》(丘维声) P82 习题8.1 10.
证明:在数域\(K\)上的\(n\)维线性空间\(V\)中,
如果每一个向量都可以由\(\AutoTuple{\alpha}{n}\)线性表出,
则\(\AutoTuple{\alpha}{n}\)是\(V\)的一个基.
\begin{proof}
在\(V\)中取一个基\(\AutoTuple{\delta}{n}\).
由题设条件可知\(\AutoTuple{\delta}{n}\)可以由\(\AutoTuple{\alpha}{n}\)线性表出,
那么由\cref{theorem:线性空间.向量组的秩的性质} 可知\begin{equation*}
	\rank\{\AutoTuple{\delta}{n}\}
	\leq
	\rank\{\AutoTuple{\alpha}{n}\}.
\end{equation*}
又因为\(\AutoTuple{\alpha}{n}\)可以由\(\AutoTuple{\delta}{n}\)线性表出,
从而\begin{equation*}
	\rank\{\AutoTuple{\alpha}{n}\}
	\leq
	\rank\{\AutoTuple{\delta}{n}\}.
\end{equation*}
于是\(\rank\{\AutoTuple{\alpha}{n}\}
= \rank\{\AutoTuple{\delta}{n}\}
= n\),
那么\(\AutoTuple{\alpha}{n}\)线性无关,
因此\(\AutoTuple{\alpha}{n}\)是\(V\)的一个基.
\end{proof}
%@see: 《高等代数(第三版 上册)》(丘维声) P80 习题3.4 4.
\end{example}

\subsection{向量的坐标,过渡矩阵}\label{section:线性空间.向量的坐标}
%@see: 《高等代数(第三版 下册)》(丘维声) P78
%@see: 《Linear Algebra Done Right (Fourth Edition)》(Sheldon Axler) P88 3.73
我们把向量\(\alpha\)由基\(\AutoTuple{\alpha}{n}\)线性表出的系数
组成的\(n\)元有序组\((\AutoTuple{a}{n})\)
称为“向量\(\alpha\)在基\(\AutoTuple{\alpha}{n}\)下的\DefineConcept{坐标}(coordinate)”,
记作\(\VectorMatrix(\alpha,(\AutoTuple{\alpha}{n}))\),
或在不致混淆的情况下简记为\(\VectorMatrix(\alpha)\).
通常把向量的坐标写成列向量形式,
即\begin{equation*}
	\VectorMatrix(\alpha,(\AutoTuple{\alpha}{n}))
	=
	\begin{bmatrix}
		a_1 \\
		\vdots \\
		a_n
	\end{bmatrix}
	\defiff
	\alpha = a_1 \alpha_1 + \dotsb + a_n \alpha_n.
\end{equation*}

由上可知,有限维线性空间\(V\)中给定一个基,
则\(V\)中每一个向量都可以唯一地表示成这个基的线性组合,
从而\(V\)的结构就很清楚了.
因此,基是研究线性空间的结构的第一条途径.

\(n\)维线性空间\(V\)中给定两个基,
我们想要知道,\(V\)中每一个向量分别在这两个基下的坐标有什么关系.

设\(\AutoTuple{\alpha}{n}\)和\(\AutoTuple{\beta}{n}\)是\(V\)的两个基,
\(V\)中向量\(\alpha\)在这两个基下的坐标分别为\begin{equation*}
	X=(\AutoTuple{x}{n})^T, \qquad
	Y=(\AutoTuple{y}{n})^T.
\end{equation*}
为了求\(X\)与\(Y\)之间的关系,
首先把这两个基之间的关系搞清楚.
由于\(\AutoTuple{\alpha}{n}\)是\(V\)的一个基,
并且\(\AutoTuple{\beta}{n}\)都是\(V\)中的元素,
那么,由\cref{theorem:线性空间.任一向量可由给定基唯一线性表出} 可知,
向量组\(\AutoTuple{\beta}{n}\)中的每一个向量\(\beta_i\)
均可唯一地表示成\(\AutoTuple{\alpha}{n}\)的线性组合,
因此有\begin{equation}\label{equation:线性空间.基变换方程组1}
%@see: 《高等代数(第三版 下册)》(丘维声) P79 (2)
	\left\{ \begin{array}{l}
		\beta_1=a_{11} \alpha_1+a_{21} \alpha_2+\dotsb+a_{n1} \alpha_n, \\
		\beta_2=a_{12} \alpha_1+a_{22} \alpha_2+\dotsb+a_{n2} \alpha_n, \\
		\hdotsfor1 \\
		\beta_n=a_{1n} \alpha_1+a_{2n} \alpha_2+\dotsb+a_{nn} \alpha_n.
	\end{array} \right.
\end{equation}
为了使推导过程简洁,
我们引入一种形式写法\footnote{
	这里之所以称之为形式写法,
	是因为我们在先前学习的矩阵乘法运算的定义
	要求参与运算的每一个矩阵、向量的每一个元素都是数域\(K\)中的元素,
	但是这里看似向量的\((\AutoTuple{\alpha}{n}),(\AutoTuple{\beta}{n})\)中的“元素”
	并不是某个域中的元素,而是线性空间\(V\)中的元素.
}\begin{equation*}
%@see: 《高等代数(第三版 下册)》(丘维声) P79 (3)
	x_1 \alpha_1 + \dotsb + x_n \alpha_n
	\defeq
	(\AutoTuple{\alpha}{n})
	\begin{bmatrix}
		x_1 \\
		\vdots \\
		x_n
	\end{bmatrix}.
\end{equation*}
我们可以把\cref{equation:线性空间.基变换方程组1}
写成\begin{equation}\label{equation:线性空间.基变换方程组2}
%@see: 《高等代数(第三版 下册)》(丘维声) P79 (5)
	(\AutoTuple{\beta}{n})
	=
	(\AutoTuple{\alpha}{n})
	A,
\end{equation}
其中\begin{equation*}
	A=\begin{bmatrix}
		a_{11} & a_{12} & \dots & a_{1n} \\
		a_{21} & a_{22} & \dots & a_{2n} \\
		\vdots & \vdots & & \vdots \\
		a_{n1} & a_{n2} & \dots & a_{nn}
	\end{bmatrix}.
\end{equation*}
我们把\(A\)称为
“基\(\AutoTuple{\alpha}{n}\)到基\(\AutoTuple{\beta}{n}\)的\DefineConcept{过渡矩阵}”.
把\cref{equation:线性空间.基变换方程组1,equation:线性空间.基变换方程组2}
称为“基\(\AutoTuple{\alpha}{n}\)到基\(\AutoTuple{\beta}{n}\)的\DefineConcept{基变换公式}”.
把\begin{equation*}
	\begin{bmatrix}
		x_1 \\ \vdots \\ x_n
	\end{bmatrix}
	= A^{-1}
	\begin{bmatrix}
		y_1 \\ \vdots \\ y_n
	\end{bmatrix}
\end{equation*}
称为“基\(\AutoTuple{\alpha}{n}\)到基\(\AutoTuple{\beta}{n}\)的向量的\DefineConcept{坐标变换公式}”.
%@see: https://mathworld.wolfram.com/TransitionMatrix.html
%@see: https://mathworld.wolfram.com/ChangeofCoordinatesMatrix.html

像这样的形式写法,是模仿矩阵乘法的定义.
因此,类似于矩阵乘法的结合律、左右分配律、乘法与数量乘法的关系的证明方法,
可以证明形式写法满足以下规则.

设\(\AutoTuple{\alpha}{n}\)与\(\AutoTuple{\beta}{n}\)是\(V\)中的两个向量组,
\(A,B\)是域\(F\)上的两个\(n\)阶矩阵,
数\(k \in F\),
则\begin{gather*}
	%@see: 《高等代数(第三版 下册)》(丘维声) P79 (6)
	[(\AutoTuple{\alpha}{n}) A] B
	= (\AutoTuple{\alpha}{n}) (A B), \\
	%@see: 《高等代数(第三版 下册)》(丘维声) P79 (7)
	(\AutoTuple{\alpha}{n}) A
	+ (\AutoTuple{\alpha}{n}) B
	= (\AutoTuple{\alpha}{n}) (A + B), \\
	%@see: 《高等代数(第三版 下册)》(丘维声) P79 (8)
	(\AutoTuple{\alpha}{n}) A
	+ (\AutoTuple{\beta}{n}) A
	= (\alpha_1+\beta_1,\dotsc,\alpha_n+\beta_n) A, \\
	%@see: 《高等代数(第三版 下册)》(丘维声) P79 (9)
	[k (\AutoTuple{\alpha}{n})] A
	= (\AutoTuple{\alpha}{n}) (k A), \\
	[k (\AutoTuple{\alpha}{n})] A
	= k [(\AutoTuple{\alpha}{n}) A],
\end{gather*}
其中\begin{gather*}
	%@see: 《高等代数(第三版 下册)》(丘维声) P79 (10)
	(\AutoTuple{\alpha}{n})
	+ (\AutoTuple{\beta}{n})
	\defeq
	(\alpha_1+\beta_1,\dotsc,\alpha_n+\beta_n), \\
	%@see: 《高等代数(第三版 下册)》(丘维声) P79 (11)
	k (\AutoTuple{\alpha}{n})
	= (\AutoTuple{k \alpha}{n}).
\end{gather*}

\begin{proposition}\label{theorem:线性空间.命题14}
%@see: 《高等代数(第三版 下册)》(丘维声) P80 命题14
设\(\AutoTuple{\alpha}{n}\)是\(V\)的一个基,
且\((\AutoTuple{\beta}{n})=(\AutoTuple{\alpha}{n})A\),
则\(\AutoTuple{\beta}{n}\)是\(V\)的一个基
当且仅当\(A\)是可逆矩阵.
\begin{proof}
由于\(\AutoTuple{\alpha}{n}\)线性无关,
并且有\begin{align*}
	k_1 \beta_1+\dotsb+k_n \beta_n
	&=(\AutoTuple{\beta}{n}) (\AutoTuple{k}{n})^T \\
	&=(\AutoTuple{\alpha}{n}) A (\AutoTuple{k}{n})^T,
\end{align*}
因此\begin{align*}
	&\text{$\AutoTuple{\beta}{n}$是$V$的一个基}
	\iff \text{$\AutoTuple{\beta}{n}$线性无关} \\
	&\iff
	k_1 \beta_1+\dotsb+k_n \beta_n=0
	\implies
	k_1=\dotsb=k_n=0 \\
	&\iff
	(\AutoTuple{\alpha}{n}) A (\AutoTuple{k}{n})^T=0
	\implies
	(\AutoTuple{k}{n})^T=0 \\
	&\iff
	A (\AutoTuple{k}{n})^T=0
	\implies
	(\AutoTuple{k}{n})^T=0 \\
	&\iff \text{齐次线性方程组$AX=0$只有零解} \\
	&\iff \abs{A}\neq0
	\iff \text{$A$是可逆矩阵}.
	\qedhere
\end{align*}
\end{proof}
\end{proposition}

\cref{theorem:线性空间.命题14} 表明:
基\(\AutoTuple{\alpha}{n}\)到基\(\AutoTuple{\beta}{n}\)的过渡矩阵是可逆矩阵.

现在可以给出向量\(\alpha\)
分别在基\(\AutoTuple{\alpha}{n}\)
与基\(\AutoTuple{\beta}{n}\)下的坐标\(X,Y\)之间的关系.
由于\begin{equation*}
	\alpha
	=(\AutoTuple{\alpha}{n}) X
	=(\AutoTuple{\beta}{n}) Y,
\end{equation*}
并且基\(\AutoTuple{\alpha}{n}\)到基\(\AutoTuple{\beta}{n}\)的过渡矩阵是\(A\),
因此\begin{equation*}
	(\AutoTuple{\alpha}{n}) X
	=(\AutoTuple{\beta}{n}) Y
	=(\AutoTuple{\alpha}{n}) A Y.
\end{equation*}
由于同一个向量由基\(\AutoTuple{\alpha}{n}\)线性表出的方式唯一,
从上式得\begin{equation*}
%@see: 《高等代数(第三版 下册)》(丘维声) P79 (12)
	X=AY,
\end{equation*}
从而\begin{equation*}
%@see: 《高等代数(第三版 下册)》(丘维声) P79 (13)
	Y=A^{-1}X.
\end{equation*}

\begin{example}
设\(\alpha_1,\alpha_2,\alpha_3\)是\(\mathbb{R}^3\)的一组基,
求:基\(\alpha_1,\frac12\alpha_2,\frac13\alpha_3\)
到基\(\alpha_1+\alpha_2,\alpha_2+\alpha_3,\alpha_3+\alpha_1\)的过渡矩阵.
\begin{solution}
设所求过渡矩阵为\(P\),
则根据定义有\begin{equation*}
	\begin{bmatrix}
		\alpha_1 & \frac12\alpha_2 & \frac13\alpha_3
	\end{bmatrix} P
	= \begin{bmatrix}
		\alpha_1+\alpha_2 & \alpha_2+\alpha_3 & \alpha_3+\alpha_1
	\end{bmatrix},
\end{equation*}
即\begin{equation*}
	\begin{bmatrix}
		\alpha_1 & \alpha_2 & \alpha_3
	\end{bmatrix}
	\begin{bmatrix}
		1 \\
		& \frac12 \\
		&& \frac13
	\end{bmatrix} P
	= \begin{bmatrix}
	\alpha_1 & \alpha_2 & \alpha_3
	\end{bmatrix}
	\begin{bmatrix}
		1 & 0 & 1 \\
		1 & 1 & 0 \\
		0 & 1 & 1
	\end{bmatrix},
\end{equation*}
所以\begin{equation*}
	P = \begin{bmatrix}
		1 \\
		& \frac12 \\
		&& \frac13
	\end{bmatrix}^{-1}
	\begin{bmatrix}
		1 & 0 & 1 \\
		1 & 1 & 0 \\
		0 & 1 & 1
	\end{bmatrix}
	= \begin{bmatrix}
		1 \\
		& 2 \\
		&& 3
	\end{bmatrix} \begin{bmatrix}
		1 & 0 & 1 \\
		1 & 1 & 0 \\
		0 & 1 & 1
	\end{bmatrix}
	= \begin{bmatrix}
		1 & 0 & 1 \\
		2 & 2 & 0 \\
		0 & 3 & 3
	\end{bmatrix}.
\end{equation*}
\end{solution}
\end{example}

\begin{example}
在\(K[x]_3\)中取两个基\(
	\alpha_1
	= x^3 + 2x^2 - x,
	\allowbreak
	\alpha_2
	= x^3 - x^2 + x + 1,
	\allowbreak
	\alpha_3
	= -x^3 + 2x^2 + x + 1,
	\allowbreak
	\alpha_4
	= -x^3 - x^2 + 1
\)和\(
	\beta_1
	= 2x^3 + x^2 + 1,
	\allowbreak
	\beta_2
	= x^2 + 2x + 2,
	\allowbreak
	\beta_3
	= -2x^3 + x^2 + x + 2,
	\allowbreak
	\beta_4
	= x^3 + 3x^2 + x + 2
\).
求基\(\AutoTuple{\alpha}{4}\)到基\(\AutoTuple{\beta}{4}\)的过渡矩阵.
\begin{solution}
显然\begin{gather*}
	(\AutoTuple{\alpha}{4})
	= (x^3,x^2,x,1)
	\begin{bmatrix}
		1 & 1 & -1 & -1 \\
		2 & -1 & 2 & -1 \\
		-1 & 1 & 1 & 0 \\
		0 & 1 & 1 & 1
	\end{bmatrix}, \\
	(\AutoTuple{\beta}{4})
	= (x^3,x^2,x,1)
	\begin{bmatrix}
		2 & 0 & -2 & 1 \\
		1 & 1 & 1 & 3 \\
		0 & 2 & 1 & 1 \\
		1 & 2 & 2 & 2
	\end{bmatrix},
\end{gather*}
于是所求过渡矩阵为\begin{equation*}
	\begin{bmatrix}
		1 & 1 & -1 & -1 \\
		2 & -1 & 2 & -1 \\
		-1 & 1 & 1 & 0 \\
		0 & 1 & 1 & 1
	\end{bmatrix}^{-1}
	\begin{bmatrix}
		2 & 0 & -2 & 1 \\
		1 & 1 & 1 & 3 \\
		0 & 2 & 1 & 1 \\
		1 & 2 & 2 & 2
	\end{bmatrix}.
\end{equation*}
由于\begin{equation*}
	\begin{bmatrix}
		1 & 1 & -1 & -1 & 2 & 0 & -2 & 1 \\
		2 & -1 & 2 & -1 & 1 & 1 & 1 & 3 \\
		-1 & 1 & 1 & 0 & 0 & 2 & 1 & 1 \\
		0 & 1 & 1 & 1 & 1 & 2 & 2 & 2
	\end{bmatrix}
	\to \begin{bmatrix}
		1 & 0 & 0 & 0 & 1 & 0 & 0 & 1 \\
		0 & 1 & 0 & 0 & 1 & 1 & 0 & 1 \\
		0 & 0 & 1 & 0 & 0 & 1 & 1 & 1 \\
		0 & 0 & 0 & 1 & 0 & 0 & 1 & 0 \\
	\end{bmatrix},
\end{equation*}
因此所求过渡矩阵为\begin{equation*}
	\begin{bmatrix}
		1 & 0 & 0 & 1 \\
		1 & 1 & 0 & 1 \\
		0 & 1 & 1 & 1 \\
		0 & 0 & 1 & 0 \\
	\end{bmatrix}.
\end{equation*}
%@Mathematica: A = {{1, 1, -1, -1}, {2, -1, 2, -1}, {-1, 1, 1, 0}, {0, 1, 1, 1}};
%@Mathematica: B = {{2, 0, -2, 1}, {1, 1, 1, 3}, {0, 2, 1, 1}, {1, 2, 2, 2}};
%@Mathematica: Inverse[A].B // MatrixForm
%@Mathematica: Join[A, B, 2] // MatrixForm
%@Mathematica: Join[A, B, 2] // RowReduce // MatrixForm
%@Mathematica: RowReduce[Join[A, B, 2]][[All, 5 ;; 8]] // MatrixForm
\end{solution}
\end{example}

\begin{example}
%@see: 《高等代数(第三版 下册)》(丘维声) P81 习题8.1 4.
把复数域\(\mathbb{C}\)看成实数域\(\mathbb{R}\)上的线性空间,
求它的一个基和维数,
以及每个复数在这个基下的坐标.
\begin{solution}
把复数域看成实数域上的线性空间\(V\),
容易看出,有限集\(S = \{1,\iu\}\)是线性空间\(V\)的一个基,
它的维数为\(\dim V = \card S = 2\),
而每个复数\(z = a + b\iu\)在这个基下的坐标为\((a,b)^T\).
\end{solution}
\end{example}

\begin{example}
%@see: 《高等代数(第三版 下册)》(丘维声) P81 习题8.1 5.
把数域\(K\)看成自身上的线性空间,求它的一个基和维数.
\begin{solution}
把数域\(K\)看成自身上的线性空间\(V\),
容易看出,\(S = \{1\}\)是线性空间\(V\)的一个基,
它的维数为\(\dim V = \card S = 1\).
\end{solution}
\end{example}

\begin{example}
%@see: 《高等代数(第三版 下册)》(丘维声) P81 习题8.1 11.
设\(X = \{\AutoTuple{x}{n}\}\),\(F\)是一个域.
把映射空间\(F^X\)看成域\(F\)上的一个线性空间,
求\(F^X\)的一个基和维数,
再求映射\(f \in F^X\)在这个基下的坐标.
\begin{solution}
任意给定\(f \in F^X\),
必有\(f = \Set{
	(x_1,f(x_1)),
	\dotsc,
	(x_n,f(x_n))
}\).

令\begin{equation*}
	f_i(x_j) \defeq \delta(i,j),
	\quad i,j=1,2,\dotsc,n,
\end{equation*}
其中\(\delta\)是克罗内克\(\delta\)函数,
即\begin{equation*}
	\delta(i,j) = \left\{ \begin{array}{cl}
		1, & i = j, \\
		0, & i \neq j.
	\end{array} \right.
\end{equation*}
那么\begin{equation*}
	f(x) = f(x_1) f_1(x) + \dotsb + f(x_n) f_n(x),
	\quad x \in X,
	\eqno(1)
\end{equation*}
这就说明,\(f\)可以由\(\AutoTuple{f}{n}\)线性表出.

显然\(\AutoTuple{f}{n}\)线性无关,
那么\(\AutoTuple{f}{n}\)是\(F^X\)的一个基,
从而有\(\dim F^X = n\).

由(1)式可知,
函数\(f\)在基\(\AutoTuple{f}{n}\)下的坐标为
\((f(x_1),\dotsc,f(x_n))\).
\end{solution}
\end{example}

\subsection{线性空间的笛卡尔和}
\begin{definition}
%@see: 《Linear Algebra and Its Applications (Second Edition)》(Peter D. Lax) P10 Definition
%@see: 《Linear Algebra Done Right (Fourth Edition)》(Sheldon Axler) P96 3.87
设\(V,W\)都是域\(F\)上的线性空间.
把定义了加法\begin{equation*}
	(v_1,w_1) + (v_2,w_2) \defeq (v_1+v_2,w_1+w_2)
\end{equation*}
和纯量乘法\begin{equation*}
	k (v_1,w_1) \defeq (k v_1,k w_1)
\end{equation*}
的集合\begin{equation*}
	\Set{
		(v,w)
		\given
		v \in V,
		w \in W
	}
\end{equation*}
称为“线性空间\(V\)和\(W\)的\DefineConcept{笛卡尔和}(Cartesian sum)”,
记作\(V \CartesianSum W\).
\end{definition}

\begin{theorem}\label{theorem:线性空间.线性空间的笛卡尔和是线性空间}
%@see: 《Linear Algebra and Its Applications (Second Edition)》(Peter D. Lax) P10
%@see: 《Linear Algebra Done Right (Fourth Edition)》(Sheldon Axler) P96 3.89
设\(V,W\)都是域\(F\)上的线性空间,
则线性空间\(V\)和\(W\)的笛卡尔和\(V \CartesianSum W\)是域\(F\)上的线性空间.
\begin{proof}
由于\(V,W\)都是线性空间,
所以由定义有:
对于\(\forall v_1,v_2,v_3 \in V,
\forall w_1,w_2,w_3 \in W,
\forall k,l \in F\),
有\begin{gather*}
	(v_1,w_1) + (v_2,w_2)
	= (v_1+v_2,w_1+w_2)
	= (v_2+v_1,w_2+w_1)
	= (v_2,w_2)+(v_1,w_1), \\
	\begin{aligned}
		((v_1,w_1) + (v_2,w_2)) + (v_3,w_3)
		&= (v_1+v_2,w_1+w_2) + (v_3,w_3) \\
		&= ((v_1+v_2)+v_3,(w_1+w_2)+w_3) \\
		&= (v_1+(v_2+v_3),w_1+(w_2+w_3)) \\
		&= (v_1,w_1) + (v_2+v_3,w_2+w_3) \\
		&= (v_1,w_1) + ((v_2,w_2) + (v_3,w_3)),
	\end{aligned} \\
	(0,0) \in V \times W
	\land
	(v_1,w_1) + (0,0)
	= (v_1+0,w_1+0)
	= (v_1,w_1), \\
	(v_1,w_1) + (-v_1,-w_1)
	= (v_1+(-v_1),w_1+(-w_1))
	= (0,0), \\
	1(v_1,w_1)
	= (1v_1,1w_1)
	= (v_1,w_1), \\
	k(l(v_1,w_1))
	= k(lv_1,lw_1)
	= (k(lv_1),k(lw_1))
	= ((kl)v_1,(kl)w_1)
	= (kl)(v_1,w_1), \\
	\begin{aligned}
		(k+l)(v_1,w_1)
		&= ((k+l)v_1,(k+l)w_1) \\
		&= (kv_1+lv_1,kw_1+lw_1) \\
		&= (kv_1,kw_1) + (lv_1,lw_1) \\
		&= k(v_1,w_1) + l(v_1,w_1),
	\end{aligned} \\
	\begin{aligned}
		k((v_1,w_1)+(v_2,w_2))
		&= k(v_1+v_2,w_1+w_2) \\
		&= (k(v_1+v_2),k(w_1+w_2)) \\
		&= (kv_1+kv_2,kw_1+kw_2) \\
		&= (kv_1,kw_1)+(kv_2,kw_2) \\
		&= k(v_1,w_1)+k(v_2,w_2).
	\end{aligned}
\end{gather*}
这就说明\(V \times W\)是域\(F\)上的一个线性空间.
\end{proof}
\end{theorem}

\begin{theorem}\label{theorem:线性空间.笛卡尔和的维数公式}
%@see: 《Linear Algebra and Its Applications (Second Edition)》(Peter D. Lax) P11 Exercise 18.
%@see: 《Linear Algebra Done Right (Fourth Edition)》(Sheldon Axler) P97 3.92
设\(V,W\)都是域\(F\)上的线性空间,
则线性空间\(V\)和\(W\)的笛卡尔和\(V \CartesianSum W\)满足\begin{equation*}
	\dim(V \CartesianSum W) = \dim V + \dim W.
\end{equation*}
%TODO proof
\end{theorem}
