\section{线性空间的结构}
\subsection{线性空间的概念与性质}
\begin{definition}
%@see: 《高等代数(第三版 下册)》(丘维声) P72 定义1
%@see: 《Linear Algebra and Its Applications (Second Edition)》(Peter D. Lax) P1
设\(V\)是一个非空集合,
\(F\)是一个域.

%@see: 《Linear Algebra Done Right (Fourth Eidition)》(Sheldon Axler) P12 1.19
定义运算\(V\times V\to V,(\alpha,\beta)\mapsto\gamma=\alpha+\beta\),
称之为\DefineConcept{加法}(addition).

定义运算\(F\times V\to V,(k,\alpha)\mapsto\beta=k\alpha\),
称之为\DefineConcept{纯量乘法}(scalar multiplication)
\footnote{
	当域\(F\)是一个数域(例如\(\mathbb{Q},\mathbb{R},\mathbb{C}\))时,
	纯量乘法又称为\DefineConcept{数量乘法}.
}.

%@see: 《Linear Algebra Done Right (Fourth Eidition)》(Sheldon Axler) P12 1.20
%@see: 《Linear Algebra Done Right (Fourth Eidition)》(Sheldon Axler) P15 1.28
如果\emph{加法}与\emph{纯量乘法}满足以下八条公理:
\begin{center}
	\begin{minipage}{.8\textwidth}
		\begin{axiom}\label{definition:线性空间.运算法则1}
		%@see: 《Linear Algebra and Its Applications (Second Edition)》(Peter D. Lax) P2 (2)
		% 加法交换律
		\((\forall\alpha,\beta\in V)
		[\alpha+\beta=\beta+\alpha]\).
		\end{axiom}
		\begin{axiom}\label{definition:线性空间.运算法则2}
		%@see: 《Linear Algebra and Its Applications (Second Edition)》(Peter D. Lax) P2 (3)
		% 加法结合律
		\((\forall\alpha,\beta,\gamma\in V)
		[(\alpha+\beta)+\gamma=\alpha+(\beta+\gamma)]\).
		\end{axiom}
		\begin{axiom}\label{definition:线性空间.运算法则3}
		%@see: 《Linear Algebra and Its Applications (Second Edition)》(Peter D. Lax) P2 (4)
		% 零元
		\(0\in V
		\land
		(\forall\alpha \in V)
		[\alpha+0=\alpha]\),
		把\(0\)称为“\(V\)的\DefineConcept{零元}(additive identity)”.
		\end{axiom}
		\begin{axiom}\label{definition:线性空间.运算法则4}
		%@see: 《Linear Algebra and Its Applications (Second Edition)》(Peter D. Lax) P2 (5)
		% 负元
		\((\forall\alpha \in V)
		(\exists\beta \in V)
		[\alpha+\beta=0]\),
		\(\beta\)称为“\(\alpha\)的\DefineConcept{负元}(additive inverse)”,
		记作\(-\alpha\).
		定义运算\(V\times V \to V,(\alpha,\beta) \mapsto \alpha-\beta \defeq \alpha+(-\beta)\),
		称之为\DefineConcept{减法}.
		\end{axiom}
		\begin{axiom}\label{definition:线性空间.运算法则5}
		%@see: 《Linear Algebra and Its Applications (Second Edition)》(Peter D. Lax) P2 (9)
		% 幺元
		\((\forall\alpha\in V)[1\alpha=\alpha]\),
		其中\(1\)是\(F\)的单位元.
		\end{axiom}
		\begin{axiom}\label{definition:线性空间.运算法则6}
		%@see: 《Linear Algebra and Its Applications (Second Edition)》(Peter D. Lax) P2 (6)
		% 纯量乘法
		\((\forall\alpha\in V)
		(\forall k,l\in F)
		[k(l\alpha)=(kl)\alpha]\).
		\end{axiom}
		\begin{axiom}\label{definition:线性空间.运算法则7}
		%@see: 《Linear Algebra and Its Applications (Second Edition)》(Peter D. Lax) P2 (8)
		% 纯量乘法
		\((\forall\alpha\in V)
		(\forall k,l\in F)
		[(k+l)\alpha=k\alpha+l\alpha]\).
		\end{axiom}
		\begin{axiom}\label{definition:线性空间.运算法则8}
		%@see: 《Linear Algebra and Its Applications (Second Edition)》(Peter D. Lax) P2 (7)
		% 纯量乘法
		\((\forall\alpha,\beta\in V)
		(\forall k\in F)
		[k(\alpha+\beta)=k\alpha+k\beta]\).
		\end{axiom}
	\end{minipage}
\end{center}
则称“\(V\)是域\(F\)上的一个\DefineConcept{线性空间}%
(\(V\) is a \emph{linear space} over field \(F\))”,
%@see: 《Linear Algebra Done Right (Fourth Eidition)》(Sheldon Axler) P12 1.21
把\(V\)中的元素称为\DefineConcept{向量}(vector)或\DefineConcept{点}(point),
把\(F\)中的元素称为\DefineConcept{标量}(scalar),
把加法与纯量乘法这两种运算统称为\DefineConcept{线性运算}(linear operation).
\end{definition}
\begin{remark}
与我们在初等代数中学习的自然数、整数、有理数、实数的加法、乘法不同,
线性空间的加法、纯量乘法是抽象的,
线性空间的加法可以是映射空间\(V^{V \times V}\)中的任意一个映射,
线性空间的纯量乘法可以是映射空间\(V^{F \times V}\)中的任意一个映射.
另外,向量空间的加法、纯量乘法和域\(F\)的加法、乘法毫无关系.
\end{remark}
\begin{remark}
线性空间\(V\)对加法成群.
\end{remark}

%@see: 《Linear Algebra Done Right (Fourth Eidition)》(Sheldon Axler) P13 1.22
当\(F = \mathbb{R}\)时,把线性空间\(V\)称为\DefineConcept{实线性空间}(real vector space).

当\(F = \mathbb{C}\)时,把线性空间\(V\)称为\DefineConcept{复线性空间}(complex vector space).

实线性空间与复线性空间,是代数结构完全不同的两个线性空间.

\begin{example}
下面列举一些常见的线性空间.
\begin{itemize}
	\item 只含零元\(0 \in V\)的线性空间\(\{0\}\),
	称为\DefineConcept{零空间}.

	%@see: 《高等代数(第三版 下册)》(丘维声) P73 例4
	\item 复数域\(\mathbb{C}\)
	可以看成是实数域\(\mathbb{R}\)上的一个线性空间,
	其加法是复数的加法,
	其数量乘法是实数与复数的乘法.

	%@see: 《高等代数(第三版 下册)》(丘维声) P73 例5
	\item 任一数域\(K\)都可以看成是自身上的线性空间,
	其加法就是数域\(K\)中的加法,
	其数量乘法就是数域\(K\)中的乘法.

	\item 集合\(\mathbb{R}^{n \times 1}\)关于向量的加法、实数与向量的纯量乘法,构成实线性空间.

	\item 集合\(\mathbb{R}^{s \times n}\)关于矩阵的加法、实数与矩阵的纯量乘法,构成实线性空间.

	%@see: 《高等代数(第三版 下册)》(丘维声) P73 例2
	%@see: 《Linear Algebra Done Right (Fourth Eidition)》(Sheldon Axler) P13 1.24
	\item 设\(F\)是一个域,\(X\)是一个非空集合,
	则映射空间\(F^X\)
	对函数的加法\[
		(f+g)(x) \defeq f(x) + g(x),
		\quad f,g \in F^X, x \in X,
	\]
	以及实数与函数的数量乘法\[
		(k f)(x) \defeq k f(x),
		\quad f \in F^X, k \in F, x \in X,
	\]
	成为\(F\)上的一个线性空间.
	\(F^X\)的零元是零函数\[
		0(x) = 0,
		\quad x \in X.
	\]
	%上式等号左边的0表示零函数,等号右边的0表示域\(F\)的零元

	%@see: 《高等代数(第三版 下册)》(丘维声) P73 例3
	\item 数域\(K\)上的一元多项式环\(K[x]\)
	对多项式的加法,以及数与多项式的乘法,
	成为\(K\)上的一个线性空间.
\end{itemize}
\end{example}

上述例子表明,线性空间这一数学模型适用性很广.
从现在开始,我们将从线性空间的定义出发,
作逻辑推理,深入揭示线性空间的性质和结构,
它们对于所有的具体的线性空间都成立.

\begin{property}
%@see: 《高等代数(第三版 下册)》(丘维声) P74
%@see: 《Linear Algebra Done Right (Fourth Eidition)》(Sheldon Axler) P14 1.26
%@see: 《Linear Algebra Done Right (Fourth Eidition)》(Sheldon Axler) P15 1.27
%@see: 《Linear Algebra Done Right (Fourth Eidition)》(Sheldon Axler) P15 1.30
%@see: 《Linear Algebra Done Right (Fourth Eidition)》(Sheldon Axler) P16 1.31
%@see: 《Linear Algebra Done Right (Fourth Eidition)》(Sheldon Axler) P16 1.32
设\(V\)是域\(F\)上的一个线性空间.
\begin{itemize}
	\item \(V\)的零元是唯一的.
	\item \(V\)中每个元素的负元是唯一的.
	%@see: 《Linear Algebra and Its Applications (Second Edition)》(Peter D. Lax) P2 (10)
	\item \((\forall\alpha\in V)[0\alpha=0]\).
	\item \((\forall k\in F)[k0=0]\).
	\item \(k\alpha=0 \implies k=0 \lor \alpha=0\).
	\item \((\forall\alpha\in V)[(-1)\alpha=-\alpha]\).
\end{itemize}
\end{property}

\begin{example}
%@see: 《高等代数(第三版 下册)》(丘维声) P81 习题8.1 1.(2)
在正实数集\(\mathbb{R}^+\)上定义加法、数量乘法:\begin{gather*}
	\oplus \defeq \Set{
		((a,b),ab)
		\given
		a,b \in \mathbb{R}^+
	}, \\
	\odot \defeq \Set{
		((k,a),a^k)
		\given
		a \in \mathbb{R}^+,
		k \in \mathbb{R}
	}.
\end{gather*}
试判断\((\mathbb{R}^+,\oplus,\odot)\)是不是实数域\(\mathbb{R}\)上的线性空间.
\begin{solution}
显然\(\oplus,\odot\)都是映射.
由于实数的乘法运算适合交换律、结合律,
所以\(\oplus\)也适合交换律、结合律.
正实数\(1\)是\(\mathbb{R}^+\)的零元,
这是因为对于任意\(a \in \mathbb{R}^+\)总有\(1 \oplus a = 1a = a\).
任意一个正实数\(a\)的倒数\(1/a\)就是\(a\)的负元.
实数\(1\)还是\(\mathbb{R}^+\)的单位元,
这是因为\(1 \odot a = a^1 = a\).
对于任意正实数\(a,b\)和任意实数\(k,l\),
显然有\begin{gather*}
	k \odot (l \odot a)
	= (a^l)^k
	= a^{k l}
	= (kl) \odot a, \\
	(k+l) \odot a
	= a^{k+l}
	= a^k a^l
	= (k \odot a) \oplus (l \odot a), \\
	k \odot (a \oplus b)
	= (ab)^k
	= a^k b^k
	= (k \odot a) \oplus (k \odot b).
\end{gather*}
综上所述,\((\mathbb{R}^+,\oplus,\odot)\)确实是实数域\(\mathbb{R}\)上的一个线性空间.
\end{solution}
\end{example}

\subsection{线性空间的线性关系}
域\(F\)上的线性空间\(V\)的有限子集,称为“\(V\)中的一个\DefineConcept{向量组}”.

向量组\(A\)的子集,称为“\(A\)的一个\DefineConcept{部分组}”.

%@see: 《高等代数(第三版 下册)》(丘维声) P75
%@see: 《Linear Algebra and Its Applications (Second Edition)》(Peter D. Lax) P4 Definition
%@see: 《Linear Algebra Done Right (Fourth Eidition)》(Sheldon Axler) P28 2.2
设\(\AutoTuple{\alpha}{s}\)是\(V\)中一个向量组,
任给\(F\)中一组元素\(\AutoTuple{k}{s}\),
向量\(k_1\alpha_1+\dotsb+k_s\alpha_s\)
称为“\(\AutoTuple{\alpha}{s}\)的一个\DefineConcept{线性组合}(linear combination)”,
称\(\AutoTuple{k}{s}\)为\DefineConcept{系数}.

%@see: 《高等代数(第三版 下册)》(丘维声) P75
对于\(\beta\in V\),
如果有\(F\)中一组元素\(\AutoTuple{c}{s}\),
使得\(\beta=c_1\alpha_1+\dotsb+c_s\alpha_s\),
则称“\(\beta\)可以由\(\AutoTuple{\alpha}{s}\)~\DefineConcept{线性表出}%
(\(\beta\) can be expressed as a linear combination of \(\AutoTuple{\alpha}{s}\))”.

\begin{definition}
%@see: 《高等代数(第三版 下册)》(丘维声) P75 定义2
%@see: 《Linear Algebra and Its Applications (Second Edition)》(Peter D. Lax) P4 Definition
%@see: 《Linear Algebra and Its Applications (Second Edition)》(Peter D. Lax) P5 Definition
%@see: 《Linear Algebra Done Right (Fourth Eidition)》(Sheldon Axler) P32 2.15
%@see: 《Linear Algebra Done Right (Fourth Eidition)》(Sheldon Axler) P33 2.17
设\(\AutoTuple{\alpha}{s}\ (s\geq1)\)是\(V\)中一个向量组.
如果有\(F\)中不全为零的元素\(\AutoTuple{k}{s}\),
使得\(k_1\alpha_1+\dotsb+k_s\alpha_s=0\),
则称“\(\AutoTuple{\alpha}{s}\)是\DefineConcept{线性相关的}%
(\(\AutoTuple{\alpha}{s}\) are \emph{linearly dependent})”;
否则称“\(\AutoTuple{\alpha}{s}\)是\DefineConcept{线性无关的}%
(\(\AutoTuple{\alpha}{s}\) are \emph{linearly independent})”.
\end{definition}

%@see: 《Linear Algebra Done Right (Fourth Eidition)》(Sheldon Axler) P32 2.15
空向量组\(\emptyset\)是线性无关的.

\begin{definition}
%@see: 《高等代数(第三版 下册)》(丘维声) P75 定义3
设\(W\)是\(V\)的任一无限子集.
如果\(W\)有一个有限子集是线性相关的,
则称“\(W\)是\DefineConcept{线性相关的}%
(\(W\) is \emph{linearly dependent})”;
如果\(W\)的任何有限子集都是线性无关的,
则称“\(W\)是\DefineConcept{线性无关的}%
(\(W\) is \emph{linearly independent})”.
\end{definition}

可以证明,
数域\(K\)上的线性方程组的理论,
和数域\(K\)上的矩阵、行列式理论,
在把数域\(K\)换成任意域\(F\)以后,
仍然成立.
\begin{property}
%@see: 《高等代数(第三版 下册)》(丘维声) P75 例6
%@see: 《高等代数(第三版 下册)》(丘维声) P75 例7
%@see: 《高等代数(第三版 下册)》(丘维声) P75 命题1
%@see: 《高等代数(第三版 下册)》(丘维声) P75 命题2
%@see: 《Linear Algebra Done Right (Fourth Eidition)》(Sheldon Axler) P33 2.19
设\(V\)是域\(F\)上的一个线性空间.
\begin{itemize}
	\item \(\text{$\alpha$线性相关}\iff\alpha=0\).
	\item 包含零向量的向量组一定线性相关.
	\item 基数大于或等于\(2\)的向量组\(W\)线性相关
	当且仅当\(W\)中至少有一个向量可以由其余向量中的有限多个线性表出.
	\item 向量\(\beta\)可以由线性无关向量组\(\AutoTuple{\alpha}{s}\)线性表出的充分必要条件是
	\(\AutoTuple{\alpha}{s},\beta\)线性相关.
\end{itemize}
\end{property}

\begin{definition}
%@see: 《高等代数(第三版 下册)》(丘维声) P76 定义4
设\(W_1,W_2\)都是\(V\)的非空子集,
如果\(W_1\)中每一个向量都可以由\(W_2\)中有限多个向量线性表出,
则称“\(W_1\)可以由\(W_2\)~\DefineConcept{线性表出}”.
如果\(W_1\)与\(W_2\)可以互相线性表出,
则称“\(W_1\)与\(W_2\)是\DefineConcept{等价的}”.
\end{definition}

容易证明,“线性表出”具有传递性,
从而“等价”也具有传递性.
显然,向量组的“等价”具有反身性与对称性.

\begin{property}\label{theorem:线性空间.性质3}
%@see: 《高等代数(第三版 下册)》(丘维声) P76 引理1
%@see: 《高等代数(第三版 下册)》(丘维声) P76 推论3
%@see: 《高等代数(第三版 下册)》(丘维声) P76 推论4
设\(V\)是域\(F\)上的一个线性空间.
\begin{itemize}
	\item 设向量组\(\AutoTuple{\beta}{r}\)
	可以由向量组\(\AutoTuple{\alpha}{s}\)线性表出,则\begin{gather*}
		r>s
		\implies
		\text{$\AutoTuple{\beta}{r}$线性相关}, \\
		\text{$\AutoTuple{\beta}{r}$线性无关}
		\implies
		r\leq s.
	\end{gather*}

	\item 等价的线性无关的向量组所含向量的个数相等.
\end{itemize}
\end{property}

\subsection{向量组的秩}
\begin{definition}
%@see: 《高等代数(第三版 下册)》(丘维声) P76 定义5
设\(V\)是域\(F\)上的一个线性空间,
\(A\)是\(V\)的一个子集,
\(a\)是\(A\)的有限子集.
如果\(a\)是线性无关的,
但是\[
	(\forall\beta \in A-a)
	[\text{$a \cup \{\beta\}$是线性相关的}],
\]
则称“\(a\)是\(A\)的一个\DefineConcept{极大线性无关组}”.
\end{definition}

\begin{property}
%@see: 《高等代数(第三版 下册)》(丘维声) P76 推论5
%@see: 《高等代数(第三版 下册)》(丘维声) P76 推论6
设\(V\)是域\(F\)上的一个线性空间.
\begin{itemize}
	\item 向量组与它的极大线性无关组等价.
	\item 向量组的任意两个极大线性无关组的基数相等.
\end{itemize}
\end{property}

\begin{definition}
%@see: 《高等代数(第三版 下册)》(丘维声) P76 定义6
向量组\(A=\{\AutoTuple{\alpha}{s}\}\)的一个极大线性无关组的基数,
称为“向量组\(A\)的\DefineConcept{秩}(rank)”,
记为\(\rank A\)或\(\rank\{\AutoTuple{\alpha}{s}\}\).
\end{definition}

\begin{property}\label{theorem:线性空间.向量组的秩的性质}
%@see: 《高等代数(第三版 下册)》(丘维声) P76 命题8
%@see: 《高等代数(第三版 下册)》(丘维声) P76 命题9
%@see: 《高等代数(第三版 下册)》(丘维声) P76 推论9
设\(V\)是域\(F\)上的一个线性空间.
\begin{itemize}
	\item 全由零向量组成的向量组的秩为零.

	\item 向量组线性无关的充分必要条件是
	它的秩等于它的基数.

	\item 设\(A,B\)都是向量组.
	如果\(A\)可以由\(B\)线性表出,
	则\(\rank A \leq \rank B\).

	\item 等价的向量组有相同的秩.
\end{itemize}
\end{property}

\subsection{线性空间的基}
\begin{definition}\label{definition:线性空间.线性空间的基}
%@see: 《高等代数(第三版 下册)》(丘维声) P76 定义7
%@see: 《Linear Algebra Done Right (Fourth Eidition)》(Sheldon Axler) P39 2.26
设\(V\)是域\(F\)上的一个线性空间,\(S \subseteq V\).
如果\begin{itemize}
	\item \(S\)线性无关,
	\item \(V\)中每一个向量都可以由\(S\)中有限多个向量线性表出,
\end{itemize}
则称“\(S\)是\(V\)的一个\DefineConcept{基}%
(\(S\) is a \emph{basis} for \(V\))”.
\end{definition}
\begin{remark}
%@credit: {647826c9-7e2a-49d1-b176-cd39b299b349} 说:除了 Hamel 基以外,还有 Schauder 基等其他定义
%@credit: {85841724-e8e0-4a39-88bf-973ade1b5e13} 说:参考《代数学(一)》(李方、邓少强、冯荣权、刘东文) P101 定义5.1.2
在\hyperref[definition:线性空间.线性空间的基]{基的定义}中,
必须要注意第二个条件中“有限多个”这个限定,
它说明这里定义的基是\emph{哈莫基}(Hamel basis).
%@see: https://zh.wikipedia.org/wiki/%E5%9F%BA_(%E7%B7%9A%E6%80%A7%E4%BB%A3%E6%95%B8)
%@see: https://en.wikipedia.org/wiki/Basis_(linear_algebra)
\end{remark}

\begin{property}\label{theorem:线性空间的结构.零空间的基是空集}
%@see: 《高等代数(第三版 下册)》(丘维声) P77
零空间的基是空集.
%TODO 无法确定这个究竟是定义还是性质
\end{property}

\begin{property}
%@see: 《高等代数(第三版 下册)》(丘维声) P77
%@see: 《Linear Algebra Done Right (Fourth Eidition)》(Sheldon Axler) P40 2.30
%@see: 《Linear Algebra Done Right (Fourth Eidition)》(Sheldon Axler) P41 2.31
域\(F\)上的任一线性空间\(V\)都有基.
%TODO proof
\end{property}

\begin{example}
%@see: 《高等代数(第三版 下册)》(丘维声) P77 例8
在数域\(K\)上全体\(s \times n\)矩阵形成的线性空间\(M_{s \times n}(K)\)中,
所有基本矩阵组成的子集\[
	\Set{
		E_{11},E_{12},\dotsc,E_{1n},
		\dotsc,
		E_{s1},E_{s2},\dotsc,E_{sn}
	}
\]是\(M_{s \times n}(K)\)的一个基.
\begin{proof}
每个\(s \times n\)矩阵\(A = (a_{ij})_{s \times n}\)都可以表示成\[
	A = \sum_{i=1}^s \sum_{j=1}^n a_{ij} E_{ij}.
\]

假设\[
	\sum_{i=1}^s \sum_{j=1}^n a_{ij} E_{ij} = 0,
\]
则矩阵\(A = (a_{ij})_{s \times n}\)是零矩阵,
从而\(a_{ij} = 0\ (i=1,2,\dotsc,s;j=1,2,\dotsc,n)\).
因此\[
	\Set{
		E_{11},E_{12},\dotsc,E_{1n},
		\dotsc,
		E_{s1},E_{s2},\dotsc,E_{sn}
	}
\]线性无关,
从而说明它是\(M_{s \times n}(K)\)的一个基.
\end{proof}
\end{example}

\begin{example}
%@see: 《高等代数(第三版 下册)》(丘维声) P77 例9
数域\(K\)上所有一元多项式形成的线性空间\(K[x]\)中,
子集\[
	\{1,x,x^2,\dotsc,x^n,\dotsc\}
\]是\(K[x]\)的一个基.
\begin{proof}
\(K[x]\)上每一个一元多项式\(f(x)\)
可以写成\(f(x)=a_0+a_1 x+a_2 x^2+\dotsb+a_n x^n\).
任取\(S\)的一个有限子集\(\{x^{i_1},\dotsc,x^{i_m}\}\).
设\(k_1 x^{i_1}+\dotsb+k_m x^{i_m}=0\),
则由一元多项式的定义得
\(k_1=\dotsb=k_m=0\),
从而这个子集线性无关,
因此\(S\)线性无关,
于是\(S\)是\(K[x]\)的一个基.
\end{proof}
\end{example}

\subsection{有限维线性空间,无限维线性空间}
\begin{definition}
%@see: 《高等代数(第三版 下册)》(丘维声) P77 定义8
%@see: 《Linear Algebra and Its Applications (Second Edition)》(Peter D. Lax) P5 Definition
%@see: 《Linear Algebra Done Right (Fourth Eidition)》(Sheldon Axler) P31 2.13
设\(V\)是域\(F\)上的一个线性空间,
\(S\)是\(V\)的一个基.
如果\(S\)是有限集,
则称“\(V\)是\DefineConcept{有限维的}(finite-dimensional)”;
否则称“\(V\)是\DefineConcept{无限维的}(infinite-dimensional)”.
\end{definition}

\begin{example}
数域\(K\)上全体\(s \times n\)矩阵\(M_{s \times n}(K)\)是有限维的.
\end{example}

\begin{example}
%@see: 《Linear Algebra Done Right (Fourth Eidition)》(Sheldon Axler) P31 2.14
数域\(K\)上全体一元多项式\(K[x]\)是无限维的.
\end{example}

\subsection{有限维线性空间的维数}
\begin{theorem}\label{theorem:线性空间.同一个线性空间的任意两个基的基数相等}
%@see: 《高等代数(第三版 下册)》(丘维声) P77 定理10
%@see: 《Linear Algebra and Its Applications (Second Edition)》(Peter D. Lax) P5 Theorem 3.
%@see: 《Linear Algebra Done Right (Fourth Eidition)》(Sheldon Axler) P44 2.34
% 原话是: Any two bases of a finite-dimensional vector space have the same length.
设\(V\)是域\(F\)上的一个线性空间.
如果\(V\)是有限维的,
则\(V\)的任意两个基的基数相等.
\begin{proof}
不妨设\(V\)有一个基包含有限多个向量\(\AutoTuple{\alpha}{n}\).
设\(S\)是\(V\)的另一个基.

假如\(\card S>n\),
则\(S\)中可取出\(n+1\)个向量\(\AutoTuple{\beta}{n+1}\),
它们可以由\(\AutoTuple{\alpha}{n}\)线性表出.
由\cref{theorem:线性空间.性质3},%引理1
可知\(\AutoTuple{\beta}{n+1}\)线性相关.
这与\(S\)线性无关矛盾,
因此\(\card S\leq n\).

设\(S=\{\AutoTuple{\beta}{m}\}\ (m\leq n)\),
由\cref{theorem:线性空间.性质3},%推论4
又可知\(m=n\).
\end{proof}
\end{theorem}

\begin{definition}
%@see: 《高等代数(第三版 下册)》(丘维声) P78 定义9
%@see: 《Linear Algebra Done Right (Fourth Eidition)》(Sheldon Axler) P44 2.35
设\(V\)是域\(F\)上的一个有限维线性空间,
则\(V\)的一个基的基数
称为“\(V\)的\DefineConcept{维数}(the \emph{dimension} of \(V\))”,
记作\(\dim_F V\),
简记为\(\dim V\).
\end{definition}

\begin{property}
零空间的维数为\(0\).
\begin{proof}
由\cref{theorem:线性空间的结构.零空间的基是空集} 立即可得.
\end{proof}
\end{property}

\begin{example}
\(\dim M_{s \times n}(K)=sn\).
\end{example}

\begin{example}
\(\dim M_n(K) = n^2\).
\end{example}

维数对于研究有限维线性空间的结构起着重要的作用.

\begin{property}
%@see: 《高等代数(第三版 下册)》(丘维声) P78 命题11
%@see: 《高等代数(第三版 下册)》(丘维声) P78 命题12
%@see: 《Linear Algebra and Its Applications (Second Edition)》(Peter D. Lax) P5 Lemma 1.
%@see: 《Linear Algebra Done Right (Fourth Eidition)》(Sheldon Axler) P35 2.22
%@see: 《Linear Algebra Done Right (Fourth Eidition)》(Sheldon Axler) P45 2.38
设\(V\)是域\(F\)上的一个线性空间.
\begin{itemize}
	\item 如果\(\dim V=n\),
	则\(V\)中任意\(n+1\)个向量都线性无关.

	\item 如果\(\dim V=n\),
	则\(V\)中任意\(n\)个线性无关的向量都是\(V\)的一个基.

	\item 如果\(V\)的有限子集\(\AutoTuple{\alpha}{s}\)是线性无关的,
	则\(s \leq n\).
\end{itemize}
%TODO proof
\end{property}

基对于研究线性空间的结构起着重要的作用.

\begin{property}\label{theorem:线性空间.任一向量可由给定基唯一线性表出}
%@see: 《高等代数(第三版 下册)》(丘维声) P78 命题13
%@see: 《Linear Algebra Done Right (Fourth Eidition)》(Sheldon Axler) P39 2.28
设\(V\)是域\(F\)上的一个线性空间,
\(\AutoTuple{\alpha}{n}\)是\(V\)的一个基,
则\(V\)中每一个向量\(\alpha\)
可以唯一地表成\(\AutoTuple{\alpha}{n}\)的线性组合.
\begin{proof}
从基的定义知道,任意一个向量\(\alpha\),均可由\(\AutoTuple{\alpha}{n}\)线性表出.
假设有如下两种表出方式:\begin{gather*}
	\alpha = x_1 \alpha_1 + x_2 \alpha_2 + \dotsb + x_n \alpha_n, \\
	\alpha = y_1 \alpha_1 + y_2 \alpha_2 + \dotsb + y_n \alpha_n.
\end{gather*}
相减得\[
	0 = (x_1 - y_1) \alpha_1 + (x_2 - y_2) \alpha_2 + \dotsb + (x_n - y_n) \alpha_n.
\]
由于\(\AutoTuple{\alpha}{n}\)线性无关,
所以\begin{equation*}
	x_1 - y_1
	= x_2 - y_2
	= \dotsb
	= x_n - y_n
	= 0.
\end{equation*}
由此可见表出方式唯一.
\end{proof}
\end{property}

\begin{example}\label{example:线性空间.生成子空间等于线性空间的向量组就是基}
%@see: 《高等代数(第三版 下册)》(丘维声) P82 习题8.1 10.
证明:在数域\(K\)上的\(n\)维线性空间\(V\)中,
如果每一个向量都可以由\(\AutoTuple{\alpha}{n}\)线性表出,
则\(\AutoTuple{\alpha}{n}\)是\(V\)的一个基.
\begin{proof}
在\(V\)中取一个基\(\AutoTuple{\delta}{n}\).
由题设条件可知\(\AutoTuple{\delta}{n}\)可以由\(\AutoTuple{\alpha}{n}\)线性表出,
那么由\cref{theorem:线性空间.向量组的秩的性质} 可知\[
	\rank\{\AutoTuple{\delta}{n}\}
	\leq
	\rank\{\AutoTuple{\alpha}{n}\}.
\]
又因为\(\AutoTuple{\alpha}{n}\)可以由\(\AutoTuple{\delta}{n}\)线性表出,
从而\[
	\rank\{\AutoTuple{\alpha}{n}\}
	\leq
	\rank\{\AutoTuple{\delta}{n}\}.
\]
于是\(\rank\{\AutoTuple{\alpha}{n}\}
= \rank\{\AutoTuple{\delta}{n}\}
= n\),
那么\(\AutoTuple{\alpha}{n}\)线性无关,
因此\(\AutoTuple{\alpha}{n}\)是\(V\)的一个基.
\end{proof}
%@see: 《高等代数(第三版 上册)》(丘维声) P80 习题3.4 4.
\end{example}

\subsection{向量的坐标,过渡矩阵}
%@see: 《高等代数(第三版 下册)》(丘维声) P78
我们把向量\(\alpha\)由基\(\AutoTuple{\alpha}{n}\)线性表出的系数
组成的\(n\)元有序组\((\AutoTuple{\alpha}{n})\)
称为“向量\(\alpha\)在基\(\AutoTuple{\alpha}{n}\)下的\DefineConcept{坐标}(coordinate)”.
通常把向量的坐标写成列向量形式.

由上可知,有限维线性空间\(V\)中给定一个基,
则\(V\)中每一个向量都可以唯一地表示成这个基的线性组合,
从而\(V\)的结构就很清楚了.
因此,基是研究线性空间的结构的第一条途径.

\(n\)维线性空间\(V\)中给定两个基,
我们想要知道,\(V\)中每一个向量分别在这两个基下的坐标有什么关系.

设\(\AutoTuple{\alpha}{n}\)和\(\AutoTuple{\beta}{n}\)是\(V\)的两个基,
\(V\)中向量\(\alpha\)在这两个基下的坐标分别为\[
	X=(\AutoTuple{x}{n})^T, \qquad
	Y=(\AutoTuple{y}{n})^T.
\]
为了求\(X\)与\(Y\)之间的关系,
首先把这两个基之间的关系搞清楚.
由于\(\AutoTuple{\alpha}{n}\)是\(V\)的一个基,
因此有\[
%@see: 《高等代数(第三版 下册)》(丘维声) P79 (2)
	\left\{ \begin{array}{l}
		\beta_1=a_{11} \alpha_1+a_{21} \alpha_2+\dotsb+a_{n1} \alpha_n, \\
		\beta_2=a_{12} \alpha_1+a_{22} \alpha_2+\dotsb+a_{n2} \alpha_n, \\
		\hdotsfor1, \\
		\beta_n=a_{1n} \alpha_1+a_{2n} \alpha_2+\dotsb+a_{nn} \alpha_n.
	\end{array} \right.
\]
为了使推导过程简洁,
我们可以把上式写成\[
%@see: 《高等代数(第三版 下册)》(丘维声) P79 (5)
	(\AutoTuple{\beta}{n})
	=
	(\AutoTuple{\alpha}{n})
	A,
\]
其中\[
	A=\begin{bmatrix}
		a_{11} & a_{12} & \dots & a_{1n} \\
		a_{21} & a_{22} & \dots & a_{2n} \\
		\vdots & \vdots & & \vdots \\
		a_{n1} & a_{n2} & \dots & a_{nn}
	\end{bmatrix}.
\]
我们把\(A\)称为
“基\(\AutoTuple{\alpha}{n}\)到基\(\AutoTuple{\beta}{n}\)的\DefineConcept{过渡矩阵}”.
%@see: https://mathworld.wolfram.com/TransitionMatrix.html
%@see: https://mathworld.wolfram.com/ChangeofCoordinatesMatrix.html

在这里,我们引入一种形式写法\[
%@see: 《高等代数(第三版 下册)》(丘维声) P79 (3)
	x_1 \alpha_1 + \dotsb + x_n \alpha_n
	\defeq
	(\AutoTuple{\alpha}{n})
	\begin{bmatrix}
		x_1 \\
		\vdots \\
		x_n
	\end{bmatrix}.
\]
像这样的形式写法,是模仿矩阵乘法的定义.
因此,类似于矩阵乘法的结合律、左右分配律、乘法与数量乘法的关系的证明方法,
可以证明形式写法满足以下规则.

设\(\AutoTuple{\alpha}{n}\)与\(\AutoTuple{\beta}{n}\)是\(V\)中的两个向量组,
\(A,B\)是域\(F\)上的两个\(n\)阶矩阵,
数\(k \in F\),
则\begin{gather*}
	%@see: 《高等代数(第三版 下册)》(丘维声) P79 (6)
	[(\AutoTuple{\alpha}{n}) A] B
	= (\AutoTuple{\alpha}{n}) (A B), \\
	%@see: 《高等代数(第三版 下册)》(丘维声) P79 (7)
	(\AutoTuple{\alpha}{n}) A
	+ (\AutoTuple{\alpha}{n}) B
	= (\AutoTuple{\alpha}{n}) (A + B), \\
	%@see: 《高等代数(第三版 下册)》(丘维声) P79 (8)
	(\AutoTuple{\alpha}{n}) A
	+ (\AutoTuple{\beta}{n}) A
	= (\alpha_1+\beta_1,\dotsc,\alpha_n+\beta_n) A, \\
	%@see: 《高等代数(第三版 下册)》(丘维声) P79 (9)
	[k (\AutoTuple{\alpha}{n})] A
	= (\AutoTuple{\alpha}{n}) (k A), \\
	[k (\AutoTuple{\alpha}{n})] A
	= k [(\AutoTuple{\alpha}{n}) A],
\end{gather*}
其中\begin{gather*}
	%@see: 《高等代数(第三版 下册)》(丘维声) P79 (10)
	(\AutoTuple{\alpha}{n})
	+ (\AutoTuple{\beta}{n})
	\defeq
	(\alpha_1+\beta_1,\dotsc,\alpha_n+\beta_n), \\
	%@see: 《高等代数(第三版 下册)》(丘维声) P79 (11)
	k (\AutoTuple{\alpha}{n})
	= (\AutoTuple{k \alpha}{n}).
\end{gather*}

\begin{proposition}\label{theorem:线性空间.命题14}
%@see: 《高等代数(第三版 下册)》(丘维声) P80 命题14
设\(\AutoTuple{\alpha}{n}\)是\(V\)的一个基,
且\((\AutoTuple{\beta}{n})=(\AutoTuple{\alpha}{n})A\),
则\(\AutoTuple{\beta}{n}\)是\(V\)的一个基
当且仅当\(A\)是可逆矩阵.
\begin{proof}
由于\(\AutoTuple{\alpha}{n}\)线性无关,
并且有\begin{align*}
	k_1 \beta_1+\dotsb+k_n \beta_n
	&=(\AutoTuple{\beta}{n}) (\AutoTuple{k}{n})^T \\
	&=(\AutoTuple{\alpha}{n}) A (\AutoTuple{k}{n})^T,
\end{align*}
因此\begin{align*}
	&\text{$\AutoTuple{\beta}{n}$是$V$的一个基}
	\iff \text{$\AutoTuple{\beta}{n}$线性无关} \\
	&\iff
	k_1 \beta_1+\dotsb+k_n \beta_n=0
	\implies
	k_1=\dotsb=k_n=0 \\
	&\iff
	(\AutoTuple{\alpha}{n}) A (\AutoTuple{k}{n})^T=0
	\implies
	(\AutoTuple{k}{n})^T=0 \\
	&\iff
	A (\AutoTuple{k}{n})^T=0
	\implies
	(\AutoTuple{k}{n})^T=0 \\
	&\iff \text{齐次线性方程组$AX=0$只有零解} \\
	&\iff \abs{A}\neq0
	\iff \text{$A$是可逆矩阵}.
	\qedhere
\end{align*}
\end{proof}
\end{proposition}

\cref{theorem:线性空间.命题14} 表明:
基\(\AutoTuple{\alpha}{n}\)到基\(\AutoTuple{\beta}{n}\)的过渡矩阵是可逆矩阵.

现在可以给出向量\(\alpha\)
分别在基\(\AutoTuple{\alpha}{n}\)
与基\(\AutoTuple{\beta}{n}\)下的坐标\(X,Y\)之间的关系.
由于\[
	\alpha
	=(\AutoTuple{\alpha}{n}) X
	=(\AutoTuple{\beta}{n}) Y,
\]
并且基\(\AutoTuple{\alpha}{n}\)到基\(\AutoTuple{\beta}{n}\)的过渡矩阵是\(A\),
因此\[
	(\AutoTuple{\alpha}{n}) X
	=(\AutoTuple{\beta}{n}) Y
	=(\AutoTuple{\alpha}{n}) A Y.
\]
由于同一个向量由基\(\AutoTuple{\alpha}{n}\)线性表出的方式唯一,
从上式得\[
%@see: 《高等代数(第三版 下册)》(丘维声) P79 (12)
	X=AY,
\]
从而\[
%@see: 《高等代数(第三版 下册)》(丘维声) P79 (13)
	Y=A^{-1}X.
\]

\begin{example}
设\(\alpha_1,\alpha_2,\alpha_3\)是\(\mathbb{R}^3\)的一组基,
求:基\(\alpha_1,\frac12\alpha_2,\frac13\alpha_3\)
到基\(\alpha_1+\alpha_2,\alpha_2+\alpha_3,\alpha_3+\alpha_1\)的过渡矩阵.
\begin{solution}
设所求过渡矩阵为\(P\),
则根据定义有\[
	\begin{bmatrix}
		\alpha_1 & \frac12\alpha_2 & \frac13\alpha_3
	\end{bmatrix} P
	= \begin{bmatrix}
		\alpha_1+\alpha_2 & \alpha_2+\alpha_3 & \alpha_3+\alpha_1
	\end{bmatrix},
\]
即\[
	\begin{bmatrix}
		\alpha_1 & \alpha_2 & \alpha_3
	\end{bmatrix}
	\begin{bmatrix}
		1 \\
		& \frac12 \\
		&& \frac13
	\end{bmatrix} P
	= \begin{bmatrix}
	\alpha_1 & \alpha_2 & \alpha_3
	\end{bmatrix}
	\begin{bmatrix}
		1 & 0 & 1 \\
		1 & 1 & 0 \\
		0 & 1 & 1
	\end{bmatrix},
\]
所以\[
	P = \begin{bmatrix}
		1 \\
		& \frac12 \\
		&& \frac13
	\end{bmatrix}^{-1}
	\begin{bmatrix}
		1 & 0 & 1 \\
		1 & 1 & 0 \\
		0 & 1 & 1
	\end{bmatrix}
	= \begin{bmatrix}
		1 \\
		& 2 \\
		&& 3
	\end{bmatrix} \begin{bmatrix}
		1 & 0 & 1 \\
		1 & 1 & 0 \\
		0 & 1 & 1
	\end{bmatrix}
	= \begin{bmatrix}
		1 & 0 & 1 \\
		2 & 2 & 0 \\
		0 & 3 & 3
	\end{bmatrix}.
\]
\end{solution}
\end{example}

\begin{example}
%@see: 《高等代数(第三版 下册)》(丘维声) P81 习题8.1 4.
把复数域\(\mathbb{C}\)看成实数域\(\mathbb{R}\)上的线性空间,
求它的一个基和维数,
以及每个复数在这个基下的坐标.
\begin{solution}
把复数域看成实数域上的线性空间\(V\),
容易看出,有限集\(S = \{1,\iu\}\)是线性空间\(V\)的一个基,
它的维数为\(\dim V = \card S = 2\),
而每个复数\(z = a + b\iu\)在这个基下的坐标为\((a,b)^T\).
\end{solution}
\end{example}

\begin{example}
%@see: 《高等代数(第三版 下册)》(丘维声) P81 习题8.1 5.
把数域\(K\)看成自身上的线性空间,求它的一个基和维数.
\begin{solution}
把数域\(K\)看成自身上的线性空间\(V\),
容易看出,\(S = \{1\}\)是线性空间\(V\)的一个基,
它的维数为\(\dim V = \card S = 1\).
\end{solution}
\end{example}

\begin{example}
%@see: 《高等代数(第三版 下册)》(丘维声) P81 习题8.1 11.
设\(X = \{\AutoTuple{x}{n}\}\),\(F\)是一个域.
把映射空间\(F^X\)看成域\(F\)上的一个线性空间,
求\(F^X\)的一个基和维数,
再求映射\(f \in F^X\)在这个基下的坐标.
\begin{solution}
任意给定\(f \in F^X\),
必有\(f = \Set{
	(x_1,f(x_1)),
	\dotsc,
	(x_n,f(x_n))
}\).

令\[
	f_i(x_j) \defeq \delta(i,j),
	\quad i,j=1,2,\dotsc,n,
\]
其中\(\delta\)是克罗内克\(\delta\)函数,
即\[
	\delta(i,j) = \left\{ \begin{array}{cl}
		1, & i = j, \\
		0, & i \neq j.
	\end{array} \right.
\]
那么\[
	f(x) = f(x_1) f_1(x) + \dotsb + f(x_n) f_n(x),
	\quad x \in X,
	\eqno(1)
\]
这就说明,\(f\)可以由\(\AutoTuple{f}{n}\)线性表出.

显然\(\AutoTuple{f}{n}\)线性无关,
那么\(\AutoTuple{f}{n}\)是\(F^X\)的一个基,
从而有\(\dim F^X = n\).

由(1)式可知,
函数\(f\)在基\(\AutoTuple{f}{n}\)下的坐标为
\((f(x_1),\dotsc,f(x_n))\).
\end{solution}
\end{example}

\subsection{线性空间的笛卡尔和}
\begin{definition}
%@see: 《Linear Algebra and Its Applications (Second Edition)》(Peter D. Lax) P10 Definition
设\(V,W\)都是域\(F\)上的线性空间.
把\[
	\Set{
		(v,w)
		\given
		v \in V,
		w \in W
	}
\]称为“线性空间\(V\)和\(W\)的\DefineConcept{笛卡尔和}(Cartesian sum)”,
记作\(V \CartesianSum W\).
\end{definition}

\begin{theorem}
%@see: 《Linear Algebra and Its Applications (Second Edition)》(Peter D. Lax) P10
设\(V,W\)都是域\(F\)上的线性空间,
则线性空间\(V\)和\(W\)的笛卡尔和\(V \CartesianSum W\)是域\(F\)上的线性空间.
%TODO proof
\end{theorem}

\begin{theorem}\label{theorem:线性空间.笛卡尔和的维数公式}
%@see: 《Linear Algebra and Its Applications (Second Edition)》(Peter D. Lax) P11 Exercise 18.
设\(V,W\)都是域\(F\)上的线性空间,
则线性空间\(V\)和\(W\)的笛卡尔和\(V \CartesianSum W\)满足\[
	\dim(V \CartesianSum W) = \dim V + \dim W.
\]
%TODO proof
\end{theorem}
