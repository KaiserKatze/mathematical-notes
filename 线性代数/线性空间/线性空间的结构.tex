\section{线性空间的结构}
\subsection{线性空间的概念与性质}
\begin{definition}
%@see: 《高等代数(第三版 下册)》(丘维声) P72 定义1
%@see: 《Linear Algebra and Its Applications (Second Edition)》(Peter D. Lax) P1
设\(V\)是一个非空集合,
\(F\)是一个域.

定义运算\(V\times V\to V,\opair{\a,\b}\mapsto\g=\a+\b\),
称之为\DefineConcept{加法}(addition).

定义运算\(F\times V\to V,\opair{k,\a}\mapsto\b=k\a\),
称之为\DefineConcept{纯量乘法}(scalar multiplication).

如果\emph{加法}与\emph{纯量乘法}满足以下八条公理:
\begin{center}
	\begin{minipage}{.8\textwidth}
		\begin{axiom}
		%@see: 《Linear Algebra and Its Applications (Second Edition)》(Peter D. Lax) P2 (2)
		\((\forall\a,\b\in V)
		[\a+\b=\b+\a]\).
		\end{axiom}
		\begin{axiom}
		%@see: 《Linear Algebra and Its Applications (Second Edition)》(Peter D. Lax) P2 (3)
		\((\forall\a,\b,\g\in V)
		[(\a+\b)+\g=\a+(\b+\g)]\).
		\end{axiom}
		\begin{axiom}
		%@see: 《Linear Algebra and Its Applications (Second Edition)》(Peter D. Lax) P2 (4)
		\(\vb0\in V
		\land
		(\forall\a \in V)
		[\a+\vb0=\a]\),
		把\(\vb0\)称为“\(V\)的\DefineConcept{零元}”.
		\end{axiom}
		\begin{axiom}
		%@see: 《Linear Algebra and Its Applications (Second Edition)》(Peter D. Lax) P2 (5)
		\((\forall\a \in V)
		(\exists \vb\eta \in V)
		[\a+\vb\eta=\vb0]\),
		\(\vb\eta\)称为“\(\a\)的\DefineConcept{负元}”,
		记作\(-\a\).
		\end{axiom}
		\begin{axiom}
		%@see: 《Linear Algebra and Its Applications (Second Edition)》(Peter D. Lax) P2 (9)
		\((\forall\a\in V)[1\a=\a]\),
		其中\(1\)是\(F\)的单位元.
		\end{axiom}
		\begin{axiom}
		%@see: 《Linear Algebra and Its Applications (Second Edition)》(Peter D. Lax) P2 (6)
		\((\forall\a\in V)
		(\forall k,l\in F)
		[k(l\a)=(kl)\a]\).
		\end{axiom}
		\begin{axiom}
		%@see: 《Linear Algebra and Its Applications (Second Edition)》(Peter D. Lax) P2 (8)
		\((\forall\a\in V)
		(\forall k,l\in F)
		[(k+l)\a=k\a+l\a]\).
		\end{axiom}
		\begin{axiom}
		%@see: 《Linear Algebra and Its Applications (Second Edition)》(Peter D. Lax) P2 (7)
		\((\forall\a,\b\in V)
		(\forall k\in F)
		[k(\a+\b)=k\a+k\b]\).
		\end{axiom}
	\end{minipage}
\end{center}
则称“\(V\)是域\(F\)上的\DefineConcept{线性空间}%
(\(V\) is a \emph{linear space} over field \(F\))”,
把\(V\)中的元素称为\DefineConcept{向量}(vector),
把加法与纯量乘法这两种运算统称为\DefineConcept{线性运算}.
\end{definition}
\begin{remark}
与我们在初等代数中学习的自然数、整数、有理数、实数的加法、乘法不同,
线性空间的加法、纯量乘法是抽象的,
线性空间的加法可以是映射空间\(V^{V \times V}\)中的任意一个映射,
线性空间的纯量乘法可以是映射空间\(V^{F \times V}\)中的任意一个映射.
另外,向量空间的加法、纯量乘法和域\(F\)的加法、乘法毫无关系.
\end{remark}
\begin{remark}
线性空间\(V\)对加法成群.
\end{remark}

当\(F = \mathbb{R}\)时,把线性空间\(V\)称为\DefineConcept{实线性空间}.
当\(F = \mathbb{C}\)时,把线性空间\(V\)称为\DefineConcept{复线性空间}.

\begin{example}
下面列举一些常见的线性空间:\begin{itemize}
	\item 只含零元的线性空间\(\{\vb0\}\),
	称为\DefineConcept{零空间}.

	\item 实线性空间与复线性空间
	是代数结构完全不同的两个线性空间.

	复数域\(\mathbb{C}\)
	可以看成是实数域\(\mathbb{R}\)上的一个线性空间,
	其加法是复数的加法,
	其数量乘法是实数与复数的乘法.

	任一数域\(K\)都可以看成是自身上的线性空间.

	\item 集合\(\mathbb{R}^{n \times 1}\)关于向量的加法、实数与向量的纯量乘法构成实线性空间.

	\item 集合\(\mathbb{R}^{s \times n}\)关于矩阵的加法、实数与矩阵的纯量乘法构成实线性空间.

	\item 映射空间\(\mathbb{R}^X\)
	对函数的加法,以及实数与函数的数量乘法,
	成为实线性空间.

	特别地,一元多项式环\(K[x]\)
	对多项式的加法,以及数与多项式的乘法,
	成为实线性空间.
\end{itemize}
\end{example}

\begin{property}
%@see: 《高等代数(第三版 下册)》(丘维声) P74
设\(V\)是域\(F\)上的一个线性空间.
\begin{itemize}
	\item \(V\)的零元是唯一的.
	\item \(V\)中每个元素的负元是唯一的.
	%@see: 《Linear Algebra and Its Applications (Second Edition)》(Peter D. Lax) P2 (10)
	\item \((\forall\a\in V)[0\a=\vb0]\).
	\item \((\forall k\in F)[k\vb0=\vb0]\).
	\item \(k\a=\vb0 \implies k=0 \lor \a=\vb0\).
	\item \((\forall\a\in V)[(-1)\a=-\a]\).
\end{itemize}
\end{property}

\subsection{线性空间的线性关系}
设\(\AutoTuple{\a}{s}\)是\(V\)中一个向量组,
任给\(F\)中一组元素\(\AutoTuple{k}{s}\),
向量\(k_1\a_1+\dotsb+k_s\a_s\)
称为“\(\AutoTuple{\a}{s}\)的一个\DefineConcept{线性组合}”,
称\(\AutoTuple{k}{s}\)为\DefineConcept{系数}.

对于\(\b\in V\),
如果有\(F\)中一组元素\(\AutoTuple{c}{s}\),
使得\(\b=c_1\a_1+\dotsb+c_s\a_s\),
则称“\(\b\)可以由\(\AutoTuple{\a}{s}\)~\DefineConcept{线性表出}”.

\begin{definition}
%@see: 《高等代数(第三版 下册)》(丘维声) P75 定义2
设\(\AutoTuple{\a}{s}\ (s\geq1)\)是\(V\)中一个向量组.
如果有\(F\)中不全为零的元素\(\AutoTuple{k}{s}\),
使得\(k_1\a_1+\dotsb+k_s\a_s=0\),
则称“\(\AutoTuple{\a}{s}\)是\DefineConcept{线性相关的}”;
否则称“\(\AutoTuple{\a}{s}\)是\DefineConcept{线性无关的}”.
\end{definition}

空向量组\(\emptyset\)是线性无关的.

\begin{definition}
%@see: 《高等代数(第三版 下册)》(丘维声) P75 定义3
设\(W\)是\(V\)的任一无限子集.
如果\(W\)有一个有限子集是线性相关的,
则称“\(W\)是\DefineConcept{线性相关的}”;
如果\(W\)的任何有限子集都是线性无关的,
则称“\(W\)是\DefineConcept{线性无关的}”.
\end{definition}

可以证明,
数域\(K\)上的线性方程组的理论,
和数域\(K\)上的矩阵、行列式理论,
在把数域\(K\)换成任意域\(F\)以后,
仍然成立.
\begin{property}
%@see: 《高等代数(第三版 下册)》(丘维声) P75 例6
%@see: 《高等代数(第三版 下册)》(丘维声) P75 例7
%@see: 《高等代数(第三版 下册)》(丘维声) P75 命题1
%@see: 《高等代数(第三版 下册)》(丘维声) P75 命题2
设\(V\)是域\(F\)上的一个线性空间.
\begin{itemize}
	\item \(\text{$\a$线性相关}\iff\a=\vb0\).
	\item 包含零向量的向量组一定线性相关.
	\item 基数大于或等于\(2\)的向量组\(W\)线性相关
	当且仅当\(W\)中至少有一个向量可以由其余向量中的有限多个线性表出.
	\item 向量\(\b\)可以由线性无关向量组\(\AutoTuple{\a}{s}\)线性表出的充分必要条件是
	\(\AutoTuple{\a}{s},\b\)线性相关.
\end{itemize}
\end{property}

\begin{definition}
%@see: 《高等代数(第三版 下册)》(丘维声) P76 定义4
设\(W_1,W_2\)都是\(V\)的非空子集,
如果\(W_1\)中每一个向量都可以由\(W_2\)中有限多个向量线性表出,
则称“\(W_1\)可以由\(W_2\)~\DefineConcept{线性表出}”.
如果\(W_1\)与\(W_2\)可以互相线性表出,
则称“\(W_1\)与\(W_2\)是\DefineConcept{等价的}”.
\end{definition}

容易证明,“线性表出”具有传递性,
从而“等价”也具有传递性.
显然,向量组的“等价”具有反身性与对称性.

\begin{property}\label{theorem:线性空间.性质3}
%@see: 《高等代数(第三版 下册)》(丘维声) P76 引理1
%@see: 《高等代数(第三版 下册)》(丘维声) P76 推论3
%@see: 《高等代数(第三版 下册)》(丘维声) P76 推论4
设\(V\)是域\(F\)上的一个线性空间.
\begin{itemize}
	\item 设向量组\(\AutoTuple{\b}{r}\)
	可以由向量组\(\AutoTuple{\a}{s}\)线性表出.
	\begin{itemize}
		\item 如果\(r>s\),
		那么向量组\(\AutoTuple{\b}{r}\)线性相关.

		\item 如果\(\AutoTuple{\b}{r}\)线性无关,
		则\(r\leq s\).
	\end{itemize}

	\item 等价的线性无关的向量组所含向量的个数相等.
\end{itemize}
\end{property}

\subsection{线性空间的基与维数}
\begin{definition}
%@see: 《高等代数(第三版 下册)》(丘维声) P76 定义5
设\(a\)是向量组\(A\)的部分组.
如果\(a\)是线性无关的,
但是对于\(\forall\b \in A-a\)
总有\(a \cup \{\b\}\)是线性相关的,
则称“\(a\)是一个\DefineConcept{极大线性无关组}”.
\end{definition}

\begin{property}
%@see: 《高等代数(第三版 下册)》(丘维声) P76 推论5
%@see: 《高等代数(第三版 下册)》(丘维声) P76 推论6
设\(V\)是域\(F\)上的一个线性空间.
\begin{itemize}
	\item 向量组与它的极大线性无关组等价.
	\item 向量组的任意两个极大线性无关组的基数相等.
\end{itemize}
\end{property}

\begin{definition}
%@see: 《高等代数(第三版 下册)》(丘维声) P76 定义6
向量组\(A=\{\AutoTuple{\a}{s}\}\)的一个极大线性无关组的基数,
称为这个向量组的\DefineConcept{秩},
记为\(\rank A\)或\(\rank\{\AutoTuple{\a}{s}\}\).
\end{definition}

\begin{property}
%@see: 《高等代数(第三版 下册)》(丘维声) P76 命题8
%@see: 《高等代数(第三版 下册)》(丘维声) P76 命题9
%@see: 《高等代数(第三版 下册)》(丘维声) P76 推论9
设\(V\)是域\(F\)上的一个线性空间.
\begin{itemize}
	\item 全由零向量组成的向量组的秩为零.

	\item 向量组线性无关的充分必要条件是
	它的秩等于它的基数.

	\item 设\(A,B\)都是向量组.
	如果\(A\)可以由\(B\)线性表出,
	则\(\rank A \leq \rank B\).

	\item 等价的向量组有相同的秩.
\end{itemize}
\end{property}

\begin{definition}\label{definition:线性空间.线性空间的基}
%@see: 《高等代数(第三版 下册)》(丘维声) P76 定义7
设\(V\)是域\(F\)上的一个线性空间,\(S \subseteq V\).
如果\begin{itemize}
	\item \(S\)线性无关,
	\item \(V\)中每一个向量都可以由\(S\)中有限多个向量线性表出,
\end{itemize}
则称“\(S\)是\(V\)的一个\DefineConcept{基}(basis)”.
\end{definition}
\begin{remark}
%@credit: {647826c9-7e2a-49d1-b176-cd39b299b349} 说:除了 Hamel 基以外,还有 Schauder 基等其他定义
%@credit: {85841724-e8e0-4a39-88bf-973ade1b5e13} 说:参考《代数学(一)》(李方、邓少强、冯荣权、刘东文) P101 定义5.1.2
在\hyperref[definition:线性空间.线性空间的基]{基的定义}中,
必须要注意第二个条件中“有限多个”这个限定,
它说明这里定义的基是\emph{哈莫基}(Hamel basis).
%@see: https://zh.wikipedia.org/wiki/%E5%9F%BA_(%E7%B7%9A%E6%80%A7%E4%BB%A3%E6%95%B8)
%@see: https://en.wikipedia.org/wiki/Basis_(linear_algebra)
\end{remark}

\begin{property}\label{theorem:线性空间的结构.零空间的基是空集}
%@see: 《高等代数(第三版 下册)》(丘维声) P77
零空间的基是空集.
\end{property}

\begin{property}
%@see: 《高等代数(第三版 下册)》(丘维声) P77
域\(F\)上的任一线性空间\(V\)都有基.
\end{property}

\begin{example}
%@see: 《高等代数(第三版 下册)》(丘维声) P77 例8
在数域\(K\)上全体\(s \times n\)矩阵形成的线性空间\(M_{s \times n}(K)\)中,
所有基本矩阵组成的子集\[
	\Set{
		\E_{11},\E_{12},\dotsc,\E_{1n},
		\dotsc,
		\E_{s1},\E_{s2},\dotsc,\E_{sn}
	}
\]是\(M_{s \times n}(K)\)的一个基.
\begin{proof}
每个\(s \times n\)矩阵\(\A = (a_{ij})_{s \times n}\)都可以表示成\[
	\A = \sum_{i=1}^s \sum_{j=1}^n a_{ij} \E_{ij}.
\]

假设\[
	\sum_{i=1}^s \sum_{j=1}^n a_{ij} \E_{ij} = \vb0,
\]
则矩阵\(\A = (a_{ij})_{s \times n}\)是零矩阵,
从而\(a_{ij} = 0\ (i=1,2,\dotsc,s;j=1,2,\dotsc,n)\).
因此\[
	\Set{
		\E_{11},\E_{12},\dotsc,\E_{1n},
		\dotsc,
		\E_{s1},\E_{s2},\dotsc,\E_{sn}
	}
\]线性无关,
从而说明它是\(M_{s \times n}(K)\)的一个基.
\end{proof}
\end{example}

\begin{example}
%@see: 《高等代数(第三版 下册)》(丘维声) P77 例9
数域\(K\)上所有一元多项式形成的线性空间\(K[x]\)中,
子集\[
	\{1,x,x^2,\dotsc,x^n,\dotsc\}
\]是\(K[x]\)的一个基.
\begin{proof}
\(K[x]\)上每一个一元多项式\(f(x)\)
可以写成\(f(x)=a_0+a_1 x+a_2 x^2+\dotsb+a_n x^n\).
任取\(S\)的一个有限子集\(\{x^{i_1},\dotsc,x^{i_m}\}\).
设\(k_1 x^{i_1}+\dotsb+k_m x^{i_m}=0\),
则由一元多项式的定义得
\(k_1=\dotsb=k_m=0\),
从而这个子集线性无关,
因此\(S\)线性无关,
于是\(S\)是\(K[x]\)的一个基.
\end{proof}
\end{example}

\begin{definition}
%@see: 《高等代数(第三版 下册)》(丘维声) P77 定义8
设\(V\)是域\(F\)上的一个线性空间,
\(S\)是\(V\)的一个基.
如果\(S\)是有限集,
则称“\(V\)是\DefineConcept{有限维的}”;
否则称“\(V\)是\DefineConcept{无限维的}”.
\end{definition}

可以看出,
数域\(K\)上全体\(s \times n\)矩阵\(M_{s \times n}(K)\)是有限维的,
数域\(K\)上全体一元多项式\(K[x]\)是无限维的.

\begin{theorem}
%@see: 《高等代数(第三版 下册)》(丘维声) P77 定理10
设\(V\)是域\(F\)上的一个线性空间.
如果\(V\)是有限维的,
则\(V\)的任意两个基的基数相等.
\begin{proof}
不妨设\(V\)有一个基包含有限多个向量\(\AutoTuple{\a}{n}\).
设\(S\)是\(V\)的另一个基.

假如\(\card S>n\),
则\(S\)中可取出\(n+1\)个向量\(\AutoTuple{\b}{n+1}\),
它们可以由\(\AutoTuple{\a}{n}\)线性表出.
由\cref{theorem:线性空间.性质3},%引理1
可知\(\AutoTuple{\b}{n+1}\)线性相关.
这与\(S\)线性无关矛盾,
因此\(\card S\leq n\).

设\(S=\{\AutoTuple{\b}{m}\}\ (m\leq n)\),
由\cref{theorem:线性空间.性质3},%推论4
又可知\(m=n\).
\end{proof}
\end{theorem}

\begin{definition}
%@see: 《高等代数(第三版 下册)》(丘维声) P78 定义9
设\(V\)是域\(F\)上的一个有限维线性空间,
则\(V\)的一个基的基数
称为“\(V\)的\DefineConcept{维数}”,
记作\(\dim_F V\),
简记为\(\dim V\).
\end{definition}

零空间的维数为\(0\).

\(\dim M_{s \times n}(K)=sn\).

维数对于研究有限维线性空间的结构起着重要的作用.

\begin{property}
%@see: 《高等代数(第三版 下册)》(丘维声) P78 命题11
%@see: 《高等代数(第三版 下册)》(丘维声) P78 命题12
设\(V\)是域\(F\)上的一个线性空间.
\begin{itemize}
	\item 如果\(\dim V=n\),
	则\(V\)中任意\(n+1\)个向量都线性无关.

	\item 如果\(\dim V=n\),
	则\(V\)中任意\(n\)个线性无关的向量都是\(V\)的一个基.
\end{itemize}
\end{property}

基对于研究线性空间的结构起着重要的作用.

\begin{property}
%@see: 《高等代数(第三版 下册)》(丘维声) P78 命题13
设\(V\)是域\(F\)上的一个线性空间,
\(\AutoTuple{\a}{n}\)是\(V\)的一个基,
则\(V\)中每一个向量\(\a\)
可以唯一地表成\(\AutoTuple{\a}{n}\)的线性组合.
\end{property}

\subsection{向量的坐标,过渡矩阵}
我们把向量\(\a\)由基\(\AutoTuple{\a}{n}\)线性表出的系数
组成的\(n\)元有序组\((\AutoTuple{\a}{n})\)
称为“向量\(\a\)在基\(\AutoTuple{\a}{n}\)下的\DefineConcept{坐标}(coordinate)”.
通常把向量的坐标写成列向量形式.

由上可知,有限维线性空间\(V\)中给定一个基,
则\(V\)中每一个向量都可以唯一地表示成这个基的线性组合,
从而\(V\)的结构就很清楚了.
因此,基是研究线性空间的结构的第一条途径.

\(n\)维线性空间\(V\)中给定两个基,
我们想要知道,\(V\)中每一个向量分别在这两个基下的坐标有什么关系.

设\(\AutoTuple{\a}{n}\)和\(\AutoTuple{\b}{n}\)是\(V\)的两个基,
\(V\)中向量\(\a\)在这两个基下的坐标分别为\[
	\vb{X}=(\AutoTuple{x}{n})^T, \qquad
	\vb{Y}=(\AutoTuple{y}{n})^T.
\]
为了求\(\vb{X}\)与\(\vb{Y}\)之间的关系,
首先把这两个基之间的关系搞清楚.
由于\(\AutoTuple{\a}{n}\)是\(V\)的一个基,
因此有\[
	\left\{ \begin{array}{l}
		\b_1=a_{11} \a_1+a_{21} \a_2+\dotsb+a_{n1} \a_n, \\
		\b_2=a_{12} \a_1+a_{22} \a_2+\dotsb+a_{n2} \a_n, \\
		\hdotsfor1, \\
		\b_n=a_{1n} \a_1+a_{2n} \a_2+\dotsb+a_{nn} \a_n.
	\end{array} \right.
\]
为了使推导过程简洁,
我们可以把上式写成\[
	(\AutoTuple{\b}{n})
	=
	(\AutoTuple{\a}{n})
	\vb{A},
\]
其中\[
	\vb{A}=\begin{bmatrix}
		a_{11} & a_{12} & \dots & a_{1n} \\
		a_{21} & a_{22} & \dots & a_{2n} \\
		\vdots & \vdots & & \vdots \\
		a_{n1} & a_{n2} & \dots & a_{nn}
	\end{bmatrix}.
\]
我们把\(\vb{A}\)称为
“基\(\AutoTuple{\a}{n}\)到基\(\AutoTuple{\b}{n}\)的\DefineConcept{过渡矩阵}”.
%@see: https://mathworld.wolfram.com/TransitionMatrix.html
%@see: https://mathworld.wolfram.com/ChangeofCoordinatesMatrix.html

\begin{proposition}\label{theorem:线性空间.命题14}
%@see: 《高等代数(第三版 下册)》(丘维声) P80 命题14
设\(\AutoTuple{\a}{n}\)是\(V\)的一个基,
且\((\AutoTuple{\b}{n})=(\AutoTuple{\a}{n})\vb{A}\),
则\(\AutoTuple{\b}{n}\)是\(V\)的一个基
当且仅当\(\vb{A}\)是可逆矩阵.
\begin{proof}
由于\(\AutoTuple{\a}{n}\)线性无关,
并且有\begin{align*}
	k_1 \b_1+\dotsb+k_n \b_n
	&=(\AutoTuple{\b}{n}) (\AutoTuple{k}{n})^T \\
	&=(\AutoTuple{\a}{n}) \vb{A} (\AutoTuple{k}{n})^T,
\end{align*}
因此\begin{align*}
	&\text{$\AutoTuple{\b}{n}$是$V$的一个基}
	\iff \text{$\AutoTuple{\b}{n}$线性无关} \\
	&\iff
	k_1 \b_1+\dotsb+k_n \b_n=\vb0
	\implies
	k_1=\dotsb=k_n=0 \\
	&\iff
	(\AutoTuple{\a}{n}) \vb{A} (\AutoTuple{k}{n})^T=\vb0
	\implies
	(\AutoTuple{k}{n})^T=\vb0 \\
	&\iff
	\vb{A} (\AutoTuple{k}{n})^T=\vb0
	\implies
	(\AutoTuple{k}{n})^T=\vb0 \\
	&\iff \text{齐次线性方程组$\vb{A}\vb{X}=\vb0$只有零解} \\
	&\iff \abs{\vb{A}}\neq0
	\iff \text{$\vb{A}$是可逆矩阵}.
	\qedhere
\end{align*}
\end{proof}
\end{proposition}

\cref{theorem:线性空间.命题14} 表明:
基\(\AutoTuple{\a}{n}\)到基\(\AutoTuple{\b}{n}\)的过渡矩阵是可逆矩阵.

现在可以给出向量\(\a\)
分别在基\(\AutoTuple{\a}{n}\)
与基\(\AutoTuple{\b}{n}\)下的坐标\(\vb{X},\vb{Y}\)之间的关系.
由于\[
	\a
	=(\AutoTuple{\a}{n}) \vb{X}
	=(\AutoTuple{\b}{n}) \vb{Y},
\]
并且基\(\AutoTuple{\a}{n}\)到基\(\AutoTuple{\b}{n}\)的过渡矩阵是\(\vb{A}\),
因此\[
	(\AutoTuple{\a}{n}) \vb{X}
	=(\AutoTuple{\b}{n}) \vb{Y}
	=(\AutoTuple{\a}{n}) \vb{A} \vb{Y}.
\]
由于同一个向量由基\(\AutoTuple{\a}{n}\)线性表出的方式唯一,
从上式得\[
	\vb{X}=\vb{A}\vb{Y},
\]
从而\[
	\vb{Y}=\vb{A}^{-1}\vb{X}.
\]

\begin{example}
设\(\a_1,\a_2,\a_3\)是\(\mathbb{R}^3\)的一组基,
求:基\(\a_1,\frac12\a_2,\frac13\a_3\)
到基\(\a_1+\a_2,\a_2+\a_3,\a_3+\a_1\)的过渡矩阵.
\begin{solution}
设所求过渡矩阵为\(\P\),
则根据定义有\[
	\begin{bmatrix}
		\a_1 & \frac12\a_2 & \frac13\a_3
	\end{bmatrix} \P
	= \begin{bmatrix}
		\a_1+\a_2 & \a_2+\a_3 & \a_3+\a_1
	\end{bmatrix},
\]
即\[
	\begin{bmatrix}
		\a_1 & \a_2 & \a_3
	\end{bmatrix}
	\begin{bmatrix}
		1 \\
		& \frac12 \\
		&& \frac13
	\end{bmatrix} \P
	= \begin{bmatrix}
	\a_1 & \a_2 & \a_3
	\end{bmatrix}
	\begin{bmatrix}
		1 & 0 & 1 \\
		1 & 1 & 0 \\
		0 & 1 & 1
	\end{bmatrix},
\]
所以\[
	\P = \begin{bmatrix}
		1 \\
		& \frac12 \\
		&& \frac13
	\end{bmatrix}^{-1}
	\begin{bmatrix}
		1 & 0 & 1 \\
		1 & 1 & 0 \\
		0 & 1 & 1
	\end{bmatrix}
	= \begin{bmatrix}
		1 \\
		& 2 \\
		&& 3
	\end{bmatrix} \begin{bmatrix}
		1 & 0 & 1 \\
		1 & 1 & 0 \\
		0 & 1 & 1
	\end{bmatrix}
	= \begin{bmatrix}
		1 & 0 & 1 \\
		2 & 2 & 0 \\
		0 & 3 & 3
	\end{bmatrix}.
\]
\end{solution}
\end{example}
