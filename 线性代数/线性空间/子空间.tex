\section{子空间及其运算}
数域\(K\)上\(n\)元齐次线性方程组\(A X = 0\)的解空间\(W\)是\(K^n\)的子空间,
意思是齐次线性方程组的解集\(W\)对于有序数组的加法与数量乘法封闭.

数域\(K\)上全体\(n\)阶矩阵\(M_n(K)\)对于矩阵的加法与数量乘法形成\(K\)上的一个线性空间.
\(K\)上全体\(n\)阶对称矩阵\(V\)对于矩阵的加法与数量乘法也形成一个线性空间.
显然\(V\)是\(M_n(K)\)的子集,
并且\(V\)的加法就是\(M_n(K)\)的加法,
\(V\)的数量乘法就是\(M_n(K)\)的数量乘法.
我们自然地希望把\(V\)叫做\(M_n(K)\)的一个子空间.

本节将介绍任意线性空间的子空间的概念、子空间的运算,以及研究如何利用子空间来探索线性空间的结构.

\subsection{子空间}
\begin{definition}
%@see: 《高等代数(第三版 下册)》(丘维声) P82 定义1
%@see: 《Linear Algebra and Its Applications (Second Edition)》(Peter D. Lax) P4 Definition
%@see: 《Linear Algebra Done Right (Fourth Edition)》(Sheldon Axler) P18 1.33
设\(V\)是域\(F\)上的一个线性空间,
\(\emptyset\neq U\subseteq V\).
如果\(U\)对于\(V\)的加法及纯量乘法运算
也形成\(F\)上的线性空间,
则称“\(U\)是\(V\)的一个\DefineConcept{子空间}(\(U\) is a \emph{subspace} of \(V\))”,
记作\(U \AlgebraSubstructure V\).
% A subset \(U\) of a linear space \(V\) is called a \emph{subspace} if sums and scalar multiples of elements of \(U\) belong to \(U\).
\end{definition}

%@see: 《Linear Algebra and Its Applications (Second Edition)》(Peter D. Lax) P4 Exercise 8.
显然\(\{0\}\)是\(V\)的一个子空间,
%@see: 《Linear Algebra Done Right (Fourth Edition)》(Sheldon Axler) P19
% The set \(\{0\}\) is the smallest subspace of V.
同时它也是\(V\)的最小子空间,
称其为“\(V\)的\DefineConcept{零子空间}”,
也记作\(0\).
另外,\(V\)显然也是\(V\)的一个子空间.
%@see: 《Linear Algebra Done Right (Fourth Edition)》(Sheldon Axler) P19
% \(V\) itself is the largest subspace of V.
同时它也是\(V\)的最大子空间,
我们把\(0\)和\(V\)统称为“\(V\)的\DefineConcept{平凡子空间}(trivial subspace)
\footnote{
	有的书仅仅将\emph{平凡子空间}定义为零子空间.
}”,
把\(V\)的其余子空间称为它的\DefineConcept{非平凡子空间}.

空集\(\emptyset\)不是\(V\)的子空间.

\begin{definition}
设\(U\)是域\(F\)上线性空间\(V\)的一个子集.
如果成立\begin{equation*}
	(\forall u_1,u_2\in U)[u_1+u_2 \in U],
\end{equation*}
则称“\(U\)对于\(V\)的加法封闭(\(U\) is closed under addition)”.
\end{definition}
\begin{definition}
设\(U\)是域\(F\)上线性空间\(V\)的一个子集.
如果成立\begin{equation*}
	(\forall u\in U)(\forall k\in F)[ku\in U],
\end{equation*}
则称“\(U\)对于\(V\)的纯量乘法封闭(\(U\) is closed under scalar multiplication)”.
\end{definition}

\begin{theorem}\label{theorem:线性空间.子空间的判定1}
%@see: 《高等代数(第三版 下册)》(丘维声) P82 定理1
%@see: 《Linear Algebra Done Right (Fourth Edition)》(Sheldon Axler) P18 1.34
域\(F\)上线性空间\(V\)的非空子集\(U\)是\(V\)的一个子空间
当且仅当\(U\)对于\(V\)的加法与纯量乘法都封闭.
\begin{proof}
必要性已由子空间的定义直接给出.
下面证明充分性.

由已知条件可知,\(V\)的加法与纯量乘法都是\(U\)的运算.
由于\(V\)是线性空间,
因此\(U\)的加法满足\hyperref[definition:线性空间.运算法则1]{交换律}、\hyperref[definition:线性空间.运算法则2]{结合律},
\(U\)的纯量乘法满足 \labelcref{definition:线性空间.运算法则5,definition:线性空间.运算法则6,definition:线性空间.运算法则7,definition:线性空间.运算法则8} 这4条运算法则.

% \cref{definition:线性空间.运算法则3}
由于\(U\)是非空集,
因此存在\(u \in U\).
由已知条件得\(0 u \in U\).
由于\(V\)是线性空间,因此\(0 u = 0\).
从而\(0 \in U\),
于是\(V\)的零元也是\(U\)的零元.

% \cref{definition:线性空间.运算法则4}
任取\(\alpha \in U\),
由已知条件得\((-1) \alpha \in U\).
由于\(V\)是线性空间,
因此\((-1) \alpha = -\alpha\),
从而\(-\alpha \in U\),
于是\(\alpha\)在\(V\)中的负元\(-\alpha\)也是\(\alpha\)在\(U\)中的负元.

综上所述,
\(U\)的加法、纯量乘法满足线性空间定义所要求的全部运算法则,
\(U\)是\(F\)上一个线性空间,
从而\(U\)是\(V\)的一个子空间.
\end{proof}
\end{theorem}
\begin{theorem}\label{theorem:线性空间.子空间的判定2}
%@see: 《Linear Algebra Done Right (Fourth Edition)》(Sheldon Axler) P18 1.34
域\(F\)上线性空间\(V\)的子集\(U\)是\(V\)的一个子空间
当且仅当
\(V\)的零元\(0\)属于\(U\),
且\(U\)对于\(V\)的加法与纯量乘法都封闭.
%TODO proof
\end{theorem}
\begin{remark}
应该注意到:
\cref{theorem:线性空间.子空间的判定1} 要求\(U\)是非空的,
而\cref{theorem:线性空间.子空间的判定2} 没有这个要求.
\end{remark}

\begin{proposition}\label{theorem:子空间.同一线性空间的两个子空间成立包含关系的必要条件}
设\(U,W\)是域\(F\)上\(n\)维线性空间\(V\)的两个子空间.
如果\(U \subseteq W\),
则\(U\)是\(W\)的一个子空间.
\end{proposition}

\begin{example}
%@see: 《高等代数(第三版 下册)》(丘维声) P83 例1
数域\(K\)上所有次数小于\(n\)的一元多项式组成的集合\(K[x]_n\)
是\(K[x]\)的一个子空间.
\begin{proof}
显然\(K[x]_n\)非空集.
由于两个次数小于\(n\)的一元多项式之和的次数仍小于\(n\),
且任一数\(k\)与一个次数小于\(n\)的一元多项式的乘积的次数仍小于\(n\),
因此\(K[x]_n\)对于多项式的加法与数量乘法都封闭,
从而\(K[x]_n\)是\(K[x]\)的一个子空间.
\end{proof}
\end{example}

\begin{example}
%@see: https://web.mit.edu/18.06/www/
%@see: https://youtu.be/l88D4r74gtM?t=1151
证明:数域\(K\)上所有\(n\)阶可逆矩阵的集合\(U\)不是\(M_n(K)\)的一个子空间.
\begin{proof}
因为零矩阵不是可逆矩阵,
所以\(U\)一定不是\(M_n(K)\)的子空间,
更何况,虽然\(
	I \defeq \begin{bmatrix}
		1 & 0 \\
		0 & 1
	\end{bmatrix}
\)和\(-I\)都是可逆矩阵,
但是\(I + (-I) = 0\)和\(0 I = 0\)都不是可逆的,
也就是说\(U\)对\(M_n(K)\)的加法、纯量乘法均不封闭.
\end{proof}
\end{example}

\begin{proposition}\label{theorem:线性空间.线性空间及其子空间的维数序关系}
%@see: 《高等代数(第三版 下册)》(丘维声) P83 命题2
%@see: 《Linear Algebra Done Right (Fourth Edition)》(Sheldon Axler) P45 2.37
设\(U\)是域\(F\)上\(n\)维线性空间\(V\)的一个子空间,
则\(\dim U \leq \dim V\).
\begin{proof}
由于\(n\)维线性空间\(V\)中任意\(n+1\)个向量都线性相关,
因此\(U\)的一个基所含向量的个数一定小于或等于\(n\),
从而\(\dim U \leq \dim V\).
\end{proof}
\end{proposition}

\begin{proposition}\label{theorem:子空间.与线性空间维数相等的子空间}
%@see: 《高等代数(第三版 下册)》(丘维声) P83 命题3
%@see: 《Linear Algebra Done Right (Fourth Edition)》(Sheldon Axler) P45 2.39
设\(U\)是域\(F\)上\(n\)维线性空间\(V\)的一个子空间.
如果\(\dim U=\dim V\),
则\(U=V\).
\begin{proof}
由于\(\dim U=\dim V=n\),
因此\(U\)的一个基\(\AutoTuple{\delta}{n}\)就是\(V\)的一个基,
从而\(V\)中任一向量\(\alpha=a_1\delta_1+\dotsb+a_n\delta_n\in U\),
因此\(V\subseteq U\).
又因为\(U\subseteq V\),
所以\(U=V\).
\end{proof}
\end{proposition}

\begin{proposition}
%@see: 《高等代数(第三版 下册)》(丘维声) P83 命题4
%@see: 《Linear Algebra Done Right (Fourth Edition)》(Sheldon Axler) P41 2.32
% 原话是: Every linearly independent list of vectors in a finite-dimensinal vector space can be extended to a basis of the vector space.
设\(U\)是域\(F\)上\(n\)维线性空间\(V\)的一个子空间,
则\(U\)的一个基可以扩充成\(V\)的一个基.
\begin{proof}
设\(\AutoTuple{\alpha}{s}\)是\(U\)的一个基,则\(s \leq n\).
如果\(s=n\),则\(\AutoTuple{\alpha}{n}\)是\(V\)的一个基.
下面设\(s<n\).
此时\(\AutoTuple{\alpha}{s}\)不是\(V\)的一个基,
于是\(V\)中至少有一个向量\(\beta_1\)
不能由\(\AutoTuple{\alpha}{s}\)线性表出,
从而\(\AutoTuple{\alpha}{s},\beta_1\)线性无关.
如果\(s+1=n\),
则已得到\(V\)的一个基.
如果\(s+1<n\),
则同理有\(\beta_2\in V\),
使得\(\AutoTuple{\alpha}{s},\beta_1,\beta_2\)线性无关.
依次递推,总能得到\(n\)个线性无关的向量
\(\AutoTuple{\alpha}{s},\AutoTuple{\beta}{r}\),
其中\(s+r=n\),
这就是\(V\)的一个基.
\end{proof}
\end{proposition}

\subsection{生成子空间}
%@see: 《高等代数(第三版 下册)》(丘维声) P83
如何构造域\(F\)上线性空间\(V\)的子空间?
在\(V\)中给了向量组\(\AutoTuple{\alpha}{s}\),
由它们的所有线性组合组成的集合\begin{equation*}
	U
	\defeq
	\Set{
		k_1\alpha_1+\dotsb+k_s\alpha_s
		\given
		\AutoTuple{k}{s}\in F
	}
\end{equation*}
是\(V\)的一个子空间,
称“\(U\)是由向量组\(\AutoTuple{\alpha}{s}\) \DefineConcept{生成的子空间}
%@see: 《Linear Algebra and Its Applications (Second Edition)》(Peter D. Lax) P4 Exercise 9.
%@see: 《Linear Algebra Done Right (Fourth Edition)》(Sheldon Axler) P29 2.4
%@see: 《Linear Algebra Done Right (Fourth Edition)》(Sheldon Axler) P29 2.7
(\(U\) is the subspace \emph{spanned by} \(\AutoTuple{\alpha}{s}\),
\(U\) is the \emph{span} of \(\AutoTuple{\alpha}{s}\))”
“向量组\(\AutoTuple{\alpha}{s}\)可以生成空间\(U\)
(\(\AutoTuple{\alpha}{s}\) \emph{span} \(U\))”,
称“\(\AutoTuple{\alpha}{s}\)是\(U\)的一个{生成向量组}(spanning list)”.
把由向量组\(\AutoTuple{\alpha}{s}\)生成的子空间\(U\)
记作\(\opair{\AutoTuple{\alpha}{s}}\)
或\(\Span\{\AutoTuple{\alpha}{s}\}\).

\begin{proposition}
%@see: 《Linear Algebra Done Right (Fourth Edition)》(Sheldon Axler) P29 2.4
由空集生成的子空间是零子空间,
即\(\Span\emptyset = 0\).
%TODO proof 利用连加号的定义证明
\end{proposition}

\begin{proposition}%\label{theorem:线性空间.生成子空间是包含向量组的最小子空间}
%@see: 《Linear Algebra Done Right (Fourth Edition)》(Sheldon Axler) P29 2.6
设\(V\)是域\(F\)上的一个线性空间.
由\(A \defeq \{\AutoTuple{\alpha}{s}\}\)
生成的子空间\(U \defeq \Span A\)
是\(V\)的一个子空间,
并且\(U\)是包含\(A\)的最小子空间,
即\begin{equation*}
	U = \bigcap\Set{ W \given W \AlgebraSubstructure V, W \supseteq A }.
\end{equation*}
\begin{proof}
由于\(0 = 0\alpha_1 + \dotsb + 0\alpha_s\),
所以\(0 \in U\).
由于\begin{equation*}
	(x_1 \alpha_1 + \dotsb + x_s \alpha_s)
	+ (y_1 \alpha_1 + \dotsb + y_s \alpha_s)
	= (x_1+y_1) \alpha_1 + \dotsb + (x_1+y_1) \alpha_s,
\end{equation*}
所以\(U\)对于\(V\)的加法封闭.
由于\begin{equation*}
	k (x_1 \alpha_1 + \dotsb + x_s \alpha_s)
	= k x_1 \alpha_1 + \dotsb + k x_s \alpha_s,
\end{equation*}
所以\(U\)对于\(V\)的纯量乘法封闭.
于是由\cref{theorem:线性空间.子空间的判定2} 可知,
\(U\)是\(V\)的一个子空间.

显然\(\alpha_j\ (j=1,2,\dotsc,s)\)可以由\(\AutoTuple{\alpha}{s}\)线性表出,
因此\(\alpha_j \in U\).
反过来,由于\(V\)的所有子空间都对\(V\)的加法和纯量乘法封闭,
所以对于\(V\)的任意一个子空间,只要它包括\(\AutoTuple{\alpha}{s}\),
那么它就包含\(U\).
因此\(U\)是\(V\)的所有子空间中
包含\(\AutoTuple{\alpha}{s}\)的最小子空间.
\end{proof}
\end{proposition}

\begin{proposition}
设\(A,B\)都是线性空间\(V\)中的向量组,
则“\(A\)与\(B\)等价”的充分必要条件是\begin{equation*}
	\Span A = \Span B.
\end{equation*}
\end{proposition}

由于任意一个向量组均可生成一个子空间,
所以这个向量组可能是线性相关的,
但是我们希望从中去除一些向量,
并获取子空间的一个基,
这就好像通过蒸发浓缩食盐水析出氯化钠晶体一样,
我们把这个计算过程称为\DefineConcept{约化}(reduce).
\begin{theorem}\label{theorem:线性空间.生成向量组的约化}
%@see: 《高等代数(第三版 下册)》(丘维声) P84 定理5
在域\(F\)上的线性空间\(V\)中,
如果\(U=\Span\{\AutoTuple{\alpha}{s}\}\),
则向量组\(\AutoTuple{\alpha}{s}\)的一个极大线性无关组是\(U\)的一个基,
从而\(\dim U=\rank\{\AutoTuple{\alpha}{s}\}\).
%TODO proof
\end{theorem}
\begin{remark}
%@see: 《Linear Algebra Done Right (Fourth Edition)》(Sheldon Axler) P40 2.30
% 原话是: Every spanning list in a vector space can be reduced to a basis of the vector space.
上述定理说明:
任意一个生成向量组均可约化为由它生成的子空间的一个基.
\end{remark}

\begin{proposition}
%@see: 《Linear Algebra Done Right (Fourth Edition)》(Sheldon Axler) P46 2.42
设\(V\)是一个\(n\)维线性空间.
如果\(V\)中一个向量组\(A \defeq \{\AutoTuple{\alpha}{n}\}\)可以生成\(V\),
那么\(A\)是\(V\)的一个基.
%TODO proof
\end{proposition}

从基的定义容易看出,
如果\(\AutoTuple{\delta}{r}\)是\(V\)的子空间\(U\)的一个基,
则\(U=\Span\{\AutoTuple{\delta}{r}\}\).
由此看出:
\begin{proposition}
%@see: 《高等代数(第三版 下册)》(丘维声) P84
\(V\)的任一有限维子空间都是由向量组生成的子空间.
\end{proposition}

\begin{proposition}
%@see: 《Linear Algebra Done Right (Fourth Edition)》(Sheldon Axler) P30 2.9
如果线性空间\(V\)可以由某个向量组生成,
那么\(V\)是有限维空间.
\end{proposition}

\begin{theorem}
%@see: 《Linear Algebra and Its Applications (Second Edition)》(Peter D. Lax) P6 Theorem 4.
域\(F\)上有限维线性空间\(V\)中的每一个线性无关向量组可以扩充为\(V\)的基.
\begin{proof}
设\(\AutoTuple{\alpha}{s}\)是域\(F\)上有限维线性空间\(V\)中的一个线性无关向量组.
如果\(\AutoTuple{\alpha}{s}\)不可以生成空间\(V\),
那么存在向量\(\beta_1\),不可以由\(\AutoTuple{\alpha}{s}\)线性表出.
记\(\alpha_{s+1} = \beta_1\).
如果\(\AutoTuple{\alpha}{s+1}\)不可以生成空间\(V\),
那么存在向量\(\beta_2\),不可以由\(\AutoTuple{\alpha}{s+1}\)线性表出.
记\(\alpha_{s+2} = \beta_2\).
循此往复,最终必定有\(\AutoTuple{\alpha}{n}\)可以生成空间\(V\),
其中\(n = \dim V\).
\end{proof}
\end{theorem}

\begin{theorem}\label{theorem:线性空间.子空间.有限维线性空间的子空间是有限维的}
%@see: 《Linear Algebra and Its Applications (Second Edition)》(Peter D. Lax) P6 Theorem 5.(a)
%@see: 《Linear Algebra Done Right (Fourth Edition)》(Sheldon Axler) P36 2.25
有限维线性空间的任意一个子空间都是有限维的.
%TODO proof
\end{theorem}

\begin{example}
求数域\(K\)上次数不大于\(3\)的四个一元多项式\begin{gather*}
	f_1(x) = x^3 - 2x^2 + 4x + 1, \\
	f_2(x) = 2x^3 - 3x^2 + 9x - 1, \\
	f_3(x) = x^3 + 6x - 5, \\
	f_4(x) = 2x^3 - 5x^2 + 7x + 5
\end{gather*}
生成的子空间的基与维数.
\begin{solution}
设\(V\)是数域\(K\)上所有次数不大于\(3\)的一元多项式组成的集合,
即\begin{equation*}
	V = \Span\{f_1(x),f_2(x),f_3(x),f_4(x)\}.
\end{equation*}
取\(V\)的标准基\(x^3,x^2,x,1\),
则\(f_1(x),f_2(x),f_3(x),f_4(x)\)在标准基下的坐标分别是\begin{equation*}
	\alpha_1 = (1,-2,4,1)^T, \\
	\alpha_2 = (2,-3,9,-1)^T, \\
	\alpha_3 = (1,0,6,-5)^T, \\
	\alpha_4 = (2,-5,7,5)^T.
\end{equation*}
令\begin{equation*}
	A \defeq (\alpha_1,\alpha_2,\alpha_3,\alpha_4)
	= \begin{bmatrix}
		1 & 2 & 1 & 2 \\
		-2 & -3 & 0 & -5 \\
		4 & 9 & 6 & 7 \\
		1 & -1 & -5 & 5 \\
	\end{bmatrix},
\end{equation*}
作初等行变换,得\begin{equation*}
	B \defeq \begin{bmatrix}
		1 & 0 & -3 & 4 \\
		0 & 1 & 2 & -1 \\
		0 & 0 & 0 & 0 \\
		0 & 0 & 0 & 0 \\
	\end{bmatrix},
\end{equation*}
于是\(f_1(x),f_2(x)\)是\(V\)的一个基,
因此\(\dim V = 2\).
\end{solution}
\end{example}

\subsection{子空间的交}
在几何空间\(V\)中,给定两个过原点的平面\(V_1,V_2\),
易知它们都是\(V\)的子空间,
它们的交线\(L = V_1 \cap V_2\)也是\(V\)的一个子空间.
于是我们大胆推测,在任意一个线性空间中,任意两个子空间的交集也是一个子空间.
\begin{theorem}
%@see: 《高等代数(第三版 下册)》(丘维声) P84 定理6
%@see: 《Linear Algebra and Its Applications (Second Edition)》(Peter D. Lax) P4 Definition
设\(V_1,V_2\)都是域\(F\)上线性空间\(V\)的子空间,
则\(V_1 \cap V_2\)也是\(V\)的子空间.
\begin{proof}
因为\(0\in V_1 \cap V_2\),
所以\(V_1 \cap v_2\)非空集.
设\(\alpha,\beta\in V_1 \cap V_2\),
则\(\alpha,\beta\in V_1\)且\(\alpha,\beta\in V_2\).
于是\(\alpha+\beta\in V_1\)且\(\alpha+\beta\in V_2\),
因此\(\alpha+\beta\in V_1 \cap V_2\),
\(V_1 \cap V_2\)对加法封闭.
同理可证\(V_1 \cap V_2\)对纯量乘法封闭.
综上所述\(V_1 \cap V_2\)是\(V\)的子空间.
\end{proof}
\end{theorem}

子空间的交适合交换律、结合律,
即\begin{equation*}
	V_1 \cap V_2
	=V_2 \cap V_1, \qquad
	(V_1 \cap V_2) \cap V_3
	=V_1 \cap (V_2 \cap V_3).
\end{equation*}
由结合律,我们知道\(V\)的若干个子空间的交
\(\bigcap_{i=1}^s V_i\)也是\(V\)的一个子空间.

子空间的交满足\begin{equation*}
	W \AlgebraSubstructure V
	\implies
	W \cap V = W.
\end{equation*}
%@credit: {DeepSeek}
这是因为\(W\)是\(V\)的子空间,从而有\(W \subseteq V\),
于是由\cref{equation:集合论.集合代数公式7-3} 可知\(W \cap V = W\).

\subsection{子空间的和}
现在我们想知道,在任意一个线性空间中,任意两个子空间的并集是不是子空间.
从上述几何空间的例子看出,
如果向量\(\alpha_1\)属于子空间\(V_1\),
向量\(\alpha_2\)属于子空间\(V_2\),
且两个向量都不属于交线\(L\),
则虽然\(\alpha_1,\alpha_2 \in V_1 \cup V_2\),
但是\(\alpha_1 + \alpha_2\)可能不属于\(V_1 \cup V_2\),
或者说\(V_1 \cup V_2\)可能对加法不封闭,
因此\(V_1 \cup V_2\)不是\(V\)的子空间.
下面我们给出严格证明.
\begin{proposition}
设\(V_1,V_2\)都是域\(F\)上线性空间\(V\)的子空间,
则\(V_1 \cup V_2\)不一定是\(V\)的子空间.
\begin{proof}
显然\(V_1=\Span\{(1,0,0),(0,1,0)\}\)
和\(V_2=\Span\{(0,1,0),(0,0,1)\}\)
都是几何空间\(V\)的子空间.
我们取\(\alpha=(1,1,0)\in V_1\),
再取\(\beta=(0,1,1)\in V_2\),
容易看出\(\alpha+\beta=(1,2,1)\)
虽然属于\(V\),
但是不属于\(V_1\),
也不属于\(V_2\),
即\(\alpha+\beta\notin V_1 \cup V_2\),
这就说明
\(V_1 \cup V_2\)对加法不封闭,
从而说明
\(V_1 \cup V_2\)不是\(V\)的子空间.
\end{proof}
\end{proposition}
那么,如果我们想构造一个包含\(V_1 \cup V_2\)的子空间,
这个子空间就应当含有\(V_1\)中任一向量\(\alpha_1\)与\(V_2\)中任一向量\(\alpha_2\)之和.
受此启发,我们应当考虑集合
\(\Set{ \alpha_1 + \alpha_2 \given \alpha_1 \in V_1,\alpha_2 \in V_2 }\).

\begin{definition}\label{definition:线性空间.子空间.子空间的和}
%@see: 《高等代数(第三版 下册)》(丘维声) P84 定理7
%@see: 《Linear Algebra and Its Applications (Second Edition)》(Peter D. Lax) P4 Definition
%@see: 《Linear Algebra Done Right (Fourth Edition)》(Sheldon Axler) P19 1.36
设\(V_1,V_2\)都是域\(F\)上线性空间\(V\)的子空间,
把\begin{equation*}
%@see: 《高等代数(第三版 下册)》(丘维声) P84 (1)
	\Set{ \alpha_1+\alpha_2 \given \alpha_1\in V_1,\alpha_2\in V_2 }
\end{equation*}
称为“\(V_1\)与\(V_2\)的\DefineConcept{和}(sum)”,
记作\(V_1+V_2\).
\end{definition}

\begin{definition}
设\(\AutoTuple{V}{s}\)都是域\(F\)上线性空间\(V\)的子空间,
把\begin{equation*}
	\Set*{ \sum_{i=1}^s \alpha_i \given \alpha_i \in V_i }
\end{equation*}
称为“\(\AutoTuple{V}{s}\)的\DefineConcept{和}(sum)”,
记作\(\sum_{i=1}^s V_i\).
\end{definition}

\begin{example}
%@see: 《Linear Algebra Done Right (Fourth Edition)》(Sheldon Axler) P20 1.38
设\(F\)是一个域,
则\begin{equation*}
	U \defeq \Set{
		(x,x,y,y)
		\given
		x,y \in F
	},
	\quad\text{和}\quad
	W \defeq \Set{
		(x,x,x,y)
		\given
		x,y \in F
	},
\end{equation*}
都是线性空间\(F^4\)的子空间,
且\begin{equation*}
	U+W = \Set{
		(a+c,a+c,b+c,b+d)
		\given
		a,b,c,d \in F
	}.
\end{equation*}
\end{example}

\begin{theorem}
%@see: 《高等代数(第三版 下册)》(丘维声) P84 定理7
设\(V_1,V_2\)都是域\(F\)上线性空间\(V\)的子空间,
则\(V_1+V_2\)是\(V\)的一个子空间.
\begin{proof}
由于\(0+0=0\),
所以\(0\in V_1+V_2\).
在\(V_1+V_2\)中任取两个向量\(\alpha,\beta\),
则\begin{equation*}
	\alpha=\alpha_1+\alpha_2, \qquad
	\beta=\beta_1+\beta_2,
\end{equation*}
其中\(\alpha_1,\beta_1\in V_1,
\alpha_2,\beta_2\in V_2\).
于是\(\alpha_1+\beta_1\in V_1,
\alpha_2+\beta_2\in V_2\).
因此\begin{equation*}
	\alpha+\beta
	=(\alpha_1+\alpha_2)+(\beta_1+\beta_2)
	=(\alpha_1+\beta_1)+(\alpha_2+\beta_2)
	\in V_1+V_2,
\end{equation*}
即\(V_1+V_2\)对于\(V\)的加法封闭.
同理可证\(V_1+V_2\)对于\(V\)的纯量乘法封闭,
因此\(V_1+V_2\)是\(V\)的一个子空间.
\end{proof}
\end{theorem}

\begin{proposition}
%@see: 《高等代数(第三版 下册)》(丘维声) P85
%@see: 《Linear Algebra Done Right (Fourth Edition)》(Sheldon Axler) P21 1.40
% \(V_1+V_2\) is the smallest subspace of \(V\) containing \(V_1,V_2\).
\(V_1+V_2\)是\(V\)中包含\(V_1 \cup V_2\)的最小子空间.
\begin{proof}
%@credit: {855486ab-2fcf-40c1-b774-09956dfb4012},{9fe3a491-385e-41e9-97cc-e0e3bc5c3b5b},{3d3dd0d4-9945-42ae-8a73-520c7170c8ac},{0e766461-e697-4ed8-95dd-e744fdd3194f},{a84f055e-f32d-418a-8d8c-0b72a4b2df78}
设\(U\)是\(V\)的子空间,
且\(U \supseteq V_1 \cup V_2\),
那么\begin{equation*}
	(\forall \alpha_1 \in V_1)
	(\forall \alpha_2 \in V_2)
	[
		\alpha_1 \in U
		\land
		\alpha_2 \in U
	].
	\eqno(1)
\end{equation*}
因为\(U\)是子空间,对加法封闭,
所以由(1)可得\begin{equation*}
	(\forall \alpha_1 \in V_1)
	(\forall \alpha_2 \in V_2)
	[
		\alpha_1 + \alpha_2 \in U
	].
	\eqno(2)
\end{equation*}
根据定义有\begin{equation*}
	(\forall \alpha \in V_1+V_2)
	(\exists \alpha_1 \in V_1)
	(\exists \alpha_2 \in V_2)
	[
		\alpha = \alpha_1 + \alpha_2
	].
	\eqno(3)
\end{equation*}
由(2)(3)可知\begin{equation*}
	(\forall \alpha \in V_1+V_2)
	[
		\alpha \in U
	],
\end{equation*}
则\(U \supseteq V_1+V_2\).
\end{proof}
\end{proposition}

%@see: 《高等代数(第三版 下册)》(丘维声) P85
子空间的和适合交换律、结合律,
即\begin{equation*}
	V_1 + V_2
	=V_2 + V_1, \qquad
	(V_1 + V_2) + V_3
	=V_1 + (V_2 + V_3).
\end{equation*}
由结合律,我们知道\(V\)的有限个子空间之和\begin{equation*}
%@see: 《高等代数(第三版 下册)》(丘维声) P85 (2)
	\sum_{i=1}^s V_i
	\defeq
	V_1 + \dotsb + V_s
	= \Set{
		\alpha_1 + \dotsb + \alpha_s
		\given
		\alpha_i \in V_i,
		i=1,2,\dotsc,s
	}
\end{equation*}也是\(V\)的一个子空间.

\begin{proposition}[吸收律]\label{theorem:子空间.子空间的和适合吸收律}
设\(W\)都是域\(F\)上线性空间\(V\)的子空间,
则\(W + V = V\).
\begin{proof}
%@credit: {DeepSeek}
这是因为\begin{itemize}
	\item 对于任意\(w \in W\)和\(v \in V\),
	必定有\(w + v \in W + V\);
	由于\(W\)是\(V\)的子空间,
	所以\(w \in V\);
	由于\(V\)对加法封闭,
	所以\(w + v \in V\);
	因此\(W + V \subseteq V\);

	\item 对于任意\(v \in V\),
	只要取\(w = 0 \in W\),
	就有\(v = 0 + v \in W + V\);
	因此\(V \subseteq W + V\);
\end{itemize}
综上所述\(W + V = V\).
\end{proof}
\end{proposition}

\begin{proposition}
%@see: 《高等代数(第三版 下册)》(丘维声) P85 命题8
设\(\AutoTuple{\alpha}{s}\)与\(\AutoTuple{\beta}{r}\)
是域\(F\)上线性空间\(V\)的两个向量组,
则\begin{equation*}
	\Span\{\AutoTuple{\alpha}{s}\}
	+\Span\{\AutoTuple{\beta}{r}\}
	=\Span\{\AutoTuple{\alpha}{s},\AutoTuple{\beta}{r}\}.
\end{equation*}
\begin{proof}
根据向量组生成的子空间的定义,以及子空间的和的定义,
得到\begin{align*}
	&\Span\{\AutoTuple{\alpha}{s}\}
	+\Span\{\AutoTuple{\beta}{r}\} \\
	&=\Set{
		(k_1\alpha_1+\dotsb+k_s\alpha_s)
		+(l_1\beta_1+\dotsb+l_r\beta_r)
		\given
		k_i,l_j\in F,
		1 \leq i \leq s,
		1 \leq j \leq r
	} \\
	&=\Span\{\AutoTuple{\alpha}{s},\AutoTuple{\beta}{r}\}.
	\qedhere
\end{align*}
\end{proof}
\end{proposition}

\begin{example}
试讨论:子空间的和、交是否满足分配律.
\begin{solution}
设\(V\)是域\(F\)上的线性空间,
\(V_1,V_2,V_3\)是\(V\)的子空间.

考虑命题\begin{gather*}
	(V_1 + V_2) \cap V_3
	= (V_1 \cap V_3) + (V_2 \cap V_3),
	\tag1 \\
	(V_1 \cap V_2) + V_3
	= (V_1 + V_3) \cap (V_2 + V_3).
	\tag2
\end{gather*}
%@credit: {腾讯元宝}
取\begin{equation*}
	% 坐标平面
	V = \mathbb{R}^2,
	\qquad
	% x轴
	V_1 = \Span\{(1,0)\},  % = \Set{ (x,0) \given x\in\mathbb{R} },
	\qquad
	% y轴
	V_2 = \Span\{(0,1)\},  % = \Set{ (0,y) \given y\in\mathbb{R} },
	\qquad
	% 直线\(y=x\)
	V_3 = \Span\{(1,1)\},  % = \Set{ (x,x) \given x\in\mathbb{R} },
\end{equation*}
则\begin{gather*}
	V_1 + V_2
	= V_1 + V_3
	= V_2 + V_3
	= V, \\
	V_1 \cap V_2
	= V_1 \cap V_3
	= V_2 \cap V_3
	= \{ (0,0) \}
	= 0, \\
	% (1)式等号左边
	(V_1 + V_2) \cap V_3
	= V \cap V_3
	= V_3, \\
	% (1)式等号右边
	(V_1 \cap V_3) + (V_2 \cap V_3)
	= 0 + 0
	= 0, \\
	% (2)式等号左边
	(V_1 \cap V_2) + V_3
	= 0 + V_3
	= V_3, \\
	% (2)式等号右边
	(V_1 + V_3) \cap (V_2 + V_3)
	= V \cap V
	= V,
\end{gather*}
显然(1)(2)式均不成立,
子空间的和、交不满足分配律!
\end{solution}
\end{example}
\begin{example}
设\(V\)是域\(F\)上的线性空间,
\(V_1,V_2,V_3\)是\(V\)的子空间.
证明:\begin{equation}
	(V_1 \cap V_3) + (V_2 \cap V_3) \subseteq (V_1 + V_2) \cap V_3.
\end{equation}
%TODO 有没有取等条件?
% 当\(V_1 \subseteq V_3\)或\(V_2 \subseteq V_3\)时,取“\(=\)”号.
\begin{proof}
%@credit: {腾讯元宝}
任取\(v \in (V_1 \cap V_3) + (V_2 \cap V_3)\),
必定存在\(v_1 \in V_1 \cap V_3\)和\(v_2 \in V_2 \cap V_3\),
使得\(v = v_1 + v_2\).

由\(v_1 \in V_1 \cap V_3\)可知\(v_1 \in V_1\),
由\(v_2 \in V_2 \cap V_3\)可知\(v_2 \in V_2\),
于是\(v = v_1 + v_2 \in V_1 + V_2\).

%@credit: {6c964576-9569-472e-969e-54699e35974b}
由\(v_1 \in V_1 \cap V_3\)可知\(v_1 \in V_3\),
由\(v_2 \in V_2 \cap V_3\)可知\(v_2 \in V_3\),
于是\(v = v_1 + v_2 \in V_3\).

由\(v \in V_1 + V_2\)
和\(v \in V_3\)
可知\(v \in (V_1 + V_2) \cap V_3\).

综上所述\begin{equation*}
	v \in (V_1 \cap V_3) + (V_2 \cap V_3)
	\implies
	v \in (V_1 + V_2) \cap V_3,
\end{equation*}
即\((V_1 \cap V_3) + (V_2 \cap V_3) \subseteq (V_1 + V_2) \cap V_3\).
\end{proof}
\end{example}
\begin{example}
设\(V\)是域\(F\)上的线性空间,
\(V_1,V_2,V_3\)是\(V\)的子空间.
证明:\begin{equation}
	(V_1 \cap V_2) + V_3 \subseteq (V_1 + V_3) \cap (V_2 + V_3).
\end{equation}
%TODO 有没有取等条件?
% 当\(V_3 \subseteq V_1\)或\(V_3 \subseteq V_2\)时,取“\(=\)”号.
\begin{proof}
%@credit: {腾讯元宝}
任取\(v \in (V_1 \cap V_2) + V_3\),
必定存在\(v_1 \in V_1 \cap V_2\)和\(v_2 \in V_3\),
使得\(v = v_1 + v_2\).

由\(v_1 \in V_1 \cap V_2\)
可知\(v_1 \in V_1\)且\(v_1 \in V_2\).

由\(v_1 \in V_1\)和\(v_2 \in V_3\)
可知\(v = v_1 + v_2 \in V_1 + V_3\).

由\(v_1 \in V_2\)和\(v_2 \in V_3\)
可知\(v = v_1 + v_2 \in V_2 + V_3\).

由\(v \in V_1 + V_3\)
和\(v \in V_2 + V_3\)
可知\(v \in (V_1 + V_3) \cap (V_2 + V_3)\).

综上所述\begin{equation*}
	v \in (V_1 \cap V_2) + V_3
	\implies
	v \in (V_1 + V_3) \cap (V_2 + V_3),
\end{equation*}
即\((V_1 \cap V_2) + V_3 \subseteq (V_1 + V_3) \cap (V_2 + V_3)\).
\end{proof}
\end{example}

\subsection{子空间的维数公式}
在几何空间\(V\)中,两个相交平面\(V_1,V_2\)的维数为\(\dim V_1 = \dim V_2 = 2\),
它们的交线的维数为\(\dim(V_1 \cap V_2) = 1\),
它们的和恰好就是\(V\),从而有\(\dim(V_1 + V_2) = 3\),
于是成立\begin{equation*}
	\dim V_1 + \dim V_2
	= \dim(V_1 + V_2) + \dim(V_1 \cap V_2).
\end{equation*}
这就让人不禁好奇,对于任意一个\(n\)维线性空间,上式是否成立?
回答是肯定的.
\begin{theorem}[子空间的维数公式]\label{theorem:线性空间.子空间.子空间的维数公式}
%@see: 《高等代数(第三版 下册)》(丘维声) P85 定理9
%@see: 《Linear Algebra and Its Applications (Second Edition)》(Peter D. Lax) P10 Theorem 7.
%@see: 《Linear Algebra Done Right (Fourth Edition)》(Sheldon Axler) P47 2.43
设\(V_1,V_2\)都是域\(F\)上线性空间\(V\)的有限维子空间,
则\(V_1 \cap V_2,V_1+V_2\)也都是有限维的子空间,
并且\begin{equation*}
%@see: 《高等代数(第三版 下册)》(丘维声) P85 (3)
	\dim V_1 + \dim V_2
	= \dim(V_1 + V_2)
	+ \dim(V_1 \cap V_2).
\end{equation*}
\begin{proof}
由于\(V_1 \cap V_2 \subseteq V_1\),
因此\(\dim(V_1 \cap V_2) \leq \dim V_1\).
设\begin{equation*}
	\dim V_1=n_1, \qquad
	\dim V_2=n_2, \qquad
	\dim(V_1 \cap V_2)=m.
\end{equation*}
在\(V_1 \cap V_2\)中取一个基\(\AutoTuple{\alpha}{m}\),
把它分别扩充成\(V_1,V_2\)的一个基:\begin{equation*}
	\AutoTuple{\alpha}{m},\AutoTuple{\beta}{n_1-m}, \qquad
	\AutoTuple{\alpha}{m},\AutoTuple{\gamma}{n_2-m},
\end{equation*}
于是\begin{align*}
	V_1+V_2
	&=\Span\{\AutoTuple{\alpha}{m},\AutoTuple{\beta}{n_1-m}\}
	+\Span\{\AutoTuple{\alpha}{m},\AutoTuple{\gamma}{n_2-m}\} \\
	&=\Span\{
		\AutoTuple{\alpha}{m},
		\AutoTuple{\beta}{n_1-m},
		\AutoTuple{\gamma}{n_2-m}
	\}.
\end{align*}
我们希望证明
\(\AutoTuple{\alpha}{m},
\AutoTuple{\beta}{n_1-m},
\AutoTuple{\gamma}{n_2-m}\)
是\(V_1+V_2\)的一个基,
从而得出\begin{align*}
	\dim(V_1+V_2)
	&=m+(n_1-m)+(n_2-m) \\
	&=n_1+n_2-m \\
	&=\dim V_1+\dim V_2-\dim(V_1 \cap V_2).
\end{align*}
假设等式\begin{equation*}
%@see: 《高等代数(第三版 下册)》(丘维声) P86 (4)
	k_1\alpha_1+\dotsb+k_m\alpha_m
	+p_1\beta_1+\dotsb+p_{n_1-m}\beta_{n_1-m}
	+q_1\gamma_1+\dotsb+q_{n_2-m}\gamma_{n_2-m}
	=0
	\eqno(1)
\end{equation*}成立,
则\begin{equation*}
%@see: 《高等代数(第三版 下册)》(丘维声) P86 (5)
	q_1\gamma_1+\dotsb+q_{n_2-m}\gamma_{n_2-m}
	=-k_1\alpha_1-\dotsb-k_m\alpha_m
	-p_1\beta_1-\dotsb-p_{n_1-m}\beta_{n_1-m}.
\end{equation*}
注意到上式左边的向量属于\(V_2\),
右边的向量属于\(V_1\),
从而左边的向量属于\(V_1 \cap V_2\),
因此它可由\(\AutoTuple{\alpha}{m}\)线性表出:\begin{equation*}
	q_1\gamma_1+\dotsb+q_{n_2-m}\gamma_{n_2-m}
	=l_1\alpha_1+\dotsb+l_m\alpha_m,
\end{equation*}
移项得\begin{equation*}
%@see: 《高等代数(第三版 下册)》(丘维声) P86 (6)
	l_1\alpha_1+\dotsb+l_m\alpha_m
	-q_1\gamma_1-\dotsb-q_{n_2-m}\gamma_{n_2-m}
	=0.
\end{equation*}
由于\(\AutoTuple{\alpha}{m},\AutoTuple{\gamma}{n_2-m}\)是\(V_2\)的一个基,
因此从上式得出\begin{equation*}
	l_1=\dotsb=l_m=0, \qquad
	q_1=\dotsb=q_{n_2-m}=0.
\end{equation*}
代入(1)式,得\begin{equation*}
	k_1\alpha_1+\dotsb+k_m\alpha_m
	+p_1\beta_1+\dotsb+p_{n_1-m}\beta_{n_1-m}
	=0.
\end{equation*}
同理可得\begin{equation*}
	k_1=\dotsb=k_m=0, \qquad
	p_1=\dotsb=p_{n_1-m}=0.
\end{equation*}
因此
\(\AutoTuple{\alpha}{m},
\AutoTuple{\beta}{n_1-m},
\AutoTuple{\gamma}{n_2-m}\)
线性无关.
\end{proof}
\end{theorem}
\begin{remark}
公式\(\dim V_1+\dim V_2
=\dim(V_1+V_2)
+\dim(V_1 \cap V_2)\)
可以类比于\(\card X_1+\card X_2
=\card(X_1 \cup X_2)
+\card(X_1 \cap X_2)\).
\end{remark}

\begin{corollary}\label{theorem:线性空间.子空间.子空间的维数公式.推论1}
%@see: 《高等代数(第三版 下册)》(丘维声) P86 推论10
设\(V_1,V_2\)都是域\(F\)上\(n\)维线性空间\(V\)的子空间,
则\begin{equation*}
	\dim(V_1+V_2)=\dim V_1+\dim V_2
	\iff
	V_1 \cap V_2=0.
\end{equation*}
\end{corollary}

\begin{corollary}\label{theorem:子空间.子空间的和的维数等于最大子空间维数的等价条件}
设\(V\)是域\(F\)上的一个线性空间,
\(U\)和\(W\)是\(V\)的两个子空间,
则\begin{equation*}
	\dim(U + W) = \max\{\dim U,\dim W\}
	\iff
	(U \subseteq W) \lor (U \supseteq W).
\end{equation*}
\begin{proof}
%@credit: {gemini}
先证必要性.
当\(U \subseteq W\)时,
由\cref{theorem:子空间.同一线性空间的两个子空间成立包含关系的必要条件}
可知\(U\)是\(W\)的一个子空间;
再由\cref{theorem:线性空间.线性空间及其子空间的维数序关系}
可知\(\dim U \leq \dim W\),
从而\(\max\{\dim U,\dim W\} = \dim W\);
最后由\cref{theorem:子空间.子空间的和适合吸收律}
可知\(U + W = W\),
从而\(\dim(U+W) = \dim W\),
因此\(\dim(U+W) = \max\{\dim U,\dim W\}\).
同理可知,当\(U \supseteq W\)时,
也成立\(\dim(U+W) = \max\{\dim U,\dim W\}\).
因此\(
	(U \subseteq W) \lor (U \supseteq W)
	\implies
	\dim(U + W) = \max\{\dim U,\dim W\}
\).

再证充分性.
假设\(\dim(U + W) = \max\{\dim U,\dim W\}\).
由\cref{theorem:线性空间.子空间.子空间的维数公式}
可知\(
	\dim(U + W)
	=  \dim U + \dim W - \dim(U \cap W)
\),
那么\(
	\dim(U \cap W)
	= \dim U + \dim W - \max\{\dim U,\dim W\}
	% \cref{equation:绝对值函数.两个数的最小值最大值的关联}
	= \min\{\dim U,\dim W\}
\).
考虑到\(U \cap W\)既是\(U\)的子空间,也是\(W\)的子空间.
当\(\dim U \leq \dim W\)时,
有\(\dim(U \cap W) = \min\{\dim U,\dim W\} = \dim U\),
那么由\cref{theorem:子空间.与线性空间维数相等的子空间}
可知\(U \cap W = U\),
从而\(U\)是\(W\)的子空间,即\(U \subseteq W\).
同理,当\(\dim W \leq \dim U\)时,
\(W\)是\(U\)的子空间,即\(U \supseteq W\).
因此\(
	\dim(U + W) = \max\{\dim U,\dim W\}
	\implies
	(U \subseteq W) \lor (U \supseteq W)
\).
\end{proof}
\end{corollary}

\begin{definition}
%@see: 《Linear Algebra and Its Applications (Second Edition)》(Peter D. Lax) P7 Definition
域\(F\)上\(n\)维线性空间\(V\)的每一个\(n-1\)维子空间,
都称为“\(V\)中的一个\DefineConcept{超平面}(hyperplane)”.
\end{definition}

\begin{example}
%@see: 《Linear Algebra and Its Applications (Second Edition)》(Peter D. Lax) P12 Exercise 21.
设\(U,V,W\)都是有限维线性空间\(X\)的子空间.
试判断命题\begin{align*}
	\dim(U+V+W)
	&= \dim U + \dim V + \dim W \\
	&\hspace{20pt} - \dim(U \cap V) - \dim (U \cap W) - \dim(V \cap W) \\
	&\hspace{20pt} + \dim(U \cap V \cap W)
\end{align*}
的真伪,并给出证明或反例.
\begin{solution}
%@credit: {腾讯元宝}
取\begin{equation*}
	% 坐标平面
	X = \mathbb{R}^2,
	\qquad
	% x轴
	U = \Span\{(1,0)\},
	\qquad
	% y轴
	V = \Span\{(0,1)\},
	\qquad
	% 直线\(y=x\)
	W = \Span\{(1,1)\},
\end{equation*}
则\begin{gather*}
	U + V + W = \mathbb{R}^2, \\
	U \cap V
	= U \cap W
	= V \cap W
	= U \cap V \cap W
	= 0, \\
	\dim(U + V + W) = 2,
	\qquad
	\dim U = \dim V = \dim W = 1, \\
	\dim(U \cap V)
	= \dim(U \cap W)
	= \dim(V \cap W)
	= \dim(U \cap V \cap W)
	= 0,
\end{gather*}
于是\begin{align*}
	\dim(U+V+W)
	&\neq \dim U + \dim V + \dim W \\
	&\hspace{20pt} - \dim(U \cap V) - \dim (U \cap W) - \dim(V \cap W) \\
	&\hspace{20pt} + \dim(U \cap V \cap W).
\end{align*}
\end{solution}
\end{example}

\subsection{子空间的直和、补空间}
假设在几何空间\(V\)中,\(V_1\)是过原点的一个平面,\(V_2\)是过原点的一条直线,
且直线\(V_2\)不在平面\(V_1\)上.
容易看出\(V_1 + V_2 = V\),
且\(V_1 + V_2\)中的每一个向量\(\alpha\)
都能被唯一地表示成\begin{equation*}
	\alpha = \alpha_1 + \alpha_2,
	\quad
	\alpha_1 \in V_1,
	\alpha_2 \in V_2.
\end{equation*}
受此启发,引出以下概念.
\begin{definition}
%@see: 《高等代数(第三版 下册)》(丘维声) P86 定义2
%@see: 《Linear Algebra and Its Applications (Second Edition)》(Peter D. Lax) P6 Definition
%@see: 《Linear Algebra Done Right (Fourth Edition)》(Sheldon Axler) P21 1.41
设\(V_1,V_2\)是域\(F\)上线性空间\(V\)的子空间,
\(U = V_1 + V_2\).
如果\begin{equation*}
	(\forall\alpha\in U)
	(\exists!\alpha_1\in V_1)
	(\exists!\alpha_2\in V_2)
	[\alpha=\alpha_1+\alpha_2],
\end{equation*}
则称“\(V_1+V_2\)是一个\DefineConcept{直和}(\(V_1+V_2\) is a \emph{direct sum})”
或“\(U\)是\(V_1\)和\(V_2\)的\DefineConcept{直和}%
(\(U\) is the \emph{direct sum} of \(V_1\) and \(V_2\))”,
记作\(U = V_1 \DirectSum V_2\).
\end{definition}

\begin{example}\label{example:线性空间.子空间.直和.例1}
在实数域上的线性空间\(\mathbb{R}^2\)中,
记\begin{equation*}
	V_1=\Span\{(1,0)\},
	\qquad
	V_2=\Span\{(0,1)\},
	\qquad
	V_3=\Span\{(1,1)\}
\end{equation*}
显然\(V_1+V_2\)和\(V_1+V_3\)都是直和.
这也说明:给定一个子空间\(V_1\),
满足条件“\(V_1+V_2\)是直和”的\(V_2\)不是唯一的.
\end{example}

\begin{theorem}\label{theorem:线性空间.子空间.直和的等价命题}
%@see: 《高等代数(第三版 下册)》(丘维声) P86 定理11
%@see: 《Linear Algebra Done Right (Fourth Edition)》(Sheldon Axler) P23 1.45
%@see: 《Linear Algebra Done Right (Fourth Edition)》(Sheldon Axler) P23 1.46
%@see: 《Linear Algebra Done Right (Fourth Edition)》(Sheldon Axler) P98 3.94
设\(V_1,V_2\)是域\(F\)上线性空间\(V\)的有限维子空间,
则下列命题互相等价:\begin{itemize}
	\item \(V_1+V_2\)是直和;
	\item \(V_1+V_2\)中零向量的表示法唯一;
	\item \(V_1 \cap V_2=0\);
	% \(V_1 \cap V_2=0\)中的\(0\)是零子空间
	\item \(\dim(V_1+V_2)=\dim V_1+\dim V_2\);
	\item \(V_1\)的一个基与\(V_2\)的一个基 合起来是\(V_1+V_2\)的一个基.
\end{itemize}
\begin{proof}
% (i) => (ii)
“\(V_1+V_2\)是直和”显然是“\(V_1+V_2\)中零向量的表示法唯一”的充分条件.
%TODO 还不理解
% 假设\(0\)有两种表示方法:\begin{gather*}
% 	0 = \alpha_1 + \alpha_2,
% 	\quad \alpha_1 \in V_1,
% 	\alpha_2 \in V_2, \\
% 	0 = \beta_1 + \beta_2,
% 	\quad \beta_1 \in V_1,
% 	\beta_1 \in V_2,
% \end{gather*}

% (ii) => (iii)
下面证明“\(V_1+V_2\)中零向量的表示法唯一”是“\(V_1 \cap V_2=0\)”的充分条件.
任取\(\alpha \in V_1 \cap V_2\),
于是零向量可以表示成\begin{equation*}
	0 = \alpha + (-\alpha),
	\quad
	\alpha \in V_1,
	-\alpha \in V_2.
\end{equation*}
由已知条件得\(\alpha = 0\).
因此\(V_1 \cap V_2 = 0\).

% (iii) => (i)
下面证明“\(V_1 \cap V_2=0\)”是“\(V_1+V_2\)是直和”的充分条件.
任取\(\alpha \in V_1 + V_2\),
假设\(\alpha\)有两种表示方法:\begin{gather*}
	\alpha = \alpha_1 + \alpha_2,
	\quad \alpha_1 \in V_1,
	\alpha_2 \in V_2, \\
	\alpha = \beta_1 + \beta_2,
	\quad \beta_1 \in V_1,
	\beta_2 \in V_2,
\end{gather*}
则\(\alpha_1 + \alpha_2
= \beta_1 + \beta_2\),
从而得到\(\alpha_1 - \beta_1
= \beta_2 - \alpha_2
\in V_1 \cap V_2\).
由于\(V_1 \cap V_2 = 0\),
因此\(\alpha_1 = \beta_1,
\alpha_2 = \beta_2\),
从而\(V_1 + V_2\)是直和.

% (iii) <=> (iv)
下面证明“\(V_1 \cap V_2=0\)”是“\(\dim(V_1+V_2)=\dim V_1+\dim V_2\)”的充分必要条件.
由\cref{theorem:线性空间.子空间.子空间的维数公式.推论1} 立即可得.

% (iv) => (v)
下面证明“\(\dim(V_1+V_2)=\dim V_1+\dim V_2\)”是“\(V_1\)的一个基与\(V_2\)的一个基 合起来是\(V_1+V_2\)的一个基”的充分条件.
设\(\AutoTuple{\alpha}{s}\)是\(V_1\)的一个基,
\(\AutoTuple{\beta}{r}\)是\(V_2\)的一个基,
则\begin{align*}
	V_1 + V_2
	&= \Span\{\AutoTuple{\alpha}{s}\} + \Span\{\AutoTuple{\beta}{r}\} \\
	&= \Span\{\AutoTuple{\alpha}{s},\AutoTuple{\beta}{r}\}.
\end{align*}
因为\(\dim(V_1 + V_2)
= \dim V_1 + \dim V_2
= s + r\),
且\(V_1 + V_2\)的每一个向量均可由\(\AutoTuple{\alpha}{s},\AutoTuple{\beta}{r}\)线性表出,
所以由\cref{example:线性空间.生成子空间等于线性空间的向量组就是基} 可知
\(\AutoTuple{\alpha}{s},\AutoTuple{\beta}{r}\)是由\(V_1 + V_2\)的一个基.

% (v) => (iv)
“\(\dim(V_1+V_2)=\dim V_1+\dim V_2\)”显然是“\(V_1\)的一个基与\(V_2\)的一个基 合起来是\(V_1+V_2\)的一个基”的必要条件.
\end{proof}
\end{theorem}
\begin{remark}
公式\(\dim(V_1+V_2)
= \dim V_1+\dim V_2
\iff V_1 \cap V_2 = 0\)
可以类比于\(\card(X_1 \cup X_2)
= \card X_1 + \card X_2
\iff X_1 \cap X_2 = \emptyset\).
\end{remark}

\begin{definition}
%@see: 《高等代数(第三版 下册)》(丘维声) P87
设\(V_1,V_2\)都是线性空间\(V\)的子空间,
如果\(V_1 \DirectSum V_2 = V\),
则称“\(V_1\)是\(V_2\)在\(V\)中的一个\DefineConcept{补空间}%
(\(V_1\) is a \emph{complement} of \(V_2\) in \(V\))”
“\(V_2\)是\(V_1\)在\(V\)中的一个{补空间}”
或“\(V_1,V_2\)互为{补空间}(\(V_1,V_2\) are complements of each other)”,
记\(V=V_1 \DirectSum V_2\).
\end{definition}

\begin{proposition}
%@see: 《高等代数(第三版 下册)》(丘维声) P87 命题12
%@see: 《Linear Algebra and Its Applications (Second Edition)》(Peter D. Lax) P6 Theorem 5.(b)
%@see: 《Linear Algebra Done Right (Fourth Edition)》(Sheldon Axler) P42 2.33
设\(V\)是域\(F\)上\(n\)维线性空间,
则\(V\)的每一个子空间\(U\)都有补空间.
\begin{proof}
从\(U\)中取一个基\(\AutoTuple{\alpha}{m}\),
把它扩充成\(V\)的一个基\(\AutoTuple{\alpha}{m},
\AutoTuple{\alpha}[m+1]{n}\),
则\begin{equation*}
	V=\Span\{\AutoTuple{\alpha}{m},\AutoTuple{\alpha}[m+1]{n}\}
	=\Span\{\AutoTuple{\alpha}{m}\}+\Span\{\AutoTuple{\alpha}[m+1]{n}\}
	=U+W,
\end{equation*}
其中\(W=\Span\{\AutoTuple{\alpha}[m+1]{n}\}\).
由于\(U\)的一个基与\(W\)的一个基合起来是\(V\)的一个基,
因此\(U+W\)是直和,
从\(V=U \DirectSum W\).
于是\(W\)是\(U\)的一个补空间.
\end{proof}
\end{proposition}

\begin{example}
举例说明:域\(F\)上线性空间\(V\)的子空间\(W\)在\(V\)中的补空间不唯一.
\begin{solution}
%@credit: {腾讯元宝}
取\begin{equation*}
	V = \mathbb{R}^3,
	\qquad
	W = \Span\{(1,0,0),(0,1,0)\},
	\qquad
	U_1 = \Span\{(0,0,1)\},
	\qquad
	U_2 = \Span\{(1,1,1)\}.
\end{equation*}
显然\begin{equation*}
	V
	= W \DirectSum U_1
	= W \DirectSum U_2,
\end{equation*}
这就说明\(U_1,U_2\)都是\(W\)在\(V\)中的补空间.
由于\(U_1 \neq U_2\),
所以\(W\)在\(V\)中的补空间不唯一.
\end{solution}
\end{example}
\begin{remark}
当\(V\)的子空间\(W\)是\(V\)的平凡子空间时,
\(W\)在\(V\)中的补空间是唯一的,
具体地说:
当\(W = V\)时,\(W\)的补空间是\(0\);
当\(W = 0\)时,\(W\)的补空间是\(V\).
\end{remark}
\begin{remark}
赋范线性空间\(V\)的任意一个子空间的正交补空间存在且唯一.
\end{remark}

\begin{example}
%@see: 《高等代数(第三版 下册)》(丘维声) P88 例2
设\(V=M_n(K)\),
其中\(K\)是数域.
分别用\(V_1,V_2\)表示\(K\)上所有\(n\)阶对称矩阵、反对称矩阵组成的子空间.
证明:\(V_1 \DirectSum V_2=V\).
\begin{proof}
首先证明\(V_1+V_2=V\).
显然有\(V_1+V_2\subseteq V\).
现在来证\(V\subseteq V_1+V_2\).
任取\(A\in V\),
有\begin{equation*}
	A=\frac{A+A^T}2+\frac{A-A^T}2.
\end{equation*}
容易验证\(\frac{A+A^T}2\)是对称矩阵,
\(\frac{A-A^T}2\)是反对称矩阵,
因此\(A\in V_1+V_2\).
从而\(V\subseteq V_1+V_2\),
所以\(V_1+V_2=V\).

然后证\(V_1+V_2\)是直和,
为此只要证\(V_1 \cap V_2=0\).
任取\(B \in V_1 \cap V_2\),
则\(B^T = B = -B\),
从而\(B = 0\).
因此\(V_1 \cap V_2=0\).

综上所述,\(V=V_1 \DirectSum V_2\).
\end{proof}
\end{example}

子空间的直和的概念可以推广到多个子空间的情形.
\begin{definition}
%@see: 《高等代数(第三版 下册)》(丘维声) P88 定义3
%@see: 《Linear Algebra and Its Applications (Second Edition)》(Peter D. Lax) P6 Definition
设\(\AutoTuple{V}{s}\)都是域\(F\)上线性空间\(V\)的子空间,
\(U\)也是\(V\)的一个子空间.
如果\(U\)中每一个向量\(\alpha\)可唯一地表示成\begin{equation*}
	\alpha = \alpha_1+\dotsb+\alpha_s,
	\quad \alpha_i \in V_i,
\end{equation*}
则称“\(V_1+\dotsb+V_s\)是一个\DefineConcept{直和}(\(V_1+\dotsb+V_s\) is a \emph{direct sum})”
或“\(U\)是\(\AutoTuple{V}{s}\)的\DefineConcept{直和}%
(\(U\) is the \emph{direct sum} of \(\AutoTuple{V}{s}\))”,
记作\(U = \BigDirectSum_{i=1}^s V_i\).
\end{definition}

\begin{theorem}
%@see: 《高等代数(第三版 下册)》(丘维声) P88 定理13
设\(\AutoTuple{V}{s}\)都是域\(F\)上线性空间\(V\)的有限维子空间,
则下列命题互相等价:\begin{itemize}
	\item \(\sum_{i=1}^s V_i\)是直和;
	\item \(\sum_{i=1}^s V_i\)中零向量的表示法唯一;
	\item \(V_i \cap \sum_{j\neq i} V_j=0\ (i=1,2,\dotsc,s)\);
	\item \(\dim\sum_{i=1}^s V_i=\sum_{i=1}^s\dim V_i\);
	\item \(V_i\ (i=1,2,\dotsc,s)\)的一个基 合起来是\(\sum_{i=1}^s V_i\)的一个基.
\end{itemize}
\end{theorem}

\begin{corollary}
%@see: 《高等代数(第三版 下册)》(丘维声) P88 推论14
设\(\AutoTuple{V}{s}\)都是域\(F\)上\(n\)维线性空间\(V\)的子空间,
则\begin{equation*}
	V=\BigDirectSum_{i=1}^s V_i
	\iff
	\text{\(V_i\ (i=1,2,\dotsc,s)\)的一个基,合起来是\(V\)的一个基}.
\end{equation*}
\end{corollary}

这个推论让我们可以利用子空间的运算来研究线性空间的结构,
它是研究线性空间的结构的第二条途径.

\begin{example}
%@see: 《高等代数(第三版 下册)》(丘维声) P89 例3
设\(V=K^4,
V_1=\Span\{\AutoTuple{\alpha}{3}\},
V_2=\Span\{\AutoTuple{\beta}{2}\}\),
其中\begin{equation*}
	\alpha_1=\begin{bmatrix} 1 \\ 2 \\ 1 \\ 0 \end{bmatrix},
	\alpha_2=\begin{bmatrix} -1 \\ 1 \\ 1 \\ 1 \end{bmatrix},
	\alpha_3=\begin{bmatrix} 0 \\ 3 \\ 2 \\ 1 \end{bmatrix},
	\beta_1=\begin{bmatrix} 2 \\ -1 \\ 0 \\ 1 \end{bmatrix},
	\beta_2=\begin{bmatrix} 1 \\ -1 \\ 3 \\ 7 \end{bmatrix},
\end{equation*}
分别求\(V_1+V_2,V_1 \cap V_2\)的一个基和维数.
\begin{solution}
因为\begin{equation*}
	V_1+V_2
	=\Span\{\AutoTuple{\alpha}{3}\}+\Span\{\AutoTuple{\beta}{2}\}
	=\Span\{\AutoTuple{\alpha}{3},\AutoTuple{\beta}{2}\},
\end{equation*}
所以向量组\(\AutoTuple{\alpha}{3},\AutoTuple{\beta}{2}\)的一个极大线性无关组就是\(V_1+V_2\)的一个基,
这个向量组的秩就是\(V_1+V_2\)的维数.
令\(A=(\AutoTuple{\alpha}{3},\AutoTuple{\beta}{2})\).
对\(A\)作一系列初等行变换,得到\begin{equation*}
	A=\begin{bmatrix}
		1 & -1 & 0 & 2 & 1 \\
		2 & 1 & 3 & -1 & -3 \\
		1 & 1 & 2 & 0 & 3 \\
		0 & 1 & 1 & 1 & 7
	\end{bmatrix}
	\to \begin{bmatrix}
		1 & 0 & 1 & 0 & -1 \\
		0 & 1 & 1 & 0 & 4 \\
		0 & 0 & 0 & 1 & 3 \\
		0 & 0 & 0 & 0 & 0
	\end{bmatrix}.
\end{equation*}
由此得出,\(\alpha_1,\alpha_2,\beta_1\)是\(V_1+V_2\)的一个基,
\(\dim(V_1+V_2)=3\).
同时也知道,\(\beta_2\)可由\(\alpha_1,\alpha_2,\beta_1\)线性表出,
其系数是线性方程组\(x_1\alpha_1+x_2\alpha_2+x_3\beta_1=\beta_2\)的解\((-1,4,3)^T\),
即\(\beta_2=-\alpha_1+4\alpha_2+3\beta_1\).
从而\(\alpha_1-4\alpha_2=3\beta_1-\beta_2\in V_1 \cap V_2\).
又因为\(\dim V_1=2,
\dim V_2=2\),
所以由子空间的维数公式有
\(\dim(V_1 \cap V_2)
=\dim V_1+\dim V_2-\dim(V_1+V_2)
=2+2-3=1\),
于是\(\alpha_1-4\alpha_2=(5,-2,-3,-4)^T\)是\(V_1 \cap V_2\)的一个基.
\end{solution}
\end{example}

从上例可以看出,
只要对矩阵\(A\)作一系列初等行变换,
把它化为若尔当阶梯形矩阵,
就可以得到子空间的基和维数等信息.
