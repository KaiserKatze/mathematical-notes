\section{商空间}
几何空间可以看成是由原点\(O\)为起点的所有向量组成的\(3\)维实线性空间\(V\).
过原点的一个平面\(W\)是\(V\)的一个\(2\)维子空间.
与\(W\)平行的每一个平面\(\pi\)都不是\(V\)的子空间,
因为\(\pi\)对加法和数量乘法都不封闭.
但是我们还是想问:\(\pi\)具有什么样的结构?\(\pi\)与\(W\)的关系如何?

在\(\pi\)上取定一个向量\(\g_0\),
\(\pi\)上每一个向量\(\g\)可以唯一地表示成\(\g_0\)与\(W\)中一个向量\(\vb\eta\)之和:
\(\g=\g_0+\vb\eta\).

反之,任取\(\vb\eta\in W\),
有\(\g_0+\vb\eta\in\pi\),
因此\(\pi=\Set{ \g_0+\vb\eta \given \vb\eta\in W }\).
我们可以把\(\pi\)记作\(\g_0+W\),
称其为“\(W\)的一个\DefineConcept{陪集}”,
把\(\g_0\)称为\DefineConcept{陪集代表}.
显然\begin{align*}
	\g\in\g_0+W
	&\iff
	\g=\g_0+\vb\eta,\vb\eta\in W \\
	&\iff
	\g-\g_0=\vb\eta\in W.
\end{align*}
由此看出,
如果在\(V\)上规定一个二元关系\(\sim\)满足\[
	\g\sim\g_0
	\defiff
	\g-\g_0\in W,
\]
那么容易验证关系\(\sim\)具有反身性、对称性和传递性,
这就是说关系\(\sim\)是等价关系,
于是\(\g_0\)所属的等价类\(\overline{\g_0}\)为\begin{align*}
	\overline{\g_0}
	&=\Set{ \g\in V \given \g\sim\g_0 } \\
	&=\Set{ \g\in V \given \g-\g_0\in W } \\
	&=\Set{ \g\in V \given \g=\g_0+\vb\eta,\vb\eta\in W } \\
	&=\Set{ \g_0+\vb\eta \given \vb\eta\in W }
	=\g_0+W.
\end{align*}
这表明陪集\(\g_0+W\)是等价类\(\overline{\g_0}\).
\(W\)本身也是\(W\)的一个陪集\(0+W\).

综上所述,
在几何空间\(V\)中,
与\(W\)平行或重合的每一个平面\(\pi\)是\(W\)的一个陪集,
也是等价关系\(\sim\)下的一个等价类.
所有等价类(即所有与\(W\)平行或重合的平面)组成的集合是几何空间的一个划分.
利用这个划分可以研究几何空间的结构.
受此启发,我们能不能给出线性空间\(V\)的一个划分,
然后利用这个划分来研究线性空间\(V\)的结构呢?
我们已经知道,要想给出线性空间\(V\)的一个划分,
就需要在\(V\)上建立一个二元等价关系,
得到的所有等价类组成的集合就是\(V\)的一个划分.

设\(V\)是域\(F\)上的一个线性空间,\(W\)是\(V\)的一个子空间.
在\(V\)上定义一个二元关系\(\sim\)满足\[
	\a\sim\b
	\defiff
	\a-\b\in W,
\]
则\(\sim\)是一个等价关系.
