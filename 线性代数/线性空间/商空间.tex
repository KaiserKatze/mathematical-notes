\section{商空间}
几何空间可以看成是由原点\(O\)为起点的所有向量组成的\(3\)维实线性空间\(V\).
过原点的一个平面\(W\)是\(V\)的一个\(2\)维子空间.
与\(W\)平行的每一个平面\(\pi\)都不是\(V\)的子空间,
因为\(\pi\)对加法和数量乘法都不封闭.
但是我们还是想问:\(\pi\)具有什么样的结构?\(\pi\)与\(W\)的关系如何?

在\(\pi\)上取定一个向量\(\g_0\),
\(\pi\)上每一个向量\(\g\)可以唯一地表示成\(\g_0\)与\(W\)中一个向量\(\vb\eta\)之和:
\(\g=\g_0+\vb\eta\).

反之,任取\(\vb\eta\in W\),
有\(\g_0+\vb\eta\in\pi\),
因此\(\pi=\Set{ \g_0+\vb\eta \given \vb\eta\in W }\).
我们可以把\(\pi\)记作\(\g_0+W\),
称其为“\(W\)的一个\emph{陪集}”,
把\(\g_0\)称为\emph{陪集代表}.
显然\begin{align*}
	\g\in\g_0+W
	&\iff
	\g=\g_0+\vb\eta,\vb\eta\in W \\
	&\iff
	\g-\g_0=\vb\eta\in W.
\end{align*}
由此看出,
如果在\(V\)上规定一个二元关系\(\sim\)满足\[
	\g\sim\g_0
	\defiff
	\g-\g_0\in W,
\]
那么容易验证关系\(\sim\)具有反身性、对称性和传递性,
这就是说关系\(\sim\)是等价关系,
于是\(\g_0\)所属的等价类\(\overline{\g_0}\)为\begin{align*}
	\overline{\g_0}
	&=\Set{ \g\in V \given \g\sim\g_0 } \\
	&=\Set{ \g\in V \given \g-\g_0\in W } \\
	&=\Set{ \g\in V \given \g=\g_0+\vb\eta,\vb\eta\in W } \\
	&=\Set{ \g_0+\vb\eta \given \vb\eta\in W }
	=\g_0+W.
\end{align*}
这表明陪集\(\g_0+W\)是等价类\(\overline{\g_0}\).
\(W\)本身也是\(W\)的一个陪集\(0+W\).

综上所述,
在几何空间\(V\)中,
与\(W\)平行或重合的每一个平面\(\pi\)是\(W\)的一个陪集,
也是等价关系\(\sim\)下的一个等价类.
所有等价类(即所有与\(W\)平行或重合的平面)组成的集合是几何空间的一个划分.
利用这个划分可以研究几何空间的结构.
受此启发,我们能不能给出线性空间\(V\)的一个划分,
然后利用这个划分来研究线性空间\(V\)的结构呢?
我们已经知道,要想给出线性空间\(V\)的一个划分,
就需要在\(V\)上建立一个二元等价关系,
得到的所有等价类组成的集合就是\(V\)的一个划分.

\subsection{陪集,陪集代表}
设\(V\)是域\(F\)上的一个线性空间,\(W\)是\(V\)的一个子空间.
在\(V\)上定义一个二元关系\(\sim\)满足\[
%@see: 《高等代数(第三版 下册)》(丘维声) P97 (3)
	\a\sim\b
	\defiff
	\a-\b\in W,
\]
则\(\sim\)是一个等价关系.

\(\a\)的等价类\begin{align*}
%@see: 《高等代数(第三版 下册)》(丘维声) P97 (4)
	\overline{\a}
	&=\Set{ \b \in V \given \b \sim \a }
	=\Set{ \b \in V \given \b-\a \in W } \\
	&=\Set{ \b \in V \given \b=\a+\g \land \g \in W } \\
	&=\Set{ \a+\g \given \g \in W }
\end{align*}
可以记作\(\a+W\),称为“\(W\)的一个\DefineConcept{陪集}”,
\(\a\)称为\DefineConcept{陪集代表}.
特别地,可以将\[
	\overline{\vb0}
	= \vb0+W
	= \Set{
		\g \in V
		\given
		\g \in W
	}
\]简记为\(W\).
于是\(\a\)的等价类\(\overline{\a}\)
就是以\(\a\)为代表的\(W\)的一个陪集\(\a+W\).
从而\[
%@see: 《高等代数(第三版 下册)》(丘维声) P98 (5)
	\vb\beta \in \vb\alpha + W
	\iff
	\vb\beta \sim \vb\alpha
	\iff
	\vb\beta - \vb\alpha \in W.
\]

应该注意到\[
%@see: 《高等代数(第三版 下册)》(丘维声) P98 (6)
	\vb\alpha + W = \vb\delta + W
	\iff
	\vb\alpha - \vb\delta \in W.
\]
可以看出,一个陪集的陪集代表不唯一,
只要\(\vb\delta\)满足\(\vb\alpha-\vb\delta \in W\),
那么\(\vb\delta\)也可以作为这个陪集的代表.

\subsection{商集,商空间}
根据\cref{definition:集合论.商集的定义},
\(W\)的全体陪集,就是\(V\)对于关系\(\sim\)的商集\(V/\kern-2pt\sim\).
因为关系\(\sim\)是利用子空间\(W\)确定的,
因此我们又把\(V/\kern-2pt\sim\)称为“线性空间\(V\)对于子空间\(W\)的商集”,
记作\(V/W\),
即\[
	V/W
	\defeq
	\Set{ \a+W \given \a \in V }.
\]

线性空间\(V\)是具有加法与纯量乘法两种运算的集合,
因此我们有理由期望\(V\)对于子空间\(W\)的商集\(V/W\)
也可以定义加法与纯量乘法两种运算.
但是由于\(V/W\)不是\(V\)的子集,
因此不能把\(V\)的两种运算直接搬到\(V/W\)中来.

\begin{theorem}\label{theorem:商空间.商空间的线性运算}
%@see: 《高等代数(第三版 下册)》(丘维声) P98 定理1
设\(W\)是域\(F\)上线性空间\(V\)的一个子空间.
在\(V\)对于子空间\(W\)的商集\(V/W\)规定运算\begin{itemize}
	\item 加法:\begin{equation}\label{equation:商空间.商空间的加法}
		%@see: 《高等代数(第三版 下册)》(丘维声) P98 (8)
		(\forall(\a+W),(\b+W)\in V/W)[(\a+W) + (\b+W) \defeq (\a+\b)+W],
	\end{equation}
	\item 纯量乘法:\begin{equation}\label{equation:商空间.商空间的纯量乘法}
		%@see: 《高等代数(第三版 下册)》(丘维声) P98 (9)
		(\forall(\a+W)\in V/W)(\forall k \in F)[k(\a+W) \defeq k\a+W],
	\end{equation}
\end{itemize}
则\(V/W\)对于这两种运算成为域\(F\)上的一个线性空间,
它的零元是\(0+W\).
\begin{proof}
\cref{equation:商空间.商空间的加法,equation:商空间.商空间的纯量乘法}
分别用到了陪集代表\(\a,\b\),
我们下面说明它们的运算结果不依赖于陪集代表的选择.
设\[
	\a_1 + W = \a + W,
	\qquad
	\b_1 + W = \b + W,
\]
则\[
	\a_1 - \a \in W,
	\qquad
	\b_1 - \b \in W.
\]
从而\begin{gather*}
	(\a_1 + \b_1) - (\a + \b)
	= (\a_1 - \a) + (\b_1 - \b)
	\in W, \\
	k \a_1 - k \a
	= k (\a_1 - \a)
	\in W,
\end{gather*}
因此\[
	(\a_1 + \b_1) + W
	= (\a + \b) + W,
	\qquad
	k \a_1 + W
	= k \a + W.
\]
这就证明上述关于商空间的加法和纯量乘法运算的定义是合理的.

容易验证线性空间定义的8条运算法则在\(V/W\)中都成立.
因此,\(V/W\)成为域\(F\)上的一个线性空间.
\end{proof}
\end{theorem}

既然\(V\)对于子空间\(W\)的商集\(V/W\)对加法和纯量乘法成为域\(F\)上的一个线性空间,
那么我们可以把线性空间\(V\)对于子空间\(W\)的商集\(V/W\)
重新命名为“\(V\)对\(W\)的\DefineConcept{商空间}”.

\subsection{商空间的维数}
\begin{theorem}\label{theorem:商空间.商空间的维数}
%@see: 《高等代数(第三版 下册)》(丘维声) P99 定理2
设\(W\)是域\(F\)上有限维线性空间\(V\)的一个子空间,
则\begin{equation}
%@see: 《高等代数(第三版 下册)》(丘维声) P99 (10)
	\dim(V/W) = \dim V - \dim W.
\end{equation}
\begin{proof}
设\(\dim V = n,
\dim W = s\).
在\(W\)中取一个基\(\AutoTuple{\a}{s}\),
把它扩充成\(V\)的一个基\[
	\AutoTuple{\a}{s},
	\allowbreak
	\AutoTuple{\a}[s+1]{n}.
\]
任取\(\b + W \in V/W\),
令\(\b = b_1 \a_1 + \dotsb + b_n \a_n\),
则\begin{align*}
	%@see: 《高等代数(第三版 下册)》(丘维声) P99 (11)
	\b + W
	&= (b_1 \a_1 + \dotsb + b_n \a_n) + W \\
	% 商空间的加法
	&= (b_1 \a_1 + W) + \dotsb + (b_s \a_s + W)
		+ (b_{s+1} \a_{s+1} + W) + \dotsb + (b_n \a_n + W) \\
	% 因为\(\AutoTuple{\a}{s} \in W\),所以\(b_i \a_i + W = W\ (i=1,2,\dotsc,s)\).
	&= W + \dotsb + W + b_{s+1} (\a_{s+1} + W) + \dotsb + b_n (\a_n + W) \\
	% 商空间的加法
	&= b_{s+1} (\a_{s+1} + W) + \dotsb + b_n (\a_n + W).
\end{align*}
这表明\(V/W\)中任一向量均可表示成\(\a_{s+1} + W,\dotsc,\a_n + W\)的线性组合.
若能证明\(\a_{s+1} + W,\dotsc,\a_n + W\)线性无关,
则它就是\(V/W\)的一个基,
从而\[
	\dim(V/W)
	= n - s
	= \dim V - \dim W.
\]
现在假设\[
%@see: 《高等代数(第三版 下册)》(丘维声) P99 (12)
	k_1 (\a_{s+1} + W)
	+ \dotsb
	+ k_{n-s} (\a_n + W)
	= W,
	% 这里就是在假设\(\a_{s+1} + W,\dotsc,\a_n + W\)的一个线性组合等于零陪集
\]
则\[
	% 商空间的加法
	(k_1 \a_{s+1} + \dotsb + k_{n-s} \a_n) + W = W,
\]
从而\[
	k_1 \a_{s+1} + \dotsb + k_{n-s} \a_n \in W,
\]
于是\[
	k_1 \a_{s+1} + \dotsb + k_{n-s} \a_n
	= l_1 \a_1 + \dotsb + l_s \a_s,
\]
即\[
%@see: 《高等代数(第三版 下册)》(丘维声) P99 (13)
	l_1 \a_1 + \dotsb + l_s \a_s
	- k_1 \a_{s+1} - \dotsb - k_{n-s} \a_n
	= \vb0.
\]
由于\(\AutoTuple{\a}{s},
\allowbreak
\AutoTuple{\a}[s+1]{n}\)线性无关,
所以\[
	l_1 = \dotsb = l_s
	= k_1 = \dotsb = k_{n-s}
	= 0.
\]
这证明了\(\a_{s+1} + W,\dotsc,\a_n + W\)是\(V/W\)中线性无关的向量组.
\end{proof}
\end{theorem}

从这里我们可以看出,
在\(W\)中取一个基\(\AutoTuple{\a}{s}\),
把它扩充成\(V\)的一个基\[
	\AutoTuple{\a}{s},
	\allowbreak
	\AutoTuple{\a}[s+1]{n},
\]
则\(\a_{s+1}+W,\dotsc,\a_n+W\)是商空间\(V/W\)的一个基.
令\(U=\Span\{\a_{s+1},\dotsc,\a_n\}\).
由于\(W\)的一个基\(\AutoTuple{\a}{s}\)
与\(U\)的一个基\(\a_{s+1},\dotsc,\a_n\)合起来是\(V\)的一个基,
因此\(V=W \oplus U\).
这个结论也可以推广到\(V\)及其子空间\(W\)都是无限维的,
而商空间\(V/W\)是有限维的情况:
\begin{theorem}\label{theorem:商空间.商空间的基与它的零元素的补空间的基之间的关系}
%@see: 《高等代数(第三版 下册)》(丘维声) P100 定理3
设\(V\)是域\(F\)上的线性空间,
\(W\)是\(V\)的一个子空间.
如果商空间\(V/W\)的一个基为\[
	\b_1+W,\dotsc,\b_t+W,
\]
令\(U=\Span\{\b_1,\dotsc,\b_t\}\),
那么\(V=W \oplus U\),
并且\(\AutoTuple{\b}{t}\)是\(U\)的一个基.
\begin{proof}
任取\(\a \in V\).
由于\(\b_1+W,\dotsc,\b_t+W\)是商空间\(V/W\)的一个基,
因此存在\(F\)中一组元素\(\AutoTuple{k}{t}\),
使得\begin{align*}
	\a+W
	&= k_1 (\b_1 + W) + \dotsb + k_t (\b_t + W) \\
	&= (k_1 \b_1 + \dotsb k_t \b_t) + W.
\end{align*}
于是\(\a - (k_1 \b_1 + \dotsb k_t \b_t) \in W\).
记\(\b = k_1 \b_1 + \dotsb + k_t \b_t\),
则\(\b \in U\)且\(\a - \b \in W\).
记\(\a - \b = \vb\delta\),
则\(\a = \vb\delta + \b \in W + U\).
于是\(V \subseteq W+U\),
从而\(V = W+U\).

任取\(\g \in W \cap U\),
则\(\g \in W\)且\(\g \in U\),
于是存在\(F\)中一组元素\(\AutoTuple{l}{t}\),
使得\[
	\g = l_1 \b_1 + \dotsb + l_t \b_t,
\]
并且\begin{align*}
	W &= \g + W
	= (l_1 \b_1 + \dotsb + l_t \b_t) + W \\
	&= l_1 (\b_1 + W) + \dotsb + l_t (\b_t + W).
\end{align*}
由于\(\b_1+W,\dotsc,\b_t+W\)线性无关,
因此从上式可知\[
	l_1 = \dotsb = l_t = 0.
\]
于是\(\g = \vb0\),
从而\(W \cap U = 0\).

综上所述,\(V = W \oplus U\).

设\[
	x_1 \b_1 + \dotsb x_t \b_t = 0,
\]
则\[
	(x_1 \b_1 + \dotsb + x_t \b_t) + W
	= 0 + W
	= W.
\]
从上面一段证得的结论可知,
\(x_1 = \dotsb = x_t = 0\).
因此\(\AutoTuple{\b}{t}\)线性无关,
那么\(\AutoTuple{\b}{t}\)是\(U\)的一个基.
\end{proof}
\end{theorem}
\cref{theorem:商空间.商空间的基与它的零元素的补空间的基之间的关系} 表明,
如果线性空间\(V\)对于子空间\(W\)的商空间\(V/W\)是有限维的,
并且我们知道商空间\(V/W\)的一个基,
那么\(V\)就有一个直和分解式.
这是可以利用商空间研究线性空间的结构的原理之一.

从\cref{theorem:商空间.商空间的维数} 看出,
对于\(n\)维线性空间\(V\),
如果它的子空间\(W\)不是零子空间,
那么\[
	\dim(V/W) < \dim V.
\]
于是我们可以利用数学归纳法来证明线性空间中有关被商空间继承的性质的结论.
这是可以利用商空间研究线性空间的结构的原理之二.

综上所述,商空间是研究线性空间结构的第四条途径.

\cref{theorem:商空间.商空间的基与它的零元素的补空间的基之间的关系} 还表明,
对于无限维线性空间\(V\)的无限维子空间\(W\),
如果商空间\(V/W\)是有限维的,
那么\(W\)在\(V\)中也有补空间.

更一般地,可以证明:
无限维线性空间\(V\)的任一子空间\(W\)都有补空间.
% 证明过程在《高等代数(第三版 下册)》(丘维声)参考文献[18]的第8章第4节的命题1

\begin{example}
%@see: 《高等代数(第三版 下册)》(丘维声) P101 习题8.4 1.
设\(U,W\)都是域\(F\)上线性空间\(V\)的子空间.
证明:如果\(V = U \oplus W\),
那么\(V\)对\(U\)的商空间\(V/U\)与\(W\)同构.
%TODO proof
\end{example}

\subsection{余维数}
\begin{definition}
%@see: 《高等代数(第三版 下册)》(丘维声) P101 定义1
设\(W\)是域\(F\)上线性空间\(V\)的一个子空间.
如果\(V\)对\(W\)的商空间\(V/W\)是有限维的,
则\(\dim(V/W)\)称为
“子空间\(W\)在\(V\)中的\DefineConcept{余维数}(codimension)”,
记作\(\codim_V W\),
简记为\(\codim W\).
\end{definition}

\subsection{典范映射}
\begin{definition}
设\(W\)是域\(F\)上线性空间\(V\)的一个子空间,
\(V/W\)是\(V\)对\(W\)的商空间.
把映射\[
	\pi\colon V \to V/W,
	\vb\alpha \mapsto \vb\alpha+W
\]称为\DefineConcept{标准映射}或\DefineConcept{典范映射}.
\end{definition}
\begin{remark}
典范映射\(\pi\)是满射.
\end{remark}
