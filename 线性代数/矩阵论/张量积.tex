\section{张量积}
\subsection{克罗内克积}
\begin{definition}
%@see: 《矩阵论》(詹兴致) P13
设\(\vb{A} = (a_{ij}) \in M_{m \times n}(\mathbb{C}),
\vb{B} \in M_{s \times t}(\mathbb{C})\).
定义:\begin{equation}
	\MatrixKroneckerTensorProduct{\vb{A}}{\vb{B}}
	\defeq
	\begin{bmatrix}
		a_{11} \vb{B} & a_{12} \vb{B} & \dots & a_{1n} \vb{B} \\
		a_{21} \vb{B} & a_{22} \vb{B} & \dots & a_{2n} \vb{B} \\
		\vdots & \vdots & & \vdots \\
		a_{m1} \vb{B} & a_{m2} \vb{B} & \dots & a_{mn} \vb{B}
	\end{bmatrix}
	\in M_{ms \times nt}(\mathbb{C}),
\end{equation}
称之为“矩阵\(\vb{A}\)与矩阵\(\vb{B}\)的\DefineConcept{克罗内克积}(Kronecker product)”.
%@see: https://mathworld.wolfram.com/KroneckerProduct.html
\end{definition}

\begin{property}
%@see: 《矩阵论》(詹兴致) P13
克罗内克积具有以下性质:\begin{gather*}
	(\forall \vb{A} \in M_{m \times n}(\mathbb{C}))
	(\forall \vb{B} \in M_{s \times t}(\mathbb{C}))
	(\forall k \in \mathbb{C})
	[
		\MatrixKroneckerTensorProduct{(k\vb{A})}{\vb{B}}
		= \MatrixKroneckerTensorProduct{\vb{A}}{(k\vb{B})}
		= k(\MatrixKroneckerTensorProduct{\vb{A}}{\vb{B}})
	], \\
	(\forall \vb{A} \in M_{m \times n}(\mathbb{C}))
	(\forall \vb{B} \in M_{s \times t}(\mathbb{C}))
	[
		(\MatrixKroneckerTensorProduct{\vb{A}}{\vb{B}})^T
		= \MatrixKroneckerTensorProduct{\vb{A}^T}{\vb{B}^T}
	], \\
	(\forall \vb{A} \in M_{m \times n}(\mathbb{C}))
	(\forall \vb{B} \in M_{s \times t}(\mathbb{C}))
	[
		(\MatrixKroneckerTensorProduct{\vb{A}}{\vb{B}})^H
		= \MatrixKroneckerTensorProduct{\vb{A}^H}{\vb{B}^H}
	], \\
	(\forall \vb{A} \in M_{m \times n}(\mathbb{C}))
	(\forall \vb{B} \in M_{s \times t}(\mathbb{C}))
	(\forall \vb{C} \in M_{p \times q}(\mathbb{C}))
	[
		\MatrixKroneckerTensorProduct{(\MatrixKroneckerTensorProduct{\vb{A}}{\vb{B}})}{\vb{C}}
		= \MatrixKroneckerTensorProduct{\vb{A}}{(\MatrixKroneckerTensorProduct{\vb{B}}{\vb{C}})}
	], \\
	(\forall \vb{A} \in M_{m \times n}(\mathbb{C}))
	(\forall \vb{B},\vb{C} \in M_{s \times t}(\mathbb{C}))
	[
		\MatrixKroneckerTensorProduct{\vb{A}}{(\vb{B}+\vb{C})}
		= (\MatrixKroneckerTensorProduct{\vb{A}}{\vb{B}})
		+ (\MatrixKroneckerTensorProduct{\vb{A}}{\vb{C}})
	], \\
	(\forall \vb{A},\vb{B} \in M_{m \times n}(\mathbb{C}))
	(\forall \vb{C} \in M_{s \times t}(\mathbb{C}))
	[
		\MatrixKroneckerTensorProduct{(\vb{A}+\vb{B})}{\vb{C}}
		= (\MatrixKroneckerTensorProduct{\vb{A}}{\vb{C}})
		+ (\MatrixKroneckerTensorProduct{\vb{B}}{\vb{C}})
	], \\
	(\forall \vb{A} \in M_{m \times n}(\mathbb{C}))
	(\forall \vb{B} \in M_{s \times t}(\mathbb{C}))
	[
		\MatrixKroneckerTensorProduct{\vb{A}}{\vb{B}} = 0
		\iff
		\vb{A} = 0 \lor \vb{B} = 0
	], \\
	(\forall \vb{A} \in M_m(\mathbb{C}))
	(\forall \vb{B} \in M_n(\mathbb{C}))
	[
		\text{$\vb{A},\vb{B}$都是对称矩阵}
		\implies
		\text{$\MatrixKroneckerTensorProduct{\vb{A}}{\vb{B}}$是对称矩阵}
	], \\
	(\forall \vb{A} \in M_m(\mathbb{C}))
	(\forall \vb{B} \in M_n(\mathbb{C}))
	[
		\text{$\vb{A},\vb{B}$都是厄米矩阵}
		\implies
		\text{$\MatrixKroneckerTensorProduct{\vb{A}}{\vb{B}}$是厄米矩阵}
	].
\end{gather*}
\end{property}

\begin{lemma}
%@see: 《矩阵论》(詹兴致) P14 引理2.1
设矩阵\(\vb{A} \in M_{m \times n}(\mathbb{C}),
\vb{B} \in M_{s \times t}(\mathbb{C}),
\vb{C} \in M_{n \times k}(\mathbb{C}),
\vb{D} \in M_{t \times r}(\mathbb{C})\),
则\begin{equation}
	(\MatrixKroneckerTensorProduct{\vb{A}}{\vb{B}})
	(\MatrixKroneckerTensorProduct{\vb{C}}{\vb{D}})
	= \MatrixKroneckerTensorProduct{(\vb{A}\vb{C})}{(\vb{B}\vb{D})}.
\end{equation}
%TODO proof
\end{lemma}

\begin{property}
%@see: 《矩阵论》(詹兴致) P14 定理2.2(i)
设矩阵\(\vb{A} \in M_m(\mathbb{C}),
\vb{B} \in M_n(\mathbb{C})\).
若\(\vb{A},\vb{B}\)都可逆,
则\(\MatrixKroneckerTensorProduct{\vb{A}}{\vb{B}}\)也可逆,
且\begin{equation}
	(\MatrixKroneckerTensorProduct{\vb{A}}{\vb{B}})^{-1}
	= \MatrixKroneckerTensorProduct{\vb{A}^{-1}}{\vb{B}^{-1}}.
\end{equation}
\end{property}

\begin{property}
%@see: 《矩阵论》(詹兴致) P14 定理2.2(ii)
设矩阵\(\vb{A} \in M_m(\mathbb{C}),
\vb{B} \in M_n(\mathbb{C})\).
若\(\vb{A},\vb{B}\)都是正规矩阵,
则\(\MatrixKroneckerTensorProduct{\vb{A}}{\vb{B}}\)也是正规矩阵.
\end{property}

\begin{property}
%@see: 《矩阵论》(詹兴致) P14 定理2.2(iii)
设矩阵\(\vb{A} \in M_m(\mathbb{C}),
\vb{B} \in M_n(\mathbb{C})\).
若\(\vb{A},\vb{B}\)都是酉矩阵,
则\(\MatrixKroneckerTensorProduct{\vb{A}}{\vb{B}}\)也是酉矩阵.
\end{property}

\begin{property}
%@see: 《矩阵论》(詹兴致) P14 定理2.2(iv)
%@see: 《矩阵论》(詹兴致) P14 定理2.2(v)
设矩阵\(\vb{A} \in M_m(\mathbb{C}),
\vb{B} \in M_n(\mathbb{C})\).
若\(\lambda\)是\(\vb{A}\)的一个特征值,
\(\vb{x}\)是\(\vb{A}\)的属于\(\lambda\)的一个特征向量,
\(\mu\)是\(\vb{B}\)的一个特征值,
\(\vb{y}\)是\(\vb{B}\)的属于\(\mu\)的一个特征向量,
则\(\lambda\mu\)是\(\MatrixKroneckerTensorProduct{\vb{A}}{\vb{B}}\)的一个特征值,
\(\MatrixKroneckerTensorProduct{\vb{x}}{\vb{y}}\)是
\(\MatrixKroneckerTensorProduct{\vb{A}}{\vb{B}}\)的
属于\(\lambda\mu\)的特征向量.
\end{property}

\begin{property}
%@see: 《矩阵论》(詹兴致) P14 定理2.2(vi)
设矩阵\(\vb{A} \in M_m(\mathbb{C}),
\vb{B} \in M_n(\mathbb{C})\),
则\begin{equation}
	\det(
		\MatrixKroneckerTensorProduct{\vb{A}}{\vb{B}}
	)
	= (\det\vb{A})^n (\det\vb{B})^m.
\end{equation}
\end{property}

\begin{property}
%@see: 《矩阵论》(詹兴致) P14 定理2.2(vii)
设矩阵\(\vb{A} \in M_m(\mathbb{C}),
\vb{B} \in M_n(\mathbb{C})\).
若\(\lambda\)是\(\vb{A}\)的一个奇异值,
\(\mu\)是\(\vb{B}\)的一个奇异值,
则\(\lambda\mu\)是\(\MatrixKroneckerTensorProduct{\vb{A}}{\vb{B}}\)的一个奇异值.
\end{property}

\begin{property}
%@see: 《矩阵论》(詹兴致) P14 定理2.2(viii)
设矩阵\(\vb{A} \in M_m(\mathbb{C}),
\vb{B} \in M_n(\mathbb{C})\),
则\begin{equation}
	\rank(
		\MatrixKroneckerTensorProduct{\vb{A}}{\vb{B}}
	)
	= (\rank\vb{A}) (\rank\vb{B}).
\end{equation}
\end{property}

\subsection{阿达玛积}
\begin{definition}
%@see: 《矩阵论》(詹兴致) P15
设\(\vb{A} = (a_{ij}) \in M_{m \times n}(\mathbb{C}),
\vb{B} = (b_{ij}) \in M_{m \times n}(\mathbb{C})\).
定义:\begin{equation}
	\MatrixHadamardProduct{\vb{A}}{\vb{B}}
	\defeq
	\begin{bmatrix}
		a_{11} b_{11} & a_{12} b_{12} & \dots & a_{1n} b_{1n} \\
		a_{21} b_{21} & a_{22} b_{22} & \dots & a_{2n} b_{2n} \\
		\vdots & \vdots & & \vdots \\
		a_{m1} b_{m1} & a_{m2} b_{m2} & \dots & a_{mn} b_{mn}
	\end{bmatrix}
	\in M_{m \times n}(\mathbb{C}),
\end{equation}
称之为“矩阵\(\vb{A}\)与矩阵\(\vb{B}\)的\DefineConcept{阿达玛积}(Hadamard product)”.
\end{definition}

\begin{lemma}
%@see: 《矩阵论》(詹兴致) P15 引理2.3
设矩阵\(\vb{A},\vb{B} \in M_n(\mathbb{C})\),
则\(\vb{A}\)与\(\vb{B}\)的阿达玛积
\(\MatrixHadamardProduct{\vb{A}}{\vb{B}}\)是
\(\vb{A}\)与\(\vb{B}\)的克罗内克积
\(\MatrixKroneckerTensorProduct{\vb{A}}{\vb{B}}\)的
位于\(1,n+2,2n+3,\dotsc,n^2\)行和列的主子矩阵.
%TODO proof
\end{lemma}

\begin{theorem}
%@see: 《矩阵论》(詹兴致) P16 定理2.4(Schur)
设矩阵\(\vb{A},\vb{B} \in M_n(\mathbb{C})\).
\begin{itemize}
	\item 若\(\vb{A},\vb{B}\)都是半正定的,
	则\(\vb{A}\)与\(\vb{B}\)的阿达玛积
	\(\MatrixHadamardProduct{\vb{A}}{\vb{B}}\)也是半正定的.
	\item 若\(\vb{A},\vb{B}\)都是正定的,
	则\(\vb{A}\)与\(\vb{B}\)的阿达玛积
	\(\MatrixHadamardProduct{\vb{A}}{\vb{B}}\)也是正定的.
\end{itemize}
\end{theorem}

\begin{definition}
%@see: 《矩阵论》(詹兴致) P16
设矩阵\(\vb{A} \in M_{m \times n}(\mathbb{C})\)可以按列分块为\((\AutoTuple{\alpha}{n})\).
定义:\begin{equation}
	\MatrixConcatColumn{\vb{A}}
	\defeq
	\begin{bmatrix}
		\alpha_1 \\
		\alpha_2 \\
		\vdots \\
		\alpha_n
	\end{bmatrix}
	\in \mathbb{C}^{mn}.
\end{equation}
\end{definition}

\begin{lemma}
%@see: 《矩阵论》(詹兴致) P16 引理2.5(i)
设矩阵\(\vb{A} \in M_{m \times n}(\mathbb{C}),
\vb{B} \in M_{n \times k}(\mathbb{C}),
\vb{C} \in M_{k \times t}(\mathbb{C})\),
则\begin{equation}
	\MatrixConcatColumn{(\vb{A}\vb{B}\vb{C})}
	= (\MatrixKroneckerTensorProduct{\vb{C}^T}{\vb{A}})
	\MatrixConcatColumn{\vb{B}}.
\end{equation}
\end{lemma}
