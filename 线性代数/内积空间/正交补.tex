\section{正交补,正交投影}
\subsection{正交补}
\begin{definition}\label{definition:正交补.利用内积构造的正交补}
%@see: 《高等代数(第三版 下册)》(丘维声) P181 定义1
设\((V,\rho)\)是实内积空间,
\(W\)是\(V\)的一个非空子集.
定义:\begin{equation*}
	W^\perp
	\defeq
	\Set{
		\alpha \in V
		\given
		(\forall \beta \in W)
		[\rho(\alpha,\beta) = 0]
	},
\end{equation*}
称之为“\(W\)的\DefineConcept{正交补}(the \emph{orthogonal complement} of \(W\))”.
\end{definition}
\begin{remark}
\cref{definition:正交补.利用内积构造的正交补} 中的“正交补”
与\cref{definition:双线性函数.利用双线性函数构造的正交补} 中的“正交补”稍微不同,
前者是后者的一个特例.
\end{remark}

\begin{property}
%@see: 《高等代数(第三版 下册)》(丘维声) P181
设\((V,\rho)\)是实内积空间,
\(W\)是\(V\)的一个非空子集,
则\(W\)的正交补\(W^\perp\)是\(V\)的一个子空间.
\begin{proof}
由\cref{theorem:双线性函数.双线性函数取值为零的条件2} 可知\(0 \in W^\perp\).
任取\(\alpha,\beta \in W^\perp\),
任取\(\gamma \in W\),
任取\(k \in \mathbb{R}\),
则\begin{gather*}
	\rho(\alpha+\beta,\gamma)
	= \rho(\alpha,\gamma) + \rho(\beta,\gamma)
	= 0 + 0 = 0, \\
	\rho(k\alpha,\gamma)
	= k\rho(\alpha,\gamma)
	= k0 = 0,
\end{gather*}
即\(W^\perp\)对加法、纯量乘法封闭.
因此\(W^\perp\)是\(V\)的子空间.
\end{proof}
\end{property}

\begin{property}\label{theorem:正交补.利用内积构造的正交补.正交补的对合律}
%@see: 《高等代数(第三版 下册)》(丘维声) P184 习题10.3 3.
%@see: 《高等代数(大学高等代数课程创新教材 第二版 下册)》(丘维声) P481 例2
设\((V,\rho)\)是\(n\)维欧几里得空间,
\(W\)是\(V\)的一个子空间,
则\(W\)的正交补\(W^\perp\)的正交补\((W^\perp)^\perp\)就是\(W\),
即\((W^\perp)^\perp = W\).
\begin{proof}
由于欧几里得空间的内积\(\rho\)是正定的对称双线性函数,
因此\(\rho\)是非退化的对称双线性函数,
由\cref{theorem:双线性函数.利用双线性函数构造的正交补.正交补的对合律} 可知
\((W^\perp)^\perp = W\).
\end{proof}
\end{property}

\begin{property}
%@see: 《高等代数(大学高等代数课程创新教材 第二版 下册)》(丘维声) P481 例3
设\(W_1,W_2\)是\(n\)维欧几里得空间\(V\)的两个子空间,
则\begin{gather}
%@see: 《高等代数(大学高等代数课程创新教材 第二版 下册)》(丘维声) P481 (14)
	(W_1 + W_2)^\perp
	= W_1^\perp \cap W_2^\perp, \\
	(W_1 \cap W_2)^\perp
	= W_1^\perp + W_2^\perp.
\end{gather}
%TODO proof
\end{property}

\begin{theorem}\label{theorem:正交补.实内积空间的正交直和分解}
%@see: 《高等代数(第三版 下册)》(丘维声) P181 定理1
设\(U\)是实内积空间\((V,\rho)\)的一个有限维子空间,
则\begin{equation*}
%@see: 《高等代数(第三版 下册)》(丘维声) P181 (2)
	V = U \DirectSum U^\perp.
\end{equation*}
\begin{proof}
先证\(V = U \DirectSum U^\perp\).
在\(U\)中取一个标准正交基\(\AutoTuple{\epsilon}{n}\).
任取\(\alpha \in V\),
令\begin{align*}
%@see: 《高等代数(第三版 下册)》(丘维声) P181 (3)
	\alpha_1
	&\defeq
	\rho(\alpha,\epsilon_1) \epsilon_1 + \dotsb + \rho(\alpha,\epsilon_m) \epsilon_m, \\
%@see: 《高等代数(第三版 下册)》(丘维声) P182 (4)
	\alpha_2
	&\defeq
	\alpha - \alpha_1.
\end{align*}
显然\(\alpha_1 \in U\)而\(\alpha_2 \in V\).
因为\begin{align*}
	\rho(\alpha_2,\epsilon_j)
	= \rho(\alpha-\alpha_1,\epsilon_j)
	= \rho(\alpha,\epsilon_j)
	- \rho(\alpha_1,\epsilon_j),
\end{align*}
其中\begin{align*}
	\rho(\alpha_1,\epsilon_j)
	&= \rho\left(
			\rho(\alpha,\epsilon_1) \epsilon_1 + \dotsb + \rho(\alpha,\epsilon_m) \epsilon_m,
			\epsilon_j
		\right) \\
	&= \rho(\alpha,\epsilon_1) \rho(\epsilon_1,\epsilon_j)
		+ \dotsb + \rho(\alpha,\epsilon_m) \rho(\epsilon_m,\epsilon_j) \\
	&= \rho(\alpha,\epsilon_j),
\end{align*}
所以\begin{equation*}
	\rho(\alpha_2,\epsilon_j)
	= \rho(\alpha,\epsilon_j) - \rho(\alpha,\epsilon_j)
	= 0,
\end{equation*}
这就说明\(\alpha_2 \in U^\perp\).
既然\(\alpha = \alpha_1 + \alpha_2\),
那么\(V = U + U^\perp\).

再证\(U \cap U^\perp = 0\).
假设\(\gamma \in U \cap U^\perp\),
则\(\gamma \in U\)且\(\gamma \in U^\perp\),
从而\(\rho(\gamma,\gamma) = 0\),
于是\(\gamma = 0\).

综上所述,\(V = U \DirectSum U^\perp\).
\end{proof}
\end{theorem}
\begin{remark}
从\cref{theorem:正交补.实内积空间的正交直和分解} 得到,
对于欧几里得空间\(V\)的任意一个非平凡子空间\(U\),
都有\(V = U \DirectSum U^\perp\),
这就说明\(U\)的一个标准正交基与\(U^\perp\)的一个标准正交基合起来就是\(V\)的一个标准正交基.
\end{remark}

\begin{example}
%@see: 《高等代数(第三版 下册)》(丘维声) P184 习题10.3 4.
证明:欧几里得空间\(\mathbb{R}^n\)(指定标准内积)的
任意一个子空间\(U\)是某一个齐次线性方程组的解空间.
%TODO proof
\end{example}

\subsection{正交投影的概念}
如果\(U\)是实内积空间\(V\)的一个子空间,
且\(V = U \DirectSum U^\perp\),
那么\(V\)中每个向量\(\alpha\)能唯一地表示成\begin{equation*}
%@see: 《高等代数(第三版 下册)》(丘维声) P182 (5)
	\alpha = \alpha_1 + \alpha_2,
	\quad \alpha_1 \in U, \alpha_2 \in U^\perp,
\end{equation*}
我们可以藉此构造一个\(V\)上的线性变换\begin{equation*}
	\vb{P}_U\colon V \to U,
	\alpha \mapsto \alpha_1,
\end{equation*}
并称之为“\(V\)在\(U\)上的\DefineConcept{正交投影}”,	% 这里是指线性变换
把\(\alpha_1\)称为“\(\alpha\)在\(U\)上的\DefineConcept{正交投影}”.	% 这里是指元素

这里可以看出:\(\beta_1 \in U\)是\(\beta\)在\(U\)上的正交投影,
当且仅当\(\beta - \beta_1 \in U^\perp\).

由\cref{theorem:正交补.实内积空间的正交直和分解} 可知,
只要\(U\)是有限维的,那么必定存在\(V\)在\(U\)上的正交投影.

\begin{example}
%@see: 《高等代数(第三版 下册)》(丘维声) P184 习题10.3 5.
设\(U\)是实内积空间\((V,\rho)\)的一个\(m\)维子空间,
在\(U\)中取一个标准正交基\(\AutoTuple{\eta}{m}\).
证明:向量\(\alpha \in V\)在\(U\)上的正交投影为\begin{equation*}
	\alpha_1 = \sum_{i=1}^m \rho(\alpha,\eta_i) \eta_i.
\end{equation*}
%TODO proof
\end{example}

\subsection{正交投影的性质}
向量\(\alpha\)在\(U\)上的正交投影\(\alpha_1\)具有什么性质呢?
从几何学中“垂线段最短”可以得到启发,猜测以下结论:
\begin{theorem}\label{theorem:正交补.垂线段最短}
%@see: 《高等代数(第三版 下册)》(丘维声) P182 定理2
设\(U\)是实内积空间\(V\)的一个有限维子空间,
向量\(\alpha \in V\),向量\(\alpha_1 \in U\),
则\(\alpha_1\)是\(\alpha\)在\(U\)上的正交投影,
当且仅当\begin{equation*}
%@see: 《高等代数(第三版 下册)》(丘维声) P182 (6)
	(\forall \gamma \in U)
	[d(\alpha,\alpha_1) \leq d(\alpha,\gamma)].
\end{equation*}
\begin{proof}
必要性.
设\(\alpha_1 \in U\)是\(\alpha\)在\(U\)上的正交投影,
则\(\alpha - \alpha_1 \in U^\perp\),
那么对于\(\forall \gamma \in U\),
有\begin{equation*}
	(\alpha - \alpha_1) \perp (\alpha_1 - \gamma);
\end{equation*}
由勾股定理有\begin{equation*}
	\VectorLengthA{\alpha - \alpha_1}^2
	+ \VectorLengthA{\alpha_1 - \gamma}^2
	= \VectorLengthA{(\alpha - \alpha_1) + (\alpha_1 - \gamma)}^2
	= \VectorLengthA{\alpha - \gamma}^2;
\end{equation*}
由此得出\(\VectorLengthA{\alpha - \alpha_1} \leq \VectorLengthA{\alpha - \gamma}\),
即\(d(\alpha,\alpha_1) \leq d(\alpha,\gamma)\).

充分性.
假设成立\((\forall \gamma \in U)[d(\alpha,\alpha_1) \leq d(\alpha,\gamma)]\).
同时假设向量\(\delta\)是\(\alpha\)在\(U\)上的正交投影,
从而必有\(\delta \in U\),
那么\(d(\alpha,\alpha_1) \leq d(\alpha,\delta)\).
再根据上述必要性证得的结论可知\(d(\alpha,\delta) \leq d(\alpha,\alpha_1)\).
于是由\(d(\alpha,\delta) \leq d(\alpha,\alpha_1) \leq d(\alpha,\delta)\)
有\(d(\alpha,\delta) = d(\alpha,\alpha_1)\).
由于\((\alpha-\delta) \in U^\perp,
(\delta-\alpha_1) \in U\),
所以\begin{equation*}
	\VectorLengthA{\alpha - \alpha_1}^2
	= \VectorLengthA{(\alpha - \delta) + (\delta - \alpha_1)}^2
	= \VectorLengthA{\alpha - \delta}^2 + \VectorLengthA{\delta - \alpha_1}^2.
\end{equation*}
由此得出\(\VectorLengthA{\delta - \alpha_1}^2 = 0\),
即\(\delta = \alpha_1\).
\end{proof}
\end{theorem}

\begin{example}
%@see: 《高等代数(第三版 下册)》(丘维声) P184 习题10.3 6.
设\(U\)是实内积空间\((V,\rho)\)的一个有限维子空间.
证明:\(V\)在\(U\)上的正交投影\(\vb{P}\)满足\begin{equation*}
	(\forall \alpha,\beta \in V)
	[
		\rho(\vb{P}\alpha,\beta)
		= \rho(\alpha,\vb{P}\beta)
	].
\end{equation*}
%TODO proof
\end{example}

\begin{example}
%@see: 《高等代数(第三版 下册)》(丘维声) P184 习题10.3 8.
%@see: 《Linear Algebra Done Right (Fourth Edition)》(Sheldon Axler) P198 6.26
设\(\AutoTuple{\epsilon}{m}\)是实内积空间\((V,\rho)\)中一个正交单位向量组.
证明:对于任意\(\alpha \in V\),
有\begin{equation}\label{equation:正交补.贝塞尔不等式}
	\sum_{i=1}^m \rho^2(\alpha,\epsilon_i)
	\leq \VectorLengthA{\alpha}^2.
\end{equation}
当且仅当\(\alpha = \sum_{i=1}^m \rho(\alpha,\epsilon_i) \epsilon_i\)时,上式取“\(=\)”号.
\begin{proof}
假设\(\alpha \in V\),
对它进行正交分解,得\(\alpha = \beta + \gamma\),
其中\(\beta = \sum_{i=1}^m \rho(\alpha,\epsilon_i) \epsilon_i,
\gamma = \alpha - \beta\).
注意到\(\rho(\gamma,\epsilon_k)
= \rho(\alpha,\epsilon_k) - \rho(\beta,\epsilon_k)
= \rho(\alpha,\epsilon_k) - \rho(\alpha,\epsilon_k) \rho(\epsilon_k,\epsilon_k)
= 0\),
从而有\(\rho(\gamma,\beta) = 0\),
于是由\hyperref[theorem:实内积空间.勾股定理]{勾股定理}可知
\(\VectorLengthA{\alpha}^2
= \VectorLengthA{\beta}^2 + \VectorLengthA{\gamma}^2
\geq \VectorLengthA{\beta}^2
= \sum_{i=1}^m \rho^2(\alpha,\epsilon_i)\).
\end{proof}
\end{example}
\begin{remark}
我们把\cref{equation:正交补.贝塞尔不等式} 称为\DefineConcept{贝塞尔不等式}(Bessel's inequality).
\end{remark}

\subsection{最佳逼近元}
从\cref{theorem:正交补.垂线段最短} 可以引出下述概念:
\begin{definition}
%@see: 《高等代数(第三版 下册)》(丘维声) P183 定义2
设\(U\)是实内积空间\(V\)的一个子空间,\(\alpha \in V\).
如果存在\(\delta \in U\),
使得对于任意\(\gamma \in U\),
都有\(d(\alpha,\delta) \leq d(\alpha,\gamma)\),
那么称“\(\delta\)是\(\alpha\)在\(U\)上的\DefineConcept{最佳逼近元}”.
\end{definition}

从\cref{theorem:正交补.垂线段最短} 立即可以得出:
\begin{proposition}
%@see: 《高等代数(第三版 下册)》(丘维声) P183
如果\(U\)是实内积空间\(V\)的有限维子空间,
那么\(V\)中任意一个向量\(\alpha\)在\(U\)上的最佳逼近元存在且唯一,
它就是\(\alpha\)在\(U\)上的正交投影.
\end{proposition}

\subsection{正交投影的应用 --- 最小二乘法}
%@see: 《高等代数(第三版 下册)》(丘维声) P183
%@see: 《高等代数》(丁南庆、刘公祥、纪庆忠、郭学军) P338 定义8.5.3
实际问题中,从观测数据列出的线性方程组\(\vb{A} \vb{x} = \vb\beta\)可能无解,
其中\(
	\vb{A} \in M_{s \times n}(\mathbb{R}),
	\vb\beta \in \mathbb{R}^s
\).
当\(\vb{A} \vb{x} = \vb\beta\)无解时,
我们还是想要找出一个向量\(\vb{x} \in \mathbb{R}^n\),
使得\(\vb{A} \vb{x}\)充分接近\(\vb\beta\),
即函数\begin{equation*}
	f(\vb{x})
	\defeq
	\VectorLengthA{\vb{A} \vb{x} - \vb\beta}^2
	=
	(\vb{A} \vb{x} - \vb\beta)^T (\vb{A} \vb{x} - \vb\beta)
\end{equation*}
取最小值,
我们把使\(f(\vb{x})\)取得最小值的向量\(\vb{x}\)
称为“线性方程组\(\vb{A} \vb{x} = \vb\beta\)的\DefineConcept{最小二乘解}(least squares solution)”.
将最小二乘解作为\(\vb{A} \vb{x} = \vb\beta\)的近似解的方法,
称为\DefineConcept{最小二乘法}(method of least squares).

容易看出:\begin{align*}
	&\text{$\vb\eta$是$\vb{A} \vb{x} = \vb\beta$的最小二乘解} \\
	&\iff
	(\forall \vb{x} \in \mathbb{R}^n)
	[f(\vb\eta) \leq f(\vb{x})] \\
	&\iff
	(\forall \vb\gamma \in \Im\vb{A})
	[\VectorLengthA{\vb{A} \vb\eta - \vb\beta} \leq \VectorLengthA{\vb\gamma - \vb\beta}] \\
	&\iff
	(\forall \vb\gamma \in \Im\vb{A})
	[d(\vb{A} \vb\eta,\vb\beta) \leq d(\vb\gamma,\vb\beta)] \\
	&\iff
	\text{$\vb{A} \vb\eta$是$\vb\beta$在$\Im\vb{A}$上的正交投影} \\
	&\iff
	\vb{A} \vb\eta - \vb\beta \in (\Im\vb{A})^\perp \\
	&\iff
	\vb{A} \vb\eta - \vb\beta \in \Ker\vb{A}^T \\
	&\iff
	\vb{A}^T (\vb{A} \vb\eta - \vb\beta) = \vb0 \\
	&\iff
	\vb{A}^T \vb{A} \vb\eta = \vb{A}^T \vb\beta \\
	&\iff
	\text{$\vb\eta$是$\vb{A}^T \vb{A} \vb{x} = \vb{A}^T \vb\beta$的解}.
\end{align*}

因为 \hyperref[example:线性方程组有解的充分必要条件.最小二乘解的存在性]{$\rank(\vb{A}^T \vb{A}, \vb{A}^T \vb\beta) = \rank(\vb{A}^T \vb{A})$},
所以线性方程组\(\vb{A}^T \vb{A} \vb{x} = \vb{A}^T \vb\beta\)有解.
下面来具体求出该方程组的解.

\begingroup  % 线性方程组\(\vb{A}^T \vb{A} \vb{x} = \vb{A}^T \vb\beta\)的解
%@see: 《高等代数》(丁南庆、刘公祥、纪庆忠、郭学军) P338 定理8.5.4
%\cref{theorem:线性方程组.齐次线性方程组的解的结构定理.推论1}
设\(\vb{A}^+\)是\(\vb{A}\)的穆尔--彭罗斯广义逆.
首先有\begin{equation*}
	\vb{A}^T \vb{A} (\vb{A}^+ \vb\beta)
	% 矩阵乘法的结合律
	= \vb{A}^T (\vb{A} \vb{A}^+) \vb\beta
	% \cref{theorem:线性方程组.广义逆的性质1}
	= \vb{A}^T (\vb{A} \vb{A}^+)^H \vb\beta
	% 矩阵乘积的转置
	= (\vb{A} \vb{A}^+ \vb{A})^T \vb\beta
	% \cref{theorem:线性方程组.广义逆的性质1}
	= \vb{A}^T \vb\beta,
\end{equation*}
这就是说\(\vb{A}^+ \vb\beta\)是\(\vb{A}^T \vb{A} \vb{x} = \vb{A}^T \vb\beta\)的一个特解.
其次,对于任意\(\vb{Z} \in \mathbb{R}^n\),有\begin{equation*}
	\vb{A}^T \vb{A} (\vb{E}_n - \vb{A}^+ \vb{A}) \vb{Z}
	= (\vb{A}^T \vb{A} - \vb{A}^T \vb{A} \vb{A}^+ \vb{A}) \vb{Z}
	= (\vb{A}^T \vb{A} - \vb{A}^T \vb{A}) \vb{Z}
	= \vb0,
\end{equation*}
其中\(\vb{E}_n\)是\(n\)阶单位矩阵,
这就说明\(
	\Im(\vb{E}_n - \vb{A}^+ \vb{A})
	\subseteq
	\Ker(\vb{A}^T \vb{A})
\);
再由\(\vb{A} = \vb{A} \vb{A}^+ \vb{A}\)
可知\(
	\vb{A}^+ \vb{A}
	= (\vb{A}^+ \vb{A})^2
\),
说明\(\vb{A}^+ \vb{A}\)是一个幂等矩阵,
并且由\cref{example:矩阵乘积的秩.矩阵相乘不变秩的特例1} 可知\(\rank(\vb{A}^+ \vb{A}) = \rank\vb{A}\),
所以\begin{align*}
	\dim\Im(\vb{E}_n - \vb{A}^+ \vb{A})
	&= \rank(\vb{E}_n - \vb{A}^+ \vb{A})
	% \cref{example:幂等矩阵.幂等矩阵的秩的性质1}
	% \(\rank\vb{A} + \rank(\vb{E} - \vb{A}) = n\)
	% \(\rank(\vb{A}^+ \vb{A}) + \rank(\vb{E}_n - \vb{A}^+ \vb{A}) = n\)
	= n - \rank(\vb{A}^+ \vb{A}) \\
	% \(\rank(\vb{A}^+ \vb{A}) = \rank\vb{A}\)
	&= n - \rank\vb{A}
	% \cref{equation:矩阵乘积的秩.实矩阵及其转置矩阵的乘积的秩}
	% \(\rank\vb{A} = \rank(\vb{A}^T \vb{A})\)
	= n - \rank(\vb{A}^T \vb{A})
	% \cref{theorem:线性方程组.齐次线性方程组的解向量个数}
	= \dim\Ker(\vb{A}^T \vb{A});
\end{align*}
% \cref{theorem:向量空间.两个非零子空间的关系2}
因此\(
	\Im(\vb{E}_n - \vb{A}^+ \vb{A})
	= \Ker(\vb{A}^T \vb{A})
\).
综上所述,\(\vb{A}^T \vb{A} \vb{x} = \vb{A}^T \vb\beta\)的解集是\begin{equation*}
	\Set*{
		\vb{A}^+ \vb\beta
		+ (\vb{E}_n - \vb{A}^+ \vb{A}) \vb{Z}
		\given
		\vb{Z} \in \mathbb{R}^n
	}.
\end{equation*}
\endgroup
容易看出,最小二乘解唯一,当且仅当\(\vb{A}^+ \vb{A} = \vb{E}_n\).
%TODO proof 尚未证明“最小二乘解唯一,当且仅当\(\vb{A}^+ \vb{A} = \vb{E}_n\)”

特别地,当\(\vb{A}\)是列满秩矩阵(即\(\rank\vb{A} = n\))
或\(\vb{A}^T \vb{A}\)是正定矩阵时,
\(\vb{A}^T \vb{A}\)可逆,
于是方程\(\vb{A}^T \vb{A} \vb{x} = \vb{A}^T \vb\beta\)有一个形式更简单的解:\begin{equation*}
	\{
		(\vb{A}^T \vb{A})^{-1} \vb{A}^T \vb\beta
	\}.
\end{equation*}
% 涉及正定矩阵的数值计算非常简便.
% 在计算时,无须交换行,也无须担心主元过小.
% 因此在此特别提及当\(\vb{A}^T \vb{A}\)是正定矩阵时的计算.
