\section{正交补,正交投影}
\subsection{正交补}
\begin{definition}\label{definition:正交补.利用内积构造的正交补}
%@see: 《高等代数(第三版 下册)》(丘维声) P181 定义1
设\((V,\rho)\)是实内积空间,
\(W\)是\(V\)的一个非空子集.
定义:\begin{equation*}
	W^\perp
	\defeq
	\Set{
		\alpha \in V
		\given
		(\forall \beta \in W)
		[\rho(\alpha,\beta) = 0]
	},
\end{equation*}
称之为“\(W\)的\DefineConcept{正交补}”.
\end{definition}
\begin{remark}
\cref{definition:正交补.利用内积构造的正交补} 中的“正交补”
与\cref{definition:双线性函数.利用双线性函数构造的正交补} 中的“正交补”稍微不同,
前者是后者的一个特例.
\end{remark}

\begin{property}
%@see: 《高等代数(第三版 下册)》(丘维声) P181
设\((V,\rho)\)是实内积空间,
\(W\)是\(V\)的一个非空子集,
则\(W\)的正交补\(W^\perp\)是\(V\)的一个子空间.
\begin{proof}
由\cref{theorem:双线性函数.双线性函数取值为零的条件2} 可知\(0 \in W^\perp\).
任取\(\alpha,\beta \in W^\perp\),
任取\(\gamma \in W\),
任取\(k \in \mathbb{R}\),
则\begin{gather*}
	\rho(\alpha+\beta,\gamma)
	= \rho(\alpha,\gamma) + \rho(\beta,\gamma)
	= 0 + 0 = 0, \\
	\rho(k\alpha,\gamma)
	= k\rho(\alpha,\gamma)
	= k0 = 0,
\end{gather*}
即\(W^\perp\)对加法、纯量乘法封闭.
因此\(W^\perp\)是\(V\)的子空间.
\end{proof}
\end{property}

\begin{theorem}\label{theorem:正交补.实内积空间的正交直和分解}
%@see: 《高等代数(第三版 下册)》(丘维声) P181 定理1
设\(U\)是实内积空间\((V,\rho)\)的一个有限维子空间,
则\begin{equation*}
%@see: 《高等代数(第三版 下册)》(丘维声) P181 (2)
	V = U \DirectSum U^\perp.
\end{equation*}
\begin{proof}
先证\(V = U \DirectSum U^\perp\).
在\(U\)中取一个标准正交基\(\AutoTuple{\epsilon}{n}\).
任取\(\alpha \in V\),
令\begin{align*}
%@see: 《高等代数(第三版 下册)》(丘维声) P181 (3)
	\alpha_1
	&\defeq
	\rho(\alpha,\epsilon_1) \epsilon_1 + \dotsb + \rho(\alpha,\epsilon_m) \epsilon_m, \\
%@see: 《高等代数(第三版 下册)》(丘维声) P182 (4)
	\alpha_2
	&\defeq
	\alpha - \alpha_1.
\end{align*}
显然\(\alpha_1 \in U\)而\(\alpha_2 \in V\).
因为\begin{align*}
	\rho(\alpha_2,\epsilon_j)
	= \rho(\alpha-\alpha_1,\epsilon_j)
	= \rho(\alpha,\epsilon_j)
	- \rho(\alpha_1,\epsilon_j),
\end{align*}
其中\begin{align*}
	\rho(\alpha_1,\epsilon_j)
	&= \rho\left(
			\rho(\alpha,\epsilon_1) \epsilon_1 + \dotsb + \rho(\alpha,\epsilon_m) \epsilon_m,
			\epsilon_j
		\right) \\
	&= \rho(\alpha,\epsilon_1) \rho(\epsilon_1,\epsilon_j)
		+ \dotsb + \rho(\alpha,\epsilon_m) \rho(\epsilon_m,\epsilon_j) \\
	&= \rho(\alpha,\epsilon_j),
\end{align*}
所以\begin{equation*}
	\rho(\alpha_2,\epsilon_j)
	= \rho(\alpha,\epsilon_j) - \rho(\alpha,\epsilon_j)
	= 0,
\end{equation*}
这就说明\(\alpha_2 \in U^\perp\).
既然\(\alpha = \alpha_1 + \alpha_2\),
那么\(V = U + U^\perp\).

再证\(U \cap U^\perp = 0\).
假设\(\gamma \in U \cap U^\perp\),
则\(\gamma \in U\)且\(\gamma \in U^\perp\),
从而\(\rho(\gamma,\gamma) = 0\),
于是\(\gamma = 0\).

综上所述,\(V = U \DirectSum U^\perp\).
\end{proof}
\end{theorem}
\begin{remark}
从\cref{theorem:正交补.实内积空间的正交直和分解} 得到,
对于欧几里得空间\(V\)的任意一个非平凡子空间\(U\),
都有\(V = U \DirectSum U^\perp\),
这就说明\(U\)的一个标准正交基与\(U^\perp\)的一个标准正交基合起来就是\(V\)的一个标准正交基.
\end{remark}

\subsection{正交投影的概念}
如果\(U\)是实内积空间\(V\)的一个子空间,
且\(V = U \DirectSum U^\perp\),
那么\(V\)中每个向量\(\alpha\)能唯一地表示成\begin{equation*}
%@see: 《高等代数(第三版 下册)》(丘维声) P182 (5)
	\alpha = \alpha_1 + \alpha_2,
	\quad \alpha_1 \in U, \alpha_2 \in U^\perp,
\end{equation*}
我们可以藉此构造一个\(V\)上的线性变换\begin{equation*}
	\vb{P}_U\colon V \to U,
	\alpha \mapsto \alpha_1,
\end{equation*}
并称之为“\(V\)在\(U\)上的\DefineConcept{正交投影}”,	% 这里是指线性变换
把\(\alpha_1\)称为“\(\alpha\)在\(U\)上的\DefineConcept{正交投影}”.	% 这里是指元素

这里可以看出:\(\beta_1 \in U\)是\(\beta\)在\(U\)上的正交投影,
当且仅当\(\beta - \beta_1 \in U^\perp\).

由\cref{theorem:正交补.实内积空间的正交直和分解} 可知,
只要\(U\)是有限维的,那么必定存在\(V\)在\(U\)上的正交投影.

\subsection{正交投影的性质}
向量\(\alpha\)在\(U\)上的正交投影\(\alpha_1\)具有什么性质呢?
从几何学中“垂线段最短”可以得到启发,猜测以下结论:
\begin{theorem}\label{theorem:正交补.垂线段最短}
%@see: 《高等代数(第三版 下册)》(丘维声) P182 定理2
设\(U\)是实内积空间\(V\)的一个有限维子空间,
向量\(\alpha \in V\),向量\(\alpha_1 \in U\),
则\(\alpha_1\)是\(\alpha\)在\(U\)上的正交投影,
当且仅当\begin{equation*}
%@see: 《高等代数(第三版 下册)》(丘维声) P182 (6)
	(\forall \gamma \in U)
	[d(\alpha,\alpha_1) \leq d(\alpha,\gamma)].
\end{equation*}
\begin{proof}
必要性.
设\(\alpha_1 \in U\)是\(\alpha\)在\(U\)上的正交投影,
则\(\alpha - \alpha_1 \in U^\perp\),
那么对于\(\forall \gamma \in U\),
有\begin{equation*}
	(\alpha - \alpha_1) \perp (\alpha_1 - \gamma);
\end{equation*}
由勾股定理有\begin{equation*}
	\VectorLengthA{\alpha - \alpha_1}^2
	+ \VectorLengthA{\alpha_1 - \gamma}^2
	= \VectorLengthA{(\alpha - \alpha_1) + (\alpha_1 - \gamma)}^2
	= \VectorLengthA{\alpha - \gamma}^2;
\end{equation*}
由此得出\(\VectorLengthA{\alpha - \alpha_1} \leq \VectorLengthA{\alpha - \gamma}\),
即\(d(\alpha,\alpha_1) \leq d(\alpha,\gamma)\).

充分性.
假设成立\((\forall \gamma \in U)[d(\alpha,\alpha_1) \leq d(\alpha,\gamma)]\).
同时假设向量\(\delta\)是\(\alpha\)在\(U\)上的正交投影,
从而必有\(\delta \in U\),
那么\(d(\alpha,\alpha_1) \leq d(\alpha,\delta)\).
再根据上述必要性证得的结论可知\(d(\alpha,\delta) \leq d(\alpha,\alpha_1)\).
于是由\(d(\alpha,\delta) \leq d(\alpha,\alpha_1) \leq d(\alpha,\delta)\)
有\(d(\alpha,\delta) = d(\alpha,\alpha_1)\).
由于\((\alpha-\delta) \in U^\perp,
(\delta-\alpha_1) \in U\),
所以\begin{equation*}
	\VectorLengthA{\alpha - \alpha_1}^2
	= \VectorLengthA{(\alpha - \delta) + (\delta - \alpha_1)}^2
	= \VectorLengthA{\alpha - \delta}^2 + \VectorLengthA{\delta - \alpha_1}^2.
\end{equation*}
由此得出\(\VectorLengthA{\delta - \alpha_1}^2 = 0\),
即\(\delta = \alpha_1\).
\end{proof}
\end{theorem}

\subsection{最佳逼近元}
从\cref{theorem:正交补.垂线段最短} 可以引出下述概念:
\begin{definition}
%@see: 《高等代数(第三版 下册)》(丘维声) P183 定义2
设\(U\)是实内积空间\(V\)的一个子空间,\(\alpha \in V\).
如果存在\(\delta \in U\),
使得对于任意\(\gamma \in U\),
都有\(d(\alpha,\delta) \leq d(\alpha,\gamma)\),
那么称“\(\delta\)是\(\alpha\)在\(U\)上的\DefineConcept{最佳逼近元}”.
\end{definition}

从\cref{theorem:正交补.垂线段最短} 立即可以得出:
\begin{proposition}
%@see: 《高等代数(第三版 下册)》(丘维声) P183
如果\(U\)是实内积空间\(V\)的有限维子空间,
那么\(V\)中任意一个向量\(\alpha\)在\(U\)上的最佳逼近元存在且唯一,
它就是\(\alpha\)在\(U\)上的正交投影.
\end{proposition}

% \subsection{正交投影的应用 --- 最小二乘法}
% 正交投影可以用于最小二乘法.
