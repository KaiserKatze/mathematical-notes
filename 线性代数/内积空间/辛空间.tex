\section{辛空间}
我们已经分别在实线性空间和复线性空间中,通过引进内积的概念,
使得这些空间具备了长度、角度、正交、距离等度量概念.
这使得我们不禁好奇,在任意域\(F\)上的线性空间\(V\)中,能不能也引进度量概念?
即便是对于实线性空间\(V\),也有不引入正定对称双线性函数作为内积的情况.
例如,作为爱因斯坦相对论基础的闵可夫斯基空间,
它是实数域上的一个4维线性空间,
指定了一个非退化的对称双线性函数作为内积,
从而使得洛伦兹变换保持向量的内积不变.

\begin{definition}
%@see: 《高等代数(第三版 下册)》(丘维声) P200 定义1
设\(V\)是域\(F\)上的一个线性空间.
如果映射\(f\colon V \times V \to F\)是一个对称双线性函数,
则称“\(f\)是\(V\)上的一个\DefineConcept{内积}”.
\end{definition}

\begin{definition}
%@see: 《高等代数(第三版 下册)》(丘维声) P200 定义1
设\(V\)是域\(F\)上的一个线性空间,
\(f\)是\(V\)上的一个内积,
则称“\((V,f)\)是一个\DefineConcept{正交空间}”.
\end{definition}

\begin{definition}
%@see: 《高等代数(第三版 下册)》(丘维声) P200 定义1
设\((V,f)\)是一个正交空间.
\begin{itemize}
	\item 如果\(f\)是非退化的,则称“\((V,f)\)是一个\DefineConcept{正则空间}”.
	\item 否则,称“\((V,f)\)是一个\DefineConcept{非正则空间}”.
\end{itemize}
\end{definition}

在正交空间中,由于不要求内积具有正定性,因此无法引进长度、角度、距离等度量概念,但是仍有正交这个概念.
\begin{definition}
%@see: 《高等代数(第三版 下册)》(丘维声) P200 定义2
设\(\alpha,\beta\)是正交空间\((V,f)\)中的两个非零向量.
如果\(f(\alpha,\beta) = 0\),
则称“\(\alpha\)与\(\beta\) \DefineConcept{正交}”,
记为\(\alpha \perp \beta\).
\end{definition}
