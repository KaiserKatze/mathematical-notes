\section{辛空间}
我们已经分别在实线性空间和复线性空间中,通过引进内积的概念,
使得这些空间具备了长度、角度、正交、距离等度量概念.
这使得我们不禁好奇,在任意域\(F\)上的线性空间\(V\)中,能不能也引进度量概念?
即便是对于实线性空间\(V\),也有不引入正定对称双线性函数作为内积的情况.
例如,作为爱因斯坦相对论基础的闵可夫斯基空间,
它是实数域上的一个4维线性空间,
指定了一个非退化的对称双线性函数作为内积,
从而使得洛伦兹变换保持向量的内积不变.

\subsection{正交空间}
\begin{definition}
%@see: 《高等代数(第三版 下册)》(丘维声) P200 定义1
设\(V\)是域\(F\)上的一个线性空间.
如果映射\(f\colon V \times V \to F\)是一个对称双线性函数,
则称“\(f\)是\(V\)上的一个\DefineConcept{内积}(inner product)”.
\end{definition}

\begin{definition}\label{definition:正交空间.正交空间}
%@see: 《高等代数(第三版 下册)》(丘维声) P200 定义1
设\(V\)是域\(F\)上的一个线性空间,
\(f\)是\(V\)上的一个内积,
则称“\((V,f)\)是一个\DefineConcept{正交空间}”.
\end{definition}

\begin{definition}
%@see: 《高等代数(第三版 下册)》(丘维声) P200 定义1
设\((V,f)\)是一个正交空间.
\begin{itemize}
	\item 如果\(f\)是非退化的,则称“\((V,f)\)是一个\DefineConcept{正则正交空间}”.
	\item 否则,称“\((V,f)\)是一个\DefineConcept{非正则正交空间}”.
\end{itemize}
\end{definition}

在正交空间中,由于不要求内积具有正定性,因此无法引进长度、角度、距离等度量概念,但是仍有正交这个概念.
\begin{definition}
%@see: 《高等代数(第三版 下册)》(丘维声) P200 定义2
设\(\alpha,\beta\)是正交空间\((V,f)\)中的两个非零向量.
如果\(f(\alpha,\beta) = 0\),
则称“\(\alpha\)与\(\beta\) \DefineConcept{正交}(orthogonal)”,
记为\(\alpha \perp \beta\).
\end{definition}

由于正交空间的内积\(f\)是对称的,
那么,从\(f(\alpha,\beta) = 0\)可推出\(f(\beta,\alpha) = 0\),
换言之,从\(\alpha \perp \beta\)可推出\(\beta \perp \alpha\).

\begin{example}
%@see: 《高等代数(第三版 下册)》(丘维声) P200
在正交空间中,一个非零向量有可能与自身正交.
例如,在\(V = \mathbb{R}^4\)中,
令\begin{equation*}
%@see: 《高等代数(第三版 下册)》(丘维声) P200 (2)
	f(\alpha,\beta) \defeq x_1 y_1 - x_2 y_2 - x_3 y_3 - x_4 y_4,
\end{equation*}
其中\(\alpha=(\AutoTuple{x}{4})^T,
\beta=(\AutoTuple{y}{4})^4\);
容易验证\(f\)是\(V\)上一个非退化的对称双线性函数.
在正交空间\((V,f)\)中,
向量\(\alpha_1=(1,1,0,0)^T\)
满足\(f(\alpha_1,\alpha_1) = 1^2 - 1^2 = 0\),
即有\(\alpha_1 \perp \alpha_1\).
\end{example}

\begin{definition}\label{definition:正交空间.迷向向量}
%@see: 《高等代数(第三版 下册)》(丘维声) P201
在正交空间\((V,f)\)中,
如果向量\(\alpha \in V\)满足\(\alpha \perp \alpha\),
则称“\(\alpha\)是\((V,f)\)的一个\DefineConcept{迷向向量}”;
否则称“\(\alpha\)是\((V,f)\)的一个\DefineConcept{非迷向向量}”.
\end{definition}

\begin{definition}
%@see: 《高等代数(第三版 下册)》(丘维声) P201 定义3
设\((V,f)\)是正交空间,
\(S\)是\(V\)的一个子集.
定义:\begin{equation*}
	S^\perp
	\defeq
	\Set{
		\alpha \in V
		\given
		(\forall \beta \in S)
		[f(\alpha,\beta) = 0]
	}
\end{equation*}
称之为“\(S\)的\DefineConcept{正交补}”.
\end{definition}

\begin{property}
%@see: 《高等代数(第三版 下册)》(丘维声) P201
设\((V,f)\)是正交空间,
\(S\)是\(V\)的一个子集,
则\(S\)的正交补\(S^\perp\)是\(V\)的一个子空间.
\end{property}

\begin{theorem}
%@see: 《高等代数(第三版 下册)》(丘维声) P201 定理1
设\((V,f)\)是有限维正则正交空间,
\(W\)是\(V\)的一个子空间,
则\begin{gather*}
	\dim W + \dim W^\perp = \dim V, \\
	(W^\perp)^\perp = W.
\end{gather*}
%TODO proof
\end{theorem}

\begin{definition}\label{definition:正交空间.正交空间的正交基}
%@see: 《高等代数(第三版 下册)》(丘维声) P201 定义4
设\(\AutoTuple{\alpha}{n}\)是有限维正交空间\((V,f)\)的一个基.
如果\(\AutoTuple{\alpha}{n}\)两两正交,
则称“\(\AutoTuple{\alpha}{n}\)是\((V,f)\)的一个\DefineConcept{正交基}”.
\end{definition}
\begin{remark}
%@see: 《高等代数(第三版 下册)》(丘维声) P201
注意\hyperref[definition:正交空间.正交空间的正交基]{正交基的定义}中,
首先要求\(\AutoTuple{\alpha}{n}\)是基,然后要求基向量两两正交.
这是因为在正交空间中两两正交的向量组有可能是线性相关的.
例如,在\(V = \mathbb{R}^4\)中,
令\begin{equation*}
%@see: 《高等代数(第三版 下册)》(丘维声) P200 (2)
	f(\alpha,\beta) \defeq x_1 y_1 - x_2 y_2 - x_3 y_3 - x_4 y_4,
\end{equation*}
其中\(\alpha=(\AutoTuple{x}{4})^T,
\beta=(\AutoTuple{y}{4})^4\),
那么,在正交空间\((V,f)\)中,
向量\(\alpha=(1,1,0,0)^T\)满足\(f(\alpha,2\alpha) = 0\),
%@Mathematica: f[x_, y_] := x[[1]] y[[1]] - x[[2]] y[[2]] - x[[3]] y[[3]] - x[[4]] y[[4]]
%@Mathematica: a = {1, 1, 0, 0}
%@Mathematica: f[a, 2 a]
%@Mathematica: f[a, k a]
即有\(\alpha \perp (2\alpha)\),
但是\(\{\alpha,2\alpha\}\)是线性相关的.
\end{remark}

\begin{theorem}\label{theorem:正交空间.特征不为2的域上的有限维正交空间一定存在正交基}
%@see: 《高等代数(第三版 下册)》(丘维声) P201 定理2
域\(F\ (\FieldChar F\neq2)\)上的有限维正交空间\((V,f)\)一定存在正交基.
%TODO proof
\begin{proof}
设\(\dim V = n\),
由\cref{theorem:双线性函数.特征不为2的域上的对称双线性函数在某个基下的度量矩阵是对角矩阵} 可知,
\(V\)中存在一个基\(\AutoTuple{\alpha}{n}\),
使得\(f\)在基\(\AutoTuple{\alpha}{n}\)下的度量矩阵\(A\)是对角矩阵\(\diag(\AutoTuple{d}{n})\).
这表明\(\AutoTuple{\alpha}{n}\)两两正交,
因此\(\AutoTuple{\alpha}{n}\)是\(V\)的一个正交基.
\end{proof}
\end{theorem}

注意如果\((V,f)\)是正则的,
\cref{theorem:正交空间.特征不为2的域上的有限维正交空间一定存在正交基} 的证明中的
度量矩阵\(A\)是满秩矩阵,
从而\(d_i\neq0\),
于是\(f(\alpha_i,\alpha_i)\neq0\),
这表明\(\alpha_i\)是非迷向向量,
于是我们得到如下结论:
\begin{proposition}
%@see: 《高等代数(第三版 下册)》(丘维声) P201
域\(F\ (\FieldChar F\neq2)\)上的有限维正则正交空间\((V,f)\)一定存在由非迷向向量组成的正交基.
\end{proposition}

\begin{definition}
%@see: 《高等代数(第三版 下册)》(丘维声) P201 定义5
设\(\AutoTuple{\epsilon}{n}\)是有限维正交空间\((V,f)\)的一个正交基.
如果\begin{equation*}
	f(\epsilon_i,\epsilon_i) \in \{-1,0,1\},
	\quad i=1,2,\dotsc,n,
\end{equation*}
则称“\(\AutoTuple{\epsilon}{n}\)是\((V,f)\)的一个\DefineConcept{标准正交基}”.
\end{definition}

\begin{proposition}
%@see: 《高等代数(第三版 下册)》(丘维声) P201
设\((V,f)\)是复数域上的有限维正交空间,
则\begin{equation*}
	f(\epsilon_i,\epsilon_i) \in \{0,1\},
	\quad i=1,2,\dotsc,n.
\end{equation*}
\end{proposition}

%@see: 《高等代数(第三版 下册)》(丘维声) P201
有限维正则正交空间\((V,f)\)中每一个向量\(\beta\)
在\(V\)的正交基\(\AutoTuple{\alpha}{n}\)下的坐标为\begin{equation*}
%@see: 《高等代数(第三版 下册)》(丘维声) P201 (5)
	\beta = \sum_{i=1}^n \frac{f(\beta,\alpha_i)}{f(\alpha_i,\alpha_i)} \alpha_i.
\end{equation*}

\begin{theorem}
%@see: 《高等代数(第三版 下册)》(丘维声) P202 定理3
设\((V,f)\)是域\(F\ (\FieldChar F\neq2)\)上的正交空间.
如果\(W\)是\(V\)的一个有限维正则子空间,
则\begin{equation*}
%@see: 《高等代数(第三版 下册)》(丘维声) P202 (6)
	V = W \DirectSum W^\perp.
\end{equation*}
%TODO proof
\end{theorem}

\subsection{辛空间}
\begin{definition}\label{definition:辛空间.辛空间}
%@see: 《高等代数(第三版 下册)》(丘维声) P202 定义6
设\(V\)是域\(F\ (\FieldChar F\neq2)\)上的一个线性空间,
\(f\)是\(V\)上的一个斜对称双线性函数,
则称“\((V,f)\)是一个\DefineConcept{辛空间}”,
称“\(f\)是\(V\)上的一个\DefineConcept{辛内积}”.
\end{definition}

\begin{definition}
%@see: 《高等代数(第三版 下册)》(丘维声) P202 定义6
设\((V,f)\)是一个辛空间.
\begin{itemize}
	\item 如果\(f\)是非退化的,则称“\((V,f)\)是一个\DefineConcept{正则辛空间}”.
	\item 否则,称“\((V,f)\)是一个\DefineConcept{非正则辛空间}”.
\end{itemize}
\end{definition}

\begin{example}
%@see: 《高等代数(第三版 下册)》(丘维声) P202
在\(V = \mathbb{R}^2\)中,
令\begin{equation*}
	f(\alpha,\beta)
	\defeq
	x_1 y_2 - x_2 y_1,
\end{equation*}
其中\(\alpha=(\AutoTuple{x}{2})^2,
\beta=(\AutoTuple{y}{2})^2\);
容易验证\(f\)是\(V\)上一个非退化的斜对称双线性函数,
\((V,f)\)是一个辛空间.
\end{example}

%@see: 《高等代数(第三版 下册)》(丘维声) P202
与正交空间一样,辛空间中有正交的概念,但没有长度、角度、距离等概念,辛空间中也有迷向向量.

\begin{proposition}
%@see: 《高等代数(第三版 下册)》(丘维声) P202
有限维正则辛空间一定是偶数维的.
\end{proposition}

\begin{theorem}\label{theorem:辛空间.辛集的存在性}
%@see: 《高等代数(第三版 下册)》(丘维声) P202
\def\MatrixChunk{\begin{bmatrix}
	0 & 1 \\
	-1 & 0
\end{bmatrix}}
在\(n\)维辛空间\((V,f)\)中
存在一个基\(\delta_1,\delta_{-1},\dotsc,\delta_r,\delta_{-r},\AutoTuple{\eta}{s}\),
使得\begin{align*}
%@see: 《高等代数(第三版 下册)》(丘维声) P203 (8)
	f(\delta_i,\delta_{-i}) &= 1,
	\quad i=1,2,\dotsc,r; \\
	f(\delta_i,\delta_j) &= 0,
	\quad i+j\neq0; \\
	f(\delta_i,\eta_k) &= 0,
	\quad i=\pm1,\pm2,\dotsc,\pm r; \\
	f(\eta_j,\eta_k) &= 0,
	\quad j,k=1,2,\dotsc,s;
\end{align*}
且\(f\)在这个基下的度量矩阵是\begin{equation*}
	\diag\left(
		\MatrixChunk,
		\dotsc,
		\MatrixChunk,
		0,
		\dotsc,
		0
	\right).
\end{equation*}
\begin{proof}
由\cref{theorem:双线性函数.特征不为2的域上的斜对称双线性函数在某个基下的度量矩阵是分块对角矩阵} 立即可得.
\end{proof}
\end{theorem}

我们把\cref{theorem:辛空间.辛集的存在性} 中的
基\(\delta_1,\delta_{-1},\dotsc,\delta_r,\delta_{-r},\AutoTuple{\eta}{s}\)
称为“辛空间\((V,f)\)的\DefineConcept{辛基}”.

容易看出,\(f\)在基\(\delta_1,\dotsc,\delta_r,\delta_{-1},\dotsc,\delta_{-r},\AutoTuple{\eta}{s}\)下的度量矩阵为
\begin{equation*}
	\begin{bmatrix}
		0 & I & 0 \\
		-I & 0 & 0 \\
		0 & 0 & 0
	\end{bmatrix},
\end{equation*}
其中\(I\)是域\(F\ (\FieldChar F\neq2)\)上的\(r\)阶单位矩阵.

\begin{theorem}
%@see: 《高等代数(第三版 下册)》(丘维声) P203 定理4
设\((V,f)\)是一个辛空间.
如果\(W\)是\(V\)的一个有限维正则子空间,
则\begin{equation*}
%@see: 《高等代数(第三版 下册)》(丘维声) P203 (9)
	V = W \DirectSum W^\perp.
\end{equation*}
%TODO proof
\end{theorem}

\begin{definition}
%@see: 《高等代数(第三版 下册)》(丘维声) P204 习题10.6 3.
设\((V,f)\)是一个有限维正则辛空间,
\(\vb{A}\)是\(V\)上的一个线性变换.
如果\(\vb{A}\)保持辛内积不变,
则称\(\vb{A}\)是\(V\)上的一个\DefineConcept{辛变换}.
\end{definition}

\begin{example}
%@see: 《高等代数(第三版 下册)》(丘维声) P204 习题10.6 3.
设\((V,f)\)是域\(F\)上的一个有限维正则辛空间,
\(\vb{A}\)是\(V\)上的一个线性变换,
则\(\vb{A}\)是\(V\)上的一个辛变换,
当且仅当\(\vb{A}\)在\(V\)的辛基下的矩阵\(A\)
满足\(A^T B A = B\),
其中\begin{equation*}
	B = \begin{bmatrix}
		0 & I \\
		-I & 0
	\end{bmatrix},
\end{equation*}
\(I\)是域\(F\)上的\(r\)阶单位矩阵.
\end{example}
