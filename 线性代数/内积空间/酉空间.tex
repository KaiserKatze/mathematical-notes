\section{酉空间}
%@see: 《高等代数(第三版 下册)》(丘维声) P193
在本节,我们研究在复线性空间中引进内积的概念,使之成为复内积空间.

\subsection{酉空间}
如何在复线性空间\(V\)中引进内积的概念呢?
如果我们照搬实线性空间内积的概念,
考虑复线性空间\(V\)上一个双线性函数\(f\),
那么对于\(V\)中任意一个非零向量\(\alpha\),
有\begin{equation*}
	f(\iu\alpha,\iu\alpha)
	= \iu^2 f(\alpha,\alpha)
	= -f(\alpha,\alpha).
\end{equation*}
若要求\((\forall \alpha \in V)[f(\alpha,\alpha) \in \mathbb{R}]\),
则\(f\)不满足正定性(因为当\(f(\alpha,\alpha) > 0\)时,必有\(f(\iu\alpha,\iu\alpha) < 0\)).
为了使复线性空间\(V\)上的内积仍具有正定性,
就不能要求它是双线性函数,
而只要求它对第一个自变量是线性的.
为了使内积在\(V\)的任意一个向量\(\alpha\)与自身组成的有序对上的函数值为实数,
需要让\(V\)上的内积\(\rho\)具有如下性质:
\begin{equation}\label{equation:酉空间.厄米性}
	(\forall \alpha,\beta \in V)
	[
		\rho(\alpha,\beta)
		= \ComplexConjugate{\rho(\beta,\alpha)}
	].
\end{equation}
我们把\cref{equation:酉空间.厄米性} 描述的性质
称为\DefineConcept{厄米性}或\DefineConcept{共轭对称性}(conjugate symmetry).
于是复线性空间上的内积的概念应当定义如下:
\begin{definition}\label{definition:酉空间.复线性空间上的内积}
%@see: 《高等代数(第三版 下册)》(丘维声) P193 定义1
%@see: 《Linear Algebra Done Right (Fourth Edition)》(Sheldon Axler) P183 6.2
设\(V\)是复数域\(\mathbb{C}\)上的一个线性空间.
如果映射\(\rho\colon V \times V \to \mathbb{R}\)满足以下性质\begin{itemize}
	\item {\rm\bf 厄米性}:\begin{equation*}
		(\forall \alpha,\beta \in V)
		[
			\rho(\alpha,\beta)
			= \ComplexConjugate{\rho(\beta,\alpha)}
		];
	\end{equation*}

	\item {\rm\bf 线性性}:\begin{gather*}
		(\forall \alpha,\beta,\gamma \in V)
		[
			\rho(\alpha+\beta,\gamma)
			= \rho(\alpha,\gamma) + \rho(\beta,\gamma)
		], \\
		(\forall \alpha,\beta \in V)
		(\forall k \in \mathbb{C})
		[
			\rho(k\alpha,\beta)
			= k\rho(\alpha,\beta)
		];
	\end{gather*}

	\item {\rm\bf 正定性}:\begin{gather*}
		(\forall \alpha \in V)
		[
			\rho(\alpha,\alpha) \geq 0
		], \\
		(\forall \alpha \in V)
		[
			\rho(\alpha,\alpha) = 0
			\iff
			\alpha = 0
		],
	\end{gather*}
\end{itemize}
则称“\(\rho\)是复线性空间\(V\)上的一个\DefineConcept{内积}(inner product)”.
\end{definition}
\begin{remark}
\hyperref[definition:欧几里得空间.实线性空间上的内积]{实线性空间上的内积}%
与\hyperref[definition:酉空间.复线性空间上的内积]{复线性空间上的内积}的
最大“差别”在于前者要求内积具有对称性,后者要求内积具有厄米性.
但是,考虑到实数的共轭就是它本身,
所以复线性空间上的内积也可以用作实线性空间上的内积.
\end{remark}

\begin{definition}
%@see: 《高等代数(第三版 下册)》(丘维声) P193 定义1
%@see: 《Linear Algebra Done Right (Fourth Edition)》(Sheldon Axler) P184 6.4
设\(V\)是一个复线性空间,\(\rho\)是\(V\)上的一个内积,
则称“\((V,\rho)\)是一个\DefineConcept{复内积空间}(complex inner product space)”
或“\((V,\rho)\)是一个\DefineConcept{酉空间}(unitary space)”.
\end{definition}

\begin{property}
%@see: 《Linear Algebra Done Right (Fourth Edition)》(Sheldon Axler) P185 6.6
设\((V,\rho)\)是一个酉空间,
则对于任意\(\alpha \in V\),
映射\(x \mapsto \rho(x,\alpha)\)是\(V\)上的线性函数.
\end{property}

\begin{property}
%@see: 《Linear Algebra Done Right (Fourth Edition)》(Sheldon Axler) P185 6.6
设\((V,\rho)\)是一个酉空间,
则对于任意\(\alpha \in V\),
有\begin{equation*}
	\rho(0,\alpha) = \rho(\alpha,0) = 0.
\end{equation*}
\end{property}

\begin{property}\label{theorem:酉空间.复线性空间上内积对第二个自变量具有半线性性}
%@see: 《高等代数(第三版 下册)》(丘维声) P193
%@see: 《Linear Algebra Done Right (Fourth Edition)》(Sheldon Axler) P185 6.6
%@see: 《矩阵分析与应用(第2版)》(张贤达) P23 注释3
设\((V,\rho)\)是一个酉空间,
则内积\(\rho\)对第二个自变量具有\DefineConcept{半线性性},
即\begin{equation}
	\rho(\alpha,k_1\beta_1+k_2\beta_2)
	= \ComplexConjugate{k_1} \rho(\alpha,\beta_1)
	+ \ComplexConjugate{k_2} \rho(\alpha,\beta_2).
\end{equation}
\begin{proof}
由内积的厄米性和它对第一个自变量的线性性,有\begin{align*}
	\rho(\alpha,k_1\beta_1+k_2\beta_2)
	&= \ComplexConjugate{\rho(k_1\beta_1+k_2\beta_2,\alpha)} \\
	&= \ComplexConjugate{k_1 \rho(\beta_1,\alpha)}
		+ \ComplexConjugate{k_2 \rho(\beta_2,\alpha)} \\
	&= \ComplexConjugate{k_1} \rho(\alpha,\beta_1)
		+ \ComplexConjugate{k_2} \rho(\alpha,\beta_2).
	\qedhere
\end{align*}
\end{proof}
\end{property}
\begin{remark}
注意与\cref{theorem:实线性空间.实线性空间上内积对第二个自变量具有线性性} 进行对比.
\end{remark}

\begin{example}
%@see: 《高等代数(第三版 下册)》(丘维声) P193 例1
在\(V = \mathbb{C}^n\)中,
%@see: 《高等代数(第三版 下册)》(丘维声) P193 (1)
令\(f(X,Y) \defeq x_1 \ComplexConjugate{y_1} + \dotsb + x_n \ComplexConjugate{y_n}\),
其中\(X=(\AutoTuple{x}{n})^T,
Y=(\AutoTuple{y}{n})^T\).
容易验证,\(f\)是\(V\)上的一个内积.
我们把这个内积称为 \DefineConcept{\(\mathbb{C}^n\)上的标准内积}.
\end{example}

\begin{example}
%@see: 《高等代数(第三版 下册)》(丘维声) P194 例2
设\(V = \tilde{C}[a,b]\)表示区间\([a,b]\)上所有连续复值函数组成的线性空间.
%@see: 《高等代数(第三版 下册)》(丘维声) P194 (2)
令\(\rho(f,g) \defeq \int_a^b f(x) \ComplexConjugate{g(x)} \dd{x}\).
容易验证,\(\rho\)是\(V\)上的一个内积.
\end{example}

\begin{example}
%@see: 《高等代数(第三版 下册)》(丘维声) P194 例3
在\(V = M_n(\mathbb{C})\)中,
%@see: 《高等代数(第三版 下册)》(丘维声) P194 (3)
令\(f(A,B) \defeq \tr(A B^H)\).
容易验证,\(f\)是\(V\)上的一个内积.
\end{example}

\begin{definition}
%@see: 《高等代数(第三版 下册)》(丘维声) P194 定义2
%@see: 《Linear Algebra Done Right (Fourth Edition)》(Sheldon Axler) P186 6.7
设\((V,\rho)\)是一个酉空间,
\(\alpha \in V\).
把非负实数\(\sqrt{\rho(\alpha,\alpha)}\)
称为“向量\(\alpha\)的\DefineConcept{长度}”,
记作\(\VectorLengthA{\alpha}\)或\(\VectorLengthN{\alpha}\).
\end{definition}

\begin{property}\label{theorem:酉空间.向量的长度具有非负性}
%@see: 《高等代数(第三版 下册)》(丘维声) P194
%@see: 《Linear Algebra Done Right (Fourth Edition)》(Sheldon Axler) P186 6.9
在酉空间\(V\)中,
零向量的长度为\(0\),
非零向量的长度是正数.
\end{property}

\begin{property}\label{theorem:酉空间.向量的长度具有齐次性}
%@see: 《高等代数(第三版 下册)》(丘维声) P194
%@see: 《Linear Algebra Done Right (Fourth Edition)》(Sheldon Axler) P186 6.9
在酉空间\((V,\rho)\)中,
对于\(\forall \alpha \in V\)
和\(\forall k \in \mathbb{C}\),
有\(\VectorLengthA{k\alpha} = \ComplexLengthA{k} \VectorLengthA{\alpha}\).
\begin{proof}
%@see: 《高等代数(第三版 下册)》(丘维声) P194 (4)
\(\VectorLengthA{k\alpha}
= \sqrt{\rho(k\alpha,k\alpha)}
= \sqrt{k \ComplexConjugate{k} \rho(\alpha,\alpha)}
= \ComplexLengthA{k} \VectorLengthA{\alpha}\).
\end{proof}
\end{property}

\begin{theorem}
%@see: 《高等代数(第三版 下册)》(丘维声) P194 定理1(柯西-布尼亚科夫斯基不等式)
%@see: 《Linear Algebra Done Right (Fourth Edition)》(Sheldon Axler) P189 6.14
在酉空间\((V,\rho)\)中,
对于\(\forall \alpha,\beta \in V\),
有\begin{equation}
	\abs{\rho(\alpha,\beta)} \leq \VectorLengthA{\alpha} \VectorLengthA{\beta}.
\end{equation}
当且仅当\(\{\alpha,\beta\}\)线性相关时,上式取“\(=\)”号.
%TODO proof
\end{theorem}

\begin{definition}
%@see: 《高等代数(第三版 下册)》(丘维声) P194 定义3
设\(\alpha,\beta\)是酉空间\((V,\rho)\)中的两个非零向量.
把\begin{equation}
%@see: 《高等代数(第三版 下册)》(丘维声) P194 (7)
	\arccos\frac{\abs{\rho(\alpha,\beta)}}{\VectorLengthA{\alpha} \VectorLengthA{\beta}}
\end{equation}
称为“\(\alpha\)与\(\beta\)的\DefineConcept{夹角}”,
记为\(\VectorAngleA{\alpha}{\beta}\)或\(\VectorAngleP{\alpha}{\beta}\).
\end{definition}

\begin{property}
%@see: 《高等代数(第三版 下册)》(丘维声) P194
设\(\alpha,\beta\)是酉空间\((V,\rho)\)中的两个非零向量,
则\(\alpha\)与\(\beta\)的夹角\(\theta = \VectorAngleA{\alpha}{\beta}\)
满足\(0 \leq \theta \leq \pi/2\).
%TODO proof
\end{property}

\begin{property}
%@see: 《高等代数(第三版 下册)》(丘维声) P195
设\(\alpha,\beta\)是酉空间\((V,\rho)\)中的两个非零向量,
则\(\alpha\)与\(\beta\)的夹角\(\theta = \VectorAngleA{\alpha}{\beta}\)
满足\(\theta = \frac\pi2 \iff \rho(\alpha,\beta) = 0\).
%TODO proof
\end{property}

\begin{definition}
%@see: 《高等代数(第三版 下册)》(丘维声) P195 定义4
%@see: 《Linear Algebra Done Right (Fourth Edition)》(Sheldon Axler) P187 6.10
设\(\alpha,\beta\)是酉空间\((V,\rho)\)中的两个非零向量.
如果\(\rho(\alpha,\beta) = 0\),
则称“\(\alpha\)与\(\beta\) \DefineConcept{正交}(orthogonal)”,
记为\(\alpha \perp \beta\).
\end{definition}

\begin{property}
%@see: 《Linear Algebra Done Right (Fourth Edition)》(Sheldon Axler) P187 6.11
在酉空间\((V,\rho)\)中,零向量与任意一个向量正交.
\end{property}

\begin{property}\label{theorem:酉空间.酉空间中不存在非零迷向向量}
%@see: 《Linear Algebra Done Right (Fourth Edition)》(Sheldon Axler) P187 6.11
在酉空间\((V,\rho)\)中,只有零向量与它本身正交.
\end{property}
\begin{remark}
\cref{theorem:酉空间.酉空间中不存在非零迷向向量} 说明:酉空间中不存在非零迷向向量.
\end{remark}

\begin{corollary}\label{theorem:酉空间.三角不等式}
%@see: 《高等代数(第三版 下册)》(丘维声) P195
%@see: 《Linear Algebra Done Right (Fourth Edition)》(Sheldon Axler) P190 6.17
在酉空间\((V,\rho)\)中,
\DefineConcept{三角不等式}(triangle inequality)成立,
即对于\(\forall \alpha,\beta \in V\),
有\begin{equation}
	\VectorLengthA{\alpha+\beta} \leq \VectorLengthA{\alpha} + \VectorLengthA{\beta}.
\end{equation}
当且仅当\(\alpha = k\beta\)或\(\beta = k\alpha\)(其中\(k\geq0\))时,上式取“\(=\)”号.
%TODO proof
\end{corollary}

\begin{corollary}\label{theorem:酉空间.勾股定理}
%@see: 《高等代数(第三版 下册)》(丘维声) P195
%@see: 《Linear Algebra Done Right (Fourth Edition)》(Sheldon Axler) P187 6.12
在酉空间\((V,\rho)\)中,
\DefineConcept{勾股定理}成立,
即对于\(\forall \alpha,\beta \in V\),
如果\(\alpha\)与\(\beta\)正交,
则\begin{equation}
	\VectorLengthA{\alpha+\beta}^2 = \VectorLengthA{\alpha}^2 + \VectorLengthA{\beta}^2.
\end{equation}
\begin{proof}
证明过程与\cref{theorem:实内积空间.勾股定理} 相同.
\end{proof}
\end{corollary}

\begin{proposition}\label{theorem:酉空间.平行四边形等式}
%@see: 《Linear Algebra Done Right (Fourth Edition)》(Sheldon Axler) P191 6.21
在酉空间\((V,\rho)\)中,
\(\alpha,\beta \in V\),
则\begin{equation*}
	\VectorLengthA{\alpha+\beta}^2
	+ \VectorLengthA{\alpha-\beta}^2
	= 2(\VectorLengthA{\alpha}^2+\VectorLengthA{\beta}^2).
\end{equation*}
\begin{proof}
直接有\begin{align*}
	\VectorLengthA{\alpha+\beta}^2
	+ \VectorLengthA{\alpha-\beta}^2
	&= \rho(\alpha+\beta,\alpha+\beta) + \rho(\alpha-\beta,\alpha-\beta) \\
	&= (\VectorLengthA{\alpha}^2+\VectorLengthA{\beta}^2+\rho(\alpha,\beta)+\rho(\beta,\alpha)) \\
	&\hspace{20pt}+(\VectorLengthA{\alpha}^2+\VectorLengthA{\beta}^2-\rho(\alpha,\beta)-\rho(\beta,\alpha)) \\
	&= 2(\VectorLengthA{\alpha}^2+\VectorLengthA{\beta}^2).
	\qedhere
\end{align*}
\end{proof}
\end{proposition}

\begin{proposition}\label{theorem:酉空间.向量的正交分解}
%@see: 《Linear Algebra Done Right (Fourth Edition)》(Sheldon Axler) P188 6.13
在酉空间\((V,\rho)\)中,
\(\alpha,\beta \in V\),
且\(\beta \neq 0\).
令\begin{equation*}
	k \defeq \frac{\rho(\alpha,\beta)}{\rho(\beta,\beta)},
	\qquad
	\gamma \defeq \alpha - k \beta,
\end{equation*}
则\(\alpha = k\beta + \gamma\)且\(\rho(\beta,\gamma) = 0\).
\end{proposition}
\begin{remark}
将一个向量\(\alpha\)化为两个正交向量之和\(k\beta + \gamma\)的过程,
称为\DefineConcept{正交分解}(orthogonal decomposition).
\end{remark}

\begin{definition}
%@see: 《高等代数(第三版 下册)》(丘维声) P195
在酉空间\((V,\rho)\)中,
\(\alpha,\beta \in V\).
把\(\VectorLengthA{\alpha-\beta}\)
称为“\(\alpha\)与\(\beta\)的\DefineConcept{距离}”,
记为\(d(\alpha,\beta)\).
\end{definition}

\subsection{有限维酉空间中的基}
我们希望在有限维酉空间\(V\)中找出一类基,
使得在这样的基下容易计算\(V\)中任意两个向量的内积,
从而易于计算长度、角度、距离等.

\begin{definition}
%@see: 《高等代数(第三版 下册)》(丘维声) P195
设\((V,\rho)\)是一个有限维酉空间,
\(A\)是\(V\)中的一个向量组,
如果\begin{equation*}
	(\forall \alpha \in A)
	[\alpha\neq0],
	\qquad
	(\forall \alpha,\beta \in A)
	[\alpha \perp \beta],
\end{equation*}
则称“\(A\)是\((V,\rho)\)中的一个\DefineConcept{正交向量组}(orthogonal list)”.
\end{definition}

\begin{definition}
%@see: 《高等代数(第三版 下册)》(丘维声) P195
%@see: 《Linear Algebra Done Right (Fourth Edition)》(Sheldon Axler) P197 6.22
设\((V,\rho)\)是一个有限维酉空间,
\(A\)是\(V\)中的一个向量组,
如果\begin{equation*}
	(\forall \alpha \in A)
	[\VectorLengthA{\alpha} = 1],
	\qquad
	(\forall \alpha,\beta \in A)
	[\alpha \perp \beta],
\end{equation*}
则称“\(A\)是\((V,\rho)\)中的一个\DefineConcept{正交单位向量组}(orthonormal list)”.
\end{definition}

\begin{proposition}\label{theorem:酉空间.正交向量组的线性组合的长度}
%@see: 《Linear Algebra Done Right (Fourth Edition)》(Sheldon Axler) P198 6.24
设\(\AutoTuple{\alpha}{s}\)是酉空间\((V,\rho)\)中一个正交向量组,
那么对于\(\forall \AutoTuple{k}{s} \in F\),
有\begin{equation*}
	\VectorLengthA{
		k_1 \alpha_1 + \dotsb + k_s \alpha_s
	}^2
	= \abs{k_1}^2 \VectorLengthA{\alpha_1}^2 + \dotsb + \abs{k_s}^2 \VectorLengthA{\alpha_s}^2.
\end{equation*}
\begin{proof}
由\hyperref[theorem:酉空间.勾股定理]{勾股定理}和\cref{theorem:酉空间.向量的长度具有齐次性} 立即可得.
\end{proof}
\end{proposition}

\begin{proposition}
%@see: 《Linear Algebra Done Right (Fourth Edition)》(Sheldon Axler) P198 6.25
在酉空间\((V,\rho)\)中,
任意一个正交向量组一定线性无关.
%TODO proof
\end{proposition}

\begin{definition}
%@see: 《高等代数(第三版 下册)》(丘维声) P195
%@see: 《Linear Algebra Done Right (Fourth Edition)》(Sheldon Axler) P199 6.27
设\((V,\rho)\)是\(n\)维酉空间,
\(A\)是\(V\)的一个基.
如果\(A\)是正交向量组,
则称“\(A\)是\((V,\rho)\)的一个\DefineConcept{正交基}(orthonormal basis)”.
\end{definition}

\begin{definition}
%@see: 《高等代数(第三版 下册)》(丘维声) P195
设\((V,\rho)\)是\(n\)维酉空间,
\(A\)是\(V\)的一个基.
如果\(A\)是正交单位向量组,
则称“\(A\)是\((V,\rho)\)的一个\DefineConcept{标准正交基}”
或“\(A\)是\((V,\rho)\)的一个\DefineConcept{规范正交基}”.
\end{definition}

% 与《高等代数(第三版 上册)》第4章第6节定理4的证明方法完全一样,
在酉空间\(V\)中也有施密特正交化过程,
它可以把一个线性无关向量组变成与之等价的正交向量组.

在\(n\)维酉空间\(V\)中,取一个基\(\AutoTuple{\alpha}{n}\),
经过施密特正交化把它变成正交基\(\AutoTuple{\beta}{n}\),
再经过单位化把它变成标准正交基\(\AutoTuple{\gamma}{n}\).
这就说明:
\begin{theorem}
%@see: 《Linear Algebra Done Right (Fourth Edition)》(Sheldon Axler) P202 6.35
有限维酉空间中,一定存在标准正交基.
%TODO proof
\end{theorem}

\begin{proposition}
%@see: 《Linear Algebra Done Right (Fourth Edition)》(Sheldon Axler) P203 6.36
有限维酉空间\(V\)中的任意一个正交单位向量组
均可以扩充成\(V\)的一个标准正交基.
%TODO proof
\end{proposition}

\subsection{酉空间中向量的内积、傅里叶展开}
\begin{proposition}
%@see: 《高等代数(第三版 下册)》(丘维声) P195
设\((V,\rho)\)是\(n\)维酉空间,
则向量组\(\AutoTuple{\eta}{n}\)是\((V,\rho)\)的一个标准正交基,
当且仅当\begin{equation*}
%@see: 《高等代数(第三版 下册)》(丘维声) P195 (8)
	\rho(\eta_i,\eta_j)
	= \delta(i,j),
	\quad i,j=1,2,\dotsc,n,
\end{equation*}
其中\(\delta\)是克罗内克\(\delta\)函数.
\end{proposition}

利用标准正交基,容易计算向量的内积.

设\((V,\rho)\)是\(n\)维酉空间,
向量\(\alpha,\beta \in V\),
向量组\(\AutoTuple{\eta}{n}\)是\((V,\rho)\)的一个标准正交基,
\(\alpha,\beta\)在基\(\AutoTuple{\eta}{n}\)下的坐标
分别是\(X=(\AutoTuple{x}{n})^T,
Y=(\AutoTuple{y}{n})^T\),
则\begin{equation*}
%@see: 《高等代数(第三版 下册)》(丘维声) P195 (9)
	\rho(\alpha,\beta)
	= \rho\left( \sum_{i=1}^n x_i \eta_i, \sum_{j=1}^n y_j \eta_j \right)
	= \sum_{i=1}^n \sum_{j=1}^n x_i \ComplexConjugate{y_j} \rho(\eta_i,\eta_j)
	= \sum_{i=1}^n x_i \ComplexConjugate{y_i}
	= Y^H X.
\end{equation*}
%@see: 《Linear Algebra Done Right (Fourth Edition)》(Sheldon Axler) P200 6.30(c)
上式还可以写成\begin{equation}
	\rho(\alpha,\beta)
	= \sum_{i=1}^n \rho(\alpha,\eta_i) \ComplexConjugate{\rho(\beta,\eta_i)}.
\end{equation}

利用标准正交基,可以借助内积,表达向量的坐标分量.

设\(\alpha\)在标准正交基\(\AutoTuple{\eta}{n}\)下的坐标为\(X=(\AutoTuple{x}{n})^T\),
则\begin{equation*}
	\alpha = \sum_{i=1}^n x_i \eta_i;
\end{equation*}
等号两边用\(\eta_j\)作内积,得\begin{equation*}
	\rho(\alpha,\eta_j)
	= \rho\left( \sum_{i=1}^n x_i \eta_i, \eta_j \right)
	= \sum_{i=1}^n x_i \rho(\eta_i,\eta_j)
	= x_j,
\end{equation*}
因此\begin{equation}\label{equation:酉空间.向量的傅里叶展开}
%@see: 《高等代数(第三版 下册)》(丘维声) P196 (10)
%@see: 《Linear Algebra Done Right (Fourth Edition)》(Sheldon Axler) P200 6.30(a)
	\alpha = \sum_{i=1}^n \rho(\alpha,\eta_i) \eta_i.
\end{equation}
我们把\cref{equation:酉空间.向量的傅里叶展开}
称为“向量\(\alpha\)的\DefineConcept{傅里叶展开}”,
其中系数\(\rho(\alpha,\eta_i)\ (i=1,2,\dotsc,n)\)
称为“向量\(\alpha\)的\DefineConcept{傅里叶系数}”.

由\cref{equation:酉空间.向量的傅里叶展开,theorem:酉空间.正交向量组的线性组合的长度} 可得\begin{equation}
%@see: 《Linear Algebra Done Right (Fourth Edition)》(Sheldon Axler) P200 6.30(b)
	\VectorLengthA{\alpha}^2
	= \sum_{i=1}^n \rho^2(\alpha,\eta_i).
\end{equation}

\begin{proposition}
%@see: 《高等代数(第三版 下册)》(丘维声) P196
在有限维酉空间\((V,\rho)\)中,
一个标准正交基到另一个标准正交基的过渡矩阵
一定是酉矩阵.
%TODO proof
\end{proposition}

\begin{proposition}
%@see: 《高等代数(第三版 下册)》(丘维声) P196
设\((V,\rho)\)是\(n\)维酉空间,
向量组\(\AutoTuple{\eta}{n}\)是\((V,\rho)\)的一个标准正交基,
\(\AutoTuple{\beta}{n}\)是\(V\)中一个向量组.
如果存在酉矩阵\(P \in M_n(\mathbb{C})\),
使得\begin{equation*}
	(\AutoTuple{\beta}{n})
	= (\AutoTuple{\eta}{n}) P,
\end{equation*}
那么\(\AutoTuple{\beta}{n}\)是\(V\)中一个标准正交基.
%TODO proof
\end{proposition}

\subsection{酉空间之间的同构}
\begin{definition}
%@see: 《高等代数(第三版 下册)》(丘维声) P196
设\((V_1,\rho_1),(V_2,\rho_2)\)都是酉空间.
如果存在从\(V_1\)到\(V_2\)的一个双射\(\sigma\),
使得\begin{gather*}
	(\forall \alpha,\beta \in V_1)
	[
		\sigma(\alpha+\beta)
		= \sigma(\alpha) + \sigma(\beta)
	], \\
	(\forall \alpha \in V_1)
	(\forall k \in \mathbb{C})
	[
		\sigma(k\alpha)
		= k \sigma(\alpha)
	], \\
	(\forall \alpha,\beta \in V_1)
	[
		\rho_2(\sigma(\alpha),\sigma(\beta))
		= \rho_1(\alpha,\beta)
	],
\end{gather*}
则称“\(\sigma\)是从\(V_1\)到\(V_2\)的一个\DefineConcept{同构}”;
并称“\(V_1\)与\(V_2\) \DefineConcept{同构}”,
记为\(V_1 \Isomorphism V_2\).
\end{definition}

\begin{theorem}\label{theorem:酉空间.两个酉空间同构的充分必要条件}
%@see: 《高等代数(第三版 下册)》(丘维声) P196
两个酉空间同构的充分必要条件是它们的维数相同.
%TODO proof
\end{theorem}

\subsection{酉变换}
\begin{definition}
%@see: 《高等代数(第三版 下册)》(丘维声) P197 定义6
设\((V,\rho)\)是一个酉空间,
\(\vb{A}\)是一个从\(V\)到\(V\)的满射.
如果\(\vb{A}\)满足\begin{equation}
	(\forall \alpha,\beta \in V)
	[
		\rho(\vb{A}\alpha,\vb{A}\beta)
		= \rho(\alpha,\beta)
	],
\end{equation}
则称“\(\vb{A}\)保持向量的内积不变”
“\(\vb{A}\)是\(V\)上的一个\DefineConcept{酉变换}”.
\end{definition}

\begin{proposition}
%@see: 《高等代数(第三版 下册)》(丘维声) P197 命题2
酉空间上的酉变换一定是可逆线性变换.
%TODO proof
\end{proposition}

\begin{proposition}
%@see: 《高等代数(第三版 下册)》(丘维声) P197 命题3
设\(\vb{A}\)是\(n\)维酉空间\(V\)上的一个线性变换,
则\begin{align*}
	&\text{$\vb{A}$是酉变换} \\
	&\iff \text{$\vb{A}$把$V$的标准正交基映成标准正交基} \\
	&\iff \text{$\vb{A}$在$V$的标准正交基下的矩阵是酉矩阵}.
\end{align*}
\end{proposition}

\begin{proposition}
%@see: 《高等代数(第三版 下册)》(丘维声) P197 命题4
有限维酉空间上的酉变换的特征值的模等于\(1\).
\begin{proof}
由\cref{equation:幺正矩阵.幺正矩阵的行列式} 立即可得.
\end{proof}
\end{proposition}

\begin{proposition}
%@see: 《高等代数(第三版 下册)》(丘维声) P197 命题5
设\(\vb{A}\)是酉空间\((V,\rho)\)上的一个酉变换.
如果\(W\)是\(\vb{A}\)的有限维不变子空间,
则\(W\)的正交补\(W^\perp\)也是\(\vb{A}\)的不变子空间.
\begin{proof}
任取\(\beta \in W^\perp\).
要证\(\vb{A}\beta \in W^\perp\).
任取\(\alpha \in W\),
由于酉变换\(\vb{A}\)是可逆的,
因此由\cref{example:线性映射.可逆线性变换的逆变换的不变子空间} 可知
\(W\)也是\(\vb{A}^{-1}\)的不变子空间,
从而\(\vb{A}^{-1}\alpha \in W\).
于是\begin{equation*}
	\rho(\vb{A}\beta,\alpha)
	= \rho(\vb{A}\beta,\vb{A}\vb{A}^{-1}\alpha)
	= \rho(\beta,\vb{A}^{-1}\alpha)
	= 0,
\end{equation*}
即\(\vb{A}\beta \in W^\perp\),
说明\(W^\perp\)是\(\vb{A}\)的不变子空间.
\end{proof}
\end{proposition}

\begin{theorem}
%@see: 《高等代数(第三版 下册)》(丘维声) P197 定理6
设\(\vb{A}\)是\(n\)维酉空间\(V\)上的酉变换,
则\(V\)中存在一个标准正交基\(S\),
使得\(\vb{A}\)在基\(S\)下的矩阵是对角矩阵,
且主对角元都是模为\(1\)的复数.
%TODO proof
\end{theorem}

\begin{definition}
%@see: 《高等代数(第三版 下册)》(丘维声) P198 推论7
设矩阵\(A,B \in M_n(\mathbb{C})\).
若存在可逆矩阵\(P \in M_n(\mathbb{C})\),
使得\begin{equation}
	P^{-1} A P = B,
\end{equation}
则称“\(A\)与\(B\) \DefineConcept{酉相似}”
或“\(A\) \DefineConcept{酉相似于} \(B\)”.
\end{definition}

\begin{corollary}
%@see: 《高等代数(第三版 下册)》(丘维声) P198 推论7
任意一个\(n\)阶酉矩阵一定酉相似于某个主对角元都是模为\(1\)的复数的对角矩阵.
%TODO proof
\end{corollary}

\subsection{厄米变换}
类比于实内积空间\(V\)上的对称变换,引出酉空间上的下述变换:
\begin{definition}
%@see: 《高等代数(第三版 下册)》(丘维声) P198 定义7
设\(\vb{A}\)是酉空间\((V,\rho)\)上的一个线性变换.
如果\begin{equation*}
	(\forall \alpha,\beta \in V)
	[
		\rho(\vb{A}\alpha,\beta)
		= \rho(\alpha,\vb{A}\beta)
	],
\end{equation*}
则称“\(\vb{A}\)是\(V\)上的一个\DefineConcept{厄米变换}”
或“\(\vb{A}\)是\(V\)上的一个\DefineConcept{自伴随变换}”.
\end{definition}

\begin{proposition}
%@see: 《高等代数(第三版 下册)》(丘维声) P198 命题8
酉空间上的厄米变换一定是线性变换.
%TODO proof
\end{proposition}

\begin{proposition}
%@see: 《高等代数(第三版 下册)》(丘维声) P198 命题9
设\(\vb{A}\)是\(n\)维酉空间\(V\)上的一个线性变换,
则\(\vb{A}\)是厄米变换,
当且仅当\(\vb{A}\)在\(V\)的某个标准正交基下的矩阵是厄米矩阵.
%TODO proof
\end{proposition}

我们来探索\(n\)维酉空间上的厄米变换的最简单形式的矩阵表示.
\begin{proposition}
%@see: 《高等代数(第三版 下册)》(丘维声) P199 命题10
酉空间上的厄米变换的特征值只要存在就一定是实数.
%TODO proof
\end{proposition}

\begin{proposition}
%@see: 《高等代数(第三版 下册)》(丘维声) P199 命题11
设\(\vb{A}\)是酉空间\(V\)上的一个厄米变换.
如果\(W\)是\(\vb{A}\)的一个不变子空间,
则\(W\)的正交补\(W^\perp\)也是\(\vb{A}\)的一个不变子空间.
\end{proposition}

\begin{theorem}
%@see: 《高等代数(第三版 下册)》(丘维声) P199 定理12
设\(\vb{A}\)是酉空间\(V\)上的一个厄米变换,
则\(V\)中存在一个标准正交基\(S\),
使得\(\vb{A}\)在基\(S\)下的矩阵是对角矩阵,
且主对角元都是实数.
%TODO proof
\end{theorem}

\begin{corollary}
%@see: 《高等代数(第三版 下册)》(丘维声) P199 推论13
任意一个\(n\)阶厄米矩阵一定酉相似于某个实对角矩阵.
\end{corollary}
