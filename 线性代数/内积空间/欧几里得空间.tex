\section{欧几里得空间}
\subsection{实线性空间上的内积}
\begin{definition}\label{definition:欧几里得空间.实线性空间上的内积}
设\(V\)是实数域\(\mathbb{R}\)上的一个线性空间.
如果映射\(\rho\colon V \times V \to \mathbb{R}\)满足以下性质\begin{itemize}
	\item {\rm\bf 对称性}:\begin{equation*}
		(\forall \alpha,\beta \in V)
		[
			\rho(\alpha,\beta)
			= \rho(\beta,\alpha)
		];
	\end{equation*}

	\item {\rm\bf 线性性}:\begin{gather*}
		(\forall \alpha,\beta,\gamma \in V)
		[
			\rho(\alpha+\beta,\gamma)
			= \rho(\alpha,\gamma) + \rho(\beta,\gamma)
		], \\
		(\forall \alpha,\beta \in V)
		(\forall k \in \mathbb{C})
		[
			\rho(k\alpha,\beta)
			= k\rho(\alpha,\beta)
		];
	\end{gather*}

	\item {\rm\bf 正定性}:\begin{gather*}
		(\forall \alpha \in V)
		[
			\rho(\alpha,\alpha) \geq 0
		], \\
		(\forall \alpha \in V)
		[
			\rho(\alpha,\alpha) = 0
			\iff
			\alpha = 0
		],
	\end{gather*}
\end{itemize}
则称“\(f\)是实线性空间\(V\)上的一个\DefineConcept{内积}(inner product)”.
\end{definition}

虽然在\hyperref[definition:欧几里得空间.实线性空间上的内积]{定义}中,
我们只规定了内积对第一个自变量具有线性性,
但是我们容易证明,内积对第二个自变量也具有线性性:
\begin{property}\label{theorem:实线性空间.实线性空间上内积对第二个自变量具有线性性}
%@see: 《高等代数(第三版 下册)》(丘维声) P193
%@see: 《Linear Algebra Done Right (Fourth Edition)》(Sheldon Axler) P185 6.6
设\(\rho\)是实线性空间\(V\)上的一个内积,
则内积\(\rho\)对第二个自变量具有线性性,
即\begin{equation}
	\rho(\alpha,k_1\beta_1+k_2\beta_2)
	= k_1 \rho(\alpha,\beta_1)
	+ k_2 \rho(\alpha,\beta_2).
\end{equation}
\begin{proof}
由内积的对称性和它对第一个自变量的线性性,有\begin{align*}
	\rho(\alpha,k_1\beta_1+k_2\beta_2)
	&= \rho(k_1\beta_1+k_2\beta_2,\alpha) \\
	&= k_1 \rho(\beta_1,\alpha)
		+ k_2 \rho(\beta_2,\alpha) \\
	&= k_1 \rho(\alpha,\beta_1)
		+ k_2 \rho(\alpha,\beta_2).
	\qedhere
\end{align*}
\end{proof}
\end{property}

\begin{proposition}\label{theorem:欧几里得空间.实内积空间上的内积的等价定义}
%@see: 《高等代数(第三版 下册)》(丘维声) P174 定义1
设\(V\)是实数域\(\mathbb{R}\)上的一个线性空间,
则“\(f\)是实线性空间\(V\)上的一个内积”的充分必要条件是
“\(f\)是\(V\)上的一个正定对称双线性函数”.
\end{proposition}

\begin{proposition}
%@see: 《高等代数(第三版 下册)》(丘维声) P174 命题1
设\(V\)是\(\mathbb{R}\)上的一个\(n\)维线性空间,
\(f\)是\(V\)上的一个双线性函数,
则\(f\)是正定对称的,
当且仅当\(f\)在\(V\)的某个基下的度量矩阵是正定对称的.
%TODO proof
\end{proposition}

\begin{example}
%@see: 《高等代数(第三版 下册)》(丘维声) P174 例1
在\(V = \mathbb{R}^3\)中,
令\(f(\alpha,\beta) \defeq a_1 b_1 + 2 a_2 b_2 + 3 a_3 b_3\),
其中\(\alpha=(\AutoTuple{a}{3})^T,
\beta=(\AutoTuple{b}{3})^T\).
容易验证,\(f\)是\(V\)上的一个内积.
\end{example}

\begin{example}
%@see: 《高等代数(大学高等代数课程创新教材 第二版 下册)》(丘维声) P461 例1
在\(V = \mathbb{R}^n\)中,
令\(f(\alpha,\beta) \defeq a_1 b_1 + a_2 b_2 + \dotsb + a_n b_n\),
其中\(\alpha=(\AutoTuple{a}{n})^T,
\beta=(\AutoTuple{b}{n})^T\).
容易验证,\(f\)是\(V\)上的一个内积.
我们把这个内积称为 \DefineConcept{\(\mathbb{R}^n\)上的标准内积}.
\end{example}

\begin{example}
%@see: 《高等代数(第三版 下册)》(丘维声) P174 例2
在\(V = M_n(\mathbb{R})\)中,
令\(f(A,B) \defeq \tr(AB^T)\).
容易验证,\(f\)是\(V\)上的一个内积.
\end{example}
\begin{example}
在\(V = M_{s \times n}(\mathbb{R})\)中,
令\(f(A,B) \defeq \tr(A^T B)\).
容易验证,\(f\)是\(V\)上的一个内积.
\end{example}

\begin{example}
%@see: 《高等代数(第三版 下册)》(丘维声) P174 例3
在\(V = C[a,b]\)中,
令\(\rho(f,g) \defeq \int_a^b f(x) g(x) \dd{x}\).
容易验证,\(\rho\)是\(V\)上的一个内积.
\end{example}

\subsection{实内积空间,欧几里得空间}
\begin{definition}
%@see: 《高等代数(第三版 下册)》(丘维声) P174 定义2
%@see: 《Linear Algebra Done Right (Fourth Edition)》(Sheldon Axler) P184 6.4
设\(V\)是一个实线性空间,
\(\rho\)是\(V\)上的一个内积,
则称“\((V,\rho)\)是一个\DefineConcept{实内积空间}(real inner product space)”.
\end{definition}

\begin{definition}
%@see: 《高等代数(第三版 下册)》(丘维声) P174 定义2
设\(V\)是一个实内积空间.
如果\(V\)是有限维的,
则称“\(V\)是一个\DefineConcept{欧几里得空间}(Euclidean space)”;
把线性空间\(V\)的维数称为“欧几里得空间\(V\)的\DefineConcept{维数}”.
\end{definition}

\begin{definition}
%@see: 《高等代数(第三版 下册)》(丘维声) P174 定义3
设\((V,\rho)\)是一个实内积空间,
\(\alpha \in V\).
把非负实数\(\sqrt{\rho(\alpha,\alpha)}\)
称为“向量\(\alpha\)的\DefineConcept{长度}”,
记作\(\VectorLengthA{\alpha}\)或\(\VectorLengthN{\alpha}\).
\end{definition}

\begin{property}\label{theorem:实内积空间.向量的长度具有非负性}
%@see: 《高等代数(第三版 下册)》(丘维声) P175
在实内积空间\(V\)中,
零向量的长度为\(0\),
非零向量的长度是正数.
\end{property}

\begin{property}\label{theorem:实内积空间.向量的长度具有齐次性}
%@see: 《高等代数(第三版 下册)》(丘维声) P175
在实内积空间\((V,\rho)\)中,
对于\(\forall \alpha \in V\)
和\(\forall k \in \mathbb{R}\),
有\(\VectorLengthA{k\alpha} = \abs{k} \VectorLengthA{\alpha}\).
\begin{proof}
\(\VectorLengthA{k\alpha}
= \sqrt{\rho(k\alpha,k\alpha)}
= \sqrt{k^2\rho(\alpha,\alpha)}
= \abs{k} \VectorLengthA{\alpha}\).
\end{proof}
\end{property}

\begin{definition}
%@see: 《高等代数(第三版 下册)》(丘维声) P175
设\((V,\rho)\)是实内积空间,
\(\alpha \in V\).
如果\(\VectorLengthA{\alpha} = 1\),
则称“\(\alpha\)是一个\DefineConcept{单位向量}”.
\end{definition}

\begin{property}
%@see: 《高等代数(第三版 下册)》(丘维声) P175
设\((V,\rho)\)是实内积空间,
\(\alpha \in V\).
如果\(\alpha\neq0\),
则\(\frac1{\VectorLengthA{\alpha}} \alpha\)是一个单位向量.
\end{property}
\begin{remark}
将一个非零向量\(\alpha\)变成单位向量\(\frac1{\VectorLengthA{\alpha}} \alpha\)的过程,
称为\DefineConcept{单位化}.
\end{remark}

\begin{theorem}
%@see: 《高等代数(第三版 下册)》(丘维声) P175 定理2(柯西-布尼亚科夫斯基不等式)
在实内积空间\((V,\rho)\)中,
对于\(\forall \alpha,\beta \in V\),
有\begin{equation}
	\abs{\rho(\alpha,\beta)} \leq \VectorLengthA{\alpha} \VectorLengthA{\beta}.
\end{equation}
当且仅当\(\{\alpha,\beta\}\)线性相关时,上式取“\(=\)”号.
%TODO proof
\end{theorem}

\begin{definition}
%@see: 《高等代数(第三版 下册)》(丘维声) P175 定义4
在实内积空间\((V,\rho)\)中,
\(\alpha,\beta\)是两个非零向量.
把\begin{equation}
%@see: 《高等代数(第三版 下册)》(丘维声) P175 (6)
	\arccos\frac{\rho(\alpha,\beta)}{\VectorLengthA{\alpha} \VectorLengthA{\beta}}
\end{equation}
称为“\(\alpha\)与\(\beta\)的\DefineConcept{夹角}”,
记为\(\VectorAngleA{\alpha}{\beta}\)或\(\VectorAngleP{\alpha}{\beta}\).
\end{definition}

\begin{property}
%@see: 《高等代数(第三版 下册)》(丘维声) P175
设\(\alpha,\beta\)是实内积空间\((V,\rho)\)中的两个非零向量,
则\(\alpha\)与\(\beta\)的夹角\(\theta = \VectorAngleA{\alpha}{\beta}\)
满足\(0 \leq \theta \leq \pi\).
%TODO proof
\end{property}

\begin{property}
%@see: 《高等代数(第三版 下册)》(丘维声) P175
设\(\alpha,\beta\)是实内积空间\((V,\rho)\)中的两个非零向量,
则\(\alpha\)与\(\beta\)的夹角\(\theta = \VectorAngleA{\alpha}{\beta}\)
满足\(\theta = \frac\pi2 \iff \rho(\alpha,\beta) = 0\).
%TODO proof
\end{property}

\begin{definition}
%@see: 《高等代数(第三版 下册)》(丘维声) P175 定义5
设\(\alpha,\beta\)是实内积空间\((V,\rho)\)中的两个非零向量.
如果\(\rho(\alpha,\beta) = 0\),
则称“\(\alpha\)与\(\beta\) \DefineConcept{正交}(orthogonal)”,
记为\(\alpha \perp \beta\).
\end{definition}

\begin{corollary}\label{theorem:实内积空间.三角不等式}
%@see: 《高等代数(第三版 下册)》(丘维声) P175 推论3
在实内积空间\((V,\rho)\)中,
\DefineConcept{三角不等式}(triangle inequality)成立,
即对于\(\forall \alpha,\beta \in V\),
有\begin{equation}
%@see: 《高等代数(第三版 下册)》(丘维声) P175 (7)
	\VectorLengthA{\alpha+\beta} \leq \VectorLengthA{\alpha} + \VectorLengthA{\beta}.
\end{equation}
当且仅当\(\alpha = k\beta\)或\(\beta = k\alpha\)(其中\(k\geq0\))时,上式取“\(=\)”号.
%TODO proof
\end{corollary}

\begin{corollary}\label{theorem:实内积空间.勾股定理}
%@see: 《高等代数(第三版 下册)》(丘维声) P176 推论4
%@see: 《Linear Algebra Done Right (Fourth Edition)》(Sheldon Axler) P187 6.12
在实内积空间\((V,\rho)\)中,
\DefineConcept{勾股定理}成立,
即对于\(\forall \alpha,\beta \in V\),
如果\(\alpha\)与\(\beta\)正交,
则\begin{equation}
%@see: 《高等代数(第三版 下册)》(丘维声) P176 (8)
	\VectorLengthA{\alpha+\beta}^2 = \VectorLengthA{\alpha}^2 + \VectorLengthA{\beta}^2.
\end{equation}
\begin{proof}
假设\(\alpha\)与\(\beta\)正交,
那么\(\rho(\alpha,\beta) = 0\),
从而\(\VectorLengthA{\alpha+\beta}^2
= \rho(\alpha+\beta,\alpha+\beta)
= \rho(\alpha,\alpha) + \rho(\beta,\beta) + \rho(\alpha,\beta) + \rho(\beta,\alpha)
= \VectorLengthA{\alpha}^2 + \VectorLengthA{\beta}^2\).
\end{proof}
\end{corollary}

\begin{definition}
%@see: 《高等代数(第三版 下册)》(丘维声) P176 定义6
在实内积空间\((V,\rho)\)中,
\(\alpha,\beta \in V\).
把\(\VectorLengthA{\alpha-\beta}\)
称为“\(\alpha\)与\(\beta\)的\DefineConcept{距离}”,
记为\(d(\alpha,\beta)\).
\end{definition}

\begin{property}
%@see: 《高等代数(第三版 下册)》(丘维声) P176
在实内积空间\((V,\rho)\)中,
距离\(d\colon V \times V \to \mathbb{R},
(\alpha,\beta) \mapsto \VectorLengthA{\alpha-\beta}\)
满足以下性质:\begin{itemize}
	\item {\rm\bf 对称性}:\begin{equation*}
		(\forall \alpha,\beta \in V)
		[
			d(\alpha,\beta) = d(\beta,\alpha)
		];
	\end{equation*}

	\item {\rm\bf 正定性}:\begin{equation*}
		(\forall \alpha,\beta \in V)
		[
			d(\alpha,\beta) \geq 0
		];
	\end{equation*}
	当且仅当\(\alpha = \beta\)时,上式取“\(=\)”号;

	\item {\rm\bf 三角不等式}:\begin{equation*}
		(\forall \alpha,\beta,\gamma \in V)
		[
			d(\alpha,\gamma) \leq d(\alpha,\beta) + d(\beta,\gamma)
		].
	\end{equation*}
\end{itemize}
\end{property}

\subsection{欧几里得空间中的基}
我们希望在欧几里得空间\(V\)中找出一类基,
使得在这样的基下容易计算\(V\)中任意两个向量的内积,
从而易于计算长度、角度、距离等.

\begin{definition}
%@see: 《高等代数(第三版 下册)》(丘维声) P176
设\((V,\rho)\)是一个欧几里得空间,
\(A\)是\(V\)中的一个向量组,
如果\begin{equation*}
	(\forall \alpha \in A)
	[\alpha\neq0],
	\qquad
	(\forall \alpha,\beta \in A)
	[\alpha \perp \beta],
\end{equation*}
则称“\(A\)是\((V,\rho)\)中的一个\DefineConcept{正交向量组}”.
\end{definition}

\begin{definition}
%@see: 《高等代数(第三版 下册)》(丘维声) P176
设\((V,\rho)\)是一个欧几里得空间,
\(A\)是\(V\)中的一个向量组,
如果\begin{equation*}
	(\forall \alpha \in A)
	[\VectorLengthA{\alpha} = 1],
	\qquad
	(\forall \alpha,\beta \in A)
	[\alpha \perp \beta],
\end{equation*}
则称“\(A\)是\((V,\rho)\)中的一个\DefineConcept{正交单位向量组}”.
\end{definition}

\begin{proposition}
%@see: 《高等代数(第三版 下册)》(丘维声) P176 命题5
在欧几里得空间\((V,\rho)\)中,
任意一个正交向量组一定线性无关.
%TODO proof
\end{proposition}

\begin{definition}
%@see: 《高等代数(第三版 下册)》(丘维声) P176
%@see: 《Linear Algebra Done Right (Fourth Edition)》(Sheldon Axler) P199 6.27
设\((V,\rho)\)是\(n\)维欧几里得空间,
\(A\)是\(V\)的一个基.
如果\(A\)是正交向量组,
则称“\(A\)是\((V,\rho)\)的一个\DefineConcept{正交基}(orthonormal basis)”.
\end{definition}

\begin{definition}
%@see: 《高等代数(第三版 下册)》(丘维声) P176
设\((V,\rho)\)是\(n\)维欧几里得空间,
\(A\)是\(V\)的一个基.
如果\(A\)是正交单位向量组,
则称“\(A\)是\((V,\rho)\)的一个\DefineConcept{标准正交基}”
或“\(A\)是\((V,\rho)\)的一个\DefineConcept{规范正交基}”.
\end{definition}

\begin{theorem}\label{theorem:欧几里得空间.欧几里得空间中向量组的施密特正交化}
%@see: 《高等代数(第三版 下册)》(丘维声) P176 定理6
%@see: 《Linear Algebra Done Right (Fourth Edition)》(Sheldon Axler) P201 6.32
设\((V,\rho)\)是欧几里得空间,
\(\AutoTuple{\alpha}{s}\)是\(V\)中一个向量组,
令\begin{equation*}
%@see: 《高等代数(第三版 下册)》(丘维声) P176 (10)
	\beta_i
	= \alpha_i
		- \sum_{j=1}^{s-1} \frac{\rho(\alpha_i,\beta_j)}{\rho(\beta_j,\beta_j)} \beta_j,
	\quad i=1,2,\dotsc,s,
\end{equation*}
则\(\AutoTuple{\beta}{s}\)是正交向量组,
并且\(\AutoTuple{\alpha}{s}\)与\(\AutoTuple{\beta}{s}\)等价.
\end{theorem}
\begin{remark}
\cref{theorem:欧几里得空间.欧几里得空间中向量组的施密特正交化} 中,
把线性无关向量组\(\AutoTuple{\alpha}{s}\)
变成与它等价的正交向量组\(\AutoTuple{\beta}{s}\)的过程,
称为\DefineConcept{施密特正交化}.
只要再将\(\beta_i\)单位化,
就可以得到一个与\(\AutoTuple{\alpha}{s}\)等价的正交单位向量组\(\AutoTuple{\eta}{s}\).
因此,给定\(n\)维欧几里得空间\(V\)中一个基\(\AutoTuple{\alpha}{n}\),
经过施密特正交化、单位化,就可以得到\(V\)的一个标准正交基\(\AutoTuple{\eta}{n}\).
\end{remark}

在\(n\)维欧几里得空间\(V\)中,取一个基\(\AutoTuple{\alpha}{n}\),
经过施密特正交化把它变成正交基\(\AutoTuple{\beta}{n}\),
再经过单位化把它变成标准正交基\(\AutoTuple{\gamma}{n}\).
这就说明:
\begin{theorem}
%@see: 《Linear Algebra Done Right (Fourth Edition)》(Sheldon Axler) P202 6.35
欧几里得空间中,一定存在标准正交基.
%TODO proof
\end{theorem}

\begin{proposition}
%@see: 《Linear Algebra Done Right (Fourth Edition)》(Sheldon Axler) P203 6.36
欧几里得空间\(V\)中的任意一个正交单位向量组
均可以扩充成\(V\)的一个标准正交基.
%TODO proof
\end{proposition}

\subsection{欧几里得空间中向量的内积、傅里叶展开}
\begin{proposition}
%@see: 《高等代数(第三版 下册)》(丘维声) P177
设\((V,\rho)\)是\(n\)维欧几里得空间,
则向量组\(\AutoTuple{\eta}{n}\)是\((V,\rho)\)的一个标准正交基,
当且仅当\begin{equation*}
%@see: 《高等代数(第三版 下册)》(丘维声) P177 (11)
	\rho(\eta_i,\eta_j)
	= \delta(i,j),
	\quad i,j=1,2,\dotsc,n,
\end{equation*}
其中\(\delta\)是克罗内克\(\delta\)函数.
\end{proposition}

利用标准正交基,容易计算向量的内积.

设\((V,\rho)\)是\(n\)维欧几里得空间,
向量\(\alpha,\beta \in V\),
向量组\(\AutoTuple{\eta}{n}\)是\((V,\rho)\)的一个标准正交基,
\(\alpha,\beta\)在基\(\AutoTuple{\eta}{n}\)下的坐标
分别是\(X=(\AutoTuple{x}{n})^T,
Y=(\AutoTuple{y}{n})^T\),
则\begin{equation*}
%@see: 《高等代数(第三版 下册)》(丘维声) P177 (12)
	\rho(\alpha,\beta)
	= \rho\left( \sum_{i=1}^n x_i \eta_i, \sum_{j=1}^n y_j \eta_j \right)
	= \sum_{i=1}^n \sum_{j=1}^n x_i y_j \rho(\eta_i,\eta_j)
	= \sum_{i=1}^n x_i y_i
	= X^T Y.
\end{equation*}

利用标准正交基,可以借助内积,表达向量的坐标分量.

设\(\alpha\)在标准正交基\(\AutoTuple{\eta}{n}\)下的坐标为\(X=(\AutoTuple{x}{n})^T\),
则\begin{equation*}
	\alpha = \sum_{i=1}^n x_i \eta_i;
\end{equation*}
等号两边用\(\eta_j\)作内积,得\begin{equation*}
	\rho(\alpha,\eta_j)
	= \rho\left( \sum_{i=1}^n x_i \eta_i, \eta_j \right)
	= \sum_{i=1}^n x_i \rho(\eta_i,\eta_j)
	= x_j,
\end{equation*}
因此\begin{equation}\label{equation:欧几里得空间.向量的傅里叶展开}
%@see: 《高等代数(第三版 下册)》(丘维声) P177 (13)
	\alpha = \sum_{i=1}^n \rho(\alpha,\eta_i) \eta_i.
\end{equation}
我们把\cref{equation:欧几里得空间.向量的傅里叶展开}
称为“向量\(\alpha\)的\DefineConcept{傅里叶展开}”,
其中系数\(\rho(\alpha,\eta_i)\ (i=1,2,\dotsc,n)\)
称为“向量\(\alpha\)的\DefineConcept{傅里叶系数}”.

\begin{proposition}
%@see: 《高等代数(第三版 下册)》(丘维声) P177 命题7
在欧几里得空间\((V,\rho)\)中,
一个标准正交基到另一个标准正交基的过渡矩阵
一定是正交矩阵.
%TODO proof
\end{proposition}

\begin{proposition}
%@see: 《高等代数(第三版 下册)》(丘维声) P177 命题8
设\((V,\rho)\)是\(n\)维欧几里得空间,
向量组\(\AutoTuple{\eta}{n}\)是\((V,\rho)\)的一个标准正交基,
\(\AutoTuple{\beta}{n}\)是\(V\)中一个向量组.
如果存在正交矩阵\(P \in M_n(\mathbb{R})\),
使得\begin{equation*}
	(\AutoTuple{\beta}{n})
	= (\AutoTuple{\eta}{n}) P,
\end{equation*}
那么\(\AutoTuple{\beta}{n}\)是\(V\)中一个标准正交基.
%TODO proof
\end{proposition}

\begin{definition}
%@see: 《高等代数(第三版 下册)》(丘维声) P180 习题10.2 10.
设\(\AutoTuple{\alpha}{m}\)是\(n\)维欧几里得空间\((V,\rho)\)中一个向量组,
把\begin{equation}
	A \defeq \begin{bmatrix}
		\rho(\alpha_1,\alpha_1) & \rho(\alpha_1,\alpha_2) & \dots & \rho(\alpha_1,\alpha_m) \\
		\rho(\alpha_2,\alpha_1) & \rho(\alpha_2,\alpha_2) & \dots & \rho(\alpha_2,\alpha_m) \\
		\vdots & \vdots & & \vdots \\
		\rho(\alpha_m,\alpha_1) & \rho(\alpha_m,\alpha_2) & \dots & \rho(\alpha_m,\alpha_m) \\
	\end{bmatrix},
\end{equation}
称为“向量组\(\AutoTuple{\alpha}{m}\)的\DefineConcept{格拉姆矩阵}”,
记为\(G(\AutoTuple{\alpha}{m})\).
把\(A\)的行列式\(\abs{A}\)
称为“向量组\(\AutoTuple{\alpha}{m}\)的\DefineConcept{格拉姆行列式}”.
\end{definition}

\begin{example}
%@see: 《高等代数(第三版 下册)》(丘维声) P180 习题10.2 10.
设\(\AutoTuple{\alpha}{m}\)是\(n\)维欧几里得空间\((V,\rho)\)中一个向量组.
证明:\begin{equation*}
	\abs{G(\AutoTuple{\alpha}{m})} \geq 0.
\end{equation*}
当且仅当\(\AutoTuple{\alpha}{m}\)线性相关时,上式取“\(=\)”号.
\end{example}

\subsection{实内积空间之间的同构}
对于同一个实线性空间\(V\),
当指定不同的内积时,
\(V\)便成为不同的实内积空间,
这些实内积空间之间有什么关系?
不同的实线性空间,
各自指定了一个内积,成为实内积空间后,
它们之间又有什么关系?

\begin{definition}
%@see: 《高等代数(第三版 下册)》(丘维声) P178 定义7
设\((V_1,\rho_1),(V_2,\rho_2)\)都是实内积空间.
如果存在从\(V_1\)到\(V_2\)的一个双射\(\sigma\),
使得\begin{gather*}
	(\forall \alpha,\beta \in V_1)
	[
		\sigma(\alpha+\beta)
		= \sigma(\alpha) + \sigma(\beta)
	], \\
	(\forall \alpha \in V_1)
	(\forall k \in \mathbb{R})
	[
		\sigma(k\alpha)
		= k \sigma(\alpha)
	], \\
	(\forall \alpha,\beta \in V_1)
	[
		\rho_2(\sigma(\alpha),\sigma(\beta))
		= \rho_1(\alpha,\beta)
	],
\end{gather*}
则称“\(\sigma\)是从\(V_1\)到\(V_2\)的一个\DefineConcept{同构}”;
并称“\(V_1\)与\(V_2\) \DefineConcept{同构}”,
记为\(V_1 \Isomorphism V_2\).
\end{definition}
\begin{remark}
从上述定义可以看出,
实内积空间之间的一个同构\(\sigma\)
首先是实线性空间之间的一个同构,
其次\(\sigma\)还保持内积,
因此\(\sigma\)既具有线性空间的同构的性质,还具有与内积有关的性质.
\end{remark}

\begin{proposition}\label{theorem:欧几里得空间.实内积空间之间的同构的等价定义}
%@see: 《高等代数(第三版 下册)》(丘维声) P179
设\((V_1,\rho_1),(V_2,\rho_2)\)都是实内积空间.
如果存在从\(V_1\)到\(V_2\)的一个满射\(\sigma\),
使得\begin{gather*}
	(\forall \alpha,\beta \in V_1)
	[
		\sigma(\alpha+\beta)
		= \sigma(\alpha) + \sigma(\beta)
	], \\
	(\forall \alpha \in V_1)
	(\forall k \in \mathbb{R})
	[
		\sigma(k\alpha)
		= k \sigma(\alpha)
	], \\
	(\forall \alpha,\beta \in V_1)
	[
		\rho_2(\sigma(\alpha),\sigma(\beta))
		= \rho_1(\alpha,\beta)
	],
\end{gather*}
则\(\sigma\)是从\(V_1\)到\(V_2\)的一个同构.
\end{proposition}

\begin{property}
%@see: 《高等代数(第三版 下册)》(丘维声) P178
设\((V_1,\rho_1),(V_2,\rho_2)\)都是实内积空间,
\(\sigma\)是从\(V_1\)到\(V_2\)的一个同构,
\(\AutoTuple{\alpha}{n}\)是\(V_1\)的一个标准正交基,
则\(\sigma(\alpha_1),\dotsc,\sigma(\alpha_n)\)是\(V_2\)的一个标准正交基.
% \begin{proof}
% 因为\(\AutoTuple{\alpha}{n}\)是\(V_1\)的一个标准正交基,
% 所以\begin{equation*}
% 	\rho_1(\alpha_i,\alpha_j)
% 	= \delta(i,j),
% 	\quad i,j=1,2,\dotsc,n,
% \end{equation*}
% 其中\(\delta\)是克罗内克\(\delta\)函数;
% 从而有\begin{equation*}
% 	\rho_2(\sigma(\alpha_i),\sigma(\alpha_j))
% 	= \rho_1(\alpha_i,\alpha_j)
% 	= \delta(i,j),
% 	\quad i,j=1,2,\dotsc,n,
% \end{equation*}
% 说明\(\sigma(\alpha_1),\dotsc,\sigma(\alpha_n)\)是\(V_2\)的一个标准正交基.
% \end{proof}
\end{property}

\begin{theorem}\label{theorem:欧几里得空间.两个欧几里得空间同构的充分必要条件}
%@see: 《高等代数(第三版 下册)》(丘维声) P178 定理9
两个欧几里得空间同构的充分必要条件是它们的维数相同.
%TODO proof
\end{theorem}
\begin{remark}
从\cref{theorem:欧几里得空间.两个欧几里得空间同构的充分必要条件} 得出:
任意一个\(n\)维欧几里得空间\(V\)
都与装备了标准内积的欧几里得空间\(\mathbb{R}^n\)同构,
并且\begin{equation*}
	\sigma\colon V \to \mathbb{R}^n,
	\alpha = \sum_{i=1}^n x_i \eta_i \mapsto (\AutoTuple{x}{n})^T
\end{equation*}
就是从\(V\)到\(\mathbb{R}^n\)的一个同构,
其中\(\AutoTuple{\eta}{n}\)是\(V\)的一个标准正交基.
\end{remark}

\begin{property}
%@see: 《高等代数(第三版 下册)》(丘维声) P179
%@see: 《高等代数(第三版 下册)》(丘维声) P181 习题10.2 13.
实内积空间之间的同构关系,具有反身性、对称性、传递性,是一个等价关系.
\end{property}
