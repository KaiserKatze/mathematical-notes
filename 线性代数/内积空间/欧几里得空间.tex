\section{欧几里得空间}
\subsection{内积}
\begin{definition}
%@see: 《高等代数(第三版 下册)》(丘维声) P174 定义1
设\(V\)是实数域\(\mathbb{R}\)上的一个线性空间.
如果\(f\)是\(V\)上的一个正定对称双线性函数,
则称\(f\)是“\(V\)上的一个\DefineConcept{内积}”.
\end{definition}

\begin{proposition}
%@see: 《高等代数(第三版 下册)》(丘维声) P174 命题1
设\(V\)是\(\mathbb{R}\)上的一个\(n\)维线性空间,
\(f\)是\(V\)上的一个双线性函数,
则\(f\)是正定对称的,
当且仅当\(f\)在\(V\)的某个基下的度量矩阵是正定对称的.
%TODO proof
\end{proposition}

\begin{example}
%@see: 《高等代数(第三版 下册)》(丘维声) P174 例1
在\(V = \mathbb{R}^3\)中,
令\(f(\alpha,\beta) \defeq a_1 b_1 + 2 a_2 b_2 + 3 a_3 b_3\),
其中\(\alpha=(\AutoTuple{a}{3})^T,
\beta=(\AutoTuple{b}{3})^T\).
容易验证,\(f\)是\(V\)上的一个内积.
\end{example}

\begin{example}
%@see: 《高等代数(第三版 下册)》(丘维声) P174 例2
在\(V = M_n(\mathbb{R})\)中,
令\(f(A,B) \defeq \tr(AB^T)\).
容易验证,\(f\)是\(V\)上的一个内积.
\end{example}

\begin{example}
%@see: 《高等代数(第三版 下册)》(丘维声) P174 例3
在\(V = C[a,b]\)中,
令\(f(f,g) \defeq \int_a^b f(x) g(x) \dd{x}\).
容易验证,\(f\)是\(V\)上的一个内积.
\end{example}

\subsection{实内积空间,欧几里得空间}
\begin{definition}
%@see: 《高等代数(第三版 下册)》(丘维声) P174 定义2
设\(V\)是一个实线性空间,
\(\rho\)是\(V\)上的一个内积,
则称“\((V,\rho)\)是一个\DefineConcept{实内积空间}”.
\end{definition}

\begin{definition}
%@see: 《高等代数(第三版 下册)》(丘维声) P174 定义2
设\(V\)是一个实内积空间.
如果\(V\)是有限维的,
则称“\(V\)是一个\DefineConcept{欧几里得空间}”;
把线性空间\(V\)的维数称为“欧几里得空间\(V\)的\DefineConcept{维数}”.
\end{definition}

\begin{definition}
%@see: 《高等代数(第三版 下册)》(丘维声) P174 定义3
设\((V,\rho)\)是一个实内积空间,
\(\alpha \in V\).
把非负实数\(\sqrt{\rho(\alpha,\alpha)}\)
称为“向量\(\alpha\)的\DefineConcept{长度}”,
记作\(\VectorLengthA{\alpha}\)或\(\VectorLengthN{\alpha}\).
\end{definition}

\begin{property}
%@see: 《高等代数(第三版 下册)》(丘维声) P175
在实内积空间\(V\)中,
零向量的长度为\(0\),
非零向量的长度是正数.
\end{property}

\begin{property}
%@see: 《高等代数(第三版 下册)》(丘维声) P175
在实内积空间\((V,\rho)\)中,
对于\(\forall \alpha \in V\)
和\(\forall k \in \mathbb{R}\),
有\(\VectorLengthA{k\alpha} = \abs{k} \VectorLengthA{\alpha}\).
\begin{proof}
\(\VectorLengthA{k\alpha}
= \sqrt{\rho(k\alpha,k\alpha)}
= \sqrt{k^2\rho(\alpha,\alpha)}
= \abs{k} \VectorLengthA{\alpha}\).
\end{proof}
\end{property}

\begin{definition}
%@see: 《高等代数(第三版 下册)》(丘维声) P175
设\((V,\rho)\)是实内积空间,
\(\alpha \in V\).
如果\(\VectorLengthA{\alpha} = 1\),
则称“\(\alpha\)是一个\DefineConcept{单位向量}”.
\end{definition}

\begin{property}
%@see: 《高等代数(第三版 下册)》(丘维声) P175
设\((V,\rho)\)是实内积空间,
\(\alpha \in V\).
如果\(\alpha\neq0\),
则\(\frac1{\VectorLengthA{\alpha}} \alpha\)是一个单位向量.
\end{property}

\begin{theorem}
%@see: 《高等代数(第三版 下册)》(丘维声) P175 定理2(柯西-布尼亚科夫斯基不等式)
在实内积空间\((V,\rho)\)中,
对于\(\forall \alpha,\beta \in V\),
有\begin{equation}
	\abs{\rho(\alpha,\beta)} \leq \VectorLengthA{\alpha} \VectorLengthA{\beta}.
\end{equation}
当且仅当\(\{\alpha,\beta\}\)线性相关时,上式取“\(=\)”号.
%TODO proof
\end{theorem}

\begin{definition}
%@see: 《高等代数(第三版 下册)》(丘维声) P175 定义4
在实内积空间\((V,\rho)\)中,
\(\alpha,\beta\)是两个非零向量.
把\begin{equation}
%@see: 《高等代数(第三版 下册)》(丘维声) P175 (6)
	\arccos\frac{\rho(\alpha,\beta)}{\VectorLengthA{\alpha} \VectorLengthA{\beta}}
\end{equation}
称为“\(\alpha\)与\(\beta\)的\DefineConcept{夹角}”,
记为\(\VectorAngleA{\alpha}{\beta}\)或\(\VectorAngleP{\alpha}{\beta}\).
\end{definition}

\begin{property}
%@see: 《高等代数(第三版 下册)》(丘维声) P175
设\(\alpha,\beta\)是实内积空间\((V,\rho)\)中的两个非零向量,
则\(\alpha\)与\(\beta\)的夹角\(\theta = \VectorAngleA{\alpha}{\beta}\)
满足\(0 \leq \theta \leq \pi\).
%TODO proof
\end{property}

\begin{property}
%@see: 《高等代数(第三版 下册)》(丘维声) P175
设\(\alpha,\beta\)是实内积空间\((V,\rho)\)中的两个非零向量,
则\(\alpha\)与\(\beta\)的夹角\(\theta = \VectorAngleA{\alpha}{\beta}\)
满足\(\theta = \frac\pi2 \iff \rho(\alpha,\beta) = 0\).
%TODO proof
\end{property}

\begin{definition}
%@see: 《高等代数(第三版 下册)》(丘维声) P175 定义5
设\(\alpha,\beta\)是实内积空间\((V,\rho)\)中的两个非零向量.
如果\(\rho(\alpha,\beta) = 0\),
则称“\(\alpha\)与\(\beta\) \DefineConcept{正交}”,
记为\(\alpha \perp \beta\).
\end{definition}

\begin{corollary}
%@see: 《高等代数(第三版 下册)》(丘维声) P175 推论3
在实内积空间\((V,\rho)\)中,
三角不等式成立,
即对于\(\forall \alpha,\beta \in V\),
有\begin{equation}
%@see: 《高等代数(第三版 下册)》(丘维声) P175 (7)
	\VectorLengthA{\alpha+\beta} \leq \VectorLengthA{\alpha} + \VectorLengthA{\beta}.
\end{equation}
%TODO 没有取等条件
%TODO proof
\end{corollary}

\begin{corollary}
%@see: 《高等代数(第三版 下册)》(丘维声) P176 推论4
在实内积空间\((V,\rho)\)中,
勾股定理成立,
即对于\(\forall \alpha,\beta \in V\),
如果\(\alpha\)与\(\beta\)正交,
则\begin{equation}
%@see: 《高等代数(第三版 下册)》(丘维声) P176 (8)
	\VectorLengthA{\alpha+\beta}^2 = \VectorLengthA{\alpha}^2 + \VectorLengthA{\beta}^2.
\end{equation}
%TODO proof
\end{corollary}

\begin{definition}
%@see: 《高等代数(第三版 下册)》(丘维声) P176 定义6
在实内积空间\((V,\rho)\)中,
\(\alpha,\beta \in V\).
把\(\VectorLengthA{\alpha-\beta}\)
称为“\(\alpha\)与\(\beta\)的\DefineConcept{距离}”,
记为\(d(\alpha,\beta)\).
\end{definition}

\begin{property}
%@see: 《高等代数(第三版 下册)》(丘维声) P176
在实内积空间\((V,\rho)\)中,
距离\(d\colon V \times V \to \mathbb{R},
(\alpha,\beta) \mapsto \VectorLengthA{\alpha-\beta}\)
满足以下性质:\begin{itemize}
	\item {\rm\bf 对称性}:\begin{equation*}
		(\forall \alpha,\beta \in V)
		[
			d(\alpha,\beta) = d(\beta,\alpha)
		];
	\end{equation*}

	\item {\rm\bf 正定性}:\begin{equation*}
		(\forall \alpha,\beta \in V)
		[
			d(\alpha,\beta) \geq 0
		];
	\end{equation*}
	当且仅当\(\alpha = \beta\)时,上式取“\(=\)”号;

	\item {\rm\bf 三角不等式}:\begin{equation*}
		(\forall \alpha,\beta,\gamma \in V)
		[
			d(\alpha,\gamma) \leq d(\alpha,\beta) + d(\beta,\gamma)
		].
	\end{equation*}
\end{itemize}
\end{property}
