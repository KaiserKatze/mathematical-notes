\section{对称变换}
\begin{definition}
%@see: 《高等代数(第三版 下册)》(丘维声) P186 定义2
设\(\vb{A}\)是实内积空间\((V,\rho)\)上的一个线性变换.
如果\begin{equation*}
	(\forall \alpha,\beta \in V)
	[
		\rho(\vb{A}\alpha,\beta)
		= \rho(\alpha,\vb{A}\beta)
	],
\end{equation*}
则称“\(\vb{A}\)是\DefineConcept{对称变换}”.
\end{definition}

\begin{proposition}
%@see: 《高等代数(第三版 下册)》(丘维声) P186 命题5
实内积空间上的对称变换一定是线性变换.
%TODO proof
\end{proposition}

\begin{proposition}
%@see: 《高等代数(第三版 下册)》(丘维声) P187 命题6
设\(\vb{A}\)是\(n\)维欧几里得空间\(V\)上的一个线性变换,
则\(\vb{A}\)是对称变换,
当且仅当\(\vb{A}\)在\(V\)的某个标准正交基下的矩阵是对称矩阵.
%TODO proof
\end{proposition}

\begin{proposition}
%@see: 《高等代数(第三版 下册)》(丘维声) P187 命题7
设\(\vb{A}\)是实内积空间\(V\)上的一个对称变换.
如果\(W\)是\(\vb{A}\)的不变子空间,
则\(W\)的正交补\(W^\perp\)也是\(\vb{A}\)的不变子空间.
%TODO proof
\end{proposition}

我们知道,实对称矩阵一定正交相似于某一个对角矩阵,由此得出:
\begin{theorem}
%@see: 《高等代数(第三版 下册)》(丘维声) P187 定理8
设\(\vb{A}\)是\(n\)维欧几里得空间\(V\)上的一个线性变换,
则\(\vb{A}\)是对称变换,
当且仅当\(V\)中存在一个标准正交基\(S\),
使得\(\vb{A}\)在基\(S\)下的矩阵是对角矩阵.
%TODO proof
\end{theorem}
