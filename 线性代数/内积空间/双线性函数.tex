\section{双线性函数}
我们已经知道,\(\mathbb{R}^n\)上一个内积
\(\VectorInnerProductDot{\alpha}{\beta}\)是\(\mathbb{R}^n\)上的一个二元实值函数,
并且它具有对称性、线性性、正定性,
不难得到\begin{gather*}
	\VectorInnerProductDot{(k_1\alpha_1+k_2\alpha_2)}{\beta}
	= k_1(\VectorInnerProductDot{\alpha_1}{\beta})
	+ k_2(\VectorInnerProductDot{\alpha_2}{\beta}), \\
	\VectorInnerProductDot{\alpha}{(k_1\beta_1+k_2\beta_2)}
	= k_1(\VectorInnerProductDot{\alpha}{\beta_1})
	+ k_2(\VectorInnerProductDot{\alpha}{\beta_2}).
\end{gather*}
受此启发,我们抽象出域\(F\)上线性空间\(V\)上的双线性函数的概念:
\begin{definition}
%@see: 《高等代数(第三版 下册)》(丘维声) P166 定义1
设\(V\)是域\(F\)上的一个线性空间.
如果映射\(f\colon V \times V \to F\)满足\begin{gather}
	\VectorInnerProductDot{(k_1\alpha_1+k_2\alpha_2)}{\beta}
	= k_1(\VectorInnerProductDot{\alpha_1}{\beta})
	+ k_2(\VectorInnerProductDot{\alpha_2}{\beta}),
		\label{equation:双线性函数.双线性函数判定条件1} \\
	\VectorInnerProductDot{\alpha}{(k_1\beta_1+k_2\beta_2)}
	= k_1(\VectorInnerProductDot{\alpha}{\beta_1})
	+ k_2(\VectorInnerProductDot{\alpha}{\beta_2}),
		\label{equation:双线性函数.双线性函数判定条件2}
\end{gather}
则称“\(f\)是\(V\)上的一个\DefineConcept{双线性函数}”,
记\begin{equation*}
	f_1(\beta) \defeq f(\alpha,\beta),
	\qquad
	f_2(\alpha) \defeq f(\alpha,\beta).
\end{equation*}
\end{definition}
\begin{remark}
条件 \labelcref{equation:双线性函数.双线性函数判定条件1} 表明:
当\(\beta\)固定时,映射\(f_2\)是\(V\)上的一个线性函数.
条件 \labelcref{equation:双线性函数.双线性函数判定条件2} 表明:
当\(\alpha\)固定时,映射\(f_1\)是\(V\)上的一个线性函数.
\end{remark}

\begin{example}
%@see: 《高等代数(第三版 下册)》(丘维声) P166 例1
设\(V = M_n(F)\).
令\(f(A,B) \defeq \tr(AB)\ (A,B \in V)\),
则\(f\)是\(V\)上的一个双线性函数.
\end{example}

\begin{example}
%@see: 《高等代数(第三版 下册)》(丘维声) P167 例2
设\(V = C[a,b]\).
令\(f(g,h) \defeq \int_a^b g(x) h(x) \dd{x}\ (g,h \in V)\),
则\(f\)是\(V\)上的一个双线性函数.
\end{example}

\begin{example}
%@see: 《高等代数(第三版 下册)》(丘维声) P167 例3
设\(V = F^n\).
令\(f(\alpha,\beta) = \VectorInnerProductDot{\alpha}{\beta}\ (\alpha,\beta \in V)\),
则\(f\)是\(V\)上的一个双线性函数.
\end{example}

\begin{definition}
%@see: 《高等代数(第三版 下册)》(丘维声) P167
设\(V\)是域\(F\)上\(n\)维线性空间,
\(f\)是\(V\)上的一个双线性函数,
在\(V\)中取一个基\(\AutoTuple{\epsilon}{n}\),
向量\(\alpha,\beta\)在基\(\AutoTuple{\epsilon}{n}\)下的坐标
分别是\(X=(\AutoTuple{x}{n})^T,
Y=(\AutoTuple{y}{n})^T\),
把\begin{equation*}
	f(\alpha,\beta)
	= f\left( \sum_{i=1}^n x_i \epsilon_i, \sum_{j=1}^n y_j \epsilon_j \right)
	= \sum_{i=1}^n \sum_{j=1}^n x_i y_j f(\epsilon_i,\epsilon_j)
\end{equation*}
中\(x_i y_j\)的系数\(f(\epsilon_i,\epsilon_j)\)作为矩阵的\((i,j)\)元素,
得到\begin{equation*}
	A \defeq \begin{bmatrix}
		f(\epsilon_1,\epsilon_1) & f(\epsilon_1,\epsilon_2) & \dots & f(\epsilon_1,\epsilon_n) \\
		f(\epsilon_2,\epsilon_1) & f(\epsilon_2,\epsilon_2) & \dots & f(\epsilon_2,\epsilon_n) \\
		\vdots & \vdots & & \vdots \\
		f(\epsilon_n,\epsilon_1) & f(\epsilon_n,\epsilon_2) & \dots & f(\epsilon_n,\epsilon_n) \\
	\end{bmatrix},
\end{equation*}
将\(A\)称为“双线性函数\(f\)在基\(\AutoTuple{\epsilon}{n}\)下的\DefineConcept{度量矩阵}”.
\end{definition}
\begin{remark}
%@see: 《高等代数(第三版 下册)》(丘维声) P167
双线性函数\(f\)的度量矩阵
由\(f\)以及基\(\AutoTuple{\epsilon}{n}\)唯一确定.
\end{remark}
