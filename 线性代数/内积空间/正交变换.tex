\section{正交变换}
\subsection{正交变换的概念}
\begin{definition}
%@see: 《高等代数(第三版 下册)》(丘维声) P185 定义1
设\((V,\rho)\)是一个实内积空间,
\(\vb{A}\)是一个从\(V\)到\(V\)的满射.
如果\(\vb{A}\)满足\begin{equation}
	(\forall \alpha,\beta \in V)
	[
		\rho(\vb{A}\alpha,\vb{A}\beta)
		= \rho(\alpha,\beta)
	],
\end{equation}
则称“\(\vb{A}\)保持向量的内积不变”
“\(\vb{A}\)是\(V\)上的一个\DefineConcept{正交变换}”.
\end{definition}

\subsection{正交变换的性质}
\begin{property}\label{theorem:正交变换.保长性1}
%@see: 《高等代数(第三版 下册)》(丘维声) P185
设\((V,\rho)\)是一个实内积空间,
\(\vb{A}\)是\(V\)上的一个正交变换,
则\(\VectorLengthA{\vb{A}\alpha} = \VectorLengthA{\alpha}\).
\end{property}

\begin{proposition}
%@see: 《高等代数(第三版 下册)》(丘维声) P185 命题1
实内积空间上的正交变换一定是线性变换.
%TODO proof
\end{proposition}

\begin{proposition}
%@see: 《高等代数(第三版 下册)》(丘维声) P185 命题2
实内积空间上的正交变换一定是可逆的.
%TODO proof
\end{proposition}

\begin{proposition}
%@see: 《高等代数(第三版 下册)》(丘维声) P185 命题3
设\(\vb{A}\)是实内积空间\(V\)上的一个线性变换,
则\(\vb{A}\)是\(V\)上的一个正交变换,
当且仅当\(\vb{A}\)是从\(V\)上的一个自同构.
%TODO proof
\end{proposition}

\begin{proposition}
%@see: 《高等代数(第三版 下册)》(丘维声) P186
设\(\vb{A}\)是实内积空间\(V\)上的一个正交变换,
则\(\vb{A}\)的逆\(\vb{A}^{-1}\)也是\(V\)上的一个正交变换.
%TODO proof
\end{proposition}

\begin{proposition}
%@see: 《高等代数(第三版 下册)》(丘维声) P186
设\(\vb{A},\vb{B}\)都是实内积空间\(V\)上的正交变换,
则\(\vb{A}\vb{B}\)也是\(V\)上的一个正交变换.
\end{proposition}

\begin{property}\label{theorem:正交变换.保角性1}
%@see: 《高等代数(第三版 下册)》(丘维声) P186
设\((V,\rho)\)是一个实内积空间,
\(\vb{A}\)是\(V\)上的一个正交变换,
则\(\VectorAngleA{\alpha}{\beta} = \VectorAngleA{\vb{A}\alpha}{\vb{A}\beta}\).
\end{property}

\begin{property}\label{theorem:正交变换.保角性2}
%@see: 《高等代数(第三版 下册)》(丘维声) P186
设\((V,\rho)\)是一个实内积空间,
\(\vb{A}\)是\(V\)上的一个正交变换,
则\(\alpha\perp\beta \implies (\vb{A}\alpha)\perp(\vb{A}\beta)\).
\end{property}

\begin{property}\label{theorem:正交变换.保长性2}
%@see: 《高等代数(第三版 下册)》(丘维声) P186
设\((V,\rho)\)是一个实内积空间,
\(\vb{A}\)是\(V\)上的一个正交变换,
则\(d(\alpha,\beta) = d(\vb{A}\alpha,\vb{A}\beta)\).
\end{property}

\subsection{正交变换的判定}
\begin{proposition}
%@see: 《高等代数(第三版 下册)》(丘维声) P186 命题4
设\(\vb{A}\)是\(n\)维欧几里得空间\(V\)上的一个线性变换,
则\begin{align*}
	&\text{$\vb{A}$是正交变换} \\
	&\iff \text{$\vb{A}$把$V$的标准正交基映成标准正交基} \\
	&\iff \text{$\vb{A}$在$V$的标准正交基下的矩阵是正交矩阵}.
\end{align*}
%TODO proof
\end{proposition}

\subsection{正交变换的分类}
\begin{definition}
%@see: 《高等代数(第三版 下册)》(丘维声) P186
设\(\vb{A}\)是\(n\)维欧几里得空间\(V\)上的一个正交变换,
\(A\)是\(\vb{A}\)在\(V\)的某个标准正交基下的矩阵.
\begin{itemize}
	\item 如果\(\DeterminantA{A} = 1\),
	则称\(\vb{A}\)是\DefineConcept{第一类变换}或\DefineConcept{旋转变换}.

	\item 如果\(\DeterminantA{A} = -1\),
	则称\(\vb{A}\)是\DefineConcept{第一类变换}.
\end{itemize}
\end{definition}
