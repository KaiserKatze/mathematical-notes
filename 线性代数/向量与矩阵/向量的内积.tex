\section{向量的内积}
\begin{definition}
设\(\vb\alpha=(\AutoTuple{a}{n})^T\)和\(\vb\beta=(\AutoTuple{b}{n})^T\)都是\(n\)维复向量.
我们把复数\begin{equation*}
	a_1b_1 + a_2b_2 + \dotsb + a_nb_n
\end{equation*}
称为“\(\vb\alpha\)与\(\vb\beta\)的\DefineConcept{内积}(inner product)”,
记作\(\VectorInnerProductDot{\vb\alpha}{\vb\beta}\).
\end{definition}

\begin{definition}
若向量\(\vb\alpha\)与\(\vb\beta\)满足\(\VectorInnerProductDot{\vb\alpha}{\vb\beta}=0\),
则称\(\vb\alpha\)与\(\vb\beta\)正交(orthogonal),
记作\(\vb\alpha\perp\vb\beta\).
\end{definition}

\begin{property}
向量内积具有以下性质:
\begin{enumerate}
	\item \(\VectorInnerProductDot{\vb\alpha}{\vb\beta} = \VectorInnerProductDot{\vb\beta}{\vb\alpha}\);
	\item \(\VectorInnerProductDot{(\vb\alpha+\vb\beta)}{\vb\gamma} = \VectorInnerProductDot{\vb\alpha}{\vb\gamma} + \VectorInnerProductDot{\vb\beta}{\vb\gamma}\);
	\item \(\VectorInnerProductDot{(k\vb\alpha)}{\vb\beta} = k (\VectorInnerProductDot{\vb\alpha}{\vb\beta})\ (k\in\mathbb{R})\);
	\item \(\vb\alpha\neq\vb0 \iff \VectorInnerProductDot{\vb\alpha}{\vb\alpha} > 0\);\(\vb\alpha=\vb0 \iff \VectorInnerProductDot{\vb\alpha}{\vb\alpha} = 0\);
	\item \(\VectorInnerProductDot{\vb0}{\vb\alpha} = 0\);
\end{enumerate}
\end{property}

\subsection{向量的长度(模、范数)与单位向量}
\begin{definition}
设\(n\)维向量\(\vb\alpha = (\AutoTuple{a}{n})\).
定义向量的\DefineConcept{长度}为\begin{equation*}
	\sqrt{\VectorInnerProductDot{\vb\alpha}{\vb\alpha}} = \sqrt{a_1^2+a_2^2+\dotsb+a_n^2}.
\end{equation*}同样地可以定义\(n\)维列向量的长度.
2维向量、3维向量的长度常被称作向量的\DefineConcept{模}(module),记作\(\VectorLengthA{\vb\alpha}\).
高维(\(n > 3\))向量的长度常被称作向量的\DefineConcept{范数}(norm),记作\(\VectorLengthN{\vb\alpha}\).
\end{definition}

\begin{property}
显然有向量的长度为非负实数,即\(\VectorLengthA{\vb\alpha}\geq0\).
\end{property}

\begin{definition}
长度为1的向量被称为\DefineConcept{单位向量}.
\end{definition}

\begin{definition}
\def\f{\frac{1}{\VectorLengthA{\vb\alpha}}}
设\(\vb\alpha\)满足\(\VectorLengthA{\vb\alpha}>0\).
用\(\f\)数乘\(\vb\alpha\),
称为“将\(\vb\alpha\) \DefineConcept{单位化}”,
得单位向量\(\f\vb\alpha\).
\end{definition}

尽管我们通常出于几何(特别是欧氏几何)的考量,
像上面一样将向量\(\vb\alpha\)的模(或范数)定义为\(\sqrt{\VectorInnerProductDot{\vb\alpha}{\vb\alpha}}\),
不过我们还可以定义其他形式的模(或范数).
观察上面的模(或范数)的定义,我们可以发现,
向量的模(或范数)实际上是满足以下3条性质的映射
\begingroup%
\(f\colon K^n \to K, \vb{x} \mapsto m\):
\begin{enumerate}
	\item {\rm\bf 非负性},
	即\((\forall \vb{x} \in K^n)[f(\vb{x}) \geq 0]\);
	\item {\rm\bf 齐次性},
	即\((\forall \vb{x} \in K^n)(\forall c \in K)[f(c \vb{x}) = \VectorLengthA{c} f(\vb{x})]\);
	\item {\rm\bf 三角不等式},
	即\((\forall \vb{x},\vb{y} \in K^n)[f(\vb{x}+\vb{y}) \leq f(\vb{x}) + f(\vb{y})]\).
\end{enumerate}

\begin{definition}\label{definition:向量与矩阵.p范数}
形如\begin{equation*}
	f\colon\mathbb{R}^n \to \mathbb{R},
	\vb{x} = \opair{\AutoTuple{x}{n}}
	\mapsto
	\sqrt[p]{\VectorLengthA{x_1}^p + \VectorLengthA{x_2}^p + \dotsb + \VectorLengthA{x_n}^p}
\end{equation*}的这一类映射,
称为 \DefineConcept{\(p\)范数},
记作\(\norm{\vb{x}}_p\).
\end{definition}

易见
\begin{gather}
	\norm{\vb{x}}_1 = \VectorLengthA{x_1} + \VectorLengthA{x_2} + \dotsb + \VectorLengthA{x_n}, \\
	\norm{\vb{x}}_2 = \sqrt{x_1^2 + x_2^2 + \dotsb + x_n^2}, \\
	\norm{\vb{x}}_\infty = \max\{\VectorLengthA{x_1},\VectorLengthA{x_2},\dotsc,\VectorLengthA{x_n}\}.
\end{gather}
\endgroup%
