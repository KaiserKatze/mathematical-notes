\section{分块矩阵的运算}
分块阵的运算服从以下规律:
\begin{enumerate}
	\item {\rm\bf 分块阵的加法}

	设\(\vb{A},\vb{B} \in M_{s \times n}(K)\),
	若将\(\vb{A}\)和\(\vb{B}\)按同样的规则分块为\begin{equation*}
		\vb{A}=(\vb{A}_{ij})_{t \times r}, \qquad
		\vb{B}=(\vb{B}_{ij})_{t \times r},
	\end{equation*}
	其中\(\vb{A}_{ij},\vb{B}_{ij}\in M_{s_i \times n_j}(K)\ (i=1,2,\dotsc,t;j=1,2,\dotsc,r)\),
	则\begin{equation*}
		\vb{A}+\vb{B}=(\vb{A}_{ij}+\vb{B}_{ij})_{t \times r}.
	\end{equation*}

	\item {\rm\bf 分块阵的数乘}

	设\(\vb{A}\in M_{s \times n}(K)\),
	若将\(\vb{A}\)分块为\begin{equation*}
		\vb{A}=(\vb{A}_{ij})_{t \times r},
	\end{equation*}
	其中\(\vb{A}_{ij}\in M_{s_i \times n_j}(K)\ (i=1,2,\dotsc,t;j=1,2,\dotsc,r)\),
	则\begin{equation*}
		k\vb{A}=(k\vb{A}_{ij})_{t \times r}.
	\end{equation*}

	\item {\rm\bf 分块阵的转置}

	设\(\vb{A}\in M_{s \times n}(K)\),
	若将\(\vb{A}\)分块为\begin{equation*}
		\vb{A}=(\vb{A}_{ij})_{t \times r},
	\end{equation*}
	其中\(\vb{A}_{ij}\in M_{s_i \times n_j}(K)\ (i=1,2,\dotsc,t;j=1,2,\dotsc,r)\),
	则\begin{equation*}
		\vb{A}^T=(\vb{A}_{ji}^T)_{r \times t}.
	\end{equation*}
	这就是说,在转置分块阵时,要将每个子块转置.

	\item {\rm\bf 分块阵的乘法}

	设\(\vb{A}\in M_{s \times n}(K),
	\vb{B}\in M_{n \times m}(K)\),
	若将\(\vb{A}\)、\(\vb{B}\)分别分块为\begin{equation*}
		\vb{A}=(\vb{A}_{ij})_{t \times r}, \qquad
		\vb{B}=(\vb{B}_{jk})_{r \times p},
	\end{equation*}
	且\(\vb{A}\)的列的分块法与\(\vb{B}\)的行的分块法一致,即\begin{equation*}
		\vb{A} = \begin{matrix}
			& \begin{matrix} n_1 & n_2 & \dots & n_r \end{matrix} \\
			\begin{matrix} s_1 \\ s_2 \\ \vdots \\ s_t \end{matrix} & \begin{bmatrix}
			\vb{A}_{11} & \vb{A}_{12} & \dots & \vb{A}_{1r} \\
			\vb{A}_{21} & \vb{A}_{22} & \dots & \vb{A}_{2r} \\
			\vdots & \vdots & & \vdots \\
			\vb{A}_{t1} & \vb{A}_{t2} & \dots & \vb{A}_{tr}
			\end{bmatrix}
		\end{matrix},
		\qquad
		\vb{B} = \begin{matrix}
			& \begin{matrix} m_1 & m_2 & \dots & m_p \end{matrix} \\
			\begin{matrix} n_1 \\ n_2 \\ \vdots \\ n_r \end{matrix} & \begin{bmatrix}
			\vb{B}_{11} & \vb{B}_{12} & \dots & \vb{B}_{1p} \\
			\vb{B}_{21} & \vb{B}_{22} & \dots & \vb{B}_{2p} \\
			\vdots & \vdots & & \vdots \\
			\vb{B}_{r1} & \vb{B}_{r2} & \dots & \vb{B}_{rp}
			\end{bmatrix},
		\end{matrix}
	\end{equation*}
	则\begin{equation*}
		\vb{A}\vb{B} = \begin{matrix}
			& \begin{matrix} m_1 & m_2 & \dots & m_p \end{matrix} \\
			\begin{matrix} s_1 \\ s_2 \\ \vdots \\ s_t \end{matrix} & \begin{bmatrix}
			\vb{C}_{11} & \vb{C}_{12} & \dots & \vb{C}_{1p} \\
			\vb{C}_{21} & \vb{C}_{22} & \dots & \vb{C}_{2p} \\
			\vdots & \vdots & & \vdots \\
			\vb{C}_{t1} & \vb{C}_{t2} & \dots & \vb{C}_{tp}
			\end{bmatrix}
		\end{matrix}.
	\end{equation*}
	其中\(\vb{C}_{ij}=\sum_{k=1}^r \vb{A}_{ik} \vb{B}_{kj}\ (i=1,2,\dotsc,t;j=1,2,\dotsc,p)\).
\end{enumerate}
\begin{remark}
下面列举几个十分常用的矩阵乘法运算:\begin{gather*}
	\vb{A} (\AutoTuple{\vb\alpha}{m})
	= (\AutoTuple{\vb{A} \vb\alpha}{m})
	\quad(\vb{A} \in M_{s \times n}(K),\vb\alpha_i \in K^n,i=1,2,\dotsc,m), \\
	(\AutoTuple{\vb\alpha}{m})
	\begin{bmatrix}
		\vb{B}_1 \\
		\vdots \\
		\vb{B}_m
	\end{bmatrix}
	= \sum_{i=1}^m \vb\alpha_i \vb{B}_i
	\quad(\vb{B}_i \in K^t,\vb\alpha_i \in K^n,i=1,2,\dotsc,m), \\
	\begin{bmatrix}
		\vb{A}_1 \\ \vb{A}_2 \\ \vdots \\ \vb{A}_s
	\end{bmatrix}
	\vb{B}
	= \begin{bmatrix}
		\vb{A}_1 \vb{B} \\
		\vb{A}_2 \vb{B} \\
		\vdots \\
		\vb{A}_s \vb{B}
	\end{bmatrix}
	\quad(\text{$\AutoTuple{\vb{A}}{s}$的列数与$\vb{B}$的行数相同}), \\
	\begin{bmatrix}
		\vb{A}_1 & \vb0 \\
		\vb0 & \vb{A}_2
	\end{bmatrix}
	\begin{bmatrix}
		\vb{B}_1 & \vb0 \\
		\vb0 & \vb{B}_2
	\end{bmatrix}
	= \begin{bmatrix}
		\vb{A}_1 \vb{B}_1 & \vb0 \\
		\vb0 & \vb{A}_2 \vb{B}_2
	\end{bmatrix}.
\end{gather*}
\end{remark}

\begin{example}
设\(n\)阶矩阵\(
	\vb{A}_k
	\defeq \begin{bmatrix}
		\vb0 & \vb{E}_{n-k} \\
		0 & \vb0
	\end{bmatrix}
	\ (n>k\geq1)
\),
其中\(\vb{E}_k\)是\(k\)阶单位矩阵.
% 这是一个幂零矩阵
证明:\begin{equation*}
	\vb{A}_1^k
	= \begin{cases}[cl]
		\vb{A}_k,	& 1 \leq k \leq n-1, \\
		\vb0,			& k = n.
	\end{cases}
\end{equation*}
%TODO proof
\end{example}
