\section{特殊矩阵}
\subsection{基本矩阵}
\begin{definition}
%@see: 《高等代数(第三版 上册)》(丘维声) P117 定义3
只有一个元素是\(1\),其余元素全为\(0\)的矩阵,称为\DefineConcept{基本矩阵}.
\((i,j)\)元素为\(1\)的基本矩阵,记作\(\vb{E}_{ij}\).
\end{definition}
\begin{example}
%@see: 《高等代数(第三版 上册)》(丘维声) P117
对于\(2\times3\)矩阵\(\vb{A}\),有\begin{align*}
	\vb{A}
	&= \begin{bmatrix}
		a_{11} & a_{12} & a_{13} \\
		a_{21} & a_{22} & a_{23}
	\end{bmatrix} \\
	&= a_{11}
	\begin{bmatrix}
		1 & 0 & 0 \\
		0 & 0 & 0
	\end{bmatrix}
	+ a_{12}
	\begin{bmatrix}
		0 & 1 & 0 \\
		0 & 0 & 0
	\end{bmatrix}
	+ a_{13}
	\begin{bmatrix}
		0 & 0 & 1 \\
		0 & 0 & 0
	\end{bmatrix} \\
	&\hspace{20pt}
	+ a_{21}
	\begin{bmatrix}
		0 & 0 & 0 \\
		1 & 0 & 0
	\end{bmatrix}
	+ a_{22}
	\begin{bmatrix}
		0 & 0 & 0 \\
		0 & 1 & 0
	\end{bmatrix}
	+ a_{23}
	\begin{bmatrix}
		0 & 0 & 0 \\
		0 & 0 & 1
	\end{bmatrix} \\
	&=
	a_{11} \vb{E}_{11} + a_{12} \vb{E}_{12} + a_{13} \vb{E}_{13}
	+ a_{21} \vb{E}_{21} + a_{22} \vb{E}_{22} + a_{23} \vb{E}_{23}.
\end{align*}
\end{example}

一般地,
对于任意一个\(s \times n\)矩阵\(\vb{A}\),
有\begin{equation*}
	\vb{A} = \sum_{i=1}^s \sum_{j=1}^n a_{ij} \vb{E}_{ij},
\end{equation*}
其中\(a_{ij}\)是\(\vb{A}\)的\((i,j)\)元素.

\subsection{三角矩阵}
\begin{definition}
%@see: 《矩阵论》(詹兴致) P2
设矩阵\(\vb{A}=(a_{ij})_n \in M_n(K)\).
\begin{enumerate}
	\item 如果\(\vb{A}\)左下角的元素全为零,即\begin{equation*}
		a_{ij} = 0
		\quad(i>j),
	\end{equation*}
	则称之为\DefineConcept{上三角形矩阵}(upper triangular matrix),
	%@see: https://mathworld.wolfram.com/UpperTriangularMatrix.html
	简称\DefineConcept{上三角阵}.

	\item 如果\(\vb{A}\)右上角的元素全为零,即\begin{equation*}
		a_{ij} = 0
		\quad(i<j),
	\end{equation*}
	则称之为\DefineConcept{下三角形矩阵}(lower triangular matrix),
	%@see: https://mathworld.wolfram.com/LowerTriangularMatrix.html
	简称\DefineConcept{下三角阵}.

	\item 如果\begin{equation*}
		a_{ij} = 0
		\quad(i>j+1),
	\end{equation*}
	则称之为\DefineConcept{黑森堡矩阵}(Hessenberg matrix).
	%@see: https://mathworld.wolfram.com/HessenbergMatrix.html
	%@see: https://en.wikipedia.org/wiki/Hessenberg_matrix
\end{enumerate}
%@see: https://mathworld.wolfram.com/TriangularMatrix.html
\end{definition}

\subsection{带状矩阵}
\begin{definition}
%@see: 《矩阵论》(詹兴致) P2
设矩阵\(\vb{A}=(a_{ij})_n \in M_n(K)\).
\begin{itemize}
	\item 如果\begin{equation*}
		a_{ij} = 0
		\quad(j-i>q),
	\end{equation*}
	则称“矩阵\(\vb{A}\)具有\DefineConcept{上带宽} \(q\)”.

	\item 如果\begin{equation*}
		a_{ij} = 0
		\quad(i-j>p),
	\end{equation*}
	则称“矩阵\(\vb{A}\)具有\DefineConcept{下带宽} \(p\)”.
\end{itemize}
\end{definition}

\begin{example}
下三角矩阵都具有上带宽\(0\).
\end{example}

\begin{example}
黑森堡矩阵都具有下带宽\(1\).
\end{example}

\begin{definition}
%@see: 《矩阵论》(詹兴致) P2
设矩阵\(\vb{A}=(a_{ij})_n \in M_n(K)\).
如果\(\vb{A}\)
具有上带宽\(q \leq n-2\)
或者
具有下带宽\(p \leq n-2\),
则称“矩阵\(\vb{A}\)是一个\DefineConcept{带状矩阵}”.
\end{definition}

\begin{example}
设\(\vb{A},\vb{B}\)都是\(n\)阶上三角阵,证明:\(\vb{A}\vb{B}\)是上三角阵.
\begin{proof}
利用数学归纳法.
当\(n=1\)时,\(\vb{A}=a\),\(\vb{B}=b\),\(\vb{A}\vb{B} = ab\),结论成立.

假设\(n=k\)时上三角阵的乘积是上三角阵.
当\(n=k+1\)时,对矩阵\(\vb{A}\)和\(b\)分块如下:\begin{equation*}
	\vb{A} = \begin{bmatrix}
		a_{11} & \vb{A}_2 \\
		\vb0 & \vb{A}_4
	\end{bmatrix},
	\qquad
	\vb{B} = \begin{bmatrix}
		b_{11} & \vb{B}_2 \\
		\vb0 & \vb{B}_4
	\end{bmatrix},
\end{equation*}
其中\(\vb{A}_4\)和\(\vb{B}_4\)都是\(k\)阶上三角阵,
由归纳假设,\(\vb{A}_4 \vb{B}_4\)是\(k\)阶上三角阵,
则\begin{equation*}
	\vb{A}\vb{B} = \begin{bmatrix}
		a_{11} b_{11} & a_{11} \vb{B}_2 + \vb{A}_2 \vb{B}_4 \\
		\vb0 & \vb{A}_4 \vb{B}_4
	\end{bmatrix},
\end{equation*}
即\(\vb{A}\vb{B}\)是\(k+1\)阶上三角阵.
\end{proof}
\end{example}

\subsection{置换矩阵}
\begin{definition}
%@see: 《矩阵论》(詹兴致) P3
设矩阵\(\vb{A}=(a_{ij})_n \in M_n(K)\).
如果\(a_{ij} \in \{0,1\}\ (i,j=1,2,\dotsc,n)\),
则称“矩阵\(\vb{A}\)是一个 \DefineConcept{0 - 1矩阵}”.
\end{definition}
\begin{definition}
%@see: 《矩阵论》(詹兴致) P3
设矩阵\(\vb{A}=(a_{ij})_n \in M_n(K)\).
如果矩阵\(\vb{A}\)是一个0 - 1矩阵,
且它的每一行每一列都恰好有一个\(1\),
则称“矩阵\(\vb{A}\)是一个\DefineConcept{置换矩阵}(permutation matrix)”.
%@see: https://mathworld.wolfram.com/PermutationMatrix.html
\end{definition}

%@Mathematica: GeneratePermutationMatrices[n_] := Table[
%					SparseArray[
%						Table[
%							{perm[[i]], i} -> 1,
%							{i, 1, n}
%						],
%						n, 0
%					],
%					{perm, Permutations[Range[n]]}
%				]
%@Mathematica: PrintPermutationMatrices[n_] := Module[
%					{permutationMatrices},
%					permutationMatrices = GeneratePermutationMatrices[n];
%					Table[
%						{Subscript["P", i], permutationMatrices[[i]] // MatrixForm},
%						{i, 1, n!}
%					] // TableForm
%				]
%@Mathematica: RenderTableOfPermutationMatrixOfOrder[n_] := Module[
%					{permutationMatrices},
%					permutationMatrices = GeneratePermutationMatrices[n];
%					TableForm[
%						Table[
%							Subscript[
%								"P",
%								Switch @@ Flatten[
%									{
%										{{permutationMatrices[[i]].permutationMatrices[[j]]}},
%										Table[{permutationMatrices[[l]], l}, {l, 1, n!}]
%									},
%									2
%								]
%							],
%							{i, 1, n!},
%							{j, 1, n!}
%						],
%						TableHeadings -> Table[Table[Subscript["P", k], {k, 1, n!}], {p, 1, 2}]
%					]
%				]

\begin{example}
2阶置换矩阵有2个:\begin{equation*}
	P_1 \defeq \begin{bmatrix}
		1 & 0 \\
		0 & 1
	\end{bmatrix},
	\qquad
	P_2 \defeq \begin{bmatrix}
		0 & 1 \\
		1 & 0
	\end{bmatrix}.
\end{equation*}
%@Mathematica: Subscript[P, 1] = {{1, 0}, {0, 1}}
%@Mathematica: Subscript[P, 2] = {{0, 1}, {1, 0}}
%@Mathematica: RenderTableOfPermutationMatrixOfOrder[2]
可以验证:\begin{equation*}
	P_1 P_1 = P_1,
	\qquad
	P_1 P_2 = P_2,
	\qquad
	P_2 P_1 = P_2,
	\qquad
	P_2 P_2 = P_1.
\end{equation*}
由此可见,这2个矩阵组成的集合对矩阵乘法成为一个交换群.
\end{example}

\begin{example}
3阶置换矩阵有6个:\begin{gather*}
	P_1 \defeq \begin{bmatrix}
		1 & 0 & 0 \\
		0 & 1 & 0 \\
		0 & 0 & 1
	\end{bmatrix},
	\qquad
	P_2 \defeq \begin{bmatrix}
		0 & 1 & 0 \\
		1 & 0 & 0 \\
		0 & 0 & 1
	\end{bmatrix},
	\qquad
	P_3 \defeq \begin{bmatrix}
		0 & 0 & 1 \\
		0 & 1 & 0 \\
		1 & 0 & 0
	\end{bmatrix},
	\\
	P_4 \defeq \begin{bmatrix}
		1 & 0 & 0 \\
		0 & 0 & 1 \\
		0 & 1 & 0
	\end{bmatrix},
	\qquad
	P_5 \defeq \begin{bmatrix}
		0 & 0 & 1 \\
		1 & 0 & 0 \\
		0 & 1 & 0
	\end{bmatrix},
	\qquad
	P_6 \defeq \begin{bmatrix}
		0 & 1 & 0 \\
		0 & 0 & 1 \\
		1 & 0 & 0
	\end{bmatrix}.
\end{gather*}
%@Mathematica: Subscript[P, 1] = {{1, 0, 0}, {0, 1, 0}, {0, 0, 1}}
%@Mathematica: Subscript[P, 2] = {{0, 1, 0}, {1, 0, 0}, {0, 0, 1}}
%@Mathematica: Subscript[P, 3] = {{0, 0, 1}, {0, 1, 0}, {1, 0, 0}}
%@Mathematica: Subscript[P, 4] = {{1, 0, 0}, {0, 0, 1}, {0, 1, 0}}
%@Mathematica: Subscript[P, 5] = {{0, 0, 1}, {1, 0, 0}, {0, 1, 0}}
%@Mathematica: Subscript[P, 6] = {{0, 1, 0}, {0, 0, 1}, {1, 0, 0}}
%@Mathematica: TableForm[Table[Subscript[P, i].Subscript[P, j] // MatrixForm, {i, 1, 6}, {j, 1, 6}], TableHeadings -> Table[Table[Subscript["P", k], {k, 1, 6}], {p, 1, 2}]]
%@Mathematica: RenderTableOfPermutationMatrixOfOrder[3]
%@Mathematica: Table[StringForm["P_`1` P_`2` = P_`3`", i, j, Switch[Subscript[P, i].Subscript[P, j], Subscript[P, 1], 1, Subscript[P, 2], 2, Subscript[P, 3], 3, Subscript[P, 4], 4, Subscript[P, 5], 5, Subscript[P, 6], 6]], {i, 1, 6}, {j, 1, 6}]
可以验证:\begin{gather*}
	P_1 P_1 = P_1,
	P_1 P_2 = P_2,
	P_1 P_3 = P_3,
	P_1 P_4 = P_4,
	P_1 P_5 = P_5,
	P_1 P_6 = P_6, \\
	P_2 P_1 = P_2,
	P_2 P_2 = P_1,
	P_2 P_3 = P_6,
	P_2 P_4 = P_5,
	P_2 P_5 = P_4,
	P_2 P_6 = P_3, \\
	P_3 P_1 = P_3,
	P_3 P_2 = P_5,
	P_3 P_3 = P_1,
	P_3 P_4 = P_6,
	P_3 P_5 = P_2,
	P_3 P_6 = P_4, \\
	P_4 P_1 = P_4,
	P_4 P_2 = P_6,
	P_4 P_3 = P_5,
	P_4 P_4 = P_1,
	P_4 P_5 = P_3,
	P_4 P_6 = P_2, \\
	P_5 P_1 = P_5,
	P_5 P_2 = P_3,
	P_5 P_3 = P_4,
	P_5 P_4 = P_2,
	P_5 P_5 = P_6,
	P_5 P_6 = P_1, \\
	P_6 P_1 = P_6,
	P_6 P_2 = P_4,
	P_6 P_3 = P_2,
	P_6 P_4 = P_3,
	P_6 P_5 = P_1,
	P_6 P_6 = P_5.
\end{gather*}
我们可以把上述结果列成表格,如\cref{table:置换矩阵.3阶置换矩阵的乘法表} 所示.
由此可见,这6个矩阵组成的集合对矩阵乘法成群,但是它不是交换群.
\begin{table}[hbt]
	\centering
	\begin{tblr}{c|*6c}
				& \(P_1\) & \(P_2\) & \(P_3\) & \(P_4\) & \(P_5\) & \(P_6\) \\ \hline
		\(P_1\) & \(P_1\) & \(P_2\) & \(P_3\) & \(P_4\) & \(P_5\) & \(P_6\) \\
		\(P_2\) & \(P_2\) & \(P_1\) & \(P_6\) & \(P_5\) & \(P_4\) & \(P_3\) \\
		\(P_3\) & \(P_3\) & \(P_5\) & \(P_1\) & \(P_6\) & \(P_2\) & \(P_4\) \\
		\(P_4\) & \(P_4\) & \(P_6\) & \(P_5\) & \(P_1\) & \(P_3\) & \(P_2\) \\
		\(P_5\) & \(P_5\) & \(P_3\) & \(P_4\) & \(P_2\) & \(P_6\) & \(P_1\) \\
		\(P_6\) & \(P_6\) & \(P_4\) & \(P_2\) & \(P_3\) & \(P_1\) & \(P_5\) \\
	\end{tblr}
	\caption{3阶置换矩阵的乘法表}
	\label{table:置换矩阵.3阶置换矩阵的乘法表}
\end{table}
\end{example}

因为置换矩阵是将单位矩阵的各行(或各列)重新排列,
所以我们有如下结论:
\begin{proposition}
\(n\)阶置换矩阵有\(n!\)个.
\end{proposition}

\subsection{对角矩阵}
\begin{definition}
如果矩阵\(\vb{A}=(a_{ij})_n \in M_n(K)\)除对角线以外的元素全为零,即\begin{equation*}
	a_{ij} = 0
	\quad(i \neq j),
\end{equation*}
那么称“\(\vb{A}\)是一个\(n\)阶\DefineConcept{对角矩阵}(diagonal matrix)”,
%@see: https://mathworld.wolfram.com/DiagonalMatrix.html
记作\(\diag(a_{11},a_{22},\dotsc,a_{nn})\).
\end{definition}

\begin{definition}
如果矩阵\(\vb{A}=(a_{ij})_{m \times n} \in M_{m \times n}(K)\)
除对角线以外的元素全为零,即\begin{equation*}
	a_{ij} = 0
	\quad(i \neq j),
\end{equation*}
那么称“\(\vb{A}\)是一个\(m \times n\) \DefineConcept{对角矩阵}”,
记作\(\diag_{m \times n}(a_{11},a_{22},\dotsc,a_{pp})\),
其中\(p = \min\{m,n\}\).
\end{definition}

\subsection{数量矩阵}
\begin{definition}
如果矩阵\(\vb{A}=(a_{ij})_n \in M_n(K)\)除对角线以外的元素全为零,且所有主对角元相同,即\begin{equation*}
	a_{ij} = \left\{ \begin{array}{cl}
		\lambda, & i = j, \\
		0, & i \neq j,
	\end{array} \right.
\end{equation*}
那么称“\(\vb{A}\)是一个\(n\)阶\DefineConcept{数量矩阵}”.
\end{definition}

\begin{proposition}
设\(\vb{A}\)是数域\(K\)上的一个\(n\)阶数量矩阵,
\(\vb{E}\)是数域\(K\)上的\(n\)阶单位矩阵,
则存在\(\lambda \in K\),
使得\(\vb{A} = \lambda \vb{E}\).
\end{proposition}

\subsection{对称矩阵,反对称矩阵}
\begin{definition}
%@see: 《Linear Algebra Done Right (Fourth Edition)》(Sheldon Axler) P337 9.11
若矩阵\(\vb{A} \in M_n(K)\)满足\begin{equation*}
    \vb{A}^T = \vb{A},
\end{equation*}
那么把\(\vb{A}\)称为\DefineConcept{对称矩阵}(symmetric matrix).
%@see: https://mathworld.wolfram.com/SymmetricMatrix.html
反之,如果\begin{equation*}
	\vb{A}^T \neq \vb{A},
\end{equation*}
那么把\(\vb{A}\)称为\DefineConcept{非对称矩阵}(asymmetric matrix).
%@see: https://mathworld.wolfram.com/AsymmetricMatrix.html
\end{definition}

\begin{definition}
如果矩阵\(\vb{A} \in M_n(K)\)满足\begin{equation*}
	\vb{A}^T = -\vb{A},
\end{equation*}
那么把\(\vb{A}\)称为\DefineConcept{反对称矩阵}(antisymmetric matrix)
%@see: https://mathworld.wolfram.com/AntisymmetricMatrix.html
或\DefineConcept{斜对称矩阵}.
\end{definition}

\begin{definition}
如果矩阵\(\vb{A} \in M_n(K)\)满足\begin{equation*}
    \vb{A}^H = \vb{A},
\end{equation*}
那么把\(\vb{A}\)称为\DefineConcept{厄米矩阵}(Hermitian matrix).
%@see: https://mathworld.wolfram.com/HermitianMatrix.html
\end{definition}

\begin{definition}
如果矩阵\(\vb{A} \in M_n(K)\)满足\begin{equation*}
	\vb{A}^H = -\vb{A},
\end{equation*}
那么把\(\vb{A}\)称为\DefineConcept{反厄米矩阵}(antihermitian matrix).
%@see: https://mathworld.wolfram.com/AntihermitianMatrix.html
\end{definition}

%\cref{theorem:对称矩阵.对称矩阵的合同类}
%\cref{theorem:反对称矩阵.反对称矩阵的合同类}

\begin{example}
设矩阵\(\vb{A} \in M_n(K)\).
试证:\(\vb{A}\vb{A}^T\)为对称矩阵.
\begin{proof}
因为\((\vb{A} \vb{A}^T)^T = (\vb{A}^T)^T \vb{A}^T = \vb{A} \vb{A}^T\),所以\(\vb{A} \vb{A}^T\)是对称矩阵.
\end{proof}
\end{example}
\begin{remark}
容易看出,\(\vb{A}^T\vb{A}\)也是对称矩阵.
\end{remark}

\begin{example}
设\(\vb{A}\)和\(\vb{B}\)是同阶对称矩阵.
试证:\(\vb{A}\vb{B}\)是对称矩阵的充分必要条件是\(\vb{A}\vb{B} = \vb{B}\vb{A}\).
\begin{proof}
因为\(\vb{A}\)和\(\vb{B}\)都是对称矩阵,所以\begin{equation*}
	\vb{A}^T = \vb{A},
	\qquad
	\vb{B}^T = \vb{B}.
\end{equation*}
在此条件下,有\begin{equation*}
	\text{$\vb{A}\vb{B}$是对称矩阵}
	\iff
	(\vb{A}\vb{B})^T
	= \vb{A}\vb{B}
	\iff
	\vb{B}^T\vb{A}^T
	= \vb{A}\vb{B}
	\iff
	\vb{B}\vb{A}
	= \vb{A}\vb{B}.
	\qedhere
\end{equation*}
\end{proof}
\end{example}
\begin{remark}
可以证明:同阶反对称矩阵的乘积是对称矩阵的充分必要条件也是\(\vb{A}\vb{B} = \vb{B}\vb{A}\).
类似地,我们还可以证明:同阶对称矩阵或同阶反对称矩阵的乘积是反对称矩阵的充分必要条件是
\(\vb{A}\vb{B} + \vb{B}\vb{A} = \vb0\).
\end{remark}

\begin{property}
反对称矩阵主对角线上的元素全为零.
\end{property}

\begin{example}
零矩阵\(\vb0\)是唯一一个既是实对称矩阵又是实反对称矩阵的矩阵.
\begin{proof}
\(\vb{A}^T = \vb{A} = -\vb{A} \implies 2\vb{A} = \vb{A}+\vb{A} = \vb0 \implies \vb{A} = \vb0\).
\end{proof}
\end{example}

\begin{example}
设\(\vb{A}\)是一个方阵,证明:\(\vb{A}+\vb{A}^T\)为对称矩阵,\(\vb{A}-\vb{A}^T\)为反对称矩阵.
\begin{proof}
因为\((\vb{A}+\vb{A}^T)^T = \vb{A}^T+\vb{A}\),
而\((\vb{A}-\vb{A}^T)^T = \vb{A}^T - \vb{A} = -(\vb{A}-\vb{A}^T)\),
所以\(\vb{A}+\vb{A}^T\)为对称矩阵,
\(\vb{A}-\vb{A}^T\)为反对称矩阵.
显然有\(\vb{A} = \frac{\vb{A} + \vb{A}^T}{2} + \frac{\vb{A} - \vb{A}^T}{2}\).
\end{proof}
\end{example}

\begin{example}
设\(\vb{A}\)是3阶实对称矩阵,\(\vb{B}\)是3阶实反对称矩阵,\(\vb{A}^2 = \vb{B}^2\).
试证:\(\vb{A} = \vb{B} = \vb0\).
\begin{proof}
设\(\vb{A} = (a_{ij})_n\),\(\vb{B} = (b_{ij})_n\).
因为\(\vb{A} = \vb{A}^T\),\(\vb{A}^2 = \vb{A}^T \vb{A}\),
所以\(\vb{A}^2\)的\(\opair{i,j}\)元素为
\(a_{1i} a_{1j} + a_{2i} a_{2j} + \dotsb + a_{ni} a_{nj}\).
因为\(\vb{B} = -\vb{B}^T\),\(\vb{B}^2 = -\vb{B}^T \vb{B}\),
所以\(\vb{B}^2\)的\(\opair{i,j}\)元素为
\(-(b_{1i} b_{1j} + b_{2i} b_{2j} + \dotsb + b_{ni} b_{nj})\).
因为\(\vb{A}^2 = \vb{B}^2\),
所以\begin{equation*}
	a_{1i} a_{1j} + a_{2i} a_{2j} + \dotsb + a_{ni} a_{nj}
	= -(b_{1i} b_{1j} + b_{2i} b_{2j} + \dotsb + b_{ni} b_{nj}).
\end{equation*}

当\(i=j\)时,上式变为\(a_{1i}^2 + a_{2i}^2 + \dotsb + a_{ni}^2
= -(b_{1i}^2 + b_{2i}^2 + \dotsb + b_{ni}^2)\),
又由\(a_{ij},b_{ij} \in \mathbb{R}\)
可知\(a_{1i}^2 + a_{2i}^2 + \dotsb + a_{ni}^2 \geq 0\),
\(-(b_{1i}^2 + b_{2i}^2 + \dotsb + b_{ni}^2) \leq 0\),
所以\begin{equation*}
	a_{1i}^2 + a_{2i}^2 + \dotsb + a_{ni}^2
	= -(b_{1i}^2 + b_{2i}^2 + \dotsb + b_{ni}^2) = 0,
\end{equation*}
进而有\begin{equation*}
	a_{1i} = a_{2i} = \dotsb = a_{ni} = b_{1i} = b_{2i} = \dotsb = b_{ni} = 0.
	\qedhere
\end{equation*}
\end{proof}
\end{example}

\subsection{幂零矩阵}
%@Mathematica: MatrixPowerTable[A_, n_] := Table[MatrixPower[A, k] // MatrixForm, {k, 1, n}] // DeleteDuplicates

\begin{definition}
设矩阵\(\vb{A} \in M_n(K)\).
若\begin{equation*}
	(\exists m\in\mathbb{N}^+)
	[\vb{A}^m = \vb0],
\end{equation*}
则称“\(\vb{A}\)是\DefineConcept{幂零矩阵}(nilpotent matrix)”.
%@see: https://mathworld.wolfram.com/NilpotentMatrix.html
称使得\(\vb{A}^m = \vb0\)成立的最小正整数\begin{equation*}
    \min\Set{ m\in\mathbb{N}^+ \given \vb{A}^m = \vb0 }
\end{equation*}为“\(\vb{A}\)的\DefineConcept{幂零指数}”.
\end{definition}
%\cref{example:幂零矩阵.幂零矩阵的行列式}
%\cref{example:幂零矩阵.幂零矩阵的特征值的性质}
%\cref{example:幂零矩阵.幂零矩阵的相似类}
%\cref{example:幂零矩阵.非零的幂零矩阵不可以相似对角化}

\begin{example}
%@Mathematica: MatrixPowerTable[{{0, 1, 1}, {0, 0, 1}, {0, 0, 0}}, 5]
矩阵\begin{equation*}
	\begin{bmatrix}
		0 & 1 & 1 \\
		0 & 0 & 1 \\
		0 & 0 & 0
	\end{bmatrix}
\end{equation*}是幂零矩阵,
因为\begin{equation*}
	\begin{bmatrix}
		0 & 1 & 1 \\
		0 & 0 & 1 \\
		0 & 0 & 0
	\end{bmatrix}^2
	=
	\begin{bmatrix}
		0 & 0 & 1 \\
		0 & 0 & 0 \\
		0 & 0 & 0
	\end{bmatrix},
	\qquad
	\begin{bmatrix}
		0 & 1 & 1 \\
		0 & 0 & 1 \\
		0 & 0 & 0
	\end{bmatrix}
	\begin{bmatrix}
		0 & 0 & 1 \\
		0 & 0 & 0 \\
		0 & 0 & 0
	\end{bmatrix}
	=
	\begin{bmatrix}
		0 & 0 & 0 \\
		0 & 0 & 0 \\
		0 & 0 & 0
	\end{bmatrix},
\end{equation*}
%@Mathematica: MatrixPowerTable[{{0, 1, 1}, {0, 0, 1}, {0, 0, 0}}, 5] // Length
它的幂零指数是\(3\).
\end{example}
\begin{example}
%@Mathematica: MatrixPowerTable[{{0, 1, 0, 0}, {0, 0, 1, 0}, {0, 0, 0, 1}, {0, 0, 0, 0}}, 5]
矩阵\(
	\begin{bmatrix}
		0 & 1 & 0 & 0 \\
		0 & 0 & 1 & 0 \\
		0 & 0 & 0 & 1 \\
		0 & 0 & 0 & 0
	\end{bmatrix}
\)是幂零矩阵,
%@Mathematica: MatrixPowerTable[{{0, 1, 0, 0}, {0, 0, 1, 0}, {0, 0, 0, 1}, {0, 0, 0, 0}}, 5] // Length
它的幂零指数是\(4\).
\end{example}

\subsection{幂幺矩阵}
\begin{definition}
设矩阵\(\vb{A} \in M_n(K)\),\(\vb{E}\)是数域\(K\)上的\(n\)阶单位矩阵.
若\begin{equation*}
	(\exists m\in\mathbb{N}^+)
	[(\vb{A}-\vb{E})^m=\vb0],
\end{equation*}
则称“\(\vb{A}\)是\DefineConcept{幂幺矩阵}(unipotent matrix)”.
%@see: https://mathworld.wolfram.com/Unipotent.html
称使得\((\vb{A}-\vb{E})^m=\vb0\)成立的最小正整数\begin{equation*}
    \min\Set{ m\in\mathbb{N}^+ \given (\vb{A}-\vb{E})^m=\vb0 }
\end{equation*}为“\(\vb{A}\)的\DefineConcept{幂幺指数}”.
\end{definition}
%\cref{example:幂幺矩阵.幂幺矩阵的特征值的性质}

\subsection{幂等矩阵}
\begin{definition}\label{definition:幂等矩阵.幂等矩阵的定义}
设矩阵\(\vb{A} \in M_n(K)\).
若\(\vb{A}^2=\vb{A}\),
则称“\(\vb{A}\)是\DefineConcept{幂等矩阵}(idempotent matrix)”.
%@see: https://mathworld.wolfram.com/IdempotentMatrix.html
\end{definition}
%\cref{example:幂等矩阵.幂等矩阵的秩的性质1}
%\cref{example:幂等矩阵.幂等矩阵的特征值的性质}
%\cref{example:幂等矩阵.幂等矩阵的相似类}

\begin{example}
单位矩阵、零矩阵都是幂等矩阵.
\end{example}

\begin{example}
%@Mathematica: MatrixPowerTable[{{0, 1, 1}, {0, 1, 1}, {0, 0, 0}}, 5]
矩阵\(
	\begin{bmatrix}
		0 & 1 & 1 \\
		0 & 1 & 1 \\
		0 & 0 & 0
	\end{bmatrix}
\)是一个幂等矩阵.
\end{example}

\subsection{对合矩阵}
\begin{definition}
设矩阵\(\vb{A} \in M_n(K)\),\(\vb{E}\)是数域\(K\)上的\(n\)阶单位矩阵.
若\(\vb{A}^2=\vb{E}\),
则称“\(\vb{A}\)是\DefineConcept{对合矩阵}(involutory matrix)”.
%@see: https://mathworld.wolfram.com/InvolutoryMatrix.html
\end{definition}
%\cref{example:对合矩阵.对合矩阵的秩的性质1}
%\cref{example:对合矩阵.对合矩阵的逆矩阵}
%\cref{example:对合矩阵.对合矩阵的相似类}

\begin{example}
%@Mathematica: MatrixPowerTable[{{0, 1}, {1, 0}}, 5]
矩阵\(
	\begin{bmatrix}
		0 & 1 \\
		1 & 0
	\end{bmatrix}
\)满足\begin{equation*}
	\begin{bmatrix}
		0 & 1 \\
		1 & 0
	\end{bmatrix}^2
	=
	\begin{bmatrix}
		0 & 1 \\
		1 & 0
	\end{bmatrix}
	\begin{bmatrix}
		0 & 1 \\
		1 & 0
	\end{bmatrix}
	= \begin{bmatrix}
		1 & 0 \\
		0 & 1
	\end{bmatrix},
\end{equation*}是一个对合矩阵.
\end{example}
\begin{example}
%@Mathematica: MatrixPowerTable[{{0, I}, {-I, 0}}, 5]
矩阵\(
	\begin{bmatrix}
		0 & \iu \\
		-\iu & 0
	\end{bmatrix},
	\qquad
	\begin{bmatrix}
		0 & -\iu \\
		\iu & 0
	\end{bmatrix}
\)都是对合矩阵.
\end{example}
\begin{example}
%@Mathematica: MatrixPowerTable[{{0, 0, 1}, {0, 1, 0}, {1, 0, 0}}, 5]
%@Mathematica: MatrixPowerTable[{{0, 1, 0}, {1, 0, 0}, {0, 0, 1}}, 5]
矩阵\begin{equation*}
	\begin{bmatrix}
		0 & 0 & 1 \\
		0 & 1 & 0 \\
		1 & 0 & 0
	\end{bmatrix},
	\qquad
	\begin{bmatrix}
		0 & 1 & 0 \\
		1 & 0 & 0 \\
		0 & 0 & 1
	\end{bmatrix}
\end{equation*}都是对合矩阵.
\end{example}

\subsection{周期矩阵}
\begin{definition}
设矩阵\(\vb{A} \in M_n(K)\),
\(\vb{E}\)是数域\(K\)上的\(n\)阶单位矩阵.
若\begin{equation*}
	(\exists m\in\mathbb{N}^+)
	[\vb{A}^m = \vb{E}],
\end{equation*}
则称“\(\vb{A}\)是\DefineConcept{周期矩阵}(periodic matrix)”.
使\(\vb{A}^m = \vb{E}\)成立的最小正整数\begin{equation*}
	\min\Set{ m\in\mathbb{N}^+ \given \vb{A}^m = \vb{E} }
\end{equation*}称为“\(\vb{A}\)的\DefineConcept{周期}”.
%@see: https://mathworld.wolfram.com/PeriodicMatrix.html
\end{definition}

\subsection{提俄普利茨矩阵}
\begin{definition}
%@see: 《矩阵论》(詹兴致) P3
设矩阵\(\vb{A}=(a_{ij})_n \in M_n(K)\).
如果存在数\begin{equation*}
	b_{-n+1},\dotsc,b_{-1},b_0,b_1,\dotsc,b_{n-1},
\end{equation*}
使得\(a_{ij} = b_{j-i}\ (i,j=1,2,\dotsc,n)\),
则称“矩阵\(\vb{A}\)是一个\DefineConcept{提俄普利茨矩阵}(Toeplitz matrix)”.
%@see: https://mathworld.wolfram.com/ToeplitzMatrix.html
%@see: https://en.wikipedia.org/wiki/Toeplitz_matrix
\end{definition}

\subsection{汉克尔矩阵}
\begin{definition}
%@see: 《矩阵论》(詹兴致) P3
设矩阵\(\vb{A}=(a_{ij})_n \in M_n(K)\).
如果存在数\begin{equation*}
	b_1,\dotsc,b_{2n-1},
\end{equation*}
使得\(a_{ij} = b_{i+j-1}\ (i,j=1,2,\dotsc,n)\),
则称“矩阵\(\vb{A}\)是一个\DefineConcept{汉克尔矩阵}(Hankel matrix)”.
%@see: https://mathworld.wolfram.com/HankelMatrix.html
%@see: https://en.wikipedia.org/wiki/Hankel_matrix
\end{definition}

\subsection{循环矩阵}
%@see: 《矩阵论》(詹兴致) P3
形如\begin{equation*}
	\begin{bmatrix}
		a_1 & a_2 & a_3 & \dots & a_n \\
		a_n & a_1 & a_2 & \dots & a_{n-1} \\
		a_{n-1} & a_n & a_1 & \dots & a_{n-2} \\
		\vdots & \vdots & \vdots & & \vdots \\
		a_2 & a_3 & a_4 \dots & a_1
	\end{bmatrix}
\end{equation*}
的矩阵,称为\DefineConcept{循环矩阵}(circulant matrix),
记为\(\Circ(a_1,a_2,a_3,\dotsc,a_n)\).
%@see: https://mathworld.wolfram.com/CirculantMatrix.html

%@Mathematica: CirculantMatrix[l_List?VectorQ] := NestList[RotateRight, RotateRight[l], Length[l] - 1]
%@Mathematica: CirculantMatrix[l_List?VectorQ, n_Integer] := NestList[RotateRight, RotateRight[Join[Table[0, {n - Length[l]}], l]], n - 1] /; n >= Length[l]

可以注意到,循环矩阵是一种特殊的提俄普利茨矩阵.

%@see: 《矩阵论》(詹兴致) P3
特别地,把\(\Circ(0,1,0,\dotsc,0)\)称为\DefineConcept{基本循环矩阵}.

\begin{proposition}
%@see: 《矩阵论》(詹兴致) P3
设\(\vb{P}\)是一个\(n\)阶置换矩阵,
\(\vb{A}\)是一个\(n\)阶循环矩阵,
则\begin{equation*}
%@see: 《矩阵论》(詹兴致) P3 (1.5)
	\vb{A} = \sum_{k=0}^{n-1} a_{k+1} \vb{P}^k.
\end{equation*}
\end{proposition}

\subsection{傅里叶矩阵}
当\(w \in \mathbb{C}\)满足\(w^n = 1\)时,
我们把形如\begin{equation*}
	\frac1{\sqrt{n}}
	% 这里的常系数是用来将后面的矩阵中的列向量组变成正交规范组
	\begin{bmatrix}
		1 & 1 & 1 & \dots & 1 \\
		1 & w & w^2 & \dots & w^{n-1} \\
		1 & w^2 & w^4 & \dots & w^{2(n-1)} \\
		\vdots & \vdots & \vdots & & \vdots \\
		1 & w^{n-1} & w^{2(n-1)} & \dots & w^{(n-1)^2}
	\end{bmatrix}
\end{equation*}
的矩阵称为\DefineConcept{傅里叶矩阵}(Fourier matrix),
记为\(\vb{F}_n\).
%@see: https://mathworld.wolfram.com/FourierMatrix.html

\begin{example}
%@Mathematica: FourierMatrix[4] // MatrixForm
矩阵\(
	\frac12
	\begin{bmatrix}
		1 & 1 & 1 & 1 \\
		1 & \iu & -1 & -\iu \\
		1 & -1 & 1 & -1 \\
		1 & -\iu & -1 & \iu
	\end{bmatrix}
\)是一个4阶傅里叶矩阵.
\end{example}

容易看出,傅里叶矩阵\(\vb{F}_n\)的\((i,j)\)元素是\(
	w^{(i-1)(j-1)}
	\ (i,j=1,2,\dotsc,n)
\).
我们还可以看出,
\(\vb{F}_n\)的第\(j\)列向量\(\alpha_j\)与它的第\(k\)列向量\(\alpha_k\)
满足\(\alpha_j^H \alpha_k = 0\),
这就说明傅里叶矩阵\(\vb{F}_n\)的列向量组两两正交,
从而有\(\vb{F}_n^H \vb{F}_n = \vb{E}\).

%@see: 《Introduction to Linear Algebra (2016)》(Gilbert Strang) P448 (4)
容易证明:\begin{equation*}
	\vb{F}_{2n}
	= \begin{bmatrix}
		\vb{E} & \vb{D} \\
		\vb{E} & -\vb{D}
	\end{bmatrix}
	\begin{bmatrix}
		\vb{F}_n & \vb0 \\
		\vb0 & \vb{F}_n
	\end{bmatrix}
	\vb{P},
\end{equation*}
其中\(n\)是偶数,
对角矩阵\(
	\vb{D} \defeq \diag(1,w,w^2,\dotsc,w^{n-1})
\),
而\(\vb{P}\)是一个置换矩阵
(它把奇数行排到前\(n/2\)行,把偶数行排到后\(n/2\)行).
