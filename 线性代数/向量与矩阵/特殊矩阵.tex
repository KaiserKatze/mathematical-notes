\section{特殊矩阵}
\subsection{三角矩阵}
\begin{definition}
设矩阵\(\A=(a_{ij})_n \in M_n(K)\).
\begin{enumerate}
	\item 如果\(\A\)左下角的元素全为零,即\[
		a_{ij} = 0
		\quad(i>j),
	\]
	则称之为\DefineConcept{上三角形矩阵}(upper triangular matrix),
	%@see: https://mathworld.wolfram.com/UpperTriangularMatrix.html
	简称\DefineConcept{上三角阵}.

	\item 如果\(\A\)右上角的元素全为零,即\[
		a_{ij} = 0
		\quad(i<j),
	\]
	则称之为\DefineConcept{下三角形矩阵}(lower triangular matrix),
	%@see: https://mathworld.wolfram.com/LowerTriangularMatrix.html
	简称\DefineConcept{下三角阵}.
\end{enumerate}
%@see: https://mathworld.wolfram.com/TriangularMatrix.html
\end{definition}

\begin{example}
设\(\A,\B\)都是\(n\)阶上三角阵,证明:\(\A\B\)是上三角阵.
\begin{proof}
利用数学归纳法.
当\(n=1\)时,\(\A=a\),\(\B=b\),\(\A\B = ab\),结论成立.

假设\(n=k\)时上三角阵的乘积是上三角阵.
当\(n=k+1\)时,对矩阵\(\A\)和\(b\)分块如下:\[
	\A = \begin{bmatrix}
		a_{11} & \A_2 \\
		\vb0 & \A_4
	\end{bmatrix},
	\qquad
	\B = \begin{bmatrix}
		b_{11} & \B_2 \\
		\vb0 & \B_4
	\end{bmatrix},
\]
其中\(\A_4\)和\(\B_4\)都是\(k\)阶上三角阵,
由归纳假设,\(\A_4 \B_4\)是\(k\)阶上三角阵,
则\[
	\A\B = \begin{bmatrix}
		a_{11} b_{11} & a_{11} \B_2 + \A_2 \B_4 \\
		\vb0 & \A_4 \B_4
	\end{bmatrix},
\]
即\(\A\B\)是\(k+1\)阶上三角阵.
\end{proof}
\end{example}

\subsection{对角矩阵}
\begin{definition}
如果矩阵\(\A=(a_{ij})_n \in M_n(K)\)除对角线以外的元素全为零,即\[
	a_{ij} = 0
	\quad(i \neq j),
\]
那么把\(\A\)称为\DefineConcept{对角矩阵}(diagonal matrix),
%@see: https://mathworld.wolfram.com/DiagonalMatrix.html
记作\(\diag(a_{11},a_{22},\dotsc,a_{nn})\).
\end{definition}

\subsection{对称矩阵,厄米矩阵}
\begin{definition}
若矩阵\(\A \in M_n(K)\)满足\[
    \A^T = \A,
\]
那么把\(\A\)称为\DefineConcept{对称矩阵}(symmetric matrix).
%@see: https://mathworld.wolfram.com/SymmetricMatrix.html
反之,如果\[
	\A^T \neq \A,
\]
那么把\(\A\)称为\DefineConcept{非对称矩阵}(asymmetric matrix).
%@see: https://mathworld.wolfram.com/AsymmetricMatrix.html
\end{definition}

\begin{definition}
如果矩阵\(\A \in M_n(K)\)满足\[
    \A^H = \A,
\]
那么把\(\A\)称为\DefineConcept{厄米矩阵}(Hermitian matrix).
%@see: https://mathworld.wolfram.com/HermitianMatrix.html
\end{definition}

\begin{example}
设矩阵\(\A \in M_n(K)\).
试证:\(\A\A^T\)为对称矩阵.
\begin{proof}
因为\((\A \A^T)^T = (\A^T)^T \A^T = \A \A^T\),所以\(\A \A^T\)是对称矩阵.
\end{proof}
\end{example}
\begin{remark}
容易看出,\(\A^T\A\)也是对称矩阵.
\end{remark}

\begin{example}
设\(\vb{A}\)和\(\vb{B}\)是同阶对称矩阵.
试证:\(\vb{A}\vb{B}\)是对称矩阵的充分必要条件是\(\vb{A}\vb{B} = \vb{B}\vb{A}\).
\begin{proof}
因为\(\vb{A}\)和\(\vb{B}\)都是对称矩阵,所以\[
	\vb{A}^T = \vb{A},
	\qquad
	\vb{B}^T = \vb{B}.
\]
在此条件下,有\[
	\text{$\vb{A}\vb{B}$是对称矩阵}
	\iff
	(\vb{A}\vb{B})^T
	= \vb{A}\vb{B}
	\iff
	\vb{B}^T\vb{A}^T
	= \vb{A}\vb{B}
	\iff
	\vb{B}\vb{A}
	= \vb{A}\vb{B}.
	\qedhere
\]
\end{proof}
\end{example}
\begin{remark}
可以证明:同阶反对称矩阵的乘积是对称矩阵的充分必要条件也是\(\vb{A}\vb{B} = \vb{B}\vb{A}\).
类似地,我们还可以证明:同阶对称矩阵或同阶反对称矩阵的乘积是反对称矩阵的充分必要条件是
\(\vb{A}\vb{B} + \vb{B}\vb{A} = \vb0\).
\end{remark}

\subsection{反对称矩阵}
\begin{definition}
如果矩阵\(\A \in M_n(K)\)满足\[
	\A^T = -\A,
\]
那么把\(\A\)称为\DefineConcept{反对称矩阵}(antisymmetric matrix)
%@see: https://mathworld.wolfram.com/AntisymmetricMatrix.html
或\DefineConcept{斜对称矩阵}.
\end{definition}

\begin{definition}
如果矩阵\(\A \in M_n(K)\)满足\[
	\A^H = -\A,
\]
那么把\(\A\)称为\DefineConcept{反厄米矩阵}(antihermitian matrix).
%@see: https://mathworld.wolfram.com/AntihermitianMatrix.html
\end{definition}

\begin{property}
反对称矩阵主对角线上的元素全为零.
\end{property}

\begin{example}
零矩阵\(\vb0\)是唯一一个既是实对称矩阵又是实反对称矩阵的矩阵.
\begin{proof}
\(\A^T = \A = -\A \implies 2\A = \A+\A = \vb0 \implies \A = \vb0\).
\end{proof}
\end{example}

\begin{example}
设\(\A\)是一个方阵,证明:\(\A+\A^T\)为对称矩阵,\(\A-\A^T\)为反对称矩阵.
\begin{proof}
因为\((\A+\A^T)^T = \A^T+\A\),
而\((\A-\A^T)^T = \A^T - \A = -(\A-\A^T)\),
所以\(\A+\A^T\)为对称矩阵,
\(\A-\A^T\)为反对称矩阵.
显然有\(\A = \frac{\A + \A^T}{2} + \frac{\A - \A^T}{2}\).
\end{proof}
\end{example}

\begin{example}
设\(\A\)是3阶实对称矩阵,\(\B\)是3阶实反对称矩阵,\(\A^2 = \B^2\).
试证:\(\A = \B = \vb0\).
\begin{proof}
设\(\A = (a_{ij})_n\),\(\B = (b_{ij})_n\).
因为\(\A = \A^T\),\(\A^2 = \A^T \A\),
所以\(\A^2\)的\(\opair{i,j}\)元素为
\(a_{1i} a_{1j} + a_{2i} a_{2j} + \dotsb + a_{ni} a_{nj}\).
因为\(\B = -\B^T\),\(\B^2 = -\B^T \B\),
所以\(\B^2\)的\(\opair{i,j}\)元素为
\(-(b_{1i} b_{1j} + b_{2i} b_{2j} + \dotsb + b_{ni} b_{nj})\).
因为\(\A^2 = \B^2\),
所以\[
	a_{1i} a_{1j} + a_{2i} a_{2j} + \dotsb + a_{ni} a_{nj}
	= -(b_{1i} b_{1j} + b_{2i} b_{2j} + \dotsb + b_{ni} b_{nj}).
\]

当\(i=j\)时,上式变为\(a_{1i}^2 + a_{2i}^2 + \dotsb + a_{ni}^2
= -(b_{1i}^2 + b_{2i}^2 + \dotsb + b_{ni}^2)\),
又由\(a_{ij},b_{ij} \in \mathbb{R}\)
可知\(a_{1i}^2 + a_{2i}^2 + \dotsb + a_{ni}^2 \geq 0\),
\(-(b_{1i}^2 + b_{2i}^2 + \dotsb + b_{ni}^2) \leq 0\),
所以\[
	a_{1i}^2 + a_{2i}^2 + \dotsb + a_{ni}^2
	= -(b_{1i}^2 + b_{2i}^2 + \dotsb + b_{ni}^2) = 0,
\]
进而有\[
	a_{1i} = a_{2i} = \dotsb = a_{ni} = b_{1i} = b_{2i} = \dotsb = b_{ni} = 0.
	\qedhere
\]
\end{proof}
\end{example}

\subsection{幂零矩阵}
\begin{definition}
设矩阵\(\A \in M_n(K)\).
若\[
	(\exists m\in\mathbb{N}^+)
	[\A^m = \vb0],
\]
则称“\(\A\)是\DefineConcept{幂零矩阵}(nilpotent matrix)”.
%@see: https://mathworld.wolfram.com/NilpotentMatrix.html
称使得\(\A^m = \vb0\)成立的最小正整数\[
    \min\Set{ m\in\mathbb{N}^+ \given \A^m = \vb0 }
\]为“\(\A\)的\DefineConcept{幂零指数}”.
\end{definition}
%\cref{example:幂零矩阵.幂零矩阵的行列式}
%\cref{example:幂零矩阵.幂零矩阵的特征值的性质}

\subsection{幂幺矩阵}
\begin{definition}
设矩阵\(\A \in M_n(K)\),\(\E\)是数域\(K\)上的\(n\)阶单位矩阵.
若\[
	(\exists m\in\mathbb{N}^+)
	[(\A-\E)^m=\vb0],
\]
则称“\(\A\)是\DefineConcept{幂幺矩阵}(unipotent matrix)”.
%@see: https://mathworld.wolfram.com/Unipotent.html
称使得\((\A-\E)^m=\vb0\)成立的最小正整数\[
    \min\Set{ m\in\mathbb{N}^+ \given (\A-\E)^m=\vb0 }
\]为“\(\A\)的\DefineConcept{幂幺指数}”.
\end{definition}
%\cref{example:幂幺矩阵.幂幺矩阵的特征值的性质}

\subsection{幂等矩阵}
\begin{definition}
设矩阵\(\A \in M_n(K)\).
若\(\A^2=\A\),
则称“\(\A\)是\DefineConcept{幂等矩阵}(idempotent matrix)”.
%@see: https://mathworld.wolfram.com/IdempotentMatrix.html
\end{definition}
%\cref{example:幂等矩阵.幂等矩阵的秩的性质1}
%\cref{example:幂等矩阵.幂等矩阵的特征值的性质}

\subsection{对合矩阵}
\begin{definition}
设矩阵\(\A \in M_n(K)\),\(\E\)是数域\(K\)上的\(n\)阶单位矩阵.
若\(\A^2=\E\),
则称“\(\A\)是\DefineConcept{对合矩阵}(involutory matrix)”.
%@see: https://mathworld.wolfram.com/InvolutoryMatrix.html
\end{definition}
%\cref{example:对合矩阵.对合矩阵的秩的性质1}

\subsection{周期矩阵}
\begin{definition}
设矩阵\(\A \in M_n(K)\),
\(\E\)是数域\(K\)上的\(n\)阶单位矩阵.
若\[
	(\exists m\in\mathbb{N}^+)
	[\A^m = \E],
\]
则称“\(\A\)是\DefineConcept{周期矩阵}(periodic matrix)”.
使\(\A^m = \E\)成立的最小正整数\[
	\min\Set{ m\in\mathbb{N}^+ \given \A^m = \E }
\]称为“\(\A\)的\DefineConcept{周期}”.
%@see: https://mathworld.wolfram.com/PeriodicMatrix.html
\end{definition}

% \subsection{汉克尔矩阵}
%TODO
%@see: https://mathworld.wolfram.com/HankelMatrix.html
