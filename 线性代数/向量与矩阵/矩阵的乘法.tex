\section{矩阵的乘法}
\subsection{矩阵乘法的概念}
\begin{definition}
设\(\vb{A} = (a_{ij})_{s \times n}\),
\(\vb{B} = (b_{ij})_{n \times m}\),
\(\vb{C} = (c_{ij})_{s \times m}\).
如果满足\begin{equation*}
	c_{ij} = \sum_{k=1}^n {a_{ik} b_{kj}},
	\quad
	i=1,2,\dotsc,s;j=1,2,\dotsc,m,
\end{equation*}
则称矩阵\(\vb{C}\)是“\(\vb{A}\)与\(\vb{B}\)的\DefineConcept{乘积}”,
记作\(\vb{C} = \vb{A} \vb{B}\).
\end{definition}
\begin{remark}
如果我们分别对\(\vb{A}\)和\(\vb{B}\)做行分块和列分块,得\begin{equation*}
	\vb{A}=(\AutoTuple{\vb\alpha}{s}[,][T])^T, \qquad
	\vb{B}=(\AutoTuple{\vb\beta}{m}),
\end{equation*}
那么\begin{itemize}
	%@see: 《Linear Algebra Done Right (Fourth Edition)》(Sheldon Axler) P75 3.46
	\item \(\vb{A}\)与\(\vb{B}\)的乘积\(\vb{A}\vb{B}\)的\((i,j)\)元素,
	等于\(\vb{A}\)的第\(i\)行向量\(\vb\alpha_i^T\)与\(\vb{B}\)的第\(j\)列向量\(\vb\beta_j\)的乘积,
	即\begin{equation*}
		c_{ij} = \vb\alpha_i^T \vb\beta_j,
		\quad
		i=1,2,\dotsc,s;j=1,2,\dotsc,m;
	\end{equation*}

	%@see: 《Linear Algebra Done Right (Fourth Edition)》(Sheldon Axler) P75 3.48
	%@see: 《Linear Algebra Done Right (Fourth Edition)》(Sheldon Axler) P76 3.50
	\item \(\vb{A}\vb{B}\)的第\(k\)列向量,
	等于\(\vb{A}\)与\(\vb{B}\)的第\(k\)列向量\(\vb\beta_k\)的乘积,
	即\begin{equation*}
		\MatrixEntry{(\vb{A}\vb{B})}{*,k}
		= \vb{A} (\MatrixEntry{\vb{B}}{*,k}),
		\quad k=1,2,\dotsc,m.
	\end{equation*}
	这说明\(\vb{C}\)中每一列都是\(\vb{A}\)中各列的一个线性组合.

	%@see: 《Linear Algebra Done Right (Fourth Edition)》(Sheldon Axler) P80 Exercises 3C 8.
	%@see: 《Linear Algebra Done Right (Fourth Edition)》(Sheldon Axler) P80 Exercises 3C 9.
	\item \(\vb{A}\vb{B}\)的第\(k\)行向量,
	等于\(\vb{A}\)的第\(k\)行向量\(\vb\alpha_k^T\)与\(\vb{B}\)的乘积,
	即\begin{equation*}
		\MatrixEntry{(\vb{A}\vb{B})}{k,*}
		= (\MatrixEntry{\vb{A}}{k,*}) \vb{B},
		\quad k=1,2,\dotsc,s.
	\end{equation*}
	这说明\(\vb{C}\)中每一行都是\(\vb{B}\)中各行的一个线性组合.
\end{itemize}
\end{remark}
\begin{remark}
如果我们分别对\(\vb{A}\)和\(\vb{B}\)做列分块和行分块,
\begingroup%
\def\mx{\vb{\xi}}%
\def\mz{\vb{\zeta}}%
得\begin{equation*}
	\vb{A}=(\AutoTuple{\mx}{n}), \qquad
	\vb{B}=(\AutoTuple{\mz}{n}[,][T])^T,
\end{equation*}
那么有\begin{equation}
	\mx_i \mz_i^T
	= \begin{bmatrix}
		a_{1i} b_{i1} & a_{1i} b_{i2} & \dots & a_{1i} b_{im} \\
		a_{2i} b_{i1} & a_{2i} b_{i2} & \dots & a_{2i} b_{im} \\
		\vdots & \vdots & & \vdots \\
		a_{si} b_{i1} & a_{si} b_{i2} & \dots & a_{si} b_{im} \\
	\end{bmatrix},
	\quad
	i=1,2,\dotsc,n,
\end{equation}
于是\begin{equation*}
	\vb{A}\vb{B}=\sum_{i=1}^n \mx_i \mz_i^T.
\end{equation*}
\endgroup%
\end{remark}

\begin{proposition}
矩阵乘法不满足交换律.
\end{proposition}

\begin{definition}
设矩阵\(\vb{A},\vb{B} \in M_n(K)\).
如果\begin{equation*}
	\vb{A} \vb{B} = \vb{B} \vb{A},
\end{equation*}
则称“\(\vb{A}\)与\(\vb{B}\)~\DefineConcept{可交换}”
或“\(\vb{A}\)与\(\vb{B}\)的乘积服从交换律”.
\end{definition}

\begin{example}
矩阵\begin{equation*}
	\begin{bmatrix}
		1 & 0 & 0 \\
		0 & -1 & 0 \\
		0 & 0 & -1
	\end{bmatrix}
	\quad\text{与}\quad
	\begin{bmatrix}
		1 & 0 & 0 \\
		0 & 0 & 1 \\
		0 & 1 & 0
	\end{bmatrix}
\end{equation*}可交换,这是因为\begin{equation*}
	\begin{bmatrix}
		1 & 0 & 0 \\
		0 & -1 & 0 \\
		0 & 0 & -1
	\end{bmatrix}
	\begin{bmatrix}
		1 & 0 & 0 \\
		0 & 0 & 1 \\
		0 & 1 & 0
	\end{bmatrix}
	= \begin{bmatrix}
		1 & 0 & 0 \\
		0 & 0 & -1 \\
		0 & -1 & 0
	\end{bmatrix}
	= \begin{bmatrix}
		1 & 0 & 0 \\
		0 & 0 & 1 \\
		0 & 1 & 0
	\end{bmatrix}
	\begin{bmatrix}
		1 & 0 & 0 \\
		0 & -1 & 0 \\
		0 & 0 & -1
	\end{bmatrix}.
\end{equation*}
\end{example}

\begin{example}
举例说明:非零矩阵的乘积可能是零矩阵.
\begin{solution}
矩阵\begin{equation*}
	\vb{A} = \begin{bmatrix}
		0 & 0 & 0 \\
		a_{21} & 0 & 0 \\
		a_{31} & a_{32} & 0 \\
	\end{bmatrix}
	\quad\text{和}\quad
	\vb{B} = \begin{bmatrix}
		b_{11} & b_{12} & b_{13} \\
		0 & b_{22} & b_{23} \\
		0 & 0 & b_{33} \\
	\end{bmatrix}
\end{equation*}可以都不是零矩阵,
但他们的乘积\(\vb{A}\vb{B}\)一定是零矩阵.
\end{solution}
\end{example}

\begin{example}
举例说明:矩阵乘法不满足消去律,即\begin{equation*}
	\vb{A} \vb{B} = \vb{A} \vb{C}
	\notimplies
	\vb{A} = \vb0 \lor \vb{B} = \vb{C}.
\end{equation*}
\begin{solution}
取\begin{equation*}
	\vb{A} = \begin{bmatrix}
		1 & 0 \\
		1 & 0
	\end{bmatrix},
	\qquad
	\vb{B} = \begin{bmatrix}
		0 & 0 \\
		0 & 1
	\end{bmatrix},
	\qquad
	\vb{C} = \begin{bmatrix}
		0 & 0 \\
		0 & 2
	\end{bmatrix},
\end{equation*}
显然\begin{equation*}
	\vb{A}\vb{B}
	= \vb{A}\vb{C}
	= \begin{bmatrix}
		0 & 0 \\
		0 & 0
	\end{bmatrix},
\end{equation*}
但是\(\vb{A}\neq\vb0\)且\(\vb{B}\neq\vb{C}\).
\end{solution}
\end{example}

\subsection{矩阵乘法的运算规则}
\begin{theorem}
矩阵乘法满足结合律.
\begin{proof}
设\(\vb{A} = (a_{ij})_{s \times n},
\vb{B} = (b_{ij})_{n \times m},
\vb{C} = (c_{ij})_{m \times r}\).
显然\((\vb{A}\vb{B})\vb{C}\)与\(\vb{A}(\vb{B}\vb{C})\)同型,都是\(s \times r\)矩阵.
由于\begin{align*}
	\MatrixEntry{((\vb{A}\vb{B})\vb{C})}{i,j}
	&= \sum_{l=1}^m (\MatrixEntry{(\vb{A}\vb{B})}{i,l}) \cdot c_{lj} \\
	&= \sum_{l=1}^m \left( \sum_{k=1}^n a_{ik} b_{kl} \right) c_{lj} \\
	&= \sum_{l=1}^m \left( \sum_{k=1}^n a_{ik} b_{kl} c_{lj} \right), \\
	\MatrixEntry{(\vb{A}(\vb{B}\vb{C}))}{i,j}
	&= \sum_{k=1}^n a_{ik} \cdot (\MatrixEntry{(\vb{B}\vb{C})}{k,j}) \\
	&= \sum_{k=1}^n a_{ik} \left( \sum_{l=1}^m b_{kl} c_{lj} \right) \\
	&= \sum_{k=1}^n \left( \sum_{l=1}^m a_{ik} b_{kl} c_{lj} \right) \\
	&= \sum_{l=1}^m \left( \sum_{k=1}^n a_{ik} b_{kl} c_{lj} \right),
\end{align*}
于是\((\vb{A}\vb{B})\vb{C} = \vb{A}(\vb{B}\vb{C})\).
\end{proof}
\end{theorem}

\begin{definition}
设\(\vb{A}\in M_n(K)\).
若有\begin{equation*}
	\vb{A}(i,j) = \left\{ \begin{array}{cl}
		1, & i=j, \\
		0, & i\neq j,
	\end{array} \right.
\end{equation*}
则称“\(\vb{A}\)是\DefineConcept{单位矩阵}(identity matrix)”,记作\(\vb{E}\).
%@see: https://mathworld.wolfram.com/IdentityMatrix.html
\end{definition}

\begin{property}
矩阵的乘法满足以下性质:
\begin{gather}
	(\forall\vb{A},\vb{B}\in M_{s\times n}(K))(\forall k\in K)[k(\vb{A}\vb{B}) = (k\vb{A})\vb{B} = \vb{A}(k\vb{B})], \\
	(\forall\vb{A},\vb{B},\vb{C}\in M_{s\times n}(K))[\vb{A}(\vb{B}+\vb{C}) = \vb{A}\vb{B} + \vb{A}\vb{C}], \label{equation:矩阵的乘法.左分配律} \\
	(\forall\vb{A},\vb{B},\vb{C}\in M_{s\times n}(K))[(\vb{A}+\vb{B})\vb{C} = \vb{A}\vb{C} + \vb{B}\vb{C}], \label{equation:矩阵的乘法.右分配律} \\
	(\forall\vb{A}\in M_{s\times n}(K))[\vb0_{q \times s} \vb{A} = \vb0_{q \times n}], \\
	(\forall\vb{A}\in M_{s\times n}(K))[\vb{A} \vb0_{n \times p} = \vb0_{s \times p}], \\
	(\forall\vb{A}\in M_{s\times n}(K))[\vb{E}_s \vb{A} = \vb{A}], \\
	(\forall\vb{A}\in M_{s\times n}(K))[\vb{A} \vb{E}_n = \vb{A}].
\end{gather}
\end{property}

\subsection{矩阵的幂}
\begin{definition}
设\(\vb{A}\in M_n(K)\).
定义:
\begin{align}
	\vb{A}^0 &\defeq \vb{E}, \\
	\vb{A}^k &\defeq \underbrace{\vb{A}\vb{A}\dotsm\vb{A}}_{\text{$k$个}}.
\end{align}
%@see: https://mathworld.wolfram.com/MatrixPower.html
\end{definition}

\begin{theorem}
指数律成立,
即\begin{gather}
	(\forall\vb{A} \in M_n(K))
	(k,l \in \mathbb{N})
	[\vb{A}^k\vb{A}^l = \vb{A}^{k+l}], \\
	(\forall\vb{A} \in M_n(K))
	(k,l \in \mathbb{N})
	[(\vb{A}^k)^l = \vb{A}^{kl}].
\end{gather}
\end{theorem}

\begin{proposition}
设\(\vb{A},\vb{B} \in M_n(K)\),
则\begin{equation}
	(\vb{A}\vb{B})^k = \vb{A}(\vb{B}\vb{A})^{k-1}\vb{B},
	\quad k=1,2,\dotsc.
\end{equation}
\end{proposition}
\begin{proposition}
设\(\vb{A},\vb{B} \in M_n(K)\).
若\(\vb{A}\)与\(\vb{B}\)可交换,则\begin{equation*}
	(\vb{A}\vb{B})^k = \vb{A}^k\vb{B}^k.
\end{equation*}
\end{proposition}
\begin{remark}
注意,当\(\vb{A}\)、\(\vb{B}\)不可交换时,通常有\begin{equation*}
	(\vb{A}\vb{B})^k \neq \vb{A}^k\vb{B}^k.
\end{equation*}
\end{remark}

\begin{example}
设\(\vb{A}=\diag(\AutoTuple{a}{n}),
\vb{B}=\diag(\AutoTuple{b}{n})\).
那么\begin{equation*}
	\vb{A}\vb{B} = \diag(a_1b_1,a_2b_2,\dotsc,a_nb_n).
\end{equation*}
\end{example}
\begin{remark}
如果\(\vb{A}\)是一个对角阵,其主对角线上的元素各不相同,
则与\(\vb{A}\)可交换的矩阵必定也是一个对角阵.
\end{remark}

\begin{example}
设二阶矩阵\(\vb{A}=\begin{bmatrix} 1 & \lambda \\ 0 & 1 \end{bmatrix}\).
试证\(\vb{A}^n=\begin{bmatrix} 1 & n\lambda \\ 0 & 1 \end{bmatrix}\).
\begin{proof}
用数学归纳法.
显然\(n=1\)时,命题成立.
接下来我们再验证\(n=2\)时,命题是否成立.
因为\begin{equation*}
	\vb{A}^2
	= \begin{bmatrix}
		1 & \lambda \\
		0 & 1
	\end{bmatrix}^2
	= \begin{bmatrix}
		1\cdot0+\lambda\cdot0 & 1\cdot\lambda+\lambda\cdot1 \\
		0\cdot1+1\cdot0 & 0\cdot\lambda+1\cdot1
	\end{bmatrix}
	= \begin{bmatrix}
		1 & 2\lambda \\
		0 & 1
	\end{bmatrix},
\end{equation*}
于是\(n=2\)时命题成立.

假设\(n=k\ (k\geq1)\)时命题成立,
那么,当\(n=k+1\)时,有\begin{equation*}
	A^{k+1}
	= A A^k
	= \begin{bmatrix}
		1 & \lambda \\
		0 & 1
	\end{bmatrix}
	\begin{bmatrix}
		1 & k\lambda \\
		0 & 1
	\end{bmatrix}
	= \begin{bmatrix}
		1\cdot1+\lambda\cdot0 & 1\cdot k\lambda+\lambda\cdot1 \\
		0\cdot1+1\cdot0 & 0\cdot k\lambda+1\cdot1
	\end{bmatrix}
	= \begin{bmatrix}
		1 & (k+1)\lambda \\
		0 & 1
	\end{bmatrix}.
\end{equation*}
因此,命题\(\vb{A}^n=\begin{bmatrix} 1 & n\lambda \\ 0 & 1 \end{bmatrix}\)
当\(n=1,2,\dotsc\)时总成立.
\end{proof}
\end{example}

\begin{example}
设\(\vb{A},\vb{B},\vb{C} \in M_n(K)\),则有\begin{equation*}
	(\vb{A} + \vb{B} + \vb{C})^2
	= \vb{A}^2 + \vb{B}^2 + \vb{C}^2 + \vb{A}\vb{B} + \vb{B}\vb{A} + \vb{A}\vb{C} + \vb{C}\vb{A} + \vb{B}\vb{C} + \vb{C}\vb{B}.
\end{equation*}
\end{example}

\begin{theorem}
设矩阵\(\vb{A} \in M_n(K)\),
则\begin{itemize}
	\item 对于任意非负整数\(k\),\(\vb{A}\)与\(\vb{A}^k\)可交换;
	\item 对于数域\(K\)上的任意一个一元多项式\(f(x)\),\(\vb{A}\)与\(f(\vb{A})\)可交换.
	\item 对于数域\(K\)上的任意两个一元多项式\(f(x)\)和\(g(x)\),\(f(\vb{A})\)与\(g(\vb{A})\)可交换.
\end{itemize}
\end{theorem}

\begin{theorem}
如果\(g(x)\)和\(h(x)\)是两个多项式,
设\(l(x) = g(x) + h(x)\),\(m(x) = g(x) h(x)\),
则\begin{equation*}
	l(\vb{A}) = g(\vb{A}) + h(\vb{A}),
	\quad
	m(\vb{A}) = g(\vb{A}) h(\vb{A}).
\end{equation*}
\end{theorem}

\begin{example}
设\begin{equation*}
	\vb{A} = \begin{bmatrix}
	\cos t & \sin t \\
	-\sin t & \cos t
	\end{bmatrix}.
\end{equation*}
令\begin{equation*}
	\vb{B} = \begin{bmatrix}
		\cos t & 0 \\
		0 & \cos t
	\end{bmatrix},
	\qquad
	\vb{C} = \begin{bmatrix}
		0 & \sin t \\
		-\sin t & 0
	\end{bmatrix},
\end{equation*}
则\(\vb{A}=\vb{B}+\vb{C}\).
因为\begin{equation*}
	\vb{B}\vb{C} = \begin{bmatrix}
		\cos t & 0 \\
		0 & \cos t
	\end{bmatrix}
	\begin{bmatrix}
		0 & \sin t \\
		-\sin t & 0
	\end{bmatrix}
	= \begin{bmatrix}
		0 & \cos t \sin t \\
		-\cos t \sin t & 0
	\end{bmatrix},
\end{equation*}\begin{equation*}
	\vb{C}\vb{B} = \begin{bmatrix}
		0 & \sin t \\
		-\sin t & 0
	\end{bmatrix}
	\begin{bmatrix}
		\cos t & 0 \\
		0 & \cos t
	\end{bmatrix}
	= \begin{bmatrix}
		0 & \cos t \sin t \\
		-\cos t \sin t & 0
	\end{bmatrix},
\end{equation*}
所以\(\vb{B}\vb{C}=\vb{C}\vb{B}\),\(\vb{B}\)与\(\vb{C}\)可交换.
由牛顿二项式定理有,\begin{equation*}
	\vb{A}^n=(\vb{B}+\vb{C})^n
	=\sum_{k=0}^n C_n^k \vb{B}^{n-k} \vb{C}^k.
\end{equation*}
\end{example}

\begin{example}
设\(\vb{A},\vb{B},\vb{x} \in M_n(K)\).
证明:若\(\vb{A}\vb{x}=\vb{x}\vb{B}\),则对任意多项式\begin{equation*}
	f(x) = a_0 + a_1 x + a_2 x^2 + \dotsb + a_k x^k,
	\quad
	a_0,a_1,a_2,\dotsc,a_k \in K,
\end{equation*}
总有\begin{equation*}
	f(\vb{A}) \vb{x} = \vb{x} f(\vb{B}).
\end{equation*}
\begin{proof}
因为\begin{equation*}
	\vb{A}\vb{x} = \vb{x}\vb{B},
\end{equation*}
所以\begin{equation*}
	\vb{A}^2 \vb{x} = \vb{A}(\vb{A}\vb{x}) = \vb{A}(\vb{x}\vb{B}),
	\qquad
	\vb{x} \vb{B}^2 = (\vb{x}\vb{B})\vb{B} = (\vb{A}\vb{x})\vb{B}.
\end{equation*}
以此类推,可证\begin{equation*}
	\vb{A}^n \vb{x} = \vb{x} \vb{B}^n,
	\quad n=1,2,\dotsc.
\end{equation*}

因为\begin{equation*}
	f(\vb{A}) = a_0 \vb{E} + a_1 \vb{A} + a_2 \vb{A}^2 + \dotsb + a_k \vb{A}^k,
\end{equation*}
根据左分配律,有\begin{equation*}
	f(\vb{A}) \vb{x} = a_0 \vb{x} + a_1 \vb{A}\vb{x} + a_2 \vb{A}^2 \vb{x} + \dotsb + a_k \vb{A}^k \vb{x}.
	\eqno(1)
\end{equation*}
同理,根据右分配律,有\begin{equation*}
	\vb{x} f(\vb{B}) = a_0 \vb{x} + a_1 \vb{x}\vb{B} + a_2 \vb{x} \vb{B}^2 + \dotsb + a_k \vb{x} \vb{B}^k.
	\eqno(2)
\end{equation*}
因为(1)与(2)中各项逐项相等,
故\(f(\vb{A}) \vb{x} = \vb{x} f(\vb{B})\).
\end{proof}
\end{example}

\subsection{矩阵乘积的转置}
\begin{theorem}\label{theorem:矩阵.矩阵乘积的转置}
设\(\vb{A}\in M_{s\times n}(K),
\vb{B}\in M_{n \times t}(K)\),
则有\begin{equation*}
	(\vb{A}\vb{B})^T = \vb{B}^T \vb{A}^T.
\end{equation*}
\begin{proof}
假设\begin{equation*}
	\vb{A}=(a_{ij})_{s \times n}
	=(\AutoTuple{\vb\alpha}{s})^T, \qquad
	\vb{B}=(b_{ij})_{n \times t}
	=(\AutoTuple{\vb\beta}{t}),
\end{equation*}
其中\(\vb\alpha_i\in K^n\ (i=1,2,\dotsc,s)\)是行向量,
\(\vb\beta_j\in K^n\ (j=1,2,\dotsc,t)\)是列向量.
又假设\begin{equation*}
	\vb{A}\vb{B}=(c_{ij})_{s \times t}, \qquad
	\vb{B}^T\vb{A}^T=(d_{ij})_{t \times s}.
\end{equation*}
那么\begin{gather*}
	c_{ij}  % \(\vb{A}\)的第\(i\)行,\(\vb{B}\)的第\(j\)列
	= \VectorInnerProductDot{\vb\alpha_i}{\vb\beta_j}
	= \sum_{k=1}^n a_{ik}b_{kj}, \\
	d_{ij}  % \(\vb{B}^T\)的第\(i\)行,\(\vb{A}^T\)的第\(j\)列
	= \VectorInnerProductDot{\vb\beta_i}{\vb\alpha_j}  % 相当于\(\vb{B}\)的第\(i\)列,\(\vb{A}\)的第\(j\)行
	= \sum_{k=1}^n a_{jk}b_{ki},
\end{gather*}
可见\(c_{ij}=d_{ji}\ (i=1,2,\dotsc,s;j=1,2,\dotsc,t)\).
因此,\((\vb{A}\vb{B})^T = \vb{B}^T \vb{A}^T\).
\end{proof}
\end{theorem}

\begin{corollary}
\((\vb{A}_1 \vb{A}_2 \dotsm \vb{A}_n)^T = \vb{A}_n^T \dotsm \vb{A}_2^T \vb{A}_1^T\).
\end{corollary}
