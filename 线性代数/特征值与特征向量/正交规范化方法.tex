\section{正交规范化方法}
在许多实际问题中,我们需要构造正交矩阵,于是我们要设法求标准正交基.

\begin{figure}[htb]
%@see: 《高等代数(第三版 上册)》(丘维声) P149 图4-2
	\centering
	\begin{tikzpicture}[scale=3,>=Stealth,->]
		\draw(0,0)--(2,1)node[right]{\(\vb\alpha_2\)};
		\draw(0,0)--(0,1)node[left]{\(\vb\beta_2\)};
		\draw(2,1)--(0,1)node[midway,above]{\(k\vb\alpha_1\)};
		\draw(0,0)--(1,0)node[below]{\(\vb\alpha_1=\vb\beta_1\)};
	\end{tikzpicture}
	\caption{}
	\label{figure:正交规范化方法.二维几何空间中两不共线向量的正交规范化}
\end{figure}

平面上给定两个不共线的向量\(\vb\alpha_1,\vb\alpha_2\),
如\cref{figure:正交规范化方法.二维几何空间中两不共线向量的正交规范化},
我们很容易找到一个正交向量组:\begin{align*}
	\vb\beta_1 &= \vb\alpha_1, \\
	\vb\beta_2 &= \vb\alpha_2 + k \vb\beta_1,
\end{align*}
其中\(k\)是待定系数.

为了求出待定系数\(k\),
在上式两边用\(\vb\beta_1\)作内积,得\[
	\vb\beta_2\cdot\vb\beta_1
	= (\vb\alpha_2 + k~\vb\beta_1)\cdot\vb\beta_1.
\]
从而\[
	0 = \vb\alpha_2\cdot\vb\beta_1 + k~\vb\beta_1\cdot\vb\beta_1.
\]
因此\[
%@see: 《高等代数(第三版 上册)》(丘维声) P149 (14)
	k = -\frac{\vb\alpha_2\cdot\vb\beta_1}{\vb\beta_1\cdot\vb\beta_1}.
\]
于是\[
	\vb\beta_2
	= \vb\alpha_2
	- \frac{\vb\alpha_2\cdot\vb\beta_1}{\vb\beta_1\cdot\vb\beta_1}~\vb\beta_1.
\]

从几何上的这个例子受到启发,
对于欧几里得空间\(\mathbb{R}^n\),
我们可以从一个线性无关的向量组出发,
构造一个正交向量组.

\begin{theorem}
%@see: 《高等代数(第三版 上册)》(丘维声) P149 定理4
设\(\vb\alpha_1,\vb\alpha_2,\dotsc,\vb\alpha_m\)是\(\mathbb{R}^n\)中的一个线性无关组,
令\begin{align*}
	\vb\beta_1 &= \vb\alpha_1, \\
	\vb\beta_k &= \vb\alpha_k - \sum_{i=1}^{k-1}
		\frac{\VectorInnerProductDot{\vb\alpha_k}{\vb\beta_i}}{\VectorInnerProductDot{\vb\beta_i}{\vb\beta_i}} \vb\beta_i,
	\quad k=2,3,\dotsc,m,
\end{align*}
则\(\AutoTuple{\vb\beta}{m}\)是一个正交向量组,
且\[
	\{\AutoTuple{\vb\alpha}{j}\}
	\cong
	\{\AutoTuple{\vb\beta}{j}\},
	\quad j=1,2,\dotsc,m.
\]
再将之单位化(或规范化)得\[
	\g_k = \frac1{\abs{\vb\beta_k}} \vb\beta_k,
	\quad k=1,2,\dotsc,m,
\]
则\(\AutoTuple{\vb\gamma}{m}\)是一个规范正交组,且满足\[
	\{\AutoTuple{\vb\alpha}{j}\}
	\cong
	\{\AutoTuple{\vb\gamma}{j}\},
	\quad j=1,2,\dotsc,m.
\]
\end{theorem}
