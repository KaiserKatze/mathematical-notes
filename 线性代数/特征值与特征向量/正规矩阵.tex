\section{正规矩阵}
\begin{definition}
若矩阵\(\vb{A} \in M_n(\mathbb{C})\)及其共轭转置\(\vb{A}^H\)满足\begin{equation*}
	\vb{A} \vb{A}^H = \vb{A}^H \vb{A},
\end{equation*}
则称“\(\vb{A}\)是一个\DefineConcept{正规矩阵}(normal matrix)”.
%@see: https://mathworld.wolfram.com/NormalMatrix.html
\end{definition}

\begin{example}
矩阵\(
	\vb{A}
	\defeq
	\begin{bmatrix}
		1 & -1 \\
		1 & 1 \\
	\end{bmatrix}
\)
就是一个正规矩阵.
\end{example}

\begin{property}
实正交矩阵、实对称矩阵、实反对称矩阵都是实正规矩阵.
%TODO proof
\end{property}

\begin{property}
对角矩阵、厄米矩阵、反厄米矩阵、
酉矩阵都是正规矩阵.
%TODO proof
\end{property}

\begin{property}
%@see: 《矩阵分析(第3版)》(史荣昌、魏丰) P118 引理3.6.1
设\(\vb{A} \in M_n(\mathbb{C})\)是正规矩阵,
则每一个与\(\vb{A}\)酉相似的矩阵都是一个正规矩阵.
\end{property}
% 正规矩阵的酉相似类

\begin{property}
%@see: 《矩阵分析(第3版)》(史荣昌、魏丰) P118 引理3.6.2
设\(\vb{A} \in M_n(\mathbb{C})\)是正规矩阵.
如果\(\vb{A}\)是三角矩阵,
则\(\vb{A}\)是对角矩阵.
\end{property}

\begin{theorem}%[正规矩阵的结构定理]
%@see: 《矩阵分析(第3版)》(史荣昌、魏丰) P119 定理3.6.3
设矩阵\(\vb{A} \in M_n(\mathbb{C})\),
则\(\vb{A}\)是正规矩阵,
当且仅当
存在酉矩阵\(\vb{U} \in M_n(\mathbb{C})\),
使得\begin{equation*}
	\vb{U}^H \vb{A} \vb{U} = \diag(\lambda_1,\dotsc,\lambda_n),
\end{equation*}
其中\(\lambda_1,\dotsc,\lambda_n\)是\(\vb{A}\)的特征值.
\end{theorem}
% 一个矩阵是正规矩阵当且仅当它可以酉相似对角化

\begin{corollary}
%@see: 《矩阵分析(第3版)》(史荣昌、魏丰) P119 推论3.6.1
设\(\vb{A} \in M_n(\mathbb{C})\)是正规矩阵,
\(\lambda\)是\(\vb{A}\)的一个特征值,
\(\vb{x}\)是\(\vb{A}\)的属于\(\lambda\)的一个特征向量,
则\(\ComplexConjugate{\lambda}\)是\(\vb{A}^H\)的特征值,
\(\vb{x}\)是\(\vb{A}^H\)的属于\(\ComplexConjugate{\lambda}\)的一个特征向量.
\end{corollary}

\begin{corollary}
%@see: 《矩阵分析(第3版)》(史荣昌、魏丰) P119 推论3.6.2
设\(\vb{A} \in M_n(\mathbb{C})\)是正规矩阵,
则\(\vb{A}\)有\(n\)个线性无关的特征向量.
% 因为酉矩阵\(\vb{P}\)的列向量组是一个正交向量组
\end{corollary}
%TODO 正规矩阵的属于不同特征值的特征向量正交吗?
%\cref{theorem:特征值与特征向量.实对称矩阵2}

\begin{corollary}
%@see: 《矩阵分析(第3版)》(史荣昌、魏丰) P119 推论3.6.3
设\(\vb{A} \in M_n(\mathbb{C})\)是正规矩阵,
则\(\vb{A}\)的属于不同特征值的特征子空间是互相正交的.
\end{corollary}

\begin{theorem}\label{theorem:正规矩阵.厄米矩阵的特征值}
%@see: 《矩阵分析(第3版)》(史荣昌、魏丰) P122 定理3.6.4
设\(\vb{A} \in M_n(\mathbb{C})\)是正规矩阵,
则\(\vb{A}\)是厄米矩阵,当且仅当它的特征值都是实数.
\end{theorem}

\begin{theorem}\label{theorem:正规矩阵.反厄米矩阵的特征值}
%@see: 《矩阵分析(第3版)》(史荣昌、魏丰) P122 定理3.6.4
设\(\vb{A} \in M_n(\mathbb{C})\)是正规矩阵,
则\(\vb{A}\)是反厄米矩阵,当且仅当它的特征值都是纯虚数.
\end{theorem}

\begin{theorem}\label{theorem:正规矩阵.酉矩阵的特征值}
%@see: 《矩阵分析(第3版)》(史荣昌、魏丰) P122 定理3.6.4
设\(\vb{A} \in M_n(\mathbb{C})\)是正规矩阵,
则\(\vb{A}\)是酉矩阵,当且仅当它的特征值都是绝对值等于\(1\)的复数.
\end{theorem}

\begin{example}
%@see: 《矩阵分析(第3版)》(史荣昌、魏丰) P122 例3.6.4
设矩阵\(\vb{A} \in M_n(\mathbb{C})\)满足\(\vb{A}^H = \vb{A}\),
且存在自然数\(k\)使得\(\vb{A}^k = \vb0\).
证明:\(\vb{A} = \vb0\).
\end{example}

\begin{example}
%@see: 《矩阵分析(第3版)》(史荣昌、魏丰) P123 例3.6.5
设\(\vb{U} \in M_n(\mathbb{C})\)是一个\(n\)阶酉矩阵,
且\(\vb{U} - \vb{E}\)可逆.
证明:\(\vb{A} \defeq (\vb{U} - \vb{E})^{-1} (\vb{U} + \vb{E})\)是反厄米矩阵.
\end{example}

\begin{theorem}
%@see: 《矩阵分析(第3版)》(史荣昌、魏丰) P123 定理3.6.5
%@see: 《高等代数(第三版 上册)》(丘维声) P202 习题6.1 13.
设\(\vb{A}, \vb{B} \in M_n(\mathbb{C})\)都是正规矩阵,
则\(\vb{A}, \vb{B}\)可以同时酉对角化
(即存在\(n\)阶酉矩阵\(\vb{U} \in M_n(\mathbb{C})\),
存在\(n\)阶对角矩阵\(\vb\Lambda_1, \vb\Lambda_2 \in M_n(\mathbb{C})\),
使得\(
	\vb{U}^H \vb{A} \vb{U} = \vb\Lambda_1,
	\vb{U}^H \vb{B} \vb{U} = \vb\Lambda_2
\))
当且仅当
\(\vb{A}\)与\(\vb{B}\)可交换
(即\(\vb{A} \vb{B} = \vb{B} \vb{A}\)).
\end{theorem}
