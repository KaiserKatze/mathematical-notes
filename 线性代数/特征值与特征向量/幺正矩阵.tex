\section{幺正矩阵}
\begin{definition}\label{definition:幺正矩阵.幺正矩阵的定义}
%@see: 《高等代数(第三版 下册)》(丘维声) P196 定义5
设\(\vb{Q} \in M_n(\mathbb{C})\),
\(\vb{E}\)是复数域上的\(n\)阶单位矩阵.
若\(\vb{Q}\)满足\begin{equation}\label{equation:幺正矩阵.幺正矩阵的定义式}
	\vb{Q}^H \vb{Q} = \vb{E},
\end{equation}
则称“\(\vb{Q}\)是\(n\)阶\DefineConcept{幺正矩阵}”
或“\(Q\)是\(n\)阶\DefineConcept{酉矩阵}(unitary matrix)”.
%@see: https://mathworld.wolfram.com/UnitaryMatrix.html
\end{definition}

\begin{property}
设\(\vb{Q} \in M_n(\mathbb{C})\),
则\begin{itemize}
	\item \(\vb{A}\)的行列式\(\det\vb{A}\)的模等于\(1\),
	即\begin{equation}\label{equation:幺正矩阵.幺正矩阵的行列式}
		\abs{\det\vb{A}}=1.
	\end{equation}

	\item \(\vb{A}\)可逆.

	\item \(\vb{Q}\)的共轭转置矩阵\(\vb{Q}^H\)和它的逆矩阵\(\vb{Q}^{-1}\)
	满足\begin{equation}\label{equation:幺正矩阵.幺正矩阵的转置等于幺正矩阵的逆}
		\vb{Q}^H = -\vb{Q}^{-1}.
	\end{equation}
\end{itemize}
%TODO proof
\end{property}

\begin{example}
%@see: 《线性代数》(张慎语、周厚隆) P114 第五章综合判断题 (14)
举例说明:元素全是实数的幺正矩阵的特征值不是\(\pm1\).
\begin{solution}
取\(\vb{A}
= \begin{bmatrix}
	0 & 1 \\
	-1 & 0
\end{bmatrix} \in M_n(\mathbb{C})\),
显然\(\vb{A}^T \vb{A}
= \begin{bmatrix}
	0 & -1 \\
	1 & 0
\end{bmatrix}
\begin{bmatrix}
	0 & 1 \\
	-1 & 0
\end{bmatrix}
= \begin{bmatrix}
	1 & 0 \\
	0 & 1
\end{bmatrix}\),
但是由\begin{equation*}
	\abs{\lambda\vb{E}-\vb{A}}
	= \begin{vmatrix}
		\lambda & -1 \\
		1 & \lambda
	\end{vmatrix}
	= \lambda^2 + 1
	= 0
\end{equation*}解得\(\lambda=\pm\iu\).
\end{solution}
\end{example}
\begin{remark}
从上面这个例子可以看出:在实数域上,正交矩阵可能没有特征值.
\end{remark}
\begin{example}
%@see: 《线性代数》(张慎语、周厚隆) P114 第五章综合判断题 (15)
%@see: 《高等代数(第三版 上册)》(丘维声) P180 习题5.5 10.(1)
设\(\vb{A}\)是幺正矩阵.
证明:\(\vb{A}\)的特征值是模为\(1\)的复数.
\begin{proof}
设\(\lambda\)是幺正矩阵\(\vb{A}\)的一个特征值,
\(\vb\alpha\)是\(\vb{A}\)的属于特征值\(\lambda\)的一个特征向量,
即成立\begin{equation*}
	\vb{A} \vb\alpha
	= \lambda \vb\alpha.
	\eqno(1)
\end{equation*}
对(1)式取转置得\begin{equation*}
	\vb\alpha^T \vb{A}^T
	= \lambda \vb\alpha^T.
	\eqno(2)
\end{equation*}
将(1)(2)两式相乘,得\begin{equation*}
	(\vb\alpha^T \vb{A}^T) (\vb{A} \vb\alpha)
	= (\lambda \vb\alpha^T) (\lambda \vb\alpha),
\end{equation*}
再利用矩阵乘法的结合律和幺正矩阵的定义,得\begin{equation*}
	\vb\alpha^T \vb\alpha
	= \lambda^2 \vb\alpha^T \vb\alpha,
\end{equation*}
移项得\begin{equation*}
	(1-\lambda^2) \vb\alpha^T \vb\alpha = 0.
	\eqno(3)
\end{equation*}
因为\(\vb\alpha\neq\vb0\),
所以\(\vb\alpha^T \vb\alpha \neq 0\),
那么(3)式可以化为\begin{equation*}
	1-\lambda^2 = 0,
	\quad\text{或}\quad
	\lambda^2 = 1,
\end{equation*}
因此\(\vb{A}\)的特征值只要存在,就一定是模为\(1\)的复数.
\end{proof}
\end{example}
\begin{remark}
从上面这个例子可以看出:在实数域上,正交矩阵的特征值只要存在就只能是\(\pm1\).
\end{remark}
