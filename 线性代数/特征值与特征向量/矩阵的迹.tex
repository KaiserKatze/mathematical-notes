\section{矩阵的迹}
\begin{definition}
%@see: 《解析几何》(丘维声) P160 定义2.2
矩阵\(\vb{A}=(a_{ij})_{s \times n}\)
主对角线上元素之和称为“\(\vb{A}\)的\DefineConcept{迹}(trace)”,
%@see: https://mathworld.wolfram.com/MatrixTrace.html
记作\(\tr\vb{A}\),
即\begin{equation*}
%@see: 《解析几何》(丘维声) P160 (2.5)
	\tr\vb{A}
	\defeq
	\sum_{i=1}^m a_{ii},
\end{equation*}
其中\(m = \min\{s,n\}\).
\end{definition}

\begin{property}\label{theorem:矩阵的迹.性质1}
%@see: 《解析几何》(丘维声) P160 性质2.1
已知矩阵\(\vb{A},\vb{B} \in M_{s \times n}(K)\),
则\begin{gather}
	%@see: 《高等代数(第三版 上册)》(丘维声) P170 (2)
	\tr(\vb{A}+\vb{B}) = \tr\vb{A} + \tr\vb{B}, \\
	%@see: 《高等代数(第三版 上册)》(丘维声) P170 (3)
	(\forall k \in K)[\tr(k \vb{A}) = k \tr\vb{A}].
\end{gather}
\begin{proof}
设\(\vb{A}=(a_{ij})_{s \times n},
\vb{B}=(b_{ij})_{s \times n}\),
取\(m = \min\{s,n\}\),
那么\begin{equation*}
	\tr(\vb{A}+\vb{B}) = \sum_{i=1}^m (a_{ii}+b_{ii})
	= \sum_{i=1}^m a_{ii}
	+ \sum_{i=1}^m b_{ii}
	= \tr\vb{A} + \tr\vb{B},
\end{equation*}\begin{equation*}
	\tr(k \vb{A}) = \sum_{i=1}^m (k a_{ii})
	= k \sum_{i=1}^m a_{ii}
	= k \tr\vb{A}.
	\qedhere
\end{equation*}
\end{proof}
\end{property}
\begin{remark}
\cref{theorem:矩阵的迹.性质1} 说明:
矩阵的迹具有“线性性”.
\end{remark}

\begin{property}\label{theorem:矩阵的迹.矩阵转置不变迹}
已知矩阵\(\vb{A} \in M_{s \times n}(K)\),
则\begin{equation}
	\tr\vb{A} = \tr(\vb{A}^T).
\end{equation}
%TODO proof
\end{property}

\begin{property}\label{theorem:矩阵的迹.矩阵乘积交换次序不变迹}
%@see: 《高等代数(第三版 上册)》(丘维声) P170 (4)
%@see: 《解析几何》(丘维声) P160 性质2.2
已知矩阵\(\vb{A},\vb{B} \in M_n(K)\),
则\begin{equation}
	\tr(\vb{A} \vb{B}) = \tr(\vb{B} \vb{A}).
\end{equation}
\begin{proof}
设\(\vb{A} = (a_{ij})_n,
\vb{B} = (b_{ij})_n\),
则\begin{gather*}
	\tr(\vb{A} \vb{B})
	= \sum_{i=1}^n (\vb{A} \vb{B})(i,i)
	= \sum_{i=1}^n \sum_{k=1}^n a_{ik} b_{ki}, \\
	\tr(\vb{B} \vb{A})
	= \sum_{k=1}^n (\vb{B} \vb{A})(k,k)
	= \sum_{k=1}^n \sum_{i=1}^n b_{ki} a_{ik},
\end{gather*}
利用加法结合律可得\begin{equation*}
	\sum_{i=1}^n \sum_{k=1}^n a_{ik} b_{ki}
	= \sum_{k=1}^n \sum_{i=1}^n b_{ki} a_{ik},
\end{equation*}
于是\(\tr(\vb{A} \vb{B}) = \tr(\vb{B} \vb{A})\).
\end{proof}
\end{property}

\begin{example}
举例说明:实矩阵\(\vb{A}\)满足\(\tr(\vb{A}^2)<0\).
\begin{solution}
取\(\vb{A} = \begin{bmatrix}
	0 & 1 \\
	-1 & 0
\end{bmatrix}\),
则\(\vb{A}^2 = \begin{bmatrix}
	-1 & 0 \\
	0 & -1
\end{bmatrix}\),
从而有\(\tr(\vb{A}^2) = -2 < 0\).
\end{solution}
%@Mathematica: Tr[MatrixPower[{{0, 1}, {-1, 0}}, 2]]
\end{example}

\begin{example}
%@see: 《高等代数(第三版 上册)》(丘维声) P171 习题5.4 9.
证明:如果数域\(K\)上的\(n\)阶矩阵\(\vb{A},\vb{B}\)满足\begin{equation*}
	\vb{A} \vb{B} - \vb{B} \vb{A}=\vb{A},
\end{equation*}
则\(\vb{A}\)不可逆.
\begin{proof}
用反证法.
假设\(\vb{A}\)可逆,\(\vb{E}\)是数域\(K\)上的\(n\)阶单位矩阵,
那么\begin{gather*}
	\vb{E} = \vb{A}\vb{A}^{-1}
	= (\vb{A} \vb{B} - \vb{B} \vb{A})\vb{A}^{-1} % 把\(\vb{A} \vb{B} - \vb{B} \vb{A}\)代入\(\vb{A}\)
	= \vb{A} \vb{B} \vb{A}^{-1}-\vb{B},
\end{gather*}
从而有\(\tr(\vb{A} \vb{B} \vb{A}^{-1}-\vb{B}) = \tr\vb{E}\),
但是\begin{align*}
	\tr(\vb{A} \vb{B} \vb{A}^{-1}-\vb{B})
	&= \tr(\vb{A} \vb{B} \vb{A}^{-1})-\tr\vb{B}
		\tag{\cref{theorem:矩阵的迹.性质1}} \\
	&= \tr(\vb{A}^{-1}(\vb{A} \vb{B}))-\tr\vb{B}
		\tag{\cref{theorem:矩阵的迹.矩阵乘积交换次序不变迹}} \\
	&= \tr\vb{B}-\tr\vb{B}
	= 0
	< n = \tr\vb{E},
\end{align*}
所以\(\vb{A}\)不可逆.
\end{proof}
\end{example}

\begin{property}
设\(\vb{A}\)是可逆矩阵,
则\begin{equation}
	\tr(\vb{A}^*) = \abs{\vb{A}}~\tr(\vb{A}^{-1}).
\end{equation}
\begin{proof}
由\cref{theorem:逆矩阵.逆矩阵的唯一性} 可知,
\(\vb{A}^* = \abs{\vb{A}}~\vb{A}^{-1}\).
于是由\cref{theorem:矩阵的迹.性质1} 可知\begin{equation*}
	\tr(\vb{A}^*) = \tr(\abs{\vb{A}}~\vb{A}^{-1}) = \abs{\vb{A}}~\tr(\vb{A}^{-1}).
	\qedhere
\end{equation*}
\end{proof}
\end{property}

\begin{property}
已知矩阵\(\vb{A} \in M_{s \times n}(K)\),
则\begin{equation}
	\tr(\vb{A}\vb{A}^T) = \tr(\vb{A}^T\vb{A}).
\end{equation}
\begin{proof}
在\cref{theorem:矩阵的迹.矩阵乘积交换次序不变迹} 中,用\(\vb{A}^T\)代\(\vb{B}\)便得.
\end{proof}
\end{property}

\begin{property}
已知矩阵\(\vb{A},\vb{B} \in M_n(K)\),
且\(\vb{A},\vb{B}\)均为实对称矩阵,
则\begin{equation}
	\tr(\vb{A} \vb{B})^2 \leq \tr(\vb{A}^2 \vb{B}^2).
\end{equation}
%TODO proof
% \begin{proof}
% %@see: https://www.bilibili.com/video/BV1s12RY9EMx/
% 因为\(\vb{A},\vb{B}\)均是实对称矩阵,
% 所以\begin{equation*}
% 	\vb{A} = \vb{A}^T,
% 	\qquad
% 	\vb{B} = \vb{B}^T.
% \end{equation*}
% 因为\hyperref[theorem:矩阵的迹.矩阵乘积交换次序不变迹]{矩阵乘积交换次序不变迹},
% 所以\begin{equation*}
% 	\tr(\vb{A}^2 \vb{B}^2)
% 	= \tr[(\vb{A} \vb{A} \vb{B}) \vb{B}]
% 	= \tr(\vb{B} \vb{A} \vb{A} \vb{B}).
% 	\qquad
% 	\tr(\vb{B} \vb{A})
% 	= \tr(\vb{A} \vb{B}).
% \end{equation*}
% % 因为\hyperref[theorem:矩阵的迹.矩阵转置不变迹]{矩阵转置不变迹},
% 所以\begin{equation*}
% \end{equation*}
% 由\hyperref[theorem:矩阵的迹.性质1]{迹的线性性}可知,
% 要证\(\tr(\vb{A} \vb{B})^2 \leq \tr(\vb{A}^2 \vb{B}^2)\),
% 即证\(\tr(\vb{A} \vb{A} \vb{B} \vb{B} - \vb{A} \vb{B} \vb{A} \vb{B}) \geq 0\)
% \end{proof}
\end{property}

\begin{example}
设\(\vb{A}\)是数域\(K\)上的\(n\)阶对称矩阵,
\(\vb{B}\)是数域\(K\)上的\(n\)阶反对称矩阵.
证明:\begin{equation}
	\tr(\vb{A} \vb{B}) = 0.
\end{equation}
\begin{proof}
设\(\vb{A}\vb{B}\)的\((i,j)\)元素为\(c_{ij}\),
\(\vb{B}\vb{A}\)的\((i,j)\)元素为\(d_{ij}\),
即\begin{equation*}
	c_{ij} = \sum_{k=1}^n a_{ik} b_{kj},
	\qquad
	d_{ij} = \sum_{k=1}^n b_{ik} a_{kj}.
\end{equation*}
那么由迹的定义有\begin{align*}
	\tr(\vb{A}\vb{B})
	&= \sum_{i=1}^n c_{ii}
	= \sum_{i=1}^n \sum_{j=1}^n a_{ij} b_{ji}, \\
	\tr(\vb{B}\vb{A})
	&= \sum_{i=1}^n d_{ii}
	= \sum_{i=1}^n \sum_{j=1}^n b_{ij} a_{ji}.
\end{align*}
相加得\begin{equation*}
	\tr(\vb{A}\vb{B}) + \tr(\vb{B}\vb{A})
	= \sum_{i=1}^n \sum_{j=1}^n (a_{ij} b_{ji} + b_{ij} a_{ji}).
	\eqno(1)
\end{equation*}
因为\(\vb{A}\)是对称矩阵,所以\(a_{ij} = a_{ji}\).
因为\(\vb{B}\)是反对称矩阵,所以\(b_{ij} = -b_{ji}\).
那么(1)式化为\begin{equation*}
	\tr(\vb{A}\vb{B}) + \tr(\vb{B}\vb{A})
	= \sum_{i=1}^n \sum_{j=1}^n (a_{ij} b_{ji} - b_{ji} a_{ij})
	= 0.
\end{equation*}
因为\(\tr(\vb{A}\vb{B}) = \tr(\vb{B}\vb{A})\),
所以\(\tr(\vb{A}\vb{B}) = 0\).
\end{proof}
\end{example}
