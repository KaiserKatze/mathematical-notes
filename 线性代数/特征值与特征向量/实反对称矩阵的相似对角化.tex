\section{实反对称矩阵的相似对角化}
\begin{theorem}
%@see: 《线性代数》(张慎语、周厚隆) P113 习题5.3 6.
实反对称矩阵的特征值为零或纯虚数.
\begin{proof}
设\(\vb{A} \in M_n(\mathbb{R})\)满足\(\vb{A}^T=-\vb{A}\).
又设\(\mathbb{C} \ni \lambda_0 = a_0 + \iu b_0\ (a_0,b_0 \in \mathbb{R})\)是\(\vb{A}\)的任意一个特征值,
\(\mathbb{C}^{n \times 1} \ni \vb{x}_0=(\AutoTuple{c}{n})^T \neq \vb0\)是
\(\vb{A}\)属于特征值\(\lambda_0\)的特征向量,
即\begin{gather}
\vb{A}\vb{x}_0 = \lambda_0\vb{x}_0, \tag1
\end{gather}
在(1)式两端左乘\(\ComplexConjugate{\vb{x}_0}^T\),
得\begin{gather}
	\ComplexConjugate{\vb{x}_0}^T \vb{A} \vb{x}_0
	= \lambda_0\ \ComplexConjugate{\vb{x}_0}^T \vb{x}_0, \tag2
\end{gather}
取共轭转置,得\begin{gather}
\ComplexConjugate{\vb{x}_0}^T \vb{A}^T \vb{x}_0
= \ComplexConjugate{\lambda_0}\ \ComplexConjugate{\vb{x}_0}^T \vb{x}_0, \tag3
\end{gather}
由于\(\vb{A}\)是实反对称矩阵,即\(\vb{A}^T = -\vb{A}\),所以\begin{gather}
	\ComplexConjugate{\vb{x}_0}^T \vb{A} \vb{x}_0
	= -\ComplexConjugate{\lambda_0}\ \ComplexConjugate{\vb{x}_0}^T \vb{x}_0, \tag4
\end{gather}
其中,\(\ComplexConjugate{\vb{x}_0}^T \vb{x}_0
= \ComplexConjugate{c_1}c_1 + \ComplexConjugate{c_2}c_2 + \dotsb + \ComplexConjugate{c_n}{c_n} > 0\).
由(2)式与(4)式,得\begin{equation*}
	\lambda_0\ \ComplexConjugate{\vb{x}_0}^T \vb{x}_0
	= -\ComplexConjugate{\lambda_0}\ \ComplexConjugate{\vb{x}_0}^T \vb{x}_0,
\end{equation*}\begin{equation*}
	\lambda_0 = -\ComplexConjugate{\lambda_0},
\end{equation*}\begin{equation*}
	a_0 + \iu b_0 = -(a_0 - \iu b_0) = -a_0 + \iu b_0,
\end{equation*}\begin{equation*}
	\Re \lambda_0 = a_0 = 0.
\end{equation*}
也就是说\(\lambda_0\)要么为零要么为纯虚数.
\end{proof}
\end{theorem}
