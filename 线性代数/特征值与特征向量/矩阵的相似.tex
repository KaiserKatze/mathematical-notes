\section{矩阵的相似}
有时候,我们会遇到这样的问题:
已知\(\vb{A}\)是数域\(K\)上的一个\(n\)阶方阵,求\(\vb{A}^m\).
这时候,如果存在数域\(K\)上的一个\(n\)阶可逆矩阵\(\vb{P}\),
使得\(\vb{P}^{-1}\vb{A}\vb{P} = \vb{B}\),并且\(\vb{B}^m\)容易计算,
那么我们就可以利用矩阵的乘法结合律得到以下结果:\begin{equation*}
	\vb{A}^m
	= (\vb{P}^{-1}\vb{B}\vb{P})^m
	= \underbrace{
			(\vb{P}^{-1}\vb{B}\vb{P})
			(\vb{P}^{-1}\vb{B}\vb{P})
			\dotsm
			(\vb{P}^{-1}\vb{B}\vb{P})
		}_{\text{$m$个}}
	= \vb{P}^{-1}\vb{B}^m\vb{P}.
\end{equation*}

\subsection{矩阵相似的概念}
\begin{definition}
%@see: 《线性代数》(张慎语、周厚隆) P97 定义3
%@see: 《高等代数(第三版 上册)》(丘维声) P169 定义1
设\(\vb{A}\)、\(\vb{B}\)是两个\(n\)阶矩阵.
若存在可逆矩阵\(\vb{P}\),
使得\begin{equation}\label{equation:特征值与特征向量.相似矩阵的定义}
	\vb{P}^{-1} \vb{A} \vb{P} = \vb{B}
\end{equation}
则称“\(\vb{A}\)与\(\vb{B}\)~\DefineConcept{相似}%
(\(\vb{A}\) is \emph{similar} to \(\vb{B}\))”,
%@see: https://mathworld.wolfram.com/SimilarMatrices.html
记作\(\vb{A}\sim\vb{B}\).
\end{definition}
\begin{example}
%@see: 《高等代数(第三版 上册)》(丘维声) P171 习题5.4 6.
证明:单位矩阵只与它本身相似.
\begin{proof}
设\(\vb{A} \in M_n(K)\),
\(\vb{E}\)是数域\(K\)上的\(n\)阶单位矩阵,
且\(\vb{E} \sim \vb{A}\).
根据矩阵相似的定义,
存在可逆\(\vb{P} \in M_n(K)\),
使得\(\vb{P}^{-1} \vb{E} \vb{P} = \vb{A}\).
又因为单位矩阵可以与任意同阶矩阵交换,
所以\(\vb{A}
= \vb{P}^{-1} \vb{E} \vb{P}
= \vb{P}^{-1} \vb{P} \vb{E}
= \vb{E} \vb{E}
= \vb{E}\).
\end{proof}
\end{example}

\begin{example}
%@see: https://www.bilibili.com/video/BV1dA29YeEz8/
设\begin{equation*}
	\vb{A} = \begin{bmatrix}
		1 & 2 & 0 \\
		0 & 1 & 3 \\
		0 & 0 & 1
	\end{bmatrix},
	\qquad
	\vb{B} = \begin{bmatrix}
		1 & -1 & -2 \\
		0 & 1 & 1 \\
		0 & 0 & 1
	\end{bmatrix}.
\end{equation*}
证明:\(\vb{A} \sim \vb{B}\).
\begin{solution}
要想证明\(\vb{A} \sim \vb{B}\),
我们需要找出满足等式\(\vb{P}^{-1} \vb{A} \vb{P} = \vb{B}\)的可逆矩阵\(\vb{P}\),
也就是要解矩阵方程\(\vb{A} \vb{P} = \vb{P} \vb{B}\).
将\(\vb{P}\)按列分块为\((\vb\alpha_1,\vb\alpha_2,\vb\alpha_3)\),
则有\((\vb{A} \vb\alpha_1,\vb{A} \vb\alpha_2,\vb{A} \vb\alpha_3) = (\vb\alpha_1,\vb\alpha_2,\vb\alpha_3) \vb{B}\),
即\begin{gather*}
	\vb{A} \vb\alpha_1 = \vb\alpha_1, \tag1 \\
	\vb{A} \vb\alpha_2 = -\vb\alpha_1 + \vb\alpha_2, \tag2 \\
	\vb{A} \vb\alpha_3 = -2\vb\alpha_1 + \vb\alpha_2 + a_3. \tag3
\end{gather*}

由(1)式有\begin{equation*}
	(\vb{A} - \vb{E}) \vb\alpha_1 = \vb0,
\end{equation*}
它的系数矩阵为\begin{equation*}
	\vb{A} - \vb{E}
	= \begin{bmatrix}
		0 & 2 & 0 \\
		0 & 0 & 3 \\
		0 & 0 & 0
	\end{bmatrix}
	\to \begin{bmatrix}
		0 & 1 & 0 \\
		0 & 0 & 1 \\
		0 & 0 & 0
	\end{bmatrix},
\end{equation*}
所以\(\vb\alpha_1 = \begin{bmatrix}
	k_1 \\
	0 \\
	0
\end{bmatrix}
\ (\text{$k_1$是任意常数})\).

由(2)式有\begin{equation*}
	(\vb{A} - \vb{E}) \vb\alpha_2 = -\vb\alpha_1,
\end{equation*}
它的增广矩阵为\begin{equation*}
	(\vb{A} - \vb{E},-\vb\alpha_1)
	= \begin{bmatrix}
		0 & 2 & 0 & -k_1 \\
		0 & 0 & 3 & 0 \\
		0 & 0 & 0 & 0
	\end{bmatrix},
\end{equation*}
所以\(\vb\alpha_2 = \begin{bmatrix}
	k_2 \\
	-\frac12 k_1 \\
	0
\end{bmatrix}
\ (\text{$k_2$是任意常数})\).

由(3)式有\begin{equation*}
	(\vb{A} - \vb{E}) \vb\alpha_3 = -2\vb\alpha_1 + \vb\alpha_2,
\end{equation*}
它的增广矩阵为\begin{equation*}
	(\vb{A} - \vb{E},-2\vb\alpha_1 + \vb\alpha_2)
	= \begin{bmatrix}
		0 & 2 & 0 & k_2 - 2 k_1 \\
		0 & 0 & 3 & -\frac12 k_1 \\
		0 & 0 & 0 & 0
	\end{bmatrix},
\end{equation*}
所以\(\vb\alpha_3 = \begin{bmatrix}
	k_3 \\
	\frac12 k_2 - k_1 \\
	-\frac16 k_1
\end{bmatrix}
\ (\text{$k_3$是任意常数})\).

于是\begin{equation*}
	\vb{P} = (\vb\alpha_1,\vb\alpha_2,\vb\alpha_3)
	= \begin{bmatrix}
		k_1 & k_2 & k_3 \\
		0 & -\frac12 k_1 & \frac12 k_2 - k_1 \\
		0 & 0 & -\frac16 k_1
	\end{bmatrix}
	\quad(\text{$k_1,k_2,k_3$是任意常数}).
\end{equation*}
由于\(\vb{P}\)是可逆矩阵,
所以\(\abs{\vb{P}} = \frac12 k_1^3 \neq 0\),
说明\(k_1\)必须满足约束性条件\(k_1 \neq 0\),
因此所求可逆矩阵\(\vb{P}\)为\begin{equation*}
	\vb{P} = \begin{bmatrix}
		k_1 & k_2 & k_3 \\
		0 & -\frac12 k_1 & \frac12 k_2 - k_1 \\
		0 & 0 & -\frac16 k_1
	\end{bmatrix}
	\quad(\text{$k_1,k_2,k_3$是任意常数,且$k_1\neq0$}).
\end{equation*}

既然可逆矩阵\(\vb{P}\)存在,
那么\(\vb{A} \sim \vb{B}\).
\end{solution}
\end{example}

\subsection{矩阵相似的性质}
\begin{proposition}
%@see: 《高等代数(第三版 上册)》(丘维声) P169 命题1
%@see: 《线性代数》(张慎语、周厚隆) P97 性质2
设矩阵\(\vb{A}_1,\vb{A}_2,\vb{B}_1,\vb{B}_2,\vb{P} \in M_n(K)\),\(\vb{P}\)可逆,
且\begin{equation*}
	\vb{B}_1=\vb{P}^{-1}\vb{A}_1\vb{P}, \qquad
	\vb{B}_2=\vb{P}^{-1}\vb{A}_2\vb{P},
\end{equation*}
则\begin{gather}
	\vb{B}_1 + \vb{B}_2 = \vb{P}^{-1} (\vb{A}_1 + \vb{A}_2) \vb{P}, \\
	\vb{B}_1 \vb{B}_2 = \vb{P}^{-1} (\vb{A}_1 \vb{A}_2) \vb{P}, \\
	m\in\mathbb{N} \implies \vb{B}_1^m = \vb{P}^{-1}\vb{A}_1^m\vb{P}.
		\label{equation:相似矩阵.利用相似性简化计算}
\end{gather}
\end{proposition}
\begin{corollary}\label{theorem:相似矩阵.相似矩阵的多项式相似}
设\(f(x)\)是数域\(K\)上的一个一元多项式,
矩阵\(\vb{A},\vb{B} \in M_n(K)\),且\(\vb{A} \sim \vb{B}\),
则\begin{equation*}
	f(\vb{A}) \sim f(\vb{B}).
\end{equation*}
\end{corollary}
\begin{example}
举例说明:数域\(K\)上的一个一元多项式\(f(x)\)和矩阵\(\vb{A},\vb{B} \in M_n(K)\)满足\begin{equation*}
	f(\vb{A}) \sim f(\vb{B}),
\end{equation*}
但不满足\(\vb{A} \sim \vb{B}\).
\begin{solution}
%@credit: {0275c083-b4f8-46fa-96e2-cce85388d500}
取\(f(x) = x^2 - x,
\vb{A} = \vb{E},
\vb{B} = \vb0\),
其中\(\vb{E}\)是数域\(K\)上的\(n\)阶单位矩阵.
显然\begin{equation*}
	f(\vb{A}) = \vb{E}^2 - \vb{E} = \vb0,
	\qquad
	f(\vb{B}) = \vb0^2 - \vb0 = \vb0,
\end{equation*}
但是\(\vb{E}\)与\(\vb0\)不相似.
\end{solution}
\end{example}

\begin{property}\label{theorem:特征值与特征向量.矩阵相似的必要条件1}
%@see: 《线性代数》(张慎语、周厚隆) P97 性质1
%@see: 《高等代数(第三版 上册)》(丘维声) P169 1°
相似矩阵的行列式相等.
\begin{proof}
设\(\vb{A},\vb{B} \in M_n(K)\).
假设\(\vb{A}\sim\vb{B}\),
那么存在数域\(K\)上\(n\)阶可逆矩阵\(\vb{P}\),
使得\begin{equation*}
	\vb{P}^{-1}\vb{A}\vb{P}=\vb{B},
\end{equation*}
两端取行列式,得\begin{equation*}
	\abs{\vb{B}} = \abs{\vb{P}^{-1}\vb{A}\vb{P}}
	= \abs{\vb{P}^{-1}}\abs{\vb{A}}\abs{\vb{P}}
	= \abs{\vb{P}}^{-1}\abs{\vb{A}}\abs{\vb{P}}
	= \abs{\vb{A}}.
	\qedhere
\end{equation*}
\end{proof}
\end{property}
\begin{proposition}
%@see: 《高等代数(第三版 上册)》(丘维声) P169 2°
设\(\vb{A},\vb{B} \in M_n(K)\).
若\(\vb{A}\sim\vb{B}\),则\(\vb{A}\)和\(\vb{B}\)同为可逆或不可逆.
\begin{proof}
由\cref{theorem:特征值与特征向量.矩阵相似的必要条件1} 立即可得.
\end{proof}
\end{proposition}
\begin{remark}
如果\(\vb{A},\vb{B}\)可逆,
那么对\begin{equation*}
	\vb{P}^{-1} \vb{A} \vb{P} = \vb{B}
\end{equation*}取逆,
由\cref{theorem:逆矩阵.矩阵乘积的逆2} 得\begin{equation*}
	\vb{P}^{-1} \vb{A}^{-1} \vb{P}
	= (\vb{P}^{-1} \vb{A} \vb{P})^{-1}
	= \vb{B}^{-1},
\end{equation*}
即\(\vb{A}^{-1} \sim \vb{B}^{-1}\),
于是\cref{equation:相似矩阵.利用相似性简化计算}
可以推广为\(m\in\mathbb{Z} \implies \vb{B}^m = \vb{P}^{-1}\vb{A}^m\vb{P}\).
这也说明:
相似矩阵的同次幂也相似.
同理,我们也可以把\cref{theorem:相似矩阵.相似矩阵的多项式相似} 的前提条件
“\(f(x)\)是数域\(K\)上的一个一元多项式”
推广为“\(f(x)\)是数域\(K\)上的一个一元罗朗多项式”.
%@see: https://mathworld.wolfram.com/LaurentPolynomial.html
\end{remark}

\begin{property}\label{theorem:特征值与特征向量.矩阵相似的必要条件3}
%@see: 《线性代数》(张慎语、周厚隆) P97 性质3
相似矩阵有相同的特征多项式,从而有相同的特征值.
\begin{proof}
设\(\vb{A},\vb{B} \in M_n(K)\).
假设\(\vb{A}\sim\vb{B}\),
那么存在数域\(K\)上\(n\)阶可逆矩阵\(\vb{P}\),
使得\begin{equation*}
	\vb{P}^{-1}\vb{A}\vb{P}=\vb{B},
\end{equation*}
于是\begin{equation*}
	\abs{\lambda\vb{E}-\vb{B}}
	=\abs{\vb{P}^{-1}(\lambda\vb{E}-\vb{A})\vb{P}}
	=\abs{\vb{P}^{-1}}\abs{\lambda\vb{E}-\vb{A}}\abs{\vb{P}}
	=\abs{\lambda\vb{E}-\vb{A}}.
	\qedhere
\end{equation*}
\end{proof}
\end{property}
\begin{remark}
\cref{theorem:特征值与特征向量.矩阵相似的必要条件3} 只是矩阵相似的必要不充分条件.
下面我们举出一条反例,不相似的两个矩阵有相同的特征多项式和特征值.
取\begin{equation*}
	\vb{A} = \begin{bmatrix} 2 & 1 \\ 0 & 2 \end{bmatrix},
	\quad\text{和}\quad
	\vb{B} = \begin{bmatrix} 2 & 0 \\ 0 & 2 \end{bmatrix},
\end{equation*}
显然两者的特征多项式相同,都是\((\lambda-2)^2\),
故而两者的特征值也相同,都是\(\lambda=2\ (\text{二重})\).
但\(\vb{A}\)与\(\vb{B}\)不相似,
这是因为\(\vb{B}=2\vb{E}\)是数乘矩阵,可以和所有二阶矩阵交换,
那么对任意二阶可逆矩阵\(\vb{P}\)都有\(\vb{P}^{-1}\vb{B}\vb{P}=\vb{B}\vb{P}^{-1}\vb{P}=\vb{B}\),
即\(\vb{B}\)只能与自身相似,
\(\vb{A}\)与\(\vb{B}\)不相似.
\end{remark}

\begin{property}\label{theorem:特征值与特征向量.相似矩阵的迹的不变性}
%@see: 《高等代数(第三版 上册)》(丘维声) P170 4°
相似矩阵有相同的迹.
\begin{proof}
假设\(\vb{A}\sim\vb{B}\),
那么存在数域\(K\)上\(n\)阶可逆矩阵\(\vb{P}\),
使得\begin{equation*}
	\vb{P}^{-1}\vb{A}\vb{P}=\vb{B},
\end{equation*}
于是由\cref{theorem:矩阵的迹.矩阵乘积交换次序不变迹} 有\begin{equation*}
	\tr\vb{B}
	= \tr(\vb{P}^{-1}\vb{A}\vb{P})
	= \tr(\vb{P}(\vb{P}^{-1}\vb{A}))
	= \tr\vb{A}.
	\qedhere
\end{equation*}
\end{proof}
\end{property}

\begin{property}\label{theorem:特征值与特征向量.相似矩阵的秩的不变性}
%@see: 《高等代数(第三版 上册)》(丘维声) P170 3°
相似矩阵有相同的秩.
\begin{proof}
由\cref{theorem:矩阵乘积的秩.与可逆矩阵相乘不变秩} 可得.
\end{proof}
\end{property}
\begin{example}
%@see: 《2018年全国硕士研究生入学统一考试(数学一)》一选择题/第5题/选项(B)
证明:矩阵\begin{equation*}
	\vb{A} = \begin{bmatrix}
		1 & 1 & 0 \\
		0 & 1 & 1 \\
		0 & 0 & 1
	\end{bmatrix}
	\quad\text{与}\quad
	\vb{B} = \begin{bmatrix}
		1 & 0 & -1 \\
		0 & 1 & 1 \\
		0 & 0 & 1
	\end{bmatrix}
\end{equation*}不相似.
\begin{proof}
%@see: https://www.bilibili.com/video/BV1E9s2eSEYR/
因为\begin{gather*}
	\vb{A}-\vb{E} = \begin{bmatrix}
		0 & 1 & 0 \\
		0 & 0 & 1 \\
		0 & 0 & 0
	\end{bmatrix}
	\qquad
	\vb{B}-\vb{E} = \begin{bmatrix}
		0 & 0 & -1 \\
		0 & 0 & 1 \\
		0 & 0 & 0
	\end{bmatrix}, \\
	\rank(\vb{A}-\vb{E}) = 2
	\neq
	\rank(\vb{B}-\vb{E}) = 1,
\end{gather*}
所以\(\vb{A}-\vb{E} \not\sim \vb{B}-\vb{E}\),
于是\(\vb{A} \not\sim \vb{B}\).
\end{proof}
\end{example}

\begin{property}
相似矩阵与原矩阵等价,即\(\vb{A}\sim\vb{B} \implies \vb{A}\cong\vb{B}\).
\begin{proof}
由于\hyperref[theorem:特征值与特征向量.相似矩阵的秩的不变性]{相似矩阵的秩的不变性},
而\hyperref[theorem:矩阵乘积的秩.矩阵等价的充分必要条件]{秩相等的矩阵等价},
所以相似矩阵必定等价.
\end{proof}
\end{property}

\begin{remark}
由\cref{theorem:特征值与特征向量.矩阵相似的必要条件1,theorem:特征值与特征向量.相似矩阵的迹的不变性,theorem:特征值与特征向量.相似矩阵的秩的不变性} 可知,
数域\(K\)上的\(n\)阶方阵的行列式、秩、迹
都是相似关系下的不变量,
我们把这三个量统称为\DefineConcept{相似不变量}.
\end{remark}

\subsection{相似类}
\begin{property}\label{theorem:特征值与特征向量.相似关系是等价关系}
%@see: 《线性代数》(张慎语、周厚隆) P97
%@see: 《高等代数(第三版 上册)》(丘维声) P169
数域\(K\)上的\(n\)阶矩阵之间的相似关系,
是数域\(K\)上的全体\(n\)阶矩阵\(M_n(K)\)上的等价关系,
因为它满足:\begin{itemize}
	\item {\rm\bf 反身性}:
	\((\forall \vb{A} \in M_n(K))
	[\vb{A}\sim\vb{A}]\).

	\item {\rm\bf 对称性}:
	\((\forall \vb{A},\vb{B} \in M_n(K))
	[\vb{A} \sim \vb{B} \implies \vb{B} \sim \vb{A}]\).

	\item {\rm\bf 传递性}:
	\((\forall \vb{A},\vb{B},\vb{C} \in M_n(K))
	[\vb{A} \sim \vb{B}, \vb{B} \sim \vb{C} \implies \vb{A} \sim \vb{C}]\).
\end{itemize}
\begin{proof}
在\cref{equation:特征值与特征向量.相似矩阵的定义} 中,
令\(\vb{A}=\vb{B}\)、\(\vb{P}=\vb{E}\),得\(\vb{E}\vb{A}\vb{E}=\vb{A}\),
即有相似矩阵的反身性成立.

再在\cref{equation:特征值与特征向量.相似矩阵的定义} 中取\(\vb{Q}=\vb{P}^{-1}\),
得\(\vb{A} = \vb{Q}^{-1}(\vb{P}^{-1}\vb{A}\vb{P})\vb{Q} = \vb{Q}^{-1}\vb{B}\vb{Q}\),即有相似矩阵的对称性成立.

设\(\vb{P}_1^{-1}\vb{A}\vb{P}_1=\vb{B},
\vb{P}_2^{-1}\vb{B}\vb{P}_2=\vb{C}\),
于是\(\vb{P}_2^{-1}(\vb{P}_1^{-1}\vb{A}\vb{P}_1)\vb{P}_2=\vb{C}\).
取\(\vb{Q}=\vb{P}_1\vb{P}_2\),\(\vb{Q}\)是可逆矩阵,且\(\vb{Q}^{-1}\vb{A}\vb{Q}=\vb{C}\),所以\(\vb{A}\sim\vb{C}\),
即有相似矩阵的传递性成立.
\end{proof}
\end{property}

\begin{definition}
%@see: 《高等代数(第三版 上册)》(丘维声) P169
把矩阵\(\vb{A} \in M_n(K)\)在相似关系下的等价类\begin{equation*}
	\Set{ \vb{B} \in M_n(K) \given \vb{A}\sim\vb{B} }
\end{equation*}称为“矩阵\(\vb{A}\)的\DefineConcept{相似类}”.
\end{definition}

\begin{example}
%@see: 《高等代数(第三版 上册)》(丘维声) P171 习题5.4 1.
设矩阵\(\vb{A},\vb{B} \in M_n(K)\).
证明:如果\(\vb{A} \sim \vb{B}\),则\begin{gather}
	(\forall k \in K)
	[k\vb{A} \sim k\vb{B}], \\
	\vb{A}^T \sim \vb{B}^T.
\end{gather}
\begin{proof}
假设可逆矩阵\(\vb{P}\)满足\begin{equation*}
	\vb{P}^{-1}\vb{A}\vb{P} = \vb{B},
\end{equation*}
那么由矩阵运算规律可知,
对于\(\forall k \in K\),
成立\begin{equation*}
	\vb{P}^{-1}(k\vb{A})\vb{P}
	= k(\vb{P}^{-1}\vb{A}\vb{P})
	= k\vb{B},
	\quad\text{和}\quad
	\vb{P}^T\vb{A}^T(\vb{P}^T)^{-1}
	= (\vb{P}^{-1}\vb{A}\vb{P})^T
	= \vb{B}^T,
\end{equation*}
因此\(k\vb{A} \sim k\vb{B}\)且\(\vb{A}^T \sim \vb{B}^T\).
\end{proof}
\end{example}
\begin{example}
举例说明:即使矩阵\(\vb{A},\vb{B} \in M_n(K)\)相似,还是有\(\vb{A} + \vb{A}^T\)与\(\vb{B} + \vb{B}^T\)不相似.
\begin{solution}
取\begin{equation*}
	\vb{A} = \begin{bmatrix}
		1 & 0 \\
		0 & -1
	\end{bmatrix},
	\qquad
	\vb{B} = \begin{bmatrix}
		-3 & -2 \\
		4 & 3
	\end{bmatrix}.
\end{equation*}
\end{solution}
%@Mathematica: A = {{1, 0}, {0, -1}}
%@Mathematica: B = {{-3, -2}, {4, 3}}
%@Mathematica: Eigenvalues[A + Transpose[A]]
%@Mathematica: Eigenvalues[B + Transpose[B]]
\end{example}
\begin{example}
设矩阵\(\vb{A}\)可逆,\(\vb{A}^T\)是\(\vb{A}\)的转置.
证明:\(\vb{A}\vb{A}^T \sim \vb{A}^T\vb{A}\).
\begin{proof}
取\(\vb{P}=\vb{A}^{-1}\),
则\(\vb{P}^{-1}=\vb{A}\),\(\vb{P}\vb{A}=\vb{A}\vb{P}=\vb{E}\),
于是\begin{equation*}
	\vb{P}(\vb{A}\vb{A}^T)\vb{P}^{-1}
	= (\vb{P}\vb{A})(\vb{A}^T\vb{P}^{-1})
	= \vb{A}^T\vb{P}^{-1}
	= \vb{A}^T\vb{A},
\end{equation*}
故\(\vb{A}\vb{A}^T \sim \vb{A}^T\vb{A}\).
\end{proof}
\end{example}
\begin{example}
%@see: 《高等代数(第三版 上册)》(丘维声) P171 习题5.4 2.
%@see: 《线性代数》(张慎语、周厚隆) P105 习题5.2 5.
设矩阵\(\vb{A},\vb{B} \in M_n(K)\).
证明:如果\(\vb{A}\)可逆,则\(\vb{A}\vb{B} \sim \vb{B}\vb{A}\).
\begin{proof}
因为\(\vb{A}^{-1}(\vb{A}\vb{B})\vb{A}
= \vb{B}\vb{A}\),
所以\(\vb{A}\vb{B} \sim \vb{B}\vb{A}\).
\end{proof}
\end{example}
\begin{example}\label{example:相似矩阵.分块对角矩阵的相似性}
%@see: 《高等代数(第三版 上册)》(丘维声) P171 习题5.4 3.
设矩阵\(\vb{A}_1,\vb{B}_1 \in M_s(K),
\vb{A}_2,\vb{B}_2 \in M_n(K)\).
证明:如果\(\vb{A}_1 \sim \vb{B}_1,\vb{A}_2 \sim \vb{B}_2\),
则\begin{equation*}
	\begin{bmatrix}
		\vb{A}_1 & \vb0 \\
		\vb0 & \vb{A}_2
	\end{bmatrix}
	\sim \begin{bmatrix}
		\vb{B}_1 & \vb0 \\
		\vb0 & \vb{B}_2
	\end{bmatrix}.
\end{equation*}
\begin{proof}
假设可逆矩阵\(\vb{P}_1 \in M_s(K)\)
和可逆矩阵\(\vb{P}_2 \in M_n(K)\)满足\begin{equation*}
	\vb{P}_1^{-1} \vb{A}_1 \vb{P}_1 = \vb{B}_1,
	\qquad
	\vb{P}_2^{-1} \vb{A}_2 \vb{P}_2 = \vb{B}_2,
\end{equation*}
那么\begin{equation*}
	\begin{bmatrix}
		\vb{P}_1^{-1} & \vb0 \\
		\vb0 & \vb{P}_2^{-1}
	\end{bmatrix}
	\begin{bmatrix}
		\vb{A}_1 & \vb0 \\
		\vb0 & \vb{A}_2
	\end{bmatrix}
	\begin{bmatrix}
		\vb{P}_1 & \vb0 \\
		\vb0 & \vb{P}_2
	\end{bmatrix}
	= \begin{bmatrix}
		\vb{P}_1^{-1} \vb{A}_1 \vb{P}_1 & \vb0 \\
		\vb0 & \vb{P}_2^{-1} \vb{A}_2 \vb{P}_2
	\end{bmatrix}
	= \begin{bmatrix}
		\vb{B}_1 & \vb0 \\
		\vb0 & \vb{B}_2
	\end{bmatrix}.
	\qedhere
\end{equation*}
\end{proof}
\end{example}
\begin{example}
%@see: 《高等代数(第三版 上册)》(丘维声) P171 习题5.4 7.
证明:数量矩阵只与它本身相似.
\begin{proof}
设\(\vb{A} \in M_n(K)\),
\(\vb{E}\)是数域\(K\)上的\(n\)阶单位矩阵,
\(k \in K\),
且\(k\vb{E} \sim \vb{A}\).
根据矩阵相似的定义,
存在可逆\(\vb{P} \in M_n(K)\),
使得\(\vb{P}^{-1} (k\vb{E}) \vb{P} = \vb{A}\),
于是\(\vb{A} = k\vb{E}\).
\end{proof}
\end{example}
\begin{example}\label{example:幂等矩阵.幂等矩阵的相似类}
%@see: 《高等代数(第三版 上册)》(丘维声) P171 习题5.4 10.
证明:与幂等矩阵相似的矩阵仍是幂等矩阵.
\begin{proof}
假设\(\vb{A}\)是数域\(K\)上的一个\(n\)阶幂等矩阵,
即\(\vb{A}^2=\vb{A}\).
假设\(\vb{A}\)与数域\(K\)上的某个\(n\)阶矩阵\(\vb{B}\)相似,
那么存在可逆矩阵\(\vb{P}\),使得\begin{equation*}
	\vb{P}^{-1}\vb{A}\vb{P} = \vb{B},
\end{equation*}
从而有\begin{equation*}
	\vb{P}\vb{B}^2\vb{P}^{-1}
	= (\vb{P}\vb{B}\vb{P}^{-1})(\vb{P}\vb{B}\vb{P}^{-1})
	= \vb{A}^2
	= \vb{A}
	= \vb{P}\vb{B}\vb{P}^{-1},
\end{equation*}
于是\(\vb{B}^2=\vb{B}\),
即\(\vb{B}\)也是幂等矩阵.
\end{proof}
\end{example}
\begin{example}\label{example:对合矩阵.对合矩阵的相似类}
%@see: 《高等代数(第三版 上册)》(丘维声) P171 习题5.4 11.
证明:与对合矩阵相似的矩阵仍是对合矩阵.
\begin{proof}
假设\(\vb{E}\)是数域\(K\)上的\(n\)阶单位矩阵,
\(\vb{A}\)是数域\(K\)上的一个\(n\)阶对合矩阵,
即\(\vb{A}^2=\vb{E}\).
假设\(\vb{A}\)与数域\(K\)上的某个\(n\)阶矩阵\(\vb{B}\)相似,
那么存在可逆矩阵\(\vb{P}\),使得\begin{equation*}
	\vb{P}^{-1}\vb{A}\vb{P} = \vb{B},
\end{equation*}
从而有\begin{equation*}
	\vb{P}\vb{B}^2\vb{P}^{-1}
	= (\vb{P}\vb{B}\vb{P}^{-1})(\vb{P}\vb{B}\vb{P}^{-1})
	= \vb{A}^2
	= \vb{E},
\end{equation*}
于是\(\vb{B}^2=\vb{E}\),
即\(\vb{B}\)也是对合矩阵.
\end{proof}
\end{example}
\begin{example}\label{example:幂零矩阵.幂零矩阵的相似类}
%@see: 《高等代数(第三版 上册)》(丘维声) P171 习题5.4 12.
证明:与幂零矩阵相似的矩阵仍是幂零矩阵.
\begin{proof}
假设\(\vb{A}\)是数域\(K\)上的一个以\(m\)为幂零指数的\(n\)阶幂零矩阵,
即\(\vb{A}^m=\vb0\).
假设\(\vb{A}\)与数域\(K\)上的某个\(n\)阶矩阵\(\vb{B}\)相似,
那么存在可逆矩阵\(\vb{P}\),使得\begin{equation*}
	\vb{P}^{-1}\vb{A}\vb{P} = \vb{B},
\end{equation*}
从而有\begin{equation*}
	\vb{P}\vb{B}^m\vb{P}^{-1}
	= (\vb{P}\vb{B}\vb{P}^{-1})^m
	= \vb{A}^m
	= \vb0,
\end{equation*}
于是\(\vb{B}^m=\vb0\),
即\(\vb{B}\)也是以\(m\)为幂零指数的幂零矩阵.
\end{proof}
\end{example}
