\section{实对称矩阵的相似对角化}
由上节讨论我们知道,\(n\)阶矩阵分成可以相似对角化与不可以相似对角化两类.
实际上,矩阵的相似概念与数域有关,矩阵能否相似对角化也与数域有关.
因为一般矩阵的特征值是复数,即使\(\vb{A}\)的元素都是实数,也可能没有实数特征值.
例如,\(\vb{A} = \begin{bmatrix} 0 & -1 \\ 1 & 0 \end{bmatrix}\)是一个二阶实矩阵,
由于\begin{equation*}
	\abs{\lambda\vb{E}-\vb{A}}
	= \begin{vmatrix}
		\lambda & 1 \\
		-1 & \lambda
	\end{vmatrix}
	= (\lambda+\iu)(\lambda-\iu),
\end{equation*}
可以看出\(\vb{A}\)有两个特征值\(\pm\iu\),
其对应的特征向量\((\iu,1)^T\)和\((\iu,-1)^T\)是复向量,
因此\(\vb{A}\)在复数域上可以相似对角化,
但不存在可逆实阵\(\vb{P}\)使得\(\vb{P}^{-1}\vb{A}\vb{P}\)为对角阵.

\begin{theorem}\label{theorem:特征值与特征向量.实对称矩阵1}
%@see: 《线性代数》(张慎语、周厚隆) P109 定理6
实对称矩阵的特征值都是实数.
\begin{proof}
设\(\vb{A} \in M_n(\mathbb{R})\)满足\(\vb{A}^T=\vb{A}\).
显然\(\vb{A}\)的特征多项式\(\abs{\lambda\vb{E}-\vb{A}}\)在复数范围内有\(n\)个根.
假设\(\lambda_0\in\mathbb{C}\)是\(\vb{A}\)的任意一个特征值,
则存在\(n\)维复向量\(\vb{x}_0=(\AutoTuple{c}{n})^T \neq \vb0\),使得
\begin{gather}
	\vb{A}\vb{x}_0 = \lambda_0 \vb{x}_0, \tag1
\end{gather}
用\(\vb{x}_0\)的共轭转置向量\(\ComplexConjugate{\vb{x}_0}^T
=(\ComplexConjugate{c_1},\ComplexConjugate{c_2},\dotsc,\ComplexConjugate{c_n})\)
左乘(1)式两端,
得\begin{gather}
	\ComplexConjugate{\vb{x}_0}^T \vb{A} \vb{x}_0 = \lambda \ComplexConjugate{\vb{x}_0}^T \vb{x}_0, \tag2
\end{gather}
其中\(\vb{A}^T=\vb{A}=\ComplexConjugate{\vb{A}}\),
\(\ComplexConjugate{\vb{x}_0}^T \vb{x}_0
= \ComplexConjugate{c_1}c_1 + \ComplexConjugate{c_2}c_2 + \dotsb + \ComplexConjugate{c_n}{c_n} \in \mathbb{R}^+\).

又因为\(\ComplexConjugate{\vb{x}_0}^T \vb{A} \vb{x}_0 \in \mathbb{C}\)取转置不变,
且实对称矩阵满足\(\vb{A}^T = \vb{A} = \ComplexConjugate{\vb{A}}\),
所以\begin{equation*}
	\ComplexConjugate{\vb{x}_0}^T \vb{A} \vb{x}_0
	= (\ComplexConjugate{\vb{x}_0}^T \vb{A} \vb{x}_0)^T
	= \vb{x}_0^T \vb{A}^T \ComplexConjugate{\vb{x}_0}
	= \ComplexConjugate{\ComplexConjugate{\vb{x}_0}^T} \ComplexConjugate{\vb{A}} \ComplexConjugate{\vb{x}_0}
	= \ComplexConjugate{\ComplexConjugate{\vb{x}_0}^T \vb{A} \vb{x}_0},
\end{equation*}
说明\(\ComplexConjugate{\vb{x}_0}^T \vb{A} \vb{x}_0 \in \mathbb{R}\),进而有\(\lambda_0 \in \mathbb{R}\).
\end{proof}
\end{theorem}

\begin{theorem}\label{theorem:特征值与特征向量.实对称矩阵2}
%@see: 《线性代数》(张慎语、周厚隆) P109 定理7
实对称矩阵的不同特征值所对应的特征向量正交.
\begin{proof}
设\(\vb{A}\)是实对称矩阵,
\(\lambda_1\neq\lambda_2\)是\(\vb{A}\)的两个不同的特征值,
\(\vb{x}_1\neq\vb0\),
\(\vb{x}_2\neq\vb0\)分别是\(\vb{A}\)对应于\(\lambda_1\)、\(\lambda_2\)的特征向量,
则\(\vb{x}_1\)、\(\vb{x}_2\)都是实向量,
\begin{align*}
	\vb{A}\vb{x}_1 &= \lambda_1\vb{x}_1, \tag1 \\
	\vb{A}\vb{x}_2 &= \lambda_2\vb{x}_2. \tag2
\end{align*}
对(1)式左乘\(\vb{x}_2^T\),得\begin{gather}
	\vb{x}_2^T \vb{A} \vb{x}_1 = \lambda_1 \vb{x}_2^T \vb{x}_1, \tag3
\end{gather}
对(2)式左乘\(\vb{x}_1^T\),得\begin{gather}
	\vb{x}_1^T \vb{A} \vb{x}_2 = \lambda_2 \vb{x}_1^T \vb{x}_2, \tag4
\end{gather}
(3)式取转置,
得\(\lambda_1(\vb{x}_2^T \vb{x}_1)^T = (\vb{x}_2^T \vb{A} \vb{x}_1)^T\),
又由\(\vb{A}=\vb{A}^T\),
得\begin{gather}
	\lambda_1 \vb{x}_1^T \vb{x}_2 = \vb{x}_1^T \vb{A}^T \vb{x}_2 = \vb{x}_1^T \vb{A} \vb{x}_2, \tag5
\end{gather}
(5)式减(4)式,得
\begin{gather}
	(\lambda_2-\lambda_1)\vb{x}_1^T\vb{x}_2=0. \tag6
\end{gather}
因为\(\lambda_2 \neq \lambda_1\),
所以\(\vb{x}_1^T \vb{x}_2 = 0\),
即\((\vb{x}_1,\vb{x}_2) = 0\),
也就是\(\vb{x}_1\)与\(\vb{x}_2\)正交.
\end{proof}
\end{theorem}
\begin{remark}
一般地,对于\(n\)阶实对称矩阵\(\vb{A}\),
属于\(\vb{A}\)的同一特征值的一组线性无关的特征向量不一定相互正交,
可用施密特正交化方法将其正交化,得到\(\vb{A}\)的属于该特征值的正交特征向量组.
由以上定理,\(\vb{A}\)的几个属于不同特征值的正交特征向量组仍构成正交组.
特别地,\(\vb{A}\)有\(n\)个正交的特征向量,\(\vb{A}\)相似于对角形矩阵.
\end{remark}

\begin{example}
%@see: 《2022年全国硕士研究生入学统一考试(数学一)》一选择题/第5题/选项(D)
设矩阵\(\vb{A} \in M_n(K)\).
举例说明:“\(\vb{A}\)的属于不同特征值的特征向量相互正交”是
“\(\vb{A}\)可以相似对角化”的既不充分也不必要条件.
\begin{solution}
先证伪必要性.
取\(\vb\xi_1 = (1,0,0)^T,
\vb\xi_2 = (1,1,0)^T,
\vb\xi_3 = (1,0,1)^T\),
则\(\AutoTuple{\vb\xi}{3}\)是两两不正交的线性无关的向量组.
记矩阵\(\vb{P} = (\AutoTuple{\vb\xi}{3})\),
那么\(\vb{P}\)是可逆矩阵.
令\(\vb\Lambda = \diag(1,-1,0)\),
那么矩阵\(\vb{A} = \vb{P} \vb\Lambda \vb{P}^{-1}\)可以相似对角化,
但是\(\vb{A}\)的特征向量\(\AutoTuple{\vb\xi}{3}\)不相互正交.

再证伪充分性.
取\(\vb{A} = \begin{bmatrix}
	1 & 0 & 0 \\
	0 & 0 & 1 \\
	0 & 0 & 0
\end{bmatrix}\),
则\(\vb{A}\)有两个特征值\(1\)和\(0\ (\text{$2$重})\).
由于\(\Ker(0\vb{E}-\vb{A}) = \Ker\vb{A} = 3 - \rank\vb{A} = 1\),
所以由\cref{theorem:矩阵可以相似对角化的充分必要条件.定理3} 可知\(\vb{A}\)不可以相似对角化.
解\((0\vb{E}-\vb{A}) \vb{x} = \vb0\)得
\(\vb\xi_1 = (0,1,0)^T\)是\(\vb{A}\)的属于特征值\(0\)的一个特征向量.
解\((\vb{E}-\vb{A}) \vb{x} = \vb0\)得
\(\vb\xi_2 = (1,0,0)^T\)是\(\vb{A}\)的属于特征值\(1\)的一个特征向量.
显然\(\vb{A}\)的属于不同特征值的特征向量\(\vb\xi_1\)与\(\vb\xi_2\)是相互正交的.
\end{solution}
\end{example}

\begin{example}
%@see: 《1995年全国硕士研究生入学统一考试(数学一)》八解答题
设3阶实对称矩阵\(\vb{A}\)的特征值为
\(\lambda_1=-1,\lambda_2=\lambda_3=1\),
对应于\(\lambda_1\)的特征向量为\(\vb{\xi}_1=(0,1,1)^T\),
求\(\vb{A}\).
\begin{solution}
设属于\(\lambda=1\)的特征向量为\(\vb{\xi}=(x_1,x_2,x_3)^T\),
由于\hyperref[theorem:特征值与特征向量.实对称矩阵2]{实对称矩阵的不同特征值所对应的特征向量相互正交},
故\begin{equation*}
	\vb{\xi}^T \vb{\xi}_1 = x_2+x_3 = 0,
\end{equation*}
于是\(\vb{\xi}_2=(1,0,0)^T,
\vb{\xi}_3=(0,1,-1)^T\)
是属于\(\lambda=1\)的线性无关的特征向量,
也就是说\begin{equation*}
	\vb{A}(\vb{\xi}_1,\vb{\xi}_2,\vb{\xi}_3)
	=(-\vb{\xi}_1,\vb{\xi}_2,\vb{\xi}_3).
\end{equation*}
那么\begin{equation*}
	\vb{A}
	=(-\vb{\xi}_1,\vb{\xi}_2,\vb{\xi}_3)
	(\vb{\xi}_1,\vb{\xi}_2,\vb{\xi}_3)^{-1}
	=\begin{bmatrix}
		0 & 1 & 0 \\
		-1 & 0 & 1 \\
		-1 & 0 & -1
	\end{bmatrix}
	\begin{bmatrix}
		0 & \frac{1}{2} & \frac{1}{2} \\
		1 & 0 & 0 \\
		0 & \frac{1}{2} & -\frac{1}{2}
	\end{bmatrix}
	=\begin{bmatrix}
		1 & 0 & 0 \\
		0 & 0 & -1 \\
		0 & -1 & 0
	\end{bmatrix}.
\end{equation*}
\end{solution}
\end{example}

\begin{theorem}\label{theorem:特征值与特征向量.实对称矩阵3}
%@see: 《线性代数》(张慎语、周厚隆) P110 定理8
若\(\vb{A}\)为\(n\)阶实对称矩阵,则一定存在正交矩阵\(\vb{Q}\),使得\(\vb{\Lambda} = \vb{Q}^{-1}\vb{A}\vb{Q}\)为对角形矩阵.
\begin{proof}
用数学归纳法.
当\(n=1\)时,矩阵\(\vb{A}\)是一个实数\(a_{11}\),定理成立.
假设当\(n=k-1\)时定理成立,下面证明当\(n=k\)时定理也成立.

由\cref{theorem:特征值与特征向量.实对称矩阵1},
\(\vb{A}\)的特征值全为实数.
假设\(\lambda_1\)是\(\vb{A}\)的特征值,
并且相应地存在非零实向量\(\vb{x}_1=(\AutoTuple{c}{n})^T\),
使得\begin{equation*}
	\vb{A}\vb{x}_1=\lambda_1\vb{x}_1.
\end{equation*}
不妨设\(c_1\neq0\),则\(n\)元向量组\begin{equation*}
	\vb{x}_1,\vb{x}_2=(0,1,\dotsc,0)^T,\dotsc,\vb{x}_n=(0,0,\dotsc,1)^T
\end{equation*}线性无关.
对向量组\(X=\{\AutoTuple{\vb{x}}{n}\}\)
用施密特正交规范化方法可得规范正交组\(Y=\{\AutoTuple{\vb{y}}{n}\}\),
则\(\vb{P}=(\AutoTuple{\vb{y}}{n})\)是正交矩阵,
其中\(\vb{y}_1=\abs{\vb{x}_1}^{-1}\vb{x}_1\)是\(\vb{A}\)的特征向量.

因为\(\mathbb{R}^n\)中任意\(n+1\)个向量线性相关,
故任意向量都可由\(Y\)线性表出,
即\begin{align*}
	\vb{A}\vb{y}_1 &= \lambda_1\vb{y}_1 = \lambda_1\vb{y}_1 + 0\vb{y}_2 + \dotsb + 0\vb{y}_n, \\
	\vb{A}\vb{y}_k &= b_{1k}\vb{y}_1 + b_{2k}\vb{y}_2 + \dotsb + b_{nk}\vb{y}_n \quad(k=2,3,\dotsc,n),
\end{align*}
由分块矩阵乘法,\begin{equation*}
	\vb{A}(\AutoTuple{\vb{y}}{n})
	= (\AutoTuple{\vb{y}}{n})
	\begin{bmatrix}
		\lambda_1 & b_{11} & \dots & b_{1n} \\
		0 & b_{22} & \dots & b_{2n} \\
		\vdots & \vdots & & \vdots \\
		0 & b_{n2} & \dots & b_{nn}
	\end{bmatrix},
\end{equation*}
令\(\vb{P}=(\AutoTuple{\vb{y}}{n})\),
显然\(\vb{P}\)是正交阵,
而\(\vb{P}^{-1}\vb{A}\vb{P}=\vb{P}^T\vb{A}\vb{P}=\begin{bmatrix}
	\lambda_1 & \vb\alpha \\
	\vb0 & \vb{B}
\end{bmatrix}\).

由\((\vb{P}^T\vb{A}\vb{P})^T=\vb{P}^T\vb{A}^T\vb{P}=\vb{P}^T\vb{A}\vb{P}\)可知\(\begin{bmatrix}
	\lambda_1 & \vb\alpha \\
	\vb0 & \vb{B}
\end{bmatrix}
= \begin{bmatrix}
	\lambda_1 & \vb0 \\
	\vb\alpha^T & \vb{B}^T
\end{bmatrix}\),
于是\(\vb\alpha=\vb0\),\(\vb{B}=\vb{B}^T\),也就是说\(\vb{B}\)是\(n-1\)阶实对称矩阵.
又由归纳假设,存在\(n-1\)阶正交阵\(\vb{M}\),使得\(\vb{M}^{-1}\vb{B}\vb{M}\)成为对角阵.

令\(\vb{Q}=\vb{P}\begin{bmatrix} 1 & \vb0 \\ \vb0 & \vb{M} \end{bmatrix}\),
因为\(\vb{Q}\vb{Q}^T=\vb{Q}^T\vb{Q}=\vb{E}\),所以\(\vb{Q}\)是正交矩阵,
而\begin{align*}
	\vb{Q}^{-1}\vb{A}\vb{Q}
	&=\begin{bmatrix}
		1 & \vb0 \\
		\vb0 & \vb{M}^{-1}
	\end{bmatrix}\vb{P}^{-1}\vb{A}\vb{P}\begin{bmatrix}
		1 & \vb0 \\
		\vb0 & \vb{M}
	\end{bmatrix}=\begin{bmatrix}
		1 & \vb0 \\
		\vb0 & \vb{M}^{-1}
	\end{bmatrix}\begin{bmatrix}
		\lambda_1 & \vb0 \\
		\vb0 & \vb{B}
	\end{bmatrix}\begin{bmatrix}
		1 & \vb0 \\
		\vb0 & \vb{M}
	\end{bmatrix} \\
	&=\begin{bmatrix}
		\lambda_1 & \vb0 \\
		\vb0 & \vb{M}^{-1}\vb{B}\vb{M}
	\end{bmatrix}
	=\diag(\AutoTuple{\lambda}{n}).
\end{align*}
由上可知当\(n=k\)时定理也成立.
\end{proof}
\end{theorem}

\cref{theorem:特征值与特征向量.实对称矩阵3} 表明,
实对称矩阵总可以相似对角化,
或者说,实对称矩阵总是\DefineConcept{正交相似}({orthogonally similar})于某个对角形矩阵.

从\cref{theorem:特征值与特征向量.矩阵相似的必要条件3} 我们已经知道
\(\vb{A}\sim\vb{B} \implies \abs{\lambda\vb{E}-\vb{A}}=\abs{\lambda\vb{E}-\vb{B}}\).
但对于一般的同阶矩阵\(\vb{A},\vb{B}\),
我们不能肯定\cref{theorem:特征值与特征向量.矩阵相似的必要条件3} 的逆命题一定成立.
但是对于实对称矩阵来说,
只要加上一个额外条件“\(\vb{A}\)与\(\vb{B}\)的特征值相同”,
就可以依靠\cref{theorem:特征值与特征向量.实对称矩阵3} 证明\(\vb{A}\)与\(\vb{B}\)相似.
\begin{corollary}
%@see: 《线性代数》(张慎语、周厚隆) P113 习题5.3 1(2)
%@see: 《线性代数》(张慎语、周厚隆) P113 习题5.3 9(1)
设\(\vb{A},\vb{B}\)是同阶实对称矩阵,
则\begin{equation*}
	\vb{A}\sim\vb{B}
	\iff
	\abs{\lambda\vb{E}-\vb{A}}=\abs{\lambda\vb{E}-\vb{B}}.
\end{equation*}
\end{corollary}

\begin{corollary}
%@see: 《线性代数》(张慎语、周厚隆) P110 推论
\(n\)阶实对称矩阵\(\vb{A}\)存在\(n\)个正交的单位特征向量.
\end{corollary}

\begin{remark}
\color{red}
对于实对称矩阵\(\vb{A}\),求正交矩阵\(\vb{Q}\),使得\(\vb{Q}^{-1}\vb{A}\vb{Q}\)为对角形矩阵的方法:
\begin{enumerate}
	\item 求出\(\vb{A}\)的全部不同的特征值\(\AutoTuple{\lambda}{m}\);
	\item 求出\((\lambda_i\vb{E}-\vb{A})\vb{x}=\vb0\)的基础解系,将其正交化,
	得到\(\vb{A}\)属于\(\lambda_i\ (i=1,2,\dotsc,m)\)的正交特征向量,
	共求出\(\vb{A}\)的\(n\)个正交特征向量;
	\item 将以上\(n\)个正交特征向量单位化,由所得向量作为列构成正交矩阵\(\vb{Q}\),则\begin{equation*}
		\vb{Q}^{-1}\vb{A}\vb{Q} = \vb{Q}^T \vb{A} \vb{Q} = \diag(\AutoTuple{\lambda}{n}).
	\end{equation*}
\end{enumerate}
\end{remark}

\begin{example}
%@see: 《线性代数》(张慎语、周厚隆) P112 例5
设\(\vb{A}\)为\(n\)阶实对称矩阵,\(\vb{A}\)是对合矩阵,证明:存在正交矩阵\(\vb{Q}\),使得\begin{equation*}
	\vb{Q}^{-1}\vb{A}\vb{Q}=\begin{bmatrix} \vb{E}_r \\ & -\vb{E}_{n-r} \end{bmatrix}.
\end{equation*}
\begin{proof}
因为\(\vb{A}\)为\(n\)阶实对称矩阵,
则\(\vb{A}\)有\(n\)个实特征值,
\(\vb{A}\)有\(n\)个正交的单位特征向量,
适当调整它们的顺序,可以构成正交矩阵\(\vb{Q}\),
满足\begin{gather}
	\vb{Q}^{-1}\vb{A}\vb{Q}=\diag(\AutoTuple{\lambda}{n}), \tag1
\end{gather}
其中,\(\lambda_i>0\ (i=1,2,\dotsc,r),
\lambda_i\leq0\ (i=r+1,r+2,\dotsc,n)\).
对(1)式两端分别平方,
又由\(\vb{A}\)是对合矩阵,满足\(\vb{A}^2=\vb{E}\),
得\begin{equation*}
	\vb{Q}^{-1}\vb{A}^2\vb{Q}
	= \vb{Q}^{-1}\vb{E}\vb{Q}
	= \vb{E}
	= \diag(\lambda_1^2,\lambda_2^2,\dotsc,\lambda_n^2),
\end{equation*}
于是\(\lambda_i^2=1\ (i=1,2,\dotsc,n)\),
进而有\begin{equation*}
	\lambda_i= \begin{cases}
		1, & i=1,2,\dotsc,r, \\
		-1, & i=r+1,r+2,\dotsc,n.
	\end{cases}
	\qedhere
\end{equation*}
\end{proof}
\end{example}

\begin{example}
设\(\vb{A}\)是4阶实对称矩阵,且\(\vb{A}^2+\vb{A}=\vb0\).
若\(\rank\vb{A}=3\),求\(\vb{A}\)的特征值以及与\(\vb{A}\)相似的对角阵.
\begin{solution}
\(\vb{A}\)是实对称矩阵,根据\cref{theorem:特征值与特征向量.实对称矩阵3},
\(\vb{A}\)一定可以相似对角化,不妨设\(\vb{A}\vb{x}=\lambda\vb{x}\ (\vb{x}\neq0)\),
那么\(\vb{A}^2\vb{x}=\lambda^2\vb{x}\),\((\vb{A}^2+\vb{A})\vb{x}=(\lambda^2+\lambda)\vb{x}\).
因为\(\vb{A}^2+\vb{A}=\vb0\),所以\((\lambda^2+\lambda)\vb{x}=\vb0\),\(\lambda^2+\lambda=0\),
解得\(\vb{A}\)的特征值为\(\lambda=0,-1\).
又因为\(\rank\vb{A}=3\),所以\(\vb{A}\)具有3个非零特征值,
因此与\(\vb{A}\)相似的对角阵为\(\diag(-1,-1,-1,0)\).
\end{solution}
\end{example}

\begin{example}
设\(\vb{A}\)是特征值仅为1与0的\(n\)阶实对称矩阵,证明:\(\vb{A}\)是幂等矩阵.
\begin{proof}
\def\M{\begin{bmatrix} \vb{E}_r \\ & \vb0_{n-r} \end{bmatrix}}%
因为\(\vb{A}\)是实对称矩阵,所以存在正交矩阵\(\vb{Q}\)使得\begin{equation*}
	\vb{Q}^{-1}\vb{A}\vb{Q} = \M,
\end{equation*}
从而有\begin{equation*}
	\vb{A} = \vb{Q}\M\vb{Q}^{-1},
\end{equation*}
进而有\begin{equation*}
	\vb{A}^2 = \vb{Q}\M\vb{Q}^{-1}\vb{Q}\M\vb{Q}^{-1} = \vb{Q}\M\vb{Q}^{-1} = \vb{A}.
	\qedhere
\end{equation*}
\end{proof}
\end{example}

\begin{example}
%@see: 《线性代数》(张慎语、周厚隆) P113 习题5.3 7.
设\(\vb{A}\)为\(n\)阶实对称矩阵,满足\(\vb{A}^2=\vb0\),证明:\(\vb{A}=\vb0\).
\begin{proof}
因为\(\vb{A}\)是实对称矩阵,所以存在正交矩阵\(\vb{Q}\)使得\begin{equation*}
	\vb{Q}^{-1}\vb{A}\vb{Q} = \diag(\AutoTuple{\lambda}{n}) = \vb{\Lambda},
\end{equation*}
从而有\(\vb{A} = \vb{Q}\vb{\Lambda}\vb{Q}^{-1}\),\(\vb{A}^2 = (\vb{Q}\vb{\Lambda}\vb{Q}^{-1})^2 = \vb{Q}\vb{\Lambda}^2\vb{Q}^{-1} = \vb0\),那么\begin{equation*}
	\vb{\Lambda}^2 = \diag(\lambda_1^2,\lambda_2^2,\dotsc,\lambda_n^2) = \vb{Q}^{-1}\vb0\vb{Q} = \vb0,
\end{equation*}\begin{equation*}
	\lambda_1=\lambda_2=\dotsb=\lambda_n = 0,
\end{equation*}
所以\(\vb{A}=\vb0\).
\end{proof}
\end{example}

\begin{example}
%@see: 《2017年全国硕士研究生入学统一考试(数学一)》一选择题/第5题
设\(\vb\alpha\)是实数域上的\(n\)维单位列向量,
\(\vb{E}\)是实数域上的\(n\)阶单位矩阵.
判断矩阵\begin{equation*}
	\vb{E} - k \vb\alpha \vb\alpha^T
	\quad(k\in\mathbb{R})
\end{equation*}是否可逆.
\begin{solution}
由\cref{theorem:矩阵的迹.矩阵乘积交换次序不变迹} 可知
\(\tr(\vb\alpha \vb\alpha^T)
= \tr(\vb\alpha^T \vb\alpha)\).
因为\(\vb\alpha\)是单位向量,
所以\begin{equation*}
	\tr(\vb\alpha \vb\alpha^T)
	= \vb\alpha^T \vb\alpha
	= 1.
\end{equation*}
由\cref{example:矩阵乘积的秩.两个向量的乘积的秩} 可知\begin{equation*}
	\rank(\vb\alpha \vb\alpha^T) = 1.
\end{equation*}
由于\((\vb\alpha \vb\alpha^T)^T = \vb\alpha \vb\alpha^T\),
\(\vb\alpha \vb\alpha^T\)是实对称矩阵,
所以由\cref{theorem:特征值与特征向量.实对称矩阵3} 可知
\(\vb\alpha \vb\alpha^T\)可以相似对角化.
不妨设\begin{equation*}
	\vb\alpha \vb\alpha^T
	\sim
	\diag(\AutoTuple{\lambda}{n}).
\end{equation*}
由于\(\rank(\vb\alpha \vb\alpha^T) = 1\),
所以\(\AutoTuple{\lambda}{n}\)中必有\(n-1\)个是零,
因此不妨设\(\lambda_1 \neq 0\)
而\(\lambda_2 = \dotsb = \lambda_n = 0\).
又因为\begin{equation*}
	\tr(\vb\alpha \vb\alpha^T)
	= \lambda_1 + \lambda_2 + \dotsb + \lambda_n
	= \lambda_1
	= 1,
\end{equation*}
所以存在正交矩阵\(\vb{P}\)使得\begin{equation*}
	\diag(1,0,\dotsc,0)
	= \vb{P}^{-1} (\vb\alpha \vb\alpha^T) \vb{P}.
\end{equation*}
因为\begin{equation*}
	\vb{P}^{-1} (\vb{E} - k \vb\alpha \vb\alpha^T) \vb{P}
	= \vb{E} - k \diag(1,0,\dotsc,0)
	= \diag(1-k,1,\dotsc,1),
\end{equation*}
所以\begin{equation*}
	\abs{\vb{E} - k \vb\alpha \vb\alpha^T}
	= \abs{\vb{P}^{-1} (\vb{E} - k \vb\alpha \vb\alpha^T) \vb{P}}
	= \abs{\diag(1-k,1,\dotsc,1)}
	= 1-k,
\end{equation*}
因此\begin{equation*}
	\text{矩阵$\vb{E} - k \vb\alpha \vb\alpha^T$可逆}
	\iff
	k\neq1.
\end{equation*}
\end{solution}
\end{example}
