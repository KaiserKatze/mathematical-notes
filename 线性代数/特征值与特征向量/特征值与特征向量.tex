\section{矩阵的特征值与特征向量}
首先看一个实际例子.
\begin{example}
%@see: 《线性代数》(张慎语、周厚隆) P91 例1
设点\(M\)在新、旧坐标系的坐标分别为\((x',y',z')^T\)与\((x,y,z)^T\),
则在旋转变换下有\begin{equation*}
	\left\{ \begin{array}{l}
		x' = l_1 x + l_2 y + l_3 z, \\
		y' = m_1 x + m_2 y + m_3 z, \\
		z' = n_1 x + n_2 y + n_3 z
	\end{array} \right.
	\quad\text{或}\quad
	\begin{bmatrix}
		x' \\ y' \\ z'
	\end{bmatrix}
	= \begin{bmatrix}
		l_1 & l_2 & l_3 \\
		m_1 & m_2 & m_3 \\
		n_1 & n_2 & n_3
	\end{bmatrix}
	\begin{bmatrix}
		x \\ y \\ z
	\end{bmatrix},
\end{equation*}
记\begin{equation*}
	\vb{Y} \defeq \begin{bmatrix}
		x' \\ y' \\ z'
	\end{bmatrix},
	\qquad
	\vb{A} \defeq \begin{bmatrix}
		l_1 & l_2 & l_3 \\
		m_1 & m_2 & m_3 \\
		n_1 & n_2 & n_3
	\end{bmatrix},
	\qquad
	\vb{X} \defeq \begin{bmatrix}
		x \\ y \\ z
	\end{bmatrix},
\end{equation*}
于是又有\begin{equation*}
	\vb{Y}=\vb{A}\vb{X},
\end{equation*}
这里\(\vb{A}\)的第一、二、三列分别是坐标轴\(Ox,Oy,Oz\)在\(Ox'y'z'\)的方向余弦,
以后我们会证明\(\vb{A}\)是一个正交矩阵.
可以注意到,如果点\(M\)在\(z\)轴上,当它绕\(z\)轴旋转时,
点\(M\)保持不动,也就是说\(\vb{A}\vb{X}=1\vb{X}\).
\end{example}
从这个例子可以看出,有的时候,一个向量左乘一个矩阵得到的结果,就跟它乘以一个标量一样.

\subsection{特征值,特征向量}
\begin{definition}
%@see: 《高等代数(第三版 上册)》(丘维声) P172 定义1
%@see: 《线性代数》(张慎语、周厚隆) P92 定义1
设矩阵\(\vb{A} \in M_n(K)\).
如果\begin{equation*}
	(\exists \lambda \in K)
	(\exists \vb{x} \in K^n - \{\vb0\})
	[\vb{A}\vb{x} = \lambda\vb{x}],
\end{equation*}
则称“数\(\lambda\)是矩阵\(\vb{A}\)的一个\DefineConcept{特征值}(eigenvalue)”
%@see: https://mathworld.wolfram.com/Eigenvalue.html
“向量\(\vb{x}\)是矩阵\(\vb{A}\)的(属于特征值\(\lambda\)的)一个\DefineConcept{特征向量}(eigenvector)”.
%@see: https://mathworld.wolfram.com/Eigenvector.html
\end{definition}
\begin{remark}
零向量不是任何一个矩阵的特征向量.
\end{remark}
\begin{example}
设\begin{equation*}
	\vb{A} = \begin{bmatrix}
		1 & 1 \\ 1 & 1
	\end{bmatrix},
	\qquad
	\vb\alpha = \begin{bmatrix}
		1 \\ 1
	\end{bmatrix}.
\end{equation*}
由于\begin{equation*}
	\vb{A} \vb\alpha
	= \begin{bmatrix}
		1 & 1 \\ 1 & 1
	\end{bmatrix}
	\begin{bmatrix}
		1 \\ 1
	\end{bmatrix}
	= \begin{bmatrix}
		2 \\ 2
	\end{bmatrix}
	= 2 \begin{bmatrix}
		1 \\ 1
	\end{bmatrix}
	= 2 \vb\alpha,
\end{equation*}
所以\(2\)是\(\vb{A}\)的一个特征值,
\(\vb\alpha\)是\(\vb{A}\)的属于特征值\(2\)的一个特征向量.
\end{example}
\begin{example}
设\(\vb{E}\)是数域\(K\)上的\(n\)阶单位矩阵,
那么对于\(K^n\)中的任意一个非零向量\(\vb{x}\),
都成立\begin{equation*}
	\vb{E} \vb{x} = 1 \vb{x},
\end{equation*}
也就是说,\(1\)是单位矩阵的唯一一个特征值,
任意一个非零向量都是单位矩阵的特征向量.
\end{example}

\begin{example}
%@see: 《高等代数(第三版 上册)》(丘维声) P172
根据\cref{equation:解析几何.平面坐标系的点的右手直角坐标变换公式I到II.矩阵形式1},
在平面上绕原点\(O\)的转角为\(\pi/3\)的旋转变换可以表示为实数域上的2阶矩阵\begin{equation*}
	\vb{A}
	= \begin{bmatrix}
		\cos(\pi/3) & -\sin(\pi/3) \\
		\sin(\pi/3) & \cos(\pi/3)
	\end{bmatrix}
	= \begin{bmatrix}
		1/2 & -\sqrt3/2 \\
		\sqrt3/2 & 1/2
	\end{bmatrix}.
\end{equation*}
显然平面上任一非零向量在旋转一次后都不会变成它的倍数,
因此在\(\mathbb{R}^2\)中不存在非零列向量\(\vb\alpha\)
满足\(\vb{A}\vb\alpha=\lambda\vb\alpha\),
也就是说\(\vb{A}\)没有特征值、特征向量.

这个例子说明:不是所有数域上的矩阵都有特征值、特征向量.
但是我们只要把这个矩阵看成是复数域上的2阶矩阵,
就会发现\(\frac12(1\pm\iu\sqrt3)\)是它的两个特征值.
%@Mathematica: Eigenvalues[{{Cos[Pi/3], -Sin[Pi/3]}, {Sin[Pi/3], Cos[Pi/3]}}]
%@Mathematica: Eigenvectors[{{Cos[Pi/3], -Sin[Pi/3]}, {Sin[Pi/3], Cos[Pi/3]}}]
我们将在下面证明:复数域上的矩阵必定存在特征值、特征向量.
\end{example}

\begin{example}\label{example:特征值与特征向量.各行元素之和相同的矩阵的特征值与特征向量}
设矩阵\(\vb{A} \in M_n(K)\),
\(\vb{A}\)的各行元素之和均为\(\lambda_0\).
证明:\(\lambda_0\)是矩阵\(\vb{A}\)的特征值,
\(n\)维列向量\(\vb{x}_0=(1,\dotsc,1)^T\)是矩阵\(\vb{A}\)属于\(\lambda_0\)的特征向量.
\begin{proof}
记\begin{equation*}
	\vb{A}
	= \begin{bmatrix}
		a_{11} & a_{12} & \dots & a_{1n} \\
		a_{21} & a_{22} & \dots & a_{2n} \\
		\vdots & \vdots & & \vdots \\
		a_{n1} & a_{n2} & \dots & a_{nn}
	\end{bmatrix},
\end{equation*}
那么\begin{equation*}
	\vb{A} \vb{x}_0
	= \begin{bmatrix}
		a_{11} & a_{12} & \dots & a_{1n} \\
		a_{21} & a_{22} & \dots & a_{2n} \\
		\vdots & \vdots & & \vdots \\
		a_{n1} & a_{n2} & \dots & a_{nn}
	\end{bmatrix}
	\begin{bmatrix}
		1 \\ 1 \\ \vdots \\ 1
	\end{bmatrix}
	= \begin{bmatrix}
		a_{11} + a_{12} + \dotsb + a_{1n} \\
		a_{21} + a_{22} + \dotsb + a_{2n} \\
		\vdots \\
		a_{n1} + a_{n2} + \dotsb + a_{nn}
	\end{bmatrix}
	= \begin{bmatrix}
		\lambda_0 \\ \lambda_0 \\ \vdots \\ \lambda_0
	\end{bmatrix}
	= \lambda_0
	\begin{bmatrix}
		1 \\ 1 \\ \vdots \\ 1
	\end{bmatrix},
\end{equation*}
由此可见,\(\lambda_0\)是\(\vb{A}\)的特征值,
\(n\)维列向量\(\vb{x}_0=(1,\dotsc,1)^T\)是\(\vb{A}\)属于\(\lambda_0\)的特征向量.
\end{proof}
\end{example}

\begin{example}\label{example:矩阵的特征值与特征向量.零是奇异矩阵的特征值}
%@see: 《高等代数(第三版 上册)》(丘维声) P179 习题5.5 9.
设\(\vb{A} \in M_n(K)\).
证明:\(0\)是\(\vb{A}\)的一个特征值当且仅当\(\abs{\vb{A}}=0\).
\begin{proof}
由\cref{theorem:矩阵的特征值与特征向量.与特征多项式和特征子空间的联系} 有\begin{align*}
	&\text{\(0\)是\(\vb{A}\)的一个特征值} \\
	&\iff
	\abs{0\vb{E}-\vb{A}}
	= \abs{-\vb{A}}
	= (-1)^n \abs{\vb{A}}
	= 0 \\
	&\iff
	\abs{\vb{A}} = 0.
	\qedhere
\end{align*}
\end{proof}
\end{example}
\begin{remark}
如果\(0\)是矩阵\(\vb{A}\)的一个特征值,
那么矩阵\(\vb{A}\)的属于特征值\(0\)的每个特征向量都是齐次方程\(\vb{A}\vb{x}=\vb0\)的解,
这就说明\(\vb{A}\vb{x}=\vb0\)有非零解,
由\cref{theorem:线性方程组.齐次线性方程组有非零解的充分必要条件} 可知\(\rank\vb{A} < n\).
但是,即便一个矩阵的特征值全是零,也不能说明它的秩为零.
例如,幂零矩阵\begin{equation*}
	\begin{bmatrix}
		0 & 1 & 0 & 0 \\
		0 & 0 & 1 & 0 \\
		0 & 0 & 0 & 1 \\
		0 & 0 & 0 & 0
	\end{bmatrix}
\end{equation*}的特征值全为零(见\cref{example:幂零矩阵.幂零矩阵的特征值的性质}),
但是它有一个三阶非零子式\begin{equation*}
	\begin{vmatrix}
		1 & 0 & 0 \\
		0 & 1 & 0 \\
		0 & 0 & 1
	\end{vmatrix},
\end{equation*}
所以它的秩为\(3\).
要想保证一个矩阵的秩等于它的非零特征值个数,
需要验证它是否\hyperref[definition:相似对角化.相似对角化]{可以相似对角化},
这是因为\hyperref[theorem:特征值与特征向量.相似矩阵的迹的不变性]{秩是相似不变量}.
如果一个矩阵不可以相似对角化,则我们至多像上面提到的那样,
根据这个矩阵是否存在零特征值,判断它是否不满秩.
\end{remark}
\begin{example}\label{example:矩阵的特征值与特征向量.零不是非奇异矩阵的特征值}
%@see: 《高等代数(第三版 上册)》(丘维声) P179 习题5.5 8.(1)
设\(\vb{A}\)是数域\(K\)上一个\(n\)阶可逆矩阵.
证明:如果\(\vb{A}\)有特征值,则\(\vb{A}\)的特征值不等于\(0\).
\begin{proof}
由\cref{example:矩阵的特征值与特征向量.零是奇异矩阵的特征值} 可知,
\(0\)是奇异矩阵的特征值,
而可逆矩阵必定是非奇异矩阵,
所以\(\vb{A}\)的特征值只要存在就不可能等于\(0\).
\end{proof}
\end{example}

\begin{proposition}\label{theorem:矩阵的特征值与特征向量.特征子空间1}
%@see: 《线性代数》(张慎语、周厚隆) P92 性质1
设\(\vb{A} \in M_n(K)\).
如果\(\vb{x}_1,\vb{x}_2\)是\(\vb{A}\)的属于同一个特征值\(\lambda_0\)的特征向量,
且\(\vb{x}_1 + \vb{x}_2 \neq \vb0\),
则\(\vb{x}_1 + \vb{x}_2\)也是\(\vb{A}\)的属于\(\lambda_0\)的特征向量.
\begin{proof}
\(\vb{A} (\vb{x}_1 + \vb{x}_2)
= \vb{A} \vb{x}_1 + \vb{A} \vb{x}_2
= \lambda_0 \vb{x}_1 + \lambda_0 \vb{x}_2
= \lambda_0 (\vb{x}_1 + \vb{x}_2)\).
\end{proof}
\end{proposition}
\begin{proposition}\label{theorem:矩阵的特征值与特征向量.特征子空间2}
%@see: 《线性代数》(张慎语、周厚隆) P92 性质2
设\(\vb{A} \in M_n(K)\).
如果\(\vb{x}_0\)是\(\vb{A}\)的属于特征值\(\lambda_0\)的特征向量,
且\(k \in K-\{0\}\),
则\(k \vb{x}_0\)也是\(\vb{A}\)的属于\(\lambda_0\)的特征向量.
\begin{proof}
\(\vb{A} (k \vb{x}_0)
= k (\vb{A} \vb{x}_0)
= k (\lambda_0 \vb{x}_0)
= \lambda_0 (k \vb{x}_0)\).
\end{proof}
\end{proposition}
\begin{example}\label{example:矩阵的特征值与特征向量.矩阵的多项式的特征值1}
%@see: 《线性代数》(张慎语、周厚隆) P96 例6
%@see: 《高等代数(第三版 上册)》(丘维声) P180 习题5.5 11.(1)
设\(\vb{A} \in M_n(K)\).
证明:如果\(\lambda_0\)是\(\vb{A}\)的特征值,
则对于任意\(k \in K\),
都有\(k\lambda_0\)是\(k\vb{A}\)的特征值.
\begin{proof}
假设\(\vb{A} \vb{x}_0 = \lambda_0 \vb{x}_0\),
则\((k\vb{A}) \vb{x}_0
= k(\vb{A} \vb{x}_0)
= k(\lambda_0 \vb{x}_0)
= (k\lambda_0) \vb{x}_0\).
\end{proof}
\end{example}
\begin{example}\label{example:矩阵的特征值与特征向量.矩阵的多项式的特征值2}
%@see: 《线性代数》(张慎语、周厚隆) P92 性质3
%@see: 《高等代数(第三版 上册)》(丘维声) P180 习题5.5 11.(2)
设\(\vb{A} \in M_n(K)\).
证明:如果\(\lambda_0\)是\(\vb{A}\)的特征值,
则对于任意\(m\in\mathbb{N}\),
都有\(\lambda_0^m\)是\(\vb{A}^m\)的特征值.
\begin{proof}
由定义,存在\(\vb{x}_0\neq\vb0\),
使得\(\vb{A}\vb{x}_0 = \lambda_0\vb{x}_0\),
则\begin{gather*}
	\vb{A}^0 \vb{x}_0
	= \vb{E} \vb{x}_0
	= 1 \vb{x}_0
	= \lambda_0^0 \vb{x}_0, \\
	\vb{A}^2\vb{x}_0 = \vb{A}(\vb{A}\vb{x}_0)
	= \vb{A}(\lambda_0\vb{x}_0)
	= \lambda_0(\vb{A}\vb{x}_0)
	= \lambda_0(\lambda_0\vb{x}_0)
	= \lambda_0^2\vb{x}_0.
\end{gather*}
用数学归纳法.
假设\(\vb{A}^{m-1}\vb{x}_0 = \lambda_0^{m-1}\vb{x}_0\ (m\geq3)\)成立,
则\begin{equation*}
	\vb{A}^m\vb{x}_0 = \vb{A}(\vb{A}^{m-1}\vb{x}_0)
	= \vb{A}(\lambda_0^{m-1}\vb{x}_0)
	= \lambda_0^{m-1}(\vb{A}\vb{x}_0)
	= \lambda_0^{m-1}(\lambda_0\vb{x}_0)
	= \lambda_0^m\vb{x}_0.
	\qedhere
\end{equation*}
\end{proof}
\end{example}
\begin{example}\label{example:矩阵的特征值与特征向量.矩阵的多项式的特征值3}
%@see: 《线性代数》(张慎语、周厚隆) P97 习题5.1 4.
%@see: 《高等代数(第三版 上册)》(丘维声) P180 习题5.5 8.(2)
设\(\vb{A}\)是数域\(K\)上的一个可逆矩阵.
证明:如果\(\lambda_0\)是\(\vb{A}\)的一个特征值,
则\(\lambda_0^{-1}\)是\(\vb{A}^{-1}\)的一个特征值.
\begin{proof}
假设非零向量\(\vb{x}_0\)满足
\(\lambda_0 \vb{x}_0 = \vb{A} \vb{x}_0\),
左乘\(\vb{A}^{-1}\)得\begin{equation*}
	\vb{A}^{-1} (\lambda_0 \vb{x}_0)
	= \vb{A}^{-1} (\vb{A} \vb{x}_0),
\end{equation*}
这里\begin{equation*}
	\vb{A}^{-1} (\lambda_0 \vb{x}_0)
	= \lambda_0 (\vb{A}^{-1} \vb{x}_0),
	\qquad
	\vb{A}^{-1} (\vb{A} \vb{x}_0)
	= (\vb{A}^{-1} \vb{A}) \vb{x}_0
	= \vb{E} \vb{x}_0
	= \vb{x}_0,
\end{equation*}
于是\(\lambda_0 (\vb{A}^{-1} \vb{x}_0) = \vb{x}_0\),
由\cref{example:矩阵的特征值与特征向量.零不是非奇异矩阵的特征值} 可知
\(\lambda_0 \neq 0\),
那么有\(\vb{A}^{-1} \vb{x}_0 = \lambda_0^{-1} \vb{x}_0\).
\end{proof}
\end{example}
\begin{example}\label{example:矩阵的特征值与特征向量.伴随矩阵的特征值}
设\(\vb{A}\)是数域\(K\)上的一个可逆矩阵.
证明:如果\(\lambda_0\)是\(\vb{A}\)的一个特征值,
\(\vb{x}_0\)是\(\vb{A}\)的属于\(\lambda_0\)的一个特征向量,
则\(\lambda_0^{-1} \abs{\vb{A}}\)是\(\vb{A}^*\)的一个特征值,
\(\vb{x}_0\)是\(\vb{A}^*\)的属于\(\lambda_0^{-1} \abs{\vb{A}}\)的一个特征向量.
\begin{proof}
% 由\cref{example:矩阵的特征值与特征向量.矩阵的多项式的特征值3} 可知
% \(\lambda_0^{-1}\)是\(\vb{A}^{-1}\)的一个特征值.
% 由\cref{example:矩阵的特征值与特征向量.矩阵的多项式的特征值1} 可知,
% 对于任意\(k \in K\),
% \(k \lambda_0^{-1}\)是\(k \vb{A}^{-1}\)的一个特征值.
% 于是由 \hyperref[theorem:逆矩阵.逆矩阵的唯一性]{\(\vb{A}^* = \abs{\vb{A}} \vb{A}^{-1}\)}
% 可知\(\lambda_0^{-1} \abs{\vb{A}}\)是\(\vb{A}^*\)的一个特征值.
在 \hyperref[equation:行列式.伴随矩阵.恒等式1]{\(\vb{A}^* \vb{A} = \abs{\vb{A}} \vb{E}\)}
等号两边右乘向量\(\vb{x}_0\),
得\begin{equation*}
	\vb{A}^* \vb{A} \vb{x}_0
	= \abs{\vb{A}} \vb{x}_0.
\end{equation*}
因为\(\vb{A} \vb{x}_0 = \lambda_0 \vb{x}_0\),
所以上式可以化为\begin{equation*}
	\vb{A}^* (\lambda_0 \vb{x}_0)
	= \abs{\vb{A}} \vb{x}_0.
\end{equation*}
由\cref{example:矩阵的特征值与特征向量.零不是非奇异矩阵的特征值} 可知
\(\lambda_0 \neq 0\),
所以上式又可以化为\begin{equation*}
	\vb{A}^* \vb{x}_0
	= \frac{\abs{\vb{A}}}{\lambda_0} \vb{x}_0,
\end{equation*}
这就说明\(\lambda_0^{-1} \abs{\vb{A}}\)是\(\vb{A}^*\)的一个特征值,
\(\vb{x}_0\)是\(\vb{A}^*\)的属于\(\lambda_0^{-1} \abs{\vb{A}}\)的一个特征向量.
\end{proof}
\end{example}
\begin{example}\label{example:矩阵的特征值与特征向量.矩阵的多项式的特征值4}
%@see: 《高等代数(第三版 上册)》(丘维声) P180 习题5.5 11.(3)
%@see: 《矩阵论》(詹兴致) P4 定理1.3(谱映射定理)
设\(\vb{A} \in M_n(K)\),
\(\lambda_0\)是\(\vb{A}\)的一个特征值,
\(f(x)\)是数域\(K\)上的一个一元多项式.
证明:\(f(\lambda_0)\)是\(f(\vb{A})\)的一个特征值.
\begin{proof}
假设\(\vb{A} \vb{x}_0 = \lambda_0 \vb{x}_0\),
那么由\cref{example:矩阵的特征值与特征向量.矩阵的多项式的特征值1,example:矩阵的特征值与特征向量.矩阵的多项式的特征值2}
可知对于任意\(a_n \in K\)和任意\(n\in\mathbb{N}\)总成立\begin{equation*}
	(a_n \vb{A}^n) \vb{x}_0
	= (a_n \lambda_0^n) \vb{x}_0.
\end{equation*}
于是\begin{align*}
	f(\vb{A}) \vb{x}_0
	&= (a_0 \vb{E} + a_1 \vb{A} + a_2 \vb{A}^2 + \dotsb + a_m \vb{A}^m) \vb{x}_0 \\
	&= a_0 \vb{E} \vb{x}_0 + a_1 \vb{A} \vb{x}_0 + a_2 \vb{A}^2 \vb{x}_0 + \dotsb + a_m \vb{A}^m \vb{x}_0 \\
	&= a_0 \vb{x}_0 + a_1 \lambda_0 \vb{x}_0 + a_2 \lambda_0^2 \vb{x}_0 + \dotsb + a_m \lambda_0^m \vb{x}_0 \\
	&= (a_0 + a_1 \lambda_0 + a_2 \lambda_0^2 + \dotsb + a_m \lambda_0^m) \vb{x}_0 \\
	&= f(\lambda_0) \vb{x}_0.
	\qedhere
\end{align*}
\end{proof}
\end{example}
\begin{example}
%@see: 《高等代数(第三版 上册)》(丘维声) P179 习题5.5 3.
设矩阵\(\vb{A} \in M_n(\mathbb{C})\)的各个元素均为实数,
\(\lambda_0\)是\(\vb{A}\)的一个特征值,
\(\vb{x}_0\)是\(\vb{A}\)的属于特征值\(\lambda_0\)的一个特征向量.
证明:\(\ComplexConjugate{\lambda_0}\)也是\(\vb{A}\)的一个特征值,
\(\ComplexConjugate{\vb{x}_0}\)是\(\vb{A}\)的属于特征值\(\ComplexConjugate{\lambda_0}\)的一个特征向量.
\begin{proof}
根据定义有\(\ComplexConjugate{\vb{A}} = \vb{A}\)和\(\vb{A} \vb{x}_0 = \lambda_0 \vb{x}_0\),
那么\begin{equation*}
	\vb{A}~\ComplexConjugate{\vb{x}_0}
	= \ComplexConjugate{\vb{A}~\vb{x}_0}
	= \ComplexConjugate{\lambda_0~\vb{x}_0}
	= \ComplexConjugate{\lambda_0}~\ComplexConjugate{\vb{x}_0}.
	\qedhere
\end{equation*}
\end{proof}
\end{example}

\begin{example}\label{example:特征值与特征向量.各行元素之和相同的矩阵的伴随矩阵的特征值与特征向量}
设矩阵\(\vb{A} \in M_n(K)\),
\(\vb{A}\)的各行元素之和均为\(\lambda_0\).
证明:\(\vb{A}\)的伴随矩阵\(\vb{A}^*\)的各行元素之和等于\(\lambda_0^{-1} \abs{\vb{A}}\).
\begin{proof}
由\cref{example:特征值与特征向量.各行元素之和相同的矩阵的特征值与特征向量} 可知,
\(\lambda_0\)是矩阵\(\vb{A}\)的特征值,
\(n\)维列向量\(\vb{x}_0=(1,\dotsc,1)^T\)是矩阵\(\vb{A}\)属于\(\lambda_0\)的特征向量.
那么由\cref{example:矩阵的特征值与特征向量.伴随矩阵的特征值} 可知,
\(\lambda_0^{-1} \abs{\vb{A}}\)是\(\vb{A}^*\)的一个特征值,
\(\vb{x}_0\)是\(\vb{A}^*\)的属于\(\lambda_0^{-1} \abs{\vb{A}}\)的一个特征向量,
即\begin{equation*}
	\vb{A}^*
	\begin{bmatrix}
		1 \\ 1 \\ \vdots \\ 1
	\end{bmatrix}
	= \frac{\abs{\vb{A}}}{\lambda_0}
	\begin{bmatrix}
		1 \\ 1 \\ \vdots \\ 1
	\end{bmatrix}.
\end{equation*}
又因为\begin{equation*}
	\vb{A}^*
	\begin{bmatrix}
		1 \\ 1 \\ \vdots \\ 1
	\end{bmatrix}
	= \begin{bmatrix}
		A_{11} & A_{21} & \dots & A_{n1} \\
		A_{12} & A_{22} & \dots & A_{n2} \\
		\vdots & \vdots & & \vdots \\
		A_{1n} & A_{2n} & \dots & A_{nn}
	\end{bmatrix}
	\begin{bmatrix}
		1 \\ 1 \\ \vdots \\ 1
	\end{bmatrix}
	= \begin{bmatrix}
		A_{11} + A_{21} + \dotsb + A_{n1} \\
		A_{12} + A_{22} + \dotsb + A_{n2} \\
		\vdots \\
		A_{1n} + A_{2n} + \dotsb + A_{nn}
	\end{bmatrix},
\end{equation*}
所以\begin{equation*}
	\begin{bmatrix}
		A_{11} + A_{21} + \dotsb + A_{n1} \\
		A_{12} + A_{22} + \dotsb + A_{n2} \\
		\vdots \\
		A_{1n} + A_{2n} + \dotsb + A_{nn}
	\end{bmatrix}
	= \frac{\abs{\vb{A}}}{\lambda_0}
	\begin{bmatrix}
		1 \\ 1 \\ \vdots \\ 1
	\end{bmatrix}
	= \begin{bmatrix}
		\lambda_0^{-1} \abs{\vb{A}} \\ \lambda_0^{-1} \abs{\vb{A}} \\ \vdots \\ \lambda_0^{-1} \abs{\vb{A}}
	\end{bmatrix}.
	\qedhere
\end{equation*}
\end{proof}
\end{example}

\subsection{特征矩阵,特征多项式,特征方程}
\begin{proposition}
设\(\vb{A} \in M_n(K)\),
\(\AutoTuple{\lambda}{n}\)都是\(\vb{A}\)的特征值,
\(\AutoTuple{\vb{x}}{n}\)分别是\(\vb{A}\)的属于特征值\(\AutoTuple{\lambda}{n}\)的特征向量.
记\(\vb{X} = (\AutoTuple{\vb{x}}{n})\),
则\begin{equation*}
	\vb{A} \vb{X} = \vb{X} \vb{\Lambda},
\end{equation*}
其中\(\vb{\Lambda} = \diag(\AutoTuple{\lambda}{n})\).
\end{proposition}

\begin{definition}
%@see: 《线性代数》(张慎语、周厚隆) P93 定义2
%@see: 《高等代数(第三版 上册)》(丘维声) P173
设矩阵\(\vb{A} \in M_n(K)\),\(\lambda \in K\),
\(\vb{E}\)是数域\(K\)上的\(n\)阶单位矩阵.
把\begin{equation*}
	\lambda\vb{E}-\vb{A}
\end{equation*}称为“\(\vb{A}\)的属于特征值\(\lambda\)的\DefineConcept{特征矩阵}”.
把\(\vb{A}\)的属于特征值\(\lambda\)的特征矩阵的行列式\begin{equation*}
	\abs{\lambda\vb{E}-\vb{A}}
\end{equation*}称为“\(\vb{A}\)的属于特征值\(\lambda\)的\DefineConcept{特征多项式}(characteristic polynomial)”.
%@see: https://mathworld.wolfram.com/CharacteristicPolynomial.html
把关于\(\lambda\)的方程\begin{equation*}
	\abs{\lambda\vb{E}-\vb{A}}=0
\end{equation*}称为“\(\vb{A}\)的属于特征值\(\lambda\)的\DefineConcept{特征方程}(characteristic equation)”.
%@see: https://mathworld.wolfram.com/CharacteristicEquation.html
\end{definition}

\begin{theorem}\label{theorem:矩阵的特征值与特征向量.与特征多项式和特征子空间的联系}
%@see: 《高等代数(第三版 上册)》(丘维声) P174 定理1
%@see: 《线性代数》(张慎语、周厚隆) P93 性质4
设矩阵\(\vb{A} \in M_n(K)\),\(\lambda_0 \in K\),
\(\vb{E}\)是数域\(K\)上的\(n\)阶单位矩阵,
则\begin{align*}
	&\text{\(\lambda_0\)是矩阵\(\vb{A}\)的一个特征值} \\
	&\iff \text{\(\lambda_0\)是矩阵\(\vb{A}\)的特征多项式\(\abs{\lambda\vb{E}-\vb{A}}\)在\(K\)中的一个根} \\
	&\iff \abs{\lambda_0 \vb{E} - \vb{A}} = 0, \\
	&\text{\(\vb{x}_0\)是矩阵\(\vb{A}\)的属于特征值\(\lambda_0\)的一个特征向量} \\
	&\iff \text{\(\vb{x}_0\)是齐次线性方程组\((\lambda_0\vb{E}-\vb{A})\vb{x}=\vb0\)的一个非零解} \\
	&\iff \vb{x}_0 \in \Ker(\lambda_0 \vb{E} - \vb{A}) - \{\vb0\}.
\end{align*}
%TODO proof
\end{theorem}

\begin{example}\label{example:特征值与特征向量.交换矩阵乘积后非零特征值不变}
%@see: 《高等代数(第三版 上册)》(丘维声) P180 习题5.5 13.
设\(\vb{A} \in M_{s \times n}(K),
\vb{B} \in M_{n \times s}(K)\).
证明:\(\vb{A} \vb{B}\)与\(\vb{B} \vb{A}\)有相同的非零特征值.
\begin{proof}
\begin{proof}[证法一]
%@see: 《高等代数(第三版 上册)》(丘维声) P260 习题5.5 13.
假设\(\lambda\)是\(\vb{A} \vb{B}\)的一个非零特征值,
即存在非零向量\(\vb\alpha\),
使得\begin{equation*}
	(\vb{A} \vb{B}) \vb\alpha = \lambda \vb\alpha.
\end{equation*}
左乘\(B\),得\begin{equation*}
	(\vb{B} \vb{A}) (\vb{B} \vb\alpha) = \lambda (B \vb\alpha).
\end{equation*}
因此\(\lambda\)也是\(\vb{B} \vb{A}\)的非零特征值.
\end{proof}
\begin{proof}[证法二]
%@see: https://www.bilibili.com/video/BV17UDtYrExu/
% 假设\(\lambda=0\),
% 那么\begin{equation*}
% 	\abs{\lambda \vb{E}_s - \vb{A} \vb{B}}
% 	= \abs{\vb{A} \vb{B}},
% 	\qquad
% 	\abs{\lambda \vb{E}_n - \vb{B} \vb{A}}
% 	= \abs{\vb{B} \vb{A}}.
% \end{equation*}
假设\(\lambda\neq0\),
那么由\hyperref[theorem:行列式.性质2.推论2]{行列式的性质}有\begin{equation*}
	\abs{\lambda \vb{E}_s - \vb{A} \vb{B}}
	= \abs{\lambda (\vb{E}_s - (\lambda^{-1} \vb{A}) \vb{B})}
	= \lambda^s \abs{\vb{E}_s - (\lambda^{-1} \vb{A}) \vb{B}},
\end{equation*}
同理可得\begin{equation*}
	\abs{\lambda \vb{E}_n - \vb{B} \vb{A}}
	= \lambda^n \abs{\vb{E}_n - \vb{B} (\lambda^{-1} \vb{A})}.
\end{equation*}
由\cref{example:逆矩阵.行列式降阶定理的重要应用1} 有\begin{equation*}
	\abs{\vb{E}_s - (\lambda^{-1} \vb{A}) \vb{B}}
	= \abs{\vb{E}_n - \vb{B} (\lambda^{-1} \vb{A})},
\end{equation*}
那么\begin{equation*}
	\lambda^n \abs{\lambda \vb{E}_s - \vb{A} \vb{B}}
	= \lambda^s \abs{\lambda \vb{E}_n - \vb{B} \vb{A}}.
\end{equation*}
于是\(\abs{\lambda \vb{E}_s - \vb{A} \vb{B}} = 0
\iff
\abs{\lambda \vb{E}_n - \vb{B} \vb{A}} = 0\),
由\cref{theorem:矩阵的特征值与特征向量.与特征多项式和特征子空间的联系} 可知\begin{equation*}
	\text{\(\lambda\)是\(\vb{A} \vb{B}\)的非零特征值}
	\iff
	\text{\(\lambda\)是\(\vb{B} \vb{A}\)的非零特征值},
\end{equation*}
即\(\vb{A} \vb{B}\)与\(\vb{B} \vb{A}\)有相同的非零特征值,且它们的代数重数相同.
\end{proof}\let\qed\relax
\end{proof}
%\cref{example:单位矩阵与两矩阵乘积之差.单位矩阵与两矩阵乘积之差的行列式}
\end{example}
\begin{example}\label{example:单位矩阵与两矩阵乘积之差.单位矩阵与两矩阵乘积之差的行列式}
设\(\vb{A} \in M_{m \times n}(K),
\vb{B} \in M_{n \times m}(K)\),
且\(m \geq n\).
求证:\begin{equation}
	\abs{\lambda\vb{E}_m-\vb{A}\vb{B}} = \lambda^{m-n} \abs{\lambda\vb{E}_n-\vb{B}\vb{A}}.
\end{equation}
\begin{proof}
当\(\lambda\neq0\)时,考虑下列分块矩阵:\begin{equation*}
	\begin{bmatrix}
		\lambda\vb{E}_m & \vb{A} \\
		\vb{B} & \vb{E}_n
	\end{bmatrix}.
\end{equation*}
因为\(\lambda\vb{E}_m,\vb{E}_n\)都是可逆矩阵,
故由\hyperref[theorem:逆矩阵.行列式降阶定理]{降阶公式}可得\begin{equation*}
	\abs{\vb{E}_n} \cdot \abs{\lambda\vb{E}_m - \vb{A} (\vb{E}_n)^{-1} \vb{B}}
	= \abs{\lambda\vb{E}_m} \cdot \abs{\vb{E}_n - \vb{B} (\lambda\vb{E}_m)^{-1} \vb{A}},
\end{equation*}
即有\(\abs{\lambda\vb{E}_m-\vb{A}\vb{B}} = \lambda^{m-n} \abs{\lambda\vb{E}_n-\vb{B}\vb{A}}\)成立.

当\(\lambda=0\)时,若\(m>n\),则\begin{equation*}
	\rank(\vb{A}\vb{B}) \leq \min\{\rank\vb{A},\rank\vb{B}\} \leq \min\{m,n\} < m,
\end{equation*}
故\(\abs{-\vb{A}\vb{B}}=0\),结论也成立;
若\(m = n\),则由\cref{theorem:行列式.矩阵乘积的行列式} 可知结论也成立.

事实上,\(\lambda=0\)的情形也可通过摄动法由\(\lambda\neq0\)的情形来得到.
\end{proof}
%\cref{example:逆矩阵.行列式降阶定理的重要应用1}
%\cref{example:单位矩阵与两矩阵乘积之差.单位矩阵与两矩阵乘积之差的秩}
% 本例还有其他证法:
% 你可以将\(\vb{A}\)化为等价标准型来证明,
% 或先证\(\vb{A}\)非异的情形,再用摄动法进行讨论.
\end{example}

\begin{example}
%@see: https://www.bilibili.com/video/BV1vGmuYkEt3/
设\(\vb{A}\)是数域\(K\)上的2阶矩阵,
且\(\vb{A}^2 \vb\alpha - (p + q) \vb{A} \vb\alpha + p q \vb\alpha = \vb0\).
求\(\vb{A}\)的特征值与对应的特征向量.
\begin{solution}
由题意有\begin{equation*}
	(\vb{A} - p \vb{E})(\vb{A} - q \vb{E}) \vb\alpha = \vb0.
	\eqno(1)
\end{equation*}
% 矩阵\((\vb{A} - p \vb{E})\)与\((\vb{A} - q \vb{E})\)可交换
根据定义有\(0\)是\(\vb{A} - p \vb{E}\)的特征值,也是\(\vb{A} - q \vb{E}\)的特征值,
那么\begin{equation*}
	\abs{\vb{A} - p \vb{E}}
	= \abs{\vb{A} - q \vb{E}}
	= 0;
\end{equation*}
由\cref{theorem:矩阵的特征值与特征向量.与特征多项式和特征子空间的联系}
可知\(p\)和\(q\)都是\(\vb{A}\)的特征值.
又因为\begin{gather*}
	(\vb{A} - p \vb{E})(\vb{A} - q \vb{E}) \vb\alpha
	= \vb{A} (\vb{A} - q \vb{E}) \vb\alpha - p (\vb{A} - q \vb{E}) \vb\alpha
	= \vb0, \\
	\vb{A} (\vb{A} - q \vb{E}) \vb\alpha = p (\vb{A} - q \vb{E}) \vb\alpha,
\end{gather*}
所以\((\vb{A} - q \vb{E}) \vb\alpha\)是\(\vb{A}\)的属于\(p\)的特征向量.
同理可知\((\vb{A} - p \vb{E}) \vb\alpha\)是\(\vb{A}\)的属于\(q\)的特征向量.
\end{solution}
\end{example}
\begin{remark}
在上例中,可以根据等式\(\vb{A}^2 \vb\alpha - (p + q) \vb{A} \vb\alpha + p q \vb\alpha = \vb0\)
写出\begin{equation*}
	\vb{A} (\vb\alpha,\vb{A} \vb\alpha)
	= (\vb{A} \vb\alpha,\vb{A}^2 \vb\alpha)
	= (\vb{A} \vb\alpha,(p+q) \vb{A} \vb\alpha - p q \vb\alpha)
	= (\vb\alpha,\vb{A} \vb\alpha)
	\begin{bmatrix}
		0 & -p q \\
		1 & p+q
	\end{bmatrix}.
\end{equation*}
\end{remark}

\begin{example}
%@see: 《高等代数(第三版 上册)》(丘维声) P179 习题5.5 7.
%@see: 《线性代数》(张慎语、周厚隆) P96 例7
设\(\vb{A} \in M_n(K)\),
证明:\begin{equation*}
	\text{\(\lambda\)是\(\vb{A}\)的特征值}
	\iff
	\text{\(\lambda\)是\(\vb{A}^T\)的特征值}.
\end{equation*}
\begin{proof}
由\cref{theorem:行列式.性质1,theorem:矩阵的转置.性质1,theorem:矩阵的转置.性质2,theorem:矩阵的转置.性质3}
有\begin{equation*}
	\abs{\lambda\vb{E}-\vb{A}^T}
	= \abs{(\lambda\vb{E}-\vb{A}^T)^T}
	= \abs{(\lambda\vb{E})^T-(\vb{A}^T)^T}
	= \abs{\lambda\vb{E}-\vb{A}},
\end{equation*}
因此\(\vb{A}\)与\(\vb{A}^T\)具有相同的特征值.
\end{proof}
\end{example}
% \begin{example}
% 举例说明:矩阵\(\vb{A} \in M_n(K)\)与它的转置矩阵\(\vb{A}^T\)的特征向量不同.
% %TODO
% \end{example}
\begin{example}
试讨论:在什么条件下,一个矩阵的任一特征向量总是它的转置矩阵的特征向量.
\begin{solution}
设\(\vb{A} \in M_n(K)\).
假设\(\vb{A}\)的特征向量都是\(\vb{A}^T\)的特征向量,
即\begin{equation*}
	(\forall\lambda_1 \in K)
	(\forall\vb{x} \in K^n)
	[
		\vb{A} \vb{x} = \lambda_1 \vb{x}
		\implies
		\vb{A}^T \vb{x} = \lambda_2 \vb{x}
	],
\end{equation*}
那么有\begin{equation*}
	(\vb{A}^T\vb{x})^T=(\lambda_2\vb{x})^T
	\implies
	\vb{x}^T\vb{A}=\lambda_2\vb{x}^T
	\implies
	(\vb{x}^T\vb{A})\vb{x}=(\lambda_2\vb{x}^T)\vb{x},
\end{equation*}\begin{equation*}
	\vb{x}^T(\vb{A}\vb{x})=\vb{x}^T(\lambda_1\vb{x})=\lambda_1\vb{x}^T\vb{x}=\lambda_2\vb{x}^T\vb{x}
	\implies
	(\lambda_1-\lambda_2)\vb{x}^T\vb{x}=0,
\end{equation*}
因为\(\vb{x}\neq\vb0\),\(\vb{x}^T\vb{x}\neq0\),
所以\(\lambda_1-\lambda_2=0\),\(\lambda_1=\lambda_2\).
那么\begin{equation*}
	\vb{A}\vb{x}=\lambda_1\vb{x}=\lambda_2\vb{x}=\vb{A}^T\vb{x},
\end{equation*}
从而\((\vb{A}-\vb{A}^T)\vb{x}=\vb0\),\(\vb{A}=\vb{A}^T\).
也就是说,当且仅当\(\vb{A}\)是对称矩阵时,\(\vb{A}\)的特征向量都是\(\vb{A}^T\)的特征向量.
\end{solution}
\end{example}

\begin{example}
%@see: 《高等代数(第三版 上册)》(丘维声) P175 例2
设\begin{equation*}
	\vb{A} = \begin{bmatrix}
		2 & -2 & 2 \\
		-2 & -1 & 4 \\
		2 & 4 & -1
	\end{bmatrix}
\end{equation*}是数域\(K\)上的矩阵,
求\(\vb{A}\)的全部特征值和特征向量.
\begin{solution}
因为\begin{equation*}
	\abs{\lambda\vb{E}-\vb{A}}
	= \begin{vmatrix}
		\lambda-2 & 2 & -2 \\
		2 & \lambda+1 & -4 \\
		-2 & -4 & \lambda+1
	\end{vmatrix}
	= (\lambda-3)^2 (\lambda+6),
\end{equation*}
所以\(\vb{A}\)的全部特征值是
\(3\ (\text{二重}),-6\).

对于特征值\(3\),解齐次线性方程组\((3\vb{E}-\vb{A}) \vb{x} = \vb0\):\begin{equation*}
	\begin{bmatrix}
		1 & 2 & -2 \\
		2 & 4 & -4 \\
		-2 & -4 & 4
	\end{bmatrix}
	\to \begin{bmatrix}
		1 & 2 -2 \\
		0 & 0 & 0 \\
		0 & 0 & 0
	\end{bmatrix},
\end{equation*}
它的基础解系为\(\vb{x}_1 = (-2,1,0)^T,
\vb{x}_2 = (2,0,1)^T\).
因此\(\vb{A}\)的属于\(3\)的全部特征向量是\begin{equation*}
	\Set{
		k_1 \vb{x}_1 + k_2 \vb{x}_2
		\given
		\text{$k_1,k_2 \in K$且$k_1,k_2$不全为$0$}
	}.
\end{equation*}

对于特征值\(-6\),解齐次线性方程组\((-6\vb{E}-\vb{A}) \vb{x} = \vb0\)
得它的基础解系为\(\vb{x}_3 = (1,2,-2)^T\).
因此\(\vb{A}\)的属于\(-6\)的全部特征向量是\begin{equation*}
	\Set{
		k_3 \vb{x}_3
		\given
		k_3 \in K-\{0\}
	}.
\end{equation*}
\end{solution}
\end{example}

\begin{example}
%@see: 《2025年全国硕士研究生入学统一考试(数学一)》三解答题/第21题
设矩阵\(\vb{A} = \begin{bmatrix}
	0 & -1 & 2 \\
	-1 & 0 & 2 \\
	-1 & -1 & a
\end{bmatrix}\),
且\(1\)是\(\vb{A}\)的特征多项式的重根.
\begin{itemize}
	\item 求\(a\)的值.
	\item 求所有满足\(\vb{A} \vb\alpha = \vb\alpha + \vb\beta,
	\vb{A}^2 \vb\alpha = \vb\alpha + 2\vb\beta\)的非零列向量\(\vb\alpha,\vb\beta\).
\end{itemize}
\begin{solution}
由题意有\begin{equation*}
	\abs{\lambda\vb{E}-\vb{A}}
	= \begin{vmatrix}
		\lambda & 1 & -2 \\
		1 & \lambda & -2 \\
		1 & 1 & \lambda-a
	\end{vmatrix}
	%@Mathematica: -CharacteristicPolynomial[{{0, -1, 2}, {-1, 0, 2}, {-1, -1, a}}, \[Lambda]] // Factor
	= (\lambda-1)[\lambda^2+(1-a)\lambda+(4-a)].
\end{equation*}
因为\(1\)是\(\abs{\lambda\vb{E}-\vb{A}}\)的重根,
所以\(1\)是\(\lambda^2+(1-a)\lambda+(4-a)\)的根,
即\(1^2+(1-a)\cdot1+(4-a)
= 2a-6
= 0\),
解得\(a=3\).

由\(\vb{A} \vb\alpha = \vb\alpha + \vb\beta,
\vb{A}^2 \vb\alpha = \vb\alpha + 2\vb\beta\)
可得\(\vb{A}^2 \vb\alpha - 2 \vb{A} \vb\alpha = -\vb\alpha\),
即\((\vb{A}-\vb{E})^2 \vb\alpha = \vb0\).
换言之,\(\vb\alpha\)是方程组\((\vb{A}-\vb{E})^2 \vb{x} = \vb0\)的非零解.
因为\((\vb{A}-\vb{E})^2 = \begin{bmatrix}
	-1 & -1 & 2 \\
	-1 & -1 & 2 \\
	-1 & -1 & 2
\end{bmatrix}^2
= \vb0\),
所以\(\vb\alpha\)可以是任意非零列向量,
即\(\vb\alpha = (k_1,k_2,k_3)^T\),
其中\(k_1,k_2,k_3\)是待定常数.

由\(\vb{A} \vb\alpha = \vb\alpha + \vb\beta\)
可得\begin{equation*}
	\vb\beta = (\vb{A}-\vb{E}) \vb\alpha
	= \begin{bmatrix}
		-1 & -1 & 2 \\
		-1 & -1 & 2 \\
		-1 & -1 & 2
	\end{bmatrix}
	\begin{bmatrix}
		k_1 \\ k_2 \\ k_3
	\end{bmatrix}
	= (2k_3 - k_2 - k_1)
	(1,1,1)^T,
\end{equation*}
其中\(k_1 + k_2 \neq 2 k_3\),且\(k_1,k_2,k_3\)不全为零.
\end{solution}
\end{example}

\subsection{特征子空间}
%\cref{theorem:矩阵的特征值与特征向量.特征子空间1,theorem:矩阵的特征值与特征向量.特征子空间2}
既然\(\vb{A}\)的属于\(\lambda_0\)的特征向量\(\vb{x}_0\)
就是齐次线性方程组\((\lambda_0\vb{E}-\vb{A})\vb{x}=\vb0\)的非零解,
那么由\cref{theorem:线性方程组.齐次线性方程组的解的线性组合也是解} 可知,
\(\vb{A}\)的属于同一个特征值\(\lambda_0\)的
特征向量\(\AutoTuple{\vb{x}}{t}\)的非零线性组合
\(\sum_{i=1}^t k_i \vb{x}_i\)
也是\(\vb{A}\)的属于\(\lambda_0\)的特征向量.
于是可知,齐次线性方程组\((\lambda_0\vb{E}-\vb{A})\vb{x}=\vb0\)的解集是\(K^n\)的子空间.

\begin{definition}
%@see: 《高等代数(第三版 上册)》(丘维声) P174
设\(\lambda \in K\)是矩阵\(\vb{A} \in M_n(K)\)的一个特征值,
我们把齐次线性方程\begin{equation*}
	(\lambda\vb{E}-\vb{A})\vb{x}=\vb0
\end{equation*}的解空间\(\Ker(\lambda\vb{E}-\vb{A})\)
称为“矩阵\(\vb{A}\)的属于特征值\(\lambda\)的\DefineConcept{特征子空间}(eigenspace)”.
%@see: https://mathworld.wolfram.com/Eigenspace.html
\end{definition}

\begin{proposition}
%@see: 《高等代数(第三版 上册)》(丘维声) P174
矩阵\(\vb{A}\)的属于特征值\(\lambda\)的特征子空间的全体非零向量
恰是矩阵\(\vb{A}\)的属于特征值\(\lambda\)的全体特征向量.
\end{proposition}

\begin{proposition}
%@see: 《高等代数(第三版 上册)》(丘维声) P177 命题2
%@see: 《线性代数》(张慎语、周厚隆) P97 习题5.1 5.
设\(\vb{A} \in M_n(K)\),
则\(\vb{A}\)的特征多项式\(\abs{\lambda\vb{E}-\vb{A}}\)是\(\lambda\)的\(n\)次多项式,
\(\lambda^n\)的系数是\(1\),
\(\lambda^{n-1}\)的系数等于\(-\tr\vb{A}\),
常数项为\((-1)^n \abs{\vb{A}}\),
\(\lambda^{n-k}\ (1\leq k<n)\)的系数是\begin{equation*}
	(-1)^k \sum_{1 \leq i_1 < \dotsb < i_k \leq n} \MatrixMinor{\vb{A}}{
		i_1,i_2,\dotsc,i_k \\
		i_1,i_2,\dotsc,i_k
	}.
\end{equation*}
\begin{proof}
\(\vb{A}\)的特征多项式为\begin{equation*}
	\abs{\lambda\vb{E}-\vb{A}}
	= \begin{vmatrix}
		\lambda-a_{11} & 0-a_{12} & \dots & 0-a_{1n} \\
		0-a_{21} & \lambda-a_{22} & \dots & 0-a_{2n} \\
		\vdots & \vdots && \vdots \\
		0-a_{n1} & 0-a_{n2} & \dots & \lambda-a_{nn}
	\end{vmatrix}.
\end{equation*}
由\cref{theorem:行列式.性质3} 可知,
\(\vb{A}\)的特征多项式等于\(2^n\)个行列式之和,
其中两个是\begin{equation*}
	\begin{vmatrix}
		\lambda & 0 & \dots & 0 \\
		0 & \lambda & \dots & 0 \\
		\vdots & \vdots && \vdots \\
		0 & 0 & \dots & \lambda
	\end{vmatrix}
	= \lambda^n,
	\qquad
	\begin{vmatrix}
		-a_{11} & -a_{12} & \dots & -a_{1n} \\
		-a_{21} & -a_{22} & \dots & -a_{2n} \\
		\vdots & \vdots && \vdots \\
		-a_{n1} & -a_{n2} & \dots & -a_{nn}
	\end{vmatrix}
	= (-1)^n \abs{\vb{A}},
\end{equation*}
其余行列式是这样的行列式 --- 它的第\(j_1,j_2,\dotsc,j_{n-k}\)列是含有\(\lambda\)的列,
其余列是不含\(\lambda\)的列(它们是\(-\vb{A}\)的列):\begin{equation*}
	\begin{vmatrix}
		-a_{11} & \dots & 0 & \dots & 0 & \dots & 0 & \dots & -a_{1n} \\
		\vdots & & \vdots & & \vdots & & \vdots & & \vdots \\
		-a_{j_1 1} & \dots & \lambda & \dots & 0 & \dots & 0 & \dots & -a_{j_1 n} \\
		\vdots & & \vdots & & \vdots & & \vdots & & \vdots \\
		-a_{j_2 1} & \dots & 0 & \dots & \lambda & \dots & 0 & \dots & -a_{j_2 n} \\
		\vdots & & \vdots & & \vdots & & \vdots & & \vdots \\
		-a_{j_{n-k} 1} & \dots & 0 & \dots & 0 & \dots & \lambda & \dots & -a_{j_{n-k} n} \\
		\vdots & & \vdots & & \vdots & & \vdots & & \vdots \\
		-a_{n1} & \dots & 0 & \dots & 0 & \dots & 0 & \dots & -a_{nn} \\
	\end{vmatrix}.
	\eqno(1)
\end{equation*}
利用\hyperref[theorem:行列式.拉普拉斯定理]{拉普拉斯定理},
按第\(j_1,j_2,\dotsc,j_{n-k}\)列展开(1)式,
这\(n-k\)列元素组成的\(n-k\)阶子式只有一个不为零,
其余\(n-k\)阶子式全为零(因为它们都含有元素全为零的行).
令\begin{equation*}
	\Set{ j_1',j_2',\dotsc,j_k' }
	= \Set{ 1,2,\dotsc,n } - \Set{ j_1,j_2,\dotsc,j_{n-k} },
\end{equation*}
且\(j_1'<j_2'<\dotsb<j_k'\),
则(1)式等于\begin{align*}
	&\hspace{-20pt}
	\begin{vmatrix}
		\lambda & 0 & \dots & 0 \\
		0 & \lambda & \dots & 0 \\
		\vdots & \vdots & & \vdots \\
		0 & 0 & \dots & \lambda
	\end{vmatrix}
	(-1)^{(j_1+j_2+\dotsb+j_{n-k})+(j_1+j_2+\dotsb+j_{n-k})}
	\MatrixMinor{(-\vb{A})}{
		j_1',j_2',\dotsc,j_k' \\
		j_1',j_2',\dotsc,j_k'
	} \\
	&= \lambda^{n-k} (-1)^k
	\MatrixMinor{\vb{A}}{
		j_1',j_2',\dotsc,j_k' \\
		j_1',j_2',\dotsc,j_k'
	}.
\end{align*}
由于\(1 \leq j_1' < j_2' < \dotsb < j_k' \leq n\),
因此\(\abs{\lambda\vb{E}-\vb{A}}\)中\(\lambda^{n-k}\)的系数就是
\(\vb{A}\)的所有\(k\)阶主子式之和与\((-1)^k\)的乘积\begin{equation*}
	(-1)^k
	\sum_{1 \leq j_1' < j_2' < \dotsb < j_k' \leq n}
	\MatrixMinor{\vb{A}}{
		j_1',j_2',\dotsc,j_k' \\
		j_1',j_2',\dotsc,j_k'
	}.
	\qedhere
\end{equation*}
\end{proof}
\end{proposition}

\begin{example}
%@see: https://www.bilibili.com/video/BV1tz4y177ys/
设矩阵\(\vb{A} \in M_3(K)\),
\(\vb{E}\)是数域\(K\)上的3阶单位矩阵,
且\begin{equation*}
	\abs{\vb{A}-2\vb{E}}
	= \abs{\vb{A}-3\vb{E}}
	= \abs{\vb{A}-4\vb{E}}
	= 3.
\end{equation*}
求\(\abs{\vb{A}-\vb{E}}\).
\begin{solution}
记\(f(\lambda) = \abs{\vb{A}-\lambda\vb{E}}\),
那么由题意有\begin{equation*}
	f(2) = f(3) = f(4) = 3,
\end{equation*}
由于\(f(\lambda)\)的变量\(\lambda\)的最高次项\(\lambda^3\)的系数是\(-1\),
所以\begin{equation*}
	f(\lambda) = 3 - (\lambda-2)(\lambda-3)(\lambda-4).
\end{equation*}
用\(1\)代\(\lambda\)得\begin{equation*}
	\abs{\vb{A}-\vb{E}}
	= f(1)
	= 3 - (1-2)(1-3)(1-4)
	= 9.
\end{equation*}
\end{solution}
\end{example}
\begin{example}
%@see: https://www.bilibili.com/video/BV1qQymYUEBj/
设矩阵\(\vb{A} \in M_3(K)\),
\(\vb{E}\)是数域\(K\)上的3阶单位矩阵,
且\begin{equation*}
	\abs{\vb{E}-\vb{A}} = 1,
	\qquad
	\abs{-\vb{E}-\vb{A}} = -1,
	\qquad
	\abs{2\vb{E}-\vb{A}} = 2.
\end{equation*}
求\(\abs{3\vb{E}-\vb{A}}\).
\begin{solution}
记\(f(\lambda) = \abs{\lambda\vb{E}-\vb{A}}\),
那么由题意有\(\lambda=-1,1,2\)是方程\begin{equation*}
	f(\lambda) = \lambda
\end{equation*}的根,
由于\(f(\lambda)\)的变量\(\lambda\)的最高次项\(\lambda^3\)的系数是\(+1\),
所以\begin{equation*}
	f(\lambda) - \lambda
	= (\lambda+1)(\lambda-1)(\lambda-2).
\end{equation*}
用\(3\)代\(\lambda\)得\begin{equation*}
	\abs{3\vb{E}-\vb{A}} - 3
	= (3+1)(3-1)(3-2),
\end{equation*}
整理得\(\abs{3\vb{E}-\vb{A}} = 11\).
\end{solution}
\end{example}

\subsection{求解特征值和特征向量的一般程序}
% 大数学家高斯在1799年证明了以下\DefineConcept{代数基本定理}:
\begin{lemma}[代数基本定理]
任何\(n\ (n>0)\)次多项式有且仅有\(n\)个复根,其中规定\(m\)重根算\(m\)个根.
\end{lemma}
我们将会在\cref{example:留数定理.利用儒歇定理证明代数基本定理} 给出代数基本定理的具体证明过程.

%@see: 《线性代数》(张慎语、周厚隆) P93
求解矩阵\(\vb{A} \in M_n(K)\)的特征值和特征向量的一般程序:\begin{enumerate}
	\item 计算特征多项式\(\abs{\lambda\vb{E}-\vb{A}}\).

	\item 判别多项式\(\abs{\lambda\vb{E}-\vb{A}}\)(即方程\(\abs{\lambda\vb{E}-\vb{A}}=0\))
	在数域\(K\)中有没有根.
	\begin{itemize}
		\item 如果它没有根,则没有特征值、特征向量.
		\item 如果它有根,则它在\(K\)中的全部根\(\AutoTuple{\lambda}{n}\)就是\(\vb{A}\)的全部特征值.
		\item 如果\(K = \mathbb{C}\),
		那么根据代数基本定理,任意\(n\)阶矩阵的特征多项式有且仅有\(n\)个复根.
	\end{itemize}

	\item 对于每个不同的特征值\(\lambda_j\ (j=1,2,\dotsc,n)\),
	求出齐次线性方程组\((\lambda_j \vb{E} - \vb{A})\vb{x} = \vb0\)的一个基础解系\(\AutoTuple{\vb{x}}{t}\),
	则\(\vb{A}\)的属于\(\lambda_j\)的全部特征向量为\(k_1 \vb{x}_1 + k_2 \vb{x}_2 + \dotsb + k_t \vb{x}_t\)
	(其中\(\AutoTuple{k}{t}\)是不全为零的任意常数).
\end{enumerate}

\begin{example}
%@see: 《线性代数》(张慎语、周厚隆) P94 例3
求实数域上的3阶矩阵\begin{equation*}
	\vb{A} = \begin{bmatrix}
		2 & -1 & 2 \\
		5 & -3 & 3 \\
		-1 & 0 & -2
	\end{bmatrix}
\end{equation*}的特征值与对应的特征向量.
\begin{solution}
\(\vb{A}\)的特征多项式\begin{align*}
	\abs{\lambda\vb{E}-\vb{A}}
	&= \begin{vmatrix}
		\lambda-2 & 1 & -2 \\
		-5 & \lambda+2 & -3 \\
		1 & 0 & \lambda+2
	\end{vmatrix} \\
	&= \lambda^3 + 3\lambda^2 + 3\lambda + 1
	= (\lambda+1)^3,
\end{align*}
%@Mathematica: CharacteristicPolynomial[{{2, -1, 2}, {5, -3, 3}, {-1, 0, -2}}, x]
特征值为\(\lambda=-1\ (\text{三重})\).
解方程组\((-\vb{E}-\vb{A})\vb{x} = \vb0\),对系数矩阵施行初等行变换:\begin{equation*}
	-\vb{E}-\vb{A} = \begin{bmatrix}
		-3 & 1 & -2 \\
		-5 & 2 & -3 \\
		1 & 0 & 1
	\end{bmatrix} \to \begin{bmatrix}
		1 & 0 & 1 \\
		5 & 2 & -3 \\
		-3 & 1 & -2
	\end{bmatrix} \to \begin{bmatrix}
		1 & 0 & 1 \\
		0 & 1 & 1 \\
		0 & 0 & 0
	\end{bmatrix}.
\end{equation*}
可知\(\rank(-\vb{E}-\vb{A}) = 2\),
\(\dim\Ker(-\vb{E}-\vb{A}) = 1\).
令\(x_3 = -1\),得基础解系\begin{equation*}
	\vb{x}_1 = (1,1,-1)^T,
\end{equation*}
那么属于\(-1\)的全部特征向量为\(k (1,1,-1)^T\),
\(k\)是非零的任意常数.
\end{solution}
\end{example}

\begin{example}
求实数域上的3阶矩阵\begin{equation*}
	\vb{A} = \begin{bmatrix}
		-1 & 0 & 0 \\
		8 & 2 & 4 \\
		8 & 3 & 3
	\end{bmatrix}
\end{equation*}的特征值与对应的特征向量.
\begin{solution}
\(\vb{A}\)的特征多项式\begin{equation*}
	\abs{\lambda\vb{E}-\vb{A}}
	= \begin{vmatrix}
		\lambda+1 & 0 & 0 \\
		-8 & \lambda-2 & -4 \\
		-8 & -3 & \lambda-3
	\end{vmatrix}
	= (\lambda+1)^2 (\lambda-6),
\end{equation*}
故\(\vb{A}\)的特征值为\(\lambda_1=-1\ (\text{二重}),
\lambda_2=6\).

当\(\lambda=-1\)时,解方程组\((-\vb{E}-\vb{A})\vb{x} = \vb0\),\begin{equation*}
	-\vb{E}-\vb{A} = \begin{bmatrix}
		0 & 0 & 0 \\
		-8 & -3 & -4 \\
		-8 & -3 & -4
	\end{bmatrix} \to \begin{bmatrix}
		-8 & -3 & -4 \\
		0 & 0 & 0 \\
		0 & 0 & 0
	\end{bmatrix},
\end{equation*}
可知\(\rank(-\vb{E}-\vb{A}) = 1\),
方程组\((-\vb{E}-\vb{A})\vb{x} = \vb0\)的解空间有2个基向量.
分别令\(x_2 = 8, x_3 = 0\)和\(x_2 = 0, x_3 = 2\),得基础解系\begin{equation*}
	\vb{x}_1 = \begin{bmatrix} -3 \\ 8 \\ 0 \end{bmatrix},
	\qquad
	\vb{x}_2 = \begin{bmatrix} -1 \\ 0 \\ 2 \end{bmatrix},
\end{equation*}
故属于\(-1\)的全部特征向量为\(k_1 \vb{x}_1 + k_2 \vb{x}_2\),
\(k_1,k_2\)是不全为零的任意常数.

当\(\lambda=6\)时,解方程组\((6\vb{E}-\vb{A})\vb{x} = \vb0\),\begin{equation*}
	6\vb{E}-\vb{A} = \begin{bmatrix}
		7 & 0 & 0 \\
		-8 & 4 & -4 \\
		-8 & -3 & 3
	\end{bmatrix} \to \begin{bmatrix}
		1 & 0 & 0 \\
		0 & 1 & -1 \\
		0 & 0 & 0
	\end{bmatrix},
\end{equation*}
可知\(\rank(6\vb{E}-\vb{A}) = 2\),方程组\((6\vb{E}-\vb{A})\vb{x} = \vb0\)的解空间有1个基向量.
令\(x_3 = 1\),得\(x_1 = 0, x_2 = 1\),基础解系\begin{equation*}
	\vb{x}_3 = (0,1,1)^T,
\end{equation*}
故属于\(6\)的全部特征向量为\(k_3 \vb{x}_3\),\(k_3\)是任意非零常数.
\end{solution}
\end{example}

从特征值、特征向量的性质可以看出,
矩阵\(\vb{A}\)的一个特征值对应若干个线性无关的特征向量;
但反之,一个特征向量只能属于一个特征值.
事实上,设\(\vb{x}_0\)为某个矩阵\(\vb{A}\)的特征向量,若有\(\lambda_1,\lambda_2\)满足\begin{equation*}
	\vb{A}\vb{x}_0=\lambda_1\vb{x}_0,
	\quad
	\vb{A}\vb{x}_0=\lambda_2\vb{x}_0,
\end{equation*}
则必有\(\lambda_1\vb{x}_0=\lambda_2\vb{x}_0\)或\((\lambda_1-\lambda_2)\vb{x}_0=\vb0\),
因为\(\vb{x}_0\neq\vb0\),所以\(\lambda_1-\lambda_2=0\),\(\lambda_1=\lambda_2\).

前面已经知道,
矩阵\(\vb{A}\)的同一个特征值\(\lambda_0\)对应的特征向量的非零线性组合
仍为\(\vb{A}\)的属于\(\lambda_0\)的特征向量.
那么,\(\vb{A}\)的不同特征值对应的特征向量的非零线性组合又如何呢?
\begin{example}
%@see: 《线性代数》(张慎语、周厚隆) P95 例5
设\(\lambda_1,\lambda_2\)是矩阵\(\vb{A}\)的两个不同的特征值,
\(\vb{x}_1,\vb{x}_2\)分别是\(\lambda_1,\lambda_2\)对应的特征向量.
证明:\(\vb{x}_1+\vb{x}_2\)不是\(\vb{A}\)的特征向量.
\begin{proof}
\(\vb{A}\vb{x}_1 = \lambda_1\vb{x}_1\),\(\vb{A}\vb{x}_2 = \lambda_2\vb{x}_2\),
假设\(\vb{A}(\vb{x}_1+\vb{x}_2) = \lambda_0(\vb{x}_1+\vb{x}_2)\),则\begin{equation*}
	\vb{A}\vb{x}_1+\vb{A}\vb{x}_2 =\lambda_1\vb{x}_1+\lambda_2\vb{x}_2 = \lambda_0\vb{x}_1+\lambda_0\vb{x}_2,
\end{equation*}\begin{equation*}
	(\lambda_0-\lambda_1)\vb{x}_1+(\lambda_0-\lambda_2)\vb{x}_2 = \vb0,
\end{equation*}
在上式左右两端同乘\(\vb{A}\)和\(\lambda_1\)可得\begin{equation*}
	\left\{ \begin{array}{l}
		(\lambda_0-\lambda_1)\vb{A}\vb{x}_1+(\lambda_0-\lambda_2)\vb{A}\vb{x}_2 = (\lambda_0-\lambda_1)\lambda_1\vb{x}_1 + (\lambda_0-\lambda_2)\lambda_2\vb{x}_2 = \vb0, \\
		(\lambda_0-\lambda_1)\lambda_1\vb{x}_1+(\lambda_0-\lambda_2)\lambda_1\vb{x}_2 = \vb0,
	\end{array} \right.
\end{equation*}\begin{equation*}
	(\lambda_0-\lambda_2)(\lambda_2-\lambda_1)\vb{x}_2 = \vb0,
\end{equation*}
因为\(\vb{x}_2\neq\vb0\),所以\((\lambda_0-\lambda_2)(\lambda_2-\lambda_1)=0\);
又因为\(\lambda_2\neq\lambda_1\),所以\(\lambda_0=\lambda_2\).
同理有\begin{equation*}
	(\lambda_0-\lambda_1)\lambda_2\vb{x}_1+(\lambda_0-\lambda_2)\lambda_2\vb{x}_2 = \vb0
	\implies
	(\lambda_0-\lambda_1)(\lambda_1-\lambda_2)\vb{x}_1 = \vb0,
\end{equation*}
因为\(\vb{x}_1\neq\vb0\),所以\(\lambda_0=\lambda_1\).

于是导出\(\lambda_1=\lambda_2\),与题设矛盾,说明\(\vb{x}_1+\vb{x}_2\)不是\(\vb{A}\)的特征向量.
\end{proof}
\end{example}

\begin{example}\label{example:幂零矩阵.幂零矩阵的特征值的性质}
%@see: 《高等代数(第三版 上册)》(丘维声) P179 习题5.5 4.
证明:数域\(K\)上的\(n\)阶幂零矩阵的特征值都是\(0\).
\begin{proof}
设\(\vb{A}\)是数域\(K\)上的以正整数\(m\)为幂零指数的\(n\)阶幂零矩阵.
由\cref{example:幂零矩阵.幂零矩阵的行列式} 可知\begin{equation*}
	\abs{0\vb{E}-\vb{A}}
	= \abs{-\vb{A}}
	= (-1)^n \abs{\vb{A}}
	= 0,
\end{equation*}
所以\(0\)是\(\vb{A}\)的一个特征值.
假设\(\lambda_0\)是\(\vb{A}\)的任意一个特征值,
即存在\(\vb{x}_0 \in K^n-\{\vb0\}\)
使得\begin{equation*}
	\vb{A}\vb{x}_0 = \lambda_0\vb{x}_0.
	\eqno(1)
\end{equation*}
在(1)式等号两边同时左乘\(m-1\)次\(\vb{A}\),
便得\begin{equation*}
	\vb{A}^m\vb{x}_0 = \lambda_0^m\vb{x}_0.
	\eqno(2)
\end{equation*}
因为\(\vb{A}^m=\vb0\)且\(\vb{x}_0\neq\vb0\),
所以由(2)式解得\(\lambda_0=0\),
这就说明:\(\vb{A}\)的特征值都是\(0\).
\end{proof}
\end{example}
\begin{remark}
\cref{example:幂零矩阵.幂零矩阵的特征值的性质} 说明:
特征值全为零的矩阵不一定是零矩阵,还可能是幂零矩阵.
\end{remark}

\begin{example}\label{example:幂幺矩阵.幂幺矩阵的特征值的性质}
证明:数域\(K\)上的\(n\)阶幂幺矩阵的特征值都是\(1\).
\begin{proof}
设\(\vb{E}\)是数域\(K\)上的\(n\)阶单位矩阵,
\(\vb{A}\)是数域\(K\)上的以正整数\(m\)为幂幺指数的\(n\)阶幂幺矩阵,
等价成立:\((\vb{A}-\vb{E})\)是数域\(K\)上的以正整数\(m\)为幂零指数的\(n\)阶幂零矩阵.
由\cref{example:幂零矩阵.幂零矩阵的特征值的性质} 可知\((\vb{A}-\vb{E})\)的特征值全为零,
即\(\abs{0\vb{E}-(\vb{A}-\vb{E})}
= \abs{\vb{E}-\vb{A}}
= 0\),
可见\(\vb{A}\)的特征值都是\(1\).
\end{proof}
\end{example}
\begin{remark}
\cref{example:幂幺矩阵.幂幺矩阵的特征值的性质} 说明:
特征值全为一的矩阵不一定是单位矩阵,还可能是幂幺矩阵.
\end{remark}
%credit: {61d1026b-642e-438a-9506-08e3e7865f96} 说:“任意一个特征值全为0的矩阵,要么是零矩阵,要么是幂零矩阵”和“任意一个特征值全为1的矩阵,要么是单位矩阵,要么是幂幺矩阵”成立
%TODO 这两个命题的证明要用到若尔当标准型的构造或哈密顿--凯莱定理.

\begin{example}\label{example:幂等矩阵.幂等矩阵的特征值的性质}
%@see: 《高等代数(第三版 上册)》(丘维声) P179 习题5.5 5.
%@see: 《线性代数》(张慎语、周厚隆) P96 例8
证明:数域\(K\)上的\(n\)阶幂等矩阵一定有特征值,且其特征值必为0或1.
\begin{proof}
设\(\vb{A}\)是幂等矩阵,即有\(\vb{A}^2=\vb{A}\).
设\(\vb{A}\vb{x}_0=\lambda_0\vb{x}_0\ (\vb{x}_0\neq\vb0)\),
则\begin{equation*}
	\vb{A}^2\vb{x}_0
	=\vb{A}(\vb{A}\vb{x}_0)
	=\vb{A}(\lambda_0\vb{x}_0)
	=\lambda_0(\vb{A}\vb{x}_0)
	=\lambda_0(\lambda_0\vb{x}_0)
	=\lambda_0^2\vb{x}_0.
\end{equation*}
因为\begin{equation*}
	\vb{A}^2=\vb{A}
	\iff
	\vb{A}^2-\vb{A}=\vb0
	\implies
	\vb{A}^2\vb{x}_0-\vb{A}\vb{x}_0
	=(\vb{A}^2-\vb{A})\vb{x}_0
	=\vb0\vb{x}_0
	=\vb0,
\end{equation*}
所以\begin{equation*}
	\lambda_0^2\vb{x}_0-\lambda_0\vb{x}_0
	=(\lambda_0^2-\lambda_0)\vb{x}_0
	=\lambda_0(\lambda_0-1)\vb{x}_0
	=\vb0,
\end{equation*}
进一步有\(\lambda_0(\lambda_0-1)=0\),
所以\(\lambda_0=0\)或\(\lambda_0=1\).
\end{proof}
\end{example}

\begin{example}
%@see: 《高等代数(第三版 上册)》(丘维声) P179 习题5.5 6.
设数域\(\mathbb{C}\)上的\(n\)阶方阵\(\vb{A}\)
与数域\(\mathbb{C}\)上的\(n\)阶单位矩阵\(\vb{E}\)
满足\(\vb{A}^m=\vb{E}\),
其中\(m\)是正整数.
证明:\(\vb{A}\)的全体特征值是\(\Set{ z\in\mathbb{R} \given z^m = 1 }\).
%TODO proof
\end{example}

\begin{example}\label{example:矩阵乘积的秩.两个向量的乘积的特征值和特征向量}
设\(\vb\alpha,\vb\beta\)是\(n\)维非零列向量.
证明:求矩阵\(\vb\alpha\vb\beta^T\)的特征值和特征向量.
\begin{proof}
显然有\((\vb\alpha\vb\beta^T)\vb\alpha = \vb\alpha(\vb\beta^T\vb\alpha)\),
那么根据定义可知\(\vb\beta^T\vb\alpha\)就是矩阵\(\vb\alpha\vb\beta^T\)的特征值,
而\(\vb\alpha\)是\(\vb\alpha\vb\beta^T\)属于\(\vb\beta^T\vb\alpha\)的一个特征向量.

又由\cref{example:行列式.两个向量的乘积矩阵的行列式} 可知\(\abs{\vb\alpha\vb\beta^T} = 0\),
所以\(\abs{0\cdot\vb{E}-\vb\alpha\vb\beta^T}=0\),
也就是说\(0\)也是矩阵\(\vb\alpha\vb\beta^T\)的特征值.
再由\cref{example:矩阵乘积的秩.两个向量的乘积的秩} 可知\(\rank(\vb\alpha\vb\beta^T) = 1\),
于是根据\cref{theorem:线性方程组.齐次线性方程组的解向量个数} 可知
\((\vb\alpha\vb\beta^T)\vb{x}=\vb0\)的解空间的维数为\(n-1\),
这就是说\(0\)是矩阵\(\vb\alpha\vb\beta^T\)的\(n-1\)重特征值.
最后,我们来求\(\vb\alpha\vb\beta^T\)属于\(0\)的特征向量,
解方程组\((\vb\alpha\vb\beta^T)\vb{x}=\vb0\),
左乘\(\vb\alpha^T\)得\(\vb\alpha^T\vb\alpha\vb\beta^T\vb{x}=0\),
由于\(\vb\alpha\neq\vb0\),\(\vb\alpha^T\vb\alpha>0\),
消去便得\(\vb\beta^T\vb{x}=0\),
因此由\cref{theorem:线性方程组.同解方程组.特例1} 可知
\(\vb\beta^T\vb{x}=0\)与\((\vb\alpha\vb\beta^T)\vb{x}=\vb0\)同解,
\(\vb\beta^T\vb{x}=0\)的解空间就是\(\vb\alpha\vb\beta^T\)的属于\(0\)的特征子空间.
\end{proof}
\end{example}

\begin{example}
设\(\vb{A} = \begin{bmatrix} 2 & -1 & 2 \\ 5 & -3 & 3 \\ -1 & 0 & -2 \end{bmatrix}\),
求\(\vb{A}\)的特征值与对应的特征向量.
\begin{solution}
\(\vb{A}\)的特征多项式\begin{equation*}
	\abs{\lambda\vb{E}-\vb{A}}
	= \begin{vmatrix}
		\lambda-2 & 1 & -2 \\
		-5 & \lambda+3 & -3 \\
		1 & 0 & \lambda+2
	\end{vmatrix}
	= (\lambda+1)^3,
\end{equation*}
解\(\abs{\lambda\vb{E}-\vb{A}}=0\)得\(\lambda=-1\ (\text{三重})\).

当\(\lambda=-1\)时,解方程组\((-\vb{E}-\vb{A})\vb{x}=\vb0\),\begin{equation*}
	-\vb{E}-\vb{A} = \begin{bmatrix} -3 & 1 & -2 \\ -5 & 2 & -3 \\ 1 & 0 & 1 \end{bmatrix}
	\to \begin{bmatrix} 1 & 0 & 1 \\ 5 & 2 & -3 \\ -3 & 1 & -2 \end{bmatrix}
	\to \begin{bmatrix} 1 & 0 & 1 \\ 0 & 1 & 1 \\ 0 & 0 & 0 \end{bmatrix},
\end{equation*}
\(\rank(-\vb{E}-\vb{A})=2\),
令\(x_3=1\),
得基础解系\begin{equation*}
	\vb{x}_1=\begin{bmatrix} -1 \\ -1 \\ 1 \end{bmatrix},
\end{equation*}
属于\(-1\)的全部特征向量为\(k\vb{x}_1\)(\(k\)为非零的任意常数).
\end{solution}
\end{example}

\begin{example}
设\(\vb{A} = \begin{bmatrix} -1 & 0 & 0 \\ 8 & 2 & 4 \\ 8 & 3 & 3 \end{bmatrix}\),
求\(\vb{A}\)的特征值与对应的特征向量.
\begin{solution}
\(\vb{A}\)的特征多项式\begin{equation*}
	\vb\alpha = \begin{vmatrix}
		\lambda+1 & 0 & 0 \\
		-8 & \lambda-2 & -4 \\
		-8 & -3 & \lambda-3
	\end{vmatrix}
	= (\lambda+1)(\lambda^2-5\lambda-6)
	= (\lambda+1)^2(\lambda-6).
\end{equation*}
令\(\vb\alpha = 0\)可得\(\vb{A}\)的特征值为\(\lambda_1=-1\ (\text{二重})\),\(\lambda_2=6\).

当\(\lambda=-1\)时,解方程组\((-\vb{E}-\vb{A})\vb{x}=\vb0\),\begin{equation*}
	-\vb{E}-\vb{A}
	= \begin{bmatrix} 0 & 0 & 0 \\ -8 & -3 & -4 \\ -8 & -3 & -4 \end{bmatrix}
	\to \begin{bmatrix} -8 & -3 & -4 \\ 0 & 0 & 0 \\ 0 & 0 & 0 \end{bmatrix}.
\end{equation*}
分别令\(\left\{ \begin{array}{l} x_2=8 \\ x_3=0 \end{array} \right.\)
和\(\left\{ \begin{array}{l} x_2=0 \\ x_3=2 \end{array} \right.\),
得基础解系\begin{equation*}
	\vb{x}_1 = \begin{bmatrix} -3 \\ 8 \\ 0 \end{bmatrix},
	\quad
	\vb{x}_2 = \begin{bmatrix} -1 \\ 0 \\ 2 \end{bmatrix},
\end{equation*}
属于\(-1\)的全部特征向量为\(k_1\vb{x}_1+k_2\vb{x}_2\)(\(k_1,k_2\)为不全为零的任意常数);

当\(\lambda=6\)时,解方程组\((6\vb{E}-\vb{A})\vb{x}=\vb0\),\begin{equation*}
	6\vb{E}-\vb{A}
	= \begin{bmatrix} 7 & 0 & 0 \\ -8 & 4 & -4 \\ -8 & -3 & 3 \end{bmatrix}
	\to \begin{bmatrix} 1 & 0 & 0 \\ 0 & 1 & -1 \\ 0 & 0 & 0 \end{bmatrix},
\end{equation*}
令\(x_3=1\)得\(x_1=0\),\(x_2=1\),
基础解系\begin{equation*}
	\vb{x}_3 = \begin{bmatrix} 0 \\ 1 \\ 1 \end{bmatrix},
\end{equation*}
属于\(6\)的全部特征向量为\(k_3\vb{x}_3\)(\(k_3\)为任意常数).
\end{solution}
\end{example}

\begin{example}
设矩阵\(\vb{A} = (a_{ij})_n \in \mathbb{C}^n\),
但\(a_{ij} \in \mathbb{R}\ (i,j=1,2,\dotsc,n)\).
证明:如果\(\lambda_0\in\mathbb{C}\)是\(\vb{A}\)的一个特征值,
\(\vb{x}_0\)是\(\vb{A}\)属于\(\lambda_0\)的一个特征向量,
那么\(\ComplexConjugate{\lambda_0}\)也是\(\vb{A}\)的一个特征值,
且\(\ComplexConjugate{\vb{x}_0}\)是\(\vb{A}\)属于\(\ComplexConjugate{\lambda_0}\)的一个特征向量.
\begin{proof}
在\(\vb{A}\vb{x}_0=\lambda_0\vb{x}_0\)两边取共轭得
\(\ComplexConjugate{\vb{A}}\ComplexConjugate{\vb{x}_0}
=\ComplexConjugate{\lambda_0}\ComplexConjugate{\vb{x}_0}\).
又因为\(\vb{A}=\ComplexConjugate{\vb{A}}\),
因此\(\vb{A}\ComplexConjugate{\vb{x}_0}=\ComplexConjugate{\lambda_0}\ComplexConjugate{\vb{x}_0}\).
这就表明\(\ComplexConjugate{\lambda_0}\)也是\(\vb{A}\)的一个特征值,
\(\ComplexConjugate{\vb{x}_0}\)是\(\vb{A}\)的属于\(\ComplexConjugate{\lambda_0}\)的一个特征向量.
\end{proof}
\end{example}

\begin{example}
求复数域上矩阵\begin{equation*}
	\vb{A} = \begin{bmatrix}
		4 & 7 & -3 \\
		-2 & -4 & 2 \\
		-4 & -10 & 4
	\end{bmatrix}
\end{equation*}的全部特征值和特征向量.
\begin{solution}
\(\vb{A}\)的特征多项式为\begin{equation*}
	\abs{\lambda\vb{E}-\vb{A}}
	= \begin{vmatrix}
		\lambda-4 & -7 & 3 \\
		2 & \lambda+4 & -2 \\
		4 & 10 & \lambda-4
	\end{vmatrix}
	= \lambda^3 - 4\lambda^2 + 6\lambda - 4
	= (\lambda-2)(\lambda^2-2\lambda+2).
\end{equation*}
令\(\abs{\lambda\vb{E}-\vb{A}}=0\)解得\(\lambda=2,1\pm\iu\).

当\(\lambda=2\)时,解方程组\((2\vb{E}-\vb{A})\vb{x}=\vb0\),\begin{equation*}
	2\vb{E}-\vb{A} = \begin{bmatrix}
		-2 & -7 & 3 \\
		2 & 6 & -2 \\
		4 & 10 & -2
	\end{bmatrix} \to \begin{bmatrix}
		2 & 4 & 0 \\
		0 & 1 & -1 \\
		0 & 0 & 0
	\end{bmatrix}.
\end{equation*}
令\(x_2=x_3=1\)得\(x_1=-2\),基础解系为\begin{equation*}
	\vb{x}_1 = (-2,1,1)^T,
\end{equation*}
属于\(2\)的全部特征向量为\(k_1\vb{x}_1\ (k_1\in\mathbb{C}-\{0\})\).

当\(\lambda=1+\iu\)时,解方程组\([(1+\iu)\vb{E}-\vb{A}]\vb{x}=\vb0\),\begin{equation*}
	(1+\iu)\vb{E}-\vb{A} = \begin{bmatrix}
		-3+\iu & -7 & 3 \\
		2 & 5+\iu & -2 \\
		4 & 10 & -3+\iu
	\end{bmatrix}
	\to \def\arraystretch{1.5}\begin{bmatrix}
		1 & 0 & \frac{1}{2}-\iu \\
		0 & 1 & -\frac{1}{2}+\frac{1}{2}\iu \\
		0 & 0 & 0
	\end{bmatrix}.
\end{equation*}
令\(x_3=-2\)得\(x_1=1-2\iu,x_2=-1+\iu\),
基础解系为\begin{equation*}
	\vb{x}_2 = (1-2\iu,-1+\iu,-2)^T,
\end{equation*}
属于\(1+\iu\)的全部特征向量为\(k_2\vb{x}_2\ (k_2\in\mathbb{C}-\{0\})\).

当\(\lambda=1-\iu\)时,
\(\vb{x}_2\)也是它的一个特征向量,
那么属于\(1-\iu\)的全部特征向量为\(k_3\vb{x}_2\ (k_3\in\mathbb{C}-\{0\})\).
\end{solution}
\end{example}
