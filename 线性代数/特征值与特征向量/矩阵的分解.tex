\section{CR分解}
\DefineConcept{CR分解}(column-row factorization)的目的,
是将一个矩阵分解成一个列满秩矩阵和一个行满秩矩阵的乘积.
\begin{theorem}
%@see: 《Linear Algebra Done Right (Fourth Eidition)》(Sheldon Axler) P78 3.56
设\(\vb{A} \in M_{m \times n}(K)\)且\(\rank\vb{A} = c \geq 1\),
那么存在一个矩阵\(\vb{C} \in M_{m \times c}(K)\),
存在一个矩阵\(\vb{R} \in M_{c \times n}(K)\),
使得\(\vb{A} = \vb{C} \vb{R}\).
\begin{proof}
将矩阵\(\vb{A}\)按列分块,
得\(\vb{A} = (\AutoTuple{\alpha}{n})\).
由\cref{theorem:线性空间.生成向量组的约化} 可知
\(\vb{A}\)的列向量组可以约化为它的列空间的一个基,
由列秩与列空间的维数的定义以及两者之间的关系可知,
这个基的基数必定是\(c\).
这个基(不妨记为\(\AutoTuple{\alpha}{c}\))的全部\(c\)个基向量
可以组成一个\(m \times c\)矩阵\(\vb{C} = (\AutoTuple{\alpha}{c})\).

显然\(\vb{A}\)的每一个列向量都可以由\(\vb{C}\)的列向量组线性表出,
即\begin{equation*}
	\alpha_k = x_{1k} \alpha_1 + \dotsb + x_{ck} \alpha_c,
	\quad k=1,2,\dotsc,n.
\end{equation*}
那么只要记\begin{equation*}
	\vb{R} = \begin{bmatrix}
		x_{11} & \dots & x_{1n} \\
		\vdots & & \vdots \\
		x_{c1} & \dots & x_{cn}
	\end{bmatrix},
\end{equation*}
便有\(\vb{A} = \vb{C} \vb{R}\).
\end{proof}
\end{theorem}

\section{QR分解}
QR分解的目的,是将一个实满秩矩阵分解成一个正定矩阵和一个主对角元都是正数的上三角矩阵的乘积.

\section{LU分解}
LU分解的目的,是将一个矩阵分解成一个下三角矩阵和一个上三角矩阵的乘积.

\begin{theorem}
设\(\vb{A} = (a_{ij})_n \in M_n(\mathbb{R})\),
存在下三角阵\(\vb{L} = (l_{ij})_n\)和上三角阵\(\vb{U} = (u_{ij})_n\),
使得\(\vb{A} = \vb{L} \vb{U}\),
其中\(l_{ii} = 1\ (i=1,2,\dotsc,n),
l_{ij} = 0\ (i<j),
u_{ij} = 0\ (i>j)\).
\end{theorem}

举例来说,令\begin{equation*}
	\vb{A} = \begin{bmatrix}
		a_{11} & a_{12} \\
		a_{21} & a_{22}
	\end{bmatrix}
	= \begin{bmatrix}
		1 & 0 \\
		l_{21} & 1
	\end{bmatrix}
	\begin{bmatrix}
		u_{11} & u_{12} \\
		0 & u_{22}
	\end{bmatrix}
	= \vb{L} \vb{U},
\end{equation*}
得\begin{equation*}
	\left.\begin{array}{r}
		1 \cdot u_{11} + 0 \cdot 0 = a_{11} \\
		1 \cdot u_{12} + 0 \cdot u_{22} = a_{12} \\
		l_{21} u_{11} + 1 \cdot 0 = a_{21} \\
		l_{21} u_{12} + 1 \cdot u_{22} = a_{22}
	\end{array}\right\}
	\implies
	\left\{\begin{array}{l}
		u_{11} = a_{11}, \\
		u_{12} = a_{12}, \\
		l_{21} = a_{21} / u_{11}, \\
		u_{22} = a_{22} - l_{21} u_{12}.
	\end{array}\right.
\end{equation*}

又令\begin{equation*}
	\vb{A} = \begin{bmatrix}
		a_{11} & a_{12} & a_{13} \\
		a_{21} & a_{22} & a_{23} \\
		a_{31} & a_{32} & a_{33}
	\end{bmatrix}
	= \begin{bmatrix}
		1 & 0 & 0 \\
		l_{21} & 1 & 0 \\
		l_{31} & l_{32} & 1
	\end{bmatrix}
	\begin{bmatrix}
		u_{11} & u_{12} & u_{13} \\
		0 & u_{22} & u_{23} \\
		0 & 0 & u_{33}
	\end{bmatrix} = \vb{L} \vb{U},
\end{equation*}
得\begin{equation*}
	\left\{\begin{array}{l}
		u_{11} = a_{11}, \\
		u_{12} = a_{12}, \\
		u_{13} = a_{13}, \\
		l_{21} = a_{21} / u_{11}, \\
		l_{31} = a_{31} / u_{11}, \\
		u_{22} = a_{22} - l_{21} \cdot u_{12}, \\
		u_{23} = a_{23} - l_{21} \cdot u_{13}, \\
		l_{32} = (a_{32} - l_{31} \cdot u_{12}) / u_{22}, \\
		u_{33} = a_{33} - (l_{31} \cdot u_{13} + l_{32} \cdot u_{23}).
	\end{array}\right.
\end{equation*}


\section{谱分解}
\begin{lemma}\label{theorem:矩阵分解.复方阵酉相似于上三角阵}
%@see: 《矩阵论》(詹兴致) P6 定理1.5(Schur酉三角化)
设矩阵\(\vb{A} \in M_n(\mathbb{C})\),
则\(\vb{A}\)酉相似于某个上三角矩阵.
\begin{proof}
用数学归纳法.
当\(n=1\)时,结论显然成立.
假设当\(n=k-1\ (k\geq2)\)时,结论成立.
当\(n=k\)时,
任取\(\vb{A}\)的某一个特征值\(\lambda_1\),
再任取属于\(\lambda_1\)的某一个单位特征向量\(\vb{x}_1\),
将\(\vb{x}_1\)扩充成\(\mathbb{C}^n\)的一个标准正交基\(\AutoTuple{\vb{x}}{n}\),
然后令\(\vb{U}_1 = (\AutoTuple{\vb{x}}{n})\),
则\(\vb{U}_1\)是酉矩阵,
且\begin{equation*}
	\vb{U}_1^H \vb{A} \vb{U}_1
	= \begin{bmatrix}
		\lambda & \vb{y}^H \\
		\vb0 & \vb{A}_1
	\end{bmatrix},
\end{equation*}
其中\(\vb{A}_1 \in M_{n-1}(\mathbb{C})\).
由归纳假设,
存在酉矩阵\(\vb{U}_2 \in M_{n-1}(\mathbb{C})\)使得\(\vb{U}_2^H \vb{A}_1 \vb{U}_2\)是上三角矩阵.
令\(\vb{U} = \vb{U}_1 \diag(1,\vb{U}_2) \in M_n(\mathbb{C})\),
则\(\vb{U}\)是酉矩阵,
且\begin{equation*}
	\vb{U}^H \vb{A} \vb{U}
	= \begin{bmatrix}
		\lambda & \vb{y}^H \vb{U}_2 \\
		\vb0 & \vb{U}_2^H \vb{A}_1 \vb{U}_2
	\end{bmatrix}
\end{equation*}是上三角矩阵.
\end{proof}
\end{lemma}

\begin{theorem}\label{theorem:矩阵分解.谱分解}
%@see: 《矩阵论》(詹兴致) P2 定理1.1
设\(\vb{A} \in M_n(\mathbb{C})\)是正规矩阵,
则存在酉矩阵\(\vb{U} \in M_n(\mathbb{C})\)满足\begin{equation*}
	\vb{A} = \vb{U} \diag(\AutoTuple{\lambda}{n}) \vb{U}^H,
\end{equation*}
其中\(\AutoTuple{\lambda}{n}\)是\(\vb{A}\)的特征值.
\begin{proof}
根据\cref{theorem:矩阵分解.复方阵酉相似于上三角阵},
存在酉矩阵\(\vb{U}\)和上三角阵\(\vb{T}\)满足\(\vb{A} = \vb{U} \vb{T} \vb{U}^H\).
因为\(\vb{A}\)是正规矩阵,\(\vb{A} \vb{A}^H = \vb{A}^H \vb{A}\),
所以\(\vb{T} \vb{T}^H = \vb{T}^H \vb{T}\).
逐个比较\(\vb{T} \vb{T}^H\)和\(\vb{T}^H \vb{T}\)的主对角线元素,
可以看出\(\vb{T}\)的每一个非主对角线元素都是零,
\(\vb{T}\)是对角阵.
\end{proof}
\end{theorem}
\begin{remark}
\cref{theorem:矩阵分解.谱分解} 说明
每一个正规矩阵都酉相似于一个对角矩阵.
\end{remark}

\section{奇异值分解}
\begin{theorem}
设矩阵\(\vb{A} \in M_{m \times n}(\mathbb{R})\),
则存在\(m\)阶正交矩阵\(\vb{U}\)、\(n\)阶正交矩阵\(\vb{V}\)和\(m \times n\)对角阵\(\vb\Sigma\),
使得\begin{equation*}
	\vb{A} = \vb{U} \vb\Sigma \vb{V}^T,
\end{equation*}
其中\(\vb\Sigma = (\sigma_{ij})_{m \times n}\)的元素\(\sigma_{ij}\)满足\begin{equation*}
	\sigma_{ij} = \left\{ \begin{array}{cc}
	0, & i \neq j, \\
	s_i \geq 0, & i = j.
	\end{array} \right.
\end{equation*}
\rm
这里,矩阵\(\vb\Sigma\)的对角元\(s_i\)
称为\(\vb{A}\)的\DefineConcept{奇异值}(通常按\(s_i \geq s_{i+1}\)排列),
\(\vb{U}\)的列分块向量
称为\(\vb{A}\)的\DefineConcept{左奇异向量},
\(\vb{V}\)的列分块向量
称为\DefineConcept{右奇异向量}.
\begin{proof}
由于\(\vb{A}^T \vb{A} \in M_n(\mathbb{R})\),故可作谱分解,即存在正交矩阵\(\vb{V}\),使得\begin{equation*}
	\vb{V}^{-1}\vb{A}\vb{V} = \vb{V}^T\vb{A}\vb{V} = \diag(\lambda_1,\lambda_2,\dotsc,\lambda_n),
\end{equation*}
其中\(\vb{V}=(\AutoTuple{\vb{v}}{n})\)中的列分块向量\(\vb{v}_i\)是\(\vb{A}^T \vb{A}\)对应于特征值\(\lambda_i\)的特征向量,
而\(\{\AutoTuple{\vb{v}}{n}\}\)构成\(\mathbb{R}^n\)的一组标准正交基.

注意到\(\vb{A}^T \vb{A}\)是半正定矩阵\footnote{当\(\vb{A}^T \vb{A}\)是可逆矩阵时,\(\vb{A}^T \vb{A}\)是正定矩阵.},
故其特征值\(\lambda_i\geq0\).

考虑映射\(\vb{A}_{m \times n}\colon \mathbb{R}^n \to \mathbb{R}^m, \vb{x} \mapsto \vb{A}\vb{x}\),
设\(\rank\vb{A} = r\),
将\(\vb{A}\)作用到\(\mathbb{R}^n\)的标准正交基\(\{\AutoTuple{\vb{v}}{n}\}\)上,
利用维数公式,得\begin{equation*}
\dim(\ker \vb{A}) + \dim(\Img \vb{A}) = n,
\end{equation*}
可知\(\vb{A}\vb{v}_1,\vb{A}\vb{v}_2,\dotsc,\vb{A}\vb{v}_n\)这\(n\)个向量中有\(r\)个向量构成\(\mathbb{R}^m\)的一组部分基,
而\(\vb{A}\vb{v}_{r+1} = \vb{A}\vb{v}_{r+2} = \dotsb = \vb{A}\vb{v}_n = 0\).

有\(\vb{A}^T \vb{A} \vb{v}_j = \lambda_j \vb{v}_j\),又有\begin{equation*}
	\VectorInnerProductDot{\vb{v}_i}{\vb{v}_j}
	= \vb{v}_i^T \vb{v}_j
	= \left\{ \begin{array}{lc}
		1, & i=j, \\
		0, & i \neq j.
	\end{array} \right.
\end{equation*}
所以,当\(i \neq j\)时,\begin{equation*}
	\VectorInnerProductDot{\vb{A} \vb{v}_i}{\vb{A} \vb{v}_j}
	= \vb{v}_i^T \vb{A}^T \vb{A} \vb{v}_j
	= \lambda_j \vb{v}_i^T \vb{v}_j
	= 0;
\end{equation*}
而当\(i = j\)时,\begin{equation*}
	\abs{\vb{A}\vb{v}_i}^2
	= \VectorInnerProductDot{\vb{A} \vb{v}_i}{\vb{A} \vb{v}_j}
	= \lambda_i \vb{v}_i^T \vb{v}_i
	= \lambda_i.
\end{equation*}
也就是说,向量组\(\{\vb{A}\vb{v}_1,\vb{A}\vb{v}_2,\dotsc,\vb{A}\vb{v}_r\}\)是两两正交的.
单位化该向量组,又记\begin{equation*}
	\vb{u}_i = \frac{\vb{A}\vb{v}_i}{\abs{\vb{A}\vb{v}_i}}
	= \frac{\vb{A}\vb{v}_i}{\sqrt{\lambda_i}}
	\vb{v}uad(i=1,2,\dotsc,r),
\end{equation*}
于是\(\vb{A}\vb{v}_i = s_i \vb{u}_i\),其中\(s_i = \sqrt{\lambda_i}\).

将\(\vb{u}_1,\vb{u}_2,\dotsc,\vb{u}_r\)扩充成\(\mathbb{R}^m\)的标准正交基
\(\{\vb{u}_1,\vb{u}_2,\dotsc,\vb{u}_r,\vb{u}_{r+1},\dotsc,\vb{u}_m\}\),
在这组基下,有\begin{equation*}
	\vb{A}\vb{V} = \vb{A}(\AutoTuple{\vb{v}}{n}) = \begin{bmatrix}
		s_1 \vb{u}_1 \\
		& \ddots \\
		& & s_r \vb{u}_r \\
		& & & 0 \\
		& & & & \ddots \\
		& & & & & 0
	\end{bmatrix}
	= \vb{U} \vb\Sigma,
\end{equation*}
其中\(\vb{U} = (\vb{u}_1,\vb{u}_2,\dotsc,\vb{u}_m)\),
\(\vb\Sigma = \diag(s_1,\dotsc,s_r,0,\dotsc,0)\),
将上式两边右乘\(\vb{V}^{-1}\),即得\(\vb{A} = \vb{U}\vb\Sigma\vb{V}^T\).
\end{proof}
\end{theorem}

\begin{example}
对矩阵\(\vb{A} = \begin{bmatrix} 0 & 1 \\ 1 & 1 \\ 1 & 0 \end{bmatrix}\)进行奇异值分解.
\begin{solution}
经计算\begin{equation*}
	\vb{A}^T \vb{A} = \begin{bmatrix} 2 & 1 \\ 1 & 2 \end{bmatrix},
\end{equation*}
其特征值是\(\lambda_1 = 3\)和\(\lambda_2 = 1\).
\(\vb{A}^T \vb{A}\)属于特征值\(\lambda_1\)的特征向量为
\(\vb{v}_1 = \begin{bmatrix} 1/\sqrt{2} \\ 1/\sqrt{2} \end{bmatrix}\);
\(\vb{A}^T \vb{A}\)属于特征值\(\lambda_2\)的特征向量为
\(\vb{v}_2 = \begin{bmatrix} -1/\sqrt{2} \\ 1/\sqrt{2} \end{bmatrix}\).

同时有\begin{equation*}
	\vb{A} \vb{A}^T = \begin{bmatrix} 1 & 1 & 0 \\ 1 & 2 & 1 \\ 0 & 1 & 1 \end{bmatrix},
\end{equation*}
其特征值是\(\mu_1 = 3\)、\(\mu_2 = 1\)、\(\mu_3 = 0\).
\(\vb{A} \vb{A}^T\)属于特征值\(\mu_1\)的特征向量为
\(\vb{u}_1 = \begin{bmatrix} 1/\sqrt{6} \\ 2/\sqrt{6} \\ 1/\sqrt{6} \end{bmatrix}\);
\(\vb{A} \vb{A}^T\)属于特征值\(\mu_2\)的特征向量为
\(\vb{u}_2 = \begin{bmatrix} 1/\sqrt{2} \\ 0 \\ -1/\sqrt{2} \end{bmatrix}\);
\(\vb{A} \vb{A}^T\)属于特征值\(\mu_3\)的特征向量为
\(\vb{u}_3 = \begin{bmatrix} 1/\sqrt{3} \\ -1/\sqrt{3} \\ 1/\sqrt{3} \end{bmatrix}\).

再根据\(s_i = \sqrt{\lambda_i}\)求得奇异值\(s_1 = \sqrt{3}\)和\(s_2 = 1\).

于是\begin{gather*}
	\vb{U} = (\vb{u}_1,\vb{u}_2,\vb{u}_3)
	= \begin{bmatrix}
		1/\sqrt{6} & 1/\sqrt{2} & 1/\sqrt{3} \\
		2/\sqrt{6} & 0 & -1/\sqrt{3} \\
		1/\sqrt{6} & -1/\sqrt{2} & 1/\sqrt{3}
	\end{bmatrix}, \\
	\vb{V} = (\vb{v}_1,\vb{v}_2,\vb{v}_3)
	= \begin{bmatrix}
		1/\sqrt{2} & -1/\sqrt{2} \\
		1/\sqrt{2} & 1/\sqrt{2}
	\end{bmatrix}, \\
	\vb\Sigma = \begin{bmatrix}
		\sqrt{3} & 0 \\
		0 & 1 \\
		0 & 0
	\end{bmatrix}.
\end{gather*}
\end{solution}
\end{example}

\begin{theorem}
%@see: 《矩阵论》(詹兴致) P7 定理1.6(奇异值分解)
设矩阵\(\vb{A} \in M_{m \times n}(\mathbb{C})\),
则存在酉矩阵\(\vb{U} \in M_m(\mathbb{C})\)、酉矩阵\(\vb{V} \in M_n(\mathbb{C})\)使得\begin{equation*}
	\vb{U} \vb{A} \vb{V} = \vb\Sigma,
\end{equation*}
其中\(\vb\Sigma = \diag_{m \times n}(\AutoTuple{s}{p}),
s_1 \geq \dotsb \geq s_p \geq 0,
p = \min\{m,n\}\).
%TODO proof
\end{theorem}

\section{极分解}
\begin{theorem}
任意实方阵\(\vb{A}\)可表为\begin{equation*}
	\vb{A} = \vb{S}\vb\Omega = \vb\Omega_1 \vb{S}_1,
\end{equation*}
其中\(\vb{S}\)和\(\vb{S}_1\)为半正定实对称方阵,
\(\vb\Omega\)与\(\vb\Omega_1\)为实正交方阵,
而且\(\vb{S}\)和\(\vb{S}_1\)都是唯一的.
\begin{proof}
当\(\vb{A}\)可逆时,
\(\vb{A}^T \vb{A}\)是正定阵,
存在正定阵\(\vb{S}_1\),
使得\(\vb{A}^T \vb{A} = \vb{S}_1^2\),
于是\(\vb{A} = \vb{A} \vb{S}_1^{-1} \vb{S}_1\),
注意到\((\vb{A} \vb{S}_1^{-1})^T (\vb{A} \vb{S}_1^{-1})
= (\vb{S}_1^{-1})^T \vb{A}^T \vb{A} \vb{S}_1^{-1} = \vb{E}\),
即\(\vb{A} \vb{S}_1^{-1}\)正交,
那么只需要令\(\vb\Omega_1 = \vb{A} \vb{S}_1^{-1}\)
即有\(\vb{A} = \vb\Omega_1 \vb{S}_1\).

当\(\vb{A}\)不可逆时,可以运用正交相似标准型;
也可以运用扰动法,
即令\(\vb{S}_1(t) = \vb{S}_1 + t\vb{E}\),
则当\(t\)充分大时,
\(\vb{S}_1(t)\)可逆.
\end{proof}
\end{theorem}

%@see: https://mathworld.wolfram.com/QRDecomposition.html
%@see: https://o-o-sudo.github.io/numerical-methods/qr-.html
%@see: https://reference.wolfram.com/language/guide/MatrixDecompositions.html.zh
%@see: https://blog.csdn.net/Insomnia_X/article/details/126787580
%@see: https://mathworld.wolfram.com/SchurDecomposition.html
%@see: https://mathworld.wolfram.com/JordanMatrixDecomposition.html
