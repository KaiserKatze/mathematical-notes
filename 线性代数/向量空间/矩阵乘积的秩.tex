\section{矩阵乘积的秩}
令\begin{equation*}
	\vb{A}=\begin{bmatrix}
		1 & 0 \\
		0 & 0
	\end{bmatrix}, \qquad
	\vb{B}=\begin{bmatrix}
		0 & 0 \\
		1 & 0
	\end{bmatrix}, \qquad
	\vb{C}=\begin{bmatrix}
		1 & 1 \\
		0 & 1
	\end{bmatrix},
\end{equation*}
则\begin{equation*}
	\vb{A} \vb{B}=\begin{bmatrix}
		1 & 0 \\
		0 & 0
	\end{bmatrix}
	\begin{bmatrix}
		0 & 0 \\
		1 & 0
	\end{bmatrix}
	= \begin{bmatrix}
		0 & 0 \\
		0 & 0
	\end{bmatrix};
\end{equation*}
于是\(\rank(\vb{A} \vb{B})=0\),而\(\rank\vb{A}=1\),\(\rank\vb{B}=1\).

又\begin{equation*}
	\vb{A}\vb{C}=\begin{bmatrix}
		1 & 0 \\
		0 & 0
	\end{bmatrix}
	\begin{bmatrix}
		1 & 1 \\
		0 & 1
	\end{bmatrix}
	= \begin{bmatrix}
		1 & 1 \\
		0 & 0
	\end{bmatrix};
\end{equation*}
于是\(\rank(\vb{A}\vb{C})=1\),而\(\rank\vb{A}=1\),\(\rank\vb{C}=2\).

从上述例子,我们猜测:
对于任意矩阵\(\vb{A},\vb{B}\),总有\begin{equation*}
	\rank(\vb{A} \vb{B}) \leq \rank\vb{A}, \qquad
	\rank(\vb{A} \vb{B}) \leq \rank\vb{B}.
\end{equation*}

\subsection{矩阵乘积的秩}
\begin{theorem}\label{theorem:线性方程组.矩阵乘积的秩}
%@see: 《线性代数》(张慎语、周厚隆) P78 定理7
%@see: 《高等代数(第三版 上册)》(丘维声) P121 定理1
设\(\vb{A} \in M_{s \times t}(K),
\vb{B} \in M_{t \times n}(K)\),
那么\begin{equation*}
	\rank(\vb{A}\vb{B}) \leq \min\{\rank\vb{A},\rank\vb{B}\}.
\end{equation*}
\begin{proof}
记\(\vb{C} \defeq \vb{A}\vb{B}\).
显然\(\vb{C} \in M_{s \times n}(K)\).
将\(\vb{C}\)、\(\vb{B}\)分别进行列分块得\begin{equation*}
	\vb{C} = (\AutoTuple{\vb\gamma}{n}),
	\qquad
	\vb{B} = (\AutoTuple{\vb\beta}{n}),
\end{equation*}
其中\(\vb\beta_i \in K^t\ (i=1,2,\dotsc,n)\),
\(\vb\gamma_i \in K^s\ (i=1,2,\dotsc,n)\),
则\begin{equation*}
	(\AutoTuple{\vb\gamma}{n})
	= \vb{A} (\AutoTuple{\vb\beta}{n})
	= (\AutoTuple{\vb{A}\vb\beta}{n}),
\end{equation*}
于是\(\vb\gamma_i = \vb{A} \vb\beta_i\ (i=1,2,\dotsc,n)\).

假设\(\rank\vb{B} = r\).
由\cref{example:向量空间.若部分组向量个数多于全组的秩则部分组必线性相关},
\(\vb{B}\)的任意\(r+1\)个列向量
\(\vb\beta_{k_1},\vb\beta_{k_2},\dotsc,\vb\beta_{k_{r+1}}\)线性相关,
也就是说,存在不全为零的数\(l_1,l_2,\dotsc,l_{r+1}\in K\),使得\begin{equation*}
	l_1 \vb\beta_{k_1} + l_2 \vb\beta_{k_2} + \dotsb + l_{r+1} \vb\beta_{k_{r+1}} = \vb0,
\end{equation*}
从而有\begin{equation*}
	l_1 \vb\gamma_{k_1} + l_2 \vb\gamma_{k_2} + \dotsb + l_{r+1} \vb\gamma_{k_{r+1}}
	= \vb{A}(l_1 \vb\beta_{k_1} + l_2 \vb\beta_{k_2} + \dotsb + l_{r+1} \vb\beta_{k_{r+1}})
	= \vb0,
\end{equation*}
这就是说\(\vb{C}\)的任意\(r+1\)个列向量也线性相关,
那么\(\RankC\vb{C} \ngeq r+1\),
因此\begin{equation*}
	\rank(\vb{A}\vb{B})
	\leq r = \rank\vb{B}.
\end{equation*}
利用这个结论,我们还可以得到\begin{equation*}
	\rank(\vb{A}\vb{B})
	= \rank(\vb{A}\vb{B})^T
	= \rank(\vb{B}^T \vb{A}^T)
	\leq \rank \vb{A}^T
	= \rank \vb{A}.
\end{equation*}
综上所述\(\rank(\vb{A}\vb{B}) \leq \min\{\rank\vb{A},\rank\vb{B}\}\).
\end{proof}
\end{theorem}
\begin{remark}
我们可以看出\cref{theorem:向量空间.向量组的秩的比较1}
与\cref{theorem:线性方程组.矩阵乘积的秩} 之间存在内在联系:
由\cref{theorem:向量空间.线性表出2的等价条件} 可知\begin{equation*}
	\rank(\AutoTuple{\vb\alpha}{s})
	= \rank((\AutoTuple{\vb\beta}{t})\vb{Q}),
\end{equation*}
而由可知\begin{equation*}
	\rank((\AutoTuple{\vb\beta}{t})\vb{Q})
	\leq \rank(\AutoTuple{\vb\beta}{t}),
\end{equation*}
于是\begin{equation*}
	\rank(\AutoTuple{\vb\alpha}{s})
	\leq \rank(\AutoTuple{\vb\beta}{t}).
\end{equation*}
\end{remark}

\begin{corollary}\label{theorem:矩阵乘积的秩.与可逆矩阵相乘不变秩}
%@see: 《线性代数》(张慎语、周厚隆) P78 推论1
设矩阵\(\vb{A} \in M_{s \times n}(K)\).
\begin{itemize}
	\item 如果\(\vb{P}\)是数域\(K\)上的\(s\)阶可逆矩阵,则\(\rank\vb{A} = \rank(\vb{P}\vb{A})\).
	\item 如果\(\vb{Q}\)是数域\(K\)上的\(n\)阶可逆矩阵,则\(\rank\vb{A} = \rank(\vb{A}\vb{Q})\).
\end{itemize}
\begin{proof}
因为\(\vb{A} = (\vb{P}^{-1} \vb{P}) \vb{A} = \vb{P}^{-1} (\vb{P} \vb{A})\),
由\cref{theorem:线性方程组.矩阵乘积的秩},
有\begin{equation*}
	\rank\vb{A} = \rank(\vb{P}^{-1}(\vb{P}\vb{A})) \leq \rank(\vb{P}\vb{A}) \leq \rank\vb{A},
\end{equation*}
所以\(\rank\vb{A} = \rank(\vb{P}\vb{A})\).
同理可得\(\rank\vb{A} = \rank(\vb{A}\vb{Q})\).
\end{proof}
\end{corollary}

\begin{theorem}\label{theorem:矩阵乘积的秩.多行少列矩阵与少行多列矩阵的乘积的行列式}
设矩阵\(\vb{A} \in M_{m \times n}(K),
\vb{B} \in M_{n \times m}(K)\),
\(m > n\),
那么\begin{equation*}
	\abs{\vb{A} \vb{B}} = 0.
\end{equation*}
\begin{proof}
当\(m>n\)时,
根据\cref{theorem:线性方程组.矩阵的秩的性质2},有\begin{equation*}
	\rank\vb{A},\rank\vb{B} \leq \min\{m,n\} = n;
\end{equation*}
再根据\cref{theorem:线性方程组.矩阵乘积的秩},有\begin{equation*}
	\rank(\vb{A} \vb{B}) \leq \min\{\rank\vb{A},\rank\vb{B}\} = n < m,
\end{equation*}
也就是说,矩阵\(\vb{A} \vb{B}\)不满秩;
那么根据\cref{theorem:向量空间.满秩方阵的行列式非零} 可知\(\abs{\vb{A} \vb{B}} = 0\).
\end{proof}
%\cref{equation:线性方程组.柯西比内公式}
\end{theorem}
\begin{remark}
在\cref{example:行列式.两个向量的乘积矩阵的行列式} 我们看到
维数相同的一个列向量与一个行向量的乘积的行列式等于零.
\end{remark}

\begin{example}\label{example:矩阵乘积的秩.矩阵相乘不变秩的特例1}
%@see: {de3029b8-10a6-4ae5-8f64-108dae1c10a9}
设\(\vb{A} \in M_{s \times n}(K), \vb{B} \in M_{n \times s}(K)\),
且\(\vb{A} \vb{B} \vb{A} = \vb{A}\).
证明:\begin{equation*}
	\rank(\vb{A} \vb{B})
	= \rank(\vb{B} \vb{A})
	= \rank\vb{A}.
\end{equation*}
\begin{proof}
因为\(
	\rank\vb{A}
	= \rank(\vb{A} \vb{B} \vb{A})
	\leq \rank(\vb{A} \vb{B})
	\leq \rank\vb{A}
\),
所以\(\rank(\vb{A} \vb{B}) = \rank\vb{A}\).
同理可得\(\rank(\vb{B} \vb{A}) = \rank\vb{A}\).
\end{proof}
\end{example}

\subsection{等价标准型}
\begin{definition}
设矩阵\(\vb{A} \in M_{s \times n}(K)\),
\(\rank\vb{A} = r\),
\(\vb{E}_r\)是\(r\)阶单位矩阵.
我们把分块矩阵\begin{equation*}
	\begin{bmatrix}
		\vb{E}_r & \vb0 \\
		\vb0 & \vb0
	\end{bmatrix}
\end{equation*}
称为“\(\vb{A}\)的\DefineConcept{等价标准型}”.
\end{definition}

下面我们证明任意一个矩阵的等价标准型总是存在.
\begin{theorem}\label{theorem:矩阵乘积的秩.等价标准型的存在性}
矩阵\(\vb{A}\)满足\(\rank\vb{A}=r\)的充分必要条件是:
存在可逆矩阵\(\vb{P},\vb{Q}\),使得\begin{equation*}
	\vb{P} \vb{A} \vb{Q}
	= \begin{bmatrix}
		\vb{E}_r & \vb0 \\
		\vb0 & \vb0
	\end{bmatrix} = \vb{B}.
\end{equation*}
\begin{proof}
充分性.
如果可逆矩阵\(\vb{P},\vb{Q}\)使得\begin{equation*}
	\vb{P}\vb{A}\vb{Q}
	= \begin{bmatrix}
		\vb{E}_r & \vb0 \\
		\vb0 & \vb0
	\end{bmatrix},
\end{equation*}
把上式等号左边的可逆矩阵\(\vb{P}\)、\(\vb{Q}\)分别视作对矩阵\(\vb{A}\)的初等行变换和初等列变换,
那么,根据\cref{theorem:线性方程组.初等变换不变秩},
所得矩阵\(\vb{B}\)的秩与原矩阵\(\vb{A}\)相同,
即\begin{equation*}
	\rank\vb{A} = \rank\vb{B} = r.
	\qedhere
\end{equation*}
%\cref{theorem:线性方程组.非零矩阵可经初等行变换化为若尔当阶梯形矩阵}
%TODO proof 未证明必要性
\end{proof}
\end{theorem}

\begin{theorem}\label{theorem:矩阵乘积的秩.矩阵等价的充分必要条件}
设\(\vb{A}\)与\(\vb{B}\)都是\(s \times n\)矩阵,
则\(\vb{A} \cong \vb{B}\)的充分必要条件是:
\(\rank\vb{A} = \rank\vb{B}\).
\begin{proof}
必要性.
因为\(\vb{A}\)可经一系列初等变换化为\(\vb{B}\),
根据\cref{theorem:线性方程组.初等变换不变秩},初等变换不改变矩阵的秩,
所以\(\rank\vb{A} = \rank\vb{B}\).

充分性.
已知\(\rank\vb{A} = \rank\vb{B} = r\).
对\(\vb{A}\)作初等行变换可将其化简为仅前\(r\)行不为零的阶梯形矩阵\(\vb{C}\),
同样对\(\vb{C}\)作初等列变换可化简为\(\vb{A}\)的等价标准型.
对\(\vb{B}\)也可作初等变换化为等价标准型.
那么存在\(s\)阶可逆矩阵\(\vb{P}_1\)和\(\vb{P}_2\),
存在\(n\)阶可逆矩阵\(\vb{Q}_1\)和\(\vb{Q}_2\),
使得\begin{equation*}
	\vb{P}_1 \vb{A} \vb{Q}_1 = \vb{P}_2 \vb{A} \vb{Q}_2
	= \begin{bmatrix} \vb{E}_r & \vb0 \\ \vb0 & \vb0 \end{bmatrix},
\end{equation*}
令\(\vb{P} = \vb{P}_2^{-1} \vb{P}_1\),\(\vb{Q} = \vb{Q}_1 \vb{Q}_2^{-1}\),
则\(\vb{P}\)和\(\vb{Q}\)可逆,
\(\vb{P} \vb{A} \vb{Q} = \vb{B}\),
从而\(\vb{A} \cong \vb{B}\).
\end{proof}
\end{theorem}

\begin{example}\label{example:矩阵乘积的秩.可交换矩阵之和的秩}
设\(\vb{A},\vb{B}\in M_n(K)\),
且\(\vb{A} \vb{B}=\vb{B}\vb{A}\).
证明:\begin{equation*}
	\rank(\vb{A}+\vb{B})\leq\rank\vb{A}+\rank\vb{B}-\rank(\vb{A} \vb{B}).
\end{equation*}
\begin{proof}
考虑\begin{equation*}
	\begin{bmatrix}
		\vb{E} & \vb{E} \\
		\vb0 & \vb{E}
	\end{bmatrix}
	\begin{bmatrix}
		\vb{A} & \vb0 \\
		\vb0 & \vb{B}
	\end{bmatrix}
	\begin{bmatrix}
		\vb{E} & -\vb{B} \\
		\vb{E} & \vb{A}
	\end{bmatrix}
	= \begin{bmatrix}
		\vb{A}+\vb{B} & -\vb{A} \vb{B}+\vb{B}\vb{A} \\
		\vb{B} & \vb{B}\vb{A}
	\end{bmatrix}
	= \begin{bmatrix}
		\vb{A}+\vb{B} & \vb0 \\
		\vb{B} & \vb{A} \vb{B}
	\end{bmatrix}.
\end{equation*}
由\cref{equation:矩阵的秩.分块矩阵的秩的等式1} 可知\begin{equation*}
	\rank\vb{A}+\rank\vb{B}
	= \rank\begin{bmatrix}
		\vb{A} & \vb0 \\
		\vb0 & \vb{B}
	\end{bmatrix}
	\geq \rank\begin{bmatrix}
		\vb{A}+\vb{B} & \vb0 \\
		\vb{B} & \vb{B}\vb{A}
	\end{bmatrix}
	\geq \rank(\vb{A}+\vb{B}) + \rank(\vb{A} \vb{B}).
	\qedhere
\end{equation*}
\end{proof}
\end{example}

\begin{example}\label{example:矩阵乘积的秩.两个向量的乘积的秩}
设\(\vb\alpha,\vb\beta\)是\(n\)维非零列向量.
证明:\(\rank(\vb\alpha\vb\beta^T)=1\).
\begin{proof}
因为\(\vb\alpha,\vb\beta\neq\vb0\),
所以\(\rank\vb\alpha=\rank\vb\beta=1\),
且\(\vb\alpha\vb\beta^T\neq\vb0\),
从而\(\rank(\vb\alpha\vb\beta^T)>0\),
再根据\cref{theorem:线性方程组.矩阵乘积的秩} 可知
\(\rank(\vb\alpha\vb\beta^T)
\leq \min\{\rank\vb\alpha,\rank\vb\beta^T\}
= 1\),
因此\(\rank(\vb\alpha\vb\beta^T)=1\).
\end{proof}
\end{example}

\begin{example}\label{example:矩阵乘积的秩.分块矩阵的秩的等式2}
设\(\vb{A} \in M_{s \times n}(K),
\vb{B} \in M_{s \times m}(K)\).
证明:\begin{equation}
	\max\{\rank\vb{A},\rank\vb{B}\} \leq \rank(\vb{A},\vb{B}) \leq \rank\vb{A} + \rank\vb{B}.
\end{equation}
%@credit: {e9b17d8d-3be5-4f44-9c7a-a5e6122a69e2} 提出取等条件如下:
当且仅当\(\vb{A}\)的列空间包含于\(\vb{B}\)的列空间,或\(\vb{B}\)的列空间包含于\(\vb{A}\)的列空间时,成立\begin{equation*}
	\max\{\rank\vb{A},\rank\vb{B}\} = \rank(\vb{A},\vb{B}).
\end{equation*}
当且仅当\((\vb{A},\vb{B})\)的列空间等于\(\vb{A}\)的列空间与\(\vb{B}\)的列空间的直和时,成立\begin{equation*}
	\rank(\vb{A},\vb{B}) = \rank\vb{A} + \rank\vb{B}.
\end{equation*}
%TODO 取等条件留待以后证明
\begin{proof}
\def\as{\AutoTuple{\vb\alpha}{n}}
\def\bs{\AutoTuple{\vb\beta}{m}}
\def\asi{\vb\alpha_{i_1},\dotsc,\vb\alpha_{i_r}}
\def\bsj{\vb\beta_{j_1},\dotsc,\vb\beta_{j_t}}
设\(\rank\vb{A} = r,
\rank\vb{B} = t\).
对\(\vb{A}\)、\(\vb{B}\)分别按列分块得\begin{equation*}
	\vb{A} = (\as),
	\qquad
	\vb{B} = (\bs).
\end{equation*}
易见\begin{equation*}
	(\vb{A},\vb{B}) = (\as,\bs),
\end{equation*}且\begin{equation*}
	\rank\{\as\} = r,
	\qquad
	\rank\{\bs\} = t.
\end{equation*}

假设\(\as\)可由其极大线性无关组\(\asi\)线性表出,
\(\bs\)可由其极大线性无关组\(\bsj\)线性表出,
那么向量组\begin{equation*}
	V_1=\{\as\}\cup\{\bs\}
\end{equation*}可由向量组\begin{equation*}
	V_2=\{\asi\}\cup\{\bsj\}
\end{equation*}线性表出,
即\(V_1 \subseteq \Span V_2\),
则由\cref{theorem:向量空间.向量组的秩的比较1} 有\begin{equation*}
	\rank V_1
	\leq
	\rank V_2
	\leq
	r+t;
\end{equation*}
于是\(\rank(\vb{A},\vb{B}) = \rank V_1 \leq r+t\).

又因为\hyperref[theorem:向量空间.向量组的秩的比较2]{部分组的秩总是小于或等于全组的秩},
而\begin{equation*}
	\{\asi\},\{\bsj\} \subseteq V_1,
\end{equation*}
所以\begin{equation*}
	\rank\{\asi\},\rank\{\bsj\} \leq \rank V_1,
\end{equation*}
于是\(\max\{\rank\vb{A},\rank\vb{B}\} \leq \rank(\vb{A},\vb{B})\).
\end{proof}
\end{example}

\begin{example}\label{example:矩阵乘积的秩.任意同型矩阵之和的秩}
%\cref{example:矩阵乘积的秩.可交换矩阵之和的秩}
设\(\vb{A}\)、\(\vb{B}\)都是\(s \times n\)矩阵.
证明:\begin{equation*}
	\rank(\vb{A}+\vb{B}) \leq \rank\vb{A} + \rank\vb{B}.
\end{equation*}
\begin{proof}
\def\asi{\vb\alpha_{i_1},\vb\alpha_{i_2},\dotsc,\vb\alpha_{i_r}}
\def\bsj{\vb\beta_{j_1},\vb\beta_{j_2},\dotsc,\vb\beta_{j_t}}
设\(\rank\vb{A} = r\),\(\rank\vb{B} = t\).
对\(\vb{A}\)、\(\vb{B}\)分别按列分块得\begin{equation*}
	\vb{A} = (\AutoTuple{\vb\alpha}{n}), \qquad
	\vb{B} = (\AutoTuple{\vb\beta}{m}),
\end{equation*}
则\begin{equation*}
	\vb{A} + \vb{B} = (\vb\alpha_1 + \vb\beta_1,\vb\alpha_2 + \vb\beta_2,\dotsc,\vb\alpha_n + \vb\beta_n).
\end{equation*}
由于\(\AutoTuple{\vb\alpha}{n}\)可由其极大线性无关组\(\asi\)线性表出,
\(\AutoTuple{\vb\beta}{m}\)可由其极大线性无关组\(\bsj\)线性表出,
故\begin{equation*}
	\vb\alpha_1 + \vb\beta_1,\vb\alpha_2 + \vb\beta_2,\dotsc,\vb\alpha_n + \vb\beta_n
\end{equation*}可由向量组\begin{equation*}
	\asi,\bsj
\end{equation*}线性表出,
结论显然成立.
\end{proof}
\end{example}
\begin{remark}
\cref{example:矩阵乘积的秩.任意同型矩阵之和的秩} 的意义在于:
既然由\cref{theorem:线性方程组.矩阵的秩的性质3}
可知一个矩阵乘以非零常数不变秩,
那么对于一个由矩阵构成的一次多项式,
我们可以把其中的某些项消掉,
像这样:\begin{equation*}
	\rank(x_1\vb{A}+x_2\vb{B}) + \rank(y_1\vb{A}+y_2\vb{B})
	\geq
	\max\left\{
		\rank\left( k \vb{A} \right),
		\rank\left( k \vb{B} \right)
	\right\},
\end{equation*}
其中\(k = \begin{vmatrix}
	x_1 & x_2 \\
	y_1 & y_2
\end{vmatrix}\);
如果进一步有\(k \neq 0\),
则有\begin{equation*}
	\rank(x_1\vb{A}+x_2\vb{B}) + \rank(y_1\vb{A}+y_2\vb{B})
	\geq
	\max\left\{
		\rank\vb{A},
		\rank\vb{B}
	\right\}.
\end{equation*}
\end{remark}

\begin{example}\label{example:矩阵乘积的秩.矩阵的一次多项式的秩之和}
%\cref{example:矩阵乘积的秩.矩阵的一次多项式的秩之和.取等条件1}
设\(\vb{A}\)是\(n\)阶矩阵.
证明:\(n \leq \rank(\vb{A} + \vb{E}) + \rank(\vb{A} - \vb{E})\).
\begin{proof}
由于\(\rank(\vb{A} - \vb{E}) = \rank(\vb{E} - \vb{A})\),
所以\begin{equation*}
	\rank(\vb{A} + \vb{E}) + \rank(\vb{A} - \vb{E})
	= \rank(\vb{A} + \vb{E}) + \rank(\vb{E} - \vb{A}).
\end{equation*}
由\cref{example:矩阵乘积的秩.可交换矩阵之和的秩} 可知\begin{equation*}
	\rank(\vb{A} + \vb{E}) + \rank(\vb{E} - \vb{A})
	\geq \rank(\vb{A} + \vb{E} + \vb{E} - \vb{A})
	= \rank(2\vb{E})
	= \rank\vb{E}
	= n.
\end{equation*}
因此\(\rank(\vb{A} + \vb{E}) + \rank(\vb{A} - \vb{E}) \geq n\).
\end{proof}
\end{example}
\begin{example}\label{example:矩阵乘积的秩.矩阵的多项式的各个互素因式的秩之和}
设\(n\)阶矩阵\(\vb{A}\)满足\(\vb{A}^2 - 3 \vb{A} - 10 \vb{E} = \vb0\).
证明:\begin{equation*}
	\rank(\vb{A} - 5\vb{E}) + \rank(\vb{A} + 2\vb{E}) = n.
\end{equation*}
\begin{proof}
因为\(\vb{A}\)满足\(\vb{A}^2 - 3 \vb{A} - 10 \vb{E} = (\vb{A} - 5\vb{E})(\vb{A} + 2\vb{E}) = \vb0\),
所以由\cref{example:矩阵乘积的秩.乘积为零的两个矩阵的秩之和} 可知\begin{equation*}
	\rank(\vb{A} - 5\vb{E}) + \rank(\vb{A} + 2\vb{E}) \leq n.
\end{equation*}
又因为\begin{align*}
	&\rank(\vb{A} - 5\vb{E}) + \rank(\vb{A} + 2\vb{E}) \\
	&\geq \rank[(5\vb{E} - \vb{A}) + (\vb{A} + 2\vb{E})] \\
	&= \rank(7\vb{E})
	= n,
\end{align*}
所以\(\rank(\vb{A} - 5\vb{E}) + \rank(\vb{A} + 2\vb{E}) = n\).
\end{proof}
\end{example}
\begin{example}
%@credit: {5a781423-ba4e-4629-ac1a-eac743a4d445},{8b6edada-f2fd-4ae5-9020-eb533149a54c}
设数域\(K\)上的一个一元多项式\(f(x)\)可以分解为\(m\)个互素的多项式的乘积,
即\begin{equation*}
	f(x) = p_1(x) \cdot p_2(x) \dotsm p_m(x).
\end{equation*}
证明:如果矩阵\(\vb{A} \in M_n(K)\)满足\(f(\vb{A}) = \vb0\),
则\begin{equation*}
	\sum_{k=1}^m \rank(p_k(A))
	= (m-1)n.
\end{equation*}
%TODO proof 具体证明过程参考2024年10月22日凌晨在[小飞机群]的聊天记录
\end{example}

\begin{example}
设\(\vb{A} \in M_{s \times n}(K)\ (s \neq n)\).
证明:\(\det(\vb{A} \vb{A}^T) \det(\vb{A}^T \vb{A}) = 0\).
\begin{proof}
由\cref{theorem:矩阵乘积的秩.多行少列矩阵与少行多列矩阵的乘积的行列式} 可知,
要么成立\(\det(\vb{A} \vb{A}^T) = 0\),
要么成立\(\det(\vb{A}^T \vb{A}) = 0\),
但总归有\begin{equation*}
	\det(\vb{A} \vb{A}^T) \det(\vb{A}^T \vb{A}) = 0.
	\qedhere
\end{equation*}
\end{proof}
\end{example}

\begin{example}
设\(\vb{A}\)是\(m \times n\)矩阵,
\(\vb{B}\)是\(n \times m\)矩阵,
\(\vb{E}\)是\(m\)阶单位矩阵,
且\(\vb{A} \vb{B} = \vb{E}\).
求\(\vb{A}\)与\(\vb{B}\)的秩.
\begin{solution}
假设\(m > n\),
由\cref{theorem:矩阵乘积的秩.多行少列矩阵与少行多列矩阵的乘积的行列式}
可知\(\abs{\vb{A} \vb{B}} = 0\),
但是由题设条件\(\vb{A} \vb{B} = \vb{E}\)可知\begin{equation*}
	\abs{\vb{A} \vb{B}} = \abs{\vb{E}} = 1 \neq 0,
\end{equation*}
矛盾,故必有\(m \leq n\).
那么\begin{equation*}
	\rank\vb{A},\rank\vb{B} \leq \min\{m,n\} = m,
\end{equation*}
从而有\begin{equation*}
	\min\{\rank\vb{A},\rank\vb{B}\} \leq m.
\end{equation*}
由\cref{theorem:线性方程组.矩阵乘积的秩} 可知\begin{equation*}
	\rank(\vb{A} \vb{B}) \leq \min\{\rank\vb{A},\rank\vb{B}\},
\end{equation*}
即\(\rank(\vb{A} \vb{B}) \leq m\).
由题设条件\(\vb{A} \vb{B} = \vb{E}\)可知\begin{equation*}
	\rank(\vb{A} \vb{B}) = \rank\vb{E} = m.
\end{equation*}
那么\begin{equation*}
	m \leq \rank\vb{A} \leq m,
	\qquad
	m \leq \rank\vb{B} \leq m,
\end{equation*}
于是\(\rank\vb{A} = \rank\vb{B} = m\).
\end{solution}
\end{example}

\begin{example}
计算行列式\(\det\vb{D}\),
其中\begin{equation*}
	\vb{D} = \begin{bmatrix}
		1 & \cos(\alpha_1-\alpha_2) & \cos(\alpha_1-\alpha_3) & \dots & \cos(\alpha_1-\alpha_n) \\
		\cos(\alpha_1-\alpha_2) & 1 & \cos(\alpha_2-\alpha_3) & \dots & \cos(\alpha_2-\alpha_n) \\
		\cos(\alpha_1-\alpha_3) & \cos(\alpha_2-\alpha_3) & 1 & \dots & \cos(\alpha_3-\alpha_n) \\
		\vdots & \vdots & \vdots & & \vdots \\
		\cos(\alpha_1-\alpha_n) & \cos(\alpha_2-\alpha_n) & \cos(\alpha_3-\alpha_n) & \dots & 1
	\end{bmatrix}.
\end{equation*}
\begin{solution}
记\(\vb{D} = (d_{ij})_n\).
由\cref{equation:函数.三角函数.和积互化公式2} 可知\begin{equation*}
	d_{ij} = \cos(\alpha_i-\alpha_j)
	= \cos\alpha_i\cos\alpha_j+\sin\alpha_i\sin\alpha_j,
	\quad i,j=1,2,\dotsc,n.
\end{equation*}
记\begin{equation*}
	\vb{A} = \begin{bmatrix}
		\cos\alpha_1 & \cos\alpha_2 & \dots & \cos\alpha_n \\
		\sin\alpha_1 & \sin\alpha_2 & \dots & \sin\alpha_n
	\end{bmatrix},
\end{equation*}那么\(\vb{D} = \vb{A}^T \vb{A}\).

由\cref{theorem:线性方程组.矩阵的秩的性质2} 可知,
\(\rank\vb{A} = \rank\vb{A}^T \leq \min\{n,2\}\).
由\cref{theorem:线性方程组.矩阵乘积的秩} 可知,\begin{equation*}
	\rank(\vb{A}^T\vb{A}) \leq \min\left\{\rank\vb{A}^T,\rank\vb{A}\right\} = \rank\vb{A}.
\end{equation*}

当\(n=1\)时,\(\abs{\vb{D}}=1\).

当\(n=2\)时,\begin{equation*}
	\abs{\vb{D}}
	= \begin{vmatrix}
		1 & \cos(\alpha_1-\alpha_2) \\
		\cos(\alpha_1-\alpha_2) & 1
	\end{vmatrix}
	= 1 - \cos^2(\alpha_1-\alpha_2).
\end{equation*}

当\(n>2\)时,
\(\rank(\vb{A}^T\vb{A}) \leq \rank\vb{A} \leq2\),
\(\vb{D} = \vb{A}^T \vb{A}\)不满秩,
故由\cref{theorem:向量空间.满秩方阵的行列式非零} 有\(\abs{\vb{D}}=0\).
\end{solution}
\end{example}

\begin{example}
设\(\vb\alpha_1=(1,2,-1,0)^T,\vb\alpha_2=(1,1,0,2)^T,\vb\alpha_3=(2,1,1,a)^T\).
若\(\dim\opair{\AutoTuple{\vb\alpha}{3}}=2\),求\(a\).
\begin{solution}
除了利用\cref{equation:线性方程组.子空间的维数与向量组的秩的联系} 在将矩阵\begin{equation*}
	\vb{A} = (\vb\alpha_1,\vb\alpha_2,\vb\alpha_3)
	= \begin{bmatrix}
		1 & 1 & 2 \\
		2 & 1 & 1 \\
		-1 & 0 & 1 \\
		0 & 2 & a
	\end{bmatrix}
\end{equation*}化为阶梯形矩阵以后,
根据\(\rank\{\AutoTuple{\vb\alpha}{3}\}=2\)求出\(a\)的值这种方法以外,
我们还可以利用本节\hyperref[definition:线性方程组.矩阵的秩的定义]{矩阵的秩的定义},
得出\(\rank\vb{A}=\dim\opair{\AutoTuple{\vb\alpha}{3}}=2\)这一结论,
从而根据\cref{definition:线性方程组.矩阵的秩的定义} 可知,
\(\vb{A}\)中任意3阶子式全都为零.
对于\(\vb{A}\)这么一个\(4\times3\)矩阵,
任意去掉不含\(a\)的一行(不妨去掉第一行)得到一个行列式必为零:\begin{equation*}
	\begin{vmatrix}
	2 & 1 & 1 \\
	-1 & 0 & 1 \\
	0 & 2 & a
	\end{vmatrix}
	= a - 6 = 0,
\end{equation*}
解得\(a = 6\).
\end{solution}
\end{example}
