\section{矩阵的秩}
\subsection{矩阵的行秩与列秩}
为了求解向量组的秩,我们可以把向量组看成矩阵的行向量组或列向量组,
利用矩阵的性质,得出这个矩阵的行向量组、列向量组的秩,
最后得到所求向量组的秩.

\begin{definition}\label{definition:线性方程组.行秩与列秩的定义}
%@see: 《线性代数》(张慎语、周厚隆) P76 定义11(1)
设\(\A\)是数域\(K\)上的\(s \times n\)矩阵.
\begin{itemize}
	\item \(\A\)的列向量组生成的子空间\[
		\Set{
			\A\x \given \x=(x_1,\dotsc,x_n)^T \in K^n
		}
	\]
	称为“\(\A\)的\DefineConcept{列空间}(column space)”,
	记为\(\SpanC\A\).

	有时候我们把矩阵的列空间称为它的\DefineConcept{像空间},
	记作\(\Img\A\).

	\item \(\A\)的行向量组生成的子空间\[
		\Set{
			\x^T\A \given \x=(x_1,\dotsc,x_s)^T \in K^s
		}
	\]
	称为“\(\A\)的\DefineConcept{行空间}(row space)”,
	记为\(\SpanR\A\).

	\item \(\A\)的行向量组的秩
	称为“\(\A\)的\DefineConcept{行秩}(row rank)”,
	记为\(\RankR\A\).

	\item \(\A\)的列向量组的秩
	称为“\(\A\)的\DefineConcept{列秩}(column rank)”,
	记为\(\RankC\A\).
\end{itemize}
%@see: https://mathworld.wolfram.com/RowSpace.html
%@see: https://mathworld.wolfram.com/ColumnSpace.html
\end{definition}

矩阵的列秩等于它的列空间的维数,它的行秩等于它的行空间的维数,即\[
	\RankC\A=\dim(\SpanC\A), \qquad
	\RankR\A=\dim(\SpanR\A).
\]

\begin{proposition}\label{theorem:向量空间.矩阵的行秩与列秩分别等于它的转置矩阵的列秩与行秩}
设\(\A\)是矩阵,那么\begin{gather}
	\RankR\A=\RankC\A^T, \\
	\RankC\A=\RankR\A^T.
\end{gather}
\begin{proof}
显然\(\A\)的行向量组就是\(\A^T\)的列向量组,
而\(\A\)的列向量组就是\(\A^T\)的行向量组.
\end{proof}
\end{proposition}

\begin{definition}
设矩阵\(\A \in M_{s \times n}(K)\).
\begin{itemize}
	\item 若\(\RankR\A = s\),
	则称“\(\A\)为\DefineConcept{行满秩矩阵}(row full rank matrix)”.
	\item 若\(\RankC\A = n\),
	则称“\(\A\)为\DefineConcept{列满秩矩阵}(column full rank matrix)”.
\end{itemize}
\end{definition}

现在我们来研究一个问题:矩阵的列秩与行秩之间有什么联系?

\subsection{矩阵的秩}
\begin{definition}\label{definition:线性方程组.矩阵的秩的定义}
%@see: 《线性代数》(张慎语、周厚隆) P76 定义11(2)
数域\(K\)上的\(s \times n\)矩阵\(\A\)的最高阶非零子式的阶数,
% the highest order of nonzero minor of \(\A\)
称为“矩阵\(\A\)的\DefineConcept{秩}(rank)”,
记为\(\rank\A\).
%@see: https://mathworld.wolfram.com/MatrixRank.html
\end{definition}

\begin{proposition}\label{theorem:向量空间.零矩阵的秩为零}
零矩阵的秩为零,即\(\rank\vb0 = 0\).
\end{proposition}
\begin{proposition}\label{theorem:向量空间.秩为零的矩阵必为零矩阵}
秩为零的矩阵必为零矩阵.
\end{proposition}

\begin{property}\label{theorem:线性方程组.矩阵的秩的性质2}
设\(\A\)是数域\(K\)上的\(s \times n\)矩阵,则\(0 \leq \rank\A \leq \min\{s,n\}\).
\end{property}

\begin{property}\label{theorem:线性方程组.矩阵的秩的性质3}
设\(\A\)是数域\(K\)上的矩阵,\(k \in K-\{0\}\),
则\(\rank(k\A) = \rank\A\).
\end{property}

\begin{property}
设矩阵\(\A \in M_{s \times n}(K)\),
则\begin{equation*}
	\rank(\A,\vb0) = \rank\A.
\end{equation*}
\end{property}

\begin{theorem}
设\(\A\in M_{s \times n}(K)\).
如果\(\A\)有一个\(k\)阶非零子式,那么\(\rank\A\geq k\).
\begin{proof}
用反证法.
假设\(\rank\A<k\),
也就是说\(\A\)的非零子式的最高阶数\(r\)小于\(k\),
但是根据前提条件,\(\A\)有一个\(k\)阶非零子式,\(k>r\),
这就和\hyperref[definition:线性方程组.矩阵的秩的定义]{矩阵的秩的定义}矛盾!
因此\(\rank\A\geq k\).
\end{proof}
\end{theorem}
\begin{corollary}
设矩阵\(\A \in M_{s \times n}(K)\),
\(\E\)是数域\(K\)上的\(s\)阶单位矩阵,
则\begin{equation*}
	\rank(\A,\E) = s.
\end{equation*}
\end{corollary}

\begin{example}
求矩阵\(\A = \begin{bmatrix} 1 & 7 \\ 2 & 6 \\ -3 & 1 \end{bmatrix}\)的秩、行秩与列秩.
\begin{solution}
因为至少存在一个\(\A\)的二阶子式不为零,即\[
	\begin{vmatrix} 1 & 7 \\ 2 & 6 \end{vmatrix} = -8 \neq 0,
\]
且\(\A\)没有三阶子式,所以\(\rank\A = 2\).

写出\(\A\)的行向量组\[
	(1,7), \qquad
	(2,6), \qquad
	(-3,1),
\]
分别转置后,列成一个矩阵(显然就是\(\A^T\)),
作初等行变换化成阶梯形矩阵\[
	\A^T = \begin{bmatrix}
		1 & 2 & -3 \\
		7 & 6 & 1
	\end{bmatrix} \to \begin{bmatrix}
		1 & 0 & 17/2 \\
		0 & 4 & -11
	\end{bmatrix} = \B_1.
\]阶梯形矩阵\(\B_1\)有2行不为零,故矩阵\(\A\)的行秩为2.

写出\(\A\)的列向量组\[
	(1,2,-3)^T,
	(7,6,1)^T,
\]列成一个矩阵(显然就是\(\A\)本身),
作初等行变换化成阶梯形矩阵\[
	\A = \begin{bmatrix} 1 & 7 \\ 2 & 6 \\ -3 & 1 \end{bmatrix}
	\to \begin{bmatrix} 1 & 0 \\ 0 & 1 \\ 0 & 0 \end{bmatrix} = \B_2.
\]
阶梯形矩阵\(\B_2\)有2行不为零,故矩阵\(\A\)的列秩为2.
\end{solution}
\end{example}
\begin{remark}
从这个例子可以看出,矩阵\(\A = \begin{bmatrix} 1 & 7 \\ 2 & 6 \\ -3 & 1 \end{bmatrix}\)的秩、行秩与列秩全都相等.
这让我们不禁好奇:是否任意矩阵的秩、行秩与列秩也全都相等?
\end{remark}

\subsection{矩阵的秩、行秩、列秩的关系}
\begin{lemma}\label{theorem:向量空间.矩阵的秩与行秩和列秩的关系.引理}
%@see: 《线性代数》(张慎语、周厚隆) P77 引理
设矩阵\(\A = (a_{ij})_{s \times n}\)的列秩等于\(\A\)的列数\(n\),
则\(\A\)行秩、秩都等于\(n\).
\begin{proof}
对\(\A\)分别进行列分块与行分块:\[
	\A = (\AutoTuple{\a}{n})
	= (\AutoTuple{\A}{s}[,][T])^T.
\]
根据\cref{theorem:向量空间.秩与线性相关性的关系},
由于\(\RankC\A=n\),
所以\(\A\)的列向量组线性无关,
也就是说齐次线性方程组\[
	x_1 \a_1 + x_2 \a_2 + \dotsb + x_n \a_n = \z
\]只有零解,或者说\(\A\x=\z\)只有零解.
进而根据\cref{theorem:线性方程组.方程个数少于未知量个数的齐次线性方程组必有非零解},
在\(\A\x=\z\)中,方程个数\(s\)不少于未知量个数\(n\),即\(s \geq n\).

设\(\RankR\A=t\).
根据\cref{theorem:向量空间.子空间维数.命题4},
因为\(\A\)的行向量的维数是\(n\),
所以\(t \leq n\).

设\(\A\)的行极大线性无关组是
\(\A_{i_1},\A_{i_2},\dotsc,\A_{i_t}\),
其中\(1 \leq i_1 < i_2 < \dotsb < i_t \leq s\).

假设\(t < n\),
则\(\A\)可通过初等行变换化为\[
	\begin{bmatrix}
		\A_{i_1} \\ \A_{i_2} \\ \vdots \\ \A_{i_t} \\ \z_{(s-t) \times n}
	\end{bmatrix},
\]
于是,\(\A\x=\z\)表示的齐次线性方程组中,
非零方程的个数\(t\)小于未知量个数\(n\),
那么根据\cref{theorem:线性方程组.方程个数少于未知量个数的齐次线性方程组必有非零解},
\(\A\x=\z\)有非零解,
这与我们前面得到的结论“\(\A\x=\z\)只有零解”矛盾,
所以假设\(t < n\)不成立,
因此\(t = n\).
这就是说\[
	\RankR\A=\RankC\A=n.
\]

既然\(t=n\),
那么\(\A\)的行极大线性无关组恰好可以构成一个方阵,
而这个方阵的行列式为\[
	d=\begin{vmatrix} \A_{i_1} \\ \A_{i_2} \\ \vdots \\ \A_{i_n} \end{vmatrix}.
\]
根据\cref{theorem:线性方程组.克拉默法则},
因为\(\A\x=\z\)只有零解作为它的唯一解,
所以\(d\neq0\).
于是\(d\)是\(\A\)的一个\(n\)阶非零子式.
\(\A\)是一个\(s \times n\)矩阵,
不可能有阶数大于\(n\)的子式,
因此\(\rank\A = n\).
\end{proof}
\end{lemma}

\begin{theorem}\label{theorem:向量空间.矩阵的秩与行秩和列秩的关系.定理}
%@see: 《线性代数》(张慎语、周厚隆) P77 定理5
矩阵的行秩、列秩、秩都相等.
\begin{proof}
如果矩阵\(\A = \z\),
则\(\rank\A=\RankR\A=\RankC\A=0\),
结论成立.

当\(\A\neq\z\)时,
设\(\rank\A=r\),
则根据\hyperref[definition:线性方程组.矩阵的秩的定义]{矩阵的秩的定义},
\(\A\)的所有\(t\ (t > r)\)阶子式全为零;
且\(\A\)有一个\(r\)阶子式不为零,
从而该\(r\)阶子式的列向量组线性无关,
它们的延伸组也线性无关,
\(\A\)有\(r\)列线性无关,
于是\(\RankC\A=p \geq r\);
由\cref{theorem:向量空间.矩阵的秩与行秩和列秩的关系.引理},
\(\A\)的列极大线性无关组构成的矩阵有一非零\(p\)阶子式,
也是\(\A\)的子式,
故\(p \leq r\);
综上得\(p = r\).
因此\(\rank\A=\RankC\A\).
同样地,将这一结论用于\(\A^T\),
得\(\rank\A^T=\RankC\A^T=\RankR\A\).
\end{proof}
\end{theorem}

\begin{theorem}\label{theorem:向量空间.转置不变秩}
%@see: 《高等代数(第三版 上册)》(丘维声) P83 推论6
设\(\A\)是矩阵,那么\(\rank\A = \rank\A^T\).
\begin{proof}
根据\cref{theorem:向量空间.矩阵的秩与行秩和列秩的关系.定理,theorem:向量空间.矩阵的行秩与列秩分别等于它的转置矩阵的列秩与行秩},
有\(\rank\A=\RankC\A=\RankR\A^T=\rank\A^T\).
\end{proof}
\end{theorem}

\begin{example}
设\(\A,\B \in M_n(K)\),
则\[
	\rank\begin{bmatrix}
		\A \\ \B
	\end{bmatrix}
	= \rank\begin{bmatrix}
		\A \\ \B
	\end{bmatrix}^T
	= \rank(\A^T,\B^T).
\]
\end{example}
\begin{remark}
%@credit: {ba6d5dc2-c9a6-4590-bfd1-1cd33fae3bd3}
需要注意的是,矩阵\((\A,\B)\)与\((\A^T,\B^T)\)的秩不一定相等.
例如,取\[
	\A = \begin{bmatrix}
		1 & 0 \\
		0 & 0
	\end{bmatrix},
	\qquad
	\B = \begin{bmatrix}
		0 & 1 \\
		0 & 0
	\end{bmatrix},
\]
然而\[
	\rank(\A,\B)
	= \rank\begin{bmatrix}
		1 & 0 & 0 & 1 \\
		0 & 0 & 0 & 0
	\end{bmatrix}
	= 1
	\neq
	\rank(\A^T,\B^T)
	= \rank\begin{bmatrix}
		1 & 0 & 0 & 0 \\
		0 & 0 & 1 & 0
	\end{bmatrix}
	= 2.
\]
\end{remark}

\subsection{初等变换对矩阵的秩的影响}
\begin{theorem}\label{theorem:线性方程组.初等变换不变秩}
%@see: 《高等代数(第三版 上册)》(丘维声) P81 定理2
%@see: 《高等代数(第三版 上册)》(丘维声) P82 定理3
%@see: 《高等代数(第三版 上册)》(丘维声) P83 推论7
初等变换不改变矩阵的秩.
\begin{proof}
由\cref{theorem:向量空间.利用初等行变换求取列极大线性无关组的依据},
初等行变换不改变矩阵的列秩;
同理,初等列变换不改变矩阵的行秩;
再由\cref{theorem:向量空间.矩阵的秩与行秩和列秩的关系.定理},
初等变换不改变矩阵的秩.
\end{proof}
\end{theorem}

\begin{example}
%@see: 《高等代数(第三版 上册)》(丘维声) P86 习题3.5 12.
%@see: 《高等代数学习指导书(第二版 上册)》(丘维声) P115 例8
设\(\A \in M_{s \times n}(K),
\B \in M_{l \times m}(K)\).
试证:\begin{equation}
	\rank\begin{bmatrix}
		\A & \z \\
		\z & \B
	\end{bmatrix}
	= \rank\A + \rank\B.
	\label{equation:矩阵的秩.分块矩阵的秩的等式1}
\end{equation}
\begin{proof}
\def\J{\vb{J}}
若\(\A\)或\(\B\)是零矩阵,
则必有\(\rank\A=0\)或\(\rank\B=0\),
这时候\cref{equation:矩阵的秩.分块矩阵的秩的等式1} 显然成立.
下面我们假设\(\A\)和\(\B\)的秩都大于零,
对矩阵\(\begin{bmatrix}
	\A & \z \\
	\z & \B
\end{bmatrix}\)的前\(s\)行、后\(l\)行分别作初等行变换,化成\[
	\begin{bmatrix}
		\J_r & \z \\
		\z & \z \\
		\z & \J_t \\
		\z & \z
	\end{bmatrix},
	\eqno(1)
\]
其中\(\J_r\)是\(r \times n\)阶的阶梯形矩阵,且\(r\)行都是非零行,\(\rank\A=r\);
\(\J_t\)是\(t \times m\)阶的阶梯形矩阵,且\(t\)行都是非零行,\(\rank\B=t\).
对矩阵(1)作一系列的两行互换,化成\[
	\begin{bmatrix}
		\J_r & \z \\
		\z & \J_t \\
		\z & \z
	\end{bmatrix},
	\eqno(2)
\]
矩阵(2)是阶梯形矩阵,它有\(r+t\)个非零行.
因此\[
	\rank\begin{bmatrix}
		\A & \z \\
		\z & \B
	\end{bmatrix}
	= r+t
	= \rank\A+\rank\B.
	\qedhere
\]
\end{proof}
\end{example}

现在我们来对\cref{equation:矩阵的秩.分块矩阵的秩的等式1} 做一番推广,
我们用一个非零矩阵\(\C\)取代分块对角矩阵\(\begin{bmatrix}
	\A & \z \\
	\z & \B
\end{bmatrix}\)的一个副对角,
得到分块上三角矩阵\(\begin{bmatrix}
	\A & \C \\
	\z & \B
\end{bmatrix}\)或分块下三角矩阵\(\begin{bmatrix}
	\A & \z \\
	\C & \B
\end{bmatrix}\),
接下来我们研究这类矩阵的秩的取值范围.
\begin{example}
%@see: 《高等代数(第三版 上册)》(丘维声) P87 习题3.5 13.
%@see: 《高等代数学习指导书(第二版 上册)》(丘维声) P116 例9
%@see: 《高等代数(第四版)》(谢启鸿 姚慕生) P157 例3.62
%@see: 《高等代数(第四版)》(谢启鸿 姚慕生) P170 例3.90
设\(\A \in M_{s \times n}(K),
\B \in M_{l \times m}(K),
\C \in M_{s \times m}(K)\).
试证:\begin{equation}
	\rank\begin{bmatrix}
		\A & \C \\
		\z & \B
	\end{bmatrix} \geq \rank\A + \rank\B.
	\label{equation:矩阵的秩.分块矩阵的秩的不等式}
\end{equation}
当且仅当,
关于\(\vb{X} \in M_{n \times m}(K)\)和\(\vb{Y} \in M_{s \times l}(K)\)的矩阵方程
\(\C=\A\vb{X}+\vb{Y}\B\)有解时,
上式取“\(=\)”号.
\begin{proof}
首先证明不等式成立.
设\(\rank\A=r_1,
\rank\B=r_2\),
则存在可逆矩阵\(\P_1,\P_2,\Q_1,\Q_2\),
使得\[
	\P_1\A\Q_1 = \begin{bmatrix}
		\E_{r_1} & \vb0 \\
		\vb0 & \vb0
	\end{bmatrix},
	\qquad
	\P_2\B\Q_2 = \begin{bmatrix}
		\E_{r_2} & \vb0 \\
		\vb0 & \vb0
	\end{bmatrix}.
\]
于是\[
	\begin{bmatrix}
		\P_1 & \vb0 \\
		\vb0 & \P_2
	\end{bmatrix}
	\begin{bmatrix}
		\A & \C \\
		\vb0 & \B
	\end{bmatrix}
	\begin{bmatrix}
		\Q_1 & \vb0 \\
		\vb0 & \Q_2
	\end{bmatrix}
	= \begin{bmatrix}
		\P_1\A\Q_1 & \P_1\C\Q_2 \\
		\vb0 & \P_2\B\Q_2
	\end{bmatrix}
	= \begin{bmatrix}
		\E_{r_1} & \vb0 & \C_{11} & \C_{12} \\
		\vb0 & \vb0 & \C_{21} & \C_{22} \\
		\vb0 & \vb0 & \E_{r_2} & \vb0 \\
		\vb0 & \vb0 & \vb0 & \vb0
	\end{bmatrix}.
\]
做初等变换,
用\(\E_{r_1}\)消去同行的矩阵,
用\(\E_{r_2}\)消去同列的矩阵,
再将\(\C_{22}\)对换到第二行第二列,
得到:\[
	\begin{bmatrix}
		\E_{r_1} & \vb0 & \C_{11} & \C_{12} \\
		\vb0 & \vb0 & \C_{21} & \C_{22} \\
		\vb0 & \vb0 & \E_{r_2} & \vb0 \\
		\vb0 & \vb0 & \vb0 & \vb0
	\end{bmatrix}
	\to \begin{bmatrix}
		\E_{r_1} & \vb0 & \vb0 & \vb0 \\
		\vb0 & \C_{22} & \vb0 & \vb0 \\
		\vb0 & \vb0 & \E_{r_2} & \vb0 \\
		\vb0 & \vb0 & \vb0 & \vb0
	\end{bmatrix}.
\]
于是,由\cref{equation:矩阵的秩.分块矩阵的秩的等式1} 有\[
	\rank\begin{bmatrix}
		\A & \C \\
		\z & \B
	\end{bmatrix}
	= \rank\E_{r_1} + \rank\C_{22} + \rank\E_{r2}
	\geq r_1 + r_2
	= \rank\A + \rank\B.
\]

然后证明取等条件.
\begin{itemize}
	\item 充分性.
	假设\(\vb{X} = \vb{X}_0, \vb{Y} = \vb{Y}_0\)是矩阵方程\(\A\vb{X}+\vb{Y}\B=\C\)的解,
	则\begin{align*}
		&\begin{bmatrix}
			\E & -\vb{Y}_0 \\
			\vb0 & \E
		\end{bmatrix}
		\begin{bmatrix}
			\A & \C \\
			\vb0 & \B
		\end{bmatrix}
		\begin{bmatrix}
			\E & -\vb{X}_0 \\
			\vb0 & \E
		\end{bmatrix}
		= \begin{bmatrix}
			\A & \C-\vb{Y}_0\B \\
			\vb0 & \B
		\end{bmatrix}
		\begin{bmatrix}
			\E & -\vb{X}_0 \\
			\vb0 & \E
		\end{bmatrix} \\
		&\hspace{20pt}= \begin{bmatrix}
			\A & \C-\vb{Y}_0\B-\A\vb{X}_0 \\
			\vb0 & \B
		\end{bmatrix}
		= \begin{bmatrix}
			\A & \vb0 \\
			\vb0 & \B
		\end{bmatrix},
	\end{align*}
	于是\[
		\rank\begin{bmatrix}
			\A & \C \\
			\vb0 & \B
		\end{bmatrix}
		= \rank\begin{bmatrix}
			\A & \vb0 \\
			\vb0 & \B
		\end{bmatrix}
		= \rank \A + \rank \B.
	\]

	\item 必要性.
	假设\(\rank \A=r_1,
	\rank \B=r_2\),
	且\[
		\P_1 \A \Q_1
		= \begin{bmatrix}
			\E_{r_1} & \vb0 \\
			\vb0 & \vb0
		\end{bmatrix},
		\qquad
		\P_2 \B \Q_2
		= \begin{bmatrix}
			\E_{r_2} & \vb0 \\
			\vb0 & \vb0
		\end{bmatrix},
	\]
	其中\(\P_1,\Q_1,\P_2,\Q_2\)是可逆矩阵.
	注意到问题的条件和结论在等价变换\[
		\A \mapsto \P_1 \A \Q_1, \qquad
		\B \mapsto \P_2 \B \Q_2, \qquad
		\C \mapsto \P_1 \C \Q_2, \qquad
		\vb{X} \mapsto \Q_1^{-1} \vb{X} \Q_2, \qquad
		\vb{Y} \mapsto \P_1 \vb{Y} \P_2^{-1},
	\]下保持不变,
	故不妨从一开始就假设\(\A,\B\)是等价标准型,
	即\[
		\A = \begin{bmatrix}
			\E_{r_1} & \vb0 \\
			\vb0 & \vb0
		\end{bmatrix}, \qquad
		\B = \begin{bmatrix}
			\E_{r_2} & \vb0 \\
			\vb0 & \vb0
		\end{bmatrix}.
	\]
	再对\(\C,\vb{X},\vb{Y}\)相应地分块,
	得到\[
		\C = \begin{bmatrix}
			\C_{11} & \C_{12} \\
			\C_{21} & \C_{22}
		\end{bmatrix},
		\qquad
		\vb{X} = \begin{bmatrix}
			\vb{X}_{11} & \vb{X}_{12} \\
			\vb{X}_{21} & \vb{X}_{22}
		\end{bmatrix},
		\qquad
		\vb{Y} = \begin{bmatrix}
			\vb{Y}_{11} & \vb{Y}_{12} \\
			\vb{Y}_{21} & \vb{Y}_{22}
		\end{bmatrix}.
	\]
	作初等变换:\[
		\begin{bmatrix}
			\A & \C \\
			\vb0 & \B
		\end{bmatrix}
		= \begin{bmatrix}
			\E_{r_1} & \vb0 & \C_{11} & \C_{12} \\
			\vb0 & \vb0 & \C_{21} & \C_{22} \\
			\vb0 & \vb0 & \E_{r_2} & \vb0 \\
			\vb0 & \vb0 & \vb0 & \vb0
		\end{bmatrix}
		\to \begin{bmatrix}
			\E_{r_1} & \vb0 & \vb0 & \vb0 \\
			\vb0 & \vb0 & \vb0 & \C_{22} \\
			\vb0 & \vb0 & \E_{r_2} & \vb0 \\
			\vb0 & \vb0 & \vb0 & \vb0
		\end{bmatrix}.
	\]
	因为\(\rank\begin{bmatrix}
		\A & \C \\
		\vb0 & \B
	\end{bmatrix}
	= \rank \A + \rank \B
	= r_1 + r_2\),
	所以\(\C_{22}\)必为零矩阵.
	于是矩阵方程\(\A\vb{X}+\vb{Y}\B=\C\)或者说\[
		\C = \begin{bmatrix}
			\C_{11} & \C_{12} \\
			\C_{21} & \vb0
		\end{bmatrix}
		= \begin{bmatrix}
			\vb{X}_{11} + \vb{Y}_{11} & \vb{X}_{12} \\
			\vb{Y}_{21} & \vb0
		\end{bmatrix}
		= \begin{bmatrix}
			\vb{X}_{11} & \vb{X}_{12} \\
			\vb0 & \vb0
		\end{bmatrix}
		+ \begin{bmatrix}
			\vb{Y}_{11} & \vb0 \\
			\vb{Y}_{21} & \vb0
		\end{bmatrix}
	\]有解.
	例如\(\vb{X}_{11} = \C_{11}, \vb{X}_{12} = \C_{12}, \vb{Y}_{11} = \vb0, \vb{Y}_{21} = \C_{21}\).
	\qedhere
\end{itemize}
\end{proof}
%@see: https://math.stackexchange.com/questions/4941561/whats-the-equality-condition-of-inequality-def-rank-operatornamerank-rank
\end{example}

\subsection{满秩矩阵}
\begin{definition}
%@see: 《线性代数》(张慎语、周厚隆) P76
设矩阵\(\A \in M_n(K)\).
若\(\rank\A = n\),
则称“\(\A\)是\DefineConcept{满秩矩阵}(full rank matrix)”,
或“\(\A\)是\DefineConcept{非退化矩阵}(non-degenerate matrix)”.
\end{definition}

%应该注意到,“可逆矩阵”“非奇异矩阵”“满秩矩阵”“非退化矩阵”是从不同侧重点对同一类矩阵的四种称谓.

\begin{theorem}\label{theorem:向量空间.满秩方阵的行列式非零}
%@see: 《高等代数(第三版 上册)》(丘维声) P84 推论9
设\(\A\in M_n(K)\).
\(\A\)是满秩矩阵的充分必要条件是\(\abs{\A}\neq0\).
\begin{proof}
\(\rank\A=n
\iff \text{\(A\)的非零子式的最高阶数为\(n\)}
\iff \abs{\A}\neq0\).
\end{proof}
\end{theorem}

\begin{corollary}
%@see: 《高等代数(第三版 上册)》(丘维声) P84 推论10
设\(\A\in M_{s \times n}(K)\)且\(\rank\A=r\).
那么,\(\A\)的\(r\)阶非零子式
所在的列构成\(\A\)的列向量组的一个极大线性无关组,
所在的行构成\(\A\)的行向量组的一个极大线性无关组.
\begin{proof}
将\(\A\)按列分块为\((\a_1,\a_2,\dotsc,\a_n)\),
并假设\(\rank\A = r\),
且\[
	\MatrixMinor\A{
		i_1,i_2,\dotsc,i_r \\
		j_1,j_2,\dotsc,j_r
	} \neq 0.
\]
于是这个\(r\)阶子式的列向量组线性无关.
那么它的延伸组\(\a_{j_1},\a_{j_2},\dotsc,\a_{j_r}\)线性无关.
由于\(\A\)的列秩为\(r\),
因此\(\a_{j_1},\a_{j_2},\dotsc,\a_{j_r}\)构成\(\A\)的列向量组的一个极大线性无关组.
\end{proof}
\end{corollary}
