\section{向量组的秩}
\subsection{向量组的等价关系}
\begin{definition}\label{definition:向量空间.线性表出2}
%@see: 《线性代数》(张慎语、周厚隆) P72 定义7
%@see: 《高等代数(第三版 上册)》(丘维声) P72 定义2
在\(K^n\)中,如果向量组\(A\)的每个向量都可由向量组\(B\)线性表出,
即\begin{equation*}
	(\forall \vb\alpha \in A)[\vb\alpha \in \Span B],
\end{equation*}或\begin{equation*}
	A\subseteq\Span B,
\end{equation*}
则称“向量组\(A\)可由向量组\(B\)~\DefineConcept{线性表出}”
或“向量组\(B\)可以\DefineConcept{线性表出}向量组\(A\)”;
否则称“向量组\(A\)不可由向量组\(B\)~线性表出”
或“向量组\(B\)不可以线性表出向量组\(A\)”.
\end{definition}

\begin{proposition}\label{theorem:向量空间.线性表出2的等价条件}
在\(K^n\)中,向量组\(A=\{\AutoTuple{\vb\alpha}{s}\}\)
可由向量组\(B=\{\AutoTuple{\vb\beta}{t}\}\)线性表出,
当且仅当存在矩阵\(\vb{Q} \in M_{t \times s}(K)\),
使得\begin{equation*}
	(\AutoTuple{\vb\alpha}{s}) = (\AutoTuple{\vb\beta}{t})~\vb{Q}.
\end{equation*}
\begin{proof}
假设向量组\(\{\AutoTuple{\vb\alpha}{s}\}\)由向量组\(\{\AutoTuple{\vb\beta}{t}\}\)线性表出,
即存在\begin{equation*}
	x_{ij} \in K\ (i=1,2,\dotsc,s;j=1,2,\dotsc,t),
\end{equation*}
使得\begin{equation*}
	\left\{ \begin{array}{l}
		\vb\alpha_1
		= x_{11} \vb\beta_1 + x_{12} \vb\beta_2 + \dotsb + x_{1t} \vb\beta_t
		= (\AutoTuple{\vb\beta}{t})
		(x_{11},\dotsc,x_{1t})^T, \\
		\vb\alpha_2
		= x_{21} \vb\beta_1 + x_{22} \vb\beta_2 + \dotsb + x_{2t} \vb\beta_t
		= (\AutoTuple{\vb\beta}{t})
		(x_{21},\dotsc,x_{2t})^T, \\
		\hdotsfor{1} \\
		\vb\alpha_s
		= x_{s1} \vb\beta_1 + x_{s2} \vb\beta_2 + \dotsb + x_{st} \vb\beta_t
		= (\AutoTuple{\vb\beta}{t})
		(x_{s1},\dotsc,x_{st})^T,
	\end{array} \right.
\end{equation*}
那么\begin{align*}
	(\AutoTuple{\vb\alpha}{s})
	&= (
		(\AutoTuple{\vb\beta}{t})
		(x_{11},\dotsc,x_{1t})^T,
		\dotsc,
		(\AutoTuple{\vb\beta}{t})
		(x_{s1},\dotsc,x_{st})^T
	) \\
	&= (\AutoTuple{\vb\beta}{t})
	((x_{11},\dotsc,x_{1t})^T,\dotsc,(x_{s1},\dotsc,x_{st})^T),
\end{align*}
这里\(\vb{Q} = ((x_{11},\dotsc,x_{1t})^T,\dotsc,(x_{s1},\dotsc,x_{st})^T)\).
\end{proof}
\end{proposition}
\begin{remark}
%@see: 《Linear Algebra Done Right (Fourth Edition)》(Sheldon Axler) P76 3.51
\cref{theorem:向量空间.线性表出2的等价条件} 说明:
矩阵\(\vb{A}\)与\(\vb{B}\)的乘积\(\vb{A} \vb{B}\)的列向量组可以由\(\vb{A}\)的列向量组线性表出.
同理可证:
\(\vb{A} \vb{B}\)的行向量组可以由\(\vb{B}\)的行向量组线性表出.
\end{remark}

\begin{proposition}\label{theorem:向量空间.线性表出2的自反性}
非空向量组\(A\)总可由它本身线性表出,
即\begin{equation*}
	A \subseteq \Span A.
\end{equation*}
\begin{proof}
设\(A=\{\AutoTuple{\vb\alpha}{s}\}\ (s\geq1)\),
显然\begin{equation*}
	\vb\alpha_i=0\vb\alpha_1+\dotsb+0\vb\alpha_{i-1}+1\vb\alpha_i+0\vb\alpha_{i+1}+\dotsb+0\vb\alpha_s,
	\quad i=1,2,\dotsc,s;
\end{equation*}
这就是说\(\vb\alpha_i\ (i=1,2,\dotsc,s)\)可由\(A\)线性表出,即\begin{equation*}
	\vb\alpha_i\in\opair{\AutoTuple{\vb\alpha}{s}},
	\quad i=1,2,\dotsc,s.
\end{equation*}
于是\(A\)可由\(A\)线性表出.
\end{proof}
\end{proposition}

\begin{proposition}\label{theorem:向量空间.线性表出2的必要条件}
设\(A,B\)都是向量组,
则\begin{equation*}
	\text{\(A\)可由\(B\)线性表出}
	\implies
	\Span A \subseteq \Span B.
\end{equation*}
\begin{proof}
假设\(A = \{\AutoTuple{\vb\alpha}{s}\},
B = \{\AutoTuple{\vb\beta}{t}\}\).

任意取定\(\vb\alpha\in\Span A\),
那么一定存在\(\AutoTuple{k}{s}\in K\)满足\begin{equation*}
	\vb\alpha = \sum_{i=1}^s k_i \vb\alpha_i;
\end{equation*}
又设\(l_{ij}\in K\ (i=1,2,\dotsc,s;j=1,2,\dotsc,t)\)满足\begin{equation*}
	\vb\alpha_i = \sum_{j=1}^t l_{ij} \vb\beta_j,
	\quad i=1,2,\dotsc,s;
\end{equation*}
那么\begin{equation*}
	\vb\alpha = \sum_{i=1}^s k_i \left(
		\sum_{j=1}^t l_{ij} \vb\beta_j
	\right)
	= \sum_{j=1}^t \left(
		\sum_{i=1}^s k_i l_{ij}
	\right) \vb\beta_j,
\end{equation*}
这就是说\(\vb\alpha\in\Span B\),
于是\(\Span A\subseteq\Span B\).
\end{proof}
\end{proposition}

\begin{proposition}\label{theorem:向量空间.线性表出2的充分条件}
如果\(\Span A\subseteq\Span B\),
那么向量组\(A\)可由\(B\)线性表出.
\begin{proof}
因为\(\Span A\subseteq\Span B\),\(A\subseteq\Span A\),
所以\(A\subseteq\Span B\).
\end{proof}
\end{proposition}

于是,依据\cref{theorem:向量空间.线性表出2的必要条件,%
theorem:向量空间.线性表出2的充分条件},
我们可以说:\begin{equation}
	\text{向量组\(A\)可由\(B\)线性表出}
	\iff
	\Span A\subseteq\Span B.
\end{equation}
从而“线性表出”和集合的“包含”关系一样,具有自反性和传递性:
\begin{enumerate}
	\item \(\Span A\subseteq\Span A\).

	这由\cref{theorem:向量空间.线性表出2的自反性} 立即可得.

	\item \(\Span A\subseteq\Span B\land\Span B\subseteq\Span C\implies\Span A\subseteq\Span C\).

	即便不利用集合的包含关系,我们也可以证明线性表出的传递性.
	具体来说,
	假设向量组\(A=\Set{\AutoTuple{\vb\alpha}{s}}\)可以由向量组\(B=\Set{\AutoTuple{\vb\beta}{r}}\)线性表出,
	且\(B\)可以由向量组\(C=\Set{\AutoTuple{\vb\gamma}{m}}\)线性表出.
	在向量组\(A\)中任取一个向量\(\vb\alpha_i\),则\begin{equation*}
		\vb\alpha_i = \sum_{j=1}^r k_j \vb\beta_j.
	\end{equation*}
	又由于\(\vb\beta_j\)可以由向量组\(C\)线性表出,因此\begin{equation*}
		\vb\beta_j = \sum_{t=1}^m l_{jt} \vb\gamma_t,
		\quad j=1,\dotsc,r.
	\end{equation*}
	从而\begin{equation*}
		\vb\alpha_i = \sum_{j=1}^r k_j \vb\beta_j
		= \sum_{j=1}^r k_j \left(
			\sum_{t=1}^m l_{jt} \vb\gamma_t
		\right)
		= \sum_{t=1}^m \left(
			\sum_{j=1}^r k_j l_{jt}
		\right) \vb\gamma_t.
	\end{equation*}
	于是\(\vb\alpha_i\)可以由\(C\)线性表出,
	从而\(A\)可以由\(C\)线性表出.
\end{enumerate}

\begin{definition}\label{definition:向量空间.向量组等价的定义}
%@see: 《线性代数》(张慎语、周厚隆) P72 定义7
%@see: 《高等代数(第三版 上册)》(丘维声) P72 定义2
如果向量组\(A\)与\(B\)可以相互线性表出,
即\begin{equation*}
	\Span A\subseteq\Span B
	\land
	\Span B\subseteq\Span A,
\end{equation*}
或\begin{equation*}
	\Span A = \Span B,
\end{equation*}
则称“\(A\)与\(B\) \DefineConcept{等价}”,
记作\(A \cong B\).
\end{definition}
\begin{proposition}\label{theorem:向量空间.向量组等价的等价条件}
在\(K^n\)中,向量组\(A=\{\AutoTuple{\vb\alpha}{s}\}\)
与向量组\(B=\{\AutoTuple{\vb\beta}{t}\}\)等价,
当且仅当存在
矩阵\(\vb{P} \in M_{s \times t}(K)\)、
矩阵\(\vb{Q} \in M_{t \times s}(K)\),
使得\begin{equation*}
	(\AutoTuple{\vb\beta}{t}) = (\AutoTuple{\vb\alpha}{s})~\vb{P},
	\quad\text{且}\quad
	(\AutoTuple{\vb\alpha}{s}) = (\AutoTuple{\vb\beta}{t})~\vb{Q}.
\end{equation*}
\begin{proof}
由\cref{theorem:向量空间.线性表出2的等价条件} 易得.
\end{proof}
%@credit: {5f4d2f8a-fc8b-4798-85d6-98670f6761e7} 说
% 不能把上述条件浓缩为“存在一个行满秩矩阵或列满秩矩阵\(\vb{Q}\)使得\((\AutoTuple{\vb\alpha}{s}) = (\AutoTuple{\vb\beta}{t})~\vb{Q}\)”.
% 特别是考虑到向量组\(A\)和向量组\(B\)可能是线性相关的,那样就完全不需要一个满秩过渡矩阵.
\end{proposition}

\begin{property}\label{theorem:向量空间.向量组的等价的性质}
对于任意向量组\(A,B,C\subseteq K^n\)来说,
\begin{itemize}
	\item “向量组的等价”具有自反性,即\(A \cong A\).
	\item “向量组的等价”具有对称性,即\(A \cong B \implies B \cong A\).
	\item “向量组的等价”具有传递性,即\(A \cong B \land B \cong C \implies A \cong C\).
\end{itemize}
\end{property}
“向量组的等价”是向量组之间的一种等价关系.

另外,应该注意到,即便有\begin{equation*}
	A \subseteq \Span A = \Span B \supseteq B,
\end{equation*}成立,不见得就有\(A=B\)一定成立.

\begin{theorem}\label{theorem:线性方程组.部分组可由全组线性表出}
%@see: 《线性代数》(张慎语、周厚隆) P72
部分组可由全组线性表出.
\begin{proof}
设数域\(K\)上一个向量组\(A=\{\AutoTuple{\vb\alpha}{s}\}\),
从中任取\(t\ (t \leq s)\)个向量组成向量组\begin{equation*}
	B=\{\AutoTuple{\vb\alpha}{t}\}.
\end{equation*}
欲证部分组可由全组线性表出,
即证\(\forall \vb\alpha_j \in B\),
\(\exists \AutoTuple{k}{j},\dotsc,k_s \in K\),
使得\begin{equation*}
	\vb\alpha_j = k_1 \vb\alpha_1 + k_2 \vb\alpha_2 + \dotsb + k_j \vb\alpha_j + \dotsb + k_s \vb\alpha_s.
\end{equation*}
显然只要取\begin{equation*}
	k_i = \left\{ \begin{array}{cl}
		1, & i=j, \\
		0, & i \neq j,
	\end{array} \right.
\end{equation*}
便可令上式成立.
\end{proof}
\end{theorem}
对于\cref{theorem:线性方程组.部分组可由全组线性表出},
从\(A \subseteq B\)出发,
我们还可以利用\cref{theorem:向量空间.线性表出2的自反性},
结合\(B \subseteq \Span B\),
根据集合包含关系的传递性,
就可以得到\(A \subseteq \Span B\).
因此\begin{equation*}
	A \subseteq B \implies A \subseteq \Span B.
\end{equation*}

\begin{theorem}
%@see: 《线性代数》(张慎语、周厚隆) P72
设\(A\)是向量组,且\(\card A > 1\).
\(A\)线性相关的充分必要条件是:
\(A\)可由某个部分组线性表出.
\begin{proof}
必要性.
因为\(\card A > 1\),
所以由\cref{theorem:线性方程组.向量组线性相关的充分必要条件1} 可知
\begin{equation*}
	\text{\(A\)线性相关}
	\iff
	(\exists \vb\alpha \in A)[\vb\alpha \in \Span(A-\{\vb\alpha\})].
\end{equation*}
又因为\(\card A > 1\),
所以\((\forall\vb\beta\in A)[\card(A-\{\vb\beta\})>0]\),
那么只要令\(B=A-\{\vb\alpha\}\),
就必然有\begin{equation*}
	\emptyset \neq B \subseteq A = \{\vb\alpha\} \cup B, \qquad
	\{\vb\alpha\}\subseteq\Span B, \qquad
	B \subseteq \Span B
\end{equation*}同时成立.
因此\(A\subseteq\Span B\),
这就是说\(A\)可由\(B\)线性表出.

充分性.
因为\(\card A > 1\),
所以\((\exists B)[\emptyset \neq B \subset A]\).
假设\(B\)是\(A\)的一个非空真子集,且\(A\)可由它线性表出,
即\(\emptyset \neq B \subset A\)且\(\Span A \subseteq \Span B\).
显然\(A-B\neq\emptyset\).
由于\hyperref[theorem:线性方程组.部分组可由全组线性表出]{部分组可由全组线性表出},
所以\begin{equation*}
	A-B \subseteq A
	\implies
	A-B \subseteq \Span A
	\implies
	A-B \subseteq \Span B.
\end{equation*}
我们可以笃定:存在向量\(\vb\gamma \in A-B\),使得\(\vb\gamma\)可由向量组\(B\)线性表出,
即\(\vb\gamma \in \Span B\).
因为\(\vb\gamma \in A-B \implies A-\{\vb\gamma\} \supseteq B\),
所以\(\Span B \subseteq \Span(A-\{\vb\gamma\})\),
那么\(\vb\gamma\)也可由向量组\((A-\{g\})\)线性表出,
即\(\vb\gamma \in \Span(A-\{g\})\).
于是由\cref{theorem:线性方程组.向量组线性相关的充分必要条件1} 可知
向量组\(A\)线性相关.
\end{proof}
\end{theorem}

\begin{theorem}\label{theorem:向量空间.可由比自己基数小的向量组线性表出的向量组线性相关}
%@see: 《线性代数》(张慎语、周厚隆) P72 定理2
%@see: 《高等代数(第三版 上册)》(丘维声) P74 引理1
设向量组\(A=\{\AutoTuple{\vb\alpha}{s}\}\)可由\(B=\{\AutoTuple{\vb\beta}{t}\}\)线性表出.
如果\(s>t\),则\(A\)线性相关.
\begin{proof}
欲证\(A\)线性相关,须找到不全为零的\(s\)个数\(\AutoTuple{k}{s}\)使得\begin{equation*}
	k_1 \vb\alpha_1 + k_2 \vb\alpha_2 + \dotsb + k_s \vb\alpha_s = \vb0.
\end{equation*}
因为向量组\(A\)可由\(B\)线性表出,即有\begin{equation*}
	\left\{ \begin{array}{l}
		\vb\alpha_1 = c_{11} \vb\beta_1 + c_{21} \vb\beta_2 + \dotsb + c_{t1} \vb\beta_t, \\
		\vb\alpha_2 = c_{12} \vb\beta_1 + c_{22} \vb\beta_2 + \dotsb + c_{t2} \vb\beta_t, \\
		\hdotsfor{1} \\
		\vb\alpha_s = c_{1s} \vb\beta_1 + c_{2s} \vb\beta_2 + \dotsb + c_{ts} \vb\beta_t.
	\end{array} \right.
\end{equation*}代入可得\begin{equation*}
	\sum_{j=1}^s k_j \vb\alpha_j
	=\sum_{j=1}^s k_j \sum_{i=1}^t c_{ij} \vb\beta_i
	=\sum_{j=1}^s \sum_{i=1}^t k_j c_{ij} \vb\beta_i
	=\sum_{i=1}^t \vb\beta_i \sum_{j=1}^s k_j c_{ij}
	=\vb0.
\end{equation*}
如此只需证存在不全为零的\(s\)个数\(\AutoTuple{k}{s}\)
使得对于任意\(i=1,2,\dotsc,t\)都有\begin{equation*}
	\sum_{j=1}^s k_j c_{ij} = 0.
\end{equation*}
而关于\(k_i\ (i=1,2,\dotsc,s)\)的齐次线性方程组
\begin{equation*}
	\left\{ \begin{array}{l}
		c_{11} k_1 + c_{12} k_2 + \dotsb + c_{1s} k_s = 0, \\
		c_{21} k_1 + c_{22} k_2 + \dotsb + c_{2s} k_s = 0, \\
		\hdotsfor{1} \\
		c_{t1} k_1 + c_{t2} k_2 + \dotsb + c_{ts} k_s = 0.
	\end{array} \right.
\end{equation*}中方程数\(t\)小于未知量个数\(s\),必有非零解.
\end{proof}
\end{theorem}

\begin{corollary}
%@see: 《线性代数》(张慎语、周厚隆) P73 推论1
任意\(n+1\)个\(n\)维向量线性相关.
\begin{proof}
由\cref{theorem:向量空间.任一向量可由基本向量组唯一线性表出},
\(K^n\)中任意\(n+1\)个\(n\)维向量\(A=\{\AutoTuple{\vb\alpha}{n+1}\}\)
可由基本向量组线性表出;
这两个向量组中的向量个数满足\(n+1>n\),
由\cref{theorem:向量空间.可由比自己基数小的向量组线性表出的向量组线性相关},
向量组\(A\)线性相关.
\end{proof}
\end{corollary}

\begin{corollary}\label{theorem:向量空间.线性无关向量组的基数不大于可以线性表出它的任意向量组的基数}
%@see: 《线性代数》(张慎语、周厚隆) P73 推论2
%@see: 《高等代数(第三版 上册)》(丘维声) P74 推论3
若线性无关向量组\(A=\{\AutoTuple{\vb\alpha}{s}\}\)可由\(B=\{\AutoTuple{\vb\beta}{t}\}\)线性表出,
则\(s \leq t\).
\begin{proof}
用反证法.
假设\(s > t\),
由\cref{theorem:向量空间.可由比自己基数小的向量组线性表出的向量组线性相关},
因为向量组\(A\)可由\(B\)线性表出,
所以向量组\(A\)线性相关,矛盾!
故\(s \leq t\).
\end{proof}
\end{corollary}

\begin{corollary}\label{theorem:向量空间.两个等价的线性无关向量组含有相同的向量个数}
%@see: 《线性代数》(张慎语、周厚隆) P73 推论3
%@see: 《高等代数(第三版 上册)》(丘维声) P74 推论4
两个等价的线性无关向量组含有相同的向量个数,即\begin{equation*}
	A \cong B \implies \card A = \card B.
\end{equation*}
\begin{proof}
设\(A=\{\AutoTuple{\vb\alpha}{s}\}\)%
与\(B=\{\AutoTuple{\vb\beta}{t}\}\)%
都线性无关,且\(A \cong B\).
因为\(A\)可由\(B\)线性表出,
由\cref{theorem:向量空间.线性无关向量组的基数不大于可以线性表出它的任意向量组的基数},
\(s \leq t\);
同理可得\(t \leq s\);
于是\(s = t\).
\end{proof}
\end{corollary}
\begin{remark}
含有相同个数向量的两个向量组不一定等价.
例如,取向量组\(A=\{(0,1)\},
B=\{(1,0)\}\).
易知向量组\(B\)不可以线性表出向量组\(A\),
向量组\(A\)也不可以线性表出向量组\(B\).
\end{remark}

%\begin{example}
%在数域\(K\)上,满足\begin{equation*}
%\abs{a_{ii}} > \sum_{\substack{1 \leq j \leq n \\ j \neq i}} \abs{a_{ij}}
%\quad (i=1,2,\dotsc,n)
%\end{equation*}的\(n\)阶矩阵\(\vb{A} = (a_{ij})_n\)称为\DefineConcept{主对角占优矩阵}.
%证明:\(\vb{A}\)的列向量组\(\AutoTuple{\vb\alpha}{n}\)的秩等于\(n\).
%\begin{proof}
%假设\(\AutoTuple{\vb\alpha}{n}\)线性相关,
%则在\(K\)中有一组不全为0的数\(\AutoTuple{k}{n}\),
%使得\begin{equation*}
%	k_1 \vb\alpha_1 + k_2 \vb\alpha_2 + \dotsb + k_n \vb\alpha_n = \vb0.
%\end{equation*}
%不妨设\(\abs{k_l} = \max\{\abs{k_1},\abs{k_2},\dotsc,\abs{k_n}\}\neq0\).
%由\begin{equation*}
%	k_1 a_{l1} + k_2 a_{l2} + \dotsb + k_l a_{ll} + \dotsb + k_n a_{ln} = 0,
%\end{equation*}
%可得\begin{equation*}
%	a_{ll} = -\frac{1}{k_l} (k_1 a_{l1} + \dotsb + k_{l-1} a_{l,l-1} + k_{l+1} a_{l,l+1} + \dotsb + k_n a_{ln})
%	= - \sum_{\substack{1 \leq j \leq n \\ j \neq l}} \frac{k_j}{k_l} a_{lj},
%\end{equation*}\begin{equation*}
%	\abs{a_{ll}} \leq \sum_{\substack{1 \leq j \leq n \\ j \neq l}} \frac{\abs{k_j}}{\abs{k_l}} \abs{a_{lj}}
%	\leq \sum_{\substack{1 \leq j \leq n \\ j \neq l}} \abs{a_{lj}}.
%\end{equation*}
%这与已知条件矛盾!
%因此\(\AutoTuple{\vb\alpha}{n}\)线性无关,
%\(\rank\{\AutoTuple{\vb\alpha}{n}\} = n\).
%\end{proof}
%\end{example}
%TODO 后移到【向量的秩】

\subsection{极大线性无关组的概念}
\begin{definition}\label{definition:线性方程组.极大线性无关组的定义}
%@see: 《线性代数》(张慎语、周厚隆) P73 定义8
%@see: 《高等代数(第三版 上册)》(丘维声) P72 定义1
在\(K^n\)中,设\(B\)是\(A\)的一个部分组.
如果\begin{itemize}
	\item \(B\)线性无关,
	\item \(A\)可由\(B\)线性表出,
\end{itemize}
则称“\(B\)是\(A\)的一个\DefineConcept{极大线性无关组}(maximally linearly independent subset)”.
%@see: https://mathworld.wolfram.com/MaximallyLinearlyIndependent.html
\end{definition}

\begin{example}\label{example:向量空间.单向量组的极大线性无关组}
求向量组\(\{\vb\alpha\}\)的极大线性无关组.
\begin{solution}
显然有\(\Powerset\{\vb\alpha\} = \{ \emptyset, \{\vb\alpha\} \}\),
也就是说\(\{\vb\alpha\}\)的部分组只有\(\emptyset\)和\(\{\vb\alpha\}\),
于是它的极大线性无关组也只能是这两者中的一个.
因为\(\card\{\vb\alpha\} = 1 > 0 = \card\emptyset\),
所以只需要讨论\(\{\vb\alpha\}\)是不是极大线性无关组.

当\(\vb\alpha=\vb0\)时,\(\{\vb\alpha\}\)线性相关,
不能满足\hyperref[definition:线性方程组.极大线性无关组的定义]{极大线性无关组的定义},
故\(\{\vb\alpha\}\)的极大线性无关组是\(\emptyset\).

当\(\vb\alpha\neq\vb0\)时,\(\{\vb\alpha\}\)线性无关,
所以\(\{\vb\alpha\}\)的极大线性无关组是它本身.
\end{solution}
\end{example}

\begin{theorem}\label{theorem:线性方程组.向量组与其极大线性无关组等价}
%@see: 《高等代数(第三版 上册)》(丘维声) P73 命题1
向量组与其极大线性无关组等价.
\begin{proof}
由\cref{theorem:线性方程组.部分组可由全组线性表出} 可知,
作为部分组,极大线性无关组可由全组线性表出;
再根据\hyperref[definition:线性方程组.极大线性无关组的定义]{极大线性无关组的定义},
全组可由极大线性无关组线性表出;
因此,根据\hyperref[definition:向量空间.向量组等价的定义]{向量组等价的定义},
全组与极大线性无关组等价.
\end{proof}
\end{theorem}
\begin{remark}
\cref{theorem:线性方程组.向量组与其极大线性无关组等价} 说明,
向量组\(\AutoTuple{\vb\alpha}{s}\)可以由它的一个极大线性无关组线性表出.
再根据线性表出的传递性得,
\(W=\opair{\AutoTuple{\vb\alpha}{s}}\)中的每个向量%
可以由\(\AutoTuple{\vb\alpha}{s}\)的一个极大线性无关组线性表出,
此时表出方式就唯一了.
\end{remark}

\begin{corollary}\label{theorem:线性空间.向量组的任意两个极大线性无关组等价且等势}
%@see: 《线性代数》(张慎语、周厚隆) P73 定理3
%@see: 《高等代数(第三版 上册)》(丘维声) P73 推论2
%@see: 《高等代数(第三版 上册)》(丘维声) P75 推论5
向量组的任何两个极大线性无关组等价,
且包含相同个数的向量.
\begin{proof}
设\(A\)与\(B\)是向量组\(C\)的两个极大线性无关组.
根据\cref{theorem:线性方程组.向量组与其极大线性无关组等价},
\(A \cong C\),\(B \cong C\).
再由\cref{theorem:向量空间.向量组的等价的性质},向量组等价具有对称性和传递性,
于是\(A \cong B\).
\end{proof}
\end{corollary}

\begin{theorem}
在\(K^n\)中,任意向量组的极大线性无关组的向量个数不大于\(n\)个.
\begin{proof}
根据定义,任意向量组的极大线性无关组是线性无关的,
而向量个数大于维数的向量组总是线性相关,
故任意向量组的极大线性无关组的向量个数总是不大于其维数\(n\)的.
\end{proof}
\end{theorem}

\subsection{向量组的秩}
\begin{definition}
向量组\(A = \{\AutoTuple{\vb\alpha}{s}\}\)的极大线性无关组所含向量的个数,
称为向量组的\DefineConcept{秩}(rank),
记为\(\rank A\)或\(\rank\{\AutoTuple{\vb\alpha}{s}\}\),
即\begin{equation*}
	\text{\(A'\)是\(A\)的极大线性无关组}
	\implies
	[\rank A = \card A'].
\end{equation*}
\end{definition}

\begin{property}
空集\(\emptyset\)的秩为零,即\(\rank\emptyset = 0\).
\begin{proof}
由于\(A\subseteq\emptyset\iff A=\emptyset\),
所以\(\emptyset\)的极大线性无关组就是它本身,
\(\rank\emptyset=\card\emptyset=0\).
\end{proof}
\end{property}

\begin{property}
零向量组的秩为零,即\(\rank\{\vb0\}=0\).
\begin{proof}
由\cref{example:向量空间.单向量组的极大线性无关组},
\(\{\vb0\}\)的极大线性无关组是\(\emptyset\),
故\(\rank\{\vb0\} = \card\emptyset = 0\).
\end{proof}
\end{property}

\begin{proposition}\label{theorem:向量组的秩.并集的秩}
设\(A\)是\(K^n\)中的向量组.
\begin{itemize}
	\item 如果向量\(\vb\alpha\)可由\(A\)线性表出,
	则\begin{equation*}
		\rank(A \cup \{\vb\alpha\}) = \rank A.
	\end{equation*}

	\item 如果向量\(\vb\alpha\)不可由\(A\)线性表出,
	则\begin{equation*}
		\rank(A \cup \{\vb\alpha\}) = \rank A + 1.
	\end{equation*}
\end{itemize}
\begin{proof}
假设\(A'\)是\(A\)的一个极大线性无关组.

如果向量\(\vb\alpha\)可由\(A\)线性表出,
那么\(\vb\alpha\)可由\(A'\)线性表出,
所以\(A \cup \{\vb\alpha\}\)中的每一个向量都可由\(A'\)线性表出,
于是\begin{equation*}
	\rank(A \cup \{\vb\alpha\})
	= \card A'
	= \rank A.
\end{equation*}

如果向量\(\vb\alpha\)不可由\(A\)线性表出,
那么\(A' \cup \{\vb\alpha\}\)线性无关,
且\(A \cup \{\vb\alpha\}\)可由\(A' \cup \{\vb\alpha\}\)线性表出,
这就说明\(A' \cup \{\vb\alpha\}\)一定是\(A \cup \{\vb\alpha\}\)的一个极大线性无关组.
于是\begin{equation*}
	\rank(A \cup \{\vb\alpha\})
	= \card(A' \cup \{\vb\alpha\})
	= \card A' + 1
	= \rank A + 1.
	\qedhere
\end{equation*}
\end{proof}
\end{proposition}

\begin{corollary}\label{theorem:向量空间.秩与线性相关性的关系}
%@see: 《线性代数》(张慎语、周厚隆) P73 推论4
%@see: 《高等代数(第三版 上册)》(丘维声) P75 命题6
设向量组\(A\).
\begin{itemize}
	\item 如果\(\rank A=\card A\),则向量组\(A\)线性无关.
	\item 如果\(\rank A<\card A\),则向量组\(A\)线性相关.
\end{itemize}
\begin{proof}
\(\text{\(A\)线性无关}
	\iff \text{\(A\)的极大线性无关组是它本身}
	\iff \rank A = \card A\).
\end{proof}
\end{corollary}

\begin{theorem}\label{theorem:向量空间.向量组的秩的比较1}
%@see: 《线性代数》(张慎语、周厚隆) P73 推论5
%@see: 《高等代数(第三版 上册)》(丘维声) P75 命题7
设向量组\(A\)可由\(B\)线性表出,
则\(\rank A \leq \rank B\).
\begin{proof}
设\(A=\{\AutoTuple{\vb\alpha}{s}\}\),
\(B=\{\AutoTuple{\vb\beta}{t}\}\),
\(\rank A = r\),\(\rank B = u\).
因为\(A\)可由\(B\)线性表出,即\begin{equation*}
	\vb\alpha_k = \sum_{i=1}^t l_{ki} \vb\beta_i,
	\quad k=1,2,\dotsc,s.
\end{equation*}
设\(A'=\{\AutoTuple{\vb\alpha}{r}\}\)%
和\(B'=\{\AutoTuple{\vb\beta}{u}\}\)%
分别是\(A\)和\(B\)的极大线性无关组,
则\(B\)可由\(B'\)线性表出,即\begin{equation*}
	\vb\beta_i = \sum_{j=1}^u b_{ij} \vb\beta_j,
	\quad i=1,2,\dotsc,t;
\end{equation*}
所以有\begin{equation*}
	\vb\alpha_k = \sum_{i=1}^t l_{ki} \sum_{j=1}^u b_{ij} \vb\beta_j
	= \sum_{i=1}^t \sum_{j=1}^u l_{ki} b_{ij} \vb\beta_j
	= \sum_{j=1}^u \vb\beta_j \sum_{i=1}^t l_{ki} b_{ij},
	\quad k=1,2,\dotsc,s.
\end{equation*}

特别地,\(A'\)可由\(B'\)线性表出,
由\cref{theorem:向量空间.线性无关向量组的基数不大于可以线性表出它的任意向量组的基数},
则有\(r \leq u\),即\(\rank A \leq \rank B\).
\end{proof}
\end{theorem}
\begin{remark}
我们可以把\cref{theorem:向量空间.向量组的秩的比较1} 的证明思路绘制如下:\begin{equation*}
	\color{gray}
	\left. \begin{array}{r}
		\rank A = \rank A' = \card A' \\
		\left. \begin{array}{r}
			A' \subseteq A \\
			{\color{black} A \subseteq \Span B} \\
			\Span B \subseteq \Span B'
		\end{array} \right\}
		\implies
		A' \subseteq \Span B'
		\implies
		\card A' \leq \card B' \\
		\rank B = \rank B' = \card B'
	\end{array} \right\}
	\implies
	{\color{black} \rank A \leq \rank B}.
\end{equation*}
\end{remark}
\begin{example}
举例说明:纵然向量组\(A,B\)满足\(\rank A \leq \rank B\),也不能断定\(A\)可以由\(\vb{B}\)线性表出.
\begin{solution}
取\(A = \{(1,0,0),(0,1,0)\},
B = \{(0,0,1)\}\),
可见\(\rank A = 2,
\rank B = 1\),
但是\(B\)不能由\(A\)线性表出.
\end{solution}
\end{example}

\begin{corollary}\label{theorem:向量空间.向量组的秩的比较2}
部分组的秩总是小于或等于全组的秩.
\begin{proof}
因为\hyperref[theorem:线性方程组.部分组可由全组线性表出]{部分组总可由全组线性表出},
所以\cref{theorem:向量空间.向量组的秩的比较1} 可知,
部分组的秩总是小于或等于全组的秩.
\end{proof}
\end{corollary}

\begin{theorem}\label{theorem:向量组的秩.等价向量组的秩相等}
%@see: 《线性代数》(张慎语、周厚隆) P74 推论6
%@see: 《高等代数(第三版 上册)》(丘维声) P76 推论8
等价向量组的秩相等.
秩相等的向量组却不一定等价.
\begin{proof}
先证“等价向量组的秩相等”.
设向量组\(A\)与\(B\)等价,
则\(A\)可由\(B\)线性表出,
那么由\cref{theorem:向量空间.向量组的秩的比较1} 可得%
\(\rank A \leq \rank B\);
同理可得\(\rank A \geq \rank B\);
因此,\(\rank A = \rank B\).

再证“秩相等的向量组却不一定等价”.
设\(A=\{(0,1)\},
B=\{(1,0)\}\).
虽然\begin{equation*}
	\rank A = \rank B = 1,
\end{equation*}
但\(A\)与\(B\)显然不等价.
\end{proof}
\end{theorem}

\begin{proposition}\label{theorem:向量组的秩.向量组等价的充分必要条件}
设\(A,B\)都是\(K^n\)中的向量组,
则\(A\)与\(B\)等价的充分必要条件是\begin{equation*}
	\rank A = \rank B = \rank(A \cup B).
\end{equation*}
\begin{proof}
必要性.
假设\(A \cong B\)成立.
因为\hyperref[theorem:向量组的秩.等价向量组的秩相等]{等价向量组的秩相等},
所以\(\rank A = \rank B\).
因为\(A\)和\(B\)均可由\(B\)线性表出,
从而\(A \cup B\)也可由\(B\)线性表出;
同时\(B\)作为\(A \cup B\)的部分组自然可由\(A \cup B\)线性表出,
所以\(A \cup B\)与\(B\)等价,
于是\(\rank(A \cup B) = \rank B\).

充分性.
用反证法.
假设\(\rank A = \rank(A \cup B)\)成立,
但是\(B\)不可由\(A\)线性表出.
因为\(B\)不可由\(A\)线性表出,
所以\begin{equation*}
	(\exists\vb\beta \in B)  		% 向量组 B 中存在向量 \vb\beta
	[\vb\beta \notin \Span A],	% 向量 \vb\beta 不在向量组 A 的线性生成空间中
\end{equation*}
于是根据\cref{theorem:向量组的秩.并集的秩}
可知\(\rank(A \cup B) > \rank A\),
与假设矛盾,
因此\begin{equation*}
	\rank A = \rank(A \cup B)
	\implies
	\text{\(B\)可由\(A\)线性表出}.
\end{equation*}
同理\begin{equation*}
	\rank B = \rank(A \cup B)
	\implies
	\text{\(A\)可由\(B\)线性表出}.
\end{equation*}
综上所述\begin{equation*}
	\rank A = \rank(A \cup B) = \rank B
	\implies
	A \cong B.
	\qedhere
\end{equation*}
\end{proof}
\end{proposition}

\begin{example}\label{example:向量空间.若部分组向量个数多于全组的秩则部分组必线性相关}
证明:在秩为\(r\)的向量组中,任意\(r+1\)个向量必线性相关.
\begin{proof}
设向量组\(\AutoTuple{\vb\alpha}{s}\)的秩为\(r\).
假设部分组\(\AutoTuple{\vb\alpha}{r+1}\)线性无关,
那么由\cref{theorem:向量空间.秩与线性相关性的关系} 得\begin{equation*}
	\rank\{\AutoTuple{\vb\alpha}{r+1}\} = r+1.
\end{equation*}
因为\hyperref[theorem:向量空间.向量组的秩的比较2]{部分组的秩总是小于或等于全组的秩},
而这与\begin{equation*}
	r+1 = \rank\{\AutoTuple{\vb\alpha}{r+1}\} \leq \rank\{\AutoTuple{\vb\alpha}{s}\} = r,
\end{equation*}矛盾,
所以部分组\(\AutoTuple{\vb\alpha}{r+1}\)一定线性相关.
\end{proof}
\end{example}

\begin{example}
设向量组\(\AutoTuple{\vb\alpha}{s}\)的秩为\(r\).
如果\(\AutoTuple{\vb\alpha}{r}\)线性无关,证明:
\(\AutoTuple{\vb\alpha}{r}\)
是\(\AutoTuple{\vb\alpha}{s}\)的一个极大线性无关组.
\begin{proof}
设\begin{equation*}
	A=\{\AutoTuple{\vb\alpha}{s}\},
	\qquad
	B=\{\AutoTuple{\vb\alpha}{r}\}.
\end{equation*}
要证\(B\)是\(A\)的一个极大线性无关组,
须证\(A\)的任意向量可由\(B\)线性表出.

\begin{enumerate}
	\item 显然地,\(\vb\alpha_i\ (i=1,2,\dotsc,r)\)可由\(B\)线性表出.

	\item 根据上例,在秩为\(r\)的向量组中,
	任意\(r+1\)个向量必线性相关,
	那么向量组\begin{equation*}
		A_i = \{\AutoTuple{\vb\alpha}{r},\vb\alpha_i\}\quad(i=r+1,\dotsc,s)
	\end{equation*}必线性相关.

	又因为\(B\)线性无关,
	所以\(\vb\alpha_i\ (i=r+1,\dotsc,s)\)可由\(B\)线性表出.
\end{enumerate}

综上所述,\(A\)的任意向量可由\(B\)线性表出,且\(B\)线性无关,
根据极大线性无关组的定义,\(B\)是\(A\)的一个极大线性无关组.
\end{proof}
\end{example}

\begin{example}
向量组\(\AutoTuple{\vb\alpha}{r+1}\)与部分组\(\AutoTuple{\vb\alpha}{r}\)的秩相等.
证明:\(\vb\alpha_{r+1}\)可由\(\AutoTuple{\vb\alpha}{r}\)线性表出.
\begin{proof}
记\(A=\{\AutoTuple{\vb\alpha}{r+1}\}\),
\(B=\{\AutoTuple{\vb\alpha}{r}\}\).
设\(B\)的极大线性无关组为\begin{equation*}
	B'=\{\AutoTuple{\vb\alpha}{t}\},
	\quad 0 \leq t \leq r.
\end{equation*}
由题意有\(\rank A = \rank B = \rank B' = \card B' = t\).

由上例可知,因为\(\rank A = t\),\(B'\)线性无关,
所以\(B'\)是\(A\)的一个极大线性无关组.
那么向量组\(A\)中的向量\(\vb\alpha_{r+1}\)可以由极大线性无关组\(B'\)线性表出.
又由于\(B'\)是\(B\)的部分组,故\(B'\)可由\(B\)线性表出.
总而言之,\(A\)可由\(B\)线性表出.
\end{proof}
\end{example}

\begin{example}
%@see: https://www.bilibili.com/video/BV19N4y1b7aS/
设向量组\(A = \{\AutoTuple{\vb\alpha}{r}\}\)线性无关,
且可以由向量组\(B = \{\AutoTuple{\vb\beta}{s}\}\ (s>r)\)线性表出.
证明:存在\(\vb\beta \in B\),
使得\begin{equation*}
	\rank(\vb\beta,\AutoTuple{\vb\alpha}[2]{r}) = r.
\end{equation*}
%TODO proof
\end{example}

\subsection{极大线性无关组的求解}
\begin{theorem}\label{theorem:向量空间.利用初等行变换求取列极大线性无关组的依据}
%@see: 《线性代数》(张慎语、周厚隆) P75 定理4
设矩阵\begin{equation*}
	\vb{A}=(\AutoTuple{\vb\alpha}{m})
\end{equation*}
经一系列初等行变换化为矩阵\begin{equation*}
	\vb{B}=(\AutoTuple{\vb\beta}{m}),
\end{equation*}
则\(\vb\alpha_{j_1},\vb\alpha_{j_2},\dotsc,\vb\alpha_{j_r}\)为\(\vb{A}\)的列极大线性无关组的充分必要条件是:
\(\vb\beta_{j_1},\vb\beta_{j_2},\dotsc,\vb\beta_{j_r}\)为\(\vb{B}\)的列极大线性无关组.
\begin{proof}
假设矩阵\(\widetilde{\vb{A}}=(\vb\alpha_{j_1},\vb\alpha_{j_2},\dotsc,\vb\alpha_{j_r},\vb\alpha_l)\)经相同的初等行变换化为\begin{equation*}
	\widetilde{\vb{B}}=(\vb\beta_{j_1},\vb\beta_{j_2},\dotsc,\vb\beta_{j_r},\vb\beta_l) \quad(l=1,2,\dotsc,m).
\end{equation*}
考虑以下四个向量形式的线性方程组
\begin{gather}
	x_1 \vb\alpha_{j_1} + x_2 \vb\alpha_{j_2} + \dotsb + x_r \vb\alpha_{j_r} = \vb0, \tag1 \\
	x_1 \vb\beta_{j_1} + x_2 \vb\beta_{j_2} + \dotsb + x_r \vb\beta_{j_r} = \vb0, \tag2 \\
	y_1 \vb\alpha_{j_1} + y_2 \vb\alpha_{j_2} + \dotsb + y_r \vb\alpha_{j_r} = \vb\alpha_l, \tag3 \\
	y_1 \vb\beta_{j_1} + y_2 \vb\beta_{j_2} + \dotsb + y_r \vb\beta_{j_r} = \vb\beta_l, \tag4
\end{gather}
其中(1)与(2)同解,(3)与(4)同解.

必要性.
当\(\vb\alpha_{j_1},\vb\alpha_{j_2},\dotsc,\vb\alpha_{j_r}\)是\(\vb{A}\)的列极大线性无关组时,
(1)仅有零解,(3)有解.于是(2)仅有零解,(4)有解,
从而\(\vb\beta_{j_1},\vb\beta_{j_2},\dotsc,\vb\beta_{j_r}\)线性无关,
\(\vb\beta_l\ (l=1,2,\dotsc,m)\)可由其线性表出;
由极大线性无关组定义,
\(\vb\beta_{j_1},\vb\beta_{j_2},\dotsc,\vb\beta_{j_r}\)是\(\vb{B}\)的列极大线性无关组.

同理可证充分性.
\end{proof}
\end{theorem}

\cref{theorem:向量空间.利用初等行变换求取列极大线性无关组的依据} 告诉我们,
要想求出一组向量\(\AutoTuple{\vb\alpha}{s}\)的极大线性无关组,
可以构造矩阵\(\vb{A}=(\AutoTuple{\vb\alpha}{s})\),
再利用高斯消元法得到阶梯形矩阵\(\vb{B}\),
找出\(\vb{B}\)的非零首元所在的列\(
	\vb\beta_{j_1}, \allowbreak
	\vb\beta_{j_2}, \allowbreak
	\dotsc, \allowbreak
	\vb\beta_{j_r}
\),
回过头找出\(\vb{A}\)中对应的列\(\vb\alpha_{j_1},\vb\alpha_{j_2},\dotsc,\vb\alpha_{j_r}\),
那么\(\vb\alpha_{j_1},\vb\alpha_{j_2},\dotsc,\vb\alpha_{j_r}\)就是\(\AutoTuple{\vb\alpha}{s}\)的极大线性无关组.
%@see: https://math.stackexchange.com/a/164021/591741

\begin{example}
求列向量组\begin{equation*}
	\vb\alpha_1 = \begin{bmatrix} -1 \\ 1 \\ 0 \\ 0 \end{bmatrix},
	\vb\alpha_2 = \begin{bmatrix} -1 \\ 2 \\ -1 \\ 1 \end{bmatrix},
	\vb\alpha_3 = \begin{bmatrix} 0 \\ -1 \\ 1 \\ -1 \end{bmatrix},
	\vb\alpha_4 = \begin{bmatrix} 1 \\ -1 \\ 2 \\ 3 \end{bmatrix},
	\vb\alpha_5 = \begin{bmatrix} 2 \\ -6 \\ 4 \\ 1 \end{bmatrix}
\end{equation*}的秩与一个极大线性无关组.
\begin{solution}
对矩阵\(\vb{A} = (\AutoTuple{\vb\alpha}{5})\)作初等行变换化为阶梯形矩阵:
\begin{align*}
	\vb{A} &= \begin{bmatrix}
		-1 & -1 & 0 & 1 & 2 \\
		1 & 2 & -1 & -3 & -6 \\
		0 & -1 & 1 & 2 & 4 \\
		0 & 1 & -1 & 3 & 1 \\
	\end{bmatrix}
	\xlongrightarrow{\begin{array}{c}
		(2\text{行}) \addeq 1 \times (1\text{行}) \\
		(4\text{行}) \addeq (3\text{行})
	\end{array}}
	\begin{bmatrix}
		-1 & -1 & 0 & 1 & 2 \\
		0 & 1 & -1 & -2 & -4 \\
		0 & -1 & 1 & 2 & 4 \\
		0 & 0 & 0 & 5 & 5 \\
	\end{bmatrix} \\
	&\xlongrightarrow{\begin{array}{c}
		(3\text{行}) \addeq (2\text{行}) \\
		(4\text{行}) \diveq 5
	\end{array}}
	\begin{bmatrix}
		-1 & -1 & 0 & 1 & 2 \\
		0 & 1 & -1 & -2 & -4 \\
		0 & 0 & 0 & 0 & 0 \\
		0 & 0 & 0 & 1 & 1 \\
	\end{bmatrix} \\
	&\xlongrightarrow{\begin{array}{c} \text{交换(3行)与(4行)} \end{array}}
	\begin{bmatrix}
		-1 & -1 & 0 & 1 & 2 \\
		0 & 1 & -1 & -2 & -4 \\
		0 & 0 & 0 & 1 & 1 \\
		0 & 0 & 0 & 0 & 0 \\
	\end{bmatrix}
	= \vb{B}.
\end{align*}
若按列分块有\(\vb{B} = (\AutoTuple{\vb\beta}{5})\).
阶梯形矩阵\(\vb{B}\)有3行不为零,故\begin{equation*}
	\rank\{\AutoTuple{\vb\alpha}{5}\}=3.
\end{equation*}
又因为\(\vb{B}\)的非零首元分别位于1、2、4列,
则\(\vb\beta_1,\vb\beta_2,\vb\beta_4\)是\(\vb{B}\)的一个列极大线性无关组,
相应地,\(\vb\alpha_1,\vb\alpha_2,\vb\alpha_4\)是\(\vb{A}\)的一个列极大线性无关组,
即\(\{\AutoTuple{\vb\alpha}{5}\}\)的极大线性无关组.
\end{solution}
\end{example}

\begingroup
%@credit: 我原本以为只靠向量个数和秩两个信息就可以推断一个向量组含有多少个极大线性无关组,但是 {ce603838-a24d-4616-9395-d7b223e8cb72} 告诉我以下两个例子
\begin{example}
设向量组\(A = \{(1,0)^T,(0,1)^T,(1,1)^T\}\).
求\(A\)的不相同的极大线性无关组的个数.
\begin{solution}
容易看出\(\rank A = 2\).
\(A\)一共有\(C_3^2 = 3\)个含有\(2\)个向量的部分组,分别是\begin{equation*}
	A_1 \defeq \{(1,0)^T,(0,1)^T\},
	\qquad
	A_2 \defeq \{(1,0)^T,(1,1)^T\},
	\qquad
	A_3 \defeq \{(0,1)^T,(1,1)^T\}.
\end{equation*}
因为\(
	\rank A_1
	= \rank A_2
	= \rank A_3
	= 2
\),
所以\(A_1,A_2,A_3\)都是\(A\)的极大线性无关组,
\(A\)有3个不同的极大线性无关组.
\end{solution}
\end{example}
\begin{example}
设向量组\(A = \{(1,0)^T,(0,1)^T,(1,0)^T\}\).
\begin{solution}
容易看出\(\rank A = 2\).
乍一看可能会以为本例的\(A\)和上例一样也有\(C_3^2 = 3\)个含有\(2\)个向量的部分组,
但是\(A\)实际上只有\(2\)个向量(即\((1,0)^T\)和\((0,1)^T\)).
因此\(A\)只有1个极大线性无关组.
\end{solution}
\end{example}
\begin{remark}
从上面两个例子可以看出:
一个向量组的极大线性无关组不一定是唯一的;
单纯依据一个向量组中的向量个数和向量组的秩,
不能断定这个向量组的不同的极大线性无关组的个数.
\end{remark}
%@Mathematica: DependentVectorListQ[A_?MatrixQ (* 向量组 *)] := MatrixRank[A] < Length[A]
%@Mathematica: IndependentVectorListQ[A_?MatrixQ (* 向量组 *)] := MatrixRank[A] == Length[A]
%@Mathematica: AllVectorLists[A_?MatrixQ (* 向量组 *)] := Module[
%					{n = Length[A], indexes},
%					indexes = Subsets[Range[n], n];
%					(* 输出给定向量组的部分组 *)
%					Table[A[[index]], {index, indexes}]
%				];
%@Mathematica: MaximallyLinearIndependentSubset[A_?MatrixQ (* 向量组 *)] := Module[
%					{n, r, indexes},
%					n = Length[A];
%					r = MatrixRank[A];
%					indexes = Subsets[Range[n], r];
%					Select[Table[A[[index]], {index, indexes}], IndependentVectorListQ]
%				];
\endgroup
