\section{西尔维斯特不等式}
\begin{theorem}
设\(\A \in M_{s \times n}(K),
\B \in M_{n \times t}(K)\),
则\begin{equation}\label{equation:线性方程组.西尔维斯特不等式}
	\rank\A + \rank\B - n \leq \rank(\A\B).
\end{equation}
当且仅当\[
	\rank\begin{bmatrix}
		\A & \vb0 \\
		\E_n & \B
	\end{bmatrix}
	= \rank\begin{bmatrix}
		\A & \vb0 \\
		\vb0 & \B
	\end{bmatrix}
\]时,
\cref{equation:线性方程组.西尔维斯特不等式} 取“=”号.
\begin{proof}
由\cref{equation:矩阵的秩.分块矩阵的秩的等式1},\[
	\rank\begin{bmatrix}
		\E_n & \z \\
		\z & \A\B
	\end{bmatrix}
	= n + \rank(\A\B).
	\eqno(1)
\]
又因为\[
	\begin{bmatrix}
		\B & \E_n \\
		\z & \A
	\end{bmatrix}
	= \begin{bmatrix}
		\E_n & \z \\
		\A & \E_s
	\end{bmatrix}
	\begin{bmatrix}
		\E_n & \z \\
		\z & \A\B
	\end{bmatrix}
	\begin{bmatrix}
		\E_n & -\B \\
		\z & \E_t
	\end{bmatrix}
	\begin{bmatrix}
		\z & \E_s \\
		-\E_t & \z
	\end{bmatrix},
\]
而\[
	\begin{bmatrix}
		\E_n & \z \\
		\A & \E_s
	\end{bmatrix}, \qquad
	\begin{bmatrix}
		\E_n & -\B \\
		\z & \E_t
	\end{bmatrix},
	\quad\text{和}\quad
	\begin{bmatrix}
		\z & \E_s \\
		-\E_t & \z
	\end{bmatrix}
\]这三个矩阵都是满秩矩阵,
所以\[
	\rank\begin{bmatrix}
		\E_n & \z \\
		\z & \A\B
	\end{bmatrix}
	= \rank\begin{bmatrix}
		\B & \E_n \\
		\z & \A
	\end{bmatrix}.
	\eqno(2)
\]
再由\cref{equation:矩阵的秩.分块矩阵的秩的不等式} 有\[
	\rank\begin{bmatrix}
		\B & \E_n \\
		\z & \A
	\end{bmatrix}
	\geq \rank\A+\rank\B.
	\eqno(3)
\]
因此,\(\rank\A + \rank\B \leq n + \rank(\A\B)\).
%@see: https://math.stackexchange.com/a/2414197/591741
%@see: http://www.m-hikari.com/imf-password2009/33-36-2009/luIMF33-36-2009.pdf
\end{proof}
%@see: https://math.stackexchange.com/questions/872587/equality-case-in-the-frobenius-rank-inequality
\end{theorem}

我们把\cref{equation:线性方程组.西尔维斯特不等式}
称为\DefineConcept{西尔维斯特不等式}(Sylvester rank inequality).

\begin{example}\label{example:矩阵乘积的秩.乘积为零的两个矩阵的秩之和}
%@see: 《高等代数(第三版 上册)》(丘维声) P143 习题4.5 1.
设\(\A \in M_{s \times n}(K),
\B \in M_{n \times m}(K)\).
如果\(\A\B=\vb0\),
那么\[
	\rank\A + \rank\B \leq n.
\]
\begin{proof}
由\hyperref[equation:线性方程组.西尔维斯特不等式]{西尔维斯特不等式}立即可得.
\end{proof}
\end{example}

我们可以利用\hyperref[equation:线性方程组.西尔维斯特不等式]{西尔维斯特不等式}证明一个重要结论:
\begin{proposition}\label{theorem:向量空间.用列满秩矩阵左乘任一矩阵不变秩}
设\(\A \in M_{m \times s}(K),
\B \in M_{s \times n}(K)\).
\begin{itemize}
	\item 如果\(\A\)是列满秩矩阵,则\(\rank(\A\B) = \rank\B\).
	\item 如果\(\B\)是行满秩矩阵,则\(\rank(\A\B) = \rank\A\).
\end{itemize}
\begin{proof}
假设\(\A\)是列满秩矩阵,
即\(\rank\A = s\).
由\hyperref[equation:线性方程组.西尔维斯特不等式]{西尔维斯特不等式}有\[
	\rank(\A\B) \geq \rank\B + \rank\A - s
	= \rank\B. % 代入\(\rank\A = s\)
	\eqno(1)
\]
又由\cref{theorem:线性方程组.矩阵乘积的秩} 可知\[
	\rank(\A\B) \leq \rank\B.
	\eqno(2)
\]
由(1)(2)两式便有\(\rank(\A\B) = \rank\B\).

同理可证:如果\(\B\)是行满秩矩阵,则\(\rank(\A\B) = \rank\A\).
\end{proof}
\end{proposition}
\begin{remark}
\cref{theorem:向量空间.用列满秩矩阵左乘任一矩阵不变秩} 说明:
对于任意一个矩阵,我们用一个列满秩矩阵左乘它,不变秩;
用一个行满秩矩阵右乘它,也不变秩.
%@credit: {439f21f7-fd12-4996-b112-dcbb8b467950} 给出逆命题不成立的反例
但是要注意\cref{theorem:向量空间.用列满秩矩阵左乘任一矩阵不变秩} 的逆命题并不成立,
只要取\(\A = \B = \vb0\),就有\(\rank(\A \B) = \rank\A = \rank\B = 0\),
但是零矩阵\(\A,\B\)显然既不是列满秩矩阵也不是行满秩矩阵.
%@credit: {523653db-1ec1-4b3a-972f-e44311ded599} 给出了条件增强后逆命题仍不成立的反例(\(\A = \B\)是一个幂等矩阵)
%@credit: {de3029b8-10a6-4ae5-8f64-108dae1c10a9} 给出了下面用到的具体的幂等矩阵
即便增加一个条件 --- “\(\rank(\A \B)\neq0\)”,
也不能断定\cref{theorem:向量空间.用列满秩矩阵左乘任一矩阵不变秩} 的逆命题一定成立,
这是因为只要取\(\A = \B
= \begin{bmatrix}
	1 & 0 \\
	0 & 0
\end{bmatrix}\)
(实际上对于任意一个不满秩的\hyperref[definition:幂等矩阵.幂等矩阵的定义]{幂等矩阵}~\(\A\),
逆命题始终不成立),
就有\(\A \B = \A\),
从而有\(\rank(\A \B) = \rank\A = \rank\B\),
但是\(\A,\B\)显然既不是列满秩矩阵也不是行满秩矩阵.
% 我们只要把\cref{theorem:向量空间.用列满秩矩阵左乘任一矩阵不变秩} 中的各个矩阵转置,
% 得到\(\A_1 = \A^T \in M_{n \times s}(K),
% \B_1 = \B^T \in M_{s \times m}(K),
% \A_1\B_1 = \A^T\B^T = (\B\A)^T\),
% 易见\(\B_1\)是行满秩矩阵,且\[
% 	\rank(\A_1\B_1)
% 	= \rank(\B\A)^T
% 	= \rank(\B\A)
% 	= \rank\A
% 	= \rank\A^T.
% \]
\end{remark}
\begin{corollary}\label{theorem:西尔维斯特不等式.分块矩阵的秩的等式3}
%@see: 《2018年全国硕士研究生入学统一考试(数学一)》一选择题/第6题
设\(\A \in M_{m \times n}(K),
\B \in M_{n \times t}(K),
\C \in M_{s \times m}(K)\),
则\begin{gather*}
	\rank\begin{bmatrix}
		\A \\
		\C \A
	\end{bmatrix}
	= \rank\A, \\
	\rank(\A,\A \B)
	= \rank\A.
\end{gather*}
\begin{proof}
由\cref{theorem:向量空间.用列满秩矩阵左乘任一矩阵不变秩} 立即可得.
\end{proof}
\end{corollary}
\begin{example}
设\(\A \in M_{m \times n}(K),
\B \in M_{n \times t}(K)\).
举例说明:\(\rank(\A,\B \A) \neq \rank\A\).
\begin{solution}
取\[
%@Mathematica: A = {{1, 0}, {0, 0}}
	\A = \begin{bmatrix}
		1 & 0 \\
		0 & 0
	\end{bmatrix},
	\qquad
%@Mathematica: B = {{1, 0}, {1, 1}}
	\B = \begin{bmatrix}
		1 & 0 \\
		1 & 1
	\end{bmatrix},
\]
那么\[
%@Mathematica: B.A
	\B \A = \begin{bmatrix}
		1 & 0 \\
		1 & 0
	\end{bmatrix},
	\qquad
%@Mathematica: Join[A, B.A, 2]
	(\A,\B \A) = \begin{bmatrix}
		1 & 0 & 1 & 0 \\
		0 & 0 & 1 & 0
	\end{bmatrix},
\]
%@Mathematica: MatrixRank[Join[A, B.A, 2]]
%@Mathematica: MatrixRank[A]
于是\(\rank(\A,\B \A) = 2 \neq 1 = \rank\A\).
\end{solution}
\end{example}
\begin{example}\label{example:向量空间.等秩矩阵的行向量组的等价性}
设\(\A \in M_{s \times n}(K),
\B \in M_{m \times n}(K)\),
且\(\rank\A
= \rank\begin{bmatrix}
	\A \\ \B
\end{bmatrix}\),
则\(\A\)的行向量组的任意一个极大线性无关组
都是\(\begin{bmatrix}
	\A \\ \B
\end{bmatrix}\)的行向量组的极大线性无关组.
\begin{proof}
%@credit: {5a781423-ba4e-4629-ac1a-eac743a4d445},{6c964576-9569-472e-969e-54699e35974b},{5f4d2f8a-fc8b-4798-85d6-98670f6761e7},{6f21d9e6-edca-4b6f-9dff-364f3d62dcce}
设\(A' = \AutoTuple{\vb\alpha}{r}\)是\(\A\)的行向量组的一个极大线性无关组.
用反证法.
假设\(\B\)的某一个行向量\(\vb\beta\)不可以由\(A'\)线性表出,
即向量组\(A' \cup \{\vb\beta\}\)线性无关,
那么\begin{equation*}
	\rank\begin{bmatrix}
		\A \\ \B
	\end{bmatrix}
	\geq \card(A' \cup \{\vb\beta\})
	= r + 1,
\end{equation*}
与题设矛盾!
因此\(\B\)的行向量组可以由\(A'\)线性表出,
\(A'\)是\(\begin{bmatrix}
	\A \\ \B
\end{bmatrix}\)的极大线性无关组.
\end{proof}
\end{example}

\begin{example}
%@see: 《高等代数(第三版 上册)》(丘维声) P143 习题4.5 2.
设\(\A\)是数域\(K\)上的\(n\)阶非零矩阵.
证明:“存在一个\(n \times m\)非零矩阵\(\B\),使得\(\A\B=\vb0\)”的充分必要条件为
\(\abs{\A}=0\).
\begin{proof}
%@credit: {8b6edada-f2fd-4ae5-9020-eb533149a54c},{c5f76621-34f0-4114-845f-e36475300576},{5f4d2f8a-fc8b-4798-85d6-98670f6761e7},{ce603838-a24d-4616-9395-d7b223e8cb72}
必要性.
假设存在一个\(n \times m\)非零矩阵\(\B\),使得\(\A\B=\vb0\).
那么只要任取\(\B\)的一个非零列向量\(\vb\beta\),
就有\(\A\vb\beta = \vb0\),
即\(\vb\beta\)是齐次方程\(\A\vb{x}=\vb0\)的非零解,
故\(\rank\A<n\),
从而有\(\abs{\A}=0\).

充分性.
假设\(\abs{\A}=0\),
则\(\rank\A = r < n\),
在\(\A\)的核空间\(\Ker\A\)中
任取一个非零向量\(\vb\beta\),
% 从而成立\(\A\vb\beta=\vb0\),
构成一个\(n \times m\)矩阵\(\B\),
使得\(\B\)的每一列都是\(\vb\beta\),
那么\(\A\B=\vb0\).
\end{proof}
\end{example}
\begin{example}
%@see: 《高等代数(第三版 上册)》(丘维声) P143 习题4.5 3.
设\(\A\)是数域\(K\)上的\(n\)阶方阵,
\(\B\)是数域\(K\)上的\(n \times m\)行满秩矩阵,
\(\E\)是数域\(K\)上的\(n\)阶单位矩阵.
证明:\begin{itemize}
	\item 如果\(\A\B=\vb0\),则\(\A=\vb0\).
	\item 如果\(\A\B=\B\),则\(\A=\E\).
\end{itemize}
\begin{proof}
因为\(\B\)是行满秩矩阵,
由\cref{theorem:向量空间.用列满秩矩阵左乘任一矩阵不变秩} 可知,
\(\rank(\A\B) = \rank\A\).
假设\(\A\B=\vb0\),
则\[
	\rank\A=\rank(\A\B)=\rank\vb0=0,
\]
那么由\cref{theorem:向量空间.秩为零的矩阵必为零矩阵} 可知,
\(\A=\vb0\).

假设\(\A\B=\B\),
则\((\A-\E)\B=\vb0\),
那么由上述结论可知\(\A-\E=\vb0\),
因此\(\A=\E\).
\end{proof}
\end{example}

\begin{example}
%@see: 《高等代数(第三版 上册)》(丘维声) P143 习题4.5 9.
设矩阵\(\A \in M_n(K)\ (n\geq2)\),\(\A^*\)是\(\A\)的伴随矩阵.
证明:\begin{equation}\label{equation:伴随矩阵.伴随矩阵的秩}
	\rank\A^* = \left\{ \begin{array}{cl}
		n, & \rank\A=n, \\
		1, & \rank\A=n-1, \\
		0, & \rank\A<n-1.
	\end{array} \right.
\end{equation}
\begin{proof}
根据恒等式 \labelcref{equation:行列式.伴随矩阵.恒等式1} 有\[
	\A \A^* = \abs{\A} \E,
	\eqno(1)
\]
于是\[
	\abs{\A \A^*} = \abs{\abs{\A} \E}.
	\eqno(2)
\]
根据\cref{theorem:行列式.矩阵乘积的行列式},
(2)式左边可以分解为\[
	\abs{\A \A^*} = \abs{\A} \abs{\A^*}.
\]
根据\cref{theorem:行列式.性质2.推论2},
(2)式右边可以分解为\[
	\abs{\abs{\A} \E} = \abs{\A}^n \abs{\E} = \abs{\A}^n.
\]
于是我们得到\[
	\abs{\A} \abs{\A^*} = \abs{\A}^n.
	\eqno(3)
\]

下面我们根据矩阵\(\A\)的秩的取值分类讨论.
\begin{enumerate}
	\item 当\(\rank\A = n\)时,
	由\cref{theorem:向量空间.满秩方阵的行列式非零} 有\(\abs{\A} \neq 0\);
	于是(3)式可化简得\[
		\abs{\A^*}
		= \abs{\A}^{n-1} \neq 0,
	\]
	再次利用\cref{theorem:向量空间.满秩方阵的行列式非零} 便知\(\rank\A^* = n\).

	\item 当\(\rank\A = n-1\)时,
	因为\(\rank\A<n\),
	所以\(\abs{\A} = 0\),
	那么由(1)式可知\(\A \A^* = \z\),
	那么利用\hyperref[equation:线性方程组.西尔维斯特不等式]{西尔维斯特不等式}便得\[
		\rank\A + \rank\A^* \leq n,
	\]
	移项得\[
		\rank\A^*
		\leq n - \rank\A
		= n - (n-1)
		= 1.
	\]
	又因为\(n > 1\),
	\(\rank\A = n-1 > 0\),
	根据矩阵的秩的定义,\(\A\)有一个\(n-1\)阶子式不等于零,
	再根据伴随矩阵的定义,这个子式是\(\A^*\)的一个元素,从而\(\A^*\neq\z\),
	\(\rank\A^*>0\).
	因此\(\rank\A^* = 1\).

	\item 当\(\rank\A < n-1\)时,
	\(\A\)的所有\(n-1\)阶子式全为零,
	也就是说\(\A\)的任意一个元素的代数余子式为零,
	根据伴随矩阵的定义,\(\A^* = \z\),
	因此\(\rank\A^* = 0\).
	\qedhere
\end{enumerate}
\end{proof}
\end{example}
\begin{remark}
从\cref{equation:伴随矩阵.伴随矩阵的秩} 还推导出以下结论:
%@see: 《高等代数(第三版 上册)》(丘维声) P143 习题4.5 8.
%@see: 《高等代数学习指导书(第三版)》(姚慕生、谢启鸿) P63 例2.26
\begin{equation}\label{equation:伴随矩阵.伴随矩阵的行列式}
	\abs{\A^*}
	= \abs{\A}^{n-1}.
\end{equation}
\end{remark}

\begin{example}
%@see: 《高等代数(第三版 上册)》(丘维声) P143 习题4.5 10.
%@see: 《高等代数学习指导书(第三版)》(姚慕生、谢启鸿) P64 例2.27
设矩阵\(\A \in M_n(K)\ (n\geq2)\),\(\A^*\)是\(\A\)的伴随矩阵.
证明:\begin{equation}\label{equation:伴随矩阵.伴随矩阵的伴随}
	(\A^*)^* = \left\{ \begin{array}{cl}
		\abs{\A}^{n-2} \A, & n\geq3, \\
		\A, & n=2.
	\end{array} \right.
\end{equation}
\begin{proof}
当\(n=2\)时,
设\(\A = \begin{bmatrix}
	a_{11} & a_{12} \\
	a_{21} & a_{22}
\end{bmatrix}\),
那么\[
	\A^* = \begin{bmatrix}
		A_{11} & A_{21} \\
		A_{12} & A_{22}
	\end{bmatrix}
	= \begin{bmatrix}
		a_{22} & - a_{12} \\
		- a_{21} & a_{11}
	\end{bmatrix}.
\]
从而\[
	(\A^*)^* = \begin{bmatrix}
		a_{11} & a_{12} \\
		a_{21} & a_{22}
	\end{bmatrix}
	= \A.
\]

下面证明当\(n\geq3\)时,成立\((\A^*)^* = \abs{\A}^{n-2} \A\).

假设\(\A\)可逆.
那么由\cref{theorem:逆矩阵.逆矩阵的唯一性}
可知\(\A^* = \abs{\A} \A^{-1}\).
由\cref{equation:伴随矩阵.伴随矩阵的行列式}
可知\(\abs{\A^*} = \abs{\A}^{n-1}\).
由\cref{theorem:逆矩阵.伴随矩阵的逆与逆矩阵的伴随}
可知\((\A^*)^{-1} = \abs{\A}^{-1} \A\).
于是\begin{align*}
	(\A^*)^*
	&= \abs{\A^*} (\A^*)^{-1} \\
	&= \abs{\A}^{n-1} \cdot \abs{\A}^{-1} \A \\
	&= \abs{\A}^{n-2} \A
	\quad(n\geq2).
\end{align*}

假设\(\A\)不可逆,即\(\rank\A<n\),
那么由\cref{equation:伴随矩阵.伴随矩阵的秩}
可知\(\rank\A^* \leq 1 < 2 \leq n-1\),
从而\(\rank(\A^*)^* = 0\),
即\((\A^*)^* = \vb0\).
这时\(\abs{\A}=0\),
因此\((\A^*)^* = \abs{\A}^{n-2} \A\)仍然成立.
\end{proof}
\end{example}

我们还可以将\hyperref[equation:线性方程组.西尔维斯特不等式]{西尔维斯特不等式}进行如下的推广.
\begin{theorem}
设\(\A \in M_{s \times n}(K),
\B \in M_{n \times m}(K),
\C \in M_{m \times t}(K)\),
则\begin{equation}\label{equation:线性方程组.弗罗贝尼乌斯不等式}
	\rank(\A\B\C) \geq \rank(\A\B) + \rank(\B\C) - \rank\B.
\end{equation}
\begin{proof}
利用初等变换,有\[
	\begin{bmatrix}
		\B & \z \\
		\z & \A\B\C
	\end{bmatrix}
	\to \begin{bmatrix}
		\B & \z \\
		\A\B & \A\B\C
	\end{bmatrix}
	\to \begin{bmatrix}
		\B & -\B\C \\
		\A\B & \z
	\end{bmatrix}
	\to \begin{bmatrix}
		\B\C & \B \\
		\z & \A\B
	\end{bmatrix},
\]
于是\[
	\rank\B + \rank(\A\B\C)
	= \rank\begin{bmatrix}
		\B & \z \\
		\z & \A\B\C
	\end{bmatrix}
	= \rank\begin{bmatrix}
		\B\C & \B \\
		\z & \A\B
	\end{bmatrix}
	\geq \rank(\A\B) + \rank(\B\C).
	\qedhere
\]
\end{proof}
\end{theorem}

我们把\cref{equation:线性方程组.弗罗贝尼乌斯不等式}
称为\DefineConcept{弗罗贝尼乌斯不等式}(Frobenius rank inequality).

\begin{example}\label{example:对合矩阵.对合矩阵的秩的性质1}
%@see: 《高等代数(第三版 上册)》(丘维声) P143 习题4.5 4.
设\(\A\)是数域\(K\)上的\(n\)阶对合矩阵,
即有\(\A^2=\E\),
则\[
	\rank(\E+\A)+\rank(\E-\A)=n.
\]
\begin{proof}
因为利用初等变换可以得到\begin{align*}
	&\hspace{-20pt}
	\begin{bmatrix}
		\E+\A & \z \\
		\z & \E-\A
	\end{bmatrix}
	\to \begin{bmatrix}
		\E+\A & \z \\
		\A(\E+\A) & \E-\A
	\end{bmatrix}
	= \begin{bmatrix}
		\E+\A & \z \\
		\A+\E & \E-\A
	\end{bmatrix} \\
	&\to \begin{bmatrix}
		\E+\A & \E-\A \\
		\z & \z
	\end{bmatrix}
	\to \begin{bmatrix}
		\E+\A & 2\E \\
		\z & \z
	\end{bmatrix}
	\to \begin{bmatrix}
		\E+\A & \E \\
		\z & \z
	\end{bmatrix}
	\to \begin{bmatrix}
		\A & \E \\
		\z & \z
	\end{bmatrix},
\end{align*}
所以\[
	\rank(\E+\A)+\rank(\E-\A)
	=\rank\begin{bmatrix}
		\E+\A & \z \\
		\z & \E-\A
	\end{bmatrix}
	= \rank\begin{bmatrix}
		\A & \E \\
		\z & \z
	\end{bmatrix}
	= n.
	\qedhere
\]
\end{proof}
\end{example}

\begin{example}\label{example:幂等矩阵.幂等矩阵的秩的性质1}
%@see: 《高等代数(第三版 上册)》(丘维声) P143 习题4.5 5.
设\(\A\)是数域\(K\)上的\(n\)阶幂等矩阵,
即有\(\A^2=\A\).
证明:\[
	\rank\A+\rank(\E-\A)=n.
\]
\begin{proof}
由于\[
	\A^2=\A
	\iff
	\A^2-\A=\z
	\iff
	\rank(\A^2-\A)=0.
\]
又因为利用初等变换可以得到\[
	\begin{bmatrix}
		\A & \z \\
		\z & \E_n-\A
	\end{bmatrix}
	\to \begin{bmatrix}
		\A & \z \\
		\A & \E_n-\A
	\end{bmatrix}
	\to \begin{bmatrix}
		\A & \A \\
		\A & \E_n
	\end{bmatrix}
	\to \begin{bmatrix}
		\A-\A^2 & \z \\
		\A & \E_n
	\end{bmatrix}
	\to \begin{bmatrix}
		\A-\A^2 & \z \\
		\z & \E_n
	\end{bmatrix},
\]
所以\(\rank\A+\rank(\E_n-\A)
=\rank(\A-\A^2)+n
=n\).
\end{proof}
\end{example}

\begin{example}\label{example:单位矩阵与两矩阵乘积之差.单位矩阵与两矩阵乘积之差的秩}
设\(\A \in M_{m \times n}(K),
\B \in M_{n \times m}(K)\).
证明:\[
	\rank(\E_m-\A\B)-\rank(\E_n-\B\A)=m-n.
\]
\begin{proof}
因为利用初等变换可以得到\begin{align*}
	\begin{bmatrix}
		\E_m-\A\B & \z \\
		\z & \E_n
	\end{bmatrix}
	&\to \begin{bmatrix}
		\E_m-\A\B & \A \\
		\z & \E_n
	\end{bmatrix}
	\to \begin{bmatrix}
		\E_m & \A \\
		\B & \E_n
	\end{bmatrix} \\
	&\to \begin{bmatrix}
		\E_m & \z \\
		\B & \E_n-\B\A
	\end{bmatrix}
	\to \begin{bmatrix}
		\E_m & \z \\
		\z & \E_n-\B\A
	\end{bmatrix},
\end{align*}
所以\[
	\rank(\E_m-\A\B)+n=m+\rank(\E_n-\B\A),
\]
移项便得\(\rank(\E_m-\A\B)-\rank(\E_n-\B\A)=m-n\).
\end{proof}
%\cref{example:逆矩阵.行列式降阶定理的重要应用1}
%\cref{example:单位矩阵与两矩阵乘积之差.单位矩阵与两矩阵乘积之差的行列式}
\end{example}

\begin{example}
设\(\A,\B,\C \in M_n(K)\),
\(\rank\C=n\),
\(\A(\B\A+\C)=\z\).
证明:\[
	\rank(\B\A+\C)=n-\rank\A.
\]
\begin{proof}
因为\(\A(\B\A+\C)=\z\),
所以根据\cref{equation:线性方程组.西尔维斯特不等式} 有\[
	\rank(\B\A+\C)+\rank\A \leq n.
\]
又利用初等变换可以得到\[
	\begin{bmatrix}
		\B\A+\C & \z \\
		\z & \A
	\end{bmatrix}
	\to \begin{bmatrix}
		\B\A+\C & \B\A \\
		\z & \A
	\end{bmatrix}
	\to \begin{bmatrix}
		\C & \B\A \\
		-\A & \A
	\end{bmatrix},
\]
于是\[
	\rank(\B\C+\C)
	+\rank\A
	= \rank\begin{bmatrix}
		\B\A+\C & \z \\
		\z & \A
	\end{bmatrix}
	= \rank\begin{bmatrix}
		\C & \B\A \\
		-\A & \A
	\end{bmatrix}
	\geq \rank\C = n.
	\qedhere
\]
\end{proof}
\end{example}

\begin{example}
%@see: 《2021年全国硕士研究生入学统一考试(数学一)》一选择题/第7题
设\(\A,\B \in M_n(\mathbb{R})\),
证明:\begin{equation*}
	\rank\begin{bmatrix}
		\A & \vb0 \\
		\vb0 & \A^T \A
	\end{bmatrix}
	= \rank\begin{bmatrix}
		\A & \A \B \\
		\vb0 & \A^T
	\end{bmatrix}
	= \rank\begin{bmatrix}
		\A & \vb0 \\
		\B \A & \A^T
	\end{bmatrix}
	= 2 \rank\A,
\end{equation*}
并举例说明\(\rank\begin{bmatrix}
	\A & \B \A \\
	\vb0 & \A \A^T
\end{bmatrix}
\neq 2 \rank\A\).
\begin{proof}
因为\begin{gather*}
	\rank\begin{bmatrix}
		\A & \vb0 \\
		\vb0 & \A^T \A
	\end{bmatrix}
	= \rank\A + \rank(\A^T \A), \\
	\begin{bmatrix}
		\A & \A \B \\
		\vb0 & \A^T
	\end{bmatrix}
	= \begin{bmatrix}
		\A & \vb0 \\
		\vb0 & \A^T
	\end{bmatrix}
	\begin{bmatrix}
		\E & \B \\
		\vb0 & \E
	\end{bmatrix}
	\implies
	\rank\begin{bmatrix}
		\A & \A \B \\
		\vb0 & \A^T
	\end{bmatrix}
	= \rank\begin{bmatrix}
		\A & \vb0 \\
		\vb0 & \A^T
	\end{bmatrix}
	= \rank\A + \rank\A^T, \\
	\begin{bmatrix}
		\A & \vb0 \\
		\B \A & \A^T
	\end{bmatrix}
	= \begin{bmatrix}
		\E & \vb0 \\
		\B & \E
	\end{bmatrix}
	\begin{bmatrix}
		\A & \vb0 \\
		\vb0 & \A^T
	\end{bmatrix}
	\implies
	\rank\begin{bmatrix}
		\A & \vb0 \\
		\B \A & \A^T
	\end{bmatrix}
	= \rank\begin{bmatrix}
		\A & \vb0 \\
		\vb0 & \A^T
	\end{bmatrix}
	= \rank\A + \rank\A^T,
\end{gather*}
而由\cref{theorem:向量空间.转置不变秩} 可知\(\rank\A = \rank\A^T\),
且由\cref{equation:矩阵乘积的秩.实矩阵及其转置矩阵的乘积的秩} 可知\(\rank(\A^T \A) = \rank\A\),
所以\begin{equation*}
	\rank\begin{bmatrix}
		\A & \vb0 \\
		\vb0 & \A^T \A
	\end{bmatrix}
	= \rank\begin{bmatrix}
		\A & \A \B \\
		\vb0 & \A^T
	\end{bmatrix}
	= \rank\begin{bmatrix}
		\A & \vb0 \\
		\B \A & \A^T
	\end{bmatrix}
	= 2 \rank\A.
\end{equation*}

取\(\A = \begin{bmatrix}
	0 & 1 \\
	0 & 0
\end{bmatrix},
\B = \begin{bmatrix}
	0 & 1 \\
	1 & 0
\end{bmatrix}\),
则\(\rank\A = 1\),
而\begin{equation*}
	\rank\begin{bmatrix}
		\A & \B \A \\
		\vb0 & \A \A^T
	\end{bmatrix}
	= \rank\begin{bmatrix}
		0 & 1 & 0 & 0 \\
		0 & 0 & 0 & 1 \\
		0 & 0 & 1 & 0 \\
		0 & 0 & 0 & 0
	\end{bmatrix}
	= 3 \neq 2,
\end{equation*}
所以\(\rank\begin{bmatrix}
	\A & \B \A \\
	\vb0 & \A \A^T
\end{bmatrix}
\neq 2 \rank\A\).
\end{proof}
\end{example}
\begin{example}
%@see: 《2023年全国硕士研究生入学统一考试(数学一)》一选择题/第5题
\def\M{\vb{M}}
设矩阵\(\A,\B,\C \in M_n(K)\)满足\(\A \B \C = \vb0\),
\(\E\)是数域\(K\)上的\(n\)阶单位矩阵.
试比较以下三个矩阵的秩的序关系:\begin{equation*}
	\M_1 \defeq \begin{bmatrix}
		\vb0 & \A \\
		\B \C & \E
	\end{bmatrix},
	\qquad
	\M_2 \defeq \begin{bmatrix}
		\A \B & \C \\
		\vb0 & \E
	\end{bmatrix},
	\qquad
	\M_3 \defeq \begin{bmatrix}
		\E & \A \B \\
		\A \B & \vb0
	\end{bmatrix}.
\end{equation*}
\begin{solution}
因为\begin{align*}
	\begin{bmatrix}
		\E & -\A \\
		\vb0 & \E
	\end{bmatrix}
	\begin{bmatrix}
		\vb0 & \A \\
		\B \C & \E
	\end{bmatrix}
	&= \begin{bmatrix}
		% 第一行第一列的元素等于\(\vb0 - \A \B \C = \vb0\)
		% 第一行第二列的元素等于\(\A - \A = \vb0\)
		\vb0 & \vb0 \\
		\B \C & \E
	\end{bmatrix}, \\
	\begin{bmatrix}
		\E & -\C \\
		\vb0 & \E
	\end{bmatrix}
	\begin{bmatrix}
		\A \B & \C \\
		\vb0 & \E
	\end{bmatrix}
	&= \begin{bmatrix}
		\A \B & \vb0 \\
		\vb0 & \E
	\end{bmatrix}, \\
	\begin{bmatrix}
		\E & \vb0 \\
		-\A \B & \E
	\end{bmatrix}
	\begin{bmatrix}
		\E & \A \B \\
		\A \B & \vb0
	\end{bmatrix}
	\begin{bmatrix}
		\E & -\A \B \\
		\vb0 & \E
	\end{bmatrix}
	&= \begin{bmatrix}
		\E & \vb0 \\
		\vb0 & -\A \B \A \B
	\end{bmatrix},
\end{align*}
而\begin{align*}
	\rank\M_1
	&= \rank\begin{bmatrix}
		\vb0 & \vb0 \\
		\B \C & \E
	\end{bmatrix}
	= n, \\
	\rank\M_2
	&= \rank\begin{bmatrix}
		\A \B & \vb0 \\
		\vb0 & \E
	\end{bmatrix}
	= \rank(\A\B) + n
	%\cref{theorem:线性方程组.矩阵的秩的性质2}
	\geq n, \\
	\rank\M_3
	&= \rank\begin{bmatrix}
		\E & \vb0 \\
		\vb0 & -\A \B \A \B
	\end{bmatrix}
	= \rank(\A \B \A \B) + n
	%\cref{theorem:线性方程组.矩阵的秩的性质2}
	\geq n,
\end{align*}
%\cref{theorem:线性方程组.矩阵乘积的秩}
这里\(\rank(\A \B) \geq \rank(\A \B \A \B)\),
所以\(\rank\M_2 \geq \rank\M_3 \geq \rank\M_1\).
\end{solution}
\end{example}
