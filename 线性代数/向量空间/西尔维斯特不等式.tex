\section{西尔维斯特不等式}
\begin{theorem}
设\(\vb{A} \in M_{s \times n}(K),
\vb{B} \in M_{n \times t}(K)\),
则\begin{equation}\label{equation:线性方程组.西尔维斯特不等式}
	\rank\vb{A} + \rank\vb{B} - n \leq \rank(\vb{A}\vb{B}).
\end{equation}
当且仅当\begin{equation*}
	\rank\begin{bmatrix}
		\vb{A} & \vb0 \\
		\vb{E}_n & \vb{B}
	\end{bmatrix}
	= \rank\begin{bmatrix}
		\vb{A} & \vb0 \\
		\vb0 & \vb{B}
	\end{bmatrix}
\end{equation*}时,
\cref{equation:线性方程组.西尔维斯特不等式} 取“=”号.
\begin{proof}
由\cref{equation:矩阵的秩.分块矩阵的秩的等式1},\begin{equation*}
	\rank\begin{bmatrix}
		\vb{E}_n & \vb0 \\
		\vb0 & \vb{A}\vb{B}
	\end{bmatrix}
	= n + \rank(\vb{A}\vb{B}).
	\eqno(1)
\end{equation*}
又因为\begin{equation*}
	\begin{bmatrix}
		\vb{B} & \vb{E}_n \\
		\vb0 & \vb{A}
	\end{bmatrix}
	= \begin{bmatrix}
		\vb{E}_n & \vb0 \\
		\vb{A} & \vb{E}_s
	\end{bmatrix}
	\begin{bmatrix}
		\vb{E}_n & \vb0 \\
		\vb0 & \vb{A}\vb{B}
	\end{bmatrix}
	\begin{bmatrix}
		\vb{E}_n & -\vb{B} \\
		\vb0 & \vb{E}_t
	\end{bmatrix}
	\begin{bmatrix}
		\vb0 & \vb{E}_s \\
		-\vb{E}_t & \vb0
	\end{bmatrix},
\end{equation*}
而\begin{equation*}
	\begin{bmatrix}
		\vb{E}_n & \vb0 \\
		\vb{A} & \vb{E}_s
	\end{bmatrix}, \qquad
	\begin{bmatrix}
		\vb{E}_n & -\vb{B} \\
		\vb0 & \vb{E}_t
	\end{bmatrix},
	\quad\text{和}\quad
	\begin{bmatrix}
		\vb0 & \vb{E}_s \\
		-\vb{E}_t & \vb0
	\end{bmatrix}
\end{equation*}这三个矩阵都是满秩矩阵,
所以\begin{equation*}
	\rank\begin{bmatrix}
		\vb{E}_n & \vb0 \\
		\vb0 & \vb{A}\vb{B}
	\end{bmatrix}
	= \rank\begin{bmatrix}
		\vb{B} & \vb{E}_n \\
		\vb0 & \vb{A}
	\end{bmatrix}.
	\eqno(2)
\end{equation*}
再由\cref{equation:矩阵的秩.分块矩阵的秩的不等式} 有\begin{equation*}
	\rank\begin{bmatrix}
		\vb{B} & \vb{E}_n \\
		\vb0 & \vb{A}
	\end{bmatrix}
	\geq \rank\vb{A}+\rank\vb{B}.
	\eqno(3)
\end{equation*}
因此,\(\rank\vb{A} + \rank\vb{B} \leq n + \rank(\vb{A}\vb{B})\).
%@see: https://math.stackexchange.com/a/2414197/591741
%@see: http://www.m-hikari.com/imf-password2009/33-36-2009/luIMF33-36-2009.pdf
\end{proof}
%@see: https://math.stackexchange.com/questions/872587/equality-case-in-the-frobenius-rank-inequality
\end{theorem}

我们把\cref{equation:线性方程组.西尔维斯特不等式}
称为\DefineConcept{西尔维斯特不等式}(Sylvester rank inequality).

\begin{example}\label{example:西尔维斯特不等式.可逆矩阵的正整数次幂可逆}
设\(\vb{A} \in M_n(K)\)是可逆矩阵.
证明:\(\vb{A}^m\ (m=2,3,\dotsc)\)都是可逆矩阵.
\begin{proof}
首先证明\(\vb{A}^2\)是可逆矩阵.
由\cref{theorem:线性方程组.矩阵乘积的秩} 可知
\(\rank\vb{A}^2 \leq \rank\vb{A} = n\).
再由\hyperref[equation:线性方程组.西尔维斯特不等式]{西尔维斯特不等式}可知
\(\rank\vb{A}^2 \geq 2\rank\vb{A} - n = n\).
因此\(\rank\vb{A}^2 = n\).
接下来运用数学归纳法容易证得\(\rank\vb{A}^m = n\ (m=2,3,\dotsc)\).
\end{proof}
\end{example}

\begin{example}\label{example:矩阵乘积的秩.乘积为零的两个矩阵的秩之和}
%@see: 《高等代数(第三版 上册)》(丘维声) P143 习题4.5 1.
设\(\vb{A} \in M_{s \times n}(K),
\vb{B} \in M_{n \times m}(K)\).
如果\(\vb{A}\vb{B}=\vb0\),
那么\begin{equation*}
	\rank\vb{A} + \rank\vb{B} \leq n.
\end{equation*}
\begin{proof}
由\hyperref[equation:线性方程组.西尔维斯特不等式]{西尔维斯特不等式}立即可得.
\end{proof}
\end{example}

我们可以利用\hyperref[equation:线性方程组.西尔维斯特不等式]{西尔维斯特不等式}证明一个重要结论:
\begin{proposition}\label{theorem:向量空间.用列满秩矩阵左乘任一矩阵不变秩}
设\(\vb{A} \in M_{m \times s}(K),
\vb{B} \in M_{s \times n}(K)\).
\begin{itemize}
	\item 如果\(\vb{A}\)是列满秩矩阵,则\(\rank(\vb{A}\vb{B}) = \rank\vb{B}\).
	\item 如果\(\vb{B}\)是行满秩矩阵,则\(\rank(\vb{A}\vb{B}) = \rank\vb{A}\).
\end{itemize}
\begin{proof}
假设\(\vb{A}\)是列满秩矩阵,
即\(\rank\vb{A} = s\).
由\hyperref[equation:线性方程组.西尔维斯特不等式]{西尔维斯特不等式}有\begin{equation*}
	\rank(\vb{A}\vb{B}) \geq \rank\vb{B} + \rank\vb{A} - s
	= \rank\vb{B}. % 代入\(\rank\vb{A} = s\)
	\eqno(1)
\end{equation*}
又由\cref{theorem:线性方程组.矩阵乘积的秩} 可知\begin{equation*}
	\rank(\vb{A}\vb{B}) \leq \rank\vb{B}.
	\eqno(2)
\end{equation*}
由(1)(2)两式便有\(\rank(\vb{A}\vb{B}) = \rank\vb{B}\).

同理可证:如果\(\vb{B}\)是行满秩矩阵,则\(\rank(\vb{A}\vb{B}) = \rank\vb{A}\).
\end{proof}
\end{proposition}
\begin{remark}
\cref{theorem:向量空间.用列满秩矩阵左乘任一矩阵不变秩} 说明:
对于任意一个矩阵,我们用一个列满秩矩阵左乘它,不变秩;
用一个行满秩矩阵右乘它,也不变秩.
%@credit: {439f21f7-fd12-4996-b112-dcbb8b467950} 给出逆命题不成立的反例
但是要注意\cref{theorem:向量空间.用列满秩矩阵左乘任一矩阵不变秩} 的逆命题并不成立,
只要取\(\vb{A} = \vb{B} = \vb0\),就有\(\rank(\vb{A} \vb{B}) = \rank\vb{A} = \rank\vb{B} = 0\),
但是零矩阵\(\vb{A},\vb{B}\)显然既不是列满秩矩阵也不是行满秩矩阵.
%@credit: {523653db-1ec1-4b3a-972f-e44311ded599} 给出了条件增强后逆命题仍不成立的反例(\(\vb{A} = \vb{B}\)是一个幂等矩阵)
%@credit: {de3029b8-10a6-4ae5-8f64-108dae1c10a9} 给出了下面用到的具体的幂等矩阵
即便增加一个条件 --- “\(\rank(\vb{A} \vb{B})\neq0\)”,
也不能断定\cref{theorem:向量空间.用列满秩矩阵左乘任一矩阵不变秩} 的逆命题一定成立,
这是因为只要取\(\vb{A} = \vb{B}
= \begin{bmatrix}
	1 & 0 \\
	0 & 0
\end{bmatrix}\)
(实际上对于任意一个不满秩的\hyperref[definition:幂等矩阵.幂等矩阵的定义]{幂等矩阵}~\(\vb{A}\),
逆命题始终不成立),
就有\(\vb{A} \vb{B} = \vb{A}\),
从而有\(\rank(\vb{A} \vb{B}) = \rank\vb{A} = \rank\vb{B}\),
但是\(\vb{A},\vb{B}\)显然既不是列满秩矩阵也不是行满秩矩阵.
% 我们只要把\cref{theorem:向量空间.用列满秩矩阵左乘任一矩阵不变秩} 中的各个矩阵转置,
% 得到\(\vb{A}_1 = \vb{A}^T \in M_{n \times s}(K),
% \vb{B}_1 = \vb{B}^T \in M_{s \times m}(K),
% \vb{A}_1\vb{B}_1 = \vb{A}^T\vb{B}^T = (\vb{B}\vb{A})^T\),
% 易见\(\vb{B}_1\)是行满秩矩阵,且\begin{equation*}
% 	\rank(\vb{A}_1\vb{B}_1)
% 	= \rank(\vb{B}\vb{A})^T
% 	= \rank(\vb{B}\vb{A})
% 	= \rank\vb{A}
% 	= \rank\vb{A}^T.
% \end{equation*}
\end{remark}
\begin{corollary}\label{theorem:西尔维斯特不等式.分块矩阵的秩的等式3}
%@see: 《2018年全国硕士研究生入学统一考试(数学一)》一选择题/第6题
设\(\vb{A} \in M_{m \times n}(K),
\vb{B} \in M_{n \times t}(K),
\vb{C} \in M_{s \times m}(K)\),
则\begin{gather*}
	\rank\begin{bmatrix}
		\vb{A} \\
		\vb{C} \vb{A}
	\end{bmatrix}
	= \rank\vb{A}, \\
	\rank(\vb{A},\vb{A} \vb{B})
	= \rank\vb{A}.
\end{gather*}
\begin{proof}
由\cref{theorem:向量空间.用列满秩矩阵左乘任一矩阵不变秩} 立即可得.
\end{proof}
\end{corollary}
\begin{example}
设\(\vb{A} \in M_{m \times n}(K),
\vb{B} \in M_{n \times t}(K)\).
举例说明:\(\rank(\vb{A},\vb{B} \vb{A}) \neq \rank\vb{A}\).
\begin{solution}
取\begin{equation*}
%@Mathematica: A = {{1, 0}, {0, 0}}
	\vb{A} = \begin{bmatrix}
		1 & 0 \\
		0 & 0
	\end{bmatrix},
	\qquad
%@Mathematica: B = {{1, 0}, {1, 1}}
	\vb{B} = \begin{bmatrix}
		1 & 0 \\
		1 & 1
	\end{bmatrix},
\end{equation*}
那么\begin{equation*}
%@Mathematica: B.A
	\vb{B} \vb{A} = \begin{bmatrix}
		1 & 0 \\
		1 & 0
	\end{bmatrix},
	\qquad
%@Mathematica: Join[A, B.A, 2]
	(\vb{A},\vb{B} \vb{A}) = \begin{bmatrix}
		1 & 0 & 1 & 0 \\
		0 & 0 & 1 & 0
	\end{bmatrix},
\end{equation*}
%@Mathematica: MatrixRank[Join[A, B.A, 2]]
%@Mathematica: MatrixRank[A]
于是\(\rank(\vb{A},\vb{B} \vb{A}) = 2 \neq 1 = \rank\vb{A}\).
\end{solution}
\end{example}
\begin{example}\label{example:向量空间.等秩矩阵的行向量组的等价性}
设\(\vb{A} \in M_{s \times n}(K),
\vb{B} \in M_{m \times n}(K)\),
且\(\rank\vb{A}
= \rank\begin{bmatrix}
	\vb{A} \\ \vb{B}
\end{bmatrix}\),
则\(\vb{A}\)的行向量组的任意一个极大线性无关组
都是\(\begin{bmatrix}
	\vb{A} \\ \vb{B}
\end{bmatrix}\)的行向量组的极大线性无关组.
\begin{proof}
\begin{proof}[证法一]
%@credit: {5a781423-ba4e-4629-ac1a-eac743a4d445},{5f4d2f8a-fc8b-4798-85d6-98670f6761e7},{6f21d9e6-edca-4b6f-9dff-364f3d62dcce}
设\(A' = \AutoTuple{\vb\alpha}{r}\)是\(\vb{A}\)的行向量组的一个极大线性无关组.
用反证法.
假设\(\vb{B}\)的某一个行向量\(\vb\beta\)不可以由\(A'\)线性表出,
即向量组\(A' \cup \{\vb\beta\}\)线性无关,
那么\begin{equation*}
	\rank\begin{bmatrix}
		\vb{A} \\ \vb{B}
	\end{bmatrix}
	\geq \card(A' \cup \{\vb\beta\})
	= r + 1,
\end{equation*}
与题设矛盾!
因此\(\vb{B}\)的行向量组可以由\(A'\)线性表出,
\(A'\)是\(\begin{bmatrix}
	\vb{A} \\ \vb{B}
\end{bmatrix}\)的极大线性无关组.
\end{proof}
\begin{proof}[证法二]
%@credit: {6c964576-9569-472e-969e-54699e35974b}
显然\(\vb{A}\)的行向量组张成的空间\(\SpanR\vb{A}\)是\(\begin{bmatrix}
	\vb{A} \\ \vb{B}
\end{bmatrix}\)的行向量组张成的空间\(\SpanR\begin{bmatrix}
	\vb{A} \\ \vb{B}
\end{bmatrix}\)的子集.
因为\(\rank\vb{A}
= \rank\begin{bmatrix}
	\vb{A} \\ \vb{B}
\end{bmatrix}\),
所以由\cref{theorem:线性方程组.齐次线性方程组的解向量个数} 可知
\(\dim\SpanR\vb{A}
= \dim\SpanR\begin{bmatrix}
	\vb{A} \\ \vb{B}
\end{bmatrix}\).
再由\cref{theorem:向量空间.两个非零子空间的关系2} 可知
\(\SpanR\vb{A} = \SpanR\begin{bmatrix}
	\vb{A} \\ \vb{B}
\end{bmatrix}\).
\end{proof}\let\qed\relax
\end{proof}
\end{example}

\begin{example}
%@see: 《高等代数(第三版 上册)》(丘维声) P143 习题4.5 2.
设\(\vb{A}\)是数域\(K\)上的\(n\)阶非零矩阵.
证明:“存在一个\(n \times m\)非零矩阵\(\vb{B}\),使得\(\vb{A}\vb{B}=\vb0\)”的充分必要条件为
\(\abs{\vb{A}}=0\).
\begin{proof}
%@credit: {8b6edada-f2fd-4ae5-9020-eb533149a54c},{c5f76621-34f0-4114-845f-e36475300576},{5f4d2f8a-fc8b-4798-85d6-98670f6761e7},{ce603838-a24d-4616-9395-d7b223e8cb72}
必要性.
假设存在一个\(n \times m\)非零矩阵\(\vb{B}\),使得\(\vb{A}\vb{B}=\vb0\).
那么只要任取\(\vb{B}\)的一个非零列向量\(\vb\beta\),
就有\(\vb{A}\vb\beta = \vb0\),
即\(\vb\beta\)是齐次方程\(\vb{A}\vb{x}=\vb0\)的非零解,
故\(\rank\vb{A}<n\),
从而有\(\abs{\vb{A}}=0\).

充分性.
假设\(\abs{\vb{A}}=0\),
则\(\rank\vb{A} = r < n\),
在\(\vb{A}\)的核空间\(\Ker\vb{A}\)中
任取一个非零向量\(\vb\beta\),
% 从而成立\(\vb{A}\vb\beta=\vb0\),
构成一个\(n \times m\)矩阵\(\vb{B}\),
使得\(\vb{B}\)的每一列都是\(\vb\beta\),
那么\(\vb{A}\vb{B}=\vb0\).
\end{proof}
\end{example}
\begin{example}
%@see: 《高等代数(第三版 上册)》(丘维声) P143 习题4.5 3.
设\(\vb{A}\)是数域\(K\)上的\(n\)阶方阵,
\(\vb{B}\)是数域\(K\)上的\(n \times m\)行满秩矩阵,
\(\vb{E}\)是数域\(K\)上的\(n\)阶单位矩阵.
证明:\begin{itemize}
	\item 如果\(\vb{A}\vb{B}=\vb0\),则\(\vb{A}=\vb0\).
	\item 如果\(\vb{A}\vb{B}=\vb{B}\),则\(\vb{A}=\vb{E}\).
\end{itemize}
\begin{proof}
因为\(\vb{B}\)是行满秩矩阵,
由\cref{theorem:向量空间.用列满秩矩阵左乘任一矩阵不变秩} 可知,
\(\rank(\vb{A}\vb{B}) = \rank\vb{A}\).
假设\(\vb{A}\vb{B}=\vb0\),
则\begin{equation*}
	\rank\vb{A}=\rank(\vb{A}\vb{B})=\rank\vb0=0,
\end{equation*}
那么由\cref{theorem:向量空间.秩为零的矩阵必为零矩阵} 可知,
\(\vb{A}=\vb0\).

假设\(\vb{A}\vb{B}=\vb{B}\),
则\((\vb{A}-\vb{E})\vb{B}=\vb0\),
那么由上述结论可知\(\vb{A}-\vb{E}=\vb0\),
因此\(\vb{A}=\vb{E}\).
\end{proof}
\end{example}

\begin{example}
%@see: 《高等代数(第三版 上册)》(丘维声) P143 习题4.5 9.
设矩阵\(\vb{A} \in M_n(K)\ (n\geq2)\),\(\vb{A}^*\)是\(\vb{A}\)的伴随矩阵.
证明:\begin{equation}\label{equation:伴随矩阵.伴随矩阵的秩}
	\rank\vb{A}^* = \left\{ \begin{array}{cl}
		n, & \rank\vb{A}=n, \\
		1, & \rank\vb{A}=n-1, \\
		0, & \rank\vb{A}<n-1.
	\end{array} \right.
\end{equation}
\begin{proof}
根据恒等式 \labelcref{equation:行列式.伴随矩阵.恒等式1} 有\begin{equation*}
	\vb{A} \vb{A}^* = \abs{\vb{A}} \vb{E},
	\eqno(1)
\end{equation*}
于是\begin{equation*}
	\abs{\vb{A} \vb{A}^*} = \abs{\abs{\vb{A}} \vb{E}}.
	\eqno(2)
\end{equation*}
根据\cref{theorem:行列式.矩阵乘积的行列式},
(2)式左边可以分解为\begin{equation*}
	\abs{\vb{A} \vb{A}^*} = \abs{\vb{A}} \abs{\vb{A}^*}.
\end{equation*}
根据\cref{theorem:行列式.性质2.推论2},
(2)式右边可以分解为\begin{equation*}
	\abs{\abs{\vb{A}} \vb{E}} = \abs{\vb{A}}^n \abs{\vb{E}} = \abs{\vb{A}}^n.
\end{equation*}
于是我们得到\begin{equation*}
	\abs{\vb{A}} \abs{\vb{A}^*} = \abs{\vb{A}}^n.
	\eqno(3)
\end{equation*}

下面我们根据矩阵\(\vb{A}\)的秩的取值分类讨论.
\begin{enumerate}
	\item 当\(\rank\vb{A} = n\)时,
	由\cref{theorem:向量空间.满秩方阵的行列式非零} 有\(\abs{\vb{A}} \neq 0\);
	于是(3)式可化简得\begin{equation*}
		\abs{\vb{A}^*}
		= \abs{\vb{A}}^{n-1} \neq 0,
	\end{equation*}
	再次利用\cref{theorem:向量空间.满秩方阵的行列式非零} 便知\(\rank\vb{A}^* = n\).

	\item 当\(\rank\vb{A} = n-1\)时,
	因为\(\rank\vb{A}<n\),
	所以\(\abs{\vb{A}} = 0\),
	那么由(1)式可知\(\vb{A} \vb{A}^* = \vb0\),
	那么利用\hyperref[equation:线性方程组.西尔维斯特不等式]{西尔维斯特不等式}便得\begin{equation*}
		\rank\vb{A} + \rank\vb{A}^* \leq n,
	\end{equation*}
	移项得\begin{equation*}
		\rank\vb{A}^*
		\leq n - \rank\vb{A}
		= n - (n-1)
		= 1.
	\end{equation*}
	又因为\(n > 1\),
	\(\rank\vb{A} = n-1 > 0\),
	根据矩阵的秩的定义,\(\vb{A}\)有一个\(n-1\)阶子式不等于零,
	再根据伴随矩阵的定义,这个子式是\(\vb{A}^*\)的一个元素,从而\(\vb{A}^*\neq\vb0\),
	\(\rank\vb{A}^*>0\).
	因此\(\rank\vb{A}^* = 1\).

	\item 当\(\rank\vb{A} < n-1\)时,
	\(\vb{A}\)的所有\(n-1\)阶子式全为零,
	也就是说\(\vb{A}\)的任意一个元素的代数余子式为零,
	根据伴随矩阵的定义,\(\vb{A}^* = \vb0\),
	因此\(\rank\vb{A}^* = 0\).
	\qedhere
\end{enumerate}
\end{proof}
\end{example}
\begin{remark}
从\cref{equation:伴随矩阵.伴随矩阵的秩} 还推导出以下结论:
%@see: 《高等代数(第三版 上册)》(丘维声) P143 习题4.5 8.
%@see: 《高等代数学习指导书(第三版)》(姚慕生、谢启鸿) P63 例2.26
\begin{equation}\label{equation:伴随矩阵.伴随矩阵的行列式}
	\abs{\vb{A}^*}
	= \abs{\vb{A}}^{n-1}.
\end{equation}
\end{remark}

\begin{example}
%@see: 《高等代数(第三版 上册)》(丘维声) P143 习题4.5 10.
%@see: 《高等代数学习指导书(第三版)》(姚慕生、谢启鸿) P64 例2.27
设矩阵\(\vb{A} \in M_n(K)\ (n\geq2)\),\(\vb{A}^*\)是\(\vb{A}\)的伴随矩阵.
证明:\begin{equation}\label{equation:伴随矩阵.伴随矩阵的伴随}
	(\vb{A}^*)^* = \left\{ \begin{array}{cl}
		\abs{\vb{A}}^{n-2} \vb{A}, & n\geq3, \\
		\vb{A}, & n=2.
	\end{array} \right.
\end{equation}
\begin{proof}
当\(n=2\)时,
设\(\vb{A} = \begin{bmatrix}
	a_{11} & a_{12} \\
	a_{21} & a_{22}
\end{bmatrix}\),
那么\begin{equation*}
	\vb{A}^* = \begin{bmatrix}
		A_{11} & A_{21} \\
		A_{12} & A_{22}
	\end{bmatrix}
	= \begin{bmatrix}
		a_{22} & - a_{12} \\
		- a_{21} & a_{11}
	\end{bmatrix}.
\end{equation*}
从而\begin{equation*}
	(\vb{A}^*)^* = \begin{bmatrix}
		a_{11} & a_{12} \\
		a_{21} & a_{22}
	\end{bmatrix}
	= \vb{A}.
\end{equation*}

下面证明当\(n\geq3\)时,成立\((\vb{A}^*)^* = \abs{\vb{A}}^{n-2} \vb{A}\).

假设\(\vb{A}\)可逆.
那么由\cref{theorem:逆矩阵.逆矩阵的唯一性}
可知\(\vb{A}^* = \abs{\vb{A}} \vb{A}^{-1}\).
由\cref{equation:伴随矩阵.伴随矩阵的行列式}
可知\(\abs{\vb{A}^*} = \abs{\vb{A}}^{n-1}\).
由\cref{theorem:逆矩阵.伴随矩阵的逆与逆矩阵的伴随}
可知\((\vb{A}^*)^{-1} = \abs{\vb{A}}^{-1} \vb{A}\).
于是\begin{align*}
	(\vb{A}^*)^*
	&= \abs{\vb{A}^*} (\vb{A}^*)^{-1} \\
	&= \abs{\vb{A}}^{n-1} \cdot \abs{\vb{A}}^{-1} \vb{A} \\
	&= \abs{\vb{A}}^{n-2} \vb{A}
	\quad(n\geq2).
\end{align*}

假设\(\vb{A}\)不可逆,即\(\rank\vb{A}<n\),
那么由\cref{equation:伴随矩阵.伴随矩阵的秩}
可知\(\rank\vb{A}^* \leq 1 < 2 \leq n-1\),
从而\(\rank(\vb{A}^*)^* = 0\),
即\((\vb{A}^*)^* = \vb0\).
这时\(\abs{\vb{A}}=0\),
因此\((\vb{A}^*)^* = \abs{\vb{A}}^{n-2} \vb{A}\)仍然成立.
\end{proof}
\end{example}

我们还可以将\hyperref[equation:线性方程组.西尔维斯特不等式]{西尔维斯特不等式}进行如下的推广.
\begin{theorem}
设\(\vb{A} \in M_{s \times n}(K),
\vb{B} \in M_{n \times m}(K),
\vb{C} \in M_{m \times t}(K)\),
则\begin{equation}\label{equation:线性方程组.弗罗贝尼乌斯不等式}
	\rank(\vb{A}\vb{B}\vb{C}) \geq \rank(\vb{A}\vb{B}) + \rank(\vb{B}\vb{C}) - \rank\vb{B}.
\end{equation}
\begin{proof}
利用初等变换,有\begin{equation*}
	\begin{bmatrix}
		\vb{B} & \vb0 \\
		\vb0 & \vb{A}\vb{B}\vb{C}
	\end{bmatrix}
	\to \begin{bmatrix}
		\vb{B} & \vb0 \\
		\vb{A}\vb{B} & \vb{A}\vb{B}\vb{C}
	\end{bmatrix}
	\to \begin{bmatrix}
		\vb{B} & -\vb{B}\vb{C} \\
		\vb{A}\vb{B} & \vb0
	\end{bmatrix}
	\to \begin{bmatrix}
		\vb{B}\vb{C} & \vb{B} \\
		\vb0 & \vb{A}\vb{B}
	\end{bmatrix},
\end{equation*}
于是\begin{equation*}
	\rank\vb{B} + \rank(\vb{A}\vb{B}\vb{C})
	= \rank\begin{bmatrix}
		\vb{B} & \vb0 \\
		\vb0 & \vb{A}\vb{B}\vb{C}
	\end{bmatrix}
	= \rank\begin{bmatrix}
		\vb{B}\vb{C} & \vb{B} \\
		\vb0 & \vb{A}\vb{B}
	\end{bmatrix}
	\geq \rank(\vb{A}\vb{B}) + \rank(\vb{B}\vb{C}).
	\qedhere
\end{equation*}
\end{proof}
\end{theorem}

我们把\cref{equation:线性方程组.弗罗贝尼乌斯不等式}
称为\DefineConcept{弗罗贝尼乌斯不等式}(Frobenius rank inequality).

\begin{example}
%@see: 《2025年全国硕士研究生入学统一考试(数学一)》一选择题/第7题
设\(n\)阶矩阵\(\vb{A},\vb{B},\vb{C}\)满足
\(\rank\vb{A} + \rank\vb{B} + \rank\vb{C} = \rank(\vb{A}\vb{B}\vb{C}) + 2n\).
证明:\begin{gather*}
	\rank(\vb{A}\vb{B}\vb{C}) + n = \rank(\vb{A}\vb{B}) + \rank\vb{C}, \\
	\rank(\vb{A}\vb{B}) + n = \rank\vb{A} + \rank\vb{B}.
\end{gather*}
\begin{proof}
由\hyperref[equation:线性方程组.西尔维斯特不等式]{西尔维斯特不等式}
\(\rank\vb{A} + \rank\vb{B} - n \leq \rank(\vb{A}\vb{B})\)可知\begin{gather*}
	\rank(\vb{A}\vb{B}\vb{C}) + n
	\geq \rank(\vb{A}\vb{B}) + \rank\vb{C}, \\
	\rank(\vb{A}\vb{B}\vb{C}) + n
	= (\rank\vb{A} + \rank\vb{B} - n) + \rank\vb{C}
	\leq \rank(\vb{A}\vb{B}) + \rank\vb{C},
\end{gather*}
从而有\(\rank(\vb{A}\vb{B}\vb{C}) + n = \rank(\vb{A}\vb{B}) + \rank\vb{C}\),
进而有\begin{equation*}
	\rank(\vb{A}\vb{B}) + n
	= \rank(\vb{A}\vb{B}\vb{C}) + n - \rank\vb{C} + n
	= \rank\vb{A} + \rank\vb{B}.
	\qedhere
\end{equation*}
\end{proof}
\end{example}

\begin{example}\label{example:对合矩阵.对合矩阵的秩的性质1}
%@see: 《高等代数(第三版 上册)》(丘维声) P143 习题4.5 4.
设\(\vb{A}\)是数域\(K\)上的\(n\)阶对合矩阵,
即有\(\vb{A}^2=\vb{E}\),
则\begin{equation*}
	\rank(\vb{E}+\vb{A})+\rank(\vb{E}-\vb{A})=n.
\end{equation*}
\begin{proof}
因为利用初等变换可以得到\begin{align*}
	&\hspace{-20pt}
	\begin{bmatrix}
		\vb{E}+\vb{A} & \vb0 \\
		\vb0 & \vb{E}-\vb{A}
	\end{bmatrix}
	\to \begin{bmatrix}
		\vb{E}+\vb{A} & \vb0 \\
		\vb{A}(\vb{E}+\vb{A}) & \vb{E}-\vb{A}
	\end{bmatrix}
	= \begin{bmatrix}
		\vb{E}+\vb{A} & \vb0 \\
		\vb{A}+\vb{E} & \vb{E}-\vb{A}
	\end{bmatrix} \\
	&\to \begin{bmatrix}
		\vb{E}+\vb{A} & \vb{E}-\vb{A} \\
		\vb0 & \vb0
	\end{bmatrix}
	\to \begin{bmatrix}
		\vb{E}+\vb{A} & 2\vb{E} \\
		\vb0 & \vb0
	\end{bmatrix}
	\to \begin{bmatrix}
		\vb{E}+\vb{A} & \vb{E} \\
		\vb0 & \vb0
	\end{bmatrix}
	\to \begin{bmatrix}
		\vb{A} & \vb{E} \\
		\vb0 & \vb0
	\end{bmatrix},
\end{align*}
所以\begin{equation*}
	\rank(\vb{E}+\vb{A})+\rank(\vb{E}-\vb{A})
	=\rank\begin{bmatrix}
		\vb{E}+\vb{A} & \vb0 \\
		\vb0 & \vb{E}-\vb{A}
	\end{bmatrix}
	= \rank\begin{bmatrix}
		\vb{A} & \vb{E} \\
		\vb0 & \vb0
	\end{bmatrix}
	= n.
	\qedhere
\end{equation*}
\end{proof}
\end{example}

\begin{example}\label{example:幂等矩阵.幂等矩阵的秩的性质1}
%@see: 《高等代数(第三版 上册)》(丘维声) P143 习题4.5 5.
%@see: 《高等代数》(丁南庆、刘公祥、纪庆忠、郭学军) P151 习题2 57.(2)
设\(\vb{A}\)是数域\(K\)上的\(n\)阶幂等矩阵,
即有\(\vb{A}^2=\vb{A}\).
证明:\begin{equation*}
	\rank\vb{A}+\rank(\vb{E}-\vb{A})=n.
\end{equation*}
\begin{proof}
由于\begin{equation*}
	\vb{A}^2=\vb{A}
	\iff
	\vb{A}^2-\vb{A}=\vb0
	\iff
	\rank(\vb{A}^2-\vb{A})=0.
\end{equation*}
又因为利用初等变换可以得到\begin{equation*}
	\begin{bmatrix}
		\vb{A} & \vb0 \\
		\vb0 & \vb{E}_n-\vb{A}
	\end{bmatrix}
	\to \begin{bmatrix}
		\vb{A} & \vb0 \\
		\vb{A} & \vb{E}_n-\vb{A}
	\end{bmatrix}
	\to \begin{bmatrix}
		\vb{A} & \vb{A} \\
		\vb{A} & \vb{E}_n
	\end{bmatrix}
	\to \begin{bmatrix}
		\vb{A}-\vb{A}^2 & \vb0 \\
		\vb{A} & \vb{E}_n
	\end{bmatrix}
	\to \begin{bmatrix}
		\vb{A}-\vb{A}^2 & \vb0 \\
		\vb0 & \vb{E}_n
	\end{bmatrix},
\end{equation*}
所以\(\rank\vb{A}+\rank(\vb{E}_n-\vb{A})
=\rank(\vb{A}-\vb{A}^2)+n
=n\).
\end{proof}
\end{example}

\begin{example}\label{example:单位矩阵与两矩阵乘积之差.单位矩阵与两矩阵乘积之差的秩}
设\(\vb{A} \in M_{m \times n}(K),
\vb{B} \in M_{n \times m}(K)\).
证明:\begin{equation*}
	\rank(\vb{E}_m-\vb{A}\vb{B})-\rank(\vb{E}_n-\vb{B}\vb{A})=m-n.
\end{equation*}
\begin{proof}
因为利用初等变换可以得到\begin{align*}
	\begin{bmatrix}
		\vb{E}_m-\vb{A}\vb{B} & \vb0 \\
		\vb0 & \vb{E}_n
	\end{bmatrix}
	&\to \begin{bmatrix}
		\vb{E}_m-\vb{A}\vb{B} & \vb{A} \\
		\vb0 & \vb{E}_n
	\end{bmatrix}
	\to \begin{bmatrix}
		\vb{E}_m & \vb{A} \\
		\vb{B} & \vb{E}_n
	\end{bmatrix} \\
	&\to \begin{bmatrix}
		\vb{E}_m & \vb0 \\
		\vb{B} & \vb{E}_n-\vb{B}\vb{A}
	\end{bmatrix}
	\to \begin{bmatrix}
		\vb{E}_m & \vb0 \\
		\vb0 & \vb{E}_n-\vb{B}\vb{A}
	\end{bmatrix},
\end{align*}
所以\begin{equation*}
	\rank(\vb{E}_m-\vb{A}\vb{B})+n=m+\rank(\vb{E}_n-\vb{B}\vb{A}),
\end{equation*}
移项便得\(\rank(\vb{E}_m-\vb{A}\vb{B})-\rank(\vb{E}_n-\vb{B}\vb{A})=m-n\).
\end{proof}
%\cref{example:逆矩阵.行列式降阶定理的重要应用1}
%\cref{example:单位矩阵与两矩阵乘积之差.单位矩阵与两矩阵乘积之差的行列式}
\end{example}

\begin{example}
设\(\vb{A},\vb{B},\vb{C} \in M_n(K)\),
\(\rank\vb{C}=n\),
\(\vb{A}(\vb{B}\vb{A}+\vb{C})=\vb0\).
证明:\begin{equation*}
	\rank(\vb{B}\vb{A}+\vb{C})=n-\rank\vb{A}.
\end{equation*}
\begin{proof}
因为\(\vb{A}(\vb{B}\vb{A}+\vb{C})=\vb0\),
所以根据\cref{equation:线性方程组.西尔维斯特不等式} 有\begin{equation*}
	\rank(\vb{B}\vb{A}+\vb{C})+\rank\vb{A} \leq n.
\end{equation*}
又利用初等变换可以得到\begin{equation*}
	\begin{bmatrix}
		\vb{B}\vb{A}+\vb{C} & \vb0 \\
		\vb0 & \vb{A}
	\end{bmatrix}
	\to \begin{bmatrix}
		\vb{B}\vb{A}+\vb{C} & \vb{B}\vb{A} \\
		\vb0 & \vb{A}
	\end{bmatrix}
	\to \begin{bmatrix}
		\vb{C} & \vb{B}\vb{A} \\
		-\vb{A} & \vb{A}
	\end{bmatrix},
\end{equation*}
于是\begin{equation*}
	\rank(\vb{B}\vb{C}+\vb{C})
	+\rank\vb{A}
	= \rank\begin{bmatrix}
		\vb{B}\vb{A}+\vb{C} & \vb0 \\
		\vb0 & \vb{A}
	\end{bmatrix}
	= \rank\begin{bmatrix}
		\vb{C} & \vb{B}\vb{A} \\
		-\vb{A} & \vb{A}
	\end{bmatrix}
	\geq \rank\vb{C} = n.
	\qedhere
\end{equation*}
\end{proof}
\end{example}

\begin{example}
%@see: 《2021年全国硕士研究生入学统一考试(数学一)》一选择题/第7题
设\(\vb{A},\vb{B} \in M_n(\mathbb{R})\),
证明:\begin{equation*}
	\rank\begin{bmatrix}
		\vb{A} & \vb0 \\
		\vb0 & \vb{A}^T \vb{A}
	\end{bmatrix}
	= \rank\begin{bmatrix}
		\vb{A} & \vb{A} \vb{B} \\
		\vb0 & \vb{A}^T
	\end{bmatrix}
	= \rank\begin{bmatrix}
		\vb{A} & \vb0 \\
		\vb{B} \vb{A} & \vb{A}^T
	\end{bmatrix}
	= 2 \rank\vb{A},
\end{equation*}
并举例说明\(\rank\begin{bmatrix}
	\vb{A} & \vb{B} \vb{A} \\
	\vb0 & \vb{A} \vb{A}^T
\end{bmatrix}
\neq 2 \rank\vb{A}\).
\begin{proof}
因为\begin{gather*}
	\rank\begin{bmatrix}
		\vb{A} & \vb0 \\
		\vb0 & \vb{A}^T \vb{A}
	\end{bmatrix}
	= \rank\vb{A} + \rank(\vb{A}^T \vb{A}), \\
	\begin{bmatrix}
		\vb{A} & \vb{A} \vb{B} \\
		\vb0 & \vb{A}^T
	\end{bmatrix}
	= \begin{bmatrix}
		\vb{A} & \vb0 \\
		\vb0 & \vb{A}^T
	\end{bmatrix}
	\begin{bmatrix}
		\vb{E} & \vb{B} \\
		\vb0 & \vb{E}
	\end{bmatrix}
	\implies
	\rank\begin{bmatrix}
		\vb{A} & \vb{A} \vb{B} \\
		\vb0 & \vb{A}^T
	\end{bmatrix}
	= \rank\begin{bmatrix}
		\vb{A} & \vb0 \\
		\vb0 & \vb{A}^T
	\end{bmatrix}
	= \rank\vb{A} + \rank\vb{A}^T, \\
	\begin{bmatrix}
		\vb{A} & \vb0 \\
		\vb{B} \vb{A} & \vb{A}^T
	\end{bmatrix}
	= \begin{bmatrix}
		\vb{E} & \vb0 \\
		\vb{B} & \vb{E}
	\end{bmatrix}
	\begin{bmatrix}
		\vb{A} & \vb0 \\
		\vb0 & \vb{A}^T
	\end{bmatrix}
	\implies
	\rank\begin{bmatrix}
		\vb{A} & \vb0 \\
		\vb{B} \vb{A} & \vb{A}^T
	\end{bmatrix}
	= \rank\begin{bmatrix}
		\vb{A} & \vb0 \\
		\vb0 & \vb{A}^T
	\end{bmatrix}
	= \rank\vb{A} + \rank\vb{A}^T,
\end{gather*}
而由\cref{theorem:向量空间.转置不变秩} 可知\(\rank\vb{A} = \rank\vb{A}^T\),
且由\cref{equation:矩阵乘积的秩.实矩阵及其转置矩阵的乘积的秩} 可知\(\rank(\vb{A}^T \vb{A}) = \rank\vb{A}\),
所以\begin{equation*}
	\rank\begin{bmatrix}
		\vb{A} & \vb0 \\
		\vb0 & \vb{A}^T \vb{A}
	\end{bmatrix}
	= \rank\begin{bmatrix}
		\vb{A} & \vb{A} \vb{B} \\
		\vb0 & \vb{A}^T
	\end{bmatrix}
	= \rank\begin{bmatrix}
		\vb{A} & \vb0 \\
		\vb{B} \vb{A} & \vb{A}^T
	\end{bmatrix}
	= 2 \rank\vb{A}.
\end{equation*}

取\(\vb{A} = \begin{bmatrix}
	0 & 1 \\
	0 & 0
\end{bmatrix},
\vb{B} = \begin{bmatrix}
	0 & 1 \\
	1 & 0
\end{bmatrix}\),
则\(\rank\vb{A} = 1\),
而\begin{equation*}
	\rank\begin{bmatrix}
		\vb{A} & \vb{B} \vb{A} \\
		\vb0 & \vb{A} \vb{A}^T
	\end{bmatrix}
	= \rank\begin{bmatrix}
		0 & 1 & 0 & 0 \\
		0 & 0 & 0 & 1 \\
		0 & 0 & 1 & 0 \\
		0 & 0 & 0 & 0
	\end{bmatrix}
	= 3 \neq 2,
\end{equation*}
所以\(\rank\begin{bmatrix}
	\vb{A} & \vb{B} \vb{A} \\
	\vb0 & \vb{A} \vb{A}^T
\end{bmatrix}
\neq 2 \rank\vb{A}\).
\end{proof}
\end{example}
\begin{example}
%@see: 《2023年全国硕士研究生入学统一考试(数学一)》一选择题/第5题
设矩阵\(\vb{A},\vb{B},\vb{C} \in M_n(K)\)满足\(\vb{A} \vb{B} \vb{C} = \vb0\),
\(\vb{E}\)是数域\(K\)上的\(n\)阶单位矩阵.
试比较以下三个矩阵的秩的序关系:\begin{equation*}
	\vb{M}_1 \defeq \begin{bmatrix}
		\vb0 & \vb{A} \\
		\vb{B} \vb{C} & \vb{E}
	\end{bmatrix},
	\qquad
	\vb{M}_2 \defeq \begin{bmatrix}
		\vb{A} \vb{B} & \vb{C} \\
		\vb0 & \vb{E}
	\end{bmatrix},
	\qquad
	\vb{M}_3 \defeq \begin{bmatrix}
		\vb{E} & \vb{A} \vb{B} \\
		\vb{A} \vb{B} & \vb0
	\end{bmatrix}.
\end{equation*}
\begin{solution}
因为\begin{align*}
	\begin{bmatrix}
		\vb{E} & -\vb{A} \\
		\vb0 & \vb{E}
	\end{bmatrix}
	\begin{bmatrix}
		\vb0 & \vb{A} \\
		\vb{B} \vb{C} & \vb{E}
	\end{bmatrix}
	&= \begin{bmatrix}
		% 第一行第一列的元素等于\(\vb0 - \vb{A} \vb{B} \vb{C} = \vb0\)
		% 第一行第二列的元素等于\(\vb{A} - \vb{A} = \vb0\)
		\vb0 & \vb0 \\
		\vb{B} \vb{C} & \vb{E}
	\end{bmatrix}, \\
	\begin{bmatrix}
		\vb{E} & -\vb{C} \\
		\vb0 & \vb{E}
	\end{bmatrix}
	\begin{bmatrix}
		\vb{A} \vb{B} & \vb{C} \\
		\vb0 & \vb{E}
	\end{bmatrix}
	&= \begin{bmatrix}
		\vb{A} \vb{B} & \vb0 \\
		\vb0 & \vb{E}
	\end{bmatrix}, \\
	\begin{bmatrix}
		\vb{E} & \vb0 \\
		-\vb{A} \vb{B} & \vb{E}
	\end{bmatrix}
	\begin{bmatrix}
		\vb{E} & \vb{A} \vb{B} \\
		\vb{A} \vb{B} & \vb0
	\end{bmatrix}
	\begin{bmatrix}
		\vb{E} & -\vb{A} \vb{B} \\
		\vb0 & \vb{E}
	\end{bmatrix}
	&= \begin{bmatrix}
		\vb{E} & \vb0 \\
		\vb0 & -\vb{A} \vb{B} \vb{A} \vb{B}
	\end{bmatrix},
\end{align*}
而\begin{align*}
	\rank\vb{M}_1
	&= \rank\begin{bmatrix}
		\vb0 & \vb0 \\
		\vb{B} \vb{C} & \vb{E}
	\end{bmatrix}
	= n, \\
	\rank\vb{M}_2
	&= \rank\begin{bmatrix}
		\vb{A} \vb{B} & \vb0 \\
		\vb0 & \vb{E}
	\end{bmatrix}
	= \rank(\vb{A}\vb{B}) + n
	%\cref{theorem:线性方程组.矩阵的秩的性质2}
	\geq n, \\
	\rank\vb{M}_3
	&= \rank\begin{bmatrix}
		\vb{E} & \vb0 \\
		\vb0 & -\vb{A} \vb{B} \vb{A} \vb{B}
	\end{bmatrix}
	= \rank(\vb{A} \vb{B} \vb{A} \vb{B}) + n
	%\cref{theorem:线性方程组.矩阵的秩的性质2}
	\geq n,
\end{align*}
%\cref{theorem:线性方程组.矩阵乘积的秩}
这里\(\rank(\vb{A} \vb{B}) \geq \rank(\vb{A} \vb{B} \vb{A} \vb{B})\),
所以\(\rank\vb{M}_2 \geq \rank\vb{M}_3 \geq \rank\vb{M}_1\).
\end{solution}
\end{example}
