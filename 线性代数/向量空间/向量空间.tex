\section{向量空间}
为了直接用线性方程组的系数和常数项判断方程组有没有解,有多少解,
我们在前面给出了用系数行列式判断\(n\)个方程的\(n\)元线性方程组有唯一解的充分必要条件.
这一判定方法只适用于方程数目与未知量数目相等的线性方程组;
而且,当系数行列式等于零时,只能得出方程组无解或有无穷多解的结论,
没有办法区分什么时候无解,什么时候有无穷多解.
对于任意的线性方程组,有没有这样一种判定方法:
直接依据它的系数和常数项,给出它有没有解,有多少解呢?
为此我们需要探讨和建立线性方程组的进一步的理论.
这一理论还将使我们弄清楚线性方程组有无穷多个解时解的结构.

\subsection{向量空间}
设\(K\)是数域,\(n\)是任意给定的一个正整数.
令\[
	K^n \defeq \Set{ (\AutoTuple{a}{n}) \given a_i \in K\ (i=1,2,\dotsc,n) }.
\]

如果\[
	a_i=b_i
	\quad(i=1,2,\dotsc,n),
\]
则称“\(K^n\)中的两个元素\((\AutoTuple{a}{n})\)与\((\AutoTuple{b}{n})\)相等”.

在\(K^n\)中规定“加法”运算如下:
\begin{equation}\label{equation:向量空间.向量的加法.定义式}
	(\AutoTuple{a}{n}) + (\AutoTuple{b}{n})
	\defeq (a_1+b_1,a_2+b_2,\dotsc,a_n+b_n).
\end{equation}

在\(K\)的元素与\(K^n\)的元素之间规定“数量乘法”运算如下:
\begin{equation}\label{equation:向量空间.向量的数量乘法.定义式}
	k (\AutoTuple{a}{n})
	\defeq (k a_1,k a_2,\dotsc,k a_n).
\end{equation}

容易验证,上述加法和数量乘法满足下述8条运算法则:
\begin{enumerate}
	\item 加法交换律,即\((\forall \a,\b \in K^n)[\a+\b=\b+\a]\).

	\item 加法结合律,即\((\forall \a,\b,\g \in K^n)[(\a+\b)+\g=\a+(\b+\g)]\).

	\item 记\(\z=(0,0,\dotsc,0)\),\[
		(\forall\a \in K^n)[\a+\z = \z+\a = \a].
	\]
	称\(\z\)为“\(K^n\)的\DefineConcept{零元}(zero element)”.

	\item \(\forall\a=(\AutoTuple{a}{n}) \in K^n\),令\[
		-\a \defeq (\AutoTuple{-a}{n}),
	\]
	则\(-\a \in K^n\)且\[
		\a+-\a
		= -\a+\a
		= \z;
	\]
	称\(-\a\)为“\(\a\)的\DefineConcept{负元}(negative element)”.

	\item \((\forall \a \in K^n)[1 \a=\a]\).

	\item \((\forall \a \in K^n)(\forall k,l \in K)[k (l \a)=(kl) \a]\).

	\item \((\forall \a \in K^n)(\forall k,l \in K)[(k+l) \a=k \a+l \a]\).

	\item \((\forall \a,\b \in K^n)(\forall k \in K)[k (\a+\b)=k \a+k \b]\).
\end{enumerate}

\begin{definition}
数域\(K\)上全体\(n\)元组组成的集合\(K^n\),
连同定义在它上面的加法运算和数量乘法运算,
及其满足的8条运算法则一起,
称为“数域\(K\)上的一个\(n\)维\DefineConcept{向量空间}(vector space)”.
\(K^n\)的元素称为“\(n\)维\DefineConcept{向量}(vector)”.

对于\(K^n\)中的任意一个向量\(\a=(\AutoTuple{a}{n})\),
称数\[
	a_i\quad(i=1,2,\dotsc,n)
\]为“\(\a\)的第\(i\)个\DefineConcept{分量}”.
\end{definition}

在\(n\)维向量空间\(K^n\)中,我们可以额外定义减法运算如下:
\begin{equation}\label{equation:向量空间.向量的减法.定义式}
	\a-\b \defeq \a+(-\b).
\end{equation}

在\(n\)维向量空间\(K^n\)中,容易验证下述4条性质:
\begin{property}
\((\forall\a \in K^n)[0\cdot\a=\z]\).
\end{property}

\begin{property}
\((\forall\a \in K^n)[(-1)\cdot\a=-\a]\).
\end{property}

\begin{property}
\((\forall k \in K)[k\z=\z]\).
\end{property}

\begin{property}
\(k\a=\z \implies k=0 \lor \a=\z\).
\end{property}

把\(n\)元组写成一行,得\[
	(\AutoTuple{a}{n})
	\quad\text{或}\quad
	\begin{bmatrix}
		a_1 & a_2 & \dots & a_n
	\end{bmatrix},
\]
称之为“\(n\)维\DefineConcept{行向量}(row vector)”.

把\(n\)元组写成一列,得\[
	\begin{bmatrix} a_1 \\ a_2 \\ \vdots \\ a_n \end{bmatrix},
\]
称之为“\(n\)维\DefineConcept{列向量}(column vector)”;
不过,我们有时候会为了方便排版,把列向量写成\[
	(a_1,a_2,\dotsc,a_n)^T.
\]

\(K^n\)可以看成是
全体\(n\)维行向量
组成的向量空间,
也可以看成是
全体\(n\)维列向量
组成的向量空间.
两者并没有本质的区别,
只是它们的元素的写法不同而已.

由有限个\(n\)维行向量构成的集合,
称为“\(n\)维\DefineConcept{行向量组}”.
由有限个\(n\)维列向量构成的集合,
称为“\(n\)维\DefineConcept{列向量组}”.
\(n\)维行向量组和\(n\)维列向量组
统称\(n\)维\DefineConcept{向量组}.
从本质上看,向量组就是\(n\)维向量空间\(K^n\)的有限子集.

称满足
\[
	e_{ij} = \left\{ \begin{array}{ll}
		1, & i=j, \\
		0, & i \neq j
	\end{array} \right.
\]
的向量组
\[
	\e_i = \begin{bmatrix}
		e_{1i} \\ e_{2i} \\ \vdots \\ e_{ni}
	\end{bmatrix}
	\quad(i=1,2,\dotsc,n)
\]为“\(K^n\)的\DefineConcept{基本向量组}”.

\subsection{线性组合,线性表出}
%@see: 《线性代数》(张慎语、周厚隆) P67 定义5
\begin{definition}\label{definition:向量空间.线性组合}
在\(K^n\)中,给定向量组\(A=\{\AutoTuple{\a}{s}\}\).
任给\(K\)中一组数\(\AutoTuple{k}{s}\),
我们把\[
	k_1 \a_1 + k_2 \a_2 + \dotsb + k_s \a_s
\]
称为“向量组\(A\)的一个\DefineConcept{线性组合}(linear combination)”,
把\(\AutoTuple{k}{s}\)称为\DefineConcept{系数}.
\end{definition}

\begin{definition}\label{definition:向量空间.线性表出1}
在\(K^n\)中,给定向量组\(A=\{\AutoTuple{\a}{s}\}\).
对于向量\(\b \in K^n\),
如果存在\(K\)中一组数\(\AutoTuple{c}{s}\),
使得\[
	\b = c_1 \a_1 + c_2 \a_2 + \dotsb + c_s \a_s,
\]
则称“向量\(\b\)可由向量组\(A\)~\DefineConcept{线性表出}”;
否则称“向量\(\b\)不可由向量组\(A\)线性表出”.
\end{definition}

现在,利用向量的加法运算和数量乘法运算,
我们可以把数域\(K\)上\(n\)元线性方程组 \labelcref{equation:线性方程组.线性方程组的代数形式}
写成
\begin{equation}\label{equation:线性方程组.线性方程组的向量形式}
	x_1 \a_1 + x_2 \a_2 + \dotsb + x_n \a_n = \b,
\end{equation}
其中\[
	\a_j=(a_{1j},a_{2j},\dotsc,a_{sj})^T,
	\quad
	j=1,2,\dotsc,n.
\]
于是,\begin{align*}
	&\text{数域\(K\)上线性方程组\(x_1 \a_1 + x_2 \a_2 + \dotsb + x_n \a_n = \b\)有解} \\
	&\iff \text{\(K\)中存在一组数\(\AutoTuple{c}{n}\),使得\(c_1 \a_1 + c_2 \a_2 + \dotsb + c_n \a_n = \b\)成立} \\
	&\iff \text{\(\b\)可以由\(\AutoTuple{\a}{n}\)线性表出}.
\end{align*}
这样我们把线性方程组有没有解的问题归结为:
常数项列向量\(\b\)能不能由系数矩阵的列向量组线性表出.
这个结论有两方面的意义:
一方面,为了从理论上研究线性方程组有没有解,
就需要去研究\(\b\)能否由\(\AutoTuple{\a}{n}\)线性表出;
另一方面,对于\(K^n\)中给定的向量组\(\AutoTuple{\a}{n}\),
以及给定的\(\b\),
为了判断\(\b\)能否由\(\AutoTuple{\a}{n}\)线性表出,
就可以去判断线性方程组\(x_1 \a_1 + x_2 \a_2 + \dotsb + x_n \a_n = \b\)是否有解.

\subsection{线性子空间}
在\(K^n\)中,从理论上如何判断任一向量\(\b\)能否由向量组\(\AutoTuple{\a}{n}\)线性表出?
从线性表出的定义知道,这需要考察\(\b\)是否等于\(\AutoTuple{\a}{n}\)的某一个线性组合.
为此,我们把\(\AutoTuple{\a}{n}\)的所有线性组合组成一个集合\(W\),即\[
	W \defeq \Set{ k_1 \a_1 + k_2 \a_2 + \dotsb + k_s \a_s \given k_i \in K, i=1,2,\dotsc,s }.
\]
如果我们能够把\(W\)的结构研究清楚,那么就比较容易判断\(\b\)是否属于\(W\),
也就是判断\(\b\)能否由\(\AutoTuple{\a}{n}\)线性表出.

现在我们来研究\(W\)的结构.
任取\(\a,\g\in W\),设\[
	\a=a_1\a_1+a_2\a_2+\dotsb+a_s\a_s, \qquad
	\g=b_1\a_1+b_2\a_2+\dotsb+b_s\a_s,
\]
则\begin{align*}
	\a+\g
	&=(a_1\a_1+a_2\a_2+\dotsb+a_s\a_s)+(b_1\a_1+b_2\a_2+\dotsb+b_s\a_s) \\
	&=(a_1+b_1)\a_1+(a_2+b_2)\a_2+\dotsb+(a_s+b_s)\a_s,
\end{align*}
从而\(\a+\g\in W\).

再任取\(k\in W\),则\begin{align*}
	k\a
	&=k(a_1\a_1+a_2\a_2+\dotsb+a_s\a_s) \\
	&=(ka_1)\a_1+(ka_2)\a_2+\dotsb+(ka_s)\a_s,
\end{align*}
从而\(k\a\in W\).

受此启发,我们引出如下概念.
\begin{definition}
\(K^n\)的一个非空子集\(U\)如果满足:
\begin{enumerate}
	\item \(U\)对\(K^n\)的加法封闭,即\[
		(\forall \a,\b \in U)[\a+\b \in U];
	\]
	\item \(U\)对\(K^n\)的数量乘法封闭,即\[
		(\a \in U)(k \in K)[k\a \in U];
	\]
\end{enumerate}
那么称\(U\)是“\(K^n\)的一个\DefineConcept{线性子空间}(linear subspace)”,
简称为\DefineConcept{子空间}(subspace).
\end{definition}
零空间\(\{\vb0\}\)是\(K^n\)的一个子空间,
因此我们又称之为“\(K^n\)的\DefineConcept{零子空间}(zero subspace)”.

类\(\Set{ x \given x\neq\{\vb0\} \land \text{\(x\)是\(K^n\)的子空间} }\)%
中的每一个\(x\)都称为“\(K^n\)的\DefineConcept{非零子空间}”.

\(K^n\)也是其自身的一个子空间.

\begin{proposition}
任意一个线性子空间总含有零向量.
\begin{proof}
假设存在一个线性子空间\(U\),不含有零向量.
由于\(U\)非空,不妨设非零向量\(\a\)是\(U\)的元素,即\(\a \in U\).
那么根据线性子空间的定义,有\[
	(-1)\cdot\a = -\a \in U.
\]
从而\[
	\vb0 = \a+(-\a) \in U.
\]
矛盾!
因此,\((\forall U)[\text{\(U\)是线性子空间} \implies \vb0 \in U]\).
\end{proof}
\end{proposition}

从上面的讨论知道,在\(K^n\)中,
向量组\(A=\{\AutoTuple{\a}{s}\}\)的所有线性组合组成的集合\(W\)是\(K^n\)的一个子空间,
称它为“\(\AutoTuple{\a}{s}\)生成的子空间”,
或“\(\AutoTuple{\a}{s}\)的\DefineConcept{线性生成空间}(linear span)”,
记作\(\opair{\AutoTuple{\a}{s}}\)或\(\Span A\),即\[
	\Span A
	\defeq
	\Set*{
		\sum_{i=1}^s k_i \a_i
		\given
		\AutoTuple{k}{s} \in K
	}.
\]
%@see: https://mathworld.wolfram.com/VectorSpaceSpan.html
%@see: https://math.stackexchange.com/questions/185255/span-of-an-empty-set-is-the-zero-vector/

%@see: 《代数学(一)》(李方、邓少强、冯荣权、刘东文) P102
易见\(\Span\emptyset = \{\vb0\}\).

于是,我们得出结论,以下三个命题等价:
\begin{enumerate}
	\item 数域\(K\)上的\(n\)元线性方程组\(x_1 \a_1 + x_2 \a_2 + \dotsb + x_n \a_n = \b\)有解.
	\item 向量\(\b\)可以由向量组\(A=\{\AutoTuple{\a}{n}\}\)线性表出.
	\item 向量\(\b\in\Span A=\opair{\AutoTuple{\a}{n}}\).
\end{enumerate}

\begin{theorem}\label{theorem:向量空间.任一向量可由基本向量组唯一线性表出}
\(K^n\)中任一向量都可由基本向量组唯一地线性表出.
\begin{proof}
对于任意一个向量\(\a=(\AutoTuple{a}{n})^T\),
线性方程组\(x_1 \e_1 + x_2 \e_2 + \dotsb + x_n \e_n = \a\)的系数行列式为
\[
\begin{vmatrix}
	1 & 0 & \dots & 0 \\
	0 & 1 & \dots & 0 \\
	\vdots & \vdots & & \vdots \\
	0 & 0 & \dots & 1
\end{vmatrix}
= 1 \neq 0,
\]
那么,根据\hyperref[theorem:线性方程组.克拉默法则]{克拉默法则},
上述线性方程组有唯一解.
由此可知,\(K^n\)中任一向量\(\a\)都能由基本向量组线性表出,且表出方式唯一.
事实上,由于\[
	a_1 \begin{bmatrix}
		1 \\ 0 \\ 0 \\ \vdots \\ 0
	\end{bmatrix}
	+ a_2 \begin{bmatrix}
		0 \\ 1 \\ 0 \\ \vdots \\ 0
	\end{bmatrix}
	+ \dotsb + a_n \begin{bmatrix}
		0 \\ 0 \\ 0 \\ \vdots \\ 1
	\end{bmatrix}
	= \begin{bmatrix}
		a_1 \\ a_2 \\ a_3 \\ \vdots \\ a_n
	\end{bmatrix},
\]
因此,用基本向量组标出向量\(\a\)的方式为\[
	\a = a_1 \e_1 + a_2 \e_2 + \dotsb + a_n \e_n.
	\qedhere
\]
\end{proof}
\end{theorem}
