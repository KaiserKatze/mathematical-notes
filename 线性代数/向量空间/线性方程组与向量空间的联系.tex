\section{线性方程组与向量空间的联系}
我们可以利用向量、矩阵表记线性方程组.

\begin{definition}
将\cref{equation:线性方程组.线性方程组的代数形式} 的系数按原位置构成的\(s \times n\)矩阵\[
	\A = \begin{bmatrix}
		a_{11} & a_{12} & \dots & a_{1n} \\
		a_{21} & a_{22} & \dots & a_{2n} \\
		\vdots & \vdots & & \vdots \\
		a_{n1} & a_{n2} & \dots & a_{nn}
	\end{bmatrix}
\]叫做\DefineConcept{系数矩阵}(coefficient matrix).

特别地,如果\(s = n\)(即系数矩阵\(\A\)是一个方阵),
则系数矩阵的行列式\(\abs{\A}\)叫做\DefineConcept{系数行列式}.
\end{definition}

为使表述简明,常用向量、矩阵表示线性方程组.
若记\[
	\x=\begin{bmatrix}
		x_1 \\ x_2 \\ \vdots \\ x_n
	\end{bmatrix},
	\quad
	\b=\begin{bmatrix}
		b_1 \\ b_2 \\ \vdots \\ b_n
	\end{bmatrix},
	\quad
	\a_j=\begin{bmatrix}
		a_{1j} \\ a_{2j} \\ \vdots \\ a_{sj}
	\end{bmatrix},
	\quad
	j=1,2,\dotsc,n.
\]
\(\A\)的列分块阵为\(\A = (\a_1,\a_2,\dotsc,\a_n)\),
则\cref{equation:线性方程组.线性方程组的代数形式} 有以下两种等价表示:
\begin{enumerate}
	\item 矩阵形式\[
		\A \x = \b.
	\]
	\item 向量形式\[
		x_1 \a_1 + x_2 \a_2 + \dotsb + x_n \a_n = \b.
	\]
\end{enumerate}






% \begin{example}
% 设\(\A\in M_{s \times n}(\mathbb{R})\).
% 证明:齐次线性方程组\(\A\x=\z\)与\((\A^T\A)\x=\z\)同解.
% \begin{proof}
% \def\a{\vb{\xi}}
% \def\b{\vb{\eta}}
% 设\(\a\)是\(\A\x=\z\)的任意一个解,
% 则\(\A\a=\z\),于是\[
% 	(\A^T\A)\a=\A^T(\A\a)=\A^T\z=\z,
% \]
% 这就是说\(\a\)是\((\A^T\A)\x=\z\)的一个解.

% 又设\(\b\)是\((\A^T\A)\x=\z\)的任意一个解,
% 则\[
% 	(\A^T\A)\b=\z.
% 	\eqno(1)
% \]
% 在(1)式等号两边同时左乘\(\b^T\)得\[
% 	\b^T(\A^T\A)\b=(\A\b)^T(\A\b)=0.
% 	\eqno(2)
% \]

% 假设\(\A\b=(\AutoTuple{c}{s})^T\).
% 由(2)式有\[
% 	(\AutoTuple{c}{s}) (\AutoTuple{c}{s})^T
% 	= \AutoTuple{c}{n}[+][2]
% 	= 0.
% \]
% 由于\(\AutoTuple{c}{n}\in\mathbb{R}\),
% 所以\(\AutoTuple{c}{n}[=]=0\),
% \(\A\b=\z\),
% 这就是说\(\b\)是\(\A\x=\z\)的一个解.

% 综上所述,\((\A^T\A)\x=\z\)与\(\A\x=\z\)同解.
% \end{proof}
% \end{example}
