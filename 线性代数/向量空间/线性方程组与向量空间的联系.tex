\section{线性方程组与向量空间的联系}
我们可以利用向量、矩阵表记线性方程组.

\begin{definition}
将\cref{equation:线性方程组.线性方程组的代数形式} 的系数按原位置构成的\(s \times n\)矩阵\[
	\vb{A} = \begin{bmatrix}
		a_{11} & a_{12} & \dots & a_{1n} \\
		a_{21} & a_{22} & \dots & a_{2n} \\
		\vdots & \vdots & & \vdots \\
		a_{n1} & a_{n2} & \dots & a_{nn}
	\end{bmatrix}
\]叫做\DefineConcept{系数矩阵}(coefficient matrix).

特别地,如果\(s = n\)(即系数矩阵\(\vb{A}\)是一个方阵),
则系数矩阵的行列式\(\abs{\vb{A}}\)叫做\DefineConcept{系数行列式}.
\end{definition}

为使表述简明,常用向量、矩阵表示线性方程组.
若记\[
	\vb{x}=\begin{bmatrix}
		x_1 \\ x_2 \\ \vdots \\ x_n
	\end{bmatrix},
	\quad
	\vb\beta=\begin{bmatrix}
		b_1 \\ b_2 \\ \vdots \\ b_n
	\end{bmatrix},
	\quad
	\vb\alpha_j=\begin{bmatrix}
		a_{1j} \\ a_{2j} \\ \vdots \\ a_{sj}
	\end{bmatrix},
	\quad
	j=1,2,\dotsc,n.
\]
\(\vb{A}\)的列分块阵为\(\vb{A} = (\vb\alpha_1,\vb\alpha_2,\dotsc,\vb\alpha_n)\),
则\cref{equation:线性方程组.线性方程组的代数形式} 有以下两种等价表示:
\begin{enumerate}
	\item {\rm\bf 矩阵形式}:
	\begin{equation}
		\vb{A} \vb{x} = \vb\beta.
	\end{equation}
	\item {\rm\bf 向量形式}:
	\begin{equation}
		x_1 \vb\alpha_1 + x_2 \vb\alpha_2 + \dotsb + x_n \vb\alpha_n = \vb\beta.
	\end{equation}
\end{enumerate}
