\section{高斯--塞德尔迭代法}
将线性方程组\(A X = B\)的系数矩阵\(A\)分成三部分:\begin{equation*}
%@see: 《数值分析(第5版)》(李庆扬、王能超、易大义) P188 (2.1)
	A = D - L - U,
\end{equation*}
其中\begin{align*}
	D
	&\defeq
	\begin{bmatrix}
		a_{11} \\
		& a_{22} \\
		&& \ddots \\
		&&& a_{nn} \\
	\end{bmatrix}, \\
	L
	&\defeq
	- \begin{bmatrix}
		0 \\
		a_{21} & 0 \\
		\vdots & \vdots & \ddots \\
		a_{n-1,1} & a_{n-1,2} & \dots & 0 \\
		a_{n1} & a_{n2} & \dots & a_{n,n-1} & 0 \\
	\end{bmatrix}, \\
	U
	&\defeq
	- \begin{bmatrix}
		0 & a_{12} & \dots & a_{1,n-1} & a_{1n} \\
		& 0 & \dots & a_{2,n-1} & a_{2n} \\
		&& \ddots & \vdots & \vdots \\
		&&& 0 & a_{n-1,n} \\
		&&&& 0 \\
	\end{bmatrix}.
\end{align*}

假设\(D \neq 0\)且\(D - L = A + U\)是可逆的.
因为\begin{align*}
	A X = B
	&\iff
	(A + U) X = B + U X \\
	&\iff
	(D - L) X = U X + B \\
	&\iff
	X = (D - L)^{-1} U X + (D - L)^{-1} B,
\end{align*}
所以只要令\(
	G \defeq (D - L)^{-1} U,
	C \defeq (D - L)^{-1} B
\),
就可以写出递推公式\begin{equation}
%@see: 《数值分析(第5版)》(李庆扬、王能超、易大义) P188 (2.4)
	X_{k+1}
	\defeq
	G X_k + C
	\quad(k=0,1,2,\dotsc),
\end{equation}
或\begin{equation}
%@see: 《数值分析(第5版)》(李庆扬、王能超、易大义) P189 (2.5)
	x_{k+1,i}
	= \frac{1}{a_{ii}}
	\left(
		b_i
		- \sum_{j=1}^{i-1} a_{ij} x_{k+1,j}
		- \sum_{j=i+1}^{n} a_{ij} x_{k,j}
	\right)
	\quad(i=1,2,\dotsc,n;k=0,1,2,\dotsc),
\end{equation}
或\begin{equation}
%@see: 《数值分析(第5版)》(李庆扬、王能超、易大义) P189 (2.6)
	\begin{cases}
		x_{k+1,i}
		= x_{k,i} + \increment x_i, \\
		\increment x_i
		= \frac{1}{a_{ii}}
		\left(
			b_i
			- \sum_{j=1}^{i-1} a_{ij} x_{k+1,j}
			- \sum_{j=i}^{n} a_{ij} x_{k,j}
		\right)
	\end{cases}
	\quad(i=1,2,\dotsc,n;k=0,1,2,\dotsc),
\end{equation}
其中\(X_k \defeq (x_{k,1},\dotsc,x_{k,n})\).

雅克比迭代法不使用变量的最新信息计算\(x_{k+1,i}\),
而高斯--塞德尔迭代法在计算\(x_{k+1,i}\)时,
利用了已经计算出的最新分量\(x_{k+1,j}\ (j=1,2,\dotsc,i-1)\).
高斯--塞德尔迭代法可以看作是雅克比迭代法的一种改进.
容易看出,该方法每迭代一次只需要计算一次矩阵和向量的乘法.
