\section{迭代法的基本概念}
\subsection{一阶定常迭代法}
设\(A \in M_n(F)\)是非奇异矩阵.
当\(A\)是低阶的、稠密的矩阵时,列主元消去法是求解线性方程组\(A x = b\)的有效方法.
但是,当\(A\)的阶数\(n\)很大时,或者当\(A\)是稀疏矩阵时,
就应该使用迭代法求解线性方程组\(A x = b\)了.

设\begin{equation*}
	A \defeq \begin{bmatrix}
		a_{11} & a_{12} & \dots & a_{1n} \\
		a_{21} & a_{22} & \dots & a_{2n} \\
		\vdots & \vdots & & \vdots \\
		a_{n1} & a_{n2} & \dots & a_{nn} \\
	\end{bmatrix},
	\qquad
	X \defeq \begin{bmatrix}
		x_1 \\ x_2 \\ \vdots \\ x_n
	\end{bmatrix},
	\qquad
	B \defeq \begin{bmatrix}
		b_1 \\ b_2 \\ \vdots \\ b_n
	\end{bmatrix}.
\end{equation*}
在求解线性方程组\(A X = B\)
或者\(
	\sum_{j=1}^n a_{ij} x_j = b_i
	\ (i=1,2,\dotsc,n)
\)时,可以将其变形,得到\begin{equation}
	x_i
	= \frac{1}{a_{ii}}
	\left(
		b_i
		- \sum_{j=1}^n a_{ij} x_j
		+ a_{ii} x_i
	\right)
	\quad(i=1,2,\dotsc,n)
\end{equation}
或者\begin{equation}\label{equation:用迭代法求线性方程组的数值解.一般理论.线性方程组的等价形式}
	X = D^{-1} (B - A X + D X)
	= D^{-1} (D - A) X
		+ D^{-1} B,
\end{equation}
其中\(D \defeq \diag(a_{11},a_{22},\dotsc,a_{nn})\).

我们可以比照\cref{equation:用迭代法求线性方程组的数值解.一般理论.线性方程组的等价形式}
写出递推公式\begin{equation}\label{equation:用迭代法求线性方程组的数值解.一般理论.递推公式}
%@see: 《数值分析(第5版)》(李庆扬、王能超、易大义) P181 (1.6)
%@see: 《数值分析(第5版)》(李庆扬、王能超、易大义) P184 (1.11)
	X_{k+1}
	\defeq
	M X_k + N
	\quad(k=0,1,2,\dotsc),
\end{equation}
其中\(
	M \defeq D^{-1} (D - A),
	N \defeq D^{-1} B
\).
把\(D\)称为\DefineConcept{分裂矩阵}.
把\(M\)称为\DefineConcept{迭代矩阵}.

%@see: 《数值分析(第5版)》(李庆扬、王能超、易大义) P181 定义1
我们把递推公式 \labelcref{equation:用迭代法求线性方程组的数值解.一般理论.递推公式}
所展示的迭代方法称为\DefineConcept{一阶定常迭代法}.
之所以称之为“一阶”,是因为递推公式只描绘了\(X_{k+1}\)和\(X_k\)的关系,
\(X_{k+1}\)的取值只依赖于前一步\(X_k\),而不依赖于更早的步骤(例如\(X_{k-1},X_{k-2}\)等).
之所以称之为“定常”,是因为递推公式中迭代矩阵\(M\)和常数项\(N\)在整个迭代过程中保持不变,不随迭代步数\(k\)而改变.
一阶定常迭代法形式简单、易于编程实现,但缺点是收敛速度较慢,甚至对于某些问题来说根本不收敛.
常见的一阶定常迭代法包括我们即将展开介绍的“雅克比迭代法”“高斯--塞德尔迭代法”“超松弛迭代法”.

容易验证:
当\begin{equation*}
	A = \begin{bmatrix}
		8 & -3 & 2 \\
		4 & 11 & -1 \\
		6 & 3 & 12 \\
	\end{bmatrix},
	\qquad
	B = \begin{bmatrix}
		20 \\ 33 \\ 36
	\end{bmatrix}
\end{equation*}
时,可以算出线性方程组\(A X = B\)的精确解为\(X^* = (3,2,1)^T\),
而如果令\begin{equation*}
	M \defeq \begin{bmatrix}
		0 & 3/8 & -2/8 \\
		-4/11 & 0 & 1/11 \\
		-6/12 & -3/12 & 0 \\
	\end{bmatrix},
	\qquad
	N \defeq \begin{bmatrix}
		20/8 \\ 33/11 \\ 36/12
	\end{bmatrix},
\end{equation*}
再取初始值\(X_0 \defeq (0,0,0)^T\),
最后代入\cref{equation:用迭代法求线性方程组的数值解.一般理论.递推公式} 反复迭代,
在迭代10次以后可以得到\begin{equation*}
	X_{10} = (
		\num{3.000032},
		\num{1.999838},
		\num{0.999881}
	)^T.
\end{equation*}
从这个例子我们可以看出:
有的线性方程组可以使用迭代法产生一个近似解向量序列\(\{X_k\}_{k\geq0}\),逐步逼近它的精确解.
但是必须说明的是:不是所有线性方程组都可以用迭代法逼近精确解\(X^*\).
例如,当\begin{equation*}
	M = \begin{bmatrix}
		0 & 2 \\
		3 & 0 \\
	\end{bmatrix},
	\qquad
	N = \begin{bmatrix}
		5 \\ 5
	\end{bmatrix}
\end{equation*}
时,利用\cref{equation:用迭代法求线性方程组的数值解.一般理论.递推公式}
得到的向量序列为\(
	(0,0)^T,
	(5,5)^T,
	(15,20)^T,
	(45,50)^T,
	\dotsc
\),
这显然无法逼近线性方程组的精确解\(X^* \defeq (-3,-4)^T\).

为了判断在什么情况下可以由递推公式 \labelcref{equation:用迭代法求线性方程组的数值解.一般理论.递推公式}
得到方程 \labelcref{equation:用迭代法求线性方程组的数值解.一般理论.线性方程组的等价形式} 的精确解\(X^*\),
我们需要知道经过迭代以后,\(X_k\)的误差是否逐渐缩小.
于是,我们把\(\epsilon_k \defeq X_k - X^*\)称为“第\(k\)步的\DefineConcept{误差向量}”.
如果随着迭代步数\(k\)增加,\(\epsilon_k\)逐渐“收敛”为零向量,
那么我们就可以宣称迭代法 \labelcref{equation:用迭代法求线性方程组的数值解.一般理论.递推公式} 是有效的,
或者说当迭代步数\(k\)足够多以后,就能利用这种方法得到足够精度的近似解.

由递推公式 \labelcref{equation:用迭代法求线性方程组的数值解.一般理论.递推公式} 可知\begin{equation*}
	\epsilon_{k+1} = M \epsilon_k
	\quad(k=0,1,2,\dotsc),
\end{equation*}
所以\begin{equation*}
	\epsilon_k
	= M^k \epsilon_0.
\end{equation*}
从这里可以看出,除非我们一开始就将猜中精确解,将精确解作为初始向量\(X_0\)开始迭代,
否则,误差向量\(\epsilon_k\)是否“收敛”为零向量,完全取决于迭代矩阵\(M\)的\(k\)次幂是否“收敛”为零矩阵.
为此,我们需要建立向量序列与矩阵序列的收敛性的概念与判别法.

\subsection{向量序列的收敛性}
为了考察向量序列的收敛性,就必须研究度量两个向量的“距离”的方法,这就要用到向量的范数.
\begin{definition}
%@see: 《数值分析(第5版)》(李庆扬、王能超、易大义) P182 定义2
设\(V\)是数域\(K\)上的一个\(n\)维向量空间(即\(V \defeq K^n\)),
向量\(\alpha \defeq (\AutoTuple{a}{n})^T \in V\).
如果向量序列\(\{\beta_k \defeq (b_{k1},\dotsc,b_{kn})^T\}_{k\geq0}\)
满足\begin{equation*}
	\lim_{k\to\infty} b_{ki} = a_i
	\quad(i=1,2,\dotsc,n),
\end{equation*}
那么称“向量序列\(\{\beta_k\}_{k\geq0}\)收敛于\(\alpha\)”,
记作\(\lim_{k\to\infty} \beta_k = \alpha\).
\end{definition}

\begin{theorem}
%@see: 《数值分析(第5版)》(李庆扬、王能超、易大义) P182
设\(V\)是数域\(K\)上的一个\(n\)维向量空间,
那么对于\(V\)的任意一个范数\(p\),
以下命题等价:\begin{itemize}
	\item 向量序列\(\{\beta_k\}_{k\geq0}\)收敛于\(\alpha \in V\);
	\item \(\lim_{k\to\infty} p( \beta_k - \alpha ) = 0\).
\end{itemize}
%TODO
\end{theorem}

\subsection{矩阵序列的收敛性}
类似地,让我们给出矩阵序列的收敛性的定义,以及矩阵序列的收敛性与矩阵范数的关系.
\begin{definition}
%@see: 《数值分析(第5版)》(李庆扬、王能超、易大义) P182 定义3
设\(V\)是数域\(K\)上的\(n\)阶全矩阵环(即\(K \defeq M_n(K)\)),
矩阵\(A \defeq (a_{ij})_n \in V\).
如果矩阵序列\(\{B_k \defeq (b_{kij})_n\}_{k\geq0}\)
满足\begin{equation*}
	\lim_{k\to\infty} b_{kij} = a_{ij}
	\quad(i,j=1,2,\dotsc,n),
\end{equation*}
那么称“矩阵序列\(\{B_k\}_{k\geq0}\)收敛于\(A\)”,
记作\(\lim_{k\to\infty} B_k = A\).
\end{definition}

\begin{example}
%@see: 《数值分析(第5版)》(李庆扬、王能超、易大义) P182 例2
实数域上的二阶矩阵序列\begin{equation*}
	B_k
	\defeq
	\begin{bmatrix}
		\lambda & 1 \\
		0 & \lambda \\
	\end{bmatrix}^k
	=
	\begin{bmatrix}
		\lambda^k & k \lambda^{k-1} \\
		0 & \lambda^k \\
	\end{bmatrix}
\end{equation*}
当\(\abs{\lambda} < 1\)时收敛于零矩阵.
\end{example}

\begin{theorem}
%@see: 《数值分析(第5版)》(李庆扬、王能超、易大义) P182 定理1
设\(V\)是数域\(K\)上的\(n\)阶全矩阵环(即\(K \defeq M_n(K)\)),
矩阵\(A \defeq (a_{ij})_n \in V\).
对于\(V\)上任意一个\(L_p\)范数,
以下命题等价:\begin{itemize}
	\item 矩阵序列\(\{B_k\}_{k\geq0}\)收敛于\(A \in V\);
	\item \(\lim_{k\to\infty} \MatrixNorm{ B_k - A }_p = 0\).
\end{itemize}
%TODO
\end{theorem}

\begin{theorem}
%@see: 《数值分析(第5版)》(李庆扬、王能超、易大义) P182 定理2
实数域上的\(n\)阶矩阵序列\(\{B_k\}_{k\geq0}\)
满足\(\lim_{k\to\infty} B_k = 0\)的充分必要条件是
对于任意向量\(\alpha \in \mathbb{R}^n\)
%@see: 《数值分析(第5版)》(李庆扬、王能超、易大义) P182 (1.7)
有\(\lim_{k\to\infty} B_k \alpha = 0\).
%TODO
\end{theorem}

\subsection{矩阵乘幂序列的收敛性}
下面考虑一种特殊的矩阵序列的收敛性,这种序列由某个给定矩阵的幂构成.
\begin{theorem}
%@see: 《数值分析(第5版)》(李庆扬、王能超、易大义) P183 定理3
设矩阵\(B \in M_n(\mathbb{R})\),
则以下命题等价:\begin{itemize}
	\item 矩阵序列\(\{B^k\}_{k\geq0}\)收敛于零矩阵;
	\item 矩阵\(B\)的谱半径为\(\SpecRad(B) < 1\ (k=0,1,2,\dotsc)\);
	\item 至少存在一种矩阵范数\(f\),使得\(f(B) < 1\ (k=0,1,2,\dotsc)\).
\end{itemize}
%TODO
\end{theorem}

\begin{theorem}
%@see: 《数值分析(第5版)》(李庆扬、王能超、易大义) P183 定理4
设矩阵\(B \in M_n(\mathbb{R})\),
那么对于任意一种矩阵范数\(f\),
实数列\(\{f(B^k)^{1/k}\}_{k\geq0}\)收敛于\(B\)的谱半径\(\SpecRad(B)\).
%TODO
\end{theorem}

\subsection{迭代法的收敛性}
在建立向量序列与矩阵序列的收敛性的概念与判别法以后,我们现在就可以定义迭代法的收敛性了.
\begin{definition}
%@see: 《数值分析(第5版)》(李庆扬、王能超、易大义) P181 定义1 (2)
给定线性方程组 \labelcref{equation:用迭代法求线性方程组的数值解.一般理论.线性方程组的等价形式}
及其一阶定常迭代法 \labelcref{equation:用迭代法求线性方程组的数值解.一般理论.递推公式}.
如果存在向量\(X^*\),向量序列\(\{X_k\}_{k\geq0}\)收敛于\(X^*\),
则称“迭代法 \labelcref{equation:用迭代法求线性方程组的数值解.一般理论.递推公式} \DefineConcept{收敛}”,
否则称“迭代法 \labelcref{equation:用迭代法求线性方程组的数值解.一般理论.递推公式} \DefineConcept{发散}”.
\end{definition}

可以证明,当迭代法 \labelcref{equation:用迭代法求线性方程组的数值解.一般理论.递推公式} 收敛时,
向量序列\(\{X_k\}_{k\geq0}\)收敛于精确解\(X^*\).

下面给出迭代法 \labelcref{equation:用迭代法求线性方程组的数值解.一般理论.递推公式} 收敛的充分必要条件.
\begin{theorem}%[一阶定常迭代法基本定理]
%@see: 《数值分析(第5版)》(李庆扬、王能超、易大义) P184 定理5
给定线性方程组 \labelcref{equation:用迭代法求线性方程组的数值解.一般理论.线性方程组的等价形式}
及其一阶定常迭代法 \labelcref{equation:用迭代法求线性方程组的数值解.一般理论.递推公式}.
对于任意取定的初始向量\(X_0\),
迭代法 \labelcref{equation:用迭代法求线性方程组的数值解.一般理论.递推公式} 收敛的充分必要条件是
迭代矩阵\(M\)的谱半径\(\SpecRad(M)<1\).
\end{theorem}

由于矩阵的谱半径总是不大于矩阵的范数,
所以我们可以利用矩阵范数建立判别迭代法收敛的充分不必要条件.
\begin{theorem}
%@see: 《数值分析(第5版)》(李庆扬、王能超、易大义) P185 定理6(迭代法收敛的充分条件)
给定线性方程组 \labelcref{equation:用迭代法求线性方程组的数值解.一般理论.线性方程组的等价形式}
及其一阶定常迭代法 \labelcref{equation:用迭代法求线性方程组的数值解.一般理论.递推公式}.
如果迭代矩阵\(M\)的某个范数\(q \defeq \MatrixNorm{M} < 1\),
则\begin{itemize}
	\item 迭代法收敛,即对任意取定初始向量\(X_0\)有\(\lim_{k\to\infty} X_k = X^*\)且\(X^* = M X^* + N\);
	\item \(\MatrixNorm{X^* - X_k} \leq q^k \MatrixNorm{X^* - X_0}\);
	\item \(\MatrixNorm{X^* - X_k} \leq \frac{q}{1-q} \MatrixNorm{X_k - X_{k-1}}\);
	\item \(\MatrixNorm{X^* - X_k} \leq \frac{q^k}{1-q} \MatrixNorm{X_1 - X_0}\).
\end{itemize}
\end{theorem}
由于上述定理仅是迭代法 \labelcref{equation:用迭代法求线性方程组的数值解.一般理论.递推公式} 收敛的充分不必要条件,
所以即便已知迭代矩阵\(M\)的常见范数(即\(
	\MatrixNorm{M}_\infty,
	\MatrixNorm{M}_1,
	\MatrixNorm{M}_2,
	\MatrixNorm{M}_F
\)都大于\(1\),
也不能断言迭代法不收敛,
也就是说\(M\)的谱半径完全有可能小于\(1\).

\subsection{迭代法的收敛速度}
下面考察迭代法 \labelcref{equation:用迭代法求线性方程组的数值解.一般理论.递推公式} 的收敛速度.
假设迭代法 \labelcref{equation:用迭代法求线性方程组的数值解.一般理论.递推公式} 是收敛的,
初始误差\(\epsilon_0 \neq 0\).
记\begin{equation*}
	X^* \defeq \lim_{k\to\infty} X_k,
	\qquad
	\epsilon_k \defeq X_k - X^*.
\end{equation*}
那么由\(
	\epsilon_k = M^k \epsilon_0,
	\epsilon_0 = X_0 - X^*
\)可得\begin{equation*}
	\norm{\epsilon_k}
	\leq
	\MatrixNorm{M^k}
	\norm{\epsilon_0},
\end{equation*}
于是\begin{equation*}
	\frac{
		\norm{\epsilon_k}
	}{
		\norm{\epsilon_0}
	}
	\leq
	\MatrixNorm{M^k}.
\end{equation*}
根据矩阵范数的定义,有\begin{equation*}
	\MatrixNorm{M^k}
	= \max_{0 \neq \epsilon_0 \in \mathbb{C}^n} \frac{ \norm{M^k \epsilon_0}_p }{ \norm{\epsilon_0}_p }
	= \max_{0 \neq \epsilon_0 \in \mathbb{C}^n} \frac{ \norm{\epsilon_k}_p }{ \norm{\epsilon_0}_p },
\end{equation*}
也就是说\(\MatrixNorm{M^k}\)是迭代\(k\)次以后,误差向量\(\epsilon_k\)的范数与初始误差向量\(\epsilon_0\)的范数之比的最大值.
那么,迭代\(k\)次以后,平均每次迭代的误差向量的范数的“压缩率”可以看成是\(\MatrixNorm{M^k}^{1/k}\).
如果要求迭代\(k\)次以后达到指定精度\(\sigma\)(我们常常取\(\sigma \defeq 10^{-s} \ll 1\),这里\(s\)是某个正整数),
即\begin{equation*}
	\norm{\epsilon_k}
	\leq \sigma \norm{\epsilon_0},
	\quad\text{即}\quad
	\frac{
		\norm{\epsilon_k}
	}{
		\norm{\epsilon_0}
	}
	\leq
	\MatrixNorm{M^k}
	\leq
	\sigma,
\end{equation*}
那么相当于要求\(\MatrixNorm{M^k}^{1/k} \leq \sigma^{1/k}\),
取对数得\begin{equation*}
	\ln\MatrixNorm{M^k}^{1/k} \leq \ln\sigma^{1/k} = \frac1k \ln\sigma,
\end{equation*}
即\begin{equation}
%@see: 《数值分析(第5版)》(李庆扬、王能超、易大义) P187 (1.12)
	k
	\geq \frac{ -\ln\sigma }{ -\ln\MatrixNorm{M^k}^{1/k} }
	= \frac{ s \ln10 }{ -\ln\MatrixNorm{M^k}^{1/k} }.
\end{equation}
可以看出,迭代次数\(k\)与\(-\ln\MatrixNorm{M^k}^{1/k}\)成反比.

据此,我们可以定义迭代法的平均收敛速度.
\begin{definition}
%@see: 《数值分析(第5版)》(李庆扬、王能超、易大义) P187 定义4
把\begin{equation}
%@see: 《数值分析(第5版)》(李庆扬、王能超、易大义) P187 (1.13)
	R_k(M)
	\defeq
	-\ln\MatrixNorm{M^k}^{1/k}
\end{equation}
称为“迭代法 \labelcref{equation:用迭代法求线性方程组的数值解.一般理论.递推公式} 的\DefineConcept{平均收敛速度}”.
\end{definition}

由于平均收敛速度\(R_k(M)\)依赖于迭代次数以及矩阵范数的选取,给计算分析带来不便,
加之我们已知\(\lim_{k\to\infty} \MatrixNorm{M^k}^{1/k} = \SpecRad(M)\),
从而有\(\lim_{k\to\infty} R_k(M) = -\ln\SpecRad(M)\).
据此,我们可以定义迭代法的渐进收敛速度.
\begin{definition}
%@see: 《数值分析(第5版)》(李庆扬、王能超、易大义) P187 定义5
把\begin{equation}
%@see: 《数值分析(第5版)》(李庆扬、王能超、易大义) P187 (1.14)
	R(M)
	\defeq
	-\ln\SpecRad(M)
\end{equation}
称为“迭代法 \labelcref{equation:用迭代法求线性方程组的数值解.一般理论.递推公式} 的\DefineConcept{渐进收敛速度}”.
\end{definition}
渐进收敛速度\(R(M)\)与迭代次数、矩阵范数的选取都没有关系,
它反映了迭代次数趋于无穷时,迭代法的渐进性质.
当\(\SpecRad(M)\)越小时,迭代法收敛越快.

我们常常使用不等式\begin{equation}
%@see: 《数值分析(第5版)》(李庆扬、王能超、易大义) P187 (1.15)
	k \geq \frac{s \ln10}{R(M)}
\end{equation}
估计迭代法 \labelcref{equation:用迭代法求线性方程组的数值解.一般理论.递推公式} 所需的迭代次数.
