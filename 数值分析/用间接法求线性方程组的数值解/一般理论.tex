\section{迭代法的基本概念}
\subsection{一阶定常迭代法}
设\(A \in M_n(F)\)是非奇异矩阵.
当\(A\)是低阶的、稠密的矩阵时,列主元消去法是求解线性方程组\(A x = b\)的有效方法.
但是,当\(A\)的阶数\(n\)很大时,或者当\(A\)是稀疏矩阵时,
就应该使用迭代法求解线性方程组\(A x = b\)了.

设\begin{equation*}
	A \defeq \begin{bmatrix}
		a_{11} & a_{12} & \dots & a_{1n} \\
		a_{21} & a_{22} & \dots & a_{2n} \\
		\vdots & \vdots & & \vdots \\
		a_{n1} & a_{n2} & \dots & a_{nn} \\
	\end{bmatrix},
	\qquad
	X \defeq \begin{bmatrix}
		x_1 \\ x_2 \\ \vdots \\ x_n
	\end{bmatrix},
	\qquad
	B \defeq \begin{bmatrix}
		b_1 \\ b_2 \\ \vdots \\ b_n
	\end{bmatrix}.
\end{equation*}
在求解线性方程组\(A X = B\)
或者\(
	\sum_{j=1}^n a_{ij} x_j = b_i
	\ (i=1,2,\dotsc,n)
\)时,可以将其变形,得到\begin{equation}
	x_i
	= \frac{1}{a_{ii}}
	\left(
		b_i
		- \sum_{j=1}^n a_{ij} x_j
		+ a_{ii} x_i
	\right)
	\quad(i=1,2,\dotsc,n)
\end{equation}
或者\begin{equation}\label{equation:用迭代法求线性方程组的数值解.一般理论.线性方程组的等价形式}
	X = D^{-1} (B - A X + D X)
	= D^{-1} (D - A) X
		+ D^{-1} B,
\end{equation}
其中\(D \defeq \diag(a_{11},a_{22},\dotsc,a_{nn})\).

我们可以比照\cref{equation:用迭代法求线性方程组的数值解.一般理论.线性方程组的等价形式}
写出递推公式\begin{equation}\label{equation:用迭代法求线性方程组的数值解.一般理论.递推公式}
%@see: 《数值分析(第5版)》(李庆扬、王能超、易大义) P181 (1.6)
	X_{k+1}
	\defeq
	M X_k + N
	\quad(k=1,2,\dotsc),
\end{equation}
其中\(
	M \defeq D^{-1} (D - A),
	N \defeq D^{-1} B
\).
把\(D\)称为\DefineConcept{分裂矩阵}.
把\(M\)称为\DefineConcept{迭代矩阵}.

%@see: 《数值分析(第5版)》(李庆扬、王能超、易大义) P181 定义1
我们把递推公式 \labelcref{equation:用迭代法求线性方程组的数值解.一般理论.递推公式}
所展示的迭代方法称为\DefineConcept{一阶定常迭代法}.
之所以称之为“一阶”,是因为递推公式只描绘了\(X_{k+1}\)和\(X_k\)的关系,
\(X_{k+1}\)的取值只依赖于前一步\(X_k\),而不依赖于更早的步骤(例如\(X_{k-1},X_{k-2}\)等).
之所以称之为“定常”,是因为递推公式中迭代矩阵\(M\)和常数项\(N\)在整个迭代过程中保持不变,不随迭代步数\(k\)而改变.
一阶定常迭代法形式简单、易于编程实现,但缺点是收敛速度较慢,甚至对于某些问题来说根本不收敛.
常见的一阶定常迭代法包括我们即将展开介绍的“雅克比迭代法”“高斯--塞德尔迭代法”“超松弛迭代法”.

容易验证:
当\begin{equation*}
	A = \begin{bmatrix}
		8 & -3 & 2 \\
		4 & 11 & -1 \\
		6 & 3 & 12 \\
	\end{bmatrix},
	\qquad
	B = \begin{bmatrix}
		20 \\ 33 \\ 36
	\end{bmatrix}
\end{equation*}
时,可以算出线性方程组\(A X = B\)的精确解为\(X^* = (3,2,1)^T\),
而如果令\begin{equation*}
	M \defeq \begin{bmatrix}
		0 & 3/8 & -2/8 \\
		-4/11 & 0 & 1/11 \\
		-6/12 & -3/12 & 0 \\
	\end{bmatrix},
	\qquad
	N \defeq \begin{bmatrix}
		20/8 \\ 33/11 \\ 36/12
	\end{bmatrix},
\end{equation*}
再取初始值\(X_0 \defeq (0,0,0)^T\),
最后代入\cref{equation:用迭代法求线性方程组的数值解.一般理论.递推公式} 反复迭代,
在迭代10次以后可以得到\begin{equation*}
	X_{10} = (
		\num{3.000032},
		\num{1.999838},
		\num{0.999881}
	)^T.
\end{equation*}
从这个例子我们可以看出:
有的线性方程组可以使用迭代法产生一个近似解向量序列\(\{X_k\}_{k\geq0}\),逐步逼近它的精确解.
但是必须说明的是:不是所有线性方程组都可以用迭代法逼近精确解\(X^*\).
例如,当\begin{equation*}
	M = \begin{bmatrix}
		0 & 2 \\
		3 & 0 \\
	\end{bmatrix},
	\qquad
	N = \begin{bmatrix}
		5 \\ 5
	\end{bmatrix}
\end{equation*}
时,利用\cref{equation:用迭代法求线性方程组的数值解.一般理论.递推公式}
得到的向量序列为\(
	(0,0)^T,
	(5,5)^T,
	(15,20)^T,
	(45,50)^T,
	\dotsc
\),
这显然无法逼近线性方程组的精确解\(X^* \defeq (-3,-4)^T\).

为了判断在什么情况下可以由递推公式 \labelcref{equation:用迭代法求线性方程组的数值解.一般理论.递推公式}
得到方程 \labelcref{equation:用迭代法求线性方程组的数值解.一般理论.线性方程组的等价形式} 的精确解\(X^*\),
我们需要知道经过迭代以后,\(X_k\)的误差是否逐渐缩小.
于是,我们把\(\epsilon_k \defeq X_k - X^*\)称为“第\(k\)步的\DefineConcept{误差向量}”.
如果随着迭代步数\(k\)增加,\(\epsilon_k\)逐渐“收敛”为零向量,
那么我们就可以宣称迭代法 \labelcref{equation:用迭代法求线性方程组的数值解.一般理论.递推公式} 是有效的,
或者说当迭代步数\(k\)足够多以后,就能利用这种方法得到足够精度的近似解.

由递推公式 \labelcref{equation:用迭代法求线性方程组的数值解.一般理论.递推公式} 可知\begin{equation*}
	\epsilon_{k+1} = M \epsilon_k
	\quad(k=0,1,2,\dotsc),
\end{equation*}
所以\begin{equation*}
	\epsilon_k
	= M^k \epsilon_0.
\end{equation*}
从这里可以看出,除非我们一开始就将猜中精确解,将精确解作为初始向量\(X_0\)开始迭代,
否则,误差向量\(\epsilon_k\)是否“收敛”为零向量,完全取决于迭代矩阵\(M\)的\(k\)次幂是否“收敛”为零矩阵.
为此,我们需要建立向量序列与矩阵序列的收敛性的概念与判别法.
