\subsection{雅克比迭代法、高斯--塞德尔迭代法的收敛性}
\begin{theorem}
%@see: 《数值分析(第5版)》(李庆扬、王能超、易大义) P190 定理7
设线性方程组\(A X = B\)的系数矩阵\(A = D - L - U\)和它的对角部分\(D\)都是非奇异矩阵,
则\begin{itemize}
	\item 雅克比迭代法收敛,当且仅当迭代矩阵\(J \defeq D^{-1} (L + U)\)的谱半径\(\SpecRad(J) < 1\);
	\item 高斯--塞德尔迭代法收敛,当且仅当迭代矩阵\(G \defeq (D - L)^{-1} U\)的谱半径\(\SpecRad(G) < 1\).
\end{itemize}
%TODO proof
\end{theorem}

\begin{theorem}
%@see: 《数值分析(第5版)》(李庆扬、王能超、易大义) P192 定理9
设线性方程组\(A X = B\)的系数矩阵\(A\)是严格对角占优矩阵或不可约弱对角占优矩阵,
则雅克比迭代法、高斯--赛德尔迭代法均收敛.
%TODO proof
\end{theorem}

\begin{theorem}
%@see: 《数值分析(第5版)》(李庆扬、王能超、易大义) P192 定理10
设线性方程组\(A X = B\)的系数矩阵\(A = D - L - U\)是对称矩阵,且主对角元都是正数,
则\begin{itemize}
	\item 雅克比迭代法收敛,当且仅当\(A\)和\(2 D - A\)都是正定矩阵;
	\item 高斯--塞德尔迭代法收敛,当且仅当\(A\)是正定矩阵.
\end{itemize}
%TODO proof
\end{theorem}
