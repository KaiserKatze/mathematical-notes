\section{雅克比迭代法}
%@see: https://mathworld.wolfram.com/JacobiMethod.html
将线性方程组\(A X = B\)的系数矩阵\(A\)分成三部分:\begin{equation*}
%@see: 《数值分析(第5版)》(李庆扬、王能超、易大义) P188 (2.1)
	A = D - L - U,
\end{equation*}
其中\begin{align*}
	D
	&\defeq
	\begin{bmatrix}
		a_{11} \\
		& a_{22} \\
		&& \ddots \\
		&&& a_{nn} \\
	\end{bmatrix}, \\
	L
	&\defeq
	- \begin{bmatrix}
		0 \\
		a_{21} & 0 \\
		\vdots & \vdots & \ddots \\
		a_{n-1,1} & a_{n-1,2} & \dots & 0 \\
		a_{n1} & a_{n2} & \dots & a_{n,n-1} & 0 \\
	\end{bmatrix}, \\
	U
	&\defeq
	- \begin{bmatrix}
		0 & a_{12} & \dots & a_{1,n-1} & a_{1n} \\
		& 0 & \dots & a_{2,n-1} & a_{2n} \\
		&& \ddots & \vdots & \vdots \\
		&&& 0 & a_{n-1,n} \\
		&&&& 0 \\
	\end{bmatrix}.
\end{align*}

假设\(D \neq 0\),即\(a_{ii} \neq 0\ (i=1,2,\dotsc,n)\).
令\(
	J \defeq D^{-1} (L + U),
	C \defeq D^{-1} B
\),
那么递推公式 \labelcref{equation:用迭代法求线性方程组的数值解.一般理论.递推公式}
可以写成\begin{equation}
%@see: 《数值分析(第5版)》(李庆扬、王能超、易大义) P188 (2.2)
	X_{k+1}
	\defeq
	J X_k + C
	\quad(k=0,1,2,\dotsc),
\end{equation}
或\begin{equation}
%@see: 《数值分析(第5版)》(李庆扬、王能超、易大义) P188 (2.3)
	x_{k+1,i}
	= \frac{1}{a_{ii}} \left( b_i - \sum_{\substack{1 \leq j \leq n \\ j \neq i}} a_{ij} x_{k,j} \right)
	\quad(i=1,2,\dotsc,n;k=0,1,2,\dotsc),
\end{equation}
其中\(X_k \defeq (x_{k,1},\dotsc,x_{k,n})\).

可以看出,雅克比迭代法计算公式简单,每迭代一次只需要计算一次矩阵和向量的乘法,且计算过程中系数矩阵\(A\)始终不变.

\begin{theorem}
%@see: 《数值分析(第5版)》(李庆扬、王能超、易大义) P190 定理7 (1)
设线性方程组\(A X = B\)的系数矩阵\(A = D - L - U\)和它的对角部分\(D\)都是非奇异矩阵,
则雅克比迭代法收敛的充分必要条件是迭代矩阵\(J\)的谱半径\(\SpecRad(J) < 1\).
\end{theorem}
