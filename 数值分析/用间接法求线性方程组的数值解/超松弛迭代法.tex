\section{超松弛迭代法}
%@see: https://mathworld.wolfram.com/SuccessiveOverrelaxationMethod.html
\subsection{逐次超松弛迭代法的概念}
假设\(D \neq 0\),即\(a_{ii} \neq 0\ (i=1,2,\dotsc,n)\).
选取\(\frac1\omega (D - \omega L)\)为分裂矩阵,
其中\(\omega > 0\)是待定参数,称为\DefineConcept{松弛因子},
便可得到迭代矩阵\begin{equation*}
	L_\omega
	\defeq
	(D - \omega L)^{-1}
	[(1-\omega) D + \omega U],
\end{equation*}
那么递推公式 \labelcref{equation:用迭代法求线性方程组的数值解.一般理论.递推公式}
可以写成\begin{equation}
%@see: 《数值分析(第5版)》(李庆扬、王能超、易大义) P194 (3.1)
	X_{k+1}
	\defeq
	L_\omega X_k + C
	\quad(k=0,1,2,\dotsc),
\end{equation}
其中\(C \defeq \omega (D - \omega L)^{-1} B\),
或\begin{equation}
%@see: 《数值分析(第5版)》(李庆扬、王能超、易大义) P194 (3.2)
	x_{k+1,i}
	= x_{k,i} + \frac{\omega}{a_{ii}} \left( b_i - \sum_{j=1}^{i-1} a_{ij} x_{k+1,j} - \sum_{j=i}^n a_{ij} x_{k,j} \right)
	\quad(i=1,2,\dotsc,n;k=0,1,2,\dotsc),
\end{equation}
或\begin{equation}
%@see: 《数值分析(第5版)》(李庆扬、王能超、易大义) P194 (3.3)
	\begin{cases}
		x_{k+1,i} = x_{k,i} + \increment x_i, \\
		\increment x_i
		= \frac{\omega}{a_{ii}} \left( b_i - \sum_{j=1}^{i-1} a_{ij} x_{k+1,j} - \sum_{j=i}^n a_{ij} x_{k,j} \right)
	\end{cases}
	\quad(i=1,2,\dotsc,n;k=0,1,2,\dotsc),
\end{equation}
其中\(X_k \defeq (x_{k,1},\dotsc,x_{k,n})\).

可以注意到,当\(\omega = 1\)时,逐次超松弛迭代法成为高斯--塞德尔迭代法.
因此逐次超松弛迭代法实际上是高斯--塞德尔迭代法的一种修正,
具体地说,首先用高斯--塞德尔迭代法计算出\begin{equation}
%@see: 《数值分析(第5版)》(李庆扬、王能超、易大义) P195 (3.4)
	\widetilde{x}_{k+1,i}
	\defeq \frac{1}{a_{ii}} \left( b_i - \sum_{j=1}^{i-1} a_{ij} x_{k+1,j} - \sum_{j=i+1}^n a_{ij} x_{k,j} \right),
\end{equation}
再将\(x_{k,i}\)与\(\widetilde{x}_{k+1,i}\)加权平均,得到修正后的结果\begin{equation}
%@see: 《数值分析(第5版)》(李庆扬、王能超、易大义) P195 (3.5)
	x_{k+1,i} \defeq (1-\omega) x_{k,i} + \omega \widetilde{x}_{k+1,i}.
\end{equation}

逐次超松弛迭代法每迭代一次只需要计算一次矩阵和向量的乘法.
当\(\omega > 1\)时,称之为超松弛法.
当\(\omega < 1\)时,称之为低松弛法.

在计算机实现时,可以用\begin{equation}
	\max_{1 \leq i \leq n} \abs{\increment x_i}
	= \max_{1 \leq i \leq n} \abs{x_{k+1,i} - x_{k,i}}
	< \epsilon
\end{equation}
控制迭代终止,
或者用\begin{equation}
	\norm{B - A X_k}_\infty
	< \epsilon
\end{equation}
控制迭代终止.
