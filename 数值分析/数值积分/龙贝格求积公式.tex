\section{龙贝格求积公式}
%@see: https://mathworld.wolfram.com/RombergIntegration.html
上一节介绍的复合求积方法可以提高求积精度.
在实际计算中,如果无法预知需要划分的区间个数,
那么可以运用迭代思想,在发现精度不够时,将步长减半.
具体地,假设一开始将区间\([a,b]\)划分为\(n\)等份,


% 变步长的优势:在无法预知需要划分的区间个数时,使用变步长方法,可以计算出符合精度要求的近似值,并且可以随时停止计算,节省算力
% 变步长的劣势:只能求\(T_n,T_{2n},T_{4n},...\),不能求\(T_3,T_7,T_{30}\).

\begin{equation}
	S_n = \frac{4}{3} T_{2n} - \frac{1}{3} T_n.
\end{equation}

\begin{equation}
	C_n = \frac{16}{15} S_{2n} - \frac{1}{15} S_n.
\end{equation}

\begin{equation}
	R_n = \frac{64}{63} C_{2n} - \frac{1}{63} C_n.
\end{equation}
