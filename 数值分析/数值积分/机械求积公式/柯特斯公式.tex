\section{牛顿--柯特斯公式}
\subsection{柯特斯系数}
对于定积分\(\int_a^b f(x) \dd{x}\),
假设我们将积分区间\([a,b]\)划为\(n\)等份,
步长为\(h = \frac{b-a}{n}\),
选取等距节点\(x_k \defeq a + k h\ (k=0,1,2,\dotsc,n)\),
构造插值型求积公式\begin{equation}\label{equation:柯特斯公式}
%@see: 《数值分析(第5版)》(李庆扬、王能超、易大义) P103 (2.1)
	I_n
	\defeq
	(b-a) \sum_{k=0}^n C^{(n)}_k f(x_k),
\end{equation}
其中\begin{equation}
%@see: 《数值分析(第5版)》(李庆扬、王能超、易大义) P103 (2.2)
	C^{(n)}_k
	\defeq
	\frac{h}{b-a} \int_0^n \prod_{\substack{0 \leq j \leq n \\ j \neq k}} \frac{t-j}{k-j} \dd{t}
	= \frac{(-1)^{n-k}}{n k! (n-k)!} \int_0^n \prod_{\substack{0 \leq j \leq n \\ j \neq k}} (t-j) \dd{t}.
\end{equation}

当\(n=1\)时,
有\begin{align*}
	C^{(1)}_0
\end{align*}
这就是梯形公式

当\(n=2\)时,
有

\begin{equation*}
	\frac{b-a}{6}
	\left[
		f(a)
		+ 4 f\left( \frac{a+b}{2} \right)
		+ f(b)
	\right]
\end{equation*}
我们将其称为\DefineConcept{辛普森公式}.
% 辛普森公式给出了一个具有单峰结构的函数

当\(n=4\)时,
% 4阶柯特斯公式给出了一个具有双峰结构的函数

% 柯特斯系数表,当\(n\geq8\)时出现负值,说明牛顿--柯特斯公式不稳定
% 类似地,拉格朗日插值、勒让德也不稳定

% 为什么用三次样条插值?因为它可以让拟合函数在不同子区间之间的节点处平滑过渡。

% 辛普森公式的推导,辛普森公式的余项 (重要)
