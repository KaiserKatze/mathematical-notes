\section{复合求积公式}
由于牛顿--柯特斯公式在\(n\geq8\)时不具有稳定性,
故不可能通过提高阶的方法来提高求积精度.
为了提高精度,通常可以把积分区间分成若干个子区间,
再在每个子区间上运用低阶求积公式.
这种方法称为复合求积法.
本节只讨论复合梯形公式与复合辛普森公式.

\subsection{复合梯形公式}
将区间\([a,b]\)划分为\(n\)等份,
分点\(
	x_k \defeq a + k h
	\ (k=0,1,2,\dotsc,n)
\),
其中步长为\(
	h \defeq \frac{b - a}{n}
\).
因为\(
%@see: 《数值分析(第5版)》(李庆扬、王能超、易大义) P106 (3.1)
	\int_a^b f(x) \dd{x}
	= \sum_{k=0}^{n-1} \int_{x_k}^{x_{k+1}} f(x) \dd{x}
\),
所以只要在每个子区间\([x_k,x_{k+1}]\ (k=0,1,2,\dotsc,n-1)\)上
运用梯形公式 \labelcref{equation:数值积分.梯形公式},
将\(
	\frac{h}{2} ( f(x_k) + f(x_{k+1}) )
\)
作为定积分\(
	\int_{x_k}^{x_{k+1}} f(x) \dd{x}
\)的近似值,
于是我们可以将\begin{equation}\label{equation:复合求积公式.复合梯形公式}
%@see: 《数值分析(第5版)》(李庆扬、王能超、易大义) P106 (3.2)
	\begin{aligned}
		T_n
		&\defeq \frac{h}{2} \sum_{k=0}^{n-1} ( f(x_k) + f(x_{k+1}) )
		\\
		&= \frac{h}{2}
		\left(
			f(a) + 2 \sum_{k=1}^{n-1} f(x_k) + f(b)
		\right)
	\end{aligned}
\end{equation}
作为定积分\(
	\int_a^b f(x) \dd{x}
\)的近似值.
我们把\cref{equation:复合求积公式.复合梯形公式}
称为\DefineConcept{复合梯形公式}.

复合梯形公式 \labelcref{equation:复合求积公式.复合梯形公式} 的余项为\begin{equation}%\label{equation:复合求积公式.复合梯形公式的余项1}
	R_n(f)
	\defeq
	-\frac{h^3}{12} \sum_{k=0}^{n-1} f''(\eta_k)
	\quad(x_k < \eta_k < x_{k+1}).
\end{equation}
如果\(f \in C^2[a,b]\)且\begin{equation*}
	\min_{0 \leq k \leq n-1} f''(\eta_k)
	\leq
	\frac1n \sum_{k=0}^{n-1} f''(\eta_k)
	\leq
	\max_{0 \leq k \leq n-1} f''(\eta_k),
\end{equation*}
则存在\(\eta \in (a,b)\)使得\(
	f''(\eta)
	= \frac1n \sum_{k=0}^{n-1} f''(\eta_k)
\),
这时复合梯形公式 \labelcref{equation:复合求积公式.复合梯形公式} 的余项可以写成\begin{equation}\label{equation:复合求积公式.复合梯形公式的余项2}
%@see: 《数值分析(第5版)》(李庆扬、王能超、易大义) P107 (3.3)
	R_n(f)
	\defeq
	-\frac{(b-a) h^2}{12} f''(\eta)
	\quad(a < \eta < b).
\end{equation}
由此可见,对2阶连续可导函数运用复合梯形公式产生的求积误差是\(h^2\)阶无穷小,
且\(
	\lim_{n\to\infty} T_n
	= \int_a^b f(x) \dd{x}
\)(即复合梯形公式是收敛的).
实际上,即便只知道\(f \in C[a,b]\),也可以得到\(T_n\)的收敛性,
这是因为只要把\(T_n\)改写为\begin{equation*}
	T_n = \frac12 \left( \frac{b-a}{n} \sum_{k=0}^{n-1} f(x_k) + \frac{b-a}{n} \sum_{k=1}^n f(x_k) \right),
\end{equation*}
当\(n\to\infty\)时,
上式右端括号内的两个和式均收敛到积分\(\int_a^b f(x) \dd{x}\).

由于\(T_n\)的求积系数总是正数,
所以复合梯形公式 \labelcref{equation:复合求积公式.复合梯形公式} 是稳定的.
%TODO proof

\subsection{复合辛普森公式}
将区间\([a,b]\)划分为\(n\)等份,
分点\(
	x_k \defeq a + k h
	\ (k=0,1,2,\dotsc,n)
\),
其中步长为\(
	h \defeq \frac{b - a}{n}
\).
将每个子区间\([x_k,x_{k+1}]\ (k=0,1,2,\dotsc,n-1)\)的中点记为\(x_{k+\frac12}\),
那么我们可以将\begin{equation}\label{equation:复合求积公式.复合辛普森公式}
%@see: 《数值分析(第5版)》(李庆扬、王能超、易大义) P107 (3.4)
%@see: 《数值分析(第5版)》(李庆扬、王能超、易大义) P107 (3.5)
	\begin{aligned}
		S_n
		&\defeq \frac{h}{6} \sum_{k=0}^{n-1} ( f(x_k) + 4 f(x_{k+\frac12}) + f(x_{k+1}) )
		\\
		&= \frac{h}{6}
		\left(
			f(a) + 4 \sum_{k=0}^{k-1} f(x_{k+\frac12}) + 2 \sum_{k=1}^{n-1} f(x_k) + f(b)
		\right)
	\end{aligned}
\end{equation}
作为定积分\(
	\int_a^b f(x) \dd{x}
\)的近似值.
我们把\cref{equation:复合求积公式.复合辛普森公式}
称为\DefineConcept{复合辛普森公式}.

复合辛普森公式 \labelcref{equation:复合求积公式.复合辛普森公式} 的余项为\begin{equation}
	R_n(f)
	\defeq
	-\frac{h}{180} \left( \frac{h}{2} \right)^4
	\sum_{k=0}^{n-1} f^{(4)}(\eta_k)
	\quad(x_k < \eta_k < x_{k+1}).
\end{equation}
当\(f \in C^4[a,b]\)时,
复合辛普森公式 \labelcref{equation:复合求积公式.复合辛普森公式} 的余项为\begin{equation}
%@see: 《数值分析(第5版)》(李庆扬、王能超、易大义) P107 (3.6)
	R_n(f)
	= -\frac{b-a}{180} \left( \frac{h}{2} \right)^4 f^{(4)}(\eta)
	\quad(a < \eta < b).
\end{equation}
