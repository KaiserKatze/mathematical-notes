\section{龙贝格求积公式}
%@see: https://mathworld.wolfram.com/RombergIntegration.html
%@see: https://zhuanlan.zhihu.com/p/1895142933969216910
上一节介绍的复合求积方法可以提高求积精度.
在实际计算中,如果无法预知需要划分的区间个数,
那么可以运用迭代思想,在发现精度不够时,将步长减半.
具体地,假设一开始将区间\([a,b]\)划分为\(n\)等份,
分点\(
	x_k \defeq a + k h
	\ (k=0,1,2,\dotsc,n)
\),
运用复合梯形公式\begin{equation*}
	T_n
	= \frac{h}{2}
	\left(
		f(a) + 2 \sum_{k=1}^{n-1} f(x_k) + f(b)
	\right),
\end{equation*}
其中步长为\(
	h \defeq \frac{b - a}{n}
\).
如果这时我们发现精度不够,
则将区间\([a,b]\)划分为\(2n\)等份,
在\(x_k\)与\(x_{k+1}\)之间新增分点\(x_{k+1/2}\)(\(k=0,1,2,\dotsc,n-1\)),
运用复合梯形公式\begin{equation*}
	T_{2n}
	= \frac{h'}{2}
	\left(
		f(a) + 2 \sum_{k=1}^{n-1} f(x_k) + 2 \sum_{k=0}^{n-1} f(x_{k+1/2}) + f(b)
	\right),
\end{equation*}
其中步长为\(
	h' \defeq \frac{h}{2} = \frac{b - a}{2n}
\).
显然\(T_{2n}\)可以改写为一个递推公式\begin{align*}
%@see: 《数值分析(第5版)》(李庆扬、王能超、易大义) P110 (4.1)
	T_{2n}
	&= \frac{h'}{2}
	\left(
		f(a) + 2 \sum_{k=1}^{n-1} f(x_k) + 2 \sum_{k=0}^{n-1} f(x_{k+1/2}) + f(b)
	\right) \\
	&= \frac{h'}{2}
	\left(
		f(a) + 2 \sum_{k=1}^{n-1} f(x_k) + f(b)
	\right)
	+ \frac{h'}{2} \cdot 2 \sum_{k=0}^{n-1} f(x_{k+1/2}) \\
	&= \frac{1}{2} \cdot \frac{h}{2}
	\left(
		f(a) + 2 \sum_{k=1}^{n-1} f(x_k) + f(b)
	\right)
	+ \frac{h}{2} \sum_{k=0}^{n-1} f(x_{k+1/2}) \\
	&= \frac{1}{2} T_n + \frac{h}{2} \sum_{k=0}^{n-1} f(x_{k+1/2}).
\end{align*}
假设被积函数\(f\)在\([a,b]\)上是光滑的,
记\(I \defeq \int_a^b f(x) \dd{x}\),
那么由\cref{equation:复合求积公式.复合梯形公式的余项2} 可知\begin{equation*}
	I - T_n
	= - \frac{b-a}{12} h^2 f''(\eta)
	\quad(a \leq \eta \leq b).
\end{equation*}
如果把\(T_n\)看作关于\(h\)的函数,
记\(g(h) \defeq T_n\),
则\(g(h) = I + \frac{b-a}{12} h^2 f''(\eta)\),
易知\begin{equation*}
	\lim_{h\to0} g(h)
	= g(0)
	= I.
\end{equation*}
这就证明梯形公式的余项可以展成幂级数,
因此我们得到如下结论.
\begin{theorem}
%@see: 《数值分析(第5版)》(李庆扬、王能超、易大义) P111 定理4
设\(f \in C^\infty[a,b]\),
则\begin{equation*}
%@see: 《数值分析(第5版)》(李庆扬、王能超、易大义) P111 (4.2)
	g(h)
	= I + \sum_{k=1}^\infty a_k h^{2k},
\end{equation*}
其中系数\(a_k\ (k=1,2,\dotsc)\)与\(h\)的取值无关.
\end{theorem}

上述定理表明,用\(g(h)\)近似\(I\)产生的绝对误差是比\(h^2\)高阶的无穷小\(o(h^2)\ (h\to0)\),
而用\begin{equation*}
	g^*(h) \defeq \frac{ 4 g(h/2) - g(h) }{3}
\end{equation*}
近似\(I\)产生的绝对误差是\(o(h^4)\),
这比复合梯形公式\(g(h)\)的误差阶\(o(h^2)\)提高了,
容易验证\(g^*(h)\)就是复合辛普森公式 \labelcref{equation:复合求积公式.复合辛普森公式},
即\begin{equation}\label{equation:龙贝格求积公式.复合梯形公式外推成复合辛普森公式}
	S_n = \frac{4}{3} T_{2n} - \frac{1}{3} T_n.
\end{equation}

上面所说的,将计算积分的近似值的误差阶由\(o(h^{2n})\)提高到\(o(h^{2n+2})\)的方法,
称为\DefineConcept{理查德外推算法}(Richardson extrapolation).
%@see: https://mathworld.wolfram.com/RichardsonExtrapolation.html
%@see: https://en.wikipedia.org/wiki/Richardson_extrapolation
这是数值分析中的一个重要的技巧:
只要真值与近似值的误差可以表示成某个参数(例如\(h\))的幂级数,
就都可以使用理查德外推算法,提高精度.

类似地,我们可以验证复合辛普森公式与复合柯特斯公式之间的关系为\begin{equation}\label{equation:龙贝格求积公式.复合辛普森公式外推成复合柯特斯公式}
%@see: 《数值分析(第5版)》(李庆扬、王能超、易大义) P111 (4.7)
	C_n = \frac{16}{15} S_{2n} - \frac{1}{15} S_n.
\end{equation}
显然复合柯特斯公式的误差阶为\(o(h^6)\).

再进一步,继续运用外推技巧,就可以得到误差阶为\(o(h^8)\)的近似公式\begin{equation}\label{equation:龙贝格求积公式.复合柯特斯公式外推成复合龙贝格公式}
	R_n \defeq \frac{64}{63} C_{2n} - \frac{1}{63} C_n.
\end{equation}

下面我们将\cref{equation:龙贝格求积公式.复合梯形公式外推成复合辛普森公式,equation:龙贝格求积公式.复合辛普森公式外推成复合柯特斯公式,equation:龙贝格求积公式.复合柯特斯公式外推成复合龙贝格公式}
统一成一个公式.
令\begin{equation*}
	T_0(h) \defeq T_n,
	\qquad
	T_1(h) \defeq S_n,
	\qquad
	T_2(h) \defeq C_n,
	\qquad
	T_3(h) \defeq R_n,
\end{equation*}
则\begin{equation}
%@see: 《数值分析(第5版)》(李庆扬、王能超、易大义) P112 (4.9)
	T_m(h) = \frac{4^m}{4^m - 1} T_{m-1}(h/2) - \frac{1}{4^m - 1} T_{m-1}(h).
\end{equation}

用\(T_0^{(k)}\)表示二分\(k\)次后求得的梯形值(利用复合梯形公式求得的近似值),
用\(T_m^{(k)}\)表示序列\(\{T_0^{(k)}\}\)的\(m\)次加速值,
则根据上述递推公式可得\begin{equation}\label{equation:龙贝格求积公式.龙贝格求积算法}
%@see: 《数值分析(第5版)》(李庆扬、王能超、易大义) P112 (4.9)
	T_m^{(k)}
	= \frac{4^m}{4^m - 1} T_{m-1}^{(k+1)}
	- \frac{1}{4^m - 1} T_{m-1}^{(k)}
	\quad(k=1,2,\dotsc).
\end{equation}
我们把递推公式 \labelcref{equation:龙贝格求积公式.龙贝格求积算法}
称为\DefineConcept{龙贝格求积算法}.

% 变步长的优势:在无法预知需要划分的区间个数时,使用变步长方法,可以计算出符合精度要求的近似值,并且可以随时停止计算,节省算力
% 变步长的劣势:只能求\(T_n,T_{2n},T_{4n},...\),不能求\(T_3,T_7,T_{30}\).





