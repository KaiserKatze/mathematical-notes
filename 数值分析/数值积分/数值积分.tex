\section{数值积分}
\subsection{数值积分的基本思想}
设函数\(f\)在区间\([a,b]\)上黎曼可积,
那么我们可以将\begin{equation}\label{equation:数值积分.梯形公式}
%@see: 《数值分析(第5版)》(李庆扬、王能超、易大义) P97 (1.1)
	\frac{f(a) + f(b)}{2} (b-a)
\end{equation}
作为定积分\(\int_a^b f(x) \dd{x}\)的近似值,
也可以将\begin{equation}\label{equation:数值积分.中矩形公式}
%@see: 《数值分析(第5版)》(李庆扬、王能超、易大义) P98 (1.2)
	f\left( \frac{a+b}{2} \right) (b-a)
\end{equation}
作为定积分\(\int_a^b f(x) \dd{x}\)的近似值.
我们把\cref{equation:数值积分.梯形公式} 称为\DefineConcept{梯形公式},
把\cref{equation:数值积分.中矩形公式} 称为\DefineConcept{中矩形公式}.

一般地,我们常常将\begin{equation}\label{equation:数值积分.机械求积公式}
	\sum_{k=0}^n A_k f(x_k)
\end{equation}
作为定积分\(\int_a^b f(x) \dd{x}\)的近似值,
其中\(x_k\)称为\DefineConcept{求积节点},
\(A_k\)称为\DefineConcept{求积系数}或“伴随节点\(x_k\)的\DefineConcept{权}”.
可以证明:\(A_k\)仅与\(x_k\)的选取有关,而不依赖于被积函数\(f\)的具体形式.

我们把上述数值积分方法称为\DefineConcept{机械求积}.
它的特点是将积分求值问题归结为被积函数值的计算,
避开了牛顿--莱布尼茨公式需要寻求原函数的困难,
很适合在计算机上使用.

\subsection{代数精度的概念}
数值求积方法是近似方法.
为了保证计算精度,我们自然希望求积公式能对“尽可能多”的函数准确成立,这就引出了“代数精度”的概念.
\begin{definition}
%@see: 《数值分析(第5版)》(李庆扬、王能超、易大义) P99 定义1
如果求积公式 \labelcref{equation:数值积分.机械求积公式}
对于次数不超过\(m\)的一元多项式均能准确地成立,
即\begin{equation*}
	\int_a^b x^i \dd{x}
	= \frac{1}{i+1} (b^{i+1} - a^{i+1})
	= \sum_{k=0}^n A_k x_k^i
	\quad(i=0,1,2,\dotsc,m),
\end{equation*}
但是对于\(m+1\)次一元多项式就不准确成立,
即\begin{equation*}
	\int_a^b x^{m+1} \dd{x}
	= \frac{1}{m+2} (b^{m+2} - a^{m+2})
	\neq \sum_{k=0}^n A_k x_k^{m+1},
\end{equation*}
则称“求积公式 \labelcref{equation:数值积分.机械求积公式}
具有\(m\)次\DefineConcept{代数精度}”.
\end{definition}

显然,梯形公式 \labelcref{equation:数值积分.梯形公式}
和中矩形公式 \labelcref{equation:数值积分.中矩形公式}
均具有1次代数精度.
