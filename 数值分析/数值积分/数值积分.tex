\section{数值积分}
\subsection{数值积分的基本思想}
设函数\(f\)在区间\([a,b]\)上黎曼可积,
那么我们可以将\begin{equation}\label{equation:数值积分.梯形公式}
%@see: 《数值分析(第5版)》(李庆扬、王能超、易大义) P97 (1.1)
	\frac{f(a) + f(b)}{2} (b-a)
\end{equation}
作为定积分\(\int_a^b f(x) \dd{x}\)的近似值,
也可以将\begin{equation}\label{equation:数值积分.中矩形公式}
%@see: 《数值分析(第5版)》(李庆扬、王能超、易大义) P98 (1.2)
	f\left( \frac{a+b}{2} \right) (b-a)
\end{equation}
作为定积分\(\int_a^b f(x) \dd{x}\)的近似值.
我们把\cref{equation:数值积分.梯形公式} 称为\DefineConcept{梯形公式},
把\cref{equation:数值积分.中矩形公式} 称为\DefineConcept{中矩形公式}.

一般地,我们常常将\begin{equation}\label{equation:数值积分.机械求积公式}
	\sum_{k=0}^n A_k f(x_k)
\end{equation}
作为定积分\(\int_a^b f(x) \dd{x}\)的近似值.
我们把\(x_k\)称为\DefineConcept{求积节点},
把\(A_k\)称为\DefineConcept{求积系数}或“伴随节点\(x_k\)的\DefineConcept{权}”.

可以证明:\(A_k\)仅与\(x_k\)的选取有关,而不依赖于被积函数\(f\)的具体形式.

我们把上述数值积分方法称为\DefineConcept{机械求积}.
它的特点是将积分求值问题归结为被积函数值的计算,
避开了牛顿--莱布尼茨公式需要寻求原函数的困难,
很适合在计算机上使用.

\subsection{代数精度的概念}
数值求积方法是近似方法.
为了保证计算精度,我们自然希望求积公式能对“尽可能多”的函数准确成立,这就引出了“代数精度”的概念.
\begin{definition}
%@see: 《数值分析(第5版)》(李庆扬、王能超、易大义) P99 定义1
如果求积公式 \labelcref{equation:数值积分.机械求积公式}
对于次数不超过\(m\)的一元多项式均能准确地成立,
即\begin{equation*}
	\int_a^b x^i \dd{x}
	= \frac{1}{i+1} (b^{i+1} - a^{i+1})
	= \sum_{k=0}^n A_k x_k^i
	\quad(i=0,1,2,\dotsc,m),
\end{equation*}
但是对于\(m+1\)次一元多项式就不准确成立,
即\begin{equation*}
	\int_a^b x^{m+1} \dd{x}
	= \frac{1}{m+2} (b^{m+2} - a^{m+2})
	\neq \sum_{k=0}^n A_k x_k^{m+1},
\end{equation*}
则称“求积公式 \labelcref{equation:数值积分.机械求积公式}
具有\(m\)次\DefineConcept{代数精度}”.
\end{definition}

显然,梯形公式 \labelcref{equation:数值积分.梯形公式}
和中矩形公式 \labelcref{equation:数值积分.中矩形公式}
均具有1次代数精度.

\subsection{插值型求积公式}
给定一组节点\begin{equation*}
	a \leq x_0 < x_1 < x_2 < \dotsb < x_n \leq b,
\end{equation*}
如果已知函数\(f\)在这些节点的值,
那么我们可以构造拉格朗日插值多项式\begin{equation*}
%@see: 《数值分析(第5版)》(李庆扬、王能超、易大义) P26 (2.9)
	L_n(x)
	= \sum_{k=0}^n y_k l_k(x),
\end{equation*}
其中\(y_k \defeq f(x_k)\ (k=0,1,2,\dotsc,n)\).
由于代数多项式\(L_n\)的原函数是容易求出的,
即\begin{equation*}
	\int_a^b L_n(x) \dd{x}
	= \int_a^b \sum_{k=0}^n y_k l_k(x) \dd{x}
	= \sum_{k=0}^n y_k \int_a^b l_k(x) \dd{x},
\end{equation*}
所以可以将\begin{equation*}
	I_n \defeq \int_a^b L_n(x) \dd{x}
\end{equation*}
作为积分\(I \defeq \int_a^b f(x) \dd{x}\)的近似值,
这时求积系数\(A_k\)可以通过插值基函数\(l_k\)积分得到的,
即\begin{equation}\label{equation:数值积分.插值型求积公式的求积系数}
%@see: 《数值分析(第5版)》(李庆扬、王能超、易大义) P100 (1.6)
	A_k \defeq \int_a^b l_k(x) \dd{x}
	\quad(k=0,1,2,\dotsc,n).
\end{equation}
我们把采用上述方法构造出来的求积公式\(
%@see: 《数值分析(第5版)》(李庆扬、王能超、易大义) P100 (1.5)
	\sum_{k=0}^n A_k f(x_k)
\)
称为“函数\(f\)的\(n\)阶\DefineConcept{(拉格朗日)插值型求积公式}”.

插值型求积公式的余项为\begin{equation}
%@see: 《数值分析(第5版)》(李庆扬、王能超、易大义) P101 (1.7)
	R[f] \defeq \int_a^b R_n(x) \dd{x},
\end{equation}
其中\begin{equation*}
%@see: 《数值分析(第5版)》(李庆扬、王能超、易大义) P26 (2.12)
%@see: 《数值分析(第5版)》(李庆扬、王能超、易大义) P26 (2.10)
% \cref{equation:拉格朗日插值.拉格朗日插值余项}
	R_n(x) \defeq \frac{f^{(n+1)}(\xi)}{(n+1)!} \omega_{n+1}(x)
	\quad(a<\xi<b),
	\qquad
	\omega_{n+1}(x)
	\defeq
	\prod_{i=0}^n (x - x_i).
\end{equation*}

如果求积公式 \labelcref{equation:数值积分.机械求积公式} 是一个插值型求积公式,
那么它至少具有\(n\)次代数精度,
这是因为对于任意一个次数不超过\(n\)的多项式函数\(f\),它的余项\(R[f]\)等于零.

反过来,如果求积公式 \labelcref{equation:数值积分.机械求积公式} 至少具有\(n\)次代数精度,
那么它一定是一个插值型求积公式,
这时求积公式 \labelcref{equation:数值积分.机械求积公式}
对于每一个插值基函数\(l_k\)都准确成立,
即\begin{equation*}
	\int_a^b l_k(x) \dd{x}
	= \sum_{j=0}^n A_j l_k(x_j).
\end{equation*}
注意到\(l_k(x_j) = \delta_{kj}\),
所以上式等号右端实际上等于\(A_k\),
从而有\cref{equation:数值积分.插值型求积公式的求积系数} 成立.

综上所述,我们有以下结论.
\begin{theorem}
%@see: 《数值分析(第5版)》(李庆扬、王能超、易大义) P101 定理1
求积公式 \labelcref{equation:数值积分.机械求积公式} 至少具有\(n\)次代数精度
的充分必要条件是
它是插值型求积公式.
\end{theorem}
