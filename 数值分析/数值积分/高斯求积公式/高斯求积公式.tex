\section{高斯求积公式}
\subsection{高斯求积公式的概念}
当我们用求积公式\(\sum_{k=0}^n A_k f(x_k)\)近似\(\int_a^b f(x) \dd{x}\)时,
\(\sum_{k=0}^n A_k f(x_k)\)实际上含有\(2(n+1)\)个待定参数:\begin{equation*}
	x_0,x_1,\dotsc,x_n
	\quad\text{和}\quad
	A_0,A_1,\dotsc,A_n.
\end{equation*}
只不过,在前几节中,我们讨论的是“机械”求积公式,
这类公式的特点是\(x_k\)是等距节点,
这就使\(x_0,x_1,\dotsc,x_n\)不再是待定参数,
这时插值型求积公式\(\sum_{k=0}^n A_k f(x_k)\)至少具有\(n\)次代数精度.
但是,如果适当选取求积节点,那么有可能使求积公式具有\(2n+1\)次代数精度.

下面我们首先考虑一个例子.
\begin{example}
%@see: 《数值分析(第5版)》(李庆扬、王能超、易大义) P117 例8
\setcounter{equation}{0}%
\renewcommand{\theequation}{\arabic{equation}}%
假设我们想要用求积公式\(A_0 f(x_0) + A_1 f(x_1)\)近似\(\int_{-1}^1 f(x) \dd{x}\),
假设该求积公式对\(x^n\ (n=0,1,2,3)\)精确成立,
则有\begin{gather}
%@see: 《数值分析(第5版)》(李庆扬、王能超、易大义) P117 (6.2)
	A_0 + A_1 = 2, \\
	A_0 x_0 + A_1 x_1 = 0, \\
	A_0 x_0^2 + A_1 x_1^2 = 2/3, \\
	A_0 x_0^3 + A_1 x_1^3 = 0.
\end{gather}
用(4)式减去(2)式乘\(x_0^2\),
得\begin{equation}
	A_1 x_1 (x_1^2 - x_0^2) = 0,
\end{equation}
由此得\(x_1 = \pm x_0\).

用\(x_0\)乘(1)式减(2)式,
得\begin{equation}
	A_1 (x_0 - x_1) = 2 x_0.
\end{equation}
用(3)式减去\(x_0\)乘(2)式,
得\begin{equation}
	A_1 x_1 (x_1 - x_0) = 2/3.
\end{equation}
将(6)式代入(7)式,得\begin{equation}
	x_0 x_1 = -1/3.
\end{equation}
由此可知\(x_0\)与\(x_1\)异号,故\begin{equation}
	x_1 = - x_0.
\end{equation}
将(9)式代回(8)式,得\(x_1^2 = 1/3\),
于是可以取\(x_0 = -\sqrt3/3\)和\(x_1 = \sqrt3/3\),
那么\begin{equation}
	A_1 = \frac{2 x_0}{x_0 - x_1}
	= 1.
\end{equation}
再将上式代回(1)式,得\(A_0 = 1\).

综上所述,我们可以用\(f(-\sqrt3/3) + f(\sqrt3/3)\)近似\(\int_{-1}^1 f(x) \dd{x}\).

当\(f(x) = x^4\)时,
有\begin{equation*}
	f(-\sqrt3/3) + f(\sqrt3/3)
	= 2/9
	\neq
	\int_{-1}^1 f(x) \dd{x}
	= 2/5,
\end{equation*}
所以求积公式\(f(-\sqrt3/3) + f(\sqrt3/3)\)的代数精度为\(3\).
\end{example}

实际上求积公式\(A_0 f(x_0) + A_1 f(x_1)\)的代数精度不可能超过\(3\),
这是因为对于任意给定的两个积分节点\(x_0,x_1 \in [-1,1]\),
4次多项式函数\(f(x) \defeq (x - x_0)^2 (x - x_1)^2\)的积分为\begin{equation*}
	\int_{-1}^1 f(x) \dd{x}
	% = 2 x_0^2 x_1^2 + \frac23 x_0^2 + \frac83 x_0 x_1 + \frac23 x_1^2 + \frac25
	= \frac23 \left( x_0 + x_1 \right)^2
	+ 2 \left( x_0 x_1 + \frac13 \right)^2
	+ \frac{8}{45}
	> 0,
\end{equation*}
而\(f(x_0) = f(x_1) = 0\),
故近似值为\(A_0 f(x_0) + A_1 f(x_1) = 0\),
它表明两个节点的求积公式的最高代数精度为\(3\).
以此类推,我们可以如下结论.
\begin{theorem}
含\(n+1\)个节点的求积公式\(\sum_{k=0}^n A_k f(x_k)\)的代数精度最高为\(2n+1\)次.
%TODO
\end{theorem}

\begin{definition}
%@see: 《数值分析(第5版)》(李庆扬、王能超、易大义) P118 定义4
如果在近似积分\(\int_a^b \rho(x) f(x) \dd{x}\)时,
求积公式\(\sum_{k=0}^n A_k f(x_k)\)具有\(2n+1\)次代数精度,
则把它称为\DefineConcept{高斯求积公式},
把它的节点\(x_k\ (k=0,1,2,\dotsc,n)\)称为\DefineConcept{高斯点}.
\end{definition}

根据上述定义,要使求积公式具有\(2n+1\)次代数精度,
则必定成立\begin{equation}\label{equation:高斯求积公式.高斯求积公式的待定参数方程组}
%@see: 《数值分析(第5版)》(李庆扬、王能超、易大义) P118 (6.5)
	\sum_{k=0}^n A_k f(x_k)
	= \int_a^b \rho(x) x^m \dd{x}
	\quad(m=0,1,2,\dotsc,2n,2n+1).
\end{equation}
理论上,只要给定权函数\(\rho\),就可以求出右边的积分,
再根据上面的方程组解出\(A_k\)和\(x_k\).
但是由于上式是关于\(A_k\)和\(x_k\)的非线性方程组,
在\(n>1\)时求解就已经很困难了,
因此我们通常是在已经提前确定了节点\(x_k\)的位置以后,
再根据上述方程组求解\(A_k\),
这时它变成关于\(A_k\)的线性方程组.
下面讨论如何选取节点\(x_k\),才能使求积公式具有\(2n+1\)次代数精度.

\subsection{利用插值多项式确定求积节点}
给定\(n+1\)个节点\(a = x_0 < x_1 < \dotsb < x_n = b\),
被积函数\(f\)的拉格朗日插值多项式为\begin{equation*}
	L_n(x) \defeq \sum_{k=0}^n f(x_k) ~ l_k(x),
\end{equation*}
其中\begin{equation*}
	l_k(x) \defeq \prod_{0 \leq i \leq n, i \neq k} \frac{x - x_i}{x_k - x_i}
	\quad(k=0,1,2,\dotsc,n),
\end{equation*}
而插值余项为\begin{equation*}
	R_n(x) \defeq \frac{1}{(n+1)!} ~ f^{(n+1)}(\xi(x)) ~ \omega_{n+1}(x)
	\quad(a < \xi(x) < b),
\end{equation*}
其中\begin{equation*}
	\omega_{n+1}(x)
	\defeq
	\prod_{i=0}^n (x - x_i),
\end{equation*}
则\begin{equation*}
	f(x) = L_n(x) + R_n(x).
\end{equation*}
用\(\rho(x)\)乘上式,并积分,得\begin{equation*}
	\int_a^b \rho(x) ~ f(x) \dd{x}
	= \int_a^b \rho(x) ~ L_n(x) \dd{x}
	+ \int_a^b \rho(x) ~ R_n(x) \dd{x},
\end{equation*}
再令\begin{equation*}
	A_k \defeq \int_a^b \rho(x) ~ l_k(x) \dd{x},
\end{equation*}
则\begin{equation*}
%@see: 《数值分析(第5版)》(李庆扬、王能超、易大义) P118 (6.6)
	\int_a^b \rho(x) ~ f(x) \dd{x}
	= \sum_{k=0}^n A_k f(x_k) + R[f],
\end{equation*}
其中\begin{equation*}
	R[f] \defeq \frac{1}{(n+1)!} \int_a^b \rho(x) ~ f^{(n+1)}(\xi(x)) ~ \omega_{n+1}(x) \dd{x}
\end{equation*}
是求积余项.

因为\(m \le n < n+1\),
所以\(f\)的\(n+1\)阶导数\(f^{(n+1)}\)在整个区间\([a,b]\)上恒为零.
因此,当\(f(x) = x^m\ (m=0,1,2,\dotsc,n)\)时,有\(R[f] = 0\),
此时有\begin{equation*}
	\int_a^b \rho(x) ~ f(x) \dd{x}
	= \sum_{k=0}^n A_k f(x_k).
\end{equation*}
这就说明求积公式至少具有\(n\)次代数精度.

假设求积公式具有\(2n+1\)次代数精度,
那么对于\(f(x) \defeq x^{2n+1}\)应该有\(R[f] = 0\).
因为\begin{equation*}
	f^{(n+1)}(x) = \frac{(2n+1)!}{n!} x^n,
\end{equation*}
再由\cref{example:牛顿插值.幂函数的插值余项的中值} 有\begin{equation*}
%@credit: {gemini}
	\xi(x)
	= \left[
		\frac{n! (n+1)!}{(2n+1)!}
		\sum_{\substack{p_0 + p_1 + \dotsb + p_n + p = n \\ p_0,p_1,\dotsc,p_n,p \geq 0}}
		x_0^{p_0} x_1^{p_1} \dotsm x_n^{p_n} x^p
	\right]^{1/n},
\end{equation*}
所以\begin{equation*}
	f^{(n+1)}(\xi(x))
	= (n+1)!
		\sum_{\substack{p_0 + p_1 + \dotsb + p_n + p = n \\ p_0,p_1,\dotsc,p_n,p \geq 0}}
		x_0^{p_0} x_1^{p_1} \dotsm x_n^{p_n} x^p
\end{equation*}
是一个\(n\)次多项式函数.
因此,要使求积公式具有\(2n+1\)次代数精度,
必须有\begin{equation*}
	R[f]
	= \sum_{\substack{p_0 + p_1 + \dotsb + p_n + p = n \\ p_0,p_1,\dotsc,p_n,p \geq 0}}
			x_0^{p_0} x_1^{p_1} \dotsm x_n^{p_n}
			\int_a^b \rho(x) ~ x^p ~ \omega_{n+1}(x) \dd{x}
	= 0.
\end{equation*}
可以看出,只要有\begin{equation*}
	\int_a^b \rho(x) ~ x^p ~ \omega_{n+1}(x) \dd{x} = 0
	\quad(p=0,1,2,\dotsc,n),
\end{equation*}
便有\(R[f] = 0\).
这就要求\(\omega_{n+1}\)与每一个次数不超过\(n\)的幂函数
在区间\([a,b]\)上带权\(\rho\)正交.
综上所述,我们可以得出如下结论.
\begin{theorem}
%@see: 《数值分析(第5版)》(李庆扬、王能超、易大义) P119 定理5
设函数\(f \in C[a,b]\),
给定常数\begin{equation}\label{equation:高斯求积公式.高斯求积公式的求积系数}
	A_k \defeq \int_a^b \rho(x) ~ l_k(x) \dd{x}
	\quad(k=0,1,2,\dotsc,n),
\end{equation}
其中\(l_k\ (k=0,1,2,\dotsc,n)\)是节点\(x_0,x_1,\dotsc,x_n\)上的
\(k\)次\hyperref[equation:拉格朗日插值.拉格朗日插值基函数]{插值基函数},
则求积公式\(\sum_{k=0}^n A_k f(x_k)\)是高斯求积公式,
当且仅当\(
	\omega_{n+1}(x)
	\defeq
	\prod_{i=0}^n (x - x_i)
\)
与任意一个次数不超过\(n\)的多项式函数\(p\)在区间\([a,b]\)上带权\(\rho\)正交,
即\begin{equation*}
%@see: 《数值分析(第5版)》(李庆扬、王能超、易大义) P119 (6.7)
	\int_a^b \rho(x) ~ p(x) ~ \omega_{n+1}(x) \dd{x} = 0.
\end{equation*}
%TODO proof
\end{theorem}

上述定理表明:区间\([a,b]\)上带权\(\rho\)的\(n+1\)次正交多项式的零点
就是求积公式\(\sum_{k=0}^n A_k f(x_k)\)的高斯点.
于是,在已知求积节点\(x_0,x_1,\dotsc,x_n\)以后,
高斯求积公式的求积系数\(A_0,A_1,\dotsc,A_n\)有两种计算方法:
要么解方程组 \labelcref{equation:高斯求积公式.高斯求积公式的待定参数方程组},
要么利用\cref{equation:高斯求积公式.高斯求积公式的求积系数} 计算得到.

% \begin{example}
% %@see: 《数值分析(第5版)》(李庆扬、王能超、易大义) P120 例9
% 确定高斯求积公式\begin{equation*}
% 	\int_0^1 \sqrt{x} f(x) \dd{x}
% 	\approx A_0 f(x_0) + A_1 f(x_1)
% \end{equation*}
% 的系数\(A_0,A_1\)和节点\(x_0,x_1\).
% \end{example}

下面讨论高斯求积公式的稳定性和收敛性.
\begin{theorem}
%@see: 《数值分析(第5版)》(李庆扬、王能超、易大义) P121 定理6
高斯求积公式的求积系数\(A_0,A_1,\dotsc,A_n\)都是正数.
%TODO proof
\end{theorem}

\begin{corollary}
%@see: 《数值分析(第5版)》(李庆扬、王能超、易大义) P121 推论
高斯求积公式是稳定的.
%TODO proof
\end{corollary}

\begin{theorem}
%@see: 《数值分析(第5版)》(李庆扬、王能超、易大义) P121 定理7
设\(f \in C[a,b]\),
则高斯求积公式是收敛的,
即\begin{equation*}
	\lim_{n\to\infty} \sum_{k=0}^n A_k f(x_k)
	= \int_a^b \rho(x) f(x) \dd{x}.
\end{equation*}
%TODO proof
\end{theorem}

高斯求积公式与机械求积公式的最大区别:
机械求积公式中未知待定的量仅有求积系数,
高斯求积公式中未知待定的量除了求积系数以外还有求积节点.

\subsection{高斯--勒让德求积公式}
我们知道勒让德多项式是区间\([-1,1]\)上的正交多项式.
如果我们要求的定积分中,
权函数为\(\rho(x) \defeq 1\),积分区间为\([-1,1]\),
那么可以将勒让德多项式\(P_{n+1}\)的零点
\(x_0,x_1,\dotsc,x_n\)作为求积节点,
于是求积公式\begin{equation*}
	\int_{-1}^1 f(x) \dd{x}
	\approx
	\sum_{k=0}^n A_k f(x_k)
\end{equation*}
就一定是一个高斯求积公式,
这时我们把它称为\DefineConcept{高斯--勒让德求积公式}.

% 非线性方程组的求解:因式分解、替换、消元法

% 高斯--勒让德求积公式 重要

% 辛普森公式对节点的取法没有要求
% 高斯求积公式对节点的取法有要求,节点必须是某个正交多项式(例如勒让德多项式、第一类切比雪夫多项式等)的零点
% 高斯求积公式的精度比辛普森公式的精度更高,所需的节点数更少
