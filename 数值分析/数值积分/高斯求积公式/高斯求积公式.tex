\section{高斯求积公式}
当我们用求积公式\(\sum_{k=0}^n A_k f(x_k)\)近似\(\int_a^b f(x) \dd{x}\)时,
\(\sum_{k=0}^n A_k f(x_k)\)实际上含有\(2(n+1)\)个待定参数:\begin{equation*}
	x_0,x_1,\dotsc,x_n
	\quad\text{和}\quad
	A_0,A_1,\dotsc,A_n.
\end{equation*}
只不过,在前几节中,我们讨论的是“机械”求积公式,
这类公式的特点是\(x_k\)是等距节点,
这就使\(x_0,x_1,\dotsc,x_n\)不再是待定参数,
这时插值型求积公式\(\sum_{k=0}^n A_k f(x_k)\)至少具有\(n\)次代数精度.
但是,如果适当选取求积节点,那么有可能使求积公式具有\(2n+1\)次代数精度.

下面我们首先考虑一个例子.
\begin{example}
\setcounter{equation}{0}%
\renewcommand{\theequation}{\arabic{equation}}%
假设我们想要用求积公式\(A_0 f(x_0) + A_1 f(x_1)\)近似\(\int_{-1}^1 f(x) \dd{x}\),
假设该求积公式对\(x^n\ (n=0,1,2,3)\)精确成立,
则有\begin{gather}
%@see: 《数值分析(第5版)》(李庆扬、王能超、易大义) P117 (6.2)
	A_0 + A_1 = 2, \\
	A_0 x_0 + A_1 x_1 = 0, \\
	A_0 x_0^2 + A_1 x_1^2 = 2/3, \\
	A_0 x_0^3 + A_1 x_1^3 = 0.
\end{gather}
用(4)式减去(2)式乘\(x_0^2\),
得\begin{equation}
	A_1 x_1 (x_1^2 - x_0^2) = 0,
\end{equation}
由此得\(x_1 = \pm x_0\).

用\(x_0\)乘(1)式减(2)式,
得\begin{equation}
	A_1 (x_0 - x_1) = 2 x_0.
\end{equation}
用(3)式减去\(x_0\)乘(2)式,
得\begin{equation}
	A_1 x_1 (x_1 - x_0) = 2/3.
\end{equation}
将(6)式代入(7)式,得\begin{equation}
	x_0 x_1 = -1/3.
\end{equation}
由此可知\(x_0\)与\(x_1\)异号,故\begin{equation}
	x_1 = - x_0.
\end{equation}
将(9)式代回(8)式,得\(x_1^2 = 1/3\),
于是可以取\(x_0 = -\sqrt3/3\)和\(x_1 = \sqrt3/3\),
那么\begin{equation}
	A_1 = \frac{2 x_0}{x_0 - x_1}
	= 1.
\end{equation}
再将上式代回(1)式,得\(A_0 = 1\).

综上所述,我们可以用\(f(-\sqrt3/3) + f(\sqrt3/3)\)近似\(\int_{-1}^1 f(x) \dd{x}\).

当\(f(x) = x^4\)时,
有\begin{equation*}
	f(-\sqrt3/3) + f(\sqrt3/3)
	= 2/9
	\neq
	\int_{-1}^1 f(x) \dd{x}
	= 2/5,
\end{equation*}
所以求积公式\(f(-\sqrt3/3) + f(\sqrt3/3)\)的代数精度为\(3\).
\end{example}




高斯求积公式与机械求积公式的最大区别:
机械求积公式中未知待定的量仅有求积系数,
高斯求积公式中未知待定的量除了求积系数以外还有求积节点.

% 非线性方程组的求解:因式分解、替换、消元法

% 高斯--勒让德求积公式 重要

% 辛普森公式对节点的取法没有要求
% 高斯求积公式对节点的取法有要求,节点必须是某个正交多项式(例如勒让德多项式、第一类切比雪夫多项式等)的零点
% 高斯求积公式的精度比辛普森公式的精度更高,所需的节点数更少
