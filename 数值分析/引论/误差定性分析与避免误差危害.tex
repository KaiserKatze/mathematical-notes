\section{误差定性分析与避免误差危害}
数值运算中的误差分析是个很重要又很复杂的问题.
上一节我们虽然讨论不精确数据运算结果的误差限,
但是它只适用于简单情形,
然而一个工程或科学计算问题往往需要运算成千上万次,
由于每一步运算都有误差,
所以每一步都做误差分析是不可能的,也是不科学的,
这是因为误差积累有正有负,
绝对值有大有小,
都按最坏情况估计误差限,
得到的结果就会比实际误差大很多,
这种保守的误差估计不能反映实际误差积累.
考虑到误差分布的随机性,
有人用概率统计方法,
将数据和运算中的舍入误差视为服从某种分布的随机变量,
然后确定计算结果的误差分布,
这样得到的误差估计更接近实际.
我们把这种方法称为\DefineConcept{概率分析法}.

\subsection{算法的数值稳定性}
%@see: 《数值分析(第5版)》(李庆扬、王能超、易大义) P10 定义3
用一个算法进行计算,
如果初始数据或输入数据的误差在计算中传播,使得计算结果的舍入误差增长很快,
那么就说这个算法是\DefineConcept{数值不稳定的};
反之,如果计算结果的舍入误差不增长,
则称这个算法是\DefineConcept{数值稳定的}.

\subsection{病态问题与条件数}
给定一个数值问题,
输入数据的微小扰动(即误差)
引起输出数据(即问题解)相对误差很大,
那么这个数值问题就叫做\DefineConcept{病态问题}.
例如,计算函数值\(f(x)\)时,
假设\(x\)有扰动\(\increment x = x^* - x\),
\(x\)的相对误差为\(\frac{\increment x}{x}\),
函数值\(f(x)\)的相对误差为\(\frac{f(x^*) - f(x)}{f(x)}\),
又假设差商\(\frac{f(x^*) - f(x)}{\increment x}\)约等于微商\(f'(x)\),
那么\(f(x)\)与\(x\)的相对误差比值为\begin{equation*}
%@see: 《数值分析(第5版)》(李庆扬、王能超、易大义) P10 (3.3)
	\abs{\frac{f(x^*) - f(x)}{f(x)}}
	\bigg/
	\abs{\frac{\increment x}{x}}
	= \abs{\frac{x}{f(x)} \cdot \frac{f(x^*) - f(x)}{\increment x}}
	\approx \abs{\frac{x f'(x)}{f(x)}}.
\end{equation*}
我们把\begin{equation*}
	C_p
	\defeq
	\abs{\frac{x f'(x)}{f(x)}}
\end{equation*}
称为计算函数值问题的\DefineConcept{条件数}.
自变量相对误差\(\frac{\increment x}{x}\)一般不会太大.
如果条件数\(C_p\)很大,
将引起函数值相对误差\(\frac{f(x^*) - f(x)}{f(x)}\)很大,
出现这种情况的问题就是病态问题.

例如,取\(f(x) \defeq x^n\),
则\(C_p = \abs{\frac{x \cdot n x^{n-1}}{x^n}} = n\),
这表示计算结果的相对误差相交于输入数据可能放大\(n\)倍.
具体地,取\(n \defeq 10\),
令准确值\(x \defeq 1\),令近似值\(x^* \defeq 1.02\),
则自变量相对误差为\(2\%\),
函数值相对误差为\(24\%\),
这时\(f(x)\)的计算问题可以认为是病态的.

一般地,当条件数\(C_p \geq 10\)时,
我们就认为问题是病态的,
并且\(C_p\)越大病态越严重.

\begin{example}

%@see: 《数值分析(第5版)》(李庆扬、王能超、易大义) P11 例6
求解线性方程组\begin{equation*}
	\begin{cases}
		x + \alpha y = 1, \\
		\alpha x + y = 0.
	\end{cases}
\end{equation*}
\begin{solution}
当\(\alpha = 1\)时,系数行列式为零,方程无解.
但是当\(\alpha \neq 1\)时,解得\begin{equation*}
	x = \frac{1}{1-\alpha^2},
	\qquad
	y = -\frac{\alpha}{1-\alpha^2}.
\end{equation*}
当\(\alpha \approx 1\)时,若输入数据\(\alpha\)有微小扰动(误差),则解的误差就会很大》
实际上,把\(x = \frac{1}{1-\alpha^2}\)看成\(\alpha\)的函数,
那么条件数为\begin{equation*}
	C_p = \abs{\frac{\alpha x'(\alpha)}{x(\alpha)}}
	= \abs{\frac{2\alpha^2}{1-\alpha^2}}.
\end{equation*}
假如输入数据的准确值为\(\alpha = 0.99\),
那么条件数为\(C_p \approx 100\),
因此本问题是病态的.
\end{solution}
\end{example}

应该注意到的是,病态问题不是计算方法引起的,
而是数值问题自身固有的,
因此,对数值问题首先要分清问题是否病态,
对于病态问题就必须采取相应的特殊方法以减少误差危害.

\subsection{避免误差危害}
数值计算中,通常不采用数值不稳定算法.
在设计算法时,还应该尽量避免误差危害,防止有效数字损失.
通常要避免两个相近的数相减、用绝对值很小的数做除数,
还要注意运算次序和减少运算次数.

\begin{example}
计算关于\(x\)的方程\(x^2-16x+1=0\)的数值解.
\begin{solution}
我们首先写出方程\(x^2-16x+1=0\)的解析解:\(
	x_1 = 8 + \sqrt{63},
	x_2 = 8 - \sqrt{63}
\),
其中\(x_1\)的数值很容易计算,
但是\(x_2\)的数值计算就需要特殊处理了.
显然\(x_2\)的数值有两种计算方式,
一种是直接计算减法,得到\(x_2 \approx 8 - 7.94 = 0.06\),
这里计算结果只有一位有效数字;
另一种是对解析式作等值变形,得到\(x_2 = \frac{1}{8 + \sqrt{63}} \approx \frac{1}{15.94} \approx 0.0627\),
现在计算结果就有三位有效数字了.
可以看出,第一种算法涉及两个相近数相减,产生了很大的误差,
反观第二种算法,通过等值变形,误差得以减小.
\end{solution}
\end{example}

一般地,如果\(x_1\)和\(x_2\)很接近,
则可以使用\(\log\frac{x_1}{x_2}\)代替\(\log x_1 - \log x_2\);
如果\(x\)很大,
则可以使用\(\frac{1}{\sqrt{x+1} - \sqrt{x}}\)代替\(\sqrt{x+1} - \sqrt{x}\);
如果\(f(x)\)和\(f(x^*)\)很接近,
则可以使用次数足够大的泰勒多项式\(f'(x^*) (x-x^*) + \frac{f''(x^*)}{2} (x-x^*)^2 + \dotsb\)
代替\(f(x) - f(x^*)\).
如果计算公式实在无法变形,那么在计算前就应该增加输入数据的有效位数.
在计算机上进行计算时,为了增加计算精度,就应该选用双精度浮点数,而非单精度浮点数.
