\section{数值计算的误差}
\subsection{误差来源与分类}
用计算机解决科学计算问题首先要建立数学模型,
它是对被描述的实际问题进行抽象、简化得到的,因而是近似的.
我们把数学模型与实际问题之间出现的这种误差,
称为\DefineConcept{模型误差}.

在数学模型中往往还有一些根据观测得到的物理量,
例如温度、长度、电压等,
这些参量显然也包含误差.
这种由观测产生的误差称为\DefineConcept{观测误差}.

当数学模型不能得到精确解时,
通常需要用数值方法求它的近似解,
其近似解与精确解之间的误差称为\DefineConcept{截断误差}或\DefineConcept{方法误差}.

有了求解数学问题的计算公式以后,
就可以用计算机做数值计算了.
但是由于计算机的字长有限,
在计算机上表示原始数据、中间计算数据和输出数据,
以及将二进制数据转化为十进制数据,或将十进制数据转化为二进制数据时,
都会产生\DefineConcept{舍入误差}.

研究计算结果的误差是否满足精度要求就是误差估计问题.
数值分析主要讨论算法的截断误差、舍入误差.

\subsection{误差与有效数字}
%@see: 《数值分析(第5版)》(李庆扬、王能超、易大义) P4 定义1
假设\(x\)是准确值,\(x^*\)是\(x\)的一个近似值,
则称\begin{equation}
	e^* \defeq x^* - x
\end{equation}
是近似值的\DefineConcept{绝对误差}.

通常我们不能算出准确值\(x\),
也不能算出误差\(e^*\)的准确值,
只能根据测量工具或计算情况估计出误差的绝对值不超过某个整数\(\epsilon^*\),
也就是误差绝对值的一个上界.
我们把\(\epsilon^*\)称为近似值的\DefineConcept{绝对误差限},
它总是正数.
例如,用毫米刻度的米尺测量一根棍子的长度,假设棍子长度的准确值是\(x\),
测量时读出的与棍子长度接近的刻度是\(x^*\),
\(x^*\)是\(x\)的近似值,
它的绝对误差限是\(\qty{0.5}{\milli\meter}\),
可知\(\abs{x^* - x} \leq \qty{0.5}{\milli\meter}\).
如果读出的长度是\(\qty{765}{\milli\meter}\),
则有\(\abs{765-x} \leq 0.5\),
从这个不等式我们仍不知道准确的\(x\)是多少,
但是可以得出结论\(764.5 \leq x \leq 765.5\).
对于一般情形,我们常常把不等式\begin{equation}
	\abs{x^* - x} \leq \epsilon^*
\end{equation}
写成\begin{equation}
	x = x^* \pm \epsilon^*.
\end{equation}

绝对误差限的大小还不能完全表示近似值的好坏.
例如有两个量\(x = 10\pm1\)和\(y = 1000\pm5\),
虽然后者的绝对误差限是前者的5倍,
但是\(5/1000 = 0.5\%\)比\(1/10 = 10\%\)要小得多,
这说明\(y^*\)近似\(y\)的程度
比\(x^*\)近似\(x\)的程度要好得多.
也就是说,除了考虑误差的大小以外,
还应该考虑准确值\(x\)本身的大小.

我们把准确值的误差\(e^*\)与准确值\(x\)的比值\begin{equation}
	e^*_r
	\defeq \frac{e^*}{x}
	= \frac{x^* - x}{x}
\end{equation}
称为近似值\(x^*\)的\DefineConcept{相对误差}.

在实际计算中,由于真值\(x\)总是不知道的,
当\(e^* / x^*\)较小时,
通常取\begin{equation}
	e^*_r
	\defeq \frac{e^*}{x^*}
	= \frac{x^* - x}{x^*}
\end{equation}
作为\(x^*\)的相对误差,
此时\begin{equation*}
	\frac{e^*}{x}
	- \frac{e^*}{x^*}
	= \frac{e^* (x^* - x)}{x^* x}
	= \frac{(e^*)^2}{x^* (x^* - e^*)}
	= \frac{(e^* / x^*)^2}{1 - (e^* / x^*)}
\end{equation*}
是\(e^*_r\)的平方项级,故可忽略不计.

相对误差\((e^* / x^*)\)跟绝对误差\(e^*\)一样,也可正可负.
我们把相对误差的绝对值上界称为\DefineConcept{相对误差限},
记作\(\epsilon^*_r\),
即\begin{equation}
	\epsilon^*_r
	\defeq \frac{\epsilon^*}{\abs{x^*}}.
\end{equation}

根据定义,\(10\pm1\)和\(1000\pm5\)的相对误差限分别是\(10\%\)和\(0.5\%\),
由此可见后者的近似程度比前者的近视程度好.

需要注意的是,
绝对误差、绝对误差限是有量纲的,
相对误差、相对误差限是无量纲的.

在\(p\)进制下,
当准确值\(x\)有多位数时,
常常按四舍五入的原则得到\(x\)的前几位近似值\(x^*\).
如果近似值\(x^*\)的绝对误差限是某一位的半个单位,
且该位到\(x^*\)的第一位非零数字共有\(n\)位,
则称\(x^*\)有\(n\)位\DefineConcept{有效数字}.
我们常常把\(x^*\)表示为\begin{equation*}
	x^*
	= \pm p^m \times (
		a_1
		+ a_2 \times p^{-1}
		+ \dotsb
		+ a_n \times p^{-(n-1)}
	),
\end{equation*}
其中\(a_i\ (i=1,2,\dotsc,n)\)
是\(0\)到\(p-1\)中的一个数字,
\(a_1\neq0\),
\(m\)是整数,
且\begin{equation*}
	\abs{x - x^*}
	\leq \frac12 \times p^{m-n+1}.
\end{equation*}

\begin{example}
%@see: 《数值分析(第5版)》(李庆扬、王能超、易大义) P6 例1
按四舍五入原则写出下列十进制数字的具有5位有效数字的近似数:\begin{equation*}
	187.932~5, \qquad
	0.037~855~51, \qquad
	8.000~033, \qquad
	2.718~281~8.
\end{equation*}
\begin{solution}
根据定义,上述十进制数字的具有5位有效数字的近似数分别是:\begin{equation*}
	187.93, \qquad
	0.037~856, \qquad
	8.000~0, \qquad
	2.718~3.
\end{equation*}
\end{solution}
\end{example}

关于有效数字与相对误差限的关系,有下面的定理.
\begin{theorem}\label{theorem:数值分析.数值计算误差.有效数字与相对误差限的关系}
%@see: 《数值分析(第5版)》(李庆扬、王能超、易大义) P6 定理1
给定近似数\(x^*\),用\(p\)进制下的科学计数法可以将\(x^*\)表示为\begin{equation*}
	x^*
	= \pm p^m \times (
		a_1
		+ a_2 \times p^{-1}
		+ \dotsb
		+ a_n \times p^{-(n-1)}
	),
\end{equation*}
其中\(a_i\ (i=1,2,\dotsc,n)\)
是\(0\)到\(p-1\)中的一个数字,
\(a_1\neq0\),
\(m\)是整数.
\begin{itemize}
	\item 如果\(x^*\)具有\(n\)位有效数字,
	则\(x^*\)的相对误差限为\begin{equation}
		\epsilon^*_r
		\leq \frac1{2 a_1} \times p^{-(n-1)}.
	\end{equation}

	\item 如果\(x^*\)的相对误差限为\begin{equation*}
		\epsilon^*_r
		\leq \frac1{2(a_1+1)} \times p^{-(n-1)},
	\end{equation*}
	则\(x^*\)至少具有\(n\)位有效数字.
\end{itemize}
\end{theorem}
\cref{theorem:数值分析.数值计算误差.有效数字与相对误差限的关系} 说明,
一个近似数的有效位数越多,它的相对误差限就越小.

\begin{example}
%@see: 《数值分析(第5版)》(李庆扬、王能超、易大义) P7 例3
在十进制下,要使\(\sqrt{20}\)的近似值的相对误差限小于\(0.1\%\),应该取几位有效数字?
\begin{solution}
假设应取\(n\)位有效数字,
那么由\cref{theorem:数值分析.数值计算误差.有效数字与相对误差限的关系} 可知\begin{equation*}
	\epsilon^*_r \leq \frac1{2a_1} \times 10^{-(n-1)}.
\end{equation*}
由于\(4 < \sqrt{20} < 5\),
所以\(a_1 = 4\),
故只要取\(n = 4\),就有\begin{equation*}
	\epsilon^*_r
	\leq 0.125 \times 10^{-3}
	< 10^{-3}
	= 0.1\%,
\end{equation*}
因此,只要对\(\sqrt{20}\)的近似值取4位有效数字,
即\(\sqrt{20} \approx 4.472\),
其相对误差限就小于\(0.1\%\).
\end{solution}
\end{example}

\subsection{数值运算的误差估计}
设两个近似数\(x^*_1\)与\(x^*_2\)的误差限分别是\(\epsilon(x^*_1)\)与\(\epsilon(x^*_2)\),
则它们进行加减乘除运算,得到的误差限分别满足不等式\begin{gather*}
	\epsilon(x^*_1 \pm x^*_2)
	\leq \epsilon(x^*_1) + \epsilon(x^*_2), \\
	\epsilon(x^*_1 x^*_2)
	\leq \abs{x^*_1} \epsilon(x^*_2) + \abs{x^*_2} \epsilon(x^*_1), \\
	\epsilon(x^*_1 / x^*_2)
	\leq \frac{\abs{x^*_1} \epsilon(x^*_2) + \abs{x^*_2} \epsilon(x^*_1)}{\abs{x^*_2}^2}
	\quad(x^*_2 \neq 0).
\end{gather*}

更一般的情况是,当自变量有误差时,计算函数值也产生误差,
其误差值可以利用函数的泰勒展开式进行估计.

设\(f\)是一元可微函数,
\(x^*\)是\(x\)的近似值,
由\begin{equation*}
	% \cref{equation:微分中值定理.泰勒公式.多项式1}
	% \cref{equation:微分中值定理.泰勒公式.余项1}
	f(x) = f(x^*) + f'(x^*) (x-x^*)
	+ \dotsb
	+ \frac{f^{(n)}(x^*)}{n!} (x-x^*)^n
	+ \frac{f^{(n+1)}(\xi)}{(n+1)!} (x-x^*)^{n+1}
	\footnote{
		这里不写成\(f(x^*)\)的泰勒公式,是因为那样做会导致泰勒多项式中出现\(f'(x),f''(x)\)等含有未知真值的表达式.
		我们现在采取的写法,则\(f'(x^*),f''(x^*)\)等都是已知的,剩下的\(x-x^*\)可以通过放缩用绝对误差限代替.
	}
\end{equation*}
可知\begin{equation*}
	% \(f(x^*) - f(x)\)才是用\(f(x^*)\)近似\(f(x)\)产生的绝对误差,这里是绝对误差的相反数
	f(x) - f(x^*)
	= f'(x^*) (x-x^*)
	+ \dotsb
	+ \frac{f^{(n)}(x^*)}{n!} (x-x^*)^n
	+ \frac{f^{(n+1)}(\xi)}{(n+1)!} (x-x^*)^{n+1},
\end{equation*}
其中\(\xi\)在\(x\)与\(x^*\)之间.
对上式取绝对值,
考虑到\(\abs{x^* - x} \leq \epsilon^* \defeq \epsilon(x^*)\),
得\begin{align*}
	\abs{f(x^*) - f(x)}
	&= \abs{f'(x^*)} \abs{x-x^*}
	+ \dotsb
	+ \abs{\frac{f^{(n)}(x^*)}{n!}} \abs{x-x^*}^n
	+ \abs{\frac{f^{(n+1)}(\xi)}{(n+1)!}} \abs{x-x^*}^{n+1} \\
	&\leq \abs{f'(x^*)} \epsilon(x^*)
	+ \dotsb
	+ \abs{\frac{f^{(n)}(x^*)}{n!}} \epsilon^n(x^*)
	+ \abs{\frac{f^{(n+1)}(\xi)}{(n+1)!}} \epsilon^{n+1}(x^*).
\end{align*}
我们通常只需要按\((x-x^*)\)的幂展开的1阶泰勒公式,即\begin{equation*}
	\abs{f(x^*) - f(x)}
	\leq \abs{f'(x^*)} \epsilon(x^*)
	+ \frac{\abs{f''(\xi)}}{2!} \epsilon^2(x^*).
\end{equation*}
当\(f'(x^*)\)与\(f''(x^*)\)的比值不太大时,
我们还可以忽略\(\epsilon(x^*)\)的高次项,
于是得到计算函数的绝对误差限为\begin{equation}
	\epsilon(f(x^*))
	\approx \abs{f'(x^*)} \epsilon(x^*).
\end{equation}

设\(f\)是多元可微函数,
\(x^*_1,\dotsc,x^*_m\)分别是\(x_1,\dotsc,x_m\)的近似值,
那么由\hyperref[theorem:多元函数微分法.多元函数的泰勒公式]{多元函数的泰勒公式}可知,
用\(A^* \defeq f(x^*_1,\dotsc,x^*_m)\)近似\(A \defeq f(x_1,\dotsc,x_m)\)产生的绝对误差为\begin{align*}
	A^* - A
	&= f(x^*_1,\dotsc,x^*_m) - f(x_1,\dotsc,x_m) \\
	&\approx \sum_{k=1}^m \left( \pdv{f(x^*_1,\dotsc,x^*_m)}{x_k} \right) (x^*_k - x_k)
	= \sum_{k=1}^m \left( \pdv{f}{x_k} \right)^* e^*_k,
\end{align*}
于是绝对误差限为\begin{equation}\label{equation:数值计算的误差.数值运算的绝对误差限}
%@see: 《数值分析(第5版)》(李庆扬、王能超、易大义) P8 (2.3)
	\epsilon(A^*)
	\approx \sum_{k=1}^m \abs{\left( \pdv{f}{x_k} \right)^*} \epsilon(x^*_k),
\end{equation}
而\(A^*\)的相对误差限为\begin{equation}\label{equation:数值计算的误差.数值运算的相对误差限}
%@see: 《数值分析(第5版)》(李庆扬、王能超、易大义) P8 (2.4)
	\epsilon^*_r
	= \epsilon_r(A^*)
	= \frac{\epsilon(A^*)}{\abs{A^*}}
	\approx \sum_{k=1}^m \abs{\left( \pdv{f}{x_k} \right)^*} \frac{\epsilon(x^*_k)}{\abs{A^*}}.
\end{equation}

\begin{example}
%@see: 《数值分析(第5版)》(李庆扬、王能超、易大义) P8 例4
已经测得某场地长为\(l^* = \qty{110}{\meter}\),宽为\(d^* = \qty{80}{\meter}\),
且\(
	\abs{l-l^*} \leq \qty{0.2}{\meter},
	\abs{d-d^*} \leq \qty{0.1}{\meter}
\).
试求面积\(s = l d\)的绝对误差限和相对误差限.
\begin{solution}
由\(s = l d\)可知\(\pdv{s}{l} = d, \pdv{s}{d} = l\).
由\cref{equation:数值计算的误差.数值运算的绝对误差限} 可知\begin{equation*}
	\epsilon(s^*)
	\approx \abs{\left( \pdv{s}{l} \right)^*} \epsilon(l^*)
	+ \abs{\left( \pdv{s}{d} \right)^*} \epsilon(d^*),
\end{equation*}
其中\begin{equation*}
	\left( \pdv{s}{l} \right)^*
	= d^*
	= \qty{80}{\meter},
	\qquad
	\left( \pdv{s}{d} \right)^*
	= l^*
	= \qty{110}{\meter},
	\qquad
	\epsilon(l^*) = \qty{0.2}{\meter},
	\qquad
	\epsilon(d^*) = \qty{0.1}{\meter},
\end{equation*}
于是绝对误差限为\begin{equation*}
	\epsilon(s^*)
	\approx 80 \times (0.2) + 110 \times (0.1)
	= \qty{27}{\meter\squared},
\end{equation*}
相对误差限为\begin{equation*}
	\epsilon_r(s^*)
	= \frac{\epsilon(s^*)}{\abs{s^*}}
	= \frac{\epsilon(s^*)}{l^* d^*}
	\approx \frac{27}{8800}
	= 0.31\%.
\end{equation*}
\end{solution}
\end{example}
