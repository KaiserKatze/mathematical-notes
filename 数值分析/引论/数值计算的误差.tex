\section{数值计算的误差}
\subsection{误差来源与分类}
用计算机解决科学计算问题首先要建立数学模型,
它是对被描述的实际问题进行抽象、简化得到的,因而是近似的.
我们把数学模型与实际问题之间出现的这种误差,
称为\DefineConcept{模型误差}.

在数学模型中往往还有一些根据观测得到的物理量,
例如温度、长度、电压等,
这些参量显然也包含误差.
这种由观测产生的误差称为\DefineConcept{观测误差}.

当数学模型不能得到精确解时,
通常需要用数值方法求它的近似解,
其近似解与精确解之间的误差称为\DefineConcept{截断误差}或\DefineConcept{方法误差}.

有了求解数学问题的计算公式以后,
就可以用计算机做数值计算了.
但是由于计算机的字长有限,
在计算机上表示原始数据、中间计算数据和输出数据,
以及将二进制数据转化为十进制数据,或将十进制数据转化为二进制数据时,
都会产生\DefineConcept{舍入误差}.

研究计算结果的误差是否满足精度要求就是误差估计问题.
数值分析主要讨论算法的截断误差、舍入误差.

\subsection{误差与有效数字}
%@see: 《数值分析(第5版)》(李庆扬、王能超、易大义) P4 定义1
假设\(x\)是准确值,\(x^*\)是\(x\)的一个近似值,
则称\begin{equation}
	e^* \defeq x^* - x
\end{equation}
是近似值的\DefineConcept{绝对误差}.

通常我们不能算出准确值\(x\),
也不能算出误差\(e^*\)的准确值,
只能根据测量工具或计算情况估计出误差的绝对值不超过某个整数\(\epsilon^*\),
也就是误差绝对值的一个上界.
我们把\(\epsilon^*\)称为近似值的\DefineConcept{绝对误差限},
它总是正数.
例如,用毫米刻度的米尺测量一根棍子的长度,假设棍子长度的准确值是\(x\),
测量时读出的与棍子长度接近的刻度是\(x^*\),
\(x^*\)是\(x\)的近似值,
它的绝对误差限是\(\qty{0.5}{\milli\meter}\),
可知\(\abs{x^* - x} \leq \qty{0.5}{\milli\meter}\).
如果读出的长度是\(\qty{765}{\milli\meter}\),
则有\(\abs{765-x} \leq 0.5\),
从这个不等式我们仍不知道准确的\(x\)是多少,
但是可以得出结论\(764.5 \leq x \leq 765.5\).
对于一般情形,我们常常把不等式\(\abs{x^* - x} \leq \epsilon^*\)
写成\begin{equation}
	x = x^* \pm \epsilon^*.
\end{equation}

绝对误差限的大小还不能完全表示近似值的好坏.
例如有两个量\(x = 10\pm1\)和\(y = 1000\pm5\),
虽然后者的绝对误差限是前者的5倍,
但是\(5/1000 = 0.5\%\)比\(1/10 = 10\%\)要小得多,
这说明\(y^*\)近似\(y\)的程度
比\(x^*\)近似\(x\)的程度要好得多.
也就是说,除了考虑误差的大小以外,
还应该考虑准确值\(x\)本身的大小.

我们把准确值的误差\(e^*\)与准确值\(x\)的比值\begin{equation}
	e^*_r
	\defeq \frac{e^*}{x}
	= \frac{x^* - x}{x}
\end{equation}
称为近似值\(x^*\)的\DefineConcept{相对误差}.

在实际计算中,由于真值\(x\)总是不知道的,
当\(e^* / x^*\)较小时,
通常取\begin{equation}
	e^*_r
	\defeq \frac{e^*}{x^*}
	= \frac{x^* - x}{x^*}
\end{equation}
作为\(x^*\)的相对误差,
此时\begin{equation*}
	\frac{e^*}{x}
	- \frac{e^*}{x^*}
	= \frac{e^* (x^* - x)}{x^* x}
	= \frac{(e^*)^2}{x^* (x^* - e^*)}
	= \frac{(e^* / x^*)^2}{1 - (e^* / x^*)}
\end{equation*}
是\(e^*_r\)的平方项级,故可忽略不计.

相对误差跟绝对误差一样,也可正可负,
它的绝对值上界叫做相对误差限,
记作\(\epsilon^*_r\),
即\begin{equation}
	\epsilon^*_r
	\defeq \frac{\epsilon^*}{\abs{x^*}}.
\end{equation}

根据定义,\(10\pm1\)和\(1000\pm5\)的相对误差限分别是\(10\%\)和\(0.5\%\),
由此可见后者的近似程度比前者的近视程度好.

在\(p\)进制下,
当准确值\(x\)有多位数时,
常常按四舍五入的原则得到\(x\)的前几位近似值\(x^*\).
如果近似值\(x^*\)的绝对误差限是某一位的半个单位,
且该位到\(x^*\)的第一位非零数字共有\(n\)位,
则称\(x^*\)有\(n\)位\DefineConcept{有效数字}.
我们常常把\(x^*\)表示为\begin{equation*}
	x^*
	= \pm p^m \times (
		a_1
		+ a_2 \times p^{-1}
		+ \dotsb
		+ a_n \times p^{-(n-1)}
	),
\end{equation*}
其中\(a_i\ (i=1,2,\dotsc,n)\)
是\(0\)到\(p-1\)中的一个数字,
\(a_1\neq0\),
\(m\)是整数,
且\begin{equation*}
	\abs{x - x^*}
	\leq \frac12 \times p^{m-n+1}.
\end{equation*}

\begin{example}
%@see: 《数值分析(第5版)》(李庆扬、王能超、易大义) P6 例1
按四舍五入原则写出下列十进制数字的具有5位有效数字的近似数:\begin{equation*}
	187.932~5, \qquad
	0.037~855~51, \qquad
	8.000~033, \qquad
	2.718~281~8.
\end{equation*}
\begin{solution}
根据定义,上述十进制数字的具有5位有效数字的近似数分别是:\begin{equation*}
	187.93, \qquad
	0.037~856, \qquad
	8.000~0, \qquad
	2.718~3.
\end{equation*}
\end{solution}
\end{example}

关于有效数字与相对误差限的关系,有下面的定理.
\begin{theorem}
%@see: 《数值分析(第5版)》(李庆扬、王能超、易大义) P6 定理1
设近似数\begin{equation*}
	x^*
	= \pm p^m \times (
		a_1
		+ a_2 \times p^{-1}
		+ \dotsb
		+ a_n \times p^{-(n-1)}
	),
\end{equation*}
其中\(a_i\ (i=1,2,\dotsc,n)\)
是\(0\)到\(p-1\)中的一个数字,
\(a_1\neq0\),
\(m\)是整数.
\begin{itemize}
	\item 如果\(x^*\)具有\(n\)位有效数字,
	则\(x^*\)的相对误差限为\begin{equation}
		\epsilon^*_r
		\leq \frac1{2 a_1} \times p^{-(n-1)}.
	\end{equation}

	\item 如果\(x^*\)的相对误差限为\begin{equation*}
		\epsilon^*_r
		\leq \frac1{2(a_1+1)} \times p^{-(n-1)},
	\end{equation*}
	则\(x^*\)至少具有\(n\)位有效数字.
\end{itemize}
\end{theorem}
上述定理说明,一个近似数的有效位数越多,它的相对误差限就越小.
