\section{数值计算中算法设计的技术}
%@see: 《数值分析(第5版)》(李庆扬、王能超、易大义) P13
在数值计算中,算法设计的好坏,不但影响计算结果的精度,还会影响计算时间的长度.
下面给出几个具有代表性的算法,让我们一起学习它们的基本原则.

\subsection{多项式求值的秦九韶算法}
让我们考虑下面这个问题:
给定\(n\)次多项式\begin{equation*}
	p(x)
	\defeq
	a_0 x^n + a_1 x^{n-1} + \dotsb + a_{n-1} x + a_n
	\quad(a_0\neq0),
\end{equation*}
求\(p(x^*)\).

若直接计算每一项\(a_i x^{n-i}\)再相加,
总共需要\begin{equation*}
	\sum_{i=0}^n (n-i)
	= \frac{n(n+1)}{2}
	= O(n^2)
\end{equation*}
次乘法和\(n\)次加法.

但是,如果利用递推公式\begin{align*}
%@see: 《数值分析(第5版)》(李庆扬、王能超、易大义) P13 (4.1)
	b_0 &= a_0, \\
	b_i &= b_{i-1} x^* + a_i
	\quad(i=1,2,\dotsc,n),
\end{align*}
计算出\(b_n\),那么就会发现\(p(x^*)\)恰与\(b_n\)相等.
这个算法便是秦九韶算法.
用它计算\(n\)次多项式\(p(x^*)\)只需要\(n\)次乘法和\(n\)次加法,
乘法次数由\(O(n^2)\)降为\(O(n)\),
并且只需要\(n+2\)个存储单元.
秦九韶算法是计算多项式函数值的最佳算法,
它是南宋数学家秦九韶于1247年提出的.
% 国外称此算法为 Horner 算法(1819年提出).

秦九韶算法还可以用于计算\(p'(x^*)\).
由\begin{equation*}
	p(x)
	\defeq
	a_0 x^n + a_1 x^{n-1} + \dotsb + a_{n-1} x + a_n
	\quad(a_0\neq0)
\end{equation*}
求导得\begin{equation*}
	p'(x)
	= n a_0 x^{n-1} + (n-1) a_1 x^{n-2} + \dotsb + a_{n-1}.
\end{equation*}
