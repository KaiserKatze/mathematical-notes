\section{数值分析的对象、作用与特点}
数值分析研究用计算机求解各种数学问题的数值计算方法及其理论与软件实现,
用计算机求解科学技术问题通常经历以下步骤:
\begin{enumerate}
	\item 根据实际问题建立数学模型.
	\item 由数学模型给出数值计算方法.
	\item 根据计算方法编制算法程序,在计算机上算出结果.
\end{enumerate}

能用计算机计算的数值问题,
是指输入数据(即问题中的自变量与原始数据)
与输出数据(结果)之间函数关系的一个确定而无歧义的描述,
输入输出数据可用有限维向量表示.

有的问题,例如求解线性方程组,属于数值问题.
而有的问题,例如给定常微分方程,求未知函数的解析表达式,就不属于数值问题.

数值计算的基本单位称为算法元,
它由算子、输入元和输出元组成.
算子可以是简单操作,
例如算术运算、逻辑运算;
也可以是宏操作,
例如向量运算、数组传输、基本初等函数求值等.
输入元和输出元通常视作向量.
有限个算法元的序列称为一个进程.
一个数值问题的算法是指按规定顺序执行一个或多个完整的进程,
通过它们将输入元变换成输出元.
按同时运行的进程个数,
可以将算法分为串行算法和并行算法两类;
其中,同一时间只有一个进程在运行的算法,称为串行算法;
反之,同一时间有若干个进程在运行的算法,称为并行算法.

一个给定的数值问题,
可以有许多不同的算法.
虽然它们都能给出近似答案,
但是所需的计算量和得到的精确程度可能相差很大.
一个面向计算机、有可靠理论分析、计算复杂性好的算法,就是一个好算法.
理论分析主要是连续系统的离散化及离散型方程的数值问题求解,
它包括误差分析、稳定性、收敛性等基本概念,
它刻画了算法的可靠性、准确性.
计算复杂性包含计算时间复杂性与存储空间复杂性两个方面.
在同一规模、同一精度条件下,计算时间少的算法计算时间复杂性好,
而占用内存空间少的算法存储空间复杂性好.
计算复杂性实际上就是算法中计算量与存储量的分析.
同一问题的不同算法的计算复杂性可能差别很大.
例如在解\(n\)阶线性方程组时,
若依照克拉默法则用行列式解法则需要算\(n+1\)个\(n\)阶行列式的值,
具体到\(n=20\)的情况下就需要\num{9.7e21}次乘除法运算,
但是若用高斯列主元消去法,则只需要做3060次乘除运算,
且\(n\)越大两种算法所需运算次数差距就越大.
这表明算法研究的重要性,
也说明只提高计算机速度而不改进和选用好的算法是不行的.

综上所述,数值分析是研究数值问题的算法,概括起来有四点:\begin{enumerate}
	\item 面向计算机,要根据计算机的特点,提供切实可行的有效算法(即计算机能直接处理的逻辑运算和算术运算);
	\item 建立在相应的数学理论基础上,有可靠的理论分析,能任意逼近并达到精度要求,对近似算法要保证收敛性和数值稳定性,还要对误差进行分析;
	\item 要有好的计算复杂性,节省计算时间和存储空间;
	\item 要有数值实验,通过实验证明算法是行之有效的.
\end{enumerate}
