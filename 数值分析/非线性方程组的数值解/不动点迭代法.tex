\section{不动点迭代法}
\subsection{不动点迭代法的概念}
将方程\(f(x) = 0\)改写成等价的形式\(
%@see: 《数值分析(第5版)》(李庆扬、王能超、易大义) P215 (2.1)
	x = \phi(x)
\),
那么\(f\)的每一个零点都是\(\phi\)的一个不动点,
%\cref{definition:迭代数列.不动点}
求\(f\)的零点就等价于求\(\phi\)的不动点.
于是我们可以选定一个初始近似值\(x_0\),并给出递推公式\begin{equation}\label{definition:不动点迭代法.递推公式}
%@see: 《数值分析(第5版)》(李庆扬、王能超、易大义) P215 (2.2)
	x_{k+1} \defeq \phi(x_k)
	\quad(k=0,1,2,\dotsc).
\end{equation}
我们把\(\phi\)称为\DefineConcept{迭代函数}.
如果对于任意\(x_0 \in [a,b]\),
由递推公式 \labelcref{definition:不动点迭代法.递推公式} 确定的点列\(\{x_k\}_{k\geq0}\)收敛于\(x^*\),
则称“迭代方程 \labelcref{definition:不动点迭代法.递推公式} \DefineConcept{收敛}”.

不动点迭代法是一种逐次逼近法,
其基本思想是将隐式方程\(x = \phi(x)\)
归结为一组显式的计算公式 \labelcref{definition:不动点迭代法.递推公式},
也就是说,该迭代过程实质上是一个逐步显式化的过程.

\subsection{不动点迭代法的收敛性}
