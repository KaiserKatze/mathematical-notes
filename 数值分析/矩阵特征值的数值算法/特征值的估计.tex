\section{特征值的估计}
\begin{definition}
%@see: 《数值分析(第5版)》(李庆扬、王能超、易大义) P242 定义1
设矩阵\(A = (a_{ij})_n \in M_n(\mathbb{C})\),
记\(
	r_i
	\defeq
	\sum_{\substack{1 \leq j \leq n \\ j \neq i}} \abs{ a_{ij} }
	\ (i=1,2,\dotsc,n)
\).
把复平面上以\(a_{ii}\)为圆心、以\(r_i\)为半径的所有圆盘\begin{equation*}
	D_i
	\defeq
	\Set{
		z \in \mathbb{C}
		\given
		\abs{ z - a_{ii} } \leq r_i
	}
	\quad(i=1,2,\dotsc,n)
\end{equation*}
称为“矩阵\(A\)的一个\DefineConcept{格什戈林圆盘}(Gershgorin disk)”.
\end{definition}
\begin{theorem}
%@see: 《数值分析(第5版)》(李庆扬、王能超、易大义) P242 定理5(格什戈林圆盘定理)
%@see: 《Matrix Computations 4th Edition》(Gene H. Golub, Charles F. Van Loan) P442 Theorem 8.1.3 (Gershgorin)
设矩阵\(A = (a_{ij})_n \in M_n(\mathbb{C})\),
则\(A\)的任意一个特征值\(\lambda\)必定属于\(A\)的全体格什戈林圆盘的并集,
即有\begin{equation*}
%@see: 《数值分析(第5版)》(李庆扬、王能超、易大义) P242 (1.4)
	\abs{ \lambda - a_{ii} }
	\leq r_i = \sum_{\substack{1 \leq j \leq n \\ j \neq i}} \abs{ a_{ij} }
	\quad(i=1,2,\dotsc,n).
\end{equation*}
%@see: https://mathworld.wolfram.com/GershgorinCircleTheorem.html
\end{theorem}

\begin{theorem}
%@see: 《数值分析(第5版)》(李庆扬、王能超、易大义) P242 定理5(格什戈林圆盘定理)
设矩阵\(A = (a_{ij})_n \in M_n(\mathbb{C})\),
\(S\)是\(A\)的\(m\)个格什戈林圆盘的并集,
\(S'\)是\(A\)的其余\(n-m\)个格什戈林圆盘的并集.
如果\(S\)是连通的,
且\(S\)与\(S'\)是隔离的,
则\(S\)内恰好含有\(A\)的\(m\)个特征值.
\end{theorem}

\begin{corollary}
%@see: 《数值分析(第5版)》(李庆扬、王能超、易大义) P242 定理5(格什戈林圆盘定理)
设矩阵\(A = (a_{ij})_n \in M_n(\mathbb{C})\),
\(S\)是\(A\)的\(1\)个格什戈林圆盘的并集,
\(S'\)是\(A\)的其余\(n-1\)个格什戈林圆盘的并集.
如果\(S\)是连通的,
且\(S\)与\(S'\)是隔离的,
则\(S\)内恰好含有\(A\)的\(1\)个特征值.
\end{corollary}
