\section{三次样条插值}
%@see: 《数值分析(第5版)》(李庆扬、王能超、易大义) P41
上面讨论的分段低次插值函数都有一致收敛性,但光滑性较差.
在设计高速飞机的机翼形线、船体放样等型值线时,我们往往要求插值函数具有二阶光滑度(即插值函数需要具有二阶连续导数).
早期工程师制图时,把富有弹性的细长木条(所谓样条)用压铁固定在样点,在其他地方让它自由完全,
然后沿着木条画下曲线,称之为“样条曲线”.
样条曲线实际上是由分段三次曲线拼接而成,
在连接点(即样点)上要求二阶导数连续.

\subsection{三次样条函数}
\begin{definition}
%@see: 《数值分析(第5版)》(李庆扬、王能超、易大义) P41 定义3
给定节点\(a = x_0 < x_1 < \dotsb < x_n = b\),
如果函数\(S \in C^2[a,b]\)在每个小区间\([x_j,x_{j+1}]\)上是三次多项式,
那么称“\(S\)是节点\(x_0,x_1,\dotsc,x_n\)上的\DefineConcept{三次样条函数}”.
\end{definition}

\begin{definition}
若在节点\(x_j\)上给定函数值\(y_j = f(x_j)\ (j=0,1,2,\dotsc,n)\),
并成立\begin{equation*}
%@see: 《数值分析(第5版)》(李庆扬、王能超、易大义) P42 (6.1)
	S(x_j) = y_j
	\quad(j=0,1,2,\dotsc,n),
\end{equation*}
则称“\(S\)是\DefineConcept{三次样条插值函数}”.
\end{definition}
