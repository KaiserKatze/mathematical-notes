\section{三次样条插值}
%@see: 《数值分析(第5版)》(李庆扬、王能超、易大义) P41
上面讨论的分段低次插值函数都有一致收敛性,但光滑性较差.
在设计高速飞机的机翼形线、船体放样等型值线时,我们往往要求插值函数具有二阶光滑度(即插值函数需要具有二阶连续导数).
早期工程师制图时,把富有弹性的细长木条(所谓样条)用压铁固定在样点,在其他地方让它自由完全,
然后沿着木条画下曲线,称之为“样条曲线”.
样条曲线实际上是由分段三次曲线拼接而成,
在连接点(即样点)上要求二阶导数连续.

\subsection{三次样条函数}
\begin{definition}
%@see: 《数值分析(第5版)》(李庆扬、王能超、易大义) P41 定义3
给定节点\(a = x_0 < x_1 < \dotsb < x_n = b\),
如果函数\(S \in C^2[a,b]\)在每个小区间\([x_j,x_{j+1}]\)上是三次多项式,
那么称“\(S\)是节点\(x_0,x_1,\dotsc,x_n\)上的\DefineConcept{三次样条函数}”.
\end{definition}

\begin{definition}
给定节点\(a = x_0 < x_1 < \dotsb < x_n = b\),
设\(S\)是节点\(x_0,x_1,\dotsc,x_n\)上的三次样条函数.
若在节点\(x_j\)上给定函数值\(y_j = f(x_j)\ (j=0,1,2,\dotsc,n)\),
并成立\begin{equation}\label{equation:三次样条插值.插值条件}
%@see: 《数值分析(第5版)》(李庆扬、王能超、易大义) P42 (6.1)
	S(x_j) = y_j
	\quad(j=0,1,2,\dotsc,n),
\end{equation}
则称“\(S\)是\DefineConcept{三次样条插值函数}”.
\end{definition}

由定义可知,要想求出\(S\),
就必须在每个小区间\([x_j,x_{j+1}]\)上确定4个待定参数.
由于共有\(n\)个小区间,
所以一共需要确定\(4n\)个参数.

因为\(S\)在\([a,b]\)上二阶导数连续,
在节点\(x_j\ (j=1,2,\dotsc,n-1)\)满足\begin{equation}\label{equation:三次样条插值.中间条件}
%@see: 《数值分析(第5版)》(李庆扬、王能超、易大义) P42 (6.2)
	S(x_j^-) = S(x_j^+),
	\qquad
	S'(x_j^-) = S'(x_j^+),
	\qquad
	S''(x_j^-) = S''(x_j^+).
\end{equation}
这里一共有\(3n-3\)个条件,
再加上\(S\)满足插值条件 \labelcref{equation:三次样条插值.插值条件},
一共有\(4n-2\)个条件,
因此还需要加上2个条件才能确定\(S\).
通常可以在区间\([a,b]\)的端点\(a = x_0, b = x_n\)的基础上各加一个条件(称之为\DefineConcept{边界条件}).
边界条件可以根据实际问题的要求予以确定.

常见的边界条件有以下三种类型:\begin{enumerate}
	\item 已知两端的一阶导数值,即\begin{equation}\label{equation:三次样条插值.边界条件1}
	%@see: 《数值分析(第5版)》(李庆扬、王能超、易大义) P42 (6.3)
		S'(x_0) = f'_0,
		\qquad
		S'(x_n) = f'_n;
	\end{equation}

	\item 已知两端的二阶导数值\begin{equation}\label{equation:三次样条插值.边界条件2}
	%@see: 《数值分析(第5版)》(李庆扬、王能超、易大义) P42 (6.4)
		S''(x_0) = f''_0,
		\qquad
		S''(x_n) = f''_n;
	\end{equation}
	特别地,把\begin{equation}\label{equation:三次样条插值.边界条件3}
	%@see: 《数值分析(第5版)》(李庆扬、王能超、易大义) P42 (6.4)'
		S''(x_0) = S''(x_n) = 0
	\end{equation}
	称为\DefineConcept{自然边界条件};

	\item 如果已知\(f\)以\(x_n-x_0\)为周期,
	那么\(S\)必须满足以下周期性边界条件\begin{equation}\label{equation:三次样条插值.边界条件4}
	%@see: 《数值分析(第5版)》(李庆扬、王能超、易大义) P42 (6.5)
		S(x_0^+) = S(x_n^-)
		= y_0 = y_n,
		\qquad
		S'(x_0^+) = S'(x_n^-),
		\qquad
		S''(x_0^+) = S''(x_n^-),
	\end{equation}
	由此确定的样条函数\(S\)称为\DefineConcept{周期样条函数}.
\end{enumerate}

\subsection{样条插值函数的建立}
构造满足插值条件 \labelcref{equation:三次样条插值.插值条件}
及相应边界条件的三次样条插值函数\(S\)的表达式可以有多种方法.
例如,可以直接利用分段三次厄米插值,
只要假定\(S'(x_j) = m_j\ (j=0,1,2,\dotsc,n)\),
再由插值条件 \labelcref{equation:三次样条插值.插值条件}
可得\begin{equation*}
%@see: 《数值分析(第5版)》(李庆扬、王能超、易大义) P42 (6.6)
	S(x) = \sum_{j=0}^n (y_j \alpha_j(x) + m_j \beta_j(x)),
\end{equation*}
其中\(\alpha_j(x),\beta_j(x)\)是
由\cref{equation:厄米插值.厄米插值基函数1,equation:厄米插值.厄米插值基函数2,equation:厄米插值.厄米插值基函数3,equation:厄米插值.厄米插值基函数4}
表示的厄米插值基函数,
利用内节点二阶光滑条件 \labelcref{equation:三次样条插值.中间条件}
及相应边界条件
\labelcref{equation:三次样条插值.边界条件1,equation:三次样条插值.边界条件2,equation:三次样条插值.边界条件3,equation:三次样条插值.边界条件4},
就可以得到一个关于\(m_j\ (j=0,1,2,\dotsc,n)\)的三对角方程组,
只要求出\(m_j\)就能得到所求的三次样条函数\(S\).

下面我们讨论另一种构造方法:
利用\(S\)的二阶导数值\(S''(x_j) = M_j\ (j=0,1,2,\dotsc,n)\)表达\(S\).
由于\(S\)在区间\([x_j,x_{j+1}]\)上是三次多项式,
故\(S''(x)\)在\([x_j,x_{j+1}]\)上是线性函数,
可以表示为\begin{equation}
%@see: 《数值分析(第5版)》(李庆扬、王能超、易大义) P43 (6.7)
	S''(x) = M_j \frac{x_{j+1} - x}{h_j} + M_{j+1} \frac{x - x_j}{h_j}.
\end{equation}
对\(S''\)积分两次并利用\(S(x_j) = y_j\)和\(S(x_{j+1}) = y_{j+1}\),
可以定出积分常数,
于是得到三次样条表达式\begin{equation}\label{equation:三次样条插值.用二阶导数值表示三次样条插值函数}
%@see: 《数值分析(第5版)》(李庆扬、王能超、易大义) P43 (6.8)
	\begin{aligned}
		S(x) &= M_j \frac{(x_{j+1} - x)^3}{6 h_j} + M_{j+1} \frac{(x - x_j)^3}{6 h_j} \\
		&+ \left( y_j - \frac{M_j h_j^2}{6} \right) \frac{x_{j+1} - x}{h_j}
		+ \left( y_{j+1} - \frac{M_{j+1} h_j^2}{6} \right) \frac{x - x_j}{h_j},
	\end{aligned}
\end{equation}
其中\(j=0,1,2,\dotsc,n-1\).
这里\(M_j\ (j=0,1,2,\dotsc,n)\)是未知的.
为了确定\(M_j\)的取值,
对\(S\)求导得\begin{equation}
%@see: 《数值分析(第5版)》(李庆扬、王能超、易大义) P43 (6.9)
	S'(x) = -M_j \frac{(x_{j+1} - x)^2}{2 h_j}
	+ M_{j+1} \frac{(x - x_j)^2}{2 h_j}
	+ \frac{y_{j+1} - y_j}{h_j}
	- \frac{M_{j+1} - M_j}{6} h_j;
\end{equation}
由此可以求得\begin{equation*}
	S'(x_j^+)
	= -\frac{h_j}{3} M_j
	- \frac{h_j}{6} M_{j+1}
	+ \frac{y_{j+1} - y_j}{h_j}.
\end{equation*}
类似地可以求出\(S\)在区间\([x_{j-1},x_j]\)上的表达式,
进而有\begin{equation*}
	S'(x_j^-)
	= \frac{h_{j-1}}{6} M_{j-1}
	+ \frac{h_{j-1}}{3} M_j
	+ \frac{y_j - y_{i-1}}{h_{j-1}}.
\end{equation*}
利用\(S'(x_j^+) = S'(x_j^-)\)可得\begin{equation}\label{equation:三次样条插值.一阶光滑的必要条件}
%@see: 《数值分析(第5版)》(李庆扬、王能超、易大义) P43 (6.10)
	\mu_j M_{j-1} + 2 M_j + \lambda_j M_{j+1}
	= d_j
	\quad(j=1,2,\dotsc,n-1),
\end{equation}
其中\begin{align}
%@see: 《数值分析(第5版)》(李庆扬、王能超、易大义) P43 (6.11)
	\mu_j
	&= \frac{h_{j-1}}{h_{j-1} + h_j}
	\quad(j=1,2,\dotsc,n-1), \\
	\lambda_j
	&= \frac{h_j}{h_{j-1} + h_j}
	\quad(j=1,2,\dotsc,n-1), \\
	d_j
	&= 6 \frac{f[x_j,x_{j+1}] - f[x_{j-1},x_j]}{h_{j-1} + h_j}
	= 6 f[x_{j-1},x_j,x_{j+1}]
	\quad(j=1,2,\dotsc,n-1).
\end{align}

由边界条件 \labelcref{equation:三次样条插值.边界条件1} 可以导出两个方程\begin{equation}\label{equation:三次样条插值.边界条件1的必要条件}
%@see: 《数值分析(第5版)》(李庆扬、王能超、易大义) P43 (6.12)
	\begin{cases}
		2 M_0 + M_1 = \frac{6}{h_0} ( f[x_0,x_1] - f'_0 ), \\
		M_{n-1} + 2 M_n = \frac{6}{h_{n-1}} ( f'_n - f[x_{n-1},x_n] ).
	\end{cases}
\end{equation}
如果令\(
	\lambda_0 \defeq 1,
	d_0 \defeq \frac{6}{h_0} (f[x_0,x_1] - f'_0),
	\mu_n = 1,
	d_n = \frac{6}{h_{n-1}} (f'_n - f[x_{n-1},x_n])
\),
那么\cref{equation:三次样条插值.一阶光滑的必要条件,equation:三次样条插值.边界条件1的必要条件}
可以写成矩阵形式\begin{equation}\label{equation:三次样条插值.边界条件1对应的线性方程组}
%@see: 《数值分析(第5版)》(李庆扬、王能超、易大义) P44 (6.13)
	\begin{bmatrix}
		2 & \lambda_0 \\
		\mu_1 & 2 & \lambda_1 \\
		& \ddots & \ddots & \ddots \\
		&& \mu_{n-1} & 2 & \lambda_{n-1} \\
		&&& \mu_n & 2
	\end{bmatrix}
	\begin{bmatrix}
		M_0 \\ M_1 \\ \vdots \\ M_{n-1} \\ M_n
	\end{bmatrix}
	= \begin{bmatrix}
		d_0 \\ d_1 \\ \vdots \\ d_{n-1} \\ d_n
	\end{bmatrix}.
\end{equation}

由边界条件 \labelcref{equation:三次样条插值.边界条件2} 可以导出端点方程\begin{equation}\label{equation:三次样条插值.边界条件2的必要条件}
%@see: 《数值分析(第5版)》(李庆扬、王能超、易大义) P44 (6.14)
	M_0 = f''_0,
	\qquad
	M_n = f''_n.
\end{equation}
如果令\(
	\lambda_0 \defeq 0,
	\mu_n \defeq 0,
	d_0 \defeq 2 f''_0,
	d_n \defeq 2 f''_n
\),
则\cref{equation:三次样条插值.一阶光滑的必要条件,equation:三次样条插值.边界条件2的必要条件}
也可以写成\cref{equation:三次样条插值.边界条件1对应的线性方程组} 的形式.

由边界条件 \labelcref{equation:三次样条插值.边界条件4} 可以导出\begin{equation}\label{equation:三次样条插值.边界条件4的必要条件}
%@see: 《数值分析(第5版)》(李庆扬、王能超、易大义) P44 (6.15)
	M_0 = M_n,
	\qquad
	\lambda_n M_1 + \mu_n M_{n-1} + 2 M_n = d_n,
\end{equation}
其中\begin{equation*}
	\lambda_n = \frac{h_0}{h_{n-1} + h_0},
	\qquad
	\mu_n = 1 - \lambda_n
	= \frac{h_{n-1}}{h_{n-1} + h_0},
	\qquad
	d_n = 6 \frac{f[x_0,x_1] - f[x_{n-1},x_n]}{h_0 + h_{n-1}}.
\end{equation*}
这时\cref{equation:三次样条插值.一阶光滑的必要条件,equation:三次样条插值.边界条件4的必要条件}
可以写成矩阵形式\begin{equation}\label{equation:三次样条插值.边界条件4对应的线性方程组}
%@see: 《数值分析(第5版)》(李庆扬、王能超、易大义) P44 (6.16)
	\begin{bmatrix}
		2 & \lambda_1 &&& 2 \\
		\mu_2 & 2 & \lambda_2 \\
		& \ddots & \ddots & \ddots \\
		&& \mu_{n-1} & 2 & \lambda_{n-1} \\
		\lambda_n &&& \mu_n & 2 \\
	\end{bmatrix}
	\begin{bmatrix}
		M_0 \\ M_1 \\ \vdots \\ M_{n-1} \\ M_n
	\end{bmatrix}
	= \begin{bmatrix}
		d_0 \\ d_1 \\ \vdots \\ d_{n-1} \\ d_n
	\end{bmatrix}.
\end{equation}
\cref{equation:三次样条插值.边界条件1对应的线性方程组,equation:三次样条插值.边界条件4对应的线性方程组}
都是关于\(M_j\ (j=0,1,2,\dotsc,n)\)的三对角线性方程组.
\(M_j\)在力学上可以解释为细梁在\(x_j\)截面处的弯矩,因此我们称之为\(S\)的\DefineConcept{矩},
继而把\cref{equation:三次样条插值.边界条件1对应的线性方程组,equation:三次样条插值.边界条件4对应的线性方程组}
称为\DefineConcept{三弯矩方程}.
在三弯矩方程的系数矩阵中,元素\(\lambda_i,\mu_i\)均已完全确定,
并且它们满足\(
	\lambda_j \geq 0,
	\mu_j \geq 0,
	\lambda_j + \mu_j = 1
\).
因此三弯矩方程的系数矩阵是严格对角占优矩阵,
%TODO 什么是{严格对角占优矩阵}?
于是它有唯一解.
求解三弯矩方程可以采用追赶法,
%TODO 追赶法在《数值分析(第5版)》(李庆扬、王能超、易大义)第5章第3节
将计算结果代入\cref{equation:三次样条插值.用二阶导数值表示三次样条插值函数} 即可.

\subsection{样条插值函数的收敛性与误差限}
\begin{theorem}
%@see: 《数值分析(第5版)》(李庆扬、王能超、易大义) P46 定理5
设\(f \in C^4[a,b]\),
三次样条函数\(S\)满足边界条件
\labelcref{equation:三次样条插值.边界条件1}
或 \labelcref{equation:三次样条插值.边界条件2},
令\(
	h \defeq \max_{0 \leq i \leq n-1} h_i,
	h_i \defeq x_{i+1} - x_i
	\ (i=0,1,2,\dotsc,n-1)
\),
则有\begin{equation*}
%@see: 《数值分析(第5版)》(李庆扬、王能超、易大义) P46 (6.17)
	\max_{a \leq x \leq b} \abs{f^{(k)}(x) - S^{(k)}(x)}
	\leq C_k \max_{a \leq x \leq b} \abs{f^{(4)}(x)} h^{4-k},
\end{equation*}
其中\(
	k=0,1,2;
	C_0 = \frac{5}{384},
	C_1 = \frac{1}{24},
	C_2 = \frac{3}{8}
\).
\end{theorem}
