\section{三次样条插值}
%@see: 《数值分析(第5版)》(李庆扬、王能超、易大义) P41
上面讨论的分段低次插值函数都有一致收敛性,但光滑性较差.
在设计高速飞机的机翼形线、船体放样等型值线时,我们往往要求插值函数具有二阶光滑度(即插值函数需要具有二阶连续导数).
早期工程师制图时,把富有弹性的细长木条(所谓样条)用压铁固定在样点,在其他地方让它自由完全,
然后沿着木条画下曲线,称之为“样条曲线”.
样条曲线实际上是由分段三次曲线拼接而成,
在连接点(即样点)上要求二阶导数连续.

\subsection{三次样条函数}
\begin{definition}
%@see: 《数值分析(第5版)》(李庆扬、王能超、易大义) P41 定义3
给定节点\(a = x_0 < x_1 < \dotsb < x_n = b\),
如果函数\(S \in C^2[a,b]\)在每个小区间\([x_j,x_{j+1}]\)上是三次多项式,
那么称“\(S\)是节点\(x_0,x_1,\dotsc,x_n\)上的\DefineConcept{三次样条函数}”.
\end{definition}

\begin{definition}
给定节点\(a = x_0 < x_1 < \dotsb < x_n = b\),
设\(S\)是节点\(x_0,x_1,\dotsc,x_n\)上的三次样条插值函数.
若在节点\(x_j\)上给定函数值\(y_j = f(x_j)\ (j=0,1,2,\dotsc,n)\),
并成立\begin{equation}\label{equation:三次样条插值.三次样条插值条件}
%@see: 《数值分析(第5版)》(李庆扬、王能超、易大义) P42 (6.1)
	S(x_j) = y_j
	\quad(j=0,1,2,\dotsc,n),
\end{equation}
则称“\(S\)是\DefineConcept{三次样条插值函数}”.
\end{definition}

由定义可知,要想求出\(S\),
就必须在每个小区间\([x_j,x_{j+1}]\)上确定4个待定参数.
由于共有\(n\)个小区间,
所以一共需要确定\(4n\)个参数.
因为\(S\)在\([a,b]\)上二阶导数连续,
在节点\(x_j\ (j=1,2,\dotsc,n-1)\)满足\begin{equation*}
%@see: 《数值分析(第5版)》(李庆扬、王能超、易大义) P42 (6.2)
	S(x_j^-) = S(x_j^+),
	\qquad
	S'(x_j^-) = S'(x_j^+),
	\qquad
	S''(x_j^-) = S''(x_j^+).
\end{equation*}
这里一共有\(3n-3\)个条件,
再加上\(S\)满足插值条件 \labelcref{equation:三次样条插值.三次样条插值条件},
一共有\(4n-2\)个条件,
因此还需要加上2个条件才能确定\(S\).
通常可以在区间\([a,b]\)的端点\(a = x_0, b = x_n\)的基础上各加一个条件(称之为\DefineConcept{边界条件}).
边界条件可以根据实际问题的要求予以确定.

常见的边界条件有以下三种类型:\begin{enumerate}
	\item 已知两端的一阶导数值,即\begin{equation*}
	%@see: 《数值分析(第5版)》(李庆扬、王能超、易大义) P42 (6.3)
		S'(x_0) = f'_0,
		\qquad
		S'(x_n) = f'_n;
	\end{equation*}

	\item 已知两端的二阶导数值\begin{equation*}
	%@see: 《数值分析(第5版)》(李庆扬、王能超、易大义) P42 (6.4)
		S''(x_0) = f''_0,
		\qquad
		S''(x_n) = f''_n;
	\end{equation*}
	特别地,把\begin{equation*}
	%@see: 《数值分析(第5版)》(李庆扬、王能超、易大义) P42 (6.4)'
		S''(x_0) = S''(x_n) = 0
	\end{equation*}
	称为\DefineConcept{自然边界条件};

	\item 已知\(f\)以\(x_n-x_0\)为周期,从而有\begin{equation*}
	%@see: 《数值分析(第5版)》(李庆扬、王能超、易大义) P42 (6.5)
		S(x_0^+) = S(x_n^-),
		\qquad
		S'(x_0^+) = S'(x_n^-),
		\qquad
		S''(x_0^+) = S''(x_n^-),
		\qquad
		y_0 = y_n,
	\end{equation*}
	并将\(S\)称为\DefineConcept{周期样条函数}.
\end{enumerate}
