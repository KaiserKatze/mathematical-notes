\section{拉格朗日插值}
\subsection{多项式插值的存在性}
设\begin{equation*}
	a \leq x_0 < x_1 < \dotsb < x_n \leq b,
	\alpha \leq y_0 < y_1 < \dotsb < y_n \leq \beta,
\end{equation*}
现在要求次数不超过\(n\)的多项式\begin{equation*}
	P(x) = a_0 + a_1 x + a_2 x^2 + \dotsb + a_n x^n,
\end{equation*}
使得\begin{equation}\label{equation:插值法.用于插值的多项式需要满足的方程组}
%@see: 《数值分析(第5版)》(李庆扬、王能超、易大义) P23 (1.3)
	P(x_i) = y_i
	\quad(i=0,1,2,\dotsc,n)
\end{equation}成立,
那么可以建立关于\(a_0,\dotsc,a_n\)的\(n+1\)元线性方程组\begin{equation}\label{equation:插值法.用于插值的多项式的系数需要满足的线性方程组}
%@see: 《数值分析(第5版)》(李庆扬、王能超、易大义) P23 (1.4)
	\begin{cases}
		a_0 + a_1 x_0 + a_2 x_0^2 + \dotsb + a_n x_0^n = y_0, \\
		a_0 + a_1 x_1 + a_2 x_1^2 + \dotsb + a_n x_1^n = y_0, \\
		\hdotsfor{1}, \\
		a_0 + a_1 x_n + a_2 x_n^2 + \dotsb + a_n x_n^n = y_0, \\
	\end{cases}
\end{equation}
此方程组的系数矩阵就是范德蒙德矩阵\begin{equation*}
%@see: 《数值分析(第5版)》(李庆扬、王能超、易大义) P23 (1.5)
	A = \begin{bmatrix}
		1 & x_0 & \dots & x_0^n \\
		1 & x_1 & \dots & x_1^n \\
		\vdots & \vdots & & \vdots \\
		1 & x_n & \dots & x_n^n \\
	\end{bmatrix}.
\end{equation*}
由于\(x_i\)互不相等,所以范德蒙德行列式\begin{equation*}
	\det A
	= \prod_{0 \leq j < i \leq n+1} (x_i - x_j)
	\neq 0,
\end{equation*}
于是方程 \labelcref{equation:插值法.用于插值的多项式的系数需要满足的线性方程组} 的解存在且唯一,
也就是说,满足条件 \labelcref{equation:插值法.用于插值的多项式需要满足的方程组} 的多项式存在且唯一.

虽然只要求解方程 \labelcref{equation:插值法.用于插值的多项式的系数需要满足的线性方程组} 就能得到插值多项式函数\(P\),
但是这种解法十分繁琐,通常是不与采用的,下面我们介绍构造插值多项式的简便方法.

\subsection{拉格朗日插值多项式}
如前所述,给定\(n+1\)个节点\(x_0 < x_1 < \dotsb < x_n\),
要求\(n\)次插值多项式\(L_n\),
假定它满足\begin{equation*}
%@see: 《数值分析(第5版)》(李庆扬、王能超、易大义) P25 (2.6)
	L_n(x_j) = y_j
	\quad(j=0,1,2,\dotsc,n).
\end{equation*}

为了构造\(L_n\),我们首先需要定义\(n\)次插值基函数.
\begin{definition}
%@see: 《数值分析(第5版)》(李庆扬、王能超、易大义) P25 定义1
设\(n+1\)个\(n\)次多项式\(l_j\ (j=0,1,2,\dotsc,n)\)
在\(n+1\)个节点\(x_0 < x_1 < \dotsb < x_n\)上满足\begin{equation*}
%@see: 《数值分析(第5版)》(李庆扬、王能超、易大义) P25 (2.7)
	l_j(x_k) = \begin{cases}
		1, & k=j, \\
		0, & k \neq j,
	\end{cases}
	\quad(j,k=0,1,2,\dotsc,n),
\end{equation*}
则称多项式\(l_j\ (j=0,1,2,\dotsc,n)\)是
节点\(x_0,x_1,\dotsc,x_n\)上的
\DefineConcept{\(n\)次插值基函数}.
\end{definition}
\begin{remark}
拉格朗日插值法用到的基函数是一组正交基.
\end{remark}

显然\begin{equation}
%@see: 《数值分析(第5版)》(李庆扬、王能超、易大义) P26 (2.8)
	l_k(x)
	= \prod_{0 \leq i \leq n, i \neq k} \frac{(x - x_i)}{(x_k - x_i)}
	\quad(k=0,1,2,\dotsc,n).
\end{equation}
于是所求插值多项式为\begin{equation}
%@see: 《数值分析(第5版)》(李庆扬、王能超、易大义) P26 (2.9)
	L_n(x)
	= \sum_{k=0}^n y_k l_k(x).
\end{equation}
我们把\(L_n\)称为\DefineConcept{拉格朗日插值多项式}.

若记\begin{equation}
%@see: 《数值分析(第5版)》(李庆扬、王能超、易大义) P26 (2.10)
	\omega_{n+1}(x)
	\defeq
	\prod_{i=0}^n (x - x_i),
\end{equation}
则有\begin{equation*}
	\omega_{n+1}'(x_k)
	= \prod_{0 \leq i \leq n, i \neq k} (x_k - x_i),
\end{equation*}
于是\(L_n\)又可写为\begin{equation}
%@see: 《数值分析(第5版)》(李庆扬、王能超、易大义) P26 (2.11)
	L_n(x) = \sum_{k=0}^n y_k \frac{\omega_{n+1}(x)}{(x - x_k) \omega_{n+1}'(x_k)}.
\end{equation}

\begin{example}
给定数据\begin{center}
	\begin{tblr}{c|*4c}
		\hline
		\(i\) & 0 & 1 & 2 & 3 \\
		\hline
		\(x_i\) & 0 & 1 & 2 & 3 \\
		\(y_i\) & 0 & 1 & 5 & 14 \\
		\hline
	\end{tblr}
\end{center}
求三次拉格朗日插值多项式\(L_3\).
\begin{solution}
基函数为\begin{align*}
	l_1(x)
	&= \frac{(x-0)(x-2)(x-3)}{(1-0)(1-2)(1-3)}
	= \frac{x(x-2)(x-3)}{2}, \\
	l_2(x)
	&= \frac{(x-0)(x-1)(x-3)}{(2-0)(2-1)(2-3)}
	= \frac{x(x-1)(x-3)}{-2}, \\
	l_3(x)
	&= \frac{(x-0)(x-1)(x-2)}{(3-0)(3-1)(3-2)}
	= \frac{x(x-1)(x-2)}{6}.
\end{align*}
于是所求插值多项式为\begin{align*}
	L_3(x)
	&= \sum_{k=0}^3 y_k l_k(x)
	= 0 \cdot l_0(x)
	+ 1 \cdot l_1(x)
	+ 5 \cdot l_2(x)
	+ 14 \cdot l_3(x) \\
	&= \frac{x(x-2)(x-3)}{2}
	+ 5 \frac{x(x-1)(x-3)}{-2}
	+ 14 \frac{x(x-1)(x-2)}{6} \\
	&= \frac13 x^3 + \frac12 x^2 + \frac16 x
	= \frac16 x(x+1)(2x+1).
\end{align*}
\end{solution}
%@Mathematica: Expand[InterpolatingPolynomial[{{0,0},{1,1},{2,5},{3,14}},x]]
%@Mathematica: Expand[x(x-2)(x-3)/2+5x(x-1)(x-3)/(-2)+14x(x-1)(x-2)/6]
\end{example}

\subsection{插值余项与误差估计}
%@see: 《数值分析(第5版)》(李庆扬、王能超、易大义) P26
在闭区间\([a,b]\)上用\(L_n\)近似\(f\),
则其截断误差为\begin{equation*}
	R_n(x) \defeq f(x) - L_n(x).
\end{equation*}
我们把\(R_n\)称为“拉格朗日插值多项式\(L_n\)的\DefineConcept{余项}”.

%@see: 《数值分析(第5版)》(李庆扬、王能超、易大义) P26 定理2
给定结点\(a \leq x_0 < x_1 < \dotsb + x_n \leq b\),
设函数\(f \in C^n[a,b] \cap D^{n+1}(a,b)\),
\(L_n\)是用来近似\(f\)的插值多项式,
那么由\hyperref[equation:微分中值定理.泰勒公式.余项1]{带有拉格朗日余项的泰勒公式}可知,
对于任意\(x \in [a,b]\),
插值余项满足\begin{equation}\label{equation:拉格朗日插值.拉格朗日插值余项}
%@see: 《数值分析(第5版)》(李庆扬、王能超、易大义) P26 (2.12)
	R_n(x) = \frac{f^{(n+1)}(\xi)}{(n+1)!} \omega_{n+1}(x),
\end{equation}
其中\(\xi \in (a,b)\)且依赖于\(x\)的取值.
记\(M \defeq \max_{a \leq x \leq b} \abs{f^{(n+1)}(x)}\),
那么用插值多项式\(L_n\)近似\(f\)的误差限为\begin{equation*}
	% \sup_{a \leq x \leq b} \abs{R_n(x)}
	\frac{M}{(n+1)!} \abs{\omega_{n+1}(x)}.
\end{equation*}

\begin{example}
%@see: 《数值分析(第5版)》(李庆扬、王能超、易大义) P28 例2
已知\begin{equation*}
	\sin0.32 \approx \num{0.314567},
	\qquad
	\sin0.34 \approx \num{0.333487},
	\qquad
	\sin0.36 \approx \num{0.352274}.
\end{equation*}
用线性插值法和抛物插值法分别计算\(\sin\num{0.3367}\)的近似值,并估计截断误差.
\begin{solution}
首先采用线性插值法,
拉格朗日插值多项式为\begin{equation*}
	L_1(x) = \num{0.314567} \frac{x-0.34}{0.32-0.34}
	+ \num{0.333487} \frac{x-0.32}{0.34-0.32},
\end{equation*}
于是\begin{equation*}
	L_1(\num{0.3367})
	\approx \num{0.330365}.
\end{equation*}
其截断误差\begin{equation*}
	\abs{R_1(x)}
	\leq \frac{M_2}{2} \abs{(x-0.32)(x-0.34)},
\end{equation*}
其中\(M_2 = \max_{0.32 \leq x \leq 0.34} \abs{f''(x)}\).
因为对\(f(x) = \sin x\)求导可得\(f''(x) = -\sin x\),
所以\begin{equation*}
	M_2
	= \max_{0.32 \leq x \leq 0.34} \abs{\sin x}
	= \sin0.34
	\leq \num{0.3335},
\end{equation*}
于是截断误差为\begin{equation*}
	\abs{R_1(\num{0.3367})}
	\leq \frac{\num{0.3335}}{2} \abs{(\num{0.3367}-0.32)(\num{0.3367}-0.34)}
	\leq \num{9.2e-6},
\end{equation*}

接着采用抛物插值法,
拉格朗日插值多项式为\begin{align*}
	L_2(x) &= \num{0.314567} \frac{(x-0.34)(x-0.36)}{(0.32-0.34)(0.32-0.36)} \\
	&+ \num{0.333487} \frac{(x-0.32)(x-0.36)}{(0.34-0.32)(0.34-0.36)} \\
	&+ \num{0.352274} \frac{(x-0.32)(x-0.34)}{(0.36-0.32)(0.36-0.34)},
\end{align*}
%@Mathematica: 0.314567(x-0.34)(x-0.36)/((0.32-0.34)(0.32-0.36))+0.333487(x-0.32)(x-0.36)/((0.34-0.32)(0.34-0.36))+0.352274(x-0.32)(x-0.34)/((0.36-0.32)(0.36-0.34)) /. x->0.3367
于是\begin{equation*}
	L_2(\num{0.3367})
	\approx \num{0.330374}.
\end{equation*}
其截断误差为\begin{equation*}
	\abs{R_2(\num{0.3367})}
	\leq \num{2.0e-6}.
\end{equation*}
\end{solution}
\end{example}
