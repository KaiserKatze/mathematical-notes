\section{牛顿插值}
当数据增加或减少时,
拉格朗日插值法就非常不方便了,
这是因为之前的计算结果全都作废了,计算要全部从头开始.
因此,为了计算简便,就可以使用牛顿插值法.

\subsection{差商}
\begin{definition}
%@see: 《数值分析(第5版)》(李庆扬、王能超、易大义) P30 定义2
设函数\(f\colon \mathbb{R} \to \mathbb{R}\).
\begin{itemize}
	\item 把\(f(x_i)\)
	称为“函数\(f\)关于点\(x_i\)的\DefineConcept{零阶差商}”,
	记作\(f[x_i]\).

	\item 把\begin{equation*}
		\frac{f(x_j) - f(x_i)}{x_j - x_i}
	\end{equation*}
	称为“函数\(f\)关于点\(x_i,x_j\)的\DefineConcept{一阶差商}”,
	记作\(f[x_i,x_j]\).

	\item 把\begin{equation*}
		\frac{f[x_i,x_k] - f[x_i,x_j]}{x_k - x_j}
	\end{equation*}
	称为“函数\(f\)关于点\(x_i,x_j,x_k\)的\DefineConcept{二阶差商}”,
	记作\(f[x_i,x_j,x_k]\).

	\item 把\begin{equation*}
	%@see: 《数值分析(第5版)》(李庆扬、王能超、易大义) P30 (3.3)
		\frac{f[x_0,\dotsc,x_{k-2},x_k] - f[x_0,\dotsc,x_{k-2},x_{k-1}]}{x_k - x_{k-1}}
	\end{equation*}
	称为“函数\(f\)关于点\(x_0,\dotsc,x_k\)的 \DefineConcept{\(k\)阶差商}”,
	记作\(f[x_0,\dotsc,x_k]\).
\end{itemize}
\end{definition}
\begin{remark}
在\(k\)阶差商的定义式中,
分子中的被减数\(f[x_0,\dotsc,x_{k-2},x_k]\)中没有符号\(x_{k-1}\),
与此同时,
分子中的减数\(f[x_0,\dotsc,x_{k-2},x_{k-1}]\)中没有符号\(x_k\),
这正好与分母\(x_k - x_{k-1}\)的顺序相反.
\end{remark}

\begin{property}
%@see: 《数值分析(第5版)》(李庆扬、王能超、易大义) P30
\(k\)阶差商\(f[x_0,\dotsc,x_k]\)
可以表示为函数值\(f(x_0),\dotsc,f(x_k)\)的线性组合,
即\begin{equation}
%@see: 《数值分析(第5版)》(李庆扬、王能超、易大义) P30 (3.4)
	f[x_0,\dotsc,x_k]
	= \sum_{j=0}^k
	\frac{f(x_j)}{
		\prod_{0 \leq i \leq k, i \neq j}
		(x_j - x_i)
	}.
\end{equation}
\end{property}

\begin{property}[差商的对称性]
%@see: 《数值分析(第5版)》(李庆扬、王能超、易大义) P30
\(k\)阶差商\(f[x_0,\dotsc,x_k]\)的取值
与数据的编号次序无关,
即\begin{equation}
	f[x_0,\dotsc,x_i,\dotsc,x_j,\dotsc,x_k]
	= f[x_0,\dotsc,x_j,\dotsc,x_i,\dotsc,x_k].
\end{equation}
\end{property}

\begin{property}
\(k\)阶差商\(f[x_0,\dotsc,x_k]\)还可以表示为\begin{equation}\label{equation:牛顿插值.差商的计算式}
%@see: 《数值分析(第5版)》(李庆扬、王能超、易大义) P30 (3.3)'
	f[x_0,\dotsc,x_k]
	= \frac{f[x_1,\dotsc,x_{k-1},x_k] - f[x_0,x_1,\dotsc,x_{k-1}]}{x_k - x_0}.
\end{equation}
\end{property}

\begin{property}
若\(f\)在\([a,b]\)上存在\(n\)阶导数,
且\(x_0,\dotsc,x_n \in (a,b)\),
则\begin{equation*}%\label{equation:牛顿插值.差商与导数的关系}
%@see: 《数值分析(第5版)》(李庆扬、王能超、易大义) P30 (3.5)
	f[x_0,\dotsc,x_n]
	= \frac{f^{(n)}(\xi)}{n!}
	\quad(a \leq \xi \leq b).
\end{equation*}
\end{property}

当我们利用\cref{equation:牛顿插值.差商的计算式} 逐阶计算差商,
可得\begin{align*}
	f[x_0,x_1,x_2]
	&= \frac{f[x_1,x_2] - f[x_0,x_1]}{x_2 - x_0}, \\
	f[x_1,x_2,x_3]
	&= \frac{f[x_2,x_3] - f[x_1,x_2]}{x_3 - x_1}, \\
	f[x_0,x_1,x_2,x_3]
	&= \frac{f[x_1,x_2,x_3] - f[x_0,x_1,x_2]}{x_3 - x_0}.
\end{align*}
根据上述计算结果,我们可以列出一张表:\begin{center}
	\begin{tblr}{*3c|*3c}
		\hline
		\(k\) & \(x_k\) & \(f(x_k)\) & 一阶差商 & 二阶差商 & 三阶差商 \\
		\hline
		0 & \(x_0\) & \(f(x_0)\) \\
		1 & \(x_1\) & \(f(x_1)\) & \(f[x_0,x_1]\) \\
		2 & \(x_2\) & \(f(x_2)\) & \(f[x_1,x_2]\) & \(f[x_0,x_1,x_2]\) \\
		3 & \(x_3\) & \(f(x_3)\) & \(f[x_2,x_3]\) & \(f[x_1,x_2,x_3]\) & \(f[x_0,x_1,x_2,x_3]\) \\
		\hline
	\end{tblr}
\end{center}
我们把上述表格称为\DefineConcept{差商表}.

\subsection{牛顿差商插值多项式}
我们把\begin{align*}
%@see: 《数值分析(第5版)》(李庆扬、王能超、易大义) P31 (3.6)
	N_n(x)
	&\defeq
	\sum_{j=0}^n f[x_0,\dotsc,x_j] \prod_{i=0}^{j-1} (x-x_i) \\
	&= f(x_0)
	+ f[x_0,x_1] (x-x_0)
	+ f[x_0,x_1,x_2] (x-x_0)(x-x_1) \\
	&\hspace{20pt}+ \dotsb
	+ f[x_0,\dotsc,x_n] (x-x_0)\dotsm(x-x_{n-1})
\end{align*}
称为\DefineConcept{牛顿差商插值多项式},
把截断误差\begin{align*}
%@see: 《数值分析(第5版)》(李庆扬、王能超、易大义) P31 (3.7)
	R_n(x)
	&\defeq
	f(x) - N_n(x) \\
	&= f[x,x_0,\dotsc,x_n] \omega_{n+1}(x)
\end{align*}
称为“牛顿差商插值多项式\(N_n\)的\DefineConcept{余项}”.
