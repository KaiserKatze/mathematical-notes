历史上,天文学家、物理学家通过观测或实验获得了大量数据.
对于这些数据,我们总是希望从中找出表示某种内在规律的数量关系.
例如,天文学家开普勒在分析天文观测数据时发现行星轨道周期的平方
与它的轨道半长轴的立方成正比(此即开普勒第三定律).
%@see: https://mathshistory.st-andrews.ac.uk/HistTopics/Keplers_laws/
在计算机尚未发明的年代,
有的函数虽有解析表达式,
但是由于计算复杂,使用不方便,
所以数学家会编制函数表,
例如三角函数表、对数表、平方根表、立方根表等,
接着借助函数表,
构造一个既能反映数量关系又便于计算的简单函数\(P\),
用\(P\)近似\(f\).
通常选用的简单函数包括多项式函数、分段多项式函数、有理分式、三角多项式等,
我们把这类函数称为\DefineConcept{插值函数},
把自变量\(x_0,\dotsc,x_n\)称为\DefineConcept{插值节点},
把包含插值节点的区间\([a,b]\)称为\DefineConcept{插值区间},
把求解插值函数的方法称为\DefineConcept{插值法}.
