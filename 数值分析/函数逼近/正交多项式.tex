\section{正交多项式}
\subsection{正交函数族,正交多项式函数族}
\begin{definition}
%@see: 《数值分析(第5版)》(李庆扬、王能超、易大义) P57 定义5
设区间\(D \subseteq \mathbb{R}\),
\(f,g\)都是\(D\)上的连续函数,
\(\rho\)是\(D\)上的一个权函数.
如果函数\(f\)与\(g\)的带权\(\rho\)的内积为零,
即\begin{equation*}
%@see: 《数值分析(第5版)》(李庆扬、王能超、易大义) P57 (2.1)
	(f,g)
	= \int_D \rho(x) f(x) g(x) \dd{x}
	= 0,
\end{equation*}
则称“\(f\)与\(g\)在\(D\)上\DefineConcept{带权\(\rho\)正交}”.
\end{definition}

\begin{definition}
%@see: 《数值分析(第5版)》(李庆扬、王能超、易大义) P57 定义5
设区间\(D \subseteq \mathbb{R}\)上的一个连续函数族\(\{\phi_n\}_{n\geq0}\)满足\begin{equation*}
%@see: 《数值分析(第5版)》(李庆扬、王能超、易大义) P57 (2.2)
	(\phi_i,\phi_j)
	= \int_D \rho(x) \phi_i(x) \phi_j(x) \dd{x}
	= \begin{cases}[cl]
		0, & i \neq j, \\
		A_j, & i = j,
	\end{cases}
\end{equation*}
其中\(A_j > 0\ (j=0,1,2,\dotsc)\)是常数,
则称“\(\{\phi_n\}_{n\geq0}\)是\(D\)上的一个\DefineConcept{带权\(\rho\)正交函数族}”.
特别地,如果\(A_j = 1\ (j=0,1,2,\dotsc)\),
则称“\(\{\phi_n\}_{n\geq0}\)是\(D\)上的一个\DefineConcept{(带权\(\rho\))标准正交函数族}”.
\end{definition}

\begin{example}
三角函数族\begin{equation*}
	1,\cos x,\sin x,\cos2x,\sin2x,\dotsc
\end{equation*}
就是在区间\([-\pi,\pi]\)上的正交函数族,
这是因为\begin{gather*}
	(1,1) = 2\pi, \\
	(\sin m x,\sin m x)
	= (\cos m x,\cos m x)
	= \pi
	\quad(m=1,2,\dotsc), \\
	(1,\cos m x)
	= (1,\sin m x)
	= 0
	\quad(m=1,2,\dotsc), \\
	(\cos m x,\sin n x)
	= 0
	\quad(m,n=1,2,\dotsc), \\
	(\cos m x,\cos n x)
	= (\sin m x,\sin n x)
	= 0
	\quad(m,n=1,2,\dotsc;m \neq n).
\end{gather*}
\end{example}

\begin{definition}
%@see: 《数值分析(第5版)》(李庆扬、王能超、易大义) P57 定义6
设\(\{\phi_n\}_{n\geq0}\)是区间\(D \subseteq \mathbb{R}\)上的首项系数\(a_n\neq0\)的\(n\)次多项式函数族,
\(\rho\)是\(D\)上的一个权函数.
如果\(\{\phi_n\}_{n\geq0}\)是带权\(\rho\)的正交函数族,
则称“\(\{\phi_n\}_{n\geq0}\)是\(D\)上的一个\DefineConcept{带权\(\rho\)正交多项式函数族}”.
\end{definition}

只要给定区间\(D\)和权函数\(\rho\),
我们总可以基于一族线性无关的幂函数族\(\{t_n\}_{n\geq0}\)
(例如实数域上一元多项式环的标准基\(\{1,x,\dotsc,x^n,\dotsc\}\)),
通过逐个正交化,构造出带权\(\rho\)正交多项式函数族\(\{\phi_n\}_{n\geq0}\):\begin{equation*}
	\phi_0(x) \defeq 1,
	\qquad
	\phi_n(x) \defeq x^n - \sum_{j=0}^{n-1} \frac{
			(
				t_n,
				\phi_j
			)
		}{
			(
				\phi_j,
				\phi_j
			)
		}
		\phi_j(x)
	\quad(n=1,2,\dotsc).
\end{equation*}
这样得到的正交多项式\(\phi_n\)的最高项系数为\(1\).
这是因为,如果假设\(c_0,c_1,\dotsc,c_n \in \mathbb{R}\)满足\begin{equation*}
	c_0 \phi_0(x) + c_1 \phi_1(x) + \dotsb + c_n \phi_n(x) = 0,
\end{equation*}
那么用\(\rho(x) \phi_j(x)\ (j=0,1,2,\dotsc,n)\)依次去乘上式并积分得\begin{equation*}
	\sum_{i=0}^n c_i \int_D \rho(x) \phi_i(x) \phi_j(x) \dd{x} = 0
	\quad(j=0,1,2,\dotsc,n).
\end{equation*}
由正交性有,当\(i \neq j\)时有\(\int_D \rho(x) \phi_i(x) \phi_j(x) \dd{x} = 0\),
于是上式化为\begin{equation*}
	c_j \int_D \rho(x) \phi_j(x) \phi_j(x) \dd{x} = 0
	\quad(j=0,1,2,\dotsc,n).
\end{equation*}
由于\begin{equation*}
	(\phi_j,\phi_j)
	= \int_D \rho(x) \phi_j^2(x) \dd{x}
	> 0
	\quad(j=0,1,2,\dotsc,n),
\end{equation*}
故\(c_j = 0\ (j=0,1,2,\dotsc,n)\).
由此可知\(\phi_0,\phi_1,\dotsc,\phi_n\)线性无关.
%TODO 没有证明\(\phi_n\)一定是最高项系数为\(1\)的幂函数
可以证明:
\begin{enumerate}
	\item 区间\(D\)上的次数不超过\(n\)的任意一个多项式函数\(P\)
	均可表示成\(\phi_0,\phi_1,\dotsc,\phi_n\)的线性组合,
	即\begin{equation*}
		P(x) = \sum_{j=0}^n c_j \phi_j(x).
	\end{equation*}

	\item 函数\(\phi_n\)与区间\(D\)上次数小于\(n\)的任意一个多项式函数\(P\)正交,
	即\begin{equation*}
		(\phi_n,P)
		= \int_D \rho(x) \phi_n(x) P(x) \dd{x}
		= 0.
	\end{equation*}
\end{enumerate}

正交多项式还有一些重要性质.
\begin{theorem}
%@see: 《数值分析(第5版)》(李庆扬、王能超、易大义) P58 定理4
设\(\{\phi_n\}_{n\geq0}\)是\(D\)上的一个带权\(\rho\)正交多项式函数族,
那么\begin{equation}
	\phi_{n+1}(x)
	= (x - \alpha_n) \phi_n(x)
	- \beta_n \phi_{n-1}(x)
	\quad(n=0,1,2,\dotsc),
\end{equation}
其中\begin{gather*}
	\phi_0(x) = 1,
	\qquad
	\phi_{-1}(x) = 0, \\
	\alpha_n
	= \frac{(x \phi_n,\phi_n)}{(\phi_n,\phi_n)}
	\quad(n=1,2,\dotsc), \\
	\beta_n
	= \frac{(\phi_n,\phi_n)}{(\phi_{n-1},\phi_{n-1})}
	\quad(n=1,2,\dotsc).
\end{gather*}
%TODO proof
% 这里\((x \phi_n,\phi_n) = \int_D x \phi_n^2(x) \rho(x) \dd{x}\).
\end{theorem}

\begin{theorem}
%@see: 《数值分析(第5版)》(李庆扬、王能超、易大义) P58 定理5
设\(\{\phi_n\}_{n\geq0}\)是区间\(D\)上的一个带权\(\rho\)正交多项式函数族,
则\(\phi_n\ (n\geq1)\)在\(D\)的内部\(\TopoInterior{D}\)内有\(n\)个零点.
%TODO proof
\end{theorem}

\subsection{勒让德多项式}
当区间为\(D \defeq [-1,1]\),权函数为\(\rho(x) \defeq 1\)时,
由\(\{1,x,\dotsc,x^n,\dotsc\}\)正交化得到的多项式函数族
称为\DefineConcept{勒让德多项式函数族}(Legendre polynomials).
%@see: https://mathworld.wolfram.com/LegendrePolynomial.html
这是勒让德于1785年引进的.
1814年罗德里克给出了勒让德多项式的简单表达式\begin{equation}
%@see: 《数值分析(第5版)》(李庆扬、王能超、易大义) P59 (2.5)
	P_0(x) \defeq 1,
	\qquad
	P_n(x) \defeq \frac1{n!2^n} \dv[n]{x} (x^2-1)^n
	\quad(n=1,2,\dotsc).
\end{equation}

勒让德多项式还可以表示为\begin{equation}
	P_n(z)
	= \frac1{2\pi\iu} \oint_\Gamma (1 - 2 t z + t^2)^{-1/2} t^{-n-1} \dd{t},
\end{equation}
其中围线\(\Gamma\)包围原点且沿逆时针方向行进.

由于\((x^2-1)^n\)是\(2n\)次多项式,
求\(n\)阶导数后得\begin{equation*}
	P_n(x) = \frac1{n!2^n} (2n)(2n-1)\dotsm(n+1)x^n + a_{n-1} x^{n-1} + \dotsb + a_0,
\end{equation*}
于是首项\(x^n\)的系数为\(a_n = \frac{(2n)!}{(n!)^2 2^n}\).
显然最高项系数为\(1\)的勒让德多项式为\begin{equation*}
%@see: 《数值分析(第5版)》(李庆扬、王能超、易大义) P59 (2.6)
	\tilde{P}_n(x)
	= \frac{n!}{(2n)!} \dv[n]{x} (x^2-1)^n.
\end{equation*}

勒让德多项式具有以下几个重要性质.
\begin{property}
%@see: 《数值分析(第5版)》(李庆扬、王能超、易大义) P59 性质1(正交性)
设\(\{P_n\}_{n\geq0}\)是勒让德多项式函数族,
则\begin{equation}
%@see: 《数值分析(第5版)》(李庆扬、王能超、易大义) P59 (2.7)
	\int_{-1}^1 P_n(x) P_m(x) \dd{x}
	= \begin{cases}
		0, & m \neq n, \\
		\frac2{2n+1}, & m = n.
	\end{cases}
\end{equation}
\end{property}

\begin{property}
%@see: 《数值分析(第5版)》(李庆扬、王能超、易大义) P60 性质2(奇偶性)
设\(\{P_n\}_{n\geq0}\)是勒让德多项式函数族,
则当\(n\)是偶数时\(P_n\)是偶函数,
当\(n\)是奇数时\(P_n\)是奇函数,
即\begin{equation}
%@see: 《数值分析(第5版)》(李庆扬、王能超、易大义) P60 (2.8)
	P_n(-x) = (-1)^n P_n(x).
\end{equation}
\end{property}

\begin{property}
%@see: 《数值分析(第5版)》(李庆扬、王能超、易大义) P60 性质3(递推关系)
设\(\{P_n\}_{n\geq0}\)是勒让德多项式函数族,
则\begin{gather}
%@see: 《数值分析(第5版)》(李庆扬、王能超、易大义) P61 (2.9)
	P_0(x) = 1,
	\qquad
	P_1(x) = x, \\
	P_{n+1}(x)
	= \frac{2n+1}{n+1} x P_n(x)
	- \frac{n}{n+1} P_{n-1}(x)
	\quad(n=1,2,\dotsc).
\end{gather}
\end{property}

利用递推公式可以推出:\begin{align*}
	P_2(x) &= (3x^2-1)/2, \\
	P_3(x) &= (5x^3-3x)/2, \\
	P_4(x) &= (35x^4-30x^2+3)/8, \\
	P_5(x) &= (63x^5-70x^3+15x)/8, \\
	P_6(x) &= (231x^6-315x^4+105x^2-5)/16, \\
	&\vdots
\end{align*}

\begin{property}
%@see: 《数值分析(第5版)》(李庆扬、王能超、易大义) P60 性质4
设\(\{P_n\}_{n\geq0}\)是勒让德多项式函数族,
则\(P_n\)在区间\([-1,1]\)内有\(n\)个不同的实零点.
\end{property}

\subsection{第一类切比雪夫多项式}
当区间为\(D \defeq [-1,1]\),权函数为\(\rho(x) \defeq \frac1{\sqrt{1-x^2}}\)时,
由\(\{1,x,\dotsc,x^n,\dotsc\}\)正交化得到的多项式函数族
称为\DefineConcept{第一类切比雪夫多项式函数族}(Chebyshev polynomials of the first kind),
%@see: https://mathworld.wolfram.com/ChebyshevPolynomialoftheFirstKind.html
它可以表示为\begin{equation}
%@see: 《数值分析(第5版)》(李庆扬、王能超、易大义) P61 (2.10)
	T_n(x) \defeq \cos(n \arccos x)
	\quad(-1 \leq x \leq 1).
\end{equation}

第一类切比雪夫多项式还可以表示为\begin{equation}
	T_n(z)
	= \frac1{4\pi\iu} \oint_\Gamma \frac{(1 - t^2) t^{-n-1}}{1 - 2 t z + t^2} \dd{t},
\end{equation}
其中围线\(\Gamma\)包围原点且沿逆时针方向行进.

第一类切比雪夫多项式具有以下几个重要性质.
\begin{property}
%@see: 《数值分析(第5版)》(李庆扬、王能超、易大义) P61 性质1(递推关系)
设\(\{T_n\}_{n\geq0}\)是第一类切比雪夫多项式函数族,
则\begin{gather}
%@see: 《数值分析(第5版)》(李庆扬、王能超、易大义) P61 (2.11)
	T_0(x) = 1,
	\qquad
	T_1(x) = x,
	\qquad
	T_{n+1}(x) = 2x T_n(x) - T_{n-1}(x)
	\quad(n=1,2,\dotsc).
\end{gather}
\end{property}
\begin{remark}
由递推关系可以推得\(T_n(x)\)的最高次项系数为\(2^{n-1}\ (n=1,2,\dotsc)\).
\end{remark}

利用递推公式可以推出:\begin{align*}
	T_2(x) &= 2x^2-1, \\
	T_3(x) &= 4x^3-3x, \\
	T_4(x) &= 8x^4-8x^2+1, \\
	T_5(x) &= 16x^5-20x^3+5x, \\
	T_6(x) &= 32x^6-48x^4+18x^2-1, \\
	&\vdots
\end{align*}

\begin{property}
%@see: 《数值分析(第5版)》(李庆扬、王能超、易大义) P62 性质2
设\(\{T_n\}_{n\geq0}\)是第一类切比雪夫多项式函数族,
则\(T_i\)和\(T_j\)在\([-1,1]\)上带权\(\rho\)正交,
且\begin{equation}
%@see: 《数值分析(第5版)》(李庆扬、王能超、易大义) P62 (2.12)
	\def\arraystretch{1.5}
	\int_{-1}^1 \frac{T_i(x) T_j(x)}{\sqrt{1-x^2}} \dd{x}
	= \begin{cases}
		0, & i \neq j, \\
		\frac\pi2, & i = j \neq 0, \\
		\pi, & i = j = 0.
	\end{cases}
\end{equation}
\end{property}

\begin{property}
%@see: 《数值分析(第5版)》(李庆扬、王能超、易大义) P62 性质3
设\(\{T_n\}_{n\geq0}\)是第一类切比雪夫多项式函数族,
那么\(T_{2k}(x)\)只含\(x\)的偶次幂,
而\(T_{2k+1}(x)\)只含\(x\)的奇次幂.
\end{property}

\begin{property}
%@see: 《数值分析(第5版)》(李庆扬、王能超、易大义) P62 性质4
设\(\{T_n\}_{n\geq0}\)是第一类切比雪夫多项式函数族,
则\(T_n\)在\([-1,1]\)上有\(n\)个零点:\begin{equation}
	x_k = \cos\frac{2k-1}{2n}\pi
	\quad(k=1,2,\dotsc,n).
\end{equation}
\end{property}

\begin{property}
%@see: 《数值分析(第5版)》(李庆扬、王能超、易大义) P62 性质5
设\(\{T_n\}_{n\geq0}\)是第一类切比雪夫多项式函数族,
则\(T_n\)的首项\(x^n\)的系数为\(2^{n-1}\ (n=1,2,\dotsc)\).
\end{property}

若令\begin{equation*}
	\tilde{T}_0(x) \defeq 1,
	\qquad
	\tilde{T}_n(x) \defeq \frac1{2^{n-1}} T_n(x)
	\quad(n=1,2,\dotsc),
\end{equation*}
则\(\tilde{T}_n\)的首项系数为\(1\).

\begin{theorem}
%@see: 《数值分析(第5版)》(李庆扬、王能超、易大义) P62 定理6
设\(\tilde{H}_n\)是所有次数不大于\(n\)的、首项系数为\(1\)的多项式的集合,
则对于任意函数\(P \in \tilde{H}_n\),
有\begin{equation*}
	\max_{-1 \leq x \leq 1} \abs{\tilde{T}_n(x)}
	\leq \max_{-1 \leq x \leq 1} \abs{P(x)},
\end{equation*}
且\begin{equation*}
	\max_{-1 \leq x \leq 1} \abs{\tilde{T}_n(x)}
	= \frac1{2^{n-1}}.
\end{equation*}
\end{theorem}
\begin{remark}
上述定理表明,
在所有首项系数为\(1\)的\(n\)次多项式的集合\(\tilde{H}_n\)中,
有\begin{equation*}
	\norm{\tilde{T}_n}_\infty
	= \min_{P \in \tilde{H}_n} \norm{P}_\infty,
\end{equation*}
所有\(\tilde{T}_n\)是\(\tilde{H}_n\)中最大值最小的多项式,
即\begin{equation*}
%@see: 《数值分析(第5版)》(李庆扬、王能超、易大义) P63 (2.13)
	\max_{-1 \leq x \leq 1} \abs{\tilde{T}_n(x)}
	= \min_{P \in \tilde{H}_n} \max_{-1 \leq x \leq 1} \abs{P(x)}
	= \frac1{2^{n-1}}.
\end{equation*}
利用这个结论,可以求\(P \in H_n\)在\(H_{n-1}\)中的最佳一致逼近多项式.
\end{remark}

由于切比雪夫多项式是在区间\([-1,1]\)上定义的,
所以对于任意给定的闭区间\([a,b]\),
要通过变换\begin{equation*}
%@see: 《数值分析(第5版)》(李庆扬、王能超、易大义) P63 (2.14)
	x = \frac{b-a}{2} t + \frac{a+b}{2}
\end{equation*}
将\([a,b]\)映成\([-1,1]\).

\subsection{切比雪夫多项式零点插值}
切比雪夫多项式\(T_n\)在区间\([-1,1]\)上有\(n\)个零点\begin{equation*}
	x_k = \cos\frac{2k-1}{2n}\pi
	\quad(k=1,2,\dotsc,n)
\end{equation*}
和\(n+1\)个极值点(包括端点)\begin{equation*}
	x_k = \cos\frac{k}{n}\pi
	\quad(k=0,1,2,\dotsc,n).
\end{equation*}
这两组点合称为\DefineConcept{切比雪夫点}.
它们在插值中有重要作用.
切比雪夫点恰好是单位圆周上等距分布的点的横坐标.
这些点的横坐标在接近区间\([-1,1]\)的端点处是密集的.

利用切比雪夫点做插值,可以使插值区间最大误差最小化.
具体来说,对于区间\([-1,1]\)上的\(n+1\)阶连续可导函数\(f\),
如果将\(T_{n+1}\)的零点\(
	x_k = \cos\frac{2k+1}{2(n+1)}\pi
	\ (k=0,1,2,\dotsc,n)
\)作为插值节点,
那么有\begin{equation*}
	\max_{-1 \leq x \leq 1} \abs{\omega_{n+1}(x)}
	= \max_{-1 \leq x \leq 1} \abs{\tilde{T}_{n+1}(x)}
	= \frac1{2^n}.
\end{equation*}

\begin{theorem}
%@see: 《数值分析(第5版)》(李庆扬、王能超、易大义) P64 定理7
设插值节点\(x_0,x_1,\dotsc,x_n\)是第一类切比雪夫多项式\(T_{n+1}\)的零点,
被插函数\(f\)在\([-1,1]\)上具有\(n+1\)阶连续导数,
\(L_n\)是拉格朗日插值多项式,
则\begin{equation*}
%@see: 《数值分析(第5版)》(李庆扬、王能超、易大义) P64 (2.15)
	\max_{-1 \leq x \leq 1} \abs{f(x) - L_n(x)}
	\leq \frac{1}{(n+1)! 2^n} \norm{f^{(n+1)}(x)}_\infty.
\end{equation*}
\end{theorem}

\subsection{第二类切比雪夫多项式}
当区间为\(D \defeq [-1,1]\),权函数为\(\rho(x) \defeq \sqrt{1-x^2}\)时,
由\(\{1,x,\dotsc,x^n,\dotsc\}\)正交化得到的多项式函数族
称为\DefineConcept{第二类切比雪夫多项式函数族}(Chebyshev polynomials of the second kind),
%@see: https://mathworld.wolfram.com/ChebyshevPolynomialoftheSecondKind.html
它可以表示为\begin{equation}
%@see: 《数值分析(第5版)》(李庆扬、王能超、易大义) P66 (2.16)
	U_n(x)
	\defeq
	\frac{\sin((n+1) \arccos x)}{\sqrt{1-x^2}}.
\end{equation}

第二类切比雪夫多项式还可以表示为\begin{equation}
	U_n(z)
	= \frac1{2\pi\iu} \oint_\Gamma \frac{1-t^{n+1}}{(1 - 2 z t + t^2) t^{n+1}} \dd{t},
\end{equation}
其中围线\(\Gamma\)包围原点且沿逆时针方向行进.

\begin{property}
%@see: 《数值分析(第5版)》(李庆扬、王能超、易大义) P66
设\(\{U_n\}_{n\geq0}\)是第二类切比雪夫多项式函数族,
则\begin{gather}
	U_0(x) = 1,
	\qquad
	U_1(x) = 2x,
	\qquad
	U_{n+1}(x) = 2x U_n(x) - U_{n-1}(x)
	\quad(n=1,2,\dotsc).
\end{gather}
\end{property}

\begin{property}
%@see: 《数值分析(第5版)》(李庆扬、王能超、易大义) P66
设\(\{U_n\}_{n\geq0}\)是第二类切比雪夫多项式函数族,
则\(U_i\)和\(U_j\)在\([-1,1]\)上带权\(\rho\)正交,
且\begin{equation}
	\def\arraystretch{1.5}
	\int_{-1}^1 U_i(x) U_j(x) \sqrt{1-x^2} \dd{x}
	= \begin{cases}
		0, & i \neq j, \\
		\frac\pi2, & i = j.
	\end{cases}
\end{equation}
\end{property}

\subsection{拉盖尔多项式}
当区间为\(D \defeq [0,+\infty)\),权函数为\(\rho(x) \defeq e^{-x}\)时,
由\(\{1,x,\dotsc,x^n,\dotsc\}\)正交化得到的多项式函数族
称为\DefineConcept{拉盖尔多项式}(Laguerre polynomials),
%@see: https://mathworld.wolfram.com/LaguerrePolynomial.html
它可以表示为\begin{equation}
%@see: 《数值分析(第5版)》(李庆扬、王能超、易大义) P66 (2.17)
	L_n(x)
	\defeq
	e^x \dv[n]{x} (x^n e^{-x}).
\end{equation}

拉盖尔多项式还可以表示为\begin{equation}
	L_n(z) = \frac1{2\pi\iu} \oint_\Gamma \frac{e^{-zt/(1-t)}}{(1-t)t^{n+1}} \dd{t},
\end{equation}
其中围线\(\Gamma\)包围原点且沿逆时针方向行进.

\begin{property}
%@see: 《数值分析(第5版)》(李庆扬、王能超、易大义) P66
设\(\{L_n\}_{n\geq0}\)是拉盖尔多项式函数族,
则\begin{gather}
	L_0(x) = 1,
	\qquad
	L_1(x) = 1-x, \\
	L_{n+1}(x) = (1+2n-x) L_n(x) - n^2 L_{n-1}(x)
	\quad(n=1,2,\dotsc).
\end{gather}
\end{property}

\begin{property}
%@see: 《数值分析(第5版)》(李庆扬、王能超、易大义) P66
设\(\{L_n\}_{n\geq0}\)是拉盖尔多项式函数族,
则\(L_i\)和\(L_j\)在\([-1,1]\)上带权\(\rho\)正交,
且\begin{equation}
	\def\arraystretch{1.5}
	\int_0^{+\infty} L_i(x) L_j(x) e^{-x} \dd{x}
	= \begin{cases}
		0, & i \neq j, \\
		(n!)^2, & i = j.
	\end{cases}
\end{equation}
\end{property}

\subsection{厄米多项式}
当区间为\(D \defeq (-\infty,+\infty)\),权函数为\(\rho(x) \defeq e^{-x^2}\)时,
由\(\{1,x,\dotsc,x^n,\dotsc\}\)正交化得到的多项式函数族
称为\DefineConcept{厄米多项式}(Hermite polynomials),
%@see: https://mathworld.wolfram.com/HermitePolynomial.html
它可以表示为\begin{equation}
%@see: 《数值分析(第5版)》(李庆扬、王能超、易大义) P66 (2.18)
	H_n(x)
	\defeq
	(-1)^n e^{x^2} \dv[n]{x} e^{-x^2}.
\end{equation}

厄米多项式还可以表示为\begin{equation}
	H_n(z) = \frac{n!}{2\pi\iu} \oint_\Gamma e^{-t^2 + 2 t z} t^{-n-1} \dd{t}.
\end{equation}
其中围线\(\Gamma\)包围原点且沿逆时针方向行进.

\begin{property}
%@see: 《数值分析(第5版)》(李庆扬、王能超、易大义) P66
设\(\{H_n\}_{n\geq0}\)是厄米多项式函数族,
则\begin{gather}
	H_0(x) = 1,
	\qquad
	H_1(x) = 2x, \\
	H_{n+1}(x) = 2x H_n(x) - 2n H_{n-1}(x)
	\quad(n=1,2,\dotsc).
\end{gather}
\end{property}

\begin{property}
%@see: 《数值分析(第5版)》(李庆扬、王能超、易大义) P66
设\(\{H_n\}_{n\geq0}\)是厄米多项式函数族,
则\(H_i\)和\(H_j\)在\([-1,1]\)上带权\(\rho\)正交,
且\begin{equation}
	\def\arraystretch{1.5}
	\int_{-\infty}^{+\infty} H_i(x) H_j(x) e^{-x^2} \dd{x}
	= \begin{cases}
		0, & i \neq j, \\
		2^n n! \sqrt{\pi}, & i = j.
	\end{cases}
\end{equation}
\end{property}
