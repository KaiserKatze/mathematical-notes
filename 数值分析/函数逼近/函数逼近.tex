\section{函数逼近的基本概念}
函数逼近问题:
给定函数\(f\)(称之为目标函数),
我们想要在某个函数空间\(\Phi\)中找到一个函数\(\phi\)
使得误差\(f\)与\(\phi\)之间的距离在某种度量意义下最小.
\begin{definition}
% 这是我自己构思的定义
设\(D_f \subseteq D_\phi \subseteq \mathbb{R}\),
取定\(\Phi \subseteq \mathbb{R}^{D_\phi}\),
给定函数\(f\colon D_f \to \mathbb{R}\),
给定\(\Phi\)的一个度量\(\rho\),
把\begin{equation*}
	\argmin_{\phi \in \Phi} \rho(f,\phi \SetRestrict D_f)
\end{equation*}
称为“函数\(f\)在\(D_\phi\)上的\DefineConcept{最佳逼近函数}”.
\end{definition}

我们通常将函数空间\(\Phi\)限定为某个区间上的连续函数族\(C[a,b]\),
或者进一步限定为某个区间上的次数不超过\(n\)的多项式函数族\(H_n[a,b]\).
