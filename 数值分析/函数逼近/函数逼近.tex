\section{函数逼近的基本概念}
函数逼近问题:
给定函数\(f\)(称之为目标函数),
我们想要在某个函数空间\(\Phi\)中找到一个函数\(\phi\)
使得误差\(f\)与\(\phi\)之间的距离在某种度量意义下最小.
\begin{definition}
% 这是我自己构思的定义
设\(D_f \subseteq D_\phi \subseteq \mathbb{R}\),
取定\(\Phi \subseteq \mathbb{R}^{D_\phi}\),
给定函数\(f\colon D_f \to \mathbb{R}\),
给定\(\Phi\)的一个度量\(\rho\),
把\begin{equation*}
	\argmin_{\phi \in \Phi} \rho(f,\phi \SetRestrict D_f)
\end{equation*}
称为“函数\(f\)在\(D_\phi\)上的\DefineConcept{最佳逼近函数}”.
\end{definition}

我们通常将函数空间\(\Phi\)限定为某个区间上的连续函数族\(C[a,b]\),
或者进一步限定为某个区间上的次数不超过\(n\)的多项式函数族\(H_n[a,b]\).

\begin{definition}
%@see: 《数值分析(第5版)》(李庆扬、王能超、易大义) P55 定义4
设\(D \subseteq \mathbb{R}\).
如果函数\(\rho\colon D \to [0,+\infty)\)满足\begin{itemize}
	\item 积分\(\int_D x^k \rho(x) \dd{x}\ (k=0,1,2,\dotsc)\)存在且有限;
	\item 对于任意函数\(g\colon D \to [0,+\infty)\)
	总有\(\int_D g(x) \rho(x) \dd{x} = 0\)
	蕴含\((\forall x \in D)[g(x) = 0]\),
\end{itemize}
则称“\(\rho\)是定义域\(D\)上的一个\DefineConcept{权函数}”.
\end{definition}
\begin{remark}
上述定义中,函数\(\rho\)的定义域\(D\)既可以是有限区间,也可以是无限区间.
\end{remark}

\begin{definition}
%@see: 《数值分析(第5版)》(李庆扬、王能超、易大义) P55 例2
设\(f,g \in C[a,b]\).
对于\([a,b]\)上的任意一个权函数\(\rho\),
我们把积分\begin{equation*}
%@see: 《数值分析(第5版)》(李庆扬、王能超、易大义) P55 (1.15)
	\int_a^b \rho(x) f(x) g(x) \dd{x}
\end{equation*}
称为“函数\(f\)与\(g\)的\DefineConcept{带权\(\rho\)的内积}”,
记作\((f,g)_\rho\).
\end{definition}

显然两个函数的带权内积满足内积的四条性质,
由它导出的范数\begin{equation*}
%@see: 《数值分析(第5版)》(李庆扬、王能超、易大义) P55 (1.16)
	\norm{f}_{2,\rho}
	\defeq
	\sqrt{(f,f)_\rho}
	= \sqrt{
		\int_a^b \rho(x) f^2(x) \dd{x}
	}
\end{equation*}
称为“函数\(f\)的\DefineConcept{带权\(\rho\)的范数}”.
