\section{函数逼近的基本概念}
函数逼近问题:
给定函数\(f\)(称之为目标函数),
我们想要在某个函数空间\(\Phi\)中找到一个函数\(\phi\)
使得误差\(f\)与\(\phi\)之间的距离在某种度量意义下最小.
\begin{definition}
% 这是我自己构思的定义
设\(D_f \subseteq D \subseteq \mathbb{R}\),
给定函数\(f\colon D_f \to \mathbb{R}\)(\(f\)可能是连续的,也可能离散的),
取定\(\Phi \subseteq \mathbb{R}^D\),
给定\(\Phi\)的一个度量\(\rho\),
把\begin{equation*}
	\argmin_{\phi \in \Phi} \rho(f,\phi \SetRestrict D_f)
\end{equation*}
称为“函数\(f\)在\(D\)上的\DefineConcept{最佳逼近函数}”.
\end{definition}

我们通常将函数空间\(\Phi\)限定为区间\([a,b]\)上的连续函数族\(C[a,b]\),
或者进一步限定为区间\([a,b]\)上的次数不超过\(n\)的多项式函数族\(H_n[a,b]\).
在这种情况下,把\begin{equation*}
	\argmin_{\phi \in H_n} \rho(f,\phi \SetRestrict D_f)
\end{equation*}
称为“函数\(f\)在\(D\)上的\DefineConcept{最佳逼近多项式}”.

如果度量\(\rho\)取为 \hyperref[equation:范数.连续函数的无穷范数]{\(L_\infty\)范数},
那么把\begin{equation*}
	\argmin_{\phi \in H_n} \max_{x \in D_f} \abs{f(x) - \phi(x)}
\end{equation*}
称为“函数\(f\)在\(D\)上的\DefineConcept{最佳一致逼近多项式}”.

如果度量\(\rho\)取为 \hyperref[equation:范数.连续函数的L2范数]{\(L_2\)范数},
且\(f\)是连续的,
那么把\begin{equation*}
	\argmin_{\phi \in H_n} \int_{D_f} (f(x) - \phi(x))^2 \dd{x}
\end{equation*}
称为“函数\(f\)在\(D\)上的\DefineConcept{最佳平方逼近多项式}”.

如果度量\(\rho\)取为 \hyperref[equation:范数.连续函数的L2范数]{\(L_2\)范数},
且\(f\)是离散的,
那么把\begin{equation*}
	\argmin_{\phi \in H_n} \sum_{x \in D_f} (f(x) - \phi(x))^2
\end{equation*}
称为“函数\(f\)在\(D\)上的\DefineConcept{最小二乘拟合}”.

本章将着重讨论实际应用多且便于计算的最佳平方逼近与最小二乘拟合.
