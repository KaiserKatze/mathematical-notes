\section{最佳平方逼近}
\subsection{最佳平方逼近的概念,法方程,误差}
\begin{definition}
设区间\(D \subseteq \mathbb{R}\),
\(\rho\)是\(D\)上的一个权函数,
给定连续函数\(f\colon D \to \mathbb{R}\),
取定\(\Phi \subseteq C(D)\),
把\begin{equation*}
%@see: 《数值分析(第5版)》(李庆扬、王能超、易大义) P67 (3.1)
	\argmin_{\phi \in \Phi} \int_D \rho(x) (f(x) - \phi(x))^2 \dd{x}
\end{equation*}
称为“函数\(f\)在\(D\)上的\DefineConcept{最佳平方逼近函数}”.
\end{definition}

假设连续函数族\(\Phi\)可以由一个基\(\{\phi_0,\phi_1,\dotsc,\phi_n\}\)线性表出,
那么\begin{equation*}
%@see: 《数值分析(第5版)》(李庆扬、王能超、易大义) P67 (3.2)
	\argmin_{\phi \in \Phi} \int_D \rho(x) (f(x) - \phi(x))^2 \dd{x}
	= \argmin_{a_0,a_1,\dotsc,a_n} \int_D \rho(x) \left( \sum_{j=0}^n a_j \phi_j(x) - f(x) \right)^2 \dd{x}.
\end{equation*}
由于\begin{equation*}
	I(a_0,a_1,\dotsc,a_n)
	\defeq
	\int_D \rho(x) \left( \sum_{j=0}^n a_j \phi_j(x) - f(x) \right)^2 \dd{x}
\end{equation*}
是关于\(a_0,a_1,\dotsc,a_n\)的二次函数,
那么利用多元函数求极值的必要条件可得\begin{equation*}
	\pdv{I}{a_k}
	= 2 \int_D \rho(x) \left( \sum_{j=0}^n a_j \phi_j(x) - f(x) \right) \phi_k(x) \dd{x}
	= 0
	\quad(k=0,1,2,\dotsc,n),
\end{equation*}
于是有\begin{equation}\label{equation:最佳平方逼近.法方程}
%@see: 《数值分析(第5版)》(李庆扬、王能超、易大义) P67 (3.3)
	\sum_{j=0}^n a_j (\phi_k,\phi_j)
	= (f,\phi_k)
	\quad(k=0,1,2,\dotsc,n)
\end{equation}
或\begin{equation}%\label{equation:最佳平方逼近.法方程.矩阵形式}
	\begin{bmatrix}
		(\phi_0,\phi_0) & (\phi_0,\phi_1) & \dots & (\phi_0,\phi_n) \\
		(\phi_1,\phi_0) & (\phi_1,\phi_1) & \dots & (\phi_1,\phi_n) \\
		\vdots & \vdots & & \vdots \\
		(\phi_n,\phi_0) & (\phi_n,\phi_1) & \dots & (\phi_n,\phi_n) \\
	\end{bmatrix}
	\begin{bmatrix}
		a_0 \\ a_1 \\ \vdots \\ a_n
	\end{bmatrix}
	= \begin{bmatrix}
		(f,\phi_0) \\
		(f,\phi_1) \\
		\vdots \\
		(f,\phi_n) \\
	\end{bmatrix}.
\end{equation}
\cref{equation:最佳平方逼近.法方程}
是关于\(a_0,a_1,\dotsc,a_n\)的线性方程组,
我们称之为\DefineConcept{法方程}.
由于\(\phi_0,\phi_1,\dotsc,\phi_n\)线性无关,
所以\(\phi_0,\phi_1,\dotsc,\phi_n\)的格拉姆行列式
\(\DeterminantA{G(\phi_0,\phi_1,\dotsc,\phi_n)} \neq 0\),
于是线性方程组 \labelcref{equation:最佳平方逼近.法方程} 有唯一解.
不妨设线性方程组 \labelcref{equation:最佳平方逼近.法方程} 的解是\(a_k = a^*_k\ (k=0,1,2,\dotsc,n)\),
记\(
	\phi^*
	\defeq \sum_{k=0}^n a^*_k \phi_k
\),
对于任意连续函数\(\phi \in \Phi\),
考虑\begin{align*}
	&\hspace{-15pt}
	\Delta
	\defeq
	\norm{f - \phi}_2^2 - \norm{f - \phi^*}_2^2 \\
	&= \int_D \rho(x) (f(x) - \phi(x))^2 \dd{x}
		- \int_D \rho(x) (f(x) - \phi^*(x))^2 \dd{x} \\
	&= \int_D \rho(x) (\phi(x) - \phi^*(x))^2 \dd{x}
		+ 2 \int_D \rho(x) (\phi^*(x) - \phi(x)) (f(x) - \phi^*(x)) \dd{x}.
\end{align*}
由于\(\phi^*\)的系数\(a^*_k\)是线性方程组 \labelcref{equation:最佳平方逼近.法方程} 的解,
所以\begin{equation*}
	\int_D \rho(x) (f(x) - \phi^*(x)) \phi_k(x) \dd{x} = 0
	\quad(k=0,1,2,\dotsc,n),
\end{equation*}
从而有\begin{equation*}
	\int_D \rho(x) (\phi^*(x) - \phi(x)) (f(x) - \phi^*(x)) \dd{x} = 0,
\end{equation*}
于是\begin{equation*}
	\Delta
	= \int_D \rho(x) (\phi(x) - \phi^*(x))^2 \dd{x}
	\geq 0,
\end{equation*}
因此\begin{equation*}
%@see: 《数值分析(第5版)》(李庆扬、王能超、易大义) P67 (3.4)
	\int_D \rho(x) (f(x) - \phi^*(x))^2 \dd{x}
	\leq \int_D \rho(x) (f(x) - \phi(x))^2 \dd{x}.
\end{equation*}
这就说明函数\(\phi^*\)就是\(f\)在\(D\)上的最佳平方逼近函数.

\subsection{用多项式函数族作最佳平方逼近}
设\(H_n\)是区间\([0,1]\)上的次数不超过\(n\)的多项式函数族.
给定函数\(f \in C[0,1]\).
若取\begin{equation*}
	\phi_k(x) \defeq x^k,
	\qquad
	\rho(x) \defeq 1,
\end{equation*}
则\begin{align*}
	(\phi_j,\phi_k)
	&= \int_0^1 x^j \cdot x^k \dd{x}
	= \int_0^1 x^{j+k} \dd{x}
	= \frac1{j+k+1},
	\\
	(f,\phi_k)
	&= \int_0^1 f(x) \cdot x^k \dd{x},
\end{align*}
显然\(1,x,x^2,\dotsc,x^n\)的格拉姆矩阵\(G(1,x,x^2,\dotsc,x^n)\)
就是希尔伯特矩阵\begin{equation*}
%@see: 《数值分析(第5版)》(李庆扬、王能超、易大义) P68 (3.6)
	H
	\defeq
	\begin{bmatrix}
		1 & 1/2 & \dots & 1/(n+1) \\
		1/2 & 1/3 & \dots & 1/(n+2) \\
		\vdots & \vdots && \vdots \\
		1/(n+1) & 1/(n+2) & \dots & 1/(2n+1) \\
	\end{bmatrix}
\end{equation*}
记\(
	b_k \defeq (f,\phi_k),
	\alpha \defeq (a_0,a_1,\dotsc,a_n)^T,
	\beta \defeq (b_0,b_1,\dotsc,b_n)^T
\),
则\begin{equation*}
%@see: 《数值分析(第5版)》(李庆扬、王能超、易大义) P68 (3.7)
	H \alpha = \beta
\end{equation*}
的解\(\alpha^* \defeq (a^*_0,a^*_1,\dotsc,a^*_n)^T\)
就是\(f\)在区间\([0,1]\)上的最佳平方逼近多项式.

用\(\{1,x,x^2,\dotsc,x^n\}\)作基,
求最佳平方逼近多项式,当\(n\)较大时,
法方程的系数矩阵 --- 希尔伯特矩阵 --- 是高度病态的,
%TODO 为什么是高度病态的?见《数值分析(第5版)》(李庆扬、王能超、易大义)第五章
直接求解法方程是相当困难的,
因此我们通常采用正交多项式函数作基.

\subsection{用正交函数族作最佳平方逼近}
设\(f \in C[a,b]\),
\(\Phi \defeq \Span\{\phi_0,\phi_1,\dotsc,\phi_n\}\).
若\(\phi_0,\phi_1,\dotsc,\phi_n\)是正交函数族,
则法方程的系数矩阵是非奇异矩阵,
法方程的解为\begin{equation*}
	\alpha^*
	\defeq
	(a^*_0,a^*_1,\dotsc,a^*_n),
\end{equation*}
其中\begin{equation*}
%@see: 《数值分析(第5版)》(李庆扬、王能超、易大义) P69 (3.8)
	a^*_k
	\defeq
	\frac{(f,\phi_k)}{(\phi_k,\phi_k)}.
\end{equation*}
于是\(f\)在\(\Phi\)中的最佳平方逼近函数为\begin{equation*}
%@see: 《数值分析(第5版)》(李庆扬、王能超、易大义) P69 (3.9)
	S^*(x)
	\defeq
	\sum_{k=0}^n \frac{(f,\phi_k)}{(\phi_k,\phi_k)} \phi_k(x).
\end{equation*}
