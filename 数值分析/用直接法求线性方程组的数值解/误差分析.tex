\section{误差分析}
%@see: 《数值分析(第5版)》(李庆扬、王能超、易大义) P168 定义7
给定线性方程组\(A X = B\),
如果系数矩阵\(A\)或常数项\(B\)的微小变化,
引起解\(X\)的巨大变化,
则称“线性方程组\(A X = B\)是\DefineConcept{病态的}”或“\(A\)是\DefineConcept{病态的}”;
否则称“线性方程组\(A X = B\)是\DefineConcept{良态的}”或“\(A\)是\DefineConcept{良态的}”.

应该注意,病态性质是系数矩阵的内禀性质.
接下来我们希望找出刻画矩阵的病态性质的量.

考虑线性方程组\(
%@see: 《数值分析(第5版)》(李庆扬、王能超、易大义) P168 (5.3)
	A x = b
\),其中\(A\)是非奇异矩阵.
如果系数矩阵是精确的,常数项有误差\(\delta b\),
那么解也会相应地有误差\(\delta x\),
于是\(A (x + \delta x) = b + \delta b\).
由此可得\begin{equation*}
%@see: 《数值分析(第5版)》(李庆扬、王能超、易大义) P168 (5.4)
	\delta x = A^{-1} (b + \delta b) - A^{-1} b
	= A^{-1} \delta b,
	\qquad
	\norm{\delta x}
	\leq \MatrixNorm{A^{-1}} \norm{\delta b \vphantom{A^{-1}}}.
\end{equation*}
又因为\begin{equation*}
	\norm{b}
	\leq \MatrixNorm{A} \norm{x},
	\qquad
	\frac{1}{\norm{x}}
	\leq \frac{\MatrixNorm{A}}{\norm{b}},
\end{equation*}
于是我们可以得出如下结论.
\begin{theorem}
%@see: 《数值分析(第5版)》(李庆扬、王能超、易大义) P168 定理21
设\(A \in M_n(\mathbb{R})\)是非奇异矩阵.
若\(A x = b \neq 0\)且\(A (x + \delta x) = b + \delta b\),
则\begin{equation*}
	\frac{
		\norm{ \delta x }
	}{
		\norm{ x }
	}
	\leq
	\MatrixNorm{ A^{-1} } \MatrixNorm{ A \vphantom{A^{-1}} }
	\frac{
		\norm{ \delta b }
	}{
		\norm{ b }
	}.
\end{equation*}
\end{theorem}
\begin{remark}
上式给出了解的相对误差的上界,
常数项\(b\)的相对误差可能在解中放大\(\MatrixNorm{ A^{-1} } \MatrixNorm{ A \vphantom{A^{-1}} }\)倍.
\end{remark}

假设常数项是精确的,系数矩阵有误差\(\delta A\),
那么解也会相应地有误差\(\delta x\),
%@see: 《数值分析(第5版)》(李庆扬、王能超、易大义) P168 (5.6)
于是有\begin{equation*}
	(A + \delta A) (x + \delta x) = b,
\end{equation*}
即\((A + \delta A) \delta x = - \delta A x\).
如果\(\delta A\)不受限制的话,那么\(A + \delta A\)可能是一个奇异矩阵.
显然\begin{equation*}
	A + \delta A
	= A + A A^{-1} \delta A
	= A (I + A^{-1} \delta A).
\end{equation*}
根据定理20,%@see: 《数值分析(第5版)》(李庆扬、王能超、易大义) P166 定理20
当\(\norm{A^{-1} \delta A} < 1\)时,
\(I + A^{-1} \delta A\)可逆,
于是\begin{equation*}
	\delta x = - (I + A^{-1} \delta A)^{-1} A^{-1} \delta A x.
\end{equation*}
因此\begin{equation*}
	\norm{\delta x}
	\leq \frac{
		\MatrixNorm{A^{-1}} \MatrixNorm{\delta A} \norm{x}
	}{
		1 - \norm{A^{-1} \delta A}
	}.
\end{equation*}
设\(\MatrixNorm{A^{-1}} \norm{\delta A \vphantom{A^{-1}}} < 1\),
则\begin{equation*}
%@see: 《数值分析(第5版)》(李庆扬、王能超、易大义) P168 (5.7)
	\frac{\norm{\delta x}}{\norm{x}}
	\leq \frac{
		\MatrixNorm{A^{-1}}
		\MatrixNorm{A}
		\frac{
			\MatrixNorm{\delta A}
		}{
			\MatrixNorm{A}
		}
	}{
		1
		- \MatrixNorm{A^{-1}}
		\MatrixNorm{A}
		\frac{
			\MatrixNorm{\delta A}
		}{
			\MatrixNorm{A}
		}
	}.
\end{equation*}

\begin{theorem}
%@see: 《数值分析(第5版)》(李庆扬、王能超、易大义) P169 定理22
设\(A\)是非奇异矩阵.
若\(A x = b \neq 0\)且\((A + \delta A) (x + \delta x) = b\),
那么当\(\MatrixNorm{A^{-1}} \norm{\delta A \vphantom{A^{-1}}} < 1\)时,
有\begin{equation*}
%@see: 《数值分析(第5版)》(李庆扬、王能超、易大义) P168 (5.7)
	\frac{\norm{\delta x}}{\norm{x}}
	\leq \frac{
		\MatrixNorm{A^{-1}}
		\MatrixNorm{A}
		\frac{
			\MatrixNorm{\delta A}
		}{
			\MatrixNorm{A}
		}
	}{
		1
		- \MatrixNorm{A^{-1}}
		\MatrixNorm{A}
		\frac{
			\MatrixNorm{\delta A}
		}{
			\MatrixNorm{A}
		}
	}.
\end{equation*}
\end{theorem}
\begin{remark}
如果\(\delta A\)充分小,那么在\(\MatrixNorm{A^{-1}} \norm{\delta A \vphantom{A^{-1}}} < 1\)的情况下,
上式说明系数矩阵\(A\)的相对误差\(
	\frac{
		\MatrixNorm{\delta A}
	}{
		\MatrixNorm{A}
	}
\)可能在解中放大\(\MatrixNorm{ A^{-1} } \MatrixNorm{ A \vphantom{A^{-1}} }\)倍.
\end{remark}

总之,从上述两个定理我们可以看出,
\(\MatrixNorm{ A^{-1} } \MatrixNorm{ A \vphantom{A^{-1}} }\)越小,
由系数矩阵或常数项的相对误差引起的解的相对误差就越小;
反过来,\(\MatrixNorm{ A^{-1} } \MatrixNorm{ A \vphantom{A^{-1}} }\)越大,
解的相对误差就越大.
所以\(\MatrixNorm{ A^{-1} } \MatrixNorm{ A \vphantom{A^{-1}} }\)
实际上刻画了线性方程组的病态程度,
于是我们可以引出下述定义.

\begin{definition}
%@see: 《数值分析(第5版)》(李庆扬、王能超、易大义) P169 定义8
设\(A\)是一个非奇异矩阵.
把\begin{equation*}
	\cond_p A
	\defeq
	\MatrixNorm{ A^{-1} }_p \MatrixNorm{ A \vphantom{A^{-1}} }_p
\end{equation*}
称为“矩阵\(A\)的\DefineConcept{条件数}”.
\end{definition}

现在我们就可以定量地定义病态性质了.
\begin{definition}
%@see: 《数值分析(第5版)》(李庆扬、王能超、易大义) P169
给定线性方程组\(A X = B\),
如果矩阵\(A\)的条件数\(\cond A \gg 1\),
则称“线性方程组\(A X = B\)是\DefineConcept{病态的}”或“\(A\)是\DefineConcept{病态的}”;
否则称“线性方程组\(A X = B\)是\DefineConcept{良态的}”或“\(A\)是\DefineConcept{良态的}”.
\end{definition}

\begin{theorem}
%@see: 《数值分析(第5版)》(李庆扬、王能超、易大义) P169
设\(A \in M_n(\mathbb{C})\),
则\begin{equation*}
	\cond_2 A
	= \sqrt{
		\frac{
			\max\Spec(A^H A)
		}{
			\min\Spec(A A^H)
		}
	}.
\end{equation*}
%TODO proof
\end{theorem}

\begin{theorem}
%@see: 《数值分析(第5版)》(李庆扬、王能超、易大义) P169
设\(A \in M_n(\mathbb{C})\)是对称矩阵,
则\begin{equation*}
	\cond_2 A
	= \frac{
		\abs{\lambda_1}
	}{
		\abs{\lambda_n}
	},
\end{equation*}
其中\(\lambda_1\)分别是\(A\)的绝对值最大的特征值,
\(\lambda_n\)分别是\(A\)的绝对值最小的特征值.
%TODO proof
\end{theorem}

\begin{theorem}
%@see: 《数值分析(第5版)》(李庆扬、王能超、易大义) P169
设\(A \in M_n(\mathbb{R})\)是非奇异矩阵,
则\(\cond_p A \geq 1\).
\begin{proof}
因为\(L_p\)范数是矩阵范数,满足次可乘性,
所以\begin{equation*}
	\cond_p A
	\equiv \MatrixNorm{ A^{-1} }_p \MatrixNorm{ A \vphantom{A^{-1}} }_p
	\geq \MatrixNorm{ A^{-1} A }_p
	= \MatrixNorm{ I }_p
	% 单位矩阵的任意一个自然范数都等于1.
	= 1.
	\qedhere
\end{equation*}
\end{proof}
\end{theorem}

\begin{theorem}
%@see: 《数值分析(第5版)》(李庆扬、王能超、易大义) P169
设\(A \in M_n(\mathbb{R})\)是非奇异矩阵,
那么对于任意非零实数\(c\),
有\begin{equation*}
	\cond_p (c A)
	= \cond_p A.
\end{equation*}
\begin{proof}
% \cref{theorem:逆矩阵.数与矩阵乘积的逆}
因为\((c A)^{-1} = c^{-1} A^{-1}\),
% 矩阵范数的齐次性
且\(\MatrixNorm{ c A }_p = c \MatrixNorm{A}_p\),
所以\begin{align*}
	\cond_p (c A)
	&= \MatrixNorm{ (c A)^{-1} }_p \MatrixNorm{ c A \vphantom{A^{-1}} }_p
	= \MatrixNorm{ c^{-1} A^{-1} }_p \MatrixNorm{ c A \vphantom{A^{-1}} }_p
	= c^{-1} \MatrixNorm{ A^{-1} }_p \cdot c \MatrixNorm{ A \vphantom{A^{-1}} }_p \\
	&= \MatrixNorm{ A^{-1} }_p \MatrixNorm{ A \vphantom{A^{-1}} }_p
	= \cond_p A.
	\qedhere
\end{align*}
\end{proof}
\end{theorem}

\begin{theorem}
%@see: 《数值分析(第5版)》(李庆扬、王能超、易大义) P169
设\(A \in M_n(\mathbb{R})\)是正交矩阵,
则\(\cond_2 A = 1\).
%TODO proof
\end{theorem}

\begin{theorem}
%@see: 《数值分析(第5版)》(李庆扬、王能超、易大义) P169
设\(A \in M_n(\mathbb{R})\)是非奇异矩阵,
\(Q \in M_n(\mathbb{R})\)是正交矩阵,
则\begin{equation*}
	\cond_2 (Q A)
	= \cond_2 (A Q)
	= \cond_2 A.
\end{equation*}
%TODO proof
\end{theorem}

%@see: 《数值分析(第5版)》(李庆扬、王能超、易大义) P170
在理论上,要判别一个矩阵是否病态,可以通过计算条件数\(\cond A\),
但是在实际应用中,计算\(A^{-1}\)是比较费劲的.
这时,要想发现矩阵的病态情况,有以下几种方法:\begin{itemize}
	\item 如果在使用列主元消去法对矩阵\(A\)做LU分解时,某一列主元特别小,那么\(A\)很有可能是病态的;
	\item 如果矩阵\(A\)的行列式的绝对值很小,或者\(A\)的某些行(或列)近似线性相关,那么\(A\)很有可能是病态的;
	\item 如果矩阵\(A\)的各元素的数量级相差很大,并且没有规律,那么\(A\)很有可能是病态的.
\end{itemize}

在计算机上使用数值方法求解线性方程组\(A x = b\)时,
为了缓解病态问题,有以下方法:\begin{itemize}
	\item 采用高精度算术运算(例如使用双精度浮点数);
	\item 进行预处理,选取非奇异矩阵\(P,Q\)(通常令\(P,Q\)为对角阵或三角阵),
	将\(A x = b\)转化为等价线性方程组\begin{equation*}
		\begin{cases}
			P A Q y = P b, \\
			y = Q^{-1} x,
		\end{cases}
	\end{equation*}
	同时保证\begin{equation*}
		\cond(P A Q) < \cond A;
	\end{equation*}
	\item 当矩阵\(A\)的元素大小不均时,
	对\(A\)的行(或列)引入适当的比例因子,
	使\(A\)的所有行(或列)向量的\(L_\infty\)范数相等.
\end{itemize}

假设\(x^*\)是线性方程组\(A x = b\)的近似解,
我们想知道,如果\(r \defeq b - A x^*\)足够小,
那么可不可以说“\(x^*\)是\(A x = b\)的一个较好的近似解”呢?
\begin{theorem}
%@see: 《数值分析(第5版)》(李庆扬、王能超、易大义) P172 定理23(事后误差估计)
设\(A \in M_n(\mathbb{R})\)是非奇异矩阵,
\(x^*\)是线性方程组\(A x = b \neq 0\)的近似解,
记\(r \defeq b - A x^*\),
则\begin{equation*}
%@see: 《数值分析(第5版)》(李庆扬、王能超、易大义) P172 (5.11)
	\frac{ \norm{x - x^*} }{ \norm{x} }
	\leq \frac{ \norm{r} }{ \norm{b} } \cdot \cond A.
\end{equation*}
%TODO proof
\end{theorem}
上述定理说明:
近似解的精度,不仅依赖于\(r\)的范数,
还依赖于\(A\)的条件数.
当\(A\)是病态矩阵时,
即使\(r\)的范数很小,
也不能保证\(x^*\)是高精度的近似解.
