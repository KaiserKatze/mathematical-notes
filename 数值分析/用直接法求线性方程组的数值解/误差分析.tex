\section{误差分析}
%@see: 《数值分析(第5版)》(李庆扬、王能超、易大义) P168 定义7
给定线性方程组\(A X = B\),
如果系数矩阵\(A\)或常数项\(B\)的微小变化,
引起解\(X\)的巨大变化,
则称“线性方程组\(A X = B\)是\DefineConcept{病态的}”或“\(A\)是\DefineConcept{病态的}”;
否则称“线性方程组\(A X = B\)是\DefineConcept{良态的}”或“\(A\)是\DefineConcept{良态的}”.

应该注意,病态性质是系数矩阵的内禀性质.
接下来我们希望找出刻画矩阵的病态性质的量.

考虑线性方程组\(
%@see: 《数值分析(第5版)》(李庆扬、王能超、易大义) P168 (5.3)
	A x = b
\),其中\(A\)是非奇异矩阵.
如果系数矩阵是精确的,常数项有误差\(\delta b\),
那么解也会相应地有误差\(\delta x\),
于是\(A (x + \delta x) = b + \delta b\).
由此可得\begin{equation*}
%@see: 《数值分析(第5版)》(李庆扬、王能超、易大义) P168 (5.4)
	\delta x = A^{-1} (b + \delta b) - A^{-1} b
	= A^{-1} \delta b,
	\qquad
	\norm{\delta x}
	\leq \MatrixNorm{A^{-1}} \norm{\delta b \vphantom{A^{-1}}}.
\end{equation*}
又因为\begin{equation*}
	\norm{b}
	\leq \MatrixNorm{A} \norm{x},
	\qquad
	\frac{1}{\norm{x}}
	\leq \frac{\MatrixNorm{A}}{\norm{b}},
\end{equation*}
于是我们可以得出如下结论.
\begin{theorem}
%@see: 《数值分析(第5版)》(李庆扬、王能超、易大义) P168 定理21
设\(A \in M_n(\mathbb{R})\)是非奇异矩阵.
若\(A x = b \neq 0\)且\(A (x + \delta x) = b + \delta b\),
则\begin{equation*}
	\frac{
		\norm{ \delta x }
	}{
		\norm{ x }
	}
	\leq
	\MatrixNorm{ A^{-1} } \MatrixNorm{ A \vphantom{A^{-1}} }
	\frac{
		\norm{ \delta b }
	}{
		\norm{ b }
	}.
\end{equation*}
\end{theorem}
\begin{remark}
上式给出了解的相对误差的上界,
常数项\(b\)的相对误差可能在解中放大\(\MatrixNorm{ A^{-1} } \MatrixNorm{ A \vphantom{A^{-1}} }\)倍.
\end{remark}

假设常数项是精确的,系数矩阵有误差\(\delta A\),
那么解也会相应地有误差\(\delta x\),
%@see: 《数值分析(第5版)》(李庆扬、王能超、易大义) P168 (5.6)
于是有\begin{equation*}
	(A + \delta A) (x + \delta x) = b,
\end{equation*}
即\((A + \delta A) \delta x = - \delta A x\).
如果\(\delta A\)不受限制的话,那么\(A + \delta A\)可能是一个奇异矩阵.
显然\begin{equation*}
	A + \delta A
	= A + A A^{-1} \delta A
	= A (I + A^{-1} \delta A).
\end{equation*}
根据定理20,%@see: 《数值分析(第5版)》(李庆扬、王能超、易大义) P166 定理20
当\(\norm{A^{-1} \delta A} < 1\)时,
\(I + A^{-1} \delta A\)可逆,
于是\begin{equation*}
	\delta x = - (I + A^{-1} \delta A)^{-1} A^{-1} \delta A x.
\end{equation*}
因此\begin{equation*}
	\norm{\delta x}
	\leq \frac{
		\MatrixNorm{A^{-1}} \MatrixNorm{\delta A} \norm{x}
	}{
		1 - \norm{A^{-1} \delta A}
	}.
\end{equation*}
设\(\MatrixNorm{A^{-1}} \norm{\delta A \vphantom{A^{-1}}} < 1\),
则\begin{equation*}
%@see: 《数值分析(第5版)》(李庆扬、王能超、易大义) P168 (5.7)
	\frac{\norm{\delta x}}{\norm{x}}
	\leq \frac{
		\MatrixNorm{A^{-1}}
		\MatrixNorm{A}
		\frac{
			\MatrixNorm{\delta A}
		}{
			\MatrixNorm{A}
		}
	}{
		1
		- \MatrixNorm{A^{-1}}
		\MatrixNorm{A}
		\frac{
			\MatrixNorm{\delta A}
		}{
			\MatrixNorm{A}
		}
	}.
\end{equation*}

\begin{theorem}
%@see: 《数值分析(第5版)》(李庆扬、王能超、易大义) P169 定理22
设\(A\)是非奇异矩阵.
若\(A x = b \neq 0\)且\((A + \delta A) (x + \delta x) = b\),
那么当\(\MatrixNorm{A^{-1}} \norm{\delta A \vphantom{A^{-1}}} < 1\)时,
有\begin{equation*}
%@see: 《数值分析(第5版)》(李庆扬、王能超、易大义) P168 (5.7)
	\frac{\norm{\delta x}}{\norm{x}}
	\leq \frac{
		\MatrixNorm{A^{-1}}
		\MatrixNorm{A}
		\frac{
			\MatrixNorm{\delta A}
		}{
			\MatrixNorm{A}
		}
	}{
		1
		- \MatrixNorm{A^{-1}}
		\MatrixNorm{A}
		\frac{
			\MatrixNorm{\delta A}
		}{
			\MatrixNorm{A}
		}
	}.
\end{equation*}
\end{theorem}
\begin{remark}
如果\(\delta A\)充分小,那么在\(\MatrixNorm{A^{-1}} \norm{\delta A \vphantom{A^{-1}}} < 1\)的情况下,
上式说明系数矩阵\(A\)的相对误差\(
	\frac{
		\MatrixNorm{\delta A}
	}{
		\MatrixNorm{A}
	}
\)可能在解中放大\(\MatrixNorm{ A^{-1} } \MatrixNorm{ A \vphantom{A^{-1}} }\)倍.
\end{remark}

总之,从上述两个定理我们可以看出,
\(\MatrixNorm{ A^{-1} } \MatrixNorm{ A \vphantom{A^{-1}} }\)越小,
由系数矩阵或常数项的相对误差引起的解的相对误差就越小;
反过来,\(\MatrixNorm{ A^{-1} } \MatrixNorm{ A \vphantom{A^{-1}} }\)越大,
解的相对误差就越大.
所以\(\MatrixNorm{ A^{-1} } \MatrixNorm{ A \vphantom{A^{-1}} }\)
实际上刻画了线性方程组的病态程度,
于是我们可以引出下述定义.

\begin{definition}
%@see: 《数值分析(第5版)》(李庆扬、王能超、易大义) P169 定义8
设\(A\)是一个非奇异矩阵.
把\begin{equation*}
	\cond_p A
	\defeq
	\MatrixNorm{ A^{-1} }_p \MatrixNorm{ A \vphantom{A^{-1}} }_p
\end{equation*}
称为“矩阵\(A\)的\DefineConcept{条件数}”.
\end{definition}

现在我们就可以定量地定义病态性质了.
\begin{definition}
给定线性方程组\(A X = B\),
如果矩阵\(A\)的条件数\(\cond A \gg 1\),
则称“线性方程组\(A X = B\)是\DefineConcept{病态的}”或“\(A\)是\DefineConcept{病态的}”;
否则称“线性方程组\(A X = B\)是\DefineConcept{良态的}”或“\(A\)是\DefineConcept{良态的}”.
\end{definition}
