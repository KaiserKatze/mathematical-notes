\chapter{拓扑学}
拓扑学是极限的推广,通常分为点集拓扑学和代数拓扑学.
本章主要介绍点集拓扑学,它为几何学提供了基本语言.

\begingroup
\def\T{\mathfrak T}%
\def\oT{\overline{\mathfrak T}}%

\section{拓扑的基本概念}
\subsection{拓扑的定义}
\begin{definition}\label{definition:拓扑学.开集公理定义的拓扑空间}
已知非空集合\(X\),\(\T \subseteq \Powerset X\).
若有\begin{enumerate}
\item \(\emptyset,X \in \T\).
\item \(\T\)的\cemph{任意并}仍然属于\(\T\),即\(\T_1 \subseteq \T \implies \bigcup_{A \in \T_1} A \in \T\). 
\item \(\T\)的\cemph{有限交}仍然属于\(\T\),即\(A,B \in \T \implies A \cap B \in \T\).
\end{enumerate}
则称“\(\T\)为\(X\)的一个\DefineConcept{拓扑}”,%
称“\((X,\T)\)为一个(开集公理定义的)\DefineConcept{拓扑空间}(topological space)”,%
或称“集合\(X\)是一个相对于拓扑\(\T\)而言的拓扑空间”\footnote{%
不特别强调拓扑\(\T\)时,可以简单地表述为“集合\(X\)是一个拓扑空间”.%
},%
\(\T\)的任一元素都称为拓扑空间\((X,\T)\)(或\(X\))中的一个\DefineConcept{开集}(open set).
\end{definition}

\begin{definition}\label{definition:拓扑学.闭集公理定义的拓扑空间}
已知非空集合\(X\),\(\oT \subseteq \Powerset X\).
若有\begin{enumerate}
\item \(\emptyset,X \in \oT\).
\item \(\oT\)的\cemph{有限并}仍然属于\(\oT\).
\item \(\oT\)的\cemph{任意交}仍然属于\(\oT\).
\end{enumerate}
则称“\((X,\oT)\)为一个闭集公理定义的\DefineConcept{拓扑空间}”,%
\(\oT\)的任一元素都称为拓扑空间\((X,\T)\)(或\(X\))中的一个\DefineConcept{闭集}(closed set),%
\(\oT\)称为一个\DefineConcept{闭集族}.
\end{definition}

\cref{definition:拓扑学.开集公理定义的拓扑空间,definition:拓扑学.闭集公理定义的拓扑空间} 是等价的,这是因为对于集合\(X\),若有开集族\(\T\)则对应闭集族\[
\oT = \Set{ F \given F^C \in \T };
\]若有闭集族\(\oT\)则对应开集族\[
\T = \Set{ G \given G^C \in \oT };
\]且二者显然互逆确定.
也就是说,当我们说拓扑空间\(X\)时,就自动配上了对应的开集和闭集.

\begin{theorem}
已知拓扑空间\(X\),若\(A \subseteq X\),则\[
A\text{是开集} \iff A^C\text{是闭集}.
\]
\end{theorem}

\endgroup
