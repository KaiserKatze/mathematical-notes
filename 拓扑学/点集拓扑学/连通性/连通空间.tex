\section{连通空间}
我们先通过直观的方式考察一个例子.
在实数空间\(\mathbb{R}\)中,
\((0,1)\)和\([1,2)\)这两个区间尽管互不相交,
但是它们的并\((0,1)\cup[1,2)=(0,2)\)却可以构成一个“整体”.
再看\((0,1)\)和\((1,2)\)这两个区间,
它们的并\((0,1)\cup(1,2)\)却明显是两个“部分”.
产生上述不同情况的原因在于,
在前一种情况中,区间\((0,1)\)有一个聚点\(1\)在\([1,2)\)中;
在后一种情况中,两个区间中的任何一个都没有聚点属于另一个区间.
于是我们可以抽象出如下概念:
\begin{definition}
%@see: 《点集拓扑讲义(第四版)》(熊金城) P122 定义4.1.1
设\(A,B\)都是拓扑空间\(X\)的子集.
如果\begin{equation*}
	(A \cap \overline{B}) \cup (B \cap \overline{A}) = \emptyset,
\end{equation*}
则称“\(A\)与\(B\)是\DefineConcept{隔离的}”.
\end{definition}
