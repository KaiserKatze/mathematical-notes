\section{连通空间}
我们先通过直观的方式考察一个例子.
在实数空间\(\mathbb{R}\)中,
\((0,1)\)和\([1,2)\)这两个区间尽管互不相交,
但是它们的并\((0,1)\cup[1,2)=(0,2)\)却可以构成一个“整体”.
再看\((0,1)\)和\((1,2)\)这两个区间,
它们的并\((0,1)\cup(1,2)\)却明显是两个“部分”.
产生上述不同情况的原因在于,
在前一种情况中,区间\((0,1)\)有一个聚点\(1\)在\([1,2)\)中;
在后一种情况中,两个区间中的任何一个都没有聚点属于另一个区间.
于是我们可以抽象出如下概念:
\begin{definition}
%@see: 《点集拓扑讲义(第四版)》(熊金城) P122 定义4.1.1
设\(A,B\)都是拓扑空间\(X\)的子集.
如果\begin{equation*}
	(A \cap \TopoClosureL{B}) \cup (B \cap \TopoClosureL{A}) = \emptyset,
\end{equation*}
则称“\(A\)与\(B\)是\DefineConcept{隔离的}”
或“\(A,B\)是\(X\)的隔离子集”.
\end{definition}

应用上述术语,我们就可以说:
在实数空间\(\mathbb{R}\)中,
\((0,1)\)和\((1,2)\)是隔离的,
\((0,1)\)和\([1,2)\)不是隔离的.

\begin{example}
%@see: 《点集拓扑讲义(第四版)》(熊金城) P122
证明:平庸空间中任何两个非空子集都不是隔离的.
%TODO proof
\end{example}

\begin{example}
%@see: 《点集拓扑讲义(第四版)》(熊金城) P123
证明:离散空间中任何两个互斥子集都是隔离的.
%TODO proof
\end{example}

\begin{definition}
%@see: 《点集拓扑讲义(第四版)》(熊金城) P123 定义4.1.2
设\(X\)是一个拓扑空间.
如果\(X\)中两个非空隔离子集\(A,B\)使得\(X = A \cup B\),
则称“\(X\)是\DefineConcept{非连通的}”
或“\(X\)是\DefineConcept{非连通空间}”;
否则,称“\(X\)是\DefineConcept{连通的}”
或“\(X\)是\DefineConcept{连通空间}”.
\end{definition}

\begin{example}
%@see: 《点集拓扑讲义(第四版)》(熊金城) P123
证明:包含多于一个点的离散空间是非连通的.
%TODO proof
\end{example}

\begin{example}
%@see: 《点集拓扑讲义(第四版)》(熊金城) P123
证明:每一个平庸空间都是连通的.
%TODO proof
\end{example}

\begin{theorem}
%@see: 《点集拓扑讲义(第四版)》(熊金城) P123 定理4.1.1
设\(X\)是一个拓扑空间,
则下列命题等价:\begin{itemize}
	\item \(X\)是非连通的;
	\item
	\item \(X\)中存在两个非空闭子集\(A,B\)
	使得\(A \cap B = \emptyset\)和\(A \cup B = X\)同时成立;

	\item \(X\)中存在两个非空开子集\(A,B\)
	使得\(A \cap B = \emptyset\)和\(A \cup B = X\)同时成立;

	\item \(X\)中存在着一个既开又闭的非空真子集.
\end{itemize}
%TODO proof
\end{theorem}

\begin{example}
%@see: 《点集拓扑讲义(第四版)》(熊金城) P124 例4.1.1
证明:有理数集\(\mathbb{Q}\)作为实数空间\(\mathbb{R}\)的子空间是非连通的.
\begin{proof}
由于对于任何一个无理数\(r \in \mathbb{R}-\mathbb{Q}\),
集合\((-\infty,r)\cap\mathbb{Q}
= (-\infty,r]\cap\mathbb{Q}\)
%TODO 为什么上式中两个集合相等?
是\(\mathbb{Q}\)中一个既开又闭的非空真子集,
所以有理数集\(\mathbb{Q}\)作为实数空间\(\mathbb{R}\)的子空间是非连通的.
\end{proof}
\end{example}

\begin{theorem}
%@see: 《点集拓扑讲义(第四版)》(熊金城) P124 定理4.1.2
实数空间\(\mathbb{R}\)是连通的.
%TODO proof
\end{theorem}

\begin{definition}
%@see: 《点集拓扑讲义(第四版)》(熊金城) P124 定义4.1.3
设\(Y\)是拓扑空间\(X\)的一个子集.
如果\(Y\)作为\(X\)的子空间是一个连通空间,
则称“\(Y\)是\(X\)的一个\DefineConcept{连通子集}”;
否则,称“\(Y\)是\(X\)的一个\DefineConcept{非连通子集}”.
\end{definition}

不难注意到,拓扑空间\(X\)的子集\(Y\)是否连通,只取决于子空间\(Y\)的拓扑,于是我们有如下结论:
\begin{proposition}
%@see: 《点集拓扑讲义(第四版)》(熊金城) P124
如果\(Y \subseteq Z \subseteq X\),
则\(Y\)是\(X\)的连通子集,
当且仅当\(Y\)是\(Z\)的连通子集.
%TODO proof
\end{proposition}

\begin{theorem}
%@see: 《点集拓扑讲义(第四版)》(熊金城) P124 定理4.1.3
设\(Y\)是拓扑空间\(X\)的一个子集,
且\(A,B \subseteq Y\),
则\(A\)和\(B\)是子空间\(Y\)中的隔离子集,
当且仅当它们是拓扑空间\(X\)中的隔离子集.
%TODO proof
\end{theorem}

\begin{corollary}
%@see: 《点集拓扑讲义(第四版)》(熊金城) P124
设\(Y\)是拓扑空间\(X\)的一个子集,
则\(Y\)是\(X\)的一个非连通子集,
当且仅当\(X\)中存在两个非空隔离子集\(A,B\)
使得\(A \cup B = Y\).
%TODO proof
\end{corollary}

\begin{theorem}
%@see: 《点集拓扑讲义(第四版)》(熊金城) P125 定理4.1.4
设\(Y\)是拓扑空间\(X\)中的一个连通子集.
如果\(X\)中有隔离子集\(A,B\)
使得\(Y \subseteq A \cup B\),
则要么成立\(Y \subseteq A\)要么成立\(Y \subseteq B\).
%TODO proof
\end{theorem}

\begin{theorem}
%@see: 《点集拓扑讲义(第四版)》(熊金城) P125 定理4.1.5
设\(Y\)是拓扑空间\(X\)中的一个连通子集,
\(Z\)是一个集合,
且\(Y \subseteq Z \subseteq \TopoClosureL{Y}\),
则\(Z\)是\(X\)的一个连通子集.
%TODO proof
\end{theorem}

\begin{theorem}
%@see: 《点集拓扑讲义(第四版)》(熊金城) P125 定理4.1.6
设\(\{Y_\gamma\}_{\gamma \in \Gamma}\)是拓扑空间\(X\)的连通子集构成的一个子集族.
如果\(\bigcap_{\gamma \in \Gamma} Y_\gamma \neq \emptyset\),
则\(\bigcup_{\gamma \in \Gamma} Y_\gamma\)是\(X\)的一个连通子集.
%TODO proof
\end{theorem}

\begin{theorem}
%@see: 《点集拓扑讲义(第四版)》(熊金城) P125 定理4.1.7
设\(Y\)是拓扑空间\(X\)中的一个子集.
如果对于任意\(x,y \in Y\),
\(X\)中存在一个连通子集\(Y_{xy}\)
使得\(x,y \in Y_{xy} \subseteq Y\),
则\(Y\)是\(X\)中的一个连通子集.
%TODO proof
\end{theorem}
