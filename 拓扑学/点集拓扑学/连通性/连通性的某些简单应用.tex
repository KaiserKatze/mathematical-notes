\section{连通性的某些简单应用}
我们知道,实数集\(\mathbb{R}\)中区间可以分为九类:\begin{gather*}
	(-\infty,+\infty),
	(a,+\infty),
	[a,+\infty),
	(-\infty,a),
	(-\infty,a],
	(a,b),
	(a,b],
	[a,b),
	[a,b].
\end{gather*}

在\cref{theorem:连通空间.实数空间是连通空间} 中
我们证明了实数空间\(\mathbb{R}\)是一个连通空间.
由于\((a,+\infty),(-\infty,a),(a,b)\)都同胚于\(\mathbb{R}\)
%TODO 写出对应的同胚映射
所以这3个区间也都是连通的.
由于\begin{gather*}
	\TopoClosureL{(a,+\infty)}
	= [a,+\infty), \\
	\TopoClosureL{(-\infty,a)}
	= (-\infty,a], \\
	(a,b) \subseteq [a,b) \subseteq [a,b], \\
	(a,b) \subseteq (a,b] \subseteq [a,b],
\end{gather*}
那么根据\cref{theorem:连通空间.连通性的夹逼准则} 可知
区间\([a,+\infty),(-\infty,a],[a,b),(a,b],[a,b]\)都是连通的.

假设\(E\)是\(\mathbb{R}\)的一个子集,
且\(E\)中至少有两个点.
如果\(E\)不是一个区间,
则存在\(a,b \in E\)满足\(a < b\),
使得\([a,b] \not\subseteq E\).
换言之,
如果\(E\)不是一个区间,
则存在\(a,b \in E\)满足\(a < c < b\),
使得\(c \notin E\),
若令\(A \defeq (-\infty,c) \cap E,
B \defeq (c,+\infty) \cap E\),
则有\(A,B\)都是\(E\)的非空的开集,
并且有\(A \cup B = E\)和\(A \cap B = \emptyset\)同时成立,
因此\(E\)不连通.
此外,当\(E\)是空集或单点集时,
\(E\)显然是一个区间.

综上以上两个方面,我们已经证明了:
\begin{theorem}
%@see: 《点集拓扑讲义(第四版)》(熊金城) P131 定理4.2.1
设\(E\)是实数空间\(\mathbb{R}\)的一个子集,
\(E\)是一个连通子集,
当且仅当\(E\)是一个区间.
%TODO proof
\end{theorem}

\begin{theorem}\label{theorem:连通空间.从连通空间到实数域的连续映射1}
%@see: 《点集拓扑讲义(第四版)》(熊金城) P131 定理4.2.2
设\(X\)是一个连通空间,
\(f\colon X \to \mathbb{R}\)是一个连续映射,
则\(f(X)\)是\(\mathbb{R}\)中的一个区间.
%TODO proof
\end{theorem}

\begin{theorem}\label{theorem:连通空间.从连通空间到实数域的连续映射2}
%@see: 《点集拓扑讲义(第四版)》(熊金城) P131 定理4.2.2
设\(X\)是一个连通空间,
\(f\colon X \to \mathbb{R}\)是一个连续映射,
则\begin{equation*}
	(\forall a,b \in X)
	(\forall t\in\mathbb{R})
	(\exists \xi \in X)
	[
		f(a) \leq t \leq f(b)
		\implies
		f(\xi) = t
	].
\end{equation*}
%TODO proof
\end{theorem}
根据\cref{theorem:连通空间.从连通空间到实数域的连续映射2} 立即可以推出介值定理和不动点定理.

%@see: 《点集拓扑讲义(第四版)》(熊金城) P131 定理4.2.3(介值定理)
%@see: 《点集拓扑讲义(第四版)》(熊金城) P131 定理4.2.4(不动点定理)

\begin{example}
%@see: 《点集拓扑讲义(第四版)》(熊金城) P131
证明:在欧氏平面\(\mathbb{R}^2\)中,单位圆周\(S^1\)是连通的.
%TODO proof
\end{example}

\begin{definition}
%@see: 《点集拓扑讲义(第四版)》(熊金城) P132
在欧氏空间\(\mathbb{R}^2\)中,
点\(x=(x_1,x_2) \in S^1\).
把\((-x_1,-x_2)\)称为“点\(x\)的\DefineConcept{对径点}”,
记作\(-x\).
把映射\begin{equation*}
	r\colon S^1 \to S^1, x \mapsto -x
\end{equation*}
称为\DefineConcept{对径映射}.
\end{definition}

\begin{example}\label{example:连通空间.从单位圆周到实数域的连续映射在一对对径点的值相等}
%@see: 《点集拓扑讲义(第四版)》(熊金城) P132 定理4.2.5(Borsuk-Ulam定理)
设\(f\colon S^1 \to \mathbb{R}\)是一个连续映射.
证明:在\(S^1\)中存在一对对径点\(x\)和\(-x\),使得\(f(x) = f(-x)\).
%TODO proof
\end{example}

\begin{example}
%@see: 《点集拓扑讲义(第四版)》(熊金城) P132 定理4.2.6
证明:\(n\ (n>1)\)维欧氏空间\(\mathbb{R}^n\)的子集\(\mathbb{R}^n-\{0\}\)是一个连通子集.
%TODO proof
\end{example}

下面再给出一个利用拓扑不变性质判定两个空间不同胚的例子:
\begin{example}
%@see: 《点集拓扑讲义(第四版)》(熊金城) P132 定理4.2.7
证明:欧氏平面\(\mathbb{R}^2\)和实数空间\(\mathbb{R}\)不同胚.
%TODO proof
\end{example}

%TODO 下面3个定理的证明需要用到代数拓扑学(例如同调论、同伦论)

\begin{theorem}
%@see: 《点集拓扑讲义(第四版)》(熊金城) P133 定理4.2.8(Brouwer不动点定理)
设\(E^n\)是\(n\)维闭球体,
\(f\colon E^n \to E^n\)是一个连续映射,
则存在\(z \in E^n\),使得\(f(z) = z\).
%TODO proof
\end{theorem}

\begin{theorem}
%@see: 《点集拓扑讲义(第四版)》(熊金城) P133 定理4.2.9(Borsuk-Ulam定理)
设\(f\colon S^n \to \mathbb{R}^m\ (n \geq m)\)是一个连续映射,
则存在\(x \in S^n\),使得\(f(x) = f(-x)\).
%TODO proof
\end{theorem}

\begin{theorem}
%@see: 《点集拓扑讲义(第四版)》(熊金城) P133 定理4.2.10
如果\(n \neq m\),则欧氏空间\(\mathbb{R}^n\)与\(\mathbb{R}^m\)不同胚.
%TODO proof
\end{theorem}
