\section{道路连通空间}
下面讨论\DefineConcept{道路连通性}(path connectedness).

\begin{definition}
%@see: 《点集拓扑讲义(第四版)》(熊金城) P142 定义4.5.1
设\(X\)是一个拓扑空间,
\(f\colon [0,1] \to X\)一个连续映射,
\(a=f(0),
b=f(1)\),
则称“\(f\)是拓扑空间\(X\)中从\(a\)到\(b\)的一条\DefineConcept{道路}”,
把\(f\)的像\(f([0,1])\)称为“拓扑空间\(X\)中的一条\DefineConcept{曲线}”
或“拓扑空间\(X\)中的一条\DefineConcept{弧}”,
把\(a\)称为“道路\(f\)的\DefineConcept{起点}”或“曲线\(f([0,1])\)的{起点}”,
把\(b\)称为“道路\(f\)的\DefineConcept{终点}”或“曲线\(f([0,1])\)的{终点}”.
\end{definition}

\begin{definition}
%@see: 《点集拓扑讲义(第四版)》(熊金城) P142 定义4.5.1
起点和终点相同的道路,称为\DefineConcept{回路},并且此时它的起点称为\DefineConcept{基点}.
\end{definition}

\begin{definition}
%@see: 《点集拓扑讲义(第四版)》(熊金城) P142 定义4.5.2
设\(X\)是一个拓扑空间.
如果对于任意\(a,b \in X\),\(X\)中存在一条从\(a\)到\(b\)的一条道路,
则称\(X\)是一个\DefineConcept{道路连通空间}.
\end{definition}

\begin{definition}
%@see: 《点集拓扑讲义(第四版)》(熊金城) P142 定义4.5.2
设\(Y\)是道路连通空间\(X\)的一个子集.
如果\(Y\)作为\(X\)的子空间是一个道路连通空间,
则称“\(Y\)是\(X\)的一个\DefineConcept{道路连通子集}”.
\end{definition}

实数空间\(\mathbb{R}\)是道路连通的.
这是因为任取\(a,b \in \mathbb{R}\),
则连续映射\(f\colon [0,1] \to \mathbb{R}, t \mapsto a+t(b-a)\)
便是\(\mathbb{R}\)中以\(a\)为起点、以\(b\)为终点的一条道路.
容易验证,\(\mathbb{R}\)中任何一个区间都是道路连通的.

\begin{theorem}\label{theorem:道路连通空间.道路连通空间一定是连通空间}
%@see: 《点集拓扑讲义(第四版)》(熊金城) P142 定理4.5.1
如果拓扑空间\(X\)是一个道路连通空间,
则\(X\)必然是一个连通空间.
%TODO proof
\end{theorem}

连通空间可以不是道路连通的.
例如,拓扑学家的正弦曲线\(S\)的闭包\(\TopoClosureL{S}\)是一个连通空间,但是它不是一个道路连通空间.

道路连通性与局部连通性之间更没有必然的蕴含关系.
例如,离散空间都是局部连通的,然而包含多于一个点的离散空间都不是连通空间,当然也就不是道路连通空间了.

\begin{theorem}
%@see: 《点集拓扑讲义(第四版)》(熊金城) P143 定理4.5.2
设\(X,Y\)都是拓扑空间,
\(X\)是道路连通的,
\(f\colon X \to Y\)是一个连续映射,
则\(f(X)\)是道路连通的.
%TODO proof
\end{theorem}
\begin{remark}
由上述定理可知,拓扑空间的道路连通性是一个拓扑不变性质,也是一个可商性质.
\end{remark}

\begin{theorem}
%@see: 《点集拓扑讲义(第四版)》(熊金城) P143 定理4.5.3
设\(\AutoTuple{X}{n}\)是\(n\ (n\geq1)\)个道路连通空间,
则拓扑积空间\(\AutoTuple{X}{n}[\times]\)也是道路连通空间.
%TODO proof
\end{theorem}

根据上述定理容易证明:\(n\)维欧氏空间\(\mathbb{R}^n\)是一个道路连通空间.

\begin{theorem}\label{theorem:道路连通空间.粘结引理}
%@see: 《点集拓扑讲义(第四版)》(熊金城) P144 定理4.5.4
设\(A,B\)是拓扑空间\(X\)中的两个开集(或闭集),
\(X = A \cup B\),
\(Y\)是一个拓扑空间,
\(f_1\colon A \to Y\)和\(f_2\colon B \to Y\)都是连续映射,
且满足\(f_1 \SetRestrict (A \cap B) = f_2 \SetRestrict (A \cap B)\),
定义映射\(f\colon X \to Y\)使之满足\begin{equation*}
	f(x) \defeq \left\{ \begin{array}{cl}
		f_1(x), & x \in A, \\
		f_2(x), & x \in B,
	\end{array} \right.
\end{equation*}
则\(f\)是一个连续映射.
%TODO proof
\end{theorem}

\begin{definition}
%@see: 《点集拓扑讲义(第四版)》(熊金城) P144 定义4.5.3
设\(X\)是一个拓扑空间,\(a,b \in X\).
如果\(X\)中有一条从\(a\)到\(b\)的道路,
则称“\(a\)和\(b\)是\DefineConcept{道路连通的}”.
\end{definition}

容易证明,两点之间的道路连通关系,满足自反性、对称性、传递性,是一个等价关系.

\begin{definition}
%@see: 《点集拓扑讲义(第四版)》(熊金城) P145 定义4.5.4
设\(X\)是一个拓扑空间,点\(a \in X\),
把\(a\)在道路连通关系下的等价类\begin{equation*}
	\Set{
		x \in X
		\given
		\text{$x$和$a$是道路连通的}
	}
\end{equation*}
称为“拓扑空间\(X\)的一个\DefineConcept{道路连通分支}”.
\end{definition}

\begin{definition}
%@see: 《点集拓扑讲义(第四版)》(熊金城) P145 定义4.5.4
设\(X\)是一个拓扑空间,
\(Y\)是\(X\)的一个子集.
\(Y\)作为\(X\)的子空间的每一个道路连通分支,
称为“拓扑空间\(X\)的子集\(Y\)的一个\DefineConcept{道路连通分支}”.
\end{definition}

\begin{proposition}
%@see: 《点集拓扑讲义(第四版)》(熊金城) P145
拓扑空间\(X\ (\neq\emptyset)\)的每一个道路连通分支都不是空集.
%TODO proof
\end{proposition}

\begin{proposition}
%@see: 《点集拓扑讲义(第四版)》(熊金城) P145
拓扑空间\(X\ (\neq\emptyset)\)的不同的道路连通分支互斥.
%TODO proof
\end{proposition}

\begin{proposition}
%@see: 《点集拓扑讲义(第四版)》(熊金城) P145
拓扑空间\(X\ (\neq\emptyset)\)的所有道路连通分支之并就是\(X\)本身.
%TODO proof
\end{proposition}

\begin{proposition}
%@see: 《点集拓扑讲义(第四版)》(熊金城) P145
设\(X\)是一个拓扑空间,
点\(a,b \in X\),
则\(a,b\)属于\(X\)的同一个道路连通分支,
当且仅当\(a\)和\(b\)道路连通.
%TODO proof
\end{proposition}

\begin{proposition}
%@see: 《点集拓扑讲义(第四版)》(熊金城) P145
设\(X\)是一个拓扑空间,
\(Y\)是\(X\)的一个子集,
点\(a,b \in Y\),
则\(a,b\)属于\(Y\)的同一个道路连通分支,
当且仅当\(Y\)中有一个从\(a\)到\(b\)的道路.
%TODO proof
\end{proposition}

由定义可知,拓扑空间中每一个道路连通分支\(A\),都是一个道路连通子集;
由\cref{theorem:道路连通空间.道路连通空间一定是连通空间} 可知
\(A\)也是一个连通子集;
由\cref{theorem:连通分支.连通分支的性质1} 可知,
\(A\)必然包含于某个连通分支.

作为\cref{theorem:道路连通空间.道路连通空间一定是连通空间} 在某种特定情形下的一个逆命题,
我们有如下命题:
\begin{proposition}
%@see: 《点集拓扑讲义(第四版)》(熊金城) P146 定理4.5.5
\(n\)维欧氏空间\(\mathbb{R}^n\)的任何一个连通开集都是道路连通的.
%TODO proof
\end{proposition}

\begin{proposition}
%@see: 《点集拓扑讲义(第四版)》(熊金城) P146 推论4.5.6
\(n\)维欧氏空间\(\mathbb{R}^n\)中任何开集的每一个道路连通分支同时也是它的一个连通分支.
%TODO proof
\end{proposition}

\begin{definition}
%@see: 《点集拓扑讲义(第四版)》(熊金城) P147 习题 5.
设\(X\)是一个拓扑空间,\(x \in X\).
如果\(x\)的每一个邻域\(U\)中,都包含着\(x\)的某一个道路连通的邻域\(V\),
则称“拓扑空间\(X\)在点\(x\)是\DefineConcept{局部道路连通的}”.
\end{definition}
\begin{definition}
%@see: 《点集拓扑讲义(第四版)》(熊金城) P147 习题 5.
设\(X\)是一个拓扑空间.
如果\(X\)在它的每一个点都是局部道路连通的,
则称“拓扑空间\(X\)是\DefineConcept{局部道路连通空间}”.
\end{definition}
\begin{definition}
%@see: 《点集拓扑讲义(第四版)》(熊金城) P147 习题 5.
设\(X\)是一个拓扑空间,
\(Y\)是\(X\)的一个子集.
如果\(Y\)作为\(X\)的子空间是一个道路连通空间,
则称“\(Y\)是\(X\)的一个\DefineConcept{局部道路连通子集}”.
\end{definition}
