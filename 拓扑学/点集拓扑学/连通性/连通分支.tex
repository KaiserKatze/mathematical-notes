\section{连通分支}
\begin{definition}
%@see: 《点集拓扑讲义(第四版)》(熊金城) P134 定理4.3.1
设\(X\)是一个拓扑空间,点\(a,b \in X\).
如果\(X\)中有一个连通子集\(Y \supseteq \{a,b\}\),
则称“点\(a\)和\(b\)是\DefineConcept{连通的}”.
\end{definition}

容易证明,两点之间的连通关系,满足自反性、对称性、传递性,是一个等价关系.

\begin{definition}
%@see: 《点集拓扑讲义(第四版)》(熊金城) P134 定义4.3.2
设\(X\)是一个拓扑空间,点\(a \in X\),
把\(a\)在连通关系下的等价类\begin{equation*}
	\Set{
		x \in X
		\given
		\text{$x$和$a$是连通的}
	}
\end{equation*}
称为“拓扑空间\(X\)的一个\DefineConcept{连通分支}”.
\end{definition}

\begin{definition}
%@see: 《点集拓扑讲义(第四版)》(熊金城) P134 定义4.3.2
设\(X\)是一个拓扑空间,
\(Y\)是\(X\)的一个子集.
\(Y\)作为\(X\)的子空间的每一个连通分支,
称为“拓扑空间\(X\)的子集\(Y\)的一个\DefineConcept{连通分支}”.
\end{definition}

\begin{proposition}
%@see: 《点集拓扑讲义(第四版)》(熊金城) P135
拓扑空间\(X\ (\neq\emptyset)\)的每一个连通分支都不是空集.
%TODO proof
\end{proposition}

\begin{proposition}
%@see: 《点集拓扑讲义(第四版)》(熊金城) P135
拓扑空间\(X\ (\neq\emptyset)\)的不同的连通分支互斥.
%TODO proof
\end{proposition}

\begin{proposition}
%@see: 《点集拓扑讲义(第四版)》(熊金城) P135
拓扑空间\(X\ (\neq\emptyset)\)的所有连通分支之并就是\(X\)本身.
%TODO proof
\end{proposition}

\begin{proposition}
%@see: 《点集拓扑讲义(第四版)》(熊金城) P135
设\(X\)是一个拓扑空间,
点\(a,b \in X\),
则\(a,b\)属于\(X\)的同一个连通分支,
当且仅当\(a\)和\(b\)连通.
%TODO proof
\end{proposition}

\begin{proposition}
%@see: 《点集拓扑讲义(第四版)》(熊金城) P135
设\(X\)是一个拓扑空间,
\(Y\)是\(X\)的一个子集,
点\(a,b \in Y\),
则\(a,b\)属于\(Y\)的同一个连通分支,
当且仅当\(Y\)有一个连通子集同时包含\(a\)和\(b\).
%TODO proof
\end{proposition}

\begin{theorem}
%@see: 《点集拓扑讲义(第四版)》(熊金城) P135 定理4.3.1
设\(X\)是一个拓扑空间,
\(C\)是\(X\)的一个连通分支,
则\begin{itemize}
	\item 如果\(Y\)是\(X\)的一个连通子集,
	并且\(Y \cap C \neq \emptyset\),
	则\(Y \subseteq C\);

	\item \(C\)是一个连通子集;

	\item \(C\)是一个闭集.
\end{itemize}
%TODO proof
\end{theorem}

一般地说,连通分支可以不是开集.
例如,考虑有理数集\(\mathbb{Q}\)(作为实数空间\(\mathbb{R}\)的子空间),
假设\(x,y\in\mathbb{Q}\)满足\(x<y\),
如果\(\mathbb{Q}\)的一个子集\(E\)同时包含\(x,y\),
令\begin{equation*}
	A \defeq (-\infty,r) \cap E,
	B \defeq (r,+\infty) \cap E,
\end{equation*}
其中\(r\)是任何一个无理数,且\(x<r<y\),
此时易见\(A,B\)都是\(E\)的非空开集,并且\(E = A \cup B\),
因此\(E\)是非连通的.
以上论述说明:\(\mathbb{Q}\)中任意一个包含着多于一个点的集合都是不连通的,
或者说,\(\mathbb{Q}\)的连通分支都是单点集.
然而,\(\mathbb{Q}\)中每一个单点集都不是开集.
