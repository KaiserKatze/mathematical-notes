\section{局部连通空间}
引进新的概念之前,我们先来考察一个例子.
\begin{example}
%@see: 《点集拓扑讲义(第四版)》(熊金城) P139 例4.4.1
在欧氏空间\(\mathbb{R}^2\)中,
令\begin{equation*}
	S \defeq \Set*{
		(x,\sin(1/x))
		\given
		x\in(0,1]
	},
	\qquad
	T \defeq \{0\}\times[-1,1],
\end{equation*}
把\(S\)称为\DefineConcept{拓扑学家的正弦曲线}.
显然\(S\)是区间\((0,1]\)在一个连续映射下的像,因此\(S\)是连通的.
此外,容易验证\(\TopoClosureL{S} = S \cup T\),
因此\(\TopoClosureL{S}\)也是连通的.
尽管如此,倘若我们查看\(\TopoClosureL{S}\)中的点,容易发现它们明显地分为两类:
\(S\)中每一个点的任何一个“较小的”邻域中都包含着一个连通的邻域,
而\(T\)中的每一个点的任何一个邻域都是不连通的.
\end{example}
我们用以下术语将这两个类型的点区别开来.
\begin{definition}
%@see: 《点集拓扑讲义(第四版)》(熊金城) P139 定义4.4.1
设\(X\)是一个拓扑空间,\(x \in X\).
如果\(x\)的每一个邻域\(U\)中,都包含着\(x\)的某一个连通的邻域\(V\),
则称“拓扑空间\(X\)在点\(x\)是\DefineConcept{局部连通的}”.
\end{definition}
\begin{definition}
%@see: 《点集拓扑讲义(第四版)》(熊金城) P139 定义4.4.1
设\(X\)是一个拓扑空间.
如果\(X\)在它的每一个点都是局部连通的,
则称“拓扑空间\(X\)是\DefineConcept{局部连通空间}”.
\end{definition}

根据上述定义,拓扑学家的正弦曲线\(S\)的闭包\(\TopoClosureL{S}\)
在其属于\(S\)的每一个点都是局部连通的,
而在其属于\(T\)的每一个点都不是局部连通的,
因此,尽管\(\TopoClosureL{S}\)是一个连通空间,但它却不是一个局部连通空间.

局部连通的拓扑空间也不必是连通的.
例如,每一个离散空间都是局部连通空间,但是包含多于一个点的离散空间却不是连通空间.
又例如,\(n\)维欧氏空间\(\mathbb{R}^n\)的任何一个开子空间都是局部连通的
(这是因为每一个球形邻域都同胚于整个欧氏空间\(\mathbb{R}^n\),因而是连通的),
特别地,欧氏空间\(\mathbb{R}^n\)本身是局部连通的;
另一方面,欧氏空间\(\mathbb{R}^n\)中两个互斥的非空开集的并作为子空间就一定不是连通的.

\begin{proposition}
%@see: 《点集拓扑讲义(第四版)》(熊金城) P140
拓扑空间\(X\)在点\(x \in X\)是局部连通的,
当且仅当\(x\)的所有连通邻域构成点\(x\)的一个邻域基.
%TODO proof
\end{proposition}

\begin{theorem}
%@see: 《点集拓扑讲义(第四版)》(熊金城) P140 定理4.4.1
设\(X\)是一个拓扑空间,则以下命题等价:\begin{itemize}
	\item \(X\)是一个局部连通空间;
	\item \(X\)的任何一个开集的任何一个连通分支都是开集;
	\item \(X\)有一个基,它的每一个元素都是连通的.
\end{itemize}
%TODO proof
\end{theorem}

\begin{corollary}
%@see: 《点集拓扑讲义(第四版)》(熊金城) P140
局部连通空间的每一个连通分支都是开集.
%TODO proof
\end{corollary}

\begin{theorem}
%@see: 《点集拓扑讲义(第四版)》(熊金城) P140 定理4.4.2
设\(X,Y\)都是拓扑空间,
\(X\)是局部连通的,
\(f\colon X \to Y\)是一个连续开映射,
则\(f(X)\)是一个局部连通空间.
%TODO proof
\end{theorem}
\begin{remark}
%@see: 《点集拓扑讲义(第四版)》(熊金城) P141
由上述定理可知,拓扑空间的局部连通性是一个拓扑不变性质.
\end{remark}

\begin{theorem}
%@see: 《点集拓扑讲义(第四版)》(熊金城) P141 定理4.4.3
设\(\AutoTuple{X}{n}\)是\(n\ (n\geq1)\)个局部连通空间,
则拓扑积空间\(\AutoTuple{X}{n}[\times]\)也是局部连通空间.
%TODO proof
\end{theorem}

应用这些定理,有些事情说起来就会简单得多.
例如,由于所有开区间构成实数空间\(\mathbb{R}\)的一个基,所以它是局部连通的.
又例如,\(n\)维欧氏空间\(\mathbb{R}^n\)是\(n\)个实数空间的积空间,所以它也是局部连通的.
