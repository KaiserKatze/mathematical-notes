\section{可数性公理}
我们知道,基和邻域基,对于确定拓扑空间的拓扑和验证映射的连续性,都有着重要的意义,
它们的元素的“个数”越少,讨论起来就越是方便.
因此我们试图对拓扑空间的基或邻域基的基数加以限制,
但又希望加了限制的拓扑空间仍能兼容绝大多数常见的拓扑空间,如欧氏空间、度量空间等.
以下的讨论表明,将基或邻域基限定为可数集是恰当的.

\subsection{第二可数性公理}
\begin{definition}
%@see: 《点集拓扑讲义(第四版)》(熊金城) P148
设\(\B\)是拓扑空间\(X\)的一个基.
如果\(\B\)是可数的,
则称“\(\B\)是拓扑空间\(X\)的一个\DefineConcept{可数基}”.
\end{definition}

\begin{definition}
%@see: 《点集拓扑讲义(第四版)》(熊金城) P148 定义5.1.1
设\(X\)是一个拓扑空间.
如果\(X\)有一个可数基,
则称“\(X\)满足\DefineConcept{第二可数性公理}”
“\(X\)是一个满足第二可数性公理的空间”
或“\(X\)是一个 \DefineConcept{\(A_2\)空间}”.
\end{definition}

\begin{example}
%@see: 《点集拓扑讲义(第四版)》(熊金城) P148 定理5.1.1
证明:实数空间\(\mathbb{R}\)满足第二可数性公理.
%TODO proof
\end{example}

\begin{example}
%@see: 《点集拓扑讲义(第四版)》(熊金城) P149
证明:含有不可数个点的离散空间不满足第二可数性公理.
\begin{proof}
因为离散空间的每一个单点子集都是开集,
而一个单点集不能表为异于自身的非空集合的并,
因此离散空间的每一个基必定包含它的所有单点子集.
\end{proof}
\end{example}

\subsection{第一可数性公理}
\begin{definition}
%@see: 《点集拓扑讲义(第四版)》(熊金城) P148
\def\Vx{\mathscr{V}_x}
设\(\Vx\)是点\(x\)的一个邻域基.
如果\(\Vx\)是可数的,
则称“\(\Vx\)是点\(x\)的一个\DefineConcept{可数邻域基}”.
\end{definition}

\begin{definition}
%@see: 《点集拓扑讲义(第四版)》(熊金城) P149 定义5.1.2
设\(X\)是一个拓扑空间.
如果\(X\)的每一个点都有一个可数邻域基,
则称“\(X\)满足\DefineConcept{第一可数性公理}”
“\(X\)是一个满足第一可数性公理的空间”
或“\(X\)是一个 \DefineConcept{\(A_1\)空间}”.
\end{definition}

\begin{theorem}
%@see: 《点集拓扑讲义(第四版)》(熊金城) P149 定理5.1.2
每一个度量空间都满足第一可数性公理.
%TODO proof
\end{theorem}

\begin{example}
%@see: 《点集拓扑讲义(第四版)》(熊金城) P149 例5.1.1
证明:含有不可数个点的可数补空间不满足第一可数性公理.
%TODO proof
\end{example}

\subsection{两个可数性公理的关系}
\begin{theorem}
%@see: 《点集拓扑讲义(第四版)》(熊金城) P149 定理5.1.3
每一个满足第二可数性公理的拓扑空间都满足第一可数性公理.
%TODO proof
\end{theorem}

\begin{example}
%@see: 《点集拓扑讲义(第四版)》(熊金城) P150
举例说明:满足第一可数性公理的拓扑空间不一定满足第二可数性公理.
\begin{solution}
显然任意一个离散空间均满足第一可数性公理,
但是含有不可数个点的离散空间不满足第二可数性公理.
\end{solution}
\end{example}

\subsection{可遗传性质}
\begin{theorem}
%@see: 《点集拓扑讲义(第四版)》(熊金城) P150 定理5.1.4
设\(X,Y\)都是拓扑空间,
\(f\colon X \to Y\)是一个满的连续开映射.
\begin{itemize}
	\item 如果\(X\)满足第二可数性公理,则\(Y\)也满足第二可数性公理.
	\item 如果\(X\)满足第一可数性公理,则\(Y\)也满足第一可数性公理.
\end{itemize}
%TODO proof
\end{theorem}
\begin{remark}
%@see: 《点集拓扑讲义(第四版)》(熊金城) P150
由上述定理可知,“拓扑空间是否满足第一可数性公理、第二可数性公理”是一个拓扑不变性质.
\end{remark}

\begin{definition}
%@see: 《点集拓扑讲义(第四版)》(熊金城) P150
设\(P(x)\)是一个谓词公式,
\(X\)是一个拓扑空间.
如果
	只要\(X\)满足\(P(X)\),
	就有\(X\)的任意一个子空间\(Y\)也满足\(P(Y)\),
则称“\(P\)是一个\DefineConcept{可遗传性质}”.
\end{definition}

\begin{proposition}
%@see: 《点集拓扑讲义(第四版)》(熊金城) P150
拓扑空间的离散性和平庸性都是可遗传性质.
\end{proposition}

\begin{proposition}
%@see: 《点集拓扑讲义(第四版)》(熊金城) P150
拓扑空间的连通性不是可遗传性质.
\end{proposition}

\begin{definition}
%@see: 《点集拓扑讲义(第四版)》(熊金城) P150
设\(P(x)\)是一个谓词公式,
\(X\)是一个拓扑空间.
如果
	只要\(X\)满足\(P(X)\),
	就有\(X\)的任意一个开子空间\(Y\)也满足\(P(Y)\),
则称“\(P\)是一个\DefineConcept{对于开子空间可遗传性质}”.
\end{definition}

\begin{definition}
%@see: 《点集拓扑讲义(第四版)》(熊金城) P150
设\(P(x)\)是一个谓词公式,
\(X\)是一个拓扑空间.
如果
	只要\(X\)满足\(P(X)\),
	就有\(X\)的任意一个闭子空间\(Y\)也满足\(P(Y)\),
则称“\(P\)是一个\DefineConcept{对于闭子空间可遗传性质}”.
\end{definition}

\begin{proposition}
%@see: 《点集拓扑讲义(第四版)》(熊金城) P151
拓扑空间的局部连通性是对于开子空间可遗传性质.
\end{proposition}

\begin{theorem}
%@see: 《点集拓扑讲义(第四版)》(熊金城) P151 定理5.1.5
设\(X\)是一个拓扑空间.
\begin{itemize}
	\item 如果\(X\)满足第二可数性公理,则\(X\)的任意一个子空间也满足第二可数性公理.
	\item 如果\(X\)满足第一可数性公理,则\(X\)的任意一个子空间也满足第一可数性公理.
\end{itemize}
%TODO proof
\end{theorem}
\begin{remark}
%@see: 《点集拓扑讲义(第四版)》(熊金城) P150
由上述定理可知,“拓扑空间是否满足第一可数性公理、第二可数性公理”是一个可遗传性质.
\end{remark}

\begin{theorem}
%@see: 《点集拓扑讲义(第四版)》(熊金城) P151 定理5.1.6
设\(\AutoTuple{X}{n}\)是一族拓扑空间,
\(X\)是\(\AutoTuple{X}{n}\)的积空间.
\begin{itemize}
	\item 如果\(\AutoTuple{X}{n}\)均满足第二可数性公理,则\(X\)也满足第二可数性公理.
	\item 如果\(\AutoTuple{X}{n}\)均满足第一可数性公理,则\(X\)也满足第一可数性公理.
\end{itemize}
%TODO proof
\end{theorem}
\begin{remark}
%@see: 《点集拓扑讲义(第四版)》(熊金城) P150
由上述定理可知,“拓扑空间是否满足第一可数性公理、第二可数性公理”是一个有限可积性质.
\end{remark}

\begin{example}
%@see: 《点集拓扑讲义(第四版)》(熊金城) P151 推论5.1.7
证明:\(n\)维欧氏空间\(\mathbb{R}^n\)的每一个子空间都满足第二可数性公理.
%TODO proof
\end{example}

\begin{theorem}
%@see: 《点集拓扑讲义(第四版)》(熊金城) P152 定理5.1.8
设\(X\)是一族非空拓扑空间\(\{X_\gamma\}_{\gamma \in \Gamma}\)的积空间,
则“\(X\)满足第二可数性公理”的充分必要条件是:
指标集\(\Gamma\)中有一个可数子集\(\Gamma_1\),
使得当\(\alpha \in \Gamma_1\)时\(X_\alpha\)满足第二可数性公理,
并且当\(\alpha \in \Gamma - \Gamma_1\)时\(X_\alpha\)是平庸空间.
%TODO proof
\end{theorem}

\begin{theorem}
%@see: 《点集拓扑讲义(第四版)》(熊金城) P155 习题 7.
设\(X\)是一族非空拓扑空间\(\{X_\gamma\}_{\gamma \in \Gamma}\)的积空间,
则“\(X\)满足第一可数性公理”的充分必要条件是:
指标集\(\Gamma\)中有一个可数子集\(\Gamma_1\),
使得当\(\alpha \in \Gamma_1\)时\(X_\alpha\)满足第一可数性公理,
并且当\(\alpha \in \Gamma - \Gamma_1\)时\(X_\alpha\)是平庸空间.
%TODO proof
\end{theorem}

\subsection{满足第一可数性公理的空间中序列的性质}
\begin{theorem}
%@see: 《点集拓扑讲义(第四版)》(熊金城) P153 定理5.1.9
设\(X\)是一个拓扑空间.
如果点\(x \in X\)有一个可数邻域基,
则点\(x \in X\)有一个可数邻域基\(\{U_n\}_{n\in\mathbb{Z}^+}\)
满足\(U_i \supseteq U_{i+1}\ (i=1,2,\dotsc)\).
%TODO proof
\end{theorem}

\begin{theorem}
%@see: 《点集拓扑讲义(第四版)》(熊金城) P153 定理5.1.10
设\(X\)是一个满足第一可数性公理的空间,\(A\)是\(X\)的一个子集,点\(x \in X\),
则\(x\)是\(A\)的一个聚点,
当且仅当在集合\(A-\{x\}\)中有一个序列收敛于\(x\).
%TODO proof
\end{theorem}

\begin{theorem}
%@see: 《点集拓扑讲义(第四版)》(熊金城) P153 定理5.1.11
设\(X,Y\)都是拓扑空间,
\(X\)满足第一可数性公理,
点\(x \in X\),
映射\(f\colon X \to Y\),
则“\(f\)在点\(x\)连续”的充分必要条件是:
如果\(X\)中的序列\(\{x_n\}\)收敛于\(x\),
则\(Y\)中的序列\(\{f(x_n)\}\)收敛于\(f(x)\).
%TODO proof
\end{theorem}

\begin{theorem}
%@see: 《点集拓扑讲义(第四版)》(熊金城) P153 定理5.1.12
设\(X,Y\)都是拓扑空间,
\(X\)满足第一可数性公理,
映射\(f\colon X \to Y\),
则“\(f\)是一个连续映射”的充分必要条件是:
对于任意一点\(x \in X\),
只要\(X\)中的序列\(\{x_n\}\)收敛于\(x\),
就有\(Y\)中的序列\(\{f(x_n)\}\)收敛于\(f(x)\).
%TODO proof
\end{theorem}
