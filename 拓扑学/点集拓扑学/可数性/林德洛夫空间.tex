\section{林德洛夫空间}
\begin{definition}
%@see: 《点集拓扑讲义(第四版)》(熊金城) P160 定义5.3.1
设\(\sfA\)是一个集族,\(B\)是一个集合.
如果\(\bigcup \sfA \supseteq B\),
则称“集族\(\sfA\)是集合\(B\)的一个\DefineConcept{覆盖}”.
\end{definition}
\begin{definition}
%@see: 《点集拓扑讲义(第四版)》(熊金城) P160 定义5.3.1
设集族\(\sfA\)是集合\(B\)的一个覆盖.
如果\(\sfA\)是可数集,
则称“集族\(\sfA\)是集合\(B\)的一个\DefineConcept{可数覆盖}”.
\end{definition}
\begin{definition}
%@see: 《点集拓扑讲义(第四版)》(熊金城) P160 定义5.3.1
设集族\(\sfA\)是集合\(B\)的一个覆盖.
如果\(\sfA\)是有限集,
则称“集族\(\sfA\)是集合\(B\)的一个\DefineConcept{有限覆盖}”.
\end{definition}
\begin{definition}
%@see: 《点集拓扑讲义(第四版)》(熊金城) P160 定义5.3.1
设集族\(\sfA\)是集合\(B\)的一个覆盖.
如果\(\sfA\)的一个子族\(\sfA_1\)也是集合\(B\)的覆盖,
则称“子集族\(\sfA_1\)是覆盖\(\sfA\)(关于集合\(B\))的一个\DefineConcept{子覆盖}”.
\end{definition}
\begin{definition}
%@see: 《点集拓扑讲义(第四版)》(熊金城) P160 定义5.3.1
设\(X\)是一个拓扑空间,
\(B\)是\(X\)的一个子集,
\(\sfA\)是\(X\)中的一个开集族.
如果\(\sfA\)是\(B\)的一个覆盖,
则称“\(\sfA\)是\(B\)的一个\DefineConcept{开覆盖}”.
\end{definition}
\begin{definition}
%@see: 《点集拓扑讲义(第四版)》(熊金城) P160 定义5.3.1
设\(X\)是一个拓扑空间,
\(B\)是\(X\)的一个子集,
\(\sfA\)是\(X\)中的一个闭集族.
如果\(\sfA\)是\(B\)的一个覆盖,
则称“\(\sfA\)是\(B\)的一个\DefineConcept{闭覆盖}”.
\end{definition}

\begin{definition}
%@see: 《点集拓扑讲义(第四版)》(熊金城) P161 定义5.3.2
设\(X\)是一个拓扑空间.
如果\(X\)的每一个开覆盖都有一个可数子覆盖,
则称“拓扑空间\(X\) \DefineConcept{具有林德洛夫性质}”
或“\(X\)是一个\DefineConcept{林德洛夫空间}”.
\end{definition}

\begin{example}
%@see: 《点集拓扑讲义(第四版)》(熊金城) P161
证明:含有不可数个点的离散空间一定不是林德洛夫空间.
\begin{proof}
设\(X\)是一个含有不可数个点的离散空间.
因为\(X\)中所有单点子集\begin{equation*}
	\sfA \defeq \Set{
		A
		\given
		A = \{a\},
		a \in X
	}
\end{equation*}
构成\(X\)的一个开覆盖,
而\(\sfA\)没有任何可数子覆盖.
\end{proof}
\end{example}

\begin{theorem}[林德洛夫定理]\label{theorem:林德洛夫空间.林德洛夫定理}
%@see: 《点集拓扑讲义(第四版)》(熊金城) P161 定理5.3.1(Lindelof定理)
任何一个满足第二可数性公理的拓扑空间都是林德洛夫空间.
%TODO proof
\end{theorem}

\begin{corollary}
%@see: 《点集拓扑讲义(第四版)》(熊金城) P162 推论5.3.2
满足第二可数性公理的拓扑空间的每一个子空间都是林德洛夫空间.
%TODO proof
\end{corollary}

\begin{example}
%@see: 《点集拓扑讲义(第四版)》(熊金城) P162
证明:\(n\)维欧氏空间\(\mathbb{R}^n\)的每一个子空间都是林德洛夫空间.
%TODO proof
\end{example}

\begin{example}
%@see: 《点集拓扑讲义(第四版)》(熊金城) P162
举例说明:林德洛夫空间可以不满足第二可数性公理.
%TODO proof
\end{example}

\begin{example}
%@see: 《点集拓扑讲义(第四版)》(熊金城) P162 例5.3.1
举例说明:林德洛夫空间可以不满足第二可数性公理.
%TODO proof
\end{example}

\begin{example}
%@see: 《点集拓扑讲义(第四版)》(熊金城) P162 例5.3.1
举例说明:即便拓扑空间\(X\)的每一个子空间都是林德洛夫空间,但是\(X\)不满足第二可数性公理.
%TODO proof
\end{example}

\begin{theorem}
%@see: 《点集拓扑讲义(第四版)》(熊金城) P162 定理5.3.3
设\(X\)是一个度量空间.
如果\(X\)是林德洛夫空间,
则\(X\)满足第二可数性公理.
%TODO proof
\end{theorem}

\begin{example}
%@see: 《点集拓扑讲义(第四版)》(熊金城) P163 例5.3.2
举例说明:林德洛夫空间的子空间可以不是林德洛夫空间.
%TODO
\end{example}
\begin{remark}
上例说明:拓扑空间的林德洛夫性质不是可遗传性质.
\end{remark}

\begin{example}
%@see: 《点集拓扑讲义(第四版)》(熊金城) P163
举例说明:两个林德洛夫空间的积空间可以不是林德洛夫空间.
%TODO
\end{example}

\begin{theorem}
%@see: 《点集拓扑讲义(第四版)》(熊金城) P163 定理5.3.4
林德洛夫空间的每一个闭子空间都是林德洛夫空间.
%TODO proof
\end{theorem}

\begin{theorem}
%@see: 《点集拓扑讲义(第四版)》(熊金城) P163 定理5.3.5
设拓扑空间\(X\)的任意一个子空间都是林德洛夫空间.
如果\(A\)是\(X\)的一个不可数子集,
则\(A\)中必定含有\(A\)的某个聚点.
%TODO proof
\end{theorem}

\begin{proposition}
%@see: 《点集拓扑讲义(第四版)》(熊金城) P164
设\(X\)是一个满足第二可数性公理的拓扑空间,
则\(X\)的每一个不可数子集\(A\)都含有\(A\)的某个聚点.
%TODO proof
\end{proposition}
