\section{可分空间}
\begin{definition}
%@see: 《Real Analysis Modern Techniques and Their Applications Second Edition》(Gerald B. Folland) P13
%@see: 《点集拓扑讲义(第四版)》(熊金城) P156 定义5.2.1
设\((X,\T)\)是拓扑空间,\(A \subseteq X\).
若\(\TopoClosureL{A}=X\),
则称“\(A\)是\(X\)的\DefineConcept{稠密子集}(dense subset)”
或“\(A\)在\(X\)中是\DefineConcept{稠密的}(\(A\) is \emph{dense} in \(X\))”.
\end{definition}

以下定理从一个侧面说明了讨论拓扑空间中稠密子集的意义.
\begin{theorem}
%@see: 《点集拓扑讲义(第四版)》(熊金城) P156 定理5.2.1
设\(X\)是一个拓扑空间,\(D\)是\(X\)中一个稠密子集,
\(f,g\)都是从\(X\)到\(\mathbb{R}\)的连续映射.
如果\(f \SetRestrict D = g \SetRestrict D\),
则\(f = g\).
%TODO proof
\end{theorem}

我们也希望讨论“有着较少点数的”稠密子集的拓扑空间,
例如具有有限稠密子集的拓扑空间.但这类拓扑空间过于简单,
大部分我们感兴趣的拓扑空间都不是这种情形,讨论起来意思不大.
例如一个度量空间如果有一个有限的稠密子集的话,那么这个空间一定就是一个离散空间.
相反,后继的讨论表明,许多重要的拓扑空间都有可数稠密子集.

\begin{definition}
%@see: 《Real Analysis Modern Techniques and Their Applications Second Edition》(Gerald B. Folland) P14
%@see: 《基础拓扑学讲义》(尤承业) P17
%@see: 《点集拓扑讲义(第四版)》(熊金城) P156 定义5.2.2
设\((X,\T)\)是拓扑空间.
若\(X\)存在一个可数稠密子集,
则称“拓扑空间\(X\)是\DefineConcept{可分的}(separable)”
或“\(X\)是\DefineConcept{可分空间}(separable space)”.
%@see: https://mathworld.wolfram.com/SeparableSpace.html
\end{definition}

\begin{theorem}\label{theorem:可分空间.满足第二可数性公理的空间都是可分空间}
%@see: 《点集拓扑讲义(第四版)》(熊金城) P157 定理5.2.2
每一个满足第二可数性公理的空间都是可分空间.
%TODO proof
\end{theorem}

\begin{example}
%@see: 《点集拓扑讲义(第四版)》(熊金城) P157
证明:含有不可数个点的离散空间一定不是可分的.
\begin{proof}
在含有不可数个点的离散空间中,任意一个可数子集的闭包都等于它自身,而不可能等于整个空间.
\end{proof}
\end{example}

\begin{corollary}
%@see: 《点集拓扑讲义(第四版)》(熊金城) P157 推论5.2.3
满足第二可数性公理的拓扑空间的每一个子空间都是可分空间.
\begin{proof}
因为“拓扑空间是否满足第二可数性公理”是一个可遗传性质,
所以由\cref{theorem:可分空间.满足第二可数性公理的空间都是可分空间} 可知,
满足第二可数性公理的拓扑空间的每一个子空间都是可分空间.
\end{proof}
\end{corollary}

\begin{example}
%@see: 《点集拓扑讲义(第四版)》(熊金城) P157
证明:\(n\)维欧氏空间\(\mathbb{R}^n\)中的每一个子空间(包括它自己)都是可分空间.
%TODO proof
\end{example}

\begin{example}
%@see: 《点集拓扑讲义(第四版)》(熊金城) P157 例5.2.1
设\((X,\T)\)是一个拓扑空间,
任取一个不属于\(X\)的元素\(\infty\)
(例如我们可以取\(\infty \defeq X\)),
令\begin{align*}
	X^* &\defeq X \cup \{\infty\}, \\
	\T^* &\defeq \Set{
		A \cup \{\infty\}
		\given
		A \in \T
	} \cup \{\emptyset\}.
\end{align*}
证明:\begin{itemize}
	\item \((X^*,\T^*)\)是一个拓扑空间;
	\item \((X^*,\T^*)\)是一个可分空间;
	\item \((X^*,\T^*)\)满足第二可数性公理,当且仅当\((X,\T)\)满足第二可数性公理;
	\item \(\T = \T^* \TopoRestrict X\);
	\item \((X,\T)\)是\((X^*,\T^*)\)的一个子空间.
\end{itemize}
%TODO proof
\end{example}
\begin{remark}
%@see: 《点集拓扑讲义(第四版)》(熊金城) P157
%@see: 《点集拓扑讲义(第四版)》(熊金城) P158
由上例可知:\begin{itemize}
	\item 可分空间可以不满足第二可数性公理;
	\item 可分空间的子空间可以不是可分空间,换言之,拓扑空间的可分性不是可遗传性质.
\end{itemize}
\end{remark}

\begin{theorem}
%@see: 《点集拓扑讲义(第四版)》(熊金城) P158 定理5.2.4
每一个可分度量空间都满足第二可数性公理.
%TODO proof
\end{theorem}

\begin{corollary}
%@see: 《点集拓扑讲义(第四版)》(熊金城) P159 推论5.2.5
可分度量空间的每一个子空间都是可分空间.
%TODO proof
\end{corollary}

\begin{example}
%@see: 《基础拓扑学讲义》(尤承业) P18
实数余有限拓扑空间\((\mathbb{R},\T_f)\)是可分的,
事实上它的任一无穷子集都是稠密的:
\(\mathbb{Q}\)就是它的一个可数稠密子集.
但是实数余可数拓扑空间\((\mathbb{R},\T_c)\)是不可分的,
因为它的任一可数集都是闭集,不可能稠密.
\end{example}

\begin{remark}
应当注意,当我们把一个度量空间看作拓扑空间时,
空间的拓扑是由度量诱导出来的拓扑,
而一个集合是不是一个某一个点的邻域,
无论是按\cref{definition:度量空间.邻域的概念},
还是按\cref{definition:拓扑学.点的分类},
都是一回事.
\end{remark}
