\section{拓扑空间中的序列}
\subsection{序列}
\begin{definition}
%@see: 《点集拓扑讲义(第四版)》(熊金城) P91 定义2.7.1
把从自然数集\(\omega\)的非空子集\(D\)到集合\(X\)的映射\[
	S\colon D \to X, n \mapsto x_n
\]称为“\(X\)中的一个\DefineConcept{序列}”,
记作\(\{x_n\}_{n \in D}\).
\end{definition}

特别地,当\(D = \omega\)时,数列\(\{x_n\}_{n \in D}\)可以记作\(\{x_n\}_{n\geq0}\);
当\(D = \Set{ n \in \omega \given n \geq 1}\)时,
数列\(\{x_n\}_{n \in D}\)可以记作\(\{x_n\}_{n\geq1}\);
以此类推.

当序列\(\{x_n\}_{n \in D}\)的值域是一个单元素集时,
称“\(\{x_n\}_{n \in D}\)是一个\DefineConcept{常值序列}”.

\subsection{序列的单调性}
\begin{definition}
%@see: 《测度论讲义(第三版)》(严加安) P2 1.1.4
%@see: 《实变函数论(第三版)》(周民强) P9 定义1.8
设\(\{x_n\}_{n \in D}\)是集合\(X\)中的一个序列.

若\[
	(\forall n\in\omega)
	[x_n \subseteq x_{n+1}],
\]
则称“\(\{x_n\}_{n \in D}\)是\(X\)中的一个\DefineConcept{单调增序列}”.

若\[
	(\forall n\in\omega)
	[x_n \supseteq x_{n+1}],
\]
则称“\(\{x_n\}_{n \in D}\)是\(X\)中的一个\DefineConcept{单调减序列}”.

我们将“单调增序列”与“单调减序列”统称为\DefineConcept{单调序列}.
\end{definition}

\begin{definition}
设\(\{x_n\}_{n \in D}\)是集合\(X\)中的一个序列.
定义:\begin{align*}
	\bigcup_{k=n}^m x_k
	&\defeq
	\bigcup_{n \leq k \leq m, k\in\mathbb{N}} x_k, \\
	\bigcup_{k=n}^\infty x_k
	&\defeq
	\bigcup_{k \geq n, k\in\mathbb{N}} x_k, \\
	\bigcap_{k=n}^m x_k
	&\defeq
	\bigcap_{n \leq k \leq m, k\in\mathbb{N}} x_k, \\
	\bigcap_{k=n}^\infty x_k
	&\defeq
	\bigcap_{k \geq n, k\in\mathbb{N}} x_k.
\end{align*}
\end{definition}

\begin{definition}
%@see: 《实变函数论(第三版)》(周民强) P9 定义1.8
设\(\{x_n\}_{n \in D}\)是集合\(X\)中的一个序列.
定义:\[
	\lim_{n\to\infty} x_n
	\defeq
	\left\{ \def\arraystretch{2} \begin{array}{cl}
		\bigcup_{n=1}^\infty x_n, & \text{$\{x_n\}$是单调增序列}, \\
		\bigcap_{n=1}^\infty x_n, & \text{$\{x_n\}$是单调减序列},
	\end{array} \right.
\]
称其为“\(\{x_n\}_{n \in D}\)的\DefineConcept{极限}”.
\end{definition}

\begin{definition}
%@see: 《测度论讲义(第三版)》(严加安) P2 1.1.4
设\(\{x_n\}_{x \in D}\)是集合\(X\)中的一个序列.

定义:\[
	\limsup_{n\to\infty} x_n
	\defeq
	\bigcap_{n=1}^\infty
	\bigcup_{k=n}^\infty
	x_k,
\]
称其为“\(\{x_n\}_{x \in D}\)的\DefineConcept{上极限}”.

定义:\[
	\liminf_{n\to\infty} x_n
	\defeq
	\bigcup_{n=1}^\infty
	\bigcap_{k=n}^\infty
	x_k,
\]
称其为“\(\{x_n\}_{x \in D}\)的\DefineConcept{下极限}”.
\end{definition}
容易看出,\(\limsup_{n\to\infty} x_n\)的任一元素属于无限多个\(x_n\),
而\(\liminf_{n\to\infty} x_n\)的任一元素至多不属于有限多个\(x_n\).

\begin{example}
%@see: 《实变函数论(第三版)》(周民强) P10 例7
设\(E,F\)都是集合,
集合\(X\)中的序列\(\{x_k\}_{k\geq1}\)满足\[
	x_k = \left\{ \begin{array}{cl}
		E, & \text{$k$是奇数}, \\
		F, & \text{$k$是偶数}
	\end{array} \right.
	\quad(k=1,2,\dotsc),
\]
从而我们有\[
	\limsup_{k\to\infty} x_k
	= E \cup F, \qquad
	\liminf_{k\to\infty} x_k
	= E \cap F.
\]
\end{example}

\begin{proposition}
%@see: 《实变函数论(第三版)》(周民强) P10
设\(E\)是集合,\(\{x_n\}_{n\geq1}\)是集合\(X\)中的一个序列,
则\begin{gather}
	E - \limsup_{n\to\infty} x_n = \liminf_{n\to\infty} (E - x_n); \\
	E - \liminf_{n\to\infty} x_n = \limsup_{n\to\infty} (E - x_n).
\end{gather}
\end{proposition}

\begin{proposition}
%@see: 《测度论讲义(第三版)》(严加安) P2 1.1.4
设\(\{x_n\}_{n\geq1}\)是集合\(X\)中的一个序列,
则\begin{equation}
	\liminf_{n\to\infty} x_n
	\subseteq
	\limsup_{n\to\infty} x_n.
\end{equation}
%@see: https://math.stackexchange.com/questions/107931/lim-sup-and-lim-inf-of-sequence-of-sets
\end{proposition}

\begin{theorem}
%@see: 《测度论讲义(第三版)》(严加安) P2 1.1.4
设\(\{x_n\}_{n\geq1}\)是集合\(X\)中的一个序列.
如果\[
	\liminf_{n\to\infty} x_n
	= \limsup_{n\to\infty} x_n
	= A,
\]
则\(\lim_{n\to\infty} x_n = A\).
\end{theorem}

\subsection{收敛序列}
\begin{definition}\label{definition:序列.序列的聚点}
%@see: 《点集拓扑讲义(第四版)》(熊金城) P91 定义2.7.2
设\(\{x_n\}_{n \in D}\)是拓扑空间\(X\)中的一个序列,\(x \in X\).
如果对于\(x\)的每一个邻域\(U\),
存在\(N \in \omega\),
使得当\(n > N\)时
有\(x_n \in U\),
则称“点\(x\)是序列\(\{x_n\}_{n \in D}\)的一个\DefineConcept{聚点}”
“序列\(\{x_n\}_{n \in D}\)收敛于\(x\)”,
记作\(\lim_{n\to\infty} x_n = x\);
并称“序列\(\{x_n\}_{n \in D}\)是一个\DefineConcept{收敛序列}”.
\end{definition}

拓扑空间中序列的收敛性质与我们在数学分析中熟悉的有很大的差别.
例如,平庸空间中任何一个序列都收敛,并且它们收敛于这个空间中的每一个点.
这时极限的唯一性就无法保证了.

\subsection{子序列}
\begin{definition}\label{definition:序列.子序列}
%@see: 《点集拓扑讲义(第四版)》(熊金城) P91 定义2.7.3
设\(S\)和\(S_1\)是拓扑空间\(X\)中的两个序列.
如果存在一个严格单调增加的映射\(\sigma\colon \omega \to \omega\),
使得\(S_1 = S \circ \sigma\),
则称“序列\(S_1\)是序列\(S\)的一个\DefineConcept{子序列}”.
\end{definition}

\subsection{序列与子序列的性质}
我们已经看到,我们以前熟悉的序列的许多性质,对于拓扑空间中的序列来说,是不满足的.
但总有一些性质还保留着,其中最主要的,就是以下三个定理.

\begin{theorem}
%@see: 《点集拓扑讲义(第四版)》(熊金城) P92 定理2.7.1
设\(\{x_n\}_{n\geq0}\)是拓扑空间\(X\)中的一个序列,
则\begin{itemize}
	\item 如果\(\{x_n\}_{n\geq0}\)是一个常值序列,
	即\((\forall n\geq0)[x_n=x]\),
	则\(\lim_{n\to\infty} x_n = x\).
	\item 如果序列\(\{x_n\}_{n\geq0}\)收敛于\(x \in X\),
	则序列\(\{x_n\}_{n\geq0}\)的每一个子序列也收敛于\(x\).
\end{itemize}
%TODO proof
\end{theorem}

\begin{theorem}\label{theorem:序列.去心邻域内收敛于中心的序列的聚点1}
%@see: 《点集拓扑讲义(第四版)》(熊金城) P92 定理2.7.2
设\(X\)是一个拓扑空间,\(A \subseteq X\),\(x \in X\).
如果存在\(A-\{x\}\)中的一个序列\(\{x_n\}_{n\geq0}\),
\(\{x_n\}_{n\geq0}\)收敛于\(x\),
则\(x\)是\(A\)的一个聚点.
%TODO proof
\end{theorem}

\begin{proposition}\label{theorem:序列.去心邻域内收敛于中心的序列的聚点2}
%@see: 《点集拓扑讲义(第四版)》(熊金城) P92 例2.7.1
\cref{theorem:序列.去心邻域内收敛于中心的序列的聚点1} 的逆命题不成立.
%TODO proof
% \begin{proof}
% 设\(X\)是一个不可数集,
% 考虑它的拓扑为可数补拓扑.
% 这时\[
% 	\text{$X$的一个子集$X_1$是闭集}
% 	\iff
% 	X_1=X
% 	\lor
% 	\text{$X_1$是可数集}.
% \]
% \end{proof}
\end{proposition}

\cref{theorem:序列.去心邻域内收敛于中心的序列的聚点2} 表明,
在一般的拓扑空间中,不能像在数学分析中那样,通过序列收敛的性质来刻画聚点.

\begin{theorem}\label{theorem:序列.连续映射的定义域中的序列与值域中的序列之间的关系1}
%@see: 《点集拓扑讲义(第四版)》(熊金城) P93 定理2.7.3
设\(X\)和\(Y\)是两个拓扑空间,映射\(f\colon X \to Y\),
则\begin{itemize}
	\item 如果\(f\)在点\(x_0 \in X\)连续,
	则\[
		\text{$X$中的一个序列$\{x_n\}_{n\geq0}$收敛于$x_0$}
		\implies
		\text{$Y$中的序列$\{f(x_n)\}_{n\geq0}$收敛于$f(x_0)$}.
	\]

	\item 如果\(f\)连续,
	则\[
		\text{$X$中的一个序列$\{x_n\}_{n\geq0}$收敛于$x \in X$}
		\implies
		\text{$Y$中的序列$\{f(x_n)\}_{n\geq0}$收敛于$f(x)$}.
	\]
\end{itemize}
%TODO proof
\end{theorem}

\begin{proposition}\label{theorem:序列.连续映射的定义域中的序列与值域中的序列之间的关系2}
%@see: 《点集拓扑讲义(第四版)》(熊金城) P93 例2.7.2
\cref{theorem:序列.连续映射的定义域中的序列与值域中的序列之间的关系1} 的逆命题不成立.
%TODO proof
\end{proposition}

\cref{theorem:序列.连续映射的定义域中的序列与值域中的序列之间的关系2} 表明,
在一般的拓扑空间中,不能像在数学分析中那样,通过序列收敛的性质来刻画映射的连续性.

至于在什么样的条件下,
\cref{theorem:序列.去心邻域内收敛于中心的序列的聚点1,theorem:序列.连续映射的定义域中的序列与值域中的序列之间的关系1} 的逆命题成立,
我们今后还要进行进一步的研究.

在度量空间中,序列的收敛可以通过度量来加以描述.
\begin{theorem}\label{theorem:序列.度量空间中的收敛序列}
%@see: 《点集拓扑讲义(第四版)》(熊金城) P94 定理2.7.4
设\((X,\rho)\)是一个度量空间,
\(\{x_n\}_{n\geq0}\)是\(X\)中的一个序列,\(x \in X\),
则以下命题等价:\begin{itemize}
	\item 序列\(\{x_n\}_{n\geq0}\)收敛于\(x\);
	\item 对于任意给定的实数\(\epsilon>0\),存在\(N\in\omega\),使得当\(n>N\)时有\(\rho(x_n,x)<\epsilon\);
	\item \(\lim_{n\to\infty} \rho(x_n,x) = 0\).
\end{itemize}
%TODO proof
\end{theorem}

% \begin{theorem}
% %@see: 《Real Analysis Modern Techniques and Their Applications Second Edition》(Folland) P14
% 设\(X\)是一个度量空间,\(E \subseteq X\),\(x \in X\),
% 则以下三个命题等价:\begin{itemize}
% 	\item \(x \in \overline{E}\).
% 	\item \((\forall r>0)[B(x,r) \cap E \neq \emptyset]\).
% 	\item 在\(E\)中存在一个序列\(\{x_n\}\)收敛于\(x\).
% \end{itemize}
% \begin{proof}
% 假设\(B(x,r) \cap E = \emptyset\),
% 则\(X-B(x,r)\)就是一个闭集,它包含\(E\)但不包括\(x\),
% 于是\(x \notin \overline{E}\).
% 再假设\(x \notin \overline{E}\),
% 因为\(X-\overline{E}\)是开集,
% 存在\(r>0\)使得\(B(x,r) \subseteq X-\overline{E} \subseteq X-E\).
% 因此\(x \in \overline{E} \iff (\forall r>0)[B(x,r) \cap E \neq \emptyset]\).

% 假设\((\forall r>0)[B(x,r) \cap E \neq \emptyset]\)成立,
% 对于\(\forall n\in\mathbb{N}\),
% 存在\(x_n \in B(x,n^{-1}) \cap E\),
% 使得\(x_n \to x\).
% 另一方面,假设\(B(x,r) \cap E = \emptyset\),
% 则\((\forall y \in E)[\rho(y,x) \geq r]\),
% 于是\(E\)中没有一个序列可以收敛于\(x\).
% 因此\((\forall r>0)[B(x,r) \cap E \neq \emptyset]
% \iff
% \text{在\(E\)中存在一个序列\(\{x_n\}\)收敛于\(x\)}\).
% \end{proof}
% \end{theorem}

\begin{example}
%@see: 《点集拓扑讲义(第四版)》(熊金城) P95 习题 3.
设\(X\)是一个度量空间.
证明:\begin{itemize}
	\item \(X\)中的任何一个收敛序列都只有唯一的极限;
	\item \cref{theorem:序列.去心邻域内收敛于中心的序列的聚点1} 的逆命题成立;
	\item 对于任何一个拓扑空间\(Y\)以及任何一个映射\(f\colon X \to Y\),\cref{theorem:序列.连续映射的定义域中的序列与值域中的序列之间的关系1} 的逆命题成立.
\end{itemize}
%TODO proof
\end{example}
