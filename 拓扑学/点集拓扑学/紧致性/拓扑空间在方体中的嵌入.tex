\section{拓扑空间在方体中的嵌入}
设\(X\)是一个拓扑空间,
\(\Gamma\)是一个集合.
根据定义,
从\(\Gamma\)到\(X\)的映射空间\(X^\Gamma\)
可以看作拓扑空间族\(\{X_\gamma\}_{\gamma \in \Gamma}\)的笛卡尔积
(其中每一个\(X_\gamma\)都等于\(X\)),
因此它必有一个积拓扑.

\begin{definition}
%@see: 《点集拓扑讲义(第四版)》(熊金城) P233
设\(X\)是一个拓扑空间,
\(\Gamma\)是一个集合.
把从\(\Gamma\)到\(X\)的映射空间\(X^\Gamma\)的积拓扑
称为“\(X^\Gamma\)的\DefineConcept{点式收敛拓扑}”.
\end{definition}

\begin{definition}
%@see: 《点集拓扑讲义(第四版)》(熊金城) P233 定义7.8.1
设\(\Gamma\)是一个集合,
从\(\Gamma\)到单位闭区间\([0,1]\)的映射空间\([0,1]^\Gamma\),连同它的点式收敛拓扑,
称为一个\DefineConcept{方体}.
\end{definition}

显然,\(n\)维欧氏空间\(\mathbb{R}^n\)中的单位方体是我们这里所说的方体的一个特例.

由于我们熟悉单位闭区间\([0,1]\)的拓扑特性,
所以方体\([0,1]^\Gamma\)的某些拓扑性质也容易得知.

\begin{proposition}
%@see: 《点集拓扑讲义(第四版)》(熊金城) P233
任意一个方体都是连通的紧致的提赫诺夫空间.
%TODO proof
\end{proposition}

\begin{proposition}
%@see: 《点集拓扑讲义(第四版)》(熊金城) P233
如果\(\Gamma\)是可数集,
那么方体\([0,1]^\Gamma\)是满足第二可数性公理的可度量化空间.
%TODO proof
\end{proposition}

然而,方体中究竟都包含了哪些类型的拓扑空间?
或者说,哪种拓扑空间可以嵌入到方体中去?
这是我们在本节要研究的问题.
可以注意到,
所有提赫诺夫空间,或者说所有紧致的豪斯多夫空间的每一个子空间,
都可以被“装”在某一个方体之中.

之前  % 《点集拓扑讲义(第四版)》(熊金城)第6章第6节
我们证明了
每一个满足第二可数性公理的\(T_3\)空间都可以嵌入希尔伯特空间.
那个定理的证明过程可以一般化,
这恰为我们解决现在的课题提供了一条有效的途径.

\begin{definition}
%@see: 《点集拓扑讲义(第四版)》(熊金城) P233 定义7.8.2
设\(X\)是一个拓扑空间,
\(F\)是一族映射,
\(F\)中的每一个元素都是从拓扑空间\(X\)到某一个拓扑空间的一个映射.
\begin{itemize}
	\item 如果对于任意\(a,b \in X\),
	只要满足\(a \neq b\),
	就存在\(f \in F\),
	使得\(f(a) \neq f(b)\),
	则称“\(F\)是一个\DefineConcept{区别点的映射族}”.

	\item 如果对于\(X\)中任意一个点\(x\)和\(X\)中任意一个不含点\(x\)的闭集\(B\),
	存在\(f \in F\),
	使得\(f(x) \notin \TopoClosureL{f(B)}\),
	则称“\(F\)是一个\DefineConcept{区别点和闭集的映射族}”.
\end{itemize}
\end{definition}

\begin{lemma}
%@see: 《点集拓扑讲义(第四版)》(熊金城) P234 引理7.8.1(嵌入引理)
设\(X\)是拓扑空间族\(\{X_\gamma\}_{\gamma \in \Gamma}\)的积空间,
\(Y\)是一个拓扑空间,
\(f\)是从\(Y\)到\(X\)的映射,
\(p_\gamma\)是从\(X\)到坐标集\(X_\gamma\)的投射,
令\begin{equation*}
	F \defeq \Set{
		p_\alpha \circ f
		\given
		\alpha \in \Gamma
	},
\end{equation*}
则\begin{itemize}
	\item \(f\)是一个连续映射,当且仅当\(F\)是一个连续映射族;
	\item \(f\)是一个单射,当且仅当\(F\)是一个区别点的映射族;
	\item 如果\(F\)是一个区别点和闭集的映射族,
	则\(g\colon Y \to f(Y), y \mapsto f(y)\)是一个开映射.
\end{itemize}
%TODO proof
\end{lemma}

\begin{theorem}
%@see: 《点集拓扑讲义(第四版)》(熊金城) P235 定理7.8.2(嵌入定理)
设\(X\)是一个拓扑空间,
则\(X\)是一个提赫诺夫空间,
当且仅当\(X\)可以嵌入某一个方体.
%TODO proof
\end{theorem}

\begin{theorem}
%@see: 《点集拓扑讲义(第四版)》(熊金城) P235 定理7.8.3
设\(X\)是一个拓扑空间,
则\(X\)是一个提赫诺夫空间,
当且仅当\(X\)可以嵌入某一个紧致的豪斯多夫空间.
%TODO proof
\end{theorem}
