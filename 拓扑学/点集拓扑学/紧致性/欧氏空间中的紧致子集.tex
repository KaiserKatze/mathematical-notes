\section{\texorpdfstring{$n$}{n}维欧氏空间中的紧致子集}
\begin{definition}
%@see: 《点集拓扑讲义(第四版)》(熊金城) P210 定义7.3.1
设\((X,\rho)\)是一个度量空间,
\(A\)是\(X\)的一个子集.
如果存在实数\(M>0\)使得\(\rho(x,y)<M\)对于任意\(x,y \in A\)成立,
则称“\(A\)是\(X\)的一个\DefineConcept{有界子集}”.
\end{definition}
\begin{definition}
%@see: 《点集拓扑讲义(第四版)》(熊金城) P210 定义7.3.1
设\((X,\rho)\)是一个度量空间.
如果\(X\)是自身的一个有界子集,
则称“度量空间\((X,\rho)\)是\DefineConcept{有界的}”
或“\((X,\rho)\)是一个\DefineConcept{有界度量空间}”.
\end{definition}

\begin{theorem}
%@see: 《点集拓扑讲义(第四版)》(熊金城) P210 定理7.3.1
紧致度量空间是有界的.
%TODO proof
\end{theorem}

\begin{corollary}
%@see: 《点集拓扑讲义(第四版)》(熊金城) P210
度量空间中的每一个紧致子集都是它的一个有界子集.
%TODO proof
\end{corollary}

\begin{lemma}
%@see: 《点集拓扑讲义(第四版)》(熊金城) P211 引理7.3.2
在实数空间\(\mathbb{R}\)中,
单位闭区间\([0,1]\)是一个紧致空间.
%TODO proof
\end{lemma}

\begin{proposition}
%@see: 《点集拓扑讲义(第四版)》(熊金城) P212
在实数空间\(\mathbb{R}\)中,
任意一个闭区间\([a,b]\)都是一个紧致空间.
%TODO proof
\end{proposition}

\begin{proposition}
%@see: 《点集拓扑讲义(第四版)》(熊金城) P212
在\(n\)维欧氏空间\(\mathbb{R}^n\)中,
任意一个闭方体\([a,b]^n\)都是一个紧致空间.
%TODO proof
\end{proposition}

\begin{theorem}
%@see: 《点集拓扑讲义(第四版)》(熊金城) P212 定理7.3.3
设\(A\)是\(n\)维欧氏空间\(\mathbb{R}^n\)中的一个子集,
则\(A\)是一个紧致子集,
当且仅当\(A\)是一个有界闭集.
%TODO proof
\end{theorem}

\begin{theorem}\label{theorem:欧氏空间中的紧致子集.从非空紧致空间到实数域的连续映射1}
%@see: 《点集拓扑讲义(第四版)》(熊金城) P212 定理7.3.4
设\(X\)是一个非空的紧致空间,
\(f\colon X \to \mathbb{R}\)是一个连续映射,
则存在\(x_1,x_2 \in X\),
使得对于任意\(x \in X\),
有\(f(x_1) \leq f(x) \leq f(x_2)\).
%TODO proof
%\cref{theorem:连通空间.从连通空间到实数域的连续映射1}
%\cref{theorem:连通空间.从连通空间到实数域的连续映射2}
\end{theorem}

\begin{example}
%@see: 《点集拓扑讲义(第四版)》(熊金城) P212
证明:\(n\)维单位球面\(S^n\)是一个紧致空间.
%TODO proof
\end{example}

\begin{example}
%@see: 《点集拓扑讲义(第四版)》(熊金城) P212
证明:\(n\)维欧氏空间\(\mathbb{R}^n\)不是紧致的.
%TODO proof
\end{example}

\begin{theorem}
%@see: 《点集拓扑讲义(第四版)》(熊金城) P213 定理7.3.5
设\(m,n\)都是正整数,
则\(n\)维单位球面\(S^n\)与\(m\)维欧氏空间\(\mathbb{R}^m\)不同胚.
\end{theorem}
