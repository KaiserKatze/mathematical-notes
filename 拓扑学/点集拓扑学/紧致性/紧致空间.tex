\section{紧致空间}
\begin{definition}
%@see: 《点集拓扑讲义(第四版)》(熊金城) P198 定义7.1.1
设\(X\)是一个拓扑空间.
如果\(X\)的每一个开覆盖都有一个有限子覆盖,
则称“拓扑空间\(X\)是\DefineConcept{紧致的}”
“拓扑空间\(X\)具有\DefineConcept{紧致性}”
或“\(X\)是一个\DefineConcept{紧致空间}”.
\end{definition}

\begin{proposition}
%@see: 《点集拓扑讲义(第四版)》(熊金城) P198
每一个紧致空间都是一个林德洛夫空间,反之不然.
%TODO proof
\end{proposition}

\begin{example}
%@see: 《点集拓扑讲义(第四版)》(熊金城) P198 例7.1.1
证明:实数空间\(\mathbb{R}\)不是一个紧致空间.
%TODO proof
\end{example}

\begin{definition}
%@see: 《点集拓扑讲义(第四版)》(熊金城) P198 定义7.1.2
设\(X\)是一个拓扑空间,
\(Y\)是\(X\)的一个子集.
如果\(Y\)作为\(X\)的子空间是一个紧致空间,
则称“\(Y\)是拓扑空间\(X\)的一个\DefineConcept{紧致子集}”.
\end{definition}

根据定义,拓扑空间\(X\)的一个子集\(Y\)是\(X\)的紧致子集,
意味着
\(Y\)的由子空间\(Y\)中的开集构成的每一个开覆盖都有有限子覆盖.
这并不明显地意味着
\(Y\)的由\(X\)中的开集构成的构成的每一个覆盖都有有限子覆盖.
\begin{theorem}
%@see: 《点集拓扑讲义(第四版)》(熊金城) P199 定理7.1.1
设\(X\)是一个拓扑空间,
\(Y\)是\(X\)的一个子集,
则\(Y\)是\(X\)的一个紧致子集,
当且仅当\(Y\)的由\(X\)中的开集构成的每一个覆盖都有有限子覆盖.
%TODO proof
\end{theorem}

下面介绍关于紧致性的一个等价说法.
\begin{definition}
%@see: 《点集拓扑讲义(第四版)》(熊金城) P199 定义7.1.3
设\(\sfA\)是一个集族.
如果\(\sfA\)的每一个有限子族都有非空的交,
即\begin{equation*}
	\text{$\sfA_1$是$\sfA$的有限子族}
	\implies
	\bigcap \sfA \neq \emptyset,
\end{equation*}
则称“\(\sfA\)是一个\DefineConcept{具有有限交性质的集族}”.
\end{definition}

\begin{theorem}
%@see: 《点集拓扑讲义(第四版)》(熊金城) P199 定理7.1.2
设\(X\)是一个拓扑空间,
则\(X\)是一个紧致空间,
当且仅当\(X\)中每一个具有有限交性质的闭集族都有非空的交.
%TODO proof
\end{theorem}

如果\(\B\)是紧致空间\(X\)的一个基,
那么由\(\B\)中的元素构成的\(X\)的一个覆盖当然是一个开覆盖,
因此\(\B\)有有限子覆盖.
下述定理指出,为了验证拓扑空间的紧致性,
只要验证由它的某一个基中的元素组成的覆盖有有限子覆盖.
\begin{theorem}
%@see: 《点集拓扑讲义(第四版)》(熊金城) P201 定理7.1.3
设\(\B\)是拓扑空间\(X\)的一个基,
并且\(X\)的由\(\B\)中的元素构成的每一个覆盖有一个有限子覆盖,
则\(X\)是一个紧致空间.
%TODO proof
\end{theorem}

\begin{theorem}
%@see: 《点集拓扑讲义(第四版)》(熊金城) P201 定理7.1.4
设\(X,Y\)都是拓扑空间,
\(f\colon X \to Y\)是一个连续映射.
如果\(A\)是\(X\)的一个紧致子集,
则\(f(A)\)是\(Y\)的一个紧致子集.
%TODO proof
\end{theorem}
%@see: 《点集拓扑讲义(第四版)》(熊金城) P202
上述定理说明,拓扑空间的紧致性是在连续映射下保持不变的性质,
因此它是拓扑不变性质,也是可商性质.

\begin{example}
%@see: 《点集拓扑讲义(第四版)》(熊金城) P202
证明:实数空间\(\mathbb{R}\)中的每一个开区间都不是紧致空间.
\begin{proof}
由于实数空间\(\mathbb{R}\)不是一个紧致空间,
而每一个开区间都是与之同配的,
所以每一个开区间作为子空间都不是紧致空间.
\end{proof}
\end{example}

\begin{theorem}
%@see: 《点集拓扑讲义(第四版)》(熊金城) P202 定理7.1.5
紧致空间中的每一个闭子集都是紧致子集.
%TODO proof
\end{theorem}

\begin{theorem}\label{theorem:紧致空间.拓扑空间的一点紧化}
%@see: 《点集拓扑讲义(第四版)》(熊金城) P202 定理7.1.6
任意一个拓扑空间必定是某个紧致空间的开子空间.
\begin{proof}
设\((X,\T)\)是一个拓扑空间,
任取一个不属于\(X\)的元素\(\infty\),
令\begin{align*}
	X^* &\defeq X \cup \{\infty\}, \\
	\T^* &\defeq \T \cup \T_1 \cup \{X^*\}, \\
	\T_1 &\defeq \Set{
		E \subseteq X^*
		\given
		\text{$X^* - E$是$X$中的一个紧致闭集}
	}.
\end{align*}

首先验证\(\T^*\)是\(X^*\)的一个拓扑.
\begin{itemize}
	\item 根据定义,立即有\(X^* \in \T^*\)和\(\emptyset \in \T \subseteq \T^*\).

	\item 任取\(A^*,B^* \in \T^*\).
	容易看出,要么\(A^*,B^*\)中至少有一个等于\(X^*\),
	要么\(A^*,B^*\)都不等于\(X^*\).
	假设\(A^*,B^*\)中至少有一个等于\(X^*\),
	则\(A^* \cap B^*\)等于\(A^* \cap X^* = B^*\)或\(X^* \cap B^* = A^*\),
	因而\(A^* \cap B^* \in \T^*\).
	假设\(A^*\)和\(B^*\)均不等于\(X^*\),则\(A^*,B^* \in \T \cup \T_1\).
	接下来进一步分为三种情况讨论.
	\begin{itemize}
		\item 如果\(A^*,B^* \in \T\),
		则有\begin{equation*}
			A^* \cap B^* \in \T \subseteq \T^*.
		\end{equation*}

		\item 如果\(A^*,B^* \in \T_1\),
		则\begin{equation*}
			X^* - (A^* \cap B^*)
			= (X^* - A^*) \cup (X^* - B^*)
		\end{equation*}
		作为\(X\)中的两个紧致闭集的并,也是一个紧致闭集,
		所以\(A^* \cap B^* \in \T_1 \subseteq \T^*\).

		\item 如果\(A^*,B^*\)既不同时属于\(\T\)也不同时属于\(\T_1\),
		不妨设\(A^* \in \T\)而\(B^* \in \T_1\),
		则\(X\)中的一个开集\(B\),
		使得\(B^* = B \cup \{\infty\}\),
		从而有\(A^* \cap B^*
		= A^* \cap B
		\in \T\).
	\end{itemize}
	总之,只要\(A^*,B^* \in \T^*\),便有\(A^* \cap B^* \in \T^*\).

	\item 取\(\T^*\)的一个子族\(\sfA\)使得\(\bigcup \sfA \notin \{\emptyset,X^*\}\).
	这时必有\(\sfA \neq \emptyset\)和\(X^* \notin \sfA\).
	\begin{itemize}
		\item 当\(\sfA \subseteq \T\)时,显然\(\bigcup \sfA \in \T \subseteq \T^*\).

		\item 当\(\sfA \subseteq \T_1\)时,则\begin{equation*}
			X^* - \bigcup \sfA = \bigcap_{A \in \sfA} (X^* - A)
		\end{equation*}
		是\(X\)中的一个闭集,
		并且对于任意\(A_0 \in \sfA\)有\begin{equation*}
			X^* - \bigcup \sfA \subseteq X^* - A_0.
		\end{equation*}
		因此\(x^* - \bigcup \sfA\)
		作为紧致空间\(X^* - A_0\)中的一个闭集也是紧致的,
		所以\begin{equation*}
			\bigcup \sfA \in \T_1 \subseteq \T^*.
		\end{equation*}

		\item 当\(\sfA\)既非\(\T\)的子集亦非\(\T_1\)的子集,则有\begin{equation*}
			\sfA_1 \defeq \sfA \cap \T \neq \emptyset,
			\qquad
			\sfA_2 \defeq \sfA \cap \T_1 \neq \emptyset.
		\end{equation*}
		令\begin{equation*}
			B_1 \defeq \bigcup \sfA_1,
			\qquad
			B_2 \defeq \bigcup \sfA_2,
		\end{equation*}
		则有\(\bigcup \sfA = B_1 \cup B_2\).
		此时必有\(B_1 \in \T\)和\(B_2 \in \T_1\),
		再由\(\infty \notin X^* - B_2\)
		有\begin{equation*}
			X^* - (B_1 \cup B_2)
			= (X^* - B_1) \cap (X^* - B_2)
			= (X - B_1) \cap (X^* - B_2),
		\end{equation*}
		它是紧致空间\(X^* - B_2\)的一个闭集,
		因此\begin{equation*}
			\bigcup \sfA
			= B_1 \cup B_2
			\in \T_1
			\subseteq \T^*.
		\end{equation*}
	\end{itemize}
	总之,只要\(\sfA \subseteq \T^*\),便有\(\bigcup \sfA  \in \T^*\).
\end{itemize}

\def\sfC{\mathscr{C}}
\def\wsfC{\tilde{\sfC}}
接下来验证\((X^*,\T^*)\)是一个紧致空间.
设\(\sfC\)是\(X^*\)的一个开覆盖,
则存在\(C \in \sfC\)使得\(\infty \in C\).
不妨设\(C \neq X^*\),
于是\(C \in \T_1\),
因此\(X^* - C\)是紧致的,
并且\(\sfC-\{C\}\)是它的一个开覆盖.
于是\(\sfC-\{C\}\)有一个有限子族\(\wsfC\)覆盖\(X^* - C\).
易见\(\wsfC \cup \{C\}\)是\(\sfC\)的一个有限子族并且覆盖\(X^*\).

最后,显然有\(\T = \T^* \TopoRestrict X\),且\(X\)是\(X^*\)中的一个开集.
这就说明\((X,\T)\)是\((X^*,\T^*)\)的一个开子空间.
\end{proof}
\end{theorem}
\begin{definition}
%@see: 《点集拓扑讲义(第四版)》(熊金城) P202 定理7.1.6
在\cref{theorem:紧致空间.拓扑空间的一点紧化} 的证明过程中,
由拓扑空间\((X,\T)\)构造出来的紧致空间\((X^*,\T^*)\),
称为“拓扑空间\((X,\T)\)的\DefineConcept{一点紧化}”.
%@see: 《点集拓扑讲义(第四版)》(熊金城) P157 例5.2.1
\end{definition}

由于非紧致空间是它的一点紧化的一个子空间,
所以紧致性不是可遗传性质.

以下定理表明紧致性是可积性质.
\begin{theorem}
%@see: 《点集拓扑讲义(第四版)》(熊金城) P204 定理7.1.7
设\(\AutoTuple{X}{n}\)是\(n\ (n\geq1)\)个紧致空间,
则积空间\(\AutoTuple{X}{n}[\times]\)也是一个紧致空间.
\end{theorem}
