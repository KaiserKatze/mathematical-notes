\section{局部紧致空间,仿紧致空间}
\begin{definition}
%@see: 《点集拓扑讲义(第四版)》(熊金城) P222 定义7.6.1
设\(X\)是一个拓扑空间.
如果\(X\)中的每一个点都有一个紧致的邻域,
则称“拓扑空间\(X\)是\DefineConcept{局部紧致的}”
“拓扑空间\(X\)具有\DefineConcept{局部紧致性}”
或“\(X\)是一个\DefineConcept{局部紧致空间}”.
\end{definition}

\begin{proposition}
%@see: 《点集拓扑讲义(第四版)》(熊金城) P222
任意一个紧致空间都是一个局部紧致空间.
\begin{proof}
设\(X\)是一个紧致空间,
显然\(X\)就是\(X\)的任意一个点的紧致邻域,
那么根据定义,\(X\)就是一个局部紧致空间.
\end{proof}
\end{proposition}

\begin{example}
%@see: 《点集拓扑讲义(第四版)》(熊金城) P222
证明:\(n\)维欧氏空间\(\mathbb{R}^n\)是局部紧致空间.
\begin{proof}
显然\(\mathbb{R}^n\)中任意一个球形邻域的闭包都是紧致的,
所以\(\mathbb{R}^n\)是局部紧致空间.
\end{proof}
\end{example}

\begin{theorem}
%@see: 《点集拓扑讲义(第四版)》(熊金城) P223 定理7.6.1
任意一个局部紧致的豪斯多夫空间都是一个正则空间.
%TODO proof
\end{theorem}

\begin{theorem}
%@see: 《点集拓扑讲义(第四版)》(熊金城) P223 定理7.6.2
设\(X\)是一个局部紧致的正则空间,点\(x \in X\),
则\(x\)的所有紧致邻域构成的集族是
拓扑空间\(X\)在点\(x\)的一个邻域基.
%TODO proof
\end{theorem}

\begin{corollary}
%@see: 《点集拓扑讲义(第四版)》(熊金城) P223 推论7.6.3
设\(X\)是一个局部紧致的豪斯多夫空间,点\(x \in X\),
则\(x\)的所有紧致邻域构成的集族是
拓扑空间\(X\)在点\(x\)的一个邻域基.
%TODO proof
\end{corollary}

\begin{theorem}
%@see: 《点集拓扑讲义(第四版)》(熊金城) P223 定理7.6.4
任意一个局部紧致的正则空间都是一个完全正则空间.
%TODO proof
\end{theorem}

\begin{definition}
%@see: 《点集拓扑讲义(第四版)》(熊金城) P224 定义7.6.2
设集族\(\sfA,\sfB\)都是集合\(X\)的覆盖.
如果\(\sfA\)中的每一个元素都包含于\(\sfB\)的某一个元素中,
则称“\(\sfA\)是\(\sfB\)的一个\DefineConcept{加细}”.
\end{definition}

\begin{proposition}
%@see: 《点集拓扑讲义(第四版)》(熊金城) P225
如果集族\(\sfA\)是集族\(\sfB\)的一个子覆盖,
则\(\sfA\)是\(\sfB\)的一个加细.
%TODO proof
\end{proposition}

\begin{definition}
%@see: 《点集拓扑讲义(第四版)》(熊金城) P225 定义7.6.3
设\(X\)是一个拓扑空间,
\(A\)是\(X\)的一个子集,
\(\sfA\)是\(A\)的一个覆盖.
如果对于任意一个\(x \in A\),
\(x\)有一个邻域\(U\)仅与\(\sfA\)中有限个元素有非空的交,
即\(\Set{
	A \in \sfA
	\given
	A \cap U \neq \emptyset
}\)是一个有限集,
则称“\(\sfA\)是集合\(A\)的一个\DefineConcept{局部有限覆盖}”.
\end{definition}

\begin{proposition}
%@see: 《点集拓扑讲义(第四版)》(熊金城) P225
任意一个有限覆盖都是一个局部有限覆盖.
\end{proposition}

\begin{example}
%@see: 《点集拓扑讲义(第四版)》(熊金城) P225
在实数空间\(\mathbb{R}\)中,
取\begin{equation*}
	\sfA \defeq \Set{
		(n-1,n+1)
		\given
		n\in\mathbb{Z}
	},
	\qquad
	\sfB \defeq \Set{
		(-n,n)
		\given
		n\in\mathbb{Z}
	},
\end{equation*}
则\(\sfA\)和\(\sfB\)都是\(\mathbb{R}\)的开覆盖,
并且\(\sfA\)是\(\sfB\)的一个加细,
而\(\sfB\)不是\(\sfA\)的加细,
并且\(\sfA\)是一个局部有限覆盖,
而\(\sfB\)却不是局部有限的.
\end{example}

\begin{definition}
%@see: 《点集拓扑讲义(第四版)》(熊金城) P225 定义7.6.4
设\(X\)是一个拓扑空间.
如果\(X\)的每一个开覆盖都有一个局部有限的开覆盖是它的加细,
则称“拓扑空间\(X\)是\DefineConcept{仿紧致的}”
“拓扑空间\(X\)具有\DefineConcept{仿紧致性}”
或“\(X\)是一个\DefineConcept{仿紧致空间}”.
\end{definition}

\begin{example}
%@see: 《点集拓扑讲义(第四版)》(熊金城) P225
证明:紧致空间是仿紧致的.
%TODO proof
\end{example}

\begin{example}
%@see: 《点集拓扑讲义(第四版)》(熊金城) P225
证明:离散空间是仿紧致的.
\begin{proof}
因为所有单点集构成的集族既是离散空间的一个开覆盖,
也是他的任何一个开覆盖的局部有限的加细.
\end{proof}
\end{example}

\begin{theorem}
%@see: 《点集拓扑讲义(第四版)》(熊金城) P225 定理7.6.5
任意一个仿紧致的正则空间都是一个正规空间.
%TODO proof
\end{theorem}

\begin{theorem}
%@see: 《点集拓扑讲义(第四版)》(熊金城) P226 定理7.6.6
任意一个仿紧致的豪斯多夫空间都既是一个正则空间,也是一个正规空间.
%TODO proof
\end{theorem}

\begin{lemma}
%@see: 《点集拓扑讲义(第四版)》(熊金城) P226 引理7.6.7
设\(X\)是一个满足第二可数性公理的局部紧致的豪斯多夫空间,
则\(X\)有一个开覆盖\(\{\AutoTuple{V}{0}\}\)满足
\(V_i\)的闭包\(\TopoClosureL{V_i}\)是包含于\(V_{i+1}\)的紧致子集.
%TODO proof
\end{lemma}

\begin{theorem}
%@see: 《点集拓扑讲义(第四版)》(熊金城) P228 定理7.6.8
任意一个满足第二可数性公理的局部紧致的豪斯多夫空间都是一个仿紧致空间.
%TODO proof
\end{theorem}

\begin{proposition}
%@see: 《点集拓扑讲义(第四版)》(熊金城) P229
设\(X\)是一个局部紧致的豪斯多夫空间,
\(X\)还是一个林德洛夫空间,
则\(X\)有一个开覆盖\(\{\AutoTuple{V}{0}\}\)满足
\(V_i\)的闭包\(\TopoClosureL{V_i}\)是包含于\(V_{i+1}\)的紧致子集.
%TODO proof
\end{proposition}
