\section{压缩映射原理}
%\cref{example:收敛准则.压缩映射原理1,example:收敛准则.压缩映射原理2}
%@see: https://www.math.cuhk.edu.hk/course_builder/1819/math3060/

\begin{definition}
%@see: 《数学分析(第7版 第二卷)》(卓里奇) P29 定义1
设映射\(f\colon X \to X\).
如果点\(a \in X\)满足\(f(a) = a\),
则称“点\(a\)是\(f\)的一个\DefineConcept{不动点}(fixed point)”.
\end{definition}

\begin{definition}
%@see: 《数学分析(第7版 第二卷)》(卓里奇) P29 定义2
设\((X,\rho)\)是一个度量空间,
映射\(f\colon X \to X\).
如果\begin{equation*}
	(\exists q\in\mathbb{R})
	(\forall x_1,x_2 \in X)
	[
		0 < q < 1
		\implies
		\rho(f(x_1),f(x_2))
		\leq
		q \cdot \rho(x_1,x_2)
	],
\end{equation*}
则称“\(f\)是\(X\)上的\DefineConcept{压缩映射}(contraction mapping)”.
\end{definition}

\begin{theorem}[皮卡--巴拿赫不动点原理]
%@see: 《数学分析(第7版 第二卷)》(卓里奇) P29 定理(皮卡--巴拿赫不动点原理)
设\((X,\rho)\)是一个完备度量空间,
则\(X\)上的压缩映射具有唯一的不动点.
%@see: https://wuli.wiki/online/ConMap.html
\end{theorem}

\begin{definition}
设\((X,\rho)\)是一个度量空间.
把集合\begin{equation*}
	\Set{ f \in X^X \given \text{$f$是$X$上的压缩映射} }
\end{equation*}称为“\(X\)上的\DefineConcept{压缩映射空间}”.
\end{definition}

\begin{definition}
%@see: 《数学分析(第7版 第二卷)》(卓里奇) P30 命题(关于不动点的稳定性)
设\((X,\rho)\)是一个度量空间,
\((\Omega,\T)\)是拓扑空间,
\(\{f_t\}_{t\in\Omega}\)是一个映射族.
如果\begin{equation*}
	(\exists q\in\mathbb{R})
	(\forall t\in\Omega)
	(\forall x_1,x_2 \in X)
	[
		0 < q < 1
		\implies
		\rho(f_t(x_1),f_t(x_2))
		\leq
		q \cdot \rho(x_1,x_2)
	],
\end{equation*}
则称“映射族\(\{f_t\}_{t\in\Omega}\)是\DefineConcept{一致压缩的}”.
\end{definition}

\begin{proposition}[关于不动点的稳定性]
%@see: 《数学分析(第7版 第二卷)》(卓里奇) P30 命题(关于不动点的稳定性)
设\((X,\rho)\)是一个完备度量空间,
\((\Omega,\T)\)是拓扑空间,
\(\tilde{X}\)是\(X\)上的压缩映射空间,
映射\(f\colon \Omega \to \tilde{X}\),
且\begin{itemize}
	\item 映射族\(\{f_t\}_{t\in\Omega}\)是一致压缩的,
	\item 对\(\forall x \in X\),
	映射\(g_x\colon \Omega \to X, t \mapsto f_t(x)\)在某个点\(t_0 \in \Omega\)连续,
	即\(\lim_{t \to t_0} g_x(t) = g_x(t_0)\),
\end{itemize}
那么方程\(x = f_t(x)\)的解\(a(t) \in X\)在点\(t_0\)连续地依赖于\(t\),
即\(\lim_{t \to t_0} a(t) = a(t_0)\).
\end{proposition}
