\section{积空间(有限情形)}
\subsection{度量积空间}
给定两个拓扑空间,我们首先可以得到一个集合作为它们的笛卡尔积.
我们想要知道:如何按照某种自然的方式给定这个笛卡尔积一个拓扑,使之成为拓扑空间?

为此,我们先对度量空间中的同类问题进行研究.
首先回顾\cref{example:度量空间.欧氏空间} 中
\(n\)维欧氏空间\(\mathbb{R}^n\)中的度量是如何通过实数空间中的度量来定义的.
将欧氏空间的通常度量推广到有限个度量空间的笛卡尔积,不会产生任何困难.
\begin{definition}\label{definition:有限情形下的积空间.度量积空间}
%@see: 《点集拓扑讲义(第四版)》(熊金城) P104 定义3.2.1
\def\MatricSpaceCartesianProduct{(X_1,\rho_1),\allowbreak(X_2,\rho_2),\allowbreak\dotsc,\allowbreak(X_n,\rho_n)}
设\(\MatricSpaceCartesianProduct\)是\(n\ (n\geq1)\)个度量空间.
令\begin{gather*}
	X \defeq X_1 \times X_2 \times \dotsb \times X_n, \\
	\rho\colon X \times X \to \mathbb{R},
	(x,y) \mapsto \sqrt{\sum_{i=1}^n \rho_i(x_i,y_i)^2},
	\quad x=(\AutoTuple{x}{n}),y=(\AutoTuple{y}{n}) \in X.
\end{gather*}
把\(\rho\)称为“笛卡尔积\(X = X_1 \times X_2 \times \dotsb \times X_n\)的\DefineConcept{积度量}”.
把\((X,\rho)\)称为“度量空间\(\MatricSpaceCartesianProduct\)的\DefineConcept{度量积空间}”,
在不致混淆的情况下简称为\DefineConcept{积空间}.
\end{definition}

\begin{proposition}
\cref{definition:有限情形下的积空间.度量积空间} 中的积度量\(\rho\)是\(X\)的一个度量.
%TODO proof 需要用到施瓦茨引理
\end{proposition}

根据上述定义可知,\(n\)维欧氏空间\(\mathbb{R}^n\)就是\(n\)个实数空间\(\mathbb{R}\)的度量积空间.

\subsection{由积度量诱导出来的拓扑的性质}
现在来考察积度量所诱导出来的拓扑具有什么样的性质,
以便我们得到在拓扑空间中应该如何引出积空间中的概念的启示.
