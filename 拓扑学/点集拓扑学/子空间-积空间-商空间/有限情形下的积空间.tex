\section{积空间(有限情形)}
\subsection{度量积空间}
给定两个拓扑空间,我们首先可以得到一个集合作为它们的笛卡尔积.
我们想要知道:如何按照某种自然的方式给定这个笛卡尔积一个拓扑,使之成为拓扑空间?

为此,我们先对度量空间中的同类问题进行研究.
首先回顾\cref{example:度量空间.欧氏空间} 中
\(n\)维欧氏空间\(\mathbb{R}^n\)中的度量是如何通过实数空间中的度量来定义的.
将欧氏空间的通常度量推广到有限个度量空间的笛卡尔积,不会产生任何困难.
\begin{definition}\label{definition:有限情形下的积空间.度量积空间}
%@see: 《点集拓扑讲义(第四版)》(熊金城) P104 定义3.2.1
\def\MatricSpaceCartesianProduct{(X_1,\rho_1),\allowbreak(X_2,\rho_2),\allowbreak\dotsc,\allowbreak(X_n,\rho_n)}
设\(\MatricSpaceCartesianProduct\)是\(n\ (n\geq1)\)个度量空间.
令\begin{gather*}
	X \defeq X_1 \times X_2 \times \dotsb \times X_n, \\
	\rho\colon X \times X \to \mathbb{R},
	(x,y) \mapsto \sqrt{\sum_{i=1}^n \rho_i(x_i,y_i)^2},
	\quad x=(\AutoTuple{x}{n}),y=(\AutoTuple{y}{n}) \in X.
\end{gather*}
把\(\rho\)称为“笛卡尔积\(X = X_1 \times X_2 \times \dotsb \times X_n\)的\DefineConcept{积度量}”.
把\((X,\rho)\)称为“度量空间\(\MatricSpaceCartesianProduct\)的\DefineConcept{度量积空间}”,
在不致混淆的情况下简称为\DefineConcept{积空间}.
\end{definition}

\begin{proposition}
\cref{definition:有限情形下的积空间.度量积空间} 中的积度量\(\rho\)是\(X\)的一个度量.
%TODO proof 需要用到施瓦茨引理
\end{proposition}

根据上述定义可知,\(n\)维欧氏空间\(\mathbb{R}^n\)就是\(n\)个实数空间\(\mathbb{R}\)的度量积空间.

\subsection{由积度量诱导出来的拓扑的性质}
现在来考察积度量所诱导出来的拓扑具有什么样的性质,
以便我们得到在拓扑空间中应该如何引出积空间中的概念的启示.
\begin{theorem}\label{theorem:有限情形下的积空间.由积度量诱导出来的拓扑的基}
%@see: 《点集拓扑讲义(第四版)》(熊金城) P104 定理3.2.1
\def\MatricSpaceCartesianProduct{(X_1,\rho_1),\allowbreak(X_2,\rho_2),\allowbreak\dotsc,\allowbreak(X_n,\rho_n)}
设\((X,\rho)\)是\(n\ (n\geq1)\)个度量空间\(\MatricSpaceCartesianProduct\)的度量积空间,
\(\T_i\ (i=1,2,\dotsc,n)\)是由\(\rho_i\)诱导出来的拓扑,
\(\T\)是由\(\rho\)诱导出来的拓扑,
则\(X\)的子集族\begin{equation*}
	\B \defeq \Set{
		U_1 \times U_2 \times \dotsb \times U_n
		\given
		U_i \in \T_i, i=1,2,\dotsc,n
	}
\end{equation*}
是\(X\)的拓扑\(\T\)的一个基.
%TODO proof
\end{theorem}

\subsection{拓扑积空间}
受到\cref{theorem:有限情形下的积空间.由积度量诱导出来的拓扑的基} 的启示,
我们因而引入有限个拓扑空间的积空间这一概念:
\begin{theorem}\label{theorem:有限情形下的积空间.积拓扑存在且唯一}
%@see: 《点集拓扑讲义(第四版)》(熊金城) P105 定理3.2.2
\def\MatricSpaceCartesianProduct{(X_1,\T_1),\allowbreak(X_2,\T_2),\allowbreak\dotsc,\allowbreak(X_n,\T_n)}
设\(\MatricSpaceCartesianProduct\)是\(n\ (n\geq1)\)个拓扑空间.
令\begin{gather*}
	X \defeq X_1 \times X_2 \times \dotsb \times X_n,
\end{gather*}
则\(X\)有唯一一个拓扑\(\T\)
以\(X\)的子集族\begin{equation*}
	\B \defeq \Set{
		U_1 \times U_2 \times \dotsb \times U_n
		\given
		U_i \in \T_i, i=1,2,\dotsc,n
	}
\end{equation*}
为它的一个基.
%TODO proof
\end{theorem}

\begin{definition}\label{definition:有限情形下的积空间.拓扑积空间}
%@see: 《点集拓扑讲义(第四版)》(熊金城) P106 定义3.2.2
\def\MatricSpaceCartesianProduct{(X_1,\T_1),\allowbreak(X_2,\T_2),\allowbreak\dotsc,\allowbreak(X_n,\T_n)}
设\(\MatricSpaceCartesianProduct\)是\(n\ (n\geq1)\)个拓扑空间.
令\begin{gather*}
	X \defeq X_1 \times X_2 \times \dotsb \times X_n.
\end{gather*}
把\cref{theorem:有限情形下的积空间.积拓扑存在且唯一} 中指出的拓扑\(\T\)
称为“笛卡尔积\(X = X_1 \times X_2 \times \dotsb \times X_n\)的\DefineConcept{积拓扑}”.
把\((X,\T)\)称为“拓扑空间\(\MatricSpaceCartesianProduct\)的\DefineConcept{拓扑积空间}”,
在不致混淆的情况下简称为\DefineConcept{积空间}.
\end{definition}

设\(\AutoTuple{X}{n}\)是\(n\ (n\geq1)\)个度量空间,
则笛卡尔积\(X = X_1 \times X_2 \times \dotsb \times X_n\)
可以由两种方式得到它的拓扑:
一种方式是先是求出\(X\)的积度量,再用这个积度量诱导出\(X\)的拓扑;
另一种方式是先用\(X_i\)的度量诱导出\(X_i\)的拓扑,再求出\(X\)的积拓扑.
\cref{theorem:有限情形下的积空间.由积度量诱导出来的拓扑的基} 实际上指明这两种拓扑是一致的,即
\begin{theorem}
%@see: 《点集拓扑讲义(第四版)》(熊金城) P106 定理3.2.3
设\(X\)是\(n\ (n\geq1)\)个度量空间\(\AutoTuple{X}{n}\)的度量积空间,
当把各个\(X_i\)以及\(X\)都视作拓扑空间时,
\(X\)就是\(\AutoTuple{X}{n}\)的拓扑积空间.
\end{theorem}

显然,\(n\)维欧氏空间\(\mathbb{R}^n\)就是\(n\)个实数空间\(\mathbb{R}\)的拓扑积空间.

\begin{theorem}\label{theorem:有限情形下的积空间.拓扑积空间的基}
%@see: 《点集拓扑讲义(第四版)》(熊金城) P106 定理3.2.4
设\(X\)是\(n\ (n\geq1)\)个拓扑空间\(\AutoTuple{X}{n}\)的拓扑积空间,
\(\B_i\)是\(X_i\)的一个基,
则\(X\)的子集族\begin{equation*}
	\tilde{\B} \defeq \Set{
		B_1 \times B_2 \times \dotsb \times B_n
		\given
		B_i \in \B_i, i=1,2,\dotsc,n
	}
\end{equation*}
是拓扑空间\(X\)的一个基.
%TODO proof
\end{theorem}

\begin{example}%\label{example:拓扑基.全体开方体是n维欧氏空间的一个基}
%@see: 《点集拓扑讲义(第四版)》(熊金城) P107 例3.2.1
证明:\(n\)维欧氏空间\(\mathbb{R}^n\)中所有开方体\begin{equation*}
	\tilde{\B} \defeq \Set{
		B_1 \times B_2 \times \dotsb \times B_n
		\given
		\text{$B_i = (a_i,b_i)$是开区间}, i=1,2,\dotsc,n
	}
\end{equation*}
构成\(\mathbb{R}^n\)的一个基.
\begin{proof}
由\cref{example:拓扑基.全体开区间是实数空间的一个基,theorem:有限情形下的积空间.拓扑积空间的基} 立即可得.
\end{proof}
\end{example}

\begin{theorem}\label{theorem:有限情形下的积空间.由各坐标空间中的开集在投射的逆下的像组成的子基}
%@see: 《点集拓扑讲义(第四版)》(熊金城) P107 定理3.2.5
\def\MatricSpaceCartesianProduct{(X_1,\T_1),\allowbreak(X_2,\T_2),\allowbreak\dotsc,\allowbreak(X_n,\T_n)}
设\((X,\T)\)是\(n\ (n\geq1)\)个拓扑空间\(\MatricSpaceCartesianProduct\)的拓扑积空间,
映射\(p_i\)是从\(X\)到\(X_i\)的投射,
则\(X\)的子集族\begin{equation*}
	\S \defeq \Set{
		p_i^{-1}(U_i)
		\given
		U_i \in \T_i, i=1,2,\dotsc,n
	}
\end{equation*}
是\(X\)的一个子基.
%TODO proof
\end{theorem}

\subsection{开映射,闭映射}
\begin{definition}
%@see: 《点集拓扑讲义(第四版)》(熊金城) P108 定义3.2.3
设\(X,Y\)都是拓扑空间,
映射\(f\colon X \to Y\).
\begin{itemize}
	\item 如果对于\(X\)中的任意一个开集\(U\),
	\(f(U)\)是\(Y\)中的开集,
	则称“\(f\)是一个\DefineConcept{开映射}”.
	\item 如果对于\(X\)中的任意一个闭集\(U\),
	\(f(U)\)是\(Y\)中的闭集,
	则称“\(f\)是一个\DefineConcept{闭映射}”.
\end{itemize}
\end{definition}

\begin{theorem}\label{theorem:有限情形下的积空间.投射是开映射}
%@see: 《点集拓扑讲义(第四版)》(熊金城) P108 定理3.2.6
设\(X\)是\(n\ (n\geq1)\)个拓扑空间\(\AutoTuple{X}{n}\)的拓扑积空间,
则从\(X\)到\(X_i\)的投射\(p_i\)是满的连续开映射.
%TODO proof
%\cref{theorem:一般情形下的积空间.投射是开映射}
\end{theorem}

\begin{theorem}\label{theorem:有限情形下的积空间.投射与映射的复合的连续性}
%@see: 《点集拓扑讲义(第四版)》(熊金城) P108 定理3.2.7
设\(X\)是\(n\ (n\geq1)\)个拓扑空间\(\AutoTuple{X}{n}\)的拓扑积空间,
\(Y\)是一个拓扑空间,
\(p_i\)是从\(X\)到\(X_i\)的投射,
则映射\(f\colon Y \to X\)连续,
当且仅当
复合映射\(p_i \circ f\ (i=1,2,\dotsc,n)\)都连续.
%TODO proof
%\cref{theorem:一般情形下的积空间.投射与映射的复合的连续性}
\end{theorem}

\begin{theorem}\label{theorem:有限情形下的积空间.积拓扑是使所有投射都连续的最小拓扑}
%@see: 《点集拓扑讲义(第四版)》(熊金城) P109 定理3.2.8
设\(X\)是\(n\ (n\geq1)\)个拓扑空间\(\AutoTuple{X}{n}\)的拓扑积空间,
\(\T\)是\(X\)的积拓扑.
如果对于\(X\)的一个拓扑\(\wT\)而言,
当\(i=1,2,\dotsc,n\)时,
从\(X\)到\(X_i\)的投射\(p_i\)总是连续映射,
那么\(\wT \supseteq \T\).
%TODO proof
%\cref{theorem:一般情形下的积空间.积拓扑是使所有投射都连续的最小拓扑}
\end{theorem}

\begin{theorem}
%@see: 《点集拓扑讲义(第四版)》(熊金城) P109 定理3.2.9
设\(\AutoTuple{X}{n}\)是\(n\ (n\geq2)\)个拓扑空间,
则拓扑积空间\(X_1 \times X_2 \times \dotsb \times X_{n-1} \times X_n\)
同胚于拓扑积空间\((X_1 \times X_2 \times \dotsb \times X_{n-1}) \times X_n\).
%TODO proof
\end{theorem}
上述定理说明:
尽管\(X_1 \times X_2 \times \dotsb \times X_{n-1} \times X_n\)
和\((X_1 \times X_2 \times \dotsb \times X_{n-1}) \times X_n\)
作为集合可以是完全不同的,但是,在同胚的意义下,两者却别无二致.
上述定理还说明:
在同胚的意义下,有限个拓扑空间的积空间可以通过归纳的方式予以定义.
