\section{商空间}
将一条橡皮筋的两个端点“粘合”起来,便可得到一个橡皮圈.
将一块正方形橡皮的一对边“粘合”起来,便可得到一根橡皮管,
继续将橡皮管两端“粘合”起来,便可得到一个橡皮轮胎.
像这样,从一个给定的图形出发,构造出一个新图形的办法可以一般化.

\begin{proposition}
%@see: 《点集拓扑讲义(第四版)》(熊金城) P116 定义3.4.1
设\((X,\T)\)是一个拓扑空间,
\(Y\)是一个集合,
\(f\colon X \to Y\)是一个满射,
则\(Y\)的子集族\begin{equation*}
	\Set{
		U \subseteq Y
		\given
		f^{-1}(U) \in \T
	}
\end{equation*}
是\(Y\)的一个拓扑.
\end{proposition}

\begin{definition}
%@see: 《点集拓扑讲义(第四版)》(熊金城) P116 定义3.4.1
设\((X,\T)\)是一个拓扑空间,
\(Y\)是一个集合,
\(f\colon X \to Y\)是一个满射.
把\begin{equation*}
	\Set{
		U \subseteq Y
		\given
		f^{-1}(U) \in \T
	}
\end{equation*}
称为“\(Y\)的(相对于满射\(f\)而言的)\DefineConcept{商拓扑}”.
\end{definition}
