\section{分离性公理,子空间、积空间和商空间}
之前讨论的分离性公理有
\(T_0\)、\(T_1\)、\(T_2\)、\(T_3\)、\(T_{3.5}\)、\(T_4\)以及正则和正规等,
它们都是经由开集或者经由通过开集定义的概念来陈述的,
所以它们必然都会是拓扑不变性质.

\begin{theorem}
%@see: 《点集拓扑讲义(第四版)》(熊金城) P186 定理6.5.1
设\(X,Y\)是同胚的两个拓扑空间.
如果\(X\)是一个完全正则空间,
则\(Y\)也是一个完全正则空间.
%TODO proof
\end{theorem}

\(T_0\)、\(T_1\)、\(T_2\)、\(T_3\)、\(T_{3.5}\)以及完全正则和正则等,都是可遗传性质.

\begin{theorem}
%@see: 《点集拓扑讲义(第四版)》(熊金城) P187 定理6.5.2
正则空间的每一个子空间都是正则空间.
%TODO proof
\end{theorem}

\(T_0\)、\(T_1\)、\(T_2\)、\(T_3\)、\(T_{3.5}\)以及完全正则和正则等,都是可积性质.
正规和\(T_4\)不是有限可积性质.

\begin{theorem}
%@see: 《点集拓扑讲义(第四版)》(熊金城) P187 定理6.5.3
任意一族豪斯多夫空间的积空间都是豪斯多夫空间.
%TODO proof
\end{theorem}

\begin{lemma}
%@see: 《点集拓扑讲义(第四版)》(熊金城) P187 引理6.5.4
任意一族豪斯多夫空间的积空间都是豪斯多夫空间.
%TODO proof
\end{lemma}

\begin{theorem}
%@see: 《点集拓扑讲义(第四版)》(熊金城) P188 定理6.5.5
设\(\AutoTuple{X}{n}\)是\(n\ (n\geq1)\)个完全正则空间,
则积空间\(\AutoTuple{X}{n}[\times]\)也是完全正则空间.
%TODO proof
\end{theorem}

\begin{theorem}
%@see: 《点集拓扑讲义(第四版)》(熊金城) P189 定理6.5.6
任意一族完全正则空间的积空间都是完全正则空间.
%TODO proof
\end{theorem}

所有分离性公理都不是可商性质.
