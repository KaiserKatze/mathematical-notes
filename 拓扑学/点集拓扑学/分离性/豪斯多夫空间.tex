\section{\texorpdfstring{\(T_0\)}{T0}空间,\texorpdfstring{\(T_1\)}{T1}空间,豪斯多夫空间}
现在让我们回到之前提出的一个问题:
什么样的拓扑空间的拓扑可以由它的某一个度量诱导出来?
回答这个问题势必要求我们对度量空间的拓扑性质有充分的了解.
读者将会发现,本章中所提到的各个分离性公理,
实际上是模仿度量空间的拓扑性质逐步建立起来的.
对各个分离性的充分研究将使我们能够在本章为上述问题做一个比较深刻的(虽然不是完全的)回答.

\begin{definition}
%@see: 《点集拓扑讲义(第四版)》(熊金城) P166 定义6.1.1
设\(X\)是一个拓扑空间.
如果\(X\)中任意两个不同的点中必有一个点有一个开邻域不含另一个点,
则称“\(X\)是一个 \DefineConcept{\(T_0\)空间}”.
\end{definition}

\begin{example}
%@see: 《点集拓扑讲义(第四版)》(熊金城) P166
证明:含有不少于两个点的平庸空间不是\(T_0\)空间.
%TODO proof
\end{example}

\begin{theorem}
%@see: 《点集拓扑讲义(第四版)》(熊金城) P166 定理6.1.1
拓扑空间\(X\)是\(T_0\)空间,
当且仅当\(X\)中任意两个不同的单点集有不同的闭包.
%TODO proof
\end{theorem}

\begin{definition}
%@see: 《点集拓扑讲义(第四版)》(熊金城) P167 定义6.1.2
设\(X\)是一个拓扑空间.
如果\(X\)中任意两个不同的点中每一个点都有一个开邻域不含另一个点,
则称“\(X\)是一个 \DefineConcept{\(T_1\)空间}”.
\end{definition}

\begin{proposition}
%@see: 《点集拓扑讲义(第四版)》(熊金城) P167
\(T_1\)空间一定是\(T_0\)空间,反之不然.
%TODO proof
\end{proposition}

\begin{theorem}
%@see: 《点集拓扑讲义(第四版)》(熊金城) P167 定理6.1.2
设\(X\)是一个拓扑空间,
则以下命题等价:\begin{itemize}
	\item \(X\)是一个\(T_1\)空间;
	\item \(X\)中每一个单点集都是闭集;
	\item \(X\)中每一个有限子集都是闭集.
\end{itemize}
%TODO proof
\end{theorem}

\begin{theorem}
%@see: 《点集拓扑讲义(第四版)》(熊金城) P168 定理6.1.3
设\(X\)是一个\(T_1\)空间,
则点\(x \in X\)是\(X\)的子集\(A\)的一个聚点,
当且仅当\(x\)的每一个邻域中都含有\(A\)中的无限多个点.
%TODO proof
\end{theorem}

\begin{theorem}
%@see: 《点集拓扑讲义(第四版)》(熊金城) P168 定理6.1.4
设\(X\)是一个\(T_1\)空间,
则\(X\)中的一个由有限个点构成的序列\(\{x_n\}\)收敛于点\(x \in X\),
当且仅当存在正整数\(N\),使得当\(n>N\)时,成立\(x_n=x\).
%TODO proof
\end{theorem}

\begin{definition}
%@see: 《点集拓扑讲义(第四版)》(熊金城) P169 定义6.1.3
设\(X\)是一个拓扑空间.
如果\(X\)中任意两个不同的点各自有一个开邻域使得这两个开邻域互斥,
则称“\(X\)是一个 \DefineConcept{\(T_2\)空间}”
或“\(X\)是一个\DefineConcept{豪斯多夫空间}”.
\end{definition}

\begin{proposition}
%@see: 《点集拓扑讲义(第四版)》(熊金城) P169
\(T_2\)空间一定是\(T_1\)空间,反之不然.
\end{proposition}

\begin{theorem}
%@see: 《点集拓扑讲义(第四版)》(熊金城) P169 定理6.1.5
豪斯多夫空间中任意一个收敛序列只有一个聚点.
%TODO proof
\end{theorem}

%@see: 《点集拓扑讲义(第四版)》(熊金城) P169
在\(T_1\)空间中,可能存在一个收敛序列同时有多个聚点.
