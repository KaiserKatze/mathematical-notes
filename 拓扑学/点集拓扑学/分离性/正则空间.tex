\section{正则空间,正规空间,\texorpdfstring{\(T_3\)}{T3}空间,\texorpdfstring{\(T_4\)}{T4}空间}
\begin{definition}
%@see: 《点集拓扑讲义(第四版)》(熊金城) P171 定义6.2.1
设\(X\)是一个拓扑空间,
且\(A\)是\(X\)的子集.
如果集合\(Y\)包含于\(A\)的内部,
则称“\(A\)是集合\(Y\)的一个\DefineConcept{邻域}(neighborhood)”.
\end{definition}
\begin{definition}
%@see: 《点集拓扑讲义(第四版)》(熊金城) P171 定义6.2.1
设\(X\)是一个拓扑空间,
\(A\)是\(X\)的子集,
\(A\)是集合\(Y\)的一个邻域.
\begin{itemize}
	\item 如果\(A\)是一个开集,
	则称“\(A\)是集合\(Y\)的一个\DefineConcept{开邻域}(open neighborhood)”.
	\item 如果\(A\)是一个闭集,
	则称“\(A\)是集合\(Y\)的一个\DefineConcept{闭邻域}(open neighborhood)”.
\end{itemize}
\end{definition}

\begin{definition}
%@see: 《点集拓扑讲义(第四版)》(熊金城) P171 定义6.2.2
设\(X\)是一个拓扑空间.
如果\(X\)中任意一个点\(x\)和任意一个不含点\(x\)的闭集都各有一个开邻域,且两者互斥,
则称“拓扑空间\(X\)是\DefineConcept{正则的}”
“拓扑空间\(X\)具有\DefineConcept{正则性}”
或“\(X\)是一个\DefineConcept{正则空间}”.
\end{definition}

\begin{theorem}
%@see: 《点集拓扑讲义(第四版)》(熊金城) P171 定理6.2.1
设\(X\)是一个拓扑空间,
则\(X\)是一个正则空间,
当且仅当对于任意一点\(x \in X\)和\(x\)的任意一个开邻域\(U\),
存在\(x\)的一个开邻域\(V\),
使得\(V\)的闭包是\(U\)的一个子集.
%TODO proof
\end{theorem}

\begin{definition}
%@see: 《点集拓扑讲义(第四版)》(熊金城) P172 定义6.2.3
设\(X\)是一个拓扑空间.
如果\(X\)中任意两个互斥的闭集各有一个开邻域,并且这两个邻域也互斥,
则称“拓扑空间\(X\)是\DefineConcept{正规的}”
“拓扑空间\(X\)具有\DefineConcept{正规性}”
或“\(X\)是一个\DefineConcept{正规空间}”.
\end{definition}

\begin{theorem}
%@see: 《点集拓扑讲义(第四版)》(熊金城) P172 定理6.2.2
设\(X\)是一个拓扑空间,
则\(X\)是一个正规空间,
当且仅当对于任意一个闭集\(A \subseteq X\)和\(A\)的任意一个开邻域\(U\),
存在\(A\)的一个开邻域\(V\),
使得\(V\)的闭包是\(U\)的一个子集.
%TODO proof
\end{theorem}

%@see: 《点集拓扑讲义(第四版)》(熊金城) P172
“拓扑空间是否正则空间或正规空间”
与“拓扑空间是否\(T_0\)空间、\(T_1\)空间或\(T_2\)空间”
之间并无必然的蕴含关系.

\begin{example}
%@see: 《点集拓扑讲义(第四版)》(熊金城) P172 例6.2.1
举例说明:尽管拓扑空间\(X\)既是正则空间也是正规空间,但\(X\)不是\(T_0\)空间.
%TODO
\end{example}

\begin{example}
%@see: 《点集拓扑讲义(第四版)》(熊金城) P172 例6.2.2
举例说明:尽管拓扑空间\(X\)是豪斯多夫空间,但\(X\)既非正则空间亦非正规空间.
%TODO
\end{example}

%@see: 《点集拓扑讲义(第四版)》(熊金城) P174
“拓扑空间是否正则空间”
与“拓扑空间是否正规空间”
之间也没有必然的蕴含关系.

\begin{example}
%@see: 《点集拓扑讲义(第四版)》(熊金城) P174 例6.2.3
举例说明:尽管拓扑空间\(X\)是正规空间,但\(X\)不是正则空间.
%TODO
\end{example}

\begin{definition}
%@see: 《点集拓扑讲义(第四版)》(熊金城) P174 定义6.2.4
正则的\(T_1\)空间称为 \DefineConcept{\(T_3\)空间}.
\end{definition}

\begin{definition}
%@see: 《点集拓扑讲义(第四版)》(熊金城) P174 定义6.2.4
正规的\(T_1\)空间称为 \DefineConcept{\(T_4\)空间}.
\end{definition}

由于\(T_1\)空间中每一个单点集都是闭集,
因此\(T_4\)空间一定是\(T_3\)空间,
\(T_3\)空间一定是豪斯多夫空间.

\begin{example}
%@see: 《点集拓扑讲义(第四版)》(熊金城) P174
举例说明:尽管拓扑空间\(X\)是\(T_3\)空间,但\(X\)不是\(T_4\)空间.
%TODO
\end{example}

\begin{theorem}
%@see: 《点集拓扑讲义(第四版)》(熊金城) P174 定理6.2.3
每一个度量空间都是\(T_4\)空间.
%TODO proof
\end{theorem}
