\section{点式收敛拓扑}
%@see: 《点集拓扑讲义(第四版)》(熊金城) P249
设\(X\)是一个集合,
\(Y\)是一个拓扑空间.
把从\(X\)到\(Y\)的映射空间\(Y^X\)
可以看作拓扑空间族\(\{Y_x\}_{x \in X}\)的笛卡尔积
(其中每一个\(Y_x\)都等于\(Y\)),
对于任意\(x \in X\),
把从\(Y^X\)到坐标集\(Y_x\)的投射
称为“\(Y^X\)在点\(x\)的\DefineConcept{赋值映射}”,
记为\(e_x\).
显然,对于任意映射\(f \in Y^X\),有\(e_x(f) = f(x)\).
%\cref{theorem:一般情形下的积空间.由各坐标空间中的开集在投射的逆下的像组成的子基}
我们已经知道,\(Y^X\)的积拓扑\(\T\)是以\(
	\S = \Set{
		e_x^{-1}(U)
		\given
		\text{$U$是$Y$的一个开集},
		x \in X
	}
\)为子基的拓扑.
我们把\(Y^X\)的积拓扑\(\T\)
称为它的\DefineConcept{点式收敛拓扑},
将拓扑空间\((Y^X,\T)\)
称为“从\(X\)到\(Y\)的\DefineConcept{带有点式收敛拓扑的映射空间}”.

由于带有点式收敛拓扑的映射空间本身便是一类特别的积空间,
因此关于积空间的一般结论全部适用于它,无需另行证明,
我们简要归纳如下:\begin{itemize}
	\item 带有点式收敛拓扑的映射空间\(Y^X\)满足性质\(\phi\),
	当且仅当
		拓扑空间\(Y\)满足性质\(\phi\),
	其中性质\(\phi\)可以是\(T_0\)、\(T_1\)、豪斯多夫、\(T_3\)、提赫诺夫、正则、完全正则、紧致.

	\item 带有点式收敛拓扑的映射空间\(Y^X\)满足性质\(\phi\),
	当且仅当
		要么\(Y\)是一个平庸空间,
		要么\(X\)是一个可数集并且\(Y\)也满足性质\(\phi\),
	其中性质\(\phi\)可以是第一可数性公理、第二可数性公理.

	\item 带有点式收敛拓扑的映射空间\(Y^X\)中的
		序列\(\{f_n\}_{n \in \mathbb{N}}\)
		收敛于映射\(f \in Y^X\),
	当且仅当
		对于任意一个\(x \in X\),
		拓扑空间\(Y\)中的序列\(\{f_n(x)\}_{n \in \mathbb{N}}\)收敛于\(f(x)\).
\end{itemize}

\begin{definition}
%@see: 《点集拓扑讲义(第四版)》(熊金城) P250
设\(X\)和\(Y\)是两个拓扑空间.
把集合\(
	\Set{
		f \in Y^X
		\given
		\text{$f$是连续的}
	}
\)称为“从拓扑空间\(X\)到拓扑空间\(Y\)的\DefineConcept{连续映射空间}(continuous function space)”,
记作\(\ContinuousFunctionSpace(X,Y)\).
\end{definition}

\begin{definition}
%@see: 《点集拓扑讲义(第四版)》(熊金城) P250 定义9.1.1
设\(X\)和\(Y\)是两个拓扑空间,
\(\ContinuousFunctionSpace(X,Y)\)是从\(X\)到\(Y\)的连续映射空间.
如果把\(\ContinuousFunctionSpace(X,Y)\)看作带有点式收敛拓扑的映射空间\(Y^X\)的子空间,
则称“\(\ContinuousFunctionSpace(X,Y)\)是从拓扑空间\(X\)到拓扑空间\(Y\)的\DefineConcept{带有点式收敛拓扑的连续映射空间}”.
把\(Y^X\)的点式收敛拓扑在\(\ContinuousFunctionSpace(X,Y)\)上的限制
称为“\(\ContinuousFunctionSpace(X,Y)\)的\DefineConcept{点式收敛拓扑}”.
\end{definition}

从\(X\)到\(Y\)的带有点式收敛拓扑的连续映射空间
\(\ContinuousFunctionSpace(X,Y)\)作为\(Y^X\)的子空间,
自然可以继承\(Y^X\)的许多可遗传性质,
我们就不在此一一枚举了.

\begin{theorem}
%@see: 《点集拓扑讲义(第四版)》(熊金城) P250 定理9.1.1
设\(X\)是一个提赫诺夫空间,
则从\(X\)到实数空间\(\mathbb{R}\)的带有点式收敛拓扑的连续映射空间\(\ContinuousFunctionSpace(X,\mathbb{R})\)是
带有点式收敛拓扑的映射空间\(\mathbb{R}^X\)中的一个稠密子集.
%TODO proof
\end{theorem}

\begin{corollary}
%@see: 《点集拓扑讲义(第四版)》(熊金城) P251 推论9.1.2
设\(X\)是一个提赫诺夫空间,
映射\(f\colon X \to \mathbb{R}\),
则对于任意给定正数\(\epsilon\)和\(X\)中有限个点\(\AutoTuple{x}{n}\),
存在一个连续映射\(g\),
使得\(\abs{f(x_i) - g(x_i)} < \epsilon\)
对于\(i=1,2,\dotsc,n\)成立.
\end{corollary}

上述推论使我们直观地感受到,
所谓点式收敛拓扑,
便是刻画映射在有限个点“逼近”的拓扑,
它也使我们感觉得讨论映射空间不同拓扑(映射的不同“逼近”方式)的必要性.

\begin{example}
%@see: 《点集拓扑讲义(第四版)》(熊金城) P251 习题 1.
设\(X\)是一个可数的提赫诺夫空间,
\(\mathbb{R}^X\)是带有点式收敛拓扑的映射空间.
证明:如果\(f \in \mathbb{R}^X\),
则存在\(\ContinuousFunctionSpace(X,\mathbb{R})\)中的一个序列\(\{f_n\}_{n \in \mathbb{N}}\)收敛于\(f\).
%TODO proof
\end{example}

\begin{example}
%@see: 《点集拓扑讲义(第四版)》(熊金城) P252 习题 2.
设区间\(I = [0,1]\),
\(\mathbb{R}^I\)是带有点式收敛拓扑的映射空间,
\(\mathbb{R}^I\)中的
序列\(\{f_n\}_{n \in \mathbb{N}}\)满足
	对于任意\(x \in I\)有\(f_n(x) = x^n\).
证明:序列\(\{f_n\}_{n \in \mathbb{N}}\)收敛,但是它的极限不是一个连续映射.
%TODO proof
\end{example}

\begin{example}
%@see: 《点集拓扑讲义(第四版)》(熊金城) P252 习题 3.
设\(X\)是一个提赫诺夫空间.
试讨论:从拓扑空间\(X\)到欧氏平面\(Y \defeq \mathbb{R}^2\)的带有点式收敛拓扑的连续映射空间\(\ContinuousFunctionSpace(X,Y)\)
是不是带有点式收敛拓扑的映射空间\(Y^X\)中的一个稠密子集.
%TODO
\end{example}

\begin{example}
%@see: 《点集拓扑讲义(第四版)》(熊金城) P252 习题 3.
设\(X\)是一个提赫诺夫空间.
试讨论:从拓扑空间\(X\)到单位圆周\(Y \defeq S^1\)的带有点式收敛拓扑的连续映射空间\(\ContinuousFunctionSpace(X,Y)\)
是不是带有点式收敛拓扑的映射空间\(Y^X\)中的一个稠密子集.
%TODO
\end{example}
