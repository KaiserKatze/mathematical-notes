\section{点式收敛拓扑}
%@see: 《点集拓扑讲义(第四版)》(熊金城) P249
设\(X\)是一个拓扑空间,
\(\Gamma\)是一个集合.
把从\(\Gamma\)到\(X\)的映射空间\(X^\Gamma\)
可以看作拓扑空间族\(\{X_\gamma\}_{\gamma \in \Gamma}\)的笛卡尔积
(其中每一个\(X_\gamma\)都等于\(X\)),
对于任意\(\gamma \in \Gamma\),
把从\(X^\Gamma\)到坐标集\(X_\gamma\)的投射
称为“\(X^\Gamma\)在点\(\gamma\)的\DefineConcept{赋值映射}”,
记为\(e_\gamma\).
显然,对于任意映射\(f \in X^\Gamma\),有\(e_\gamma(f) = f(\gamma)\).
%\cref{theorem:一般情形下的积空间.由各坐标空间中的开集在投射的逆下的像组成的子基}
我们已经知道,\(X^\Gamma\)的积拓扑\(\T\)是以\(
	\S = \Set{
		e_\gamma^{-1}(U)
		\given
		\text{$U$是$X$的一个开集},
		\gamma \in \Gamma
	}
\)为子基的拓扑.
我们把\(X^\Gamma\)的积拓扑\(\T\)
称为它的\DefineConcept{点式收敛拓扑},
将拓扑空间\((X^\Gamma,\T)\)称为\DefineConcept{从\(\Gamma\)到\(X\)的带有点式收敛拓扑的映射空间}.

由于带有点式收敛拓扑的映射空间本身便是一类特别的积空间,
因此关于积空间的一般结论全部适用于它,无需另行证明,
我们简要归纳如下:\begin{itemize}
	\item 带有点式收敛拓扑的映射空间\(X^\Gamma\)满足性质\(\phi\),
	当且仅当
		拓扑空间\(X\)满足性质\(\phi\),
	其中性质\(\phi\)可以是\(T_0\)、\(T_1\)、豪斯多夫、\(T_3\)、提赫诺夫、正则、完全正则、紧致.

	\item 带有点式收敛拓扑的映射空间\(X^\Gamma\)满足性质\(\phi\),
	当且仅当
		要么\(X\)是一个平庸空间,
		要么\(\Gamma\)是一个可数集并且\(X\)也满足性质\(\phi\),
	其中性质\(\phi\)可以是第一可数性公理、第二可数性公理.

	\item 带有点式收敛拓扑的映射空间\(X^\Gamma\)中的
		序列\(\{f_n\}_{n \in \mathbb{N}}\)
		收敛于映射\(f \in X^\Gamma\),
	当且仅当
		对于任意一个\(\gamma \in \Gamma\),
		拓扑空间\(X\)中的序列\(\{f_n(\gamma)\}_{n \in \mathbb{N}}\)收敛于\(f(\gamma)\).
\end{itemize}

\begin{definition}
%@see: 《点集拓扑讲义(第四版)》(熊金城) P250 定义9.1.1
设\(X\)和\(Y\)是两个拓扑空间,
\(
	\C(X,Y)
	\defeq
	\Set{
		f \in Y^X
		\given
		\text{$f$是连续的}
	}
\).
如果把\(\C(X,Y)\)看作带有点式收敛拓扑的映射空间\(Y^X\)的子空间,
则称“\(\C(X,Y)\)是\DefineConcept{从拓扑空间\(X\)到拓扑空间\(Y\)的带有点式收敛拓扑的连续映射空间}”,
把\(\C(X,Y)\)的拓扑称为\DefineConcept{点式收敛拓扑}.
\end{definition}

\(\C(X,Y)\)作为\(Y^X\)的子空间,
自然可以继承\(Y^X\)的许多可遗传性质,
我们就不在此一一枚举了.

\begin{theorem}
%@see: 《点集拓扑讲义(第四版)》(熊金城) P250 定理9.1.1
设\(X\)是一个提赫诺夫空间,
则从\(X\)到实数空间\(\mathbb{R}\)的带有点式收敛拓扑的连续映射空间\(\C(X,\mathbb{R})\)是
带有点式收敛拓扑的映射空间\(\mathbb{R}^X\)中的一个稠密子集.
%TODO proof
\end{theorem}

\begin{corollary}
%@see: 《点集拓扑讲义(第四版)》(熊金城) P251 推论9.1.2
设\(X\)是一个提赫诺夫空间,
映射\(f\colon X \to \mathbb{R}\),
则对于任意给定正数\(\epsilon\)和\(X\)中有限个点\(\AutoTuple{x}{n}\),
存在一个连续映射\(g\),
使得\(\abs{f(x_i) - g(x_i)} < \epsilon\)
对于\(i=1,2,\dotsc,n\)成立.
\end{corollary}

上述推论使我们直观地感受到,
所谓点式收敛拓扑,
便是刻画映射在有限个点“逼近”的拓扑,
它也使我们感觉得讨论映射空间不同拓扑(映射的不同“逼近”方式)的必要性.

\begin{example}
%@see: 《点集拓扑讲义(第四版)》(熊金城) P251 习题 1.
设\(X\)是一个可数的提赫诺夫空间,
\(\mathbb{R}^X\)是带有点式收敛拓扑的映射空间.
证明:如果\(f \in \mathbb{R}^X\),
则存在\(\C(X,\mathbb{R})\)中的一个序列\(\{f_n\}_{n \in \mathbb{N}}\)收敛于\(f\).
%TODO proof
\end{example}

\begin{example}
%@see: 《点集拓扑讲义(第四版)》(熊金城) P252 习题 2.
设区间\(I = [0,1]\),
\(\mathbb{R}^I\)是带有点式收敛拓扑的映射空间,
\(\mathbb{R}^I\)中的
序列\(\{f_n\}_{n \in \mathbb{N}}\)满足
	对于任意\(x \in I\)有\(f_n(x) = x^n\).
证明:序列\(\{f_n\}_{n \in \mathbb{N}}\)收敛,但是它的极限不是一个连续映射.
%TODO proof
\end{example}

\begin{example}
%@see: 《点集拓扑讲义(第四版)》(熊金城) P252 习题 3.
设\(X\)是一个提赫诺夫空间.
试讨论:从拓扑空间\(X\)到欧氏平面\(Y \defeq \mathbb{R}^2\)的带有点式收敛拓扑的连续映射空间\(\C(X,Y)\)
是不是带有点式收敛拓扑的映射空间\(Y^X\)中的一个稠密子集.
%TODO
\end{example}

\begin{example}
%@see: 《点集拓扑讲义(第四版)》(熊金城) P252 习题 3.
设\(X\)是一个提赫诺夫空间.
试讨论:从拓扑空间\(X\)到单位圆周\(Y \defeq S^1\)的带有点式收敛拓扑的连续映射空间\(\C(X,Y)\)
是不是带有点式收敛拓扑的映射空间\(Y^X\)中的一个稠密子集.
%TODO
\end{example}
