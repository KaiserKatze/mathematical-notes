\section{紧致-开拓扑}
这一节讨论拓扑空间之间的所有映射构成的集合中的一个新的拓扑 --- 紧致开拓扑
--- 并且指出在某些中药的情形下它与一致收敛拓扑的关联.

\subsection{子集族开拓扑}
\begin{definition}
%@see: 《点集拓扑讲义(第四版)》(熊金城) P255
设\(X\)和\(Y\)是两个集合.
对于任意\(E \subseteq X\)和\(B \subseteq Y\),
定义:\begin{equation}
	W(E,B)
	\defeq
	\Set{
		f \in Y^X
		\given
		f(E) \subseteq B
	}.
\end{equation}
\end{definition}

\def\E{\mathscr{E}}

\begin{definition}\label{definition:映射空间.子集族开拓扑}
%@see: 《点集拓扑讲义(第四版)》(熊金城) P255 定义9.3.1
设\(X\)是一个集合,\(Y\)是一个拓扑空间,
\(\E\)是\(X\)的一个子集族,
\(Y^X\)是从\(X\)到\(Y\)的映射空间,
取\(Y^X\)的一个子集族\begin{equation}
	\S_\E
	\defeq
	\Set{
		W(E,U) \subseteq Y^X
		\given
		E \in \E,
		\text{$U$是$Y$的一个开集}
	}.
\end{equation}
把以\(\S_\E\)为子基的拓扑
称为“\(Y^X\)的 \DefineConcept{\(\E\)-开拓扑}”,
记作\(\T_\E\).
把拓扑空间\((Y^X,\T_\E)\)
称为“从\(X\)到\(Y\)的\DefineConcept{带有\(\E\)-开拓扑的映射空间}”.
\end{definition}

\begin{proposition}
\(\bigcup \S_\E = Y^X\).
\begin{proof}
显然,对于任意一个\(E \in \E\),
有\(
	W(E,Y)
	= \Set{
		f \in Y^X
		\given
		f(E) \subseteq Y
	}
	= Y^X
\).
%TODO proof
\end{proof}
\end{proposition}

\begin{proposition}
\(\T_\E\)是\(Y^X\)的唯一一个以\(\S_\E\)为子基的拓扑.
%TODO proof
\end{proposition}

\begin{proposition}
\def\P{\mathscr{P}}
设\(X\)是一个集合,\(Y\)是一个拓扑空间,
取\(X\)中所有单点子集\begin{equation*}
	\sfA \defeq \Set{
		A
		\given
		A = \{a\},
		a \in X
	},
\end{equation*}
那么\(Y^X\)的点式收敛拓扑恰好就是以\(\sfA\)为子基的拓扑\(\T_{\sfA}\).
%TODO proof
\end{proposition}
\begin{remark}
基于上述命题的结论,我们常常把点式收敛拓扑称为\DefineConcept{点开拓扑}.
\end{remark}

\begin{proposition}
如果\(\E_1\)和\(\E_2\)都是\(X\)的子集族,
并且\(\E_1 \subseteq \E_2\),
则\(\S_{\E_1} \subseteq \S_{\E_2}\)和\(\T_{\E_1} \subseteq \T_{\E_2}\).
%TODO proof
% \begin{proof}
% 由\cref{definition:映射空间.子集族开拓扑} 易知.
% \end{proof}
\end{proposition}

%@see: 《点集拓扑讲义(第四版)》(熊金城) P256
我们曾经说过,所谓点式收敛拓扑(即点开拓扑)便是刻画映射在有限个点处“逼近”的拓扑.
与之类似,\(\E\)-开拓扑可以认为是要求映射在集族\(\E\)的元素上“逼近”的拓扑.
另一方面,点式收敛拓扑明显与映射的定义域\(X\)中的拓扑毫无关系,
然而我们现在却可以通过集族\(\E\)的选取,
使得映射的定义域\(X\)中的拓扑介入到\(\E\)-开拓扑中去.

\subsection{紧致开拓扑}
%@see: 《点集拓扑讲义(第四版)》(熊金城) P256
在不同类型的\(\E\)-开拓扑中,最为重要的一种便是紧致开拓扑.
\begin{definition}
%@see: 《点集拓扑讲义(第四版)》(熊金城) P256 定义9.3.2
设\(X\)和\(Y\)是两个拓扑空间,
\(\sfC\)是\(X\)的全体紧致子集,
\(Y^X\)是从\(X\)到\(Y\)的映射空间.
把\(Y^X\)的\(\sfC\)-开拓扑\(\T_\sfC\)
称为“\(Y^X\)的\DefineConcept{紧致开拓扑}”.
把拓扑空间\((Y^X,\T_\sfC)\)称为
“从\(X\)到\(Y\)的\DefineConcept{带有紧致开拓扑的映射空间}”.
\end{definition}

\begin{definition}
%@see: 《点集拓扑讲义(第四版)》(熊金城) P256 定义9.3.2
设\(X\)和\(Y\)是两个拓扑空间,
\(Y^X\)是从\(X\)到\(Y\)的带有紧致开拓扑的映射空间.
如果把从\(X\)到\(Y\)的连续映射空间\(\ContinuousFunctionSpace(X,Y)\)看作\(Y^X\)的子空间,
则称“\(\ContinuousFunctionSpace(X,Y)\)是从拓扑空间\(X\)到拓扑空间\(Y\)的\DefineConcept{带有紧致开拓扑的连续映射空间}”.
把\(Y^X\)的紧致开拓扑在\(\ContinuousFunctionSpace(X,Y)\)上的限制
称为“\(\ContinuousFunctionSpace(X,Y)\)的\DefineConcept{紧致开拓扑}”.
\end{definition}

\subsection{点式收敛拓扑与紧致开拓扑的关联}
%@see: 《点集拓扑讲义(第四版)》(熊金城) P256
因为任意一个单点集都是紧致子集,
所以\(X\)中所有单点子集\(\sfA\)与\(X\)的全体紧致子集\(\sfC\)满足\(\sfA \subseteq \sfC\),
从而有以下定理:
\begin{theorem}
%@see: 《点集拓扑讲义(第四版)》(熊金城) P257 定理9.3.1
设\(X\)和\(Y\)是两个拓扑空间,
\(Y^X\)是从\(X\)到\(Y\)的映射空间,
\(\T_\sfA\)是\(Y^X\)的点式收敛拓扑,
\(\T_\sfC\)是\(Y^X\)的紧致开拓扑,
则\begin{itemize}
	\item \(\T_\sfA \subseteq \T_\sfC\);
	\item 对于任意\(x \in X\),
	\(Y^X\)在\(x\)的赋值映射\(e_x\)对于\(Y^X\)的紧致开拓扑而言是一个连续映射.
\end{itemize}
%TODO proof
\end{theorem}

\subsection{紧致开拓扑与拓扑不变性质}
根据同样的理由可见,
当\(Y\)是\(T_0\)空间、\(T_1\)空间或豪斯多夫空间时,
带有紧致开拓扑的映射空间\(Y^X\)
以及带有紧致开拓扑的连续映射空间\(\ContinuousFunctionSpace(X,Y)\)
相应也是\(T_0\)空间、\(T_1\)空间或豪斯多夫空间.
但是当\(Y\)是正规的、满足第一可数性公理或满足第二可数性公理时,
并不蕴含\(Y^X\)和\(\ContinuousFunctionSpace(X,Y)\)具有相应性质.
不过,当\(Y\)是正则空间时,
\(\ContinuousFunctionSpace(X,Y)\)却仍然是正则空间.

\begin{lemma}
%@see: 《点集拓扑讲义(第四版)》(熊金城) P257 引理9.3.2
设\(X\)是一个拓扑空间,\(Y \subseteq X\).
如果\(\S\)是\(X\)的一个子基,
并且对于任意一个\(y \in Y\)和\(\S\)中任意一个包含\(y\)的元素\(S\),
存在\(\S\)中的一个包含\(y\)的元素\(T\)
使得\(T\)在拓扑空间\(X\)中的闭包\(\TopoClosureL{T}\)包含于\(S\),
则\(Y\)作为\(X\)的子空间是一个正则空间.
%TODO proof
\end{lemma}

\begin{theorem}
%@see: 《点集拓扑讲义(第四版)》(熊金城) P258 定理9.3.3
设\(X\)和\(Y\)都是拓扑空间.
如果\(Y\)是一个正则空间,
则带有紧致开拓扑的连续映射空间\(\ContinuousFunctionSpace(X,Y)\)也是一个正则空间.
%TODO proof
\end{theorem}

\subsection{一致收敛拓扑与紧致开拓扑的关联}
我们现在来指出一致收敛拓扑和紧致开拓扑两者之间的一个关联.
\begin{theorem}
%@see: 《点集拓扑讲义(第四版)》(熊金城) P259 定理9.3.4
设\(X\)是一个紧致空间,\((Y,\rho)\)是一个度量空间,
则连续映射空间\(\ContinuousFunctionSpace(X,Y)\)的一致收敛拓扑和紧致开拓扑相同.
%TODO proof
\end{theorem}

\begin{example}
%@see: 《点集拓扑讲义(第四版)》(熊金城) P260 习题 1.
设\(X\)和\(Y\)是一个拓扑空间,
\(X_1\)是\(X\)的一个子集.
证明:对于\(\ContinuousFunctionSpace(X,Y)\)和\(\ContinuousFunctionSpace(X_1,Y)\)的紧致开拓扑而言,
映射\(
	r\colon \ContinuousFunctionSpace(X,Y) \to \ContinuousFunctionSpace(X_1,Y),
	f \mapsto (f \SetRestrict X_1)
\)是连续的.
%TODO proof
\end{example}

\begin{example}
%@see: 《点集拓扑讲义(第四版)》(熊金城) P261 习题 2.
设\(X\)和\(Y\)是两个拓扑空间.
证明:对于连续映射空间\(\ContinuousFunctionSpace(X,Y)\)的紧致开拓扑而言,
赋值映射\(
	e\colon \ContinuousFunctionSpace(X,Y) \times X \to Y,
	(f,x) \mapsto f(x)
\)是连续的.
%TODO proof
\end{example}
