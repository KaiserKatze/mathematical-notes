\section{紧致-开拓扑}
这一节讨论拓扑空间之间的所有映射构成的集合中的一个新的拓扑 --- 紧致开拓扑
--- 并且指出在某些中药的情形下它与一致收敛拓扑的关联.

\begin{definition}
%@see: 《点集拓扑讲义(第四版)》(熊金城) P255
设\(X\)和\(Y\)是两个集合.
对于任意\(E \subseteq X\)和\(B \subseteq Y\),
定义:\begin{equation}
	W(E,B)
	\defeq
	\Set{
		f \in Y^X
		\given
		f(E) \subseteq B
	}.
\end{equation}
\end{definition}

\def\E{\mathscr{E}}

\begin{definition}
%@see: 《点集拓扑讲义(第四版)》(熊金城) P255 定义9.3.1
设\(X\)是一个集合,\(Y\)是一个拓扑空间,
\(\E\)是\(X\)的一个子集族,
\(Y^X\)是从\(X\)到\(Y\)的映射空间,
取\(Y^X\)的一个子集族\begin{equation}
	\S_\E
	\defeq
	\Set{
		W(E,U) \subseteq Y^X
		\given
		E \in \E,
		\text{$U$是$Y$的一个开集}
	}.
\end{equation}
把以\(\S_\E\)为子基的拓扑
称为“\(Y^X\)的 \DefineConcept{\(\E\)-开拓扑}”,
记作\(\T_\E\).
把拓扑空间\((Y^X,\T_\E)\)
称为\DefineConcept{从\(X\)到\(Y\)的带有\(\E\)-开拓扑的映射空间}.
\end{definition}

\begin{proposition}
\(\bigcup \S_\E = Y^X\).
\begin{proof}
显然,对于任意一个\(E \in \E\),
有\(
	W(E,Y)
	= \Set{
		f \in Y^X
		\given
		f(E) \subseteq Y
	}
	= Y^X
\).
%TODO proof
\end{proof}
\end{proposition}

\begin{proposition}
\(\T_\E\)是\(Y^X\)的唯一一个以\(\S_\E\)为子基的拓扑.
%TODO proof
\end{proposition}

\begin{proposition}
\def\P{\mathscr{P}}
设\(X\)是一个集合,\(Y\)是一个拓扑空间,
取\(X\)中所有单点子集\begin{equation*}
	\sfA \defeq \Set{
		A
		\given
		A = \{a\},
		a \in X
	},
\end{equation*}
那么\(Y^X\)的点式收敛拓扑恰好就是以\(\sfA\)为子基的拓扑\(\T_{\sfA}\).
%TODO proof
\end{proposition}
\begin{remark}
基于上述命题的结论,我们常常把点式收敛拓扑称为\DefineConcept{点开拓扑}.
\end{remark}
