\section{图的矩阵表示}
\subsection{图的邻接矩阵}
我们首先介绍“邻接矩阵”,它表示的是图中任意两个顶点之间的邻接关系.
\begin{definition}
%@see: 《离散数学》(邓辉文) P182 定义6-29
设\(G = (V,E)\)是图,
顶点集\(V = \{\AutoTuple{v}{n}\}\).
将\begin{equation*}
	\vb{A}(G) \defeq (a_{ij})_n
\end{equation*}
称为“\(G\)的\DefineConcept{邻接矩阵}(adjacency matrix)”,
其中,
当\(G\)是有向图时,
\(a_{ij}\)表示以\(v_i\)为起点、\(v_j\)为终点的有向边的数目;
当\(G\)是无向图时,
\(a_{ij},a_{ji}\)均表示以\(v_i,v_j\)为端点的无向边的数目.
\end{definition}

显然,无向图的邻接矩阵是对称矩阵.

从一个图的邻接矩阵容易得出每个顶点的出度、入度、度数.
于有向图\(G\)而言,
\(\vb{A}(G)\)中第\(i\)行元素之和,就是第\(i\)个顶点\(v_i\)的出度,
\(\vb{A}(G)\)中第\(j\)列元素之和,就是第\(j\)个顶点\(v_j\)的入度.

从邻接矩阵还可以得出从顶点\(v_i\)到顶点\(v_j\)、路长为\(l\)的路的数目.
\begin{theorem}
%@see: 《离散数学》(邓辉文) P183 定理6-10
设\(\vb{A}\)是图\(G\)的邻接矩阵,\(l\geq1\),
则\(\vb{A}\)的\(l\)次幂\(\vb{A}^l\)的\((i,j)\)元素\(a^{(l)}_{ij}\)
就是从顶点\(v_i\)到顶点\(v_j\)、路长为\(l\)的路的数目.
%TODO proof
\end{theorem}

\subsection{图的可达矩阵}
接下来介绍“可达矩阵”,它表示的是图中任意两个顶点之间的可达关系.
\begin{definition}
%@see: 《离散数学》(邓辉文) P183 定义6-30
设\(G = (V,E)\)是图,
顶点集\(V = \{\AutoTuple{v}{n}\}\).
将\begin{equation*}
	\vb{P}(A) \defeq (p_{ij})_n
\end{equation*}
称为“\(G\)的\DefineConcept{可达矩阵}(accessible matrix)”,
其中,\begin{equation*}
	p_{ij}
	= \left\{ \begin{array}{cl}
		1, & \text{$v_i$可达$v_j$}, \\
		0, & \text{其他}
	\end{array} \right.
	\quad(i,j=1,2,\dotsc,n).
\end{equation*}
\end{definition}

容易从图的邻接矩阵\(\vb{A}(G)\)得出它的可达矩阵\(\vb{P}(G)\).
一个非常有效的算法是 Warshall 算法.
%TODO 什么是 Warshall 算法?

根据可达矩阵的定义,\(\vb{P}(G)\)的主对角线上的元素全为\(1\),
这是因为任意顶点均可达它自己.

从一个图的可达矩阵容易得出图的连通性.

\subsection{图的关联矩阵}
下面介绍“关联矩阵”,它表示的是图中顶点与边之间的关联关系.
\begin{definition}
%@see: 《离散数学》(邓辉文) P184 定义6-31
设\(G = (V,E)\)是无向图,
顶点集\(V = \{\AutoTuple{v}{n}\}\),
边集\(E = \{\AutoTuple{e}{m}\}\).
将\begin{equation*}
	\vb{M}(A) \defeq (m_{ij})_n
\end{equation*}
称为“\(G\)的\DefineConcept{关联矩阵}(incidence matrix)”,
其中,\(m_{ij}\)表示顶点\(v_i\)与边\(e_j\)的关联次数.
\end{definition}

从一个图的关联矩阵容易得出顶点的度数、是否存在多重边、是否存在孤立点等性质.

\begin{definition}
%@see: 《离散数学》(邓辉文) P184 定义6-32
设\(G = (V,E)\)是无自环的有向图,
顶点集\(V = \{\AutoTuple{v}{n}\}\),
边集\(E = \{\AutoTuple{e}{m}\}\).
将\begin{equation*}
	\vb{M}(A) \defeq (m_{ij})_n
\end{equation*}
称为“\(G\)的\DefineConcept{关联矩阵}(incidence matrix)”,
其中,\begin{equation*}
	m_{ij}
	= \left\{ \begin{array}{rl}
		1, & \text{$v_i$是$e_j$的起点}, \\
		-1, & \text{$v_i$是$e_j$的终点}, \\
		0, & \text{$v_i$与$e_j$不关联}
	\end{array} \right.
	\quad(i=1,2,\dotsc,n;j=1,2,\dotsc,m).
\end{equation*}
\end{definition}
