\section{计数原理与排列组合}
\subsection{排列、组合的基本概念}
从若干个元素中取出几个或全部的一种排法,
称作是一个\DefineConcept{排列}(permutation).
例如,从\(a,b,c,d\)四个字母里一次取出两个的排列数有12个,即\begin{gather*}
%@Mathematica: SolvePermutationProblem[list_, n_] := Module[
%					{results},
%					results = Permutations[list, {n}];
%					Print[StringForm[
%						"从`1`这`2`个字母里一次取出`3`个的排列数有`4`个:\n`5`",
%						list, Length[list], n, Length[results], results
%					]];
%				]
%@Mathematica: SolvePermutationProblem[{a, b, c, d}, 2]
	ab, \quad
	ac, \quad
	ad, \quad
	bc, \quad
	bd, \quad
	cd, \\
	ba, \quad
	ca, \quad
	da, \quad
	cb, \quad
	db, \quad
	dc;
\end{gather*}
其中每一个都代表两个字母的不同的排法.

从若干个元素中取出几个或全部的一种选法,
称作是一个\DefineConcept{组合}(combination).
例如,从\(a,b,c,d\)四个字母里一次取出两个的组合数有6个,即\begin{equation*}
	ab, \quad
	ac, \quad
	ad, \quad
	bc, \quad
	bd, \quad
	cd;
\end{equation*}
其中每一个都代表两个字母的不同的选法.

从这些例子中我们看到,组合仅与每个选法所含元素的个数有关,
而排列还要考虑元素在每一个排法中的次序.
例如,从四个字母\(a,b,c,d\)中选三个字母可以得到\(abc\)这样一种组合,
可以作以下六种不同的排列:\begin{equation*}
	abc, \quad
	acd, \quad
	bca, \quad
	bac, \quad
	cab, \quad
	cba.
\end{equation*}

\subsection{排列组合的基本原理}
在讨论排列与组合的一般性命题之前,我们先通过几个例子来说明两个重要的原则.
\begin{itemize}
	\item \DefineConcept{加法原理}(addition principle):
	如果做一件事,完成它可以有\(n\)类办法,
	在第一类办法中有\(m_1\)种不同的方法,
	在第二类办法中有\(m_2\)种不同的方法,……,
	在第\(n\)类办法中有\(m_n\)种不同的方法,那么完成这件事共有\begin{equation*}
		N = m_1 + m_2 + \dotsb + m_n
	\end{equation*}种不同的方法.

	\item \DefineConcept{乘法原理}(multiplication principle):
	如果做一件事,完成它需要分成\(n\)个步骤,
	做第一步有\(m_1\)种不同的方法,
	做第二步有\(m_2\)种不同的方法,……,
	做第\(n\)步有\(m_n\)种不同的方法,那么完成这件事共有\begin{equation*}
		N = m_1 \times m_2 \times \dotsm \times m_n
	\end{equation*}种不同的方法.
\end{itemize}
我们把加法原理、乘法原理统称为\DefineConcept{计数原理}.
计数原理实际上与\hyperref[definition:基数.基数算术的定义]{基数算术的定义}密切相关.

假设一个班有\(m_1\)个男生、\(m_2\)个女生,
那么根据加法原理可知,
这个班一共有\(m_1 + m_2\)个学生.
像这样,首先将计数的元素划分成若干个不同的类,再分类计数,最后相加,
这种计数方式称为\emph{分类处理}.

假设从地点\(A\)到地点\(B\)有\(m_1\)条路线,
从地点\(B\)到地点\(C\)有\(m_2\)条路线,
那么根据乘法原理可知,
从地点\(A\)先到地点\(B\)再到地点\(C\)就有\(m_1 m_2\)条路线.
像这种,计数时首先分成几个独立的步骤,分别计算每一步的数目,最后相乘,
这种计数方式称为\emph{分步处理}.

在实际运用计数原理时,分类处理、分步处理可能会混合使用.

\begin{example}
有10艘汽艇往返于利物浦与都柏林之间,问某人来回乘坐不同汽艇的方式有多少种?
\begin{solution}
去时有10种方式;对于其中每一种,因为不能乘坐同一条船,回来时有9种方式可选;
因此,一共有\(10 \times 9 = 90\)种方法来完成这两段路程.
\end{solution}
\end{example}

\begin{example}
3个旅行者来到一个小镇,镇上有4家旅店,若他们分别住不同的旅店,问投宿的方式有多少种?
\begin{solution}
第一个人可以有4种选择;
在他选定后,第二个人有3种选择;
从而这两个人共有\(4 \times 3\)种选择,
对于其中每一种选择,第三个人有2家旅店供选择;
所以,投宿的方式共有\(4 \times 3 \times 2 = 24\)种.
\end{solution}
\end{example}

我们现在来求从\(n\)个不同元素中一次取出\(k\)个的排列数.
这可以看成是用我们手上的\(n\)个不同元素去填满\(k\)个空位的不同方式的种数.

填第一个位置有\(n\)种方式,
因为可以取\(n\)个元素中的任意一个.
在这个位置用任意一种方式填好后,
填第二个位置有\(n-1\)种方式.
由于填第一个位置的每一种方式,都可与填第二个位置的每一种方式组合,
所以填前两个位置的方式有\(n(n-1)\)种.
前两个位置以任意一种方式填好后,
填第三个位置有\(n-2\)种方式.
同理,填前三个位置的方式共有\(n(n-1)(n-2)\)种.
继续这样的方法,我们注意到每填一个新的位置,就产生一个新的因子.
而且在每一步上,因子的个数总与剩余位置的个数相同;
于是,我们得出填满\(k\)个位置的方式种数等于\begin{equation*}
	\underbrace{n(n-1)(n-2)\dotsm[n-(k-1)]}_{\text{$k$个因子}}.
\end{equation*}
这就是所求从\(n\)个元素一次取出\(k\)个元素的排列数,
我们可以利用阶乘记号将它表示为\begin{equation*}
	\frac{n!}{(n-k)!}.
\end{equation*}

我们还可以推断出如下结论:
从\(n\)个元素一次取出全部\(n\)个元素的排列数为\(n!\).

以后我们用符号\(A_n^k\)表示从\(n\)个元素一次取出\(k\)个的排列数,即\begin{equation}
	A_n^k \defeq \frac{n!}{(n-k)!}.
\end{equation}
特别地,\(A_n^n \equiv n!\).

从\(n\)个元素一次取出\(k\)个元素的排列数也可用下面的思路求得.
按规定,我们用\(A_n^k\)代表从\(n\)个元素一次取出\(k\)个元素的排列数.
假设我们首先组成\(n\)个元素一次取出\(k-1\)个元素的所有排列,
那么这些排列一共有\(A_n^{k-1}\)种.
在这些排列的每一个的后面,我们放上剩下的\(n-(k-1)\)个元素中的任意一个;
每进行一次这样的操作,我们便得到\(n\)个元素一次取\(k\)个的一种排列,
因此,\(n\)个元素一次取\(k\)个的排列数为\(A_n^{k-1} \times (n-k+1)\)种,即\begin{equation*}
	A_n^k = A_n^{k-1} \times (n-k+1).
\end{equation*}
在上式中用\(k-1\)代替\(k\),我们得\begin{equation*}
	A_n^{k-1} = A_n^{k-2} \times (n-k+2);
\end{equation*}
依次递推,直到\begin{equation*}
	A_n^2 = A_n^1 \times (n-1),
\end{equation*}\begin{equation*}
	A_n^1 = n.
\end{equation*}
将上述各式的等号左右两边分别相乘,且约去两边相同的因子,便得\begin{equation*}
	A_n^k = n(n-1)\dotsm(n-k+2)(n-k+1).
\end{equation*}

接下来我们想求从\(n\)个不同的元素中一次取出\(k\)的组合数.
我们用符号\(C_n^k\)表示所求组合数.
每一个这样的组合都由一组\(k\)个不同的元素组成,
而这组元素本身又可以组成\(k!\)个不同的排列.
因此\(C_n^k \times k!\)便等于\(n\)个元素一次取\(k\)个的排列数,即\begin{equation*}
	C_n^k \times k! \equiv A_n^k
	= n(n-1)(n-2)\dotsm(n-k+2)(n-k+1);
\end{equation*}于是\begin{align}
	C_n^k &= \frac{n(n-1)(n-2)\dotsm(n-k+2)(n-k+1)}{k!} \\
	&= \frac{n!}{k! (n-k)!}.
\end{align}

需要注意到,当\(k=n\)时,有\begin{equation*}
	A_n^n = \frac{n!}{(n-n)!} = \frac{n!}{0!} = n!,
\end{equation*}\begin{equation*}
	C_n^n = \frac{n!}{n! (n-n)!} = \frac{n!}{n! 0!} = \frac{1}{0!} = 1;
\end{equation*}
这说明,符号\(0!\)的取值应该等于\(1\).
这就是为什么我们在\hyperref[definition:数列.阶乘的定义]{阶乘的定义}中特别规定\(0!\equiv1\).
%特别地,规定:当\(n < k\)时,\(C_n^k = 0\).

\begin{example}
将\(m+n\)颗小球分为两袋,一袋含\(m\)颗小球,另一袋含\(n\)颗小球,求可能的分法种数.
\begin{solution}
不难看出,这样的分法种数,等于从\(m+n\)颗小球中一次取\(m\)颗的组合数,
如此,可将取出的\(m\)颗小球装入一袋,将剩下的\((m+n)-m=n\)颗小球装入另一袋.
因此,所求的分法种数为\begin{equation*}
	\frac{(m+n)!}{m! n!}.
\end{equation*}

特别地,当\(m=n\)时,两个袋子中所含小球颗数相同.
如果认为“两个袋子没有次序”或者说“袋子互换,分法种数不变”,那么分法种数为\begin{equation*}
	\frac{(2n)!}{(n!)^2 2!}.
\end{equation*}
\end{solution}
\end{example}

\begin{example}
将\(m+n+p\)颗小球分为三袋,且各袋分别含\(m,n,p\)颗小球,求可能的分法种数.
\begin{solution}
首先将全部小球分到两个口袋里,各袋分别含\(m,n+p\)颗小球,分法种数为\begin{equation*}
	\frac{(m+n+p)!}{m!(n+p)!};
\end{equation*}
然后将第二袋中的\(n+p\)颗小球再分为两袋,各袋分别含\(n,p\)颗小球,分法种数为\begin{equation*}
	\frac{(n+p)!}{n! p!};
\end{equation*}
所以,全部\(m+n+p\)颗小球分成三袋,各袋分别含\(m,n,p\)颗小球的分法种数为\begin{equation*}
	\frac{(m+n+p)!}{m!(n+p)!} \cdot \frac{(n+p)!}{n! p!}
	= \frac{(m+n+p)!}{m! n! p!}.
\end{equation*}

特别地,当\(m=n=p\)时,三个袋子中所含小球颗数相同.
如果认为“三个袋子有次序”或者说“袋子互换,就得到新的分法”,那么分法种类为\begin{equation*}
	\frac{(3n)!}{(n!)^3};
\end{equation*}
反之,如果认为“三个袋子没有次序”,那么分法种类为\begin{equation*}
	\frac{(3n)!}{(n!)^3 3!}.
\end{equation*}
\end{solution}
\end{example}

到目前为止,我们考虑的元素(如小球)常常被看作是不同的;
但有的时候,所给元素中有一部分元素是相同的.
相同的元素是无法区分的,不管怎么排布这些元素,都对排法种数没有影响.

\begin{example}
排列\(n\)颗小球,其中\(p\)颗是红球,\(q\)颗是黄球,\(r\)颗是蓝球,
剩余的\(n-p-q-r\)颗小球具有互不相同的彩色花纹,求可能的排列数.
\begin{solution}
当提到一组小球具有某种特征(如小球是红色的)时,
我们认为这组小球是相同的、无法区分的.
基于这条约定,我们来求解可能的排列数.

设所求排列数为\(x\).
如果用\(p\)颗花纹各异的小球代替上述\(p\)颗红球,
那么对于\(x\)个排列中的任一个,
不改变其他小球的位置,我们可以作\(p!\)个新排列;
于是如果对\(x\)个排列中的每一个,都作这样的替换,
我们就得到\(x \cdot p!\)种排列.
同理,如果在此基础上继续用\(q\)颗花纹各异的小球代替\(q\)颗黄球,会得到\(x \cdot p! \cdot q!\)种排列.
再用\(r\)颗花纹各异的小球代替\(r\)颗蓝球,会得到\(x \cdot p! \cdot q! \cdot r!\)种排列.
现在,这\(n\)颗小球全都互不相同了,它们的全排列数为\(n!\),于是有\begin{equation*}
	n! = x \cdot p! \cdot q! \cdot r!,
\end{equation*}即有\begin{equation*}
	x = \frac{n!}{p! q! r!}.
\end{equation*}
\end{solution}
\end{example}

\begin{example}
用数字\(1,2,3,4,3,2,1\)可以组成多少个七位数,且奇数总在奇数位上.
\begin{solution}
奇数\(1,3,3,1\)有\(\frac{4!}{2! 2!}\)种方式排列在它们的四个位置上;
偶数\(2,4,2\)有\(\frac{3!}{2!}\)种方式排列在它们的三个位置上;
奇数的每一种排列都能与偶数的每一种排列组合,因此,所求排列数为\begin{equation*}
	\frac{4!}{2! 2!} \times \frac{3!}{2!} = 18.
\end{equation*}
\end{solution}
\end{example}

现在我们来求从\(n\)个元素中一次取出\(k\)个的排列数,
其中,取出的元素可以重复任意多次.
我们可以把这个问题考虑为这样的情况:
有\(n\)个不同的元素,去填满\(k\)个位置,并且每一个元素都可以任意次地重复使用.
这样的排列一共有多少种呢?
显而易见的是,填第一个位置有\(n\)种方式.
当第一个位置填好后,第二个位置仍有\(n\)种填法,
这是因为占据第一个位置的那个元素可以在第二个位置上重复使用.
因此,填满前两个位置的方式一共有\(n \times n = n^2\)种.
同理,填第三个位置还是有\(n\)种方式;
所以,填满前三个位置的方式一共有\(n^3\)种.
按这样的方法进行,我们注意到\(n\)的指数总与剩余位置的个数相同;
由此可知,一共有\(n^k\)种方式填满所给的\(k\)个位置.

\begin{example}
将5件奖品颁发给4位选手,如果每位选手都可以得到全部奖品,
问一共有多少种分发方式.
\begin{solution}
第一件奖品可以有4种颁发方式.
第二件奖品仍有4种颁发方式,
这是因为得到第一件奖品的那位选手仍可以获得第二件奖品.
因此,前两件奖品有\(4^2\)种颁发方式,
前三件奖品有\(4^3\)种颁发方式,以此类推,
全部5件奖品有\(4^5=1024\)种颁发方式.
\end{solution}
\end{example}

我们再来求从\(n\)个元素中一次取出若干个以至于全部元素的所有选择方式的种数.

每一个元素都有两种处理方式,要么取出,要么不取.
并且,任意一个元素的每一种处理方式,都可以与任何另外一个元素的每一种处理方式组合,
所以,\(n\)个元素便有\begin{equation*}
	\underbrace{2 \times 2 \times \dotsm \times 2}_{\text{$n$个}}
	= 2^n
\end{equation*}种处理方式.
但是这里包括了“\(n\)个元素都不取”这样一种选择,
除去这种情况,\(n\)个元素的取法共有\(2^n-1\)种.
有时候将这称为“\(n\)个元素的\DefineConcept{全组合数}”.

\begin{example}
某人有6位朋友,如果他要邀请至少一位朋友吃饭,有多少种不同的请法.
\begin{solution}
他必从6位朋友中选部分或全部,故有\(2^6-1=63\)种请法.
\end{solution}
\end{example}

\begin{example}
求从\(p+q+r+\dotsb\)个元素中取出至少一个元素的取法种数,
其中\(p\)个元素是相同的一类,\(q\)个元素是相同的另一类,
\(r\)个元素是相同的第三类,等等.
\begin{solution}
这\(p\)个元素可以有\(p+1\)种处理方式,因为我们可以从中取出\(0,1,2,\dotsc,p\)个;
同理,这\(q\)个元素可以有\(q+1\)种处理方式,这\(r\)个元素可以有\(r+1\)种处理方式;以此类推.
因此,所有的元素便有\begin{equation*}
	(p+1)(q+1)(r+1)\dotsm
\end{equation*}种处理方式.
但这里包括了任何元素都不取的一种选择,
除去这种情况,所求取法种数为\begin{equation*}
	-1+(p+1)(q+1)(r+1)\dotsm.
\end{equation*}
\end{solution}
\end{example}

%\begin{theorem}[抽屉原理]\label{theorem:排列组合.抽屉原理}
%抽屉原理有以下几种形式:
%\begin{enumerate}
%\item 把\(n+1\)个元素放入\(n\)个集合内,则一定有一个集合里有两个或两个以上的元素.
%\item 把\(m\)个元素任意放入\(n\ (n<m)\)个集合里,则一定有一个集合里至少有\(k\)个元素,其中\begin{equation*}
%k = \left\{ \begin{array}{ll}
%m/n, & m \pmod n = 0, \\
%\floor{m/n}+1, & m \pmod n \neq 0.
%\end{array} \right.
%\end{equation*}
%\item 把无穷多个元素放入有限个集合里,则一定有一个集合里含有无穷多个元素.
%\end{enumerate}
%\end{theorem}
%\hyperref[theorem:排列组合.抽屉原理]{抽屉原理}有时候也称作\DefineConcept{鸽巢原理};
%因它最先是由狄利克雷明确地提出来的,因此也可称其为\DefineConcept{狄利克雷原理}.

\begin{example}
将\(n\)个相同的小球,放入\(m\)个不同的袋子中,求可能的放法种数.
\begin{solution}
与其考虑“将\(n\)个相同的小球,放入\(m\)个不同的袋子中”,
不如研究“将\(n\)个相同的球与\(m-1\)个相同的木板排列在一条直线上”,
把\(m-1\)个木板作为分隔,恰能将球分入\(m\)个袋子中,
于是所求可能的方法种数就是\begin{equation*}
	C_{n+m-1}^n
	= \frac{
		(n + m-1)!
	}{
		n! (m-1)!
	}.
\end{equation*}
% 以下是错误解法:
% 与其考虑“将\(n\)个相同的小球,放入\(m\)个不同的袋子中”,
% 不如探究“将\(n\)个球排在一条直线上,形成\(n+1\)个空位
% (每两个球之间有1个空位,最左侧的球的左边、最右侧的球的右边也都各算作1个空位),
% 再将\(m-1\)个木板插入上述\(n+1\)个空位”.
\end{solution}
%@see: https://brilliant.org/wiki/integer-equations-star-and-bars/
%@see: https://artofproblemsolving.com/wiki/index.php/Ball-and-urn
%@see: https://www.geeksforgeeks.org/competitive-programming/stars-and-bars-algorithms-for-competitive-programming/
%@see: https://codeforces.com/blog/entry/143449
\end{example}

\begin{example}
将\(n\)个相同的小球,放入\(m\)个不同的袋子中,每个袋子至少放一个球,求可能的放法种数.
\begin{solution}
假设\(n \geq m\).
“将\(n\)个相同的小球,放入\(m\)个不同的袋子中,每个袋子至少放一个球”
相当于提前从\(n\)个相同的小球中取出\(m\)个,在\(m\)个袋子中各放\(1\)个,
接着考虑“将\(n-m\)个相同的小球,放入\(m\)个不同的袋子中”,
因此可能的方法种数为\(C_{n-1}^{m-1}\).
\end{solution}
\end{example}

\subsection{组合数的性质}
\begin{property}\label{theorem:组合数性质1}
\(C_n^k = C_n^{n-k}\).
\begin{proof}
\(
C_n^{n-k}
= \frac{n!}{(n-k)! [n-(n-k)]!}
= \frac{n!}{k! (n-k)!}
= C_n^k
\).
\end{proof}
\end{property}
这就是说,从\(n\)个元素中一次取出\(k\)个的组合数,
等于从\(n\)个元素中一次取出\(n-k\)个的组合数.
像这样的两类组合称为\DefineConcept{互补}.

\cref{theorem:组合数性质1} 对于简化运算有很大用处.

可以注意到,如果我们把所有组合数\(C_n^k\)按照\(n\)相同的写成一行,再按\(k\)从小到大排列,
就能得到一个美妙的三角形,如\cref{figure:排列组合.杨辉三角} 所示.
在这个三角形中,在任意一行中,除开两端的数字1以外,所有“中间”数字都是它“肩上”两个数字的和.
像这样表示组合数的关系的图示叫做\DefineConcept{杨辉三角}或\DefineConcept{帕斯卡三角}.
%@Mathematica: Column[Table[Binomial[i, j], {i, 0, 8}, {j, 0, i}]]
\begin{figure}
	\centering
	\begin{tikzpicture}[
		bino/.style={
			% The shape:
			circle,
			% The size:
			minimum size=6mm,
			% The border:
			very thick,
			draw=red!50!black!50, % 50% red and 50% black,
			% and that mixed with 50% white
			% The filling:
			top color=white, % a shading that is white at the top...
			bottom color=red!50!black!20, % and something else at the bottom
			% Font
			%font=\itshape
	}, node distance=5mm]
		\node(C00)[bino]{1};
		\node(C10)[bino,below=of C00]{1};
		\node(C11)[bino,right=of C10]{1};
		\node(C20)[bino,below=of C10]{1};
		\node(C21)[bino,right=of C20]{2};
		\node(C22)[bino,right=of C21]{1};
		\node(C30)[bino,below=of C20]{1};
		\node(C31)[bino,right=of C30]{3};
		\node(C32)[bino,right=of C31]{3};
		\node(C33)[bino,right=of C32]{1};
		\node(C40)[bino,below=of C30]{1};
		\node(C41)[bino,right=of C40]{4};
		\node(C42)[bino,right=of C41]{6};
		\node(C43)[bino,right=of C42]{4};
		\node(C44)[bino,right=of C43]{1};
		\draw(C00)--(C10)--(C20)--(C30)--(C40)
			(C00)--(C11)--(C22)--(C33)--(C44);
		\draw(C10)--(C21)--(C11)
			(C20)--(C31)--(C21)--(C32)--(C22)
			(C30)--(C41)--(C31)--(C42)--(C32)--(C43)--(C33);
	\end{tikzpicture}
	\caption{}
	\label{figure:排列组合.杨辉三角}
\end{figure}

下面我们给出对杨辉三角所指出的经验规律的严格证明.
\begin{property}\label{theorem:组合数性质2}
%@see: 《高等代数与解析几何(第三版 上册)》(孟道骥) P10 习题 1.(2)
\(C_{n-1}^{k-1} + C_{n-1}^k = C_n^k\).
\begin{proof}
直接计算得
\begin{align*}
	C_{n-1}^{k-1} + C_{n-1}^k
	&= \frac{(n-1)!}{(k-1)! (n-k)!} + \frac{(n-1)!}{k! (n-k-1)!} \\
	&= \frac{(n-1)!}{(k-1)! (n-k-1)!} \left( \frac{1}{n-k} + \frac{1}{k} \right) \\
	&= \frac{(n-1)!}{(k-1)! (n-k-1)!} \cdot \frac{n}{(n-k)k} \\
	&= \frac{n!}{k! (n-k)!}
	= C_n^k. \qedhere
\end{align*}
\end{proof}
\end{property}
因此,在\(n\)不太大的时候,我们可以直接画出杨辉三角,从图中找到\(C_n^k\)的数值.

\begin{property}\label{theorem:组合数性质3}
%@see: 《高等代数与解析几何(第三版 上册)》(孟道骥) P10 习题 1.(1)
\(\sum_{i=0}^n C_n^i = 2^n\).
\begin{proof}
当\(n=0\)时,\(\sum_{i=0}^0 C_0^i = C_0^0 = \frac{0!}{0! \cdot 0!} = 1 = 2^0\)成立.
假设\(n=k\)时,结论仍成立,即\begin{equation*}
\sum_{i=0}^k C_k^i
= C_k^0 + C_k^1 + \dotsb + C_k^k = 2^k.
\end{equation*}那么\begin{equation*}
\begin{array}{*{14}{c}}
& C_k^0 &+& C_k^1 &+& \dotsb &+& C_k^{k-1} &+& C_k^k && &=& 2^k \\
+) & && C_k^0 &+& \dotsb &+& C_k^{k-2} &+& C_k^{k-1} &+& C_k^k &=& 2^k \\ \hline
& C_k^0 &+& C_{k+1}^1 &+& \dotsb &+& C_{k+1}^{k-1} &+& C_{k+1}^k &+& C_k^k &=& 2 \cdot 2^k
\end{array}
\end{equation*}又因为\(C_k^0 = C_{k+1}^0 = C_k^k = C_{k+1}^{k+1} = 1\),所以\begin{equation*}
\sum_{i=0}^{k+1} C_{k+1}^i = 2^{k+1}
\end{equation*}成立.
\end{proof}
\end{property}

\begin{property}
%@see: 《高等代数与解析几何(第三版 上册)》(孟道骥) P10 习题 1.(3)
\(
	\sum_{i=0}^k C_{n+i}^i
	= C_{n+k+1}^k
\).
\end{property}

\begin{property}\label{theorem:组合数性质4}
\(\sum_{k=0}^n (-1)^k C_n^k = 0\).
\end{property}

\begin{property}\label{theorem:组合数性质5}
\(\sum_{k=0}^{\floor{n/2}} C_n^{2k} = 2^{n-1}\).
\end{property}
\begin{property}\label{theorem:组合数性质6}
\(\sum_{k=0}^{\floor{(n-1)/2}} C_n^{2k+1} = 2^{n-1}\).
\end{property}

\begin{property}\label{theorem:组合数性质7}
\(C_{n+m}^k = \sum_{r=0}^{k} C_n^r C_m^{k-r}\).
\end{property}

\begin{property}\label{theorem:组合数性质8}
%@see: 《高等代数与解析几何(第三版 上册)》(孟道骥) P10 习题 1.(4)
\(\sum_{k=0}^n (C_n^k)^2 = C_{2n}^n\).
\end{property}

\begin{property}\label{theorem:组合数性质9}
\(C_n^{r_1} C_{n-r_1}^{r_2} \dotsm C_{n-(r_1+r_2+\dotsb+r_{k-1})}^{r_k}
= \frac{n!}{r_1! r_2! \dotsm r_k!}\).
\end{property}

\begin{example}
求:当\(k\)取何值时,\(C_n^k\)最大.
\begin{solution}
我们首先研究几个特例.
注意到\begin{equation*}
	C_2^0 = 1, \qquad
	C_2^1 = 2, \qquad
	C_2^2 = 1,
\end{equation*}\begin{equation*}
	C_3^0 = 1, \qquad
	C_3^1 = 3, \qquad
	C_3^2 = 3, \qquad
	C_3^3 = 1,
\end{equation*}
似乎可以总结出如下两条规律:
\begin{itemize}
	\item 随着\(k\)逐渐增加,\(C_n^k\)先逐渐增大,再逐渐减小.
	\item 有的组合数在取邻近的两个不同的\(k\)值时,可能取得同样的最大值.
\end{itemize}
下面我们为这两条经验规律给出严格的证明.

因为\begin{equation*}
	C_n^k = \frac{n(n-1)(n-2)\dotsm(n-k+2)(n-k+1)}{1\cdot2\cdot3\dotsm(k-1)k},
\end{equation*}\begin{equation*}
	C_n^{k-1} = \frac{n(n-1)(n-2)\dotsm(n-k+2)}{1\cdot2\cdot3\dotsm(k-1)},
\end{equation*}
所以\begin{equation*}
	C_n^k = C_n^{k-1} \cdot \frac{n-k+1}{k}
	= C_n^{k-1} \cdot \left( \frac{n+1}{k} - 1 \right).
\end{equation*}
令\(C_n^k > C_n^{k-1}\),化简得\begin{equation*}
	\frac{n+1}{k} - 1 > 1,
\end{equation*}
考虑到\(0 \leq k \leq n\),
上式又可化简为\(2k < n+1\),
即\(k < (n+1)/2\).
这就是说,当\(0 \leq k < (n+1)/2\)时,
随着\(k\)逐渐增加,\(C_n^k\)逐渐增加;
当\((n+1)/2 < k \leq n\)时,
随着\(k\)逐渐增加,\(C_n^k\)逐渐减小.
由此可知,当\(k\)取不大于\((n+1)/2\)的最大整数\(\floor{(n+1)/2}\)时,\(C_n^k\)取得最大值.

若\(n\)为偶数,不妨设\(n = 2m\ (m\in\mathbb{N})\),则\begin{equation*}
	\frac{n+1}{2} = \frac{2m+1}{2}
	= m+\frac{1}{2}.
\end{equation*}
于是,只要取\(k = m = n/2\),我们便得\(C_n^k\)的最大值:\begin{equation*}
	C_n^{\frac{n}{2}}.
\end{equation*}

若\(n\)为奇数,不妨设\(n = 2m+1\ (m\in\mathbb{N})\),则\begin{equation*}
	\frac{n+1}{2} = \frac{2m+2}{2} = m+1.
\end{equation*}
于是,只要取\(k = m+1 = (n+1)/2\),我们便得\(C_n^k\)的最大值;
但是,由\cref{theorem:组合数性质1} 可知,\begin{equation*}
	C_n^{\frac{n+1}{2}}
	= C_n^{n-\frac{n+1}{2}}
	= C_n^{\frac{n-1}{2}};
\end{equation*}
于是,在取\(k = (n-1)/2 = m\)时,也可取得\(C_n^k\)的最大值.
也就是说,当\(k\)取\((n\pm1)/2\)时,我们便得\(C_n^k\)的最大值:\begin{equation*}
	C_n^{\frac{n+1}{2}}
	= C_n^{\frac{n-1}{2}}.
\end{equation*}

综上所述,我们有\begin{equation}
	\max_{0 \leq k \leq n} C_n^k = \left\{ \begin{array}{cl}
		C_n^{n/2}, & \text{$n$是偶数}, \\
		C_n^{(n-1)/2} = C_n^{(n+1)/2}, & \text{$n$是奇数}.
	\end{array} \right.
\end{equation}
\end{solution}
\end{example}
