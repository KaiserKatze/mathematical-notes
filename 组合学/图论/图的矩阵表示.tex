\section{图的矩阵表示}
\subsection{图的邻接矩阵}
我们首先介绍“邻接矩阵”,它表示的是图中任意两个顶点之间的邻接关系.
\begin{definition}
%@see: 《离散数学》(邓辉文) P182 定义6-29
设\(G = (V,E)\)是图,
顶点集\(V = \{\AutoTuple{v}{n}\}\).
将\begin{equation*}
	\vb{A}(G) \defeq (a_{ij})_n
\end{equation*}
称为“\(G\)的\DefineConcept{邻接矩阵}(adjacency matrix)”,
其中,
当\(G\)是有向图时,
\(a_{ij}\)表示以\(v_i\)为起点、\(v_j\)为终点的有向边的数目;
当\(G\)是无向图时,
\(a_{ij},a_{ji}\)均表示以\(v_i,v_j\)为端点的无向边的数目.
\end{definition}

显然,无向图的邻接矩阵是对称矩阵.

从一个图的邻接矩阵容易得出每个顶点的出度、入度、度数.
于有向图\(G\)而言,
\(\vb{A}(G)\)中第\(i\)行元素之和,就是第\(i\)个顶点\(v_i\)的出度,
\(\vb{A}(G)\)中第\(j\)列元素之和,就是第\(j\)个顶点\(v_j\)的入度.

从邻接矩阵还可以得出从顶点\(v_i\)到顶点\(v_j\)、路长为\(l\)的路的数目.
\begin{theorem}
%@see: 《离散数学》(邓辉文) P183 定理6-10
设\(\vb{A}\)是图\(G\)的邻接矩阵,\(l\geq1\),
则\(\vb{A}\)的\(l\)次幂\(\vb{A}^l\)的\((i,j)\)元素\(a^{(l)}_{ij}\)
就是从顶点\(v_i\)到顶点\(v_j\)、路长为\(l\)的路的数目.
%TODO proof
\end{theorem}

\subsection{图的可达矩阵}
接下来介绍“可达矩阵”,它表示的是图中任意两个顶点之间的可达关系.
\begin{definition}
%@see: 《离散数学》(邓辉文) P183 定义6-30
设\(G = (V,E)\)是图,
顶点集\(V = \{\AutoTuple{v}{n}\}\).
将\begin{equation*}
	\vb{P}(G) \defeq (p_{ij})_n
\end{equation*}
称为“\(G\)的\DefineConcept{可达矩阵}(accessible matrix)”,
其中,\begin{equation*}
	p_{ij}
	= \left\{ \begin{array}{cl}
		1, & \text{$v_i$可达$v_j$}, \\
		0, & \text{其他}
	\end{array} \right.
	\quad(i,j=1,2,\dotsc,n).
\end{equation*}
\end{definition}

容易从图的邻接矩阵\(\vb{A}(G)\)得出它的可达矩阵\(\vb{P}(G)\).
一个非常有效的算法是 Warshall 算法.
%TODO 什么是 Warshall 算法?

根据可达矩阵的定义,\(\vb{P}(G)\)的主对角线上的元素全为\(1\),
这是因为任意顶点均可达它自己.

从一个图的可达矩阵容易得出图的连通性.

\subsection{图的关联矩阵}
下面介绍“关联矩阵”,它表示的是图中顶点与边之间的关联关系.
\begin{definition}
%@see: 《离散数学》(邓辉文) P184 定义6-31
设\(G = (V,E)\)是无向图,
顶点集\(V = \{\AutoTuple{v}{n}\}\),
边集\(E = \{\AutoTuple{e}{m}\}\).
将\begin{equation*}
	\vb{M}(G) \defeq (m_{ij})_n
\end{equation*}
称为“\(G\)的\DefineConcept{关联矩阵}(incidence matrix)”,
其中,\(m_{ij}\)表示顶点\(v_i\)与边\(e_j\)的关联次数.
\end{definition}

从一个图的关联矩阵容易得出顶点的度数、是否存在多重边、是否存在孤立点等性质.

\begin{definition}
%@see: 《离散数学》(邓辉文) P184 定义6-32
设\(G = (V,E)\)是无自环的有向图,
顶点集\(V = \{\AutoTuple{v}{n}\}\),
边集\(E = \{\AutoTuple{e}{m}\}\).
将\begin{equation*}
	\vb{M}(G) \defeq (m_{ij})_{n \times m}
\end{equation*}
称为“\(G\)的\DefineConcept{关联矩阵}(incidence matrix)”,
其中,\begin{equation*}
	m_{ij}
	\defeq
	\left\{ \begin{array}{rl}
		1, & \text{$v_i$是$e_j$的起点}, \\
		-1, & \text{$v_i$是$e_j$的终点}, \\
		0, & \text{$v_i$与$e_j$不关联}
	\end{array} \right.
	\quad(i=1,2,\dotsc,n;j=1,2,\dotsc,m).
\end{equation*}
\end{definition}
\begin{remark}
关联矩阵的定义取决于个人使用习惯,
有的人或许会把\(\vb{M}(G)\)定义为\(m \times n\)矩阵,
然后将关联矩阵的元素定义为\begin{equation*}
	m_{ij}
	\defeq
	\left\{ \begin{array}{rl}
		-1, & \text{$v_i$是$e_j$的起点}, \\
		1, & \text{$v_i$是$e_j$的终点}, \\
		0, & \text{$v_i$与$e_j$不关联}
	\end{array} \right.
	\quad(i=j=1,2,\dotsc,m;1,2,\dotsc,n).
\end{equation*}
\end{remark}
\begin{remark}
可以看出,有向图的关联矩阵是一个稀疏矩阵,
其中只有\(2m\)个非零元素.
\end{remark}

\begin{example}
%@see: 《Introduction to Linear Algebra (2016)》(Gilbert Strang) P145 Problem Set 3.2 32
基尔霍夫电流定律\(\vb{A}^T \vb{y} = \vb0\)描述了一个简单的物理图景:
对于无源电路中任意一个顶点,流进的电流等于流出的电流.
我们可以为\cref{figure:图的矩阵表示.基尔霍夫电流定律1} 表示的电路图\(G = (V,E)\)
写出相应的关联矩阵:\begin{equation*}
	\vb{A}
	\defeq \vb{M}(G)
	= \begin{bmatrix}
		-1 & 1 & 0 & 0 \\
		0 & -1 & 1 & 0 \\
		1 & 0 & -1 & 0 \\
		-1 & 0 & 0 & 1 \\
		0 & -1 & 0 & 1 \\
		0 & 0 & -1 & 1
	\end{bmatrix}.
\end{equation*}

\begin{figure}[hbt]
	\centering
	\def\n{3}  % 控制顶点数
	\def\b{180}  % 控制图像绕其几何中心旋转的角度(相位)
	\def\scale{2}  % 放大图形
	\begin{tikzpicture}
		\pgfmathsetmacro{\a}{360/\n}
		\foreach \j in {1,...,\n} {
			\fill({\scale*sin(\j*\a+\b)},{\scale*cos(\j*\a+\b)})coordinate(A\j)circle(2pt);
		}
		\draw(A1)node[left]{$v_1$};
		\draw(A2)node[right]{$v_2$};
		\draw(A3)node[below]{$v_3$};
		\fill(0,0)coordinate(A4)circle(2pt)node[below left]{$v_4$};
		\begin{scope}[-{Latex[length=3mm,width=0pt 10]}]
			\draw(A1)--(A2)node[midway]{$y_1$};
			\draw(A1)--(A4)node[midway]{$y_4$};
			\draw(A2)--(A3)node[midway]{$y_2$};
			\draw(A2)--(A4)node[midway]{$y_5$};
			\draw(A3)--(A1)node[midway]{$y_3$};
			\draw(A3)--(A4)node[midway]{$y_6$};
		\end{scope}
	\end{tikzpicture}
	\caption{}
	\label{figure:图的矩阵表示.基尔霍夫电流定律1}
\end{figure}

从矩阵\(\vb{A}\)中选取若干个行向量,组成\(\vb{A}\)的一个子阵\(\vb{A}'\),
就表示\(G\)的一个子图\(G' = (V,E')\),
其中\(E' \subseteq E\).

假设电路图\(G\)中顶点\(\AutoTuple{v}{4}\)的电势分别是\(\AutoTuple{x}{4}\),
记\(\vb{x} \defeq (\AutoTuple{x}{4})^T\),
那么\(\vb{A} \vb{x}\)就表示各边上的电势差.
解方程\(\vb{A} \vb{x} = \vb0\)
可得\begin{equation*}
	\Ker\vb{A} = \Span\{(1,1,1,1)^T\}.
\end{equation*}

在实际应用中,我们尝尝将某个顶点“接地”,把这个顶点的电势记作\(0\).
假设将\(v_4\)接地,
即\(x_4 = 0\),
那么方程\(\vb{A} \vb{x} = \vb0\)
可以化简为\begin{equation*}
	\begin{bmatrix}
		-1 & 1 & 0 \\
		0 & -1 & 1 \\
		1 & 0 & -1 \\
		-1 & 0 & 0 \\
		0 & -1 & 0 \\
		0 & 0 & -1
	\end{bmatrix}
	\begin{bmatrix}
		x_1 \\ x_2 \\ x_3
	\end{bmatrix}
	= \vb0.
\end{equation*}

\(\vb{A}\)的行最简形为\begin{equation*}
	\vb{R} \defeq \begin{bmatrix}
		1 & 0 & 0 & -1 \\
		0 & 1 & 0 & -1 \\
		0 & 0 & 1 & -1 \\
		0 & 0 & 0 & 0 \\
		0 & 0 & 0 & 0 \\
		0 & 0 & 0 & 0
	\end{bmatrix}.
\end{equation*}
由此可见\(\rank\vb{A} = 3\)
(应该注意到,当我们把\(G\)看作无向图时,
\(G\)一共有4个圈,它们分别是\(
	v_1 v_2 v_3 v_1,
	v_1 v_2 v_4 v_1,
	v_1 v_3 v_4 v_1,
	v_2 v_3 v_4 v_2
\),
而\(\rank\vb{A}\)恰好等于\(G\)中的{圈}的个数减\(1\)).

假设电路图\(G\)中有向边\(\AutoTuple{y}{6}\)都表示电流,
记\(\vb{y} \defeq (\AutoTuple{y}{6})^T\),
那么\(\vb{A}^T \vb{y}\)就表示各顶点的电荷量.
解方程\(\vb{A}^T \vb{y} = \vb0\)
可得\begin{equation*}
	\Ker\vb{A}^T = \Span\{
		(1,1,0,-1,0,1)^T,\allowbreak
		(1,0,0,-1,1,0)^T,\allowbreak
		(1,1,1,0,0,0)^T
	\}.
\end{equation*}

\(\vb{A}^T\)的行最简形为\begin{equation*}
	\vb{R}' \defeq \begin{bmatrix}
		1 & 0 & -1 & 0 & -1 & -1 \\
		0 & 1 & -1 & 0 & 0 & -1 \\
		0 & 0 & 0 & 1 & 1 & 1 \\
		0 & 0 & 0 & 0 & 0 & 0
	\end{bmatrix}.
\end{equation*}

\(\vb{A}\)的每一个线性相关的行向量组,
都对应着电路图\(G\)(看作无向图)的一个圈.

\(\vb{A}\)的行向量组的每一个极大线性无关组,
都对应着电路图\(G\)的一个生成树.

对于有源电路,
我们可以将它具体分为两种情形:一种是在电路图中加入电压源(影响顶点之间的电势差),
一种是在图中加入电流源(影响流进流出顶点的电流).
对于加入电流源的情形,
只需将基尔霍夫电流定律修正为\(\vb{A}^T \vb{y} = \vb\beta\),
其中,向量\(\vb\beta\)的分量\(b_i\ (i=1,2,3,4,5,6)\)表示流进流出顶点\(v_i\)的电流总和;
当\(b_i > 0\)时,说明电流源在向顶点\(v_i\)输出电流;
当\(b_i < 0\)时,说明电流源在从顶点\(v_i\)输入电流.
\end{example}
