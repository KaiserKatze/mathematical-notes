\section{欧拉图}
欧拉图是欧拉于1736年研究七桥问题时考虑的一种图,由此得名.
\begin{definition}
%@see: 《离散数学》(邓辉文) P191 定义7-1
设\(G = (V,E)\)是图.
\begin{itemize}
	\item \(G\)中经过所有边一次且仅一次的路,
	称为\DefineConcept{欧拉轨迹}(Eulerian trail)
	或\DefineConcept{欧拉路}.

	\item \(G\)中经过所有边一次且仅一次的回路,
	称为\DefineConcept{欧拉回路}(Eulerian circuit).

	\item 存在欧拉回路的图,称为\DefineConcept{欧拉图}(Eulerian graph).
\end{itemize}
\end{definition}

显然,欧拉回路是欧拉轨迹,但欧拉轨迹不一定是欧拉回路.

\begin{theorem}[欧拉定理]
%@see: 《离散数学》(邓辉文) P191 定理7-1(欧拉定理)
设\(G\)是非平凡连通无向图,
则\(G\)是欧拉图的充分必要条件是
\(G\)的每个顶点的度数均为偶数.
%TODO proof
\end{theorem}

欧拉定理给出了一个连通图存在欧拉回路的充分必要条件.
于是接下来的问题就成了要如何具体找出一条这样的回路.
1921年 Fleury 给出了一个求解欧拉回路的算法.
%TODO

下面再介绍几个定理,它们都类似于欧拉定理.
\begin{theorem}
%@see: 《离散数学》(邓辉文) P192 定理7-2
设\(G\)是弱连通图,
则\(G\)是欧拉图的充分必要条件是
\(G\)中每个顶点的入度等于出度.
%TODO proof
\end{theorem}

\begin{example}
%@see: 《离散数学》(邓辉文) P192 例7-1
设\(G_1,G_2\)是\(n\)阶完全图\(K_n\ (n\geq4)\)的两个不同的子图.
若\(G_1,G_2\)都是欧拉图,
则\(G_1\)和\(G_2\)的对称差\(G_1 \symdiff G_2\)的每一个连通分量都是欧拉图.
%TODO proof
\end{example}

\begin{corollary}
%@see: 《离散数学》(邓辉文) P192 定理7-3
设\(G\)是连通无向图,
则\(G\)中存在欧拉轨迹的充分必要条件是
\(G\)中度数为奇数的顶点个数要么是\(0\)要么是\(2\).
%TODO proof
\end{corollary}

由上述推论可知,七桥问题无解,
在\cref{figure:图论.七桥问题} 中不存在欧拉轨迹.

\begin{theorem}
%@see: 《离散数学》(邓辉文) P192 定理7-4
设\(G\)是弱连通图,
则\(G\)中存在欧拉轨迹的充分必要条件是
\(G\)中每个顶点的入度等于出度.
%TODO proof
\end{theorem}

\begin{theorem}
%@see: 《离散数学》(邓辉文) P192 定理7-4
设\(G\)是弱连通图,
则\(G\)中存在欧拉轨迹的充分必要条件是:\begin{itemize}
	\item \(G\)中存在一个顶点\(v_1\)满足\(\degEx v_1 - \degIn v_1 = 1\);
	\item \(G\)中存在一个顶点\(v_2\)满足\(\degIn v_2 - \degEx v_2 = 1\);
	\item 除了\(v_1,v_2\)以外的其余每一个顶点\(v\)都满足\(\degEx v = \degIn v\).
\end{itemize}
%TODO proof
\end{theorem}

%TODO
% 中国邮递员问题(Chinese postman problem)
% 假设一位邮递员从邮局选好邮件去投递,然后返回邮局,
% 要求该邮递员必须经过他负责的每一条街至少一次,
% 要为他设计一条路线,使得总路程最短.
% 该问题的本质是,在允许添加多重边的情况下,在边赋权图中,求最短欧拉回路的问题.
% 该问题首次由管梅谷于1962年提出并研究,他提出了“奇偶点图上作业法”.
% 1973年,匈牙利数学家 Edmonds 和 Johnson 对中国邮递员问题给出了一种有效算法.
% 1995年,王树禾研究了多邮递员中国邮路问题(k-Postman Chinese postman problem).
