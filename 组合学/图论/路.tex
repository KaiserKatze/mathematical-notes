\section{路,回路}
在途中,经常需要考虑从一个顶点出发,沿着一些边连续移动到另一个顶点的问题.
这就是路的概念,它与七桥问题密切相关.

\subsection{路}
\begin{definition}
%@see: 《离散数学》(邓辉文) P175 定义6-13
在任意一个图\(G = (V,E)\)中,
将顶点、边交替出现的序列\begin{equation*}
	L: v_0 e_1 v_1 e_2 v_2 \dotso v_{i-1} e_i v_i \dotso e_l v_l
\end{equation*}称为“从\(v_0\)到\(v_l\)的一条\DefineConcept{路}(walk,way)”,
将\(v_0\)称为“路\(L\)的\DefineConcept{起点}”,
将\(v_l\)称为“路\(L\)的\DefineConcept{终点}”,
将\(L\)所经过的边数\(l\)称为\DefineConcept{路长}(length of walk)或\DefineConcept{跳数}(hop number).
特别地,只有一个顶点、路长为\(0\)的序列,称为\DefineConcept{平凡路}.
\end{definition}

在不引起混淆的情况下,可以将路\(L: v_0 e_1 v_1 e_2 v_2 \dotso v_{i-1} e_i v_i \dotso e_l v_l\)
简记为\begin{equation*}
	L: v_0 v_1 v_2 \dotso v_{i-1} v_i \dotso v_l
\end{equation*}
或\begin{equation*}
	L: e_1 e_2 \dotso e_i \dotso e_l.
\end{equation*}

\begin{definition}
%@see: 《离散数学》(邓辉文) P175
顶点不重复的路,称为\DefineConcept{路径}(path).
\end{definition}

\begin{definition}
%@see: 《离散数学》(邓辉文) P175
边不重复的路,称为\DefineConcept{轨迹}(trail).
\end{definition}

\begin{proposition}
%@see: 《离散数学》(邓辉文) P175
路径一定是轨迹,轨迹不一定是路径.
%TODO proof
\end{proposition}

\begin{definition}
%@see: 《离散数学》(邓辉文) P175 定义6-14
在图\(G = (V,E)\)中,
把从顶点\(u\)到顶点\(v\)、边数最少的路的路长
称为“\(u\)到\(v\)的\DefineConcept{距离}(distance)”,
记作\(d(u,v)\).
当从\(u\)到\(v\)的路不存在时,
记\(d(u,v) = \infty\).
\end{definition}

显然,对于任意顶点\(u,v \in V\)有\(d(u,v) \geq 0\).

\begin{definition}
%@see: 《离散数学》(邓辉文) P175 定义6-14
设图\(G = (V,E)\).
把\begin{equation*}
	\max_{u,v \in V} d(u,v)
\end{equation*}称为“图\(G\)的\DefineConcept{直径}(diameter)”,
记作\(\diam G\).
\end{definition}

\subsection{回路}
\begin{definition}
%@see: 《离散数学》(邓辉文) P176 定义6-15
起点与终点相同的路,称为\DefineConcept{回路}(circuit).
\end{definition}

\begin{definition}
%@see: 《离散数学》(邓辉文) P176 定义6-15
边不重复的回路,称为\DefineConcept{简单回路}(simple circuit)或\DefineConcept{闭迹}(closed trail).
\end{definition}

\begin{definition}
%@see: 《离散数学》(邓辉文) P176 定义6-15
除了起点、终点重合以外,其余顶点各个不同的简单回路,
称为\DefineConcept{圈}或\DefineConcept{环}(cycle).
\end{definition}
\begin{remark}
注意区分“环”与“自环”这两个概念:
自环是边,环是路.
\end{remark}

由于回路的路长至少是1,
所以平凡路不是回路.

\begin{example}
举例说明:简单回路不一定是圈.
%TODO
\end{example}

\begin{definition}
%@see: 《离散数学》(邓辉文) P176
有\(n\)个顶点的圈,称为 \DefineConcept{\(n\)阶圈},记作\(C_n\).
\end{definition}

\begin{definition}
%@see: 《离散数学》(邓辉文) P176
给\(n-1\)阶圈\(C_{n-1}\)增加一个顶点\(u\),
再使\(u\)与\(C_{n-1}\)的每个顶点邻接,
得到的图称为 \DefineConcept{\(n\)阶轮图},记作\(W_n\).
\end{definition}

\begin{theorem}
%@see: 《离散数学》(邓辉文) P176 定理6-3
在无向图\(G = (V,E)\)中,
若任意\(v \in V\)有\(\deg v \geq 2\),
则\(G\)中存在圈.
%TODO proof
\end{theorem}

\begin{example}
%@see: 《离散数学》(邓辉文) P177 习题6.4 5.
设无向图\(G\)的任意两个顶点之间都存在一条路.
证明:\(G\)的任意两条最长轨迹存在公共顶点.
%TODO proof
\end{example}

\begin{example}
%@see: 《离散数学》(邓辉文) P177 习题6.4 6.
设\(G\)是简单有向图,记\(k = \max\{\degEx G,\degIn G\}\).
证明:\(G\)中存在路长至少为\(k\)的轨迹.
%TODO proof
\end{example}

\begin{example}
%@see: 《离散数学》(邓辉文) P177 习题6.4 7.
证明:在一个没有回路的竞赛图\(G\)中,对于任意顶点\(u,v\),总有\(\degEx u \neq \degEx v\).
%TODO proof
\end{example}

\begin{example}
%@see: 《离散数学》(邓辉文) P177 习题6.4 8.
设有向图\(G = (V,E)\)满足\((\forall v \in V)[\degIn v \geq 2]\).
证明:\(G\)中至少含有两个不同的圈.
%TODO proof
\end{example}

\subsection{杜克路}
%@see: https://doc.sagemath.org/html/en/reference/combinat/sage/combinat/path_tableaux/dyck_path.html
%@see: https://mathweb.ucsd.edu/~duqiu/files/Nankai18.pdf
%@see: https://users.monash.edu/~heikod/cpg2020/CPGparts1and2.pdf
%@see: https://pages.stat.wisc.edu/~callan/papers/polygon_dissections/polygon_dissections.pdf
\begin{definition}
%@see: 《组合数学及其应用》(曾光) P87 定义2.28(格路径)
在二维平面\(\mathbb{Z}^2\)沿整数格点按一定步法行走的路径称为\DefineConcept{格路径},
把相邻两个格点之间的线段称为格路径的\DefineConcept{步},
所走的步数称为格路径的\DefineConcept{长度}.
\end{definition}
\begin{definition}
%@see: 《组合数学及其应用》(曾光) P87 定义2.29(Dyck路)
长为\(2n\)的\DefineConcept{杜克路}(Dyck path)
%@see: https://mathworld.wolfram.com/DyckPath.html
是指只用“上步”\((1,1)\)和“下步”\((1,-1)\),
从\((0,0)\)到\((2n,0)\)的
不允许走到\(x\)轴下方的格路径,
把\(n\)称为杜克路的\DefineConcept{阶}.
\end{definition}

\begin{example}
%@see: 《组合数学及其应用》(曾光) P87 例2.21
3阶杜克路有且仅有5种情况,如\cref{figure:杜克路.3阶杜克路} 所示.
\begin{figure}[hbt]
	\centering
	\def\subwidth{.15\linewidth}
	\def\step{10pt}
	\begin{subfigure}[b]{\subwidth}
		\centering
		\DrawDyckPath[\step]{0,1,2,3,2,1,0}
		\caption{}
	\end{subfigure}~\begin{subfigure}[b]{\subwidth}
		\centering
		\DrawDyckPath[\step]{0,1,2,1,2,1,0}
		\caption{}
	\end{subfigure}~\begin{subfigure}[b]{\subwidth}
		\centering
		\DrawDyckPath[\step]{0,1,2,1,0,1,0}
		\caption{}
	\end{subfigure}~\begin{subfigure}[b]{\subwidth}
		\centering
		\DrawDyckPath[\step]{0,1,0,1,2,1,0}
		\caption{}
	\end{subfigure}~\begin{subfigure}[b]{\subwidth}
		\centering
		\DrawDyckPath[\step]{0,1,0,1,0,1,0}
		\caption{}
	\end{subfigure}
	\caption{3阶杜克路的所有5种情况}
	\label{figure:杜克路.3阶杜克路}
\end{figure}
\end{example}

\begin{definition}
把\begin{equation*}
	\frac{(2n)!}{(n+1)! \cdot n!}
	\quad(n=1,2,\dotsc)
\end{equation*}
称为“第\(n\)个\DefineConcept{卡塔兰数}(Catalan number)”,
记作\(C_n\).
%@see: https://mathworld.wolfram.com/CatalanNumber.html
\end{definition}
\begin{table}[htb]
	\centering
	\begin{tblr}{c|*9c}
		\hline
		\(n\) & \(1\) & \(2\) & \(3\) & \(4\) & \(5\) & \(6\) & \(7\) & \(8\) & \(9\) \\
		\hline
		\(C_n\) & \(1\) & \(2\) & \(5\) & \(14\) & \(42\) & \(132\)
		& \(429\) & \(1430\) & \(4862\) \\
		\hline
	\end{tblr}
	\caption{前9个卡塔兰数}
%@Mathematica: Table[CatalanNumber[n], {n, 1, 10}]
\end{table}
\begin{theorem}
%@see: 《组合数学及其应用》(曾光) P88 定理2.26
全体\(n\)阶杜克路的基数等于第\(n\)个卡塔兰数.
\end{theorem}
