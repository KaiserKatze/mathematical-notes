\section{有向树,根树}
\subsection{有向树}
\begin{definition}
%@see: 《离散数学》(邓辉文) P203 定义7-6
设\(G = (V,E)\)是有向图.
若\(G\)的基础图是一棵树,
则称“\(G\)是一棵\DefineConcept{有向树}(directed tree)”.
\end{definition}

\begin{definition}
%@see: 《离散数学》(邓辉文) P203
设\(G\)是有向树,结点\(v \in G\).
\begin{itemize}
	\item \(v\)的前趋元素,称为“\(v\)的\DefineConcept{父结点}(parent)”.
	\item \(v\)的后继元素,称为“\(v\)的\DefineConcept{子结点}(child)”.
\end{itemize}
\end{definition}

\subsection{根树}
\begin{definition}
%@see: 《离散数学》(邓辉文) P203 定义7-7
若有向树\(T\)恰有1个结点入度为\(0\),而其余结点入度均为\(1\),
则称“\(T\)是一棵\DefineConcept{根树}(rooted tree)”.
\end{definition}

\begin{definition}
%@see: 《离散数学》(邓辉文) P203
设\(T\)是根树.
\begin{itemize}
	\item 将\(T\)中入度为\(0\)的结点称为“\(T\)的\DefineConcept{根结点}(root)”.
	\item 将\(T\)中出度为\(0\)的结点称为“\(T\)的\DefineConcept{叶子结点}(leaf)”.
	\item 将\(T\)中出度大于\(0\)的结点称为“\(T\)的\DefineConcept{分支结点}”.
	\item 将\(T\)中除了根结点以外的分支结点称为“\(T\)的\DefineConcept{内部结点}”.
\end{itemize}
\end{definition}

\begin{definition}
%@see: 《离散数学》(邓辉文) P203
设\(T\)是根树,结点\(v \in T\).
\begin{itemize}
	\item 如果结点\(u \in T\)和\(v\)的父结点相同,
	则称“\(u\)是\(v\)的\DefineConcept{兄弟结点}(sibling)”.
	\item 从根结点到\(v\)的路径上所经过的所有分支结点,
	%TODO 为什么“祖先”的定义要以“分支结点”为基础?
	%TODO \(v\)的祖先是否包括\(v\)自己,难不成根据\(v\)是叶子结点还是分支结点,还要进一步讨论?
	%TODO 我估计这里默认在讨论结点\(v\)的祖先时,是把这个结点当做叶子结点来看待的,那就不应该包括它.
	称为“\(v\)的\DefineConcept{祖先}(ancestor)”.
	\item \(v\)可达的全体结点,
	称为“\(v\)的\DefineConcept{后代}(offspring, descendants)”.
\end{itemize}
\end{definition}

可以证明,在根树中,从根结点到任意一个结点,有且仅有一条路径.

\begin{definition}
%@see: 《离散数学》(邓辉文) P204
设\(T\)是根树,结点\(v \in T\).
从根结点到\(v\)的路径的路长,
称为“\(v\)的\DefineConcept{层次}(level)”.
如果\(v\)的层次等于\(l\),
则称“\(v\)是第\(l\)层结点”.
\end{definition}

由定义有,根结点是第\(0\)层结点,根结点的子结点(如若存在)就是第\(1\)层结点,以此类推.

\begin{definition}
%@see: 《离散数学》(邓辉文) P204
设\(T\)是根树,结点\(u,v,u',v' \in T\),
\(u'\)是\(u\)的父结点,\(v'\)是\(v\)的父结点.
如果\(u'\)的层次等于\(v'\)的层次,
则称“\(u\)是\(v\)的\DefineConcept{堂兄弟结点}(cousin)”.
\end{definition}

\begin{definition}
%@see: 《离散数学》(邓辉文) P204
设\(T\)是根树.
\(T\)中各个结点的层次的最大值,
称为“\(T\)的\DefineConcept{深度}(depth)”
或“\(T\)的\DefineConcept{高度}(height)”.
\end{definition}

\begin{definition}
%@see: 《离散数学》(邓辉文) P204
设\(T\)是根树.
\begin{itemize}
	\item \(T\)的所有叶子结点的层次之和,称为\(T\)的\DefineConcept{外部路径长度}.
	\item \(T\)的所有内部结点的层次之和,称为\(T\)的\DefineConcept{内部路径长度}.
\end{itemize}
\end{definition}

\begin{definition}
%@see: 《离散数学》(邓辉文) P204 定义7-8
设\(G = (V,E)\)是一棵根树,结点\(v \in V\).
由结点\(v\)及其所有后代导出的子图,
称为“\(G\)的\DefineConcept{子根树}(rooted subtree)”,
简称为\DefineConcept{子树}(subtree).
\end{definition}

\subsection{多叉树}
\begin{definition}
%@see: 《离散数学》(邓辉文) P204 定义7-9
设\(G = (V,E)\)是一棵根树.
若\(\max_{v \in V} \degEx v = m\),
则称“\(G\)是一棵 \DefineConcept{\(m\)叉树}”.
\end{definition}

\begin{definition}
%@see: 《离散数学》(邓辉文) P204
设\(T\)是\(m\)叉树.
\begin{itemize}
	\item 若对于任意结点\(v\),均有\(\degEx v = m\)或\(\degEx v = 0\),
	则称“\(G\)是\DefineConcept{完全\(m\)叉树}”.

	\item 所有叶子结点的层次都相同的完全\(m\)叉树,
	称为\DefineConcept{正则\(m\)叉树}或\DefineConcept{满\(m\)叉树}.
\end{itemize}
\end{definition}

\begin{property}
%@see: 《离散数学》(邓辉文) P204 性质1
\(m\)叉树的第\(i\)层至多有\(m^i\)个结点.
%TODO proof
\end{property}

\begin{property}
%@see: 《离散数学》(邓辉文) P204 性质2
高度为\(h\)的\(m\ (m\geq2)\)叉树至多有\(\frac{m^{h+1}-1}{m-1}\)个结点.
%TODO proof
\end{property}

\begin{property}
%@see: 《离散数学》(邓辉文) P204 性质3
一棵有\(l\)个叶子结点的\(m\)叉树的高度至少为\(\log_m l\).
%TODO proof
\end{property}

\begin{property}
%@see: 《离散数学》(邓辉文) P204 性质4
若一棵完全\(m\)叉树有\(l\)个叶子结点、\(t\)个分支结点,
则\begin{equation*}
	(m-1)t=l-1.
\end{equation*}
%TODO proof
\end{property}

\begin{property}
%@see: 《离散数学》(邓辉文) P204 性质5
若二叉树有\(l\)个叶子结点,则出度为\(2\)的结点有\(l-1\)个.
%TODO proof
\end{property}

\begin{property}
%@see: 《离散数学》(邓辉文) P204 性质6
有\(l\)个叶子结点的完全二叉树有\(2l-1\)个结点.
%TODO proof
\end{property}

\begin{property}
%@see: 《离散数学》(邓辉文) P204 性质7
若完全二叉树有\(t\)个分支结点,
则它的外部路径长度\(E\)与它的内部路径长度\(I\)满足\begin{equation*}
	E = I + 2t.
\end{equation*}
%TODO proof
\end{property}

\subsection{叶赋权多叉树,哈夫曼树}
\begin{definition}
%@see: 《离散数学》(邓辉文) P205 定义7-10
设\(G = (V,E)\)是一棵\(m\)叉树,
映射\(\omega\colon W \to \mathbb{R}\),
其中\(W\)是\(G\)的全体叶子结点,
则称“\((G,\omega)\)是\DefineConcept{叶赋权\(m\)叉树}”,
把\(\omega\)称为“\(G\)的\DefineConcept{叶子赋权映射}”.
\end{definition}

叶赋权\(m\)叉树给它的每一个叶子结点都赋予了一个非负实数权.

\begin{definition}
%@see: 《离散数学》(邓辉文) P205 定义7-11
设\(G = (V,E)\)是一棵叶赋权\(m\)叉树,
其全部\(l\)个叶子结点上的权分别是\begin{equation*}
	w_i
	\quad(i=1,2,\dotsc,l).
\end{equation*}
将从根结点到权为\(w_i\)的叶子结点的路径的路长
分别记为\begin{equation*}
	L(w_i)
	\quad(i=1,2,\dotsc,l).
\end{equation*}
把\begin{equation*}
	\sum_{i=1}^l w_i \cdot L(w_i)
\end{equation*}称为“\(G\)的\DefineConcept{权}”,
记为\(W(G)\).
对于任意一个叶子结点\(v \in W\),
把\(\omega(v)\)称为“\(v\)的\DefineConcept{权}(weight)”.
\end{definition}

\begin{definition}
%@see: 《离散数学》(邓辉文) P206 定义7-12
设\begin{equation*}
	\mathcal{T}
	\defeq
	\Set{
		T \given \text{$T$是一棵叶赋权二叉树,其$l$个叶子结点上的权分别是$\AutoTuple{w}{l}$}
	},
\end{equation*}
把\(\mathcal{T}\)中权最小的那棵叶赋权二叉树
称为\DefineConcept{最优二叉树}
或\DefineConcept{哈夫曼树}(Huffman tree).
\end{definition}

\subsection{有序树}
在根树的定义中,同一个结点的不同子结点之间是没有先后顺序的,这与现实生活中的家族树不一样.
当我们需要考虑兄弟结点之间的先后顺序时,我们就要用到“有序树”了.
\begin{definition}
%@see: 《离散数学》(邓辉文) P207 定义7-13
设\(G = (V,E)\)是一棵根树.
若对每一个结点的所有子结点都规定了先后顺序,
则称“\(G\)是一棵\DefineConcept{有序树}(ordered tree)”.
\end{definition}

\subsection{定位二叉树}
\begin{definition}
%@see: 《离散数学》(邓辉文) P207 定义7-14
设\(G = (V,E)\)是一棵有序二叉树.
若对每一个结点的所有儿子结点规定一个左右位置,
则称“\(G\)是一棵\DefineConcept{定位二叉树}(positional binary tree)”.
\end{definition}
