\section{平面图}
\subsection{平面图的定义}
%@see: 《离散数学》(邓辉文) P211 定义7-16
设\(G\)是无向图.
如果可以将\(G\)画在一个平面上,
同时使得任意两条相异边在非端点处不相交,
则称“\(G\)是一个\DefineConcept{平面图}(planar graph)”.

设\(G\)是平面图,
\(G\)的图形表示如果可以画在一个平面上,
同时使得任意两条相异边在非端点处不相交,
则称其为“\(G\)的\DefineConcept{平面嵌入}(planar embedding)”
或“\(G\)的\DefineConcept{平面表示}”.

\begin{definition}
%@see: 《离散数学》(邓辉文) P211 定义7-17
设\(G\)是简单平面图.
若在\(G\)的任意两个不相邻的顶点\(u\)和\(v\)之间增加一条边,
所得到的图\(G+uv\)是非平面图,
则称“\(G\)是一个\DefineConcept{极大平面图}(maximal planar graph)”.
\end{definition}

\begin{definition}
%@see: 《离散数学》(邓辉文) P211 定义7-18
设\(G\)是非平面图.
若在\(G\)中任意删除一条边,所得到的图是平面图,
则称“\(G\)是一个\DefineConcept{极小非平面图}(minimal nonplanar graph)”.
\end{definition}

\begin{definition}
%@see: 《离散数学》(邓辉文) P211 定义7-19
设\(G\)是平面图.
由\(G\)的若干条边所围成的连通\footnote{
	一个区域是连通的,是指“在其内部可以随意走动而不必穿过任何一条边”.
}区域,
%TODO 这个定义似乎很不严谨,需要忖度一下如何重新表述
称为“\(G\)的一个\DefineConcept{面}(face)”,
围成面的回路称为这个面的\DefineConcept{边界}(boundary).
\end{definition}
\begin{remark}
任何一个平面图都有一个由若干条边向外围成的一个面,它是唯一一个无限面.
\end{remark}

平面图的两个面相邻,是指这两个面有公共边界.

\subsection{欧拉公式}
\begin{theorem}
%@see: 《离散数学》(邓辉文) P212 定理7-10(欧拉公式)
任意\((n,m)\)连通无向图的面数\(r\)满足\begin{equation}
	r = m - n + 2.
\end{equation}
%TODO proof
\end{theorem}
\begin{remark}
在欧拉公式中,“连通”这一条件是必不可少的.
\end{remark}

\begin{corollary}
%@see: 《离散数学》(邓辉文) P212 推论1
对于任意\((n,m)\)简单连通平面图\(G\),有\[
	n\geq3 \implies m\leq3n-6.
\]
%TODO proof
\end{corollary}

\begin{example}
%@see: 《离散数学》(邓辉文) P212 例7-16
证明:\(K_5\)不是平面图.
%TODO proof
\end{example}

\begin{corollary}
%@see: 《离散数学》(邓辉文) P212 推论2
对于任意\((n,m)\)简单连通平面图\(G\),
若\(G\)不含\(K_3\)子图,且\(n\geq3\),
则\(m\leq2n-4\).
%TODO proof
\end{corollary}

\begin{example}
%@see: 《离散数学》(邓辉文) P212 例7-17
证明:\(K_{3,3}\)不是平面图.
%TODO proof
\end{example}

\begin{theorem}
%@see: 《离散数学》(邓辉文) P212 定理7-11
任何简单平面图必有一个度数小于等于\(5\)的顶点.
%TODO proof
\end{theorem}

\subsection{库拉托夫斯基定理}
波兰数学家库拉托夫斯基于1930年给出了判定平面的充分必要条件.

\begin{theorem}
%@see: 《离散数学》(邓辉文) P213 定理7-12(Kuratowski, 1930)
无向图\(G\)是平面图的充分必要条件是:
\(G\)没有同胚于\(K_5\)和\(K_{3,3}\)的子图.
%TODO proof
\end{theorem}

\begin{example}
%@see: 《离散数学》(邓辉文) P213
证明:彼得森图不是平面图.
%TODO proof
\end{example}

\subsection{平面图的对偶图}
对平面图的面的研究,可以转换为对其对偶图的顶点的研究.
\begin{definition}
%@see: 《离散数学》(邓辉文) P213 定义7-21
设\(G\)是平面图.
\(G\)的\DefineConcept{对偶图}(记作\(G^*\))构造如下:\begin{itemize}
	\item 在\(G\)的每一个面内取一个点,作为\(G^*\)的顶点;
	\item 若\(G\)内的两个面有公共边界\(e\),
	则在\(G^*\)中将这两个面对应的顶点连接起来得到一条边\(e^*\),
	使得\(e^*\)与\(e\)相交一次而与\(G\)的其余边不相交.
	若\(e\)是\(G\)的桥,
	则\(e^*\)是\(G^*\)的吊环.
	若\(e\)是\(G\)的吊环,
	则\(e^*\)是\(G^*\)的桥.
%TODO 什么是吊环???未定义!
\end{itemize}
\end{definition}

%@see: 《离散数学》(邓辉文) P214
根据定义可知,任意平面图的对偶图都是连通平面图.

%@see: 《离散数学》(邓辉文) P214
设\(G\)是\((n,m)\)平面图,有\(r\)个面,则\(G^*\)是\((r,m)\)平面图,有\(n\)个面.

%@see: 《离散数学》(邓辉文) P214
连通平面图\(G\)的对偶图\(G^*\)的对偶图\(G^{**}\)就是\(G\)本身.
非联通平面图\(G\)的对偶图\(G^*\)的对偶图\(G^{**}\)可能与\(G\)不同构.
