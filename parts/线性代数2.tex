\chapter{多项式环}
\section{多项式}
\subsection{贝祖定理}
\begin{theorem}
设\(a,b\in\mathbb{Z}\),则\(a\)与\(b\)互素的充要条件是:
存在\(u,v\in\mathbb{Z}\),使得\[
	u a + v b = 1.
\]
\end{theorem}

\begin{corollary}
设\(f(x)\)和\(g(x)\)是\(\mathbb{P}[x]\)中两个不全为0的多项式,
则\(f(x)\)与\(g(x)\)互素(即\(f(x)\)与\(g(x)\)在\(\mathbb{C}\)上没有公共根)的充要条件是:
存在\(u(x),v(x)\in\mathbb{P}[x]\),使得\[
	u(x) f(x) + v(x) g(x) = 1.
\]
\end{corollary}

\begin{example}
设矩阵\(\A\)满足\(\A^3+\E=2\A\),其中\(\E\)是单位矩阵,
证明:\(2\A^2+\A-\E\)可逆.
\begin{proof}
令\(f(x)=x^3-2x+1\),\(g(x)=2x^2+x-1\),
因式分解可得\[
	f(x) = (x-1)(x^2+x-1),
\]\[
	g(x) = (2x-1)(x+1).
\]
显然\(f(x)\)与\(g(x)\)在\(\mathbb{C}\)上没有公共根,互素.
故根据贝祖定理,存在\(u(x),v(x)\in\mathbb{P}[x]\),使得\[
u(x) \cdot (x^3-2x+1) + v(x) \cdot (2x^2+x-1) = 1,
\]代入矩阵\(\A\),并注意到\(\A^3-2\A+\E=\z\),得到\[
v(\A) \cdot (2\A^2+\A-\E) = \E,
\]也就是说,矩阵\(2\A^2+\A-\E\)可逆,其逆矩阵为\(v(\A)\),而\(v(\A)\)可以通过辗转相除法得到.
\end{proof}
\end{example}

\begin{example}
设\(\A\)是数域\(\mathbb{P}\)上的\(n\)阶方阵,证明:若\(\A^2=\E\),则\[
\rank(\A+\E)+\rank(\A-\E)=n.
\]
\begin{proof}
由于\(x+1\)与\(x-1\)互素,存在\(u(x),v(x)\in\mathbb{P}[x]\),使得\[
u(x) \cdot (x+1) + v(x) \cdot (x-1) = 1.
\]代入矩阵\(\A\),得\[
u(\A) (\A+\E) + v(\A) (\A-\E) = \E.
\]

考虑\(2n\)阶方阵\begin{align*}
\begin{bmatrix}
\A+\E & \z \\
\z & \A-\E
\end{bmatrix}
&\xlongrightarrow{\text{(2列)}+=u(\A)\times\text{(1列)}} \begin{bmatrix}
\A+\E & u(\A) (\A+\E) \\
\z & \A-\E
\end{bmatrix} \\
&\xlongrightarrow{\text{(1行)}+=v(\A)\times\text{(2行)}} \begin{bmatrix}
\A+\E & \E \\
\z & \A-\E
\end{bmatrix} \\
&\to \begin{bmatrix}
\z & \E \\
\A^2-\E & \z
\end{bmatrix} = \begin{bmatrix}
\z & \E \\
\z & \z
\end{bmatrix}.
\end{align*}
于是\(\rank(\A+\E)+\rank(\A-\E)=n\).
\end{proof}
\end{example}

\section{整除性,带余除法}
从一元多项式环的通用性质看到,
我们应当尽可能多地得到\(K[x]\)中有关加法和乘法的等式,
为此需要研究一元多项式环\(K[x]\)的结构.
从本节开始我们将主要研究\(K[x]\)的结构,其中\(K\)是任一数域.

观察\(K[x]\)中两个多项式\(f(x)\)与\(g(x)\)之间有什么关系:\[
	f(x)=x^2-1, \qquad
	g(x)=x-1.
\]
显然,\[
	f(x)=(x+1) g(x).
\]
由此我们抽象出“整除”的概念.

\begin{definition}
%@see: 《高等代数(第三版 下册)》(丘维声) P10. 定义1
设\(f,g \in K[x]\).
如果存在\(h \in K[x]\),使得\[
	f(x) = h(x) g(x),
\]
则称“\(g(x)\) \DefineConcept{整除} \(f(x)\)”,
记作\(g(x) \vert f(x)\),
又称“\(g(x)\)是\(f(x)\)的\DefineConcept{因式}”
“\(f(x)\)是\(g(x)\)的\DefineConcept{倍式}”;
否则称“\(g(x)\)不能整除\(f(x)\)”.
\end{definition}

容易看出下列事实:
\begin{enumerate}
	\item \(0 \vert f(x) \iff f(x) = 0\).
	\item \((\forall f \in K[x])[f(x) \vert 0]\).
	\item \((\forall b \in K - \{0\})(\forall f \in K[x])[b \vert f(x)]\).
\end{enumerate}

\begin{example}
证明:整除关系具有传递性,即在\(K[x]\)中,\[
	f(x) \vert g(x) \land g(x) \vert h(x)
	\implies
	f(x) \vert h(x).
\]
%TODO
\end{example}

\begin{definition}
%@see: 《高等代数(第三版 下册)》(丘维声) P10. 定义2
在\(K[x]\)中,如果\(f(x) \vert g(x)\)且\(g(x) \vert f(x)\),
则称“\(f(x)\)与\(g(x)\) \DefineConcept{相伴}”,
记作\(f(x) \sim g(x)\).
\end{definition}

\begin{proposition}
%@see: 《高等代数(第三版 下册)》(丘维声) P10. 命题1
在\(K[x]\)中,\(f(x) \sim g(x)\)当且仅当存在\(c \in K-\{0\}\),使得\[
	f(x) = c g(x).
\]
\end{proposition}

\begin{proposition}
%@see: 《高等代数(第三版 下册)》(丘维声) P10. 命题2
在\(K[x]\)中,如果\(g(x) \vert f_i(x)\ (i=1,2,\dotsc,s)\),
则对于任意\(u_i \in K[x]\ (i=1,2,\dotsc,s)\),有\[
	g(x) \vert (u_1(x) f_1(x) + u_2(x) f_2(x) + \dotsb + u_s(x) f_s(x)).
\]
\end{proposition}

\begin{theorem}\label{theorem:多项式.带余除法}
%@see: 《高等代数(第三版 下册)》(丘维声) P11. 定理3
对于\(K[x]\)中任意两个多项式\(f(x)\)与\(g(x)\),其中\(g(x)\neq0\),
则在\(K[x]\)中存在唯一的一对多项式\(h(x),r(x)\),使得\[
	f(x)=h(x) g(x) + r(x)
	\land
	\deg r(x) < \deg g(x).
\]
\end{theorem}

\cref{theorem:多项式.带余除法} 中的\(h(x)\)称为“\(g(x)\)除\(f(x)\)的\DefineConcept{商式}”,
\(r(x)\)称为“\(g(x)\)除\(f(x)\)的\DefineConcept{余式}”.

\begin{corollary}
%@see: 《高等代数(第三版 下册)》(丘维声) P12. 推论4
设\(f,g \in K[x]\),且\(g(x) \neq 0\),
则\(g(x) \vert f(x)\)当且仅当\(g(x)\)除\(f(x)\)的余式为零.
\end{corollary}

利用带余除法可以证明:
对于\(K[x]\)中的多项式\(f(x),g(x)\),
如果在\(K[x]\)中,\(g(x)\)不能整除\(f(x)\),
那么把数域\(K\)扩大成数域\(F\)后,
在\(F[x]\)中,\(g(x)\)仍然不能整除\(f(x)\).

\begin{proposition}\label{theorem:多项式.整除性不随数域的扩大而改变}
%@see: 《高等代数(第三版 下册)》(丘维声) P12. 命题5
设\(F,K\)都是数域,且\(F \supseteq K\).
如果\(f,g \in K[x]\),那么\[
	\text{在\(K[x]\)中成立\(g(x) \vert f(x)\)}
	\iff
	\text{在\(F[x]\)中成立\(g(x) \vert f(x)\)}.
\]
\end{proposition}

\cref{theorem:多项式.整除性不随数域的扩大而改变} 表明,整除性不随数域的扩大而改变.

\begin{example}
设\(f(x) = 2x^3+5x^2+5\),
\(g(x) = x^2+2x-1\),
求用\(g(x)\)除\(f(x)\)的商式与余式.
\begin{solution}
我们可以参考整数除法的竖式,作出如下计算:
\[
	\begin{array}{r|*4r|l}
		x^2+2x-1 &
		2x^3 & +3x^2 & & +5
		& 2x-1 \\
		& 2x^3 & +4x^2 & -2x & \\ \cline{2-5}
		& & -x^2 & +2x & +5 \\
		& & -x^2 & -2x & +1 \\ \cline{3-5}
		& & & 4x & +4
	\end{array}
\]
因此\[
	2x^3+3x^2+5=(2x-1)(x^2+2x-1)+(4x+4),
\]
即\(g(x)\)除\(f(x)\)的商式是\(2x-1\),余式是\(4x+4\).
\end{solution}
\end{example}

\section{最大公因式}
从上一节知道,数域\(K\)上的一元多项式环\(K[x]\)具有带余除法,这是\(K[x]\)的一个重要性质.
这一节我们要由此出发推导出\(K[x]\)的另一个重要性质:
\(K[x]\)中任何两个多项式都有最大公因式,
并且\(f(x)\)与\(g(x)\)的最大公因式可以表成\(f(x)\)与\(g(x)\)的倍式和.

在\(K[x]\)中,如果\(c(x)\)既是\(f(x)\)的因式,又是\(g(x)\)的因式,
则称“\(c(x)\)是\(f(x)\)与\(g(x)\)的一个\DefineConcept{公因式}”.

\section{最小公倍式}
\begin{definition}
在\(K[x]\)中,如果\(c(x)\)既是\(f(x)\)的倍式,又是\(g(x)\)的倍式,
则称“\(c(x)\)是\(f(x)\)与\(g(x)\)的一个\DefineConcept{公倍式}”.
\end{definition}

\begin{definition}
%@see: 《高等代数(第三版 下册)》(丘维声) P22 习题7.3 10.
设\(f(x),g(x) \in K[x]\),
\(m(x)\)是\(f(x)\)与\(g(x)\)的一个公倍式.
如果\(f(x)\)与\(g(x)\)的任一公倍式都是\(m(x)\)的倍式,
则称“\(m(x)\)是\(f(x)\)与\(g(x)\)的一个\DefineConcept{最小公倍式}”.
\end{definition}

我们约定,用\[
	[f(x), g(x)]
\]表示首项系数是\(1\)的那个最小公倍式.

\begin{example}
%@see: 《高等代数(第三版 下册)》(丘维声) P22 习题7.3 10.(1)
%@see: 《高等代数创新教材(下册)》(丘维声) P35 例10(1)
证明:\(K[x]\)中任意两个多项式都有最小公倍式,
并且在相伴的意义下是唯一的.
%TODO proof
\begin{proof}
首先,零多项式的倍式只有零多项式,
从而任一多项式\(f\)与\(0\)的最小公倍式是\(0\).

下面设\(f,g\)都是数域\(K\)上的非零多项式.
令\(d=(f,g)\),
则存在\(u,v \in K[x]\)
使得\(f=ud,g=vd\).
又令\(m=uvd\),
显然\(f \mid m\)且\(g \mid m\),
也就是说\(m\)是\(f\)与\(g\)的一个公倍式.
设\(c \in K[x]\)是\(f\)与\(g\)的一个公倍式,
则存在\(p,q \in K[x]\)
使得\(c=pf,c=qg\).
于是\(pf=qg\),
从而\(pud=qvd\),
消去\(d\)便得\(pu=qv\),
可见\(u \mid qv\);
但是由\cref{example:最大公因式.最大公因式除多项式的商式互素}
有\((u,v)=1\);
因此根据\cref{theorem:多项式.互素.性质1}
必有\(u \mid q\),
从而存在\(h \in K[x]\)
使得\(q=hu\).
于是\(c=hug=hm\),
因此\(m \mid c\),
也就是说\(m\)是\(f\)与\(g\)的最小公倍式.

假设\(m_1,m_2\)都是\(f\)与\(g\)的最小公倍式,
则\(m_1 \mid m_2,
m_2 \mid m_1\),
因此\(m_1 \sim m_2\).
\end{proof}
\end{example}

\begin{example}
%@see: 《高等代数(第三版 下册)》(丘维声) P22 习题7.3 10.(2)
%@see: 《高等代数创新教材(下册)》(丘维声) P35 例10(2)
证明:如果\(f(x),g(x)\)的首项系数都是\(1\),
则\[
	[f(x),g(x)]
	= \frac{f(x) g(x)}{(f(x),g(x))}.
\]
%TODO proof
\end{example}

\section{不可约多项式,唯一因式分解定理}
我们已经知道,数域\(K\)上的一元多项式环\(K[x]\)具有带余除法.
由此推导出,\(K[x]\)中任意两个多项式都有最大公因式.
现在我们利用这些结论来研究\(K[x]\)的结构.
与整数环\(\mathbb{Z}\)类比:
每一个正整数都能表示成有限多个素数的乘积.
我们不禁发问:\(K[x]\)中每一个多项式是否能表示成有限多个具有类似“素数”那样的性质的多项式的乘积?
联系我们对素数的定义,
对于一个大于\(1\)的正整数\(p\),如果它的正因子只有\(1\)和\(p\),那么称其为素数.
我们可以给出如下概念:
\begin{definition}
%@see: 《高等代数(第三版 下册)》(丘维声) P24. 定义1
\(K[x]\)中一个次数大于零的多项式\(p(x)\),
如果它在\(K[x]\)中的因式只有零次多项式和\(p(x)\)的相伴元,
则称“\(p(x)\)是数域\(K\)上的一个\DefineConcept{不可约多项式}”;
否则称“\(p(x)\)是数域\(K\)上的一个\DefineConcept{可约多项式}”.
\end{definition}

\section{重因式}
上一节我们已证明\(K[x]\)中每一个次数大于零的多项式\(f(x)\)能唯一地分解成
数域\(K\)上有限多个不可约多项式的乘积.
如果\(f(x)\)的分解式中每一个不可约因式只出现\(1\)次,
这种情形是特别重要的情形.
这一节我们要给出识别这种情形的一个统一的方法.

\begin{definition}
%@see: 《高等代数(第三版 下册)》(丘维声) P29 定义1
设\(f(x),p(x) \in K[x]\).
如果\begin{enumerate}
	\item \(p(x)\)是不可约多项式,
	\item \(p^k(x) \mid f(x)\),
	\item \(p^{k+1}(x) \nmid f(x)\),
\end{enumerate}
那么称“\(p(x)\)是\(f(x)\)的~\DefineConcept{\(k\)重因式}”.

如果\(k=0\),则\(p(x) \nmid f(x)\),因此\(p(x)\)不是\(f(x)\)的因式.
如果\(k=1\),则把\(p(x)\)称为“\(f(x)\)的\DefineConcept{单因式}”.
如果\(k>1\),则把\(p(x)\)称为“\(f(x)\)的\DefineConcept{重因式}”.
\end{definition}

显然,如果\(f(x)\)的标准分解式为\[
	f(x) = c p_1^{r_1}(x) p_2^{r_2}(x) \dotsm p_m^{r_m}(x),
\]
则\(p_i^{r_i}(x)\ (i=1,2,\dotsc,m)\)是\(f(x)\)的\(r_i\)重因式.
指数\(r_i = 1\)的那些不可约因式是单因式,
指数\(r_i > 1\)的那些不可约因式是重因式.
因此,\(f(x)\)的分解式中每一个不可约因式只出现\(1\)的情形也就是\(f(x)\)没有重因式的情形.
如何判别一个多项式有没有重因式呢?
由于没有一般的方法来求一个多项式的标准分解式,
因此我们必须寻找别的方法来判断一个多项式有没有重因式.

我们先来看一个简单例子,以便从中受到启发.

设\(f(x) = (x+1)^3 \in \mathbb{R}[x]\),
这时\(f(x)\)有重因式.
如果我们把\(f(x)\)看成数学分析中讨论的多项式函数,
那么对\(f(x)\)可以求导数,得\(f'(x) = 3(x+1)^2\).
于是\((f(x),f'(x)) = (x+1)^2\).
从这个例子受到启发,
有可能运用导数概念以及最大公因式的求法来讨论一个多项式有没有重因式的问题.
由于我们现在讲的多项式是任意数域\(K\)上一个不定元的多项式,
而数学分析中的多项式函数是实变量\(x\)的函数,
其导数概念涉及极限概念,
因此我们不能直接引用数学分析中多项式函数的导数概念,
我们必须给任意数域\(K\)上一元多项式的导数下个定义,
当然这个定义是从数学分析中多项式函数的导数公式得到启发的.

\begin{definition}\label{definition:多项式.导数}
%@see: 《高等代数(第三版 下册)》(丘维声) P30 定义2
对于\(K[x]\)中的多项式\[
	f(x) = a_n x^n + a_{n-1} x^{n-1} + \dotsb + a_1 x + a_0,
\]
我们把\(K[x]\)中的多项式\[
	n a_n x^{n-1} + (n-1) a_{n-1} x^{n-2} + \dotsb + a_1
\]
叫做“\(f(x)\)的\DefineConcept{一阶导数}”,记作\(f'(x)\).
我们还把\(f'(x)\)的一阶导数称为“\(f(x)\)的\DefineConcept{二阶导数}”,记作\(f''(x)\);
把\(f''(x)\)的一阶导数称为“\(f(x)\)的\DefineConcept{三阶导数}”,记作\(f'''(x)\);
把\(f'''(x)\)的一阶导数称为“\(f(x)\)的\DefineConcept{四阶导数}”,记作\(f^{(4)}(x)\);
以此类推.
\end{definition}

从\cref{definition:多项式.导数} 立即得出,
一个\(n\)次多项式的导数是一个\(n-1\)次多项式,
它的\(n\)阶导数是\(K\)中一个非零数,
它的\(n+1\)阶导数等于零.
零多项式的导数是零多项式.

根据\cref{definition:多项式.导数},可以验证得到\(K[x]\)中多项式的导数的基本公式:\begin{gather}
	[f(x)+g(x)]' = f'(x) + g'(x), \\
	[c f(x)]' = c f'(x), \quad c \in K, \\
	[f(x) g(x)]' = f'(x) g(x) + f(x) g'(x), \\
	[f^m(x)]' = m f^{m-1}(x) f'(x).
\end{gather}

让我们回头再看一遍之前举的简单例子,
不可约多项式\(x+1\)是\(f(x) = (x+1)^3\)的\(3\)重因式.
由于按\cref{definition:多项式.导数} 和上述公式可得出,
\(f'(x) = 3(x+1)^2\),
因此\(x+1\)是\(f'(x)\)的\(2\)重因式.
我们从这个例子得出的结论具有一般性.

\begin{theorem}\label{theorem:多项式.多项式及其导数的重因式}
%@see: 《高等代数(第三版 下册)》(丘维声) P30 定理1
设\(K\)是数域,在\(K[x]\)中,
如果不可约多项式\(p(x)\)是\(f(x)\)的一个\(k\ (k\geq1)\)重因式,
则\(p(x)\)是\(f(x)\)的导数\(f'(x)\)的一个\(k-1\)重因式.
特别地,多项式\(f(x)\)的单因式不是\(f(x)\)的导数\(f'(x)\)的因式.
\end{theorem}

\begin{corollary}\label{theorem:多项式.不可约多项式是重因式的充分必要条件}
%@see: 《高等代数(第三版 下册)》(丘维声) P31 推论2
设\(K\)是数域,在\(K[x]\)中,不可约多项式\(p(x)\)是\(f(x)\)的重因式的充分必要条件是:
\(p(x)\)是\(f(x)\)与\(f'(x)\)的公因式.
\end{corollary}
从\cref{theorem:多项式.不可约多项式是重因式的充分必要条件} 立即得到:
\(K[x]\)中次数大于零的多项式\(f(x)\)有重因式的充分必要条件是\(f(x)\)及其导数\(f'(x)\)
有次数大于零的公因式.
于是我们有下述定理.
\begin{theorem}\label{theorem:多项式.高次多项式没有重因式的充分必要条件}
%@see: 《高等代数(第三版 下册)》(丘维声) P31 定理3
设\(K\)是数域,\(K[x]\)中次数大于零的多项式\(f(x)\)没有重因式的充分必要条件是:
\(f(x)\)与它的导数\(f'(x)\)互素.
\end{theorem}

\cref{theorem:多项式.高次多项式没有重因式的充分必要条件} 表明,
判断数域\(K\)上的一个多项式\(f(x)\)有没有重因式,
只要利用辗转相除法去计算最大公因式\((f(x),f'(x))\).
不仅如此,由于在数域扩大时,两个多项式的互素性不改变,一个多项式的导数也不改变,
因此我们还有下述结论.

\begin{proposition}
%@see: 《高等代数(第三版 下册)》(丘维声) P31 命题4
设\(F,K\)都是数域,\(F \supseteq K\).
对于\(f \in K[x]\),
\(f(x)\)在\(K[x]\)中没有重因式的充分必要条件是:
\(f(x)\)有无重因式不会随数域的扩大而改变,
即当把\(f(x)\)看成\(F[x]\)中的多项式时,
\(f(x)\)在\(F[x]\)中没有重因式.
\end{proposition}

在一些问题中,如果多项式\(f(x)\)有重因式,
我们希望求出一个多项式\(g(x)\),
它没有重因式,
并且在不计重数时,它与\(f(x)\)含有完全相同的不可约因式.
下面我们来讨论如何求解\(g(x)\).

设\(K[x]\)中的多项式\(f(x)\)的标准分解式是\[
	f(x) = c p_1^{r_1}(x) p_2^{r_2}(x) \dotsm p_m^{r_m}(x),
\]
根据\cref{theorem:多项式.多项式及其导数的重因式} 得\[
	f'(x) = p_1^{r_1-1}(x) p_2^{r_2-1}(x) \dotsm p_m^{r_m-1}(x) h(x),
\]
其中\(h(x)\)不能被\(p_i(x)\ (i=1,2,\dotsc,m)\)整除.
于是我们可以利用辗转相除法求得最大公因式\[
	(f(x),f'(x))
	= p_1^{r_1-1}(x) p_2^{r_2-1}(x) \dotsm p_m^{r_m-1}(x).
\]
因此用\((f(x),f'(x))\)除\(f(x)\)所得商式是\[
	c p_1(x) p_2(x) \dotsm p_m(x),
\]
把这个商式记作\(g(x)\),
我们便得到一个没有重因式的多项式\(g(x)\),
它与\(f(x)\)含有完全相同的不可约因式(不计重数).

去掉\(f(x)\)的不可约因式的重数有不少好处.
例如,为了求\(f(x)\)的所有不可约因式,
我们可以先用上述方法得到一个没有重因式的多项式\(g(x)\),
它与\(f(x)\)含有完全相同的不可约因式(不计重数),
但由于\(g(x)\)的次数小于\(f(x)\)的次数,
所以\(g(x)\)的不可约因式可能比较容易求得.
如果我们求出了\(g(x)\)的一个不可约因式\(p_i(x)\),
那么用带余除法可求出\(p_i(x)\)在\(f(x)\)中的重数.
又如,在实际问题中常常需要求出一个多项式\(f(x)\)的根,
由于有些求多项式的根的算法只对没有重因式的多项式适用,
因此我们可以先去掉\(f(x)\)的不可约因式的重数,
得到一个没有重因式的多项式\(g(x)\),
而\(g(x)\)与\(f(x)\)有完全相同的根(不计重数).

\begin{example}
证明:\(\mathbb{Q}[x]\)中的多项式\[
	f(x) = 1+x+\frac{x^2}{2!}+\dotsb+\frac{x^n}{n!}
\]没有重因式.
\begin{proof}
求\(f(x)\)的导数得\[
	f'(x) = 1+x+\dotsm+\frac{x^{n-1}}{(n-1)!}.
\]
于是\[
	f(x) = f'(x) + \frac{x^n}{n!}.
\]
那么\[
	(f(x),f'(x))
	= \left(
		f'(x)+\frac{x^n}{n!},
		f'(x)
	\right)
	= \left(
		\frac{x^n}{n!},
		f'(x)
	\right).
\]
由于\(\frac{x^n}{n!}\)的不可约因式只有\(x\)(不计重数),
而\(x \nmid f'(x)\),所以\[
	\left(
		\frac{x^n}{n!},
		f'(x)
	\right)
	= 1,
\]
从而\((f(x),f'(x))=1\).
因此,\(f(x)\)没有重因式.
\end{proof}
\end{example}

\section{多项式的根}
从唯一因式分解定理知道,
\(K[x]\)中每一个次数大于零的多项式都能唯一地分解成数域\(K\)上有限多个不可约多项式的乘积.
由此看出,不可约多项式之于\(K[x]\)正如砖块之于城市,
这促使我们取搞清楚\(K[x]\)中不可约多项式有哪些.
我们已经知道,\(K[x]\)中每一个一次多项式都是不可约的.
于是需要进一步研究的是,
\(K[x]\)中有没有次数大于\(1\)的不可约多项式?
显然,在\(K[x]\)中,如果\(p(x)\)是次数大于\(1\)的不可约多项式,则\(p(x)\)没有一次因式.
从这点受到启发,首先需要研究\(K[x]\)中一个多项式\(f(x)\)有一次因式的充分必要条件.
为此,我们需要用一次多项式去除\(f(x)\),观察它的余式.

\section{实数域上的不可约多项式}
这一节我们要找出实数域上的所有不可约多项式.
由于每一个复数都可以表示成\(a+b\iu\)的形式,
其中\(a,b\)都是实数,
因此我们可以利用复数域上多项式的信息来研究实数域上的不可约多项式.

\begin{theorem}
%@see: 《高等代数(第三版 下册)》(丘维声) P40 定理1
设\(f(x)\)是实系数多项式,
如果\(c\)是\(f(x)\)的一个复根,
则\(c\)的共轭复数\(\overline{c}\)也是\(f(x)\)的一个复根.
\begin{proof}
设\(f(x)=a_n x^n+a_{n-1} x^{n-1}+\dotsb+a_1 x+a_0\),
其中\(a_i\in\mathbb{R}\ (i=0,1,\dotsc,n)\).
因为\(c\)是\(f(x)\)的复根,
所以\[
	f(c)=a_n c^n+a_{n-1} c^{n-1}+\dotsb+a_1 c+a_0=0.
\]
在上式两边取共轭,得\[
	a_n \overline{c}^n+a_{n-1} \overline{c}^{n-1}+\dotsb+a_1 \overline{c}+a_0=0,
\]
因此\(\overline{c}\)是\(f(x)\)的一个复根.
\end{proof}
\end{theorem}

\section{有理数域上的不可约多项式}
这一节讨论有理数域上的不可约多项式有哪些,
如何判别一个有理系数多项式是否不可约.
这些问题的回答比复系数多项式和实系数多项式困难得多.

在\cref{section:多项式.多项式的根}的开头,
我们曾指出,
在\(K[x]\)中,
如果一个次数大于1的多项式\(p(x)\)不可约,
则\(p(x)\)没有一次因式,
从而\(p(x)\)在\(K\)中没有根,
这样就可以缩小讨论\(\mathbb{Q}[x]\)中不可约多项式的范围.
那么如何判别\(\mathbb{Q}[x]\)中次数大于1的多项式\(f(x)\)有没有有理根呢?
显然\(f(x)\)有有理根当且仅当\(f(x)\)在\(\mathbb{Q}[x]\)中的相伴元也有有理根.
因此很自然地选取\(f(x)\)在\(\mathbb{Q}[x]\)中的一个最简单的相伴元来研究.
例如,设\(f(x)=\frac12x^4+\frac13x^3-2x+1\),
则\(g(x)=3x^4+2x^3-12x+6\)就是\(f(x)\)的一个相伴元.
注意到\(g(x)\)是整系数多项式,
而它的各项系数的最大公因数只有\(\pm1\).
受此启发,我们可以给出如下概念.

\begin{definition}
%@see: 《高等代数(第三版 下册)》(丘维声) P42 定义1
一个非零的整系数多项式\[
	g(x)=b_n x^n+\dotsb+b_1 x+b_0,
\]
如果它的各项系数的最大公因数只有\(\pm1\),
则称“\(g(x)\)是\DefineConcept{本原的}”
或“\(g(x)\)是一个\DefineConcept{本原多项式}”.
\end{definition}

\begin{proposition}
任一非零的有理系数多项式都与一个本原多项式相伴.
\begin{proof}
只需求出有理系数多项式\[
	f(x)=a_n x^n+\dotsb+a_1 x+a_0
\]的各项系数的分母的最小公倍数\(m\),
提取公因数\(\frac1m\)得到\[
	f(x)=\frac1m(m a_n x^n+\dotsb+m a_1 x+m a_0);
\]
接着求出括号内多项式的各项系数的最大公因数\(c\),
就有\[
	f(x)=\frac{c}{m}(b_n x^n+\dotsb+b_1 x+b_0),
\]
其中\(b_i=\frac{m a_i}{c}\ (i=0,1,\dotsc,n)\),
\(g(x)=b_n x^n+\dotsb+b_1 x+b_0\)就是与\(f(x)\)相伴的本原多项式.
\end{proof}
\end{proposition}

我们不禁想要知道,
一个非零的有理系数多项式
可以与几个本原多项式相伴?

\begin{lemma}\label{theorem:多项式.有理数域上的不可约多项式.引理1}
%@see: 《高等代数(第三版 下册)》(丘维声) P42 引理1
两个本原多项式\(f(x)\)和\(g(x)\)在\(Q[x]\)中相伴
当且仅当\(f(x)=\pm g(x)\).
\begin{proof}
充分性是显然的.
下面证必要性.
设\(f(x),g(x)\)是相伴的本原多项式,
则存在\(r\in\mathbb{Q}-\{0\}\),
使得\(f(x)=r g(x)\).
设\[
	f(x)=\sum_{i=0}^n a_i x^i, \qquad
	g(x)=\sum_{i=0}^n b_i x^i,
\]
其中\(a_i,b_i\in\mathbb{Z}\ (i=0,1,\dotsc,n)\).
假设\(r\neq\pm1\),
不妨设\(r=\frac{q}{p}\),
其中\((p,q)=1\).
于是\(p,q\)两者中至少有一个不等于\(\pm1\).
不妨设\(p\neq\pm1\),
从而有\(p f(x)=q g(x)\).
比较各项系数可知\(p a_i=q b_i\ (i=0,1,\dotsc,n)\).
于是\(p \mid q b_i\).
因为\((p,q)=1\),
所以根据\cref{theorem:多项式.互素.性质1}
有\(p \mid b_i\),
这与“\(g(x)\)是本原多项式”矛盾.
因此\(r=\pm1\),
\(f(x)=\pm g(x)\).
\end{proof}
\end{lemma}

\cref{theorem:多项式.有理数域上的不可约多项式.引理1}
告诉我们,对于一个非零的有理系数多项式\(f(x)\),
与它在\(\mathbb{Q}[x]\)中相伴的本原多项式有且仅有两个,
它们相差一个正负号.
现在我们来研究本原多项式在\(\mathbb{Q}[x]\)中的不可约性问题.
为此首先介绍本原多项式的一个重要性质.

\begin{lemma}[高斯引理]
%@see: 《高等代数(第三版 下册)》(丘维声) P43 引理2
两个本原多项式的乘积还是本原多项式.
\begin{proof}
设\(
	f(x)=\sum_{i=0}^n a_i x^i,
	g(x)=\sum_{i=0}^n b_i x^i
\)是两个本原多项式,
又设\(
	h(x) = f(x) g(x) = \sum_{i=0}^{n+m} c_i x^i
\),
其中\(c_k=\sum_{i+j=k} a_i b_j\ (k=0,1,\dotsc,n+m)\).

假如\(h(x)\)不是本原多项式,
那么存在一个素数\(p\),
使得\(p\)是\(h(x)\)各项系数的公因式,
即\(p \mid c_k\ (k=0,1,\dotsc,n+m)\).
因为\(f(x)\)是本原的,
所以\(p\)不能同时整除\(f(x)\)的各项系数,
也就是说,存在\(k\ (0\leq k\leq n)\)满足\[
	p \mid a_0,
	p \mid a_1,
	\dotsc
	p \mid a_{k-1},
	p \nmid a_k.
	\eqno(1)
\]
同理,存在\(l\ (0\leq l\leq m)\)满足\[
	p \mid b_0,
	p \mid b_1,
	\dotsc
	p \mid b_{l-1},
	p \nmid b_l.
	\eqno(2)
\]

考虑\(h(x)\)的\(k+l\)次项的系数\[
	c_{k+l}
	= a_{k+l} b_0
	+ a_{k+l-1} b_1
	+ \dotsb
	+ a_1 b_{k+l-1}
	+ a_0 b_{k+l}.
\]
由(1)(2)两式可知\(p \nmid c_{k+l}\)
(注意从\(p \nmid a_k\)且\(p \nmid b_l\)
可以推出\(p \nmid a_k b_l\)),矛盾!
因此\(h(x)\)是本原多项式.
\end{proof}
\end{lemma}

\section{多元多项式环}
\subsection{多元多项式}
\begin{definition}
%@see: 《高等代数(第三版 下册)》(丘维声) P50 定义1
设\(K\)是一个数域,
用不属于\(K\)的\(n\)个符号\(\AutoTuple{x}{n}\)作表达式\[
	\sum_{\AutoTuple{i}{n}}
	a_{i_1 \dotsm i_n}
	x_1^{i_1} \dotsm x_n^{i_n},
\]
其中\(a_{i_1 \dotsm i_n} \in K\),
\(\AutoTuple{i}{n}\)是非负整数,
上式中的每一项称为一个\DefineConcept{单项式},
上式称为\DefineConcept{数域\(K\)上的\(n\)元多项式}.
如果它具有下述性质:
只有有限多个单项式的系数不为零,
并且两个这种形式的表达式相等当且仅当它们除去系数为零的单项式外含有完全相同的单项式,
而系数为零的单项式允许任意删去或添入.
这时,符号\(\AutoTuple{x}{n}\)称为\(n\)个\DefineConcept{无关不定元}.

在数域\(K\)上的\(n\)元多项式中,
如果两个单项式的幂指数都对应相等,
则称这两个单项式为\DefineConcept{同类项}.
我们约定\(n\)元多项式中的单项式都是不同类的,
即要把同类项合并成一项.

如果数域\(K\)上一个\(n\)元多项式的所有系数全为零,
则称它为\DefineConcept{零多项式},记为\(0\).

我们把\(i_1+\dotsb+i_n\)
称为“单项式\(a_{i_1 \dotsm i_n}
x_1^{i_1} \dotsm x_n^{i_n}\)的\DefineConcept{次数}”.

一个\(n\)元多项式\(f(x_1,\dotsc,x_n)\)的系数不为零的单项式的次数的最大值,
称为“\(f(x_1,\dotsc,x_n)\)的\DefineConcept{次数}”.

零多项式的全次数规定为\(-\infty\).
\end{definition}

\begin{example}
\(5x_1^4+3x_1^3x_2+2x_1x_2x_3^2+x_2^3+x_2x_3\)
是3元4次多项式,
其中单项式\(5x_1^4,3x_1^3x_2,2x_1x_2x_3^2\)的次数都是4.
\end{example}

数域\(K\)上所有\(n\)元多项式组成的集合,
记作\(K[x_1,\dotsc,x_n]\).

在\(K[x_1,\dotsc,x_n]\)中定义加法与乘法如下:
\begin{gather}
	\begin{split}
		&\hspace{-20pt}
		\sum_{\AutoTuple{i}{n}}
		a_{i_1 \dotsm i_n}
		x_1^{i_1} \dotsm x_n^{i_n}
		+
		\sum_{\AutoTuple{i}{n}}
		b_{i_1 \dotsm i_n}
		x_1^{i_1} \dotsm x_n^{i_n} \\
		&\defeq
		\sum_{\AutoTuple{i}{n}}
		(a_{i_1 \dotsm i_n} + b_{i_1 \dotsm i_n})
		x_1^{i_1} \dotsm x_n^{i_n},
	\end{split} \\
	\begin{split}
		&\hspace{-20pt}
		\left(
		\sum_{\AutoTuple{i}{n}}
		a_{i_1 \dotsm i_n}
		x_1^{i_1} \dotsm x_n^{i_n}
		\right) \left(
		\sum_{\AutoTuple{j}{n}}
		b_{j_1 \dotsm j_n}
		x_1^{j_1} \dotsm x_n^{j_n}
		\right) \\
		&\defeq
		\sum_{\AutoTuple{s}{n}}
		c_{s_1 \dotsm s_n}
		x_1^{s_1} \dotsm x_n^{s_n},
	\end{split}
\end{gather}
其中\begin{equation}
	c_{s_1 \dotsm s_n}
	= \sum_{i_1+j_1=s_1}
	\sum_{i_2+j_2=s_2}
	\dotso
	\sum_{i_n+j_n=s_n}
	a_{i_1 \dotsm i_n}
	b_{j_1 \dotsm j_n}.
\end{equation}
不难验证\(K[x_1,\dotsc,x_n]\)对于如上定义的加法与乘法成为一个环.
它的零元是零多项式.
它有单位元,即零次多项式\(1\).
它是交换环.
我们把这个环称为\DefineConcept{数域\(K\)上的\(n\)元多项式环}.

显然有\begin{equation}
	\deg(f+g)
	\leq
	\max\{\deg f,\deg g\}.
\end{equation}

先来对\(n\)元多项式\(f(x_1,\dotsc,x_n)\)的各项规定一个排列顺序,
从而给出首项的概念.

每一类单项式\(a_{i_1 \dotsm i_n} x_1^{i_1} \dotsm x_n^{i_n}\)
对应一个\(n\)元有序非负整数组\((i_1,\dotsc,i_n)\),
这个对应是双射.
为了给出各类单项式之间的一个排列顺序的方法,
就只需要对\(n\)元有序非负整数组定义一个先后顺序.

对于两个\(n\)元有序非负整数组
\((i_1,\dotsc,i_n)\)
和\((j_1,\dotsc,j_n)\),
如果\[
	i_1=j_1,
	i_2=j_2,
	\dotsc,
	i_{s-1}=j_{s-1},
	i_s>j_s
	\quad(1\leq s\leq n)
\]
则称“\((i_1,\dotsc,i_n)\)~\DefineConcept{先于}~\((j_1,\dotsc,j_n)\)”,
记作\((i_1,\dotsc,i_n)>(j_1,\dotsc,j_n)\).

由上述定义立即看出,
对于任意两个\(n\)元有序非负整数组
\((i_1,\dotsc,i_n)\)
和\((j_1,\dotsc,j_n)\),
关系\begin{gather*}
	(i_1,\dotsc,i_n)>(j_1,\dotsc,j_n), \\
	(i_1,\dotsc,i_n)=(j_1,\dotsc,j_n), \\
	(j_1,\dotsc,j_n)>(i_1,\dotsc,i_n)
\end{gather*}
中有且仅有一个成立.

这里关系“\(>\)”具有传递性,
即,
如果\((i_1,\dotsc,i_n)>(j_1,\dotsc,j_n)\)
且\((j_1,\dotsc,j_n)>(k_1,\dotsc,k_n)\),
那么\((i_1,\dotsc,i_n)>(k_1,\dotsc,k_n)\).
这是因为\(i_l-k_l=(i_l-j_l)+(j_l-k_l)\).

\begin{example}
由\((4,2,3,3)>(4,2,2,4)\)和\((4,2,2,4)>(4,1,4,3)\)
可得\((4,2,3,3)>(4,1,4,3)\).
\end{example}

这样我们的确给出了\(n\)元有序非负整数组之间的一个顺序.
相应地,\(n\)元各类单项式之间也有了一个先后顺序.
这种排列顺序的方法是模仿字典中单词的排列原则给出的,
因而称之为\DefineConcept{字典排列法}.

\begin{example}
多项式\(2x_1^4x_2x_3+x_1x_2^5x_3+6x_1^3\)
按字典排列法写出来就是
\(2x_1^4x_2x_3+6x_1^3+x_1x_2^5x_3\).
\end{example}

按字典排列法写出来的第一个系数不为零的单项式称为\(n\)元多项式的\DefineConcept{首项}.

\begin{example}
多项式\(2x_1^4x_2x_3+x_1x_2^5x_3+6x_1^3\)的首项
是\(2x_1^4x_2x_3\).
要注意,首项不一定具有最大的次数.
多项式\(2x_1^4x_2x_3+x_1x_2^5x_3+6x_1^3\)的次数是7,
而它的首项的次数是6.
\end{example}

\begin{theorem}\label{theorem:多项式.多元多项式环.两个非零多项式的乘积的首项等于它们的首项的乘积}
%@see: 《高等代数(第三版 下册)》(丘维声) P52 定理1
在\(K[x_1,\dotsc,x_n]\)中两个非零多项式的乘积的首项等于它们的首项的乘积.
\begin{proof}
设\(f(x_1,\dotsc,x_n),g(x_1,\dotsc,x_n)\)是\(K[x_1,\dotsc,x_n]\)中两个非零多项式.
设\(f(x_1,\dotsc,x_n)\)的首项是\(a x_1^{p_1} x_2^{p_2} \dotsm x_n^{p_n}\ (a\neq0)\),
\(g(x_1,\dotsc,x_n)\)的首项是\(b x_1^{q_1} x_2^{q_2} \dotsm x_n^{q_n}\ (b\neq0)\).
为了证明\(fg\)的首项是
\(ab x_1^{p_1+q_1} x_2^{p_2+q_2} \dotsm x_n^{p_n+q_n}\),
只要证明\[
	(p_1+q_1,p_2+q_2,\dotsc,p_n+q_n)
\]先于\(fg\)中其他单项式的幂指数组就行了.
\(fg\)的其他单项式的幂指数组只有三种可能情形:\[
	(p_1+j_1,p_2+j_2,\dotsc,p_n+j_n)
\]或\[
	(i_1+q_1,i_2+q_2,\dotsc,i_n+q_n)
\]或\[
	(i_1+j_1,i_2+j_2,\dotsc,i_n+j_n),
\]
其中\[
	(p_1,p_2,\dotsc,p_n)
	>
	(i_1,i_2,\dotsc,i_n), \qquad
	(q_1,q_2,\dotsc,q_n)
	>
	(j_1,j_2,\dotsc,j_n).
\]
显然有\begin{gather*}
	(p_1+q_1,p_2+q_2,\dotsc,p_n+q_n)
	>
	(i_1+q_1,i_2+q_2,\dotsc,i_n+q_n), \\
	(i_1+q_1,i_2+q_2,\dotsc,i_n+q_n)
	>
	(i_1+j_1,i_2+j_2,\dotsc,i_n+j_n),
\end{gather*}
于是由传递性得\[
	(p_1+q_1,p_2+q_2,\dotsc,p_n+q_n)
	>
	(i_1+j_1,i_2+j_2,\dotsc,i_n+j_n).
\]
这就证明了
\(ab x_1^{p_1+q_1} x_2^{p_2+q_2} \dotsm x_n^{p_n+q_n}\)
不可能与\(fg\)中其他的单项式相消,
而且它先于\(fg\)中其他的单项式,
它就是\(fg\)的首项.
\end{proof}
\end{theorem}

从\cref{theorem:多项式.多元多项式环.两个非零多项式的乘积的首项等于它们的首项的乘积}
可以推得以下三个命题.

\begin{proposition}
%@see: 《高等代数(第三版 下册)》(丘维声) P52 定理1
在\(K[x_1,\dotsc,x_n]\)中两个非零多项式的乘积仍是非零多项式.
\end{proposition}

\begin{proposition}
%@see: 《高等代数(第三版 下册)》(丘维声) P52 定理1
\(K[x_1,\dotsc,x_n]\)是无零因子环.
\end{proposition}

\begin{proposition}
%@see: 《高等代数(第三版 下册)》(丘维声) P52 定理1
在\(K[x_1,\dotsc,x_n]\)中,消去律成立.
\end{proposition}

\begin{corollary}
%@see: 《高等代数(第三版 下册)》(丘维声) P52 推论2
在\(K[x_1,\dotsc,x_n]\)中,
如果\(f_i\neq0\ (i=1,2,\dotsc,m)\),
则\(f_1 f_2 \dotsm f_m\)的首项等于它们的首项的乘积.
\begin{proof}
对\cref{theorem:多项式.多元多项式环.两个非零多项式的乘积的首项等于它们的首项的乘积}
运用数学归纳法可以证得.
\end{proof}
\end{corollary}

\subsection{齐次多项式}
\begin{definition}
%@see: 《高等代数(第三版 下册)》(丘维声) P53 定义2
设\(g(x_1,\dotsc,x_n)\)是数域\(K\)上的\(n\)元多项式.
如果\(g(x_1,\dotsc,x_n)\)的每个系数不为零的单项式都是\(m\)次的,
则称其为~\DefineConcept{\(m\)次齐次多项式}.
\end{definition}

\begin{example}
\(2x_1^4+3x_1^2x_2x_3+x_1x_2x_3^2\)是一个4次齐次多项式.
\end{example}

\begin{proposition}
\(K[x_1,\dotsc,x_n]\)中任意两个齐次多项式的乘积仍是齐次多项式,
它的次数等于这两个多项式的次数的和.
\end{proposition}

对于任何一个\(n\)元多项式\(f(x_1,\dotsc,x_n)\),
如果把\(f\)中所有次数相同的单项式并在一起,
则\(f\)可以唯一地表示成\[
	f(x_1,\dotsc,x_n)
	=\sum_{i=0}^m
	f_i(x_1,\dotsc,x_n),
\]
其中\(m=\deg f\),
\(f_i(x_1,\dotsc,x_n)\)是\(i\)次齐次多项式,
它称为“\(f(x_1,\dotsc,x_n)\)的~\DefineConcept{\(i\)次齐次成分}”.

\begin{theorem}
%@see: 《高等代数(第三版 下册)》(丘维声) P53 定理3
设\(f(x_1,\dotsc,x_n),g(x_1,\dotsc,x_n) \in K[x_1,\dotsc,x_n]\),
则\begin{equation}
	\deg(fg)=\deg f+\deg g.
\end{equation}
\begin{proof}
若\(f,g\)中有一个是零多项式,
则\(\deg(fg)=\deg f+\deg g\)成立.

现在设\(f\neq0,g\neq0,\deg f=m,\deg g=s\),
则\[
	f=f_0+f_1+\dotsb+f_m, \qquad
	g=g_0+g_1+\dotsb+g_s,
\]
其中\(f_i\)是\(f\)的\(i\)次齐次成分,
\(g_j\)是\(g\)的\(j\)次齐次成分,
\(f_m\neq0\),
\(g_s\neq0\).
我们有\[
	fg
	=(f_0 g_0+\dotsb+f_0 g_s)
	+(f_1 g_0+\dotsb+f_1 g_s)
	+\dotsb
	+(f_m g_0+\dotsb+f_m g_s),
\]
其中\(f_i g_j\)是\(i+j\)次齐次多项式.
因为\(f_m\neq0,g_s\neq0\),
所以\(f_m g_s\neq0\).
于是\(f_m g_s\)是\(m+s\)次齐次多项式.
从而有\(\deg(fg)=m+s=\deg f+\deg g\).
\end{proof}
\end{theorem}

\begin{remark}
当\(n>1\)时,
\(K[x_1,\dotsc,x_n]\)中没有带余除法,
但是唯一因式分解定理仍然成立.
\end{remark}

和数域\(K\)上的一元多项式环\(K[x]\)具有通用性质一样,
数域\(K\)上的\(n\)元多项式环\(K[x_1,\dotsc,x_n]\)也具有通用性质:
\begin{theorem}
%@see: 《高等代数(第三版 下册)》(丘维声) P53 定理4
设\(K\)是一个数域,
\(R\)是一个带有单位元\(e\)的交换环,
并且\(R\)有一个子环\(R_1\)(含有\(e\)),
\(\tau\)是\(K\)到\(R_1\)的一个环同构映射,
\(t_1,\dotsc,t_n\)是\(R\)的元素,
令\[
	\sigma_{t_1,\dotsc,t_n}
	\colon
	K[x_1,\dotsc,x_n] \to R,
	f(x_1,\dotsc,x_n) \mapsto f(t_1,\dotsc,t_n),
\]
其中\[
	f(x_1,\dotsc,x_n)
	= \sum_{i_1,\dotsc,i_n}
	a_{i_1 \dotsm i_n}
	x_1^{i_1} \dotsm x_n^{i_n},
\]
而\[
	f(t_1,\dotsc,t_n)
	= \sum_{i_1,\dotsc,i_n}
	\tau(a_{i_1 \dotsm i_n})
	t_1^{i_1} \dotsm t_n^{i_n},
\]
则\(\sigma_{t_1,\dotsc,t_n}\)是\(K[x_1,\dotsc,x_n]\)到\(R\)的一个映射,
它满足\(\sigma_{t_1,\dotsc,t_n}(x_i)=t_i\ (i=1,2,\dotsc,n)\),
且它保持加法、乘法运算,
即\begin{gather*}
	f(x_1,\dotsc,x_n)
	+g(x_1,\dotsc,x_n)
	=h(x_1,\dotsc,x_n), \\
	f(x_1,\dotsc,x_n)
	g(x_1,\dotsc,x_n)
	=p(x_1,\dotsc,x_n),
\end{gather*}
那么\begin{gather*}
	f(t_1,\dotsc,t_n)
	+g(t_1,\dotsc,t_n)
	=h(t_1,\dotsc,t_n), \\
	f(t_1,\dotsc,t_n)
	g(t_1,\dotsc,t_n)
	=p(t_1,\dotsc,t_n).
\end{gather*}
\rm
我们把映射\(\sigma_{t_1,\dotsc,t_n}\)称为
“\(x_1,\dotsc,x_n\)用\(t_1,\dotsc,t_n\)代入”.
\begin{proof}
证明过程参考\cref{theorem:多项式.多项式环的同构映射}.
\end{proof}
\end{theorem}

\(K[x_1,\dotsc,x_n]\)中所有零次多项式
添上零多项式组成的子集是
\(K[x_1,\dotsc,x_n]\)的一个子环,
它与\(K\)是环同构的,
因此\(x_1,\dotsc,x_n\)
可以用\(K[x_1,\dotsc,x_n]\)中任意\(n\)个元素代入,
这种代入是保持加法与乘法运算.

特别重要的一种情形是:
\(x_1,\dotsc,x_n\)
用\(K\)中任意\(n\)个元素
\(c_1,\dotsc,c_n\)代入.
由此我们可引进多元多项式函数的概念.

设\(f(x_1,\dotsc,x_n)\)是数域\(K\)上的一个\(n\)元多项式,
对于\(K\)中任意\(n\)个元素\(c_1,\dotsc,c_n\),
将\(x_1,\dotsc,x_n\)用\(c_1,\dotsc,c_n\)代入,
得\(f(c_1,\dotsc,c_n) \in K\).
于是\(n\)元多项式\(f(x_1,\dotsc,x_n)\)确定了
从\(K^n\)到\(K\)的一个映射
(即\(K\)上的\(n\)元函数),
仍用\(f\)表示,即\[
	f\colon
	K^n \to K,
	(c_1,\dotsc,c_n)
	\mapsto
	f(c_1,\dotsc,c_n).
\]
这种由数域\(K\)上的\(n\)元多项式确定的\(K\)上的\(n\)元函数
称为\DefineConcept{数域\(K\)上的\(n\)元多项式函数}.

对于数域\(K\)上的两个\(n\)元多项式
\(f(x_1,\dotsc,x_n)\)和\(g(x_1,\dotsc,x_n)\),
如果它们相等,
则它们确定的\(n\)元多项式函数\(f\)与\(g\)也相等;
反之亦然.

\begin{lemma}\label{theorem:多项式.多元多项式环.引理1}
%@see: 《高等代数(第三版 下册)》(丘维声) P54 引理1
设\(h(x_1,\dotsc,x_n)\)是数域\(K\)上的一个\(n\)元多项式,
如果\(h(x_1,\dotsc,x_n)\neq0\),
则\(h\)不是零函数.
\begin{proof}
对不定元的数目\(n\)运用数学归纳法.
当\(n=1\)时,
由于数域\(K\)上非零的一元多项式\(h(x)\)给出的函数不是零函数,
因此存在\(c \in K\)使得\(h(c)\neq0\).

假设命题对\(K[x_1,\dotsc,x_{n-1}]\)中的多项式成立,
现在看\(K[x_1,\dotsc,x_n]\)中的多项式\(h(x_1,\dotsc,x_n)\).
把\(h(x_1,\dotsc,x_n)\)写成\[
	h(x_1,\dotsc,x_n)
	=u_0(x_1,\dotsc,x_{n-1})
	+u_1(x_1,\dotsc,x_{n-1})
	+\dotsb
	+u_s(x_1,\dotsc,x_{n-1}) x_n^s,
\]
其中\(u_i(x_1,\dotsc,x_{n-1}) \in K[x_1,\dotsc,x_{n-1}]\ (i=0,1,\dotsc,s)\),
且\(u_s(x_1,\dotsc,x_{n-1})\neq0\).
根据归纳假设,\(u_s\)不是零函数,
因此存在\(c_1,\dotsc,c_{n-1} \in K\)
使得\(u_s(c_1,\dotsc,c_{n-1})\neq0\).
于是一元多项式环\(K[x_n]\)中的多项式\[
	h(c_1,\dotsc,c_{n-1},x_n)
	=u_0(c_1,\dotsc,c_{n-1})
	+u_1(c_1,\dotsc,c_{n-1}) x_n
	+\dotsb
	+u_s(c_1,\dotsc,c_{n-1}) x_n^s
\]是非零多项式,
因此存在\(c_n \in K\),
使得\[
	h(c_1,\dotsc,c_n)
	=u_0(c_1,\dotsc,c_{n-1})
	+u_1(c_1,\dotsc,c_{n-1}) c_n
	+\dotsb
	+u_s(c_1,\dotsc,c_{n-1}) c_n^s
	\neq0.
	\qedhere
\]
\end{proof}
\end{lemma}

\begin{theorem}
%@see: 《高等代数(第三版 下册)》(丘维声) P55 定理5
设\(f(x_1,\dotsc,x_n),g(x_1,\dotsc,x_n) \in K[x_1,\dotsc,x_n]\).
如果多项式\(f\)与\(g\)不相等,
则由它们确定的\(n\)元多项式函数\(f\)与\(g\)也不相等.
\begin{proof}
考虑多项式\(
	h(x_1,\dotsc,x_n)
	=f(x_1,\dotsc,x_n)
	-g(x_1,\dotsc,x_n)
\).
如果多项式\(f\)与\(g\)不相等,
则\(h(x_1,\dotsc,x_n)\neq0\).
根据\cref{theorem:多项式.多元多项式环.引理1},
\(h\)不是零函数,
于是存在\(c_1,\dotsc,c_n \in K\)
使得\(h(c_1,\dotsc,c_n)\neq0\).
\(x_1,\dotsc,x_n\)用\(c_1,\dotsc,c_n\)代入,
用上述式子可以推出\(f(c_1,\dotsc,c_n) \neq g(c_1,\dotsc,c_n)\),
所以映射\(f\)与\(g\)不相等.
\end{proof}
\end{theorem}

我们把数域\(K\)所有\(n\)元多项式函数组成的集合记作\(K_{npol}\),
在这个集合中规定加法与乘法如下:
对于\(\forall(c_1,\dotsc,c_n) \in K^n\)有\begin{gather*}
	(f+g)(c_1,\dotsc,c_n)
	\defeq
	f(c_1,\dotsc,c_n)+g(c_1,\dotsc,c_n), \\
	(fg)(c_1,\dotsc,c_n)
	\defeq
	f(c_1,\dotsc,c_n) g(c_1,\dotsc,c_n).
\end{gather*}
容易验证\(K_{npol}\)是一个环,
我们把它称为\DefineConcept{数域\(K\)上的\(n\)元多项式函数环}.
容易证明:
数域\(K\)上的\(n\)元多项式环\(K[x_1,\dotsc,x_n]\)
与\(K\)上的\(n\)元多项式函数环\(K_{npol}\)是同构的.
因此我们可以把数域\(K\)上的\(n\)元多项式
与\(K\)上的\(n\)元多项式函数等同看待.

设\(f(x_1,\dotsc,x_n) \in K[x_1,\dotsc,x_n]\),
对于\(c_1,\dotsc,c_n \in K\),
如果\(f(c_1,\dotsc,c_n)=0\),
则称“\((c_1,\dotsc,c_n)\)是\(f(x_1,\dotsc,x_n)\)的一个\DefineConcept{零点}”.
当\(K\)取实数域时,
若\(n=2\),
则\(f(x,y)\)的零点组成的集合就是平面上的一条\DefineConcept{代数曲线};
若\(n=3\),
则\(f(x,y,z)\)的零点组成的集合就是空间中的一个\DefineConcept{代数曲面}.
研究数域\(K\)上一组\(n\)元多项式的公共零点组成的集合,
就是代数几何的一个基本内容.

\section{对称多项式}
观察下述三元多项式\(f(x_1,x_2,x_3)\)有什么特点?
\[
	f(x_1,x_2,x_3)
	=x_1^3+x_2^3+x_3^3
	+x_1^2x_2
	+x_1^2x_3
	+x_2^2x_3
	+x_1x_2^2
	+x_1x_3^2
	+x_2x_3^2.
\]
直观上看,
\(x_1,x_2,x_3\)在\(f(x_1,x_2,x_3)\)中的地位是对称的,
即同时有\(x_1^3,x_2^3,x_3^3\)这三项,
且同时有\(x_1^2x_2,
x_1^2x_3,
x_2^2x_3,
x_1x_2^2,
x_1x_3^2,
x_2x_3^2\)这六项.
由此受到启发,
我们来研究具有这种性质的\(n\)元多项式\(f(x_1,\dotsc,x_n)\):
若\(f(x_1,\dotsc,x_n)\)含有一项\(a x_1^{i_1} \dotsm x_n^{i_n}\),
则它也含有一项\(a x_{j_1}^{i_1} \dotsm x_{j_n}^{i_n}\),
其中\(j_1 \dotso j_n\)是任意一个\(n\)元排列.

于是我们抽象出下述概念.
\begin{definition}
%@see: 《高等代数(第三版 下册)》(丘维声) P57 定义1
设\(f(x_1,\dotsc,x_n)\)是数域\(K\)上的一个\(n\)元多项式.
如果对于任意一个\(n\)元排列\(j_1 \dotso j_n\)都有\[
	f(x_{j_1},\dotsc,x_{j_n})
	=f(x_1,\dotsc,x_n),
\]
则称“\(f(x_1,\dotsc,x_n)\)是数域\(K\)上的一个\(n\)元\DefineConcept{对称多项式}”.
\end{definition}

定义表明,
在数域\(K\)上的\(n\)元多项式环\(K[x_1,\dotsc,x_n]\)中,
对于\(f(x_1,\dotsc,x_n)\),
如果任给一个\(n\)元排列\(j_1 \dotso j_n\),
不定元\(x_1,\dotsc,x_n\)用\(x_{j_1},\dotsc,x_{j_n}\)代入,
都有\(f(x_{j_1},\dotsc,x_{j_n})=f(x_1,\dotsc,x_n)\),
那么\(n\)元多项式\(f(x_1,\dotsc,x_n)\)是一个对称多项式.

容易看出,零多项式和零次多项式都是对称多项式.

在\(K[x_1,\dotsc,x_n]\)中,
我们来构造含有项\(x_1\)且项数最少的对称多项式.
由定义可知,
\(x_1+\dotsb+x_n\)就是\(n\)元对称多项式,
把它记作\(\sigma_1(x_1,\dotsc,x_n)\),
即\[
	\sigma_1(x_1,\dotsc,x_n)
	=x_1+\dotsb+x_n.
\]
我们来构造含有项\(x_1x_2\)且项数最少的对称多项式.
令\begin{align*}
	\sigma_2(x_1,\dotsc,x_n)
	&=\begin{array}[t]{l}
		x_1x_2+x_1x_3+\dotsb+x_1x_n \\
		+x_2x_3+\dotsb+x_2x_n
		+\dotsb
		+x_{n-1}x_n
	\end{array} \\
	&=\sum_{1\leq i<j\leq n} x_i x_j,
\end{align*}
则\(\sigma_2(x_1,\dotsc,x_n)\)是\(n\)元对称多项式.
同理,对于\(\forall k\in\{2,\dotsc,n-1\}\),
我们来构造含有项\(x_1 \dotsm x_k\),
且项数最少得对称多项式.
令\[
	\sigma_k(x_1,\dotsc,x_n)
	=\sum_{1\leq j_1<\dotsb<j_k\leq n}
	x_{j_1} \dotsm x_{j_k},
\]
则\(\sigma_k(x_1,\dotsc,x_n)\)是\(n\)元对称多项式.
最后,根据定义有,\[
	\sigma_n(x_1,\dotsc,x_n)
	=x_1 \dotsm x_n
\]是\(n\)元对称多项式.

我们把上述\(n\)个\(n\)元对称多项式
\(\sigma_i(x_1,\dotsc,x_n)\ (i=1,\dotsc,n)\)
统称为\(n\)元\DefineConcept{初等对称多项式}.

下面我们把数域\(K\)上所有\(n\)元对称多项式组成的集合记为\(W\).
我们想要知道\(W\)的结构是怎样的.

\begin{proposition}
%@see: 《高等代数(第三版 下册)》(丘维声) P58 命题1
\(W\)是\(K[x_1,\dotsc,x_n]\)的一个子环.
\end{proposition}

\begin{proposition}
%@see: 《高等代数(第三版 下册)》(丘维声) P59 命题2
设\(f_1,\dotsc,f_n \in W\),
则对\(K[x_1,\dotsc,x_n]\)中任意一个多项式\[
	g(x_1,\dotsc,x_n)
	=\sum_{i_1,\dotsc,i_n}
	b_{i_1 \dotso i_n}
	x_1^{i_1} \dotsm x_n^{i_n},
\]
有\[
	g(f_1,\dotsc,f_n)
	=\sum_{i_1,\dotsc,i_n}
	b_{i_1 \dotso i_n}
	f_1^{i_1} \dotsm f_n^{i_n}
	\in W.
\]
\end{proposition}

\begin{theorem}[对称多项式基本定理]
%@see: 《高等代数(第三版 下册)》(丘维声) P59 定理3
对于\(K[x_1,\dotsc,x_n]\)中任意一个对称多项式\(f(x_1,\dotsc,x_n)\),
都存在\(K[x_1,\dotsc,x_n]\)中唯一的一个多项式\(g(x_1,\dotsc,x_n)\),
使得\(f(x_1,\dotsc,x_n)=g(\sigma_1,\dotsc,\sigma_n)\).
\end{theorem}

\section{模m剩余类环}
\begin{proposition}
%@see: 《高等代数(第三版 下册)》(丘维声) P67 命题1
在\(\mathbb{Z}\)中,
若\(a\equiv b\pmod m,
c\equiv d\pmod m\),
则\[
	a+c\equiv b+d\pmod m, \qquad
	ac\equiv bd\pmod m.
\]
\begin{proof}
由已知条件,
\(m\mid(a-b),
m\mid(c-d)\).
从而\(m\mid[(a-b)+(c-d)]\),
即\(m\mid[(a+c)-(b+d)]\).
因此\(a+c\equiv b+d\pmod m\).

由于\(ac-bd
=ac-bc+bc-bd
=(a-b)c+b(c-d)\),
又有\(m\mid[(a-b)c+b(c-d)]\),
因此\(m\mid(ac-bd)\),
从而\(ac\equiv bd\pmod m\).
\end{proof}
\end{proposition}

\begin{theorem}
%@see: 《高等代数(第三版 下册)》(丘维声) P69 定理2
若\(p\)是素数,
则模\(p\)剩余类环\(\mathbb{Z}_p\)是一个域.
\begin{proof}
已知\(\mathbb{Z}_p\)是一个有单位元\(\overline1\)的交换环.
任取\(\mathbb{Z}_p\)的一个非零元\(\overline{a}\),
其中\(0<a<p\).
于是\(p \nmid a\).
又由于\(p\)是素数,
因此\((p,a)=1\).
于是存在\(u,v\in\mathbb{Z}\),
使得\(up+va=1\).
因此\[
	\overline1
	=\overline{up+va}
	=\overline{up}
	+\overline{va}
	=\overline{u}~\overline{p}
	+\overline{v}~\overline{a}
	=\overline{v}~\overline{a}.
\]
可见\(\overline{a}\)是可逆元.
所以\(\mathbb{Z}_p\)是一个域.
\end{proof}
\end{theorem}

给定素数\(p\),
我们把\(\mathbb{Z}_p\)称为\DefineConcept{模\(p\)剩余类域}.

\begin{theorem}
%@see: 《高等代数(第三版 下册)》(丘维声) P71 习题7.11 2.
若\(p\)是合数,
则模\(p\)剩余类环\(\mathbb{Z}_p\)不是域.
%TODO proof
\end{theorem}

模\(p\)剩余类域\(\mathbb{Z}_p\)与数域\(K\)有以下两个不同点:
\begin{enumerate}
	\item 数域\(K\)是无限域,
	而模\(p\)剩余类域\(\mathbb{Z}_p\)是有限域.

	\item 在\(\mathbb{Z}_p\)中,
	\(p\overline1
	=\overline{p}
	=\overline0\),
	\(l\overline1
	=\overline{l}
	\neq\overline0\ (0<l<p)\).
	在数域\(K\)中,
	有\((\forall n\in\mathbb{N}^*)[n1=n\neq0]\).
\end{enumerate}


\chapter{线性空间}
本章我们将建立一个数学模型 --- 线性空间.
我们将研究线性空间的结构.
它是研究客观世界中线性问题的一个重要理论.
即使对于非线性问题,
经过局部化后,
就可以运用线性空间的理论,
或者用线性空间的理论研究非线性问题的某一侧面.

\section{线性空间的结构}
\subsection{线性空间的概念与性质}
\begin{definition}
%@see: 《高等代数(第三版 下册)》(丘维声) P72 定义1
%@see: 《Linear Algebra and Its Applications (Second Edition)》(Peter D. Lax) P1
设\(V\)是一个非空集合,
\(F\)是一个域.

定义运算\(V\times V\to V,\opair{\alpha,\beta}\mapsto\gamma=\alpha+\beta\),
称之为\DefineConcept{加法}(addition).

定义运算\(F\times V\to V,\opair{k,\alpha}\mapsto\beta=k\alpha\),
称之为\DefineConcept{纯量乘法}(scalar multiplication)\footnote{
	当域\(F\)是一个数域(例如\(\mathbb{Q},\mathbb{R},\mathbb{C}\))时,
	纯量乘法又称为\DefineConcept{数量乘法}.
}.

如果\emph{加法}与\emph{纯量乘法}满足以下八条公理:
\begin{center}
	\begin{minipage}{.8\textwidth}
		\begin{axiom}\label{definition:线性空间.运算法则1}
		%@see: 《Linear Algebra and Its Applications (Second Edition)》(Peter D. Lax) P2 (2)
		% 加法交换律
		\((\forall\alpha,\beta\in V)
		[\alpha+\beta=\beta+\alpha]\).
		\end{axiom}
		\begin{axiom}\label{definition:线性空间.运算法则2}
		%@see: 《Linear Algebra and Its Applications (Second Edition)》(Peter D. Lax) P2 (3)
		% 加法结合律
		\((\forall\alpha,\beta,\gamma\in V)
		[(\alpha+\beta)+\gamma=\alpha+(\beta+\gamma)]\).
		\end{axiom}
		\begin{axiom}\label{definition:线性空间.运算法则3}
		%@see: 《Linear Algebra and Its Applications (Second Edition)》(Peter D. Lax) P2 (4)
		% 零元
		\(0\in V
		\land
		(\forall\alpha \in V)
		[\alpha+0=\alpha]\),
		把\(0\)称为“\(V\)的\DefineConcept{零元}”.
		\end{axiom}
		\begin{axiom}\label{definition:线性空间.运算法则4}
		%@see: 《Linear Algebra and Its Applications (Second Edition)》(Peter D. Lax) P2 (5)
		% 负元
		\((\forall\alpha \in V)
		(\exists\beta \in V)
		[\alpha+\beta=0]\),
		\(\beta\)称为“\(\alpha\)的\DefineConcept{负元}”,
		记作\(-\alpha\).
		\end{axiom}
		\begin{axiom}\label{definition:线性空间.运算法则5}
		%@see: 《Linear Algebra and Its Applications (Second Edition)》(Peter D. Lax) P2 (9)
		% 幺元
		\((\forall\alpha\in V)[1\alpha=\alpha]\),
		其中\(1\)是\(F\)的单位元.
		\end{axiom}
		\begin{axiom}\label{definition:线性空间.运算法则6}
		%@see: 《Linear Algebra and Its Applications (Second Edition)》(Peter D. Lax) P2 (6)
		% 纯量乘法
		\((\forall\alpha\in V)
		(\forall k,l\in F)
		[k(l\alpha)=(kl)\alpha]\).
		\end{axiom}
		\begin{axiom}\label{definition:线性空间.运算法则7}
		%@see: 《Linear Algebra and Its Applications (Second Edition)》(Peter D. Lax) P2 (8)
		% 纯量乘法
		\((\forall\alpha\in V)
		(\forall k,l\in F)
		[(k+l)\alpha=k\alpha+l\alpha]\).
		\end{axiom}
		\begin{axiom}\label{definition:线性空间.运算法则8}
		%@see: 《Linear Algebra and Its Applications (Second Edition)》(Peter D. Lax) P2 (7)
		% 纯量乘法
		\((\forall\alpha,\beta\in V)
		(\forall k\in F)
		[k(\alpha+\beta)=k\alpha+k\beta]\).
		\end{axiom}
	\end{minipage}
\end{center}
则称“\(V\)是域\(F\)上的一个\DefineConcept{线性空间}%
(\(V\) is a \emph{linear space} over field \(F\))”,
把\(V\)中的元素称为\DefineConcept{向量}(vector),
把\(F\)中的元素称为\DefineConcept{标量}(scalar),
把加法与纯量乘法这两种运算统称为\DefineConcept{线性运算}(linear operation).
\end{definition}
\begin{remark}
与我们在初等代数中学习的自然数、整数、有理数、实数的加法、乘法不同,
线性空间的加法、纯量乘法是抽象的,
线性空间的加法可以是映射空间\(V^{V \times V}\)中的任意一个映射,
线性空间的纯量乘法可以是映射空间\(V^{F \times V}\)中的任意一个映射.
另外,向量空间的加法、纯量乘法和域\(F\)的加法、乘法毫无关系.
\end{remark}
\begin{remark}
线性空间\(V\)对加法成群.
\end{remark}

当\(F = \mathbb{R}\)时,把线性空间\(V\)称为\DefineConcept{实线性空间}.

当\(F = \mathbb{C}\)时,把线性空间\(V\)称为\DefineConcept{复线性空间}.

实线性空间与复线性空间,是代数结构完全不同的两个线性空间.

\begin{example}
下面列举一些常见的线性空间.
\begin{itemize}
	\item 只含零元\(0 \in V\)的线性空间\(\{0\}\),
	称为\DefineConcept{零空间}.

	%@see: 《高等代数(第三版 下册)》(丘维声) P73 例4
	\item 复数域\(\mathbb{C}\)
	可以看成是实数域\(\mathbb{R}\)上的一个线性空间,
	其加法是复数的加法,
	其数量乘法是实数与复数的乘法.

	%@see: 《高等代数(第三版 下册)》(丘维声) P73 例5
	\item 任一数域\(K\)都可以看成是自身上的线性空间,
	其加法就是数域\(K\)中的加法,
	其数量乘法就是数域\(K\)中的乘法.

	\item 集合\(\mathbb{R}^{n \times 1}\)关于向量的加法、实数与向量的纯量乘法,构成实线性空间.

	\item 集合\(\mathbb{R}^{s \times n}\)关于矩阵的加法、实数与矩阵的纯量乘法,构成实线性空间.

	%@see: 《高等代数(第三版 下册)》(丘维声) P73 例2
	\item 设\(F\)是一个域,\(X\)是一个非空集合,
	则映射空间\(F^X\)
	对函数的加法\[
		(f+g)(x) \defeq f(x) + g(x),
		\quad f,g \in F^X, x \in X,
	\]
	以及实数与函数的数量乘法\[
		(k f)(x) \defeq k f(x),
		\quad f \in F^X, k \in F, x \in X,
	\]
	成为\(F\)上的一个线性空间.
	\(F^X\)的零元是零函数\[
		0(x) = 0,
		\quad x \in X.
	\]
	%上式等号左边的0表示零函数,等号右边的0表示域\(F\)的零元

	%@see: 《高等代数(第三版 下册)》(丘维声) P73 例3
	\item 数域\(K\)上的一元多项式环\(K[x]\)
	对多项式的加法,以及数与多项式的乘法,
	成为\(K\)上的一个线性空间.
\end{itemize}
\end{example}

上述例子表明,线性空间这一数学模型适用性很广.
从现在开始,我们将从线性空间的定义出发,
作逻辑推理,深入揭示线性空间的性质和结构,
它们对于所有的具体的线性空间都成立.

\begin{property}
%@see: 《高等代数(第三版 下册)》(丘维声) P74
设\(V\)是域\(F\)上的一个线性空间.
\begin{itemize}
	\item \(V\)的零元是唯一的.
	\item \(V\)中每个元素的负元是唯一的.
	%@see: 《Linear Algebra and Its Applications (Second Edition)》(Peter D. Lax) P2 (10)
	\item \((\forall\alpha\in V)[0\alpha=0]\).
	\item \((\forall k\in F)[k0=0]\).
	\item \(k\alpha=0 \implies k=0 \lor \alpha=0\).
	\item \((\forall\alpha\in V)[(-1)\alpha=-\alpha]\).
\end{itemize}
\end{property}

\subsection{线性空间的线性关系}
域\(F\)上的线性空间\(V\)的有限子集,称为“\(V\)中的一个\DefineConcept{向量组}”.

向量组\(A\)的子集,称为“\(A\)的一个\DefineConcept{部分组}”.

%@see: 《高等代数(第三版 下册)》(丘维声) P75
%@see: 《Linear Algebra and Its Applications (Second Edition)》(Peter D. Lax) P4 Definition
设\(\AutoTuple{\alpha}{s}\)是\(V\)中一个向量组,
任给\(F\)中一组元素\(\AutoTuple{k}{s}\),
向量\(k_1\alpha_1+\dotsb+k_s\alpha_s\)
称为“\(\AutoTuple{\alpha}{s}\)的一个\DefineConcept{线性组合}(linear combination)”,
称\(\AutoTuple{k}{s}\)为\DefineConcept{系数}.

%@see: 《高等代数(第三版 下册)》(丘维声) P75
对于\(\beta\in V\),
如果有\(F\)中一组元素\(\AutoTuple{c}{s}\),
使得\(\beta=c_1\alpha_1+\dotsb+c_s\alpha_s\),
则称“\(\beta\)可以由\(\AutoTuple{\alpha}{s}\)~\DefineConcept{线性表出}%
(\(\beta\) can be expressed as a linear combination of \(\AutoTuple{\alpha}{s}\))”.

\begin{definition}
%@see: 《高等代数(第三版 下册)》(丘维声) P75 定义2
%@see: 《Linear Algebra and Its Applications (Second Edition)》(Peter D. Lax) P4 Definition
%@see: 《Linear Algebra and Its Applications (Second Edition)》(Peter D. Lax) P5 Definition
设\(\AutoTuple{\alpha}{s}\ (s\geq1)\)是\(V\)中一个向量组.
如果有\(F\)中不全为零的元素\(\AutoTuple{k}{s}\),
使得\(k_1\alpha_1+\dotsb+k_s\alpha_s=0\),
则称“\(\AutoTuple{\alpha}{s}\)是\DefineConcept{线性相关的}%
(\(\AutoTuple{\alpha}{s}\) are \emph{linearly dependent})”;
否则称“\(\AutoTuple{\alpha}{s}\)是\DefineConcept{线性无关的}%
(\(\AutoTuple{\alpha}{s}\) are \emph{linearly independent})”.
\end{definition}

空向量组\(\emptyset\)是线性无关的.

\begin{definition}
%@see: 《高等代数(第三版 下册)》(丘维声) P75 定义3
设\(W\)是\(V\)的任一无限子集.
如果\(W\)有一个有限子集是线性相关的,
则称“\(W\)是\DefineConcept{线性相关的}%
(\(W\) is \emph{linearly dependent})”;
如果\(W\)的任何有限子集都是线性无关的,
则称“\(W\)是\DefineConcept{线性无关的}%
(\(W\) is \emph{linearly independent})”.
\end{definition}

可以证明,
数域\(K\)上的线性方程组的理论,
和数域\(K\)上的矩阵、行列式理论,
在把数域\(K\)换成任意域\(F\)以后,
仍然成立.
\begin{property}
%@see: 《高等代数(第三版 下册)》(丘维声) P75 例6
%@see: 《高等代数(第三版 下册)》(丘维声) P75 例7
%@see: 《高等代数(第三版 下册)》(丘维声) P75 命题1
%@see: 《高等代数(第三版 下册)》(丘维声) P75 命题2
设\(V\)是域\(F\)上的一个线性空间.
\begin{itemize}
	\item \(\text{$\alpha$线性相关}\iff\alpha=0\).
	\item 包含零向量的向量组一定线性相关.
	\item 基数大于或等于\(2\)的向量组\(W\)线性相关
	当且仅当\(W\)中至少有一个向量可以由其余向量中的有限多个线性表出.
	\item 向量\(\beta\)可以由线性无关向量组\(\AutoTuple{\alpha}{s}\)线性表出的充分必要条件是
	\(\AutoTuple{\alpha}{s},\beta\)线性相关.
\end{itemize}
\end{property}

\begin{definition}
%@see: 《高等代数(第三版 下册)》(丘维声) P76 定义4
设\(W_1,W_2\)都是\(V\)的非空子集,
如果\(W_1\)中每一个向量都可以由\(W_2\)中有限多个向量线性表出,
则称“\(W_1\)可以由\(W_2\)~\DefineConcept{线性表出}”.
如果\(W_1\)与\(W_2\)可以互相线性表出,
则称“\(W_1\)与\(W_2\)是\DefineConcept{等价的}”.
\end{definition}

容易证明,“线性表出”具有传递性,
从而“等价”也具有传递性.
显然,向量组的“等价”具有反身性与对称性.

\begin{property}\label{theorem:线性空间.性质3}
%@see: 《高等代数(第三版 下册)》(丘维声) P76 引理1
%@see: 《高等代数(第三版 下册)》(丘维声) P76 推论3
%@see: 《高等代数(第三版 下册)》(丘维声) P76 推论4
设\(V\)是域\(F\)上的一个线性空间.
\begin{itemize}
	\item 设向量组\(\AutoTuple{\beta}{r}\)
	可以由向量组\(\AutoTuple{\alpha}{s}\)线性表出,则\begin{gather*}
		r>s
		\implies
		\text{$\AutoTuple{\beta}{r}$线性相关}, \\
		\text{$\AutoTuple{\beta}{r}$线性无关}
		\implies
		r\leq s.
	\end{gather*}

	\item 等价的线性无关的向量组所含向量的个数相等.
\end{itemize}
\end{property}

\subsection{向量组的秩}
\begin{definition}
%@see: 《高等代数(第三版 下册)》(丘维声) P76 定义6
向量组\(A=\{\AutoTuple{\alpha}{s}\}\)的一个极大线性无关组的基数,
称为“向量组\(A\)的\DefineConcept{秩}(rank)”,
记为\(\rank A\)或\(\rank\{\AutoTuple{\alpha}{s}\}\).
\end{definition}

\begin{property}\label{theorem:线性空间.向量组的秩的性质}
%@see: 《高等代数(第三版 下册)》(丘维声) P76 命题8
%@see: 《高等代数(第三版 下册)》(丘维声) P76 命题9
%@see: 《高等代数(第三版 下册)》(丘维声) P76 推论9
设\(V\)是域\(F\)上的一个线性空间.
\begin{itemize}
	\item 全由零向量组成的向量组的秩为零.

	\item 向量组线性无关的充分必要条件是
	它的秩等于它的基数.

	\item 设\(A,B\)都是向量组.
	如果\(A\)可以由\(B\)线性表出,
	则\(\rank A \leq \rank B\).

	\item 等价的向量组有相同的秩.
\end{itemize}
\end{property}

\subsection{线性空间的基与维数}
\begin{definition}
%@see: 《高等代数(第三版 下册)》(丘维声) P76 定义5
设\(V\)是域\(F\)上的一个线性空间,
\(A\)是\(V\)的一个子集,
\(a\)是\(A\)的有限子集.
如果\(a\)是线性无关的,
但是\[
	(\forall\beta \in A-a)
	[\text{$a \cup \{\beta\}$是线性相关的}],
\]
则称“\(a\)是\(A\)的一个\DefineConcept{极大线性无关组}”.
\end{definition}

\begin{property}
%@see: 《高等代数(第三版 下册)》(丘维声) P76 推论5
%@see: 《高等代数(第三版 下册)》(丘维声) P76 推论6
设\(V\)是域\(F\)上的一个线性空间.
\begin{itemize}
	\item 向量组与它的极大线性无关组等价.
	\item 向量组的任意两个极大线性无关组的基数相等.
\end{itemize}
\end{property}

\begin{definition}\label{definition:线性空间.线性空间的基}
%@see: 《高等代数(第三版 下册)》(丘维声) P76 定义7
设\(V\)是域\(F\)上的一个线性空间,\(S \subseteq V\).
如果\begin{itemize}
	\item \(S\)线性无关,
	\item \(V\)中每一个向量都可以由\(S\)中有限多个向量线性表出,
\end{itemize}
则称“\(S\)是\(V\)的一个\DefineConcept{基}%
(\(S\) is a \emph{basis} for \(V\))”.
\end{definition}
\begin{remark}
%@credit: {647826c9-7e2a-49d1-b176-cd39b299b349} 说:除了 Hamel 基以外,还有 Schauder 基等其他定义
%@credit: {85841724-e8e0-4a39-88bf-973ade1b5e13} 说:参考《代数学(一)》(李方、邓少强、冯荣权、刘东文) P101 定义5.1.2
在\hyperref[definition:线性空间.线性空间的基]{基的定义}中,
必须要注意第二个条件中“有限多个”这个限定,
它说明这里定义的基是\emph{哈莫基}(Hamel basis).
%@see: https://zh.wikipedia.org/wiki/%E5%9F%BA_(%E7%B7%9A%E6%80%A7%E4%BB%A3%E6%95%B8)
%@see: https://en.wikipedia.org/wiki/Basis_(linear_algebra)
\end{remark}

\begin{property}\label{theorem:线性空间的结构.零空间的基是空集}
%@see: 《高等代数(第三版 下册)》(丘维声) P77
零空间的基是空集.
%TODO 无法确定这个究竟是定义还是性质
\end{property}

\begin{property}
%@see: 《高等代数(第三版 下册)》(丘维声) P77
域\(F\)上的任一线性空间\(V\)都有基.
\end{property}

\begin{example}
%@see: 《高等代数(第三版 下册)》(丘维声) P77 例8
在数域\(K\)上全体\(s \times n\)矩阵形成的线性空间\(M_{s \times n}(K)\)中,
所有基本矩阵组成的子集\[
	\Set{
		E_{11},E_{12},\dotsc,E_{1n},
		\dotsc,
		E_{s1},E_{s2},\dotsc,E_{sn}
	}
\]是\(M_{s \times n}(K)\)的一个基.
\begin{proof}
每个\(s \times n\)矩阵\(A = (a_{ij})_{s \times n}\)都可以表示成\[
	A = \sum_{i=1}^s \sum_{j=1}^n a_{ij} E_{ij}.
\]

假设\[
	\sum_{i=1}^s \sum_{j=1}^n a_{ij} E_{ij} = 0,
\]
则矩阵\(A = (a_{ij})_{s \times n}\)是零矩阵,
从而\(a_{ij} = 0\ (i=1,2,\dotsc,s;j=1,2,\dotsc,n)\).
因此\[
	\Set{
		E_{11},E_{12},\dotsc,E_{1n},
		\dotsc,
		E_{s1},E_{s2},\dotsc,E_{sn}
	}
\]线性无关,
从而说明它是\(M_{s \times n}(K)\)的一个基.
\end{proof}
\end{example}

\begin{example}
%@see: 《高等代数(第三版 下册)》(丘维声) P77 例9
数域\(K\)上所有一元多项式形成的线性空间\(K[x]\)中,
子集\[
	\{1,x,x^2,\dotsc,x^n,\dotsc\}
\]是\(K[x]\)的一个基.
\begin{proof}
\(K[x]\)上每一个一元多项式\(f(x)\)
可以写成\(f(x)=a_0+a_1 x+a_2 x^2+\dotsb+a_n x^n\).
任取\(S\)的一个有限子集\(\{x^{i_1},\dotsc,x^{i_m}\}\).
设\(k_1 x^{i_1}+\dotsb+k_m x^{i_m}=0\),
则由一元多项式的定义得
\(k_1=\dotsb=k_m=0\),
从而这个子集线性无关,
因此\(S\)线性无关,
于是\(S\)是\(K[x]\)的一个基.
\end{proof}
\end{example}

\begin{definition}
%@see: 《高等代数(第三版 下册)》(丘维声) P77 定义8
%@see: 《Linear Algebra and Its Applications (Second Edition)》(Peter D. Lax) P5 Definition
设\(V\)是域\(F\)上的一个线性空间,
\(S\)是\(V\)的一个基.
如果\(S\)是有限集,
则称“\(V\)是\DefineConcept{有限维的}(finite dimensional)”;
否则称“\(V\)是\DefineConcept{无限维的}(infinite dimensional)”.
\end{definition}

\begin{example}
数域\(K\)上全体\(s \times n\)矩阵\(M_{s \times n}(K)\)是有限维的.
\end{example}

\begin{example}
数域\(K\)上全体一元多项式\(K[x]\)是无限维的.
\end{example}

\begin{theorem}\label{theorem:线性空间.同一个线性空间的任意两个基的基数相等}
%@see: 《高等代数(第三版 下册)》(丘维声) P77 定理10
%@see: 《Linear Algebra and Its Applications (Second Edition)》(Peter D. Lax) P5 Theorem 3.
设\(V\)是域\(F\)上的一个线性空间.
如果\(V\)是有限维的,
则\(V\)的任意两个基的基数相等.
\begin{proof}
不妨设\(V\)有一个基包含有限多个向量\(\AutoTuple{\alpha}{n}\).
设\(S\)是\(V\)的另一个基.

假如\(\card S>n\),
则\(S\)中可取出\(n+1\)个向量\(\AutoTuple{\beta}{n+1}\),
它们可以由\(\AutoTuple{\alpha}{n}\)线性表出.
由\cref{theorem:线性空间.性质3},%引理1
可知\(\AutoTuple{\beta}{n+1}\)线性相关.
这与\(S\)线性无关矛盾,
因此\(\card S\leq n\).

设\(S=\{\AutoTuple{\beta}{m}\}\ (m\leq n)\),
由\cref{theorem:线性空间.性质3},%推论4
又可知\(m=n\).
\end{proof}
\end{theorem}

\begin{definition}
%@see: 《高等代数(第三版 下册)》(丘维声) P78 定义9
设\(V\)是域\(F\)上的一个有限维线性空间,
则\(V\)的一个基的基数
称为“\(V\)的\DefineConcept{维数}”,
记作\(\dim_F V\),
简记为\(\dim V\).
\end{definition}

\begin{property}
零空间的维数为\(0\).
\begin{proof}
由\cref{theorem:线性空间的结构.零空间的基是空集} 立即可得.
\end{proof}
\end{property}

\begin{example}
\(\dim M_{s \times n}(K)=sn\).
\end{example}

\begin{example}
\(\dim M_n(K) = n^2\).
\end{example}

维数对于研究有限维线性空间的结构起着重要的作用.

\begin{property}
%@see: 《高等代数(第三版 下册)》(丘维声) P78 命题11
%@see: 《高等代数(第三版 下册)》(丘维声) P78 命题12
%@see: 《Linear Algebra and Its Applications (Second Edition)》(Peter D. Lax) P5 Lemma 1.
设\(V\)是域\(F\)上的一个线性空间.
\begin{itemize}
	\item 如果\(\dim V=n\),
	则\(V\)中任意\(n+1\)个向量都线性无关.

	\item 如果\(\dim V=n\),
	则\(V\)中任意\(n\)个线性无关的向量都是\(V\)的一个基.

	\item 如果\(V\)的有限子集\(\AutoTuple{\alpha}{s}\)是线性无关的,
	则\(s \leq n\).
\end{itemize}
\end{property}

基对于研究线性空间的结构起着重要的作用.

\begin{property}
%@see: 《高等代数(第三版 下册)》(丘维声) P78 命题13
设\(V\)是域\(F\)上的一个线性空间,
\(\AutoTuple{\alpha}{n}\)是\(V\)的一个基,
则\(V\)中每一个向量\(\alpha\)
可以唯一地表成\(\AutoTuple{\alpha}{n}\)的线性组合.
\begin{proof}
从基的定义知道,任意一个向量\(\alpha\),均可由\(\AutoTuple{\alpha}{n}\)线性表出.
假设有如下两种表出方式:\begin{gather*}
	\alpha = x_1 \alpha_1 + x_2 \alpha_2 + \dotsb + x_n \alpha_n, \\
	\alpha = y_1 \alpha_1 + y_2 \alpha_2 + \dotsb + y_n \alpha_n.
\end{gather*}
相减得\[
	0 = (x_1 - y_1) \alpha_1 + (x_2 - y_2) \alpha_2 + \dotsb + (x_n - y_n) \alpha_n.
\]
由于\(\AutoTuple{\alpha}{n}\)线性无关,
所以\[
	x_1 - y_1
	= x_2 - y_2
	= \dotsb
	= x_n - y_n
	= 0.
\]
由此可见表出方式唯一.
\end{proof}
\end{property}

\begin{example}\label{example:线性空间.生成子空间等于线性空间的向量组就是基}
%@see: 《高等代数(第三版 下册)》(丘维声) P82 习题8.1 10.
证明:在数域\(K\)上的\(n\)维线性空间\(V\)中,
如果每一个向量都可以由\(\AutoTuple{\alpha}{n}\)线性表出,
则\(\AutoTuple{\alpha}{n}\)是\(V\)的一个基.
\begin{proof}
在\(V\)中取一个基\(\AutoTuple{\delta}{n}\).
由题设条件可知\(\AutoTuple{\delta}{n}\)可以由\(\AutoTuple{\alpha}{n}\)线性表出,
那么由\cref{theorem:线性空间.向量组的秩的性质} 可知\[
	\rank\{\AutoTuple{\delta}{n}\}
	\leq
	\rank\{\AutoTuple{\alpha}{n}\}.
\]
又因为\(\AutoTuple{\alpha}{n}\)可以由\(\AutoTuple{\delta}{n}\)线性表出,
从而\[
	\rank\{\AutoTuple{\alpha}{n}\}
	\leq
	\rank\{\AutoTuple{\delta}{n}\}.
\]
于是\(\rank\{\AutoTuple{\alpha}{n}\}
= \rank\{\AutoTuple{\delta}{n}\}
= n\),
那么\(\AutoTuple{\alpha}{n}\)线性无关,
因此\(\AutoTuple{\alpha}{n}\)是\(V\)的一个基.
\end{proof}
%@see: 《高等代数(第三版 上册)》(丘维声) P80 习题3.4 4.
\end{example}

\subsection{向量的坐标,过渡矩阵}
%@see: 《高等代数(第三版 下册)》(丘维声) P78
我们把向量\(\alpha\)由基\(\AutoTuple{\alpha}{n}\)线性表出的系数
组成的\(n\)元有序组\((\AutoTuple{\alpha}{n})\)
称为“向量\(\alpha\)在基\(\AutoTuple{\alpha}{n}\)下的\DefineConcept{坐标}(coordinate)”.
通常把向量的坐标写成列向量形式.

由上可知,有限维线性空间\(V\)中给定一个基,
则\(V\)中每一个向量都可以唯一地表示成这个基的线性组合,
从而\(V\)的结构就很清楚了.
因此,基是研究线性空间的结构的第一条途径.

\(n\)维线性空间\(V\)中给定两个基,
我们想要知道,\(V\)中每一个向量分别在这两个基下的坐标有什么关系.

设\(\AutoTuple{\alpha}{n}\)和\(\AutoTuple{\beta}{n}\)是\(V\)的两个基,
\(V\)中向量\(\alpha\)在这两个基下的坐标分别为\[
	X=(\AutoTuple{x}{n})^T, \qquad
	Y=(\AutoTuple{y}{n})^T.
\]
为了求\(X\)与\(Y\)之间的关系,
首先把这两个基之间的关系搞清楚.
由于\(\AutoTuple{\alpha}{n}\)是\(V\)的一个基,
因此有\[
%@see: 《高等代数(第三版 下册)》(丘维声) P79 (2)
	\left\{ \begin{array}{l}
		\beta_1=a_{11} \alpha_1+a_{21} \alpha_2+\dotsb+a_{n1} \alpha_n, \\
		\beta_2=a_{12} \alpha_1+a_{22} \alpha_2+\dotsb+a_{n2} \alpha_n, \\
		\hdotsfor1, \\
		\beta_n=a_{1n} \alpha_1+a_{2n} \alpha_2+\dotsb+a_{nn} \alpha_n.
	\end{array} \right.
\]
为了使推导过程简洁,
我们可以把上式写成\[
%@see: 《高等代数(第三版 下册)》(丘维声) P79 (5)
	(\AutoTuple{\beta}{n})
	=
	(\AutoTuple{\alpha}{n})
	A,
\]
其中\[
	A=\begin{bmatrix}
		a_{11} & a_{12} & \dots & a_{1n} \\
		a_{21} & a_{22} & \dots & a_{2n} \\
		\vdots & \vdots & & \vdots \\
		a_{n1} & a_{n2} & \dots & a_{nn}
	\end{bmatrix}.
\]
我们把\(A\)称为
“基\(\AutoTuple{\alpha}{n}\)到基\(\AutoTuple{\beta}{n}\)的\DefineConcept{过渡矩阵}”.
%@see: https://mathworld.wolfram.com/TransitionMatrix.html
%@see: https://mathworld.wolfram.com/ChangeofCoordinatesMatrix.html

在这里,我们引入一种形式写法\[
%@see: 《高等代数(第三版 下册)》(丘维声) P79 (3)
	x_1 \alpha_1 + \dotsb + x_n \alpha_n
	\defeq
	(\AutoTuple{\alpha}{n})
	\begin{bmatrix}
		x_1 \\
		\vdots \\
		x_n
	\end{bmatrix}.
\]
像这样的形式写法,是模仿矩阵乘法的定义.
因此,类似于矩阵乘法的结合律、左右分配律、乘法与数量乘法的关系的证明方法,
可以证明形式写法满足以下规则.

设\(\AutoTuple{\alpha}{n}\)与\(\AutoTuple{\beta}{n}\)是\(V\)中的两个向量组,
\(A,B\)是域\(F\)上的两个\(n\)阶矩阵,
数\(k \in F\),
则\begin{gather*}
	%@see: 《高等代数(第三版 下册)》(丘维声) P79 (6)
	[(\AutoTuple{\alpha}{n}) A] B
	= (\AutoTuple{\alpha}{n}) (A B), \\
	%@see: 《高等代数(第三版 下册)》(丘维声) P79 (7)
	(\AutoTuple{\alpha}{n}) A
	+ (\AutoTuple{\alpha}{n}) B
	= (\AutoTuple{\alpha}{n}) (A + B), \\
	%@see: 《高等代数(第三版 下册)》(丘维声) P79 (8)
	(\AutoTuple{\alpha}{n}) A
	+ (\AutoTuple{\beta}{n}) A
	= (\alpha_1+\beta_1,\dotsc,\alpha_n+\beta_n) A, \\
	%@see: 《高等代数(第三版 下册)》(丘维声) P79 (9)
	[k (\AutoTuple{\alpha}{n})] A
	= (\AutoTuple{\alpha}{n}) (k A), \\
	[k (\AutoTuple{\alpha}{n})] A
	= k [(\AutoTuple{\alpha}{n}) A],
\end{gather*}
其中\begin{gather*}
	%@see: 《高等代数(第三版 下册)》(丘维声) P79 (10)
	(\AutoTuple{\alpha}{n})
	+ (\AutoTuple{\beta}{n})
	\defeq
	(\alpha_1+\beta_1,\dotsc,\alpha_n+\beta_n), \\
	%@see: 《高等代数(第三版 下册)》(丘维声) P79 (11)
	k (\AutoTuple{\alpha}{n})
	= (\AutoTuple{k \alpha}{n}).
\end{gather*}

\begin{proposition}\label{theorem:线性空间.命题14}
%@see: 《高等代数(第三版 下册)》(丘维声) P80 命题14
设\(\AutoTuple{\alpha}{n}\)是\(V\)的一个基,
且\((\AutoTuple{\beta}{n})=(\AutoTuple{\alpha}{n})A\),
则\(\AutoTuple{\beta}{n}\)是\(V\)的一个基
当且仅当\(A\)是可逆矩阵.
\begin{proof}
由于\(\AutoTuple{\alpha}{n}\)线性无关,
并且有\begin{align*}
	k_1 \beta_1+\dotsb+k_n \beta_n
	&=(\AutoTuple{\beta}{n}) (\AutoTuple{k}{n})^T \\
	&=(\AutoTuple{\alpha}{n}) A (\AutoTuple{k}{n})^T,
\end{align*}
因此\begin{align*}
	&\text{$\AutoTuple{\beta}{n}$是$V$的一个基}
	\iff \text{$\AutoTuple{\beta}{n}$线性无关} \\
	&\iff
	k_1 \beta_1+\dotsb+k_n \beta_n=0
	\implies
	k_1=\dotsb=k_n=0 \\
	&\iff
	(\AutoTuple{\alpha}{n}) A (\AutoTuple{k}{n})^T=0
	\implies
	(\AutoTuple{k}{n})^T=0 \\
	&\iff
	A (\AutoTuple{k}{n})^T=0
	\implies
	(\AutoTuple{k}{n})^T=0 \\
	&\iff \text{齐次线性方程组$AX=0$只有零解} \\
	&\iff \abs{A}\neq0
	\iff \text{$A$是可逆矩阵}.
	\qedhere
\end{align*}
\end{proof}
\end{proposition}

\cref{theorem:线性空间.命题14} 表明:
基\(\AutoTuple{\alpha}{n}\)到基\(\AutoTuple{\beta}{n}\)的过渡矩阵是可逆矩阵.

现在可以给出向量\(\alpha\)
分别在基\(\AutoTuple{\alpha}{n}\)
与基\(\AutoTuple{\beta}{n}\)下的坐标\(X,Y\)之间的关系.
由于\[
	\alpha
	=(\AutoTuple{\alpha}{n}) X
	=(\AutoTuple{\beta}{n}) Y,
\]
并且基\(\AutoTuple{\alpha}{n}\)到基\(\AutoTuple{\beta}{n}\)的过渡矩阵是\(A\),
因此\[
	(\AutoTuple{\alpha}{n}) X
	=(\AutoTuple{\beta}{n}) Y
	=(\AutoTuple{\alpha}{n}) A Y.
\]
由于同一个向量由基\(\AutoTuple{\alpha}{n}\)线性表出的方式唯一,
从上式得\[
%@see: 《高等代数(第三版 下册)》(丘维声) P79 (12)
	X=AY,
\]
从而\[
%@see: 《高等代数(第三版 下册)》(丘维声) P79 (13)
	Y=A^{-1}X.
\]

\begin{example}
设\(\alpha_1,\alpha_2,\alpha_3\)是\(\mathbb{R}^3\)的一组基,
求:基\(\alpha_1,\frac12\alpha_2,\frac13\alpha_3\)
到基\(\alpha_1+\alpha_2,\alpha_2+\alpha_3,\alpha_3+\alpha_1\)的过渡矩阵.
\begin{solution}
设所求过渡矩阵为\(P\),
则根据定义有\[
	\begin{bmatrix}
		\alpha_1 & \frac12\alpha_2 & \frac13\alpha_3
	\end{bmatrix} P
	= \begin{bmatrix}
		\alpha_1+\alpha_2 & \alpha_2+\alpha_3 & \alpha_3+\alpha_1
	\end{bmatrix},
\]
即\[
	\begin{bmatrix}
		\alpha_1 & \alpha_2 & \alpha_3
	\end{bmatrix}
	\begin{bmatrix}
		1 \\
		& \frac12 \\
		&& \frac13
	\end{bmatrix} P
	= \begin{bmatrix}
	\alpha_1 & \alpha_2 & \alpha_3
	\end{bmatrix}
	\begin{bmatrix}
		1 & 0 & 1 \\
		1 & 1 & 0 \\
		0 & 1 & 1
	\end{bmatrix},
\]
所以\[
	P = \begin{bmatrix}
		1 \\
		& \frac12 \\
		&& \frac13
	\end{bmatrix}^{-1}
	\begin{bmatrix}
		1 & 0 & 1 \\
		1 & 1 & 0 \\
		0 & 1 & 1
	\end{bmatrix}
	= \begin{bmatrix}
		1 \\
		& 2 \\
		&& 3
	\end{bmatrix} \begin{bmatrix}
		1 & 0 & 1 \\
		1 & 1 & 0 \\
		0 & 1 & 1
	\end{bmatrix}
	= \begin{bmatrix}
		1 & 0 & 1 \\
		2 & 2 & 0 \\
		0 & 3 & 3
	\end{bmatrix}.
\]
\end{solution}
\end{example}

\begin{example}
%@see: 《高等代数(第三版 下册)》(丘维声) P81 习题8.1 4.
把复数域\(\mathbb{C}\)看成实数域\(\mathbb{R}\)上的线性空间,
求它的一个基和维数,
以及每个复数在这个基下的坐标.
\begin{solution}
把复数域看成实数域上的线性空间\(V\),
容易看出,有限集\(S = \{1,\iu\}\)是线性空间\(V\)的一个基,
它的维数为\(\dim V = \card S = 2\),
而每个复数\(z = a + b\iu\)在这个基下的坐标为\((a,b)^T\).
\end{solution}
\end{example}

\begin{example}
%@see: 《高等代数(第三版 下册)》(丘维声) P81 习题8.1 5.
把数域\(K\)看成自身上的线性空间,求它的一个基和维数.
\begin{solution}
把数域\(K\)看成自身上的线性空间\(V\),
容易看出,\(S = \{1\}\)是线性空间\(V\)的一个基,
它的维数为\(\dim V = \card S = 1\).
\end{solution}
\end{example}

\begin{example}
%@see: 《高等代数(第三版 下册)》(丘维声) P81 习题8.1 11.
设\(X = \{\AutoTuple{x}{n}\}\),\(F\)是一个域.
把映射空间\(F^X\)看成域\(F\)上的一个线性空间,
求\(F^X\)的一个基和维数,
再求映射\(f \in F^X\)在这个基下的坐标.
\begin{solution}
任意给定\(f \in F^X\),
必有\(f = \Set{
	(x_1,f(x_1)),
	\dotsc,
	(x_n,f(x_n))
}\).

令\[
	f_i(x_j) \defeq \delta(i,j),
	\quad i,j=1,2,\dotsc,n,
\]
其中\(\delta\)是克罗内克\(\delta\)函数,
即\[
	\delta(i,j) = \left\{ \begin{array}{cl}
		1, & i = j, \\
		0, & i \neq j.
	\end{array} \right.
\]
那么\[
	f(x) = f(x_1) f_1(x) + \dotsb + f(x_n) f_n(x),
	\quad x \in X,
	\eqno(1)
\]
这就说明,\(f\)可以由\(\AutoTuple{f}{n}\)线性表出.

显然\(\AutoTuple{f}{n}\)线性无关,
那么\(\AutoTuple{f}{n}\)是\(F^X\)的一个基,
从而有\(\dim F^X = n\).

由(1)式可知,
函数\(f\)在基\(\AutoTuple{f}{n}\)下的坐标为
\((f(x_1),\dotsc,f(x_n))\).
\end{solution}
\end{example}

\subsection{线性空间的笛卡尔和}
\begin{definition}
%@see: 《Linear Algebra and Its Applications (Second Edition)》(Peter D. Lax) P10 Definition
设\(V,W\)都是域\(F\)上的线性空间.
把\[
	\Set{
		(v,w)
		\given
		v \in V,
		w \in W
	}
\]称为“线性空间\(V\)和\(W\)的\DefineConcept{笛卡尔和}(Cartesian sum)”,
记作\(V \CartesianSum W\).
\end{definition}

\begin{theorem}
%@see: 《Linear Algebra and Its Applications (Second Edition)》(Peter D. Lax) P10
设\(V,W\)都是域\(F\)上的线性空间,
则线性空间\(V\)和\(W\)的笛卡尔和\(V \CartesianSum W\)是域\(F\)上的线性空间.
%TODO proof
\end{theorem}

\begin{theorem}\label{theorem:线性空间.笛卡尔和的维数公式}
%@see: 《Linear Algebra and Its Applications (Second Edition)》(Peter D. Lax) P11 Exercise 18.
设\(V,W\)都是域\(F\)上的线性空间,
则线性空间\(V\)和\(W\)的笛卡尔和\(V \CartesianSum W\)满足\[
	\dim(V \CartesianSum W) = \dim V + \dim W.
\]
%TODO proof
\end{theorem}

\section{子空间及其运算}
\begin{definition}
%@see: 《高等代数(第三版 下册)》(丘维声) P82 定义1
设\(V\)是域\(F\)上的一个线性空间,
\(\emptyset\neq U\subseteq V\).
如果\(U\)对于\(V\)的加法及纯量乘法运算
也形成\(F\)上的线性空间,
则称“\(U\)是\(V\)的一个\DefineConcept{子空间}(subspace)”.
\end{definition}

显然\(\{\vb0\}\)是\(V\)的一个子空间,
称其为“\(V\)的\DefineConcept{零子空间}”,
也记作\(0\).
另外,\(V\)显然也是\(V\)的一个子空间.
我们把\(0\)和\(V\)统称为“\(V\)的\DefineConcept{平凡子空间}”,
把\(V\)的其余子空间称为它的\DefineConcept{非平凡子空间}.

\begin{theorem}\label{theorem:线性空间.子空间的判定}
%@see: 《高等代数(第三版 下册)》(丘维声) P82 定理1
域\(F\)上线性空间\(V\)的非空子集\(U\)是\(V\)的一个子空间
当且仅当\(U\)对于\(V\)的加法与纯量乘法都封闭,
即\begin{enumerate}
	\item \((\forall u_1,u_2\in U)[u_1+u_2 \in U]\);
	\item \((\forall u\in U)(\forall k\in F)[ku\in U]\).
\end{enumerate}
\end{theorem}

\begin{example}
%@see: 《高等代数(第三版 下册)》(丘维声) P83 例1
数域\(K\)上所有次数小于\(n\)的一元多项式组成的集合\(K[x]_n\)
是\(K[x]\)的一个子空间.
\end{example}

\begin{proposition}
%@see: 《高等代数(第三版 下册)》(丘维声) P83 命题2
设\(U\)是域\(F\)上\(n\)维线性空间\(V\)的一个子空间,
则\(\dim U\leq\dim V\).
\begin{proof}
由于\(n\)维线性空间\(V\)中任意\(n+1\)个向量都线性相关,
因此\(U\)的一个基所含向量的个数一定小于或等于\(n\),
从而\(\dim U\leq\dim V\).
\end{proof}
\end{proposition}

\section{线性空间的同构}
域\(F\)上\(n\)维线性空间\(V\)
与域\(F\)上\(n\)元有序组组成的线性空间\(F^n\)非常相像.
例如,对于\(F^n\)向量组\(\AutoTuple{\a}{s}\)生成的子空间\(U=\opair{\AutoTuple{\a}{s}}\),
向量组\(\AutoTuple{\a}{s}\)的一个极大线性无关组是\(U\)的一个基,
\(\dim U\)等于\(\rank\{\AutoTuple{\a}{s}\}\).
对于\(V\)中向量组生成的子空间也有同样的结论.

为什么域\(F\)上的\(n\)维线性空间\(V\)与\(F^n\)这样相像?

\begin{definition}
%@see: 《高等代数(第三版 下册)》(丘维声) P92 定义1
设\(V\)与\(V'\)都是域\(F\)上的线性空间,
\(\sigma\)是一个从\(V\)到\(V'\)的双射.
如果\begin{itemize}
	\item \((\forall\a,\b \in V)
	[\sigma(\a+\b)=\sigma(\a)+\sigma(\b)]\),
	\item \((\forall\a \in V)
	(\forall k \in F)
	[\sigma(k\a)=k\sigma(\a)]\),
\end{itemize}
那么称“\(\sigma\)是一个从\(V\)到\(V'\)的\DefineConcept{同构}(isomorphism)”
“\(V\)与\(V'\)同构(\(V\) is \emph{isomorphic} to \(V'\))”,
记为\(V \simeq V'\).
\end{definition}

\begin{property}\label{theorem:线性空间的同构.同构线性空间的性质1}
%@see: 《高等代数(第三版 下册)》(丘维声) P92 性质1
设\(V\)与\(V'\)都是域\(F\)上的线性空间,
\(0\)是\(V\)的零元,
\(0'\)是\(V'\)的零元,
\(\sigma\)是一个从\(V\)到\(V'\)的同构,
则\(\sigma(0)=0'\).
\begin{proof}
\(0\a=0 \implies \sigma(0)=\sigma(0\a)=0\sigma(\a)=0'\).
\end{proof}
\end{property}

\begin{property}\label{theorem:线性空间的同构.同构线性空间的性质2}
%@see: 《高等代数(第三版 下册)》(丘维声) P92 性质2
设\(V\)与\(V'\)都是域\(F\)上的线性空间,
\(\sigma\)是一个从\(V\)到\(V'\)的同构,
则\[
	(\forall\a\in V)[\sigma(-\a)=-\sigma(\a)].
\]
\begin{proof}
\(\sigma(-\a)=\sigma((-1)\a)=(-1)\sigma(\a)=-\sigma(\a)\).
\end{proof}
\end{property}

\begin{property}\label{theorem:线性空间的同构.同构线性空间的性质3}
%@see: 《高等代数(第三版 下册)》(丘维声) P92 性质3
设\(V\)与\(V'\)都是域\(F\)上的线性空间,
\(\sigma\)是一个从\(V\)到\(V'\)的同构,
则\[
	(\forall \AutoTuple{\a}{s} \in V)
	(\forall \AutoTuple{k}{s} \in F)
	[\sigma(k_1\a_1+\dotsb+k_s\a_s)=k_1\sigma(\a_1)+\dotsb+k_s\sigma(\a_s)].
\]
\end{property}

\begin{property}\label{theorem:线性空间的同构.同构线性空间的性质4}
%@see: 《高等代数(第三版 下册)》(丘维声) P92 性质4
设\(V\)与\(V'\)都是域\(F\)上的线性空间,
\(\sigma\)是一个从\(V\)到\(V'\)的同构,
则\(V\)中向量组\(\AutoTuple{\a}{s}\)线性相关的充分必要条件是:
\(\sigma(\a_1),\dotsc,\sigma(\a_s)\)是\(V'\)中线性相关的向量组.
\begin{proof}
因为\(\sigma\)是单射,
所以\(\sigma(\a)=\sigma(\b) \implies \a=\b\),
于是\begin{align*}
	k_1\a_1+\dotsb+k_s\a_s=0
	&\iff
	\sigma(k_1\a_1+\dotsb+k_s\a_s)=\sigma(0) \\
	&\iff
	k_1\sigma(\a_1)+\dotsb+k_s\sigma(\a_s)=0',
\end{align*}
那么\(\AutoTuple{\a}{s}\)线性相关
当且仅当\(\sigma(\a_1),\dotsc,\sigma(\a_s)\)线性相关.
\end{proof}
\end{property}

\begin{property}\label{theorem:线性空间的同构.同构线性空间的性质5}
%@see: 《高等代数(第三版 下册)》(丘维声) P92 性质5
设\(V\)与\(V'\)都是域\(F\)上的线性空间,
\(\sigma\)是一个从\(V\)到\(V'\)的同构.
如果\(\AutoTuple{\a}{n}\)是\(V\)的一个基,
则\(\sigma(\a_1),\dotsc,\sigma(\a_n)\)是\(V'\)的一个基.
\begin{proof}
由\cref{theorem:线性空间的同构.同构线性空间的性质4}
可知\(\sigma(\a_1),\dotsc,\sigma(\a_n)\)是\(V'\)的一个线性无关的向量组.
任取\(\b \in V'\),
由于\(\sigma\)是满射,
因此存在\(\a \in V\),
使得\(\sigma(\a)=\b\).
设\(\a=k_1\a_1+\dotsb+k_n\a_n\),
则\[
	\b=\sigma(\a)
	=k_1\sigma(\a_1)+\dotsb+k_n\sigma(\a_n),
\]
因此\(\sigma(\a_1),\dotsc,\sigma(\a_n)\)是\(V'\)的一个基.
\end{proof}
\end{property}

\begin{theorem}\label{theorem:线性空间的同构.线性空间同构的充分必要条件}
%@see: 《高等代数(第三版 下册)》(丘维声) P92 定理1
设\(V\)与\(V'\)都是域\(F\)上的有限维线性空间,
则\(V \simeq V'\)的充分必要条件是\(\dim V = \dim V'\).
\begin{proof}
必要性.
由\cref{theorem:线性空间的同构.同构线性空间的性质5} 立即得出.

充分性.
设\(\dim V = \dim V' = n\).
在\(V\)中取一个基\(\AutoTuple{\a}{n}\).
在\(V'\)中取一个基\(\AutoTuple{\g}{n}\).
令\[
	\sigma\colon V \to V',
	\a=\sum_{i=1}^n k_i\a_i
	\mapsto
	\sum_{i=1}^n k_i\g_1.
\]
可以看出,\(\sigma\)是一个从\(V\)到\(V'\)的同构,
\(V \simeq V'\).
\end{proof}
\end{theorem}
从\cref{theorem:线性空间的同构.线性空间同构的充分必要条件} 立即得出,
域\(F\)上任意一个\(n\)维线性空间\(V\)都与\(F^n\)同构,
并且\(V\)中每一个向量\(\a\)
对应它在\(V\)的一个基\(\AutoTuple{\a}{n}\)下的坐标\((\AutoTuple{k}{n})^T\),
这个对应关系就是从\(V\)到\(F^n\)的一个同构.
正是因为域\(F\)上\(n\)维线性空间\(V\)与\(F^n\)同构,
所以\(V\)与\(F^n\)才这么相像.
虽然它们的元素不同,但是有关线性运算的性质却完全一样.
于是我们可以利用\(F^n\)的性质来研究\(F\)上\(n\)维线性空间的性质.
线性空间的同构,是研究线性空间结构的第三条途径.

\begin{proposition}\label{theorem:线性空间的同构.子空间在同构下的像}
%@see: 《高等代数(第三版 下册)》(丘维声) P93 命题2
设\(V\)是域\(F\)上的\(n\)维线性空间,
\(U\)是\(V\)的一个子空间,
\(\AutoTuple{\a}{n}\)的\(V\)的一个基,
\(\sigma\)把\(V\)中每一个向量\(\a\)对应到它在基\(\AutoTuple{\a}{n}\)下的坐标.
令\[
	\sigma(U) \defeq \Set{ \sigma(\a) \given \a \in U },
\]
则\(\sigma(U)\)是\(F^n\)的一个子空间,
且\(\dim U = \dim\sigma(U)\).
\begin{proof}
显然\(\sigma(U)\)是非空集,
\(\sigma\)是一个从\(V\)到\(F^n\)的同构,
\(U\)对加法和纯量乘法封闭.
这就说明\(\sigma(U)\)是\(F^n\)的一个子空间.

由于\(U\)与\(\sigma(U)\)都是域\(F\)上有限维线性空间,
且\(\sigma\)在\(U\)上的限制\((\sigma \setrestrict U)\)是从\(U\)到\(\sigma(U)\)的一个同构,
因此\(\dim U = \dim\sigma(U)\).
\end{proof}
\end{proposition}

\begin{example}
%@see: 《高等代数(第三版 下册)》(丘维声) P94 例1
设\(\AutoTuple{\vb\alpha}{n}\)是域\(F\)上线性空间\(V\)的一个基,
\(\AutoTuple{\vb\beta}{s}\)是\(V\)的一个向量组,
并且\[
%@see: 《高等代数(第三版 下册)》(丘维声) P94 (5)
	(\AutoTuple{\vb\beta}{s})
	= (\AutoTuple{\vb\alpha}{n}) \A,
\]
其中\(\A\)是一个\(n \times s\)矩阵.
证明:\[
%@see: 《高等代数(第三版 下册)》(丘维声) P94 (6)
	\dim\opair{\AutoTuple{\vb\beta}{s}}
	= \rank\A.
\]
\begin{proof}
用\(\sigma\)表示从\(V\)到\(F^n\)的一个同构,
它把\(\vb\alpha \in V\)映射为\(\a\)在\(\AutoTuple{\vb\alpha}{n}\)下的坐标.
设\(\vb{A}\)的列向量组是\(\AutoTuple{\vb{A}}{s}\),那么\[
	\sigma(\vb\beta_j) = \vb{A}_j,
	\quad j=1,2,\dotsc,s.
\]
由\cref{theorem:线性空间的同构.子空间在同构下的像} 可知\begin{align*}
	&\dim\opair{\AutoTuple{\vb\beta}{s}} \\
	&= \dim\sigma\opair{\AutoTuple{\vb\beta}{s}} \\
	&= \dim\opair{\sigma(\vb\beta_1),\dotsc,\sigma(\vb\beta_s)} \\
	&= \dim\opair{\AutoTuple{A}{s}} \\
	&= \rank\A.
	\qedhere
\end{align*}
\end{proof}
\end{example}

同构是域\(F\)上线性空间之间的一个关系.
它具有反身性(因为\(V\)的恒等映射是从\(V\)到\(V\)的一个同构)、
对称性和传递性.
因此同构关系是一个等价关系,
对应的等价类称为\DefineConcept{同构类}.

\cref{theorem:线性空间的同构.线性空间同构的充分必要条件} 表明,
对于域\(F\)上的全体有限维线性空间\[
	S = \Set{ V \given \text{$V$是域$F$上的有限维线性空间} }
\]而言,
所有维数为\(0\)的线性空间 --- \(\{\vb0\}\) --- 恰好组成一个同构类,
所有1维线性空间恰好组成一个同构类,
所有2维线性空间恰好组成一个同构类,
以此类推.
可以看出,维数决定了同构类.
因此,域\(F\)上有限维线性空间的同构类与自然数之间存在一个一一对应.

\begin{proposition}
%@see: 《高等代数(第三版 下册)》(丘维声) P94 命题3
域\(F\)上线性空间之间的一个同构的逆映射也是同构.
\begin{proof}
设\(V,V'\)都是域\(F\)上的线性空间,
\(\sigma\)是从\(V\)到\(V'\)的一个同构.
显然\(\sigma^{-1}\)是从\(V'\)到\(V\)的一个双射.
任取\(\vb\alpha',\vb\beta' \in V'\),
则存在\(\vb\alpha,\vb\beta \in V\),
使得\(\vb\alpha' = \sigma(\vb\alpha),
\vb\beta' = \sigma(\vb\beta)\).
从而\(\sigma^{-1}(\vb\alpha') = \vb\alpha,
\sigma^{-1}(\vb\beta') = \vb\beta\).
于是\begin{align*}
	\sigma^{-1}(\vb\alpha' + \vb\beta')
	&= \sigma^{-1}(\sigma(\vb\alpha) + \sigma(\vb\beta)) \\
	&= \sigma^{-1}(\sigma(\vb\alpha + \vb\beta)) \\
	&= (\sigma^{-1} \circ \sigma)(\vb\alpha + \vb\beta) \\
	&= \vb1_V (\vb\alpha + \vb\beta) \\
	% \(\vb1_V\)是\(V\)上的恒等变换
	&= \vb\alpha + \vb\beta \\
	&= \sigma^{-1}(\vb\alpha') + \sigma^{-1}(\vb\beta'), \\
	\sigma^{-1}(k \vb\alpha')
	&= \sigma^{-1}(k \sigma(\vb\alpha)) \\
	&= \sigma^{-1}(\sigma(k \vb\alpha)) \\
	&= \sigma^{-1}(\sigma(k \vb\alpha)) \\
	&= k \vb\alpha
	= k \sigma^{-1}(\vb\alpha').
\end{align*}
因此\(\sigma^{-1}\)是从\(V'\)到\(V\)的一个同构.
\end{proof}
\end{proposition}

\section{商空间}
几何空间可以看成是由原点\(O\)为起点的所有向量组成的\(3\)维实线性空间\(V\).
过原点的一个平面\(W\)是\(V\)的一个\(2\)维子空间.
与\(W\)平行的每一个平面\(\pi\)都不是\(V\)的子空间,
因为\(\pi\)对加法和数量乘法都不封闭.
但是我们还是想问:\(\pi\)具有什么样的结构?\(\pi\)与\(W\)的关系如何?

在\(\pi\)上取定一个向量\(\g_0\),
\(\pi\)上每一个向量\(\g\)可以唯一地表示成\(\g_0\)与\(W\)中一个向量\(\vb\eta\)之和:
\(\g=\g_0+\vb\eta\).

反之,任取\(\vb\eta\in W\),
有\(\g_0+\vb\eta\in\pi\),
因此\(\pi=\Set{ \g_0+\vb\eta \given \vb\eta\in W }\).
我们可以把\(\pi\)记作\(\g_0+W\),
称其为“\(W\)的一个\DefineConcept{陪集}”,
把\(\g_0\)称为\DefineConcept{陪集代表}.
显然\begin{align*}
	\g\in\g_0+W
	&\iff
	\g=\g_0+\vb\eta,\vb\eta\in W \\
	&\iff
	\g-\g_0=\vb\eta\in W.
\end{align*}
由此看出,
如果在\(V\)上规定一个二元关系\(\sim\)满足\[
	\g\sim\g_0
	\defiff
	\g-\g_0\in W,
\]
那么容易验证关系\(\sim\)具有反身性、对称性和传递性,
这就是说关系\(\sim\)是等价关系,
于是\(\g_0\)所属的等价类\(\overline{\g_0}\)为\begin{align*}
	\overline{\g_0}
	&=\Set{ \g\in V \given \g\sim\g_0 } \\
	&=\Set{ \g\in V \given \g-\g_0\in W } \\
	&=\Set{ \g\in V \given \g=\g_0+\vb\eta,\vb\eta\in W } \\
	&=\Set{ \g_0+\vb\eta \given \vb\eta\in W }
	=\g_0+W.
\end{align*}
这表明陪集\(\g_0+W\)是等价类\(\overline{\g_0}\).
\(W\)本身也是\(W\)的一个陪集\(0+W\).

综上所述,
在几何空间\(V\)中,
与\(W\)平行或重合的每一个平面\(\pi\)是\(W\)的一个陪集,
也是等价关系\(\sim\)下的一个等价类.
所有等价类(即所有与\(W\)平行或重合的平面)组成的集合是几何空间的一个划分.
利用这个划分可以研究几何空间的结构.
受此启发,我们能不能给出线性空间\(V\)的一个划分,
然后利用这个划分来研究线性空间\(V\)的结构呢?
我们已经知道,要想给出线性空间\(V\)的一个划分,
就需要在\(V\)上建立一个二元等价关系,
得到的所有等价类组成的集合就是\(V\)的一个划分.

设\(V\)是域\(F\)上的一个线性空间,\(W\)是\(V\)的一个子空间.
在\(V\)上定义一个二元关系\(\sim\)满足\[
	\a\sim\b
	\defiff
	\a-\b\in W,
\]
则\(\sim\)是一个等价关系.


\chapter{线性映射}
我们在上一章研究了域\(F\)上线性空间的结构.
在许多数学分支和实际问题中都会遇到线性空间之间的映射,
并且这种映射保持加法和纯量乘法两个运算,
称其为\emph{线性映射}.
线性代数就是研究线性空间和线性映射的理论.
这一章我们来研究线性映射的理论.

\section{线性映射及其运算}
\subsection{线性映射的概念}
\begin{definition}
%@see: 《高等代数(第三版 下册)》(丘维声) P106 定义1
%@see: 《Linear Algebra Done Right (Fourth Eidition)》(Sheldon Axler) P52 3.1
设\(V\)和\(V'\)都是域\(F\)上的线性空间,
\(\vb{A}\)是从\(V\)到\(V'\)的一个映射.

如果\begin{equation*}
	(\forall\alpha,\beta\in V)
	[\vb{A}(\alpha+\beta)=\vb{A}(\alpha)+\vb{A}(\beta)],
\end{equation*}
则称“\(\vb{A}\)适合\DefineConcept{可加性}(additivity)”.

如果\begin{equation*}
	(\forall\alpha\in V)
	(\forall k\in F)
	[\vb{A}(k\alpha)=k\vb{A}(\alpha)],
\end{equation*}
则称“\(\vb{A}\)适合\DefineConcept{齐次性}(homogeneity)”.

如果\(\vb{A}\)适合可加性、齐次性,
则称“\(\vb{A}\)是从\(V\)到\(V'\)的一个\DefineConcept{线性映射}%
(\(\vb{A}\) is a \emph{linear map} from \(V\) to \(V'\))”.
%@see: https://mathworld.wolfram.com/LinearTransformation.html
\end{definition}

\begin{definition}
如果线性映射\(\vb{A}\)是单射,
则称“\(\vb{A}\)是\DefineConcept{单线性映射}(injective linear map)”.
\end{definition}

\begin{definition}
如果线性映射\(\vb{A}\)是满射,
则称“\(\vb{A}\)是\DefineConcept{满线性映射}(surjective linear map)”.
\end{definition}

\begin{definition}
线性空间\(V\)到自身的线性映射称为
“\(V\)上的\DefineConcept{线性变换}”.
\end{definition}

\begin{definition}\label{definition:线性映射.线性函数}
%@see: 《Linear Algebra Done Right (Fourth Eidition)》(Sheldon Axler) P105 3.108
%@see: 《高等代数(第三版 下册)》(丘维声) P161 定义1
域\(F\)上的线性空间\(V\)到\(F\)的线性映射称为
“\(V\)上的\DefineConcept{线性函数}(linear functional)”.
\end{definition}

\begin{example}
%@see: 《高等代数(第三版 下册)》(丘维声) P107 例1
设\(V\)和\(V'\)都是域\(F\)上的线性空间,
\(0'\)是\(V'\)的零元,
映射\(\vb{A}=V\times\{0'\}\).
我们把\(\vb{A}\)称为
“从\(V\)到\(V'\)的\DefineConcept{零映射}(zero)”,
记作\(\vb0\).
显然零映射\(\vb0\)是线性映射.
\end{example}

\begin{example}
%@see: 《高等代数(第三版 下册)》(丘维声) P107 例2
设\(V\)是域\(F\)上的线性空间,
映射\(\vb{A}\colon V\to V\)
满足\((\forall\alpha\in V)[\vb{A}(\alpha)=\alpha]\).
我们把\(\vb{A}\)称为
“\(V\)上的\DefineConcept{恒等变换}(the \emph{identity operator} on \(V\))”,
记作\(\vb1_V\)或\(\vb{I}\).
显然恒等变换\(\vb1_V\)是\(V\)上的一个线性变换.
\end{example}

\begin{example}
%@see: 《高等代数(第三版 下册)》(丘维声) P107 例3
给定\(k\in F\),
\(F\)上线性空间\(V\)到自身的一个映射\(\vb{k}(\alpha)=k\alpha\),
称为“\(V\)上由\(k\)决定的\DefineConcept{数乘变换}”,
它是\(V\)上的一个线性变换.
当\(k=0\)时,便得到零变换;
当\(k=1\)时,便得到恒等变换.
\end{example}

\begin{example}\label{example.线性映射.同构是线性映射}
设\(V\)与\(V'\)都是域\(F\)上的线性空间,
\(\sigma\)是从\(V\)到\(V'\)的一个同构.
根据同构的定义,\(\sigma\)是映射,且满足\begin{itemize}
	\item \((\forall\alpha,\beta \in V)
	[\sigma(\alpha+\beta)=\sigma(\alpha)+\sigma(\beta)]\),
	\item \((\forall\alpha \in V)
	(\forall k \in F)
	[\sigma(k\alpha)=k\sigma(\alpha)]\),
\end{itemize}
所以\(\sigma\)是从\(V\)到\(V'\)的一个线性映射.
\end{example}

\begin{example}\label{example:线性映射.左乘矩阵是线性映射}
%@see: 《高等代数(第三版 下册)》(丘维声) P107 例4
设\(A\)是域\(F\)上的一个\(s \times n\)矩阵,
令\begin{equation*}
	\vb{A}(\alpha)
	\defeq A\alpha,
	\quad \forall \alpha \in F^n,
\end{equation*}
则\(\vb{A}\)是从\(F^n\)到\(F^s\)的一个线性映射.
\end{example}

\begin{example}
%@see: 《高等代数(第三版 下册)》(丘维声) P107 例5
区间\((a,b)\)上的\(1\)阶连续可导函数族\(C^1(a,b)\)
是实数域\(\mathbb{R}\)上的线性空间\(\mathbb{R}^{(a,b)}\)的一个子空间.
求导运算\(\vb{D}\)是\(C^1(a,b)\)到\(\mathbb{R}^{(a,b)}\)的一个线性映射.
\end{example}

\begin{example}\label{example:线性映射.给定区间上的定积分是线性函数}
%@see: 《高等代数(第三版 下册)》(丘维声) P106
闭区间\([a,b]\)上全体连续函数\(C[a,b]\)对于函数的加法,以及数与函数的数量乘法,
成为实数域\(\mathbb{R}\)上的线性空间.
函数的定积分是从\(C[a,b]\)到\(\mathbb{R}\)的线性映射,
它具有下列性质:\begin{gather*}
	\int_a^b (f(x) + g(x)) \dd{x}
	= \int_a^b f(x) \dd{x} + \int_a^b f(x) \dd{x}, \\
	\int_a^b k f(x) \dd{x}
	= k \int_a^b f(x) \dd{x}.
\end{gather*}
这就说明函数的定积分保持加法、数量乘法两种运算.
函数的定积分是\(C[a,b]\)上的线性函数.
\end{example}

\begin{example}
%@see: 《高等代数(大学高等代数课程创新教材 第一版 下册)》(丘维声) P227 例8
%@see: 《高等代数(大学高等代数课程创新教材 第二版 下册)》(丘维声) P231 例8
设\(W\)是域\(F\)上线性空间\(V\)的一个子空间,
\(V/W\)是\(V\)对\(W\)的商空间.
从\(V\)到\(V/W\)的标准映射\[
	\vb\pi\colon V \to V/W,
	\alpha \mapsto \alpha+W
\]是一个线性映射,
它具有下列性质:\begin{gather*}
	\vb\pi(\alpha+\beta)
	= (\alpha+\beta)+W
	= (\alpha+W) + (\beta+W)
	= \vb\pi(\alpha) + \vb\pi(\beta), \\
	\vb\pi(k\alpha)
	= (k\alpha)+W
	= k(\alpha+W)
	= k\vb\pi(\alpha).
\end{gather*}
\end{example}

\begin{example}
%@see: 《高等代数(第三版 下册)》(丘维声) P112 习题9.1 1.(1)
判断\(K^3\)上的变换\[
	\vb{A}
	\begin{bmatrix}
		x_1 \\ x_2 \\ x_3
	\end{bmatrix}
	= \begin{bmatrix}
		x_1 - x_2 \\
		x_2 + x_3 \\
		x_3^2
	\end{bmatrix}
\]是不是线性变换.
\begin{solution}
取\(\alpha=(0,0,1),
k=2\),
则\[
	k\vb{A}(\alpha)
	= \begin{bmatrix}
		0 \\ 2 \\ 2
	\end{bmatrix}
	\neq
	\vb{A}(k\alpha)
	= \begin{bmatrix}
		0 \\ 2 \\ 4
	\end{bmatrix},
\]
\(\vb{A}\)不满足线性映射的定义,
于是\(\vb{A}\)不是线性变换.
\end{solution}
\end{example}

\begin{example}\label{example:线性映射.右乘矩阵是线性映射}
%@see: 《高等代数(第三版 下册)》(丘维声) P112 习题9.1 2.(1)
设\(A \in M_n(K)\),
令\[
	\vb{A}(X) \defeq X A,
	\quad \forall X \in M_n(K).
\]
判断\(\vb{A}\)是不是\(M_n(K)\)上的线性变换.
\begin{proof}
任取\(\alpha,\beta \in M_n(K),
k \in K\),
则\begin{gather*}
	\vb{A}(\alpha + \beta)
	= (\alpha + \beta) A
	= \alpha A + \beta A
	= \vb{A}(\alpha) + \vb{A}(\beta), \\
	\vb{A}(k \alpha)
	= k \alpha A
	= k \vb{A}(\alpha),
\end{gather*}
由此可见\(\vb{A}\)是\(M_n(K)\)上的线性变换.
\end{proof}
\end{example}

\begin{example}
%@see: 《高等代数(第三版 下册)》(丘维声) P112 习题9.1 2.(2)
设\(B,C \in M_n(K)\),
令\[
	\vb{A}(X) \defeq B X C,
	\quad \forall X \in M_n(K).
\]
判断\(\vb{A}\)是不是\(M_n(K)\)上的线性变换.
\begin{proof}
任取\(\alpha,\beta \in M_n(K),
k \in K\),
则\begin{gather*}
	\vb{A}(\alpha+\beta)
	= (B(\alpha+\beta))C
	= (B\alpha+B\beta)C
	= B \alpha C + B \beta C
	= \vb{A}\alpha + \vb{A}\beta, \\
	\vb{A}(k \alpha)
	= (B(k\alpha))C
	= k (B \alpha C)
	= k \vb{A}\alpha,
\end{gather*}
由此可见\(\vb{A}\)是\(M_n(K)\)上的线性变换.
\end{proof}
\end{example}

\begin{example}
%@see: 《高等代数(第三版 下册)》(丘维声) P112 习题9.1 3.
设\(a \in K\).
判断\(K[x]\)上的变换\[
	\vb{A} \defeq \Set{
		(f(x),f(x+a))
		\given
		f(x) \in K[x]
	}
\]是不是线性变换.
\begin{solution}
任取\(u(x),v(x) \in K[x]\),
任取\(k \in K\),
则\begin{gather*}
	\vb{A}(u(x)+v(x))
	= u(x+a) + v(x+a)
	= \vb{A}u(x) + \vb{A}v(x), \\
	\vb{A}(k u(x))
	= k u(x+a)
	= k \vb{A}u(x),
\end{gather*}
所以\(\vb{A}\)是\(K[x]\)上的线性变换.
\end{solution}
\end{example}

\begin{example}
%@see: 《高等代数(第三版 下册)》(丘维声) P112 习题9.1 4.
%@see: 《高等代数(大学高等代数课程创新教材 第二版 下册)》(丘维声) P237 例2
在正实数集\(\mathbb{R}^+\)上定义加法、数量乘法:\begin{gather*}
	\oplus \defeq \Set{
		((a,b),ab)
		\given
		a,b \in \mathbb{R}^+
	}, \\
	\odot \defeq \Set{
		((k,a),a^k)
		\given
		a \in \mathbb{R}^+,
		k \in \mathbb{R}
	}.
\end{gather*}
设\(a>0\)且\(a\neq1\).
判断从\(\mathbb{R}^+\)到\(\mathbb{R}\)的映射\[
	\vb{A} \defeq \Set{
		(x,\log_a x)
		\given
		x \in \mathbb{R}^+
	}
\]是不是线性映射.
\begin{solution}
任取\(u,v\in\mathbb{R}^+\),
任取\(k\in\mathbb{R}\),
则\begin{gather*}
	\vb{A}(u \oplus v)
	= \log_a (u v)
	= \log_a u + \log_a v
	= \vb{A}u + \vb{A}v, \\
	\vb{A}(k \odot u)
	= \log_a u^k
	= k \log_a u
	= k \vb{A}u,
\end{gather*}
所以\(\vb{A}\)是从\((\mathbb{R}^+,\oplus,\odot)\)到\((\mathbb{R},+,\cdot)\)的线性映射.
\end{solution}
%@see: 《高等代数(第三版 下册)》(丘维声) P81 习题8.1 1.(2)
\end{example}

\begin{example}
%@see: 《高等代数(第三版 下册)》(丘维声) P113 习题9.1 5.
%@see: 《高等代数(大学高等代数课程创新教材 第二版 下册)》(丘维声) P237 例3
设\(V\)是\(K[x,y]\)中所有\(m\)次齐次多项式组成的集合,
它对于多项式的加法,以及数与多项式的乘法,成为数域\(K\)上的一个线性空间.
给定数域\(K\)上的一个2阶矩阵\begin{equation*}
	A = \begin{bmatrix}
		a_{11} & a_{12} \\
		a_{21} & a_{22}
	\end{bmatrix}.
\end{equation*}
定义:\begin{equation*}
	\vb{A} \defeq \Set{
		(f(x,y),f(a_{11}x+a_{21}y,a_{12}x+a_{22}y))
		\given
		f(x,y) \in V
	}.
\end{equation*}
判断\(\vb{A}\)是不是\(V\)上的一个线性变换.
\begin{solution}
显然\(\vb{A}\)是一个映射.
任取\(u(x,y),v(x,y) \in K[x,y]\),
任取\(k \in K\),
则\begin{align*}
	\vb{A}(u(x,y) + v(x,y))
	&= u(a_{11}x+a_{21}y,a_{12}x+a_{22}y) + v(a_{11}x+a_{21}y,a_{12}x+a_{22}y) \\
	&= \vb{A}u(x,y) + \vb{A}v(x,y), \\
	\vb{A}(k u(x,y))
	&= k u(a_{11}x+a_{21}y,a_{12}x+a_{22}y) \\
	&= k \vb{A}u(x,y),
\end{align*}
所以\(\vb{A}\)是\(V\)上的线性变换.
\end{solution}
\end{example}

\begin{example}
%@see: 《高等代数(第三版 下册)》(丘维声) P113 习题9.1 6.
把\(2^m\)个元素的有限域\(F_{2^m}\)看成\(F_2\)上的线性空间.
定义:\begin{equation*}
	\vb{A} \defeq \Set{
		(x,x^2)
		\given
		x \in F_{2m}
	}.
\end{equation*}
判断\(\vb{A}\)是不是\(F_{2^m}\)上的一个线性变换.
%TODO
\end{example}

\begin{example}
%@see: 《高等代数(第三版 下册)》(丘维声) P113 习题9.1 7.
%@see: 《高等代数(大学高等代数课程创新教材 第二版 下册)》(丘维声) P238 例6
定义:\begin{equation*}
	\vb{A} \defeq \Set{
		(f(x),x f(x))
		\given
		f(x) \in K[x]
	}.
\end{equation*}
证明:\begin{itemize}
	\item \(\vb{A}\)是\(K[x]\)上的一个线性变换;
	\item \(\vb{D}\vb{A}-\vb{A}\vb{D}=\vb{I}\),其中\(\vb{D}\)表示求导数.
\end{itemize}
%TODO
\end{example}

\subsection{线性映射的性质}
由于线性映射只比同构映射少了双射这一条件,
因此同构映射的性质中,
只要它的证明没有用到单射和满射的条件,
那么对于线性映射也成立.
\begin{property}
%@see: 《高等代数(第三版 下册)》(丘维声) P107
%@see: 《Linear Algebra Done Right (Fourth Eidition)》(Sheldon Axler) P56 3.10
设\(V,V'\)都是域\(F\)上的线性空间,
\(\vb{A}\)是从\(V\)到\(V'\)的线性映射,
则\(\vb{A}\)有下述性质:
\begin{itemize}
	\item \(\vb{A}(0)=0'\),
	其中\(0\)和\(0'\)分别是\(V\)和\(V'\)的零元.

	\item \((\forall\alpha\in V)[\vb{A}(-\alpha)=-\vb{A}(\alpha)]\).

	\item \(\vb{A}(k_1\alpha_1+\dotsb+k_s\alpha_s)
	=k_1\vb{A}(\alpha_1)+\dotsb+k_s\vb{A}(\alpha_s)\).

	\item 如果\(\AutoTuple{\alpha}{s}\)是\(V\)的一个线性相关的向量组,
	则\(\vb{A}(\alpha_1),\dotsc,\vb{A}(\alpha_s)\)是\(V'\)的一个线性相关的向量组;
	但是反之不成立(线性映射可以把线性无关向量组变为线性相关向量组).

	\item 如果\(V\)是有限维的,
	且\(\AutoTuple{\alpha}{s}\)是\(V\)的一个基,
	则对于\(V\)中任一向量\(\alpha=k_1\alpha_1+\dotsb+k_s\alpha_s\),
	有\[
		\vb{A}(\alpha)
		=k_1\vb{A}(\alpha_1)+\dotsb+k_s\vb{A}(\alpha_s).
	\]
	这表明,只要知道了\(V\)的一个基\(\AutoTuple{k}{s}\)在\(\vb{A}\)下的象,
	那么\(V\)中任一向量在\(\vb{A}\)下的象就都确定了.
	或者说,\(n\)维线性空间\(V\)到\(V'\)的线性映射完全被它在\(V\)的一个基上的作用所决定.
\end{itemize}
\end{property}

\subsection{线性映射的存在性}
给了域\(F\)上任意两个线性空间\(V\)和\(V'\),
是否存在\(V\)到\(V'\)的一个线性映射?
如果\(V\)是有限维的,
那么回答是肯定的,
我们有下述结论.
\begin{theorem}\label{theorem:线性映射.线性映射的存在性}
%@see: 《高等代数(第三版 下册)》(丘维声) P108 定理1
设\(V\)和\(V'\)都是域\(F\)上的线性空间,
\(V\)的维数是\(n\),
\(V\)中取一个基\(\AutoTuple{\alpha}{n}\),
\(V'\)中任意取定\(n\)个向量\(\AutoTuple{\gamma}{n}\),
令\[
	\vb{A}\colon V\to V',
	\alpha=\sum_{i=1}^n k_i\alpha_i
	\mapsto
	\sum_{i=1}^n k_i\gamma_i,
\]
则\(\vb{A}\)是\(V\)到\(V'\)的一个线性映射,
且\(\vb{A}(\alpha_i)=\gamma_i\ (i=1,2,\dotsc,n)\).
\begin{proof}
由于\(\AutoTuple{\alpha}{n}\)是\(V\)的一个基,
因此\(\alpha\)表示成\(\AutoTuple{\alpha}{n}\)的线性组合的方式唯一,
从而\(\vb{A}\)是从\(V\)到\(V'\)的一个映射.
在\(V\)中任取两个向量\[
	\alpha = \sum_{i=1}^n a_i \alpha_i,
	\qquad
	\beta = \sum_{i=1}^n b_i \alpha_i,
\]
则\begin{align*}
	\vb{A}(\alpha + \beta)
	&= \vb{A}\left( \sum_{i=1}^n (a_i + b_i) \alpha_i \right) \\
	&= \sum_{i=1}^n (a_i + b_i) \gamma_i \\
	&= \sum_{i=1}^n a_i \gamma_i
		+ \sum_{i=1}^n b_i \gamma_i \\
	&= \vb{A}(\alpha) + \vb{A}(\beta), \\
	\vb{A}(k \alpha)
	&= \vb{A}\left( \sum_{i=1}^n (k a_i) \alpha_i \right) \\
	&= \sum_{i=1}^n (k a_i) \gamma_i \\
	&= k \sum_{i=1}^n a_i \gamma_i \\
	&= k \vb{A}(\alpha),
	\quad k \in F.
\end{align*}
因此\(\vb{A}\)是从\(V\)到\(V'\)的一个线性映射.
显然有\[
	\vb{A}(\alpha_i)
	= \vb{A}(0\alpha_1 + \dotsb + 0\alpha_{i-1}
		+ 1\alpha_i + 0\alpha_{i+1} + \dotsb + 0\alpha_n)
	= \gamma_i,
\]
其中\(i=1,2,\dotsc,n\).
\end{proof}
\end{theorem}
\begin{remark}
由于\(V\)到\(V'\)的线性映射完全被它在\(V\)上的一个基上的作用所决定,
因此\cref{theorem:线性映射.线性映射的存在性} 中满足\begin{equation*}
	\vb{A}(\alpha_i)=\gamma_i
	\quad(i=1,2,\dotsc,n)
\end{equation*}
的线性映射是唯一的.
\end{remark}

\subsection{投影的概念}
\begin{definition}\label{definition:线性映射.平行于某个子空间在另一个子空间的投影}
%@see: 《高等代数(第三版 下册)》(丘维声) P108 定理2
设\(V\)是域\(F\)上的一个线性空间,
\(U,W\)是\(V\)的两个子空间,
且\(V=U \DirectSum W\).
把\[
	\vb{P}_U
	\defeq
	\Set{
		(\alpha,\alpha_1)
		\given
		\alpha \in V,
		\alpha_1 \in U,
		(\exists \alpha_2 \in W)
		[\alpha=\alpha_1+\alpha_2]
	}
\]
称为“\(V\)平行于\(W\)在\(U\)上的\DefineConcept{投影}”.
\end{definition}
\begin{remark}
\cref{definition:线性映射.平行于某个子空间在另一个子空间的投影}
强调“平行于\(W\)”
是因为从\cref{example:线性空间.子空间.直和.例1}
可以知道\(\alpha_1\)的取值是由\(U,W\)以及\(\alpha\)共同决定的.
\end{remark}
\begin{remark}
类似地,可以定义\[
	\vb{P}_W
	\defeq
	\Set{
		(\alpha,\alpha_2)
		\given
		\alpha \in V,
		\alpha_2 \in W,
		(\exists \alpha_1 \in U)
		[\alpha=\alpha_1+\alpha_2]
	},
\]
并称之为“\(V\)平行于\(U\)在\(W\)上的投影”.
\end{remark}

\begin{theorem}\label{theorem:线性映射.投影是线性变换}
%@see: 《高等代数(第三版 下册)》(丘维声) P108 定理2
设\(V\)是域\(F\)上的一个线性空间,
\(U,W\)是\(V\)的两个子空间,
且\(V=U \DirectSum W\),
则\(V\)平行于\(W\)在\(U\)上的投影
\(\vb{P}_U\)是\(V\)上的一个线性变换,
且满足\[
%@see: 《高等代数(第三版 下册)》(丘维声) P108 (7)
	\vb{P}_U(\alpha)
	=\left\{ \begin{array}{ll}
		\alpha, & \alpha\in U, \\
		0, & \alpha\in W.
	\end{array} \right.
\]
\begin{proof}
由于\(V = U \DirectSum W\),
因此\(V\)中任意一个向量\(\alpha\)表示成\(U\)的一个向量与\(W\)的一个向量之和的方式唯一,
关系\(\vb{P}_U\)是单值的,
于是\(\vb{P}_U\)是从\(V\)到\(V\)的一个映射.
任取\(V\)中两个向量\[
	\alpha = \alpha_1 + \alpha_2,
	\qquad
	\beta = \beta_1 + \beta_2,
\]
其中\(\alpha_1,\beta_1 \in U,
\alpha_2,\beta_2 \in W\),
则\[
	\alpha_1 + \beta_1
	\in U,
	\qquad
	\alpha_2 + \beta_2
	\in W,
\]
从而\begin{align*}
	\vb{P}_U(\alpha + \beta)
	&= \vb{P}_U((\alpha_1 + \beta_1) + (\alpha_2 + \beta_2)) \\
	&= \alpha_1 + \beta_1 \\
	&= \vb{P}_U(\alpha) + \vb{P}_U(\beta), \\
	\vb{P}_U(k \alpha)
	&= \vb{P}_U(k \alpha_1 + k \alpha_2) \\
	&= k \alpha_1 \\
	&= k \vb{P}_U(\alpha),
	\quad k \in F,
\end{align*}
因此\(\vb{P}_U\)是\(V\)上的一个线性变换.

如果\(\alpha \in U\),
则\(\alpha = \alpha + 0\),
从而\(\vb{P}_U(\alpha) = \alpha\).

如果\(\alpha \in W\),
则\(\alpha = 0 + \alpha\),
从而\(\vb{P}_U(\alpha) = 0\).

设\(V\)上的线性变换\(\vb{A}\)也满足投影的定义,
任取\(\alpha \in V\),
设\[
	\alpha = \alpha_1 + \alpha_2,
	\quad
	\alpha_1 \in U,
	\alpha_2 \in W,
\]
则\[
	\vb{A}(\alpha)
	= \vb{A}(\alpha_1 + \alpha_2)
	= \vb{A}(\alpha_1) + \vb{A}(\alpha_2)
	= \alpha_1 + 0
	= \alpha_1
	= \vb{P}_U(\alpha),
\]
因此\(\vb{A} = \vb{P}_U\).
\end{proof}
\end{theorem}

\cref{theorem:线性映射.投影是线性变换} 告诉我们,
如果线性空间\(V\)可以分解成两个子空间的直和\(V = U \DirectSum W\),
那么\(V\)在子空间\(U\)上的投影\(\vb{P}_U\)
和\(V\)在子空间\(W\)上的投影\(\vb{P}_W\)
就都是\(V\)上的线性变换,
并且有\begin{equation*}
	\vb{P}_U(\alpha)
	=\left\{ \begin{array}{ll}
		\alpha, & \alpha\in U, \\
		0, & \alpha\in W,
	\end{array} \right.
	\qquad
	\vb{P}_W(\alpha)
	=\left\{ \begin{array}{ll}
		\alpha, & \alpha\in W, \\
		0, & \alpha\in U.
	\end{array} \right.
\end{equation*}
投影是非常重要的一类线性变换.

为求简便,对于任意一个线性映射\(\vb{A}\),任意一个向量\(\alpha\),
以后我们都用\(\vb{A}\alpha\)代替\(\vb{A}(\alpha)\).

\subsection{幂等变换,正交变换}
\begin{definition}
%@see: 《高等代数(第三版 下册)》(丘维声) P109
线性变换\(\vb{A}\)如果满足\(\vb{A}^2=\vb{A}\),
则称“\(\vb{A}\)是\DefineConcept{幂等变换}”.
\end{definition}

\begin{definition}
%@see: 《高等代数(第三版 下册)》(丘维声) P109
两个线性变换\(\vb{A},\vb{B}\)
如果满足\(\vb{A} \vb{B}=\vb{B} \vb{A}=\vb0\),
则称“\(\vb{A}\)与\(\vb{B}\)是\DefineConcept{正交的}”.
\end{definition}

\subsection{线性映射的加法、纯量乘法}
\begin{definition}
%@see: 《Linear Algebra Done Right (Fourth Eidition)》(Sheldon Axler) P52 3.2
%@see: 《高等代数(第三版 下册)》(丘维声) P110
设\(V,V'\)都是域\(F\)上的线性空间.

\begin{itemize}
	\item 从\(V\)到\(V'\)的所有线性映射组成的集合,
	% 称为“域\(F\)上从线性空间\(V\)到\(V'\)的\DefineConcept{线性映射空间}”,
	记作\(\Hom(V,V')\).

	\item \(V\)上的所有线性变换组成的集合,
	% 称为“域\(F\)上线性空间\(V\)上的\DefineConcept{线性变换空间}”,
	记作\(\Hom(V,V)\)\footnote{
		% 《高等代数(第四版)》(谢启鸿 姚慕生)
		在有的书上,\(\Hom(V,V')\)和\(\Hom(V,V)\)
		分别记作\(\mathcal{L}(V,V')\)和\(\mathcal{L}(V) = \mathcal{L}(V,V)\).
	}.
\end{itemize}
\end{definition}

\begin{definition}
%@see: 《高等代数(第三版 下册)》(丘维声) P110 命题5
%@see: 《Linear Algebra Done Right (Fourth Eidition)》(Sheldon Axler) P55 3.5
设\(V,V'\)都是域\(F\)上的线性空间.
对于\(\forall\vb{A},\vb{B}\in\Hom(V,V'),
\forall\alpha\in V,
\forall k\in F\),
定义:\begin{gather*}
	%@see: 《高等代数(第三版 下册)》(丘维声) P110 (9)
	(\vb{A}+\vb{B})\alpha
	\defeq
	\vb{A}\alpha+\vb{B}\alpha, \\
	%@see: 《高等代数(第三版 下册)》(丘维声) P110 (10)
	(k\vb{A})\alpha
	\defeq
	k(\vb{A}\alpha),
\end{gather*}
把\(\vb{A}+\vb{B}\)称为“\(\vb{A}\)与\(\vb{B}\)的\DefineConcept{和}(sum)”,
把\(k\vb{A}\)称为“\(k\)与\(\vb{A}\)的\DefineConcept{纯量乘积}(product)”.
\end{definition}

\begin{proposition}
%@see: 《高等代数(第三版 下册)》(丘维声) P110 命题5
设\(\vb{A},\vb{B}\)都是域\(F\)上线性空间\(V\)到\(V'\)的线性映射,
\(k\in F\),
则\begin{itemize}
	\item \(\vb{A}\)与\(\vb{B}\)的和\(\vb{A}+\vb{B}\)是\(V\)到\(V'\)的线性映射,
	\item \(k\)与\(\vb{A}\)的纯量乘积\(k\vb{A}\)是\(V\)到\(V'\)的线性映射.
\end{itemize}
\begin{proof}
显然\(\vb{A}+\vb{B}\)是从\(V\)到\(V'\)的一个映射.
对于任意\(\alpha,\beta \in V,
l \in F\),
有\begin{align*}
	(\vb{A}+\vb{B})(\alpha+\beta)
	&= \vb{A}(\alpha+\beta) + \vb{B}(\alpha+\beta) \\
	&= \vb{A}\alpha + \vb{A}\beta + \vb{B}\alpha + \vb{B}\beta \\
	&= (\vb{A}+\vb{B})\alpha + (\vb{A}+\vb{B})\beta, \\
	(\vb{A}+\vb{B})(l\alpha)
	&= \vb{A}(l\alpha) + \vb{B}(l\alpha) \\
	&= l\vb{A}\alpha + l\vb{B}\alpha \\
	&= l(\vb{A}\alpha + \vb{B}\alpha) \\
	&= l(\vb{A}+\vb{B}) \alpha,
\end{align*}
因此\(\vb{A}+\vb{B}\)是从\(V\)到\(V'\)的线性映射.

同理可证\(k\vb{A}\)是从\(V\)到\(V'\)的线性映射.
\end{proof}
\end{proposition}

\begin{proposition}
%@see: 《高等代数(第三版 下册)》(丘维声) P110
%@see: 《Linear Algebra Done Right (Fourth Eidition)》(Sheldon Axler) P55 3.6
设\(V\)和\(V'\)都是域\(F\)上的线性空间,
则从\(V\)到\(V'\)的所有线性映射组成的集合\(\Hom(V,V')\),
对线性映射的加法与纯量乘法,
成为域\(F\)上的一个线性空间.
\end{proposition}
\begin{corollary}
%@see: 《高等代数(第三版 下册)》(丘维声) P111
设\(V\)是域\(F\)上的线性空间,
则在\(V\)上的所有线性变换组成的集合\(\Hom(V,V)\),
对线性映射的加法与纯量乘法,
成为域\(F\)上的一个线性空间.
\end{corollary}

\subsection{线性映射的乘法}
\begin{definition}
%@see: 《Linear Algebra Done Right (Fourth Eidition)》(Sheldon Axler) P55 3.7
设\(V,U,W\)都是域\(F\)上的线性空间.
对于\(\forall\vb{A}\in\Hom(V,U),
\forall\vb{B}\in\Hom(U,W),
\forall\alpha \in V\),
定义:\begin{equation*}
	(\vb{B}\vb{A})\alpha
	\defeq (\vb{B} \circ \vb{A})\alpha
	= \vb{B}(\vb{A}\alpha),
\end{equation*}
把\(\vb{B}\vb{A}\)称为“\(\vb{B}\)与\(\vb{A}\)的\DefineConcept{积}”.
\end{definition}

\begin{proposition}
%@see: 《高等代数(第三版 下册)》(丘维声) P109 命题3
设\(V,U,W\)都是域\(F\)上的线性空间,
\(\vb{A}\)是\(V\)到\(U\)的一个线性映射,
\(\vb{B}\)是\(U\)到\(W\)的一个线性映射,
则\(\vb{B}\vb{A}\)是\(V\)到\(W\)的一个线性映射.
\begin{proof}
显然\(\vb{B}\vb{A}\)是从\(V\)到\(W\)的一个映射.
任取\(\alpha,\beta \in V,
k \in F\),
有\begin{align*}
	(\vb{B}\vb{A})(\alpha+\beta)
	&= \vb{B}(\vb{A}(\alpha+\beta)) \\
	&= \vb{B}(\vb{A}\alpha+\vb{A}\beta), \\
	(\vb{B}\vb{A})(k\alpha)
	&= \vb{B}(\vb{A}(k \alpha)) \\
	&= \vb{B}(k \vb{A}\alpha) \\
	&= k(\vb{B}(\vb{A}\alpha)) \\
	&= k((\vb{B}\vb{A}) \alpha),
\end{align*}
因此\(\vb{B}\vb{A}\)是从\(V\)到\(W\)的一个线性映射.
\end{proof}
\end{proposition}

\begin{proposition}\label{theorem:线性映射.线性映射的乘法适合结合律}
%@see: 《高等代数(第三版 下册)》(丘维声) P110
%@see: 《Linear Algebra Done Right (Fourth Eidition)》(Sheldon Axler) P56 3.8
线性映射的乘法适合结合律,
即\begin{equation*}
	(\forall\vb{A}\in\Hom(V_2,V_1))
	(\forall\vb{B}\in\Hom(V_3,V_2))
	(\forall\vb{C}\in\Hom(V_4,V_3))
	[
		(\vb{A}\vb{B})\vb{C}
		= \vb{A}(\vb{B}\vb{C})
	].
\end{equation*}
\begin{proof}
因为线性映射是映射,
所以由\cref{example:映射.映射的复合适合结合律} 可知,
映射的乘法适合结合律.
\end{proof}
\end{proposition}
\begin{proposition}\label{theorem:线性映射.线性映射的乘法不适合交换律}
%@see: 《高等代数(第三版 下册)》(丘维声) P110
%@see: 《Linear Algebra Done Right (Fourth Eidition)》(Sheldon Axler) P56 3.9
线性映射的乘法不适合交换律.
\begin{proof}
下面举出反例.
\def\MyPolynomialRing{\mathbb{R}[x]}%
\def\MyLinearMapSpace{\Hom(\MyPolynomialRing,\MyPolynomialRing)}%
设\(\vb{D}\in\MyLinearMapSpace\)表示对多项式求导数,
\(\vb{T}\in\MyLinearMapSpace\)表示给多项式乘以\(x^2\)因式,
那么对于\(\forall p(x) \in \MyPolynomialRing\)
有\begin{equation*}
	(\vb{T}\vb{D})p(x)
	= x^2 p'(x),
	\qquad
	(\vb{D}\vb{T})p(x)
	= x^2 p'(x) + 2x p(x).
\end{equation*}
因此\(\vb{T}\vb{D} \neq \vb{D}\vb{T}\).
\end{proof}
\end{proposition}
\begin{definition}
设\(V\)是域\(F\)上的线性空间,
\(\vb{A},\vb{B}\)都是\(V\)上的线性变换.
如果\(\vb{A}\vb{B}=\vb{B}\vb{A}\),
则称“\(\vb{A}\)与\(\vb{B}\) \DefineConcept{可交换}”.
\end{definition}

由\cref{theorem:线性映射.线性映射的乘法适合结合律} 可知,
线性变换的乘法满足结合律:\begin{equation*}
	(\forall\vb{A},\vb{B},\vb{C}\in\Hom(V,V))
	[
		(\vb{A}\vb{B})\vb{C}
		= \vb{A}(\vb{B}\vb{C})
	].
\end{equation*}
恒等变换\(\vb{I}\)满足\begin{equation*}
	(\forall\vb{A}\in\Hom(V,V))
	[\vb{I}\vb{A}=\vb{A}\vb{I}=\vb{A}].
\end{equation*}
容易验证,线性变换的乘法对于加法还满足左、右分配律:\begin{gather*}
	(\forall\vb{A},\vb{B},\vb{C}\in\Hom(V,V))
	[\vb{A}(\vb{B}+\vb{C})=\vb{A}\vb{B}+\vb{A}\vb{C}], \\
	(\forall\vb{A},\vb{B},\vb{C}\in\Hom(V,V))
	[(\vb{B}+\vb{C})\vb{A}=\vb{B}\vb{A}+\vb{C}\vb{A}].
\end{gather*}
综上所述,\(\Hom(V,V)\)的加法与乘法满足环定义的6条运算法则,
并且\(\vb{I}\)是\(\Hom(V,V)\)的乘法单位元,
因此\(\Hom(V,V)\)对于加法和乘法成为一个有单位元的环.
容易验证,
线性变换的乘法与纯量乘法满足\begin{equation*}
%@see: 《高等代数(第三版 下册)》(丘维声) P111 (11)
	(\forall k\in F)
	(\forall\vb{A},\vb{B}\in\Hom(V,V))
	[
		k(\vb{A}\vb{B})
		=(k\vb{A})\vb{B}
		=\vb{A}(k\vb{B})
	].
\end{equation*}

\begin{definition}
%@see: 《Linear Algebra Done Right (Fourth Eidition)》(Sheldon Axler) P82 3.59
%@see: 《Linear Algebra Done Right (Fourth Eidition)》(Sheldon Axler) P82 3.61
%@see: 《线性代数》(李炯生 查建国) P184
%@see: 《线性代数》(李炯生 查建国) P196
%@see: 《高等代数学(第四版)》(谢启鸿 姚慕生 吴泉水) P184
给定\(\vb{A}\in\Hom(V,W)\).
如果存在\(\vb{B}\in\Hom(W,V)\),
使得\(\vb{B}\vb{A}=\vb{I}_V\)且\(\vb{A}\vb{B}=\vb{I}_W\),
其中\(\vb{I}_V\)是\(V\)上的恒等变换,
\(\vb{I}_W\)是\(W\)上的恒等变换,
则称“\(\vb{A}\) \DefineConcept{可逆}(invertible)”
或“\(\vb{A}\)有\DefineConcept{可逆性}(invertibility)”,
称“\(\vb{B}\)是\(\vb{A}\)一个的\DefineConcept{逆}(\(\vb{B}\) is the \emph{inverse} of \(\vb{A}\))”,
记作\(\vb{A}^{-1}\).
\end{definition}
\begin{proposition}\label{theorem:线性映射.可逆线性映射有唯一逆}
%@see: 《Linear Algebra Done Right (Fourth Eidition)》(Sheldon Axler) P82 3.60
可逆线性映射有唯一逆.
\begin{proof}
假设\(\vb{B}_1,\vb{B}_2\in\Hom(W,V)\)都是可逆线性映射\(\vb{A}\in\Hom(V,W)\)的逆,
那么\begin{equation*}
	\vb{B}_1 = \vb{B}_1 \vb{I}
	= \vb{B}_1 (\vb{A} \vb{B}_2)
	= (\vb{B}_1 \vb{A}) \vb{B}_2
	= \vb{I} \vb{B}_2
	= \vb{B}_2.
	\qedhere
\end{equation*}
\end{proof}
\end{proposition}

\begin{proposition}\label{theorem:线性映射.可逆线性映射是同构}
%@see: 《Linear Algebra Done Right (Fourth Eidition)》(Sheldon Axler) P83 3.63
设\(V,V'\)都是域\(F\)上的线性空间,
映射\(\sigma\colon V \to V'\),
则\[
	\text{$\sigma$是可逆线性映射}
	\iff
	\text{$\sigma$是同构}.
\]
\begin{proof}
假设\(\sigma\)是可逆线性映射,
则存在线性映射\(\rho\colon V' \to V\)使得\(\rho\sigma = \vb{I}_V\).
由\cref{example:映射.可逆映射是双射} 可知
\(\sigma\)是双射,
于是\(\sigma\)是同构.

假设\(\sigma\)是同构,
由于\hyperref[theorem:线性空间.线性空间的同构的逆是同构]{线性空间的同构的逆映射是同构},
所以\(\rho \defeq \sigma^{-1}\)是同构,
%\cref{example.线性映射.同构是线性映射}
从而\(\rho\)是线性映射,
因此\(\sigma\)是可逆线性映射.
\end{proof}
\end{proposition}

\begin{proposition}
%@see: 《高等代数(第三版 下册)》(丘维声) P110
%@see: 《线性代数》(李炯生 查建国) P196 定理5.3.2
设\(V,V'\)都是域\(F\)上有限维线性空间.
\(V\)到\(V'\)的可逆线性映射存在的充分必要条件是
\(\dim V=\dim V'\).
\begin{proof}
由\cref{theorem:线性映射.可逆线性映射是同构} 可知,
\(V\)到\(V'\)的可逆线性映射存在,
当且仅当\(V\)与\(V'\)同构.
由\cref{theorem:线性空间的同构.线性空间同构的充分必要条件} 可知,
\(V\)与\(V'\)同构的充分必要条件是它们的维数相同.
\end{proof}
\end{proposition}

\begin{proposition}
%@see: 《高等代数(第三版 下册)》(丘维声) P110 命题4
%@see: 《高等代数(第三版 下册)》(丘维声) P107 例6
%@see: 《线性代数》(李炯生 查建国) P196 定理5.3.3
设\(V,W\)都是域\(F\)上的线性空间,
\(\vb{A}\)是从\(V\)到\(W\)的一个线性映射.
如果\(\vb{A}\)可逆,
则\(\vb{A}\)的逆\(\vb{A}^{-1}\)是从\(W\)到\(V\)的一个线性映射.
\begin{proof}
直接有\begin{align*}
	&\text{$\vb{A}$是从$V$到$W$的可逆线性映射} \\
	&\iff \text{$\vb{A}$是从$V$到$W$的同构} \\
	&\implies \text{$\vb{A}^{-1}$是从$W$到$V$的同构} \\
	&\implies \text{$\vb{A}^{-1}$是从$W$到$V$的可逆线性映射}.
	\qedhere
\end{align*}
\end{proof}
%@see: https://mathworld.wolfram.com/InvertibleLinearMap.html
\end{proposition}

\begin{proposition}
%@see: 《线性代数》(李炯生 查建国) P196 定理5.3.4
设\(U,V,W\)都是域\(F\)上的线性空间,
\(\vb{A}\colon U \to V,
\vb{B}\colon V \to W\)都是可逆线性映射,
则\(\vb{B}\vb{A}\)可逆.
%TODO proof
\end{proposition}

由于线性变换的乘法满足结合律,
因此可以定义线性变换\(\vb{A}\)的正整数指数幂:\begin{equation*}
%@see: 《高等代数(第三版 下册)》(丘维声) P111 (13)
	\vb{A}^m
	\defeq
	\underbrace{\vb{A}\cdot\vb{A}\cdot\dotsm\cdot\vb{A}}_{\text{$m$个}}.
\end{equation*}
还可以定义\(\vb{A}\)的零次幂:\begin{equation*}
%@see: 《高等代数(第三版 下册)》(丘维声) P111 (14)
	\vb{A}^0
	\defeq
	\vb{I}.
\end{equation*}
容易验证:
对于\(\forall m,n\in\mathbb{N}\),
有\begin{gather*}
%@see: 《高等代数(第三版 下册)》(丘维声) P111 (15)
	\vb{A}^m\cdot\vb{A}^n=\vb{A}^{m+n}, \\
	(\vb{A}^m)^n=\vb{A}^{mn}.
\end{gather*}
当\(\vb{A}\)可逆时,可以定义:\begin{equation*}
%@see: 《高等代数(第三版 下册)》(丘维声) P111 (16)
	\vb{A}^{-m}
	\defeq
	(\vb{A}^{-1})^m,
	\quad m\in\mathbb{N}.
\end{equation*}

设\(f(x)=a_0+a_1 x+\dotsb+a_m x^m\)是域\(F\)上的一元多项式,
\(x\)用\(V\)上的线性变换\(\vb{A}\)代入,
得\begin{equation*}
%@see: 《高等代数(第三版 下册)》(丘维声) P111 (17)
	f(\vb{A})=a_0\vb{I}+a_1\vb{A}+\dotsb+a_m\vb{A}^m.
\end{equation*}
显然,\(f(\vb{A})\)仍是\(V\)上的一个线性变换,
称“\(f(\vb{A})\)是\(\vb{A}\)的一个多项式”.
容易验证:线性变换\(\vb{A}\)的任意两个多项式
\(f(\vb{A})\)与\(g(\vb{B})\)是可交换的,
即\begin{equation*}
%@see: 《高等代数(第三版 下册)》(丘维声) P111 (18)
	f(\vb{A}) g(\vb{A})
	=g(\vb{A}) f(\vb{A}).
\end{equation*}
把\(V\)上线性变换\(\vb{A}\)的所有多项式组成的集合记作\(F[\vb{A}]\).
容易验证:
\begin{itemize}
	\item \(F[\vb{A}]\)对于线性变换的减法、乘法都封闭,
	从而\(F[\vb{A}]\)是环\(\Hom(V,V)\)的一个子环;

	\item \(F[\vb{A}]\)是交换环;

	\item \(\vb{I}\in F[\vb{A}]\).
\end{itemize}
\(F[\vb{A}]\)中所有数乘变换组成的集合是\(F[\vb{A}]\)的一个子环,
并且域\(F\)与这个子环同构,
从而\(F[\vb{A}]\)可看成是\(F\)的一个扩环.
于是根据一元多项式环\(F[x]\)的通用性质,
\(x\)可用\(F[\vb{A}]\)的任一元素代入,
从\(F[x]\)的有关加法和乘法的等式
得到\(F[\vb{A}]\)中有关加法和乘法的相应等式.

\subsection{代数,代数的维数}
\begin{definition}
%@see: 《高等代数(第三版 下册)》(丘维声) P111 定义2
%@see: 《高等代数学(第四版)》(谢启鸿 姚慕生 吴泉水) P189 定义4.2.2
设\(A\)是域\(F\)上的线性空间,
\(A\)对加法和乘法成为一个有单位元的环,
且\[
	% 乘法与数乘的相容性
	(\forall k\in F)
	(\forall\alpha,\beta\in A)
	[
		k(\alpha\beta)
		=(k\alpha)\beta
		=\alpha(k\beta)
	],
\]
则称“\(A\)是域\(F\)上的一个\DefineConcept{代数}”,
把\(A\)的乘法单位元称为“\(A\)的\DefineConcept{恒等元}”,
把\(A\)的维数\(\dim A\)称为“代数\(A\)的\DefineConcept{维数}”.
\end{definition}
\begin{remark}
我们需要注意区分两个概念:代数\(A\)的乘法单位元,域\(F\)的乘法单位元.
\end{remark}

\begin{example}
域\(F\)上线性空间\(V\)上的所有线性变换组成的集合\(\Hom(V,V)\),
对于线性变换的加法、乘法与纯量乘法,
成为域\(F\)上的一个代数.
\end{example}

\begin{example}
域\(F\)上所有\(n\)阶矩阵组成的集合\(M_n(F)\),
对于矩阵的加法、乘法与数量乘法,
成为域\(F\)上的一个代数.
\end{example}

\subsection{线性变换的性质}
利用线性变换的运算,
我们可以研究线性变换的性质.

\begin{proposition}
%@see: 《高等代数(第三版 下册)》(丘维声) P109
设\(V\)是域\(F\)上的一个线性空间,
\(U,W\)是\(V\)的两个子空间,
且\(V=U \DirectSum W\),
则\(V\)平行于\(W\)在\(U\)上的投影\(\vb{P}_U\)
和\(V\)平行于\(U\)在\(W\)上的投影\(\vb{P}_W\)
满足\begin{gather*}
	%@see: 《高等代数(第三版 下册)》(丘维声) P109 (8)
	\vb{P}_U^2
	=\vb{P}_U, \\
	\vb{P}_U \vb{P}_W
	=\vb0, \\
	\vb{P}_W \vb{P}_U
	=\vb0, \\
	\vb{P}_W^2
	=\vb{P}_W.
\end{gather*}
\begin{proof}
\(V\)在子空间\(U\)上的投影\(\vb{P}_U\)
和\(V\)在子空间\(W\)上的投影\(\vb{P}_W\)
都是从\(V\)到\(V\)的映射,
对于\(\forall\alpha_1\in U,
\forall\alpha_2\in W\),
记\(\alpha=\alpha_1+\alpha_2\),
有\begin{gather*}
	\vb{P}_U(\vb{P}_U(\alpha))
	=\vb{P}_U(\alpha_1)
	=\alpha_1
	=\vb{P}_U(\alpha), \\
	\vb{P}_U(\vb{P}_W(\alpha))
	=\vb{P}_U(\alpha_2)
	=0, \\
	\vb{P}_W(\vb{P}_U(\alpha))
	=\vb{P}_W(\alpha_1)
	=0, \\
	\vb{P}_W(\vb{P}_W(\alpha))
	=\vb{P}_W(\alpha_2)
	=\alpha_2
	=\vb{P}_W(\alpha).
	\qedhere
\end{gather*}
\end{proof}
\end{proposition}
\begin{remark}
%@see: 《高等代数(第三版 下册)》(丘维声) P109
这就说明:
% 前提
如果\(V = U \DirectSum W\),
% 结论1
那么\(\vb{P}_U,\vb{P}_W\)都是幂等变换,
% 结论2
而且\(\vb{P}_U\)与\(\vb{P}_W\)是正交的.
\end{remark}

\begin{proposition}
%@see: 《高等代数(第三版 下册)》(丘维声) P112
设\(V\)是域\(F\)上的一个线性空间,
\(U,W\)是\(V\)的两个子空间,
且\(V=U \DirectSum W\),
则\(V\)平行于\(W\)在\(U\)上的投影\(\vb{P}_U\)
和\(V\)平行于\(U\)在\(W\)上的投影\(\vb{P}_W\)
满足\[
%@see: 《高等代数(第三版 下册)》(丘维声) P111 (19)
	\vb{P}_U+\vb{P}_W=\vb{I}.
\]
\begin{proof}
对于\(\forall\alpha_1\in U,
\alpha_2\in W\),
令\(\alpha=\alpha_1+\alpha_2\),
则\[
	(\vb{P}_U+\vb{P}_W)\alpha
	=\vb{P}_U\alpha+\vb{P}_W\alpha
	=\alpha_1+\alpha_2
	=\vb{I}\alpha,
\]
因此\(\vb{P}_U+\vb{P}_W=\vb{I}\).
\end{proof}
\end{proposition}
\begin{remark}
这就说明:
如果\(V=U \DirectSum W\),
则投影\(\vb{P}_U\)与\(\vb{P}_W\)的和等于恒等变换\(\vb{I}\).
\end{remark}

\begin{example}
%@see: 《高等代数(第三版 下册)》(丘维声) P113 习题9.1 8.
%@see: 《高等代数(大学高等代数课程创新教材 第二版 下册)》(丘维声) P238 例7
设\(V\)是域\(F\)上的一个线性空间,
\(\AutoTuple{\alpha}{n}\)是\(V\)的一个基,
\(\vb{A}\)是\(V\)上的一个线性变换.
证明:\(\vb{A}\)可逆,
当且仅当\(\AutoTuple{\vb{A}\alpha}{n}\)是\(V\)的一个基.
\begin{proof}
必要性.
假设\(\vb{A}\)可逆,
那么由\cref{theorem:线性映射.可逆线性映射是同构} 可知\(\vb{A}\)是\(V\)上的自同构.
因为\(\AutoTuple{\alpha}{n}\)是线性空间\(V\)的一个基,
所以由\cref{theorem:线性空间的同构.同构线性空间的性质5} 可知,
向量组\(\AutoTuple{\vb{A}\alpha}{n}\)也是线性空间\(V\)的一个基.

充分性.
假设\(\AutoTuple{\vb{A}\alpha}{n}\)是\(V\)的一个基,
那么由\cref{theorem:线性空间.任一向量可由给定基唯一线性表出} 可知,
\(V\)中任意一个向量\(\beta\)都有唯一的分解式\begin{equation*}
	\beta
	= x_1 \vb{A}\alpha_1 + \dotsb + x_n \vb{A}\alpha_n
	= \vb{A}(x_1 \alpha_1 + \dotsb + x_n \alpha_n),
	\quad \AutoTuple{x}{n} \in F,
\end{equation*}
% 构造逆映射
于是我们可以定义映射\(\vb{B}\colon V \to V\),使之满足\begin{equation*}
	\vb{B}(x_1 \vb{A}\alpha_1 + \dotsb + x_n \vb{A}\alpha_n)
	= x_1 \alpha_1 + \dotsb + x_n \alpha_n,
\end{equation*}
显然\(\vb{B}\)是映射.
由于\begin{align*}
	&\hspace{-20pt}
	\vb{B}(
		(x_1 \vb{A}\alpha_1 + \dotsb + x_n \vb{A}\alpha_n)
		+ (y_1 \vb{A}\alpha_1 + \dotsb + y_n \vb{A}\alpha_n)
	) \\
	&= (x_1 \alpha_1 + \dotsb + x_n \alpha_n)
		+ (y_1 \alpha_1 + \dotsb + y_n \alpha_n) \\
	&= \vb{B}(x_1 \vb{A}\alpha_1 + \dotsb + x_n \vb{A}\alpha_n)
		+ \vb{B}(y_1 \vb{A}\alpha_1 + \dotsb + y_n \vb{A}\alpha_n), \\
	&\hspace{-20pt}
	\vb{B}(k(x_1 \vb{A}\alpha_1 + \dotsb + x_n \vb{A}\alpha_n))
	= \vb{B}(k x_1 \vb{A}\alpha_1 + \dotsb + k x_n \vb{A}\alpha_n) \\
	&= k x_1 \alpha_1 + \dotsb + k x_n \alpha_n
	= k (x_1 \alpha_1 + \dotsb + x_n \alpha_n) \\
	&= k \vb{B}(x_1 \vb{A}\alpha_1 + \dotsb + x_n \vb{A}\alpha_n),
\end{align*}
所以\(\vb{B}\)是线性映射.
又因为\begin{align*}
	(\vb{B}\vb{A})(x_1 \alpha_1 + \dotsb + x_n \alpha_n)
	&= \vb{B}(x_1 \vb{A}\alpha_1 + \dotsb + x_n \vb{A}\alpha_n) \\
	&= x_1 \alpha_1 + \dotsb + x_n \alpha_n, \\
	(\vb{A}\vb{B})(x_1 \vb{A}\alpha_1 + \dotsb + x_n \vb{A}\alpha_n)
	&= \vb{A}(x_1 \alpha_1 + \dotsb + x_n \alpha_n) \\
	&= x_1 \vb{A}\alpha_1 + \dotsb + x_n \vb{A}\alpha_n,
\end{align*}
所以\(\vb{B}\vb{A} = \vb{A}\vb{B} = \vb{I}_V\),
这就说明\(\vb{A}\)可逆,且\(\vb{B}\)就是\(\vb{A}\)的逆.
\end{proof}
\end{example}

\begin{example}\label{example:线性映射.强循环向量组}
%@see: 《高等代数(第三版 下册)》(丘维声) P113 习题9.1 9.
%@see: 《高等代数(大学高等代数课程创新教材 第二版 下册)》(丘维声) P239 例8
设\(V\)是域\(F\)上的线性空间,
\(\vb{A}\)是\(V\)上的一个线性变换,
\(\alpha \in V\).
证明:如果存在正整数\(m\)使得\[
	\vb{A}^{m-1} \alpha \neq 0,
	\qquad
	\vb{A}^m \alpha = 0,
\]
则\(\alpha,\vb{A}\alpha,\vb{A}^2\alpha,\dotsc,\vb{A}^{m-1}\alpha\)线性无关.
\begin{proof}
假设存在正整数\(m\)使得\[
	\vb{A}^{m-1} \alpha \neq 0,
	\qquad
	\vb{A}^m \alpha = 0,
\]
令\begin{equation*}
	x_0 \alpha
	+ x_1 \vb{A}\alpha
	+ x_2 \vb{A}^2\alpha
	+ \dotsb
	+ x_{m-2} \vb{A}^{m-2}\alpha
	+ x_{m-1} \vb{A}^{m-1}\alpha
	= 0.
	\eqno(1)
\end{equation*}
由\(\vb{A}^m \alpha = 0\)可得\begin{equation*}
	\vb{A}^{m-1} (
		x_0 \alpha
		+ x_1 \vb{A}\alpha
		+ x_2 \vb{A}^2\alpha
		+ \dotsb
		+ x_{m-2} \vb{A}^{m-2}\alpha
		+ x_{m-1} \vb{A}^{m-1}\alpha
	)
	= x_0 \vb{A}^{m-1}\alpha
	= 0.
\end{equation*}
由于\(\vb{A}^{m-1}\alpha \neq 0\),
所以\(x_0 = 0\).
将\(x_0 = 0\)代入(1)式,得\begin{equation*}
	x_1 \vb{A}\alpha
	+ x_2 \vb{A}^2\alpha
	+ \dotsb
	+ x_{m-2} \vb{A}^{m-2}\alpha
	+ x_{m-1} \vb{A}^{m-1}\alpha
	= 0.
	\eqno(2)
\end{equation*}
再由\(\vb{A}^m \alpha = 0\)可得\begin{equation*}
	\vb{A}^{m-2} (
		x_1 \vb{A}\alpha
		+ x_2 \vb{A}^2\alpha
		+ \dotsb
		+ x_{m-2} \vb{A}^{m-2}\alpha
		+ x_{m-1} \vb{A}^{m-1}\alpha
	)
	= x_1 \vb{A}^{m-1}\alpha
	= 0.
\end{equation*}
由于\(\vb{A}^{m-1}\alpha \neq 0\),
所以\(x_1 = 0\).
以此类推,可证\[
	x_2 = x_3 = \dotsb = x_{m-1} = 0,
\]
因此\(\alpha,\vb{A}\alpha,\vb{A}^2\alpha,\dotsc,\vb{A}^{m-1}\alpha\)线性无关.
\end{proof}
\end{example}

\begin{example}
%@see: 《高等代数(第三版 下册)》(丘维声) P113 习题9.1 10.
%@see: 《高等代数(大学高等代数课程创新教材 第二版 下册)》(丘维声) P239 例9
设\(\vb{A},\vb{B}\)是\(V\)上的线性变换,
且\(\vb{A}\vb{B}-\vb{B}\vb{A}=\vb{I}\).
证明:存在正整数\(k\),使得\begin{equation*}
	\vb{A}^k\vb{B}-\vb{B}\vb{A}^k = k\vb{A}^{k-1}.
\end{equation*}
%TODO proof
\end{example}

\begin{example}
%@see: 《高等代数(第三版 下册)》(丘维声) P113 习题9.1 11.
%@see: 《高等代数(大学高等代数课程创新教材 第二版 下册)》(丘维声) P239 例10
设\(V\)是域\(F\)上的一个线性空间,\(\FieldChar F \neq 2\),
\(\vb{A},\vb{B}\)是\(V\)上的幂等变换.
证明:\begin{itemize}
	\item \(\vb{A}+\vb{B}\)是幂等变换,当且仅当\(\vb{A}\vb{B}=\vb{B}\vb{A}=\vb0\);
	\item 如果\(\vb{A}\vb{B}=\vb{B}\vb{A}\),则\(\vb{A}+\vb{B}-\vb{A}\vb{B}\)也是幂等变换.
\end{itemize}
%TODO proof
\end{example}

\input{线性代数/线性映射/线性映射的核与像}
\section{线性映射的矩阵表示}
在本节,我们学习如何利用矩阵研究线性映射.

\subsection{用矩阵表示一个有限维线性空间上的线性变换}
设\(V\)是域\(F\)上的\(n\)维线性空间,
\(\vb{A}\)是\(V\)上的一个线性变换.
我们知道,\(\vb{A}\)被它在\(V\)上的一个基的作用决定.
于是取\(V\)的一个基\(\AutoTuple{\a}{n}\).
由于\(\vb{A}\a_i\in V\),
因此\(\vb{A}\a_i\)可以被\(V\)的这个基唯一地线性表出:\[
	\left\{ \begin{array}{l}
		\vb{A}\a_1=a_{11}\a_1+a_{21}\a_2+\dotsb+a_{n1}\a_n, \\
		\vb{A}\a_1=a_{11}\a_1+a_{22}\a_2+\dotsb+a_{n1}\a_n, \\
		\hdotsfor1, \\
		\vb{A}\a_n=a_{1n}\a_1+a_{2n}\a_2+\dotsb+a_{nn}\a_n.
	\end{array} \right.
\]
我们可以在形式上把上式写成\[
	(\vb{A}\a_1,\vb{A}\a_2,\dotsc,\vb{A}\a_n)
	=(\a_1,\a_2,\dotsc,\a_n)
	\begin{bmatrix}
		a_{11} & a_{12} & \dots & a_{1n} \\
		a_{21} & a_{22} & \dots & a_{2n} \\
		\vdots & \vdots && \vdots \\
		a_{n1} & a_{n2} & \dots & a_{nn}
	\end{bmatrix}.
\]
我们把上式右端的\(n\)阶矩阵\((a_{ij})_n\)记作\(A\),
把它称为“线性变换\(\vb{A}\)在基\(\AutoTuple{\a}{n}\)下的矩阵”.
\(A\)的第\(j\ (j=1,2,\dotsc,n)\)列是
\(\vb{A}\a_j\)在基\(\AutoTuple{\a}{n}\)下的坐标.
因此\(A\)由线性变换\(\vb{A}\)唯一决定.
如果我们再把\((\vb{A}\a_1,\vb{A}\a_2,\dotsc,\vb{A}\a_n)\)
简记为\(\vb{A}(\a_1,\a_2,\dotsc,\a_n)\),
那么上式可以化为\[
	\vb{A}(\a_1,\a_2,\dotsc,\a_n)
	=(\a_1,\a_2,\dotsc,\a_n)A.
\]
这就是一个\(n\)阶矩阵\(A\)
是\(V\)上线性变换\(\vb{A}\)
在基\(\AutoTuple{\a}{n}\)下的矩阵的充分必要条件.

\begin{example}
%@see: 《高等代数(第三版 下册)》(丘维声) P117 例1
在\(\mathbb{R}^\mathbb{R}\)中,
设\(V=\opair{1,\sin x,\cos x}\),
证明:
导数\(\vb{D}\)是\(V\)上的线性变换,
写出\(\vb{D}\)在基\(1,\sin x,\cos x\)下的矩阵.
\begin{proof}
因为\[
	\vb{D}(k_1\cdot1+k_2\cdot\sin x+k_3\cos x)
	=-k_3\sin x+k_2\cos x
	\in V,
\]
所以\(\vb{D}\)是\(V\)上的线性变换.
因为\[
	\left\{ \begin{array}{l}
		\vb{D}1
		=0
		=0\cdot1+0\cdot\sin x+0\cdot\cos x, \\
		\vb{D}\sin x
		=\cos x
		=0\cdot1+0\cdot\sin x+1\cdot\cos x, \\
		\vb{D}\cos x
		=-\sin x
		=0\cdot1+(-1)\cdot\sin x+0\cdot\cos x,
	\end{array} \right.
\]
所以\(\vb{D}\)在基\(1,\sin x,\cos x\)下的矩阵是\[
	D=\begin{bmatrix}
		0 & 0 & 0 \\
		0 & 0 & -1 \\
		0 & 1 & 0
	\end{bmatrix}.
	\qedhere
\]
\end{proof}
\end{example}

\subsection{用矩阵表示两个有限维线性空间之间的线性映射}
上例说明,\(n\)维线性空间\(V\)上的线性变换可以用矩阵来表示.
下面我们来讨论两个有限维线性空间之间的线性映射能不能用矩阵来表示.

设\(V\)和\(V'\)分别是域\(F\)上\(n\)维、\(s\)维线性空间,
\(\vb{A}\)是\(V\)到\(V'\)的一个线性映射.
在\(V\)中取一个基\(\AutoTuple{\a}{n}\),
在\(V'\)中取一个基\(\AutoTuple{\b}{s}\),
由于\(\vb{A}\a_i\in V'\),
因此\(\vb{A}\a_i\)可以
由\(V'\)的基\(\AutoTuple{\b}{s}\)唯一地线性表出:\[
	\left\{ \begin{array}{l}
		\vb{A}\a_1=a_{11}\b_1+a_{21}\b_2+\dotsb+a_{s1}\b_s, \\
		\vb{A}\a_1=a_{11}\b_1+a_{22}\b_2+\dotsb+a_{s1}\b_s, \\
		\hdotsfor1, \\
		\vb{A}\a_n=a_{1n}\b_1+a_{2n}\b_2+\dotsb+a_{sn}\b_s.
	\end{array} \right.
\]
我们可以在形式上把上式写成\[
	(\vb{A}\a_1,\vb{A}\a_2,\dotsc,\vb{A}\a_n)
	=(\b_1,\b_2,\dotsc,\b_s)
	\begin{bmatrix}
		a_{11} & a_{12} & \dots & a_{1n} \\
		a_{21} & a_{22} & \dots & a_{2n} \\
		\vdots & \vdots && \vdots \\
		a_{s1} & a_{s2} & \dots & a_{sn}
	\end{bmatrix}.
\]
我们把上式右端的\(s\times n\)阶矩阵\((a_{ij})_{s\times n}\)记作\(A\),
把它称为“线性映射\(\vb{A}\)
在\(V\)的基\(\AutoTuple{\a}{n}\)
和\(V'\)的基\(\AutoTuple{\b}{s}\)
下的矩阵”.
\(A\)的第\(j\ (j=1,2,\dotsc,n)\)列是
\(\vb{A}\a_j\)在基\(\AutoTuple{\b}{s}\)下的坐标.
因此\(A\)由线性映射\(\vb{A}\)唯一决定.
那么上式可以化为\[
	\vb{A}(\a_1,\a_2,\dotsc,\a_n)
	=(\b_1,\b_2,\dotsc,\b_s)A.
\]
这就是一个\(s\times n\)矩阵\(A\)
是\(V\)到\(V'\)的线性映射\(\vb{A}\)
在\(V\)的基\(\AutoTuple{\a}{n}\)
和\(V'\)的基\(\AutoTuple{\b}{s}\)下的矩阵的充分必要条件.

\subsection{线性映射空间与矩阵}
从上面看到,
域\(F\)上\(n\)维线性空间\(V\)到\(s\)维线性空间\(V'\)的
每一个线性映射\(\vb{A}\)可以用一个\(s\times n\)矩阵\(A\)表示.
我们已经知道,\(V\)到\(V'\)的所有线性映射组成的集合\(\Hom(V,V')\)
是域\(F\)上的一个线性空间.
我们又知道,\(F\)上所有\(s\times n\)矩阵组成的集合\(M_{s\times n}(F)\)
也是域\(F\)上的一个线性空间.
容易证明,\(\Hom(V,V')\)与\(M_{s\times n}(F)\)同构.

\begin{theorem}
%@see: 《高等代数(第三版 下册)》(丘维声) P119 定理1
设\(V\)和\(V'\)分别是域\(F\)上\(n\)维、\(s\)维线性空间,
则\begin{gather}
	\Hom(V,V') \Isomorphism M_{s\times n}(F), \\  % 同构
	\dim\Hom(V,V')
	=\dim M_{s\times n}(F)
	=sn.
\end{gather}
\end{theorem}

\begin{corollary}
%@see: 《高等代数(第三版 下册)》(丘维声) P119 推论2
设\(V\)是域\(F\)上的\(n\)维线性空间,
则\begin{gather}
%@see: 《高等代数(第三版 下册)》(丘维声) P119 (12)
	\Hom(V,V) \Isomorphism M_n(F), \\  % 同构
%@see: 《高等代数(第三版 下册)》(丘维声) P119 (13)
	\dim\Hom(V,V) = \left(\dim V\right)^2.
\end{gather}
\end{corollary}

在\(\Hom(V,V)\)与\(M_n(F)\)中,都有乘法运算.
我们可以进一步证明:
把线性变换\(\A\)对应到它在\(V\)的基\(\AutoTuple{\alpha}{n}\)下的矩阵\(A\)的映射\(\sigma\)还保持乘法运算.

设线性变换\(\B\)在\(V\)的基\(\AutoTuple{\alpha}{n}\)下的矩阵是\(B\).
由于\begin{align*}
%@see: 《高等代数(第三版 下册)》(丘维声) P120 (14)
	&\hspace{-20pt}
	(\A\B)(\AutoTuple{\alpha}{n}) \\
	&=\A(\B\alpha_1,\dotsc,\B\alpha_n) \\
	&=\A[(\AutoTuple{\alpha}{n})B] \\
	&=\A(b_{11}\alpha_1+\dotsb+b_{n1}\alpha_n,\dotsc,b_{1n}\alpha_1+\dotsb+b_{nn}\alpha_n) \\
	&=(b_{11}\A\alpha_1+\dotsb+b_{n1}\A\alpha_n,\dotsc,b_{1n}\A\alpha_1+\dotsb+b_{nn}\A\alpha_n) \\
	&=(\A\alpha_1,\dotsc,\A\alpha_n)B \\
	&=[\A(\AutoTuple{\alpha}{n})]B \\
	&=[(\AutoTuple{\alpha}{n})A]B \\
	&=((\AutoTuple{\alpha}{n}))(AB),
\end{align*}
所以\(\A\B\)在基\(\AutoTuple{\alpha}{n}\)下的矩阵是\(AB\).
那么\[
%@see: 《高等代数(第三版 下册)》(丘维声) P120 (15)
	\sigma(\A\B) = AB = \sigma(A) \sigma(B).
\]
这表明\(\sigma\)保持乘法运算.

从上述推导过程还可看到:\[
%@see: 《高等代数(第三版 下册)》(丘维声) P120 (16)
	\A[(\AutoTuple{\alpha}{n})B]
	= [\A(\AutoTuple{\alpha}{n})]B.
\]

显然,\(V\)上的恒等变换\(\vb{I}\)在基\(\AutoTuple{\alpha}{n}\)下的矩阵是单位矩阵\(I\),
因此\(\sigma(\vb{I})=I\).

设\(V\)上线性变换\(\A\)在基\(\AutoTuple{\alpha}{n}\)下的矩阵是\(A\).
由于\begin{align*}
	&\text{线性变换$\A$可逆} \\
	&\iff \text{存在$V$上的线性变换$B$使得$\A\B=\B\A=\vb{I}$} \\
	&\iff \text{存在$V$上的线性变换$B$使得$\sigma(\A) \sigma(\B) = \sigma(\B) \sigma(\A) = \sigma(\vb{I})$} \\
	&\iff \text{存在域$F$上$n$阶矩阵$B$使得$AB=BA=I$} \\
	&\iff \text{矩阵$A$可逆},
\end{align*}
所以,\(V\)上线性变换\(\A\)可逆,
当且仅当它在\(V\)的一个基的矩阵\(A\)可逆.
从上述推导过程还可看到,
对于线性变换\(\A,\B\),
假设它们在\(V\)的一个基下的矩阵分别是\(A,B\),
则\(\B\)是可逆线性变换\(\A\)的逆变换,
当且仅当\(B\)是可逆矩阵\(A\)的逆矩阵.

设\(\A\)是域\(F\)上\(n\)维线性空间\(V\)上的一个线性变换,
且\(\A\)在\(V\)的一个基\(\AutoTuple{\alpha}{n}\)下的矩阵是\(A\).
\(V\)中任一向量\(\alpha\)在基\(\AutoTuple{\alpha}{n}\)下的坐标记作\(X\).
由于\(\alpha=(\AutoTuple{\alpha}{n})X\),
所以\begin{align*}
	\A\alpha
	&= \A[(\AutoTuple{\alpha}{n})X]
	= [\A(\AutoTuple{\alpha}{n})]X \\
	&= [(\AutoTuple{\alpha}{n})A]X
	= (\AutoTuple{\alpha}{n})(AX).
\end{align*}
这表明\(\A\alpha\)在基\(\AutoTuple{\alpha}{n}\)下的坐标是\(AX\).

由于\(V\)中两个向量相等,
当且仅当它们在\(V\)的一个基下的坐标相等,
因此,如果向量\(\gamma\)在基\(\AutoTuple{\alpha}{n}\)下的坐标是\(Y\),
则\[
%@see: 《高等代数(第三版 下册)》(丘维声) P120 (17)
	\A\alpha=\gamma
	\iff
	AX=Y.
\]

\subsection{线性变换在不同基下的矩阵的关系}
域\(F\)上\(n\)维线性空间\(V\)上的一个线性变换\(\A\)在\(V\)的不同基下的矩阵有什么关系?

\begin{theorem}\label{theorem:线性映射的矩阵表示.线性变换在不同基下的矩阵相似}
%@see: 《高等代数(第三版 下册)》(丘维声) P120 定理3
设\(V\)是域\(F\)上\(n\)维线性空间,
\(V\)上的一个线性变换\(\A\)在\(V\)的两个基
\(\AutoTuple{\a}{n}\)与\(\AutoTuple{\b}{n}\)下的矩阵分别为\(A,B\).
从基\(\AutoTuple{\a}{n}\)到基\(\AutoTuple{\b}{n}\)的过渡矩阵是\(S\),
%@see: 《高等代数(第三版 下册)》(丘维声) P120 (18)
则\(B = S^{-1} A S\).
\end{theorem}
可以看出,同一个线性变换\(\A\)在\(V\)的不同基下的矩阵是相似的.
反之,我们有如下命题:
\begin{proposition}
%@see: 《高等代数(第三版 下册)》(丘维声) P121 命题4
如果域\(F\)上\(n\)阶矩阵\(A\)与\(B\)相似,
那么\(A\)与\(B\)可以看成是域\(F\)上\(n\)维线性空间\(V\)上的
一个线性变换\(\A\)在\(V\)的不同基下的矩阵.
\end{proposition}

\section{线性函数与对偶空间}
%TODO 对偶空间与傅里叶级数、傅里叶变换有密切的联系,详见以下视频:
%@see: https://www.bilibili.com/video/BV1xxivYzEh8/
%@see: https://mathvideos.org/2021/richard-borcherds-rings-and-modules/

本节着重讨论一种特殊的线性映射 --- \hyperref[definition:线性映射.线性函数]{线性函数}.
\subsection{线性函数}
除了\cref{example:线性映射.给定区间上的定积分是线性函数} 中举出的
给定区间上的定积分是线性函数以外,
下面我们额外举几个例子.

\begin{example}
%@see: 《高等代数(第三版 下册)》(丘维声) P160
矩阵的迹,
是\(M_n(F)\)上的一个线性函数,
它把域\(F\)上每一个\(n\)阶矩阵,
对应到\(F\)中的一个元素,
并且保持加法与数量乘法.
\end{example}

\begin{example}
%@see: 《高等代数(第三版 下册)》(丘维声) P161 例1
设\(F\)是一个域,\(\AutoTuple{a}{n} \in F\),
令\begin{equation*}
	f\colon F^n \to F,
	(\AutoTuple{x}{n}) \mapsto a_1 x_1 + \dotsb + a_n x_n,
\end{equation*}
容易验证\(f\)是\(F^n\)上的一个线性函数.
\end{example}

\begin{example}
%@see: 《高等代数(第三版 下册)》(丘维声) P161 例2
\def\Z{\mathbb{Z}_2}
设\(\Z\)是模\(2\)剩余类域.
令\begin{equation*}
	f\colon \Z^2 \to \Z,
	(x_1,x_2) \mapsto x_1^2+x_2^2.
\end{equation*}
试判断\(f\)是不是\(\Z^2\)上的一个线性函数.
\begin{solution}
任取\((x_1,x_2),(y_1,y_2)\in\Z^2\),
有\begin{align*}
	f((x_1,x_2)+(y_1,y_2))
	&= f(x_1+y_1,x_2+y_2) \\  % 剩余类域上线性空间的加法
	&= (x_1+y_1)^2+(x_2+y_2)^2 \\  % 题设
	&= x_1^2+y_1^2+x_2^2+y_2^2
		\tag{\cref{example:域.域上的特征恒等式}} \\
	&= f(x_1,x_2)+f(y_1,y_2), \\  % 题设
	f(1\cdot(x_1,x_2))
	&= f(x_1,x_2)  % 剩余类域上线性空间的纯量乘法
	= 1\cdot f(x_1,x_2), \\  % 剩余类域上的乘法
	f(0\cdot(x_1,x_2))
	&= f(0,0)  % 剩余类域上线性空间的纯量乘法
	= 0 = 0\cdot f(x_1,x_2).
\end{align*}
因此\(f\)是\(\Z^2\)上的一个线性函数.
\end{solution}
\end{example}

\subsection{对偶空间}
\begin{definition}
%@see: 《高等代数(第三版 下册)》(丘维声) P162
%@see: 《Linear Algebra Done Right (Fourth Eidition)》(Sheldon Axler) P105 3.110
设\(V\)是域\(F\)上的一个线性空间,
把\(\Hom(V,F)\)称为“\(V\)上的\DefineConcept{线性函数空间}”
或“\(V\)的\DefineConcept{对偶空间}(dual space)”,
简记为\(V^*\).
\end{definition}

\begin{proposition}\label{theorem:对偶空间.对偶空间的维数}
%@see: 《高等代数(第三版 下册)》(丘维声) P162
%@see: 《Linear Algebra Done Right (Fourth Eidition)》(Sheldon Axler) P105 3.111
设\(V\)是域\(F\)上的\(n\)维线性空间,
\(V^*\)是\(V\)的对偶空间,
%@see: 《高等代数(第三版 下册)》(丘维声) P162 (4)
则\(\dim V^* = n\),
%@see: 《高等代数(第三版 下册)》(丘维声) P162 (5)
且\(V^* \Isomorphism V\).
\begin{proof}
因为\(\dim F = 1\),
所以由\cref{theorem:线性映射.线性映射空间与矩阵空间同构1} 可知
\(\dim V^*
= (\dim V)(\dim F)
= n \cdot 1
= n\).
再由\cref{theorem:线性空间的同构.线性空间同构的充分必要条件} 可知
\(V^* \Isomorphism V\).
\end{proof}
\end{proposition}
\begin{remark}
我们可以在\(V\)与\(V^*\)之间构造一个同构:
在\(V\)中取一个基\(\AutoTuple{\alpha}{n}\),
假设它的对偶基是\(\AutoTuple{\phi}{n}\),
那么映射\begin{equation*}
%@see: 《高等代数(第三版 下册)》(丘维声) P164 (15)
	\sigma\colon V \to V^*,
	\alpha = \sum_{i=1}^n x_i \alpha_i \mapsto \sum_{i=1}^n x_i f_i
\end{equation*}
就是\(\sigma\)是从\(V\)到\(V^*\)的一个同构.
\end{remark}

\subsection{对偶基}
%\cref{theorem:线性映射.线性映射的存在性}
\begin{definition}
%@see: 《高等代数(第三版 下册)》(丘维声) P162
%@see: 《Linear Algebra Done Right (Fourth Eidition)》(Sheldon Axler) P106 3.112
设\(V\)是域\(F\)上\(n\)维线性空间,
\(\AutoTuple{\alpha}{n}\)是\(V\)的一个基.
定义映射:\begin{gather}
%@see: 《高等代数(第三版 下册)》(丘维声) P162 (6)
	\phi_i(\alpha_j)
	\defeq \left\{ \begin{array}{cl}
		1, & j = i, \\
		0, & j \neq i,
	\end{array} \right.
	\quad i=1,2,\dotsc,n,
		\label{equation:对偶空间.对偶基1} \\
%@see: 《高等代数(第三版 下册)》(丘维声) P162 (7)
	\phi_i(\alpha_j+\alpha_k)
	\defeq \phi_i(\alpha_j) + \phi_i(\alpha_k),
	\quad i=1,2,\dotsc,n,
		\label{equation:对偶空间.对偶基2} \\
	\phi_i(\lambda \alpha_j)
	\defeq \lambda \phi_i(\alpha_j),
	\quad i=1,2,\dotsc,n.
		\label{equation:对偶空间.对偶基3}
\end{gather}
把\(\AutoTuple{\phi}{n}\)
称为“\(\AutoTuple{\alpha}{n}\)的\DefineConcept{对偶基}%
(the \emph{dual basis} of \(\AutoTuple{\alpha}{n}\))”.
\end{definition}

\begin{proposition}
%@see: 《高等代数(第三版 下册)》(丘维声) P162
%@see: 《Linear Algebra Done Right (Fourth Eidition)》(Sheldon Axler) P106 3.112
%@see: 《Linear Algebra Done Right (Fourth Eidition)》(Sheldon Axler) P106 3.114
%@see: 《Linear Algebra Done Right (Fourth Eidition)》(Sheldon Axler) P107 3.116
设\(V\)是域\(F\)上\(n\)维线性空间,
\(V^*\)是\(V\)的对偶空间,
\(\AutoTuple{\alpha}{n}\)是\(V\)的一个基,
\(\AutoTuple{\phi}{n}\)是\(\AutoTuple{\alpha}{n}\)的对偶基,
则\begin{itemize}
	\item \(\AutoTuple{\phi}{n}\)都是\(V\)上的线性函数,
	从而对于\(\forall \AutoTuple{x}{n} \in F\),
	有\begin{equation*}
	%@see: 《高等代数(第三版 下册)》(丘维声) P162 (7)
		\phi_i\left( \sum_{j=1}^n x_j \alpha_j \right)
		= \sum_{j=1}^n x_j \phi_i(\alpha_j)
		= x_i,
		\quad i=1,2,\dotsc,n;
	\end{equation*}

	\item \(\AutoTuple{\phi}{n}\)是\(V^*\)的一个基;

	\item \(V\)中任意一个向量\(\alpha\)
	在基\(\AutoTuple{\alpha}{n}\)下的坐标的第\(i\)个分量,
	等于\(\AutoTuple{\alpha}{n}\)的对偶基的第\(i\)个线性函数\(\phi_i\)
	在\(\alpha\)的值\(\phi_i(\alpha)\),
	即\begin{equation*}
	%@see: 《高等代数(第三版 下册)》(丘维声) P163 (9)
		(\forall \alpha \in V)
		[\alpha = \phi_1(\alpha) \alpha_1 + \dotsb + \phi_n(\alpha) \alpha_n];
	\end{equation*}

	\item \(V^*\)中任意一个线性函数\(\phi\)
	在基\(\AutoTuple{\phi}{n}\)下的坐标的第\(i\)个分量,
	等于\(\phi\)在\(\alpha_i\)的值\(\phi(\alpha_i)\),
	即\begin{equation*}
	%@see: 《高等代数(第三版 下册)》(丘维声) P163 (11)
		(\forall \phi \in V^*)
		[\phi = \phi(\alpha_1) \phi_1 + \dotsb + \phi(\alpha_n) \phi_n].
	\end{equation*}
\end{itemize}
\begin{proof}
由\cref{equation:对偶空间.对偶基2} 可知\(\phi_i\)适合可加性,
由\cref{equation:对偶空间.对偶基3} 可知\(\phi_i\)适合齐次性,
因此\(\phi_i\)是\(V\)上的线性函数.
再由\cref{theorem:线性映射.线性映射的性质} 可知,
对于\(\forall \AutoTuple{x}{n} \in F\),
有\begin{equation*}
%@see: 《高等代数(第三版 下册)》(丘维声) P162 (7)
	\phi_i\left( \sum_{j=1}^n x_j \alpha_j \right)
	= \sum_{j=1}^n x_j \phi_i(\alpha_j)
	= x_i,
	\quad i=1,2,\dotsc,n;
\end{equation*}

令\begin{equation*}
	x_1 \phi_1 + \dotsb + x_n \phi_n = 0,
\end{equation*}
则\begin{eqnarray}
	(x_1 \phi_1 + \dotsb + x_n \phi_n) \alpha_j = 0,
	\quad j=1,2,\dotsc,n.
\end{eqnarray}
根据线性映射的加法、纯量乘法的定义得\begin{equation*}
	x_1 \phi_1(\alpha_j) + \dotsb + x_j \phi_j(\alpha_j) + \dotsb + x_n \phi_n(\alpha_j) = 0,
	\quad j=1,2,\dotsc,n.
\end{equation*}
于是\(x_j = 0\ (j=1,2,\dotsc,n)\),
这就说明\(\AutoTuple{\phi}{n}\)线性无关.
再由\cref{theorem:对偶空间.对偶空间的维数,theorem:线性空间.线性相关性3} 可知
\(\AutoTuple{\phi}{n}\)是\(V^*\)的一个基.

对于任意\(\alpha \in V\),
% 由\cref{theorem:线性空间.任一向量可由给定基唯一线性表出} 可知,
存在\(\AutoTuple{k}{n} \in F\),
使得\begin{equation*}
	\alpha = \sum_{j=1}^n k_j \alpha_j.
\end{equation*}
那么有\begin{equation*}
	\phi_i(\alpha)
	= \phi_i\left( \sum_{j=1}^n k_j \alpha_j \right)
	= k_i,
\end{equation*}
于是\begin{equation*}
	\alpha = \sum_{j=1}^n \phi_j(\alpha) \alpha_j.
\end{equation*}

对于任意\(\phi \in V^*\),
存在\(\AutoTuple{k}{n} \in F\),
使得\begin{equation*}
	\phi = \sum_{j=1}^n k_j \phi_j.
\end{equation*}
那么有\begin{equation*}
	\phi(\alpha_i)
	= \left( \sum_{j=1}^n k_j \phi_j \right)(\alpha_i)
	= \sum_{j=1}^n k_j \phi_j(\alpha_i)
	= k_i,
\end{equation*}
于是\begin{equation*}
	\phi = \sum_{j=1}^n \phi(\alpha_j) \phi_j.
	\qedhere
\end{equation*}
\end{proof}
\end{proposition}

\begin{theorem}
%@see: 《高等代数(第三版 下册)》(丘维声) P163 定理1
设\(V\)是域\(F\)上一个\(n\)维线性空间,
在\(V\)中取两个基\(\AutoTuple{\alpha}{n}\)与\(\AutoTuple{\beta}{n}\),
它们的对偶基分别为\(\AutoTuple{\phi}{n}\)与\(\AutoTuple{\psi}{n}\).
如果基\(\AutoTuple{\alpha}{n}\)到基\(\AutoTuple{\beta}{n}\)的过渡矩阵是\(A\),
%@see: 《高等代数(第三版 下册)》(丘维声) P163 (12)
则基\(\AutoTuple{\phi}{n}\)到基\(\AutoTuple{\psi}{n}\)的过渡矩阵为\((A^{-1})^T\).
\begin{proof}
由\((\AutoTuple{\beta}{n}) = (\AutoTuple{\alpha}{n}) A\)得
%@see: 《高等代数(第三版 下册)》(丘维声) P163 (13)
\((\AutoTuple{\alpha}{n}) = (\AutoTuple{\beta}{n}) A^{-1}\),
这就说明\(\alpha_j\)在基\(\AutoTuple{\beta}{n}\)下的坐标是\(A^{-1}\)的第\(j\)列,
从而坐标的第\(i\)个分量为\(\MatrixEntry{A^{-1}}{i,j}\).
由于\(\AutoTuple{\beta}{n}\)的对偶基是\(\AutoTuple{\psi}{n}\),
可知\(\alpha_j\)在基\(\AutoTuple{\beta}{n}\)下的坐标的第\(i\)个分量等于\(\psi_i(\alpha_j)\).
因此\(\MatrixEntry{A^{-1}}{i,j} = \psi_i(\alpha_j)\).

由于\(\AutoTuple{\alpha}{n}\)的对偶基是\(\AutoTuple{\phi}{n}\),
可知\(\psi_i(\alpha_j)\)等于\(\psi_i\)在\(\AutoTuple{\phi}{n}\)下的坐标的第\(j\)个分量.
%@see: 《高等代数(第三版 下册)》(丘维声) P163 (14)
由已知条件可知\((\AutoTuple{\psi}{n}) = (\AutoTuple{\phi}{n}) B\),
这就说明\(\psi_i\)在基\(\AutoTuple{\phi}{n}\)下的坐标的第\(j\)个分量等于\(\MatrixEntry{B}{j,i}\).
因此\(\psi_i(\alpha_j) = \MatrixEntry{B}{j,i}\).

综上所述,有\(\MatrixEntry{A^{-1}}{i,j}
= \psi_i(\alpha_j)
= \MatrixEntry{B}{j,i}
= \MatrixEntry{B^T}{i,j}\),
其中\(i,j=1,2,\dotsc,n\).
因此\(A^{-1} = B^T\),
换言之\(B = (A^{-1})^T\).
\end{proof}
\end{theorem}

\subsection{双重对偶空间}
给定域\(F\)上的一个\(n\)维线性空间\(V\),
我们知道\(V\)的对偶空间\(V^*\)也是域\(F\)上的一个\(n\)维线性空间,
那么我们可以考虑\(V^*\)的对偶空间\((V^*)^*\).

\begin{definition}
%@see: 《高等代数(第三版 下册)》(丘维声) P164
设\(V\)是域\(F\)上的一个线性空间,
\(V^*\)是\(V\)的对偶空间.
把\(V^*\)的对偶空间\((V^*)^*\)
称为“\(V\)的\DefineConcept{双重对偶空间}”,
简记为\(V^{**}\).
\end{definition}

\begin{corollary}
%@see: 《高等代数(第三版 下册)》(丘维声) P164
设\(V\)是域\(F\)上的一个\(n\)维线性空间,
\(V^{**}\)是\(V\)的双重对偶空间,
则\(V \Isomorphism V^{**}\).
\begin{proof}
由\cref{theorem:对偶空间.对偶空间的维数}
可知\(V \Isomorphism V^*,
V^* \Isomorphism V^{**}\),
利用传递性可得\(V \Isomorphism V^{**}\).
\end{proof}
\end{corollary}

%TODO
% \begin{proposition}
% %@see: 《高等代数(第三版 下册)》(丘维声) P164
% 设\(V\)是域\(F\)上的一个线性空间,
% \(V^*\)是\(V\)的对偶空间,
% \(V^{**}\)是\(V\)的双重对偶空间,
% \(\sigma_1\)是从\(V\)到\(V^*\)的一个同构,
% \(\sigma_2\)是从\(V^*\)到\(V^{**}\)的一个同构,
% 记\(\tau \defeq \sigma_2 \circ \sigma_1\),
% 则\begin{equation*}
% 	(\forall \alpha \in V)
% 	[\tau(\alpha) = ]
% \end{equation*}
% \end{proposition}

\subsection{对偶映射}
\begin{definition}
%@see: 《Linear Algebra Done Right (Fourth Eidition)》(Sheldon Axler) P107 3.118
设\(V,W\)都是域\(F\)上线性空间,
\(V^*,W^*\)分别是\(V,W\)的对偶空间,
\(T\)是从\(V\)到\(W\)的一个线性映射,
\(T^*\)是从\(W^*\)到\(V^*\)的映射.
如果对于\(\forall \phi \in W^*\)
都有\begin{equation*}
	T^*(\phi) = \phi \circ T,
\end{equation*}
则称“\(T^*\)是\(T\)的\DefineConcept{对偶映射}(the \emph{dual map} of \(T\))”.
\end{definition}

\begin{proposition}
%@see: 《Linear Algebra Done Right (Fourth Eidition)》(Sheldon Axler) P107
设\(V,W\)都是域\(F\)上线性空间,
\(V^*,W^*\)分别是\(V,W\)的对偶空间,
\(T\)是从\(V\)到\(W\)的一个线性映射,
\(T^*\)是\(T\)的对偶映射,
则\(T^*\)是从\(W^*\)到\(V^*\)的一个线性映射.
\begin{proof}
对于\(\forall \phi,\psi \in W^*\)和\(\forall \lambda \in F\),
有\begin{gather*}
	T^*(\phi+\psi)
	= (\phi+\psi) \circ T
	= \phi \circ T + \psi \circ T
	= T^*(\phi) + T^*(\psi), \\
	T^*(\lambda\phi)
	= (\lambda\phi) \circ T
	= \lambda (\phi \circ T)
	= \lambda T^*(\phi).
	\qedhere
\end{gather*}
\end{proof}
\end{proposition}

\begin{property}
%@see: 《Linear Algebra Done Right (Fourth Eidition)》(Sheldon Axler) P108 3.120
设\(V,W\)都是域\(F\)上线性空间,
\(T\)是从\(V\)到\(W\)的一个线性映射,
则\begin{itemize}
	\item 对于从\(V\)到\(W\)的任意一个线性映射\(S\),有\begin{equation*}
		(S+T)^* = S^* + T^*;
	\end{equation*}
	\item 对于任意\(\lambda \in F\),有\begin{equation*}
		(\lambda T)^* = \lambda T^*;
	\end{equation*}
	\item 对于从\(W\)到\(U\)的任意一个线性映射\(S\),有\begin{equation*}
		(S T)^* = T^* S^*.
	\end{equation*}
\end{itemize}
%TODO proof
\end{property}


\chapter{线性变换的特征值与特征向量}
\begingroup
\NewDocumentCommand\LambdaExp{mo}{(\lambda-\lambda_{#1})\IfValueTF{#2}{^{#2}}{}}%
\NewDocumentCommand\EigenPoly{mmo}{(#1-\lambda_{#2}\vb{I})\IfValueTF{#3}{^{#3}}{}}%
\NewDocumentCommand\RestrictedEigenPoly{mmo}{\EigenPoly{#1 \SetRestrict W_{#2}}{#2}[#3]}%
\NewDocumentCommand\LambdaExpL{m}{\LambdaExp{#1}[l_{#1}]}%
\NewDocumentCommand\EpAj{o}{\EigenPoly{\vb{A}}{j}[#1]}%
\NewDocumentCommand\RepAj{o}{\RestrictedEigenPoly{\vb{A}}{j}[#1]}%
\section{线性变换的特征值与特征向量,线性变换可对角化的条件}
\cref{theorem:线性映射的矩阵表示.线性变换在不同基下的矩阵相似} 表明,
域\(F\)上\(n\)维线性空间\(V\)上的线性变换\(\A\)在\(V\)的不同基下的矩阵是相似的.
由于相似的矩阵有相同的行列式、秩、迹、特征多项式、特征值,
因此我们可以把线性变换\(\A\)在\(V\)的某一个基下的矩阵\(A\)的行列式、秩、迹、特征多项式、特征值,
分别叫做线性变换\(\A\)的行列式、秩、迹、特征多项式、特征值.

为了更好地理解线性变换的特征值的几何意义,以及对无限维线性空间上的线性变换也考虑它的特征值,
我们给出如下的定义:
\begin{definition}\label{definition:线性变换的特征值和特征向量.线性变换的特征值和特征向量}
%@see: 《高等代数(第三版 下册)》(丘维声) P127 定义1
设\(\A\)是域\(F\)上维线性空间\(V\)上的一个线性变换.
如果\(V\)中存在一个非零向量\(\xi\),
使得\[
%@see: 《高等代数(第三版 下册)》(丘维声) P127 (1)
	\A\xi=\lambda_0\xi,
	\quad \lambda_0\in F,
\]
则称“\(\lambda_0\)是\(\A\)的一个\DefineConcept{特征值}”
“\(\xi\)是\(\A\)的属于特征值\(\lambda_0\)的一个\DefineConcept{特征向量}”.
\end{definition}
从\cref{definition:线性变换的特征值和特征向量.线性变换的特征值和特征向量} 看出,
线性变换\(\A\)的特征向量\(\xi\)有这样的“几何意义”:
\(\A\)对\(\xi\)的作用是把\(\xi\)“拉伸”或“压缩”\(\lambda_0\)倍.
这个倍数\(\lambda_0\)就是\(\A\)的一个特征值.

现在设\(V\)是域\(F\)上\(n\)维线性空间,
\(V\)中取定一个基\(\AutoTuple{\alpha}{n}\).
\(V\)上的一个线性变换\(\A\)在基\(\AutoTuple{\alpha}{n}\)下的矩阵是\(A\),
向量\(\xi\)在基\(\AutoTuple{\alpha}{n}\)下的坐标是\(X\),
\(\lambda_0\in F\).
于是\[
%@see: 《高等代数(第三版 下册)》(丘维声) P127 (2)
	\A\xi=\lambda_0\xi
	\iff
	AX=\lambda_0X.
\]
由此得出\begin{align*}
%@see: 《高等代数(第三版 下册)》(丘维声) P127 (3)
	&\text{$\lambda_0$是$\A$的一个特征值} \\
	&\iff \text{$\lambda_0$是$A$的一个特征值} \\
	&\iff \text{$\xi$是$\A$的属于特征值$\lambda_0$的一个特征向量} \\
%@see: 《高等代数(第三版 下册)》(丘维声) P127 (4)
	&\iff \text{$\xi$的坐标$X$是$A$的属于特征值$\lambda_0$的一个特征向量}.
\end{align*}
可以看出,对于有限维线性空间,
用线性变换的矩阵的特征值定义线性变换的特征值,与上述定义是一致的.
同时,我们还得到了求有限维线性空间上线性变换的\(\A\)的全部特征值和特征向量的方法:
只要取求\(\A\)在\(V\)的一个基下的矩阵\(A\)的全部特征值和特征向量.
但是要注意:
矩阵\(A\)的特征向量\(X\)是线性变换\(\A\)的特征向量\(\xi\)在基\(\AutoTuple{\alpha}{n}\)下的坐标.

\section{线性变换的不变子空间}
在上一节,我们讨论了可对角化的线性变换的标准型.
本节讨论不可以对角化的线性变换的标准型.

我们首先注意到,
\(\vb{A}\)是可对角化的线性变换,
当且仅当空间\(V\)可以分解成\(\vb{A}\)的特征子空间的直和.
受此启发,我们在研究不可对角化的线性变换\(\vb{B}\)的结构时,
也可以从这里入手,研究如何将空间\(V\)分解成与\(\vb{B}\)有关的某种特殊类型的子空间的直和.

\subsection{线性变换的不变子空间}
\begin{definition}
%@see: 《高等代数(第三版 下册)》(丘维声) P131 定义1
设\(\vb{A}\)是域\(F\)上线性空间\(V\)上的线性变换,\(W\)是\(V\)的子空间.
如果\(W\)中的向量在\(\vb{A}\)下的象仍在\(W\)中,
即\((\forall\alpha \in W)[\vb{A}\alpha \in W]\),
则称“\(W\)是\(\vb{A}\)的一个\DefineConcept{不变子空间}”,
简称 \DefineConcept{\(\vb{A}\) - 子空间}.
\end{definition}

显然,对于\(V\)上每一个线性变换\(\vb{A}\)来说,
整个空间\(V\)和零子空间\(0\),
都是\(\vb{A}\)的不变子空间,
因此将它们称为“\(\vb{A}\)的\DefineConcept{平凡不变子空间}”.
如果\(W\)是\(\vb{A}\)的不变子空间,
且\(W\)不是\(\vb{A}\)的平凡不变子空间,
则称“\(W\)是\(\vb{A}\)的一个\DefineConcept{非平凡不变子空间}”.

\begin{proposition}%\label{theorem:线性映射.线性变换的不变子空间1}
%@see: 《高等代数(第三版 下册)》(丘维声) P131 命题1
\(V\)上线性变换\(\vb{A}\)的核\(\Ker\vb{A}\)与象\(\Im\vb{A}\),
以及\(\vb{A}\)的所有特征子空间,
都是\(\vb{A}\)的不变子空间.
\begin{proof}
任取\(\alpha \in \Ker\vb{A}\),
因为\(\vb{A}\alpha = 0 \in \Ker\vb{A}\),
所以\(\Ker\vb{A}\)是\(\vb{A}\)的不变子空间.

任取\(\alpha \in \Im\vb{A}\),
因为\(\vb{A}\alpha \in \Im\vb{A}\),
所以\(\Im\vb{A}\)也是\(\vb{A}\)的不变子空间.

任取\(\alpha \in V_\lambda\),
因为\(\vb{A}\alpha = \lambda \alpha \in V_\lambda\),
所以\(V_\lambda\)还是\(\vb{A}\)的不变子空间.
\end{proof}
\end{proposition}

\begin{proposition}%\label{theorem:线性映射.线性变换的不变子空间2}
%@see: 《高等代数(第三版 下册)》(丘维声) P131 命题2
如果线性变换\(\vb{A},\vb{B}\)可交换,即\(\vb{A}\vb{B} = \vb{B}\vb{A}\),
则\(\Ker\vb{B},
\Im\vb{B}\)
以及\(\vb{B}\)的特征子空间
都是\(\vb{A}\)的不变子空间.
\begin{proof}
任取\(\alpha \in \Ker\vb{B}\),
则\(\vb{B}\alpha = 0\).
于是\[
	\vb{B}(\vb{A}\alpha)
	= (\vb{B}\vb{A})\alpha
	= (\vb{A}\vb{B})\alpha
	= \vb{A}(\vb{B}\alpha)
	= \vb{A}0
	= 0.
\]
因此\(\vb{A}\alpha \in \Ker\vb{B}\),
从而\(\Ker\vb{B}\)是\(\vb{A}\)的不变子空间.

同理可证\(\Im\vb{B}\)也是\(\vb{A}\)的不变子空间.

在\(\vb{B}\)的特征子空间\(V_\lambda\)中,任取一个向量\(\alpha\),
则\(\vb{B}\alpha = \lambda \alpha\),
从而\[
	\vb{B}(\vb{A}\alpha)
	= \vb{A}(\vb{B}\alpha)
	= \vb{A}(\lambda\alpha)
	= \lambda(\vb{A}\alpha),
\]
因此\(\vb{A}\alpha \in V_\lambda\),
从而\(V_\lambda\)是\(A\)的不变子空间.
\end{proof}
\end{proposition}

\begin{corollary}\label{theorem:线性映射.线性变换的不变子空间3}
%@see: 《高等代数(第三版 下册)》(丘维声) P132 推论3
设\(\vb{A}\)是域\(F\)上线性空间\(V\)上的线性变换,
则对于域\(F\)上任意一个一元多项式\(f(x) \in F[x]\),
都有\(\Ker f(\vb{A}),
\Im f(\vb{A})\)
以及\(f(\vb{A})\)的特征子空间
都是\(\vb{A}\)的不变子空间.
\end{corollary}

\begin{proposition}%\label{theorem:线性映射.线性变换的不变子空间4}
%@see: 《高等代数(第三版 下册)》(丘维声) P132
设\(U_1,U_2\)是域\(F\)上线性空间\(V\)上的线性变换\(\vb{A}\)的两个不变子空间,
则\(U_1 \cap U_2\)和\(U_1 + U_2\)都是\(\vb{A}\)的不变子空间.
\end{proposition}

\begin{proposition}
%@see: 《高等代数(第三版 下册)》(丘维声) P132 命题4
设\(\vb{A}\)是域\(F\)上线性空间\(V\)上的线性变换,
\(W = \Span\{\AutoTuple{\alpha}{m}\}\)是\(V\)的一个子空间,
则\(W\)是\(\vb{A}\)的不变子空间,
当且仅当\(\vb{A}\alpha_i \in W\ (i=1,2,\dotsc,m)\).
\begin{proof}
由于\(W\)中任意一个向量\(\alpha\)均可表示成\(\AutoTuple{\alpha}{m}\)的一个线性组合,
不妨设\(\alpha = k_1 \alpha_1 + \dotsb + k_m \alpha_m\),
其中\(\AutoTuple{k}{m} \in F\),
那么\begin{align*}
	&\text{$W$是$\vb{A}$的不变子空间} \\
	&\iff \alpha \in W \implies \vb{A}\alpha \in W \\
	&\iff (\forall \AutoTuple{k}{m} \in F)[\vb{A}(k_1 \alpha_1 + \dotsb + k_m \alpha_m) \in W] \\
	&\iff \vb{A}\alpha_i \in W\ (i=1,2,\dotsc,m).
	\qedhere
\end{align*}
\end{proof}
\end{proposition}

\begin{example}
%@see: 《高等代数(第三版 下册)》(丘维声) P131 习题9.5 3.
设\(V\)是复数域上的\(n\)维线性空间,
\(\vb{A},\vb{B}\)都是\(V\)上的线性变换,
且\(\vb{A}\vb{B}=\vb{B}\vb{A}\).
证明:\(\vb{A}\)与\(\vb{B}\)至少有一个公共的特征向量.
%TODO proof
% 取\(\vb{A}\)的一个特征值\(\lambda\),则\(V_\lambda\)是\(\vb{B}\)的一个不变子空间
% 取\(\vb{B} \setrestrict V_\lambda\)的一个特征值\(\mu\),
% 则存在\(\xi \in V_\lambda\),
% 使得\(\vb{B} \xi = (\vb{B} \setrestrict V_\lambda) \xi = \mu \xi\)
\end{example}

\subsection{线性变换的非平凡不变子空间的存在性}
对于域\(F\)上有限维线性空间\(V\)上的线性变换\(\vb{A}\),
它有没有非平凡不变子空间,与它的矩阵表示的形式,有密切关系.
\begin{theorem}
%@see: 《高等代数(第三版 下册)》(丘维声) P132 定理5
设\(\vb{A}\)是域\(F\)上\(n\)维线性空间\(V\)上的线性变换,
则\(\vb{A}\)有非平凡不变子空间,
当且仅当\(V\)中存在一个基,
使得\(\vb{A}\)在这个基下的矩阵是一个分块上三角矩阵.
\begin{proof}
必要性.
设\(W\)是\(\vb{A}\)的非平凡不变子空间,
在\(W\)中取一个基\(\AutoTuple{\alpha}{r}\),
把它扩充成\(V\)的一个基:\[
	\AutoTuple{\alpha}{r},
	\AutoTuple{\alpha}[r+1]{n}.
\]
那么\begin{align*}
%@see: 《高等代数(第三版 下册)》(丘维声) P132 (3)
	&\vb{A} (\AutoTuple{\alpha}{r},\AutoTuple{\alpha}[r+1]{n}) \\
	&= (\AutoTuple{\alpha}{r},\AutoTuple{\alpha}[r+1]{n})
	\begin{bmatrix}
		a_{11} & \dots & a_{1r} & a_{1,r+1} & \dots & a_{1n} \\
		\vdots & & \vdots & \vdots & & \vdots \\
		a_{r1} & \dots & a_{rr} & a_{r,r+1} & \dots & a_{rn} \\
		0 & \dots & 0 & a_{r+1,r+1} & \dots & a_{r+1,n} \\
		\vdots & & \vdots & \vdots & & \vdots \\
		0 & \dots & 0 & a_{n,r+1} & \dots & a_{nn}
	\end{bmatrix}.
\end{align*}
因此\(\vb{A}\)在基\(\AutoTuple{\alpha}{r},\AutoTuple{\alpha}[r+1]{n}\)下的矩阵是
一个分块上三角矩阵\(\begin{bmatrix}
	A_1 & A_3 \\
	0 & A_2
\end{bmatrix}\),
其中\(A_1\)是\(A \setrestrict W\)在\(W\)的一个基\(\AutoTuple{\alpha}{r}\)下的矩阵.

充分性.
设\(\vb{A}\)在\(V\)的一个基\(\AutoTuple{\alpha}{n}\)下的矩阵是
一个分块上三角矩阵\(\begin{bmatrix}
	A_1 & A_3 \\
	0 & A_2
\end{bmatrix}\),
其中\(A_1\)是\(r\)阶矩阵,
且\(0 < r < n\).
%@see: 《高等代数(第三版 下册)》(丘维声) P133 (4)
令\(W = \Span\{\AutoTuple{\alpha}{r}\}\),
则\(\vb{A}\alpha_i \in W\ (i=1,2,\dotsc,r)\).
因此\(W\)是\(\vb{A}\)的不变子空间.
显然\(W\)是非平凡的.
此时\(A_1\)是\(A \setrestrict W\)在\(W\)的基\(\AutoTuple{\alpha}{r}\)下的矩阵.
\end{proof}
\end{theorem}

\subsection{线性空间的直和分解式}
\begin{theorem}\label{theorem:线性映射.线性空间可以分解为线性变换的一些非平凡不变子空间的直和的充分必要条件}
%@see: 《高等代数(第三版 下册)》(丘维声) P133 定理6
设\(\vb{A}\)是域\(F\)上\(n\)维线性空间\(V\)上的线性变换,
则\(V\)能分解成\(\vb{A}\)的一些非平凡不变子空间的直和,
当且仅当\(V\)中存在一个基,
使得\(\vb{A}\)在这个基下的矩阵是一个分块对角矩阵.
\def\BasisV{\alpha_{11},\dotsc,\alpha_{1 r_1},\dotsc,\alpha_{s1},\dotsc,\alpha_{s r_s}}
\def\BasisWi{\alpha_{i1},\dotsc,\alpha_{ir_i}}
\begin{proof}
必要性.
设\(V\)是\(\vb{A}\)的一些非平凡不变子空间的直和:\[
%@see: 《高等代数(第三版 下册)》(丘维声) P133 (6)
	V = W_1 \DirectSum \dotsb \DirectSum W_s.
\]
在每个\(W_i\ (i=1,2,\dotsc,s)\)中取一个基\(\BasisWi\),
由上式可知\[
%@see: 《高等代数(第三版 下册)》(丘维声) P133 (7)
	\BasisV
\]是\(V\)的一个基.
由于\(W_i\)是\(\vb{A}\)的不变子空间,
因此\[
%@see: 《高等代数(第三版 下册)》(丘维声) P133 (8)
	\vb{A} (\BasisWi)
	= (\BasisWi) A_i,
	\quad i=1,2,\dotsc,s.
\]
从而\(\vb{A}\)在基\(\BasisV\)下的矩阵为\[
%@see: 《高等代数(第三版 下册)》(丘维声) P133 (9)
	\begin{bmatrix}
		A_1 \\
		& A_2 \\
		& & \ddots \\
		& & & A_s
	\end{bmatrix}.
\]

充分性.
设\(\vb{A}\)在\(V\)一个基\[
	\BasisV
\]下的矩阵\(\vb{A} = \diag\{\AutoTuple{A}{s}\}\),
其中\(A_i\)是\(r_i\)阶方阵,
而\(i=1,2,\dotsc,s\).
令\[
	W_i = \Span\{\BasisWi\}
	\quad(i=1,2,\dotsc,s).
\]
由于\[
%@see: 《高等代数(第三版 下册)》(丘维声) P133 (10)
	\vb{A} (\BasisWi) = (\BasisWi) A_i,
	\quad i=1,2,\dotsc,s,
\]
所以\(\vb{A}\alpha_{i1},\dotsc,\vb{A}\alpha_{i r_i} \in W_i\),
从而\(W_i\)是\(\vb{A}\)的不变子空间.
显然\(W_i\)是非平凡的.
由于\(W_i\)的一个基\(\BasisWi\ (i=1,2,\dotsc,s)\)合起来\(V\)的一个基,
所以\(V = W_1 \DirectSum \dotsb \DirectSum W_s\).
\end{proof}
\begin{remark}
从\cref{theorem:线性映射.线性空间可以分解为线性变换的一些非平凡不变子空间的直和的充分必要条件} 的证明过程中可以看出,
在线性变换\(\vb{A}\)的矩阵\[
	\begin{bmatrix}
		A_1 \\
		& A_2 \\
		& & \ddots \\
		& & & A_s
	\end{bmatrix}
\]中,
\(A_i\)就是\(\vb{A}\)在它的不变子空间\(W_i\)上的限制\(\vb{A} \setrestrict W_i\)
在\(W_i\)的一个基\(\BasisWi\)下的矩阵,
其中\(i=1,2,\dotsc,s\).
\end{remark}
\end{theorem}

\begin{example}
%@see: 《高等代数(第三版 下册)》(丘维声) P131 习题9.5 2.(1)
设\(W\)是线性空间\(V\)上可逆线性变换\(\vb{A}\)的有限维不变子空间.
证明:\(\vb{A} \setrestrict W\)是\(W\)上的可逆线性变换.
%TODO proof
% 先说明\(\vb{A} \setrestrict W\)是单射
% 再据此结论说明\(\vb{A} \setrestrict W\)是满射
\end{example}

\begin{example}
%@see: 《高等代数(第三版 下册)》(丘维声) P131 习题9.5 2.(2)
设\(W\)是线性空间\(V\)上可逆线性变换\(\vb{A}\)的有限维不变子空间.
证明:\(W\)是\(\vb{A}^{-1}\)的不变子空间,
且\((\vb{A} \setrestrict W)^{-1}
= \vb{A}^{-1} \setrestrict W\).
%TODO proof
% 利用【《高等代数(第三版 下册)》(丘维声) P131 习题9.5 2.(1)】的结论
% 根据定义去证\(W\)是\(\vb{A}^{-1}\)的不变子空间
% 任取\(\beta \in W\),
% 令\((\vb{A} \setrestrict W)^{-1} \beta = \gamma\)
% 则\(\gamma \in W\)
% 最后证明\(\gamma = (\vb{A}^{-1} \setrestrict W) \beta\)
\end{example}

\subsection{求解非平凡子空间的基本步骤}
虽然我们从\cref{theorem:线性映射.线性空间可以分解为线性变换的一些非平凡不变子空间的直和的充分必要条件} 可以看出,
如果\(n\)维线性空间\(V\)能分解成线性变换\(\vb{A}\)的一些非平凡不变子空间的直和,
那么\(V\)中存在一个基,
使得\(\vb{A}\)在这个基下的矩阵是一个分块对角矩阵,
但是,要如何找出\(\vb{A}\)的所有非平凡不不变子空间呢?
根据\cref{theorem:线性映射.线性变换的不变子空间3},
对于任意一个\(f(x) \in F[x]\)都有\(\Ker f(\vb{A})\)是\(\vb{A}\)的不变子空间.
受此启发,我们希望找到一些多项式\(f_1(x),\dotsc,f_s(x) \in F[x]\),
使得\[
%@see: 《高等代数(第三版 下册)》(丘维声) P134 (11)
	V = \Ker f_1(\vb{A}) \DirectSum \dotsb \DirectSum \Ker f_s(\vb{A}).
\]
为此,我们尚需研究:
对于不同的一元多项式\(f_1(x)\)和\(f_2(x)\),
不定元\(x\)用\(\vb{A}\)代入,
得到的\(f_1(\vb{A})\)与\(f_2(\vb{A})\)的核,
\(\Ker f_1(\vb{A})\)与\(\Ker f_2(\vb{A})\)之间,
有什么关系.
\begin{theorem}%\label{theorem:线性映射.线性映射多项式的核空间的直和分解式1}
%@see: 《高等代数(第三版 下册)》(丘维声) P134 定理7
设\(\vb{A}\)是域\(F\)上线性空间\(V\)上的线性变换,
而\(f(x),f_1(x),f_2(x)\)都是域\(F\)上的一元多项式.
如果\[
	f(x) = f_1(x) f_2(x)
	\quad\text{且}\quad
	(f_1(x),f_2(x)) = 1,
\]
则\[
%@see: 《高等代数(第三版 下册)》(丘维声) P134 (12)
	\Ker f(\vb{A})
	= \Ker f_1(\vb{A})
	\DirectSum \Ker f_2(\vb{A}).
\]
%TODO proof
\end{theorem}

用数学归纳法可以将上述定理推广.
\begin{corollary}\label{theorem:线性映射.线性映射多项式的核空间的直和分解式2}
%@see: 《高等代数(第三版 下册)》(丘维声) P135 推论8
设\(\vb{A}\)是域\(F\)上线性空间\(V\)上的线性变换,
\(f(x)\)以及\(f_1(x),\dotsc,f_s(x)\)都是域\(F\)上的一元多项式.
如果\[
	f(x) = f_1(x) \dotsm f_s(x)
	\quad\text{且}\quad
	\text{$f_1(x),\dotsc,f_s(x)$两两互素},
\]
则\[
%@see: 《高等代数(第三版 下册)》(丘维声) P135 (15)
	\Ker f(\vb{A})
	= \Ker f_1(\vb{A})
	\DirectSum
	\dotsb
	\DirectSum \Ker f_s(\vb{A}).
\]
\end{corollary}

\subsection{线性变换的零化多项式}
由于\(\Ker\vb0 = V\),
结合\cref{theorem:线性映射.线性映射多项式的核空间的直和分解式2} 给出的暗示,
不难想到,
如果能够找到一个多项式\(f(x)\),
使得\(f(\vb{A}) = \vb0\),
那么空间\(V\)就能分解成\[
%@see: 《高等代数(第三版 下册)》(丘维声) P135 (16)
	V = \Ker f_1(\vb{A})
	\DirectSum
	\dotsb
	\DirectSum \Ker f_s(\vb{A}),
\]
其中\(f(x) = f_1(x) \dotsm f_s(x)\),
且\(f_1(x),\dotsc,f_s(x)\)两两互素.

\begin{definition}
%@see: 《高等代数(第三版 下册)》(丘维声) P135 定义2
设\(\vb{A}\)是\(V\)上的线性变换.
如果域\(F\)上的一个一元多项式\(f(x)\)满足\(f(\vb{A}) = \vb0\),
则称“\(f(x)\)是\(\vb{A}\)的一个\DefineConcept{零化多项式}”.
\end{definition}

容易看出,零多项式是任意一个线性变换的零化多项式.
因此,我们通常讨论的零化多项式是“非零的零化多项式”.

\begin{proposition}
%@see: 《高等代数(第三版 下册)》(丘维声) P135
有限维线性空间\(V\)上任意一个线性变换都有非零的零化多项式.
%TODO proof
\end{proposition}

\begin{definition}
%@see: 《高等代数(第三版 下册)》(丘维声) P135 定义3
设\(A\)是域\(F\)上的一个\(n\)阶矩阵.
如果域\(F\)上的一个一元多项式\(f(x)\)满足\(f(A) = 0\),
则称“\(f(x)\)是矩阵\(A\)的一个\DefineConcept{零化多项式}”.
\end{definition}

\begin{proposition}
%@see: 《高等代数(第三版 下册)》(丘维声) P135
设\(\vb{A}\)是域\(F\)上\(n\)维线性空间\(V\)上的线性变换,
\(A\)是\(\vb{A}\)在\(V\)的一个基下的矩阵,
则\[
	\text{$f(x)$是$\vb{A}$的零化多项式}
	\iff
	\text{$f(x)$是$A$的零化多项式}.
\]
%TODO proof
\end{proposition}

\section{哈密顿--凯莱定理}
设数域\(K\)上的2阶矩阵\(A\)为\begin{equation*}
	A = \begin{bmatrix}
		1 & 2 \\
		0 & -1
	\end{bmatrix},
\end{equation*}
则\begin{equation*}
	A^2
	= \begin{bmatrix}
		1 & 0 \\
		0 & 1
	\end{bmatrix}
	= I,
\end{equation*}
可见\(A^2 - I = 0\),
因此\(f(\lambda) = \lambda^2 - 1\)就是\(A\)的一个零化多项式.
由于\begin{equation*}
	\abs{\lambda I - A}
	= \begin{vmatrix}
		\lambda-1 & -2 \\
		0 & \lambda+1
	\end{vmatrix}
	= (\lambda-1)(\lambda+1)
	= \lambda^2-1,
\end{equation*}
因此\(A\)的特征多项式\(f(\lambda) = \lambda^2-1\)就是\(A\)的一个零化多项式.
我们不禁好奇,是不是域\(F\)上的任意一个\(n\)阶矩阵的特征多项式都是它的零化多项式.
为了讨论这个问题,我们需要首先把域上的矩阵的概念推广维整环上的矩阵.

\subsection{\texorpdfstring{$\lambda$}{\textlambda} - 矩阵}
与域\(F\)上的矩阵类似,
我们也可以为
整环\(R\)上的矩阵
定义加法、纯量乘法、乘法,
而且这三种运算满足与域\(F\)上矩阵一样的运算法则.
类似地,我们也可以定义整环\(R\)上\(n\)阶矩阵的行列式.
而且我们在之前介绍的行列式的性质、行列式展开定理,
对于整环\(R\)上的\(n\)阶矩阵的行列式也成立.
对于整环\(R\)上的\(n\)阶矩阵\(A\),
有\begin{equation*}
	A A^* = A^* A = \abs{A} I,
\end{equation*}
其中\(A^*\)是\(A\)的伴随矩阵.

域\(F\)上的一元多项式环\(F[\lambda]\)是一个整环,
因此可以考虑环\(F[\lambda]\)上的\(n\)阶矩阵.
我们把环\(F[\lambda]\)上的\(n\)阶矩阵
称为 \DefineConcept{\(\lambda\) - 矩阵}.

\begin{example}
给定\(\lambda\) - 矩阵\[
%@see: 《高等代数(第三版 下册)》(丘维声) P138 (1)
	A(\lambda) = \begin{bmatrix}
		2 \lambda^3 + \lambda^2 + 1 & \lambda^2 - 3 \\
		\lambda^3 - 1 & 2 \lambda + 5
	\end{bmatrix},
\]
我们可以按照整环上矩阵的加法、纯量乘法,
将它改写成\begin{align*}
%@see: 《高等代数(第三版 下册)》(丘维声) P138 (2)
	A(\lambda)
	&= \begin{bmatrix}
		2 \lambda^3 & 0 \\
		\lambda^3 & 0
	\end{bmatrix}
	+ \begin{bmatrix}
		\lambda^2 & \lambda^2 \\
		0 & 0
	\end{bmatrix}
	+ \begin{bmatrix}
		0 & 0 \\
		0 & 2 \lambda
	\end{bmatrix}
	+ \begin{bmatrix}
		1 & -3 \\
		-1 & 5
	\end{bmatrix} \\
	&= \lambda^3
	\begin{bmatrix}
		2 & 0 \\
		1 & 0
	\end{bmatrix}
	+ \lambda^2
	\begin{bmatrix}
		1 & 1 \\
		0 & 0
	\end{bmatrix}
	+ \lambda
	\begin{bmatrix}
		0 & 0 \\
		0 & 2
	\end{bmatrix}
	+ \begin{bmatrix}
		1 & -3 \\
		-1 & 5
	\end{bmatrix},
\end{align*}
其中\(\lambda^k\)的系数矩阵\[
	\begin{bmatrix}
		2 & 0 \\
		1 & 0
	\end{bmatrix},
	\qquad
	\begin{bmatrix}
		1 & 1 \\
		0 & 0
	\end{bmatrix},
	\qquad
	\begin{bmatrix}
		0 & 0 \\
		0 & 2
	\end{bmatrix},
	\qquad
	\begin{bmatrix}
		1 & -3 \\
		-1 & 5
	\end{bmatrix}
\]都是域\(F\)上的矩阵.
\end{example}

假设我们把两个\(\lambda\) - 矩阵\(A(\lambda)\)和\(B(\lambda)\)
都展开成\begin{equation*}
	A(\lambda)
	= \sum_{i=0}^m \lambda^i \alpha_i,
	\qquad
	B(\lambda)
	= \sum_{i=0}^m \lambda^i \beta_i,
\end{equation*}
其中\(\alpha_i,\beta_i \in M_n(F)\),
那么根据
两个一元多项式相等的定义
以及两个\(\lambda\) - 矩阵相等的定义,
可以推出,
\(A(\lambda)\)与\(B(\lambda)\)相等,
当且仅当它们系数矩阵对应相等.

\subsection{哈密顿--凯莱定理}
有了上述准备知识以后,
我们就可以着手证明下面的哈密顿--凯莱定理了.
\begin{theorem}
%@see: 《高等代数(第三版 下册)》(丘维声) P138 Hamilton-Cayley定理
设\(A\)是域\(F\)上的\(n\)阶矩阵,
则\(A\)的特征多项式\(f(\lambda)\)是\(A\)的一个零化多项式.
%TODO proof
\end{theorem}

\begin{corollary}
%@see: 《高等代数(第三版 下册)》(丘维声) P139 Hamilton-Cayley定理
设\(\vb{A}\)是域\(F\)上\(n\)维线性空间\(V\)上的一个线性变换,
则\(\vb{A}\)的特征多项式\(f(\lambda)\)是\(\vb{A}\)的一个零化多项式.
\end{corollary}

\section{线性变换的最小多项式}
由哈密顿凯莱定理可知,
有限维线性空间\(V\)上的线性变换\(\vb{A}\)的特征多项式\(f(\lambda)\)是\(\vb{A}\)的一个零化多项式.
在本节我们来讨论\(\vb{A}\)还有没有其他零化多项式.

\subsection{线性变换的最小多项式}
\begin{definition}
%@see: 《高等代数(第三版 下册)》(丘维声) P140 定义1
设\(\vb{A}\)是域\(F\)上线性空间\(V\)上的一个线性变换.
在\(\vb{A}\)的所有非零的零化多项式中,
次数最低的首项系数为\(1\)的多项式,
称为“\(\vb{A}\)的\DefineConcept{最小多项式}”.
\end{definition}

\begin{proposition}
%@see: 《高等代数(第三版 下册)》(丘维声) P140 命题1
线性空间\(V\)上的线性变换\(\vb{A}\)的最小多项式是唯一的.
%TODO proof
\end{proposition}

\begin{proposition}
%@see: 《高等代数(第三版 下册)》(丘维声) P140 命题2
线性空间\(V\)上的线性变换\(\vb{A}\)的
任一零化多项式\(g(\lambda)\)是
\(\vb{A}\)的最小多项式\(m(\lambda)\)的倍式.
%TODO proof
\end{proposition}

\begin{proposition}
%@see: 《高等代数(第三版 下册)》(丘维声) P140 命题3
域\(F\)上有限维线性空间\(V\)上的线性变换\(\vb{A}\)的最小多项式\(m(\lambda)\)
与特征多项式\(f(\lambda)\)
在\(F\)中有相同的根(重数可以不同).
%TODO proof
\end{proposition}

\begin{definition}
%@see: 《高等代数(第三版 下册)》(丘维声) P140 定义2
设\(A\)是域\(F\)上的\(n\)阶矩阵.
在\(A\)的所有非零的零化多项式中,
次数最低的首项系数为\(1\)的多项式,
称为“\(A\)的\DefineConcept{最小多项式}”.
\end{definition}

%@see: 《高等代数(第三版 下册)》(丘维声) P141
设\(n\)维线性空间\(V\)上的线性变换\(\vb{A}\)在\(V\)的一个基下的矩阵是\(A\).
之前我们已经指出,\(g(\lambda)\)是\(\vb{A}\)的零化多项式,
当且仅当\(g(\lambda)\)是\(A\)的零化多项式.
由此可见,\(m(\lambda)\)是\(\vb{A}\)的最小多项式,
当且仅当\(m(\lambda)\)是\(A\)的做小多项式.
于是得出:
\begin{corollary}
%@see: 《高等代数(第三版 下册)》(丘维声) P141 推论4
域\(F\)上\(n\)阶矩阵\(A\)的最小多项式\(m(\lambda)\)
与特征多项式\(f(x)\)
在\(F\)中有相同的根(重数可以不同).
\end{corollary}

%@see: 《高等代数(第三版 下册)》(丘维声) P141
由于相似的矩阵可以看作\(V\)上同一个线性变换\(\vb{A}\)在\(V\)的不同基下的矩阵,
因此有如下结论:
\begin{corollary}
%@see: 《高等代数(第三版 下册)》(丘维声) P141 推论5
相似的矩阵有相同的最小多项式.
\end{corollary}

\subsection{求解最小多项式的基本步骤}
如何求解线性变换或矩阵的最小多项式呢?
一种方法是:
先找出\(\vb{A}\)的一个零化多项式\(g(\lambda)\),
然后分析\(g(\lambda)\)的哪个因式是\(\vb{A}\)的最小多项式.
\begin{definition}%\label{definition:线性映射.幂零变换}
%@see: 《高等代数(第三版 下册)》(丘维声) P141 定义3
设\(\vb{A}\)是域\(F\)上线性空间\(V\)上的线性变换.
如果存在一个正整数\(k\),
使得\(\vb{A}^k = \vb0\),
则称\(\vb{A}\)是\DefineConcept{幂零变换}.
使得\(\vb{A}^k = \vb0\)成立的最小正整数
称为“\(\vb{A}\)的\DefineConcept{幂零指数}”.
\end{definition}

\begin{proposition}%\label{theorem:线性映射.幂零变换的等价命题}
%@see: 《高等代数(第三版 下册)》(丘维声) P141
设\(\vb{A}\)是域\(F\)上线性空间\(V\)上的线性变换,
则\begin{align*}
	&\text{$\vb{A}$是幂零指数为$k$的幂零变换} \\
	&\iff \vb{A}^k = \vb0,
		(\forall r<k)[\vb{A}^r \neq \vb0] \\
	&\iff \text{$\lambda^k$是$\vb{A}$的一个零化多项式},
		(\forall r<k)[\text{$\lambda^k$不是$\vb{A}$的一个零化多项式}] \\
	&\iff \text{$\vb{A}$的最小多项式是$\lambda^k$}.
\end{align*}
如果域\(F\)上\(n\)阶矩阵\(A\)是\(\vb{A}\)在\(V\)的一个基下的矩阵,
则\begin{equation*}
	\text{$A$是幂零指数为$k$的幂零矩阵}
	\iff \text{$A$的最小多项式是$\lambda^k$}.
\end{equation*}
\end{proposition}

\begin{proposition}%\label{theorem:线性映射.幂等变换的等价命题}
%@see: 《高等代数(第三版 下册)》(丘维声) P141
设\(\vb{A}\)是域\(F\)上线性空间\(V\)上的线性变换,
则\begin{align*}
	&\text{$\vb{A}$是幂等变换} \\
	&\iff \vb{A}^2 = \vb{A} \\
	&\iff \text{$\lambda^2 - \lambda$是$\vb{A}$的一个零化多项式} \\
	&\iff \text{$\vb{A}$的最小多项式是$\lambda^2-\lambda$或$\lambda$或$\lambda-1$}.
\end{align*}
如果域\(F\)上\(n\)阶矩阵\(A\)是\(\vb{A}\)在\(V\)的一个基下的矩阵,
则\begin{equation*}
	\text{$A$是幂等变换}
	\iff \text{$A$的最小多项式是$\lambda^2-\lambda$或$\lambda$或$\lambda-1$}.
\end{equation*}
\end{proposition}

\begin{proposition}\label{theorem:线性映射.任意线性变换的最小多项式}
%@see: 《高等代数(第三版 下册)》(丘维声) P141 命题6
设\(V\)是域\(F\)上的一个线性空间,
\(\vb{A}\)是\(V\)上的线性变换,
\(\vb{B}\)是\(V\)上幂零指数为\(k\)的幂零变换,
则\begin{gather*}
	\vb{A} = a \vb{I} + \vb{B}
	\iff \text{$\vb{A}$的最小多项式是$(\lambda-a)^k$}, \\
	\vb{A} = a \vb{I}
	\iff \text{$\vb{A}$的最小多项式是$\lambda-a$}.
\end{gather*}
%TODO proof
\end{proposition}

\begin{definition}%\label{definition:线性映射.若尔当块}
设\(F\)是一个域.
对于\(\forall a \in F\)和\(\forall k \in \mathbb{N}^+\),
定义:\begin{equation}
%@see: 《高等代数(第三版 下册)》(丘维声) P142 (1)
	J_k(a)
	\defeq
	\begin{bmatrix}
		a & 1 & 0 & \dots & 0 & 0 \\
		0 & a & 1 & \dots & 0 & 0 \\
		0 & 0 & a & \dots & 0 & 0 \\
		\vdots & \vdots & \vdots & & \vdots & \vdots \\
		0 & 0 & 0 & \dots & a & 1 \\
		0 & 0 & 0 & \dots & 0 & a
	\end{bmatrix}.
\end{equation}
把\(J_k(a)\)称为一个\(k\)阶\DefineConcept{若尔当块}.
\end{definition}

\begin{example}
%@see: 《高等代数(第三版 下册)》(丘维声) P136 习题9.5 4.
设\(\vb{A}\)是域\(F\)上\(n\)维线性空间\(V\)上的一个线性变换,
\(\vb{A}\)在基\(\AutoTuple{\alpha}{n}\)下的矩阵为\(n\)阶若尔当块\(J_n(a)\).
证明:\begin{itemize}
	\item 如果\(\alpha_n\)属于\(\vb{A}\)的不变子空间\(W\),则\(W = V\);
	\item 基向量\(\alpha_1\)属于\(\vb{A}\)的任意一个非零不变子空间;
	\item \(V\)不能分解成\(\vb{A}\)的两个非平凡不变子空间的直和.
\end{itemize}
%TODO proof
\end{example}

\begin{proposition}
%@see: 《高等代数(第三版 下册)》(丘维声) P142 命题7
设\(\vb{A}\)是域\(F\)上\(k\)维线性空间\(W\)上的线性变换.
若\(\vb{A} = a \vb{I} + \vb{B}\),
其中\(\vb{B}\)是\(W\)上幂零指数为\(k\)的幂零变换,
则\(W\)中有一个基\[
	\vb{B}^{k-1}\alpha,\vb{B}^{k-2}\alpha,\dotsc,\vb{B}\alpha,\alpha,
\]
使得\(\vb{A}\)在这个基下的矩阵为\(J_k(a)\).
%TODO proof
\end{proposition}

\begin{proposition}\label{theorem:线性映射.矩阵的最小多项式}
%@see: 《高等代数(第三版 下册)》(丘维声) P142 命题8
设\(A\)是域\(F\)上的一个\(k\)阶矩阵,
则\(A \sim J_k(a)\)
当且仅当\(A\)的最小多项式是\((\lambda-a)^k\).
%TODO proof
\end{proposition}

%@see: 《高等代数(第三版 下册)》(丘维声) P143
从幂零变换和幂等变换的最小多项式,
以及\cref{theorem:线性映射.任意线性变换的最小多项式},
我们可以看出:
线性变换的最小多项式能够决定这个线性变换是什么样子.
从\cref{theorem:线性映射.矩阵的最小多项式}
可以看出:
矩阵的最小多项式能够决定这个矩阵相似于什么样的形式最简单的矩阵(即若尔当块).
这促使我们利用线性变换的最小多项式来研究线性变换的形式最简的矩阵表示.

\subsection{线性映射在非平凡子空间上的限制的最小多项式}
设\(\vb{A}\)是域\(F\)上\(n\)维线性空间\(V\)上的线性变换.
把\(\vb{A}\)的最小多项式\(m(\lambda)\)在\(F[\lambda]\)中分解成
两两不等的首项系数为\(1\)的不可约多项式方幂的乘积,
则根据\cref{theorem:线性映射.线性映射多项式的核空间的直和分解式2} 可知,
\(V\)可以分解成\(\vb{A}\)的非平凡不变子空间\(\AutoTuple{W}{s}\)的直和:\[
	V = W_1 \DirectSum \dotsb \DirectSum W_s.
\]
在这些非平凡不变子空间中各取一个基,
把它们合起来得到\(V\)的一个基,
则\(\vb{A}\)在\(V\)的这个基下的矩阵\(A\)是一个分块对角矩阵
\(A = \diag\{\AutoTuple{A}{s}\}\),
其中\(A_j\)是\(\vb{A} \setrestrict W_j\)在\(W_j\)的上述基下的矩阵.
为了使\(\vb{A}\)有形式最简单的矩阵表示,
我们自然需要在\(W_j\)中取一个合适的基,
使得\(\vb{A} \setrestrict W_j\)在\(W_j\)的这个基下的矩阵具有最简单的形式.
为此产生一个问题:
\(\vb{A} \setrestrict W_j\)的最小多项式\(m_j(\lambda)\)
与\(\vb{A}\)的最小多项式\(m(\lambda)\)有什么关系?
下面的定义回答了这个问题.
\begin{theorem}
%@see: 《高等代数(第三版 下册)》(丘维声) P143 定理9
设\(\vb{A}\)是域\(F\)上线性空间\(V\)上线性变换.
如果\(V\)能分解成\(\vb{A}\)的一些非平凡不变子空间\(\AutoTuple{W}{s}\)的直和,
即\[
	V = W_1 \DirectSum \dotsb \DirectSum W_s,
\]
则\(\vb{A}\)的最小多项式为\[
	m(\lambda)
	= [m_1(\lambda),\dotsc,m_s(\lambda)],
\]
其中\(m_j(\lambda)\)是\(W_j\)上的线性变换\(\vb{A} \setrestrict W_j\)的最小多项式,
而\([m_1(\lambda),\dotsc,m_s(\lambda)]\)是
\(m_1(\lambda),\dotsc,m_s(\lambda)\)的最小公倍式.
%TODO proof
\end{theorem}

\begin{corollary}
%@see: 《高等代数(第三版 下册)》(丘维声) P144 推论10
设\(A\)是域\(F\)上一个\(n\)阶分块对角矩阵,
即\(A = \diag\{\AutoTuple{A}{s}\}\),
则\(A\)的最小多项式为\[
	m(\lambda)
	= [m_1(\lambda),\dotsc,m_s(\lambda)],
\]
其中\(m_j(\lambda)\)是矩阵\(A_j\)的最小多项式,
而\([m_1(\lambda),\dotsc,m_s(\lambda)]\)是
\(m_1(\lambda),\dotsc,m_s(\lambda)\)的最小公倍式.
%TODO proof
\end{corollary}

\begin{definition}
%@see: 《高等代数(第三版 下册)》(丘维声) P144 定义4
由若干个若尔当块组成的分块对角矩阵
称为\DefineConcept{若尔当形矩阵}.
\end{definition}
\begin{remark}
对角矩阵可以看成是由若干个1阶若尔当块组成的若尔当形矩阵.
\end{remark}

线性变换的最小多项式在研究线性变换的结构中起着十分重要的作用.
现在先利用最小多项式给出线性变换可对角化的一个充分必要条件,
然后用最小多项式研究不可以对角化的线性变换的结构.

\begin{theorem}
%@see: 《高等代数(第三版 下册)》(丘维声) P145 定理11
设\(\vb{A}\)是域\(F\)上\(n\)维线性空间\(V\)上的线性变换,
则\(\vb{A}\)可对角化的充分必要条件是,
\(\vb{A}\)的最小多项式\(m(\lambda)\)在\(F[\lambda]\)中
能分解成不同的一次因式的乘积.
%TODO proof
\end{theorem}

\begin{corollary}
%@see: 《高等代数(第三版 下册)》(丘维声) P145 推论12
域\(F\)上\(n\)阶矩阵\(A\)可对角化的充分必要条件是,
\(A\)的最小多项式在\(F[\lambda]\)中
能分解成不同的一次因式的乘积.
\end{corollary}
\begin{remark}
在一些情形下,
利用最小多项式来判别线性变换或矩阵是否可以对角化,
论证过程往往很简洁.
\end{remark}

利用最小多项式还可以研究不可对角化的线性变换的结构.

设\(\vb{A}\)是域\(F\)上\(n\)维线性空间\(V\)上的线性变换.
如果\(\vb{A}\)的最小多项式\(m(\lambda)\)在\(F[\lambda]\)中
能够分解成一次因式的乘积:\[
%@see: 《高等代数(第三版 下册)》(丘维声) P146 (5)
	m(\lambda)
	= (\lambda-\lambda_1)^{k_1}
	(\lambda-\lambda_2)^{k_2}
	\dotsm
	(\lambda-\lambda_s)^{k_s},
\]
其中\(\AutoTuple{\lambda}{s}\)是\(F\)中两两不同的元素,
则\[
%@see: 《高等代数(第三版 下册)》(丘维声) P146 (6)
	V
	= \Ker(\vb{A} - \lambda_1 \vb{I})^{k_1}
	\DirectSum
	\Ker(\vb{A} - \lambda_2 \vb{I})^{k_2}
	\DirectSum
	\dotsb
	\DirectSum
	\Ker(\vb{A} - \lambda_s \vb{I})^{k_s}.
\]
记\(W_j = \Ker(\vb{A} - \lambda_j \vb{I})^{k_j}\),
那么由\cref{theorem:线性映射.线性变换的不变子空间3} 可知,
\(W_j\)是\(\vb{A}\)的不变子空间.
由上式可知\[
%@see: 《高等代数(第三版 下册)》(丘维声) P146 (7)
	V = W_1 \DirectSum W_2 \DirectSum \dotsb \DirectSum W_s.
\]
在\(W_j\)中取一个基,
把它们合起来得到\(V\)的一个基,
\(\vb{A}\)在\(V\)的这个基下的矩阵\(A\)是一个分块对角矩阵
\(A = \diag\{\AutoTuple{A}{s}\}\),
其中\(A_j\)是\(\vb{A} \setrestrict W_j\)在\(W_j\)的上述基下的矩阵.
既然要最简化矩阵\(A\)的形式,
就应当最简化矩阵\(A_j\)的形式.
为此我们需要研究\(W_j\)上的线性变换\(\vb{A} \setrestrict W_j\).
我们断言\(\vb{A} \setrestrict W_j\)的最小多项式是\((\lambda-\lambda_j)^{k_j}\),
理由如下.

任取\(\alpha_j \in W_j\),
由于\(W_j = \Ker(\vb{A} - \lambda_j \vb{I})^{k_j}\),
所以\[
	(\vb{A} \setrestrict W_j - \lambda_j \vb{I})^{k_j} \alpha_j
	= (\vb{A} - \lambda_j \vb{I})^{k_j} \alpha_j
	= 0,
\]
从而\((\vb{A} \setrestrict W_j - \lambda_j \vb{I})^{k_j} = \vb0\),
于是\((\lambda-\lambda_j)^{k_j}\)是\(\vb{A} \setrestrict W_j\)的一个零化多项式,
因此\(\vb{A} \setrestrict W_j\)的最小多项式为
\(m_j(\lambda) = (\lambda-\lambda_j)^{t_j}\),
其中\(t_j \leq k_j\).
于是\(\vb{A}\)的最小多项式\(m(\lambda)\)为\begin{align*}
%@see: 《高等代数(第三版 下册)》(丘维声) P146 (8)
	m(\lambda)
	&= [(\lambda-\lambda_1)^{t_1},\dotsc,(\lambda-\lambda_s)^{t_s}] \\
	&= (\lambda-\lambda_1)^{t_1} \dotsm (\lambda-\lambda_s)^{t_s}.
\end{align*}
根据\hyperref[theorem:多项式.唯一因式分解定理]{唯一因式分解定理},
立即可得\[
	t_1 = k_1,
	t_2 = k_2,
	\dotsc,
	t_s = k_s.
\]
因此\(\vb{A} \setrestrict W_j\)的最小多项式\(m_j(\lambda) = (\lambda-\lambda_j)^{k_j}\).
根据\cref{theorem:线性映射.任意线性变换的最小多项式} 可知,
\(\vb{A} \setrestrict W_j = \lambda_j \vb{I} + \vb{B}_j\),
其中\(\vb{B}_j\)是\(W_j\)上的幂零变换,且幂零指数为\(k_j\).
由于\(\vb{A} \setrestrict W_j\)在\(W_j\)的上述基下的矩阵是\(A_j\),
因此\(\vb{B}_j = \vb{A} \setrestrict W_j - \lambda_j \vb{I}\)
在\(W_j\)的上述基下的矩阵\(B_j = A_j - \lambda_j I\).
于是为了最简化矩阵\(A_j\),
就应当最简化矩阵\(B_j\).
这里就产生一个问题:
如果\(\vb{B}\)是域\(F\)上\(r\)维线性空间\(W\)上幂零指数为\(k\)的幂零变换,
那么能否在\(W\)中找到一个基,使得\(\vb{B}\)在这个基下的矩阵最简单?
我们将在下一节详细讨论这个问题.

\section{幂零变换的结构}

\section{线性变换的若尔当标准型}
现在我们利用幂零变换的结构来研究最小多项式可以分解成一次因式乘积的线性变换的结构.
\begin{theorem}\label{theorem:线性变换的结构.线性变换的若尔当标准型}
%@see: 《高等代数(第三版 下册)》(丘维声) P153 定理1
设\(\vb{A}\)是域\(F\)上\(n\)维线性空间\(V\)上的线性变换.
如果\(\vb{A}\)的最小多项式\(m(\lambda)\)
在\(F[\lambda]\)中能分解成一次因式的乘积\begin{equation}\label{equation:线性变换的若尔当标准型.线性变换的最小多项式的完全分解式}
%@see: 《高等代数(第三版 下册)》(丘维声) P153 (1)
	m(\lambda)
	= \LambdaExp{1}[l_1]
	\LambdaExp{2}[l_2]
	\dotsm
	\LambdaExp{s}[l_s],
\end{equation}
则\(V\)中存在一个基,
使得\(\vb{A}\)在这个基下的矩阵\(A\)是若尔当形矩阵,其主对角元是\(\vb{A}\)的全部特征值,
主对角元为\(\lambda_j\)的若尔当块的总数为\begin{equation*}
%@see: 《高等代数(第三版 下册)》(丘维声) P153 (2)
	N_j = \dim V - \rank\EpAj,
	\quad j=1,2,\dotsc,s,
\end{equation*}
且其中每个若尔当块的阶数不超过\(l_j\),
\(t\)阶若尔当块\(J_t(\lambda_j)\)的个数为\begin{equation*}
%@see: 《高等代数(第三版 下册)》(丘维声) P153 (3)
	N_j(t)
	= \rank\EpAj[t+1]
	+ \rank\EpAj[t-1]
	- 2 \rank\EpAj[t].
\end{equation*}
\rm
这个若尔当形矩阵\(A\)称为\(\vb{A}\)的若尔当标准型.
除去若尔当块的排序次序外,\(\vb{A}\)的若尔当标准型是唯一的.
%TODO proof
\begin{proof}
从\cref{equation:线性变换的若尔当标准型.线性变换的最小多项式的完全分解式}
可以看出,\(\AutoTuple{\lambda}{s}\)是\(\vb{A}\)的所有不同特征值,
于是\begin{equation*}
%@see: 《高等代数(第三版 下册)》(丘维声) P153 (4)
%@see: 《高等代数(第三版 下册)》(丘维声) P153 (5)
	V = W_1 \DirectSum \dotsb \DirectSum W_s,
	\eqno(1)
\end{equation*}
其中\(W_j \defeq \Ker\EpAj[l_j]\ (j=1,2,\dotsc,s)\)是\(\vb{A}\)的不变子空间.
记\(\vb{B}_j \defeq \RepAj\),
显然\(\vb{B}_j\)是\(W_j\)上的幂零变换,其幂零指数为\(l_j\).
由\cref{theorem:幂零变换的结构.幂零变换的若尔当标准型} 可知,
在\(W_j\)中存在一个基\(\mathcal{W}_j\),
使得\(\vb{B}_j\)在基\(\mathcal{W}_j\)下的矩阵\(B_j\)是若尔当形矩阵,
\(\vb{A} \SetRestrict W_j\)在基\(\mathcal{W}_j\)下的矩阵\(A_j\)等于\(B_j + \lambda_j I\),
并且\(A_j\)是主对角元是\(\lambda_j\)的若尔当形矩阵.
把\(W_j\ (j=1,2,\dotsc,s)\)的上述基合起来便得到\(V\)的一个基
\(\mathcal{V} = \mathcal{W}_1 \cup \dotsb \cup \mathcal{W}_s\),
\(\vb{A}\)在基\(\mathcal{V}\)下的矩阵\(A\)为\begin{equation*}
%@see: 《高等代数(第三版 下册)》(丘维声) P153 (6)
	A = \diag(\AutoTuple{A}{s}),
\end{equation*}
这就说明\(A\)是若尔当形矩阵,
其主对角元是\(A\)的全部特征值.

由于\(\vb{B}_j\)的幂零指数为\(l_j\),
因此由\cref{theorem:幂零变换的结构.幂零变换的若尔当标准型} 可知
\(B_j\)中每个若尔当块的阶数不超过\(l_j\),
从而\(A_j\)中每个若尔当块的阶数不超过\(l_j\).
当\(1 \leq t \leq l_j\)时,
\(A_j\)中\(t\)阶若尔当块的个数\(N_j(t)\)等于\(B_j\)中\(t\)阶若尔当块的个数,
因此\begin{align*}
%@see: 《高等代数(第三版 下册)》(丘维声) P154 (7)
	&N_j(t)
	= \rank\vb{B}_j^{t+1} + \rank\vb{B}_j^{t-1} - 2\rank\vb{B}_j^t \\
	&= \rank\RepAj[t+1]
		+ \rank\RepAj[t-1]
		- 2\rank\RepAj[t] \\
	&= (\dim W_j - \dim\Ker\RepAj[t+1] ) \\
		&\hspace{20pt}+ (\dim W_j - \dim\Ker\RepAj[t-1] ) \\
		&\hspace{20pt}- 2(\dim W_j - \dim\Ker\RepAj[t] ) \\
	&= 2\dim\Ker\RepAj[t]
		- \dim\Ker\RepAj[t+1]
		- \dim\Ker\RepAj[t-1].
	\tag2
\end{align*}
当\(i \leq l_j\)时,有\begin{align*}
%@see: 《高等代数(第三版 下册)》(丘维声) P154 (8)
	&\alpha \in \Ker\EpAj[i] \\
	&\iff \EpAj[i] \alpha = 0 \\
	&\iff \alpha \in \Ker\EpAj[l_j] = W_j,
			\EpAj[i] \alpha = 0 \\
	&\iff \alpha \in W_j,
			\RepAj[i] \alpha = 0 \\
	&\iff \alpha \in \Ker\RepAj[i],
\end{align*}
因此\begin{equation*}
%@see: 《高等代数(第三版 下册)》(丘维声) P154 (9)
	\Ker\EpAj[i] = \Ker\RepAj[i].
	\eqno(3)
\end{equation*}
当\(i = l_j + 1\)时,上式仍然成立,
这是因为从\cref{equation:线性变换的若尔当标准型.线性变换的最小多项式的完全分解式}
可得\begin{equation*}
%@see: 《高等代数(第三版 下册)》(丘维声) P154 (12)
	m(\lambda)\LambdaExp{j}
	= \LambdaExp{1}[l_1] \dotsm
	\LambdaExp{j}[l_j+1] \dotsm
	\LambdaExp{s}[l_s],
\end{equation*}
从而\(V\)可以分解成\begin{equation*}
%@see: 《高等代数(第三版 下册)》(丘维声) P154 (13)
	V = \Ker\EigenPoly{\vb{A}}{1}[l_1]
		\DirectSum \dotsb \DirectSum
		\Ker\EigenPoly{\vb{A}}{j}[l_j+1]
		\DirectSum \dotsb \DirectSum
		\Ker\EigenPoly{\vb{A}}{s}[l_s];
	\eqno(4)
\end{equation*}
任取\(\beta\in\Ker\EpAj[l_j+1]\),
那么根据(1)式可知,
\(\beta\)可以分解成\begin{equation*}
%@see: 《高等代数(第三版 下册)》(丘维声) P155 (14)
	\beta = \beta_1 + \dotsb + \beta_j + \dotsb + \beta_s,
\end{equation*}
其中\(\beta_u\in\Ker\EigenPoly{\vb{A}}{u}[l_u]\ (u=1,2,\dotsc,s)\);
由于\(\beta_j\in\Ker\EpAj[l_j]\),
因此\(\beta_j\in\Ker\EpAj[l_j+1]\),
从而\(\beta_j-\beta\in\Ker\EpAj[l_j+1]\),
再由上式可得\begin{equation*}
%@see: 《高等代数(第三版 下册)》(丘维声) P155 (15)
	0 = \beta_1 + \dotsb + (\beta_j-\beta) + \dotsb + \beta_s,
\end{equation*}
上式是\(0\)在\(V\)的直和分解式(4)中一种表达方式,
由于表法唯一,因此\(\beta_j-\beta=0\),
从而\(\beta=\beta_j\in\Ker\EpAj[l_j]\),
由此可知\(\Ker\EpAj[l_j+1] \subseteq \Ker\EpAj[l_j]\),
显然还有\(\Ker\EpAj[l_j+1] \supseteq \Ker\EpAj[l_j]\),
从而有\begin{equation*}
%@see: 《高等代数(第三版 下册)》(丘维声) P155 (16)
	\Ker\EpAj[l_j+1] = \Ker\EpAj[l_j];
	\eqno(5)
\end{equation*}
现在任取\(\eta\in\Ker\RepAj[l_j+1]\),
则\(\eta \in W_j\);
由于\(W_j = \Ker\EpAj[l_j]
= \Ker\RepAj[l_j]\),
因此\(\eta\in\Ker\RepAj[l_j]\),
从而有\begin{equation*}
%@see: 《高等代数(第三版 下册)》(丘维声) P155 (17)
	\Ker\RepAj[l_j+1] = \Ker\RepAj[l_j];
	\eqno(6)
\end{equation*}
由(3)(5)(6)三式可知\begin{equation*}
%@see: 《高等代数(第三版 下册)》(丘维声) P155 (18)
	\Ker\EpAj[l_j+1] = \Ker\RepAj[l_j+1].
\end{equation*}

于是由(2)(3)两式得\begin{align*}
%@see: 《高等代数(第三版 下册)》(丘维声) P154 (10)
	&N_j(t)
	= 2\dim\Ker\EpAj[t]
		- \dim\Ker\EpAj[t+1]
		- \dim\Ker\EpAj[t-1] \\
	&= 2(\dim V-\dim\Img\EpAj[t]) \\
		&\hspace{20pt} - (\dim V-\dim\Img\EpAj[t+1]) \\
		&\hspace{20pt} - (\dim V-\dim\Img\EpAj[t-1]) \\
	&= \rank\EpAj[t+1]
		+ \rank\EpAj[t-1]
		- 2\rank\EpAj[t].
\end{align*}

\(A_j\)中若尔当块的总数\(N_j\)
等于\(B_j\)中若尔当块的总数,
继而等于\(\vb{B}_j\)的特征子空间\((W_j)_0\)的维数,
于是\begin{align*}
%@see: 《高等代数(第三版 下册)》(丘维声) P154 (11)
	&N_j
	= \dim(W_j)_0
	= \dim\Ker\vb{B}_j
	= \dim\Ker\RepAj \\
	&= \dim\Ker\EpAj
	= \dim V - \rank\EpAj.
\end{align*}

由于\(\vb{A}\)的若尔当标准型\(A\)的主对角线上元素是\(\vb{A}\)的全部特征值,
主对角元为特征值\(\lambda_j\)的若尔当块总数\(N_j\)由\(V\)的维数与\(\rank\EpAj\)决定,
主对角元为\(\lambda_j\)的\(t\)阶若尔当块的个数\(N_j(t)\)
由\(\rank\EpAj[t+1],\rank\EpAj[t-1],\rank\EpAj[t]\)共同决定,
因此,除去若尔当块的排列次序外,\(\vb{A}\)的若尔当标准型是唯一的.
\end{proof}
\end{theorem}

\begin{corollary}
%@see: 《高等代数(第三版 下册)》(丘维声) P155 推论2
设\(A\)是域\(F\)上的\(n\)阶矩阵.
如果\(A\)的最小多项式\(m(\lambda)\)在\(F[\lambda]\)中能分解成一次因式的乘积:\begin{equation*}
%@see: 《高等代数(第三版 下册)》(丘维声) P155 (19)
	m(\lambda) = \LambdaExpL{1} \LambdaExpL{2} \dotsm \LambdaExpL{s},
\end{equation*}
则\(A\)相似于一个若尔当形矩阵,
其主对角线上的元素是\(A\)的全部特征值;
主对角元为\(\lambda_j\)的若尔当块总数为\begin{equation*}
%@see: 《高等代数(第三版 下册)》(丘维声) P155 (20)
	N_j = n - \rank\EigenPoly{A}{j},
	\quad j=1,2,\dotsc,s,
\end{equation*}
且其中每个若尔当块的阶数不超过\(l_j\),
\(t\)阶若尔当块的个数为\begin{equation*}
%@see: 《高等代数(第三版 下册)》(丘维声) P155 (21)
	N_j(t) = \rank\EigenPoly{A}{j}[t+1]
		+ \rank\EigenPoly{A}{j}[t-1]
		- 2\rank\EigenPoly{A}{j}[t];
\end{equation*}
\rm
这个若尔当形矩阵称为\(A\)的若尔当标准型.
除去若尔当块的排列次序外,\(A\)的若尔当标准型是唯一的.
\begin{proof}
由\cref{theorem:线性变换的结构.线性变换的若尔当标准型} 立即可得.
\end{proof}
\end{corollary}

\begin{proposition}
%@see: 《高等代数(第三版 下册)》(丘维声) P155
复数域上任意一个有限维线性空间上的每一个线性变换都有若尔当标准型.
\begin{proof}
我们知道,
复数域上每个次数大于0的一元多项式
都可以分解成一次因式的乘积,
那么由\cref{theorem:线性变换的结构.线性变换的若尔当标准型} 立即可得.
\end{proof}
\end{proposition}

\begin{proposition}
%@see: 《高等代数(第三版 下册)》(丘维声) P155
复数域上每一个方阵都有若尔当标准型.
\end{proposition}

\begin{corollary}\label{theorem:线性变换的结构.线性变换有若尔当标准型的充分必要条件1}
%@see: 《高等代数(第三版 下册)》(丘维声) P155 推论3
域\(F\)上\(n\)维线性空间\(V\)上的线性变换\(\vb{A}\)有若尔当标准型,
当且仅当\(\vb{A}\)的最小多项式\(m(\lambda)\)在\(F[\lambda]\)中可以分解成一次因式的乘积.
%TODO proof
\end{corollary}

在\cref{theorem:最小多项式.线性变换的特征多项式与最小多项式有相同根} 中,我们证明了:
域\(F\)上有限维线性空间\(V\)上的线性变换\(\vb{A}\)的
最小多项式\(m(\lambda)\)与特征多项式\(f(\lambda)\)
在\(F\)中有相同的根(重数可以不同).
利用类似的方法可以证明:
\begin{proposition}
%@see: 《高等代数(第三版 下册)》(丘维声) P156
域\(F\)上有限维线性空间\(V\)上的线性变换\(\vb{A}\)的
最小多项式\(m(\lambda)\)与特征多项式\(f(\lambda)\)
在\(F\)的扩域\(E\)中有相同的根(重数可以不同).
%TODO proof
%TODO 证明过程在《高等代数(第三版 下册)》(丘维声)参考文献[18]的第9章第6节的命题4、推论3
\end{proposition}
根据上述命题,\(\vb{A}\)的最小多项式\(m(\lambda)\)在\(F[\lambda]\)中可以分解成一次因式的乘积,
当且仅当\(\vb{A}\)的特征多项式\(f(\lambda)\)在\(F[\lambda]\)中能分解成一次因式的乘积.
于是从\cref{theorem:线性变换的结构.线性变换有若尔当标准型的充分必要条件1} 立即得出:
\begin{corollary}\label{theorem:线性变换的结构.线性变换有若尔当标准型的充分必要条件2}
%@see: 《高等代数(第三版 下册)》(丘维声) P156 推论4
域\(F\)上\(n\)维线性空间\(V\)上的线性变换\(\vb{A}\)有若尔当标准型,
当且仅当\(\vb{A}\)的特征多项式\(f(\lambda)\)在\(F[\lambda]\)中能分解成一次因式的乘积.
\end{corollary}

用矩阵的语言叙述\cref{theorem:线性变换的结构.线性变换有若尔当标准型的充分必要条件1,theorem:线性变换的结构.线性变换有若尔当标准型的充分必要条件2} 就是:
\begin{corollary}
%@see: 《高等代数(第三版 下册)》(丘维声) P156 推论5
域\(F\)上\(n\)阶矩阵\(A\)相似于一个若尔当形矩阵,
当且仅当\(A\)的最小多项式\(m(\lambda)\)或特征多项式\(f(\lambda)\)
在\(F[\lambda]\)中能分解成一次因式的乘积.
\end{corollary}

至此,我们完全解决了域\(F\)上\(n\)维线性空间\(V\)上的线性变换
在什么条件下能够有若尔当形矩阵这样的最简单形式的矩阵表示的问题.

\begin{example}
%@see: 《高等代数(第三版 下册)》(丘维声) P156 例1
有理数域上的矩阵\begin{equation*}
	A = \begin{bmatrix}
		2 & 3 & 2 \\
		1 & 8 & 2 \\
		-2 & -14 & -3
	\end{bmatrix}
\end{equation*}是否有若尔当标准型?
如果有,求出它的若尔当标准型.
\begin{solution}
\(A\)的特征多项式\(f(\lambda)\)是\begin{equation*}
	\abs{\lambda I-A}
	= \begin{vmatrix}
		\lambda-2 & -3 & -2 \\
		-1 & \lambda-8 & -2 \\
		2 & 14 & \lambda+3
	\end{vmatrix}
	= (\lambda-1)(\lambda-3)^2,
\end{equation*}
于是\(A\)有若尔当标准型.
\(A\)的全部特征值是\(1,3(\text{二重})\).
\end{solution}

对于特征值\(\lambda_1=1\),
它是\(f(\lambda)\)的1重根,
因此它在\(A\)的若尔当标准型的主对角线上只出现一次.

对于特征值\(\lambda_2=3\),
有\begin{equation*}
	A-3I
	= \begin{bmatrix}
		-1 & 3 & 2 \\
		1 & 5 & 2 \\
		-2 & -14 & -6
	\end{bmatrix}
	\to \begin{bmatrix}
		-1 & 3 & 2 \\
		0 & 8 & 4 \\
		0 & 0 & 0
	\end{bmatrix},
\end{equation*}
因此\(\rank(A-3I)=2\),
那么主对角元为\(3\)的若尔当块的总数为\begin{equation*}
	N(\lambda_2) = 3 - 2 = 1.
\end{equation*}

综上所述,\(A\)的若尔当标准型为\begin{equation*}
	\begin{bmatrix}
		1 & 0 & 0 \\
		0 & 3 & 1 \\
		0 & 0 & 3
	\end{bmatrix}.
\end{equation*}
\end{example}

假设域\(F\)上的\(n\)阶矩阵\(A\)有若尔当标准型,
由于除了若尔当块的排列次序外,\(A\)的若尔当标准型是唯一的,
因此我们可以引进下述概念:
\begin{definition}
%@see: 《高等代数(第三版 下册)》(丘维声) P157 定义1
设域\(F\)上的\(n\)阶矩阵\(A\)有若尔当标准型\(J\).
把\(J\)中所有若尔当块的最小多项式组成的汇集
称为“\(A\)的\DefineConcept{初等因子组}”,
简称为“\(A\)的\DefineConcept{初等因子}”.
\end{definition}

\begin{example}
%@see: 《高等代数(第三版 下册)》(丘维声) P157
矩阵\begin{equation*}
	A = \begin{bmatrix}
		3 & 1 & 0 & 0 \\
		0 & 3 & 0 & 0 \\
		0 & 0 & 5 & 0 \\
		0 & 0 & 0 & 5
	\end{bmatrix}
\end{equation*}
的初等因子是\(\{(\lambda-3)^2,(\lambda-5),(\lambda-5)\}\).
\end{example}

\begin{proposition}
%@see: 《高等代数(第三版 下册)》(丘维声) P157 命题6
域\(F\)上两个若尔当块相同,
当且仅当它们的最小多项式相等.
%TODO proof
\end{proposition}

\begin{theorem}
%@see: 《高等代数(第三版 下册)》(丘维声) P157 定理7
两个\(n\)阶复矩阵\(A,B\)相似,
当且仅当它们的初等因子相同.
%TODO proof
\end{theorem}
从上述定理可以看出:
\begin{proposition}
%@see: 《高等代数(第三版 下册)》(丘维声) P157
在\(M_n(\mathbb{C})\)中,
初等因子是相似不变量.
\end{proposition}

我们还可以把{\(\lambda\) - 矩阵}\(\lambda I-A\)通过初等变换化为对角矩阵,
来求复矩阵\(A\)的初等因子.
%TODO 《高等代数(第三版 下册)》(丘维声)参考文献[18]的第9章第8节的第342~346页,第7章第2节的例9

\begin{definition}
%@see: 《高等代数(第三版 下册)》(丘维声) P157 定义2
设\(V\)是域\(F\)上的一个线性空间,
\(\vb{A}\)是\(V\)上的一个线性变换,
\(\mathcal{V}\)是\(V\)的一个基.
如果\(\vb{A}\)在基\(\mathcal{V}\)下的矩阵是若尔当形矩阵,
则称“\(\mathcal{V}\)是\(\vb{A}\)的一个\DefineConcept{若尔当基}”.
\end{definition}

%@see: 《高等代数(第三版 下册)》(丘维声) P157
在我们已经求出\(\vb{A}\)的若尔当标准型\(J\)以后,
为了求出\(\vb{A}\)的一个若尔当基,
只要把从原来的基到若尔当基的过渡矩阵\(S\)求出即可.
由于\(J = S^{-1} A S\),
%@see: 《高等代数(第三版 下册)》(丘维声) P158 (22)
因此\(S\)是矩阵方程\(AX=XJ\)的解,
并且\(S\)是可逆矩阵.
如果\(\dim V = n\),
则\(AX=XJ\)是\(n^2\)个未知量\(x_{ij}\ (i,j=1,2,\dotsc,n)\)的
由\(n^2\)个方程组成的线性方程组,
解这个线性方程组,可以求出\(X\),
从中选出可逆矩阵(因为\(\vb{A}\)的若尔当标准型存在,所以满足方程\(AX=XJ\)的可逆矩阵一定存在),
便可作为过渡矩阵\(S\).

\endgroup

\chapter{内积空间}
迄今为止,我们对于线性空间和线性映射的研究
主要是围绕线性空间的加法、纯量乘法两种运算来进行的.
我们曾经对实数域上的\(n\)维向量空间\(\mathbb{R}^n\)定义了一个内积,
从而在\(\mathbb{R}^n\)中引进了长度、正交等度量概念.
受此启发,我们在本章讨论如何在实数域、复数域以至于任意域上的线性空间中引进度量概念,
然后研究具有度量的线性空间的结构,
并且研究与度量有关的线性变换的性质.

\section{双线性函数}
\subsection{双线性函数}
我们已经知道,\(\mathbb{R}^n\)上一个内积
\((\alpha,\beta) \mapsto \VectorInnerProductDot{\alpha}{\beta}\)是\(\mathbb{R}^n\)上的一个二元实值函数,
并且它具有对称性、线性性、正定性,
不难得到\begin{gather*}
	\VectorInnerProductDot{(k_1\alpha_1+k_2\alpha_2)}{\beta}
	= k_1(\VectorInnerProductDot{\alpha_1}{\beta})
	+ k_2(\VectorInnerProductDot{\alpha_2}{\beta}), \\
	\VectorInnerProductDot{\alpha}{(k_1\beta_1+k_2\beta_2)}
	= k_1(\VectorInnerProductDot{\alpha}{\beta_1})
	+ k_2(\VectorInnerProductDot{\alpha}{\beta_2}).
\end{gather*}
受此启发,我们抽象出域\(F\)上线性空间\(V\)上的双线性函数的概念:
\begin{definition}\label{definition:双线性函数.双线性函数的定义1}
%@see: 《高等代数(第三版 下册)》(丘维声) P166 定义1
设\(V\)是域\(F\)上的一个线性空间.
如果映射\(f\colon V \times V \to F\)满足\begin{gather}
	f(k_1\alpha_1+k_2\alpha_2,\beta)
	= k_1 f(\alpha_1,\beta)
	+ k_2 f(\alpha_2,\beta),
		\label{equation:双线性函数.双线性函数判定条件1} \\
	f(\alpha,k_1\beta_1+k_2\beta_2)
	= k_1 f(\alpha,\beta_1)
	+ k_2 f(\alpha,\beta_2),
		\label{equation:双线性函数.双线性函数判定条件2}
\end{gather}
则称“\(f\)是\(V\)上的一个\DefineConcept{双线性函数}%
(\(f\) is a \emph{bilinear function} on \(V\))”.
%@see: https://mathworld.wolfram.com/BilinearFunction.html
\end{definition}
%@see: 《高等代数(第三版 下册)》(丘维声) P166
条件 \labelcref{equation:双线性函数.双线性函数判定条件1} 表明:
当\(\beta\)固定时,
双线性函数\(f\)在\(V \times \{\beta\}\)上的限制\(f \SetRestrict (V \times \{\beta\})\),
是\(V\)上的一个线性函数.
% 在《高等代数(第三版 下册)》(丘维声)上,将这个映射的限制记为\(\beta_R\).
条件 \labelcref{equation:双线性函数.双线性函数判定条件2} 表明:
当\(\alpha\)固定时,
双线性函数\(f\)在\(\{\alpha\} \times V\)上的限制\(f \SetRestrict (\{\alpha\} \times V)\),
是\(V\)上的一个线性函数.
% 在《高等代数(第三版 下册)》(丘维声)上,将这个映射的限制记为\(\alpha_L\).
于是我们可以得到双线性函数的等价定义:
\begin{theorem}\label{definition:双线性函数.双线性函数的定义2}
%@see: 《Linear Algebra Done Right (Fourth Edition)》(Sheldon Axler) P333 9.1
设\(V\)是域\(F\)上的一个线性空间,
则“映射\(f\colon V \times V \to F\)是\(V\)上的一个双线性函数”的充分必要条件是:
对于任意\(\alpha \in V\),
\(x \mapsto f(x,\alpha)\)和\(x \mapsto f(\alpha,x)\)都是\(V\)上的线性函数.
\begin{proof}
必要性.
假设\(f\)是\(V\)上的一个双线性函数.
任意取定\(\alpha \in V\),
显然映射\(x \mapsto f(x,\alpha)\)和\(x \mapsto f(\alpha,x)\)都是从\(V\)到\(F\)的映射,
并且由\hyperref[definition:双线性函数.双线性函数的定义1]{定义}可知,
对于\(\forall x_1,x_2 \in V,
\forall k_1,k_2 \in F\),
有\begin{gather*}
	f(k_1x_1+k_2x_2,\alpha)
	= k_1 f(x_1,\alpha)
	+ k_2 f(x_2,\alpha), \\
	f(\alpha,k_1x_1+k_2x_2)
	= k_1 f(\alpha,x_1)
	+ k_2 f(\alpha,x_2),
\end{gather*}
可见\(x \mapsto f(x,\alpha)\)和\(x \mapsto f(\alpha,x)\)都适合可加性、齐次性,
因此\(x \mapsto f(x,\alpha)\)和\(x \mapsto f(\alpha,x)\)
都是\(V\)上的\hyperref[definition:线性映射.线性函数]{线性函数}.

充分性.
假设对于任意\(\alpha \in V\),
\(x \mapsto f(x,\alpha)\)和\(x \mapsto f(\alpha,x)\)都是\(V\)上的线性函数.
任意取定\(\beta \in V\),
记\(g(x) \defeq f(x,\beta)\),
那么由\hyperref[definition:线性映射.线性映射]{定义}可知,
对于\(\forall \alpha_1,\alpha_2 \in V,
\forall k_1,k_2 \in F\),
有\begin{equation*}
	g(\alpha_1+\alpha_2)=g(\alpha_1)+g(\alpha_2),
	\qquad
	g(k_1\alpha_1)=k_1g(\alpha_1),
	\qquad
	g(k_2\alpha_2)=k_2g(\alpha_2),
\end{equation*}
从而有\begin{equation*}
	g(k_1\alpha_1+k_2\alpha_2)
	= g(k_1\alpha_1) + g(k_2\alpha_2)
	= k_1g(\alpha_1) + k_2g(\alpha_2),
\end{equation*}
即\begin{equation*}
	f(k_1\alpha_1+k_2\alpha_2,\beta)
	= k_1 f(\alpha_1,\beta)
	+ k_2 f(\alpha_2,\beta).
\end{equation*}
同理可证,在任意取定\(\alpha \in V\)以后,
对于\(\forall \beta_1,\beta_2 \in V,
\forall k_1,k_2 \in F\),
有\begin{equation*}
	f(\alpha,k_1\beta_1+k_2\beta_2)
	= k_1 f(\alpha,\beta_1)
	+ k_2 f(\alpha,\beta_2).
\end{equation*}
因此\(f\)是\(V\)上的一个双线性函数.
\end{proof}
\end{theorem}

\begin{example}\label{example:双线性函数.例1}
%@see: 《高等代数(第三版 下册)》(丘维声) P166 例1
设\(V = M_n(F)\).
令\(f(A,B) \defeq \tr(AB)\ (A,B \in V)\),
则\(f\)是\(V\)上的一个双线性函数.
\end{example}

\begin{example}\label{example:双线性函数.例2}
%@see: 《高等代数(第三版 下册)》(丘维声) P167 例2
设\(V = C[a,b]\).
令\(f(g,h) \defeq \int_a^b g(x) h(x) \dd{x}\ (g,h \in V)\),
则\(f\)是\(V\)上的一个双线性函数.
\end{example}

\begin{example}\label{example:双线性函数.例3}
%@see: 《高等代数(第三版 下册)》(丘维声) P167 例3
设\(V = F^n\).
映射\(f\colon V \times V \to F,
(\alpha,\beta) \mapsto \sum_{i=1}^n a_i b_i\)是\(V\)上的一个双线性函数,
其中\(\alpha=(\AutoTuple{a}{n})^T,
\beta=(\AutoTuple{b}{n})^T\).
\end{example}

\begin{property}\label{theorem:双线性函数.双线性函数取值为零的条件1}
设\(V\)是域\(F\)上的一个线性空间,
\(f\)是\(V\)上的一个双线性函数,
则\(f(0,0) = 0\).
\begin{proof}
在 \hyperref[equation:双线性函数.双线性函数判定条件1]{$
	f(k_1\alpha_1+k_2\alpha_2,\beta)
	= k_1 f(\alpha_1,\beta)
	+ k_2 f(\alpha_2,\beta)
$} 中,
取\(k_1=k_2=1,
\alpha_1=\alpha_2=\beta=0\),
得\begin{equation*}
	f(0,0) = f(0,0) + f(0,0),
\end{equation*}
整理得\(f(0,0) = 0\).
\end{proof}
\end{property}

\begin{property}\label{theorem:双线性函数.双线性函数取值为零的条件2}
设\(V\)是域\(F\)上的一个线性空间,
\(f\)是\(V\)上的一个双线性函数,
则\begin{equation*}
	f(\alpha,0)
	= f(0,\beta)
	= 0.
\end{equation*}
\begin{proof}
%@credit: {de3029b8-10a6-4ae5-8f64-108dae1c10a9},{a84f055e-f32d-418a-8d8c-0b72a4b2df78},{6d916ea6-1441-4c5b-9e97-4f834b319710},{83c38fc7-afe0-41ba-a56b-684984744681}
在 \hyperref[equation:双线性函数.双线性函数判定条件2]{$
	f(\alpha,k_1\beta_1+k_2\beta_2)
	= k_1 f(\alpha,\beta_1)
	+ k_2 f(\alpha,\beta_2)
$} 中,
取\(k_1=k_2=1,
\beta_1=\beta_2=0\),
得\begin{equation*}
	f(\alpha,0)
	= f(\alpha,0+0)
	= f(\alpha,0) + f(\alpha,0),
\end{equation*}
整理得\(f(\alpha,0) = 0\).
同理可得\(f(0,\beta) = 0\).
\end{proof}
\end{property}
\begin{remark}
我们还可以为\cref{theorem:双线性函数.双线性函数取值为零的条件2} 给出另一种证明:
由\hyperref[definition:双线性函数.双线性函数的定义2]{双线性函数的等价定义}可知,
在固定任意一个自变量的情况下,
映射\(x \mapsto f(\alpha,x)\)和\(x \mapsto f(x,\beta)\)都是\(V\)上的线性函数,
那么由\hyperref[theorem:线性映射.线性映射的性质]{线性映射的性质}可知
这两个线性函数在\(x=0\)的值都是\(0\).
\end{remark}

\subsection{双线性函数的度量矩阵}
\begin{definition}
%@see: 《高等代数(第三版 下册)》(丘维声) P167
%@see: 《Linear Algebra Done Right (Fourth Edition)》(Sheldon Axler) P334 9.4
设\(V\)是域\(F\)上\(n\)维线性空间,
\(f\)是\(V\)上的一个双线性函数,
在\(V\)中取一个基\(\AutoTuple{\epsilon}{n}\),
向量\(\alpha,\beta\)在基\(\AutoTuple{\epsilon}{n}\)下的坐标
分别是\begin{equation}\label{equation:双线性函数.双线性函数的自变量的坐标}
	X=(\AutoTuple{x}{n})^T,
	\qquad
	Y=(\AutoTuple{y}{n})^T,
\end{equation}
把\begin{equation}\label{equation:双线性函数.双线性函数的因变量的坐标}
%@see: 《高等代数(第三版 下册)》(丘维声) P167 (1)
	f(\alpha,\beta)
	= f\left( \sum_{i=1}^n x_i \epsilon_i, \sum_{j=1}^n y_j \epsilon_j \right)
	= \sum_{i=1}^n \sum_{j=1}^n x_i y_j f(\epsilon_i,\epsilon_j)
\end{equation}
中\(x_i y_j\)的系数\(f(\epsilon_i,\epsilon_j)\)作为矩阵的\((i,j)\)元素,
得到\begin{equation}\label{equation:双线性函数.双线性函数的度量矩阵}
%@see: 《高等代数(第三版 下册)》(丘维声) P167 (2)
	A \defeq \begin{bmatrix}
		f(\epsilon_1,\epsilon_1) & f(\epsilon_1,\epsilon_2) & \dots & f(\epsilon_1,\epsilon_n) \\
		f(\epsilon_2,\epsilon_1) & f(\epsilon_2,\epsilon_2) & \dots & f(\epsilon_2,\epsilon_n) \\
		\vdots & \vdots & & \vdots \\
		f(\epsilon_n,\epsilon_1) & f(\epsilon_n,\epsilon_2) & \dots & f(\epsilon_n,\epsilon_n) \\
	\end{bmatrix},
\end{equation}
将\(A\)称为“双线性函数\(f\)在基\(\AutoTuple{\epsilon}{n}\)下的\DefineConcept{度量矩阵}”,
记为\(\BilinearMatrix(f,(\AutoTuple{\epsilon}{n}))\),
或在不致混淆的情况下简记为\(\BilinearMatrix(f)\).
\end{definition}
\begin{remark}
%@see: 《高等代数(第三版 下册)》(丘维声) P167
双线性函数\(f\)的度量矩阵
由\(f\)以及基\(\AutoTuple{\epsilon}{n}\)唯一确定.
\end{remark}
\begin{remark}
由\cref{equation:双线性函数.双线性函数的自变量的坐标,equation:双线性函数.双线性函数的因变量的坐标,equation:双线性函数.双线性函数的度量矩阵}
可得\begin{equation}\label{equation:双线性函数.双线性函数的矩阵表示}
%@see: 《高等代数(第三版 下册)》(丘维声) P167 (3)
	f(\alpha,\beta)
	= X^T A Y.
\end{equation}
\cref{equation:双线性函数.双线性函数的矩阵表示,equation:双线性函数.双线性函数的因变量的坐标}
都是双线性函数\(f\)在基\(\AutoTuple{\epsilon}{n}\)下的表达式.
\end{remark}
\begin{theorem}
%@see: 《Linear Algebra Done Right (Fourth Edition)》(Sheldon Axler) P335 9.6
设\(V\)是域\(F\)上的\(n\)维线性空间,
\(f\)是\(V\)上的一个双线性函数,
\(A\)是\(V\)上的一个线性变换,
在\(V\)中取一个基\(\AutoTuple{\epsilon}{n}\),
令\begin{equation*}
	g(u,v) \defeq f(u,Av),
	\qquad
	h(u,v) \defeq f(Au,v),
\end{equation*}
则\begin{equation*}
	\BilinearMatrix(g)
	= \BilinearMatrix(f) \LinearMapMatrix(A),
	\qquad
	\BilinearMatrix(h)
	= (\LinearMapMatrix(A))^T \BilinearMatrix(f).
\end{equation*}
%TODO proof
\end{theorem}

\begin{theorem}\label{theorem:双线性函数.双线性函数在两个基下的度量矩阵合同}
%@see: 《高等代数(第三版 下册)》(丘维声) P167 定理1
%@see: 《Linear Algebra Done Right (Fourth Edition)》(Sheldon Axler) P336 9.7
设\(f\)是域\(F\)上\(n\)维线性空间\(V\)上的一个双线性函数,
在\(V\)中取两个基\(\AutoTuple{\alpha}{n}\)与\(\AutoTuple{\beta}{n}\),
矩阵\(P \in M_n(F)\)满足\begin{equation*}
%@see: 《高等代数(第三版 下册)》(丘维声) P167 (4)
	(\AutoTuple{\beta}{n})
	= (\AutoTuple{\alpha}{n}) P,
\end{equation*}
\(f\)在这两个基下的度量矩阵分别是\(A,B\),
则\begin{equation*}
%@see: 《高等代数(第三版 下册)》(丘维声) P167 (5)
	B = P^T A P.
\end{equation*}
\begin{proof}
设\begin{equation*}
	\alpha = (\AutoTuple{\alpha}{n}) X
	= (\AutoTuple{\beta}{n}) X_0,
	\qquad
	\beta = (\AutoTuple{\alpha}{n}) Y
	= (\AutoTuple{\beta}{n}) Y_0,
\end{equation*}
则\(X = P X_0,
Y = P Y_0\),
从而\begin{equation*}
	f(\alpha,\beta)
	= X^T A Y
	= (P X_0)^T A (P Y_0)
	= X_0^T (P^T A P) Y_0.
\end{equation*}
再由\(f(\alpha,\beta) = X_0^T B Y_0\)
可得\begin{equation*}
	X_0^T B Y_0 = X_0^T (P^T A P) Y_0,
\end{equation*}
根据\(X_0,Y_0\)的任意性,
可知\(B = P^T A P\).
\end{proof}
\end{theorem}
\begin{remark}
\cref{theorem:双线性函数.双线性函数在两个基下的度量矩阵合同} 表明,
\(V\)上的双线性函数\(f\)在不同基下的度量矩阵是合同的.
由于合同矩阵有相同的秩,
因此我们把双线性函数\(f\)在\(V\)某个基下的度量矩阵的秩由\(f\)唯一确定.
\end{remark}
\begin{definition}
%@see: 《高等代数(第三版 下册)》(丘维声) P168
设\(f\)是域\(F\)上\(n\)维线性空间\(V\)上的一个双线性函数,
\(A\)是\(f\)在\(V\)中某个基下的度量矩阵.
把\(\rank A\)
称为“双线性函数\(f\)的\DefineConcept{矩阵秩}”,
记为\(\rank f\).
\end{definition}

\subsection{双线性函数空间}
%@see: 《Linear Algebra Done Right (Fourth Edition)》(Sheldon Axler) P334
由\cref{theorem:线性空间.线性空间的笛卡尔和是线性空间}
可知\(V \times V\)是一个线性空间,
这让我们想要知道:双线性函数是不是从\(V \times V\)到\(F\)的线性映射.
答案是否定的.
%@see: 《Linear Algebra Done Right (Fourth Edition)》(Sheldon Axler) P344 Exercises 9A 3.
实际上,如果\(V\)上的双线性函数\(f\)是\(V \times V\)上的一个线性函数,
那么\(f\)一定是零映射.
不过,虽然除了零映射以外的其余双线性函数都不是线性映射,
但是所有\(V\)上的双线性函数组成的集合\begin{equation}
%@see: 《Linear Algebra Done Right (Fourth Edition)》(Sheldon Axler) P334 9.3
	V^{(2)}
	\defeq
	\Set{
		f
		\given
		\text{$f$是$V$上的双线性函数}
	}
\end{equation}
在配上通常的映射加法、纯量乘法运算以后
还是可以成为域\(F\)上的一个线性空间.

\begin{proposition}
%@see: 《Linear Algebra Done Right (Fourth Edition)》(Sheldon Axler) P335 9.4
设\(V\)是域\(F\)上的\(n\)维线性空间,
则\(\dim V^{(2)} = (\dim V)^2\).
%TODO proof
\end{proposition}

\subsection{非退化双线性函数}
\begin{definition}
%@see: 《高等代数(大学高等代数课程创新教材 第二版 下册)》(丘维声) P446 例6
设\(f\)是域\(F\)上线性空间\(V\)上的一个双线性函数,
\(f\)是对称的或斜对称的.
如果\(V\)中的两个向量\(\alpha,\beta\)满足\(f(\alpha,\beta) = 0\),
则称“\(\alpha\)与\(\beta\) \DefineConcept{正交}”,
记为\(\alpha \perp \beta\).
\end{definition}

\begin{definition}
%@see: 《高等代数(第三版 下册)》(丘维声) P168 定义2
%@see: 《高等代数》(丁南庆、刘公祥、纪庆忠、郭学军) P351 定义9.1.6
%@see: 《高等代数(大学高等代数课程创新教材 第二版 下册)》(丘维声) P446 例6
设\(f\)是域\(F\)上\(n\)维线性空间\(V\)上的一个双线性函数,
\(W\)是\(V\)的一个子空间.
\begin{itemize}
	\item 定义:\begin{equation}
		\Rad_L(f,W)
		\defeq
		\Set{
			\alpha \in W
			\given
			(\forall\beta \in W)
			[f(\alpha,\beta) = 0]
		},
	\end{equation}
	称之为“\(f\)在\(W\)中的\DefineConcept{左根}(left radical)”
	或“\(f \SetRestrict W\)的\DefineConcept{左根}”.

	在不致混淆的情况下,可以将\(\Rad_L(f,W)\)简记为\(\Rad_L f\)或\(\Rad_L W\).

	\item 定义:\begin{equation}
		\Rad_R(f,W)
		\defeq
		\Set{
			\beta \in W
			\given
			(\forall\alpha \in W)
			[f(\alpha,\beta) = 0]
		},
	\end{equation}
	称之为“\(f\)在\(W\)中的\DefineConcept{右根}(right radical)”
	或“\(f \SetRestrict W\)的\DefineConcept{右根}”.

	在不致混淆的情况下,可以将\(\Rad_R(f,W)\)简记为\(\Rad_R f\)或\(\Rad_R W\).

	\item 如果\(\Rad_L(f,W) = \Rad_R(f,W)\),
	定义:\begin{equation*}
		\Rad(f,W)
		\defeq
		\Rad_L(f,W),
	\end{equation*}
	称之为“\(f\)在\(W\)中的\DefineConcept{根}(radical)”
	或“\(f \SetRestrict W\)的\DefineConcept{根}”.

	在不致混淆的情况下,可以将\(\Rad(f,W)\)简记为\(\Rad f\)或\(\Rad W\).
\end{itemize}
\end{definition}

\begin{proposition}
%@see: 《高等代数(第三版 下册)》(丘维声) P168
设\(f\)是域\(F\)上\(n\)维线性空间\(V\)上的一个双线性函数,
则\(f\)的左根\(\Rad_L f\)和右根\(\Rad_R f\)都是\(V\)的子空间.
%TODO proof
\end{proposition}

\begin{example}
%@see: 《高等代数(第三版 下册)》(丘维声) P173 习题10.1 2.
证明:如果\(V\)是有限维线性空间,则\(V\)上的双线性函数\(f\)满足\begin{equation*}
	\dim\Rad_L f
	= \dim\Rad_R f
	= \dim V - \rank f.
\end{equation*}
%TODO proof
\end{example}

\begin{example}
%@see: 《高等代数(第三版 下册)》(丘维声) P173 习题10.1 3.
\def\fL{\alpha_L}  % 双线性函数\(f\)的限制
\def\fR{\beta_R}  % 双线性函数\(f\)的限制
\def\Lf{L_f}  % 从\(V\)到\(V^*\)的线性映射
\def\Rf{R_f}  % 从\(V\)到\(V^*\)的线性映射
设\(V\)是域\(F\)上\(n\)维线性空间,
\(V^*\)是\(V\)的对偶空间,
\(f\)是\(V\)上的一个双线性函数,
记\(\alpha_L \defeq f \SetRestrict (\{\alpha\} \times V),
\beta_R \defeq f \SetRestrict (V \times \{\beta\})\),
映射\(\Lf\colon \alpha \mapsto \fL\),
映射\(\Rf\colon \beta \mapsto \fR\).
证明:\begin{itemize}
	\item \(\Lf\)和\(\Rf\)都是从\(V\)到\(V^*\)的线性映射;
	\item \(\Ker \Lf = \Rad_L V\);
	\item \(\Ker \Rf = \Rad_R V\);
	\item \(\rank \Lf = \rank \Rf = \rank f\);
	\item \(f\)是非退化的,当且仅当\(\Lf\)或\(\Rf\)是从\(V\)到\(V^*\)的同构.
\end{itemize}
%TODO proof
\end{example}

\begin{theorem}
%@see: 《高等代数(第三版 下册)》(丘维声) P168 定理2
%@see: 《高等代数》(丁南庆、刘公祥、纪庆忠、郭学军) P351 定理9.1.7
设\(f\)是域\(F\)上\(n\)维线性空间\(V\)上的一个双线性函数,
在\(V\)中取一个基\(\AutoTuple{\epsilon}{n}\),
\(A\)是\(f\)在基\(\AutoTuple{\epsilon}{n}\)下的度量矩阵,
则\begin{equation*}
	\text{$A$是满秩矩阵}
	\iff
	\Rad_L f = 0
	\iff
	\Rad_R f = 0.
\end{equation*}
%TODO proof
\end{theorem}

\begin{definition}
%@see: 《高等代数(第三版 下册)》(丘维声) P168 定义3
%@see: 《高等代数》(丁南庆、刘公祥、纪庆忠、郭学军) P351 定理9.1.7
设\(f\)是域\(F\)上\(n\)维线性空间\(V\)上的一个双线性函数.
如果\(\Rad_L f = \Rad_R f = 0\),
则称“\(f\)是\DefineConcept{非退化的}(nondegenerate)”;
否则称“\(f\)是\DefineConcept{退化的}(degenerate)”.
\end{definition}

\begin{example}
%@see: 《高等代数(第三版 下册)》(丘维声) P173 习题10.1 4.
设\(V = M_n(F)\),
令\(f(A,B) \defeq \tr(AB)\ (A,B \in V)\).
证明:\(f\)是非退化的.
%TODO proof
%\cref{example:双线性函数.例1}
\end{example}

\begin{definition}
设\(f\)是域\(F\)上\(n\)维线性空间\(V\)上的一个双线性函数,
\(U = V-\{0\}\).
\begin{itemize}
	\item 如果\(
		(\forall \alpha \in U)
		[f(\alpha,\alpha) > 0]
	\),
	则称“\(f\)是\DefineConcept{正定的}(positive definite)”.

	\item 如果\(
		(\forall \alpha \in U)
		[f(\alpha,\alpha) \geq 0]
	\),
	则称“\(f\)是\DefineConcept{半正定的}(positive semi-definite)”.

	\item 如果\(
		(\forall \alpha \in U)
		[f(\alpha,\alpha) < 0]
	\),
	则称“\(f\)是\DefineConcept{负定的}(negative definite)”.

	\item 如果\(
		(\forall \alpha \in U)
		[f(\alpha,\alpha) \leq 0]
	\),
	则称“\(f\)是\DefineConcept{半负定的}(negative semi-definite)”.

	\item 否则,称“\(f\)是\DefineConcept{不定的}(indefinite)”.
\end{itemize}
%\cref{definition:实二次型的分类.实二次型的分类}
\end{definition}

\subsection{对称双线性函数,斜对称双线性函数}
\begin{definition}
%@see: 《高等代数(第三版 下册)》(丘维声) P169 定义4
%@see: 《Linear Algebra Done Right (Fourth Edition)》(Sheldon Axler) P337 9.9
设\(f\)是域\(F\)上\(n\)维线性空间\(V\)上的一个双线性函数.
\begin{itemize}
	\item 如果\begin{equation}
	%@see: 《高等代数(第三版 下册)》(丘维声) P169 (8)
		(\forall \alpha,\beta \in V)
		[f(\alpha,\beta) = f(\beta,\alpha)],
	\end{equation}
	则称“\(f\)是\DefineConcept{对称的}”.
	\item 如果\begin{equation}
	%@see: 《高等代数(第三版 下册)》(丘维声) P169 (9)
		(\forall \alpha,\beta \in V)
		[f(\alpha,\beta) = -f(\beta,\alpha)],
	\end{equation}
	则称“\(f\)是\DefineConcept{斜对称的}”
	或“\(f\)是\DefineConcept{反对称的}”.
\end{itemize}
\end{definition}

\cref{example:双线性函数.例1,example:双线性函数.例2,example:双线性函数.例3} 中的双线性函数也都是对称的.

\begin{example}
%@see: 《高等代数(第三版 下册)》(丘维声) P169 例5
设\(V = \mathbb{R}^2\).
映射\(f\colon V \times V \to \mathbb{R},
\left(
	\begin{bmatrix}
		x_1 \\
		x_2
	\end{bmatrix},
	\begin{bmatrix}
		y_1 \\
		y_2
	\end{bmatrix}
\right)
\mapsto
\begin{vmatrix}
	x_1 & y_1 \\
	x_2 & y_2
\end{vmatrix}\)是斜对称双线性函数.
\end{example}

\begin{proposition}
%@see: 《高等代数(第三版 下册)》(丘维声) P169
设\(f\)是域\(F\)上\(n\)维线性空间\(V\)上的一个双线性函数,
则\begin{equation*}
	\text{$f$是对称的}
	\iff
	\text{$f$的度量矩阵是对称的}.
\end{equation*}
%TODO proof
\end{proposition}

\begin{proposition}
%@see: 《高等代数(第三版 下册)》(丘维声) P169
设\(f\)是域\(F\)上\(n\)维线性空间\(V\)上的一个双线性函数,
则\begin{equation*}
	\text{$f$是斜对称的}
	\iff
	\text{$f$的度量矩阵是斜对称的}.
\end{equation*}
%TODO proof
\end{proposition}

\begin{example}
%@see: 《高等代数(第三版 下册)》(丘维声) P173 习题10.1 6.
设\(f\)是域\(F\ (\FieldChar F\neq2)\)上线性空间\(V\)的双线性函数.
证明:\(f\)是斜对称的,当且仅当
对于任意\(\alpha \in V\),
都有\(f(\alpha,\alpha) = 0\).
%TODO proof
\end{example}

\begin{definition}\label{definition:双线性函数.利用双线性函数构造的正交补}
%@see: 《高等代数(第三版 下册)》(丘维声) P173 习题10.1 7.
设\(V\)是域\(F\)上的线性空间,
\(f\)是\(V\)上的双线性函数,
\(f\)是对称的或斜对称的,
\(W\)是\(V\)的一个子空间.
定义:\begin{equation*}
	W^\perp
	\defeq
	\Set{
		\alpha \in V
		\given
		(\forall \beta \in W)
		[f(\alpha,\beta) = 0]
	},
\end{equation*}
称之为“\(W\)的\DefineConcept{正交补}”.
\end{definition}

\begin{proposition}
%@see: 《高等代数(第三版 下册)》(丘维声) P173 习题10.1 7.
%@see: 《高等代数(大学高等代数课程创新教材 第二版 下册)》(丘维声) P448 例7(1)
设\(V\)是域\(F\)上的线性空间,
\(f\)是\(V\)上的双线性函数,
\(f\)是对称的或斜对称的,
\(W\)是\(V\)的一个子空间,
\(W^\perp\)是\(W\)的正交补,
则\(W^\perp\)是\(V\)的一个子空间.
\begin{proof}
由于\(f(0,\beta) = 0 f(0,\beta) = 0\),
所以\(0 \in W^\perp\).
显然\(W^\perp\)对于\(V\)的加法与纯量乘法都封闭.
于是,由\cref{theorem:线性空间.子空间的判定2} 可知
\(W^\perp\)是\(V\)的一个子空间.
\end{proof}
\end{proposition}

\begin{proposition}
%@see: 《高等代数(大学高等代数课程创新教材 第二版 下册)》(丘维声) P448 例7(2)
设\(V\)是域\(F\)上的线性空间,
\(f\)是\(V\)上的双线性函数,
\(f\)是对称的或斜对称的,
\(W\)是\(V\)的一个子空间,
\(W^\perp\)是\(W\)的正交补,
则\begin{gather*}
	\Rad W \subseteq W^\perp, \\
	\Rad W = W \cap W^\perp.
\end{gather*}
\begin{proof}
由\(\Rad W\)与\(W^\perp\)的定义立即可得.
\end{proof}
\end{proposition}

\begin{proposition}\label{theorem:双线性函数.利用双线性函数构造的正交补.正交补空间的维数}
%@see: 《高等代数(第三版 下册)》(丘维声) P173 习题10.1 8.(1)
%@see: 《高等代数(大学高等代数课程创新教材 第二版 下册)》(丘维声) P448 例8(1)
设\(V\)是域\(F\)上的\(n\)维线性空间,
\(f\)是\(V\)上的非退化的双线性函数,
\(f\)是对称的或斜对称的,
\(W\)是\(V\)的一个子空间,
则\(\dim W + \dim W^\perp = \dim V\).
\begin{proof}
设\(f\)是\(V\)上的非退化的对称双线性函数.
在\(W\)中取一个基\(\AutoTuple{\alpha}{m}\),
把它扩充成\(V\)的一个基\(\AutoTuple{\alpha}{m},\AutoTuple{\alpha}[m+1]{n}\).
设\(f\)在此基下的度量矩阵为\(A\).
由于\(f\)是非退化的,
所以\(A\)是满秩的.
对于\(V\)中向量\(\alpha = (\AutoTuple{a}{n}) x\),
有\begin{align*}
	\alpha \in W^\perp
	&\iff (\forall \beta \in W)[f(\alpha,\beta) = 0] \\
	&\iff f(\alpha,\alpha_1)
			= f(\alpha,\alpha_2)
			= \dotsb
			= f(\alpha,\alpha_m)
			= 0 \\
	&\iff x^T A \epsilon_1
			= x^T A \epsilon_2
			= \dotsb
			= x^T A \epsilon_m
			= 0 \\
	&\iff x^T A (\AutoTuple{\epsilon}{m}) = 0 \\
	&\iff (\AutoTuple{\epsilon}{m})^T A^T x = 0 \\
	&\iff \text{$x$是关于$z$的齐次线性方程组$(\AutoTuple{\epsilon}{m})^T A^T z = 0$的解}.
\end{align*}
把上述齐次线性方程组的解空间记作\(U\),
则\(\alpha \in W^\perp\),
当且仅当\(\alpha\)在上述基下的坐标\(x \in U\).
由于\(\alpha \mapsto x\)是\(V\)到\(F^n\)的一个同构映射,
且\(W^\perp\)在此同构映射下的象是\(U\),
因此\(
	\dim W^\perp
	= \dim U
	= n - \rank((\AutoTuple{\epsilon}{m})^T A^T)
	= n - \rank(\AutoTuple{\epsilon}{m})^T
	= n - m
	= \dim V - \dim W
\),
从而\(\dim W + \dim W^\perp = \dim V\).
\end{proof}
\end{proposition}

\begin{proposition}\label{theorem:双线性函数.利用双线性函数构造的正交补.正交补的对合律}
%@see: 《高等代数(第三版 下册)》(丘维声) P173 习题10.1 8.(2)
%@see: 《高等代数(大学高等代数课程创新教材 第二版 下册)》(丘维声) P448 例8(2)
设\(V\)是域\(F\)上的\(n\)维线性空间,
\(f\)是\(V\)上的非退化的双线性函数,
\(f\)是对称的或斜对称的,
\(W\)是\(V\)的一个子空间,
则\((W^\perp)^\perp = W\).
\begin{proof}
任取\(\gamma \in W\),
对于任意\(\delta \in W^\perp\),
有\(f(\delta,\gamma) = 0\).
由于\(f\)要么是对称的,要么是斜对称的,
因此\(f(\gamma,\delta) = 0\),
从而\(\gamma \in (W^\perp)^\perp\),
于是\(W \subseteq (W^\perp)^\perp\).
对于\(W^\perp\),
由\cref{theorem:双线性函数.利用双线性函数构造的正交补.正交补空间的维数} 可知,
\(\dim W^\perp + \dim(W^\perp)^\perp = \dim V\).
与\(\dim W + \dim W^\perp = \dim V\)比较可得\(\dim W = \dim(W^\perp)^\perp\).
因此\(W = (W^\perp)^\perp\).
\end{proof}
\end{proposition}

\subsection{对称双线性函数的度量矩阵}
\begin{theorem}\label{theorem:双线性函数.数域上的对称双线性函数在某个基下的度量矩阵是对角矩阵}
%@see: 《高等代数(第三版 下册)》(丘维声) P169 定理3
设\(f\)是数域\(K\)上\(n\)维线性空间\(V\)上的一个对称双线性函数,
则\(V\)中存在一个基\(\AutoTuple{\epsilon}{n}\),
使得\(f\)在基\(\AutoTuple{\epsilon}{n}\)下的度量矩阵是对角矩阵.
%TODO proof
\end{theorem}

\begin{theorem}\label{theorem:双线性函数.特征不为2的域上的对称双线性函数在某个基下的度量矩阵是对角矩阵}
%@see: 《高等代数(第三版 下册)》(丘维声) P170 定理4
%@see: 《高等代数(大学高等代数课程创新教材 第二版 下册)》(丘维声) P429 定理3
设\(f\)是域\(F\ (\FieldChar F\neq2)\)上\(n\)维线性空间\(V\)上的一个对称双线性函数,
则\(V\)中存在一个基\(\AutoTuple{\epsilon}{n}\),
使得\(f\)在基\(\AutoTuple{\epsilon}{n}\)下的度量矩阵是对角矩阵.
%TODO proof
\end{theorem}

\begin{proposition}
%@see: 《高等代数(第三版 下册)》(丘维声) P171
设\(f\)是域\(F\ (\FieldChar F\neq2)\)上\(n\)维线性空间\(V\)上的一个非退化的对称双线性函数,
则\(V\)中存在一个基\(\AutoTuple{\eta}{n}\),
使得\begin{align*}
	f(\eta_i,\eta_j)
	&= 0,
	\quad i \neq j; i,j=1,2,\dotsc,n; \\
	f(\eta_i,\eta_i)
	&\neq 0,
	\quad i=1,2,\dotsc,n.
\end{align*}
\end{proposition}

\cref{theorem:双线性函数.特征不为2的域上的对称双线性函数在某个基下的度量矩阵是对角矩阵} 的意义之一
在于它可以用来简化计算对称双线性函数在任意一对向量上的函数值.
由于\(V\)中一定存在一个基\(\AutoTuple{\eta}{n}\),
使得\(f\)在基\(\AutoTuple{\eta}{n}\)下的度量矩阵是对称矩阵\(\diag(\AutoTuple{d}{n})\),
从而对于\(\forall\alpha=x_1\eta_1+\dotsb+x_n\eta_n\)
和\(\forall\beta=y_1\eta_1+\dotsb+y_n\eta_n\),
有\begin{equation*}
%@see: 《高等代数(第三版 下册)》(丘维声) P171 (13)
	f(\alpha,\beta)
	= d_1 x_1 y_1 + \dotsb + d_n x_n y_n.
\end{equation*}

\subsection{斜对称双线性函数的度量矩阵}
%@see: 《高等代数(第三版 下册)》(丘维声) P171
下面来讨论斜对称双线性函数.
设\(f\)是域\(F\)上\(n\)维线性空间\(V\)上的一个斜对称双线性函数.

当\(\FieldChar F=2\)时,
对\(\forall \alpha,\beta \in V\),
由\hyperref[definition:域的特征.域的特征]{域的特征的定义}有\(2 f(\alpha,\beta) = 0,
f(\alpha,\beta) = -f(\alpha,\beta)\).
由于\(f\)是斜对称的,
所以\(f(\alpha,\beta)
= -f(\beta,\alpha)
= f(\beta,\alpha)\),
这就说明\(f\)是对称的.
因此,在特征为2的域上的线性空间中,\begin{equation*}
	\text{$f$是斜对称双线性函数}
	\iff
	\text{$f$是对称双线性函数}.
\end{equation*}

当\(\FieldChar F\neq2\)时,
对\(\forall \alpha,\beta \in V\),
由\(f(\alpha,\alpha) = -f(\alpha,\alpha),
2 f(\alpha,\alpha) = 0,
f(\alpha,\alpha) = 0\).

\begin{theorem}\label{theorem:双线性函数.特征不为2的域上的斜对称双线性函数在某个基下的度量矩阵是分块对角矩阵}
%@see: 《高等代数(第三版 下册)》(丘维声) P171 定理5
%@see: 《高等代数(大学高等代数课程创新教材 第二版 下册)》(丘维声) P430 定理4
\def\MatrixChunk{\begin{bmatrix}
	0 & 1 \\
	-1 & 0
\end{bmatrix}}
设\(f\)是域\(F\ (\FieldChar F\neq2)\)上的\(n\)维线性空间\(V\)上的斜对称双线性函数,
则\(V\)中存在一个基\(\delta_1,\delta_{-1},\delta_2,\delta_{-2},\dotsc,\delta_r,\delta_{-r},\eta_1,\eta_2,\dotsc,\eta_s\),
使得\begin{align*}
%@see: 《高等代数(第三版 下册)》(丘维声) P171 (15)
	f(\delta_i,\delta_{-i}) &= 1,
	\quad i=1,2,\dotsc,r; \\
	f(\delta_i,\delta_j) &= 0,
	\quad i+j\neq0; \\
	f(\alpha,\eta_k) &= 0,
	\quad \alpha \in V; k=1,2,\dotsc,s;
\end{align*}
且\(f\)在这个基下的度量矩阵是\begin{equation*}
%@see: 《高等代数(第三版 下册)》(丘维声) P171 (14)
	\diag\left(
		\MatrixChunk,
		\dotsc,
		\MatrixChunk,
		0,
		\dotsc,
		0
	\right).
\end{equation*}
%TODO proof
\end{theorem}

\begin{proposition}
%@see: 《高等代数(第三版 下册)》(丘维声) P172
\def\MatrixChunk{\begin{bmatrix}
	0 & 1 \\
	-1 & 0
\end{bmatrix}}
如果域\(F\ (\FieldChar F\neq2)\)上的\(n\)维线性空间\(V\)上的斜对称双线性函数\(f\)是非退化的,
则\(V\)中存在一个基\(\AutoTuple{\delta}{n}\),
使得\(f\)在基\(\AutoTuple{\delta}{n}\)下的度量矩阵是分块对角矩阵\begin{equation*}
	%@see: 《高等代数(第三版 下册)》(丘维声) P173 (20)
		\diag\left(
			\MatrixChunk,
			\dotsc,
			\MatrixChunk
		\right).
	\end{equation*}
\begin{proof}
由\cref{theorem:双线性函数.特征不为2的域上的斜对称双线性函数在某个基下的度量矩阵是分块对角矩阵} 可得.
\end{proof}
\end{proposition}

\begin{example}
%@see: 《高等代数(大学高等代数课程创新教材 第二版 下册)》(丘维声) P446 例6
设\(f\)是域\(F\ (\FieldChar F\neq2)\)上线性空间\(V\)上的一个对称或斜对称的双线性函数,
\(W\)是\(V\)的一个有限维真子空间,
向量\(\xi \in V-W\)与\(\Rad W\)中的每一个向量都正交.
证明:在\(W\)的陪集\(\xi+W\)中,
存在\(\eta\neq0\),
使得对于任意\(\beta \in W\),
成立\(f(\eta,\beta) = 0\).
\begin{proof}
设\(\dim W = m\).
\begin{enumerate}
	\item 假设\(f\)是对称的,
	则\(f\)在\(W\)上的限制\(f \SetRestrict W\)是\(W\)上的一个对称双线性函数.

	如果\(f \SetRestrict W = 0\),
	则\(\Rad W = W\),
	于是对于任意\(\beta \in W\),
	有\(f(\xi,\beta) = 0\),
	从而对于\(\xi+W\)中任意一个向量\(\xi+\gamma\)(其中\(\gamma \in W\))有\begin{equation*}
		f(\xi+\gamma,\beta)
		= f(\xi,\beta)
		+ f(\gamma,\beta)
		= 0.
	\end{equation*}

	如果\(f \SetRestrict W \neq 0\),
	那么由\cref{theorem:双线性函数.特征不为2的域上的对称双线性函数在某个基下的度量矩阵是对角矩阵} 可知,
	\(W\)中存在一个基\(\AutoTuple{\alpha}{m}\),
	使得\(f \SetRestrict W\)在此基下的度量矩阵为\begin{equation*}
		D = \diag(d_1,d_2,\dotsc,d_r,0,\dotsc,0),
	\end{equation*}
	其中\(d_i\neq0\ (i=1,2,\dotsc,r)\)且\(1 \leq r \leq m\).
	令\begin{equation*}
		\eta
		\defeq
		\xi - \sum_{i=1}^r \frac{f(\xi,\alpha_i)}{f(\alpha_i,\alpha_i)} \alpha_i,
	\end{equation*}
	则当\(1 \leq j \leq r\)时,
	有\begin{align*}
		f(\eta,\alpha_j)
		&= f(\xi,\alpha_j) - \sum_{i=1}^r \frac{f(\xi,\alpha_i)}{f(\alpha_i,\alpha_i)} f(\alpha_i,\alpha_j) \\
		&= f(\xi,\alpha_j) - f(\xi,\alpha_j)
		= 0;
	\end{align*}
	当\(r < j \leq m\)时,
	由于\(f(\alpha_i,\alpha_j) = 0\ (i=1,2,\dotsc,m)\),
	因此\(\alpha_j \in \Rad W\),
	从而有\begin{equation*}
		f(\eta,\alpha_j)
		= f(\xi,\alpha_j)
		= 0.
	\end{equation*}
	因此,对于\(W\)中任意一个向量\(\beta = \sum_{i=1}^m b_i \alpha_i\),
	有\begin{equation*}
		f(\eta,\beta)
		= f\left( \eta,\sum_{i=1}^m b_i \alpha_i \right)
		= \sum_{i=1}^m b_i f(\eta,\alpha_i)
		= 0.
	\end{equation*}

	\item 假设\(f\)是斜对称的.

	如果\(f \SetRestrict W = 0\),
	则与“\(f\)是对称双线性函数”这个情形一样,命题依然成立.

	如果\(f \SetRestrict W \neq 0\),
	那么由\cref{theorem:双线性函数.特征不为2的域上的斜对称双线性函数在某个基下的度量矩阵是分块对角矩阵} 可知,
	\(W\)中存在一个基\(
		\delta_1, \allowbreak
		\delta_{-1}, \allowbreak
		\delta_2, \allowbreak
		\delta_{-2}, \allowbreak
		\dotsc, \allowbreak
		\delta_r, \allowbreak
		\delta_{-r}, \allowbreak
		\eta_1, \allowbreak
		\eta_2, \allowbreak
		\dotsc, \allowbreak
		\eta_{m-2r}
	\),
	使得\(f\)在这个基下的度量矩阵是\begin{equation*}
		\def\MatrixChunk{\begin{bmatrix}
			0 & 1 \\
			-1 & 0
		\end{bmatrix}}
		\diag\left(
			\MatrixChunk,
			\dotsc,
			\MatrixChunk,
			0,
			\dotsc,
			0
		\right),
	\end{equation*}
	其中\(2 \leq 2r \leq m\).
	令\begin{equation*}
		\eta
		\defeq
		\xi + \sum_{i=1}^r (
			-f(\xi,\delta_{-i}) \delta_i
			+f(\xi,\delta_i) \delta_{-i}
		),
	\end{equation*}
	则当\(1 \leq j \leq r\)时,
	有\begin{align*}
		f(\eta,\delta_j)
		&= f(\xi,\delta_j)
		+ \sum_{i=1}^r (
			-f(\xi,\delta_{-i}) f(\delta_i,\delta_j)
			+f(\xi,\delta_i) f(\delta_{-i},\delta_j)
		) \\
		&= f(\xi,\delta_j)
		+ f(\xi,\delta_j) (-1)
		= 0, \\
		f(\eta,\delta_{-j})
		&= f(\xi,\delta_{-j})
		+ \sum_{i=1}^r (
			-f(\xi,\delta_{-i}) f(\delta_i,\delta_{-j})
			+f(\xi,\delta_i) f(\delta_{-i},\delta_{-j})
		) \\
		&= f(\xi,\delta_{-j})
		+ f(\xi,\delta_{-j}) (-1)
		= 0.
	\end{align*}
	当\(1 \leq s \leq m-2r\)时,
	由于\((\forall \beta \in W)[f(\eta_s,\beta) = 0]\),
	因此\(\eta_s \in \Rad W\),
	从而\(f(\xi,\eta_s) = 0\),
	于是\begin{equation*}
		f(\eta,\eta_s)
		= f(\xi,\eta_s)
		+ \sum_{i=1}^r (
			-f(\xi,\delta_{-i}) f(\delta_i,\eta_s)
			+f(\xi,\delta_i) f(\delta_{-i},\eta_s)
		)
		= 0.
	\end{equation*}
	综上所述,有\((\forall \beta \in W)[f(\eta,\beta) = 0]\).
	\qedhere
\end{enumerate}
\end{proof}
\end{example}

\section{欧几里得空间}
\subsection{内积}
\begin{definition}
%@see: 《高等代数(第三版 下册)》(丘维声) P174 定义1
设\(V\)是实数域\(\mathbb{R}\)上的一个线性空间.
如果\(f\)是\(V\)上的一个正定对称双线性函数,
则称\(f\)是“\(V\)上的一个\DefineConcept{内积}”.
\end{definition}

\begin{proposition}
%@see: 《高等代数(第三版 下册)》(丘维声) P174 命题1
设\(V\)是\(\mathbb{R}\)上的一个\(n\)维线性空间,
\(f\)是\(V\)上的一个双线性函数,
则\(f\)是正定对称的,
当且仅当\(f\)在\(V\)的某个基下的度量矩阵是正定对称的.
%TODO proof
\end{proposition}

\begin{example}
%@see: 《高等代数(第三版 下册)》(丘维声) P174 例1
在\(V = \mathbb{R}^3\)中,
令\(f(\alpha,\beta) \defeq a_1 b_1 + 2 a_2 b_2 + 3 a_3 b_3\),
其中\(\alpha=(\AutoTuple{a}{3})^T,
\beta=(\AutoTuple{b}{3})^T\).
容易验证,\(f\)是\(V\)上的一个内积.
\end{example}

\begin{example}
%@see: 《高等代数(第三版 下册)》(丘维声) P174 例2
在\(V = M_n(\mathbb{R})\)中,
令\(f(A,B) \defeq \tr(AB^T)\).
容易验证,\(f\)是\(V\)上的一个内积.
\end{example}

\begin{example}
%@see: 《高等代数(第三版 下册)》(丘维声) P174 例3
在\(V = C[a,b]\)中,
令\(f(f,g) \defeq \int_a^b f(x) g(x) \dd{x}\).
容易验证,\(f\)是\(V\)上的一个内积.
\end{example}

\subsection{实内积空间,欧几里得空间}
\begin{definition}
%@see: 《高等代数(第三版 下册)》(丘维声) P174 定义2
设\(V\)是一个实线性空间,
\(\rho\)是\(V\)上的一个内积,
则称“\((V,\rho)\)是一个\DefineConcept{实内积空间}”.
\end{definition}

\begin{definition}
%@see: 《高等代数(第三版 下册)》(丘维声) P174 定义2
设\(V\)是一个实内积空间.
如果\(V\)是有限维的,
则称“\(V\)是一个\DefineConcept{欧几里得空间}”;
把线性空间\(V\)的维数称为“欧几里得空间\(V\)的\DefineConcept{维数}”.
\end{definition}

\begin{definition}
%@see: 《高等代数(第三版 下册)》(丘维声) P174 定义3
设\((V,\rho)\)是一个实内积空间,
\(\alpha \in V\).
把非负实数\(\sqrt{\rho(\alpha,\alpha)}\)
称为“向量\(\alpha\)的\DefineConcept{长度}”,
记作\(\VectorLengthA{\alpha}\)或\(\VectorLengthN{\alpha}\).
\end{definition}

\begin{property}
%@see: 《高等代数(第三版 下册)》(丘维声) P175
在实内积空间\(V\)中,
零向量的长度为\(0\),
非零向量的长度是正数.
\end{property}

\begin{property}
%@see: 《高等代数(第三版 下册)》(丘维声) P175
在实内积空间\((V,\rho)\)中,
对于\(\forall \alpha \in V\)
和\(\forall k \in \mathbb{R}\),
有\(\VectorLengthA{k\alpha} = \abs{k} \VectorLengthA{\alpha}\).
\begin{proof}
\(\VectorLengthA{k\alpha}
= \sqrt{\rho(k\alpha,k\alpha)}
= \sqrt{k^2\rho(\alpha,\alpha)}
= \abs{k} \VectorLengthA{\alpha}\).
\end{proof}
\end{property}

\begin{definition}
%@see: 《高等代数(第三版 下册)》(丘维声) P175
设\((V,\rho)\)是实内积空间,
\(\alpha \in V\).
如果\(\VectorLengthA{\alpha} = 1\),
则称“\(\alpha\)是一个\DefineConcept{单位向量}”.
\end{definition}

\begin{property}
%@see: 《高等代数(第三版 下册)》(丘维声) P175
设\((V,\rho)\)是实内积空间,
\(\alpha \in V\).
如果\(\alpha\neq0\),
则\(\frac1{\VectorLengthA{\alpha}} \alpha\)是一个单位向量.
\end{property}

\begin{theorem}
%@see: 《高等代数(第三版 下册)》(丘维声) P175 定理2(柯西-布尼亚科夫斯基不等式)
在实内积空间\((V,\rho)\)中,
对于\(\forall \alpha,\beta \in V\),
有\begin{equation}
	\abs{\rho(\alpha,\beta)} \leq \VectorLengthA{\alpha} \VectorLengthA{\beta}.
\end{equation}
当且仅当\(\{\alpha,\beta\}\)线性相关时,上式取“\(=\)”号.
%TODO proof
\end{theorem}

\begin{definition}
%@see: 《高等代数(第三版 下册)》(丘维声) P175 定义4
在实内积空间\((V,\rho)\)中,
\(\alpha,\beta\)是两个非零向量.
把\begin{equation}
%@see: 《高等代数(第三版 下册)》(丘维声) P175 (6)
	\arccos\frac{\rho(\alpha,\beta)}{\VectorLengthA{\alpha} \VectorLengthA{\beta}}
\end{equation}
称为“\(\alpha\)与\(\beta\)的\DefineConcept{夹角}”,
记为\(\VectorAngleA{\alpha}{\beta}\)或\(\VectorAngleP{\alpha}{\beta}\).
\end{definition}

\begin{property}
%@see: 《高等代数(第三版 下册)》(丘维声) P175
设\(\alpha,\beta\)是实内积空间\((V,\rho)\)中的两个非零向量,
则\(\alpha\)与\(\beta\)的夹角\(\theta = \VectorAngleA{\alpha}{\beta}\)
满足\(0 \leq \theta \leq \pi\).
%TODO proof
\end{property}

\begin{property}
%@see: 《高等代数(第三版 下册)》(丘维声) P175
设\(\alpha,\beta\)是实内积空间\((V,\rho)\)中的两个非零向量,
则\(\alpha\)与\(\beta\)的夹角\(\theta = \VectorAngleA{\alpha}{\beta}\)
满足\(\theta = \frac\pi2 \iff \rho(\alpha,\beta) = 0\).
%TODO proof
\end{property}

\begin{definition}
%@see: 《高等代数(第三版 下册)》(丘维声) P175 定义5
设\(\alpha,\beta\)是实内积空间\((V,\rho)\)中的两个非零向量.
如果\(\rho(\alpha,\beta) = 0\),
则称“\(\alpha\)与\(\beta\) \DefineConcept{正交}”,
记为\(\alpha \perp \beta\).
\end{definition}

\begin{corollary}
%@see: 《高等代数(第三版 下册)》(丘维声) P175 推论3
在实内积空间\((V,\rho)\)中,
三角不等式成立,
即对于\(\forall \alpha,\beta \in V\),
有\begin{equation}
%@see: 《高等代数(第三版 下册)》(丘维声) P175 (7)
	\VectorLengthA{\alpha+\beta} \leq \VectorLengthA{\alpha} + \VectorLengthA{\beta}.
\end{equation}
%TODO 没有取等条件
%TODO proof
\end{corollary}

\begin{corollary}
%@see: 《高等代数(第三版 下册)》(丘维声) P176 推论4
在实内积空间\((V,\rho)\)中,
勾股定理成立,
即对于\(\forall \alpha,\beta \in V\),
如果\(\alpha\)与\(\beta\)正交,
则\begin{equation}
%@see: 《高等代数(第三版 下册)》(丘维声) P176 (8)
	\VectorLengthA{\alpha+\beta}^2 = \VectorLengthA{\alpha}^2 + \VectorLengthA{\beta}^2.
\end{equation}
%TODO proof
\end{corollary}

\begin{definition}
%@see: 《高等代数(第三版 下册)》(丘维声) P176 定义6
在实内积空间\((V,\rho)\)中,
\(\alpha,\beta \in V\).
把\(\VectorLengthA{\alpha-\beta}\)
称为“\(\alpha\)与\(\beta\)的\DefineConcept{距离}”,
记为\(d(\alpha,\beta)\).
\end{definition}

\begin{property}
%@see: 《高等代数(第三版 下册)》(丘维声) P176
在实内积空间\((V,\rho)\)中,
距离\(d\colon V \times V \to \mathbb{R},
(\alpha,\beta) \mapsto \VectorLengthA{\alpha-\beta}\)
满足以下性质:\begin{itemize}
	\item {\rm\bf 对称性}:\begin{equation*}
		(\forall \alpha,\beta \in V)
		[
			d(\alpha,\beta) = d(\beta,\alpha)
		];
	\end{equation*}

	\item {\rm\bf 正定性}:\begin{equation*}
		(\forall \alpha,\beta \in V)
		[
			d(\alpha,\beta) \geq 0
		];
	\end{equation*}
	当且仅当\(\alpha = \beta\)时,上式取“\(=\)”号;

	\item {\rm\bf 三角不等式}:\begin{equation*}
		(\forall \alpha,\beta,\gamma \in V)
		[
			d(\alpha,\gamma) \leq d(\alpha,\beta) + d(\beta,\gamma)
		].
	\end{equation*}
\end{itemize}
\end{property}

\section{正交补,正交投影}

\section{正交变换}
\subsection{正交变换的概念}
\begin{definition}
%@see: 《高等代数(第三版 下册)》(丘维声) P185 定义1
设\((V,\rho)\)是一个实内积空间,
\(\vb{A}\)是一个从\(V\)到\(V\)的满射.
如果\(\vb{A}\)满足\begin{equation}
	(\forall \alpha,\beta \in V)
	[
		\rho(\vb{A}\alpha,\vb{A}\beta)
		= \rho(\alpha,\beta)
	],
\end{equation}
则称“\(\vb{A}\)保持向量的内积不变”
“\(\vb{A}\)是\(V\)上的一个\DefineConcept{正交变换}”.
\end{definition}

\subsection{正交变换的性质}
\begin{property}\label{theorem:正交变换.保长性1}
%@see: 《高等代数(第三版 下册)》(丘维声) P185
设\((V,\rho)\)是一个实内积空间,
\(\vb{A}\)是\(V\)上的一个正交变换,
则\(\VectorLengthA{\vb{A}\alpha} = \VectorLengthA{\alpha}\).
\end{property}

\begin{proposition}
%@see: 《高等代数(第三版 下册)》(丘维声) P185 命题1
实内积空间上的正交变换一定是线性变换.
%TODO proof
\end{proposition}

\begin{proposition}
%@see: 《高等代数(第三版 下册)》(丘维声) P185 命题2
实内积空间上的正交变换一定是可逆的.
%TODO proof
\end{proposition}

\begin{proposition}
%@see: 《高等代数(第三版 下册)》(丘维声) P185 命题3
设\(\vb{A}\)是实内积空间\(V\)上的一个线性变换,
则\(\vb{A}\)是\(V\)上的一个正交变换,
当且仅当\(\vb{A}\)是从\(V\)上的一个自同构.
%TODO proof
\end{proposition}

\begin{proposition}
%@see: 《高等代数(第三版 下册)》(丘维声) P186
设\(\vb{A}\)是实内积空间\(V\)上的一个正交变换,
则\(\vb{A}\)的逆\(\vb{A}^{-1}\)也是\(V\)上的一个正交变换.
%TODO proof
\end{proposition}

\begin{proposition}
%@see: 《高等代数(第三版 下册)》(丘维声) P186
设\(\vb{A},\vb{B}\)都是实内积空间\(V\)上的正交变换,
则\(\vb{A}\vb{B}\)也是\(V\)上的一个正交变换.
\end{proposition}

\begin{property}\label{theorem:正交变换.保角性1}
%@see: 《高等代数(第三版 下册)》(丘维声) P186
设\((V,\rho)\)是一个实内积空间,
\(\vb{A}\)是\(V\)上的一个正交变换,
则\(\VectorAngleA{\alpha}{\beta} = \VectorAngleA{\vb{A}\alpha}{\vb{A}\beta}\).
\end{property}

\begin{property}\label{theorem:正交变换.保角性2}
%@see: 《高等代数(第三版 下册)》(丘维声) P186
设\((V,\rho)\)是一个实内积空间,
\(\vb{A}\)是\(V\)上的一个正交变换,
则\(\alpha\perp\beta \implies (\vb{A}\alpha)\perp(\vb{A}\beta)\).
\end{property}

\begin{property}\label{theorem:正交变换.保长性2}
%@see: 《高等代数(第三版 下册)》(丘维声) P186
设\((V,\rho)\)是一个实内积空间,
\(\vb{A}\)是\(V\)上的一个正交变换,
则\(d(\alpha,\beta) = d(\vb{A}\alpha,\vb{A}\beta)\).
\end{property}

\subsection{正交变换的判定}
\begin{proposition}
%@see: 《高等代数(第三版 下册)》(丘维声) P186 命题4
设\(\vb{A}\)是\(n\)维欧几里得空间\(V\)上的一个线性变换,
则\begin{align*}
	&\text{$\vb{A}$是正交变换} \\
	&\iff \text{$\vb{A}$把$V$的标准正交基映成标准正交基} \\
	&\iff \text{$\vb{A}$在$V$的标准正交基下的矩阵是正交矩阵}.
\end{align*}
%TODO proof
\end{proposition}

\subsection{正交变换的分类}
\begin{definition}
%@see: 《高等代数(第三版 下册)》(丘维声) P186
设\(\vb{A}\)是\(n\)维欧几里得空间\(V\)上的一个正交变换,
\(A\)是\(\vb{A}\)在\(V\)的某个标准正交基下的矩阵.
\begin{itemize}
	\item 如果\(\DeterminantA{A} = 1\),
	则称\(\vb{A}\)是\DefineConcept{第一类变换}或\DefineConcept{旋转变换}.

	\item 如果\(\DeterminantA{A} = -1\),
	则称\(\vb{A}\)是\DefineConcept{第一类变换}.
\end{itemize}
\end{definition}

\section{对称变换}

\section{酉空间}
%@see: 《高等代数(第三版 下册)》(丘维声) P193
在本节,我们研究在复线性空间中引进内积的概念,使之成为复内积空间.

\subsection{酉空间}
如何在复线性空间\(V\)中引进内积的概念呢?
如果我们照搬实线性空间内积的概念,
考虑复线性空间\(V\)上一个双线性函数\(f\),
那么对于\(V\)中任意一个非零向量\(\alpha\),
有\begin{equation*}
	f(\iu\alpha,\iu\alpha)
	= \iu^2 f(\alpha,\alpha)
	= -f(\alpha,\alpha).
\end{equation*}
若要求\((\forall \alpha \in V)[f(\alpha,\alpha) \in \mathbb{R}]\),
则\(f\)不满足正定性(因为当\(f(\alpha,\alpha) > 0\)时,必有\(f(\iu\alpha,\iu\alpha) < 0\)).
为了使复线性空间\(V\)上的内积仍具有正定性,
就不能要求它是双线性函数,
而只要求它对第一个自变量是线性的.
为了使内积在\(V\)的任意一个向量\(\alpha\)与自身组成的有序对上的函数值为实数,
需要让\(V\)上的内积\(\rho\)具有如下性质:
\begin{equation}\label{equation:酉空间.厄米性}
	(\forall \alpha,\beta \in V)
	[
		\rho(\alpha,\beta)
		= \ComplexConjugate{\rho(\beta,\alpha)}
	].
\end{equation}
我们把\cref{equation:酉空间.厄米性} 描述的性质
称为\DefineConcept{厄米性}或\DefineConcept{共轭对称性}(conjugate symmetry).
于是复线性空间上的内积的概念应当定义如下:
\begin{definition}\label{definition:酉空间.复线性空间上的内积}
%@see: 《高等代数(第三版 下册)》(丘维声) P193 定义1
%@see: 《Linear Algebra Done Right (Fourth Edition)》(Sheldon Axler) P183 6.2
设\(V\)是复数域\(\mathbb{C}\)上的一个线性空间.
如果映射\(\rho\colon V \times V \to \mathbb{R}\)满足以下性质\begin{itemize}
	\item {\rm\bf 厄米性}:\begin{equation*}
		(\forall \alpha,\beta \in V)
		[
			\rho(\alpha,\beta)
			= \ComplexConjugate{\rho(\beta,\alpha)}
		];
	\end{equation*}

	\item {\rm\bf 线性性}:\begin{gather*}
		(\forall \alpha,\beta,\gamma \in V)
		[
			\rho(\alpha+\beta,\gamma)
			= \rho(\alpha,\gamma) + \rho(\beta,\gamma)
		], \\
		(\forall \alpha,\beta \in V)
		(\forall k \in \mathbb{C})
		[
			\rho(k\alpha,\beta)
			= k\rho(\alpha,\beta)
		];
	\end{gather*}

	\item {\rm\bf 正定性}:\begin{gather*}
		(\forall \alpha \in V)
		[
			\rho(\alpha,\alpha) \geq 0
		], \\
		(\forall \alpha \in V)
		[
			\rho(\alpha,\alpha) = 0
			\iff
			\alpha = 0
		],
	\end{gather*}
\end{itemize}
则称“\(\rho\)是复线性空间\(V\)上的一个\DefineConcept{内积}(inner product)”.
\end{definition}
\begin{remark}
\hyperref[definition:欧几里得空间.实线性空间上的内积]{实线性空间上的内积}%
与\hyperref[definition:酉空间.复线性空间上的内积]{复线性空间上的内积}的
最大“差别”在于前者要求内积具有对称性,后者要求内积具有厄米性.
但是,考虑到实数的共轭就是它本身,
所以复线性空间上的内积也可以用作实线性空间上的内积.
\end{remark}

\begin{definition}
%@see: 《高等代数(第三版 下册)》(丘维声) P193 定义1
%@see: 《Linear Algebra Done Right (Fourth Edition)》(Sheldon Axler) P184 6.4
设\(V\)是一个复线性空间,\(\rho\)是\(V\)上的一个内积,
则称“\((V,\rho)\)是一个\DefineConcept{复内积空间}(complex inner product space)”
或“\((V,\rho)\)是一个\DefineConcept{酉空间}(unitary space)”.
\end{definition}

\begin{property}
%@see: 《Linear Algebra Done Right (Fourth Edition)》(Sheldon Axler) P185 6.6
设\((V,\rho)\)是一个酉空间,
则对于任意\(\alpha \in V\),
映射\(x \mapsto \rho(x,\alpha)\)是\(V\)上的线性函数.
\end{property}

\begin{property}
%@see: 《Linear Algebra Done Right (Fourth Edition)》(Sheldon Axler) P185 6.6
设\((V,\rho)\)是一个酉空间,
则对于任意\(\alpha \in V\),
有\begin{equation*}
	\rho(0,\alpha) = \rho(\alpha,0) = 0.
\end{equation*}
\end{property}

\begin{property}\label{theorem:酉空间.复线性空间上内积对第二个自变量具有半线性性}
%@see: 《高等代数(第三版 下册)》(丘维声) P193
%@see: 《Linear Algebra Done Right (Fourth Edition)》(Sheldon Axler) P185 6.6
设\((V,\rho)\)是一个酉空间,
则内积\(\rho\)对第二个自变量具有\DefineConcept{半线性性},
即\begin{equation}
	\rho(\alpha,k_1\beta_1+k_2\beta_2)
	= \ComplexConjugate{k_1} \rho(\alpha,\beta_1)
	+ \ComplexConjugate{k_2} \rho(\alpha,\beta_2).
\end{equation}
\begin{proof}
由内积的厄米性和它对第一个自变量的线性性,有\begin{align*}
	\rho(\alpha,k_1\beta_1+k_2\beta_2)
	&= \ComplexConjugate{\rho(k_1\beta_1+k_2\beta_2,\alpha)} \\
	&= \ComplexConjugate{k_1 \rho(\beta_1,\alpha)}
		+ \ComplexConjugate{k_2 \rho(\beta_2,\alpha)} \\
	&= \ComplexConjugate{k_1} \rho(\alpha,\beta_1)
		+ \ComplexConjugate{k_2} \rho(\alpha,\beta_2).
	\qedhere
\end{align*}
\end{proof}
\end{property}
\begin{remark}
注意与\cref{theorem:实线性空间.实线性空间上内积对第二个自变量具有线性性} 进行对比.
\end{remark}

\begin{example}
%@see: 《高等代数(第三版 下册)》(丘维声) P193 例1
在\(V = \mathbb{C}^n\)中,
%@see: 《高等代数(第三版 下册)》(丘维声) P193 (1)
令\(f(X,Y) \defeq x_1 \ComplexConjugate{y_1} + \dotsb + x_n \ComplexConjugate{y_n}\),
其中\(X=(\AutoTuple{x}{n})^T,
Y=(\AutoTuple{y}{n})^T\).
容易验证,\(f\)是\(V\)上的一个内积.
我们把这个内积称为 \DefineConcept{\(\mathbb{C}^n\)上的标准内积}.
\end{example}

\begin{example}
%@see: 《高等代数(第三版 下册)》(丘维声) P194 例2
设\(V = \tilde{C}[a,b]\)表示区间\([a,b]\)上所有连续复值函数组成的线性空间.
%@see: 《高等代数(第三版 下册)》(丘维声) P194 (2)
令\(\rho(f,g) \defeq \int_a^b f(x) \ComplexConjugate{g(x)} \dd{x}\).
容易验证,\(\rho\)是\(V\)上的一个内积.
\end{example}

\begin{example}
%@see: 《高等代数(第三版 下册)》(丘维声) P194 例3
在\(V = M_n(\mathbb{C})\)中,
%@see: 《高等代数(第三版 下册)》(丘维声) P194 (3)
令\(f(A,B) \defeq \tr(A B^H)\).
容易验证,\(f\)是\(V\)上的一个内积.
\end{example}

\begin{definition}
%@see: 《高等代数(第三版 下册)》(丘维声) P194 定义2
%@see: 《Linear Algebra Done Right (Fourth Edition)》(Sheldon Axler) P186 6.7
设\((V,\rho)\)是一个酉空间,
\(\alpha \in V\).
把非负实数\(\sqrt{\rho(\alpha,\alpha)}\)
称为“向量\(\alpha\)的\DefineConcept{长度}”,
记作\(\VectorLengthA{\alpha}\)或\(\VectorLengthN{\alpha}\).
\end{definition}

\begin{property}\label{theorem:酉空间.向量的长度具有非负性}
%@see: 《高等代数(第三版 下册)》(丘维声) P194
%@see: 《Linear Algebra Done Right (Fourth Edition)》(Sheldon Axler) P186 6.9
在酉空间\(V\)中,
零向量的长度为\(0\),
非零向量的长度是正数.
\end{property}

\begin{property}\label{theorem:酉空间.向量的长度具有齐次性}
%@see: 《高等代数(第三版 下册)》(丘维声) P194
%@see: 《Linear Algebra Done Right (Fourth Edition)》(Sheldon Axler) P186 6.9
在酉空间\((V,\rho)\)中,
对于\(\forall \alpha \in V\)
和\(\forall k \in \mathbb{C}\),
有\(\VectorLengthA{k\alpha} = \ComplexLengthA{k} \VectorLengthA{\alpha}\).
\begin{proof}
%@see: 《高等代数(第三版 下册)》(丘维声) P194 (4)
\(\VectorLengthA{k\alpha}
= \sqrt{\rho(k\alpha,k\alpha)}
= \sqrt{k \ComplexConjugate{k} \rho(\alpha,\alpha)}
= \ComplexLengthA{k} \VectorLengthA{\alpha}\).
\end{proof}
\end{property}

\begin{theorem}
%@see: 《高等代数(第三版 下册)》(丘维声) P194 定理1(柯西-布尼亚科夫斯基不等式)
%@see: 《Linear Algebra Done Right (Fourth Edition)》(Sheldon Axler) P189 6.14
在酉空间\((V,\rho)\)中,
对于\(\forall \alpha,\beta \in V\),
有\begin{equation}
	\abs{\rho(\alpha,\beta)} \leq \VectorLengthA{\alpha} \VectorLengthA{\beta}.
\end{equation}
当且仅当\(\{\alpha,\beta\}\)线性相关时,上式取“\(=\)”号.
%TODO proof
\end{theorem}

\begin{definition}
%@see: 《高等代数(第三版 下册)》(丘维声) P194 定义3
设\(\alpha,\beta\)是酉空间\((V,\rho)\)中的两个非零向量.
把\begin{equation}
%@see: 《高等代数(第三版 下册)》(丘维声) P194 (7)
	\arccos\frac{\abs{\rho(\alpha,\beta)}}{\VectorLengthA{\alpha} \VectorLengthA{\beta}}
\end{equation}
称为“\(\alpha\)与\(\beta\)的\DefineConcept{夹角}”,
记为\(\VectorAngleA{\alpha}{\beta}\)或\(\VectorAngleP{\alpha}{\beta}\).
\end{definition}

\begin{property}
%@see: 《高等代数(第三版 下册)》(丘维声) P194
设\(\alpha,\beta\)是酉空间\((V,\rho)\)中的两个非零向量,
则\(\alpha\)与\(\beta\)的夹角\(\theta = \VectorAngleA{\alpha}{\beta}\)
满足\(0 \leq \theta \leq \pi/2\).
%TODO proof
\end{property}

\begin{property}
%@see: 《高等代数(第三版 下册)》(丘维声) P195
设\(\alpha,\beta\)是酉空间\((V,\rho)\)中的两个非零向量,
则\(\alpha\)与\(\beta\)的夹角\(\theta = \VectorAngleA{\alpha}{\beta}\)
满足\(\theta = \frac\pi2 \iff \rho(\alpha,\beta) = 0\).
%TODO proof
\end{property}

\begin{definition}
%@see: 《高等代数(第三版 下册)》(丘维声) P195 定义4
%@see: 《Linear Algebra Done Right (Fourth Edition)》(Sheldon Axler) P187 6.10
设\(\alpha,\beta\)是酉空间\((V,\rho)\)中的两个非零向量.
如果\(\rho(\alpha,\beta) = 0\),
则称“\(\alpha\)与\(\beta\) \DefineConcept{正交}(orthogonal)”,
记为\(\alpha \perp \beta\).
\end{definition}

\begin{property}
%@see: 《Linear Algebra Done Right (Fourth Edition)》(Sheldon Axler) P187 6.11
在酉空间\((V,\rho)\)中,零向量与任意一个向量正交.
\end{property}

\begin{property}\label{theorem:酉空间.酉空间中不存在非零迷向向量}
%@see: 《Linear Algebra Done Right (Fourth Edition)》(Sheldon Axler) P187 6.11
在酉空间\((V,\rho)\)中,只有零向量与它本身正交.
\end{property}
\begin{remark}
\cref{theorem:酉空间.酉空间中不存在非零迷向向量} 说明:酉空间中不存在非零迷向向量.
\end{remark}

\begin{corollary}\label{theorem:酉空间.三角不等式}
%@see: 《高等代数(第三版 下册)》(丘维声) P195
%@see: 《Linear Algebra Done Right (Fourth Edition)》(Sheldon Axler) P190 6.17
在酉空间\((V,\rho)\)中,
\DefineConcept{三角不等式}(triangle inequality)成立,
即对于\(\forall \alpha,\beta \in V\),
有\begin{equation}
	\VectorLengthA{\alpha+\beta} \leq \VectorLengthA{\alpha} + \VectorLengthA{\beta}.
\end{equation}
当且仅当\(\alpha = k\beta\)或\(\beta = k\alpha\)(其中\(k\geq0\))时,上式取“\(=\)”号.
%TODO proof
\end{corollary}

\begin{corollary}\label{theorem:酉空间.勾股定理}
%@see: 《高等代数(第三版 下册)》(丘维声) P195
%@see: 《Linear Algebra Done Right (Fourth Edition)》(Sheldon Axler) P187 6.12
在酉空间\((V,\rho)\)中,
\DefineConcept{勾股定理}成立,
即对于\(\forall \alpha,\beta \in V\),
如果\(\alpha\)与\(\beta\)正交,
则\begin{equation}
	\VectorLengthA{\alpha+\beta}^2 = \VectorLengthA{\alpha}^2 + \VectorLengthA{\beta}^2.
\end{equation}
\begin{proof}
证明过程与\cref{theorem:实内积空间.勾股定理} 相同.
\end{proof}
\end{corollary}

\begin{proposition}\label{theorem:酉空间.平行四边形等式}
%@see: 《Linear Algebra Done Right (Fourth Edition)》(Sheldon Axler) P191 6.21
在酉空间\((V,\rho)\)中,
\(\alpha,\beta \in V\),
则\begin{equation*}
	\VectorLengthA{\alpha+\beta}^2
	+ \VectorLengthA{\alpha-\beta}^2
	= 2(\VectorLengthA{\alpha}^2+\VectorLengthA{\beta}^2).
\end{equation*}
\begin{proof}
直接有\begin{align*}
	\VectorLengthA{\alpha+\beta}^2
	+ \VectorLengthA{\alpha-\beta}^2
	&= \rho(\alpha+\beta,\alpha+\beta) + \rho(\alpha-\beta,\alpha-\beta) \\
	&= (\VectorLengthA{\alpha}^2+\VectorLengthA{\beta}^2+\rho(\alpha,\beta)+\rho(\beta,\alpha)) \\
	&\hspace{20pt}+(\VectorLengthA{\alpha}^2+\VectorLengthA{\beta}^2-\rho(\alpha,\beta)-\rho(\beta,\alpha)) \\
	&= 2(\VectorLengthA{\alpha}^2+\VectorLengthA{\beta}^2).
	\qedhere
\end{align*}
\end{proof}
\end{proposition}

\begin{proposition}\label{theorem:酉空间.向量的正交分解}
%@see: 《Linear Algebra Done Right (Fourth Edition)》(Sheldon Axler) P188 6.13
在酉空间\((V,\rho)\)中,
\(\alpha,\beta \in V\),
且\(\beta \neq 0\).
令\begin{equation*}
	k \defeq \frac{\rho(\alpha,\beta)}{\rho(\beta,\beta)},
	\qquad
	\gamma \defeq \alpha - k \beta,
\end{equation*}
则\(\alpha = k\beta + \gamma\)且\(\rho(\beta,\gamma) = 0\).
\end{proposition}
\begin{remark}
将一个向量\(\alpha\)化为两个正交向量之和\(k\beta + \gamma\)的过程,
称为\DefineConcept{正交分解}(orthogonal decomposition).
\end{remark}

\begin{definition}
%@see: 《高等代数(第三版 下册)》(丘维声) P195
在酉空间\((V,\rho)\)中,
\(\alpha,\beta \in V\).
把\(\VectorLengthA{\alpha-\beta}\)
称为“\(\alpha\)与\(\beta\)的\DefineConcept{距离}”,
记为\(d(\alpha,\beta)\).
\end{definition}

\subsection{有限维酉空间中的基}
我们希望在有限维酉空间\(V\)中找出一类基,
使得在这样的基下容易计算\(V\)中任意两个向量的内积,
从而易于计算长度、角度、距离等.

\begin{definition}
%@see: 《高等代数(第三版 下册)》(丘维声) P195
设\((V,\rho)\)是一个有限维酉空间,
\(A\)是\(V\)中的一个向量组,
如果\begin{equation*}
	(\forall \alpha \in A)
	[\alpha\neq0],
	\qquad
	(\forall \alpha,\beta \in A)
	[\alpha \perp \beta],
\end{equation*}
则称“\(A\)是\((V,\rho)\)中的一个\DefineConcept{正交向量组}(orthogonal list)”.
\end{definition}

\begin{definition}
%@see: 《高等代数(第三版 下册)》(丘维声) P195
%@see: 《Linear Algebra Done Right (Fourth Edition)》(Sheldon Axler) P197 6.22
设\((V,\rho)\)是一个有限维酉空间,
\(A\)是\(V\)中的一个向量组,
如果\begin{equation*}
	(\forall \alpha \in A)
	[\VectorLengthA{\alpha} = 1],
	\qquad
	(\forall \alpha,\beta \in A)
	[\alpha \perp \beta],
\end{equation*}
则称“\(A\)是\((V,\rho)\)中的一个\DefineConcept{正交单位向量组}(orthonormal list)”.
\end{definition}

\begin{proposition}\label{theorem:酉空间.正交向量组的线性组合的长度}
%@see: 《Linear Algebra Done Right (Fourth Edition)》(Sheldon Axler) P198 6.24
设\(\AutoTuple{\alpha}{s}\)是酉空间\((V,\rho)\)中一个正交向量组,
那么对于\(\forall \AutoTuple{k}{s} \in F\),
有\begin{equation*}
	\VectorLengthA{
		k_1 \alpha_1 + \dotsb + k_s \alpha_s
	}^2
	= \abs{k_1}^2 \VectorLengthA{\alpha_1}^2 + \dotsb + \abs{k_s}^2 \VectorLengthA{\alpha_s}^2.
\end{equation*}
\begin{proof}
由\hyperref[theorem:酉空间.勾股定理]{勾股定理}和\cref{theorem:酉空间.向量的长度具有齐次性} 立即可得.
\end{proof}
\end{proposition}

\begin{proposition}
%@see: 《Linear Algebra Done Right (Fourth Edition)》(Sheldon Axler) P198 6.25
在酉空间\((V,\rho)\)中,
任意一个正交向量组一定线性无关.
%TODO proof
\end{proposition}

\begin{definition}
%@see: 《高等代数(第三版 下册)》(丘维声) P195
%@see: 《Linear Algebra Done Right (Fourth Edition)》(Sheldon Axler) P199 6.27
设\((V,\rho)\)是\(n\)维酉空间,
\(A\)是\(V\)的一个基.
如果\(A\)是正交向量组,
则称“\(A\)是\((V,\rho)\)的一个\DefineConcept{正交基}(orthonormal basis)”.
\end{definition}

\begin{definition}
%@see: 《高等代数(第三版 下册)》(丘维声) P195
设\((V,\rho)\)是\(n\)维酉空间,
\(A\)是\(V\)的一个基.
如果\(A\)是正交单位向量组,
则称“\(A\)是\((V,\rho)\)的一个\DefineConcept{标准正交基}”
或“\(A\)是\((V,\rho)\)的一个\DefineConcept{规范正交基}”.
\end{definition}

% 与《高等代数(第三版 上册)》第4章第6节定理4的证明方法完全一样,
在酉空间\(V\)中也有施密特正交化过程,
它可以把一个线性无关向量组变成与之等价的正交向量组.

在\(n\)维酉空间\(V\)中,取一个基\(\AutoTuple{\alpha}{n}\),
经过施密特正交化把它变成正交基\(\AutoTuple{\beta}{n}\),
再经过单位化把它变成标准正交基\(\AutoTuple{\gamma}{n}\).
这就说明:
\begin{theorem}
%@see: 《Linear Algebra Done Right (Fourth Edition)》(Sheldon Axler) P202 6.35
有限维酉空间中,一定存在标准正交基.
%TODO proof
\end{theorem}

\begin{proposition}
%@see: 《Linear Algebra Done Right (Fourth Edition)》(Sheldon Axler) P203 6.36
有限维酉空间\(V\)中的任意一个正交单位向量组
均可以扩充成\(V\)的一个标准正交基.
%TODO proof
\end{proposition}

\subsection{酉空间中向量的内积、傅里叶展开}
\begin{proposition}
%@see: 《高等代数(第三版 下册)》(丘维声) P195
设\((V,\rho)\)是\(n\)维酉空间,
则向量组\(\AutoTuple{\eta}{n}\)是\((V,\rho)\)的一个标准正交基,
当且仅当\begin{equation*}
%@see: 《高等代数(第三版 下册)》(丘维声) P195 (8)
	\rho(\eta_i,\eta_j)
	= \delta(i,j),
	\quad i,j=1,2,\dotsc,n,
\end{equation*}
其中\(\delta\)是克罗内克\(\delta\)函数.
\end{proposition}

利用标准正交基,容易计算向量的内积.

设\((V,\rho)\)是\(n\)维酉空间,
向量\(\alpha,\beta \in V\),
向量组\(\AutoTuple{\eta}{n}\)是\((V,\rho)\)的一个标准正交基,
\(\alpha,\beta\)在基\(\AutoTuple{\eta}{n}\)下的坐标
分别是\(X=(\AutoTuple{x}{n})^T,
Y=(\AutoTuple{y}{n})^T\),
则\begin{equation*}
%@see: 《高等代数(第三版 下册)》(丘维声) P195 (9)
	\rho(\alpha,\beta)
	= \rho\left( \sum_{i=1}^n x_i \eta_i, \sum_{j=1}^n y_j \eta_j \right)
	= \sum_{i=1}^n \sum_{j=1}^n x_i \ComplexConjugate{y_j} \rho(\eta_i,\eta_j)
	= \sum_{i=1}^n x_i \ComplexConjugate{y_i}
	= Y^H X.
\end{equation*}
%@see: 《Linear Algebra Done Right (Fourth Edition)》(Sheldon Axler) P200 6.30(c)
上式还可以写成\begin{equation}
	\rho(\alpha,\beta)
	= \sum_{i=1}^n \rho(\alpha,\eta_i) \ComplexConjugate{\rho(\beta,\eta_i)}.
\end{equation}

利用标准正交基,可以借助内积,表达向量的坐标分量.

设\(\alpha\)在标准正交基\(\AutoTuple{\eta}{n}\)下的坐标为\(X=(\AutoTuple{x}{n})^T\),
则\begin{equation*}
	\alpha = \sum_{i=1}^n x_i \eta_i;
\end{equation*}
等号两边用\(\eta_j\)作内积,得\begin{equation*}
	\rho(\alpha,\eta_j)
	= \rho\left( \sum_{i=1}^n x_i \eta_i, \eta_j \right)
	= \sum_{i=1}^n x_i \rho(\eta_i,\eta_j)
	= x_j,
\end{equation*}
因此\begin{equation}\label{equation:酉空间.向量的傅里叶展开}
%@see: 《高等代数(第三版 下册)》(丘维声) P196 (10)
%@see: 《Linear Algebra Done Right (Fourth Edition)》(Sheldon Axler) P200 6.30(a)
	\alpha = \sum_{i=1}^n \rho(\alpha,\eta_i) \eta_i.
\end{equation}
我们把\cref{equation:酉空间.向量的傅里叶展开}
称为“向量\(\alpha\)的\DefineConcept{傅里叶展开}”,
其中系数\(\rho(\alpha,\eta_i)\ (i=1,2,\dotsc,n)\)
称为“向量\(\alpha\)的\DefineConcept{傅里叶系数}”.

由\cref{equation:酉空间.向量的傅里叶展开,theorem:酉空间.正交向量组的线性组合的长度} 可得\begin{equation}
%@see: 《Linear Algebra Done Right (Fourth Edition)》(Sheldon Axler) P200 6.30(b)
	\VectorLengthA{\alpha}^2
	= \sum_{i=1}^n \rho^2(\alpha,\eta_i).
\end{equation}

\begin{proposition}
%@see: 《高等代数(第三版 下册)》(丘维声) P196
在有限维酉空间\((V,\rho)\)中,
一个标准正交基到另一个标准正交基的过渡矩阵
一定是酉矩阵.
%TODO proof
\end{proposition}

\begin{proposition}
%@see: 《高等代数(第三版 下册)》(丘维声) P196
设\((V,\rho)\)是\(n\)维酉空间,
向量组\(\AutoTuple{\eta}{n}\)是\((V,\rho)\)的一个标准正交基,
\(\AutoTuple{\beta}{n}\)是\(V\)中一个向量组.
如果存在酉矩阵\(P \in M_n(\mathbb{C})\),
使得\begin{equation*}
	(\AutoTuple{\beta}{n})
	= (\AutoTuple{\eta}{n}) P,
\end{equation*}
那么\(\AutoTuple{\beta}{n}\)是\(V\)中一个标准正交基.
%TODO proof
\end{proposition}

\begin{definition}
%@see: 《矩阵论》(詹兴致) P5
设\(\alpha \in \mathbb{C}^n\).
定义:\begin{equation}
	\MatrixNorm{A}_p
	\defeq
	\left( \sum_{i=1}^n \ComplexLengthA{a_i}^p \right)^{1/p},
\end{equation}
称之为“向量\(\alpha\)的 \DefineConcept{\(L_p\)范数}”,
其中\(\alpha = (\AutoTuple{a}{n})^T\).
\end{definition}

\begin{definition}
%@see: 《矩阵论》(詹兴致) P5
设\(A \in M_{m \times n}(\mathbb{C})\),
\(A^H\)是\(A\)的共轭转置.
定义:\begin{equation}
	\MatrixNorm{A}_F
	\defeq
	\sqrt{
		\tr(A^H A)
	},
\end{equation}
称之为“矩阵\(A\)的\DefineConcept{弗罗贝尼乌斯范数}”.
\end{definition}

\begin{definition}
%@see: 《矩阵论》(詹兴致) P5
设\(A \in M_{m \times n}(\mathbb{C})\).
定义:\begin{equation}
	\MatrixNorm{A}_r
	\defeq
	\max_{1 \leq i \leq m} \sum_{j=1}^n \abs{a_{ij}},
\end{equation}
称之为“矩阵\(A\)的\DefineConcept{行和范数}”.
\end{definition}

\begin{definition}
%@see: 《矩阵论》(詹兴致) P5
设\(A \in M_{m \times n}(\mathbb{C})\).
定义:\begin{equation}
	\MatrixNorm{A}_c
	\defeq
	\max_{1 \leq j \leq n} \sum_{i=1}^n \abs{a_{ij}},
\end{equation}
称之为“矩阵\(A\)的\DefineConcept{列和范数}”.
\end{definition}

\begin{definition}
%@see: 《矩阵论》(詹兴致) P5
设\(A \in M_{m \times n}(\mathbb{C})\),
\(f\)是\(\mathbb{C}^m\)上的一个范数,
\(g\)是\(\mathbb{C}^n\)上的一个范数.
定义:\begin{equation}
	\MatrixNorm{A}
	\defeq
	\max_{0 \neq x \in \mathbb{C}^n} \frac{f(Ax)}{g(x)},
\end{equation}
称之为“矩阵\(A\)由\(f\)和\(g\)诱导的\DefineConcept{算子范数}”.
\end{definition}

\begin{definition}
%@see: 《矩阵论》(詹兴致) P5
如果由\(f\)和\(g\)诱导的算子范数
\(N\colon M_n(\mathbb{C}) \to \mathbb{R}, A \mapsto \MatrixNorm{A}\)
满足\begin{equation*}
	(\forall A,B \in M_n(\mathbb{C}))
	[
		\MatrixNorm{AB}
		\leq \MatrixNorm{A} \MatrixNorm{B}
	],
\end{equation*}
则称“由\(f\)和\(g\)诱导的算子范数\(N\)是\DefineConcept{次可乘的}”.
\end{definition}

\begin{definition}
%@see: 《矩阵论》(詹兴致) P5
设\(A \in M_{m \times n}(\mathbb{C})\).
定义:\begin{equation}
	\MatrixNorm{A}_\infty
	\defeq
	\max_{x\neq0} \frac{\MatrixNorm{Ax}_2}{\MatrixNorm{x}_2},
\end{equation}
称之为“矩阵\(A\)的\DefineConcept{谱范数}”.
\end{definition}

\begin{property}
%@see: 《矩阵论》(詹兴致) P5
如果\(B\)是\(A \in M_{m \times n}(\mathbb{C})\)的一个子矩阵,
则\(\MatrixNorm{B}_\infty \leq \MatrixNorm{A}_\infty\).
\end{property}

\begin{property}
%@see: 《矩阵论》(詹兴致) P5
设\(A \in M_{m \times n}(\mathbb{C})\),
\(X\)是一个\(m\)阶酉矩阵,
\(Y\)是一个\(n\)阶酉矩阵,
则\begin{equation}
	\MatrixNorm{A}_\infty
	= \MatrixNorm{AY}_\infty
	= \MatrixNorm{XA}_\infty.
\end{equation}
\end{property}

\subsection{酉空间之间的同构}
\begin{definition}
%@see: 《高等代数(第三版 下册)》(丘维声) P196
设\((V_1,\rho_1),(V_2,\rho_2)\)都是酉空间.
如果存在从\(V_1\)到\(V_2\)的一个双射\(\sigma\),
使得\begin{gather*}
	(\forall \alpha,\beta \in V_1)
	[
		\sigma(\alpha+\beta)
		= \sigma(\alpha) + \sigma(\beta)
	], \\
	(\forall \alpha \in V_1)
	(\forall k \in \mathbb{C})
	[
		\sigma(k\alpha)
		= k \sigma(\alpha)
	], \\
	(\forall \alpha,\beta \in V_1)
	[
		\rho_2(\sigma(\alpha),\sigma(\beta))
		= \rho_1(\alpha,\beta)
	],
\end{gather*}
则称“\(\sigma\)是从\(V_1\)到\(V_2\)的一个\DefineConcept{同构}”;
并称“\(V_1\)与\(V_2\) \DefineConcept{同构}”,
记为\(V_1 \Isomorphism V_2\).
\end{definition}

\begin{theorem}\label{theorem:酉空间.两个酉空间同构的充分必要条件}
%@see: 《高等代数(第三版 下册)》(丘维声) P196
两个酉空间同构的充分必要条件是它们的维数相同.
%TODO proof
\end{theorem}

\subsection{酉变换}
\begin{definition}
%@see: 《高等代数(第三版 下册)》(丘维声) P197 定义6
设\((V,\rho)\)是一个酉空间,
\(\vb{A}\)是一个从\(V\)到\(V\)的满射.
如果\(\vb{A}\)满足\begin{equation}
	(\forall \alpha,\beta \in V)
	[
		\rho(\vb{A}\alpha,\vb{A}\beta)
		= \rho(\alpha,\beta)
	],
\end{equation}
则称“\(\vb{A}\)保持向量的内积不变”
“\(\vb{A}\)是\(V\)上的一个\DefineConcept{酉变换}”.
\end{definition}

\begin{proposition}
%@see: 《高等代数(第三版 下册)》(丘维声) P197 命题2
酉空间上的酉变换一定是可逆线性变换.
%TODO proof
\end{proposition}

\begin{proposition}
%@see: 《高等代数(第三版 下册)》(丘维声) P197 命题3
设\(\vb{A}\)是\(n\)维酉空间\(V\)上的一个线性变换,
则\begin{align*}
	&\text{$\vb{A}$是酉变换} \\
	&\iff \text{$\vb{A}$把$V$的标准正交基映成标准正交基} \\
	&\iff \text{$\vb{A}$在$V$的标准正交基下的矩阵是酉矩阵}.
\end{align*}
\end{proposition}

\begin{proposition}
%@see: 《高等代数(第三版 下册)》(丘维声) P197 命题4
有限维酉空间上的酉变换的特征值的模等于\(1\).
\begin{proof}
由\cref{equation:幺正矩阵.幺正矩阵的行列式} 立即可得.
\end{proof}
\end{proposition}

\begin{proposition}
%@see: 《高等代数(第三版 下册)》(丘维声) P197 命题5
设\(\vb{A}\)是酉空间\((V,\rho)\)上的一个酉变换.
如果\(W\)是\(\vb{A}\)的有限维不变子空间,
则\(W\)的正交补\(W^\perp\)也是\(\vb{A}\)的不变子空间.
\begin{proof}
任取\(\beta \in W^\perp\).
要证\(\vb{A}\beta \in W^\perp\).
任取\(\alpha \in W\),
由于酉变换\(\vb{A}\)是可逆的,
因此由\cref{example:线性映射.可逆线性变换的逆变换的不变子空间} 可知
\(W\)也是\(\vb{A}^{-1}\)的不变子空间,
从而\(\vb{A}^{-1}\alpha \in W\).
于是\begin{equation*}
	\rho(\vb{A}\beta,\alpha)
	= \rho(\vb{A}\beta,\vb{A}\vb{A}^{-1}\alpha)
	= \rho(\beta,\vb{A}^{-1}\alpha)
	= 0,
\end{equation*}
即\(\vb{A}\beta \in W^\perp\),
说明\(W^\perp\)是\(\vb{A}\)的不变子空间.
\end{proof}
\end{proposition}

\begin{theorem}
%@see: 《高等代数(第三版 下册)》(丘维声) P197 定理6
设\(\vb{A}\)是\(n\)维酉空间\(V\)上的酉变换,
则\(V\)中存在一个标准正交基\(S\),
使得\(\vb{A}\)在基\(S\)下的矩阵是对角矩阵,
且主对角元都是模为\(1\)的复数.
%TODO proof
\end{theorem}

\begin{definition}
%@see: 《高等代数(第三版 下册)》(丘维声) P198 推论7
设矩阵\(A,B \in M_n(\mathbb{C})\).
若存在可逆矩阵\(P \in M_n(\mathbb{C})\),
使得\begin{equation}
	P^{-1} A P = B,
\end{equation}
则称“\(A\)与\(B\) \DefineConcept{酉相似}”
或“\(A\) \DefineConcept{酉相似于} \(B\)”.
\end{definition}

\begin{corollary}
%@see: 《高等代数(第三版 下册)》(丘维声) P198 推论7
任意一个\(n\)阶酉矩阵一定酉相似于某个主对角元都是模为\(1\)的复数的对角矩阵.
%TODO proof
\end{corollary}

\subsection{厄米变换}
类比于实内积空间\(V\)上的对称变换,引出酉空间上的下述变换:
\begin{definition}
%@see: 《高等代数(第三版 下册)》(丘维声) P198 定义7
设\(\vb{A}\)是酉空间\((V,\rho)\)上的一个线性变换.
如果\begin{equation*}
	(\forall \alpha,\beta \in V)
	[
		\rho(\vb{A}\alpha,\beta)
		= \rho(\alpha,\vb{A}\beta)
	],
\end{equation*}
则称“\(\vb{A}\)是\(V\)上的一个\DefineConcept{厄米变换}”
或“\(\vb{A}\)是\(V\)上的一个\DefineConcept{自伴随变换}”.
\end{definition}

\begin{proposition}
%@see: 《高等代数(第三版 下册)》(丘维声) P198 命题8
酉空间上的厄米变换一定是线性变换.
%TODO proof
\end{proposition}

\begin{proposition}
%@see: 《高等代数(第三版 下册)》(丘维声) P198 命题9
设\(\vb{A}\)是\(n\)维酉空间\(V\)上的一个线性变换,
则\(\vb{A}\)是厄米变换,
当且仅当\(\vb{A}\)在\(V\)的某个标准正交基下的矩阵是厄米矩阵.
%TODO proof
\end{proposition}

我们来探索\(n\)维酉空间上的厄米变换的最简单形式的矩阵表示.
\begin{proposition}
%@see: 《高等代数(第三版 下册)》(丘维声) P199 命题10
酉空间上的厄米变换的特征值只要存在就一定是实数.
%TODO proof
\end{proposition}

\begin{proposition}
%@see: 《高等代数(第三版 下册)》(丘维声) P199 命题11
设\(\vb{A}\)是酉空间\(V\)上的一个厄米变换.
如果\(W\)是\(\vb{A}\)的一个不变子空间,
则\(W\)的正交补\(W^\perp\)也是\(\vb{A}\)的一个不变子空间.
\end{proposition}

\begin{theorem}
%@see: 《高等代数(第三版 下册)》(丘维声) P199 定理12
设\(\vb{A}\)是酉空间\(V\)上的一个厄米变换,
则\(V\)中存在一个标准正交基\(S\),
使得\(\vb{A}\)在基\(S\)下的矩阵是对角矩阵,
且主对角元都是实数.
%TODO proof
\end{theorem}

\begin{corollary}
%@see: 《高等代数(第三版 下册)》(丘维声) P199 推论13
任意一个\(n\)阶厄米矩阵一定酉相似于某个实对角矩阵.
\end{corollary}

\section{辛空间}


\chapter{赋范线性空间}
\section{范数}
尽管我们通常出于几何(特别是欧氏几何)的考量,
将向量\(\alpha\)的模(或范数)定义为\(\sqrt{\VectorInnerProductDot{\alpha}{\alpha}}\),
不过我们还可以定义其他形式的模(或范数).

\subsection{范数的定义}
\begin{definition}
%@see: 《矩阵分析与应用(第2版)》(张贤达) P23 定义1.3.7(范数和赋范向量空间)
设\(V\)是域\(F\)上的一个线性空间.
如果映射\(p\colon V \to \mathbb{R}\)
满足\begin{itemize}
	\item {\rm\bf 非负性}:\begin{equation*}
		(\forall \alpha \in V)
		[p(\alpha) \geq 0],
		\qquad
		(\forall \alpha \in V)
		[
			p(\alpha) = 0
			\iff
			\alpha = 0
		];
	\end{equation*}

	\item {\rm\bf 齐次性}:\begin{equation*}
		(\forall \alpha \in V)
		(\forall k \in F)
		[
			p(k \alpha) = \abs{k} p(\alpha)
		];
	\end{equation*}

	\item {\rm\bf 三角不等式}:\begin{equation*}
		(\forall \alpha,\beta \in V)
		[
			p(\alpha+\beta) \leq p(\alpha) + p(\beta)
		],
	\end{equation*}
\end{itemize}
则称“\(f\)是线性空间\(V\)的一个\DefineConcept{范数}(norm)”
“\((V,f)\)是域\(F\)上的一个\DefineConcept{赋范线性空间}”,
在不致混淆的情况下简称“\(V\)是一个\DefineConcept{赋范线性空间}”;
对于任意向量\(\alpha \in V\),
把\(\alpha\)在\(f\)下的像
称为“向量\(\alpha\)的\(f\) \DefineConcept{范数}”,
记作\(\VectorLengthN{\alpha}\),
即\begin{equation*}
	\VectorLengthN{\alpha}
	\defeq
	f(\alpha).
\end{equation*}
\end{definition}

\subsection{半范数}
\begin{definition}
%@see: 《矩阵分析与应用(第2版)》(张贤达) P24 定义1.3.8
设\(V\)是域\(F\)上的一个线性空间.
如果映射\(p\colon V \to \mathbb{R}\)
满足\begin{itemize}
	\item {\rm\bf 非负性}:\begin{equation*}
		(\forall \alpha \in V)
		[p(\alpha) \geq 0];
	\end{equation*}

	\item {\rm\bf 齐次性}:\begin{equation*}
		(\forall \alpha \in V)
		(\forall k \in F)
		[
			p(k \alpha) = \abs{k} p(\alpha)
		];
	\end{equation*}

	\item {\rm\bf 三角不等式}:\begin{equation*}
		(\forall \alpha,\beta \in V)
		[
			p(\alpha+\beta) \leq p(\alpha) + p(\beta)
		],
	\end{equation*}
\end{itemize}
则称“\(f\)是线性空间\(V\)的一个\DefineConcept{半范数}(seminorm)”
或者“\(f\)是线性空间\(V\)的一个\DefineConcept{伪范数}(seminorm)”.
%@see: https://mathworld.wolfram.com/Seminorm.html
\end{definition}

半范数与范数的唯一区别是:
半范数不完全满足范数的非负性公理,
有可能当\(\alpha\neq0\)时成立\(p(\alpha) = 0\).
\begin{proposition}
设\(f\)是线性空间\(V\)的一个半范数.
如果\begin{equation*}
	(\forall \alpha \in V)
	[
		p(\alpha) = 0
		\iff
		\alpha = 0
	],
\end{equation*}
则\(f\)是线性空间\(V\)的一个范数.
\end{proposition}

\begin{example}
对于任意\(n\)维向量\(\alpha = (x_1,\dotsc,x_n)^T\),
只要满足\begin{equation*}
	x_1 + \dotsb + x_n = 0,
	% 即\(\alpha\)是“零均值向量”
\end{equation*}
则映射\begin{equation*}
	p(\alpha)
	\defeq
	x_1 + \dotsb + x_n
\end{equation*}
就是向量\(\alpha\)的一个半范数.
但是,显然\(p(\alpha) = 0\)并不意味着\(\alpha = 0\).
\end{example}

\begin{definition}% 向量的范数
%@see: 《数值分析(第5版)》(李庆扬、王能超、易大义) P53
设\(\vb{x} \defeq (x_1,\dotsc,x_n)^T \in \mathbb{R}^n\).
定义:\begin{align*}
	\norm{\vb{x}}_\infty
	&\defeq
	\max_{1 \leq i \leq n} \abs{x_i}, \\
	\norm{\vb{x}}_1
	&\defeq
	\sum_{i=1}^n \abs{x_i}, \\
	\norm{\vb{x}}_2
	&\defeq
	\left( \sum_{i=1}^n x_i^2 \right)^{\frac12}.
\end{align*}
\end{definition}

\begin{definition}% 连续函数的范数
%@see: 《数值分析(第5版)》(李庆扬、王能超、易大义) P53
设\(f \in C[a,b]\).
定义:\begin{align*}
	\norm{f}_\infty
	&\defeq
	\max_{a \leq x \leq b} \abs{f(x)}, \\
	\norm{f}_1
	&\defeq
	\int_a^b \abs{f(x)} \dd{x}, \\
	\norm{f}_2
	&\defeq
	\left( \int_a^b f^2(x) \right)^{\frac12}.
\end{align*}
\end{definition}

\subsection{拟范数}
\begin{definition}
%@see: 《矩阵分析与应用(第2版)》(张贤达) P24 定义1.3.9
设\(V\)是有序域\(F\)上的一个线性空间.
如果映射\(p\colon V \to \mathbb{R}\)
满足\begin{itemize}
	\item {\rm\bf 非负性}:\begin{equation*}
		(\forall \alpha \in V)
		[p(\alpha) \geq 0],
		\qquad
		(\forall \alpha \in V)
		[
			p(\alpha) = 0
			\iff
			\alpha = 0
		];
	\end{equation*}

	\item {\rm\bf 齐次性}:\begin{equation*}
		(\forall \alpha \in V)
		(\forall k \in F)
		[
			p(k \alpha) = \abs{k} p(\alpha)
		];
	\end{equation*}

	\item {\rm\bf 三角不等式}:\begin{equation*}
		(\forall \alpha,\beta \in V)
		(\exists c \in C)
		[
			p(\alpha+\beta) \leq c [p(\alpha) + p(\beta)]
		],
	\end{equation*}
	其中\(
		C
		\defeq
		\Set{
			x \in F
			\given
			x > 0,
			x \neq 1
		}
	\),
\end{itemize}
则称“\(f\)是线性空间\(V\)的一个\DefineConcept{拟范数}(quasinorm)”.
\end{definition}

拟范数与范数的唯一区别是:
拟范数不严格满足范数的三角不等式公理.

\begin{example}
%@see: 《矩阵分析与应用(第2版)》(张贤达) P24
同一个定义公式,有时给出拟范数,有时给出范数,取决于参数的变化.
例如,容易验证\begin{equation*}
%@see: 《矩阵分析与应用(第2版)》(张贤达) P24 (1.3.24)
	\norm{\alpha}_p
	\defeq
	\left( \sum_{i=1}^m x_i^p \right)^{1/p}
\end{equation*}
当\(0 < p < 1\)时给出的是拟范数,
当\(p \geq 1\)时给出的是范数.
\end{example}

\subsection{向量范数}
\begin{definition}
%@see: 《矩阵论》(詹兴致) P5
设\(\alpha \in \mathbb{C}^n\).
定义:\begin{equation}
	\MatrixNorm{A}_p
	\defeq
	\left( \sum_{i=1}^n \ComplexLengthA{a_i}^p \right)^{1/p},
\end{equation}
称之为“向量\(\alpha\)的 \DefineConcept{\(L_p\)范数}”,
其中\(\alpha = (\AutoTuple{a}{n})^T\).
\end{definition}

易见\begin{gather}
	\norm{\alpha}_1 = \VectorLengthA{x_1} + \VectorLengthA{x_2} + \dotsb + \VectorLengthA{x_n}, \\
	\norm{\alpha}_2 = \sqrt{x_1^2 + x_2^2 + \dotsb + x_n^2}, \\
	\norm{\alpha}_\infty = \max\{\VectorLengthA{x_1},\VectorLengthA{x_2},\dotsc,\VectorLengthA{x_n}\}.
\end{gather}

\subsection{矩阵范数}
由所有形状相同的矩阵组成的集合,对加法和标量乘法,成为一个线性空间.
于是我们可以在这个线性空间\(M_{m \times n}(K)\)上定义范数.

\begin{definition}
%@see: 《矩阵论》(詹兴致) P5
设\(A \in M_{m \times n}(\mathbb{C})\),
\(A^H\)是\(A\)的共轭转置.
定义:\begin{equation}
	\MatrixNorm{A}_F
	\defeq
	\sqrt{
		\tr(A^H A)
	},
\end{equation}
称之为“矩阵\(A\)的\DefineConcept{弗罗贝尼乌斯范数}”.
\end{definition}

\begin{definition}
%@see: 《矩阵论》(詹兴致) P5
设\(A \in M_{m \times n}(\mathbb{C})\).
定义:\begin{equation}
	\MatrixNorm{A}_r
	\defeq
	\max_{1 \leq i \leq m} \sum_{j=1}^n \abs{a_{ij}},
\end{equation}
称之为“矩阵\(A\)的\DefineConcept{行和范数}”.
\end{definition}

\begin{definition}
%@see: 《矩阵论》(詹兴致) P5
设\(A \in M_{m \times n}(\mathbb{C})\).
定义:\begin{equation}
	\MatrixNorm{A}_c
	\defeq
	\max_{1 \leq j \leq n} \sum_{i=1}^n \abs{a_{ij}},
\end{equation}
称之为“矩阵\(A\)的\DefineConcept{列和范数}”.
\end{definition}

\begin{definition}
%@see: 《矩阵论》(詹兴致) P5
设\(A \in M_{m \times n}(\mathbb{C})\),
\(f\)是\(\mathbb{C}^m\)上的一个范数,
\(g\)是\(\mathbb{C}^n\)上的一个范数.
定义:\begin{equation}
	\MatrixNorm{A}
	\defeq
	\max_{0 \neq x \in \mathbb{C}^n} \frac{f(Ax)}{g(x)},
\end{equation}
称之为“矩阵\(A\)由\(f\)和\(g\)诱导的\DefineConcept{算子范数}”.
\end{definition}

\begin{definition}
%@see: 《矩阵论》(詹兴致) P5
如果由\(f\)和\(g\)诱导的算子范数
\(N\colon M_n(\mathbb{C}) \to \mathbb{R}, A \mapsto \MatrixNorm{A}\)
满足\begin{equation*}
	(\forall A,B \in M_n(\mathbb{C}))
	[
		\MatrixNorm{AB}
		\leq \MatrixNorm{A} \MatrixNorm{B}
	],
\end{equation*}
则称“由\(f\)和\(g\)诱导的算子范数\(N\)是\DefineConcept{次可乘的}”.
\end{definition}

\begin{definition}
%@see: 《矩阵论》(詹兴致) P5
设\(A \in M_{m \times n}(\mathbb{C})\).
定义:\begin{equation}
	\MatrixNorm{A}_\infty
	\defeq
	\max_{x\neq0} \frac{\MatrixNorm{Ax}_2}{\MatrixNorm{x}_2},
\end{equation}
称之为“矩阵\(A\)的\DefineConcept{谱范数}”.
\end{definition}

\begin{property}
%@see: 《矩阵论》(詹兴致) P5
如果\(B\)是\(A \in M_{m \times n}(\mathbb{C})\)的一个子矩阵,
则\(\MatrixNorm{B}_\infty \leq \MatrixNorm{A}_\infty\).
\end{property}

\begin{property}
%@see: 《矩阵论》(詹兴致) P5
设\(A \in M_{m \times n}(\mathbb{C})\),
\(X\)是一个\(m\)阶酉矩阵,
\(Y\)是一个\(n\)阶酉矩阵,
则\begin{equation}
	\MatrixNorm{A}_\infty
	= \MatrixNorm{AY}_\infty
	= \MatrixNorm{XA}_\infty.
\end{equation}
\end{property}


\chapter{仿射几何与射影几何}
\section{仿射空间}
本节利用公理法描述仿射几何中的几何元素.

\subsection{仿射空间}
\begin{definition}
%@see: 《基础代数(第二卷)》(席南华) P161 定义4.1
%@see: 《代数学讲义(下册)》(李文威)
设\(A\)是一个非空集合,
\(V\)是域\(F\)上的一个线性空间.
若映射\(f\colon A \times V \to A\)满足\begin{gather*}
	(\forall p \in A)
	(\forall x,y \in V)
	[
		f(f(p,x),y)
		= f(p,x+y)
	], \\
	(\forall p \in A)
	[
		0 \in V
		\implies
		f(p,0) = p
	], \\
	(\forall p,q \in A)
	(\exists! x \in V)
	[
		f(p,x) = q
	],
\end{gather*}
则称“\(A\)是与线性空间\(V\)关联的一个\DefineConcept{仿射空间}”
“\(A\)是与线性空间\(V\)相伴的一个\DefineConcept{仿射空间}”
或“\((A,V,f)\)是一个\DefineConcept{仿射空间}”,
在不致混淆的情况下简称“\((A,V)\)是一个\DefineConcept{仿射空间}”
或“\(A\)是(域\(F\)上的)一个\DefineConcept{仿射空间}”.
把\(f\)称为“仿射空间\(A\)的\DefineConcept{加法}”,
在不致混淆的情况下把\(f(p,x)\)记为\(p + x\).
把\(A\)中的每一个元素称为“仿射空间\(A\)中的一个\DefineConcept{点}”.
对于\(A\)中任意两点\(p,q\),把满足\(f(p,x) = q\)的向量\(x \in V\)
称为“连接点\(p\)和点\(q\)的\DefineConcept{向量}”,记作\(\vec{pq}\)或\(q - p\).
把线性空间\(V\)的维数\(\dim V\)称为“仿射空间\(A\)的\DefineConcept{维数}”,记为\(\dim A\).
\end{definition}

为方便起见,我们把\((A,\mathbb{R}^n,+)\)称为“\(n\)维\DefineConcept{实仿射空间}”,
把\((A,\mathbb{C}^n,+)\)称为“\(n\)维\DefineConcept{复仿射空间}”.

\begin{property}%\label{theorem:仿射空间.仿射空间的减法1}
设\((A,V,f)\)是仿射空间,
则\(x = \vec{pq}\)当且仅当\(f(p,x) = q\).
\end{property}

\begin{property}%\label{theorem:仿射空间.仿射空间的减法2}
设\((A,V,f)\)是仿射空间,
则\(f(p,v_1) = f(p,v_2)\)当且仅当\(v_1 = v_2\).
\end{property}

\begin{proposition}\label{theorem:仿射空间.仿射空间与线性空间等势}
设\((A,V)\)是仿射空间,
则\(A\)与\(V\)等势.
\begin{proof}
由于仿射空间\(A\)是非空的,我们可以从中任意取定一个点\(o\).
然后构造一个从\(A\)到\(V\)的映射\(\sigma\colon A \to V, p \mapsto \vec{op}\).
根据仿射空间的定义,
对于\(A\)中任意一点\(p\),
满足\(f(o,v) = p\)的向量\(v \in V\)存在且唯一,
由此保证\(\sigma\)是单值的,且\(\ran\sigma \subseteq V\).
由\begin{equation*}
	(\forall p_1,p_2 \in A)
	[
		\sigma(p_1) = \sigma(p_2) = v
		\iff
		\vec{op_1} = \vec{op_2} = v
		\iff
		f(o,v) = p_1 = p_2
	]
\end{equation*}
可知\(\sigma\)是单射.
因为\begin{equation*}
	\sigma(q) = v
	\iff  % 映射\(\sigma\)的定义
	\vec{oq} = v
	\iff  % “连接两点的向量”的定义,\cref{theorem:仿射空间.仿射空间的减法1}
	f(o,v) = q,
\end{equation*}
% 又因为\(f(o,v) = f(o,v)\),
所以\(
	(\forall v \in V)
	[
		\sigma(f(o,v)) = v
	]
\),
可知\(\sigma\)是满射.
因此\(\sigma\)是从\(A\)到\(V\)的双射,
\(A\)与\(V\)等势.
\end{proof}
\end{proposition}

\begin{example}
%@see: 《基础代数(第二卷)》(席南华) P162 例4.2
设\(V\)是域\(F\)上的一个线性空间.
证明:\(V\)是一个仿射空间.
\begin{proof}
任意取定\(p,q \in V\).
对于任意\(x,y \in V\),
利用向量加法的结合律可得\begin{equation*}
	(p + x) + y
	= p + (x + y),
	\qquad
	p + 0
	= p.
\end{equation*}
同时,我们还有\begin{equation*}
	p + x = q
	\iff
	x = q - p \in V,
\end{equation*}
这就说明\(
	(\exists! x \in V)
	[
		f(p,x) = q
	]
\).
综上所述,\((V,+)\)是与\(V\)关联的仿射空间.
\end{proof}
\end{example}

\begin{example}\label{example:仿射空间.仿射空间与仿射几何的联系1}
%@see: 《基础代数(第二卷)》(席南华) P162 例4.3
设\(V\)是域\(F\)上的一个线性空间,\(W\)是\(V\)的一个子空间,
\(S\)是以向量\(\alpha \in V\)为代表、子空间\(W\)的一个陪集.
证明:\((S,+)\)是与\(W\)关联的仿射空间.
\begin{proof}
由题意有\(S = \alpha + W\).
任意取定\(p,q \in S\),
则存在\(w_1,w_2 \in W\)
使得\(p = \alpha + w_1, q = \alpha + w_2\).
由向量加法的封闭性可知\(p,q \in V\).
对于任意\(x,y \in W\),
利用向量加法的结合律可得\begin{equation*}
	(p + x) + y
	= p + (x + y),
	\qquad
	p + 0
	= p.
\end{equation*}
同时,我们还有\begin{equation*}
	p + x = q
	\iff
	x = q - p
	\iff
	x = (\alpha + w_2) - (\alpha + w_1)
	\iff
	x = w_2 - w_1 \in W,
\end{equation*}
这就说明\(
	(\exists! x \in W)
	[
		f(p,x) = q
	]
\).
综上所述,\((S,+)\)是与\(W\)关联的仿射空间.
\end{proof}
\end{example}

\subsection{仿射空间中的仿射映射与仿射同构}
\begin{definition}%\label{definition:仿射空间.仿射映射}
%@see: 《基础代数(第二卷)》(席南华) P163 定义4.4
设\(V,V'\)都是域\(F\)上的线性空间,
\(A,A'\)分别是与线性空间\(V,V'\)关联的仿射空间.
如果存在线性映射\(g \in \Hom(V,V')\)
使得映射\(f\colon A \to A'\)满足\begin{equation*}
%@see: 《基础代数(第二卷)》(席南华) P163 (4.1.1)
	(\forall p \in A)
	(\forall v \in V)
	[
		f(p + v)
		= f(p) + g(v)
	],
\end{equation*}
则称“\(f\)是从\(A\)到\(A'\)的一个\DefineConcept{仿射映射}”;
把\(g\)称为“仿射映射\(f\)的\DefineConcept{线性部分}”
或“仿射映射\(f\)的\DefineConcept{微分}”,
记作\(\dd{f}\).
\end{definition}

\begin{proposition}\label{theorem:仿射空间.仿射映射的微分的唯一性}
%@see: 《基础代数(第二卷)》(席南华) P163
设\(A,A'\)都是域\(F\)上的仿射空间,
\(p,q\)是\(A\)中两点,
\(f\)是从\(A\)到\(A'\)的一个仿射映射,
记\(p' \defeq f(p), q' \defeq f(q)\),
则\(
%@see: 《基础代数(第二卷)》(席南华) P163 (4.1.2)
	\vec{p'q'}
	= \dd{f}(\vec{pq})
\).
\end{proposition}
\begin{remark}
由\cref{theorem:仿射空间.仿射映射的微分的唯一性} 可以看出,
仿射映射的微分存在且唯一.
\end{remark}

\begin{corollary}
%@see: 《基础代数(第二卷)》(席南华) P164 命题4.7
设\(A,A'\)都是域\(F\)上的仿射空间,
\(o,p,q\)是\(A\)中三点,
\(f\)是从\(A\)到\(A'\)的一个仿射映射,
记\(o' \defeq f(o), p' \defeq f(p), q' \defeq f(q)\).
如果\(p \neq q\),
则\(
	\vec{op} \neq \vec{oq},
	\vec{o'p'} \neq \vec{o'q'}
\).
%TODO proof
\end{corollary}

\begin{definition}
设\(V,V'\)都是域\(F\)上的线性空间,
\(A,A'\)分别是与线性空间\(V,V'\)关联的仿射空间.
如果映射\(f\colon A \to A\)是从\(A\)到\(A\)的一个仿射映射,
则称“\(f\)是\(A\)上的一个\DefineConcept{仿射变换}”.
\end{definition}

\begin{definition}%\label{definition:仿射空间.平移变换}
%@see: 《基础代数(第二卷)》(席南华) P162
设\(A\)是与线性空间\(V\)关联的仿射空间,
向量\(\alpha \in V\),
把映射\begin{equation*}
	\tau_\alpha\colon A \to A,
	p \mapsto p + \alpha
\end{equation*}
称为“(仿射空间\(A\)上)由向量\(\alpha\)决定的\DefineConcept{平移变换}”.
\end{definition}

\begin{theorem}%\label{theorem:仿射空间.平移变换对乘法成群}
%@see: 《基础代数(第二卷)》(席南华) P162
设\(V\)是域\(F\)上的一个线性空间,
\(A\)是与线性空间\(V\)关联的仿射空间,
\(\alpha,\beta \in V\),
\(\tau_\alpha,\tau_\beta\)分别是\(A\)上由\(\alpha,\beta\)决定的平移变换,
\(I\)是\(A\)上的恒等映射,
则\begin{gather*}
	\tau_\alpha \tau_\beta = \tau_{\alpha + \beta}, \\
	\tau_{-\alpha} \tau_\alpha = I.
\end{gather*}
\end{theorem}
\begin{remark}
%@see: 《基础代数(第二卷)》(席南华) P162
上述定理说明:
两个平移的复合还是平移,
平移的逆映射也是平移.
全体平移对复合成群,它同构于线性空间\(V\)的加法群.
不难验证:
对于任意\(x,y \in F\),
如果令\begin{equation*}
	x \tau_\alpha + y \tau_\beta
	\defeq \tau_{x \alpha + y \beta},
\end{equation*}
则\(A\)的全体平移就成为一个线性空间(记作\(A^\sharp\)),它与\(V\)同构.
\end{remark}

\begin{definition}%\label{definition:仿射空间.伸缩变换}
%@see: 《基础代数(第二卷)》(席南华) P163 例4.6
设\(V\)是域\(F\)上的一个线性空间,
\(A\)是与线性空间\(V\)关联的仿射空间,
\(o\)是\(A\)中的一个点,
数\(\lambda \in F\),
把映射\begin{equation*}
	\delta_\lambda\colon A \to A,
	p \mapsto o + \lambda\vec{op}
\end{equation*}
称为“(仿射空间\(A\)上)以点\(o\)为中心、\(\lambda\)为伸缩率的\DefineConcept{伸缩变换}(dilatation)”.
\end{definition}

\begin{definition}%\label{definition:仿射空间.常值变换}
%@see: 《基础代数(第二卷)》(席南华) P163 例4.5(1)
设\(A\)是一个仿射空间,
点\(a \in A\).
把映射\(f\colon A \to A, x \mapsto a\)
称为“仿射空间\(A\)上的\DefineConcept{常值变换}”.
\end{definition}

\begin{proposition}%\label{theorem:仿射空间.常值变换的微分是零映射}
%@see: 《基础代数(第二卷)》(席南华) P163 例4.5(1)
仿射空间\(A\)上的常值变换的微分是零映射.
\end{proposition}

\begin{definition}%\label{definition:仿射空间.仿射同构}
%@see: 《基础代数(第二卷)》(席南华) P163 定义4.4
设\(V,V'\)都是域\(F\)上的线性空间,
\(A,A'\)分别是与线性空间\(V,V'\)关联的仿射空间.
如果从\(A\)到\(A'\)的一个仿射映射\(f\)是双射,
则称\(f\)是“从\(A\)到\(A'\)的一个\DefineConcept{仿射同构}”.
如果存在从\(A\)到\(A'\)的一个仿射同构,
则称“仿射空间\(A\)与\(A'\) \DefineConcept{同构}”,
记为\(A \Isomorphism A'\).
\end{definition}

\begin{definition}%\label{definition:仿射空间.仿射自同构}
%@see: 《基础代数(第二卷)》(席南华) P163 定义4.4
设\(V\)都是域\(F\)上的线性空间,
\(A\)分别是与线性空间\(V\)关联的仿射空间,
\(f\)是\(A\)上的一个仿射变换.
如果\(f\)是双射,
则称\(f\)是“\(A\)上的一个\DefineConcept{仿射自同构}”.
\end{definition}

\begin{proposition}%\label{theorem:仿射空间.仿射映射是双射当且仅当它的微分是双射}
%@see: 《基础代数(第二卷)》(席南华) P164 命题4.7
设\(V,V'\)都是域\(F\)上的线性空间,
\(A,A'\)分别是与线性空间\(V,V'\)关联的仿射空间,
映射\(f\colon A \to A'\),
则仿射映射\(f\)是双射,当且仅当\(f\)的微分\(\dd{f}\)是双射.
%TODO proof
\end{proposition}

\begin{proposition}\label{theorem:仿射空间.仿射空间与线性空间同构}
设\(V\)是域\(F\)上的线性空间,
\(A\)是与线性空间\(V\)关联的仿射空间,
则\(A\)与\(V\)同构.
\begin{proof}
在\cref{theorem:仿射空间.仿射空间与线性空间等势} 证明过程中,
我们已经给出了一个双射\(\sigma\colon A \to V, p \mapsto \vec{op}\),
故\(A \Isomorphism V\).
\end{proof}
\end{proposition}

\begin{proposition}%\label{theorem:仿射空间.两个仿射空间同构当且仅当它们的维数相同}
%@see: 《基础代数(第二卷)》(席南华) P164 定理4.8
设\(A\)和\(A'\)是域\(F\)上的两个仿射空间,
则\(A \Isomorphism A'\)当且仅当\(\dim A = \dim A'\).
\begin{proof}
设\(V,V'\)都是域\(F\)上的线性空间,
\(A,A'\)分别是与线性空间\(V,V'\)关联的仿射空间.
由\cref{theorem:仿射空间.仿射空间与线性空间同构} 可知,
\(A\)与\(V\)同构,\(A'\)与\(V'\)同构,
于是\(A\)与\(A'\)同构,当且仅当\(V\)与\(V'\)同构.
由\cref{theorem:线性空间的同构.线性空间同构的充分必要条件} 可知,
\(V\)与\(V'\)同构,当且仅当\(\dim V = \dim V'\).
因此\(A\)与\(A'\)同构,当且仅当\(\dim A = \dim A'\).
\end{proof}
\end{proposition}

\subsection{仿射空间中的坐标}
%@see: 《基础代数(第二卷)》(席南华) P164
把一个点\(o \in A\)与\(V\)的一个基\(\AutoTuple{e}{n}\)合在一起
称为“仿射空间\(A\)的一个\DefineConcept{坐标系}”
或“仿射空间\(A\)的一个\DefineConcept{标架}”,
记作\([o;\AutoTuple{e}{n}]\).
把\(o\)称为“标架\([o;\AutoTuple{e}{n}]\)的\DefineConcept{原点}”.
把向量\(\vec{op}\)在基\(\AutoTuple{e}{n}\)下的坐标
称为“点\(p\)在标架\([o;\AutoTuple{e}{n}]\)下的\DefineConcept{坐标}”.

只要在仿射空间\(A\)中取定一个点\(o\)(称之为原点),
就可以把点\(p\)与向量\(\vec{op}\)等同起来
(正如我们在\cref{theorem:仿射空间.仿射空间与线性空间等势} 证明过程中给出的双射\(\sigma\)那样),
然后把点与向量的加法看作向量的加法.
这种把点视同向量的做法,称为仿射空间的\DefineConcept{向量化}.
当然,这依赖原点的选取.
不过,只要取定\(n\)维仿射空间\(A\)的一个标架,
点与它的坐标的对应就是从仿射空间\(A\)到向量空间\(F^n\)的同构.

\begin{proposition}
%@see: 《基础代数(第二卷)》(席南华) P164 命题4.9
给定仿射空间\(A\)的一个标架\([o;\AutoTuple{e}{n}]\).
\begin{itemize}
	\item 如果\((\AutoTuple{x}{n})\)是点\(p\)的坐标,
	\((\AutoTuple{y}{n})\)是点\(q\)的坐标,
	那么向量\(\vec{pq}\)的坐标就是\((y_1-x_1,\dotsc,y_n-x_n)\).

	\item 给定向量\(v = a_1 e_1 + \dotsb + a_n e_n\),
	如果\((\AutoTuple{x}{n})\)是点\(p\)的坐标,
	那么点\(p + v\)的坐标是\((x_1+a_1,\dotsc,x_n+a_n)\).
\end{itemize}
\end{proposition}

我们经常需要考虑同一个点在不同标架下的坐标之间的联系.
假设\([o;\AutoTuple{e}{n}]\)和\([o';\AutoTuple{e}{n}]\)是仿射空间\(A\)的两个标架,
点\(p\)在这两个标架下的坐标分别是\((\AutoTuple{x}{n})\)和\((\AutoTuple{x'}{n})\),
点\(o'\)在标架\([o;\AutoTuple{e}{n}]\)下的坐标为\(\AutoTuple{b}{n}\),
\(A = (a_{ij})\)是基\(\AutoTuple{e}{n}\)到基\(\AutoTuple{e'}{n}\)的过渡矩阵.
由\begin{align*}
	\sum_i x_i e_i
	&= \vec{op}
	= \vec{oo'} + \vec{o'p}
	= \sum_i b_i e_i + \sum_j x'_j e'_j \\
	&= \sum_i b_i e_i + \sum_j x'_j \sum_i a_{ij} e_i \\
	&= \sum_i \left( \sum_j a_{ij} x'_j \right) e_i
	+ \sum_i b_i e_i
\end{align*}
可知\begin{equation*}
%@see: 《基础代数(第二卷)》(席南华) P165 (4.1.3)
	x_i = \sum_{j=1}^n a_{ij} x'_j + b_i
	\quad(i=1,2,\dotsc,n).
\end{equation*}
上式可以写成矩阵形式:\begin{equation*}
	X = A X' + B,
\end{equation*}
其中\(
	X \defeq (\AutoTuple{x}{n}),
	X' \defeq (\AutoTuple{x'}{n}),
	B \defeq (\AutoTuple{b}{n})
\).
由于\(A\)是可逆矩阵,
所以\begin{equation*}
	X' = A^{-1} (X - B)
	= A^{-1} X - A^{-1} B.
\end{equation*}

\subsection{仿射子空间}
\begin{definition}
%@see: 《基础代数(第二卷)》(席南华) P166 定义4.10
设\(p\)是仿射空间\((A,V,f)\)中的一个点,\(U\)是\(V\)的一个线性子空间.
把集合\begin{equation*}
	\Set{
		f(p,u)
		\given
		u \in U
	}
\end{equation*}
称为“(仿射空间\(A\)的)(过点\(p\)的、以\(U\)为方向的)一个\DefineConcept{仿射子空间}”,
记为\(p + U\).
把\(U\)称为“仿射子空间\(p + U\)的\DefineConcept{方向}”.
把\(U\)的维数\(\dim U\)称为“仿射子空间\(p + U\)的\DefineConcept{维数}”,
记为\(\dim(p + U)\).
将\(A\)的\(0\)维仿射子空间称为\DefineConcept{点}.
将\(A\)的\(1\)维仿射子空间称为\DefineConcept{直线}.
将\(A\)的\(2\)维仿射子空间称为\DefineConcept{平面}.
将\(A\)的\(\dim A-1\)维仿射子空间称为\DefineConcept{超平面}.
\end{definition}

\section{仿射几何}
\subsection{陪集的交与联}
%@see: 《高等代数与解析几何(第三版 下册)》(孟道骥) P435
设\(V\)是域\(F\)上的一个线性空间,
\(\{S_i\}_{i \in I}\)是一个有标集族.
如果\begin{equation*}
	(\forall i \in I)
	(\exists W \leq V)
	[
		\text{$S_i$是$W$的一个陪集}
	],
\end{equation*}
则称“\(\{S_i\}_{i \in I}\)是\(V\)的一个\DefineConcept{陪集族}”.

\begin{proposition}
设\(\{S_i\}_{i \in I}\)是域\(F\)上线性空间\(V\)的一个陪集族,
则\(V\)是包含\(\bigcup_{i \in I} S_i\)的一个陪集.
\begin{proof}
因为\(V\)是一个陪集,
并且\((\forall i \in I)[V \supseteq S_i]\),
所以\(V \supseteq \bigcup_{i \in I} S_i\).
\end{proof}
\end{proposition}

\begin{lemma}\label{theorem:线性空间.商空间.陪集.线性空间中全体陪集对联运算封闭}
%@see: 《高等代数与解析几何(第三版 下册)》(孟道骥) P438 定理10.2.1(1)
设\(V\)是域\(F\)上的一个线性空间,
\(W_1,W_2\)是\(V\)的两个子空间,
向量\(\alpha_1,\alpha_2 \in V\),
则\begin{equation*}
	\Gamma
	\defeq
	\bigcap\Set{
		X
		\given
		X \supseteq (\alpha_1 + W_1) \cup (\alpha_2 + W_2),
		\text{$X$是一个陪集}
	}
\end{equation*}
是陪集.
\begin{proof}
设陪集\(\beta + U\)满足\(\beta + U \supseteq (\alpha_1 + W_1) \cup (\alpha_2 + W_2)\).
由\(\beta + U \supseteq \alpha_1 + W_1\)
可得\(\alpha_1 - \beta \in U\)和\(W_1 \subseteq U\),
即\(\Span\{\alpha_1 - \beta\} + W_1 \subseteq U\).
由\(\beta + U \supseteq \alpha_2 + W_2\)
可得\(\alpha_2 - \beta \in U\)和\(W_2 \subseteq U\),
即\(\Span\{\alpha_2 - \beta\} + W_2 \subseteq U\).
%\cref{equation:集合论.集合代数公式6-8}
于是有\begin{equation*}
	\Span\{\alpha_1 - \beta,\alpha_2 - \beta\} + W_1 + W_2 \subseteq U.
\end{equation*}
令\(U_0 \defeq \Span\{\alpha_1 - \alpha_2\} + W_1 + W_2\).
由于\(
	\alpha_1 - \alpha_2
	= (\alpha_1 - \beta) - (\alpha_2 - \beta)
	\in \Span\{\alpha_1 - \beta,\alpha_2 - \beta\}
\),
所以\(
	\Span\{\alpha_1 - \alpha_2\}
	\subseteq
	\Span\{\alpha_1 - \beta,\alpha_2 - \beta\}
\),
从而有\(U \supseteq U_0\).
因此\(
	\alpha_1 + U_0
	\subseteq
	\alpha_1 + U
	= \beta + U
\),
这就说明\(\alpha_1 + U_0\)是包含\((\alpha_1 + W_1) \cup (\alpha_2 + W_2)\)的每一个陪集的子集,
于是由\cref{equation:集合论.集合代数公式6-9}
可知\begin{equation*}
	\alpha_1 + U_0
	\subseteq
	\Gamma.
\end{equation*}

由\(U_0 \supseteq W_1\)
可得\(\alpha_1 + W_1 \subseteq \alpha_1 + U_0\).
由\(U_0 \supseteq \Span\{\alpha_1 - \alpha_2\}\)
可得\(
	\alpha_2
	= \alpha_1 - (\alpha_1 - \alpha_2)
	\in \alpha_1 + \Span\{\alpha_1 - \alpha_2\}
	\subseteq \alpha_1 + U_0
\).
由\(\alpha_2 \in \alpha_1 + U_0\)
可得\(\alpha_2 + U_0 = \alpha_1 + U_0\).
由\(U_0 \supseteq W_2\)
可得\(
	\alpha_2 + W_2
	\subseteq
	\alpha_2 + U_0
	= \alpha_1 + U_0
\).
因此\begin{equation*}
	(\alpha_1 + W_1) \cup (\alpha_2 + W_2)
	\subseteq
	\alpha_1 + U_0.
\end{equation*}
由上可知\(\alpha_1 + U_0\)是包含\((\alpha_1 + W_1) \cup (\alpha_2 + W_2)\)的一个陪集,
即\begin{equation*}
	\alpha_1 + U_0
	\in
	\Set{
		X
		\given
		X \supseteq (\alpha_1 + W_1) \cup (\alpha_2 + W_2),
		\text{$X$是一个陪集}
	},
\end{equation*}
于是由\cref{equation:集合论.集合代数公式6-5}
可知\begin{equation*}
	\Gamma
	\subseteq
	\bigcap\{\alpha_1 + U_0\}
	=
	\alpha_1 + U_0.
\end{equation*}

综上所述,\(\Gamma = \alpha_1 + U_0\).
\end{proof}
\end{lemma}

\begin{definition}
%@see: 《高等代数与解析几何(第三版 下册)》(孟道骥) P436
设\(S_1\)和\(S_2\)是\(V\)中两个陪集.
把包含\(\mathscr{S} \defeq S_1 \cup S_2\)的所有陪集之交\begin{equation*}
	\bigcap\Set{
		X
		\given
		X \supseteq \mathscr{S},
		\text{$X$是一个陪集}
	}
\end{equation*}
称为“陪集\(S_1\)和\(S_2\)的\DefineConcept{联}”,
记作\(S_1 \vee S_2\).
\end{definition}

\begin{definition}
%@see: 《高等代数与解析几何(第三版 下册)》(孟道骥) P436 定义10.1.1
设\(\{S_i\}_{i \in I}\)是\(V\)中一个陪集族.
把包含\(\mathscr{S} \defeq \bigcup_{i \in I} S_i\)的所有陪集之交\begin{equation*}
	\bigcap\Set{
		X
		\given
		X \supseteq \mathscr{S},
		\text{$X$是一个陪集}
	}
\end{equation*}
称为“陪集族\(\{S_i\}_{i \in I}\)的\DefineConcept{联}”,
记作\(\bigvee_{i \in I} S_i\).
\end{definition}

\cref{theorem:线性空间.商空间.陪集.线性空间中全体陪集对联运算封闭} 说明:
线性空间\(V\)中全体陪集\(
	\mathscr{C}
	\defeq
	\Set{
		\alpha + W
		\given
		\alpha \in V,
		W \leq V
	}
\)对“陪集的联”运算封闭.

\cref{theorem:线性空间.商空间.陪集.线性空间中全体陪集对联运算封闭} 还说明:\begin{align}
	\bigvee_{i=1}^2 (\alpha_i + W_i)
	&= \alpha_1 + \Span\{\alpha_1 - \alpha_2\} + W_1 + W_2,
		\label{equation:仿射几何.两个陪集的联}
		\\
	\bigvee_{i=1}^3 (\alpha_i + W_i)
	&= \alpha_1 + \Span\{\alpha_1 - \alpha_2,\alpha_1 - \alpha_3\} + W_1 + W_2 + W_3,
		% \label{equation:仿射几何.三个陪集的联}
		\\
	\bigvee_{i=1}^n (\alpha_i + W_i)
	&= \alpha_1 + \Span\{\alpha_1 - \alpha_2,\dotsc,\alpha_1 - \alpha_n\}
		+ \sum_{i=1}^n W_i.
		% \label{equation:仿射几何.n个陪集的联}
\end{align}

\begin{property}
陪集的联运算适合交换律,即对于任意陪集\(x,y\),成立\begin{equation}
	x \vee y = y \vee z.
\end{equation}
\begin{proof}
显然\(
	x \vee y
	= \bigcap\Set{
		c \in \mathscr{C}
		\given
		c \supseteq x \cup y
	}
	= \bigcap\Set{
		c \in \mathscr{C}
		\given
		c \supseteq y \cup x
	}
	= y \vee x
\).
\end{proof}
\end{property}

\begin{property}
%@see: 《高等代数与解析几何(第三版 下册)》(孟道骥) P436
陪集的联运算适合结合律,即对于任意陪集\(x,y,z\),成立\begin{equation}
	(x \vee y) \vee z = x \vee (y \vee z).
\end{equation}
\begin{proof}
任取\(x,y,z \in \mathscr{C}\).
由定义可知\(
	x \vee y
	\supseteq
	x \cup y
\),
于是,对于任意\(c \in \mathscr{C}\),
当\(c \supseteq (x \vee y) \cup z\)时,
必有\(c \supseteq (x \cup y) \cup z\).
因此\begin{equation*}
	(x \vee y) \vee z
	= \bigcap\Set{
		c \in \mathscr{C}
		\given
		c \supseteq (x \vee y) \cup z
	}
	\supseteq
	\bigcap\Set{
		c \in \mathscr{C}
		\given
		c \supseteq (x \cup y) \cup z
	}.
\end{equation*}
与此同时,由\(x \cup y \subseteq (x \cup y) \cup z\)可知,
包含\((x \cup y) \cup z\)的任意一个陪集\(c\)必定包含\(x \cup y\),
由定义可知这样的\(c\)必定包含\(x \vee y\).
于是当\(c \supseteq (x \cup y) \cup z\)时,
必有\(c \supseteq (x \vee y) \cup z\).
因此\begin{equation*}
	(x \vee y) \vee z
	= \bigcap\Set{
		c \in \mathscr{C}
		\given
		c \supseteq (x \vee y) \cup z
	}
	\subseteq
	\bigcap\Set{
		c \in \mathscr{C}
		\given
		c \supseteq (x \cup y) \cup z
	}.
\end{equation*}
综上所述\begin{equation*}
	(x \vee y) \vee z
	= \bigcap\Set{
		c \in \mathscr{C}
		\given
		c \supseteq (x \cup y) \cup z
	}.
\end{equation*}
利用对称性可得\begin{equation*}
	x \vee (y \vee z)
	= \bigcap\Set{
		c \in \mathscr{C}
		\given
		c \supseteq x \cup (y \cup z)
	},
\end{equation*}
于是\((x \vee y) \vee z = x \vee (y \vee z)\).
\end{proof}
\end{property}

通过下面的例子可以看出“陪集的联”这个运算的几何意义.
\begin{example}
%@see: 《高等代数与解析几何(第三版 下册)》(孟道骥) P436 例10.1
设\(P,Q\)是几何空间\(V \defeq \mathbb{R}^3\)中的两个点,它们的坐标向量分别是\(\alpha,\beta\).
显然\(\{\alpha\}\)是以\(\alpha\)为代表的零空间的陪集,
\(\{\beta\}\)是以\(\beta\)为代表的零空间的陪集.
假设\(V\)的子空间\(W\)满足\(
	\{\alpha\} \vee \{\beta\}
	= \alpha + W
	= \beta + W
\),
那么\(\alpha - \beta \in W\).
显然\(U \defeq \Span\{\alpha - \beta\}\)是\(V\)的一个子空间,
并且\(\alpha + U = \beta + U\)就是包含\(\alpha\)和\(\beta\)的一个陪集,
因此\(W = U\).
这表明\(\{\alpha\} \vee \{\beta\}\)是过\(P,Q\)两点的直线.
\end{example}

\begin{example}
%@see: 《高等代数与解析几何(第三版 下册)》(孟道骥) P436 例10.2
%@see: https://math.stackexchange.com/q/5086322/591741
在几何空间\(V \defeq \mathbb{R}^3\)中,
取两个陪集\(
	S_1 \defeq \alpha + 0,
	S_2 \defeq \beta + U
\),
其中\(U\)是\(V\)的一个1维子空间,
且向量\(\alpha,\beta \in V\),
但\(\alpha \notin S_2\).
显然\(S_1\)是一个点,\(S_2\)是一条直线.
假设\(V\)的子空间\(W_1\)满足\(
	S_1 \vee S_2
	= \alpha + W_1
	= \beta + W_1
\),
那么\(\alpha - \beta \in W_1\)
且\(U \subseteq W_1\),
于是\(
	W_2
	\allowbreak
	\defeq \Span\{\alpha - \beta\} + U
	\allowbreak
	\subseteq W_1
\).
显然\(\alpha + W_2 = \beta + W_2\)是包含\(S_1\)和\(S_2\)的一个陪集,
因此\(W_2 = W_1\).
由于\(\alpha \notin S_2\),
所以\(\alpha - \beta \notin U\),
因此\(\dim W_1 = 2\).
这表明\(S_1 \vee S_2\)是过点\(S_1\)与直线\(S_2\)的平面.
\end{example}

\begin{example}
%@see: 《高等代数与解析几何(第三版 下册)》(孟道骥) P437 习题 2.
设\(V\)是域\(F\)上的一个线性空间,\(W\)是\(V\)的一个子空间,
向量\(\alpha \in W\).
证明:\((\alpha + W) \vee \{0\} = \Span\{\alpha\} + W\).
%TODO proof
\end{example}

\subsection{陪集的维数}
\begin{definition}
%@see: 《高等代数与解析几何(第三版 下册)》(孟道骥) P437 定义10.1.2
设\(V\)是域\(F\)上的一个线性空间,\(W\)是\(V\)的一个子空间,
向量\(\alpha \in V\).
定义:\begin{equation}
	\dim(\alpha + W)
	\defeq
	\dim W,
\end{equation}
称之为“陪集\(\alpha + W\)的\DefineConcept{维数}(the \emph{dimension} of \(\alpha + W\))”.
\end{definition}

\begin{example}
零陪集(即域\(F\)上线性空间\(V\)中以零向量\(0\)为代表的零子空间\(0\)的陪集\(0 \defeq 0+0\))的
维数为\(\dim 0 = 0\).
\end{example}

\subsection{仿射几何}
\begin{definition}
%@see: 《高等代数与解析几何(第三版 下册)》(孟道骥) P437 定义10.1.3
设\(V\)是域\(F\)上的一个线性空间,
\(S\)是\(V\)中一个陪集.
把\(S\)中全体陪集\begin{equation*}
	\Set{
		\alpha + W
		\given
		\alpha \in V,
		W \leq V,
		\alpha + W \subseteq S
	}
\end{equation*}
称为“\(S\)上的\DefineConcept{仿射几何}(affine geometry)”,
记作\(\mathcal{A}(S)\);
把陪集\(S\)的维数\(\dim S\)
称为“\(\mathcal{A}(S)\)的\DefineConcept{维数}”,
记作\(\dim\mathcal{A}(S)\);
将\(\mathcal{A}(S)\)中\(0\)维元素称为\DefineConcept{点};
将\(\mathcal{A}(S)\)中\(1\)维元素称为\DefineConcept{直线};
将\(\mathcal{A}(S)\)中\(2\)维元素称为\DefineConcept{平面};
将\(\mathcal{A}(S)\)中\(\dim S-1\)维元素称为\DefineConcept{超平面}.
\end{definition}
\begin{remark}
应该注意到,“线性空间\(V\)中某个陪集的仿射几何”
与“线性空间\(V\)对于某个子空间的商空间”是完全不同的两个概念.
\end{remark}

\begin{definition}
%@see: 《高等代数与解析几何(第三版 下册)》(孟道骥) P437 定义10.1.3
设\(V\)是域\(F\)上的一个线性空间,
\(R\)和\(S\)是\(V\)中两个陪集.
如果\(R \subseteq S\),
则称“\(\mathcal{A}(R)\)是\(\mathcal{A}(S)\)的\DefineConcept{子几何}”.
\end{definition}

\subsection{陪集之间的平行关系}
\begin{definition}
%@see: 《高等代数与解析几何(第三版 下册)》(孟道骥) P437 定义10.2.1
设\(V\)是域\(F\)上的一个线性空间,
\(U\)和\(W\)是\(V\)的两个子空间,
向量\(\alpha,\beta \in V\).
如果\(U \subseteq W\)或\(W \subseteq U\),
则称“陪集\(\alpha + U\)与\(\beta + W\) \DefineConcept{平行}”,
记作\((\alpha + U) \parallel (\beta + W)\);
反之,称“陪集\(\alpha + U\)与\(\beta + W\) \DefineConcept{不平行}”,
记作\((\alpha + U) \nparallel (\beta + W)\).
\end{definition}

\begin{property}
%@see: 《高等代数与解析几何(第三版 下册)》(孟道骥) P437
陪集之间的平行关系具有自反性,
即对于线性空间\(V\)中任意一个陪集\(x\),
有\(x \parallel x\).
\end{property}

\begin{property}
%@see: 《高等代数与解析几何(第三版 下册)》(孟道骥) P437
陪集之间的平行关系具有对称性,
即对于线性空间\(V\)中任意两个陪集\(x\)和\(y\),
当\(x \parallel y\)时,必有\(y \parallel x\).
\end{property}

\begin{property}
%@see: 《高等代数与解析几何(第三版 下册)》(孟道骥) P438
陪集之间的平行关系不具有传递性.
\begin{proof}
设\(V \defeq \mathbb{R}^3\).
取\begin{gather*}
	\alpha_1 \defeq (0,0,1)^T,
	\qquad
	\alpha_2 \defeq (0,0,2)^T, \\
	W_1 \defeq \Span\{(1,0,0)^T\},
	\qquad
	W_2 \defeq \Span\{(0,1,0)^T\},
	\qquad
	W \defeq W_1 + W_2, \\
	S_1 \defeq \alpha_1 + W_1,
	\qquad
	S_2 \defeq \alpha_2 + W_2,
	\qquad
	S_3 \defeq W,
\end{gather*}
则有\(
	S_1 \parallel S_3,
	S_2 \parallel S_3
\),
但是\(S_1 \nparallel S_2\).
\end{proof}
\end{property}

\subsection{仿射性质}
\begin{theorem}\label{theorem:仿射几何.陪集的交非空的充分必要条件1}
%@see: 《高等代数与解析几何(第三版 下册)》(孟道骥) P438 定理10.2.1(2)
设\(V\)是域\(F\)上的一个线性空间,
\(U\)和\(W\)是\(V\)的两个子空间,
向量\(\alpha,\beta \in V\),
则\((\alpha + U) \cap (\beta + W) \neq \emptyset\)
当且仅当\(\alpha - \beta \in U + W\).
\begin{proof}
\((\alpha + U) \cap (\beta + W) \neq \emptyset\)
当且仅当存在\(\gamma \in U\)和\(\delta \in W\),
使得\(\alpha + \gamma = \beta + \delta\),
即\(\beta - \alpha = \gamma - \delta \in U + W\).
\end{proof}
\end{theorem}

\begin{theorem}
%@see: 《高等代数与解析几何(第三版 下册)》(孟道骥) P438 定理10.2.1(3)
设\(V\)是域\(F\)上的一个线性空间,
\(U\)和\(W\)是\(V\)的两个子空间,
向量\(\alpha,\beta \in V\),
则\((\alpha + U) \cap (\beta + W) = \emptyset\)
当且仅当\(\dim((\alpha + U) \vee (\beta + W)) = \dim(U + W) + 1\).
%TODO proof
\end{theorem}

\begin{theorem}
%@see: 《高等代数与解析几何(第三版 下册)》(孟道骥) P439 定理10.2.2(1)
设\(V\)是域\(F\)上的一个线性空间,
\(U\)和\(W\)是\(V\)的两个子空间,
向量\(\alpha,\beta \in V\).
如果\(\alpha + U \subseteq \beta + W\),
则\(\dim(\alpha + U) \leq \dim(\beta + W)\).
%TODO proof
\end{theorem}

\begin{theorem}
%@see: 《高等代数与解析几何(第三版 下册)》(孟道骥) P439 定理10.2.2(1)
设\(V\)是域\(F\)上的一个线性空间,
\(U\)和\(W\)是\(V\)的两个子空间,
向量\(\alpha,\beta \in V\).
如果\(\alpha + U \subseteq \beta + W\),
且\(\dim(\alpha + U) = \dim(\beta + W)\),
则\(\alpha + U = \beta + W\).
%TODO proof
\end{theorem}

\begin{theorem}
%@see: 《高等代数与解析几何(第三版 下册)》(孟道骥) P439 定理10.2.2(2)
设\(V\)是域\(F\)上的一个线性空间,
\(U\)和\(W\)是\(V\)的两个子空间,
向量\(\alpha,\beta \in V\).
如果\((\alpha + U) \cap (\beta + W) \neq \emptyset\),
则\begin{equation*}
	\dim((\alpha + U) \vee (\beta + W))
	+ \dim((\alpha + U) \cap (\beta + W))
	= \dim(\alpha + U)
	+ \dim(\beta + W).
\end{equation*}
%TODO proof
\end{theorem}

\begin{theorem}
%@see: 《高等代数与解析几何(第三版 下册)》(孟道骥) P439 定理10.2.2(3)
设\(V\)是域\(F\)上的一个线性空间,
\(U\)和\(W\)是\(V\)的两个子空间,
向量\(\alpha,\beta \in V\).
如果\((\alpha + U) \cap (\beta + W) \neq \emptyset\),
则\((\alpha + U) \parallel (\beta + W)\)
当且仅当\(\alpha + U \subseteq \beta + W\)
或\(\alpha + U \supseteq \beta + W\).
%TODO proof
\end{theorem}

\begin{theorem}
%@see: 《高等代数与解析几何(第三版 下册)》(孟道骥) P439 定理10.2.2(3)
设\(V\)是域\(F\)上的一个线性空间,
\(U\)和\(W\)是\(V\)的两个子空间,
向量\(\alpha,\beta \in V\).
如果\((\alpha + U) \cap (\beta + W) = \emptyset\),
则\((\alpha + U) \parallel (\beta + W)\)
当且仅当\begin{equation*}
	\dim((\alpha + U) \vee (\beta + W))
	= \max\{\dim(\alpha + U),\dim(\beta + W)\} + 1.
\end{equation*}
%TODO proof
\end{theorem}
\begin{remark}
%@see: 《高等代数与解析几何(第三版 下册)》(孟道骥) P440 推论1
%@see: 《高等代数与解析几何(第三版 下册)》(孟道骥) P440 推论2
从上述定理可以看出:
在2维仿射几何中,
两个不同点的联是一条直线,
两条非平行直线的交是一个点;
在3维仿射几何中,
两个不同点的联是一条直线,
两个非平行平面的交是一条直线,
交于一点的两条直线的联是一个平面,
两条共面的非平行直线的交是一个点,
两条不同的平行直线的联是一个平面,
一个点与不包含它的一条直线的联是一个平面,
一个平面与一个不平行于它的直线的交是一个点.
\end{remark}

\begin{example}
%@see: 《高等代数与解析几何(第三版 下册)》(孟道骥) P441 习题 2.(1)
设\(x\)和\(y\)是3维仿射几何中两条异面直线.
试证:存在唯一的平面\(p\)使得\(x \subseteq p\)且\(y \parallel p\).
%TODO proof
\end{example}

\begin{example}
%@see: 《高等代数与解析几何(第三版 下册)》(孟道骥) P441 习题 2.(2)
设\(x\)和\(y\)是3维仿射几何中两条异面直线.
试证:存在唯一的平面\(p\)使得\(y \subseteq p\)且\(x \parallel p\).
%TODO proof
\end{example}

\begin{example}
%@see: 《高等代数与解析几何(第三版 下册)》(孟道骥) P441 习题 2.(3)
设\(x\)和\(y\)是3维仿射几何中两条异面直线,
平面\(p_x\)满足\(x \subseteq p_x\)且\(y \parallel p_x\),
平面\(p_y\)满足\(y \subseteq p_y\)且\(x \parallel p_y\).
试证:\(p_x \parallel p_y\).
%TODO proof
\end{example}

\begin{definition}
%@see: 《高等代数与解析几何(第三版 下册)》(孟道骥) P441 定义10.3.1
设\(A\)与\(A'\)都是仿射几何,
\(\sigma\)是从\(A\)到\(A'\)的一个双射.
如果\begin{equation*}
	(\forall S_1,S_2 \in A)
	[
		\sigma(S_1) \subseteq \sigma(S_2)
		\iff
		S_1 \subseteq S_2
	],
\end{equation*}
那么称“\(\sigma\)是从\(A\)到\(A'\)的一个\DefineConcept{同构}(isomorphism)”
“\(A\)到\(A'\)同构(\(A\) is \emph{isomorphic} to \(A'\))”,
记作\(A \Isomorphism A'\).
\end{definition}

%@see: 《高等代数与解析几何(第三版 下册)》(孟道骥) P441
显然,仿射几何的同构关系是一个等价关系.

%@see: 《高等代数与解析几何(第三版 下册)》(孟道骥) P441 性质1
仿射几何的同构保持交的运算,即\begin{equation*}
	\sigma\left( \bigcap_i S_i \right)
	= \bigcap_i \sigma(S_i).
\end{equation*}

%@see: 《高等代数与解析几何(第三版 下册)》(孟道骥) P441 性质2
仿射几何的同构保持联的运算,即\begin{equation*}
	\sigma\left( \bigvee_i S_i \right)
	= \bigcap_i \sigma(S_i).
\end{equation*}

%@see: 《高等代数与解析几何(第三版 下册)》(孟道骥) P441 性质3
仿射几何的同构保持维数不变,即\begin{equation*}
	\dim\sigma(S) = \dim S.
\end{equation*}

%@see: 《高等代数与解析几何(第三版 下册)》(孟道骥) P441 性质4
仿射几何的同构保持平行关系,即\begin{equation*}
	\sigma(S_1) \parallel \sigma(S_2)
	\iff
	S_1 \parallel S_2.
\end{equation*}

\section{射影几何}
本节利用构造法描述射影几何中的几何元素.

\subsection{射影几何}
\begin{definition}
%@see: 《高等代数与解析几何(第三版 下册)》(孟道骥) P452 定义10.5.1
设\(V\)是域\(F\)上的一个线性空间.
把\(V\)的全体子空间\begin{equation*}
	\Set{
		W
		\given
		W \AlgebraSubstructure V
	}
\end{equation*}
称为“域\(F\)上线性空间\(V\)上的\DefineConcept{射影几何}(projective geometry)”,
记为\(\mathcal{P}(V)\).
%@see: https://mathworld.wolfram.com/ProjectiveGeometry.html
\end{definition}

\begin{definition}
%@see: 《高等代数与解析几何(第三版 下册)》(孟道骥) P452 定义10.5.2
设\(W\)是线性空间\(V\)的一个子空间.
把\((\dim W - 1)\)称为“\(W\)的\DefineConcept{射影维数}”,
记为\(\pdim W\),
即\begin{equation*}
	\pdim W \defeq \dim W - 1.
\end{equation*}

将\(\mathcal{P}(V)\)中\(0\)维元素称为\DefineConcept{射影点}.

将\(\mathcal{P}(V)\)中\(1\)维元素称为\DefineConcept{射影直线}.

将\(\mathcal{P}(V)\)中\(2\)维元素称为\DefineConcept{射影平面}.

将\(\mathcal{P}(V)\)中\((\pdim V-1)\)维元素称为\DefineConcept{射影超平面}.
\end{definition}

\begin{definition}
%@see: 《高等代数与解析几何(第三版 下册)》(孟道骥) P452 定义10.5.2
设\(W\)是线性空间\(V\)的一个子空间.
如果\(W\)是射影点,
则把\(W\)中的每一个非零向量称为“射影点\(W\)的一个\DefineConcept{齐性向量}”.
\end{definition}

\begin{definition}
%@see: 《高等代数与解析几何(第三版 下册)》(孟道骥) P452 定义10.5.2
设\(V\)是域\(F\)上的一个线性空间.
把射影几何\(\mathcal{P}(V)\)中全体射影点\begin{equation*}
	\Set{
		W \AlgebraSubstructure V
		\given
		\pdim W = 0
	}
\end{equation*}
称为“由射影几何\(\mathcal{P}(V)\)决定的\DefineConcept{射影空间}”.
\end{definition}

\begin{property}
%@see: 《高等代数与解析几何(第三版 下册)》(孟道骥) P452
线性空间\(V\)的零子空间\(0\)不在由\(\mathcal{P}(V)\)决定的射影空间中.
%TODO proof
\end{property}

\begin{definition}
%@see: 《高等代数与解析几何(第三版 下册)》(孟道骥) P452
把线性空间\(V\)的零子空间\(0\)
称为“(由\(\mathcal{P}(V)\)决定的)射影空间的\DefineConcept{空子集}”.
\end{definition}

\begin{proposition}
%@see: 《高等代数与解析几何(第三版 下册)》(孟道骥) P452
设\(M,N \in \mathcal{P}(V)\),
则\(M \cap N = 0\)
当且仅当\(\pdim(M \cap N) = -1\).
\end{proposition}

\begin{definition}
%@see: 《高等代数与解析几何(第三版 下册)》(孟道骥) P452
设\(M,N \in \mathcal{P}(V)\).
如果\(M \cap N = 0\),
则称“\(M\)与\(N\)是\DefineConcept{交错的}”.
\end{definition}

%@see: 《高等代数与解析几何(第三版 下册)》(孟道骥) P452
在2维射影几何中,
两个不同点的联是一条直线,
两条不同直线的交是一个点.
在3维射影几何中,
两个不同点的联是一条直线,
两个不同平面的交是一条直线,
两条不同的相交直线的联是一个平面,
两条不同的共面直线的交是一个点,
一个点与不包含该点的一条直线的联是一个平面,
一个平面与不在该平面中的一条直线的交是一个点.

\begin{theorem}
%@see: 《高等代数与解析几何(第三版 下册)》(孟道骥) P453
设\(U\)和\(W\)是线性空间\(V\)的两个子空间,
则\begin{equation*}
	\pdim(U + W)
	+ \pdim (U \cap W)
	= \pdim U
	+ \pdim W.
\end{equation*}
\end{theorem}

\subsection{射影几何与仿射几何的联系}
%@see: 《高等代数与解析几何(第三版 下册)》(孟道骥) P453
假设我们从同一个线性空间\(V\)出发,定义了仿射几何\(\mathcal{A}(V)\)和射影几何\(\mathcal{P}(V)\).
容易看出\(\mathcal{P}(V)\)是\(\mathcal{A}(V)\)的一个子集,
后者由线性空间\(V\)中的所有陪集组成,
前者却只含有那些包含零子空间(或者说经过原点)的陪集.
这种看法是自然而平凡的.
更有趣、更深刻的思想是将仿射几何作为射影几何的一部分,
换句话说,是给仿射几何添上一些东西使其成为射影几何.

\begin{theorem}\label{theorem:射影几何.嵌入定理}
%@see: 《高等代数与解析几何(第三版 下册)》(孟道骥) P453 定理10.5.1(嵌入定理)
设\(V\)是域\(F\)上的一个线性空间,
\(H\)是射影空间\(\mathcal{P}(V)\)中一个超平面,
\(\alpha \in V - H\),
则映射\(
	\phi\colon \mathcal{A}(\alpha + H) \to \mathcal{P}(V),
	S \mapsto \Span\{S\}
\)具有以下性质:\begin{itemize}
	\item \(\phi\)是双射;

	\item \(\mathcal{A}(\alpha + H)\)在\(\phi\)下的像是
	\(\mathcal{P}(H)\)在\(\mathcal{P}(V)\)中的补集,
	即\(\phi(\mathcal{A}(\alpha + H)) = \mathcal{P}(V) - \mathcal{P}(H)\);

	\item 对于任意\(S,T \in \mathcal{A}(\alpha + H)\),
	\(S \subseteq T\)当且仅当\(\phi(S) \subseteq \phi(T)\);

	\item 如果\(\bigcap_i S_i \neq \emptyset\),
	则\(\phi\left( \bigcap_i S_i \right) = \bigcap_i \phi(S_i)\);

	\item \(\phi\left( \bigvee_i S_i \right) = \sum_i S_i\);

	\item 对于任意\(S \in \mathcal{A}(\alpha + H)\),
	有\(\dim S = \pdim \phi(S)\);

	\item 对于任意\(S,T \in \mathcal{A}(\alpha + H)\),
	\(S \parallel T\)当且仅当\begin{equation*}
		\phi(S) \cap H \subseteq \phi(T) \cap H
		\quad\text{或}\quad
		\phi(T) \cap H \subseteq \phi(S) \cap H.
	\end{equation*}
\end{itemize}
%TODO proof
\end{theorem}
\begin{remark}
下面我们用三维空间给\cref{theorem:射影几何.嵌入定理} 一个形象的几何解释.
假设\(V\)是三维实向量空间,
\(H\)是\(V\)的一个二维子空间.
由于\(\alpha\)不在\(H\)中,
所以\(\alpha + H\)是一个平行于\(H\)但不经过原点\(O\)的平面.
假设\(B\)是\(\alpha + H\)上的一个点,
则\(\phi(B)\)是经过原点\(O\)和点\(B\)的一条直线,
同时\(\phi(B)\)也是\(\mathcal{P}(V)\)中的一个射影点.
假设\(\beta + M\)是\(\alpha + H\)中的一条直线(\(\dim M = 1\)),
则\(\phi(\beta + M)\)是经过原点\(O\)和直线\(\beta + M\)的一个平面,
同时\(\phi(\beta + M)\)是\(\mathcal{P}(V)\)中的一条射影直线.
假设\(\beta + M, \gamma + M\)是\(\alpha + H\)中两条不同的直线,
显然它们是平行的,它们的交是空集,
但是它们在\(\phi\)下的像的交是\(\mathcal{P}(V)\)中的一个射影点\(M\).
\end{remark}

