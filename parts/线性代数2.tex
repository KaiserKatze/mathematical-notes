\chapter{多项式环}
\section{多项式}
\begin{definition}\label{definition:多项式.多项式的定义}
设\(K\)是一个数域,
\(x\)是一个不属于\(K\)的符号.
任意给定一个非负整数\(n\),
在\(K\)中任意取定\(\AutoTuple{a}[0]{n}\),
如果表达式\begin{equation}\label[polynomial]{equation:多项式.多项式}
	a_n x^n + a_{n-1} x^{n-1} + \dotsb + a_1 x + a_0
\end{equation}
满足\begin{enumerate}
	\item 两个这种形式的表达式相等当且仅当它们除了系数为零的项以外含有完全相同的项,
	即\begin{align*}
		&a_n x^n + a_{n-1} x^{n-1} + \dotsb + a_1 x + a_0
		= b_n y^n + b_{n-1} y^{n-1} + \dotsb + b_1 y + b_0 \\
		&\iff
		(\forall i\in\Set{0,1,\dotsc,n})[a_i = b_i = 0 \lor a_i x^i = b_i y^i].
	\end{align*}
	\item 允许从表达式中删去系数为零的项,也允许向表达式中添进系数为零的项,
\end{enumerate}
那么称之为“数域\(K\)上的一个一元\DefineConcept{多项式}(polynomial)”,
把\(x\)称为\DefineConcept{不定元}.
%@see: 《Linear Algebra Done Right (Fourth Eidition)》(Sheldon Axler) P30 2.10
\end{definition}

系数全为零的多项式称为\DefineConcept{零多项式}.
在\cref{equation:多项式.多项式} 中,
把\(a_i x^i\)称为“\(i\)次\DefineConcept{项}”,
把\(a_0\)称为\DefineConcept{零次项}或\DefineConcept{常数项}.

从\cref{definition:多项式.多项式的定义} 知道,
数域\(K\)上两个一元多项式相等当且仅当它们的同次项的系数都相等.

%@see: 《Linear Algebra Done Right (Fourth Eidition)》(Sheldon Axler) P31 2.11
设\(f(x)\)表示\cref{equation:多项式.多项式}.
如果\(a_n\neq0\),
则称“\(a_n x^n\)是多项式\(f(x)\)的\DefineConcept{首项}”;
并称“\(f(x)\)的\DefineConcept{次数}是\(n\)”
或“\(f(x)\)是\(n\)次多项式”,
记作\(\deg f\).

零多项式的次数定义为\(-\infty\),并规定:\begin{gather*}
	(-\infty)+(-\infty)=-\infty, \\
	(-\infty)+n=-\infty, \\
	-\infty<n,
\end{gather*}
其中\(n\)是任意非负整数.

零次多项式总是一个常数\(a\),它满足\(a \in K \land a \neq 0\).

我们把数域\(K\)上的所有一元多项式组成的集合记作\(K[x]\).
我们可以在\(K[x]\)中规定“加法”与“乘法”运算.
设\[
	f(x) = \sum_{i=0}^n a_i x^i, \qquad
	g(x) = \sum_{i=0}^n b_i x^i,
\]
如果\(n \ge m\),那么\begin{gather}
	f(x) + g(x) \defeq \sum_{i=0}^n (a_i+b_i) x^i, \\
	f(x) \cdot g(x) \defeq \sum_{s=0}^{n+m} \left( \sum_{i+j=s} a_i b_j \right) x^s,
\end{gather}
我们把\(f(x)+g(x)\)称为“\(f(x)\)与\(g(x)\)的\DefineConcept{和}”,
把\(f(x) \cdot g(x)\)称为“\(f(x)\)与\(g(x)\)的\DefineConcept{积}”.

容易验证上面所定义的多项式的加法与乘法满足下列运算法则:
\begin{enumerate}
	\item 加法交换律,即\[
		(\forall f,g \in K[x])[f+g=g+f].
	\]

	\item 加法结合律,即\[
		(\forall f,g,h \in K[x])[(f+g)+h=f+(g+h)].
	\]

	\item 零多项式\(0\)是加法单位元,即\[
		(\forall f \in K[x])[0+f=f+0=f].
	\]

	\item \(K[x]\)具有负元.

	设\(f(x)=\sum_{i=0}^n a_i x^i\),
	定义\(-f(x)=\sum_{i=0}^n (-a_i) x^i\),则\[
		f+(-f)=0.
	\]
	称\(-f\)为\(f\)的\DefineConcept{负元}.

	\item 乘法交换律,即\[
		(\forall f,g \in K[x])[fg=gf].
	\]

	\item 乘法结合律,即\[
		(\forall f,g,h \in K[x])[(fg)h=f(gh)].
	\]

	\item 零次多项式\(1\)是乘法单位元,即\[
		1f=f1=f.
	\]

	\item 乘法对加法的分配律,即\[
		(\forall f,g,h \in K[x])[f(g+h)=fg+fh],
	\]\[
		(\forall f,g,h \in K[x])[(g+h)f=gf+hf].
	\]

	\item 乘法消去律,即\[
		fg=fh \land f\neq0 \implies g=h.
	\]
\end{enumerate}

多项式的减法定义如下:\begin{equation}
	f-g \defeq f+(-g).
\end{equation}

%@see: 《Linear Algebra Done Right (Fourth Eidition)》(Sheldon Axler) P31 2.12
我们把数域\(K\)上的所有次数不超过\(m\ (m\geq0)\)的一元多项式组成的集合记作\(K[x]_m\).

\begin{proposition}
%@see: 《高等代数(第三版 下册)》(丘维声) P3 命题1
设\(f,g \in K[x]\),则\begin{gather}
	\deg(f \pm g) \leq \max\{\deg f, \deg g\},
	\label{equation:多项式.和的次数} \\
	\deg(fg) = \deg f + \deg g.
	\label{equation:多项式.积的次数}
\end{gather}
\begin{proof}
如果\(f=0\)或\(g=0\),
则上述两式显然成立.
下面假设\[
	f(x)
	= \sum_{i=0}^n a_i x^i
	\neq0, \qquad
	g(x)
	= \sum_{i=0}^m b_i x^i
	\neq0,
\]
其中\(a_n\neq0,b_m\neq0\).
于是\(\deg f=n,
\deg g=m\).
不妨设\(n \geq m\).
根据定义\[
	f(x) \pm g(x)
	= \sum_{i=0}^n (a_i \pm b_i) x^i,
\]
因此\[
	\deg(f \pm g)
	\leq n
	= \max\{
		\deg f,
		\deg g
	\}.
\]

又因为\(a_n b_m \neq 0\),
因此\(a_n b_m x^{n+m}\)是\(f(x) g(x)\)的首项,
从而\[
	\deg(fg) = n+m = \deg f + \deg g.
	\qedhere
\]
\end{proof}
\end{proposition}

在\(K[x]\)中,我们根据多项式的乘法的定义,以及多项式的加法和乘法满足的运算法则,有
\begin{equation}\label{equation:多项式.示例公式1}
	(2x+3)(x+5)
	=2x^2+10x+3x+15
	=2x^2+13x+15.
\end{equation}

设\(\A \in M_n(K)\),在\(K[\A]\)中,根据矩阵乘法的定义及其分配律等运算法则,有
\begin{equation}\label{equation:多项式.示例公式2}
	(2\A+3\E)(\A+5\E)
	=2\A^2+10\A\E+3\E\A+(3\E)(5\E)
	=2\A^2+13\A+15\E.
\end{equation}

以上两式的计算过程类似.
这促使我们设想:
能不能不必进行\cref{equation:多项式.示例公式2} 的计算过程,
而从\cref{equation:多项式.示例公式1}
的计算结果直接得出\cref{equation:多项式.示例公式2} 呢?

\(K[x]\)中所有零次多项式添上零多项式组成的集合\(S\),
对于多项式的减法与乘法封闭,
因此\(S\)是\(K[x]\)的一个子环.
显然\(K[x]\)的单位元\(1\)属于\(S\),
从而\(1\)也是\(S\)的单位元.
我们可以建立数域\(K\)到\(S\)的一个映射\(\sigma\):
让非零数\(a\)对应于零次多项式\(a\),
让数\(0\)对应到零次多项式,
可以证明\(\sigma\)是一个同构映射.

给定\(\A\in M_n(K)\),
形如\[
	a_m \A^m + a_{m-1} \A^{m-1} + \dotsb + a_1 \A + a_0 \E
\]的表达式称为“数域\(K\)上矩阵\(\A\)的多项式”,
其中\(m\)是非负整数,
\(\E\)是\(n\)阶单位矩阵,
\(a_i \in K\ (i=0,1,\dotsc,m)\).
把数域\(K\)上矩阵\(\A\)的所有多项式组成的集合记作\(K[\A]\),即\[
	K[\A]
	\defeq
	\Set{
		a_m \A^m + a_{m-1} \A^{m-1} + \dotsb + a_1 \A + a_0 \E
		\given
		m \in \mathbb{N}
		\land
		a_i \in K\ (i=0,1,\dotsc,m)
	}.
\]

设\(f(\A)=\sum_{i=0}^m a_i \A^i\),
\(g(\A)=\sum_{i=0}^n b_i \A^i\),
\(m \geq n\).
从矩阵的运算法则可以得到
\begin{gather}
	f(\A)-g(\A)
	= \sum_{i=0}^m (a_i - b_i) \A^i, \\
	f(\A) g(\A)
	= \sum_{s=0}^{m+n} \left( \sum_{i+j=s} a_i b_j \right) \A^s.
	\label{equation:多项式.矩阵多项式.乘法}
\end{gather}
因此\(K[\A]\)是\(M_n(K)\)的一个子环.
从\cref{equation:多项式.矩阵多项式.乘法} 容易看出\[
	f(\A) g(\A) = g(\A) f(\A);
\]
又由于\(\E \in K[\A]\),
所以\(K[\A]\)是有单位元的交换环.

\(K[\A]\)中所有数量矩阵组成的集合\(W\),
对于矩阵的减法与乘法封闭,
因此\(W\)是\(K[\A]\)的一个子环.
显然\(\E \in W\).
我们可以建立数域\(K\)到\(W\)的一个映射\(\tau\colon a \mapsto a \E\).
显然\(\tau\)是同构映射.

现在我们就可以回答前面提出的问题了.
因为\(K\)与\(W\)同构,所以我们可以用矩阵\(\A\)代入\(x\),
再把每一项的系数换成它在\(K\)到\(W\)的环同构映射\(\tau\)下的像,
就直接得到了\cref{equation:多项式.示例公式2}.

\begin{theorem}\label{theorem:多项式.多项式环的同构映射}
%@see: 《高等代数(第三版 下册)》(丘维声) P7 定理4
设\(K\)是一个数域,\(R\)是一个有单位元的交换环,
并且\(K\)到\(R\)的一个子环\(R'\)有一个环同构映射\(\tau\).
对于任意给定\(t \in R\),令\[
	\sigma_t\colon
	K[x] \to R,
	f(x)=\sum_{i=0}^n a_i x^i \mapsto \sum_{i=0}^n \tau(a_i) t^i \defeq f(t),
\]
则\(\sigma_t\)是\(K[x]\)到\(R\)的映射;
并且\(\sigma_t\)保持加法与乘法运算,即\[
	f(x)+g(x)=h(x) \land f(x) g(x) = p(x)
	\implies
	f(t)+g(t)=h(t) \land f(t) g(t) = p(t);
\]
此外,\(\sigma_t(x) = t\).
我们把映射\(\sigma_t\)叫做“\(x\)用\(t\)代入”.
\end{theorem}

我们把\cref{theorem:多项式.多项式环的同构映射}
称为“一元多项式环\(K[x]\)的通用性质”.

\cref{theorem:多项式.多项式环的同构映射} 告诉我们,
如果\(R\)是有单位元的交换环,
且\(R\)有一个子环\(R_1\)满足\cref{theorem:多项式.多项式环的同构映射} 的条件
(这时我们称“\(R\)可看成是\(K\)的一个\DefineConcept{扩环}”),
那么一元多项式环\(K[x]\)中所有通过加法与乘法表示的关系,
在不定元\(x\)用\(R\)的任一元素\(t\)代入后仍然保持.
因此我们只要把一元多项式环\(K[x]\)中有关加法与乘法的等式研究清楚了,
通过不定元\(x\)用环\(R\)中任一元素\(t\)代入,就可以得到环\(R\)中有关加法与乘法的等式.
这就是一元多项式环\(K[x]\)的通用性质的含义.

从前面的讨论知道,
\(K[x],K[\A]\)都可以作为\cref{theorem:多项式.多项式环的同构映射} 中的环\(R\).
因此,不定元\(x\)可以用\(x\)的任一多项式代入,
也可以用矩阵\(\A\)的任一多项式代入,
从\(K[x]\)中已知的有关加法和乘法的等式,
得到\(K[x]\)中另一些有关加法和乘法的等式,
或者得到\(K[\A]\)中一些有关加法和乘法的等式.

\begin{example}
%@see: 《高等代数(第三版 下册)》(丘维声) P8 例1
设\(\B\)是数域\(K\)上的\(n\)阶幂零矩阵,
其幂零指数为\(l\).
令\(\A=\E+k\B\ (k \in K)\).
证明:\(\A\)可逆,并且求\(\A^{-1}\).
\begin{proof}
在\(K[x]\)中直接计算可得\[
	(1-x)(1+x+x^2+\dotsb+x^{l-1})
	=1-x^l.
\]
不定元\(x\)用\(-k\B\)代入,就可以从上式得到\[
	(\E+k\B)[\E+(-k\B)+(-k\B)^2+\dotsb+(-k\B)^{l-1}]
	=\E-(-k\B)^l.
\]
由于\(\B^l=\vb0\),
因此从上式得\[
	(\E+k\B)[\E-k\B+k^2\B^2+\dotsb+(-k)^{l-1}\B^{l-1}]
	=\E.
\]
上式表明\(\E+k\B\)可逆,
并且\[
	(\E+k\B)^{-1}
	= \E-k\B+k^2\B^2+\dotsb+(-k)^{l-1}\B^{l-1}.
	\qedhere
\]
\end{proof}
\end{example}

\section{整除性,带余除法}
从一元多项式环的通用性质看到,
我们应当尽可能多地得到\(K[x]\)中有关加法和乘法的等式,
为此需要研究一元多项式环\(K[x]\)的结构.
从本节开始我们将主要研究\(K[x]\)的结构,其中\(K\)是任一数域.

\subsection{整除}
观察\(K[x]\)中两个多项式\(f(x)\)与\(g(x)\)之间有什么关系:\[
	f(x)=x^2-1, \qquad
	g(x)=x-1.
\]
显然,\[
	f(x)=(x+1) g(x).
\]
由此我们抽象出“整除”的概念.

\begin{definition}
%@see: 《高等代数(第三版 下册)》(丘维声) P10 定义1
设\(f,g \in K[x]\).
如果存在\(h \in K[x]\),使得\[
	f(x) = h(x) g(x),
\]
则称“\(g(x)\) \DefineConcept{整除} \(f(x)\)”,
记作\(g(x) \mid f(x)\),
又称“\(g(x)\)是\(f(x)\)的\DefineConcept{因式}”
“\(f(x)\)是\(g(x)\)的\DefineConcept{倍式}”;
否则称“\(g(x)\)不能整除\(f(x)\)”,
记作\(g(x) \nmid f(x)\).
\end{definition}

容易看出下列事实:
\begin{enumerate}
	\item 零多项式整除一个多项式当且仅当这个多项式是零多项式,
	即\[
		0 \mid f(x)
		\iff
		f(x) = 0.
	\]
	\item 任一多项式整除零多项式,
	即\[
		(\forall f \in K[x])
		[f(x) \mid 0].
	\]
	\item 非零数都是多项式的因式,
	即\[
		(\forall b \in K - \{0\})
		(\forall f \in K[x])
		[b \mid f(x)].
	\]
\end{enumerate}

\begin{proposition}\label{theorem:多项式.整除的序}
设\(f,g\)都是数域\(K\)上的非零多项式.
若\(g \mid f\),
则\(\deg g \leq \deg f\).
\begin{proof}
当\(g \mid f\)时,
根据定义,存在\(h \in K[x]\),
使得\(f = h g\).
于是由\cref{equation:多项式.积的次数} 有\[
	\deg f
	= \deg(hg)
	= \deg h + \deg g.
\]
假设\(h\)是零多项式,
则\(f\)必定也是零多项式,
矛盾!
因此\(\deg h\geq0\),
从而\(\deg f\geq\deg g\).
\end{proof}
\end{proposition}

从\cref{theorem:多项式.整除的序} 可以看出:
一个非零多项式不可能整除比它次数更低的另一个非零多项式.

\begin{example}
%@see: 《高等代数(第三版 下册)》(丘维声) P13 习题7.2 1.
证明:整除关系具有传递性,即在\(K[x]\)中,\[
	(\forall f,g,h \in K[x])
	[
		f(x) \mid g(x) \land g(x) \mid h(x)
		\implies
		f(x) \mid h(x)
	].
\]
\begin{proof}
假设\(f(x) \mid g(x), g(x) \mid h(x)\).
由定义可知,存在\(u,v \in K[x]\),
使得\[
	g(x) = u(x) f(x), \qquad
	h(x) = v(x) g(x),
\]
于是\(h(x) = v(x) u(x) f(x)\),
即\(f(x) \mid h(x)\).
\end{proof}
\end{example}

\subsection{相伴}
\begin{definition}
%@see: 《高等代数(第三版 下册)》(丘维声) P10 定义2
在\(K[x]\)中,如果\(f(x) \mid g(x)\)且\(g(x) \mid f(x)\),
则称“\(f(x)\)与\(g(x)\) \DefineConcept{相伴}”
或“\(f(x)\)是\(g(x)\)的\DefineConcept{相伴元}”,
记作\(f(x) \sim g(x)\).
\end{definition}

\begin{proposition}
%@see: 《高等代数(第三版 下册)》(丘维声) P10 命题1
在\(K[x]\)中,\(f(x) \sim g(x)\)当且仅当存在\(c \in K-\{0\}\),使得\[
	f(x) = c g(x).
\]
\begin{proof}
充分性.
假设\(f(x)=c g(x)\),其中\(c \in K-\{0\}\).
显然有\(g(x) \mid f(x)\).
又因为\(g(x)=\frac1c f(x)\),
所以\(f(x) \mid g(x)\).
因此\(f(x) \sim g(x)\).

必要性.
假设\(f(x) \sim g(x)\).
由定义有\(f(x) \mid g(x)\)和\(g(x) \mid f(x)\).
于是存在\(h_1(x),h_2(x) \in K[x]\),
使得\[
	g(x) = h_1(x) f(x), \qquad
	f(x) = h_2(x) g(x).
\]
于是\[
	f(x) = h_2(x) h_1(x) f(x).
\]
如果\(f(x)=0\),则\(g(x)=0\).
下面假设\(f(x)\neq0\).
运用消去律,
由上式可得\[
	1 = h_2(x) h_1(x).
\]
继而可得\[
	\deg h_2(x) + \deg h_1(x) = 0.
\]
因此\(\deg h_1(x) = \deg h_2(x) = 0\),
从而\(h_2(x)\)等于\(K\)中某个非零常数\(c\),
于是\(f(x) = c g(x)\).
\end{proof}
\end{proposition}

容易看出,两个多项式的相伴关系是多项式环\(K[x]\)上的等价关系.

\subsection{整除的性质}
\begin{proposition}\label{theorem:多项式.整除的线性性}
%@see: 《高等代数(第三版 下册)》(丘维声) P10 命题2
在\(K[x]\)中,如果\(g(x) \mid f_i(x)\ (i=1,2,\dotsc,s)\),
则对于任意\(u_i \in K[x]\ (i=1,2,\dotsc,s)\),有\[
	g(x) \mid (u_1(x) f_1(x) + u_2(x) f_2(x) + \dotsb + u_s(x) f_s(x)).
\]
\begin{proof}
由\(g(x) \mid f_i(x)\)可知,
存在\(h_i(x) \in K[x]\),
使得\(f_i(x) = h_i(x) g(x)\).
因此\begin{align*}
	\sum_{i=1}^s u_i(x) f_i(x)
	&= \sum_{i=1}^s u_i(x) h_i(x) g(x) \\
	&= g(x) \sum_{i=1}^s u_i(x) h_i(x),
\end{align*}
所以\(g(x) \mid (u_1(x) f_1(x) + u_2(x) f_2(x) + \dotsb + u_s(x) f_s(x))\).
\end{proof}
\end{proposition}

\subsection{带余除法}
在\(K[x]\)中,如果\(g(x)\)不能整除\(f(x)\),
那么能有什么样的结论呢?
例如,设\(f(x)=x^2,
g(x)=x-1\),
则\[
	f(x)=x^2-1+1=(x+1)g(x)+1.
\]
由此受到启发,我们可以给出如下结论.
\begin{theorem}\label{theorem:多项式.带余除法}
%@see: 《高等代数(第三版 下册)》(丘维声) P11 定理3
对于\(K[x]\)中任意两个多项式\(f(x)\)与\(g(x)\),其中\(g(x)\neq0\),
则在\(K[x]\)中存在唯一的一对多项式\(h(x),r(x)\),使得\[
	f(x) = h(x) g(x) + r(x),
	\qquad
	\deg r(x) < \deg g(x).
\]
\begin{proof}
存在性.
假设被除式\(f(x)\)的次数为\(n\),
除式\(g(x)\)的次数为\(m\),
即\(\deg f(x)=n\in\mathbb{N}\),
\(\deg g(x)=m\in\mathbb{N}\).
让我们分情况讨论:\begin{enumerate}
	\item[情形1]
	当\(m=0\)时,
	除式\(g(x)\)是零次多项式,
	不妨设\(g(x)\)等于\(K\)中某个非零常数\(b\).
	于是,只要取\(h(x) = \frac1b f(x), r(x) = 0\),
	就有\[
		f(x) = h(x) g(x) + r(x),
		\qquad
		\deg 0 < \deg g(x).
	\]

	\item[情形2]
	当\(m>0\)且\(\deg f(x) = n < m\)时,
	只要取\(h(x) = 0, r(x) = f(x)\)
	就有\[
		f(x) = h(x) g(x) + r(x), \qquad
		\deg f(x) < \deg g(x).
	\]

	\item[情形3]
	当\(m>0\)且\(\deg f(x) = n \geq m\)时,
	对被除式\(f(x)\)的次数\(n\)运用数学归纳法.

	假设对于次数小于\(n\)的被除式,
	命题的存在性部分成立.
	现在来看\(n\)次多项式\(f(x)\).

	设\(f(x),g(x)\)的首项分别是\(a_n x^n,b_m x^m\).
	于是\(a_n b_m^{-1} x^{n-m} g(x)\)的首项是\(a_n x^n\).
	令\(f_1(x) = f(x) - a_n b_m^{-1} x^{n-m} g(x)\),
	则\(\deg f_1(x) < n\).
	根据归纳假设,
	存在\(h_1(x),r_1(x) \in K[x]\),
	使得\[
		f_1(x) = h_1(x) g(x) + r_1(x), \qquad
		\deg r_1(x) < \deg g(x).
	\]
	于是\begin{align*}
		f(x)
		&= f_1(x) + a_n b_m^{-1} x^{n-m} g(x) \\
		&= [h_1(x) + a_n b_m^{-1} x^{n-m}] g(x) + r_1(x).
	\end{align*}
	因此,只需要令\(h(x) = h_1(x) + a_n b_m^{-1} x^{n-m}\),
	就有\[
		f(x) = h(x) g(x) + r_1(x),
		\qquad
		\deg r_1(x) < \deg g(x).
	\]
\end{enumerate}

唯一性.
设\(h(x),r(x),h'(x),r'(x) \in K[x]\),
使得\begin{gather*}
	f(x) = h(x) g(x) + r(x), \qquad \deg r(x) < \deg g(x), \\
	f(x) = h'(x) g(x) + r'(x), \qquad \deg r'(x) < \deg g(x).
\end{gather*}
于是有\[
	h(x) g(x) + r(x)
	= h'(x) g(x) + r'(x),
\]
即\[
	[h(x) - h'(x)] g(x) = r'(x) - r(x).
\]
那么\begin{align*}
	\deg[h(x) - h'(x)] + \deg g(x)
	&= \deg[r'(x) - r(x)] \\
	&\leq \max\{
		\deg r'(x),
		\deg r(x)
	\}
	< \deg g(x).
\end{align*}
假设\(h(x) \neq h'(x)\),
那么由上式可知\[
	\deg[h(x) - h'(x)] < 0,
\]
矛盾!
因此必有\(h(x) = h'(x)\).
从而又有\(r(x) = r'(x)\).
\end{proof}
\end{theorem}

\cref{theorem:多项式.带余除法} 中的
\(f(x)\)称为“\(g(x)\)除\(f(x)\)的\DefineConcept{被除式}(dividend)”,
\(g(x)\)称为“\(g(x)\)除\(f(x)\)的\DefineConcept{除式}(divisor)”,
\(h(x)\)称为“\(g(x)\)除\(f(x)\)的\DefineConcept{商式}(quotient)”,
\(r(x)\)称为“\(g(x)\)除\(f(x)\)的\DefineConcept{余式}(remainder)”.
%@see: https://mathworld.wolfram.com/Dividend.html
%@see: https://mathworld.wolfram.com/Divisor.html

\begin{corollary}\label{theorem:多项式.带余除法.推论}
%@see: 《高等代数(第三版 下册)》(丘维声) P12 推论4
设\(f,g \in K[x]\),且\(g(x) \neq 0\),
则\(g(x) \mid f(x)\)当且仅当\(g(x)\)除\(f(x)\)的余式为零.
\begin{proof}
由\cref{theorem:多项式.带余除法} 立即可得\begin{align*}
	g(x) \mid f(x)
	&\iff
	(\exists h \in K[x])
	[f(x) = h(x) g(x)] \\
	&\iff
	\text{$g(x)$除$f(x)$的余式是$0$}.
	\qedhere
\end{align*}
\end{proof}
\end{corollary}

利用带余除法可以证明:
对于\(K[x]\)中的多项式\(f(x),g(x)\),
如果在\(K[x]\)中,\(g(x)\)不能整除\(f(x)\),
那么把数域\(K\)扩大成数域\(F\)后,
在\(F[x]\)中,\(g(x)\)仍然不能整除\(f(x)\).

\begin{proposition}\label{theorem:多项式.整除性不随数域的扩大而改变}
%@see: 《高等代数(第三版 下册)》(丘维声) P12 命题5
设\(F,K\)都是数域,且\(F \supseteq K\).
如果\(f,g \in K[x]\),那么\[
	\text{在\(K[x]\)中成立\(g(x) \mid f(x)\)}
	\iff
	\text{在\(F[x]\)中成立\(g(x) \mid f(x)\)}.
\]
\begin{proof}
必要性.
假设在\(K[x]\)中,\(g(x) \mid f(x)\),
则存在\(h(x) \in K[x]\),
使得\(f(x) = h(x) g(x)\).
由于\(K \subseteq F\),
因此\(f(x),g(x),h(x) \in K[x]\).
从而在\(F[x]\)中,\(g(x) \mid f(x)\).

充分性.
假设在\(F[x]\)中,\(g(x) \mid f(x)\).
我们分以下两种情况讨论:\begin{enumerate}
	\item 当\(g(x)\neq0\)时,
	在\(K[x]\)中作带余除法,
	有\(h(x),r(x) \in K[x]\),
	使得\[
		f(x) = h(x) g(x) + r(x), \qquad
		\deg r(x) < \deg g(x).
	\]
	由于\(f(x),g(x),h(x),r(x) \in F[x]\),
	因此上式也可以看成是在\(F[x]\)中的带余除法.
	由于在\(F[x]\)中,
	\(g(x) \mid f(x)\),
	因此根据\cref{theorem:多项式.带余除法.推论}
	得\(r(x) = 0\).
	从而在\(K[x]\)中,有\(g(x) \mid f(x)\).

	\item 当\(g(x)=0\)时,
	从\(g(x) \mid f(x)\)得\(f(x)=0\).
	从而在\(K[x]\)中,也有\(g(x) \mid f(x)\).
\end{enumerate}
综上所述,在\(K[x]\)中,总有\(g(x) \mid f(x)\).
\end{proof}
\end{proposition}

\cref{theorem:多项式.整除性不随数域的扩大而改变} 表明,整除性不随数域的扩大而改变.

\begin{example}
%@see: 《高等代数(第三版 下册)》(丘维声) P12 例1
设\(f(x) = 2x^3+3x^2+5\),
\(g(x) = x^2+2x-1\),
求用\(g(x)\)除\(f(x)\)的商式与余式.
\begin{solution}
我们可以参考整数除法的竖式,作出如下计算:
\[
	\begin{array}{r|*4r|l}
		x^2+2x-1 &
		2x^3 & +3x^2 & & +5
		& 2x-1 \\
		& 2x^3 & +4x^2 & -2x & \\ \cline{2-5}
		& & -x^2 & +2x & +5 \\
		& & -x^2 & -2x & +1 \\ \cline{3-5}
		& & & 4x & +4
	\end{array}
\]
因此\[
	2x^3+3x^2+5=(2x-1)(x^2+2x-1)+(4x+4),
\]
即\(g(x)\)除\(f(x)\)的商式是\(2x-1\),余式是\(4x+4\).
\end{solution}
\end{example}

\begin{example}
%@see: 《高等代数(第三版 下册)》(丘维声) P12 例1
设\(f(x) = 2x^4-6x^3+3x^2-2x+5\),
\(g(x) = x-2\),
求用\(g(x)\)除\(f(x)\)的商式与余式.
\begin{solution}我们可以参考整数除法的竖式,作出如下计算:
\[
	\begin{array}{r|*5r|l}
		x{\color{red}-2} &
		2x^4 & -6x^3 & +3x^2 & -2x & +5
		& 2x^3-2x^2-x-4 \\
		& 2x^4 & {\color{red}-4}x^3 &&&& \\ \cline{2-6}
		&& -2x^3 & +3x^2 &&& \\
		&& -2x^3 & {\color{red}+4}x^2 &&& \\ \cline{3-6}
		&&& -x^2 & -2x && \\
		&&& -x^2 & {\color{red}+2}x && \\ \cline{4-6}
		&&&& -4x & +5 & \\
		&&&& -4x & {\color{red}+8} & \\ \cline{5-6}
		&&&&& {\color{red}-3}
	\end{array}
\]
注意到除式是1次多项式,
我们可以采用“综合除法”这种简易计算方法.
首先我们需要把被除式的各项系数写成一行,
再把除式的常数项的相反数写在下一行的最左边,
如下:\[
	\begin{array}{r|*5r}
		& 2 & -6 & 3 & -2 & 5 \\
		{\color{red}2} \\ \cline{2-6}
	\end{array}
\]
接下来把第一行第一列数(即被除式的首项系数)
写到横线下方对应位置,
得到\[
	\begin{array}{r|*5r}
		& 2 & -6 & 3 & -2 & 5 \\
		2 \\ \cline{2-6}
		& {\color{red}2}
	\end{array}
\]
接下来把横线下方当前排在最末的数与竖线左边的数相乘,
写在第二行第二列:\[
	\begin{array}{r|*5r}
		& 2 & -6 & 3 & -2 & 5 \\
		2 && {\color{red}4} \\ \cline{2-6}
		& 2
	\end{array}
\]
然后把第二列第一行、第二行的数字相加,
把结果写到横线下方对应位置:\[
	\begin{array}{r|*5r}
		& 2 & -6 & 3 & -2 & 5 \\
		2 && 4 \\ \cline{2-6}
		& 2 & {\color{red}-2}
	\end{array}
\]
类似地,把横线下方当前排在最末的数与竖线左边的数相乘,
写在第二行第三列:\[
	\begin{array}{r|*5r}
		& 2 & -6 & 3 & -2 & 5 \\
		2 && 4 & {\color{red}-4} \\ \cline{2-6}
		& 2 & -2
	\end{array}
\]
然后又把第三列第一行、第二行的数字相加,
把结果写到横线下方对应位置:\[
	\begin{array}{r|*5r}
		& 2 & -6 & 3 & -2 & 5 \\
		2 && 4 & -4 \\ \cline{2-6}
		& 2 & -2 & {\color{red}-1}
	\end{array}
\]
以此类推,最后我们得到:\[
	\begin{array}{r|*5r}
		& 2 & -6 & 3 & -2 & 5 \\
		2 && 4 & -4 & -2 & -8 \\ \cline{2-6}
		& \color{blue}2 & \color{blue}-2 & \color{blue}-1 & \color{blue}-4
		& {\color{red}-3}
	\end{array}
\]
我们把上式中蓝色的数字
按顺序写成一个3次多项式\(2x^3-2x^2-x-4\)
(这是因为\(\deg f(x)-\deg g(x)=3\)),
这就是\(g(x)\)除\(f(x)\)的商式;
然后我们把上式中红色的数字\(-3\)
作为\(g(x)\)除\(f(x)\)的余式;
也就是说\[
	2x^4-6x^3+3x^2-2x+5
	=(2x^3-2x^2-x-4)(x-2)-3.
\]
\end{solution}
\end{example}

%@see: https://zhuanlan.zhihu.com/p/634579122
%@see: https://mathworld.wolfram.com/LongDivision.html
%@see: https://mathworld.wolfram.com/SyntheticDivision.html
在上面这个例子中,
我们利用\DefineConcept{综合除法}(synthetic division)
求出了用一次多项式\(g(x)=x-c\)
去除任一多项式\(f(x)=a_n x^n+a_{n-1} x^{n-1}+\dotsb+a_1 x+a_0\)的商式和余式.
实际上综合除法是基于带余除法和待定系数法建立了一种简易算法.
不妨设\[
	f(x)=(x-c) q(x)+r,
\]
其中\(q(x)=b_{n-1} x^{n-1}+b_{n-2} x^{n-2}+\dotsb+b_1 x+b_0\).
那么有\[
	f(x)
	=b_{n-1} x^n
	+(b_{n-2}-c b_{n-1}) x^{n-1}
	+(b_{n-3}-c b_{n-2}) x^{n-2}
	+\dotsb
	+(b_0-c b_1) x
	+(r-c b_0).
\]
将上式与\(f(x)=a_n x^n+a_{n-1} x^{n-1}+\dotsb+a_1 x+a_0\)比较可得\[
	\left\{ \begin{array}{l}
		a_n=b_{n-1}, \\
		a_{n-1}=b_{n-2}-c b_{n-1}, \\
		a_{n-2}=b_{n-3}-c b_{n-2}, \\
		\hdotsfor1 \\
		a_1=b_0-c b_1, \\
		a_0=r-c b_0,
	\end{array} \right.
	\quad\text{即}\quad
	\left\{ \begin{array}{l}
		b_{n-1}=a_n, \\
		b_{n-2}=a_{n-1}+c b_{n-1}, \\
		b_{n-3}=a_{n-2}+c b_{n-2}, \\
		\hdotsfor1 \\
		b_0=a_1+c b_1, \\
		r=a_0+c b_0.
	\end{array} \right.
\]
于是我们可以列出下表:\[
	\begin{array}{r|*5c}
		& a_n & a_{n-1} & \dots & a_1 & a_0 \\
		c && c a_n & \dots & c b_1 & c b_0 \\ \cline{2-6}
		& a_n=b_{n-1} & a_{n-1}+c a_n=b_{n-2} & \dots & a_1+c b_1=b_0 & a_0+c b_0=r
	\end{array}
\]
这就是综合除法的原理.

最后,让我们再看一个运用综合除法的例子.
%@see: https://billcookmath.com/sage/algebra/Horners_method.html
设\(f(x) = 3x^5 - 8x^4 - 5x^3 + 26x^2 - 33x + 26\),
\(g(x) = x^3 - 2x^2 - 4x + 8\).
首先把被除式的各项系数写成一行,再把除式的非最高次的各项系数的相反数写成一列,如下:\[
	\begin{array}{r|*6r}
		& 3 & -8 & -5 & 26 & -33 & 26 \\
		2 \\
		4 \\
		-8 \\ \cline{2-7}
	\end{array}
\]
接下来把第一行第一列数(即被除式的首项系数)
写到横线下方对应位置,
得到\[
	\begin{array}{r|*6r}
		& 3 & -8 & -5 & 26 & -33 & 26 \\
		2 \\
		4 \\
		-8 \\ \cline{2-7}
		& \color{red}3
	\end{array}
\]
接下来把横线下方当前排在最末的数与竖线左边的各数依次相乘,
从第二行第二列开始依次写出乘积:\[
	\begin{array}{r|*6r}
		& 3 & -8 & -5 & 26 & -33 & 26 \\
		2 & & \color{red}6 & \color{red}12 & \color{red}-24 \\
		4 \\
		-8 \\ \cline{2-7}
		& 3
	\end{array}
\]
然后把第二列各行的数字相加,
把结果写到横线下方对应位置:\[
	\begin{array}{r|*6r}
		& 3 & -8 & -5 & 26 & -33 & 26 \\
		2 & & 6 & 12 & -24 \\
		4 \\
		-8 \\ \cline{2-7}
		& 3 & \color{red}-2
	\end{array}
\]
类似地,把横线下方当前排在最末的数与竖线左边的各数依次相乘,
从第三行第三列开始依次写出乘积:\[
	\begin{array}{r|*6r}
		& 3 & -8 & -5 & 26 & -33 & 26 \\
		2 && 6 & 12 & -24 \\
		4 &&& \color{red}-4 & \color{red}-8 & \color{red}16 \\
		-8 \\ \cline{2-7}
		& 3 & -2
	\end{array}
\]
然后又把第三列各行的数字相加,
把结果写到横线下方对应位置:\[
	\begin{array}{r|*6r}
		& 3 & -8 & -5 & 26 & -33 & 26 \\
		2 && 6 & 12 & -24 \\
		4 &&& -4 & -8 & 16 \\
		-8 \\ \cline{2-7}
		& 3 & -2 & \color{red}3
	\end{array}
\]
继续把横线下方当前排在最末的数与竖线左边的各数依次相乘,
从第四行第四列开始依次写出乘积:\[
	\begin{array}{r|*6r}
		& 3 & -8 & -5 & 26 & -33 & 26 \\
		2 && 6 & 12 & -24 \\
		4 &&& -4 & -8 & 16 \\
		-8 &&&& \color{red}6 & \color{red}12 & \color{red}-24 \\ \cline{2-7}
		& 3 & -2 & 3
	\end{array}
\]
再把第四列各行的数字相加,
把结果写到横线下方对应位置:\[
	\begin{array}{r|*6r}
		& 3 & -8 & -5 & 26 & -33 & 26 \\
		2 && 6 & 12 & -24 \\
		4 &&& -4 & -8 & 16 \\
		-8 &&&& 6 & 12 & -24 \\ \cline{2-7}
		& 3 & -2 & 3 & \color{red}0
	\end{array}
\]
最后把第五列、第六列的数字相加,
把结果写到横线下方对应位置:\[
	\begin{array}{r|*6r}
		& 3 & -8 & -5 & 26 & -33 & 26 \\
		2 && 6 & 12 & -24 \\
		4 &&& -4 & -8 & 16 \\
		-8 &&&& 6 & 12 & -24 \\ \cline{2-7}
		& \color{blue}3 & \color{blue}-2 & \color{blue}3 & \color{red}0 & \color{red}-5 & \color{red}2
	\end{array}
\]
我们把上式中蓝色的数字
按顺序写成一个2次多项式\(3x^2-2x+3\)
(这是因为\(\deg f(x)-\deg g(x)=2\)),
这就是\(g(x)\)除\(f(x)\)的商式;
然后我们把上式中红色的数字
也按顺序写成一个1次多项式\(-5x+2\),
作为\(g(x)\)除\(f(x)\)的余式;
也就是说\[
	3x^5 - 8x^4 - 5x^3 + 26x^2 - 33x + 26
	=(3x^2-2x+3)(x^3 - 2x^2 - 4x + 8)+(-5x+2).
\]

\section{最大公因式}
\subsection{最大公因式}
从上一节知道,数域\(K\)上的一元多项式环\(K[x]\)具有带余除法,这是\(K[x]\)的一个重要性质.
这一节我们要由此出发推导出\(K[x]\)的另一个重要性质:
\(K[x]\)中任何两个多项式都有最大公因式,
并且\(f(x)\)与\(g(x)\)的最大公因式可以表成\(f(x)\)与\(g(x)\)的倍式和.

\begin{definition}
%@see: 《高等代数(第三版 下册)》(丘维声) P15
在\(K[x]\)中,如果\(c(x)\)既是\(f(x)\)的因式,又是\(g(x)\)的因式,
则称“\(c(x)\)是\(f(x)\)与\(g(x)\)的一个\DefineConcept{公因式}”.
\end{definition}

\begin{definition}
%@see: 《高等代数(第三版 下册)》(丘维声) P15 定义1
设\(f(x),g(x) \in K[x]\),
\(d(x)\)是\(f(x)\)与\(g(x)\)的一个公因式.
如果\(f(x)\)与\(g(x)\)的任一公因式都是\(d(x)\)的因式,
则称“\(d(x)\)是\(f(x)\)与\(g(x)\)的一个\DefineConcept{最大公因式}”.
\end{definition}

对于任意多项式\(f(x)\),由于\(f(x) \mid f(x)\)且\(f(x) \mid 0\),
所以\(f(x)\)是\(f(x)\)与\(0\)的一个公因式.
又由于\(f(x)\)与\(0\)的任一公因式\(c(x)\)总可整除\(f(x)\),
因此\(f(x)\)是\(f(x)\)与\(0\)的一个最大公因式.
特别地,\(0\)是\(0\)与\(0\)的最大公因式.

现在我们想要知道,对于\(K[x]\)中任意两个多项式,是否存在它们的最大公因式?
如果存在,我们又该如何找出它们的最大公因式?
对于给定的两个多项式\(f(x)\)与\(g(x)\),它们的最大公因式是否唯一?
这些就是本节要讨论的问题.

我们先指出几个简单而有用的结论.
\begin{proposition}\label{theorem:多项式.最大公因式.命题1}
%@see: 《高等代数(第三版 下册)》(丘维声) P15 命题1
设\(f,g,p,q \in K[x]\).
如果\begin{equation*}
	\Set{ h(x) \given \text{\(h(x)\)是\(f(x)\)与\(g(x)\)的公因式} }
	= \Set{ r(x) \given \text{\(r(x)\)是\(p(x)\)与\(q(x)\)的公因式} },
\end{equation*}
那么\begin{equation*}
	\Set{ h(x) \given \text{\(h(x)\)是\(f(x)\)与\(g(x)\)的最大公因式} }
	= \Set{ r(x) \given \text{\(r(x)\)是\(p(x)\)与\(q(x)\)的最大公因式} }.
\end{equation*}
\begin{proof}
设\(d(x)\)是\(f(x)\)与\(g(x)\)的一个最大公因式,
则\(d(x)\)是\(p(x)\)与\(q(x)\)的一个公因式.
任取\(p(x)\)与\(q(x)\)的一个公因式\(\phi(x)\),
则\(\phi(x)\)也是\(f(x)\)与\(g(x)\)的一个公因式,
从而\(\phi(x) \mid d(x)\).
所以\(d(x)\)是\(p(x)\)与\(q(x)\)的一个最大公因式.
同理,\(p(x)\)与\(q(x)\)的任一最大公因式也是\(f(x)\)与\(g(x)\)的最大公因式.
\end{proof}
\end{proposition}

\begin{corollary}\label{theorem:多项式.最大公因式.推论2}
%@see: 《高等代数(第三版 下册)》(丘维声) P15 推论2
设\(f,g \in K[x]\),\(a,b \in K-\{0\}\),
则\begin{equation*}
	\Set{ h(x) \given \text{\(h(x)\)是\(f(x)\)与\(g(x)\)的最大公因式} }
	= \Set{ h(x) \given \text{\(h(x)\)是\(a f(x)\)与\(b g(x)\)的最大公因式} }.
\end{equation*}
\begin{proof}
显然\(f(x)\)与\(g(x)\)的任一公因式是\(a f(x)\)与\(b g(x)\)的公因式.
对于\(a f(x)\)与\(b g(x)\)的任一公因式\(c(x)\),
有\(c(x) \mid a f(x)\).
又由于\(a\neq0\),
因此\(a f(x) \mid f(x)\),
从而\(c(x) \mid f(x)\).
同理\(c(x) \mid g(x)\).
因此\(c(x)\)也是\(f(x)\)与\(g(x)\)的公因式.
于是由\cref{theorem:多项式.最大公因式.命题1} 立即得出结论.
\end{proof}
\end{corollary}

\begin{lemma}\label{theorem:多项式.最大公因式.引理1}
%@see: 《高等代数(第三版 下册)》(丘维声) P15 引理1
在\(K[x]\)中,如果多项式\(f,g,h,r\)满足\begin{equation*}
	f(x) = h(x) g(x) + r(x),
\end{equation*}
则被除式\(f\)与除式\(g\)的最大公因式就是除式\(g\)与余式\(r\)的最大公因式,
即\begin{equation*}
	\Set{ u(x) \given \text{\(u(x)\)是\(f(x)\)与\(g(x)\)的最大公因式} }
	= \Set{ u(x) \given \text{\(u(x)\)是\(g(x)\)与\(r(x)\)的最大公因式} }.
\end{equation*}
\begin{proof}
设\(d(x)\)是\(f(x)\)与\(g(x)\)的一个公因式,
则\(d(x) \mid f(x)\)且\(d(x) \mid g(x)\).
因为\begin{equation*}
	f(x) = h(x) g(x) + r(x)
	\implies
	r(x) = f(x) - h(x) g(x),
\end{equation*}
所以由\cref{theorem:多项式.整除的线性性}
得\(d(x) \mid r(x)\),
也就是说\(d(x)\)是\(g(x)\)与\(r(x)\)的一个公因式.
现在任取\(g(x)\)与\(r(x)\)的一个公因式\(c(x)\),
由\(f(x) = h(x) g(x) + r(x)\)得\(c(x) \mid f(x)\),
也就是说\(c(x)\)是\(f(x)\)与\(g(x)\)的一个公因式.
由\cref{theorem:多项式.最大公因式.命题1} 立即得出所要求的结论.
\end{proof}
\end{lemma}

\subsection{辗转相除法}
\begin{flowchart}
	\node (start) [startstop] {开始};
	\node (in1) [io, below of=start] {输入整数$m,n$};
	\node (pro1) [process, below of=in1] {$r \defeq m$除以$n$的余数};
	\node (pro2) [process, below of=pro1] {$m \defeq n$};
	\node (pro3) [process, below of=pro2] {$n \defeq r$};
	\node (dec1) [decision, below of=pro3] {$r=0$?};
	\node (out1) [io, below of=dec1] {输出$m$};
	\node (stop) [startstop, below of=out1] {结束};

	\begin{scope}[arrow]
		\draw (start) -- (in1);
		\draw (in1) -- (pro1);
		\draw (pro1) -- (pro2);
		\draw (pro2) -- (pro3);
		\draw (pro3) -- (dec1);
		\draw (dec1) -- node[anchor=east]{是} (out1);
		\draw (dec1) -- node[anchor=south]{否} ++(3,0) |- (pro1);
		\draw (out1) -- (stop);
	\end{scope}
\end{flowchart}

\begin{theorem}\label{theorem:多项式.辗转相除法}
%@see: 《高等代数(第三版 下册)》(丘维声) P16 定理3
对于\(K[x]\)中任意两个多项式\(f(x)\)与\(g(x)\),
存在它们的一个最大公因式\(d(x)\),
并且\(d(x)\)可以表示成\(f(x)\)与\(g(x)\)的倍式和,即存在\(u,v \in K[x]\),使得\begin{equation*}
	d(x) = u(x) f(x) + v(x) g(x).
\end{equation*}
\begin{proof}
假设\(g(x)=0\),
则\(f(x)\)就是\(f(x)\)与\(g(x)\)的一个最大公因式,
并且\begin{equation*}
	f(x) = 1 \cdot f(x) + 1 \cdot 0.
\end{equation*}

现在设\(g(x)\neq0\).
根据\hyperref[theorem:多项式.带余除法]{带余除法},
存在\(h_1(x),r_1(x) \in K[x]\),
使得\begin{equation*}
	f(x) = h_1(x) g(x) + r_1(x), \qquad
	\deg r_1(x) < \deg g(x).
\end{equation*}
如果\(r_1(x)\neq0\),
则用\(r_1(x)\)去除\(g(x)\),
存在\(h_2(x),r_2(x) \in K[x]\),
使得\begin{equation*}
	g(x) = h_2(x) r_1(x) + r_2(x), \qquad
	\deg r_2(x) < \deg r_1(x).
\end{equation*}
又如果\(r_2\neq0\),
则用\(r_2(x)\)去除\(r_1(x)\),
存在\(h_3(x),r_3(x) \in K[x]\),
使得\begin{equation*}
	r_1(x) = h_3(x) r_2(x) + r_3(x), \qquad
	\deg r_3(x) < \deg r_2(x).
\end{equation*}
如此辗转相除下去,
显然,所得余式的次数不断降低,
因此在有限次之后,必然有余式为零,
即\begin{equation*}\begin{array}{ll}
	r_2(x) = h_4(x) r_3(x) + r_4(x), \qquad
		&\deg r_4(x) < \deg r_3(x), \\
	\hdotsfor{2}, \\
	r_{i-2}(x) = h_i(x) r_{i-1}(x) + r_i(x), \qquad
		&\deg r_i(x) < \deg r_{i-1}(x), \\
	\hdotsfor{2}, \\
	r_{s-3}(x) = h_{s-1}(x) r_{s-2}(x) + r_{s-1}(x), \qquad
		&\deg r_{s-1}(x) < \deg r_{s-2}(x), \\
	r_{s-2}(x) = h_s(x) r_{s-1}(x) + r_s(x), \qquad
		&\deg r_s(x) < \deg r_{s-1}(x), \\
	r_{s-1}(x) = h_{s+1}(x) r_s(x) + 0,
\end{array}\end{equation*}
其中\(h_i(x),r_i(x) \in K[x]\).
由于\(r_s(x)\)是\(r_s(x)\)与\(0\)的一个最大公因式,
因此根据\cref{theorem:多项式.最大公因式.引理1},
从上述等式的最后一个式子得出:
\(r_s(x)\)是\(r_{s-1}(x)\)与\(r_s(x)\)的一个最大公因式.
于是\(r_s(x)\)是\(r_{s-2}(x)\)与\(r_{s-1}(x)\)的一个最大公因式,
从而\(r_s(x)\)是\(r_{s-3}(x)\)与\(r_{s-2}(x)\)的一个最大公因式,
依次递推,
\(r_s(x)\)是\(f(x)\)与\(g(x)\)的一个最大公因式.
这就证明了:
在对\(f(x)\)与\(g(x)\)作辗转相除时,
最后一个不等于零的余式是\(f(x)\)与\(g(x)\)的一个最大公因式.
对上述等式中倒数第二个式子得\begin{equation*}
	r_s(x) = r_{s-2}(x) - h_s(x) r_{s-1}(x),
\end{equation*}
再由倒数第三个式子得\begin{equation*}
	r_{s-1}(x) = r_{s-3}(x) - h_{s-1}(x) r_{s-2}(x),
\end{equation*}
合并以上两式得\begin{equation*}
	r_s(x) = [1 + h_s(x) h_{s-1}(x)] r_{s-2}(x) - h_s(x) r_{s-3}(x).
\end{equation*}
同理用更上面的等式逐个地消去\(r_{s-2}(x),r_{s-3}(x),\dotsc,r_1(x)\),
可得\begin{equation*}
	r_s(x) = u(x) f(x) + v(x) g(x),
\end{equation*}
其中\(u(x),v(x) \in K[x]\).
\end{proof}
\end{theorem}

\cref{theorem:多项式.辗转相除法} 给出了求两个多项式的最大公因式的方法 --- “辗转相除法”.

我们想要知道,
任意给定\(K[x]\)中的两个多项式\(f(x)\)与\(g(x)\),
它们的最大公因式是否唯一?
设\(d_1(x),d_2(x)\)都是\(f(x)\)与\(g(x)\)的最大公因式,
根据定义得\(d_1(x) \mid d_2(x)\)且\(d_2(x) \mid d_1(x)\).
因此\(d_1(x)\)与\(d_2(x)\)相伴,即\(d_1(x)\)与\(d_2(x)\)仅相差一个非零数因子.
这说明:两个多项式的最大公因式在相伴的意义下是唯一确定的.
容易看出,两个不全为零的多项式的最大公因式一定是非零多项式,
在这个情形,我们约定,用\begin{equation*}
	(f(x), g(x))
\end{equation*}表示首项系数是\(1\)的那个最大公因式.

应该注意到,
在\cref{theorem:多项式.辗转相除法} 的证明过程中,
我们证明了\(r_s(x)\)是\(f(x)\)与\(g(x)\)的一个最大公因式,
并且有\(r_s(x) = u(x) f(x) + v(x) g(x)\).
对于\(f(x)\)与\(g(x)\)的任一最大公因式\(d(x)\),
由于\(d(x)\)与\(r_s(x)\)相伴,
因此\(d(x) = c r_s(x)\),
其中\(c\)是\(K\)中某个非零数.
于是有\(d(x) = c u(x) f(x) + c v(x) g(x)\).
这表明\(d(x)\)也可以表示成\(f(x)\)与\(g(x)\)的倍式和.

由\cref{theorem:多项式.最大公因式.推论2} 得出,
当\(f(x),g(x)\)不全为零时,
对于\(a,b \in K-\{0\}\),
有\begin{equation*}
	(f(x),g(x))
	= (a f(x),b g(x)).
\end{equation*}

\begin{example}
设\(f(x)=x^3+x^2-7x+2,
g(x)=3x^2-5x-2\),
求\((f(x),g(x))\),
并且把它表示成\(f(x)\)与\(g(x)\)的倍式和.
\begin{solution}
根据上面的结论,在作辗转相除时,
可以用适当的非零数去乘被除式或者除式,简化计算.
\begin{equation*}
	\def\arraystretch{1.5}
	\begin{array}{r|*3r|*4r|l}
		3x+1 & 3x^2 & -5x & -2 & 3x^3 & +3x^2 & -21x & +6 & x+\frac83 \\
		& 3x^2 & -6x && 3x^3 & -5x^2 & -2x & \\ \cline{2-8}
		&& x & -2 && 8x^2 & -19x & +6 \\
		&& x & -2 && 8x^2 & -\frac{40}3x & -\frac{16}3 \\ \cline{2-8}
		&&& 0 &&& -\frac{17}3x & +\frac{34}3 \\
		&&& &&& x & -2 \\
	\end{array}
\end{equation*}
因为最后一个不等于零的余式是\(r_1(x) = -\frac{17}3x + \frac{34}3\),
所以\begin{equation*}
	(f(x),g(x)) = x-2.
\end{equation*}
把上述辗转相除过程写出来就是\begin{align*}
	3 f(x) = \left(x+\frac83\right) g(x) + r_1(x), \\
	g(x) = (3x+1) \left[-\frac3{17} r_1(x)\right] + 0.
\end{align*}
于是\begin{align*}
	(f(x),g(x))
	&= -\frac3{17} r_1(x) \\
	&= -\frac3{17} \left[3 f(x) - \left(x+\frac83\right) g(x)\right] \\
	&= -\frac9{17} f(x) + \frac1{17} (3x+8) g(x).
\end{align*}
\end{solution}
\end{example}

\subsection{互素}
现在我们来研究两个多项式的最大公因式是零次多项式的情形.

\begin{definition}\label{definition:多项式.互素}
%@see: 《高等代数(第三版 下册)》(丘维声) P18 定义2
设\(f,g \in K[x]\).
如果\((f(x),g(x))=1\),
则称“\(f(x)\)与\(g(x)\) \DefineConcept{互素}”.
\end{definition}

从\cref{definition:多项式.互素} 立即得出,
两个多项式互素当且仅当它们的公因式都是零次多项式,
这是因为它们的任一公因式\(c(x) \mid 1\),
所以\(\deg c(x) = 0\).

下面我们给出两个多项式互素的一个充分必要条件.
\begin{theorem}\label{theorem:多项式.两个多项式互素的充分必要条件}
%@see: 《高等代数(第三版 下册)》(丘维声) P18 定理4
%@see: 《高等代数(大学高等代数课程创新教材 第二版 下册)》(丘维声) P34 例8
设\(f,g \in K[x]\).
\(f(x)\)与\(g(x)\)互素的充分必要条件是:
存在\(u,v \in K[x]\),使得\begin{equation*}
	u(x) f(x) + v(x) g(x) = 1.
\end{equation*}
\begin{proof}
必要性.
由\cref{theorem:多项式.辗转相除法} 立即可得.

充分性.
假设\(u(x) f(x) + v(x) g(x) = 1\)成立.
因为\((f(x),g(x)) \mid f(x)\)且\((f(x),g(x)) \mid g(x)\),
所以\((f(x),g(x)) \mid 1\),
于是\((f(x),g(x)) = 1\).
\end{proof}
\end{theorem}

利用\cref{theorem:多项式.两个多项式互素的充分必要条件} 可以证明关于互素的多项式的一些重要性质.

\begin{property}\label{theorem:多项式.互素.性质1}
%@see: 《高等代数(第三版 下册)》(丘维声) P19 性质1
在\(K[x]\)中,如果\begin{equation*}
	f(x) \mid g(x) h(x)
	\quad\text{且}\quad
	(f(x),g(x))=1,
\end{equation*}
则\begin{equation*}
	f(x) \mid h(x).
\end{equation*}
\begin{proof}
当\(h(x)=0\)时,
有\(f(x) \mid h(x)\).

当\(h(x)\neq0\)时,
因为\((f(x),g(x))=1\),
所以,存在\(u(x),v(x) \in K[x]\),
使得\begin{equation*}
	u(x) f(x) + v(x) g(x) = 1.
\end{equation*}
等式两边同乘\(h(x)\),
得\begin{equation*}
	u(x) f(x) h(x) + v(x) g(x) h(x) = h(x).
\end{equation*}
因为\(f(x) \mid g(x) h(x)\),
所以用\(f(x)\)整除上式左端,
就有\(f(x) \mid h(x)\).
\end{proof}
\end{property}

\begin{property}\label{theorem:多项式.互素.性质2}
%@see: 《高等代数(第三版 下册)》(丘维声) P19 性质2
在\(K[x]\)中,如果\begin{equation*}
	f(x) \mid h(x)
	\quad\text{且}\quad
	g(x) \mid h(x)
	\quad\text{且}\quad
	(f(x),g(x))=1,
\end{equation*}
则\begin{equation*}
	f(x) g(x) \mid h(x).
\end{equation*}
\begin{proof}
因为\(f(x) \mid h(x)\),
所以存在\(p(x) \in K[x]\),
使得\(h(x) = p(x) f(x)\).
因为\(g(x) \mid h(x)\),
所以\(g(x) \mid p(x) f(x)\).
因为\((g(x),f(x))=1\),
所以\(g(x) \mid p(x)\).
因此存在\(q(x) \in K[x]\),
使得\(p(x) = q(x) g(x)\).
于是\(h(x) = q(x) g(x) f(x)\),
那么\(f(x) g(x) \mid h(x)\).
\end{proof}
\end{property}
应该注意到,当\(f(x) \mid h(x)\)、\(g(x) \mid h(x)\)且\(f(x) g(x) \mid h(x)\)时,
不一定有\((f(x),g(x))=1\).
例如,取\(f(x)=g(x)=x,h(x)=x^2\),
就有\((f(x),g(x))=x\).

\begin{property}\label{theorem:多项式.互素.性质3}
%@see: 《高等代数(第三版 下册)》(丘维声) P19 性质3
在\(K[x]\)中,如果\begin{equation*}
	(f(x),h(x))=1
	\quad\text{且}\quad
	(g(x),h(x))=1,
\end{equation*}
则\begin{equation*}
	(f(x) g(x),h(x))=1.
\end{equation*}
\begin{proof}
因为\((f(x),g(x))=1\),
\((g(x),h(x))=1\),
所以存在\(u_1(x),u_2(x),v_1(x),v_2(x) \in K[x]\),
使得\begin{gather*}
	u_1(x) f(x) + v_1(x) h(x) = 1, \\
	u_2(x) g(x) + v_2(x) h(x) = 1.
\end{gather*}
将上面两个等式相乘,
得\begin{equation*}
	u_1(x) u_2(x) f(x) g(x)
	+ [
		u_1(x) f(x) v_2(x)
		+ v_1(x) u_2(x) g(x)
		+ v_1(x) v_2(x) h(x)
	] h(x)
	= 1.
\end{equation*}
根据\cref{theorem:多项式.两个多项式互素的充分必要条件}
得\((f(x) g(x),h(x))=1\).
\end{proof}
\end{property}

\subsection{最大公因式、互素的概念推广}
最大公因式和互素的概念可以推广到\(n>2\)个多项式的情形.
\begin{definition}
%@see: 《高等代数(第三版 下册)》(丘维声) P20 定义3
在\(K[x]\)中,
如果多项式\(c(x)\)能整除多项式\(f_i(x)\ (i=1,2,\dotsc,n)\)的每一个,
那么把\(c(x)\)称为这\(n\)个多项式的一个\DefineConcept{公因式}.
\end{definition}

\begin{definition}
%@see: 《高等代数(第三版 下册)》(丘维声) P20 定义3
在\(K[x]\)中,
设多项式\(d(x)\)是\(f_i(x)\ (i=1,2,\dotsc,n)\)的一个公因式.
如果\(f_i(x)\ (i=1,2,\dotsc,n)\)的每一个公因式都能整除\(d(x)\),
那么把\(d(x)\)称为这\(n\)个多项式的一个\DefineConcept{最大公因式}.
\end{definition}

用数学归纳法可以证明,
在\(K[x]\)中,
任意\(n\geq2\)个多项式
\(f_1(x),\dotsc,f_n(x)\)的最大公因式存在,
并且如果\(d_1(x)\)是\(f_1(x),\dotsc,f_{n-1}(x)\)的一个最大公因式,
则\(d_1(x)\)与\(f_n(x)\)的最大公因式就是\(f_1(x),\dotsc,f_{n-1}(x),f_n(x)\)的最大公因式.
因此我们依然可以逐次使用辗转相除法求出\(n\)个多项式的一个最大公因式.

从定义可知,
\(n\)个多项式\(f_1(x),\dotsc,f_n(x)\)的最大公因式在相伴的意义下是唯一的.
对于\(n\)个不全为零的多项式\(f_1(x),\dotsc,f_n(x)\),
我们约定使用\begin{equation*}
	(f_1(x),\dotsc,f_n(x))
\end{equation*}表示首项系数是\(1\)的那个最大公因式.
于是我们断言\begin{equation*}
%@see: 《高等代数(第三版 下册)》(丘维声) P20 公式(6)
	(f_1(x),\dotsc,f_n(x))
	= ((f_1(x),\dotsc,f_{n-1}(x)),f_n(x)).
\end{equation*}
从上式出发,根据\cref{theorem:多项式.辗转相除法},
存在\(u_1(x),\dotsc,u_n(x) \in K[x]\),
使得\begin{equation*}
%@see: 《高等代数(第三版 下册)》(丘维声) P20 公式(7)
	u_1(x) f_1(x) + \dotsb + u_n(x) f_n(x)
	= (f_1(x),\dotsc,f_n(x)).
\end{equation*}

\begin{definition}
%@see: 《高等代数(第三版 下册)》(丘维声) P20 定义4
如果\(K[x]\)中\(n\geq2\)个多项式\(f_1(x),\dotsc,f_n(x)\)满足\begin{equation*}
	(f_1(x),\dotsc,f_n(x)) = 1,
\end{equation*}
那么称“\(f_1(x),\dotsc,f_n(x)\)~\DefineConcept{互素}”.
\end{definition}

与\cref{theorem:多项式.两个多项式互素的充分必要条件} 一样,
我们可以证明:
在\(K[x]\)中,
\(n\)个多项式\(f_1(x),\dotsc,f_n(x)\)
互素的充分必要条件是
存在\(K[x]\)中多项式\(u_1(x),\dotsc,u_n(x)\)
使得\begin{equation*}
%@see: 《高等代数(第三版 下册)》(丘维声) P20 公式(8)
	u_1(x) f_1(x) + \dotsb + u_n(x) f_n(x) = 1.
\end{equation*}
但要注意点,\(n>2\)个多项式互素时,
它们不一定两两互素.
例如,多项式\begin{equation*}
	f_1(x) = x+1, \qquad
	f_2(x) = x^2+3x+2, \qquad
	f_3(x) = x-1
\end{equation*}满足\begin{equation*}
	(f_1(x),f_2(x))=x+1, \qquad
	(f_1(x),f_2(x),f_3(x))=1,
\end{equation*}
也就是说\(f_1(x),f_2(x),f_3(x)\)互素,
但是\(f_1(x),f_2(x)\)不互素.

\subsection{数域扩张下的不变性}
我们还要指出一点,
设\(K\)与\(F\)都是数域,
并且\(K \subseteq F\).
设\(f(x),g(x) \in K[x]\),
则我们也可以把\(f(x)\)与\(g(x)\)看成是\(F[x]\)中的多项式.
注意\(f(x)\)与\(g(x)\)在\(K[x]\)中的公因式
和它们在\(F[x]\)中的公因式不一定相同.
例如,设\begin{equation*}
	f(x) = x^2+1, \qquad
	g(x) = x^3+x^2+x+1,
\end{equation*}
则\(f(x)\)与\(g(x)\)在\(\mathbb{R}[x]\)中没有一次公因式,
但是它们在\(\mathbb{C}[x]\)中有一次公因式\(x+\iu\)与\(x-\iu\).
容易看出它们在\(\mathbb{R}[x]\)中的最大公因式是\(x^2+1\),
在\(\mathbb{C}[x]\)中的最大公因式也是\(x^2+1\).
一般地,我们有如下结论.

\begin{proposition}
%@see: 《高等代数(第三版 下册)》(丘维声) P20 命题5
设\(F,K\)都是数域,且\(F \supseteq K\),
则对于\(K[x]\)中任意两个多项式\(f(x)\)与\(g(x)\),
它们在\(K[x]\)中的首项系数为\(1\)的最大公因式
与它们在\(F[x]\)中的首项系数为\(1\)的最大公因式相同.
也就是说,当数域扩大时,\(f(x)\)与\(g(x)\)的首项系数为\(1\)的最大公因式不改变.
\begin{proof}
若\(f(x)=g(x)=0\),
则\(f(x)\)与\(g(x)\)在\(K[x]\)中的最大公因式是零多项式,
在\(F[x]\)中的最大公因式也是零多项式.
下面设\(f(x)\)与\(g(x)\)不全为零.
设\(d_1(x)\)是\(f(x)\)与\(g(x)\)在\(K[x]\)中的首项系数为\(1\)的最大公因式,
设\(d_2(x)\)是\(f(x)\)与\(g(x)\)在\(F[x]\)中的首项系数为\(1\)的最大公因式.
在\(K[x]\)中对\(f(x)\)与\(g(x)\)作辗转相除法,
设最后一个不等于零的余式是\(r_s(x)\),
其首项系数为\(c\),
则\(d_1(x) = \frac1c r_s(x)\);
由于每一步带余除法也可看成是在\(F[x]\)中进行的(根据带余除法的唯一性),
因此\(r_s(x)\)也是\(f(x)\)与\(g(x)\)在\(F[x]\)中的一个最大公因式,
从而\begin{equation*}
	d_2(x) = \frac1c r_s(x)
	= d_1(x).
	\qedhere
\end{equation*}
\end{proof}
\end{proposition}

\begin{corollary}
%@see: 《高等代数(第三版 下册)》(丘维声) P21 推论6
设\(F,K\)都是数域,且\(F \supseteq K\),
\(f,g \in K[x]\),
则\(f(x)\)与\(g(x)\)在\(K[x]\)中互素的充分必要条件是:
\(f(x)\)在\(g(x)\)在\(F[x]\)中互素.
也就是说,互素性不随数域的扩大而改变.
\begin{proof}
容易看出\begin{align*}
	&\text{$f(x)$与$g(x)$在$K[x]$中互素} \\
	&\iff \text{在$K[x]$中,$(f(x),g(x))=1$} \\
	&\iff \text{在$F[x]$中,$(f(x),g(x))=1$} \\
	&\iff \text{$f(x)$与$g(x)$在$F[x]$中互素}.
	\qedhere
\end{align*}
\end{proof}
\end{corollary}

\begin{example}
%@see: 《高等代数(第三版 下册)》(丘维声) P21 习题7.3 2.
证明:在\(K[x]\)中,
如果\(d(x)\)既是\(f(x)\)与\(g(x)\)的倍式和,
又是\(f(x)\)与\(g(x)\)的一个公因式,
则\(d(x)\)是\(f(x)\)与\(g(x)\)的一个最大公因式.
\begin{proof}
设\(c(x)\)是\(f(x)\)与\(g(x)\)的一个公因式,
则\(c(x) \mid f(x)\)且\(c(x) \mid g(x)\).
又设\begin{equation*}
	d(x) = u(x) f(x) + v(x) g(x),
\end{equation*}
其中\(u(x),v(x) \in K[x]\).
那么由\cref{theorem:多项式.整除的线性性}
可知\(c(x) \mid d(x)\).
由定义可知\(d(x)\)是\(f(x)\)与\(g(x)\)的一个最大公因式.
\end{proof}
\end{example}

\begin{example}\label{example:最大公因式.最大公因式除多项式的商式互素}
%@see: 《高等代数(第三版 下册)》(丘维声) P21 习题7.3 4.
%@see: 《高等代数(大学高等代数课程创新教材 第二版 下册)》(丘维声) P31 例2
证明:在\(K[x]\)中,
如果\(f(x),g(x)\)不全为零,
则\begin{equation*}
	\left(
		\frac{f(x)}{(f(x),g(x))},
		\frac{g(x)}{(f(x),g(x))}
	\right)=1.
\end{equation*}
\begin{proof}
设\(f(x) = u(x) (f(x),g(x)),
g(x) = v(x) (f(x),g(x))\).
由\cref{theorem:多项式.辗转相除法} 可知\begin{equation*}
	(f(x),g(x)) = p(x) f(x) + q(x) g(x),
\end{equation*}
其中\(p(x),q(x) \in K[x]\).
于是\begin{align*}
	(f(x),g(x))
	&= p(x) u(x) (f(x),g(x)) + q(x) v(x) (f(x),g(x)) \\
	&= [p(x) u(x) + q(x) v(x)] (f(x),g(x)),
\end{align*}
消去\((f(x),g(x))\)得\begin{equation*}
	p(x) u(x) + q(x) v(x) = 1.
\end{equation*}
由\cref{theorem:多项式.两个多项式互素的充分必要条件} 可知
\(u(x)\)与\(v(x)\)互素,
所以\begin{equation*}
	\left(
		\frac{f(x)}{(f(x),g(x))},
		\frac{g(x)}{(f(x),g(x))}
	\right)
	= (u(x),v(x))
	= 1.
	\qedhere
\end{equation*}
\end{proof}
\end{example}

\begin{example}
%@see: 《高等代数(第三版 下册)》(丘维声) P21 习题7.3 5.
证明:在\(K[x]\)中,
如果\(f(x),g(x)\)不全为零,
并且\begin{equation*}
	u(x) f(x) + v(x) g(x) = (f(x),g(x)),
\end{equation*}
则\((u(x),v(x))=1\).
\begin{proof}
设\(f(x) = p(x) (f(x),g(x)),
g(x) = q(x) (f(x),g(x))\),
其中\(p(x),q(x) \in K[x]\).
那么\begin{align*}
	u(x) f(x) + v(x) g(x)
	&= u(x) p(x) (f(x),g(x))
	+ v(x) q(x) (f(x),g(x)) \\
	&= [u(x) p(x) + v(x) q(x)] (f(x),g(x)).
\end{align*}
根据题设有\(u(x) p(x) + v(x) q(x) = 1\),
于是\((u(x),v(x)) = 1\).
\end{proof}
\end{example}

\begin{example}
%@see: 《高等代数(第三版 下册)》(丘维声) P22 习题7.3 6.
%@see: 《高等代数(大学高等代数课程创新教材 第二版 下册)》(丘维声) P32 例4
证明:在\(K[x]\)中,
如果\((f,g)=1\),
那么\((fg,f+g)=1\).
\begin{proof}
设\((f,g)=1\),
由\cref{theorem:多项式.两个多项式互素的充分必要条件}
可知\(uf+vg=1\),
其中\(u,v \in K[x]\).
于是\begin{equation*}
	(u-v)f+v(f+g)=1;
\end{equation*}
再次利用\cref{theorem:多项式.两个多项式互素的充分必要条件}
便知\((f,f+g)=1\).
同理有\begin{equation*}
	(v-u)g+u(f+g)=1,
\end{equation*}
即\((g,f+g)=1\).
由\cref{theorem:多项式.互素.性质3}
可知\((fg,f+g)=1\).
\end{proof}
\end{example}

\begin{example}
%@see: 《高等代数(第三版 下册)》(丘维声) P22 习题7.3 7.
%@see: 《高等代数(大学高等代数课程创新教材 第二版 下册)》(丘维声) P32 例3
设\(f,g \in K[x]\),
并且\(a,b,c,d \in K\)满足\(ad-bc\neq0\).
证明:\((af+bg,cf+dg)=(f,g)\).
\begin{proof}
由\cref{theorem:多项式.整除的线性性}
可知\begin{equation*}
	u \mid f \land u \mid g
	\implies
	u \mid af+bg,
	u \mid cf+dg,
\end{equation*}
于是\begin{equation*}
	\Set{ u \given \text{$u$是$f$与$g$的公因式} }
	\subseteq
	\Set{ u \given \text{$u$是$af+bg$与$cf+dg$的公因式} }.
\end{equation*}
现在来证\begin{equation*}
	\Set{ u \given \text{$u$是$af+bg$与$cf+dg$的公因式} }
	\subseteq
	\Set{ u \given \text{$u$是$f$与$g$的公因式} }.
\end{equation*}
令\begin{equation*}
	p(af+bg)+q(cf+dg)
	= (pa+qc)f+(pb+qd)g
	= f+g,
\end{equation*}
建立关于\(p,q\)的线性方程组\begin{equation*}
	\left\{ \begin{array}{l}
		ap+cq=1, \\
		bp+dq=1.
	\end{array} \right.
\end{equation*}
因为系数行列式\(\begin{vmatrix}
	a & c \\
	b & d
\end{vmatrix}\neq0\),
所以上述线性方程组有唯一解.
这就是说\begin{equation*}
	u \mid af+bg \land u \mid cf+dg
	\implies
	u \mid f \land u \mid g.
\end{equation*}
综上所述,我们有\begin{equation*}
	\Set{ u \given \text{$u$是$af+bg$与$cf+dg$的公因式} }
	= \Set{ u \given \text{$u$是$f$与$g$的公因式} }.
\end{equation*}
那么由\cref{theorem:多项式.最大公因式.命题1}
可知\begin{equation*}
	\Set{ u \given \text{$u$是$af+bg$与$cf+dg$的最大公因式} }
	= \Set{ u \given \text{$u$是$f$与$g$的最大公因式} },
\end{equation*}
因此\((af+bg,cf+dg)=(f,g)\).
\end{proof}
\end{example}

\begin{example}
%@see: 《高等代数(第三版 下册)》(丘维声) P22 习题7.3 8.
%@see: 《高等代数(大学高等代数课程创新教材 第二版 下册)》(丘维声) P32 例5
证明:在\(K[x]\)中,如果\((f(x),g(x))=1\),
则对任意正整数\(m\),
有\((f(x^m),g(x^m))=1\).
\begin{proof}
由于\((f(x),g(x))=1\),
所以存在\(u(x),v(x) \in K[x]\),
使得\(u(x) f(x) + v(x) g(x) = 1\).
由于\(K[x]\)可看成是\(K\)的一个扩环,
因此不定元\(x\)可用\(x^m\)代入,
于是有\(u(x^m) f(x^m) + v(x^m) g(x^m) = 1\).
又因为\(u(x^m),v(x^m) \in K[x]\),
所以\((f(x^m),g(x^m))=1\).
\end{proof}
\end{example}

\begin{example}
%@see: 《高等代数(第三版 下册)》(丘维声) P22 习题7.3 9.
证明:\(K[x]\)中两个非零多项式\(f(x)\)与\(g(x)\)不互素的充分必要条件是
存在两个非零多项式\(u(x),v(x)\)
使得\begin{gather*}
	u(x) f(x) = v(x) g(x), \\
	\deg u(x) < \deg g(x), \\
	\deg v(x) < \deg f(x).
\end{gather*}
\begin{proof}
令\(h=(f,g)\).
设\(f = ph,
g = qh\),
其中\(p,q \in K[x]\).
因为\begin{gather*}
	\text{$f$与$g$互素}
	\iff
	h=1
	\iff
	\deg h=0, \\
	\text{$f,g$是非零多项式}
	\implies
	\deg h\geq0,
\end{gather*}
所以有\([\text{$f$与$g$不互素}
\iff
\deg h>0]\)成立.

先证必要性.
假设\(f\)与\(g\)不互素,
那么\(h\neq0\).
于是可以从方程\(
	uf
	= uph
	= vqh
	= vg
\)中消去\(h\)
得\(up=vq\).
容易看出,当\(u=q,v=p\)时,就有\(uf=vg\)成立.
又因为\(\deg f=\deg(ph)=\deg p+\deg h\),
而\(\deg h>0\),
所以\(\deg v=\deg f-\deg h<\deg f\);
同理\(\deg u<\deg g\).

再证充分性.
% 假设\(uf=vg,\deg u<\deg g,\deg v<\deg f\).
用反证法.
假设\(f\)与\(g\)互素,
且\(uf=vg\).
根据\cref{theorem:多项式.互素.性质1},
由于\(f \mid uf\),
所以\(f \mid vg\),
从而\(f \mid v\),
那么必有\(\deg f \leq \deg v\).
同理可得\(g \mid u\),
继而必有\(\deg g \leq \deg u\).
\end{proof}
\end{example}

\begin{example}
设矩阵\(\vb{A}\)满足\(\vb{A}^3+\vb{E}=2\vb{A}\),其中\(\vb{E}\)是单位矩阵,
证明:\(2\vb{A}^2+\vb{A}-\vb{E}\)可逆.
\begin{proof}
令\(f(x)=x^3-2x+1,
g(x)=2x^2+x-1\),
因式分解可得\begin{equation*}
	f(x) = (x-1)(x^2+x-1),
	\qquad
	g(x) = (2x-1)(x+1).
\end{equation*}
显然\(f(x)\)与\(g(x)\)在\(\mathbb{C}\)上没有公共根,互素.
故根据\cref{theorem:多项式.两个多项式互素的充分必要条件},
存在\(u(x),v(x) \in K[x]\),
使得\begin{equation*}
	u(x) \cdot (x^3-2x+1) + v(x) \cdot (2x^2+x-1) = 1,
\end{equation*}
代入矩阵\(\vb{A}\),并注意到\(\vb{A}^3-2\vb{A}+\vb{E}=\vb0\),得到\begin{equation*}
	v(\vb{A}) \cdot (2\vb{A}^2+\vb{A}-\vb{E}) = \vb{E},
\end{equation*}
也就是说,矩阵\(2\vb{A}^2+\vb{A}-\vb{E}\)可逆,
其逆矩阵为\(v(\vb{A})\),
而\(v(\vb{A})\)可以通过辗转相除法得到.
\end{proof}
\end{example}

\begin{example}\label{example:矩阵乘积的秩.矩阵的一次多项式的秩之和.取等条件1}
%\cref{example:矩阵乘积的秩.矩阵的一次多项式的秩之和}
设\(\vb{A}\)是数域\(K\)上的\(n\)阶方阵.
证明:若\(\vb{A}^2=\vb{E}\),则\begin{equation*}
	\rank(\vb{A}+\vb{E})+\rank(\vb{A}-\vb{E})=n.
\end{equation*}
\begin{proof}
由于\(x+1\)与\(x-1\)互素,
根据\cref{theorem:多项式.两个多项式互素的充分必要条件},
存在\(u(x),v(x) \in K[x]\),
使得\begin{equation*}
	u(x) \cdot (x+1) + v(x) \cdot (x-1) = 1.
\end{equation*}
代入矩阵\(\vb{A}\),
得\begin{equation*}
	u(\vb{A}) (\vb{A}+\vb{E}) + v(\vb{A}) (\vb{A}-\vb{E}) = \vb{E}.
\end{equation*}

考虑\(2n\)阶方阵
\begin{align*}
	\begin{bmatrix}
		\vb{A}+\vb{E} & \vb0 \\
		\vb0 & \vb{A}-\vb{E}
	\end{bmatrix}
	&\to
	\begin{bmatrix}
		\vb{A}+\vb{E} & u(\vb{A}) (\vb{A}+\vb{E}) \\
		\vb0 & \vb{A}-\vb{E}
	\end{bmatrix} \\
	&\to
	\begin{bmatrix}
		\vb{A}+\vb{E} & u(\vb{A}) (\vb{A}+\vb{E}) + v(\vb{A}) (\vb{A}-\vb{E}) \\
		\vb0 & \vb{A}-\vb{E}
	\end{bmatrix}
	=\begin{bmatrix}
		\vb{A}+\vb{E} & \vb{E} \\
		\vb0 & \vb{A}-\vb{E}
	\end{bmatrix} \\
	&\to
	\begin{bmatrix}
		(\vb{A}+\vb{E})-\vb{E}(\vb{A}+\vb{E}) & \vb{E} \\
		\vb0-(\vb{A}-\vb{E})(\vb{A}+\vb{E}) & \vb0
	\end{bmatrix}
	=\begin{bmatrix}
		\vb0 & \vb{E} \\
		\vb0 & \vb0
	\end{bmatrix}.
\end{align*}
于是\(\rank(\vb{A}+\vb{E})+\rank(\vb{A}-\vb{E})=n\).
\end{proof}
\end{example}

\section{最小公倍式}
\begin{definition}
在\(K[x]\)中,如果\(c(x)\)既是\(f(x)\)的倍式,又是\(g(x)\)的倍式,
则称“\(c(x)\)是\(f(x)\)与\(g(x)\)的一个\DefineConcept{公倍式}”.
\end{definition}

\begin{definition}
%@see: 《高等代数(第三版 下册)》(丘维声) P22 习题7.3 10.
设\(f(x),g(x) \in K[x]\),
\(m(x)\)是\(f(x)\)与\(g(x)\)的一个公倍式.
如果\(f(x)\)与\(g(x)\)的任一公倍式都是\(m(x)\)的倍式,
则称“\(m(x)\)是\(f(x)\)与\(g(x)\)的一个\DefineConcept{最小公倍式}”.
\end{definition}

我们约定,用\begin{equation*}
	[f(x), g(x)]
\end{equation*}表示首项系数是\(1\)的那个最小公倍式.

\begin{example}
%@see: 《高等代数(第三版 下册)》(丘维声) P22 习题7.3 10.(1)
%@see: 《高等代数(大学高等代数课程创新教材 第二版 下册)》(丘维声) P35 例10(1)
证明:\(K[x]\)中任意两个多项式都有最小公倍式,
并且在相伴的意义下是唯一的.
\begin{proof}
首先,零多项式的倍式只有零多项式,
从而任一多项式\(f\)与\(0\)的最小公倍式是\(0\).

下面设\(f,g\)都是数域\(K\)上的非零多项式.
令\(d=(f,g)\),
则存在\(u,v \in K[x]\)
使得\(f=ud,g=vd\).
又令\(m=uvd\),
显然\(f \mid m\)且\(g \mid m\),
也就是说\(m\)是\(f\)与\(g\)的一个公倍式.
设\(c \in K[x]\)是\(f\)与\(g\)的一个公倍式,
则存在\(p,q \in K[x]\)
使得\(c=pf,c=qg\).
于是\(pf=qg\),
从而\(pud=qvd\),
消去\(d\)便得\(pu=qv\),
可见\(u \mid qv\);
但是由\cref{example:最大公因式.最大公因式除多项式的商式互素}
有\((u,v)=1\);
因此根据\cref{theorem:多项式.互素.性质1}
必有\(u \mid q\),
从而存在\(h \in K[x]\)
使得\(q=hu\).
于是\(c=hug=hm\),
因此\(m \mid c\),
也就是说\(m\)是\(f\)与\(g\)的最小公倍式.

假设\(m_1,m_2\)都是\(f\)与\(g\)的最小公倍式,
则\(m_1 \mid m_2,
m_2 \mid m_1\),
因此\(m_1 \sim m_2\).
\end{proof}
\end{example}

\begin{example}
%@see: 《高等代数(第三版 下册)》(丘维声) P22 习题7.3 10.(2)
%@see: 《高等代数(大学高等代数课程创新教材 第二版 下册)》(丘维声) P35 例10(2)
证明:如果\(f(x),g(x)\)的首项系数都是\(1\),
则\begin{equation*}
	[f(x),g(x)]
	= \frac{f(x) g(x)}{(f(x),g(x))}.
\end{equation*}
%TODO proof
\end{example}

\section{不可约多项式,唯一因式分解定理}
我们已经知道,数域\(K\)上的一元多项式环\(K[x]\)具有带余除法.
由此推导出,\(K[x]\)中任意两个多项式都有最大公因式.
现在我们利用这些结论来研究\(K[x]\)的结构.
与整数环\(\mathbb{Z}\)类比:
每一个正整数都能表示成有限多个素数的乘积.
我们不禁发问:\(K[x]\)中每一个多项式是否能表示成有限多个具有类似“素数”那样的性质的多项式的乘积?
联系我们对素数的定义,
对于一个大于\(1\)的正整数\(p\),如果它的正因子只有\(1\)和\(p\),那么称其为素数.
我们可以给出如下概念:
\begin{definition}
%@see: 《高等代数(第三版 下册)》(丘维声) P24 定义1
\(K[x]\)中一个次数大于零的多项式\(p(x)\),
如果它在\(K[x]\)中的因式只有零次多项式和\(p(x)\)的相伴元,
则称“\(p(x)\)是数域\(K\)上的一个\DefineConcept{不可约多项式}(irreducible polynomial)”;
否则称“\(p(x)\)是数域\(K\)上的一个\DefineConcept{可约多项式}(reducible polynomial)”.
%@see: https://mathworld.wolfram.com/IrreduciblePolynomial.html
\end{definition}

\begin{property}\label{theorem:多项式.不可约多项式.性质1}
%@see: 《高等代数(第三版 下册)》(丘维声) P24 性质1
\(K[x]\)中不可约多项式\(p(x)\)与任一多项式\(f(x)\)的关系只有两种可能:
\begin{enumerate}
	\item 要么\(p(x) \mid f(x)\).
	\item 要么\(p(x)\)与\(f(x)\)互素.
\end{enumerate}
\begin{proof}
由于\((p(x),f(x))\)是\(p(x)\)的因式,
而\(p(x)\)不可约,
因此\((p(x),f(x))\)是零次多项式,
即\((p(x),f(x)) \sim p(x)\).
从而\((p(x),f(x))=1\),
即\(p(x) \mid (p(x),f(x))\).
利用整除的传递性得出\(p(x) \mid f(x)\).
\end{proof}
\end{property}

\begin{property}\label{theorem:多项式.不可约多项式.性质2}
%@see: 《高等代数(第三版 下册)》(丘维声) P25 性质2
在\(K[x]\)中,如果\(p(x)\)不可约,且\(p(x) \mid f(x) g(x)\),
则\(p(x) \mid f(x)\)或\(p(x) \mid g(x)\).
\begin{proof}
如果\(p(x) \mid f(x)\),
则结论成立.
下面设\(p(x) \nmid f(x)\).
由于\(p(x)\)不可约,
因此根据\cref{theorem:多项式.不可约多项式.性质1}
得\((p(x),f(x))=1\).
于是从\(p(x) \mid f(x) g(x)\)
得出\(p(x) \mid g(x)\).
\end{proof}
\end{property}

利用数学归纳法,\cref{theorem:多项式.不可约多项式.性质2} 可以推广为:
在\(K[x]\)中,如果\(p(x)\)不可约,
且\[
	p(x) \mid f_1(x) f_2(x) \dotsm f_s(x),
\]
则对于某个\(j \in \Set{1,2,\dotsc,s}\),有\(p(x) \mid f_j(x)\).

\begin{property}\label{theorem:多项式.不可约多项式.性质3}
%@see: 《高等代数(第三版 下册)》(丘维声) P25 性质3
\(K[x]\)中,\(p(x)\)不可约,当且仅当\(p(x)\)不能分解成两个次数较\(p(x)\)的次数低的多项式的乘积.
\begin{proof}
必要性.
如果\(p(x)\)不可约,
则\(p(x)\)的因式只有零次多项式和\(p(x)\)的相伴元,
因此\(p(x)\)不能分解成两个次数较\(p(x)\)的次数低的多项式的乘积.

充分性.
用反证法.
假设\(p(x)\)可约,
则\(p(x)\)有因式\(g(x)\)使得\(0<\deg g(x)<\deg p(x)\).
从而存在\(h(x) \in K[x]\)
使得\[
	p(x) = h(x) g(x),
\]
于是\[
	\deg p(x) = \deg h(x) + \deg g(x).
\]
由此推出\(0<\deg h(x)<\deg p(x)\),
这与已知条件矛盾!
\end{proof}
\end{property}

从\cref{theorem:多项式.不可约多项式.性质3} 立即得出,
\(K[x]\)中的每一个\(1\)次多项式一定是不可约多项式.

\begin{theorem}[唯一因式分解定理]\label{theorem:多项式.唯一因式分解定理}
%@see: 《高等代数(第三版 下册)》(丘维声) P25 定理1
\(K[x]\)中每一个次数大于零的多项式\(f(x)\)都能唯一地分解成数域\(K\)上有限多个不可约多项式的乘积.
\begin{proof}
先证分解式\[
	f(x) = p_1(x) p_2(x) \dotsm p_s(x)
\]的存在性.
利用数学归纳法.
因为一次多项式都是不可约的,
所以\(n=1\)时,存在性成立.
假设对于次数小于\(n\)的多项式,存在性成立.
现在来看\(n\)次多项式\(f(x)\).
如果\(f(x)\)是不可约多项式,
则存在性显然成立.
如果\(f(x)\)是可约多项式,
则有\(f(x)=f_1(x) f_2(x)\),
其中\(f_1(x) \in K[x]\),
并且\(\deg f_1(x) < \deg f(x)\),
\(\deg f_2(x) < \deg f(x)\).
由归纳假设,\(f_1(x)\)与\(f_2(x)\)都可以分解成数域\(K\)上有限多个不可约多项式的乘积,
那么只要把\(f_1(x)\)与\(f_2(x)\)的分解式合起来就可以得到\(f(x)\)的一个分解式.
于是由归纳原理,存在性普遍成立.

现在证唯一性.
假设\(f(x)\)有两个分解式:\[
	f(x) = p_1(x) p_2(x) \dotsm p_s(x),
	\quad\text{和}\quad
	f(x) = q_1(x) q_2(x) \dotsm q_t(x),
\]
其中\(p_i(x),q_j(x)\ (i=1,\dotsc,s;j=1,\dotsc,t)\)都是数域\(K\)上的不可约多项式.
我们对第一个分解式中不可约因式个数\(s\)作归纳法.
当\(s=1\)时,
\(f(x) = p_1(x)\),
则\(f(x)\)是不可约多项式.
由于不可约多项式的因式只有它的相伴元和零次多项式,
所以由\(q_1(x) \mid f(x)\)可得\(q_1(x) \sim f(x)\),
从而\(f(x) = c q_1(x)\ (c \in K-\{0\})\).
因此\(t=1\)且\(p_1(x) \sim q_1(x)\).
假设当第一个分解式的不可约因式的个数为\(s-1\)时,唯一性成立.
现在来看第一个分解式的不可约因式的个数为\(s\)的情形.
由于两个分解式相等,我们有\(p_1(x) \mid q_1(x) \dotsm q_t(x)\),
因此\(p_1(x)\)必能整除其中的一个.
不妨设\(p_1(x) \mid q_1(x)\).
因为\(q_1(x)\)也是不可约多项式,
所以\(p_1(x) \sim q_1(x)\),
即\[
	p_1(x) = c_1 q_1(x), \qquad
	c_1 \in K-\{0\},
\]
将上式代入分解式,并在等号两边消去\(q_1(x)\),可得\[
	p_2(x) \dotsm p_s(x) = c_1^{-1} q_2(x) \dotsm q_t(x).
\]
由归纳假设有\(s-1=t-1\),即\(s=t\),
并且我们只要适当排列因式的次序就能发现\[
	p_2(x) \sim c_1^{-1} q_2(x),
	p_3(x) \sim q_3(x),
	\dotsc,
	p_s(x) \sim q_s(x).
\]
也就是说\(p_i(x) \sim q_i(x)\ (i=1,\dotsc,s)\).
因此由归纳原理,唯一性普遍成立.
\end{proof}
\end{theorem}

从\hyperref[theorem:多项式.唯一因式分解定理]{唯一因式分解定理}可以看出,
\(f(x)\)的任一不可约因式一定与\(f(x)\)的分解式中的某一个不可约因式相伴.
因此,\(f(x)\)的分解式给出了它在相伴意义下的全部不可约因式.

\(K[x]\)中的唯一因式分解定理在理论上非常重要,
但是至今仍没有一个统一的方法来做因式分解,
也就是没有统一的方法求出一个次数大于零的多项式的所有不可约因式.

在多项式\(f(x)\)的分解式中,可以把每一个不可约因式的首项系数提出来,
使它们称为首项系数为\(1\)的多项式,再把相同的不可约因式的乘积写成乘幂的形式,
于是\(f(x)\)的分解式成为
\begin{equation}\label{equation:多项式.标准分解式}
	f(x) = c p_1^{r_1}(x) p_2^{r_2}(x) \dotsm p_m^{r_m}(x),
\end{equation}
其中\(c\)是\(f(x)\)的首项系数,
\(p_1(x),p_2(x),\dotsc,p_m(x)\)是不同的首项系数为\(1\)的不可约多项式,
\(\AutoTuple{r}{m}\)是正整数.
我们把分解式 \labelcref{equation:多项式.标准分解式}
称为“\(f(x)\)的\DefineConcept{标准分解式}”.

从理论研究的角度,如果已知两个多项式\(f(x)\)与\(g(x)\)的标准分解式\[
	f(x) = a p_1^{k_1}(x) \dotsm p_l^{k_l}(x) p_{l+1}^{k_{l+1}}(x) \dotsm p_m^{k_m}(x),
\]\[
	g(x) = b p_1^{t_1}(x) \dotsm p_l^{t_l}(x) q_{l+1}^{t_{l+1}}(x) \dotsm q_s^{t_s}(x),
\]
则\(f(x)\)与\(g(x)\)的最大公因式为\[
	(f(x),g(x))
	= p_1^{\min\{k_1,t_1\}}(x) \dotsm p_l^{\min\{k_l,t_l\}}(x).
\]

由于把多项式分解成不可约因式的乘积没有统一的方法,
因此上述最最大公因式的方法不能代替辗转相除法.

\begin{example}
%@see: 《高等代数(第三版 下册)》(丘维声) P27 例1
证明:\(x^2-2\)在有理数域上不可约.
\begin{proof}
若\(x^2-2\)在有理数域\(\mathbb{Q}\)上可约,
则它的标准分解式为\[
	x^2-2=(x+a)(x+b),
	\qquad a,b\in\mathbb{Q}.
\]
由于实数域\(\mathbb{R}\)是有理数域\(\mathbb{Q}\)的一个扩域,
所以上式也可以看成是\(x^2-2\)在实数域\(\mathbb{R}\)上的一个不可约因式分解.
另一方面我们知道\(x^2-2\)在\(\mathbb{R}\)上有如下的不可约因式分解:\[
	x^2-2=(x+\sqrt2)(x-\sqrt2).
\]
由\(\mathbb{R}[x]\)中的唯一因式分解定理得\[
	x+a=x+\sqrt2
	\quad\text{或}\quad
	x+a=x-\sqrt2.
\]
由此推出\(a=\sqrt2\)或\(a=-\sqrt2\),
这与\(a\in\mathbb{Q}\)矛盾!
因此\(x^2-2\)在\(\mathbb{Q}\)上不可约.
\end{proof}
\end{example}

\begin{example}
%@see: 《高等代数(第三版 下册)》(丘维声) P28 习题7.4 5.
证明:数域\(K\)上一个次数大于零的多项式\(f(x)\)
与\(K[x]\)中某一不可约多项式的正整数次幂相伴的充分必要条件是
对于任意\(g(x) \in K[x]\),
必有\((f(x),g(x))=1\),
或者存在一个正整数\(m\),
使得\(f(x) \mid g^m(x)\).
%TODO proof
\end{example}

\begin{example}
%@see: 《高等代数(第三版 下册)》(丘维声) P28 习题7.4 6.
证明:数域\(K\)上一个次数大于零的多项式\(f(x)\)
与\(K[x]\)中某一不可约多项式的正整数次幂相伴的充分必要条件是
对于任意\(g(x),h(x) \in K[x]\),
从\(f(x) \mid g(x) h(x)\)可以推出\(f(x) \mid g(x)\),
或者存在一个正整数\(m\),
使得\(f(x) \mid h^m(x)\).
%TODO proof
\end{example}

\begin{example}
%@see: 《高等代数(第三版 下册)》(丘维声) P28 习题7.4 7.
%@see: 《高等代数(大学高等代数课程创新教材 第二版 下册)》(丘维声) P41 例2
在\(K[x]\)中,设\((f,g_2)=1\),证明:\((fg_1,g_2)=(g_1,g_2)\).
\begin{proof}
易见\((g_1,g_2)\)是\(fg_1\)与\(g_2\)的一个公因式.

若\(c(x) \mid f(x) g_1(x)\)
且\(c(x) \mid g_2(x)\),
由于\((f,g_2)=1\),
因此存在\(u(x),v(x) \in K[x]\),
使得\(u(x) f(x) + v(x) g_2(x) = 1\),
那么当\(g_1(x)\neq0\)时有\[
	u(x) f(x) g_1(x) + v(x) g_1(x) g_2(x) = g_1(x),
\]
因此\(c(x) \mid g_1(x)\),
从而\(c(x) \mid (g_1,g_2)\),
因此\((fg_1,g_2)=(g_1,g_2)\).

若\(g_1(x)=0\),
则\((fg_1,g_2)=(0,g_2)=(g_1,g_2)\).
\end{proof}
\end{example}

\begin{example}
%@see: 《高等代数(大学高等代数课程创新教材 第二版 下册)》(丘维声) P42 习题7.4 4.
证明:在\(K[x]\)中,
对于不全为零的多项式\(f(x)\)与\(g(x)\),
以及任意正整数\(m\),
有\[
	(f^m(x),g^m(x))=(f(x),g(x))^m.
\]
%TODO proof
\end{example}

\section{重因式}
\subsection{重因式}
上一节我们已证明\(K[x]\)中每一个次数大于零的多项式\(f(x)\)能唯一地分解成
数域\(K\)上有限多个不可约多项式的乘积.
如果\(f(x)\)的分解式中每一个不可约因式只出现\(1\)次,
这种情形是特别重要的情形.
这一节我们要给出识别这种情形的一个统一的方法.

\begin{definition}
%@see: 《高等代数(第三版 下册)》(丘维声) P29 定义1
设\(f(x),p(x) \in K[x]\).
如果\begin{enumerate}
	\item \(p(x)\)是不可约多项式,
	\item \(p^k(x) \mid f(x)\),
	\item \(p^{k+1}(x) \nmid f(x)\),
\end{enumerate}
那么称“\(p(x)\)是\(f(x)\)的~\DefineConcept{\(k\)重因式}”.

如果\(k=0\),则\(p(x) \nmid f(x)\),因此\(p(x)\)不是\(f(x)\)的因式.
如果\(k=1\),则把\(p(x)\)称为“\(f(x)\)的\DefineConcept{单因式}”.
如果\(k>1\),则把\(p(x)\)称为“\(f(x)\)的\DefineConcept{重因式}”.
\end{definition}

显然,如果\(f(x)\)的标准分解式为\begin{equation*}
	f(x) = c p_1^{r_1}(x) p_2^{r_2}(x) \dotsm p_m^{r_m}(x),
\end{equation*}
则\(p_i^{r_i}(x)\ (i=1,2,\dotsc,m)\)是\(f(x)\)的\(r_i\)重因式.
指数\(r_i = 1\)的那些不可约因式是单因式,
指数\(r_i > 1\)的那些不可约因式是重因式.
因此,\(f(x)\)的分解式中每一个不可约因式只出现\(1\)的情形也就是\(f(x)\)没有重因式的情形.
如何判别一个多项式有没有重因式呢?
由于没有一般的方法来求一个多项式的标准分解式,
因此我们必须寻找别的方法来判断一个多项式有没有重因式.

\begin{proposition}\label{theorem:多项式.重因式的等价定义}
设\(f(x),p(x) \in K[x]\),
\(p(x)\)是不可约多项式.
\(p(x)\)是\(f(x)\)的\(k\)重因式的充分必要条件是:
存在\(g(x) \in K[x]\),
使得\(f(x) = p^k(x) g(x)\)且\(p(x) \nmid g(x)\).
\begin{proof}
必要性.
假设\(p(x)\)是\(f(x)\)的\(k\)重因式,
由定义可知,\(p^k(x) \mid f(x)\)且\(p^{k+1} \nmid f(x)\).
于是,存在\(g(x) \in K[x]\),使得\(f(x) = p^k(x) g(x)\).
用反证法,假设\(p(x) \mid g(x)\),
那么存在\(h(x) \in K[x]\),使得\(g(x) = p(x) h(x)\).
于是\(f(x) = p^{k+1}(x) h(x)\),从而\(p^{k+1}(x) \mid f(x)\),矛盾!
因此\(p(x) \nmid g(x)\).

充分性.
假设\(f(x) = p^k(x) g(x)\)且\(p(x) \nmid g(x)\),
显然有\(p^k(x) \mid f(x)\).
用反证法,假设\(p^{k+1}(x) \mid f(x)\),
那么存在\(h(x) \in K[x]\),使得\(f(x) = p^{k+1}(x) h(x)\),
于是\(p^{k+1}(x) h(x) = p^k(x) g(x)\).
因为\(p(x)\)是不可约多项式,
所以可以运用消去律得到\(p(x) h(x) = g(x)\),
从而有\(p(x) \mid g(x)\),矛盾!
因此\(p^{k+1}(x) \nmid f(x)\).
\end{proof}
\end{proposition}

\subsection{形式导数}
我们先来看一个简单例子,以便从中受到启发.

设\(f(x) = (x+1)^3 \in \mathbb{R}[x]\),
这时\(f(x)\)有重因式.
如果我们把\(f(x)\)看成数学分析中讨论的多项式函数,
那么对\(f(x)\)可以求导数,得\(f'(x) = 3(x+1)^2\).
于是\((f(x),f'(x)) = (x+1)^2\).
从这个例子受到启发,
有可能运用导数概念以及最大公因式的求法来讨论一个多项式有没有重因式的问题.
由于我们现在讲的多项式是任意数域\(K\)上一个不定元的多项式,
而数学分析中的多项式函数是实变量\(x\)的函数,
其导数概念涉及极限概念,
因此我们不能直接引用数学分析中多项式函数的导数概念,
我们必须给任意数域\(K\)上一元多项式的导数下个定义,
当然这个定义是从数学分析中多项式函数的导数公式得到启发的.

\begin{definition}\label{definition:多项式.导数}
%@see: 《高等代数(第三版 下册)》(丘维声) P30 定义2
对于\(K[x]\)中的多项式\begin{equation*}
	f(x) = a_n x^n + a_{n-1} x^{n-1} + \dotsb + a_1 x + a_0,
\end{equation*}
我们把\(K[x]\)中的多项式\begin{equation*}
	n a_n x^{n-1} + (n-1) a_{n-1} x^{n-2} + \dotsb + a_1
\end{equation*}
叫做“\(f(x)\)的\DefineConcept{一阶导数}”,记作\(f'(x)\).
我们还把\(f'(x)\)的一阶导数称为“\(f(x)\)的\DefineConcept{二阶导数}”,记作\(f''(x)\);
把\(f''(x)\)的一阶导数称为“\(f(x)\)的\DefineConcept{三阶导数}”,记作\(f'''(x)\);
把\(f'''(x)\)的一阶导数称为“\(f(x)\)的\DefineConcept{四阶导数}”,记作\(f^{(4)}(x)\);
以此类推.
\end{definition}
%\cref{example:微分中值定理.一元高次方程的根的存在性}

从\cref{definition:多项式.导数} 立即得出,
一个\(n\)次多项式的导数是一个\(n-1\)次多项式,
它的\(n\)阶导数是\(K\)中一个非零数,
它的\(n+1\)阶导数等于零.
零多项式的导数是零多项式.

根据\cref{definition:多项式.导数},可以验证得到\(K[x]\)中多项式的导数的基本公式:\begin{gather}
	[f(x)+g(x)]' = f'(x) + g'(x), \\
	[c f(x)]' = c f'(x), \quad c \in K, \\
	[f(x) g(x)]' = f'(x) g(x) + f(x) g'(x), \\
	[f^m(x)]' = m f^{m-1}(x) f'(x).
\end{gather}

\subsection{判定多项式有无重因式}
让我们回头再看一遍之前举的简单例子,
不可约多项式\(x+1\)是\(f(x) = (x+1)^3\)的\(3\)重因式.
由于按\cref{definition:多项式.导数} 和上述公式可得出,
\(f'(x) = 3(x+1)^2\),
因此\(x+1\)是\(f'(x)\)的\(2\)重因式.
我们从这个例子得出的结论具有一般性.

\begin{theorem}\label{theorem:多项式.多项式及其导数的重因式}
%@see: 《高等代数(第三版 下册)》(丘维声) P30 定理1
设\(K\)是数域,在\(K[x]\)中,
如果不可约多项式\(p(x)\)是\(f(x)\)的一个\(k\ (k\geq1)\)重因式,
则\(p(x)\)是\(f(x)\)的导数\(f'(x)\)的一个\(k-1\)重因式.
特别地,多项式\(f(x)\)的单因式不是\(f(x)\)的导数\(f'(x)\)的因式.
\begin{proof}
因为\(p(x)\)是\(f(x)\)的\(k\)重因式,
所以由\cref{theorem:多项式.重因式的等价定义} 可知,
存在\(g(x) \in K[x]\),
使得\begin{equation*}
	f(x) = p^k(x) g(x), \qquad
	p(x) \nmid g(x).
\end{equation*}
求\(f(x)\)的导数,
得\begin{equation*}
	f'(x) = p^{k-1}(x) [ k p'(x) g(x) + p(x) g'(x) ].
\end{equation*}
因为根据\cref{theorem:多项式.整除的序},
不可约多项式不能整除它的导数,
即\(p(x) \nmid k p'(x)\),
又因为\(p(x) \nmid g(x)\),
并且\(p(x)\)是不可约多项式,
所以\(p(x) \nmid k p'(x) g(x)\).
但是\(p(x) \mid p(x) g'(x)\),
所以\(p(x) \nmid [k p'(x) g(x) + p(x) g'(x)]\).
因此\(p(x)\)是\(f'(x)\)的\(k-1\)重因式.
\end{proof}
\end{theorem}

\begin{corollary}\label{theorem:多项式.不可约多项式是重因式的充分必要条件}
%@see: 《高等代数(第三版 下册)》(丘维声) P31 推论2
设\(K\)是数域,在\(K[x]\)中,不可约多项式\(p(x)\)是\(f(x)\)的重因式的充分必要条件是:
\(p(x)\)是\(f(x)\)与\(f'(x)\)的公因式.
\begin{proof}
必要性.
设不可约多项式\(p(x)\)是\(f(x)\)的\(k\)重因式,
其中\(k>1\),
则由\cref{theorem:多项式.多项式及其导数的重因式} 可知,
\(p(x)\)是\(f'(x)\)的\(k-1\)重因式,
从而\(p(x)\)是\(f(x)\)与\(f'(x)\)的公因式.

充分性.
设不可约多项式\(p(x)\)是\(f(x)\)与\(f'(x)\)的公因式.
由\cref{theorem:多项式.多项式及其导数的重因式} 可知,
\(p(x)\)不是\(f(x)\)的单因式,
所以\(p(x)\)是\(f(x)\)的重因式.
\end{proof}
\end{corollary}
从\cref{theorem:多项式.不可约多项式是重因式的充分必要条件} 立即得到:
\(K[x]\)中次数大于零的多项式\(f(x)\)有重因式的充分必要条件是\(f(x)\)及其导数\(f'(x)\)
有次数大于零的公因式.
于是我们有下述定理.
\begin{theorem}\label{theorem:多项式.高次多项式没有重因式的充分必要条件}
%@see: 《高等代数(第三版 下册)》(丘维声) P31 定理3
设\(K\)是数域,\(K[x]\)中次数大于零的多项式\(f(x)\)没有重因式的充分必要条件是:
\(f(x)\)与它的导数\(f'(x)\)互素.
\end{theorem}

\cref{theorem:多项式.高次多项式没有重因式的充分必要条件} 表明,
判断数域\(K\)上的一个多项式\(f(x)\)有没有重因式,
只要利用辗转相除法去计算最大公因式\((f(x),f'(x))\).
不仅如此,由于在数域扩大时,两个多项式的互素性不改变,一个多项式的导数也不改变,
因此我们还有下述结论.

\begin{proposition}
%@see: 《高等代数(第三版 下册)》(丘维声) P31 命题4
设\(F,K\)都是数域,\(F \supseteq K\).
对于\(f \in K[x]\),
\(f(x)\)在\(K[x]\)中没有重因式的充分必要条件是:
\(f(x)\)有无重因式不会随数域的扩大而改变,
即当把\(f(x)\)看成\(F[x]\)中的多项式时,
\(f(x)\)在\(F[x]\)中没有重因式.
\end{proposition}

在一些问题中,如果多项式\(f(x)\)有重因式,
我们希望求出一个多项式\(g(x)\),
它没有重因式,
并且在不计重数时,它与\(f(x)\)含有完全相同的不可约因式.
下面我们来讨论如何求解\(g(x)\).

设\(K[x]\)中的多项式\(f(x)\)的标准分解式是\begin{equation*}
	f(x) = c p_1^{r_1}(x) p_2^{r_2}(x) \dotsm p_m^{r_m}(x),
\end{equation*}
根据\cref{theorem:多项式.多项式及其导数的重因式} 得\begin{equation*}
	f'(x) = p_1^{r_1-1}(x) p_2^{r_2-1}(x) \dotsm p_m^{r_m-1}(x) h(x),
\end{equation*}
其中\(h(x)\)不能被\(p_i(x)\ (i=1,2,\dotsc,m)\)整除.
于是我们可以利用辗转相除法求得最大公因式\begin{equation*}
	(f(x),f'(x))
	= p_1^{r_1-1}(x) p_2^{r_2-1}(x) \dotsm p_m^{r_m-1}(x).
\end{equation*}
因此用\((f(x),f'(x))\)除\(f(x)\)所得商式是\begin{equation*}
	c p_1(x) p_2(x) \dotsm p_m(x),
\end{equation*}
把这个商式记作\(g(x)\),
我们便得到一个没有重因式的多项式\(g(x)\),
它与\(f(x)\)含有完全相同的不可约因式(不计重数).

去掉\(f(x)\)的不可约因式的重数有不少好处.
例如,为了求\(f(x)\)的所有不可约因式,
我们可以先用上述方法得到一个没有重因式的多项式\(g(x)\),
它与\(f(x)\)含有完全相同的不可约因式(不计重数),
但由于\(g(x)\)的次数小于\(f(x)\)的次数,
所以\(g(x)\)的不可约因式可能比较容易求得.
如果我们求出了\(g(x)\)的一个不可约因式\(p_i(x)\),
那么用带余除法可求出\(p_i(x)\)在\(f(x)\)中的重数.
又如,在实际问题中常常需要求出一个多项式\(f(x)\)的根,
由于有些求多项式的根的算法只对没有重因式的多项式适用,
因此我们可以先去掉\(f(x)\)的不可约因式的重数,
得到一个没有重因式的多项式\(g(x)\),
而\(g(x)\)与\(f(x)\)有完全相同的根(不计重数).

\begin{example}
证明:\(\mathbb{Q}[x]\)中的多项式\begin{equation*}
	f(x) = 1+x+\frac{x^2}{2!}+\dotsb+\frac{x^n}{n!}
\end{equation*}没有重因式.
\begin{proof}
求\(f(x)\)的导数得\begin{equation*}
	f'(x) = 1+x+\dotsm+\frac{x^{n-1}}{(n-1)!}.
\end{equation*}
于是\begin{equation*}
	f(x) = f'(x) + \frac{x^n}{n!}.
\end{equation*}
那么\begin{equation*}
	(f(x),f'(x))
	= \left(
		f'(x)+\frac{x^n}{n!},
		f'(x)
	\right)
	= \left(
		\frac{x^n}{n!},
		f'(x)
	\right).
\end{equation*}
由于\(\frac{x^n}{n!}\)的不可约因式只有\(x\)(不计重数),
而\(x \nmid f'(x)\),所以\begin{equation*}
	\left(
		\frac{x^n}{n!},
		f'(x)
	\right)
	= 1,
\end{equation*}
从而\((f(x),f'(x))=1\).
因此,\(f(x)\)没有重因式.
\end{proof}
\end{example}

\begin{example}
%@see: 《高等代数(第三版 下册)》(丘维声) P33 习题7.5 2.
设实系数多项式\(f(x)=x^3+2ax+b\).
试问:\(a,b\)应满足什么条件,
\(f(x)\)才能有重因式?
%TODO
\begin{solution}

\end{solution}
\end{example}

\begin{example}
%@see: 《高等代数(第三版 下册)》(丘维声) P33 习题7.5 4.
证明:在\(K[x]\)中,
若不可约多项式\(p(x)\)是\(f(x)\)的导数\(f'(x)\)的\(k-1\ (k\geq1)\)重因式,
并且\(p(x)\)是\(f(x)\)的因式,
则\(p(x)\)是\(f(x)\)的\(k\)重因式.
%TODO proof
\end{example}

\begin{example}
%@see: 《高等代数(第三版 下册)》(丘维声) P33 习题7.5 5.
证明:在\(K[x]\)中,
不可约多项式\(p(x)\)是\(f(x)\)的\(k\ (k\geq1)\)重因式的充分必要条件是:
\(p(x)\)是\(f(x),f'(x),\dotsc,f^{(k-1)}(x)\)的因式,
但不是\(f^{(k)}(x)\)的因式.
%TODO proof
\end{example}

\begin{example}
%@see: 《高等代数(第三版 下册)》(丘维声) P33 习题7.5 7.
证明:\(K[x]\)中一个\(n\ (n\geq1)\)次多项式\(f(x)\)能被它的导数整除的充分必要条件是:
它与一个一次因式的\(n\)次幂相伴.
%TODO proof
\end{example}

\section{多项式的根}\label{section:多项式.多项式的根}
从唯一因式分解定理知道,
\(K[x]\)中每一个次数大于零的多项式都能唯一地分解成数域\(K\)上有限多个不可约多项式的乘积.
由此看出,不可约多项式之于\(K[x]\)正如砖块之于城市,
这促使我们取搞清楚\(K[x]\)中不可约多项式有哪些.
我们已经知道,\(K[x]\)中每一个一次多项式都是不可约的.
于是需要进一步研究的是,
\(K[x]\)中有没有次数大于\(1\)的不可约多项式?
显然,在\(K[x]\)中,如果\(p(x)\)是次数大于\(1\)的不可约多项式,则\(p(x)\)没有一次因式.
从这点受到启发,首先需要研究\(K[x]\)中一个多项式\(f(x)\)有一次因式的充分必要条件.
为此,我们需要用一次多项式去除\(f(x)\),观察它的余式.

\begin{theorem}[余数定理]\label{theorem:多项式.余数定理}
%@see: 《高等代数(第三版 下册)》(丘维声) P33 定理1
在\(K[x]\)中,用\(x-a\)去除\(f(x)\)所得的余式是\(f(a)\).
\begin{proof}
作带余除法,得\begin{equation*}
	f(x) = h(x) (x-a) + r(x), \qquad
	\deg r(x) < \deg(x-a)=1.
\end{equation*}
可见\(r(x)\)要么是零多项式,要么是零次多项式.
不妨设\(r(x)=r \in K\).
于是上式成为\begin{equation*}
	f(x) = h(x) (x-a) + r, \qquad
	r \in K.
\end{equation*}
在上式中,\(x\)用\(a\)代入,得\(f(a) = r\).
因此用\(x-a\)去除\(f(x)\)所得的余式是\(f(a)\).
\end{proof}
\end{theorem}

\begin{corollary}\label{theorem:多项式.余数定理的推论}
%@see: 《高等代数(第三版 下册)》(丘维声) P34 推论2
在\(K[x]\)中,\(x-a\)整除\(f(x)\)当且仅当\(f(a)=0\).
\begin{proof}
由\cref{theorem:多项式.带余除法.推论}
和\cref{theorem:多项式.余数定理}
立即可得.
\end{proof}
\end{corollary}

从\cref{theorem:多项式.余数定理的推论} 受到启发,引出多项式的根的概念.

\begin{definition}\label{theorem:多项式.根的定义}
%@see: 《高等代数(第三版 下册)》(丘维声) P34 定义1
设\(K\)是一个数域,
\(R\)是一个有单位元的交换环,
且\(R\)可看成是\(K\)的一个扩环.
对于\(K[x]\)中一个多项式\(f(x)\),
如果\(R\)中有一个元素\(c\)使得\(f(c)=0\),
则称“\(c\)是\(f(x)\)在\(R\)中的一个\DefineConcept{根}”.
\end{definition}

多项式在复数域中的根称为\DefineConcept{复根}.
实系数多项式在实数域中的根称为\DefineConcept{实根}.
有理系数多项式在有理数域中的根称为\DefineConcept{有理根}.

从\cref{theorem:多项式.根的定义} 和\cref{theorem:多项式.余数定理的推论} 立即得到下述重要结论:
\begin{theorem}[贝祖定理]\label{theorem:多项式.贝祖定理}
%@see: 《高等代数(第三版 下册)》(丘维声) P34 定理3
在\(K[x]\)中,\(x-a\)整除\(f(x)\)当且仅当\(a\)是\(f(x)\)在\(K\)中的一个根.
\end{theorem}

从\cref{theorem:多项式.贝祖定理} 看出,
\(K[x]\)中的多项式\(f(x)\)有一次因式的充分必要条件是\(f(x)\)在\(K\)中有根.

利用根与一次因式的关系,
对于K[x]中的多项式在K中的根,我们可以定义“重根”的概念:
如果\(x-a\)是\(f(x)\)的\(k\)重因式,
那么我们把\(a \in K\)称为“\(f(x) \in K[x]\)的一个\(k\)重根”.
当\(k=1\)时,\(a\)称为\DefineConcept{单根};
当\(k>1\)时,\(a\)称为\DefineConcept{重根}.

另外,再次利用根与一次因式的关系,
我们还可以得到\(K[x]\)中的多项式在\(f(x)\)在\(K\)中的根的数目的一个上界:
\begin{theorem}\label{theorem:多项式.根的数目的上界}
%@see: 《高等代数(第三版 下册)》(丘维声) P34 定理4
\(K[x]\)中的\(n\ (n\geq0)\)次多项式在\(K\)中至多有\(n\)个根(重根按重数计算).
\begin{proof}
零次多项式没有根,因此结论成立.

设多项式\(f(x)\)的次数为\(\deg f(x)=n>0\).
把\(f(x)\)分解成不可约多项式的乘积.
根据\cref{theorem:多项式.贝祖定理} 以及根的重数的定义容易看出,
\(f(x)\)在\(K\)中的根的数目(重根按重数计算)
等于分解式中一次因式的数目(重因式按重数计算),
这个数目当然不超过\(f(x)\)的次数\(n\).
\end{proof}
\end{theorem}

从\cref{theorem:多项式.根的数目的上界} 可以得到一个重要推论:
\begin{corollary}\label{theorem:多项式.根的数目的上界.推论}
%@see: 《高等代数(第三版 下册)》(丘维声) P34 推论5
设\(K[x]\)中两个多项式\(f(x)\)与\(g(x)\)的次数都不超过\(n\).
如果\(K\)中有\(n+1\)个不同元素\(\AutoTuple{a}{n+1}\),
使得\(f(a_i)=g(a_i)\ (i=1,2,\dotsc,n,n+1)\),
则\(f(x)=g(x)\).
\begin{proof}
设\(h(x)=f(x)-g(x)\),
假设\(h(x)\neq0\),
则\begin{equation*}
	0 \leq \deg h \leq \max\{\deg f,\deg g\} \leq n.
\end{equation*}
因为\begin{equation*}
	h(a_i) = f(a_i) - g(a_i) = 0,
	\quad i=1,2,\dotsc,n+1,
\end{equation*}
所以\(h(x)\)在\(K\)中至少有\(n+1\)个不同的根.
这与\cref{theorem:多项式.根的数目的上界} 矛盾.
因此\(h(x)=0\).
于是\(f(x)=g(x)\).
\end{proof}
\end{corollary}

为了研究多项式的根,我们需要多项式函数的概念,并且研究多项式函数与多项式之间的关系.

设\(f \in K[x]\).
对于\(K\)中每一个元素\(a\),
\(x\)用\(a\)代入得\(f(a) \in K\).
于是\(K[x]\)中的一个多项式\(f(x)\)确定了\(K\)到\(K\)的一个映射\begin{equation*}
    f\colon K \to K, a \mapsto f(a).
\end{equation*}
这种由\(K[x]\)中的多项式确定的\(K\)上的函数称为\(K\)上的\DefineConcept{一元多项式函数}.

我们已经知道,\(K[x]\)中的每一个多项式都确定一个\(K\)上的一元多项式函数.
现在要问:给定\(K[x]\)中两个不相等的多项式\(f(x)\)与\(g(x)\),
它们确定的\(K\)上的一元多项式函数\(f\)和\(g\)是否不相等?
\begin{theorem}\label{theorem:多项式.多项式函数是否相等取决于多项式是否相等}
%@see: 《高等代数(第三版 下册)》(丘维声) P35 定理6
数域\(K\)上的两个多项式\(f(x)\)与\(g(x)\)如果不相等,
则它们确定的\(K\)上的一元多项式函数\(f\)与\(g\)也不相等.
\begin{proof}
证逆否命题.
设\(K\)上的一元多项式函数\(f\)与\(g\)相等,
则对\(\forall a \in K\),有\(f(a)=g(a)\).
由于\(K\)是数域,它有无穷多个元素,
于是根据\cref{theorem:多项式.根的数目的上界.推论} 得,
\(f(x)=g(x)\),
即多项式\(f(x)\)与\(g(x)\)相等.
\end{proof}
\end{theorem}

%@see: 《高等代数(第三版 下册)》(丘维声) P36 定理7
我们把数域\(K\)上的所有一元多项式函数组成的集合也记作\(K[x]\).
让一元多项式环\(K[x]\)中的多项式\(f(x)\)对应到它确定的\(K\)上的函数\(f\),
这时从一元多项式环\(K[x]\)到一元多项式函数族\(K[x]\)的一个映射,这里记为\(\sigma\).
显然\(\sigma\)是满射;
根据\cref{theorem:多项式.多项式函数是否相等取决于多项式是否相等},\(\sigma\)又是单射;
因此\(\sigma\)是双射,是一个从一元多项式环\(K[x]\)到一元多项式函数族\(K[x]\)的同构.

\section{实数域上的不可约多项式}
这一节我们要找出实数域上的所有不可约多项式.
由于每一个复数都可以表示成\(a+b\iu\)的形式,
其中\(a,b\)都是实数,
因此我们可以利用复数域上多项式的信息来研究实数域上的不可约多项式.

\begin{theorem}\label{theorem:实数域上的不可约多项式.多项式的复根的共轭也是根}
%@see: 《高等代数(第三版 下册)》(丘维声) P40 定理1
设\(f(x)\)是实系数多项式,
如果\(c\)是\(f(x)\)的一个复根,
则\(c\)的共轭复数\(\overline{c}\)也是\(f(x)\)的一个复根.
\begin{proof}
设\(f(x)=a_n x^n+a_{n-1} x^{n-1}+\dotsb+a_1 x+a_0\),
其中\(a_i\in\mathbb{R}\ (i=0,1,\dotsc,n)\).
因为\(c\)是\(f(x)\)的复根,
所以\begin{equation*}
	f(c)=a_n c^n+a_{n-1} c^{n-1}+\dotsb+a_1 c+a_0=0.
\end{equation*}
在上式两边取共轭,得\begin{equation*}
	a_n \overline{c}^n+a_{n-1} \overline{c}^{n-1}+\dotsb+a_1 \overline{c}+a_0=0,
\end{equation*}
因此\(\overline{c}\)是\(f(x)\)的一个复根.
\end{proof}
\end{theorem}

\begin{theorem}\label{theorem:实数域上的不可约多项式.实数域上的不可约多项式}
%@see: 《高等代数(第三版 下册)》(丘维声) P40 定理2
实数域上的不可约多项式都是一次多项式或判别式小于零的二次多项式.
\begin{proof}
设\(f(x)\in\mathbb{R}[x]\)是不可约的.
把\(f(x)\)看成复系数多项式,
根据代数基本定理,
\(f(x)\)有一个复根\(c\).

如果\(c\)是实数,
则\(f(x)\)是\(\mathbb{R}[x]\)中有一次因式\(x-c\).
因为\(f(x)\)不可约,
所以一定有\(f(x) \sim (x-c)\),
从而有\(f(x)=a(x-c)\),
其中\(a\)是非零实数,
因此\(f(x)\)是一次多项式.

如果\(c\)是虚数,
根据\cref{theorem:实数域上的不可约多项式.多项式的复根的共轭也是根},
\(\overline{c}\)也是\(f(x)\)的一个复根.
由于\(c\neq\overline{c}\),
所以\((x-c,x-\overline{c})=1\).
在\(\mathbb{C}[x]\)中,
\((x-c) \mid f(x),
(x-\overline{c}) \mid f(x)\),
从而根据\cref{theorem:多项式.互素.性质2} 得\begin{equation*}
	(x-c)(x-\overline{c}) \mid f(x).
\end{equation*}
于是\begin{equation*}
	(x-c)(x-\overline{c})
	=x^2-(c+\overline{c})x+c\overline{c},
\end{equation*}
而\(c+\overline{c}\)和\(c\overline{c}\)都是实数.
既然在\(\mathbb{C}[x]\)中
有\([x^2-(c+\overline{c})x+c\overline{c}] \mid f(x)\),
所以在\(\mathbb{R}[x]\)中
也有\([x^2-(c+\overline{c})x+c\overline{c}] \mid f(x)\).
由于\(f(x)\)在\(\mathbb{R}[x]\)中不可约,
因此\(f(x)=a[x^2-(c+\overline{c})x+c\overline{c}]\),
其中\(a\)是非零实数;
又因为\begin{align*}
	&\text{实系数二次多项式$f(x)=ax^2+bx+c$不可约} \\
	&\iff \text{$f(x)$在$\mathbb{R}[x]$中没有一次因式} \\
	&\iff \text{$f(x)$没有实根} \\
	&\iff b^2-4ac<0,
\end{align*}
因此\(f(x)\)是判别式小于零的二次多项式.
\end{proof}
\end{theorem}

\begin{theorem}[实系数多项式唯一因式分解定理]
%@see: 《高等代数(第三版 下册)》(丘维声) P41 定理3
每个次数大于零的实系数多项式
在实数域上都可以唯一地分解成
一次因式与判别式小于零的二次因式的乘积.
\begin{proof}
由\cref{theorem:实数域上的不可约多项式.实数域上的不可约多项式,theorem:多项式.唯一因式分解定理}
立即可得.
\end{proof}
\end{theorem}

\section{有理数域上的不可约多项式}
这一节讨论有理数域上的不可约多项式有哪些,
如何判别一个有理系数多项式是否不可约.
这些问题的回答比复系数多项式和实系数多项式困难得多.

在\cref{section:多项式.多项式的根}的开头,
我们曾指出,
在\(K[x]\)中,
如果一个次数大于1的多项式\(p(x)\)不可约,
则\(p(x)\)没有一次因式,
从而\(p(x)\)在\(K\)中没有根,
这样就可以缩小讨论\(\mathbb{Q}[x]\)中不可约多项式的范围.
那么如何判别\(\mathbb{Q}[x]\)中次数大于1的多项式\(f(x)\)有没有有理根呢?
显然\(f(x)\)有有理根当且仅当\(f(x)\)在\(\mathbb{Q}[x]\)中的相伴元也有有理根.
因此很自然地选取\(f(x)\)在\(\mathbb{Q}[x]\)中的一个最简单的相伴元来研究.
例如,设\(f(x)=\frac12x^4+\frac13x^3-2x+1\),
则\(g(x)=3x^4+2x^3-12x+6\)就是\(f(x)\)的一个相伴元.
注意到\(g(x)\)是整系数多项式,
而它的各项系数的最大公因数只有\(\pm1\).
受此启发,我们可以给出如下概念.

\begin{definition}
%@see: 《高等代数(第三版 下册)》(丘维声) P42 定义1
一个非零的整系数多项式\begin{equation*}
	g(x)=b_n x^n+\dotsb+b_1 x+b_0,
\end{equation*}
如果它的各项系数的最大公因数只有\(\pm1\),
则称“\(g(x)\)是\DefineConcept{本原的}”
或“\(g(x)\)是一个\DefineConcept{本原多项式}”.
\end{definition}

\begin{proposition}
任一非零的有理系数多项式都与一个本原多项式相伴.
\begin{proof}
只需求出有理系数多项式\begin{equation*}
	f(x)=a_n x^n+\dotsb+a_1 x+a_0
\end{equation*}的各项系数的分母的最小公倍数\(m\),
提取公因数\(\frac1m\)得到\begin{equation*}
	f(x)=\frac1m(m a_n x^n+\dotsb+m a_1 x+m a_0);
\end{equation*}
接着求出括号内多项式的各项系数的最大公因数\(c\),
就有\begin{equation*}
	f(x)=\frac{c}{m}(b_n x^n+\dotsb+b_1 x+b_0),
\end{equation*}
其中\(b_i=\frac{m a_i}{c}\ (i=0,1,\dotsc,n)\),
\(g(x)=b_n x^n+\dotsb+b_1 x+b_0\)就是与\(f(x)\)相伴的本原多项式.
\end{proof}
\end{proposition}

我们不禁想要知道,
一个非零的有理系数多项式
可以与几个本原多项式相伴?

\begin{lemma}\label{theorem:多项式.有理数域上的不可约多项式.引理1}
%@see: 《高等代数(第三版 下册)》(丘维声) P42 引理1
两个本原多项式\(f(x)\)和\(g(x)\)在\(Q[x]\)中相伴
当且仅当\(f(x)=\pm g(x)\).
\begin{proof}
充分性是显然的.
下面证必要性.
设\(f(x),g(x)\)是相伴的本原多项式,
则存在\(r\in\mathbb{Q}-\{0\}\),
使得\(f(x)=r g(x)\).
设\begin{equation*}
	f(x)=\sum_{i=0}^n a_i x^i, \qquad
	g(x)=\sum_{i=0}^n b_i x^i,
\end{equation*}
其中\(a_i,b_i\in\mathbb{Z}\ (i=0,1,\dotsc,n)\).
假设\(r\neq\pm1\),
不妨设\(r=\frac{q}{p}\),
其中\((p,q)=1\).
于是\(p,q\)两者中至少有一个不等于\(\pm1\).
不妨设\(p\neq\pm1\),
从而有\(p f(x)=q g(x)\).
比较各项系数可知\(p a_i=q b_i\ (i=0,1,\dotsc,n)\).
于是\(p \mid q b_i\).
因为\((p,q)=1\),
所以根据\cref{theorem:多项式.互素.性质1}
有\(p \mid b_i\),
这与“\(g(x)\)是本原多项式”矛盾.
因此\(r=\pm1\),
\(f(x)=\pm g(x)\).
\end{proof}
\end{lemma}

\cref{theorem:多项式.有理数域上的不可约多项式.引理1}
告诉我们,对于一个非零的有理系数多项式\(f(x)\),
与它在\(\mathbb{Q}[x]\)中相伴的本原多项式有且仅有两个,
它们相差一个正负号.
现在我们来研究本原多项式在\(\mathbb{Q}[x]\)中的不可约性问题.
为此首先介绍本原多项式的一个重要性质.

\begin{lemma}[高斯引理]\label{theorem:多项式.有理数域上的不可约多项式.引理2}
%@see: 《高等代数(第三版 下册)》(丘维声) P43 引理2
两个本原多项式的乘积还是本原多项式.
\begin{proof}
设\(
	f(x)=\sum_{i=0}^n a_i x^i,
	g(x)=\sum_{i=0}^n b_i x^i
\)是两个本原多项式,
又设\(
	h(x) = f(x) g(x) = \sum_{i=0}^{n+m} c_i x^i
\),
其中\(c_k=\sum_{i+j=k} a_i b_j\ (k=0,1,\dotsc,n+m)\).

假如\(h(x)\)不是本原多项式,
那么存在一个素数\(p\),
使得\(p\)是\(h(x)\)各项系数的公因式,
即\(p \mid c_k\ (k=0,1,\dotsc,n+m)\).
因为\(f(x)\)是本原的,
所以\(p\)不能同时整除\(f(x)\)的各项系数,
也就是说,存在\(k\ (0\leq k\leq n)\)满足\begin{equation*}
	p \mid a_0,
	p \mid a_1,
	\dotsc
	p \mid a_{k-1},
	p \nmid a_k.
	\eqno(1)
\end{equation*}
同理,存在\(l\ (0\leq l\leq m)\)满足\begin{equation*}
	p \mid b_0,
	p \mid b_1,
	\dotsc
	p \mid b_{l-1},
	p \nmid b_l.
	\eqno(2)
\end{equation*}

考虑\(h(x)\)的\(k+l\)次项的系数\begin{equation*}
	c_{k+l}
	= a_{k+l} b_0
	+ a_{k+l-1} b_1
	+ \dotsb
	+ a_1 b_{k+l-1}
	+ a_0 b_{k+l}.
\end{equation*}
由(1)(2)两式可知\(p \nmid c_{k+l}\)
(注意从\(p \nmid a_k\)且\(p \nmid b_l\)
可以推出\(p \nmid a_k b_l\)),矛盾!
因此\(h(x)\)是本原多项式.
\end{proof}
\end{lemma}

\begin{theorem}\label{theorem:多项式.有理数域上的不可约多项式.本原多项式的分解}
%@see: 《高等代数(第三版 下册)》(丘维声) P43 定理1
一个次数大于零的本原多项式\(g(x)\)在\(\mathbb{Q}\)可约
当且仅当\(g(x)\)可以分解成两个次数都比\(g(x)\)的次数低的本原多项式的乘积.
\begin{proof}
充分性是显然的.
下面证必要性.

设本原多项式\(g(x)\)在\(\mathbb{Q}\)上可约,
则存在\(g_1(x),g_2(x)\in\mathbb{Q}[x]\),
使得\(g(x) = g_1(x) g_2(x)\),
其中\(\deg g_1(x) < \deg g(x)\),
\(\deg g_2(x) < \deg g(x)\).
设\(g_i(x)=r_i h_i(x)\ (i=1,2)\),
其中\(r_1,r_2\in\mathbb{Q}^*\),
而\(h_1(x),h_2(x)\)是本原多项式,
则\begin{equation*}
	g(x) = r_1 r_2 h_1(x) h_2(x).
\end{equation*}
由于根据\cref{theorem:多项式.有理数域上的不可约多项式.引理2}
两个本原多项式的乘积\(h_1(x) h_2(x)\)也是本原多项式,
因此根据\cref{theorem:多项式.有理数域上的不可约多项式.引理1}
有\(r_1 r_2 = \pm1\).
从而\(g(x)=[\pm h_1(x)]\cdot h_2(x)\).
由于\(\deg(\pm h_1(x))
= \deg g_1(x)
< \deg g(x)\),
\(\deg h_2(x)
= \deg g_2(x)
< \deg g(x)\),
因此\(g(x)\)分解成了两个次数较低的本原多项式的乘积.
\end{proof}
\end{theorem}

\begin{corollary}\label{theorem:多项式.有理数域上的不可约多项式.本原多项式的分解.推论}
%@see: 《高等代数(第三版 下册)》(丘维声) P43 推论2
如果一个次数大于零的整系数多项式在\(\mathbb{Q}\)上可约,
则它可以分解成两个次数比它低的整系数多项式的乘积.
\begin{proof}
设\(f(x)\)是一个次数大于零的整系数多项式,
在\(\mathbb{Q}\)上可约,
则\(f(x)= r g(x)\),
其中\(r\in\mathbb{Z}^*\),
\(g(x)\)是本原多项式.
根据\cref{theorem:多项式.有理数域上的不可约多项式.本原多项式的分解}
可知\(g(x)\)可以分解为\(h_1(x),h_2(x)\)
这两个次数都比\(g(x)\)的次数低的本原多项式,
即\(g(x)=h_1(x) h_2(x)\),
从而\(f(x)=[r h_1(x)] h_2(x)\).
这表明\(f(x)\)分解成了两个次数较低的整系数多项式的乘积.
\end{proof}
\end{corollary}

\begin{theorem}\label{theorem:多项式.有理数域上的不可约多项式.高次本原多项式可唯一分解为不可约本原多项式的乘积}
%@see: 《高等代数(第三版 下册)》(丘维声) P44 定理3
每一个次数大于零的本原多项式\(g(x)\)
可以唯一地分解成\(\mathbb{Q}\)上不可约的本原多项式的乘积.
\begin{proof}
设\(g(x)\)是一个次数大于零的本原多项式.

先证可分解性.
对本原多项式的次数\(n\)作数学归纳法.
当\(n=1\)时,
显然有\(g(x)=g(x)\),
且\(g(x)\)不可约.
假设任一次数小于\(n\)的本原多项式
都可以分解成\(\mathbb{Q}\)上不可约的本原多项式的乘积.
现在来看\(n\)次本原多项式\(g(x)\).
如果\(g(x)\)在\(\mathbb{Q}\)上不可约,
则\(g(x)=g(x)\),
可分解性成立.
如果\(g(x)\)在\(\mathbb{Q}\)上可约,
那么根据\cref{theorem:多项式.有理数域上的不可约多项式.本原多项式的分解} 得,
\(g(x)=g_1(x) g_2(x)\),
其中\(g_1(x),g_2(x)\)是本原多项式,
且\(\deg g_1(x) < \deg g(x),
\deg g_2(x) < \deg g(x)\),
可分解性也成立.

再证唯一性.
根据\(\mathbb{Q}[x]\)的\hyperref[theorem:多项式.唯一因式分解定理]{唯一因式分解定理}得,
\(s=t\),
且只要适当排列因式次序
就有\(p_i(x) \sim q_i(x)\ (i=1,2,\dotsc,s)\).
由于\(p_i(x),q_i(x)\)都是本原多项式,
因此\(p_i(x)=\pm q_i(x)\ (i=1,2,\dotsc,s)\).
\end{proof}
\end{theorem}

\begin{theorem}\label{theorem:多项式.有理数域上的不可约多项式.正次整系数多项式的有理根的分子分母分别整除常数项和首项系数}
%@see: 《高等代数(第三版 下册)》(丘维声) P44 定理4
设\(f(x)=a_n x^n+a_{n-1} x^{n-1}+\dotsb+a_1 x+a_0\)
是一个次数\(n\)大于零的整系数多项式.
如果\(\frac{q}{p}\)是\(f(x)\)的一个有理根,
其中\(p,q\)是互素的整数,
那么\(p \mid a_n,
q \mid a_0\).
\begin{proof}
设\(f(x)=r f_1(x)\),
其中\(r\in\mathbb{Z}^*\),
\(f_1(x)\)是本原多项式.
又设\(\frac{q}{p}\)是\(f(x)\)的一个根,
其中\(p,q\)是互素的整数,
则\(\frac{q}{p}\)是\(f_1(x)\)的一个根.
于是在\(\mathbb{Q}[x]\)中,
有\(\left(x-\frac{q}{p}\right) \mid f_1(x)\),
从而\((px-q) \mid f_1(x)\).
由于\((p,q)=1\),
因此\(px-q\)是本原多项式.
根据\cref{theorem:多项式.有理数域上的不可约多项式.高次本原多项式可唯一分解为不可约本原多项式的乘积}
和\hyperref[theorem:多项式.有理数域上的不可约多项式.引理2]{高斯引理}得\begin{equation*}
	f_1(x)=(px-q) g(x),
\end{equation*}
其中\(g(x)=b_{n-1} x^{n-1}+\dotsb+b_1 x+b_0\)是本原多项式.
于是\begin{equation*}
	f(x)=r(px-q) g(x).
\end{equation*}
分别比较上式等号两边的首项系数与常数项,得\begin{equation*}
	a_n = r p b_{n-1}, \qquad
	a_0 = -r q b_0.
\end{equation*}
因此\(p \mid a_n,
q \mid a_0\).
\end{proof}
\end{theorem}

\cref{theorem:多项式.有理数域上的不可约多项式.正次整系数多项式的有理根的分子分母分别整除常数项和首项系数}
的证明过程表明,
如果由互素整数\(p,q\)构成的\(\frac{q}{p}\)是\(f(x)\)的一个有理根,
则存在一个整系数多项式\(g(x)\)
使得\(f(x)=(px-q) g(x)\).
于是\begin{equation*}
	f(1)=(p-q) g(1), \qquad
	f(-1)=-(p+1) g(-1).
\end{equation*}
当\(\frac{q}{p}\neq\pm1\)时,
有\begin{equation*}
	\frac{f(1)}{p-q}=g(1)\in\mathbb{Z}, \qquad
	\frac{f(-1)}{p+q}=-g(-1)\in\mathbb{Z}.
\end{equation*}
因此,如果计算出
\(\frac{f(1)}{p-q}\notin\mathbb{Z}\)
或\(\frac{f(-1)}{p+q}\notin\mathbb{Z}\),
那么可以断言\(\frac{q}{p}\)不是\(f(x)\)的根.
这个判断方法在求整系数多项式的有理根时有用.

\begin{example}
%@see: 《高等代数(第三版 下册)》(丘维声) P45 例1
判断\(f(x)=x^3+x+1\)在有理数域上是否不可约.
\begin{solution}
如果\(f(x)\)在\(\mathbb{Q}\)上可约,
则\(f(x)\)可以分解成两个次数较低的有理系数多项式的乘积.
由于\(\deg f(x)=3\),
因此\(f(x)\)在\(\mathbb{Q}[x]\)中必有一次因式,
从而\(f(x)\)必有有理根.
由于\(a_3=1,a_0=1\),
因此根据\cref{theorem:多项式.有理数域上的不可约多项式.正次整系数多项式的有理根的分子分母分别整除常数项和首项系数},
\(f(x)\)的有理根只可能是\(\pm1\).
但是\(f(1)=3,
f(-1)=-1\).
因此\(\pm1\)都不是\(f(x)\)的根.
这个矛盾表明\(f(x)\)在\(\mathbb{Q}\)上不可约.
\end{solution}
\end{example}

\begin{remark}
如果整系数多项式\(f(x)\)的次数大于3,
那么不能从“\(f(x)\)没有有理根”
便得出\(f(x)\)在\(\mathbb{Q}\)上不可约的结论.
这是因为“\(f(x)\)没有有理根”
只是说明“\(f(x)\)没有一次因式”,
但是\(f(x)\)可能有次数大于1的因式,
从而\(f(x)\)可能可约.
\end{remark}

\begin{example}
%@see: 《高等代数(第三版 下册)》(丘维声) P45 例2
求\(f(x)=3x^4+8x^3+6x^2+3x-2\)的全部有理根.
\begin{solution}
首项系数\(a_4=3\)的因子只有\(\pm1,\pm3\);
常数项\(a_0=-2\)的因子只有\(\pm1,\pm2\).
根据\cref{theorem:多项式.有理数域上的不可约多项式.正次整系数多项式的有理根的分子分母分别整除常数项和首项系数},
\(f(x)\)的有理根只可能是\(\pm1,\pm2,\pm\frac13,\pm\frac23\).

因为\(f(1)=18\neq0,
f(-1)=-4\neq0\),
所以\(\pm1\)不是\(f(x)\)的根.

因为\(\frac{f(-1)}{p+q}=-\frac43\)不是整数,
所以\(2\)不是\(f(x)\)的根.

因为\(\frac{f(1)}{p-q}=6,
\frac{f(-1)}{p+q}=4\),
无法确定\(-2\)不是\(f(x)\)的根,
所以我们需要用综合除法来进一步判断\(-2\)是不是\(f(x)\)的根:\begin{equation*}
	\begin{array}{r|*5r}
		& 3 & 8 & 6 & 3 & -2 \\
		-2 && -6 & -4 & -4 & 2 \\ \cline{2-6}
		& 3 & 2 & 2 & -1 & 0 \\
		&& -6 & 8 & -20 \\ \cline{2-5}
		& 3 & -4 & 10 & -21
	\end{array}
\end{equation*}
这表明\(-2\)是\(f(x)\)的单根.
于是\begin{equation*}
	f(x)=(x+2)(3x^3+2x^2+2x-1).
\end{equation*}

因为\(\frac{f(1)}{p-q}=9,
\frac{f(-1)}{p+q}=-1\),
无法确定\(\frac13\)不是\(f(x)\)的根,
所以我们需要用综合除法来进一步判断\(\frac13\)是不是\(f(x)\)的根:\begin{equation*}
	\def\arraystretch{1.5}
	\begin{array}{r|*5r}
		& 3 & 2 & 2 & -1 \\
		\frac13 && 1 & 1 & 1 \\ \cline{2-5}
		& 3 & 3 & 3 & 0 \\
		&& 1 & \frac43 \\ \cline{2-4}
		& 3 & 4 & \frac{13}3
	\end{array}
\end{equation*}
这表明\(\frac13\)是\(f(x)\)的单根.
于是\begin{equation*}
	f(x)=(x+2)\left(x-\frac13\right)
	(3x^2+3x+3).
\end{equation*}
由于\(x^2+x+1\)没有有理根,
所以\(f(x)\)的全部有理根是\(-2,\frac13\),
它们都是单根.
\end{solution}
\end{example}

下面来探索次数大于1的整系数多项式在\(\mathbb{Q}\)上不可约的条件.
我们来剖析两个例子:
\(x^2+4x+4=(x+2)^2\),
因此它在\(\mathbb{Q}\)上可约.
\(x^2+4x+6\)的虚根为\(-2\pm\sqrt2\iu\),
因此它没有有理根,在\(\mathbb{Q}\)上不可约.
\(x^2+4x+4\)与\(x^2+4x+6\)的共同点是
有素数2整除常数项和一次项系数,2不能整除首项系数;
不同点是2的平方能整除\(x^2+4x+4\)的常数项4,
2的平方不能整除\(x^2+4x+6\)的常数项6.

\begin{theorem}[艾森斯坦判别法]\label{theorem:多项式.有理数域上的不可约多项式.艾森斯坦判别法}
%@see: 《高等代数(第三版 下册)》(丘维声) P47 定理5
设\begin{equation*}
	f(x)=a_n x^n+\dotsb+a_1 x+a_0
\end{equation*}是一个次数\(n\)大于零的整系数多项式.
如果存在一个素数\(p\),
使得\begin{enumerate}
	\item \(p \mid a_i\ (i=0,1,\dotsc,n-1)\);
	\item \(p \nmid a_n\);
	\item \(p^2 \nmid a_0\),
\end{enumerate}
则\(f(x)\)在\(\mathbb{Q}\)上不可约.
\begin{proof}
用反证法.
假设\(f(x)\)在\(\mathbb{Q}\)上可约,
则根据\cref{theorem:多项式.有理数域上的不可约多项式.本原多项式的分解.推论} 得\begin{equation*}
	f(x)=(b_m x^m+\dotsb+b_1 x+b_0)(c_l x^l+\dotsb+c_1 x+c_0),
	\eqno(1)
\end{equation*}
其中\(b_i,c_j\in\mathbb{Z}\ (i=0,1,\dotsc,m;j=0,1,\dotsc,l),
b_m\neq0,
c_l\neq0,
m<n,
l<n,
m+l=n\).
因此\begin{equation*}
	a_n=b_m c_l, \qquad
	a_0=b_0 c_0.
\end{equation*}
已知\(p \mid a_0\),
所以\(p \mid b_0\)或\(p \mid c_0\).
又因为\(p^2 \nmid a_0\),
所以\(p\)不能同时整除\(b_0\)和\(c_0\).
不妨设\(p \mid b_0\)但\(p \nmid c_0\).
因为\(p \nmid a_n\),
所以\(p \nmid b_m\).
假设\(b_0,b_1,\dotsc,b_m\)中第一个不能被\(p\)整除的是\(b_k\),
即\begin{equation*}
	p \mid b_0,
	p \mid b_1,
	\dotsc,
	p \mid b_{k-1},
	p \nmid b_k,
	\qquad
	0<k\leq m.
	\eqno(2)
\end{equation*}
比较(1)式两边\(x^k\)的系数,
得\begin{equation*}
	a_k=b_k c_0+b_{k-1} c_1+\dotsb+b_1 c_{k-1}+b_0 c_k.
	\eqno(3)
\end{equation*}
因为\(k\leq m<n\),
所以\(p \mid a_k\).
于是从(2)(3)两式得
\(p \mid b_k c_0,
p \mid c_0\),
矛盾!
因此\(f(x)\)在\(\mathbb{Q}\)上不可约.
\end{proof}
%@see: https://brilliant.org/wiki/eisensteins-irreducibility-criterion/
%@see: https://maa.org/sites/default/files/pdf/upload_library/22/Ford/Cox-2012.pdf
\end{theorem}

\begin{corollary}
%@see: 《高等代数(第三版 下册)》(丘维声) P47 推论6
在\(\mathbb{Q}[x]\)中存在任意次数的不可约多项式.
\begin{proof}
对任意的正整数\(n\),
设\(f(x)=x^n+2\).
素数\(2\)满足\cref{theorem:多项式.有理数域上的不可约多项式.艾森斯坦判别法} 的条件,
因此\(f(x)\)在\(\mathbb{Q}\)上不可约.
\end{proof}
\end{corollary}

有时候直接用\hyperref[theorem:多项式.有理数域上的不可约多项式.艾森斯坦判别法]{艾森斯坦判别法}无法
判断\(f(x)\)是否在\(\mathbb{Q}\)上不可约,
我们可以通过不定元\(x\)用\(\mathbb{Q}[x]\)中的元素\(x+b\)代入,
得到另一个多项式\(g(x)=f(x+b)\).
对\(g(x)\)运用艾森斯坦判别法,可以判断它是否不可约.
\begin{proposition}\label{theorem:多项式.有理数域上的不可约多项式.一次代入同不可约}
%@see: 《高等代数(第三版 下册)》(丘维声) P47 命题7
设\(f(x)=a_n x^n+\dotsb+a_1 x+a_0\)是次数\(n\)大于零的整系数多项式,
任意取定\(b\in\mathbb{Z}\),
令\(g(x)=f(x+b)\),
则\(f(x)\)在\(\mathbb{Q}[x]\)上不可约的充分必要条件是
\(g(x)\)在\(\mathbb{Q}[x]\)上不可约.
\begin{proof}
先证充分性.
因为\(g(x)\)的首项是\(a_n x^n\),
所以\(\deg g(x)=n=\deg f(x)\).
假如\(f(x)\)在\(\mathbb{Q}\)上可约,
则\begin{equation*}
	f(x)=f_1(x) f_2(x), \qquad
	\deg f_1 < \deg f,
	\deg f_2 < \deg f.
\end{equation*}
\(x\)用\(x+b\)代入,
从上式得\begin{equation*}
	f(x+b)=f_1(x+b) f_2(x+b).
\end{equation*}
上式表明\(g(x)\)在\(\mathbb{Q}\)上可约.

再证必要性.
\(x\)用\(x-b\)代入,
则由\(g(x)=f(x+b)\)
得\(g(x-b)=f(x)\).
从充分性可知,
如果\(f(x)\)在\(\mathbb{Q}\)上不可约,
则\(g(x)\)在\(\mathbb{Q}\)上也不可约.
\end{proof}
\end{proposition}

\begin{example}
%@see: 《高等代数(第三版 下册)》(丘维声) P48 例3
设\(p\)是素数,
证明:分圆多项式\(f(x)=x^{p-1}+x^{p-2}+\dotsb+x+1\)在\(\mathbb{Q}\)上不可约.
\begin{proof}
我们有\begin{align*}
	(x-1) f(x)
	&= (x-1)(x^{p-1}+x^{p-2}+\dotsb+x+1) \\
	&= x^p-1.
\end{align*}
\(x\)用\(x+1\)代入,
由上式得\begin{equation*}
	x f(x+1)
	=(x+1)^p-1
	=\sum_{k=1}^p C_p^k x^k,
\end{equation*}
从而有\begin{equation*}
	f(x+1)
	=\sum_{k=1}^p C_p^k x^{k-1}.
\end{equation*}
令\(g(x)=f(x+1)\).
因为\begin{equation*}
	C_p^k
	=\frac{p(p-1)\dotsm(p-k+1)}{k!},
	\quad1\leq k<p,
\end{equation*}
并且\((p,k!)=1\),
所以\begin{equation*}
	k! \mid (p-1)\dotsm(p-k+1),
\end{equation*}
从而\(p \mid C_p^k\ (1\leq k<p)\).
于是对于\(g(x)\),
素数\(p\)满足艾森斯坦判别法的条件,
所以\(g(x)\)在\(\mathbb{Q}\)上不可约.
从而\(f(x)\)在\(\mathbb{Q}\)上不可约.
\end{proof}
\end{example}

有些整系数多项式,
即使用\cref{theorem:多项式.有理数域上的不可约多项式.一次代入同不可约}
仍不能用\hyperref[theorem:多项式.有理数域上的不可约多项式.艾森斯坦判别法]{艾森斯坦判别法}判定它是否不可约.
对于二次或三次整系数多项式\(f(x)\),
如果它可约,
则它一定有一次因式,
从而它必定有有理根.
于是,只要考察\(f(x)\)有没有有理根,
那么就可以解决二次或三次整系数多项式是否不可约的问题.
对于大于3次的整系数多项式,
则还需要考虑它是否有大于1次的因式.

\section{多元多项式环}
\subsection{多元多项式}
\begin{definition}
%@see: 《高等代数(第三版 下册)》(丘维声) P50 定义1
设\(K\)是一个数域,
用不属于\(K\)的\(n\)个符号\(\AutoTuple{x}{n}\)作表达式\begin{equation*}
	\sum_{\AutoTuple{i}{n}}
	a_{i_1 \dotsm i_n}
	x_1^{i_1} \dotsm x_n^{i_n},
\end{equation*}
其中\(a_{i_1 \dotsm i_n} \in K\),
\(\AutoTuple{i}{n}\)是非负整数,
上式中的每一项称为一个\DefineConcept{单项式},
上式称为\DefineConcept{数域\(K\)上的\(n\)元多项式}.
如果它具有下述性质:
只有有限多个单项式的系数不为零,
并且两个这种形式的表达式相等当且仅当它们除去系数为零的单项式外含有完全相同的单项式,
而系数为零的单项式允许任意删去或添入.
这时,符号\(\AutoTuple{x}{n}\)称为\(n\)个\DefineConcept{无关不定元}.

在数域\(K\)上的\(n\)元多项式中,
如果两个单项式的幂指数都对应相等,
则称这两个单项式为\DefineConcept{同类项}.
我们约定\(n\)元多项式中的单项式都是不同类的,
即要把同类项合并成一项.

如果数域\(K\)上一个\(n\)元多项式的所有系数全为零,
则称它为\DefineConcept{零多项式},记为\(0\).

我们把\(i_1+\dotsb+i_n\)
称为“单项式\(a_{i_1 \dotsm i_n}
x_1^{i_1} \dotsm x_n^{i_n}\)的\DefineConcept{次数}”.

一个\(n\)元多项式\(f(x_1,\dotsc,x_n)\)的系数不为零的单项式的次数的最大值,
称为“\(f(x_1,\dotsc,x_n)\)的\DefineConcept{次数}”.

零多项式的全次数规定为\(-\infty\).
\end{definition}

\begin{example}
\(5x_1^4+3x_1^3x_2+2x_1x_2x_3^2+x_2^3+x_2x_3\)
是3元4次多项式,
其中单项式\(5x_1^4,3x_1^3x_2,2x_1x_2x_3^2\)的次数都是4.
\end{example}

数域\(K\)上所有\(n\)元多项式组成的集合,
记作\(K[x_1,\dotsc,x_n]\).

在\(K[x_1,\dotsc,x_n]\)中定义加法与乘法如下:
\begin{gather}
	\begin{split}
		&\hspace{-20pt}
		\sum_{\AutoTuple{i}{n}}
		a_{i_1 \dotsm i_n}
		x_1^{i_1} \dotsm x_n^{i_n}
		+
		\sum_{\AutoTuple{i}{n}}
		b_{i_1 \dotsm i_n}
		x_1^{i_1} \dotsm x_n^{i_n} \\
		&\defeq
		\sum_{\AutoTuple{i}{n}}
		(a_{i_1 \dotsm i_n} + b_{i_1 \dotsm i_n})
		x_1^{i_1} \dotsm x_n^{i_n},
	\end{split} \\
	\begin{split}
		&\hspace{-20pt}
		\left(
		\sum_{\AutoTuple{i}{n}}
		a_{i_1 \dotsm i_n}
		x_1^{i_1} \dotsm x_n^{i_n}
		\right) \left(
		\sum_{\AutoTuple{j}{n}}
		b_{j_1 \dotsm j_n}
		x_1^{j_1} \dotsm x_n^{j_n}
		\right) \\
		&\defeq
		\sum_{\AutoTuple{s}{n}}
		c_{s_1 \dotsm s_n}
		x_1^{s_1} \dotsm x_n^{s_n},
	\end{split}
\end{gather}
其中\begin{equation}
	c_{s_1 \dotsm s_n}
	= \sum_{i_1+j_1=s_1}
	\sum_{i_2+j_2=s_2}
	\dotso
	\sum_{i_n+j_n=s_n}
	a_{i_1 \dotsm i_n}
	b_{j_1 \dotsm j_n}.
\end{equation}
不难验证\(K[x_1,\dotsc,x_n]\)对于如上定义的加法与乘法成为一个环.
它的零元是零多项式.
它有单位元,即零次多项式\(1\).
它是交换环.
我们把这个环称为\DefineConcept{数域\(K\)上的\(n\)元多项式环}.

显然有\begin{equation}
	\deg(f+g)
	\leq
	\max\{\deg f,\deg g\}.
\end{equation}

先来对\(n\)元多项式\(f(x_1,\dotsc,x_n)\)的各项规定一个排列顺序,
从而给出首项的概念.

每一类单项式\(a_{i_1 \dotsm i_n} x_1^{i_1} \dotsm x_n^{i_n}\)
对应一个\(n\)元有序非负整数组\((i_1,\dotsc,i_n)\),
这个对应是双射.
为了给出各类单项式之间的一个排列顺序的方法,
就只需要对\(n\)元有序非负整数组定义一个先后顺序.

对于两个\(n\)元有序非负整数组
\((i_1,\dotsc,i_n)\)
和\((j_1,\dotsc,j_n)\),
如果\begin{equation*}
	i_1=j_1,
	i_2=j_2,
	\dotsc,
	i_{s-1}=j_{s-1},
	i_s>j_s
	\quad(1\leq s\leq n)
\end{equation*}
则称“\((i_1,\dotsc,i_n)\)~\DefineConcept{先于}~\((j_1,\dotsc,j_n)\)”,
记作\((i_1,\dotsc,i_n)>(j_1,\dotsc,j_n)\).

由上述定义立即看出,
对于任意两个\(n\)元有序非负整数组
\((i_1,\dotsc,i_n)\)
和\((j_1,\dotsc,j_n)\),
关系\begin{gather*}
	(i_1,\dotsc,i_n)>(j_1,\dotsc,j_n), \\
	(i_1,\dotsc,i_n)=(j_1,\dotsc,j_n), \\
	(j_1,\dotsc,j_n)>(i_1,\dotsc,i_n)
\end{gather*}
中有且仅有一个成立.

这里关系“\(>\)”具有传递性,
即,
如果\((i_1,\dotsc,i_n)>(j_1,\dotsc,j_n)\)
且\((j_1,\dotsc,j_n)>(k_1,\dotsc,k_n)\),
那么\((i_1,\dotsc,i_n)>(k_1,\dotsc,k_n)\).
这是因为\(i_l-k_l=(i_l-j_l)+(j_l-k_l)\).

\begin{example}
由\((4,2,3,3)>(4,2,2,4)\)和\((4,2,2,4)>(4,1,4,3)\)
可得\((4,2,3,3)>(4,1,4,3)\).
\end{example}

这样我们的确给出了\(n\)元有序非负整数组之间的一个顺序.
相应地,\(n\)元各类单项式之间也有了一个先后顺序.
这种排列顺序的方法是模仿字典中单词的排列原则给出的,
因而称之为\DefineConcept{字典排列法}.

\begin{example}
多项式\(2x_1^4x_2x_3+x_1x_2^5x_3+6x_1^3\)
按字典排列法写出来就是
\(2x_1^4x_2x_3+6x_1^3+x_1x_2^5x_3\).
\end{example}

按字典排列法写出来的第一个系数不为零的单项式称为\(n\)元多项式的\DefineConcept{首项}.

\begin{example}
多项式\(2x_1^4x_2x_3+x_1x_2^5x_3+6x_1^3\)的首项
是\(2x_1^4x_2x_3\).
要注意,首项不一定具有最大的次数.
多项式\(2x_1^4x_2x_3+x_1x_2^5x_3+6x_1^3\)的次数是7,
而它的首项的次数是6.
\end{example}

\begin{theorem}\label{theorem:多项式.多元多项式环.两个非零多项式的乘积的首项等于它们的首项的乘积}
%@see: 《高等代数(第三版 下册)》(丘维声) P52 定理1
在\(K[x_1,\dotsc,x_n]\)中两个非零多项式的乘积的首项等于它们的首项的乘积.
\begin{proof}
设\(f(x_1,\dotsc,x_n),g(x_1,\dotsc,x_n)\)是\(K[x_1,\dotsc,x_n]\)中两个非零多项式.
设\(f(x_1,\dotsc,x_n)\)的首项是\(a x_1^{p_1} x_2^{p_2} \dotsm x_n^{p_n}\ (a\neq0)\),
\(g(x_1,\dotsc,x_n)\)的首项是\(b x_1^{q_1} x_2^{q_2} \dotsm x_n^{q_n}\ (b\neq0)\).
为了证明\(fg\)的首项是
\(ab x_1^{p_1+q_1} x_2^{p_2+q_2} \dotsm x_n^{p_n+q_n}\),
只要证明\begin{equation*}
	(p_1+q_1,p_2+q_2,\dotsc,p_n+q_n)
\end{equation*}先于\(fg\)中其他单项式的幂指数组就行了.
\(fg\)的其他单项式的幂指数组只有三种可能情形:\begin{equation*}
	(p_1+j_1,p_2+j_2,\dotsc,p_n+j_n)
\end{equation*}或\begin{equation*}
	(i_1+q_1,i_2+q_2,\dotsc,i_n+q_n)
\end{equation*}或\begin{equation*}
	(i_1+j_1,i_2+j_2,\dotsc,i_n+j_n),
\end{equation*}
其中\begin{equation*}
	(p_1,p_2,\dotsc,p_n)
	>
	(i_1,i_2,\dotsc,i_n), \qquad
	(q_1,q_2,\dotsc,q_n)
	>
	(j_1,j_2,\dotsc,j_n).
\end{equation*}
显然有\begin{gather*}
	(p_1+q_1,p_2+q_2,\dotsc,p_n+q_n)
	>
	(i_1+q_1,i_2+q_2,\dotsc,i_n+q_n), \\
	(i_1+q_1,i_2+q_2,\dotsc,i_n+q_n)
	>
	(i_1+j_1,i_2+j_2,\dotsc,i_n+j_n),
\end{gather*}
于是由传递性得\begin{equation*}
	(p_1+q_1,p_2+q_2,\dotsc,p_n+q_n)
	>
	(i_1+j_1,i_2+j_2,\dotsc,i_n+j_n).
\end{equation*}
这就证明了
\(ab x_1^{p_1+q_1} x_2^{p_2+q_2} \dotsm x_n^{p_n+q_n}\)
不可能与\(fg\)中其他的单项式相消,
而且它先于\(fg\)中其他的单项式,
它就是\(fg\)的首项.
\end{proof}
\end{theorem}

从\cref{theorem:多项式.多元多项式环.两个非零多项式的乘积的首项等于它们的首项的乘积}
可以推得以下三个命题.

\begin{proposition}
%@see: 《高等代数(第三版 下册)》(丘维声) P52 定理1
在\(K[x_1,\dotsc,x_n]\)中两个非零多项式的乘积仍是非零多项式.
\end{proposition}

\begin{proposition}
%@see: 《高等代数(第三版 下册)》(丘维声) P52 定理1
\(K[x_1,\dotsc,x_n]\)是无零因子环.
\end{proposition}

\begin{proposition}
%@see: 《高等代数(第三版 下册)》(丘维声) P52 定理1
在\(K[x_1,\dotsc,x_n]\)中,消去律成立.
\end{proposition}

\begin{corollary}
%@see: 《高等代数(第三版 下册)》(丘维声) P52 推论2
在\(K[x_1,\dotsc,x_n]\)中,
如果\(f_i\neq0\ (i=1,2,\dotsc,m)\),
则\(f_1 f_2 \dotsm f_m\)的首项等于它们的首项的乘积.
\begin{proof}
对\cref{theorem:多项式.多元多项式环.两个非零多项式的乘积的首项等于它们的首项的乘积}
运用数学归纳法可以证得.
\end{proof}
\end{corollary}

\subsection{齐次多项式}
\begin{definition}
%@see: 《高等代数(第三版 下册)》(丘维声) P53 定义2
设\(g(x_1,\dotsc,x_n)\)是数域\(K\)上的\(n\)元多项式.
如果\(g(x_1,\dotsc,x_n)\)的每个系数不为零的单项式都是\(m\)次的,
则称其为~\DefineConcept{\(m\)次齐次多项式}.
\end{definition}

\begin{example}
\(2x_1^4+3x_1^2x_2x_3+x_1x_2x_3^2\)是一个4次齐次多项式.
\end{example}

\begin{proposition}
\(K[x_1,\dotsc,x_n]\)中任意两个齐次多项式的乘积仍是齐次多项式,
它的次数等于这两个多项式的次数的和.
\end{proposition}

对于任何一个\(n\)元多项式\(f(x_1,\dotsc,x_n)\),
如果把\(f\)中所有次数相同的单项式并在一起,
则\(f\)可以唯一地表示成\begin{equation*}
	f(x_1,\dotsc,x_n)
	=\sum_{i=0}^m
	f_i(x_1,\dotsc,x_n),
\end{equation*}
其中\(m=\deg f\),
\(f_i(x_1,\dotsc,x_n)\)是\(i\)次齐次多项式,
它称为“\(f(x_1,\dotsc,x_n)\)的~\DefineConcept{\(i\)次齐次成分}”.

\begin{theorem}
%@see: 《高等代数(第三版 下册)》(丘维声) P53 定理3
设\(f(x_1,\dotsc,x_n),g(x_1,\dotsc,x_n) \in K[x_1,\dotsc,x_n]\),
则\begin{equation}
	\deg(fg)=\deg f+\deg g.
\end{equation}
\begin{proof}
若\(f,g\)中有一个是零多项式,
则\(\deg(fg)=\deg f+\deg g\)成立.

现在设\(f\neq0,g\neq0,\deg f=m,\deg g=s\),
则\begin{equation*}
	f=f_0+f_1+\dotsb+f_m, \qquad
	g=g_0+g_1+\dotsb+g_s,
\end{equation*}
其中\(f_i\)是\(f\)的\(i\)次齐次成分,
\(g_j\)是\(g\)的\(j\)次齐次成分,
\(f_m\neq0\),
\(g_s\neq0\).
我们有\begin{equation*}
	fg
	=(f_0 g_0+\dotsb+f_0 g_s)
	+(f_1 g_0+\dotsb+f_1 g_s)
	+\dotsb
	+(f_m g_0+\dotsb+f_m g_s),
\end{equation*}
其中\(f_i g_j\)是\(i+j\)次齐次多项式.
因为\(f_m\neq0,g_s\neq0\),
所以\(f_m g_s\neq0\).
于是\(f_m g_s\)是\(m+s\)次齐次多项式.
从而有\(\deg(fg)=m+s=\deg f+\deg g\).
\end{proof}
\end{theorem}

\begin{remark}
当\(n>1\)时,
\(K[x_1,\dotsc,x_n]\)中没有带余除法,
但是唯一因式分解定理仍然成立.
\end{remark}

和数域\(K\)上的一元多项式环\(K[x]\)具有通用性质一样,
数域\(K\)上的\(n\)元多项式环\(K[x_1,\dotsc,x_n]\)也具有通用性质:
\begin{theorem}
%@see: 《高等代数(第三版 下册)》(丘维声) P53 定理4
设\(K\)是一个数域,
\(R\)是一个带有单位元\(e\)的交换环,
并且\(R\)有一个子环\(R_1\)(含有\(e\)),
\(\tau\)是\(K\)到\(R_1\)的一个环同构映射,
\(t_1,\dotsc,t_n\)是\(R\)的元素,
令\begin{equation*}
	\sigma_{t_1,\dotsc,t_n}
	\colon
	K[x_1,\dotsc,x_n] \to R,
	f(x_1,\dotsc,x_n) \mapsto f(t_1,\dotsc,t_n),
\end{equation*}
其中\begin{equation*}
	f(x_1,\dotsc,x_n)
	= \sum_{i_1,\dotsc,i_n}
	a_{i_1 \dotsm i_n}
	x_1^{i_1} \dotsm x_n^{i_n},
\end{equation*}
而\begin{equation*}
	f(t_1,\dotsc,t_n)
	= \sum_{i_1,\dotsc,i_n}
	\tau(a_{i_1 \dotsm i_n})
	t_1^{i_1} \dotsm t_n^{i_n},
\end{equation*}
则\(\sigma_{t_1,\dotsc,t_n}\)是\(K[x_1,\dotsc,x_n]\)到\(R\)的一个映射,
它满足\(\sigma_{t_1,\dotsc,t_n}(x_i)=t_i\ (i=1,2,\dotsc,n)\),
且它保持加法、乘法运算,
即\begin{gather*}
	f(x_1,\dotsc,x_n)
	+g(x_1,\dotsc,x_n)
	=h(x_1,\dotsc,x_n), \\
	f(x_1,\dotsc,x_n)
	g(x_1,\dotsc,x_n)
	=p(x_1,\dotsc,x_n),
\end{gather*}
那么\begin{gather*}
	f(t_1,\dotsc,t_n)
	+g(t_1,\dotsc,t_n)
	=h(t_1,\dotsc,t_n), \\
	f(t_1,\dotsc,t_n)
	g(t_1,\dotsc,t_n)
	=p(t_1,\dotsc,t_n).
\end{gather*}
\rm
我们把映射\(\sigma_{t_1,\dotsc,t_n}\)称为
“\(x_1,\dotsc,x_n\)用\(t_1,\dotsc,t_n\)代入”.
\begin{proof}
证明过程参考\cref{theorem:多项式.多项式环的同构映射}.
\end{proof}
\end{theorem}

\(K[x_1,\dotsc,x_n]\)中所有零次多项式
添上零多项式组成的子集是
\(K[x_1,\dotsc,x_n]\)的一个子环,
它与\(K\)是环同构的,
因此\(x_1,\dotsc,x_n\)
可以用\(K[x_1,\dotsc,x_n]\)中任意\(n\)个元素代入,
这种代入是保持加法与乘法运算.

特别重要的一种情形是:
\(x_1,\dotsc,x_n\)
用\(K\)中任意\(n\)个元素
\(c_1,\dotsc,c_n\)代入.
由此我们可引进多元多项式函数的概念.

设\(f(x_1,\dotsc,x_n)\)是数域\(K\)上的一个\(n\)元多项式,
对于\(K\)中任意\(n\)个元素\(c_1,\dotsc,c_n\),
将\(x_1,\dotsc,x_n\)用\(c_1,\dotsc,c_n\)代入,
得\(f(c_1,\dotsc,c_n) \in K\).
于是\(n\)元多项式\(f(x_1,\dotsc,x_n)\)确定了
从\(K^n\)到\(K\)的一个映射
(即\(K\)上的\(n\)元函数),
仍用\(f\)表示,即\begin{equation*}
	f\colon
	K^n \to K,
	(c_1,\dotsc,c_n)
	\mapsto
	f(c_1,\dotsc,c_n).
\end{equation*}
这种由数域\(K\)上的\(n\)元多项式确定的\(K\)上的\(n\)元函数
称为\DefineConcept{数域\(K\)上的\(n\)元多项式函数}.

对于数域\(K\)上的两个\(n\)元多项式
\(f(x_1,\dotsc,x_n)\)和\(g(x_1,\dotsc,x_n)\),
如果它们相等,
则它们确定的\(n\)元多项式函数\(f\)与\(g\)也相等;
反之亦然.

\begin{lemma}\label{theorem:多项式.多元多项式环.引理1}
%@see: 《高等代数(第三版 下册)》(丘维声) P54 引理1
设\(h(x_1,\dotsc,x_n)\)是数域\(K\)上的一个\(n\)元多项式,
如果\(h(x_1,\dotsc,x_n)\neq0\),
则\(h\)不是零函数.
\begin{proof}
对不定元的数目\(n\)运用数学归纳法.
当\(n=1\)时,
由于数域\(K\)上非零的一元多项式\(h(x)\)给出的函数不是零函数,
因此存在\(c \in K\)使得\(h(c)\neq0\).

假设命题对\(K[x_1,\dotsc,x_{n-1}]\)中的多项式成立,
现在看\(K[x_1,\dotsc,x_n]\)中的多项式\(h(x_1,\dotsc,x_n)\).
把\(h(x_1,\dotsc,x_n)\)写成\begin{equation*}
	h(x_1,\dotsc,x_n)
	=u_0(x_1,\dotsc,x_{n-1})
	+u_1(x_1,\dotsc,x_{n-1})
	+\dotsb
	+u_s(x_1,\dotsc,x_{n-1}) x_n^s,
\end{equation*}
其中\(u_i(x_1,\dotsc,x_{n-1}) \in K[x_1,\dotsc,x_{n-1}]\ (i=0,1,\dotsc,s)\),
且\(u_s(x_1,\dotsc,x_{n-1})\neq0\).
根据归纳假设,\(u_s\)不是零函数,
因此存在\(c_1,\dotsc,c_{n-1} \in K\)
使得\(u_s(c_1,\dotsc,c_{n-1})\neq0\).
于是一元多项式环\(K[x_n]\)中的多项式\begin{equation*}
	h(c_1,\dotsc,c_{n-1},x_n)
	=u_0(c_1,\dotsc,c_{n-1})
	+u_1(c_1,\dotsc,c_{n-1}) x_n
	+\dotsb
	+u_s(c_1,\dotsc,c_{n-1}) x_n^s
\end{equation*}是非零多项式,
因此存在\(c_n \in K\),
使得\begin{equation*}
	h(c_1,\dotsc,c_n)
	=u_0(c_1,\dotsc,c_{n-1})
	+u_1(c_1,\dotsc,c_{n-1}) c_n
	+\dotsb
	+u_s(c_1,\dotsc,c_{n-1}) c_n^s
	\neq0.
	\qedhere
\end{equation*}
\end{proof}
\end{lemma}

\begin{theorem}
%@see: 《高等代数(第三版 下册)》(丘维声) P55 定理5
设\(f(x_1,\dotsc,x_n),g(x_1,\dotsc,x_n) \in K[x_1,\dotsc,x_n]\).
如果多项式\(f\)与\(g\)不相等,
则由它们确定的\(n\)元多项式函数\(f\)与\(g\)也不相等.
\begin{proof}
考虑多项式\(
	h(x_1,\dotsc,x_n)
	=f(x_1,\dotsc,x_n)
	-g(x_1,\dotsc,x_n)
\).
如果多项式\(f\)与\(g\)不相等,
则\(h(x_1,\dotsc,x_n)\neq0\).
根据\cref{theorem:多项式.多元多项式环.引理1},
\(h\)不是零函数,
于是存在\(c_1,\dotsc,c_n \in K\)
使得\(h(c_1,\dotsc,c_n)\neq0\).
\(x_1,\dotsc,x_n\)用\(c_1,\dotsc,c_n\)代入,
用上述式子可以推出\(f(c_1,\dotsc,c_n) \neq g(c_1,\dotsc,c_n)\),
所以映射\(f\)与\(g\)不相等.
\end{proof}
\end{theorem}

我们把数域\(K\)所有\(n\)元多项式函数组成的集合记作\(K_{npol}\),
在这个集合中规定加法与乘法如下:
对于\(\forall(c_1,\dotsc,c_n) \in K^n\)有\begin{gather*}
	(f+g)(c_1,\dotsc,c_n)
	\defeq
	f(c_1,\dotsc,c_n)+g(c_1,\dotsc,c_n), \\
	(fg)(c_1,\dotsc,c_n)
	\defeq
	f(c_1,\dotsc,c_n) g(c_1,\dotsc,c_n).
\end{gather*}
容易验证\(K_{npol}\)是一个环,
我们把它称为\DefineConcept{数域\(K\)上的\(n\)元多项式函数环}.
容易证明:
数域\(K\)上的\(n\)元多项式环\(K[x_1,\dotsc,x_n]\)
与\(K\)上的\(n\)元多项式函数环\(K_{npol}\)是同构的.
因此我们可以把数域\(K\)上的\(n\)元多项式
与\(K\)上的\(n\)元多项式函数等同看待.

设\(f(x_1,\dotsc,x_n) \in K[x_1,\dotsc,x_n]\),
对于\(c_1,\dotsc,c_n \in K\),
如果\(f(c_1,\dotsc,c_n)=0\),
则称“\((c_1,\dotsc,c_n)\)是\(f(x_1,\dotsc,x_n)\)的一个\DefineConcept{零点}”.
当\(K\)取实数域时,
若\(n=2\),
则\(f(x,y)\)的零点组成的集合就是平面上的一条\DefineConcept{代数曲线};
若\(n=3\),
则\(f(x,y,z)\)的零点组成的集合就是空间中的一个\DefineConcept{代数曲面}.
研究数域\(K\)上一组\(n\)元多项式的公共零点组成的集合,
就是代数几何的一个基本内容.

利用不定式\(x_1,\dotsc,x_n\)用\(K[x_1,\dotsc,x_n]\)的\(n\)个元素代入是保持运算的,
我们可以得出齐次多项式的一个特征性质.
\begin{theorem}
%@see: 《高等代数(第三版 下册)》(丘维声) P56 定理6
设\(f(x_1,\dotsc,x_n) \in K[x_1,\dotsc,x_n]\)且\(f\neq0\),
\(m\)是一个非负整数,
则\(f(x_1,\dotsc,x_n)\)是\(m\)次齐次多项式的充分必要条件是:
对于\(\forall t \in K\),有\begin{equation*}
	f(t x_1,\dotsc,t x_n)
	= t^m f(x_1,\dotsc,x_n).
\end{equation*}
\begin{proof}
必要性.
设\(f(x_1,\dotsc,x_n)\)是\(m\)次齐次多项式,
即\begin{equation*}
	f(x_1,\dotsc,x_n)
	=\sum_{i_1,\dotsc,i_n}
	a_{i_1 \dotsm i_n}
	x_1^{i_1} \dotsm x_n^{i_n},
	\qquad
	i_1+\dotsb+i_n=m.
\end{equation*}
任取\(t \in K\),
不定元\(x_1,\dotsc,x_n\)用\(t x_1,\dotsc,t x_n\)代入,
从上式得\begin{align*}
	f(t x_1,\dotsc,t x_n)
	&=\sum_{i_1,\dotsc,i_n}
	a_{i_1 \dotsm i_n}
	(t x_1)^{i_1} \dotsm (t x_n)^{i_n} \\
	&=t^{i_1+\dotsb+i_n}
	\sum_{i_1,\dotsc,i_n}
	a_{i_1 \dotsm i_n}
	x_1^{i_1} \dotsm x_n^{i_n} \\
	&=t^m f(x_1,\dotsc,x_n).
\end{align*}

充分性.
设对\(\forall t \in K\),
有\(f(t x_1,\dotsc,t x_n)
= t^m f(x_1,\dotsc,x_n)\).
设\(\deg f=s\),
则可将\(f(x_1,\dotsc,x_n)\)写成\begin{equation*}
	f=f_0+\dotsb+f_s,
\end{equation*}
其中\(f_i\)是\(f\)的\(i\)次齐次成分.
任取\(t \in K\),
\(x_1,\dotsc,x_n\)用\(t x_1,\dotsc,t x_n\)代入,
从上式得出\begin{equation*}
	f(t x_1,\dotsc,t x_n)
	=f_0(t x_1,\dotsc,t x_n)
	+f_1(t x_1,\dotsc,t x_n)
	+\dotsb
	+f_s(t x_1,\dotsc,t x_n),
\end{equation*}
那么由必要性的结论以及充分性的假设得\begin{equation*}
	t^m f(x_1,\dotsc,x_n)
	=f_0(x_1,\dotsc,x_n)
	+t f_1(x_1,\dotsc,x_n)
	+\dotsb
	+t^s f_s(x_1,\dotsc,x_n).
\end{equation*}
于是根据\(K[x_1,\dotsc,x_n]\)中两个多项式相等的规定,
我们得到\begin{equation*}
	t^m f_i(x_1,\dotsc,x_n)
	= t^i f_i(x_1,\dotsc,x_n),
	\quad
	i=0,1,\dotsc,s.
\end{equation*}
对于\(\forall i \in \{0,1,\dotsc,s\}-\{m\}\),
如果\(f_i\neq0\),
则从上式两边消去\(f_i\)得\((\forall t \in K)[t^m=t^i]\).
由\cref{theorem:多项式.多项式函数是否相等取决于多项式是否相等}
得\(x^m=x^i\),
这与\(i \neq m\)矛盾!
因此当\(i \neq m\)时必有\(f_i=0\).
所以\(f=f_m\).
已知\(f\neq0\),
因此\(f\)是\(m\)次齐次多项式.
\end{proof}
\end{theorem}

我们指出,设\(R\)是\(K\)的一个交换扩环,
任取\(R\)中\(n\)个元素\(t_1,\dotsc,t_n\),
对于\(f(x_1,\dotsc,x_n) \in K[x_1,\dotsc,x_n]\),
把\(x_1,\dotsc,x_n\)用\(t_1,\dotsc,t_n\)代入,
得到的\(f(t_1,\dotsc,t_n)\)称为“\(t_1,\dotsc,t_n\)在\(K\)上的一个多项式”.
尽管从形式上看,
\(t_1,\dotsc,t_n\)的多项式\(f(t_1,\dotsc,t_n)\)
与\(n\)元多项式\(f(x_1,\dotsc,x_n)\)类似,
但是它们有本质不同:
\(t_1,\dotsc,t_n\)的一个多项式的表法可能不唯一,
而一个\(n\)元多项式的表法唯一.

\section{对称多项式}
\subsection{对称多项式}
观察下述三元多项式\(f(x_1,x_2,x_3)\)有什么特点?
\begin{equation*}
	f(x_1,x_2,x_3)
	=x_1^3+x_2^3+x_3^3
	+x_1^2x_2
	+x_1^2x_3
	+x_2^2x_3
	+x_1x_2^2
	+x_1x_3^2
	+x_2x_3^2.
\end{equation*}
直观上看,
\(x_1,x_2,x_3\)在\(f(x_1,x_2,x_3)\)中的地位是对称的,
即同时有\(x_1^3,x_2^3,x_3^3\)这三项,
且同时有\(x_1^2x_2,
x_1^2x_3,
x_2^2x_3,
x_1x_2^2,
x_1x_3^2,
x_2x_3^2\)这六项.
由此受到启发,
我们来研究具有这种性质的\(n\)元多项式\(f(x_1,\dotsc,x_n)\):
若\(f(x_1,\dotsc,x_n)\)含有一项\(a x_1^{i_1} \dotsm x_n^{i_n}\),
则它也含有一项\(a x_{j_1}^{i_1} \dotsm x_{j_n}^{i_n}\),
其中\(j_1 \dotso j_n\)是任意一个\(n\)元排列.

于是我们抽象出下述概念.
\begin{definition}
%@see: 《高等代数(第三版 下册)》(丘维声) P57 定义1
设\(f(x_1,\dotsc,x_n)\)是数域\(K\)上的一个\(n\)元多项式.
如果对于任意一个\(n\)元排列\(j_1 \dotso j_n\)都有\begin{equation*}
	f(x_{j_1},\dotsc,x_{j_n})
	=f(x_1,\dotsc,x_n),
\end{equation*}
则称“\(f(x_1,\dotsc,x_n)\)是数域\(K\)上的一个\(n\)元\DefineConcept{对称多项式}”.
\end{definition}

定义表明,
在数域\(K\)上的\(n\)元多项式环\(K[x_1,\dotsc,x_n]\)中,
对于\(f(x_1,\dotsc,x_n)\),
如果任给一个\(n\)元排列\(j_1 \dotso j_n\),
不定元\(x_1,\dotsc,x_n\)用\(x_{j_1},\dotsc,x_{j_n}\)代入,
都有\(f(x_{j_1},\dotsc,x_{j_n})=f(x_1,\dotsc,x_n)\),
那么\(n\)元多项式\(f(x_1,\dotsc,x_n)\)是一个对称多项式.

容易看出,零多项式和零次多项式都是对称多项式.

\subsection{初等对称多项式}
在\(K[x_1,\dotsc,x_n]\)中,
我们来构造含有项\(x_1\)且项数最少的对称多项式.
由定义可知,
\(x_1+\dotsb+x_n\)就是\(n\)元对称多项式,
把它记作\(\sigma_1(x_1,\dotsc,x_n)\),
即\begin{equation*}
	\sigma_1(x_1,\dotsc,x_n)
	=x_1+\dotsb+x_n.
\end{equation*}
我们来构造含有项\(x_1x_2\)且项数最少的对称多项式.
令\begin{align*}
	\sigma_2(x_1,\dotsc,x_n)
	&=\begin{array}[t]{l}
		x_1x_2+x_1x_3+\dotsb+x_1x_n \\
		+x_2x_3+\dotsb+x_2x_n
		+\dotsb
		+x_{n-1}x_n
	\end{array} \\
	&=\sum_{1\leq i<j\leq n} x_i x_j,
\end{align*}
则\(\sigma_2(x_1,\dotsc,x_n)\)是\(n\)元对称多项式.
同理,对于\(\forall k\in\{2,\dotsc,n-1\}\),
我们来构造含有项\(x_1 \dotsm x_k\),
且项数最少得对称多项式.
令\begin{equation*}
	\sigma_k(x_1,\dotsc,x_n)
	=\sum_{1\leq j_1<\dotsb<j_k\leq n}
	x_{j_1} \dotsm x_{j_k},
\end{equation*}
则\(\sigma_k(x_1,\dotsc,x_n)\)是\(n\)元对称多项式.
最后,根据定义有,\begin{equation*}
	\sigma_n(x_1,\dotsc,x_n)
	=x_1 \dotsm x_n
\end{equation*}是\(n\)元对称多项式.

我们把上述\(n\)个\(n\)元对称多项式
\(\sigma_i(x_1,\dotsc,x_n)\ (i=1,\dotsc,n)\)
统称为\(n\)元\DefineConcept{初等对称多项式}.

\subsection{对称多项式环}
下面我们把数域\(K\)上所有\(n\)元对称多项式组成的集合记为\(W\).
我们想要知道\(W\)的结构是怎样的.

\begin{proposition}
%@see: 《高等代数(第三版 下册)》(丘维声) P58 命题1
\(W\)是\(K[x_1,\dotsc,x_n]\)的一个子环.
\begin{proof}
显然\(W\)非空集.
任取\(f(x_1,\dotsc,x_n),g(x_1,\dotsc,x_n) \in W\),
设\begin{gather*}
	h(x_1,\dotsc,x_n)
	=f(x_1,\dotsc,x_n)
	-g(x_1,\dotsc,x_n), \\
	p(x_1,\dotsc,x_n)
	=f(x_1,\dotsc,x_n)
	g(x_1,\dotsc,x_n).
\end{gather*}
任给一个\(n\)元排列\(j_1 j_2 \dotso j_n\),
\(x_1,\dotsc,x_n\)用\(x_{j_1},\dotsc,x_{j_n}\)代入,
从以上两式分别得到\begin{align*}
	h(x_{j_1},\dotsc,x_{j_n})
	&=f(x_{j_1},\dotsc,x_{j_n})
	-g(x_{j_1},\dotsc,x_{j_n}) \\
	&=f(x_1,\dotsc,x_n)
	-g(x_1,\dotsc,x_n) \\
	&=h(x_1,\dotsc,x_n), \\
	p(x_{j_1},\dotsc,x_{j_n})
	&=f(x_{j_1},\dotsc,x_{j_n})
	g(x_{j_1},\dotsc,x_{j_n}) \\
	&=f(x_1,\dotsc,x_n)
	g(x_1,\dotsc,x_n) \\
	&=p(x_1,\dotsc,x_n).
\end{align*}
因此\(h(x_1,\dotsc,x_n),p(x_1,\dotsc,x_n) \in W\).
这就说明\(W\)是\(K[x_1,\dotsc,x_n]\)的一个子环.
\end{proof}
\end{proposition}

\begin{proposition}
%@see: 《高等代数(第三版 下册)》(丘维声) P59 命题2
设\(f_1,\dotsc,f_n \in W\),
则对\(K[x_1,\dotsc,x_n]\)中任意一个多项式\begin{equation*}
	g(x_1,\dotsc,x_n)
	=\sum_{i_1,\dotsc,i_n}
	b_{i_1 \dotso i_n}
	x_1^{i_1} \dotsm x_n^{i_n},
\end{equation*}
有\begin{equation*}
	g(f_1,\dotsc,f_n)
	=\sum_{i_1,\dotsc,i_n}
	b_{i_1 \dotso i_n}
	f_1^{i_1} \dotsm f_n^{i_n}
	\in W.
\end{equation*}
\end{proposition}

\begin{theorem}[对称多项式基本定理]
%@see: 《高等代数(第三版 下册)》(丘维声) P59 定理3
对于\(K[x_1,\dotsc,x_n]\)中任意一个对称多项式\(f(x_1,\dotsc,x_n)\),
都存在\(K[x_1,\dotsc,x_n]\)中唯一的一个多项式\(g(x_1,\dotsc,x_n)\),
使得\(f(x_1,\dotsc,x_n)=g(\sigma_1,\dotsc,\sigma_n)\).
\begin{proof}
存在性.
采取首项消去法.
设对称多项式\(f(x_1,\dotsc,x_n)\)的首项是\(a x_1^{l_1} \dotsm x_n^{l_n}\),
其中\(a\neq0\),且\(l_1 \geq \dotsb \geq l_n\).
为了消去\(f(x_1,\dotsc,x_n)\)的首项,
同时又要出现\(\sigma_1,\dotsc,\sigma_n\),
我们作多项式\begin{equation*}
	\phi_1(x_1,\dotsc,x_n)
	= a_1 \sigma_1^{l_1-l_2} \sigma_2^{l_2-l_3}
	\dotsm \sigma_{n-1}^{l_{n-1}-l_n} \sigma_n^{l_n},
\end{equation*}
其中\(a_1=a\).
因为\(K[x_1,\dotsc,x_n]\)中对称多项式的乘积还是对称多项式,
所以\(\phi_1(x_1,\dotsc,x_n)\)是对称多项式.
又由于多项式的乘积的首项等于它们的首项的乘积,
因此\(\phi_1(x_1,\dotsc,x_n)\)的首项是\begin{align*}
	&a_1 x_1^{l_1-l_2} (x_1 x_2)^{l_2-l_3}
	\dotsm (x_1 x_2 \dotsm x_{n-1})^{l_{n-1}-l_n}
	(x_1 x_2 \dotsm x_{n-1} x_n)^{l_n} \\
	&= a_1 x_1^{l_1} x_2^{l_2} \dotsm x_{n-1}^{l_{n-1}} x_n^{l_n},
\end{align*}
它等于\(f(x_1,\dotsc,x_n)\)的首项.
令\begin{equation*}
	f_1(x_1,\dotsc,x_n)
	=f(x_1,\dotsc,x_n)
	-\phi_1(x_1,\dotsc,x_n),
\end{equation*}
则\(f\)的首项的幂指数组\((l_1,\dotsc,l_n)\)
先于\(f_1\)的首项的幂指数组\((p_{11},\dotsc,p_{1n})\),
并且由于对称多项式的差仍是对称多项式,
所以\(f_1(x_1,\dotsc,x_n)\)是\(K[x_1,\dotsc,x_n]\)中的对称多项式.

对\(f_1(x_1,\dotsc,x_n)\)重复上述做法,
我们又得到\(K[x_1,\dotsc,x_n]\)中的一个对称多项式\begin{equation*}
	f_2(x_1,\dotsc,x_n)
	=f_1(x_1,\dotsc,x_n)
	-\phi_2(x_1,\dotsc,x_n),
\end{equation*}
其中\begin{equation*}
	\phi_2(x_1,\dotsc,x_n)
	=a_2 \sigma_1^{p_{11}-p_{12}} \sigma_2^{p_{12}-p_{13}}
	\dotsm \sigma_{n-1}^{p_{1,n-1}-p_{1n}} \sigma_n^{p_{1n}},
\end{equation*}
%\(\phi_2(x_1,\dotsc,x_n)\)是\(\sigma_1,\dotsc,\sigma_n\)的适当方幂的乘积,
并且其系数\(a_2\)等于\(f_1\)的首项系数,
\(f_1\)的首项的幂指数组\((p_{11},\dotsc,p_{1n})\)
先于\(f_2\)的首项的幂指数组\((p_{21},\dotsc,p_{2n})\).

如此继续下去,我们得到\(K[x_1,\dotsc,x_n]\)中一系列的对称多项式\begin{equation*}
	f,
	f_1=f-\phi_1,
	f_2=f_1-\phi_2,
	\dotsc,
	f_i=f_{i-1}-\phi_i,
	\dotsc,
\end{equation*}
其中\begin{equation*}
	\phi_i(x_1,\dotsc,x_n)
	=a_i \sigma_1^{p_{i1}-p_{i2}} \sigma_2^{p_{i2}-p_{i3}}
	\dotsm \sigma_{n-1}^{p_{i,n-1}-p_{in}} \sigma_n^{p_{in}},
\end{equation*}
%\(\phi_i\)是\(\sigma_1,\dotsc,\sigma_n\)的适当方幂的乘积,
并且其系数\(a_i\)等于\(f_{i-1}\)的首项系数.
容易看出,在上述多项式序列中,它们首项的幂指数组一个比一个小,即\begin{equation*}
	(l_1,\dotsc,l_2)
	>(p_{11},\dotsc,p_{1n})
	>\dotsb
	>(p_{k1},\dotsc,p_{kn})
	>\dotsb,
\end{equation*}
于是\(l_1 \geq p_{11}\).
又因为\(f_i\)是对称多项式,
所以\(p_{11} \geq \dotsb \geq p_n\).
因此\(l_1 \geq p_{11} \geq p_{12} \geq \dotsb p_{1n}\).
满足这个条件的非负整数组\((p_{11},\dotsc,p_{1n})\)只有有限多个,
因此上述对称多项式序列中只能有有限多个\(f_i\)不为零,
换言之,存在正整数\(s\),使得\(f_s=0\).
于是\begin{equation*}
	f_1=f-\phi_1,
	f_2=f_1-\phi_2,
	\dotsc,
	f_{s-1}=f_{s-2}-\phi_{s-1},
	f_s=f_{s-1}-\phi_s=0,
\end{equation*}
从而得到\begin{equation*}
	f=\phi_1+\dotsb+\phi_s.
\end{equation*}

设\(\phi_i(x_1,\dotsc,x_n)
=a_i \sigma_1^{t_{i1}} \dotsm \sigma_n^{t_{in}}\),
令\begin{equation*}
	g(x_1,\dotsc,x_n)
	=\sum_{i=1}^s a_i x_1^{t_{i1}} \dotsm x_n^{t_{in}}
	\in K[x_1,\dotsc,x_n],
\end{equation*}
则\begin{align*}
	g(\sigma_1,\dotsc,\sigma_n)
	&=\sum_{i=1}^s a_i \sigma_1^{t_{i1}} \dotsm \sigma_n^{t_{in}} \\
	&=\sum_{i=1}^s \phi_i(x_1,\dotsc,x_n) \\
	&=f(x_1,\dotsc,x_n).
\end{align*}
存在性成立.

唯一性.
如果\(K[x_1,\dotsc,x_n]\)中有两个不同的多项式
\(g_1(x_1,\dotsc,x_n)\)和\(g_2(x_1,\dotsc,x_n)\),
使得\begin{align*}
	f(x_1,\dotsc,x_n)
	&=g_1(\sigma_1,\dotsc,\sigma_n) \\
	&=g_2(\sigma_1,\dotsc,\sigma_n),
\end{align*}
则\(g_1(\sigma_1,\dotsc,\sigma_n)-g_2(\sigma_1,\dotsc,\sigma_n)=0\).
令\begin{equation*}
	g(x_1,\dotsc,x_n)
	=g_1(x_1,\dotsc,x_n)-g_2(x_1,\dotsc,x_n),
\end{equation*}
则\(g(\sigma_1,\dotsc,\sigma_n)=0\).
由假设可知\(g(x_1,\dotsc,x_n)\neq0\),
于是由\cref{theorem:多项式.多元多项式环.引理1} 可知,
存在\(b_1,\dotsc,b_n \in K\),
使得\(g(b_1,\dotsc,b_n)\neq0\).
令\begin{equation*}
	\phi(x)=x^n-b_1x^{n-1}+\dotsb+(-1)^kb_kx^{n-k}+\dotsb+(-1)^nb_n,
\end{equation*}
设\(\phi(x)\)的\(n\)个复根是\(\alpha_1,\dotsc,\alpha_n\),
则从韦达公式推出\begin{equation*}
	b_k=\sigma_k(\alpha_1,\dotsc,\alpha_n),
	\quad
	k=1,2,\dotsc,n.
\end{equation*}
\(x_1,\dotsc,x_n\)用\(\alpha_1,\dotsc,\alpha_n\)代入,
于是\begin{equation*}
	g(\sigma_1(\alpha_1,\dotsc,\alpha_n),\dotsc,\sigma_n(\alpha_1,\dotsc,\alpha_n))=0,
\end{equation*}
即\(g(b_1,\dotsc,b_n)=0\),矛盾!
唯一性成立.
\end{proof}
\end{theorem}

对称多项式基本定理完全解决了\(K[x_1,\dotsc,x_n]\)中所有对称多项式组成的子环\(W\)的结构问题.
定理中存在性的证明是构造性的,
可以实际地利用它去求多项式\(g(x_1,\dotsc,x_n)\),
使得\begin{equation*}
	f(x_1,\dotsc,x_n)
	=g(\sigma_1,\dotsc,\sigma_n).
\end{equation*}

\begin{example}
%@see: 《高等代数(第三版 下册)》(丘维声) P61 例1
在\(K[x_1,x_2,x_3]\)中,
用初等对称多项式表示出对称多项式\begin{equation*}
	f(x_1,x_2,x_3)
	=x_1^2 x_2^2
	+x_1^2 x_3^2
	+x_2^2 x_3^2.
\end{equation*}
\begin{solution}
\(f(x_1,x_2,x_3)\)的首项是\(x_1^2 x_2^2\),
它的幂指数组为\((2,2,0)\).
作多项式\begin{equation*}
	\phi_1(x_1,x_2,x_3)
	=\sigma_1^{2-2} \sigma_2^{2-0} \sigma_3^0
	=\sigma_2^2,
\end{equation*}
令\begin{align*}
	f_1(x_1,x_2,x_3)
	&= f(x_1,x_2,x_3)
	- \phi_1(x_1,x_2,x_3) \\
	&= -2 \sigma_1 \sigma_3,
\end{align*}
于是\begin{equation*}
	f(x_1,x_2,x_3)
	=\phi_1+f_1
	=\sigma_2^2 - 2 \sigma_1 \sigma_3.
\end{equation*}
\end{solution}
\end{example}

对于较复杂的\(n\)元对称多项式\(f(x_1,\dotsc,x_n)\),
求一个多项式\(g(x_1,\dotsc,x_n)\),
使得\begin{equation*}
	f(x_1,\dotsc,x_n)
	=g(\sigma_1,\dotsc,\sigma_n),
\end{equation*}
采用待定系数法更为简便.
我们举一个例子来说明这种方法.

\begin{example}
%@see: 《高等代数(第三版 下册)》(丘维声) P62 例2
在\(K[x_1,\dotsc,x_3]\)中,
用初等对称多项式表出
含有项\(x_1^2 x_2^2\)的项数最少的
对称多项式\(f(x_1,\dotsc,x_n)\).
%TODO
% \begin{solution}
% \(f\)的首项\(x_1^2\)的幂指数组为\((2,2,0,\dotsc,0)\).
% 所以
% 对称多项式基本定理指出,
% \end{solution}
\end{example}

如果给定的对称多项式不是齐次的,
那么可以把它表示成它的齐次成分的和.
将其中每一个齐次成分看作一个对称多项式,
按照上述做法计算,
最后把所得结果相加即可.

\subsection{复数域上的多项式的重根的存在性}
对称多项式基本定理的一个重要应用是,
研究数域\(K\)上的一个多项式
在复数域中有没有重根.

设数域\(K\)上首项系数为1的多项式\begin{equation*}
	f(x)=x^n+a_{n-1} x^{n-1}+\dotsb+a_1 x+a_0
\end{equation*}在复数域中的\(n\)个根为\(c_1,\dotsc,c_n\).
记\begin{equation*}
	D(c_1,\dotsc,c_n)
	\defeq
	\prod_{1\leq j<i\leq n} (c_i-c_j)^2,
\end{equation*}
容易看出\begin{align*}
	&\text{$f(x)$在复数域中有重根} \\
	&\iff
	D(c_1,\dotsc,c_n)=0.
\end{align*}

\(f(x)\)的\(n\)个复根\(c_1,\dotsc,c_n\)是未知的,
于是我们想用\(f(x)\)的系数来表示\(D(c_1,\dotsc,c_n)\).
根据韦达公式有\begin{equation*}
	\left\{ \begin{array}{l}
		-a_{n-1}
		=c_1+\dotsb+c_n
		=\sigma_1(c_1,\dotsc,c_n), \\
		a_{n-1}
		=\sum_{1\leq i<j\leq n} c_i c_j
		=\sigma_2(c_1,\dotsc,c_n), \\
		\hdotsfor1 \\
		(-1)^n a_0
		=c_1 \dotsm c_n
		=\sigma_n(c_1,\dotsc,c_n).
	\end{array} \right.
\end{equation*}
受此启发,
如果\(D(c_1,\dotsc,c_n)\)能够用\(\sigma_1(c_1,\dotsc,c_n),
\dotsc,
\sigma_n(c_1,\dotsc,c_n)\)表示出来,
那么它就能够用\(f(x)\)的系数\(a_{n-1},a_{n-2},\dotsc,a_0\)表示出来.
注意到\(D(c_1,\dotsc,c_n)\)是关于\(c_1,\dotsc,c_n\)对称的表达式,
因此自然会想到运用对称多项式基本定理.

根据对称多项式基本定理,
数域\(K\)上\(n\)元对称多项式\begin{equation*}
	D(x_1,\dotsc,x_n)
	=\prod_{1\leq j<i\leq n} (x_i-x_j)^2
\end{equation*}
存在\(K[x_1,\dotsc,x_n]\)中唯一的一个多项式\(g(x_1,\dotsc,x_n)\),
使得\begin{equation*}
	D(x_1,\dotsc,x_n)
	=g(\sigma_1,\dotsc,\sigma_n).
\end{equation*}
不定元\(x_1,\dotsc,x_n\)分别用\(c_1,\dotsc,c_n\)代入,
于是有\begin{equation*}
	D(c_1,\dotsc,c_n)
	=g(-a_{n-1},a_{n-2},\dotsc,(-1)^n a_0).
\end{equation*}
因此我们有以下结论.

\begin{proposition}
%@see: 《高等代数(第三版 下册)》(丘维声) P63 命题4
数域\(K\)上首项系数为\(1\)的\(n\)次多项式\begin{equation*}
	f(x)=x^n+a_{n-1} x^{n-1}+\dotsb+a_1 x+a_0
\end{equation*}
在复数域中有重根的充分必要条件为\begin{equation*}
	g(-a_{n-1},a_{n-2},\dotsc,(-1)^n a_0)=0.
\end{equation*}
\end{proposition}

我们把\(f(x)\)的系数\(a_{n-1},a_{n-2},\dotsc,a_0\)的多项式\begin{equation*}
	g(-a_{n-1},a_{n-2},\dotsc,(-1)^n a_0)
\end{equation*}称为“\(f(x)\)的\DefineConcept{判别式}”,
记作\(D(f)\).

现在我们来求\(f(x)\)的判别式\(D(f)\).
\begin{align*}
	D(f)
	&= g(-a_{n-1},a_{n-2},\dotsc,(-1)^n a_0) \\
	&= D(c_1,\dotsc,c_n) \\
	&= \prod_{1\leq j<i\leq n} (c_i-c_j)^2.
\end{align*}
表达式\(\prod_{1\leq j<i\leq n} (c_i-c_j)^2\)使人联想起范德蒙德行列式\begin{equation*}
	\begin{vmatrix}
		1 & 1 & 1 & \dots & 1 \\
		x_1 & x_2 & x_3 & \dots & x_n \\
		x_1^2 & x_2^2 & x_3^2 & \dots & x_n^2 \\
		\vdots & \vdots & \vdots& & \vdots \\
		x_1^{n-1} & x_2^{n-1} & x_3^{n-1} & \dots & x_n^{n-1}
	\end{vmatrix}
	= \prod_{1 \leq j < i \leq n}(x_i-x_j).
\end{equation*}
若记\begin{equation*}
	\vb{V}(x_1,\dotsc,x_n) = \begin{bmatrix}
		1 & 1 & 1 & \dots & 1 \\
		x_1 & x_2 & x_3 & \dots & x_n \\
		x_1^2 & x_2^2 & x_3^2 & \dots & x_n^2 \\
		\vdots & \vdots & \vdots& & \vdots \\
		x_1^{n-1} & x_2^{n-1} & x_3^{n-1} & \dots & x_n^{n-1}
	\end{bmatrix},
\end{equation*}
考虑到\(\abs{\vb{V}(x_1,\dotsc,x_n)}=\abs{\vb{V}^T(x_1,\dotsc,x_n)}\),
于是有\begin{align*}
%@see: 《高等代数(第三版 下册)》(丘维声) P64 公式(26)
	D(f)
	&= \prod_{1\leq j<i\leq n} (c_i-c_j)^2 \\
	&= \abs{\vb{V}(c_1,\dotsc,c_n)} \abs{\vb{V}^T(c_1,\dotsc,c_n)} \\
	&= \abs{\vb{V}(c_1,\dotsc,c_n) \vb{V}^T(c_1,\dotsc,c_n)}.
\end{align*}
于是\begin{align}
%@see: 《高等代数(第三版 下册)》(丘维声) P64 公式(27)
	D(f)
	&= \abs{
		\begin{bmatrix}
			1 & 1 & \dots & 1 \\
			c_1 & c_2 & \dots & c_n \\
			\vdots & \vdots && \vdots \\
			c_1^{n-1} & c_2^{n-1} & \dots & c_n^{n-1}
		\end{bmatrix}
		\begin{bmatrix}
			1 & c_1 & \dots & c_1^{n-1} \\
			1 & c_2 & \dots & c_2^{n-1} \\
			\vdots & \vdots && \vdots \\
			1 & c_n & \dots & c_n^{n-1}
		\end{bmatrix}
	} \notag \\
	&= \begin{vmatrix}
		n & \sum_{i=1}^n c_i & \dots & \sum_{i=1}^n c_i^{n-1} \\
		\sum_{i=1}^n c_i & \sum_{i=1}^n c_i^2 & \dots & \sum_{i=1}^n c_i^n \\
		\vdots & \vdots && \vdots \\
		\sum_{i=1}^n c_i^{n-1} & \sum_{i=1}^n c_i^n & \dots & \sum_{i=1}^n c_i^{2n-2}
	\end{vmatrix}.
	\label{equation:多项式.对称多项式.范德蒙德}
\end{align}
上式表明,
为了求出\(D(f)\),
就需要计算\begin{equation*}
%@see: 《高等代数(第三版 下册)》(丘维声) P64 公式(28)
	\sum_{i=1}^n c_i^k,
	\quad k=0,1,\dotsc,2n-2.
\end{equation*}
由于\(f(x)\)的\(n\)个复根\(c_1,\dotsc,c_n\)是未知的,
因此必须想办法通过\(f(x)\)的系数来计算\(\sum_{i=1}^n c_i^k\).
由于\(\sum_{i=1}^n c_i^k\)是对称多项式,
因此仍然想到运用对称多项式基本定理.
为此我们考虑下列\(n\)元对称多项式\begin{equation*}
%@see: 《高等代数(第三版 下册)》(丘维声) P65 公式(29)
	s_k(x_1,\dotsc,x_n)
	=x_1^k+\dotsb+x_n^k,
	\quad k=0,1,2,\dotsc.
\end{equation*}
这些\(n\)元对称多项式称为\DefineConcept{幂和}.

根据对称多项式基本定理,
幂和\(s_k\)能表示成初等对称多项式的多项式.
具体的表示方法可以用递推公式求出.

当\(1\leq k\leq n\)时,
\begin{equation}\label{equation:多项式.对称多项式.牛顿公式1}
%@see: 《高等代数(第三版 下册)》(丘维声) P65 公式(30)
	s_k
	- \sigma_1 s_{k-1}
	+ \sigma_2 s_{k-2}
	+ \dotsb
	+ (-1)^{k-1} \sigma_{k-1} s_1
	+ (-1)^k k \sigma_k
	=0;
\end{equation}
当\(k>n\)时,
\begin{equation}\label{equation:多项式.对称多项式.牛顿公式2}
%@see: 《高等代数(第三版 下册)》(丘维声) P65 公式(31)
	s_k
	- \sigma_1 s_{k-1}
	+ \sigma_2 s_{k-2}
	+ \dotsb
	+ (-1)^{n-1} \sigma_{n-1} s_{k-n+1}
	+ (-1)^n \sigma_n s_{k-n}
	=0.
\end{equation}
我们把\cref{equation:多项式.对称多项式.牛顿公式1,equation:多项式.对称多项式.牛顿公式2}
并称为\DefineConcept{牛顿公式}.

利用牛顿公式,
可以从\(s_{k-1}(c_1,\dotsc,c_n),\dotsc,s_1(c_1,\dotsc,c_n)\)
以及\(\sigma_1(c_1,\dotsc,c_n),\dotsc,\sigma_n(c_1,\dotsc,c_n)\)
计算出\(s_k(c_1,\dotsc,c_n)\).
再从\cref{equation:多项式.对称多项式.范德蒙德}
就可以求出\(f(x)\)的判别式\(D(f)\).

多项式\(f(x)\)的判别式\(D(f)\)也称为方程\(f(x)=0\)的判别式.

\begin{remark}
在上述讨论中,\(f(x)\)的首项系数是1.
如果\(f(x)\)的首项系数是\(a_n\),
则可以先对\(a_n^{-1} f(x)\)运用上述方法,
求出它的判别式\(D(a_n^{-1} f)\),
然后规定\(f(x)\)的判别式为
\begin{equation}
%@see: 《高等代数(第三版 下册)》(丘维声) P65 公式(32)
	D(f)
	\defeq
	a_n^{2n-2} D(a_n^{-1} f).
\end{equation}
\end{remark}

\begin{example}
%@see: 《高等代数(第三版 下册)》(丘维声) P65 例3
求数域\(K\)上二次方程\(f(x)=x^2+bx+c=0\)的判别式.
\begin{solution}
设\(f(x)\)的复根是\(c_1,c_2\),
则\begin{equation*}
	\sigma_1(c_1,c_2)=-b, \qquad
	\sigma_2(c_1,c_2)=c.
\end{equation*}
根据牛顿公式有\begin{equation*}
	s_1=\sigma_1, \qquad
	s_2=\sigma_1 s_1 - 2\sigma_2
	=\sigma_1^2-2\sigma_2,
\end{equation*}
于是\begin{equation*}
	s_1(c_1,c_2)=-b, \qquad
	s_2(c_1,c_2)=b^2-2c,
\end{equation*}
因此\begin{equation*}
%@see: 《高等代数(第三版 下册)》(丘维声) P65 公式(33)
	D(f)
	=\begin{vmatrix}
		2 & -b \\
		-b & b^2-2c
	\end{vmatrix}
	=b^2-4c.
\end{equation*}
\end{solution}
\end{example}

\begin{example}
%@see: 《高等代数(第三版 下册)》(丘维声) P65 例4
求数域\(K\)上不完全三次方程\(f(x)=x^3+ax+b=0\)的判别式.
\begin{solution}
设\(f(x)\)的复根是\(c_1,c_2,c_3\),
则\begin{equation*}
	\sigma_1(c_1,c_2,c_3)=0, \qquad
	\sigma_2(c_1,c_2,c_3)=a, \qquad
	\sigma_3(c_1,c_2,c_3)=-b.
\end{equation*}
根据牛顿公式有\begin{gather*}
	s_1(c_1,c_2,c_3)=0, \qquad
	s_2(c_1,c_2,c_3)=-2a, \\
	s_3(c_1,c_2,c_3)=-3b, \qquad
	s_4(c_1,c_2,c_3)=2a^2,
\end{gather*}
所以\begin{equation*}
%@see: 《高等代数(第三版 下册)》(丘维声) P66 公式(35)
	D(f)
	=\begin{vmatrix}
		3 & 0 & -2a \\
		0 & -2a & -3b \\
		-2a & -3b & 2a^2
	\end{vmatrix}
	=-4a^3-27b^2.
\end{equation*}
\end{solution}
\end{example}

\section{模m剩余类环}
\subsection{模m同余关系}
\begin{definition}
%@see: 《高等代数(第三版 下册)》(丘维声) P67
设\(a,b,m\)都是整数.
如果\(m\mid(a-b)\),
则称“\(a\)与\(b\) \DefineConcept{模\(m\)同余}”,
记作\(a\equiv b\pmod m\).
\end{definition}
\begin{definition}
%@see: 《高等代数(第三版 下册)》(丘维声) P67
对于给定正整数\(m\ (m>1)\),
在整数集\(\mathbb{Z}\)上定义一个二元关系\begin{equation*}
	a \sim b
	\defiff
	a\equiv b\pmod m,
\end{equation*}
称之为\DefineConcept{模\(m\)同余关系}.
\end{definition}
\begin{theorem}
%@see: 《高等代数(第三版 下册)》(丘维声) P67
模\(m\)同余关系\(\sim\),具有反身性、对称性、传递性,是\(\mathbb{Z}\)上的一个等价关系.
\end{theorem}
\begin{definition}
%@see: 《高等代数(第三版 下册)》(丘维声) P67
对于每一个整数\(x\),
把\(x\)在模\(m\)同余关系\(\sim\)下的等价类,
称为一个\DefineConcept{模\(m\)剩余类},
记为\(\overline{x}\).
\end{definition}
\begin{proposition}
%@see: 《高等代数(第三版 下册)》(丘维声) P67
给定整数\(m\),
则模\(m\)剩余类一共有\(m\)个,
包括\begin{equation*}
	\overline{0}
	= \Set{ x\in\mathbb{Z} \given x\equiv0\pmod m },
	\dotsc,
	\overline{m-1}
	= \Set{ x\in\mathbb{Z} \given x\equiv m-1\pmod m }.
\end{equation*}
\end{proposition}
\begin{definition}
%@see: 《高等代数(第三版 下册)》(丘维声) P67
\(\mathbb{Z}\)对模\(m\)同余关系\(\sim\)的商集,
记作\(\mathbb{Z}_m\).
\end{definition}

\subsection{模m剩余类环}
\begin{proposition}
%@see: 《高等代数(第三版 下册)》(丘维声) P67 命题1
在\(\mathbb{Z}\)中,
若\(a\equiv b\pmod m,
c\equiv d\pmod m\),
则\[
	a+c\equiv b+d\pmod m, \qquad
	ac\equiv bd\pmod m.
\]
\begin{proof}
由已知条件,
\(m\mid(a-b),
m\mid(c-d)\).
从而\(m\mid[(a-b)+(c-d)]\),
即\(m\mid[(a+c)-(b+d)]\).
因此\(a+c\equiv b+d\pmod m\).

由于\(ac-bd
=ac-bc+bc-bd
=(a-b)c+b(c-d)\),
又有\(m\mid[(a-b)c+b(c-d)]\),
因此\(m\mid(ac-bd)\),
从而\(ac\equiv bd\pmod m\).
\end{proof}
\end{proposition}

\begin{proposition}\label{theorem:剩余类环.整数集对模m同余关系的商集对加法和乘法成环}
\(\mathbb{Z}_m\)对于加法、乘法成为一个有单位元的交换环.
\begin{proof}
\(\mathbb{Z}_m\)的加法满足交换律和结合律.
\(\overline{0}\)是零元.
\(\overline{-a}\)是\(\overline{a}\)的负元.
\(\overline{Z}_m\)的乘法满足结合律和交换律,以及对加法的分配律.
\(\overline{1}\)是单位元.
\end{proof}
\end{proposition}

\begin{definition}
% 由于\cref{theorem:剩余类环.整数集对模m同余关系的商集对加法和乘法成环},
把\(\mathbb{Z}_m\)称为\DefineConcept{模\(m\)剩余类环}.
\end{definition}

\subsection{模m剩余类环成为域的条件}
\begin{theorem}
%@see: 《高等代数(第三版 下册)》(丘维声) P69 定理2
若\(p\)是素数,
则模\(p\)剩余类环\(\mathbb{Z}_p\)是一个域.
\begin{proof}
已知\(\mathbb{Z}_p\)是一个有单位元\(\overline1\)的交换环.
任取\(\mathbb{Z}_p\)的一个非零元\(\overline{a}\),
其中\(0<a<p\).
于是\(p \nmid a\).
又由于\(p\)是素数,
因此\((p,a)=1\).
于是存在\(u,v\in\mathbb{Z}\),
使得\(up+va=1\).
因此\[
	\overline1
	=\overline{up+va}
	=\overline{up}
	+\overline{va}
	=\overline{u}~\overline{p}
	+\overline{v}~\overline{a}
	=\overline{v}~\overline{a}.
\]
可见\(\overline{a}\)是可逆元.
所以\(\mathbb{Z}_p\)是一个域.
\end{proof}
\end{theorem}

\begin{theorem}
%@see: 《高等代数(第三版 下册)》(丘维声) P71 习题7.11 2.
若\(p\)是合数,
则模\(p\)剩余类环\(\mathbb{Z}_p\)不是域.
%TODO proof
\end{theorem}

\subsection{模m剩余类域}
\begin{definition}
%@see: 《高等代数(第三版 下册)》(丘维声) P69
给定素数\(p\),
我们把\(\mathbb{Z}_p\)称为\DefineConcept{模\(p\)剩余类域}.
\end{definition}

模\(p\)剩余类域\(\mathbb{Z}_p\)与数域\(K\)有以下两个不同点:
\begin{enumerate}
	\item 数域\(K\)是无限域,
	而模\(p\)剩余类域\(\mathbb{Z}_p\)是有限域.

	\item 在\(\mathbb{Z}_p\)中,
	\(p\overline1
	=\overline{p}
	=\overline0\),
	\(l\overline1
	=\overline{l}
	\neq\overline0\ (0<l<p)\).
	在数域\(K\)中,
	有\((\forall n\in\mathbb{N}^*)[n1=n\neq0]\).
\end{enumerate}

\subsection{域的特征}
\begin{theorem}
%@see: 《高等代数(第三版 下册)》(丘维声) P70 定理3
设\(F\)是一个域,
它的单位元为\(e\),
则\begin{itemize}
	\item 要么\((\forall n\in\mathbb{N}^*)[ne\neq0]\);
	\item 要么存在一个素数\(p\),使得\(pe=0\),
	且当\(0<l<p\)时,有\(le\neq0\).
\end{itemize}
\begin{proof}
设\(n\)是使得\(ne=0\)成立的最小正整数.
假设\(n\)不是素数,
则\[
	n=n_1 n_2,
	\qquad
	0<n_1 \leq n_2<n.
\]
于是%根据习题7.1 11
\[
	(n_1 e)(n_2 e)
	=n_1[e(n_2 e)]
	=n_1[n_2(ee)]
	=n_1(n_2 e)
	=(n_1 n_2)e
	=ne=0.
\]
由于正整数\(n_1,n_2\)都小于\(n\),
因此\(n_1 e\neq0,
n_2 e\neq0\).
由于\(F\)是域,
所以\(n_1 e\)是可逆元.
于是\[
	n_2 e
	=[(n_1 e)^{-1} (n_1 e)](n_2 e)
	=(n_1 e)^{-1}
	[(n_1 e)(n_2 e)]
	=(n_1 e)^{-1} 0,
\]
矛盾!
因此\(n\)是素数.
\end{proof}
\end{theorem}

\begin{definition}\label{definition:域的特征.域的特征}
%@see: 《高等代数(第三版 下册)》(丘维声) P70 定义2
设\(F\)是一个域,
它的单位元为\(e\).
如果对于任一正整数\(n\)都有\(ne\neq0\),
那么称“域\(F\)的\DefineConcept{特征}为0”,
记作\(\FieldChar F=0\);
如果存在一个素数\(p\)使得\(pe=0\),
而当\(0<l<p\)时\(le\neq0\),
那么称“域\(F\)的\DefineConcept{特征}为\(p\)”,
记作\(\FieldChar F=p\).
\end{definition}

%@see: 《高等代数(第三版 下册)》(丘维声) P70
据此定义,域\(F\)的特征要么是零,要么是一个素数.

%@see: 《高等代数(第三版 下册)》(丘维声) P70
具体来说,模\(p\)剩余类域\(\mathbb{Z}_p\)的特征是\(p\).

%@see: 《高等代数(第三版 下册)》(丘维声) P70
任一数域的特征是零.

\begin{corollary}\label{theorem:域的特征.特征为p的域的性质1}
%@see: 《高等代数(第三版 下册)》(丘维声) P70 推论4
如果域\(F\)的特征是素数\(p\),
则\[
	ne=0
	\iff
	p \mid n.
\]
\begin{proof}
充分性.
设\(p \mid n\),
则\(n=lp\).
于是\(ne
=(lp)e
=0\).

必要性.
设\(ne=0\)
且\(n=hp+r,0\leq r<p\),
则\[
	0=ne
	=(hp+r)e
	=hpe+re
	=re.
\]
由于\(\FieldChar F=p\),
且\(r<p\),
所以由上式得\(r=0\).
从而\(n=hp\),
即\(p \mid n\).
\end{proof}
\end{corollary}

\begin{corollary}\label{theorem:域的特征.特征为p的域的性质2}
%@see: 《高等代数(第三版 下册)》(丘维声) P70 推论5
设域\(F\)的特征是素数\(p\),
任取\(a \in F-\{0\}\),
则\[
	na=0
	\iff
	p \mid n.
\]
\begin{proof}
\(na=0
\iff
n(ea)=0
\iff
(ne)a=0
\iff
ne=0
\iff
p \mid n\).
\end{proof}
\end{corollary}

\begin{example}\label{example:域.域上的特征恒等式}
%@see: 《高等代数(第三版 下册)》(丘维声) P71 习题7.11 5.
%@see: 《高等代数(大学高等代数课程创新教材 第二版 下册)》(丘维声) P134 例5
证明:在特征为素数\(p\)的域\(F\)中,成立\begin{equation*}
	(a+b)^p = a^p + b^p.
\end{equation*}
\begin{proof}
由二项式定理有\begin{equation*}
	(a+b)^p
	= \sum_{k=0}^p C_p^k a^{p-k} b^k.
\end{equation*}
由于\begin{equation*}
	C_p^k = \frac{p!}{k!(p-k)!},
	\quad k=0,1,2,\dotsc,p-1,p,
\end{equation*}
那么有\begin{equation*}
	p \mid C_p^k,
	\quad k=1,2,\dotsc,p-1,
\end{equation*}
所以\begin{equation*}
	C_p^k a^{p-k} b^k = 0,
	\quad k=1,2,\dotsc,p-1,
\end{equation*}
从而有\((a+b)^p = a^p + b^p\).
\end{proof}
\end{example}

\subsection{域上的多项式}
类似于数域\(K\)上的多项式,
我们可以定义任一域\(F\)上的多项式,
并且得出域\(F\)上的一元多项式环\(F[x]\)
和多元多项式环\(F[x_1,\dotsc,x_n]\).
不难看出,
有关数域\(K\)上一元多项式环\(K[x]\)的结论,
只要在它的证明中没有用到这个域含有无穷多个元素,
那么它对于任一域\(F\)上的一元多项式环\(F[x]\)也成立.
还需要注意,
如果域\(F\)的特征是素数\(p\),
则\(F\)的任一元素的\(p\)倍等于零.

例如,对于数域\(K\)上的两个一元多项式\(f(x)\)与\(g(x)\),
如果它们不相等,
那么由它们分别确定的多项式函数\(f\)与\(g\)也不相等.
这个结论的证明需要用到数域\(K\)有无穷多个元素.
因此这个结论对于有限域上的多项式就不成立.
譬如,在\(\mathbb{Z}_3[x]\)中,
设\(f(x)=x^3-x,
g(x)=0\).
显然\(f(x) \neq g(x)\).
但是由\(f(x)\)确定的多项式函数\(f\)满足\begin{equation*}
	f(\overline0)=\overline0, \qquad
	f(\overline1)=\overline{1^3}-\overline1=\overline0, \qquad
	f(\overline2)=\overline{2^3}-\overline2=\overline0,
\end{equation*}
因此\(f\)是零函数.
而\(g\)也是零函数.

在任一域\(F\)上的一元多项式环\(F[x]\)中,
也有不可约多项式的概念和唯一因式分解定理,
也有根与一次因式的关系,等等.


\chapter{线性空间}
本章我们将建立一个数学模型 --- 线性空间.
我们将研究线性空间的结构.
它是研究客观世界中线性问题的一个重要理论.
即使对于非线性问题,
经过局部化后,
就可以运用线性空间的理论,
或者用线性空间的理论研究非线性问题的某一侧面.

\section{线性空间的结构}
\subsection{线性空间的概念与性质}
\begin{definition}\label{definition:线性空间.线性空间的结构.线性空间的定义}
%@see: 《高等代数(第三版 下册)》(丘维声) P72 定义1
%@see: 《Linear Algebra and Its Applications (Second Edition)》(Peter D. Lax) P1
设\(V\)是一个非空集合,
\(F\)是一个域,
映射\(g\colon V \times V \to V\),
映射\(h\colon F \times V \to V\).
%@see: 《Linear Algebra Done Right (Fourth Edition)》(Sheldon Axler) P12 1.20
%@see: 《Linear Algebra Done Right (Fourth Edition)》(Sheldon Axler) P15 1.28
\begingroup
\let\labelitemi\relax
\begin{itemize}
		\item
		\vspace{-1.2cm}
		\begin{axiom}[加法交换律]\label{definition:线性空间.运算法则1}
		%@see: 《Linear Algebra and Its Applications (Second Edition)》(Peter D. Lax) P2 (2)
			\begin{equation*}
				(\forall\alpha,\beta\in V)
				[
					g(\alpha,\beta)
					= g(\beta,\alpha)
				].
			\end{equation*}
		\end{axiom}
		\begin{axiom}[加法结合律]\label{definition:线性空间.运算法则2}
		%@see: 《Linear Algebra and Its Applications (Second Edition)》(Peter D. Lax) P2 (3)
			\begin{equation*}
				(\forall\alpha,\beta,\gamma\in V)
				[
					g(g(\alpha,\beta),\gamma)
					= g(\alpha,g(\beta,\gamma))
				].
			\end{equation*}
		\end{axiom}
		\begin{axiom}[零元的存在性]\label{definition:线性空间.运算法则3}
		%@see: 《Linear Algebra and Its Applications (Second Edition)》(Peter D. Lax) P2 (4)
			\begin{equation*}
				(\forall \alpha \in V)
				(\exists \beta \in V)
				[
					g(\alpha,\beta)
					= \alpha
				].
			\end{equation*}

			如果\(\beta \in V\)满足\begin{equation*}
				(\forall \alpha \in V)
				[
					g(\alpha,\beta)
					= \alpha
				],
			\end{equation*}
			则称“\(\beta\)是\(V\)的一个\DefineConcept{零元}(additive identity)”,
			记为\(0\).
		\end{axiom}
		\begin{axiom}[负元的存在性]\label{definition:线性空间.运算法则4}
		%@see: 《Linear Algebra and Its Applications (Second Edition)》(Peter D. Lax) P2 (5)
			\begin{equation*}
				(\forall \alpha \in V)
				(\exists \beta \in V)
				[
					g(\alpha,\beta)
					= 0
				].
			\end{equation*}

			如果\(\alpha,\beta \in V\)满足\begin{equation*}
				g(\alpha,\beta)
				= 0,
			\end{equation*}
			则称“\(\beta\)是\(\alpha\)的一个\DefineConcept{负元}(additive inverse)”,
			记作\((-\alpha)\).
		\end{axiom}
		\begin{axiom}[域的单位元]\label{definition:线性空间.运算法则5}
		%@see: 《Linear Algebra and Its Applications (Second Edition)》(Peter D. Lax) P2 (9)
			设\(1\)是\(F\)的单位元,
			则\begin{equation*}
				(\forall \alpha \in V)
				[
					h(1,\alpha)
					= \alpha
				].
			\end{equation*}
		\end{axiom}
		\begin{axiom}[纯量乘法结合律]\label{definition:线性空间.运算法则6}
		%@see: 《Linear Algebra and Its Applications (Second Edition)》(Peter D. Lax) P2 (6)
			\begin{equation*}
				(\forall \alpha \in V)
				(\forall k,l \in F)
				[
					h(k,h(l,\alpha))
					= h(k l,\alpha)
				].
			\end{equation*}
		\end{axiom}
		\begin{axiom}[纯量乘法对纯量加法的分配律]\label{definition:线性空间.运算法则7}
		%@see: 《Linear Algebra and Its Applications (Second Edition)》(Peter D. Lax) P2 (8)
			\begin{equation*}
				(\forall \alpha \in V)
				(\forall k,l \in F)
				[
					h(k+l,\alpha)
					= g(
						h(k,\alpha)
						+ h(l,\alpha)
					)
				].
			\end{equation*}
		\end{axiom}
		\begin{axiom}[纯量乘法对向量加法的分配律]\label{definition:线性空间.运算法则8}
		%@see: 《Linear Algebra and Its Applications (Second Edition)》(Peter D. Lax) P2 (7)
			\begin{equation*}
				(\forall \alpha,\beta \in V)
				(\forall k\in F)
				[
					h(k,g(\alpha,\beta))
					= g(
						h(k,\alpha),
						h(k,\beta)
					)
				].
			\end{equation*}
		\end{axiom}
\end{itemize}
\endgroup
如果映射\(g\)和映射\(h\)满足上述八条公理,
则称“\((V,F,g,h)\)是一个\DefineConcept{线性空间}%
(\((V,F,g,h)\) is a \emph{linear space})”
或“\(V\)对\(f\)、\(g\)成为域\(F\)上的一个线性空间”,
在不致混淆的情况下简称“\(V\)是域\(F\)上的一个\DefineConcept{线性空间}%
(\(V\) is a \emph{linear space} over field \(F\))”,
或者进一步简称“\(V\)是一个\DefineConcept{线性空间}%
(\(V\) is a \emph{linear space})”;
%@see: 《Linear Algebra Done Right (Fourth Edition)》(Sheldon Axler) P12 1.21
把\(V\)中的每一个元素称为一个\DefineConcept{向量}(vector)或一个\DefineConcept{点}(point),
把\(F\)中的每一个元素称为一个\DefineConcept{标量}(scalar),
%@see: 《Linear Algebra Done Right (Fourth Edition)》(Sheldon Axler) P12 1.19
把映射\(g\)称为\DefineConcept{加法}(addition),
同时把任意两个向量\(\alpha,\beta\)在映射\(g\)下的像\(g(\alpha,\beta)\)记为\(\alpha + \beta\);
把映射\(h\)成为\DefineConcept{纯量乘法}(scalar multiplication)
\footnote{
	当域\(F\)是一个数域(例如\(\mathbb{Q},\mathbb{R},\mathbb{C}\))时,
	纯量乘法又称为\DefineConcept{数量乘法}.
},
同时把任意一个标量\(k\)和任意一个向量\(\alpha\)在映射\(h\)下的像\(h(k,\alpha)\)记为\(k \alpha\);
把加法与纯量乘法统称为\DefineConcept{线性运算}(linear operation).
\end{definition}
\begin{remark}
与我们在初等代数中学习的自然数、整数、有理数、实数的加法、乘法不同,
线性空间的加法、纯量乘法是抽象的,
线性空间的加法可以是映射空间\(V^{V \times V}\)中的任意一个映射,
线性空间的纯量乘法可以是映射空间\(V^{F \times V}\)中的任意一个映射.
另外,向量空间的加法、纯量乘法和域\(F\)的加法、乘法毫无关系.
\end{remark}
\begin{remark}
线性空间\(V\)对加法成群.
\end{remark}

\begin{definition}
设\(V\)是域\(F\)上的一个线性空间.
把映射\begin{equation*}
	t\colon V \times V \to V,
	(\alpha,\beta) \mapsto \alpha + (-\beta)
\end{equation*}
称为\DefineConcept{减法},
同时把任意两个向量\(\alpha,\beta\)在映射\(t\)下的像\(t(\alpha,\beta)\)记为\(\alpha - \beta\).
\end{definition}

\begin{definition}
%@see: 《Linear Algebra Done Right (Fourth Edition)》(Sheldon Axler) P13 1.22
把实数域\(\mathbb{R}\)上的每一个线性空间称为一个\DefineConcept{实线性空间}(real vector space).
\end{definition}

\begin{definition}
把复数域\(\mathbb{C}\)上的每一个线性空间称为一个\DefineConcept{复线性空间}(complex vector space).
\end{definition}

实线性空间与复线性空间,是代数结构完全不同的两个线性空间.

\begin{example}
下面列举一些常见的线性空间.
\begin{itemize}
	\item 只含零元\(0 \in V\)的线性空间\(\{0\}\),
	称为\DefineConcept{零空间}.

	%@see: 《高等代数(第三版 下册)》(丘维声) P73 例4
	\item 复数域\(\mathbb{C}\)
	可以看成是实数域\(\mathbb{R}\)上的一个线性空间,
	其加法是复数的加法,
	其数量乘法是实数与复数的乘法.

	%@see: 《高等代数(第三版 下册)》(丘维声) P73 例5
	\item 任一数域\(K\)都可以看成是自身上的线性空间,
	其加法就是数域\(K\)中的加法,
	其数量乘法就是数域\(K\)中的乘法.

	\item 集合\(\mathbb{R}^{n \times 1}\)关于向量的加法、实数与向量的纯量乘法,构成实线性空间.

	\item 集合\(\mathbb{R}^{s \times n}\)关于矩阵的加法、实数与矩阵的纯量乘法,构成实线性空间.

	\item 域\(F\)上全体\(n\)阶对称矩阵
	关于矩阵的加法、数与矩阵的纯量乘法,
	成为域\(F\)上的一个线性空间.

	\item 域\(F\)上全体\(n\)阶上三角矩阵
	关于矩阵的加法、数与矩阵的纯量乘法,
	成为域\(F\)上的一个线性空间.

	%@see: 《高等代数(第三版 下册)》(丘维声) P73 例2
	%@see: 《Linear Algebra Done Right (Fourth Edition)》(Sheldon Axler) P13 1.24
	\item 设\(F\)是一个域,\(X\)是一个非空集合,
	则映射空间\(F^X\)
	对函数的加法\begin{equation*}
		(f+g)(x) \defeq f(x) + g(x),
		\quad f,g \in F^X, x \in X,
	\end{equation*}
	以及实数与函数的数量乘法\begin{equation*}
		(k f)(x) \defeq k f(x),
		\quad f \in F^X, k \in F, x \in X,
	\end{equation*}
	成为\(F\)上的一个线性空间.
	\(F^X\)的零元是零函数\begin{equation*}
		0(x) = 0,
		\quad x \in X.
	\end{equation*}
	%上式等号左边的0表示零函数,等号右边的0表示域\(F\)的零元

	%@see: 《高等代数(第三版 下册)》(丘维声) P73 例3
	\item 数域\(K\)上的一元多项式环\(K[x]\)
	对多项式的加法,以及数与多项式的乘法,
	成为\(K\)上的一个线性空间.

	\item 数域\(K\)上所有次数小于\(n\)的一元多项式组成的集合\(K[x]_n\)
	对多项式的加法,以及数与多项式的乘法,
	成为\(K\)上的一个线性空间.
\end{itemize}
\end{example}

\begin{example}
数域\(K\)上所有次数等于\(n\)的一元多项式组成的集合
不是线性空间,
因为零多项式不属于这个集合,
它对线性运算不封闭.
\end{example}

上述例子表明,线性空间这一数学模型适用性很广.
从现在开始,我们将从线性空间的定义出发,
作逻辑推理,深入揭示线性空间的性质和结构,
它们对于所有的具体的线性空间都成立.

\begin{property}%\label{theorem:线性空间.线性空间的结构.线性空间的性质1}
%@see: 《高等代数(第三版 下册)》(丘维声) P74
%@see: 《Linear Algebra Done Right (Fourth Edition)》(Sheldon Axler) P14 1.26
线性空间的零元是唯一的.
\begin{proof}
设\(V\)是域\(F\)上的一个线性空间.
假设\(\beta_1,\beta_2\)都是\(V\)的零元,
那么由\cref{definition:线性空间.运算法则1,definition:线性空间.运算法则3}
可得\begin{equation*}
	\beta_1
	% \cref{definition:线性空间.运算法则3}
	= \beta_1 + \beta_2
	% \cref{definition:线性空间.运算法则1}
	= \beta_2 + \beta_1
	% \cref{definition:线性空间.运算法则3}
	= \beta_2.
\end{equation*}
因此\(V\)的零元是唯一的.
\end{proof}
\end{property}

\begin{property}\label{theorem:线性空间.线性空间的结构.线性空间的性质2}
%@see: 《高等代数(第三版 下册)》(丘维声) P74
%@see: 《Linear Algebra Done Right (Fourth Edition)》(Sheldon Axler) P15 1.27
线性空间中每一个向量的负元是唯一的.
\begin{proof}
设\(V\)是域\(F\)上的一个线性空间.
假设\(\beta_1,\beta_2\)都是\(\alpha\)的负元,
那么由\cref{definition:线性空间.运算法则2,definition:线性空间.运算法则3,definition:线性空间.运算法则4}
可得\begin{equation*}
	\beta_1
	% \cref{definition:线性空间.运算法则3}
	= \beta_1 + 0
	% \cref{definition:线性空间.运算法则4}
	= \beta_1 + (\alpha + \beta_2)
	% \cref{definition:线性空间.运算法则2}
	= (\beta_1 + \alpha) + \beta_2
	% \cref{definition:线性空间.运算法则4}
	= 0 + \beta_2
	% \cref{definition:线性空间.运算法则3}
	= \beta_2.
\end{equation*}
因此\(V\)中每个元素的负元是唯一的.
\end{proof}
\end{property}

\begin{property}\label{theorem:线性空间.线性空间的结构.线性空间的性质3}
%@see: 《高等代数(第三版 下册)》(丘维声) P74
%@see: 《Linear Algebra Done Right (Fourth Edition)》(Sheldon Axler) P15 1.30
%@see: 《Linear Algebra and Its Applications (Second Edition)》(Peter D. Lax) P2 (10)
零与任一向量数乘得零向量.
\begin{proof}
设\(V\)是域\(F\)上的一个线性空间.
由\cref{definition:线性空间.运算法则5,definition:线性空间.运算法则7} 可得\begin{equation*}
	0\alpha + \alpha
	% \cref{definition:线性空间.运算法则5}
	= 0\alpha + 1\alpha
	% \cref{definition:线性空间.运算法则7}
	= (0+1)\alpha
	% 域的加法
	= 1\alpha
	% \cref{definition:线性空间.运算法则5}
	= \alpha.
\end{equation*}
上式两边同时加上\((-\alpha)\)得\(
	(0\alpha + \alpha) + (-\alpha)
	= \alpha + (-\alpha)
\).
由\cref{definition:线性空间.运算法则4} 可知\(
	\alpha + (-\alpha)
	= 0
\).
于是\begin{equation*}
	(0\alpha + \alpha) + (-\alpha)
	= 0.
	\eqno(1)
\end{equation*}
由\cref{definition:线性空间.运算法则2,definition:线性空间.运算法则3,definition:线性空间.运算法则4} 可得\begin{equation*}
	(0\alpha + \alpha) + (-\alpha)
	% \cref{definition:线性空间.运算法则2}
	= 0\alpha + (\alpha + (-\alpha))
	% \cref{definition:线性空间.运算法则4}
	= 0\alpha + 0
	% \cref{definition:线性空间.运算法则3}
	= 0\alpha.
	\eqno(2)
\end{equation*}
由(1)(2)两式可得\(0\alpha = 0\).
因此\((\forall\alpha\in V)[0\alpha=0]\).
\end{proof}
\end{property}

\begin{property}\label{theorem:线性空间.线性空间的结构.线性空间的性质4}
%@see: 《高等代数(第三版 下册)》(丘维声) P74
%@see: 《Linear Algebra Done Right (Fourth Edition)》(Sheldon Axler) P16 1.31
任一标量与零向量数乘得零向量.
\begin{proof}
设\(V\)是域\(F\)上的一个线性空间.
由\cref{definition:线性空间.运算法则3,definition:线性空间.运算法则8}
可得\begin{equation*}
	k0 + k0
	% \cref{definition:线性空间.运算法则8}
	= k(0+0)
	% \cref{definition:线性空间.运算法则3}
	= k0.
\end{equation*}
上式两边同时加上\((-k0)\),得\(
	(k0 + k0) + (-k0)
	= k0 + (-k0)
\).
由\cref{definition:线性空间.运算法则4} 可知\(
	k0 + (-k0)
	= 0
\).
于是\begin{equation*}
	(k0 + k0) + (-k0)
	= 0.
	\eqno(1)
\end{equation*}
由\cref{definition:线性空间.运算法则2,definition:线性空间.运算法则3,definition:线性空间.运算法则4}
可得\begin{equation*}
	(k0 + k0) + (-k0)
	% \cref{definition:线性空间.运算法则2}
	= k0 + (k0 + (-k0))
	% \cref{definition:线性空间.运算法则4}
	= k0 + 0
	% \cref{definition:线性空间.运算法则3}
	= k0.
	\eqno(2)
\end{equation*}
由(1)(2)两式可得\begin{equation*}
	k0 = 0.
\end{equation*}
因此\((\forall k\in F)[k0=0]\).
\end{proof}
\end{property}

\begin{property}\label{theorem:线性空间.线性空间的结构.线性空间的性质5}
%@see: 《高等代数(第三版 下册)》(丘维声) P74
%@see: 《Linear Algebra Done Right (Fourth Edition)》(Sheldon Axler) P16 Exercise 1B 2
设\(V\)是域\(F\)上的一个线性空间,
标量\(k \in F\),向量\(\alpha \in V\),
则\begin{equation*}
	k\alpha=0
	\implies
	k=0 \lor \alpha=0.
\end{equation*}
\begin{proof}
由\cref{theorem:线性空间.线性空间的结构.线性空间的性质3}
可知\(k = 0 \implies k\alpha = 0\).
假设\(k\neq0\)且\(k\alpha = 0\),
则由\cref{definition:线性空间.运算法则5,definition:线性空间.运算法则6}
以及\cref{theorem:线性空间.线性空间的结构.线性空间的性质4}
可得\begin{equation*}
	\alpha
	% \cref{definition:线性空间.运算法则5}
	= 1\alpha
	% \(k\neq0\),域的定义(域中每一个非零元都是可逆元)
	= (k^{-1} k) \alpha
	% \cref{definition:线性空间.运算法则6}
	= k^{-1} (k \alpha)
	% \(k\alpha = 0\)
	= k^{-1} 0
	% \cref{theorem:线性空间.线性空间的结构.线性空间的性质4}
	= 0.
	\qedhere
\end{equation*}
\end{proof}
\end{property}

\begin{property}%\label{theorem:线性空间.线性空间的结构.线性空间的性质6}
%@see: 《高等代数(第三版 下册)》(丘维声) P74
%@see: 《Linear Algebra Done Right (Fourth Edition)》(Sheldon Axler) P16 1.32
设\(V\)是域\(F\)上的一个线性空间,
则\((\forall\alpha\in V)[(-1)\alpha=-\alpha]\).
\begin{proof}
由\cref{definition:线性空间.运算法则5,definition:线性空间.运算法则7}
以及\cref{theorem:线性空间.线性空间的结构.线性空间的性质3}
可得\begin{equation*}
	\alpha + (-1)\alpha
	% \cref{definition:线性空间.运算法则5}
	= 1\alpha + (-1)\alpha
	% \cref{definition:线性空间.运算法则7}
	= (1+(-1))\alpha
	% 域的加法
	= 0\alpha
	% \cref{theorem:线性空间.线性空间的结构.线性空间的性质3}
	= 0.
\end{equation*}
再由\hyperref[theorem:线性空间.线性空间的结构.线性空间的性质2]{负元的唯一性}可知\((-1)\alpha = -\alpha\).
\end{proof}
\end{property}

\begin{property}
%@see: 《Linear Algebra Done Right (Fourth Edition)》(Sheldon Axler) P16 Exercise 1B 1
设\(V\)是域\(F\)上的一个线性空间,
\(\alpha \in V\),
则\(-(-\alpha) = \alpha\).
\begin{proof}
由\cref{definition:线性空间.运算法则4}
可知\(\alpha\)是\((-\alpha)\)的负元.
再由\hyperref[theorem:线性空间.线性空间的结构.线性空间的性质2]{负元的唯一性}可知\(\alpha = -(\alpha)\).
\end{proof}
\end{property}

\begin{example}
假设把\cref{definition:线性空间.运算法则5} 修改为\begin{equation*}
	(\exists e \in F)
	(\forall \alpha \in V)
	[
		e \alpha = \alpha
	].
\end{equation*}
试讨论:满足\(
	(\forall \alpha \in V)
	[
		e \alpha = \alpha
	]
\)的\(e \in F\)与\(F\)的单位元\(1\)是否相等.
\begin{solution}
%@credit: {gemini},{8b6edada-f2fd-4ae5-9020-eb533149a54c},{855486ab-2fcf-40c1-b774-09956dfb4012}
假设\(e \in F\)满足
对于任意\(\alpha \in V\)成立\(e \alpha = \alpha\),
那么\(1 (e \alpha) = 1 \alpha\),
再由\cref{definition:线性空间.运算法则6} 可知\begin{equation*}
	(1 e) \alpha = 1 (e \alpha) = 1 \alpha.
\end{equation*}
根据域的单位元的定义,有\(1 e = e 1 = e\),
那么对于任意\(\alpha \in V\),成立\begin{equation*}
	(1 e) \alpha = e \alpha = \alpha.
\end{equation*}
于是对于任意\(\alpha \in V\),成立\begin{equation*}
	1 \alpha = \alpha = e \alpha,
	\eqno(1)
\end{equation*}
从而有\((1-e) \alpha = 0\).

\cref{theorem:线性空间.线性空间的结构.线性空间的性质5} 的证明过程
用到了修改之前的\cref{definition:线性空间.运算法则5},
所以我们不能直接由它得出以下结论:\begin{equation*}
	(\forall k \in F)
	(\forall \alpha \in V)
	[
		k\alpha=0 \implies k=0 \lor \alpha=0
	].
\end{equation*}

如果\(V\)只含零向量,
那么由\cref{theorem:线性空间.线性空间的结构.线性空间的性质4} 可知,
不论\((1-e)\)是否为零,
\((1-e)\alpha\)恒等于零.
这就说明,当\(V = \{0\}\)时,
\(e\)不一定是\(F\)的单位元,
事实上,\(e\)可以是\(F\)中的任意一个元素.

如果\(V\)含有非零向量,
那么,为了证明\(e=1\),只需证\begin{equation*}
	\alpha\neq0, k\alpha = 0 \implies k=0.
\end{equation*}
首先假设\(\alpha\neq0, k\alpha = 0\).
接着用反证法,假设\(k\neq0\).
鉴于域\(F\)中每个非零元\(k\)都有逆元\(k^{-1}\),
在\(k\alpha = 0\)两边同时乘以\(k^{-1}\)得\begin{equation*}
	k^{-1}(k\alpha) = k^{-1}0.
	\eqno(2)
\end{equation*}
由\cref{theorem:线性空间.线性空间的结构.线性空间的性质4} 可知\begin{equation*}
	k^{-1}0 = 0.
	\eqno(3)
\end{equation*}
由\cref{definition:线性空间.运算法则6} 可知\begin{equation*}
	k^{-1}(k\alpha)
	= (k^{-1} k)\alpha,
\end{equation*}
因为\(k^{-1} k = 1\),
所以\begin{equation*}
	(k^{-1} k)\alpha
	= 1\alpha,
\end{equation*}
于是\begin{equation*}
	k^{-1}(k\alpha)
	= 1\alpha,
\end{equation*}
再由(1)式可知\(1\alpha = \alpha\),从而有\begin{equation*}
	k^{-1}(k\alpha)
	= \alpha.
	\eqno(4)
\end{equation*}
由(2)(3)(4)可得\begin{equation*}
	\alpha = 0.
\end{equation*}
这与前提条件“\(\alpha\neq0\)”相矛盾,
因此一定有\(k=0\).
于是由\((1-e)\alpha = 0\)得\(1-e=0\),即\(e=1\).

综上所述,
在将\cref{definition:线性空间.运算法则5} 修改为\begin{equation*}
	(\exists e \in F)
	(\forall \alpha \in V)
	[
		e \alpha = \alpha
	]
\end{equation*}
以后,
要么\(e\)就是域\(F\)的单位元(即\(e = 1\)),
要么\(V\)是只含零向量的线性空间(即\(V = \{0\}\)).
\end{solution}
\end{example}

\begin{example}
%@see: 《高等代数(第三版 下册)》(丘维声) P81 习题8.1 1.(2)
在正实数集\(\mathbb{R}^+\)上定义加法、数量乘法:\begin{gather*}
	\oplus \defeq \Set{
		((a,b),ab)
		\given
		a,b \in \mathbb{R}^+
	}, \\
	\odot \defeq \Set{
		((k,a),a^k)
		\given
		a \in \mathbb{R}^+,
		k \in \mathbb{R}
	}.
\end{gather*}
试判断\((\mathbb{R}^+,\oplus,\odot)\)是不是实数域\(\mathbb{R}\)上的线性空间.
\begin{solution}
显然\(\oplus,\odot\)都是映射.
由于实数的乘法运算适合交换律、结合律,
所以\(\oplus\)也适合交换律、结合律.
正实数\(1\)是\(\mathbb{R}^+\)的零元,
这是因为对于任意\(a \in \mathbb{R}^+\)总有\(1 \oplus a = 1a = a\).
任意一个正实数\(a\)的倒数\(1/a\)就是\(a\)的负元.
实数\(1\)还是\(\mathbb{R}^+\)的单位元,
这是因为\(1 \odot a = a^1 = a\).
对于任意正实数\(a,b\)和任意实数\(k,l\),
显然有\begin{gather*}
	k \odot (l \odot a)
	= (a^l)^k
	= a^{k l}
	= (kl) \odot a, \\
	(k+l) \odot a
	= a^{k+l}
	= a^k a^l
	= (k \odot a) \oplus (l \odot a), \\
	k \odot (a \oplus b)
	= (ab)^k
	= a^k b^k
	= (k \odot a) \oplus (k \odot b).
\end{gather*}
综上所述,\((\mathbb{R}^+,\oplus,\odot)\)确实是实数域\(\mathbb{R}\)上的一个线性空间.
\end{solution}
\end{example}

\subsection{线性空间的线性关系}
域\(F\)上的线性空间\(V\)的有限子集,称为“\(V\)中的一个\DefineConcept{向量组}”.

向量组\(A\)的子集,称为“\(A\)的一个\DefineConcept{部分组}”.

%@see: 《高等代数(第三版 下册)》(丘维声) P75
%@see: 《Linear Algebra and Its Applications (Second Edition)》(Peter D. Lax) P4 Definition
%@see: 《Linear Algebra Done Right (Fourth Edition)》(Sheldon Axler) P28 2.2
设\(\AutoTuple{\alpha}{s}\)是\(V\)中一个向量组,
任给\(F\)中一组元素\(\AutoTuple{k}{s}\),
向量\(k_1\alpha_1+\dotsb+k_s\alpha_s\)
称为“\(\AutoTuple{\alpha}{s}\)的一个\DefineConcept{线性组合}(linear combination)”,
称\(\AutoTuple{k}{s}\)为\DefineConcept{系数}.

%@see: 《高等代数(第三版 下册)》(丘维声) P75
对于\(\beta\in V\),
如果有\(F\)中一组元素\(\AutoTuple{c}{s}\),
使得\(\beta=c_1\alpha_1+\dotsb+c_s\alpha_s\),
则称“\(\beta\)可以由\(\AutoTuple{\alpha}{s}\)~\DefineConcept{线性表出}%
(\(\beta\) can be expressed as a linear combination of \(\AutoTuple{\alpha}{s}\))”.

\begin{definition}
%@see: 《高等代数(第三版 下册)》(丘维声) P75 定义2
%@see: 《Linear Algebra and Its Applications (Second Edition)》(Peter D. Lax) P4 Definition
%@see: 《Linear Algebra and Its Applications (Second Edition)》(Peter D. Lax) P5 Definition
%@see: 《Linear Algebra Done Right (Fourth Edition)》(Sheldon Axler) P32 2.15
%@see: 《Linear Algebra Done Right (Fourth Edition)》(Sheldon Axler) P33 2.17
设\(\AutoTuple{\alpha}{s}\ (s\geq1)\)是\(V\)中一个向量组.
如果有\(F\)中不全为零的元素\(\AutoTuple{k}{s}\),
使得\(k_1\alpha_1+\dotsb+k_s\alpha_s=0\),
则称“\(\AutoTuple{\alpha}{s}\)是\DefineConcept{线性相关的}%
(\(\AutoTuple{\alpha}{s}\) are \emph{linearly dependent})”;
否则称“\(\AutoTuple{\alpha}{s}\)是\DefineConcept{线性无关的}%
(\(\AutoTuple{\alpha}{s}\) are \emph{linearly independent})”.
\end{definition}

%@see: 《Linear Algebra Done Right (Fourth Edition)》(Sheldon Axler) P32 2.15
空向量组\(\emptyset\)是线性无关的.

\begin{definition}
%@see: 《高等代数(第三版 下册)》(丘维声) P75 定义3
设\(W\)是\(V\)的任一无限子集.
如果\(W\)有一个有限子集是线性相关的,
则称“\(W\)是\DefineConcept{线性相关的}%
(\(W\) is \emph{linearly dependent})”;
如果\(W\)的任何有限子集都是线性无关的,
则称“\(W\)是\DefineConcept{线性无关的}%
(\(W\) is \emph{linearly independent})”.
\end{definition}

可以证明,
数域\(K\)上的线性方程组的理论,
和数域\(K\)上的矩阵、行列式理论,
在把数域\(K\)换成任意域\(F\)以后,
仍然成立.
\begin{property}\label{theorem:线性空间.线性相关性1}
%@see: 《高等代数(第三版 下册)》(丘维声) P75 例6
%@see: 《高等代数(第三版 下册)》(丘维声) P75 例7
%@see: 《高等代数(第三版 下册)》(丘维声) P75 命题1
%@see: 《高等代数(第三版 下册)》(丘维声) P75 命题2
%@see: 《Linear Algebra Done Right (Fourth Edition)》(Sheldon Axler) P33 2.19
设\(V\)是域\(F\)上的一个线性空间.
\begin{itemize}
	\item \(\text{$\alpha$线性相关}\iff\alpha=0\).
	\item 包含零向量的向量组一定线性相关.
	\item 基数大于或等于\(2\)的向量组\(W\)线性相关
	当且仅当\(W\)中至少有一个向量可以由其余向量中的有限多个线性表出.
	\item 向量\(\beta\)可以由线性无关向量组\(\AutoTuple{\alpha}{s}\)线性表出的充分必要条件是
	\(\AutoTuple{\alpha}{s},\beta\)线性相关.
\end{itemize}
\end{property}

\begin{definition}
%@see: 《高等代数(第三版 下册)》(丘维声) P76 定义4
设\(W_1,W_2\)都是\(V\)的非空子集,
如果\(W_1\)中每一个向量都可以由\(W_2\)中有限多个向量线性表出,
则称“\(W_1\)可以由\(W_2\)~\DefineConcept{线性表出}”.
如果\(W_1\)与\(W_2\)可以互相线性表出,
则称“\(W_1\)与\(W_2\)是\DefineConcept{等价的}”.
\end{definition}

容易证明,“线性表出”具有传递性,
从而“等价”也具有传递性.
显然,向量组的“等价”具有反身性与对称性.

\begin{property}\label{theorem:线性空间.线性相关性2}
%@see: 《高等代数(第三版 下册)》(丘维声) P76 引理1
%@see: 《高等代数(第三版 下册)》(丘维声) P76 推论3
%@see: 《高等代数(第三版 下册)》(丘维声) P76 推论4
设\(V\)是域\(F\)上的一个线性空间.
\begin{itemize}
	\item 设向量组\(\AutoTuple{\beta}{r}\)
	可以由向量组\(\AutoTuple{\alpha}{s}\)线性表出,则\begin{gather*}
		r>s
		\implies
		\text{$\AutoTuple{\beta}{r}$线性相关}, \\
		\text{$\AutoTuple{\beta}{r}$线性无关}
		\implies
		r\leq s.
	\end{gather*}

	\item 等价的线性无关的向量组所含向量的个数相等.
\end{itemize}
\end{property}

\subsection{向量组的秩}
\begin{definition}
%@see: 《高等代数(第三版 下册)》(丘维声) P76 定义5
设\(V\)是域\(F\)上的一个线性空间,
\(A\)是\(V\)的一个子集,
\(a\)是\(A\)的有限子集.
如果\(a\)是线性无关的,
但是\begin{equation*}
	(\forall\beta \in A-a)
	[\text{$a \cup \{\beta\}$是线性相关的}],
\end{equation*}
则称“\(a\)是\(A\)的一个\DefineConcept{极大线性无关组}”.
\end{definition}

\begin{property}
%@see: 《高等代数(第三版 下册)》(丘维声) P76 推论5
%@see: 《高等代数(第三版 下册)》(丘维声) P76 推论6
设\(V\)是域\(F\)上的一个线性空间.
\begin{itemize}
	\item 向量组与它的极大线性无关组等价.
	\item 向量组的任意两个极大线性无关组的基数相等.
\end{itemize}
\end{property}

\begin{definition}
%@see: 《高等代数(第三版 下册)》(丘维声) P76 定义6
向量组\(A=\{\AutoTuple{\alpha}{s}\}\)的一个极大线性无关组的基数,
称为“向量组\(A\)的\DefineConcept{秩}(rank)”,
记为\(\rank A\)或\(\rank\{\AutoTuple{\alpha}{s}\}\).
\end{definition}

\begin{property}\label{theorem:线性空间.向量组的秩的性质}
%@see: 《高等代数(第三版 下册)》(丘维声) P76 命题8
%@see: 《高等代数(第三版 下册)》(丘维声) P76 命题9
%@see: 《高等代数(第三版 下册)》(丘维声) P76 推论9
设\(V\)是域\(F\)上的一个线性空间.
\begin{itemize}
	\item 全由零向量组成的向量组的秩为零.

	\item 向量组线性无关的充分必要条件是
	它的秩等于它的基数.

	\item 设\(A,B\)都是向量组.
	如果\(A\)可以由\(B\)线性表出,
	则\(\rank A \leq \rank B\).

	\item 等价的向量组有相同的秩.
\end{itemize}
\end{property}

\subsection{线性空间的基}
\begin{definition}\label{definition:线性空间.线性空间的基}
%@see: 《高等代数(第三版 下册)》(丘维声) P76 定义7
%@see: 《Linear Algebra Done Right (Fourth Edition)》(Sheldon Axler) P39 2.26
设\(V\)是域\(F\)上的一个线性空间,\(S \subseteq V\).
如果\begin{itemize}
	\item \(S\)线性无关,
	\item \(V\)中每一个向量都可以由\(S\)中有限多个向量线性表出,
\end{itemize}
则称“\(S\)是\(V\)的一个\DefineConcept{基}%
(\(S\) is a \emph{basis} for \(V\))”.
\end{definition}
\begin{remark}
%@credit: {647826c9-7e2a-49d1-b176-cd39b299b349} 说:除了 Hamel 基以外,还有 Schauder 基等其他定义
%@credit: {85841724-e8e0-4a39-88bf-973ade1b5e13} 说:参考《代数学(一)》(李方、邓少强、冯荣权、刘东文) P101 定义5.1.2
在\hyperref[definition:线性空间.线性空间的基]{基的定义}中,
必须要注意第二个条件中“有限多个”这个限定,
它说明这里定义的基是\emph{哈莫基}(Hamel basis).
%@see: https://zh.wikipedia.org/wiki/%E5%9F%BA_(%E7%B7%9A%E6%80%A7%E4%BB%A3%E6%95%B8)
%@see: https://en.wikipedia.org/wiki/Basis_(linear_algebra)
\end{remark}

\begin{property}\label{theorem:线性空间的结构.零空间的基是空集}
%@see: 《高等代数(第三版 下册)》(丘维声) P77
零空间的基是空集.
%TODO 无法确定这个究竟是定义还是性质
\end{property}

\begin{property}
%@see: 《高等代数(第三版 下册)》(丘维声) P77
%@see: 《Linear Algebra Done Right (Fourth Edition)》(Sheldon Axler) P41 2.31
域\(F\)上的任一线性空间\(V\)都有基.
%TODO proof
\end{property}

\begin{example}
%@see: 《高等代数(第三版 下册)》(丘维声) P77 例8
在数域\(K\)上全体\(s \times n\)矩阵形成的线性空间\(M_{s \times n}(K)\)中,
所有基本矩阵组成的子集\begin{equation*}
	\Set{
		E_{11},E_{12},\dotsc,E_{1n},
		\dotsc,
		E_{s1},E_{s2},\dotsc,E_{sn}
	}
\end{equation*}是\(M_{s \times n}(K)\)的一个基.
\begin{proof}
每个\(s \times n\)矩阵\(A = (a_{ij})_{s \times n}\)都可以表示成\begin{equation*}
	A = \sum_{i=1}^s \sum_{j=1}^n a_{ij} E_{ij}.
\end{equation*}

假设\begin{equation*}
	\sum_{i=1}^s \sum_{j=1}^n a_{ij} E_{ij} = 0,
\end{equation*}
则矩阵\(A = (a_{ij})_{s \times n}\)是零矩阵,
从而\(a_{ij} = 0\ (i=1,2,\dotsc,s;j=1,2,\dotsc,n)\).
因此\begin{equation*}
	\Set{
		E_{11},E_{12},\dotsc,E_{1n},
		\dotsc,
		E_{s1},E_{s2},\dotsc,E_{sn}
	}
\end{equation*}线性无关,
从而说明它是\(M_{s \times n}(K)\)的一个基.
\end{proof}
\end{example}

\begin{example}
%@see: 《高等代数(第三版 下册)》(丘维声) P77 例9
数域\(K\)上所有一元多项式形成的线性空间\(K[x]\)中,
子集\begin{equation*}
	\{1,x,x^2,\dotsc,x^n,\dotsc\}
\end{equation*}是\(K[x]\)的一个基.
\begin{proof}
\(K[x]\)上每一个一元多项式\(f(x)\)
可以写成\(f(x)=a_0+a_1 x+a_2 x^2+\dotsb+a_n x^n\).
任取\(S\)的一个有限子集\(\{x^{i_1},\dotsc,x^{i_m}\}\).
设\(k_1 x^{i_1}+\dotsb+k_m x^{i_m}=0\),
则由一元多项式的定义得
\(k_1=\dotsb=k_m=0\),
从而这个子集线性无关,
因此\(S\)线性无关,
于是\(S\)是\(K[x]\)的一个基.
\end{proof}
\end{example}

\subsection{有限维线性空间,无限维线性空间}
\begin{definition}
%@see: 《高等代数(第三版 下册)》(丘维声) P77 定义8
%@see: 《Linear Algebra and Its Applications (Second Edition)》(Peter D. Lax) P5 Definition
%@see: 《Linear Algebra Done Right (Fourth Edition)》(Sheldon Axler) P31 2.13
设\(V\)是域\(F\)上的一个线性空间,
\(S\)是\(V\)的一个基.
如果\(S\)是有限集,
则称“\(V\)是\DefineConcept{有限维的}(finite-dimensional)”;
否则称“\(V\)是\DefineConcept{无限维的}(infinite-dimensional)”.
\end{definition}

\begin{example}
数域\(K\)上全体\(s \times n\)矩阵\(M_{s \times n}(K)\)是有限维的.
\end{example}

\begin{example}
%@see: 《Linear Algebra Done Right (Fourth Edition)》(Sheldon Axler) P31 2.14
数域\(K\)上全体一元多项式\(K[x]\)是无限维的.
\end{example}

\subsection{有限维线性空间的维数}
\begin{theorem}\label{theorem:线性空间.同一个线性空间的任意两个基的基数相等}
%@see: 《高等代数(第三版 下册)》(丘维声) P77 定理10
%@see: 《Linear Algebra and Its Applications (Second Edition)》(Peter D. Lax) P5 Theorem 3.
%@see: 《Linear Algebra Done Right (Fourth Edition)》(Sheldon Axler) P44 2.34
% 原话是: Any two bases of a finite-dimensional vector space have the same length.
设\(V\)是域\(F\)上的一个线性空间.
如果\(V\)是有限维的,
则\(V\)的任意两个基的基数相等.
\begin{proof}
不妨设\(V\)有一个基包含有限多个向量\(\AutoTuple{\alpha}{n}\).
设\(S\)是\(V\)的另一个基.

假如\(\card S>n\),
则\(S\)中可取出\(n+1\)个向量\(\AutoTuple{\beta}{n+1}\),
它们可以由\(\AutoTuple{\alpha}{n}\)线性表出.
由\cref{theorem:线性空间.线性相关性2},%引理1
可知\(\AutoTuple{\beta}{n+1}\)线性相关.
这与\(S\)线性无关矛盾,
因此\(\card S\leq n\).

设\(S=\{\AutoTuple{\beta}{m}\}\ (m\leq n)\),
由\cref{theorem:线性空间.线性相关性2},%推论4
又可知\(m=n\).
\end{proof}
\end{theorem}

\begin{definition}
%@see: 《高等代数(第三版 下册)》(丘维声) P78 定义9
%@see: 《Linear Algebra Done Right (Fourth Edition)》(Sheldon Axler) P44 2.35
设\(V\)是域\(F\)上的一个有限维线性空间,
则\(V\)的一个基的基数
称为“线性空间\(V\)的\DefineConcept{维数}(the \emph{dimension} of \(V\))”,
记作\(\dim_F V\),
简记为\(\dim V\).
\end{definition}

\begin{property}
零空间的维数为\(0\).
\begin{proof}
由\cref{theorem:线性空间的结构.零空间的基是空集} 立即可得.
\end{proof}
\end{property}

\begin{example}
\(\dim M_{s \times n}(K)=sn\).
\end{example}

\begin{example}
\(\dim M_n(K) = n^2\).
\end{example}

维数对于研究有限维线性空间的结构起着重要的作用.

\begin{property}\label{theorem:线性空间.线性相关性3}
%@see: 《高等代数(第三版 下册)》(丘维声) P78 命题11
%@see: 《高等代数(第三版 下册)》(丘维声) P78 命题12
%@see: 《Linear Algebra and Its Applications (Second Edition)》(Peter D. Lax) P5 Lemma 1.
%@see: 《Linear Algebra Done Right (Fourth Edition)》(Sheldon Axler) P35 2.22
%@see: 《Linear Algebra Done Right (Fourth Edition)》(Sheldon Axler) P45 2.38
设\(V\)是域\(F\)上的一个线性空间.
\begin{itemize}
	\item 如果\(\dim V=n\),
	则\(V\)中任意\(n+1\)个向量都线性无关.

	\item 如果\(\dim V=n\),
	则\(V\)中任意\(n\)个线性无关的向量都是\(V\)的一个基.

	\item 如果\(V\)的有限子集\(\AutoTuple{\alpha}{s}\)是线性无关的,
	则\(s \leq n\).
\end{itemize}
%TODO proof
\end{property}

基对于研究线性空间的结构起着重要的作用.

\begin{property}\label{theorem:线性空间.任一向量可由给定基唯一线性表出}
%@see: 《高等代数(第三版 下册)》(丘维声) P78 命题13
%@see: 《Linear Algebra Done Right (Fourth Edition)》(Sheldon Axler) P39 2.28
设\(V\)是域\(F\)上的一个线性空间,
\(\AutoTuple{\alpha}{n}\)是\(V\)的一个基,
则\(V\)中每一个向量\(\alpha\)
可以唯一地表成\(\AutoTuple{\alpha}{n}\)的线性组合.
\begin{proof}
从基的定义知道,任意一个向量\(\alpha\),均可由\(\AutoTuple{\alpha}{n}\)线性表出.
假设有如下两种表出方式:\begin{gather*}
	\alpha = x_1 \alpha_1 + x_2 \alpha_2 + \dotsb + x_n \alpha_n, \\
	\alpha = y_1 \alpha_1 + y_2 \alpha_2 + \dotsb + y_n \alpha_n.
\end{gather*}
相减得\begin{equation*}
	0 = (x_1 - y_1) \alpha_1 + (x_2 - y_2) \alpha_2 + \dotsb + (x_n - y_n) \alpha_n.
\end{equation*}
由于\(\AutoTuple{\alpha}{n}\)线性无关,
所以\begin{equation*}
	x_1 - y_1
	= x_2 - y_2
	= \dotsb
	= x_n - y_n
	= 0.
\end{equation*}
由此可见表出方式唯一.
\end{proof}
\end{property}

\begin{example}\label{example:线性空间.生成子空间等于线性空间的向量组就是基}
%@see: 《高等代数(第三版 下册)》(丘维声) P82 习题8.1 10.
证明:在数域\(K\)上的\(n\)维线性空间\(V\)中,
如果每一个向量都可以由\(\AutoTuple{\alpha}{n}\)线性表出,
则\(\AutoTuple{\alpha}{n}\)是\(V\)的一个基.
\begin{proof}
在\(V\)中取一个基\(\AutoTuple{\delta}{n}\).
由题设条件可知\(\AutoTuple{\delta}{n}\)可以由\(\AutoTuple{\alpha}{n}\)线性表出,
那么由\cref{theorem:线性空间.向量组的秩的性质} 可知\begin{equation*}
	\rank\{\AutoTuple{\delta}{n}\}
	\leq
	\rank\{\AutoTuple{\alpha}{n}\}.
\end{equation*}
又因为\(\AutoTuple{\alpha}{n}\)可以由\(\AutoTuple{\delta}{n}\)线性表出,
从而\begin{equation*}
	\rank\{\AutoTuple{\alpha}{n}\}
	\leq
	\rank\{\AutoTuple{\delta}{n}\}.
\end{equation*}
于是\(\rank\{\AutoTuple{\alpha}{n}\}
= \rank\{\AutoTuple{\delta}{n}\}
= n\),
那么\(\AutoTuple{\alpha}{n}\)线性无关,
因此\(\AutoTuple{\alpha}{n}\)是\(V\)的一个基.
\end{proof}
%@see: 《高等代数(第三版 上册)》(丘维声) P80 习题3.4 4.
\end{example}

\subsection{向量的坐标,过渡矩阵}\label{section:线性空间.向量的坐标}
%@see: 《高等代数(第三版 下册)》(丘维声) P78
%@see: 《Linear Algebra Done Right (Fourth Edition)》(Sheldon Axler) P88 3.73
我们把向量\(\alpha\)由基\(\AutoTuple{\alpha}{n}\)线性表出的系数
组成的\(n\)元有序组\((\AutoTuple{a}{n})\)
称为“向量\(\alpha\)在基\(\AutoTuple{\alpha}{n}\)下的\DefineConcept{坐标}(coordinate)”,
记作\(\VectorMatrix(\alpha,(\AutoTuple{\alpha}{n}))\),
或在不致混淆的情况下简记为\(\VectorMatrix(\alpha)\).
通常把向量的坐标写成列向量形式,
即\begin{equation*}
	\VectorMatrix(\alpha,(\AutoTuple{\alpha}{n}))
	=
	\begin{bmatrix}
		a_1 \\
		\vdots \\
		a_n
	\end{bmatrix}
	\defiff
	\alpha = a_1 \alpha_1 + \dotsb + a_n \alpha_n.
\end{equation*}

由上可知,有限维线性空间\(V\)中给定一个基,
则\(V\)中每一个向量都可以唯一地表示成这个基的线性组合,
从而\(V\)的结构就很清楚了.
因此,基是研究线性空间的结构的第一条途径.

\(n\)维线性空间\(V\)中给定两个基,
我们想要知道,\(V\)中每一个向量分别在这两个基下的坐标有什么关系.

设\(\AutoTuple{\alpha}{n}\)和\(\AutoTuple{\beta}{n}\)是\(V\)的两个基,
\(V\)中向量\(\alpha\)在这两个基下的坐标分别为\begin{equation*}
	X=(\AutoTuple{x}{n})^T, \qquad
	Y=(\AutoTuple{y}{n})^T.
\end{equation*}
为了求\(X\)与\(Y\)之间的关系,
首先把这两个基之间的关系搞清楚.
由于\(\AutoTuple{\alpha}{n}\)是\(V\)的一个基,
并且\(\AutoTuple{\beta}{n}\)都是\(V\)中的元素,
那么,由\cref{theorem:线性空间.任一向量可由给定基唯一线性表出} 可知,
向量组\(\AutoTuple{\beta}{n}\)中的每一个向量\(\beta_i\)
均可唯一地表示成\(\AutoTuple{\alpha}{n}\)的线性组合,
因此有\begin{equation}\label{equation:线性空间.基变换方程组}
%@see: 《高等代数(第三版 下册)》(丘维声) P79 (2)
	\left\{ \begin{array}{l}
		\beta_1=a_{11} \alpha_1+a_{21} \alpha_2+\dotsb+a_{n1} \alpha_n, \\
		\beta_2=a_{12} \alpha_1+a_{22} \alpha_2+\dotsb+a_{n2} \alpha_n, \\
		\hdotsfor1 \\
		\beta_n=a_{1n} \alpha_1+a_{2n} \alpha_2+\dotsb+a_{nn} \alpha_n.
	\end{array} \right.
\end{equation}
为了使推导过程简洁,
我们引入一种形式写法\footnote{
	这里之所以称之为形式写法,
	是因为我们在先前学习的矩阵乘法运算的定义
	要求参与运算的每一个矩阵、向量的每一个元素都是数域\(K\)中的元素,
	但是这里看似向量的\((\AutoTuple{\alpha}{n}),(\AutoTuple{\beta}{n})\)中的“元素”
	并不是某个域中的元素,而是线性空间\(V\)中的元素.
}\begin{equation*}
%@see: 《高等代数(第三版 下册)》(丘维声) P79 (3)
	x_1 \alpha_1 + \dotsb + x_n \alpha_n
	\defeq
	(\AutoTuple{\alpha}{n})
	\begin{bmatrix}
		x_1 \\
		\vdots \\
		x_n
	\end{bmatrix}.
\end{equation*}
我们可以把\cref{equation:线性空间.基变换方程组} 写成\begin{equation*}
%@see: 《高等代数(第三版 下册)》(丘维声) P79 (5)
	(\AutoTuple{\beta}{n})
	=
	(\AutoTuple{\alpha}{n})
	A,
\end{equation*}
其中\begin{equation*}
	A=\begin{bmatrix}
		a_{11} & a_{12} & \dots & a_{1n} \\
		a_{21} & a_{22} & \dots & a_{2n} \\
		\vdots & \vdots & & \vdots \\
		a_{n1} & a_{n2} & \dots & a_{nn}
	\end{bmatrix}.
\end{equation*}
我们把\(A\)称为
“基\(\AutoTuple{\alpha}{n}\)到基\(\AutoTuple{\beta}{n}\)的\DefineConcept{过渡矩阵}”.
%@see: https://mathworld.wolfram.com/TransitionMatrix.html
%@see: https://mathworld.wolfram.com/ChangeofCoordinatesMatrix.html

像这样的形式写法,是模仿矩阵乘法的定义.
因此,类似于矩阵乘法的结合律、左右分配律、乘法与数量乘法的关系的证明方法,
可以证明形式写法满足以下规则.

设\(\AutoTuple{\alpha}{n}\)与\(\AutoTuple{\beta}{n}\)是\(V\)中的两个向量组,
\(A,B\)是域\(F\)上的两个\(n\)阶矩阵,
数\(k \in F\),
则\begin{gather*}
	%@see: 《高等代数(第三版 下册)》(丘维声) P79 (6)
	[(\AutoTuple{\alpha}{n}) A] B
	= (\AutoTuple{\alpha}{n}) (A B), \\
	%@see: 《高等代数(第三版 下册)》(丘维声) P79 (7)
	(\AutoTuple{\alpha}{n}) A
	+ (\AutoTuple{\alpha}{n}) B
	= (\AutoTuple{\alpha}{n}) (A + B), \\
	%@see: 《高等代数(第三版 下册)》(丘维声) P79 (8)
	(\AutoTuple{\alpha}{n}) A
	+ (\AutoTuple{\beta}{n}) A
	= (\alpha_1+\beta_1,\dotsc,\alpha_n+\beta_n) A, \\
	%@see: 《高等代数(第三版 下册)》(丘维声) P79 (9)
	[k (\AutoTuple{\alpha}{n})] A
	= (\AutoTuple{\alpha}{n}) (k A), \\
	[k (\AutoTuple{\alpha}{n})] A
	= k [(\AutoTuple{\alpha}{n}) A],
\end{gather*}
其中\begin{gather*}
	%@see: 《高等代数(第三版 下册)》(丘维声) P79 (10)
	(\AutoTuple{\alpha}{n})
	+ (\AutoTuple{\beta}{n})
	\defeq
	(\alpha_1+\beta_1,\dotsc,\alpha_n+\beta_n), \\
	%@see: 《高等代数(第三版 下册)》(丘维声) P79 (11)
	k (\AutoTuple{\alpha}{n})
	= (\AutoTuple{k \alpha}{n}).
\end{gather*}

\begin{proposition}\label{theorem:线性空间.命题14}
%@see: 《高等代数(第三版 下册)》(丘维声) P80 命题14
设\(\AutoTuple{\alpha}{n}\)是\(V\)的一个基,
且\((\AutoTuple{\beta}{n})=(\AutoTuple{\alpha}{n})A\),
则\(\AutoTuple{\beta}{n}\)是\(V\)的一个基
当且仅当\(A\)是可逆矩阵.
\begin{proof}
由于\(\AutoTuple{\alpha}{n}\)线性无关,
并且有\begin{align*}
	k_1 \beta_1+\dotsb+k_n \beta_n
	&=(\AutoTuple{\beta}{n}) (\AutoTuple{k}{n})^T \\
	&=(\AutoTuple{\alpha}{n}) A (\AutoTuple{k}{n})^T,
\end{align*}
因此\begin{align*}
	&\text{$\AutoTuple{\beta}{n}$是$V$的一个基}
	\iff \text{$\AutoTuple{\beta}{n}$线性无关} \\
	&\iff
	k_1 \beta_1+\dotsb+k_n \beta_n=0
	\implies
	k_1=\dotsb=k_n=0 \\
	&\iff
	(\AutoTuple{\alpha}{n}) A (\AutoTuple{k}{n})^T=0
	\implies
	(\AutoTuple{k}{n})^T=0 \\
	&\iff
	A (\AutoTuple{k}{n})^T=0
	\implies
	(\AutoTuple{k}{n})^T=0 \\
	&\iff \text{齐次线性方程组$AX=0$只有零解} \\
	&\iff \abs{A}\neq0
	\iff \text{$A$是可逆矩阵}.
	\qedhere
\end{align*}
\end{proof}
\end{proposition}

\cref{theorem:线性空间.命题14} 表明:
基\(\AutoTuple{\alpha}{n}\)到基\(\AutoTuple{\beta}{n}\)的过渡矩阵是可逆矩阵.

现在可以给出向量\(\alpha\)
分别在基\(\AutoTuple{\alpha}{n}\)
与基\(\AutoTuple{\beta}{n}\)下的坐标\(X,Y\)之间的关系.
由于\begin{equation*}
	\alpha
	=(\AutoTuple{\alpha}{n}) X
	=(\AutoTuple{\beta}{n}) Y,
\end{equation*}
并且基\(\AutoTuple{\alpha}{n}\)到基\(\AutoTuple{\beta}{n}\)的过渡矩阵是\(A\),
因此\begin{equation*}
	(\AutoTuple{\alpha}{n}) X
	=(\AutoTuple{\beta}{n}) Y
	=(\AutoTuple{\alpha}{n}) A Y.
\end{equation*}
由于同一个向量由基\(\AutoTuple{\alpha}{n}\)线性表出的方式唯一,
从上式得\begin{equation*}
%@see: 《高等代数(第三版 下册)》(丘维声) P79 (12)
	X=AY,
\end{equation*}
从而\begin{equation*}
%@see: 《高等代数(第三版 下册)》(丘维声) P79 (13)
	Y=A^{-1}X.
\end{equation*}

\begin{example}
设\(\alpha_1,\alpha_2,\alpha_3\)是\(\mathbb{R}^3\)的一组基,
求:基\(\alpha_1,\frac12\alpha_2,\frac13\alpha_3\)
到基\(\alpha_1+\alpha_2,\alpha_2+\alpha_3,\alpha_3+\alpha_1\)的过渡矩阵.
\begin{solution}
设所求过渡矩阵为\(P\),
则根据定义有\begin{equation*}
	\begin{bmatrix}
		\alpha_1 & \frac12\alpha_2 & \frac13\alpha_3
	\end{bmatrix} P
	= \begin{bmatrix}
		\alpha_1+\alpha_2 & \alpha_2+\alpha_3 & \alpha_3+\alpha_1
	\end{bmatrix},
\end{equation*}
即\begin{equation*}
	\begin{bmatrix}
		\alpha_1 & \alpha_2 & \alpha_3
	\end{bmatrix}
	\begin{bmatrix}
		1 \\
		& \frac12 \\
		&& \frac13
	\end{bmatrix} P
	= \begin{bmatrix}
	\alpha_1 & \alpha_2 & \alpha_3
	\end{bmatrix}
	\begin{bmatrix}
		1 & 0 & 1 \\
		1 & 1 & 0 \\
		0 & 1 & 1
	\end{bmatrix},
\end{equation*}
所以\begin{equation*}
	P = \begin{bmatrix}
		1 \\
		& \frac12 \\
		&& \frac13
	\end{bmatrix}^{-1}
	\begin{bmatrix}
		1 & 0 & 1 \\
		1 & 1 & 0 \\
		0 & 1 & 1
	\end{bmatrix}
	= \begin{bmatrix}
		1 \\
		& 2 \\
		&& 3
	\end{bmatrix} \begin{bmatrix}
		1 & 0 & 1 \\
		1 & 1 & 0 \\
		0 & 1 & 1
	\end{bmatrix}
	= \begin{bmatrix}
		1 & 0 & 1 \\
		2 & 2 & 0 \\
		0 & 3 & 3
	\end{bmatrix}.
\end{equation*}
\end{solution}
\end{example}

\begin{example}
在\(K[x]_3\)中取两个基\(
	\alpha_1
	= x^3 + 2x^2 - x,
	\allowbreak
	\alpha_2
	= x^3 - x^2 + x + 1,
	\allowbreak
	\alpha_3
	= -x^3 + 2x^2 + x + 1,
	\allowbreak
	\alpha_4
	= -x^3 - x^2 + 1
\)和\(
	\beta_1
	= 2x^3 + x^2 + 1,
	\allowbreak
	\beta_2
	= x^2 + 2x + 2,
	\allowbreak
	\beta_3
	= -2x^3 + x^2 + x + 2,
	\allowbreak
	\beta_4
	= x^3 + 3x^2 + x + 2
\).
求基\(\AutoTuple{\alpha}{4}\)到基\(\AutoTuple{\beta}{4}\)的过渡矩阵.
\begin{solution}
显然\begin{gather*}
	(\AutoTuple{\alpha}{4})
	= (x^3,x^2,x,1)
	\begin{bmatrix}
		1 & 1 & -1 & -1 \\
		2 & -1 & 2 & -1 \\
		0 & 1 & 1 & 0 \\
		-1 & 1 & 1 & 1
	\end{bmatrix}, \\
	(\AutoTuple{\beta}{4})
	= (x^3,x^2,x,1)
	\begin{bmatrix}
		2 & 0 & -2 & 1 \\
		1 & 1 & 1 & 3 \\
		0 & 2 & 1 & 1 \\
		1 & 2 & 2 & 2
	\end{bmatrix},
\end{gather*}
于是所求过渡矩阵为\begin{equation*}
	\begin{bmatrix}
		1 & 1 & -1 & -1 \\
		2 & -1 & 2 & -1 \\
		0 & 1 & 1 & 0 \\
		-1 & 1 & 1 & 1
	\end{bmatrix}^{-1}
	\begin{bmatrix}
		2 & 0 & -2 & 1 \\
		1 & 1 & 1 & 3 \\
		0 & 2 & 1 & 1 \\
		1 & 2 & 2 & 2
	\end{bmatrix}.
\end{equation*}
由于\begin{equation*}
	\begin{bmatrix}
		1 & 1 & -1 & -1 & 2 & 0 & -2 & 1 \\
		2 & -1 & 2 & -1 & 1 & 1 & 1 & 3 \\
		0 & 1 & 1 & 0 & 0 & 2 & 1 & 1 \\
		-1 & 1 & 1 & 1 & 1 & 2 & 2 & 2
	\end{bmatrix}
	\to \begin{bmatrix}
		1 & 0 & 0 & 0 & \frac{13}{2} & 0 & 0 & \frac{13}{2} \\
		0 & 1 & 0 & 0 & \frac{3}{2} & 1 & 0 & \frac{3}{2} \\
		0 & 0 & 1 & 0 & -\frac{3}{2} & 1 & 1 & -\frac{1}{2} \\
		0 & 0 & 0 & 1 & \frac{15}{2} & 0 & 1 & \frac{15}{2} \\
	\end{bmatrix},
\end{equation*}
因此所求过渡矩阵为\begin{equation*}
	\begin{bmatrix}
		1 & 1 & -1 & -1 \\
		2 & -1 & 2 & -1 \\
		0 & 1 & 1 & 0 \\
		-1 & 1 & 1 & 1
	\end{bmatrix}^{-1}
	\begin{bmatrix}
		2 & 0 & -2 & 1 \\
		1 & 1 & 1 & 3 \\
		0 & 2 & 1 & 1 \\
		1 & 2 & 2 & 2
	\end{bmatrix}
	= \begin{bmatrix}
		\frac{13}{2} & 0 & 0 & \frac{13}{2} \\
		\frac{3}{2} & 1 & 0 & \frac{3}{2} \\
		-\frac{3}{2} & 1 & 1 & -\frac{1}{2} \\
		\frac{15}{2} & 0 & 1 & \frac{15}{2} \\
	\end{bmatrix}.
\end{equation*}
%@Mathematica: A = {{1, 1, -1, -1}, {2, -1, 2, -1}, {0, 1, 1, 0}, {-1, 1, 1, 1}};
%@Mathematica: B = {{2, 0, -2, 1}, {1, 1, 1, 3}, {0, 2, 1, 1}, {1, 2, 2, 2}};
%@Mathematica: Inverse[A].B // MatrixForm
%@Mathematica: Join[A, B, 2] // MatrixForm
%@Mathematica: Join[A, B, 2] // RowReduce // MatrixForm
%@Mathematica: RowReduce[Join[A, B, 2]][[All, 5 ;; 8]] // MatrixForm
\end{solution}
\end{example}

\begin{example}
%@see: 《高等代数(第三版 下册)》(丘维声) P81 习题8.1 4.
把复数域\(\mathbb{C}\)看成实数域\(\mathbb{R}\)上的线性空间,
求它的一个基和维数,
以及每个复数在这个基下的坐标.
\begin{solution}
把复数域看成实数域上的线性空间\(V\),
容易看出,有限集\(S = \{1,\iu\}\)是线性空间\(V\)的一个基,
它的维数为\(\dim V = \card S = 2\),
而每个复数\(z = a + b\iu\)在这个基下的坐标为\((a,b)^T\).
\end{solution}
\end{example}

\begin{example}
%@see: 《高等代数(第三版 下册)》(丘维声) P81 习题8.1 5.
把数域\(K\)看成自身上的线性空间,求它的一个基和维数.
\begin{solution}
把数域\(K\)看成自身上的线性空间\(V\),
容易看出,\(S = \{1\}\)是线性空间\(V\)的一个基,
它的维数为\(\dim V = \card S = 1\).
\end{solution}
\end{example}

\begin{example}
%@see: 《高等代数(第三版 下册)》(丘维声) P81 习题8.1 11.
设\(X = \{\AutoTuple{x}{n}\}\),\(F\)是一个域.
把映射空间\(F^X\)看成域\(F\)上的一个线性空间,
求\(F^X\)的一个基和维数,
再求映射\(f \in F^X\)在这个基下的坐标.
\begin{solution}
任意给定\(f \in F^X\),
必有\(f = \Set{
	(x_1,f(x_1)),
	\dotsc,
	(x_n,f(x_n))
}\).

令\begin{equation*}
	f_i(x_j) \defeq \delta(i,j),
	\quad i,j=1,2,\dotsc,n,
\end{equation*}
其中\(\delta\)是克罗内克\(\delta\)函数,
即\begin{equation*}
	\delta(i,j) = \left\{ \begin{array}{cl}
		1, & i = j, \\
		0, & i \neq j.
	\end{array} \right.
\end{equation*}
那么\begin{equation*}
	f(x) = f(x_1) f_1(x) + \dotsb + f(x_n) f_n(x),
	\quad x \in X,
	\eqno(1)
\end{equation*}
这就说明,\(f\)可以由\(\AutoTuple{f}{n}\)线性表出.

显然\(\AutoTuple{f}{n}\)线性无关,
那么\(\AutoTuple{f}{n}\)是\(F^X\)的一个基,
从而有\(\dim F^X = n\).

由(1)式可知,
函数\(f\)在基\(\AutoTuple{f}{n}\)下的坐标为
\((f(x_1),\dotsc,f(x_n))\).
\end{solution}
\end{example}

\subsection{线性空间的笛卡尔和}
\begin{definition}
%@see: 《Linear Algebra and Its Applications (Second Edition)》(Peter D. Lax) P10 Definition
%@see: 《Linear Algebra Done Right (Fourth Edition)》(Sheldon Axler) P96 3.87
设\(V,W\)都是域\(F\)上的线性空间.
把定义了加法\begin{equation*}
	(v_1,w_1) + (v_2,w_2) \defeq (v_1+v_2,w_1+w_2)
\end{equation*}
和纯量乘法\begin{equation*}
	k (v_1,w_1) \defeq (k v_1,k w_1)
\end{equation*}
的集合\begin{equation*}
	\Set{
		(v,w)
		\given
		v \in V,
		w \in W
	}
\end{equation*}
称为“线性空间\(V\)和\(W\)的\DefineConcept{笛卡尔和}(Cartesian sum)”,
记作\(V \CartesianSum W\).
\end{definition}

\begin{theorem}\label{theorem:线性空间.线性空间的笛卡尔和是线性空间}
%@see: 《Linear Algebra and Its Applications (Second Edition)》(Peter D. Lax) P10
%@see: 《Linear Algebra Done Right (Fourth Edition)》(Sheldon Axler) P96 3.89
设\(V,W\)都是域\(F\)上的线性空间,
则线性空间\(V\)和\(W\)的笛卡尔和\(V \CartesianSum W\)是域\(F\)上的线性空间.
\begin{proof}
由于\(V,W\)都是线性空间,
所以由定义有:
对于\(\forall v_1,v_2,v_3 \in V,
\forall w_1,w_2,w_3 \in W,
\forall k,l \in F\),
有\begin{gather*}
	(v_1,w_1) + (v_2,w_2)
	= (v_1+v_2,w_1+w_2)
	= (v_2+v_1,w_2+w_1)
	= (v_2,w_2)+(v_1,w_1), \\
	\begin{aligned}
		((v_1,w_1) + (v_2,w_2)) + (v_3,w_3)
		&= (v_1+v_2,w_1+w_2) + (v_3,w_3) \\
		&= ((v_1+v_2)+v_3,(w_1+w_2)+w_3) \\
		&= (v_1+(v_2+v_3),w_1+(w_2+w_3)) \\
		&= (v_1,w_1) + (v_2+v_3,w_2+w_3) \\
		&= (v_1,w_1) + ((v_2,w_2) + (v_3,w_3)),
	\end{aligned} \\
	(0,0) \in V \times W
	\land
	(v_1,w_1) + (0,0)
	= (v_1+0,w_1+0)
	= (v_1,w_1), \\
	(v_1,w_1) + (-v_1,-w_1)
	= (v_1+(-v_1),w_1+(-w_1))
	= (0,0), \\
	1(v_1,w_1)
	= (1v_1,1w_1)
	= (v_1,w_1), \\
	k(l(v_1,w_1))
	= k(lv_1,lw_1)
	= (k(lv_1),k(lw_1))
	= ((kl)v_1,(kl)w_1)
	= (kl)(v_1,w_1), \\
	\begin{aligned}
		(k+l)(v_1,w_1)
		&= ((k+l)v_1,(k+l)w_1) \\
		&= (kv_1+lv_1,kw_1+lw_1) \\
		&= (kv_1,kw_1) + (lv_1,lw_1) \\
		&= k(v_1,w_1) + l(v_1,w_1),
	\end{aligned} \\
	\begin{aligned}
		k((v_1,w_1)+(v_2,w_2))
		&= k(v_1+v_2,w_1+w_2) \\
		&= (k(v_1+v_2),k(w_1+w_2)) \\
		&= (kv_1+kv_2,kw_1+kw_2) \\
		&= (kv_1,kw_1)+(kv_2,kw_2) \\
		&= k(v_1,w_1)+k(v_2,w_2).
	\end{aligned}
\end{gather*}
这就说明\(V \times W\)是域\(F\)上的一个线性空间.
\end{proof}
\end{theorem}

\begin{theorem}\label{theorem:线性空间.笛卡尔和的维数公式}
%@see: 《Linear Algebra and Its Applications (Second Edition)》(Peter D. Lax) P11 Exercise 18.
%@see: 《Linear Algebra Done Right (Fourth Edition)》(Sheldon Axler) P97 3.92
设\(V,W\)都是域\(F\)上的线性空间,
则线性空间\(V\)和\(W\)的笛卡尔和\(V \CartesianSum W\)满足\begin{equation*}
	\dim(V \CartesianSum W) = \dim V + \dim W.
\end{equation*}
%TODO proof
\end{theorem}

\section{子空间及其运算}
数域\(K\)上\(n\)元齐次线性方程组\(A X = 0\)的解空间\(W\)是\(K^n\)的子空间,
意思是齐次线性方程组的解集\(W\)对于有序数组的加法与数量乘法封闭.

数域\(K\)上全体\(n\)阶矩阵\(M_n(K)\)对于矩阵的加法与数量乘法形成\(K\)上的一个线性空间.
\(K\)上全体\(n\)阶对称矩阵\(V\)对于矩阵的加法与数量乘法也形成一个线性空间.
显然\(V\)是\(M_n(K)\)的子集,
并且\(V\)的加法就是\(M_n(K)\)的加法,
\(V\)的数量乘法就是\(M_n(K)\)的数量乘法.
我们自然地希望把\(V\)叫做\(M_n(K)\)的一个子空间.

本节将介绍任意线性空间的子空间的概念、子空间的运算,以及研究如何利用子空间来探索线性空间的结构.

\subsection{子空间}
\begin{definition}
%@see: 《高等代数(第三版 下册)》(丘维声) P82 定义1
%@see: 《Linear Algebra and Its Applications (Second Edition)》(Peter D. Lax) P4 Definition
设\(V\)是域\(F\)上的一个线性空间,
\(\emptyset\neq U\subseteq V\).
如果\(U\)对于\(V\)的加法及纯量乘法运算
也形成\(F\)上的线性空间,
则称“\(U\)是\(V\)的一个\DefineConcept{子空间}(subspace)”,
记作\(U \AlgebraSubstructure V\).
% A subset \(U\) of a linear space \(V\) is called a \emph{subspace} if sums and scalar multiples of elements of \(U\) belong to \(U\).
\end{definition}

%@see: 《Linear Algebra and Its Applications (Second Edition)》(Peter D. Lax) P4 Exercise 8.
显然\(\{0\}\)是\(V\)的一个子空间,
%@see: 《Linear Algebra Done Right (Fourth Eidition) P19
% The set \(\{0\}\) is the smallest subspace of V.
同时它也是\(V\)的最小子空间,
称其为“\(V\)的\DefineConcept{零子空间}”,
也记作\(0\).
另外,\(V\)显然也是\(V\)的一个子空间.
%@see: 《Linear Algebra Done Right (Fourth Eidition) P19
% \(V\) itself is the largest subspace of V.
同时它也是\(V\)的最大子空间,
我们把\(0\)和\(V\)统称为“\(V\)的\DefineConcept{平凡子空间}(trivial subspace)\footnote{
	有的书仅仅将\emph{平凡子空间}定义为零子空间.
}”,
把\(V\)的其余子空间称为它的\DefineConcept{非平凡子空间}.

空集\(\emptyset\)不是\(V\)的子空间.

\begin{theorem}\label{theorem:线性空间.子空间的判定}
%@see: 《高等代数(第三版 下册)》(丘维声) P82 定理1
域\(F\)上线性空间\(V\)的非空子集\(U\)是\(V\)的一个子空间
当且仅当\(U\)对于\(V\)的加法与纯量乘法都封闭,
即\begin{itemize}
	\item \((\forall u_1,u_2\in U)[u_1+u_2 \in U]\);
	\item \((\forall u\in U)(\forall k\in F)[ku\in U]\).
\end{itemize}
\begin{proof}
必要性已由子空间的定义直接给出.
下面证明充分性.

由已知条件可知,\(V\)的加法与纯量乘法都是\(U\)的运算.
由于\(V\)是线性空间,
因此\(U\)的加法满足\hyperref[definition:线性空间.运算法则1]{交换律}、\hyperref[definition:线性空间.运算法则2]{结合律},
\(U\)的纯量乘法满足 \labelcref{definition:线性空间.运算法则5,definition:线性空间.运算法则6,definition:线性空间.运算法则7,definition:线性空间.运算法则8} 这4条运算法则.

% \cref{definition:线性空间.运算法则3}
由于\(U\)是非空集,
因此存在\(u \in U\).
由已知条件得\(0 u \in U\).
由于\(V\)是线性空间,因此\(0 u = 0\).
从而\(0 \in U\),
于是\(V\)的零元也是\(U\)的零元.

% \cref{definition:线性空间.运算法则4}
任取\(\alpha \in U\),
由已知条件得\((-1) \alpha \in U\).
由于\(V\)是线性空间,
因此\((-1) \alpha = -\alpha\),
从而\(-\alpha \in U\),
于是\(\alpha\)在\(V\)中的负元\(-\alpha\)也是\(\alpha\)在\(U\)中的负元.

综上所述,
\(U\)的加法、纯量乘法满足线性空间定义所要求的全部运算法则,
\(U\)是\(F\)上一个线性空间,
从而\(U\)是\(V\)的一个子空间.
\end{proof}
\end{theorem}

\begin{example}
%@see: 《高等代数(第三版 下册)》(丘维声) P83 例1
数域\(K\)上所有次数小于\(n\)的一元多项式组成的集合\(K[x]_n\)
是\(K[x]\)的一个子空间.
\begin{proof}
显然\(K[x]_n\)非空集.
由于两个次数小于\(n\)的一元多项式之和的次数仍小于\(n\),
且任一数\(k\)与一个次数小于\(n\)的一元多项式的乘积的次数仍小于\(n\),
因此\(K[x]_n\)对于多项式的加法与数量乘法都封闭,
从而\(K[x]_n\)是\(K[x]\)的一个子空间.
\end{proof}
\end{example}

\begin{proposition}
%@see: 《高等代数(第三版 下册)》(丘维声) P83 命题2
设\(U\)是域\(F\)上\(n\)维线性空间\(V\)的一个子空间,
则\(\dim U\leq\dim V\).
\begin{proof}
由于\(n\)维线性空间\(V\)中任意\(n+1\)个向量都线性相关,
因此\(U\)的一个基所含向量的个数一定小于或等于\(n\),
从而\(\dim U\leq\dim V\).
\end{proof}
\end{proposition}

\begin{proposition}
%@see: 《高等代数(第三版 下册)》(丘维声) P83 命题3
设\(U\)是域\(F\)上\(n\)维线性空间\(V\)的一个子空间.
如果\(\dim U=\dim V\),
则\(U=V\).
\begin{proof}
由于\(\dim U=\dim V=n\),
因此\(U\)的一个基\(\AutoTuple{\delta}{n}\)就是\(V\)的一个基,
从而\(V\)中任一向量\(\alpha=a_1\delta_1+\dotsb+a_n\delta_n\in U\),
因此\(V\subseteq U\).
又因为\(U\subseteq V\),
所以\(U=V\).
\end{proof}
\end{proposition}

\begin{proposition}
%@see: 《高等代数(第三版 下册)》(丘维声) P83 命题4
设\(U\)是域\(F\)上\(n\)维线性空间\(V\)的一个子空间,
则\(U\)的一个基可以扩充成\(V\)的一个基.
\begin{proof}
设\(\AutoTuple{\alpha}{s}\)是\(U\)的一个基,则\(s\leq n\).
如果\(s=n\),则\(\AutoTuple{\alpha}{n}\)是\(V\)的一个基.
下面设\(s<n\).
此时\(\AutoTuple{\alpha}{s}\)不是\(V\)的一个基,
于是\(V\)中至少有一个向量\(\beta_1\)
不能由\(\AutoTuple{\alpha}{s}\)线性表出,
从而\(\AutoTuple{\alpha}{s},\beta_1\)线性无关.
如果\(s+1=n\),
则已得到\(V\)的一个基.
如果\(s+1<n\),
则同理有\(\beta_2\in V\),
使得\(\AutoTuple{\alpha}{s},\beta_1,\beta_2\)线性无关.
依次递推,总能得到\(n\)个线性无关的向量
\(\AutoTuple{\alpha}{s},\AutoTuple{\beta}{r}\),
其中\(s+r=n\),
这就是\(V\)的一个基.
\end{proof}
\end{proposition}

\subsection{生成子空间}
如何构造域\(F\)上线性空间\(V\)的子空间?
在\(V\)中给了向量组\(\AutoTuple{\alpha}{s}\),
由它们的所有线性组合组成的集合\[
	U
	\defeq
	\Set{
		k_1\alpha_1+\dotsb+k_s\alpha_s
		\given
		\AutoTuple{k}{s}\in F
	}
\]是\(V\)的一个子空间,
称“\(U\)是由向量组\(\AutoTuple{\alpha}{s}\)生成的子空间
%@see: 《Linear Algebra and Its Applications (Second Edition)》(Peter D. Lax) P4 Exercise 9.
(\(U\) is the subspace \emph{spanned by} \(\AutoTuple{\alpha}{s}\))”
“向量组\(\AutoTuple{\alpha}{s}\)可以生成空间\(U\)
(\(\AutoTuple{\alpha}{s}\) \emph{span} \(U\))”,
记作\(\opair{\AutoTuple{\alpha}{s}}\)
或\(\Span\{\AutoTuple{\alpha}{s}\}\).

可以证明,由\(\AutoTuple{\alpha}{s}\)生成的子空间\(\opair{\AutoTuple{\alpha}{s}}\)
是包含\(\AutoTuple{\alpha}{s}\)的最小子空间.

\begin{theorem}
%@see: 《高等代数(第三版 下册)》(丘维声) P84 定理5
在域\(F\)上的线性空间\(V\)中,
如果\(U=\opair{\AutoTuple{\alpha}{s}}\),
则向量组\(\AutoTuple{\alpha}{s}\)的一个极大线性无关组是\(U\)的一个基,
从而\(\dim U=\rank\{\AutoTuple{\alpha}{s}\}\).
\end{theorem}

从基的定义容易看出,
如果\(\AutoTuple{\delta}{r}\)是\(V\)的子空间\(U\)的一个基,
则\(U=\opair{\AutoTuple{\delta}{r}}\).
由此看出,\(V\)的任一有限维子空间都是由向量组生成的子空间.

\begin{theorem}
%@see: 《Linear Algebra and Its Applications (Second Edition)》(Peter D. Lax) P6 Theorem 4.
域\(F\)上有限维线性空间\(V\)中的每一个线性无关向量组可以扩充为\(V\)的基.
\begin{proof}
设\(\AutoTuple{\alpha}{s}\)是域\(F\)上有限维线性空间\(V\)中的一个线性无关向量组.
如果\(\AutoTuple{\alpha}{s}\)不可以生成空间\(V\),
那么存在向量\(\beta_1\),不可以由\(\AutoTuple{\alpha}{s}\)线性表出.
记\(\alpha_{s+1} = \beta_1\).
如果\(\AutoTuple{\alpha}{s+1}\)不可以生成空间\(V\),
那么存在向量\(\beta_2\),不可以由\(\AutoTuple{\alpha}{s+1}\)线性表出.
记\(\alpha_{s+2} = \beta_2\).
循此往复,最终必定有\(\AutoTuple{\alpha}{n}\)可以生成空间\(V\),
其中\(n = \dim V\).
\end{proof}
\end{theorem}

\begin{theorem}
%@see: 《Linear Algebra and Its Applications (Second Edition)》(Peter D. Lax) P6 Theorem 5.(a)
有限维线性空间的任意一个子空间都是有限维的.
%TODO proof
\end{theorem}

\subsection{子空间的交}
在几何空间\(V\)中,给定两个过原点的平面\(V_1,V_2\),
易知它们都是\(V\)的子空间,
它们的交线\(L = V_1 \cap V_2\)也是\(V\)的一个子空间.
于是我们大胆推测,在任意一个线性空间中,任意两个子空间的交集也是一个子空间.
\begin{theorem}
%@see: 《高等代数(第三版 下册)》(丘维声) P84 定理6
%@see: 《Linear Algebra and Its Applications (Second Edition)》(Peter D. Lax) P4 Definition
设\(V_1,V_2\)都是域\(F\)上线性空间\(V\)的子空间,
则\(V_1 \cap V_2\)也是\(V\)的子空间.
\begin{proof}
因为\(0\in V_1 \cap V_2\),
所以\(V_1 \cap v_2\)非空集.
设\(\alpha,\beta\in V_1 \cap V_2\),
则\(\alpha,\beta\in V_1\)且\(\alpha,\beta\in V_2\).
于是\(\alpha+\beta\in V_1\)且\(\alpha+\beta\in V_2\),
因此\(\alpha+\beta\in V_1 \cap V_2\),
\(V_1 \cap V_2\)对加法封闭.
同理可证\(V_1 \cap V_2\)对纯量乘法封闭.
综上所述\(V_1 \cap V_2\)是\(V\)的子空间.
\end{proof}
\end{theorem}

子空间的交适合交换律、结合律,
即\[
	V_1 \cap V_2
	=V_2 \cap V_1, \qquad
	(V_1 \cap V_2) \cap V_3
	=V_1 \cap (V_2 \cap V_3).
\]
由结合律,我们知道\(V\)的若干个子空间的交
\(\bigcap_{i=1}^s V_i\)也是\(V\)的一个子空间.

子空间的交满足\[
	W \AlgebraSubstructure V
	\implies
	W \cap V = W.
\]
%@credit: {DeepSeek}
这是因为\(W\)是\(V\)的子空间,从而有\(W \subseteq V\),
于是由\cref{equation:集合论.集合代数公式7-3} 可知\(W \cap V = W\).

\subsection{子空间的和}
现在我们想知道,在任意一个线性空间中,任意两个子空间的并集是不是子空间.
从上述几何空间的例子看出,
如果向量\(\alpha_1\)属于子空间\(V_1\),
向量\(\alpha_2\)属于子空间\(V_2\),
且两个向量都不属于交线\(L\),
则虽然\(\alpha_1,\alpha_2 \in V_1 \cup V_2\),
但是\(\alpha_1 + \alpha_2\)可能不属于\(V_1 \cup V_2\),
或者说\(V_1 \cup V_2\)可能对加法不封闭,
因此\(V_1 \cup V_2\)不是\(V\)的子空间.
下面我们给出严格证明.
\begin{proposition}
设\(V_1,V_2\)都是域\(F\)上线性空间\(V\)的子空间,
则\(V_1 \cup V_2\)不一定是\(V\)的子空间.
\begin{proof}
显然\(V_1=\opair{(1,0,0),(0,1,0)}\)
和\(V_2=\opair{(0,1,0),(0,0,1)}\)
都是几何空间\(V\)的子空间.
我们取\(\alpha=(1,1,0)\in V_1\),
再取\(\beta=(0,1,1)\in V_2\),
容易看出\(\alpha+\beta=(1,2,1)\)
虽然属于\(V\),
但是不属于\(\opair{(1,0,0),(0,1,0)}\),
也不属于\(\opair{(0,1,0),(0,0,1)}\),
即\(\alpha+\beta\notin V_1 \cup V_2\),
这就说明
\(V_1 \cup V_2\)对加法不封闭,
从而说明
\(V_1 \cup V_2\)不是\(V\)的子空间.
\end{proof}
\end{proposition}
那么,如果我们想构造一个包含\(V_1 \cup V_2\)的子空间,
这个子空间就应当含有\(V_1\)中任一向量\(\alpha_1\)与\(V_2\)中任一向量\(\alpha_2\)之和.
受此启发,我们应当考虑集合
\(\Set{ \alpha_1 + \alpha_2 \given \alpha_1 \in V_1,\alpha_2 \in V_2 }\).

\begin{definition}
%@see: 《高等代数(第三版 下册)》(丘维声) P84 定理7
%@see: 《Linear Algebra and Its Applications (Second Edition)》(Peter D. Lax) P4 Definition
设\(V_1,V_2\)都是域\(F\)上线性空间\(V\)的子空间,
把\[
%@see: 《高等代数(第三版 下册)》(丘维声) P84 (1)
	\Set{ \alpha_1+\alpha_2 \given \alpha_1\in V_1,\alpha_2\in V_2 }
\]称为“\(V_1\)与\(V_2\)的\DefineConcept{和}(sum)”,
记作\(V_1+V_2\).
\end{definition}
\begin{theorem}
%@see: 《高等代数(第三版 下册)》(丘维声) P84 定理7
设\(V_1,V_2\)都是域\(F\)上线性空间\(V\)的子空间,
则\(V_1+V_2\)是\(V\)的一个子空间.
\begin{proof}
由于\(0+0=0\),
所以\(0\in V_1+V_2\).
在\(V_1+V_2\)中任取两个向量\(\alpha,\beta\),
则\[
	\alpha=\alpha_1+\alpha_2, \qquad
	\beta=\beta_1+\beta_2,
\]
其中\(\alpha_1,\beta_1\in V_1,
\alpha_2,\beta_2\in V_2\).
于是\(\alpha_1+\beta_1\in V_1,
\alpha_2+\beta_2\in V_2\).
因此\[
	\alpha+\beta
	=(\alpha_1+\alpha_2)+(\beta_1+\beta_2)
	=(\alpha_1+\beta_1)+(\alpha_2+\beta_2)
	\in V_1+V_2,
\]
即\(V_1+V_2\)对于\(V\)的加法封闭.
同理可证\(V_1+V_2\)对于\(V\)的纯量乘法封闭,
因此\(V_1+V_2\)是\(V\)的一个子空间.
\end{proof}
\end{theorem}

\begin{proposition}
%@see: 《高等代数(第三版 下册)》(丘维声) P85
%@see: 《Linear Algebra Done Right (Fourth Eidition) P21 1.40
% \(V_1+V_2\) is the smallest subspace of \(V\) containing \(V_1,V_2\).
\(V_1+V_2\)是\(V\)中包含\(V_1 \cup V_2\)的最小子空间.
\begin{proof}
%@credit: {855486ab-2fcf-40c1-b774-09956dfb4012},{9fe3a491-385e-41e9-97cc-e0e3bc5c3b5b},{3d3dd0d4-9945-42ae-8a73-520c7170c8ac},{0e766461-e697-4ed8-95dd-e744fdd3194f},{a84f055e-f32d-418a-8d8c-0b72a4b2df78}
设\(U\)是\(V\)的子空间,
且\(U \supseteq V_1 \cup V_2\),
那么\[
	(\forall \alpha_1 \in V_1)
	(\forall \alpha_2 \in V_2)
	[
		\alpha_1 \in U
		\land
		\alpha_2 \in U
	].
	\eqno(1)
\]
因为\(U\)是子空间,对加法封闭,
所以由(1)可得\[
	(\forall \alpha_1 \in V_1)
	(\forall \alpha_2 \in V_2)
	[
		\alpha_1 + \alpha_2 \in U
	].
	\eqno(2)
\]
根据定义有\[
	(\forall \alpha \in V_1+V_2)
	(\exists \alpha_1 \in V_1)
	(\exists \alpha_2 \in V_2)
	[
		\alpha = \alpha_1 + \alpha_2
	].
	\eqno(3)
\]
由(2)(3)可知\[
	(\forall \alpha \in V_1+V_2)
	[
		\alpha \in U
	],
\]
则\(U \supseteq V_1+V_2\).
\end{proof}
\end{proposition}

%@see: 《高等代数(第三版 下册)》(丘维声) P85
子空间的和适合交换律、结合律,
即\[
	V_1 + V_2
	=V_2 + V_1, \qquad
	(V_1 + V_2) + V_3
	=V_1 + (V_2 + V_3).
\]
由结合律,我们知道\(V\)的有限个子空间之和\[
%@see: 《高等代数(第三版 下册)》(丘维声) P85 (2)
	\sum_{i=1}^s V_i
	\defeq
	V_1 + \dotsb + V_s
	= \Set{
		\alpha_1 + \dotsb + \alpha_s
		\given
		\alpha_i \in V_i,
		i=1,2,\dotsc,s
	}
\]也是\(V\)的一个子空间.

子空间的和满足\[
	W \AlgebraSubstructure V
	\implies
	W + V = V.
\]
%@credit: {DeepSeek}
这是因为\begin{itemize}
	\item 对于任意\(w \in W\)和\(v \in V\),
	必定有\(w + v \in W + V\);
	由于\(W\)是\(V\)的子空间,
	所以\(w \in V\);
	由于\(V\)对加法封闭,
	所以\(w + v \in V\);
	因此\(W + V \subseteq V\);

	\item 对于任意\(v \in V\),
	只要取\(w = 0 \in W\),
	就有\(v = 0 + v \in W + V\);
	因此\(V \subseteq W + V\);
\end{itemize}
综上所述\(W + V = V\).

\begin{proposition}
%@see: 《高等代数(第三版 下册)》(丘维声) P85 命题8
设\(\AutoTuple{\alpha}{s}\)与\(\AutoTuple{\beta}{r}\)
是域\(F\)上线性空间\(V\)的两个向量组,
则\[
	\opair{\AutoTuple{\alpha}{s}}
	+\opair{\AutoTuple{\beta}{r}}
	=\opair{\AutoTuple{\alpha}{s},\AutoTuple{\beta}{r}}.
\]
\begin{proof}
根据向量组生成的子空间的定义,以及子空间的和的定义,
得到\begin{align*}
	&\opair{\AutoTuple{\alpha}{s}}
	+\opair{\AutoTuple{\beta}{r}} \\
	&=\Set{
		(k_1\alpha_1+\dotsb+k_s\alpha_s)
		+(l_1\beta_1+\dotsb+l_r\beta_r)
		\given
		k_i,l_j\in F,
		1\leq i\leq s,
		1\leq j\leq r
	} \\
	&=\opair{\AutoTuple{\alpha}{s},\AutoTuple{\beta}{r}}.
	\qedhere
\end{align*}
\end{proof}
\end{proposition}

\begin{example}
试讨论:子空间的和、交是否满足分配律.
\begin{solution}
设\(V\)是域\(F\)上的线性空间,
\(V_1,V_2,V_3\)是\(V\)的子空间.

考虑命题\begin{gather*}
	(V_1 + V_2) \cap V_3
	= (V_1 \cap V_3) + (V_2 \cap V_3),
	\tag1 \\
	(V_1 \cap V_2) + V_3
	= (V_1 + V_3) \cap (V_2 + V_3).
	\tag2
\end{gather*}
%@credit: {腾讯元宝}
取\[
	% 坐标平面
	V = \mathbb{R}^2,
	\qquad
	% x轴
	V_1 = \Span\{(1,0)\},  % = \Set{ (x,0) \given x\in\mathbb{R} },
	\qquad
	% y轴
	V_2 = \Span\{(0,1)\},  % = \Set{ (0,y) \given y\in\mathbb{R} },
	\qquad
	% 直线\(y=x\)
	V_3 = \Span\{(1,1)\},  % = \Set{ (x,x) \given x\in\mathbb{R} },
\]
则\begin{gather*}
	V_1 + V_2
	= V_1 + V_3
	= V_2 + V_3
	= V, \\
	V_1 \cap V_2
	= V_1 \cap V_3
	= V_2 \cap V_3
	= \{ (0,0) \}
	= 0, \\
	% (1)式等号左边
	(V_1 + V_2) \cap V_3
	= V \cap V_3
	= V_3, \\
	% (1)式等号右边
	(V_1 \cap V_3) + (V_2 \cap V_3)
	= 0 + 0
	= 0, \\
	% (2)式等号左边
	(V_1 \cap V_2) + V_3
	= 0 + V_3
	= V_3, \\
	% (2)式等号右边
	(V_1 + V_3) \cap (V_2 + V_3)
	= V \cap V
	= V,
\end{gather*}
显然(1)(2)式均不成立,
子空间的和、交不满足分配律!
\end{solution}
\end{example}
\begin{example}
设\(V\)是域\(F\)上的线性空间,
\(V_1,V_2,V_3\)是\(V\)的子空间.
证明:\begin{equation}
	(V_1 \cap V_3) + (V_2 \cap V_3) \subseteq (V_1 + V_2) \cap V_3.
\end{equation}
%TODO 有没有取等条件?
% 当\(V_1 \subseteq V_3\)或\(V_2 \subseteq V_3\)时,取“\(=\)”号.
\begin{proof}
%@credit: {腾讯元宝}
任取\(v \in (V_1 \cap V_3) + (V_2 \cap V_3)\),
必定存在\(v_1 \in V_1 \cap V_3\)和\(v_2 \in V_2 \cap V_3\),
使得\(v = v_1 + v_2\).

由\(v_1 \in V_1 \cap V_3\)可知\(v_1 \in V_1\),
由\(v_2 \in V_2 \cap V_3\)可知\(v_2 \in V_2\),
于是\(v = v_1 + v_2 \in V_1 + V_2\).

%@credit: {6c964576-9569-472e-969e-54699e35974b}
由\(v_1 \in V_1 \cap V_3\)可知\(v_1 \in V_3\),
由\(v_2 \in V_2 \cap V_3\)可知\(v_2 \in V_3\),
于是\(v = v_1 + v_2 \in V_3\).

由\(v \in V_1 + V_2\)
和\(v \in V_3\)
可知\(v \in (V_1 + V_2) \cap V_3\).

综上所述\[
	v \in (V_1 \cap V_3) + (V_2 \cap V_3)
	\implies
	v \in (V_1 + V_2) \cap V_3,
\]
即\((V_1 \cap V_3) + (V_2 \cap V_3) \subseteq (V_1 + V_2) \cap V_3\).
\end{proof}
\end{example}
\begin{example}
设\(V\)是域\(F\)上的线性空间,
\(V_1,V_2,V_3\)是\(V\)的子空间.
证明:\begin{equation}
	(V_1 \cap V_2) + V_3 \subseteq (V_1 + V_3) \cap (V_2 + V_3).
\end{equation}
%TODO 有没有取等条件?
% 当\(V_3 \subseteq V_1\)或\(V_3 \subseteq V_2\)时,取“\(=\)”号.
\begin{proof}
%@credit: {腾讯元宝}
任取\(v \in (V_1 \cap V_2) + V_3\),
必定存在\(v_1 \in V_1 \cap V_2\)和\(v_2 \in V_3\),
使得\(v = v_1 + v_2\).

由\(v_1 \in V_1 \cap V_2\)
可知\(v_1 \in V_1\)且\(v_1 \in V_2\).

由\(v_1 \in V_1\)和\(v_2 \in V_3\)
可知\(v = v_1 + v_2 \in V_1 + V_3\).

由\(v_1 \in V_2\)和\(v_2 \in V_3\)
可知\(v = v_1 + v_2 \in V_2 + V_3\).

由\(v \in V_1 + V_3\)
和\(v \in V_2 + V_3\)
可知\(v \in (V_1 + V_3) \cap (V_2 + V_3)\).

综上所述\[
	v \in (V_1 \cap V_2) + V_3
	\implies
	v \in (V_1 + V_3) \cap (V_2 + V_3),
\]
即\((V_1 \cap V_2) + V_3 \subseteq (V_1 + V_3) \cap (V_2 + V_3)\).
\end{proof}
\end{example}

\subsection{子空间的维数公式}
在几何空间\(V\)中,两个相交平面\(V_1,V_2\)的维数为\(\dim V_1 = \dim V_2 = 2\),
它们的交线的维数为\(\dim(V_1 \cap V_2) = 1\),
它们的和恰好就是\(V\),从而有\(\dim(V_1 + V_2) = 3\),
于是成立\[
	\dim V_1 + \dim V_2
	= \dim(V_1 + V_2) + \dim(V_1 \cap V_2).
\]
这就让人不禁好奇,对于任意一个\(n\)维线性空间,上式是否成立?
回答是肯定的.
\begin{theorem}[子空间的维数公式]\label{theorem:线性空间.子空间.子空间的维数公式}
%@see: 《高等代数(第三版 下册)》(丘维声) P85 定理9
%@see: 《Linear Algebra and Its Applications (Second Edition)》(Peter D. Lax) P10 Theorem 7.
设\(V_1,V_2\)都是域\(F\)上线性空间\(V\)的有限维子空间,
则\(V_1 \cap V_2,V_1+V_2\)也都是有限维的子空间,
并且\[
%@see: 《高等代数(第三版 下册)》(丘维声) P85 (3)
	\dim V_1+\dim V_2
	=\dim(V_1+V_2)
	+\dim(V_1 \cap V_2).
\]
\begin{proof}
由于\(V_1 \cap V_2 \subseteq V_1\),
因此\(\dim(V_1 \cap V_2) \leq \dim V_1\).
设\[
	\dim V_1=n_1, \qquad
	\dim V_2=n_2, \qquad
	\dim(V_1 \cap V_2)=m.
\]
在\(V_1 \cap V_2\)中取一个基\(\AutoTuple{\alpha}{m}\),
把它分别扩充成\(V_1,V_2\)的一个基:\[
	\AutoTuple{\alpha}{m},\AutoTuple{\beta}{n_1-m}, \qquad
	\AutoTuple{\alpha}{m},\AutoTuple{\gamma}{n_2-m},
\]
于是\begin{align*}
	V_1+V_2
	&=\opair{\AutoTuple{\alpha}{m},\AutoTuple{\beta}{n_1-m}}
	+\opair{\AutoTuple{\alpha}{m},\AutoTuple{\gamma}{n_2-m}} \\
	&=\opair{
		\AutoTuple{\alpha}{m},
		\AutoTuple{\beta}{n_1-m},
		\AutoTuple{\gamma}{n_2-m}
	}.
\end{align*}
我们希望证明
\(\AutoTuple{\alpha}{m},
\AutoTuple{\beta}{n_1-m},
\AutoTuple{\gamma}{n_2-m}\)
是\(V_1+V_2\)的一个基,
从而得出\begin{align*}
	\dim(V_1+V_2)
	&=m+(n_1-m)+(n_2-m) \\
	&=n_1+n_2-m \\
	&=\dim V_1+\dim V_2-\dim(V_1 \cap V_2).
\end{align*}
假设等式\[
%@see: 《高等代数(第三版 下册)》(丘维声) P86 (4)
	k_1\alpha_1+\dotsb+k_m\alpha_m
	+p_1\beta_1+\dotsb+p_{n_1-m}\beta_{n_1-m}
	+q_1\gamma_1+\dotsb+q_{n_2-m}\gamma_{n_2-m}
	=0
	\eqno(1)
\]成立,
则\[
%@see: 《高等代数(第三版 下册)》(丘维声) P86 (5)
	q_1\gamma_1+\dotsb+q_{n_2-m}\gamma_{n_2-m}
	=-k_1\alpha_1-\dotsb-k_m\alpha_m
	-p_1\beta_1-\dotsb-p_{n_1-m}\beta_{n_1-m}.
\]
注意到上式左边的向量属于\(V_2\),
右边的向量属于\(V_1\),
从而左边的向量属于\(V_1 \cap V_2\),
因此它可由\(\AutoTuple{\alpha}{m}\)线性表出:\[
	q_1\gamma_1+\dotsb+q_{n_2-m}\gamma_{n_2-m}
	=l_1\alpha_1+\dotsb+l_m\alpha_m,
\]
移项得\[
%@see: 《高等代数(第三版 下册)》(丘维声) P86 (6)
	l_1\alpha_1+\dotsb+l_m\alpha_m
	-q_1\gamma_1-\dotsb-q_{n_2-m}\gamma_{n_2-m}
	=0.
\]
由于\(\AutoTuple{\alpha}{m},\AutoTuple{\gamma}{n_2-m}\)是\(V_2\)的一个基,
因此从上式得出\[
	l_1=\dotsb=l_m=0, \qquad
	q_1=\dotsb=q_{n_2-m}=0.
\]
代入(1)式,得\[
	k_1\alpha_1+\dotsb+k_m\alpha_m
	+p_1\beta_1+\dotsb+p_{n_1-m}\beta_{n_1-m}
	=0.
\]
同理可得\[
	k_1=\dotsb=k_m=0, \qquad
	p_1=\dotsb=p_{n_1-m}=0.
\]
因此
\(\AutoTuple{\alpha}{m},
\AutoTuple{\beta}{n_1-m},
\AutoTuple{\gamma}{n_2-m}\)
线性无关.
\end{proof}
\end{theorem}

\begin{corollary}\label{theorem:线性空间.子空间.子空间的维数公式.推论1}
%@see: 《高等代数(第三版 下册)》(丘维声) P86 推论10
设\(V_1,V_2\)都是域\(F\)上\(n\)维线性空间\(V\)的子空间,
则\[
	\dim(V_1+V_2)=\dim V_1+\dim V_2
	\iff
	V_1 \cap V_2=0.
\]
\end{corollary}

\begin{definition}
%@see: 《Linear Algebra and Its Applications (Second Edition)》(Peter D. Lax) P7 Definition
\(n\)维线性空间\(V\)的每一个\(n-1\)维子空间,
都称为“\(V\)中的一个\DefineConcept{超平面}(hyperplane)”.
\end{definition}

\begin{example}
%@see: 《Linear Algebra and Its Applications (Second Edition)》(Peter D. Lax) P12 Exercise 21.
设\(U,V,W\)都是有限维线性空间\(X\)的子空间.
试判断命题\begin{align*}
	\dim(U+V+W)
	&= \dim U + \dim V + \dim W \\
	&\hspace{20pt} - \dim(U \cap V) - \dim (U \cap W) - \dim(V \cap W) \\
	&\hspace{20pt} + \dim(U \cap V \cap W)
\end{align*}
的真伪,并给出证明或反例.
\begin{solution}
%@credit: {腾讯元宝}
取\[
	% 坐标平面
	X = \mathbb{R}^2,
	\qquad
	% x轴
	U = \Span\{(1,0)\},
	\qquad
	% y轴
	V = \Span\{(0,1)\},
	\qquad
	% 直线\(y=x\)
	W = \Span\{(1,1)\},
\]
则\begin{gather*}
	U + V + W = \mathbb{R}^2, \\
	U \cap V
	= U \cap W
	= V \cap W
	= U \cap V \cap W
	= 0, \\
	\dim(U + V + W) = 2,
	\qquad
	\dim U = \dim V = \dim W = 1, \\
	\dim(U \cap V)
	= \dim(U \cap W)
	= \dim(V \cap W)
	= \dim(U \cap V \cap W)
	= 0,
\end{gather*}
于是\begin{align*}
	\dim(U+V+W)
	&\neq \dim U + \dim V + \dim W \\
	&\hspace{20pt} - \dim(U \cap V) - \dim (U \cap W) - \dim(V \cap W) \\
	&\hspace{20pt} + \dim(U \cap V \cap W).
\end{align*}
\end{solution}
\end{example}

\subsection{子空间的直和、补空间}
假设在几何空间\(V\)中,\(V_1\)是过原点的一个平面,\(V_2\)是过原点的一条直线,
且直线\(V_2\)不在平面\(V_1\)上.
容易看出\(V_1 + V_2 = V\),
且\(V_1 + V_2\)中的每一个向量\(\alpha\)
都能被唯一地表示成\[
	\alpha = \alpha_1 + \alpha_2,
	\quad
	\alpha_1 \in V_1,
	\alpha_2 \in V_2.
\]
受此启发,引出以下概念.
\begin{definition}
%@see: 《高等代数(第三版 下册)》(丘维声) P86 定义2
%@see: 《Linear Algebra and Its Applications (Second Edition)》(Peter D. Lax) P6 Definition
设\(V_1,V_2\)是域\(F\)上线性空间\(V\)的子空间,
\(U = V_1 + V_2\).
如果\[
	(\forall\alpha\in U)
	(\exists!\alpha_1\in V_1)
	(\exists!\alpha_2\in V_2)
	[\alpha=\alpha_1+\alpha_2],
\]
则称“\(U\)是\(V_1\)和\(V_2\)的\DefineConcept{直和}%
(\(U\) is the \emph{direct sum} of \(V_1\) and \(V_2\))”,
记作\(U = V_1 \DirectSum V_2\).
\end{definition}

\begin{example}\label{example:线性空间.子空间.直和.例1}
在实数域上的线性空间\(\mathbb{R}^2\)中,
记\(V_1=\opair{(1,0)},
V_2=\opair{(0,1)},
V_3=\opair{(1,1)}\),
显然\(V_1+V_2\)和\(V_1+V_3\)都是直和.
这也说明:给定一个子空间\(V_1\),
满足条件“\(V_1+V_2\)是直和”的\(V_2\)不是唯一的.
\end{example}

\begin{theorem}\label{theorem:线性空间.子空间.直和的等价命题}
%@see: 《高等代数(第三版 下册)》(丘维声) P86 定理11
设\(V_1,V_2\)是域\(F\)上线性空间\(V\)的有限维子空间,
则下列命题互相等价:\begin{itemize}
	\item \(V_1+V_2\)是直和;
	\item \(V_1+V_2\)中零向量的表示法唯一;
	\item \(V_1 \cap V_2=0\);
	% \(V_1 \cap V_2=0\)中的\(0\)是零子空间
	\item \(\dim(V_1+V_2)=\dim V_1+\dim V_2\);
	\item \(V_1\)的一个基与\(V_2\)的一个基 合起来是\(V_1+V_2\)的一个基.
\end{itemize}
\begin{proof}
% (i) => (ii)
“\(V_1+V_2\)是直和”显然是“\(V_1+V_2\)中零向量的表示法唯一”的充分条件.
%TODO 还不理解
% 假设\(0\)有两种表示方法:\begin{gather*}
% 	0 = \alpha_1 + \alpha_2,
% 	\quad \alpha_1 \in V_1,
% 	\alpha_2 \in V_2, \\
% 	0 = \beta_1 + \beta_2,
% 	\quad \beta_1 \in V_1,
% 	\beta_1 \in V_2,
% \end{gather*}

% (ii) => (iii)
下面证明“\(V_1+V_2\)中零向量的表示法唯一”是“\(V_1 \cap V_2=0\)”的充分条件.
任取\(\alpha \in V_1 \cap V_2\),
于是零向量可以表示成\[
	0 = \alpha + (-\alpha),
	\quad
	\alpha \in V_1,
	-\alpha \in V_2.
\]
由已知条件得\(\alpha = 0\).
因此\(V_1 \cap V_2 = 0\).

% (iii) => (i)
下面证明“\(V_1 \cap V_2=0\)”是“\(V_1+V_2\)是直和”的充分条件.
任取\(\alpha \in V_1 + V_2\),
假设\(\alpha\)有两种表示方法:\begin{gather*}
	\alpha = \alpha_1 + \alpha_2,
	\quad \alpha_1 \in V_1,
	\alpha_2 \in V_2, \\
	\alpha = \beta_1 + \beta_2,
	\quad \beta_1 \in V_1,
	\beta_2 \in V_2,
\end{gather*}
则\(\alpha_1 + \alpha_2
= \beta_1 + \beta_2\),
从而得到\(\alpha_1 - \beta_1
= \beta_2 - \alpha_2
\in V_1 \cap V_2\).
由于\(V_1 \cap V_2 = 0\),
因此\(\alpha_1 = \beta_1,
\alpha_2 = \beta_2\),
从而\(V_1 + V_2\)是直和.

% (iii) <=> (iv)
下面证明“\(V_1 \cap V_2=0\)”是“\(\dim(V_1+V_2)=\dim V_1+\dim V_2\)”的充分必要条件.
由\cref{theorem:线性空间.子空间.子空间的维数公式.推论1} 立即可得.

% (iv) => (v)
下面证明“\(\dim(V_1+V_2)=\dim V_1+\dim V_2\)”是“\(V_1\)的一个基与\(V_2\)的一个基 合起来是\(V_1+V_2\)的一个基”的充分条件.
设\(\AutoTuple{\alpha}{s}\)是\(V_1\)的一个基,
\(\AutoTuple{\beta}{r}\)是\(V_2\)的一个基,
则\begin{align*}
	V_1 + V_2
	&= \opair{\AutoTuple{\alpha}{s}} + \opair{\AutoTuple{\beta}{r}} \\
	&= \opair{\AutoTuple{\alpha}{s},\AutoTuple{\beta}{r}}.
\end{align*}
因为\(\dim(V_1 + V_2)
= \dim V_1 + \dim V_2
= s + r\),
且\(V_1 + V_2\)的每一个向量均可由\(\AutoTuple{\alpha}{s},\AutoTuple{\beta}{r}\)线性表出,
所以由\cref{example:线性空间.生成子空间等于线性空间的向量组就是基} 可知
\(\AutoTuple{\alpha}{s},\AutoTuple{\beta}{r}\)是由\(V_1 + V_2\)的一个基.

% (v) => (iv)
“\(\dim(V_1+V_2)=\dim V_1+\dim V_2\)”显然是“\(V_1\)的一个基与\(V_2\)的一个基 合起来是\(V_1+V_2\)的一个基”的必要条件.
\end{proof}
\end{theorem}

\begin{definition}
%@see: 《高等代数(第三版 下册)》(丘维声) P87
设\(V_1,V_2\)都是线性空间\(V\)的子空间,
如果\(V_1 \DirectSum V_2 = V\),
则称“\(V_1\)是\(V_2\)在\(V\)中的一个\DefineConcept{补空间}%
(\(V_1\) is a \emph{complement} of \(V_2\) in \(V\))”
“\(V_2\)是\(V_1\)在\(V\)中的一个{补空间}”
或“\(V_1,V_2\)互为{补空间}(\(V_1,V_2\) are complements of each other)”,
记\(V=V_1 \DirectSum V_2\).
\end{definition}

\begin{proposition}
%@see: 《高等代数(第三版 下册)》(丘维声) P87 命题12
%@see: 《Linear Algebra and Its Applications (Second Edition)》(Peter D. Lax) P6 Theorem 5.(b)
设\(V\)是域\(F\)上\(n\)维线性空间,
则\(V\)的每一个子空间\(U\)都有补空间.
\begin{proof}
从\(U\)中取一个基\(\AutoTuple{\alpha}{m}\),
把它扩充成\(V\)的一个基\(\AutoTuple{\alpha}{m},
\AutoTuple{\alpha}[m+1]{n}\),
则\[
	V=\opair{\AutoTuple{\alpha}{m},\AutoTuple{\alpha}[m+1]{n}}
	=\opair{\AutoTuple{\alpha}{m}}+\opair{\AutoTuple{\alpha}[m+1]{n}}
	=U+W,
\]
其中\(W=\opair{\AutoTuple{\alpha}[m+1]{n}}\).
由于\(U\)的一个基与\(W\)的一个基合起来是\(V\)的一个基,
因此\(U+W\)是直和,
从\(V=U \DirectSum W\).
于是\(W\)是\(U\)的一个补空间.
\end{proof}
\end{proposition}

\begin{example}
举例说明:域\(F\)上线性空间\(V\)的子空间\(W\)在\(V\)中的补空间不唯一.
\begin{solution}
%@credit: {腾讯元宝}
取\[
	V = \mathbb{R}^3,
	\qquad
	W = \Span\{(1,0,0),(0,1,0)\},
	\qquad
	U_1 = \Span\{(0,0,1)\},
	\qquad
	U_2 = \Span\{(1,1,1)\}.
\]
显然\[
	V
	= W \DirectSum U_1
	= W \DirectSum U_2,
\]
这就说明\(U_1,U_2\)都是\(W\)在\(V\)中的补空间.
由于\(U_1 \neq U_2\),
所以\(W\)在\(V\)中的补空间不唯一.
\end{solution}
\end{example}
\begin{remark}
当\(V\)的子空间\(W\)是\(V\)的平凡子空间时,
\(W\)在\(V\)中的补空间是唯一的,
具体地说:
当\(W = V\)时,\(W\)的补空间是\(0\);
当\(W = 0\)时,\(W\)的补空间是\(V\).
\end{remark}
\begin{remark}
赋范线性空间\(V\)的任意一个子空间的正交补空间存在且唯一.
\end{remark}

\begin{example}
%@see: 《高等代数(第三版 下册)》(丘维声) P88 例2
设\(V=M_n(K)\),
其中\(K\)是数域.
分别用\(V_1,V_2\)表示\(K\)上所有\(n\)阶对称矩阵、反对称矩阵组成的子空间.
证明:\(V_1 \DirectSum V_2=V\).
\begin{proof}
首先证明\(V_1+V_2=V\).
显然有\(V_1+V_2\subseteq V\).
现在来证\(V\subseteq V_1+V_2\).
任取\(A\in V\),
有\[
	A=\frac{A+A^T}2+\frac{A-A^T}2.
\]
容易验证\(\frac{A+A^T}2\)是对称矩阵,
\(\frac{A-A^T}2\)是反对称矩阵,
因此\(A\in V_1+V_2\).
从而\(V\subseteq V_1+V_2\),
所以\(V_1+V_2=V\).

然后证\(V_1+V_2\)是直和,
为此只要证\(V_1 \cap V_2=0\).
任取\(B \in V_1 \cap V_2\),
则\(B^T = B = -B\),
从而\(B = 0\).
因此\(V_1 \cap V_2=0\).

综上所述,\(V=V_1 \DirectSum V_2\).
\end{proof}
\end{example}

子空间的直和的概念可以推广到多个子空间的情形.
\begin{definition}
%@see: 《高等代数(第三版 下册)》(丘维声) P88 定义3
%@see: 《Linear Algebra and Its Applications (Second Edition)》(Peter D. Lax) P6 Definition
设\(\AutoTuple{V}{s}\)都是域\(F\)上线性空间\(V\)的子空间,
\(U\)也是\(V\)的一个子空间.
如果\(U\)中每一个向量\(\alpha\)可唯一地表示成\[
	\alpha = \alpha_1+\dotsb+\alpha_s,
	\quad \alpha_i \in V_i,
\]
则称“\(U\)是\(\AutoTuple{V}{s}\)的\DefineConcept{直和}%
(\(U\) is the \emph{direct sum} of \(\AutoTuple{V}{s}\))”,
记作\(U = \bigoplus_{i=1}^s V_i\).
\end{definition}

\begin{theorem}
%@see: 《高等代数(第三版 下册)》(丘维声) P88 定理13
设\(\AutoTuple{V}{s}\)都是域\(F\)上线性空间\(V\)的有限维子空间,
则下列命题互相等价:\begin{itemize}
	\item \(\sum_{i=1}^s V_i\)是直和;
	\item \(\sum_{i=1}^s V_i\)中零向量的表示法唯一;
	\item \(V_i \cap \sum_{j\neq i} V_j=0\ (i=1,2,\dotsc,s)\);
	\item \(\dim\sum_{i=1}^s V_i=\sum_{i=1}^s\dim V_i\);
	\item \(V_i\ (i=1,2,\dotsc,s)\)的一个基 合起来是\(\sum_{i=1}^s V_i\)的一个基.
\end{itemize}
\end{theorem}

\begin{corollary}
%@see: 《高等代数(第三版 下册)》(丘维声) P88 推论14
设\(\AutoTuple{V}{s}\)都是域\(F\)上\(n\)维线性空间\(V\)的子空间,
则\[
	V=\bigoplus_{i=1}^s V_i
	\iff
	\text{\(V_i\ (i=1,2,\dotsc,s)\)的一个基,合起来是\(V\)的一个基}.
\]
\end{corollary}

这个推论让我们可以利用子空间的运算来研究线性空间的结构,
它是研究线性空间的结构的第二条途径.

\begin{example}
%@see: 《高等代数(第三版 下册)》(丘维声) P89 例3
设\(V=K^4,
V_1=\opair{\AutoTuple\a3},
V_2=\opair{\AutoTuple\b2}\),
其中\[
	\alpha_1=\begin{bmatrix} 1 \\ 2 \\ 1 \\ 0 \end{bmatrix},
	\alpha_2=\begin{bmatrix} -1 \\ 1 \\ 1 \\ 1 \end{bmatrix},
	\alpha_3=\begin{bmatrix} 0 \\ 3 \\ 2 \\ 1 \end{bmatrix},
	\beta_1=\begin{bmatrix} 2 \\ -1 \\ 0 \\ 1 \end{bmatrix},
	\beta_2=\begin{bmatrix} 1 \\ -1 \\ 3 \\ 7 \end{bmatrix},
\]
分别求\(V_1+V_2,V_1 \cap V_2\)的一个基和维数.
\begin{solution}
因为\[
	V_1+V_2
	=\opair{\AutoTuple\a3}+\opair{\AutoTuple\b2}
	=\opair{\AutoTuple\a3,\AutoTuple\b2},
\]
所以向量组\(\AutoTuple\a3,\AutoTuple\b2\)的一个极大线性无关组就是\(V_1+V_2\)的一个基,
这个向量组的秩就是\(V_1+V_2\)的维数.
令\(A=(\AutoTuple\a3,\AutoTuple\b2)\).
对\(A\)作一系列初等行变换,得到\[
	A=\begin{bmatrix}
		1 & -1 & 0 & 2 & 1 \\
		2 & 1 & 3 & -1 & -3 \\
		1 & 1 & 2 & 0 & 3 \\
		0 & 1 & 1 & 1 & 7
	\end{bmatrix}
	\to \begin{bmatrix}
		1 & 0 & 1 & 0 & -1 \\
		0 & 1 & 1 & 0 & 4 \\
		0 & 0 & 0 & 1 & 3 \\
		0 & 0 & 0 & 0 & 0
	\end{bmatrix}.
\]
由此得出,\(\alpha_1,\alpha_2,\beta_1\)是\(V_1+V_2\)的一个基,
\(\dim(V_1+V_2)=3\).
同时也知道,\(\beta_2\)可由\(\alpha_1,\alpha_2,\beta_1\)线性表出,
其系数是线性方程组\(x_1\alpha_1+x_2\alpha_2+x_3\beta_1=\beta_2\)的解\((-1,4,3)^T\),
即\(\beta_2=-\alpha_1+4\alpha_2+3\beta_1\).
从而\(\alpha_1-4\alpha_2=3\beta_1-\beta_2\in V_1 \cap V_2\).
又因为\(\dim V_1=2,
\dim V_2=2\),
所以由子空间的维数公式有
\(\dim(V_1 \cap V_2)
=\dim V_1+\dim V_2-\dim(V_1+V_2)
=2+2-3=1\),
于是\(\alpha_1-4\alpha_2=(5,-2,-3,-4)^T\)是\(V_1 \cap V_2\)的一个基.
\end{solution}
\end{example}

从上例可以看出,
只要对矩阵\(A\)作一系列初等行变换,
把它化为若尔当阶梯形矩阵,
就可以得到子空间的基和维数等信息.

\section{线性空间的同构}
域\(F\)上\(n\)维线性空间\(V\)
与域\(F\)上\(n\)元有序组组成的线性空间\(F^n\)非常相像.
例如,对于\(F^n\)向量组\(\AutoTuple{\alpha}{s}\)生成的子空间\(U=\opair{\AutoTuple{\alpha}{s}}\),
向量组\(\AutoTuple{\alpha}{s}\)的一个极大线性无关组是\(U\)的一个基,
\(\dim U\)等于\(\rank\{\AutoTuple{\alpha}{s}\}\).
对于\(V\)中向量组生成的子空间也有同样的结论.

为什么域\(F\)上的\(n\)维线性空间\(V\)与\(F^n\)这样相像?

\begin{definition}
%@see: 《高等代数(第三版 下册)》(丘维声) P92 定义1
设\(V\)与\(V'\)都是域\(F\)上的线性空间,
\(\sigma\)是一个从\(V\)到\(V'\)的双射.
如果\begin{itemize}
	\item \((\forall\alpha,\beta \in V)
	[\sigma(\alpha+\beta)=\sigma(\alpha)+\sigma(\beta)]\),
	\item \((\forall\alpha \in V)
	(\forall k \in F)
	[\sigma(k\alpha)=k\sigma(\alpha)]\),
\end{itemize}
那么称“\(\sigma\)是从\(V\)到\(V'\)的一个\DefineConcept{同构}(isomorphism)”
“\(V\)与\(V'\)同构(\(V\) is \emph{isomorphic} to \(V'\))”,
记为\(V \Isomorphism V'\).

特别地,当\(V=V'\)且\(V\)和\(V'\)上定义的线性运算相同时,
称“\(\sigma\)是\(V\)上的一个\DefineConcept{自同构}(automorphism)”.
\end{definition}
\begin{remark}
讨论两个线性空间是否同构的前提是这两个线性空间的域相同.
\end{remark}

\begin{property}\label{theorem:线性空间的同构.同构线性空间的性质1}
%@see: 《高等代数(第三版 下册)》(丘维声) P92 性质1
设\(V\)与\(V'\)都是域\(F\)上的线性空间,
\(0\)是\(V\)的零元,
\(0'\)是\(V'\)的零元,
\(\sigma\)是一个从\(V\)到\(V'\)的同构,
则\(\sigma(0)=0'\).
\begin{proof}
\(0\alpha=0 \implies \sigma(0)=\sigma(0\alpha)=0\sigma(\alpha)=0'\).
\end{proof}
\end{property}

\begin{property}\label{theorem:线性空间的同构.同构线性空间的性质2}
%@see: 《高等代数(第三版 下册)》(丘维声) P92 性质2
设\(V\)与\(V'\)都是域\(F\)上的线性空间,
\(\sigma\)是一个从\(V\)到\(V'\)的同构,
则\[
	(\forall\alpha\in V)[\sigma(-\alpha)=-\sigma(\alpha)].
\]
\begin{proof}
\(\sigma(-\alpha)=\sigma((-1)\alpha)=(-1)\sigma(\alpha)=-\sigma(\alpha)\).
\end{proof}
\end{property}

\begin{property}\label{theorem:线性空间的同构.同构线性空间的性质3}
%@see: 《高等代数(第三版 下册)》(丘维声) P92 性质3
设\(V\)与\(V'\)都是域\(F\)上的线性空间,
\(\sigma\)是一个从\(V\)到\(V'\)的同构,
则\[
	(\forall \AutoTuple{\alpha}{s} \in V)
	(\forall \AutoTuple{k}{s} \in F)
	[\sigma(k_1\alpha_1+\dotsb+k_s\alpha_s)=k_1\sigma(\alpha_1)+\dotsb+k_s\sigma(\alpha_s)].
\]
\end{property}

\begin{property}\label{theorem:线性空间的同构.同构线性空间的性质4}
%@see: 《高等代数(第三版 下册)》(丘维声) P92 性质4
设\(V\)与\(V'\)都是域\(F\)上的线性空间,
\(\sigma\)是一个从\(V\)到\(V'\)的同构,
则\(V\)中向量组\(\AutoTuple{\alpha}{s}\)线性相关的充分必要条件是:
\(\sigma(\alpha_1),\dotsc,\sigma(\alpha_s)\)是\(V'\)中线性相关的向量组.
\begin{proof}
因为\(\sigma\)是单射,
所以\(\sigma(\alpha)=\sigma(\beta) \implies \alpha=\beta\),
于是\begin{align*}
	k_1\alpha_1+\dotsb+k_s\alpha_s=0
	&\iff
	\sigma(k_1\alpha_1+\dotsb+k_s\alpha_s)=\sigma(0) \\
	&\iff
	k_1\sigma(\alpha_1)+\dotsb+k_s\sigma(\alpha_s)=0',
\end{align*}
那么\(\AutoTuple{\alpha}{s}\)线性相关
当且仅当\(\sigma(\alpha_1),\dotsc,\sigma(\alpha_s)\)线性相关.
\end{proof}
\end{property}

\begin{property}\label{theorem:线性空间的同构.同构线性空间的性质5}
%@see: 《高等代数(第三版 下册)》(丘维声) P92 性质5
设\(V\)与\(V'\)都是域\(F\)上的线性空间,
\(\sigma\)是一个从\(V\)到\(V'\)的同构.
如果\(\AutoTuple{\alpha}{n}\)是\(V\)的一个基,
则\(\sigma(\alpha_1),\dotsc,\sigma(\alpha_n)\)是\(V'\)的一个基.
\begin{proof}
由\cref{theorem:线性空间的同构.同构线性空间的性质4}
可知\(\sigma(\alpha_1),\dotsc,\sigma(\alpha_n)\)是\(V'\)的一个线性无关的向量组.
任取\(\beta \in V'\),
由于\(\sigma\)是满射,
因此存在\(\alpha \in V\),
使得\(\sigma(\alpha)=\beta\).
设\(\alpha=k_1\alpha_1+\dotsb+k_n\alpha_n\),
则\[
	\beta=\sigma(\alpha)
	=k_1\sigma(\alpha_1)+\dotsb+k_n\sigma(\alpha_n),
\]
因此\(\sigma(\alpha_1),\dotsc,\sigma(\alpha_n)\)是\(V'\)的一个基.
\end{proof}
\end{property}

\begin{theorem}\label{theorem:线性空间的同构.线性空间同构的充分必要条件}
%@see: 《高等代数(第三版 下册)》(丘维声) P92 定理1
设\(V\)与\(V'\)都是域\(F\)上的有限维线性空间,
则\(V \Isomorphism V'\)的充分必要条件是\(\dim V = \dim V'\).
\begin{proof}
必要性.
由\cref{theorem:线性空间的同构.同构线性空间的性质5} 立即得出.

充分性.
设\(\dim V = \dim V' = n\).
在\(V\)中取一个基\(\AutoTuple{\alpha}{n}\).
在\(V'\)中取一个基\(\AutoTuple{\gamma}{n}\).
令\[
	\sigma\colon V \to V',
	\alpha=\sum_{i=1}^n k_i\alpha_i
	\mapsto
	\sum_{i=1}^n k_i\gamma_1.
\]
可以看出,\(\sigma\)是一个从\(V\)到\(V'\)的同构,
\(V \Isomorphism V'\).
\end{proof}
\end{theorem}
从\cref{theorem:线性空间的同构.线性空间同构的充分必要条件} 立即得出,
域\(F\)上任意一个\(n\)维线性空间\(V\)都与\(F^n\)同构,
并且\(V\)中每一个向量\(\alpha\)
对应它在\(V\)的一个基\(\AutoTuple{\alpha}{n}\)下的坐标\((\AutoTuple{k}{n})^T\),
这个对应关系就是从\(V\)到\(F^n\)的一个同构.
正是因为域\(F\)上\(n\)维线性空间\(V\)与\(F^n\)同构,
所以\(V\)与\(F^n\)才这么相像.
虽然它们的元素不同,但是有关线性运算的性质却完全一样.
于是我们可以利用\(F^n\)的性质来研究\(F\)上\(n\)维线性空间的性质.
线性空间的同构,是研究线性空间结构的第三条途径.

\begin{proposition}\label{theorem:线性空间的同构.子空间在同构下的像}
%@see: 《高等代数(第三版 下册)》(丘维声) P93 命题2
设\(V\)是域\(F\)上的\(n\)维线性空间,
\(U\)是\(V\)的一个子空间,
\(\AutoTuple{\alpha}{n}\)的\(V\)的一个基,
\(\sigma\)把\(V\)中每一个向量\(\alpha\)对应到它在基\(\AutoTuple{\alpha}{n}\)下的坐标.
令\[
	\sigma(U) \defeq \Set{ \sigma(\alpha) \given \alpha \in U },
\]
则\(\sigma(U)\)是\(F^n\)的一个子空间,
且\(\dim U = \dim\sigma(U)\).
\begin{proof}
显然\(\sigma(U)\)是非空集,
\(\sigma\)是一个从\(V\)到\(F^n\)的同构,
\(U\)对加法和纯量乘法封闭.
这就说明\(\sigma(U)\)是\(F^n\)的一个子空间.

由于\(U\)与\(\sigma(U)\)都是域\(F\)上有限维线性空间,
且\(\sigma\)在\(U\)上的限制\((\sigma \setrestrict U)\)是从\(U\)到\(\sigma(U)\)的一个同构,
因此\(\dim U = \dim\sigma(U)\).
\end{proof}
\end{proposition}

\begin{example}
%@see: 《高等代数(第三版 下册)》(丘维声) P94 例1
设\(\AutoTuple{\alpha}{n}\)是域\(F\)上线性空间\(V\)的一个基,
\(\AutoTuple{\beta}{s}\)是\(V\)的一个向量组,
并且\[
%@see: 《高等代数(第三版 下册)》(丘维声) P94 (5)
	(\AutoTuple{\beta}{s})
	= (\AutoTuple{\alpha}{n}) A,
\]
其中\(A\)是一个\(n \times s\)矩阵.
证明:\[
%@see: 《高等代数(第三版 下册)》(丘维声) P94 (6)
	\dim\opair{\AutoTuple{\beta}{s}}
	= \rank A.
\]
\begin{proof}
用\(\sigma\)表示从\(V\)到\(F^n\)的一个同构,
它把\(\alpha \in V\)映射为\(\alpha\)在\(\AutoTuple{\alpha}{n}\)下的坐标.
设\(A\)的列向量组是\(\AutoTuple{A}{s}\),那么\[
	\sigma(\beta_j) = A_j,
	\quad j=1,2,\dotsc,s.
\]
由\cref{theorem:线性空间的同构.子空间在同构下的像} 可知\begin{align*}
	&\dim\opair{\AutoTuple{\beta}{s}} \\
	&= \dim\sigma\opair{\AutoTuple{\beta}{s}} \\
	&= \dim\opair{\sigma(\beta_1),\dotsc,\sigma(\beta_s)} \\
	&= \dim\opair{\AutoTuple{A}{s}} \\
	&= \rank A.
	\qedhere
\end{align*}
\end{proof}
\end{example}

同构是域\(F\)上线性空间之间的一个关系.
它具有反身性(因为\(V\)的恒等映射是从\(V\)到\(V\)的一个同构)、
对称性和传递性.
因此同构关系是一个等价关系,
对应的等价类称为\DefineConcept{同构类}.

\cref{theorem:线性空间的同构.线性空间同构的充分必要条件} 表明,
对于域\(F\)上的全体有限维线性空间\[
	S = \Set{ V \given \text{$V$是域$F$上的有限维线性空间} }
\]而言,
域\(F\)上所有维数为\(0\)的线性空间 --- \(\{0\}\) --- 恰好组成一个同构类,
域\(F\)上所有1维线性空间恰好组成一个同构类,
域\(F\)上所有2维线性空间恰好组成一个同构类,
以此类推.
可以看出,
% Two linear spaces over the same field and of the same dimension are isomorphic.
相同域上相同维数的两个线性空间同构,
域和维数决定了同构类.
因此,域\(F\)上有限维线性空间的同构类与自然数之间存在一个一一对应.

\begin{proposition}
%@see: 《高等代数(第三版 下册)》(丘维声) P94 命题3
域\(F\)上线性空间之间的一个同构的逆映射也是同构.
\begin{proof}
设\(V,V'\)都是域\(F\)上的线性空间,
\(\sigma\)是从\(V\)到\(V'\)的一个同构.
显然\(\sigma^{-1}\)是从\(V'\)到\(V\)的一个双射.
任取\(\alpha',\beta' \in V'\),
则存在\(\alpha,\beta \in V\),
使得\(\alpha' = \sigma(\alpha),
\beta' = \sigma(\beta)\).
从而\(\sigma^{-1}(\alpha') = \alpha,
\sigma^{-1}(\beta') = \beta\).
于是\begin{align*}
	\sigma^{-1}(\alpha' + \beta')
	&= \sigma^{-1}(\sigma(\alpha) + \sigma(\beta)) \\
	&= \sigma^{-1}(\sigma(\alpha + \beta)) \\
	&= (\sigma^{-1} \circ \sigma)(\alpha + \beta) \\
	&= 1_V (\alpha + \beta) \\
	% \(1_V\)是\(V\)上的恒等变换
	&= \alpha + \beta \\
	&= \sigma^{-1}(\alpha') + \sigma^{-1}(\beta'), \\
	\sigma^{-1}(k \alpha')
	&= \sigma^{-1}(k \sigma(\alpha)) \\
	&= \sigma^{-1}(\sigma(k \alpha)) \\
	&= \sigma^{-1}(\sigma(k \alpha)) \\
	&= k \alpha
	= k \sigma^{-1}(\alpha').
\end{align*}
因此\(\sigma^{-1}\)是从\(V'\)到\(V\)的一个同构.
\end{proof}
\end{proposition}

\begin{proposition}
%@see: 《高等代数(第三版 下册)》(丘维声) P94 命题3
域\(F\)上线性空间之间的两个同构的复合还是同构.
% 原话是:两个同构映射的乘积还是同构映射.
\begin{proof}
设\(V,V',V''\)都是域\(F\)上的线性空间,
\(\sigma_1\)是从\(V\)到\(V'\)的一个同构,
\(\sigma_2\)是从\(V'\)到\(V''\)的一个同构,
则\(\sigma_2 \circ \sigma_1\)是从\(V\)到\(V''\)的一个双射.
任取\(\alpha,\beta \in V,
k \in F\),
则有\begin{align*}
	(\sigma_2 \circ \sigma_1)(\alpha + \beta)
	&= \sigma_2(\sigma_1(\alpha + \beta)) \\
	&= \sigma_2(\sigma_1(\alpha) + \sigma_1(\beta)) \\
	&= \sigma_2(\sigma_1(\alpha)) + \sigma_2(\sigma_1(\beta)), \\
	(\sigma_2 \circ \sigma_1)(k \alpha)
	&= \sigma_2(\sigma_1(k \alpha)) \\
	&= \sigma_2(k \sigma_1(\alpha)) \\
	&= k (\sigma_2 \circ \sigma_1)(\alpha).
\end{align*}
因此\(\sigma_2 \circ \sigma_1\)是从\(V\)到\(V''\)的一个同构.
\end{proof}
\end{proposition}

\section{商空间}
几何空间可以看成是由原点\(O\)为起点的所有向量组成的\(3\)维实线性空间\(V\).
过原点的一个平面\(W\)是\(V\)的一个\(2\)维子空间.
与\(W\)平行的每一个平面\(\pi\)都不是\(V\)的子空间,
因为\(\pi\)对加法和数量乘法都不封闭.
但是我们还是想问:\(\pi\)具有什么样的结构?\(\pi\)与\(W\)的关系如何?

在\(\pi\)上取定一个向量\(\g_0\),
\(\pi\)上每一个向量\(\g\)可以唯一地表示成\(\g_0\)与\(W\)中一个向量\(\vb\eta\)之和:
\(\g=\g_0+\vb\eta\).

反之,任取\(\vb\eta\in W\),
有\(\g_0+\vb\eta\in\pi\),
因此\(\pi=\Set{ \g_0+\vb\eta \given \vb\eta\in W }\).
我们可以把\(\pi\)记作\(\g_0+W\),
称其为“\(W\)的一个\emph{陪集}”,
把\(\g_0\)称为\emph{陪集代表}.
显然\begin{align*}
	\g\in\g_0+W
	&\iff
	\g=\g_0+\vb\eta,\vb\eta\in W \\
	&\iff
	\g-\g_0=\vb\eta\in W.
\end{align*}
由此看出,
如果在\(V\)上规定一个二元关系\(\sim\)满足\[
	\g\sim\g_0
	\defiff
	\g-\g_0\in W,
\]
那么容易验证关系\(\sim\)具有反身性、对称性和传递性,
这就是说关系\(\sim\)是等价关系,
于是\(\g_0\)所属的等价类\(\overline{\g_0}\)为\begin{align*}
	\overline{\g_0}
	&=\Set{ \g\in V \given \g\sim\g_0 } \\
	&=\Set{ \g\in V \given \g-\g_0\in W } \\
	&=\Set{ \g\in V \given \g=\g_0+\vb\eta,\vb\eta\in W } \\
	&=\Set{ \g_0+\vb\eta \given \vb\eta\in W }
	=\g_0+W.
\end{align*}
这表明陪集\(\g_0+W\)是等价类\(\overline{\g_0}\).
\(W\)本身也是\(W\)的一个陪集\(0+W\).

综上所述,
在几何空间\(V\)中,
与\(W\)平行或重合的每一个平面\(\pi\)是\(W\)的一个陪集,
也是等价关系\(\sim\)下的一个等价类.
所有等价类(即所有与\(W\)平行或重合的平面)组成的集合是几何空间的一个划分.
利用这个划分可以研究几何空间的结构.
受此启发,我们能不能给出线性空间\(V\)的一个划分,
然后利用这个划分来研究线性空间\(V\)的结构呢?
我们已经知道,要想给出线性空间\(V\)的一个划分,
就需要在\(V\)上建立一个二元等价关系,
得到的所有等价类组成的集合就是\(V\)的一个划分.

\subsection{陪集,陪集代表}
设\(V\)是域\(F\)上的一个线性空间,\(W\)是\(V\)的一个子空间.
在\(V\)上定义一个二元关系\(\sim\)满足\[
%@see: 《高等代数(第三版 下册)》(丘维声) P97 (3)
	\a\sim\b
	\defiff
	\a-\b\in W,
\]
则\(\sim\)是一个等价关系.

\(\a\)的等价类\begin{align*}
%@see: 《高等代数(第三版 下册)》(丘维声) P97 (4)
	\overline{\a}
	&=\Set{ \b \in V \given \b \sim \a }
	=\Set{ \b \in V \given \b-\a \in W } \\
	&=\Set{ \b \in V \given \b=\a+\g \land \g \in W } \\
	&=\Set{ \a+\g \given \g \in W }
\end{align*}
可以记作\(\a+W\),称为“\(W\)的一个\DefineConcept{陪集}”,
\(\a\)称为\DefineConcept{陪集代表}.
特别地,可以将\[
	\overline{\vb0}
	= \vb0+W
	= \Set{
		\g \in V
		\given
		\g \in W
	}
\]简记为\(W\).
于是\(\a\)的等价类\(\overline{\a}\)
就是以\(\a\)为代表的\(W\)的一个陪集\(\a+W\).
从而\[
%@see: 《高等代数(第三版 下册)》(丘维声) P98 (5)
	\vb\beta \in \vb\alpha + W
	\iff
	\vb\beta \sim \vb\alpha
	\iff
	\vb\beta - \vb\alpha \in W.
\]

应该注意到\[
%@see: 《高等代数(第三版 下册)》(丘维声) P98 (6)
	\vb\alpha + W = \vb\delta + W
	\iff
	\vb\alpha - \vb\delta \in W.
\]
可以看出,一个陪集的陪集代表不唯一,
只要\(\vb\delta\)满足\(\vb\alpha-\vb\delta \in W\),
那么\(\vb\delta\)也可以作为这个陪集的代表.

\subsection{商集,商空间}
根据\cref{definition:集合论.商集的定义},
\(W\)的全体陪集,就是\(V\)对于关系\(\sim\)的商集\(V/\kern-2pt\sim\).
因为关系\(\sim\)是利用子空间\(W\)确定的,
因此我们又把\(V/\kern-2pt\sim\)称为“线性空间\(V\)对于子空间\(W\)的商集”,
记作\(V/W\),
即\[
	V/W
	\defeq
	\Set{ \a+W \given \a \in V }.
\]

线性空间\(V\)是具有加法与纯量乘法两种运算的集合,
因此我们有理由期望\(V\)对于子空间\(W\)的商集\(V/W\)
也可以定义加法与纯量乘法两种运算.
但是由于\(V/W\)不是\(V\)的子集,
因此不能把\(V\)的两种运算直接搬到\(V/W\)中来.

\begin{theorem}\label{theorem:商空间.商空间的线性运算}
%@see: 《高等代数(第三版 下册)》(丘维声) P98 定理1
设\(W\)是域\(F\)上线性空间\(V\)的一个子空间.
在\(V\)对于子空间\(W\)的商集\(V/W\)规定运算\begin{itemize}
	\item 加法:\begin{equation}\label{equation:商空间.商空间的加法}
		%@see: 《高等代数(第三版 下册)》(丘维声) P98 (8)
		(\forall(\a+W),(\b+W)\in V/W)[(\a+W) + (\b+W) \defeq (\a+\b)+W],
	\end{equation}
	\item 纯量乘法:\begin{equation}\label{equation:商空间.商空间的纯量乘法}
		%@see: 《高等代数(第三版 下册)》(丘维声) P98 (9)
		(\forall(\a+W)\in V/W)(\forall k \in F)[k(\a+W) \defeq k\a+W],
	\end{equation}
\end{itemize}
则\(V/W\)对于这两种运算成为域\(F\)上的一个线性空间,
它的零元是\(0+W\).
\begin{proof}
\cref{equation:商空间.商空间的加法,equation:商空间.商空间的纯量乘法}
分别用到了陪集代表\(\a,\b\),
我们下面说明它们的运算结果不依赖于陪集代表的选择.
设\[
	\a_1 + W = \a + W,
	\qquad
	\b_1 + W = \b + W,
\]
则\[
	\a_1 - \a \in W,
	\qquad
	\b_1 - \b \in W.
\]
从而\begin{gather*}
	(\a_1 + \b_1) - (\a + \b)
	= (\a_1 - \a) + (\b_1 - \b)
	\in W, \\
	k \a_1 - k \a
	= k (\a_1 - \a)
	\in W,
\end{gather*}
因此\[
	(\a_1 + \b_1) + W
	= (\a + \b) + W,
	\qquad
	k \a_1 + W
	= k \a + W.
\]
这就证明上述关于商空间的加法和纯量乘法运算的定义是合理的.

容易验证线性空间定义的8条运算法则在\(V/W\)中都成立.
因此,\(V/W\)成为域\(F\)上的一个线性空间.
\end{proof}
\end{theorem}

既然\(V\)对于子空间\(W\)的商集\(V/W\)对加法和纯量乘法成为域\(F\)上的一个线性空间,
那么我们可以把线性空间\(V\)对于子空间\(W\)的商集\(V/W\)
重新命名为“\(V\)对\(W\)的\DefineConcept{商空间}”.

\subsection{商空间的维数}
\begin{theorem}\label{theorem:商空间.商空间的维数}
%@see: 《高等代数(第三版 下册)》(丘维声) P99 定理2
设\(W\)是域\(F\)上有限维线性空间\(V\)的一个子空间,
则\begin{equation}
%@see: 《高等代数(第三版 下册)》(丘维声) P99 (10)
	\dim(V/W) = \dim V - \dim W.
\end{equation}
\begin{proof}
设\(\dim V = n,
\dim W = s\).
在\(W\)中取一个基\(\AutoTuple{\a}{s}\),
把它扩充成\(V\)的一个基\[
	\AutoTuple{\a}{s},
	\allowbreak
	\AutoTuple{\a}[s+1]{n}.
\]
任取\(\b + W \in V/W\),
令\(\b = b_1 \a_1 + \dotsb + b_n \a_n\),
则\begin{align*}
	%@see: 《高等代数(第三版 下册)》(丘维声) P99 (11)
	\b + W
	&= (b_1 \a_1 + \dotsb + b_n \a_n) + W \\
	% 商空间的加法
	&= (b_1 \a_1 + W) + \dotsb + (b_s \a_s + W)
		+ (b_{s+1} \a_{s+1} + W) + \dotsb + (b_n \a_n + W) \\
	% 因为\(\AutoTuple{\a}{s} \in W\),所以\(b_i \a_i + W = W\ (i=1,2,\dotsc,s)\).
	&= W + \dotsb + W + b_{s+1} (\a_{s+1} + W) + \dotsb + b_n (\a_n + W) \\
	% 商空间的加法
	&= b_{s+1} (\a_{s+1} + W) + \dotsb + b_n (\a_n + W).
\end{align*}
这表明\(V/W\)中任一向量均可表示成\(\a_{s+1} + W,\dotsc,\a_n + W\)的线性组合.
若能证明\(\a_{s+1} + W,\dotsc,\a_n + W\)线性无关,
则它就是\(V/W\)的一个基,
从而\[
	\dim(V/W)
	= n - s
	= \dim V - \dim W.
\]
现在假设\[
%@see: 《高等代数(第三版 下册)》(丘维声) P99 (12)
	k_1 (\a_{s+1} + W)
	+ \dotsb
	+ k_{n-s} (\a_n + W)
	= W,
	% 这里就是在假设\(\a_{s+1} + W,\dotsc,\a_n + W\)的一个线性组合等于零陪集
\]
则\[
	% 商空间的加法
	(k_1 \a_{s+1} + \dotsb + k_{n-s} \a_n) + W = W,
\]
从而\[
	k_1 \a_{s+1} + \dotsb + k_{n-s} \a_n \in W,
\]
于是\[
	k_1 \a_{s+1} + \dotsb + k_{n-s} \a_n
	= l_1 \a_1 + \dotsb + l_s \a_s,
\]
即\[
%@see: 《高等代数(第三版 下册)》(丘维声) P99 (13)
	l_1 \a_1 + \dotsb + l_s \a_s
	- k_1 \a_{s+1} - \dotsb - k_{n-s} \a_n
	= \vb0.
\]
由于\(\AutoTuple{\a}{s},
\allowbreak
\AutoTuple{\a}[s+1]{n}\)线性无关,
所以\[
	l_1 = \dotsb = l_s
	= k_1 = \dotsb = k_{n-s}
	= 0.
\]
这证明了\(\a_{s+1} + W,\dotsc,\a_n + W\)是\(V/W\)中线性无关的向量组.
\end{proof}
\end{theorem}

从这里我们可以看出,
在\(W\)中取一个基\(\AutoTuple{\a}{s}\),
把它扩充成\(V\)的一个基\[
	\AutoTuple{\a}{s},
	\allowbreak
	\AutoTuple{\a}[s+1]{n},
\]
则\(\a_{s+1}+W,\dotsc,\a_n+W\)是商空间\(V/W\)的一个基.
令\(U=\Span\{\a_{s+1},\dotsc,\a_n\}\).
由于\(W\)的一个基\(\AutoTuple{\a}{s}\)
与\(U\)的一个基\(\a_{s+1},\dotsc,\a_n\)合起来是\(V\)的一个基,
因此\(V=W \oplus U\).
这个结论也可以推广到\(V\)及其子空间\(W\)都是无限维的,
而商空间\(V/W\)是有限维的情况:
\begin{theorem}\label{theorem:商空间.商空间的基与它的零元素的补空间的基之间的关系}
%@see: 《高等代数(第三版 下册)》(丘维声) P100 定理3
设\(V\)是域\(F\)上的线性空间,
\(W\)是\(V\)的一个子空间.
如果商空间\(V/W\)的一个基为\[
	\b_1+W,\dotsc,\b_t+W,
\]
令\(U=\Span\{\b_1,\dotsc,\b_t\}\),
那么\(V=W \oplus U\),
并且\(\AutoTuple{\b}{t}\)是\(U\)的一个基.
\begin{proof}
任取\(\a \in V\).
由于\(\b_1+W,\dotsc,\b_t+W\)是商空间\(V/W\)的一个基,
因此存在\(F\)中一组元素\(\AutoTuple{k}{t}\),
使得\begin{align*}
	\a+W
	&= k_1 (\b_1 + W) + \dotsb + k_t (\b_t + W) \\
	&= (k_1 \b_1 + \dotsb k_t \b_t) + W.
\end{align*}
于是\(\a - (k_1 \b_1 + \dotsb k_t \b_t) \in W\).
记\(\b = k_1 \b_1 + \dotsb + k_t \b_t\),
则\(\b \in U\)且\(\a - \b \in W\).
记\(\a - \b = \vb\delta\),
则\(\a = \vb\delta + \b \in W + U\).
于是\(V \subseteq W+U\),
从而\(V = W+U\).

任取\(\g \in W \cap U\),
则\(\g \in W\)且\(\g \in U\),
于是存在\(F\)中一组元素\(\AutoTuple{l}{t}\),
使得\[
	\g = l_1 \b_1 + \dotsb + l_t \b_t,
\]
并且\begin{align*}
	W &= \g + W
	= (l_1 \b_1 + \dotsb + l_t \b_t) + W \\
	&= l_1 (\b_1 + W) + \dotsb + l_t (\b_t + W).
\end{align*}
由于\(\b_1+W,\dotsc,\b_t+W\)线性无关,
因此从上式可知\[
	l_1 = \dotsb = l_t = 0.
\]
于是\(\g = \vb0\),
从而\(W \cap U = 0\).

综上所述,\(V = W \oplus U\).

设\[
	x_1 \b_1 + \dotsb x_t \b_t = 0,
\]
则\[
	(x_1 \b_1 + \dotsb + x_t \b_t) + W
	= 0 + W
	= W.
\]
从上面一段证得的结论可知,
\(x_1 = \dotsb = x_t = 0\).
因此\(\AutoTuple{\b}{t}\)线性无关,
那么\(\AutoTuple{\b}{t}\)是\(U\)的一个基.
\end{proof}
\end{theorem}
\cref{theorem:商空间.商空间的基与它的零元素的补空间的基之间的关系} 表明,
如果线性空间\(V\)对于子空间\(W\)的商空间\(V/W\)是有限维的,
并且我们知道商空间\(V/W\)的一个基,
那么\(V\)就有一个直和分解式.
这是可以利用商空间研究线性空间的结构的原理之一.

从\cref{theorem:商空间.商空间的维数} 看出,
对于\(n\)维线性空间\(V\),
如果它的子空间\(W\)不是零子空间,
那么\[
	\dim(V/W) < \dim V.
\]
于是我们可以利用数学归纳法来证明线性空间中有关被商空间继承的性质的结论.
这是可以利用商空间研究线性空间的结构的原理之二.

综上所述,商空间是研究线性空间结构的第四条途径.

\cref{theorem:商空间.商空间的基与它的零元素的补空间的基之间的关系} 还表明,
对于无限维线性空间\(V\)的无限维子空间\(W\),
如果商空间\(V/W\)是有限维的,
那么\(W\)在\(V\)中也有补空间.

更一般地,可以证明:
无限维线性空间\(V\)的任一子空间\(W\)都有补空间.
% 证明过程在《高等代数(第三版 下册)》(丘维声)参考文献[18]的第8章第4节的命题1

\begin{example}
%@see: 《高等代数(第三版 下册)》(丘维声) P101 习题8.4 1.
设\(U,W\)都是域\(F\)上线性空间\(V\)的子空间.
证明:如果\(V = U \oplus W\),
那么\(V\)对\(U\)的商空间\(V/U\)与\(W\)同构.
%TODO proof
\end{example}

\begin{example}
%@see: 《高等代数(第三版 下册)》(丘维声) P101 习题8.4 4.
设\(U,W\)都是域\(F\)上线性空间\(V\)的子空间.
证明:\[
	(U+W)/W \simeq U/(U \cap W).
\]
%TODO proof
\end{example}

\subsection{余维数}
\begin{definition}
%@see: 《高等代数(第三版 下册)》(丘维声) P101 定义1
设\(W\)是域\(F\)上线性空间\(V\)的一个子空间.
如果\(V\)对\(W\)的商空间\(V/W\)是有限维的,
则\(\dim(V/W)\)称为
“子空间\(W\)在\(V\)中的\DefineConcept{余维数}(codimension)”,
记作\(\codim_V W\),
简记为\(\codim W\).
\end{definition}

\subsection{典范映射}
\begin{definition}
设\(W\)是域\(F\)上线性空间\(V\)的一个子空间,
\(V/W\)是\(V\)对\(W\)的商空间.
把映射\[
	\pi\colon V \to V/W,
	\vb\alpha \mapsto \vb\alpha+W
\]称为\DefineConcept{标准映射}或\DefineConcept{典范映射}.
\end{definition}
\begin{remark}
典范映射\(\pi\)是满射.
\end{remark}


\chapter{线性映射}
我们在上一章研究了域\(F\)上线性空间的结构.
在许多数学分支和实际问题中都会遇到线性空间之间的映射,
并且这种映射保持加法和纯量乘法两个运算,
称其为\emph{线性映射}.
线性代数就是研究线性空间和线性映射的理论.
这一章我们来研究线性映射的理论.

\section{线性映射及其运算}
\subsection{线性映射的概念}
\begin{definition}
%@see: 《高等代数(第三版 下册)》(丘维声) P106 定义1
%@see: 《Linear Algebra Done Right (Fourth Eidition)》(Sheldon Axler) P52 3.1
设\(V\)和\(V'\)都是域\(F\)上的线性空间,
\(\vb{A}\)是从\(V\)到\(V'\)的一个映射.

如果\begin{equation*}
	(\forall\alpha,\beta\in V)
	[\vb{A}(\alpha+\beta)=\vb{A}(\alpha)+\vb{A}(\beta)],
\end{equation*}
则称“\(\vb{A}\)适合\DefineConcept{可加性}(additivity)”.

如果\begin{equation*}
	(\forall\alpha\in V)
	(\forall k\in F)
	[\vb{A}(k\alpha)=k\vb{A}(\alpha)],
\end{equation*}
则称“\(\vb{A}\)适合\DefineConcept{齐次性}(homogeneity)”.

如果\(\vb{A}\)适合可加性、齐次性,
则称“\(\vb{A}\)是从\(V\)到\(V'\)的一个\DefineConcept{线性映射}%
(\(\vb{A}\) is a \emph{linear map} from \(V\) to \(V'\))”.
%@see: https://mathworld.wolfram.com/LinearTransformation.html
\end{definition}

\begin{definition}
如果线性映射\(\vb{A}\)是单射,
则称“\(\vb{A}\)是\DefineConcept{单线性映射}(injective linear map)”.
\end{definition}

\begin{definition}
如果线性映射\(\vb{A}\)是满射,
则称“\(\vb{A}\)是\DefineConcept{满线性映射}(surjective linear map)”.
\end{definition}

\begin{definition}
%@see: 《高等代数(第三版 下册)》(丘维声) P106 定义1
线性空间\(V\)到自身的线性映射称为
“\(V\)上的\DefineConcept{线性变换}”
%@see: 《Linear Algebra Done Right (Fourth Eidition)》(Sheldon Axler) P133 5.1
或“\(V\)上的\DefineConcept{算子}(operator)”.
\end{definition}

\begin{definition}\label{definition:线性映射.线性函数}
%@see: 《Linear Algebra Done Right (Fourth Eidition)》(Sheldon Axler) P105 3.108
%@see: 《高等代数(第三版 下册)》(丘维声) P161 定义1
域\(F\)上的线性空间\(V\)到\(F\)的线性映射称为
“\(V\)上的\DefineConcept{线性函数}(linear functional)”.
\end{definition}

\begin{example}
%@see: 《高等代数(第三版 下册)》(丘维声) P107 例1
设\(V\)和\(V'\)都是域\(F\)上的线性空间,
\(0'\)是\(V'\)的零元,
映射\(\vb{A}=V\times\{0'\}\).
我们把\(\vb{A}\)称为
“从\(V\)到\(V'\)的\DefineConcept{零映射}(zero)”,
记作\(\vb0\).
显然零映射\(\vb0\)是线性映射.
\end{example}

\begin{example}
%@see: 《高等代数(第三版 下册)》(丘维声) P107 例2
设\(V\)是域\(F\)上的线性空间,
映射\(\vb{A}\colon V\to V\)
满足\((\forall\alpha\in V)[\vb{A}(\alpha)=\alpha]\).
我们把\(\vb{A}\)称为
“\(V\)上的\DefineConcept{恒等变换}(the \emph{identity operator} on \(V\))”,
记作\(\vb1_V\)或\(\vb{I}\).
显然恒等变换\(\vb1_V\)是\(V\)上的一个线性变换.
\end{example}

\begin{example}
%@see: 《高等代数(第三版 下册)》(丘维声) P107 例3
给定\(k\in F\),
\(F\)上线性空间\(V\)到自身的一个映射\(\vb{k}(\alpha)=k\alpha\),
称为“\(V\)上由\(k\)决定的\DefineConcept{数乘变换}”,
它是\(V\)上的一个线性变换.
当\(k=0\)时,便得到零变换;
当\(k=1\)时,便得到恒等变换.
\end{example}

\begin{example}\label{example.线性映射.同构是线性映射}
设\(V\)与\(V'\)都是域\(F\)上的线性空间,
\(\sigma\)是从\(V\)到\(V'\)的一个同构.
根据同构的定义,\(\sigma\)是映射,且满足\begin{itemize}
	\item \((\forall\alpha,\beta \in V)
	[\sigma(\alpha+\beta)=\sigma(\alpha)+\sigma(\beta)]\),
	\item \((\forall\alpha \in V)
	(\forall k \in F)
	[\sigma(k\alpha)=k\sigma(\alpha)]\),
\end{itemize}
所以\(\sigma\)是从\(V\)到\(V'\)的一个线性映射.
\end{example}

\begin{example}\label{example:线性映射.左乘矩阵是线性映射}
%@see: 《高等代数(第三版 下册)》(丘维声) P107 例4
设\(A\)是域\(F\)上的一个\(s \times n\)矩阵,
令\begin{equation*}
	\vb{A}(\alpha)
	\defeq A\alpha,
	\quad \forall \alpha \in F^n,
\end{equation*}
则\(\vb{A}\)是从\(F^n\)到\(F^s\)的一个线性映射.
\end{example}

\begin{example}
%@see: 《高等代数(第三版 下册)》(丘维声) P107 例5
区间\((a,b)\)上的\(1\)阶连续可导函数族\(C^1(a,b)\)
是实数域\(\mathbb{R}\)上的线性空间\(\mathbb{R}^{(a,b)}\)的一个子空间.
求导运算\(\vb{D}\)是\(C^1(a,b)\)到\(\mathbb{R}^{(a,b)}\)的一个线性映射.
\end{example}

\begin{example}\label{example:线性映射.给定区间上的定积分是线性函数}
%@see: 《高等代数(第三版 下册)》(丘维声) P106
闭区间\([a,b]\)上全体连续函数\(C[a,b]\)对于函数的加法,以及数与函数的数量乘法,
成为实数域\(\mathbb{R}\)上的线性空间.
函数的定积分是从\(C[a,b]\)到\(\mathbb{R}\)的线性映射,
它具有下列性质:\begin{gather*}
	\int_a^b (f(x) + g(x)) \dd{x}
	= \int_a^b f(x) \dd{x} + \int_a^b f(x) \dd{x}, \\
	\int_a^b k f(x) \dd{x}
	= k \int_a^b f(x) \dd{x}.
\end{gather*}
这就说明函数的定积分保持加法、数量乘法两种运算.
函数的定积分是\(C[a,b]\)上的线性函数.
\end{example}

\begin{example}
%@see: 《高等代数(大学高等代数课程创新教材 第一版 下册)》(丘维声) P227 例8
%@see: 《高等代数(大学高等代数课程创新教材 第二版 下册)》(丘维声) P231 例8
设\(W\)是域\(F\)上线性空间\(V\)的一个子空间,
\(V/W\)是\(V\)对\(W\)的商空间.
从\(V\)到\(V/W\)的标准映射\[
	\vb\pi\colon V \to V/W,
	\alpha \mapsto \alpha+W
\]是一个线性映射,
它具有下列性质:\begin{gather*}
	\vb\pi(\alpha+\beta)
	= (\alpha+\beta)+W
	= (\alpha+W) + (\beta+W)
	= \vb\pi(\alpha) + \vb\pi(\beta), \\
	\vb\pi(k\alpha)
	= (k\alpha)+W
	= k(\alpha+W)
	= k\vb\pi(\alpha).
\end{gather*}
\end{example}

\begin{example}
%@see: 《高等代数(第三版 下册)》(丘维声) P112 习题9.1 1.(1)
判断\(K^3\)上的变换\[
	\vb{A}
	\begin{bmatrix}
		x_1 \\ x_2 \\ x_3
	\end{bmatrix}
	= \begin{bmatrix}
		x_1 - x_2 \\
		x_2 + x_3 \\
		x_3^2
	\end{bmatrix}
\]是不是线性变换.
\begin{solution}
取\(\alpha=(0,0,1),
k=2\),
则\[
	k\vb{A}(\alpha)
	= \begin{bmatrix}
		0 \\ 2 \\ 2
	\end{bmatrix}
	\neq
	\vb{A}(k\alpha)
	= \begin{bmatrix}
		0 \\ 2 \\ 4
	\end{bmatrix},
\]
\(\vb{A}\)不满足线性映射的定义,
于是\(\vb{A}\)不是线性变换.
\end{solution}
\end{example}

\begin{example}\label{example:线性映射.右乘矩阵是线性映射}
%@see: 《高等代数(第三版 下册)》(丘维声) P112 习题9.1 2.(1)
设\(A \in M_n(K)\),
令\[
	\vb{A}(X) \defeq X A,
	\quad \forall X \in M_n(K).
\]
判断\(\vb{A}\)是不是\(M_n(K)\)上的线性变换.
\begin{proof}
任取\(\alpha,\beta \in M_n(K),
k \in K\),
则\begin{gather*}
	\vb{A}(\alpha + \beta)
	= (\alpha + \beta) A
	= \alpha A + \beta A
	= \vb{A}(\alpha) + \vb{A}(\beta), \\
	\vb{A}(k \alpha)
	= k \alpha A
	= k \vb{A}(\alpha),
\end{gather*}
由此可见\(\vb{A}\)是\(M_n(K)\)上的线性变换.
\end{proof}
\end{example}

\begin{example}
%@see: 《高等代数(第三版 下册)》(丘维声) P112 习题9.1 2.(2)
设\(B,C \in M_n(K)\),
令\[
	\vb{A}(X) \defeq B X C,
	\quad \forall X \in M_n(K).
\]
判断\(\vb{A}\)是不是\(M_n(K)\)上的线性变换.
\begin{proof}
任取\(\alpha,\beta \in M_n(K),
k \in K\),
则\begin{gather*}
	\vb{A}(\alpha+\beta)
	= (B(\alpha+\beta))C
	= (B\alpha+B\beta)C
	= B \alpha C + B \beta C
	= \vb{A}\alpha + \vb{A}\beta, \\
	\vb{A}(k \alpha)
	= (B(k\alpha))C
	= k (B \alpha C)
	= k \vb{A}\alpha,
\end{gather*}
由此可见\(\vb{A}\)是\(M_n(K)\)上的线性变换.
\end{proof}
\end{example}

\begin{example}
%@see: 《高等代数(第三版 下册)》(丘维声) P112 习题9.1 3.
设\(a \in K\).
判断\(K[x]\)上的变换\[
	\vb{A} \defeq \Set{
		(f(x),f(x+a))
		\given
		f(x) \in K[x]
	}
\]是不是线性变换.
\begin{solution}
任取\(u(x),v(x) \in K[x]\),
任取\(k \in K\),
则\begin{gather*}
	\vb{A}(u(x)+v(x))
	= u(x+a) + v(x+a)
	= \vb{A}u(x) + \vb{A}v(x), \\
	\vb{A}(k u(x))
	= k u(x+a)
	= k \vb{A}u(x),
\end{gather*}
所以\(\vb{A}\)是\(K[x]\)上的线性变换.
\end{solution}
\end{example}

\begin{example}
%@see: 《高等代数(第三版 下册)》(丘维声) P112 习题9.1 4.
%@see: 《高等代数(大学高等代数课程创新教材 第二版 下册)》(丘维声) P237 例2
在正实数集\(\mathbb{R}^+\)上定义加法、数量乘法:\begin{gather*}
	\oplus \defeq \Set{
		((a,b),ab)
		\given
		a,b \in \mathbb{R}^+
	}, \\
	\odot \defeq \Set{
		((k,a),a^k)
		\given
		a \in \mathbb{R}^+,
		k \in \mathbb{R}
	}.
\end{gather*}
设\(a>0\)且\(a\neq1\).
判断从\(\mathbb{R}^+\)到\(\mathbb{R}\)的映射\[
	\vb{A} \defeq \Set{
		(x,\log_a x)
		\given
		x \in \mathbb{R}^+
	}
\]是不是线性映射.
\begin{solution}
任取\(u,v\in\mathbb{R}^+\),
任取\(k\in\mathbb{R}\),
则\begin{gather*}
	\vb{A}(u \oplus v)
	= \log_a (u v)
	= \log_a u + \log_a v
	= \vb{A}u + \vb{A}v, \\
	\vb{A}(k \odot u)
	= \log_a u^k
	= k \log_a u
	= k \vb{A}u,
\end{gather*}
所以\(\vb{A}\)是从\((\mathbb{R}^+,\oplus,\odot)\)到\((\mathbb{R},+,\cdot)\)的线性映射.
\end{solution}
%@see: 《高等代数(第三版 下册)》(丘维声) P81 习题8.1 1.(2)
\end{example}

\begin{example}
%@see: 《高等代数(第三版 下册)》(丘维声) P113 习题9.1 5.
%@see: 《高等代数(大学高等代数课程创新教材 第二版 下册)》(丘维声) P237 例3
设\(V\)是\(K[x,y]\)中所有\(m\)次齐次多项式组成的集合,
它对于多项式的加法,以及数与多项式的乘法,成为数域\(K\)上的一个线性空间.
给定数域\(K\)上的一个2阶矩阵\begin{equation*}
	A = \begin{bmatrix}
		a_{11} & a_{12} \\
		a_{21} & a_{22}
	\end{bmatrix}.
\end{equation*}
定义:\begin{equation*}
	\vb{A} \defeq \Set{
		(f(x,y),f(a_{11}x+a_{21}y,a_{12}x+a_{22}y))
		\given
		f(x,y) \in V
	}.
\end{equation*}
判断\(\vb{A}\)是不是\(V\)上的一个线性变换.
\begin{solution}
显然\(\vb{A}\)是一个映射.
任取\(u(x,y),v(x,y) \in K[x,y]\),
任取\(k \in K\),
则\begin{align*}
	\vb{A}(u(x,y) + v(x,y))
	&= u(a_{11}x+a_{21}y,a_{12}x+a_{22}y) + v(a_{11}x+a_{21}y,a_{12}x+a_{22}y) \\
	&= \vb{A}u(x,y) + \vb{A}v(x,y), \\
	\vb{A}(k u(x,y))
	&= k u(a_{11}x+a_{21}y,a_{12}x+a_{22}y) \\
	&= k \vb{A}u(x,y),
\end{align*}
所以\(\vb{A}\)是\(V\)上的线性变换.
\end{solution}
\end{example}

\begin{example}
%@see: 《高等代数(第三版 下册)》(丘维声) P113 习题9.1 6.
把\(2^m\)个元素的有限域\(F_{2^m}\)看成\(F_2\)上的线性空间.
定义:\begin{equation*}
	\vb{A} \defeq \Set{
		(x,x^2)
		\given
		x \in F_{2m}
	}.
\end{equation*}
判断\(\vb{A}\)是不是\(F_{2^m}\)上的一个线性变换.
%TODO
\end{example}

\begin{example}
%@see: 《高等代数(第三版 下册)》(丘维声) P113 习题9.1 7.
%@see: 《高等代数(大学高等代数课程创新教材 第二版 下册)》(丘维声) P238 例6
定义:\begin{equation*}
	\vb{A} \defeq \Set{
		(f(x),x f(x))
		\given
		f(x) \in K[x]
	}.
\end{equation*}
证明:\begin{itemize}
	\item \(\vb{A}\)是\(K[x]\)上的一个线性变换;
	\item \(\vb{D}\vb{A}-\vb{A}\vb{D}=\vb{I}\),其中\(\vb{D}\)表示求导数.
\end{itemize}
%TODO
\end{example}

\subsection{线性映射的性质}
由于线性映射只比同构映射少了双射这一条件,
因此同构映射的性质中,
只要它的证明没有用到单射和满射的条件,
那么对于线性映射也成立.
\begin{property}\label{theorem:线性映射.线性映射的性质}
%@see: 《高等代数(第三版 下册)》(丘维声) P107
%@see: 《Linear Algebra Done Right (Fourth Eidition)》(Sheldon Axler) P56 3.10
设\(V,V'\)都是域\(F\)上的线性空间,
\(\vb{A}\)是从\(V\)到\(V'\)的线性映射,
则\(\vb{A}\)有下述性质:
\begin{itemize}
	\item \(\vb{A}(0)=0'\),
	其中\(0\)和\(0'\)分别是\(V\)和\(V'\)的零元.

	\item \((\forall\alpha\in V)[\vb{A}(-\alpha)=-\vb{A}(\alpha)]\).

	\item \(\vb{A}(k_1\alpha_1+\dotsb+k_s\alpha_s)
	=k_1\vb{A}(\alpha_1)+\dotsb+k_s\vb{A}(\alpha_s)\).

	\item 如果\(\AutoTuple{\alpha}{s}\)是\(V\)的一个线性相关的向量组,
	则\(\vb{A}(\alpha_1),\dotsc,\vb{A}(\alpha_s)\)是\(V'\)的一个线性相关的向量组;
	但是反之不成立(线性映射可以把线性无关向量组变为线性相关向量组).

	\item 如果\(V\)是有限维的,
	且\(\AutoTuple{\alpha}{s}\)是\(V\)的一个基,
	则对于\(V\)中任一向量\(\alpha=k_1\alpha_1+\dotsb+k_s\alpha_s\),
	有\[
		\vb{A}(\alpha)
		=k_1\vb{A}(\alpha_1)+\dotsb+k_s\vb{A}(\alpha_s).
	\]
	这表明,只要知道了\(V\)的一个基\(\AutoTuple{k}{s}\)在\(\vb{A}\)下的象,
	那么\(V\)中任一向量在\(\vb{A}\)下的象就都确定了.
	或者说,\(n\)维线性空间\(V\)到\(V'\)的线性映射完全被它在\(V\)的一个基上的作用所决定.
\end{itemize}
\end{property}

\subsection{线性映射的存在性}
给了域\(F\)上任意两个线性空间\(V\)和\(V'\),
是否存在\(V\)到\(V'\)的一个线性映射?
如果\(V\)是有限维的,
那么回答是肯定的,
我们有下述结论.
\begin{theorem}\label{theorem:线性映射.线性映射的存在性}
%@see: 《高等代数(第三版 下册)》(丘维声) P108 定理1
%@see: 《Linear Algebra Done Right (Fourth Eidition)》(Sheldon Axler) P54 3.4
设\(V\)和\(V'\)都是域\(F\)上的线性空间,
\(V\)的维数是\(n\),
\(V\)中取一个基\(\AutoTuple{\alpha}{n}\),
\(V'\)中任意取定\(n\)个向量\(\AutoTuple{\gamma}{n}\),
令\[
	\vb{A}\colon V\to V',
	\alpha=\sum_{i=1}^n k_i\alpha_i
	\mapsto
	\sum_{i=1}^n k_i\gamma_i,
\]
则\(\vb{A}\)是\(V\)到\(V'\)的一个线性映射,
且\(\vb{A}(\alpha_i)=\gamma_i\ (i=1,2,\dotsc,n)\).
\begin{proof}
由于\(\AutoTuple{\alpha}{n}\)是\(V\)的一个基,
因此\(\alpha\)表示成\(\AutoTuple{\alpha}{n}\)的线性组合的方式唯一,
从而\(\vb{A}\)是从\(V\)到\(V'\)的一个映射.
在\(V\)中任取两个向量\[
	\alpha = \sum_{i=1}^n a_i \alpha_i,
	\qquad
	\beta = \sum_{i=1}^n b_i \alpha_i,
\]
则\begin{align*}
	\vb{A}(\alpha + \beta)
	&= \vb{A}\left( \sum_{i=1}^n (a_i + b_i) \alpha_i \right) \\
	&= \sum_{i=1}^n (a_i + b_i) \gamma_i \\
	&= \sum_{i=1}^n a_i \gamma_i
		+ \sum_{i=1}^n b_i \gamma_i \\
	&= \vb{A}(\alpha) + \vb{A}(\beta), \\
	\vb{A}(k \alpha)
	&= \vb{A}\left( \sum_{i=1}^n (k a_i) \alpha_i \right) \\
	&= \sum_{i=1}^n (k a_i) \gamma_i \\
	&= k \sum_{i=1}^n a_i \gamma_i \\
	&= k \vb{A}(\alpha),
	\quad k \in F.
\end{align*}
因此\(\vb{A}\)是从\(V\)到\(V'\)的一个线性映射.
显然有\[
	\vb{A}(\alpha_i)
	= \vb{A}(0\alpha_1 + \dotsb + 0\alpha_{i-1}
		+ 1\alpha_i + 0\alpha_{i+1} + \dotsb + 0\alpha_n)
	= \gamma_i,
\]
其中\(i=1,2,\dotsc,n\).
\end{proof}
\end{theorem}
\begin{remark}
由于\(V\)到\(V'\)的线性映射完全被它在\(V\)上的一个基上的作用所决定,
因此\cref{theorem:线性映射.线性映射的存在性} 中满足\begin{equation*}
	\vb{A}(\alpha_i)=\gamma_i
	\quad(i=1,2,\dotsc,n)
\end{equation*}
的线性映射是唯一的.
\end{remark}

\subsection{投影的概念}
\begin{definition}\label{definition:线性映射.平行于某个子空间在另一个子空间的投影}
%@see: 《高等代数(第三版 下册)》(丘维声) P108 定理2
设\(V\)是域\(F\)上的一个线性空间,
\(U,W\)是\(V\)的两个子空间,
且\(V=U \DirectSum W\).
把\[
	\vb{P}_U
	\defeq
	\Set{
		(\alpha,\alpha_1)
		\given
		\alpha \in V,
		\alpha_1 \in U,
		(\exists \alpha_2 \in W)
		[\alpha=\alpha_1+\alpha_2]
	}
\]
称为“\(V\)平行于\(W\)在\(U\)上的\DefineConcept{投影}”.
\end{definition}
\begin{remark}
\cref{definition:线性映射.平行于某个子空间在另一个子空间的投影}
强调“平行于\(W\)”
是因为从\cref{example:线性空间.子空间.直和.例1}
可以知道\(\alpha_1\)的取值是由\(U,W\)以及\(\alpha\)共同决定的.
\end{remark}
\begin{remark}
类似地,可以定义\[
	\vb{P}_W
	\defeq
	\Set{
		(\alpha,\alpha_2)
		\given
		\alpha \in V,
		\alpha_2 \in W,
		(\exists \alpha_1 \in U)
		[\alpha=\alpha_1+\alpha_2]
	},
\]
并称之为“\(V\)平行于\(U\)在\(W\)上的投影”.
\end{remark}

\begin{theorem}\label{theorem:线性映射.投影是线性变换}
%@see: 《高等代数(第三版 下册)》(丘维声) P108 定理2
设\(V\)是域\(F\)上的一个线性空间,
\(U,W\)是\(V\)的两个子空间,
且\(V=U \DirectSum W\),
则\(V\)平行于\(W\)在\(U\)上的投影
\(\vb{P}_U\)是\(V\)上的一个线性变换,
且满足\[
%@see: 《高等代数(第三版 下册)》(丘维声) P108 (7)
	\vb{P}_U(\alpha)
	=\left\{ \begin{array}{ll}
		\alpha, & \alpha\in U, \\
		0, & \alpha\in W.
	\end{array} \right.
\]
\begin{proof}
由于\(V = U \DirectSum W\),
因此\(V\)中任意一个向量\(\alpha\)表示成\(U\)的一个向量与\(W\)的一个向量之和的方式唯一,
关系\(\vb{P}_U\)是单值的,
于是\(\vb{P}_U\)是从\(V\)到\(V\)的一个映射.
任取\(V\)中两个向量\[
	\alpha = \alpha_1 + \alpha_2,
	\qquad
	\beta = \beta_1 + \beta_2,
\]
其中\(\alpha_1,\beta_1 \in U,
\alpha_2,\beta_2 \in W\),
则\[
	\alpha_1 + \beta_1
	\in U,
	\qquad
	\alpha_2 + \beta_2
	\in W,
\]
从而\begin{align*}
	\vb{P}_U(\alpha + \beta)
	&= \vb{P}_U((\alpha_1 + \beta_1) + (\alpha_2 + \beta_2)) \\
	&= \alpha_1 + \beta_1 \\
	&= \vb{P}_U(\alpha) + \vb{P}_U(\beta), \\
	\vb{P}_U(k \alpha)
	&= \vb{P}_U(k \alpha_1 + k \alpha_2) \\
	&= k \alpha_1 \\
	&= k \vb{P}_U(\alpha),
	\quad k \in F,
\end{align*}
因此\(\vb{P}_U\)是\(V\)上的一个线性变换.

如果\(\alpha \in U\),
则\(\alpha = \alpha + 0\),
从而\(\vb{P}_U(\alpha) = \alpha\).

如果\(\alpha \in W\),
则\(\alpha = 0 + \alpha\),
从而\(\vb{P}_U(\alpha) = 0\).

设\(V\)上的线性变换\(\vb{A}\)也满足投影的定义,
任取\(\alpha \in V\),
设\[
	\alpha = \alpha_1 + \alpha_2,
	\quad
	\alpha_1 \in U,
	\alpha_2 \in W,
\]
则\[
	\vb{A}(\alpha)
	= \vb{A}(\alpha_1 + \alpha_2)
	= \vb{A}(\alpha_1) + \vb{A}(\alpha_2)
	= \alpha_1 + 0
	= \alpha_1
	= \vb{P}_U(\alpha),
\]
因此\(\vb{A} = \vb{P}_U\).
\end{proof}
\end{theorem}

\cref{theorem:线性映射.投影是线性变换} 告诉我们,
如果线性空间\(V\)可以分解成两个子空间的直和\(V = U \DirectSum W\),
那么\(V\)在子空间\(U\)上的投影\(\vb{P}_U\)
和\(V\)在子空间\(W\)上的投影\(\vb{P}_W\)
就都是\(V\)上的线性变换,
并且有\begin{equation*}
	\vb{P}_U(\alpha)
	=\left\{ \begin{array}{ll}
		\alpha, & \alpha\in U, \\
		0, & \alpha\in W,
	\end{array} \right.
	\qquad
	\vb{P}_W(\alpha)
	=\left\{ \begin{array}{ll}
		\alpha, & \alpha\in W, \\
		0, & \alpha\in U.
	\end{array} \right.
\end{equation*}
投影是非常重要的一类线性变换.

为求简便,对于任意一个线性映射\(\vb{A}\),任意一个向量\(\alpha\),
以后我们都用\(\vb{A}\alpha\)代替\(\vb{A}(\alpha)\).

\subsection{幂等变换,正交变换}
\begin{definition}
%@see: 《高等代数(第三版 下册)》(丘维声) P109
线性变换\(\vb{A}\)如果满足\(\vb{A}^2=\vb{A}\),
则称“\(\vb{A}\)是\DefineConcept{幂等变换}”.
\end{definition}

\begin{definition}
%@see: 《高等代数(第三版 下册)》(丘维声) P109
两个线性变换\(\vb{A},\vb{B}\)
如果满足\(\vb{A} \vb{B}=\vb{B} \vb{A}=\vb0\),
则称“\(\vb{A}\)与\(\vb{B}\)是\DefineConcept{正交的}”.
\end{definition}

\subsection{线性映射的加法、纯量乘法}
\begin{definition}
%@see: 《Linear Algebra Done Right (Fourth Eidition)》(Sheldon Axler) P52 3.2
%@see: 《高等代数(第三版 下册)》(丘维声) P110
设\(V,V'\)都是域\(F\)上的线性空间.

\begin{itemize}
	\item 从\(V\)到\(V'\)的所有线性映射组成的集合,
	% 称为“域\(F\)上从线性空间\(V\)到\(V'\)的\DefineConcept{线性映射空间}”,
	记作\(\Hom(V,V')\).

	\item \(V\)上的所有线性变换组成的集合,
	% 称为“域\(F\)上线性空间\(V\)上的\DefineConcept{线性变换空间}”,
	记作\(\Hom(V,V)\)\footnote{
		% 《高等代数(第四版)》(谢启鸿 姚慕生)
		在有的书上,\(\Hom(V,V')\)和\(\Hom(V,V)\)
		分别记作\(\mathcal{L}(V,V')\)和\(\mathcal{L}(V) = \mathcal{L}(V,V)\).
	}.
\end{itemize}
\end{definition}

\begin{definition}
%@see: 《高等代数(第三版 下册)》(丘维声) P110 命题5
%@see: 《Linear Algebra Done Right (Fourth Eidition)》(Sheldon Axler) P55 3.5
设\(V,V'\)都是域\(F\)上的线性空间.
对于\(\forall\vb{A},\vb{B}\in\Hom(V,V'),
\forall\alpha\in V,
\forall k\in F\),
定义:\begin{gather*}
	%@see: 《高等代数(第三版 下册)》(丘维声) P110 (9)
	(\vb{A}+\vb{B})\alpha
	\defeq
	\vb{A}\alpha+\vb{B}\alpha, \\
	%@see: 《高等代数(第三版 下册)》(丘维声) P110 (10)
	(k\vb{A})\alpha
	\defeq
	k(\vb{A}\alpha),
\end{gather*}
把\(\vb{A}+\vb{B}\)称为“\(\vb{A}\)与\(\vb{B}\)的\DefineConcept{和}(sum)”,
把\(k\vb{A}\)称为“\(k\)与\(\vb{A}\)的\DefineConcept{纯量乘积}(product)”.
\end{definition}

\begin{proposition}
%@see: 《高等代数(第三版 下册)》(丘维声) P110 命题5
设\(\vb{A},\vb{B}\)都是域\(F\)上线性空间\(V\)到\(V'\)的线性映射,
\(k\in F\),
则\begin{itemize}
	\item \(\vb{A}\)与\(\vb{B}\)的和\(\vb{A}+\vb{B}\)是\(V\)到\(V'\)的线性映射,
	\item \(k\)与\(\vb{A}\)的纯量乘积\(k\vb{A}\)是\(V\)到\(V'\)的线性映射.
\end{itemize}
\begin{proof}
显然\(\vb{A}+\vb{B}\)是从\(V\)到\(V'\)的一个映射.
对于任意\(\alpha,\beta \in V,
l \in F\),
有\begin{align*}
	(\vb{A}+\vb{B})(\alpha+\beta)
	&= \vb{A}(\alpha+\beta) + \vb{B}(\alpha+\beta) \\
	&= \vb{A}\alpha + \vb{A}\beta + \vb{B}\alpha + \vb{B}\beta \\
	&= (\vb{A}+\vb{B})\alpha + (\vb{A}+\vb{B})\beta, \\
	(\vb{A}+\vb{B})(l\alpha)
	&= \vb{A}(l\alpha) + \vb{B}(l\alpha) \\
	&= l\vb{A}\alpha + l\vb{B}\alpha \\
	&= l(\vb{A}\alpha + \vb{B}\alpha) \\
	&= l(\vb{A}+\vb{B}) \alpha,
\end{align*}
因此\(\vb{A}+\vb{B}\)是从\(V\)到\(V'\)的线性映射.

同理可证\(k\vb{A}\)是从\(V\)到\(V'\)的线性映射.
\end{proof}
\end{proposition}

\begin{proposition}
%@see: 《高等代数(第三版 下册)》(丘维声) P110
%@see: 《Linear Algebra Done Right (Fourth Eidition)》(Sheldon Axler) P55 3.6
设\(V\)和\(V'\)都是域\(F\)上的线性空间,
则从\(V\)到\(V'\)的所有线性映射组成的集合\(\Hom(V,V')\),
对线性映射的加法与纯量乘法,
成为域\(F\)上的一个线性空间.
\end{proposition}
\begin{corollary}
%@see: 《高等代数(第三版 下册)》(丘维声) P111
设\(V\)是域\(F\)上的线性空间,
则在\(V\)上的所有线性变换组成的集合\(\Hom(V,V)\),
对线性映射的加法与纯量乘法,
成为域\(F\)上的一个线性空间.
\end{corollary}

\subsection{线性映射的乘法}
\begin{definition}
%@see: 《Linear Algebra Done Right (Fourth Eidition)》(Sheldon Axler) P55 3.7
设\(V,U,W\)都是域\(F\)上的线性空间.
对于\(\forall\vb{A}\in\Hom(V,U),
\forall\vb{B}\in\Hom(U,W),
\forall\alpha \in V\),
定义:\begin{equation*}
	(\vb{B}\vb{A})\alpha
	\defeq (\vb{B} \circ \vb{A})\alpha
	= \vb{B}(\vb{A}\alpha),
\end{equation*}
把\(\vb{B}\vb{A}\)称为“\(\vb{B}\)与\(\vb{A}\)的\DefineConcept{积}”.
\end{definition}

\begin{proposition}
%@see: 《高等代数(第三版 下册)》(丘维声) P109 命题3
设\(V,U,W\)都是域\(F\)上的线性空间,
\(\vb{A}\)是\(V\)到\(U\)的一个线性映射,
\(\vb{B}\)是\(U\)到\(W\)的一个线性映射,
则\(\vb{B}\vb{A}\)是\(V\)到\(W\)的一个线性映射.
\begin{proof}
显然\(\vb{B}\vb{A}\)是从\(V\)到\(W\)的一个映射.
任取\(\alpha,\beta \in V,
k \in F\),
有\begin{align*}
	(\vb{B}\vb{A})(\alpha+\beta)
	&= \vb{B}(\vb{A}(\alpha+\beta)) \\
	&= \vb{B}(\vb{A}\alpha+\vb{A}\beta), \\
	(\vb{B}\vb{A})(k\alpha)
	&= \vb{B}(\vb{A}(k \alpha)) \\
	&= \vb{B}(k \vb{A}\alpha) \\
	&= k(\vb{B}(\vb{A}\alpha)) \\
	&= k((\vb{B}\vb{A}) \alpha),
\end{align*}
因此\(\vb{B}\vb{A}\)是从\(V\)到\(W\)的一个线性映射.
\end{proof}
\end{proposition}

\begin{proposition}\label{theorem:线性映射.零映射与任意线性映射之积是零映射}
设\(V,W\)都是域\(F\)上的线性空间,
\(\vb{A}\)是从\(V\)到\(W\)的一个线性映射,
则\(\vb{A}\vb0\)和\(\vb0\vb{A}\)都是零映射.
\begin{proof}
对于任意向量\(\alpha\)
和任意向量\(\beta\),
有\begin{gather*}
	(\vb{A}\vb0)(\alpha)
	= \vb{A}(\vb0(\alpha))
	% 利用零映射的定义
	= \vb{A}(0)
	%\cref{theorem:线性映射.线性映射的性质}
	= 0, \\
	(\vb0\vb{A})(\beta)
	= \vb0(\vb{A}(\beta))
	% 利用零映射的定义
	= 0.
	\qedhere
\end{gather*}
\end{proof}
\end{proposition}

\begin{proposition}\label{theorem:线性映射.线性映射的乘法适合结合律}
%@see: 《高等代数(第三版 下册)》(丘维声) P110
%@see: 《Linear Algebra Done Right (Fourth Eidition)》(Sheldon Axler) P56 3.8
线性映射的乘法适合结合律,
即\begin{equation*}
	(\forall\vb{A}\in\Hom(V_2,V_1))
	(\forall\vb{B}\in\Hom(V_3,V_2))
	(\forall\vb{C}\in\Hom(V_4,V_3))
	[
		(\vb{A}\vb{B})\vb{C}
		= \vb{A}(\vb{B}\vb{C})
	].
\end{equation*}
\begin{proof}
因为线性映射是映射,
所以由\cref{example:映射.映射的复合适合结合律} 可知,
映射的乘法适合结合律.
\end{proof}
\end{proposition}
\begin{proposition}\label{theorem:线性映射.线性映射的乘法不适合交换律}
%@see: 《高等代数(第三版 下册)》(丘维声) P110
%@see: 《Linear Algebra Done Right (Fourth Eidition)》(Sheldon Axler) P56 3.9
线性映射的乘法不适合交换律.
\begin{proof}
下面举出反例.
\def\MyPolynomialRing{\mathbb{R}[x]}%
\def\MyLinearMapSpace{\Hom(\MyPolynomialRing,\MyPolynomialRing)}%
设\(\vb{D}\in\MyLinearMapSpace\)表示对多项式求导数,
\(\vb{T}\in\MyLinearMapSpace\)表示给多项式乘以\(x^2\)因式,
那么对于\(\forall p(x) \in \MyPolynomialRing\)
有\begin{equation*}
	(\vb{T}\vb{D})p(x)
	= x^2 p'(x),
	\qquad
	(\vb{D}\vb{T})p(x)
	= x^2 p'(x) + 2x p(x).
\end{equation*}
因此\(\vb{T}\vb{D} \neq \vb{D}\vb{T}\).
\end{proof}
\end{proposition}
\begin{definition}
设\(V\)是域\(F\)上的线性空间,
\(\vb{A},\vb{B}\)都是\(V\)上的线性变换.
如果\(\vb{A}\vb{B}=\vb{B}\vb{A}\),
则称“\(\vb{A}\)与\(\vb{B}\) \DefineConcept{可交换}”.
\end{definition}

由\cref{theorem:线性映射.线性映射的乘法适合结合律} 可知,
线性变换的乘法满足结合律:\begin{equation*}
	(\forall\vb{A},\vb{B},\vb{C}\in\Hom(V,V))
	[
		(\vb{A}\vb{B})\vb{C}
		= \vb{A}(\vb{B}\vb{C})
	].
\end{equation*}
恒等变换\(\vb{I}\)满足\begin{equation*}
	(\forall\vb{A}\in\Hom(V,V))
	[\vb{I}\vb{A}=\vb{A}\vb{I}=\vb{A}].
\end{equation*}
容易验证,线性变换的乘法对于加法还满足左、右分配律:\begin{gather*}
	(\forall\vb{A},\vb{B},\vb{C}\in\Hom(V,V))
	[\vb{A}(\vb{B}+\vb{C})=\vb{A}\vb{B}+\vb{A}\vb{C}], \\
	(\forall\vb{A},\vb{B},\vb{C}\in\Hom(V,V))
	[(\vb{B}+\vb{C})\vb{A}=\vb{B}\vb{A}+\vb{C}\vb{A}].
\end{gather*}
综上所述,\(\Hom(V,V)\)的加法与乘法满足环定义的6条运算法则,
并且\(\vb{I}\)是\(\Hom(V,V)\)的乘法单位元,
因此\(\Hom(V,V)\)对于加法和乘法成为一个有单位元的环.
容易验证,
线性变换的乘法与纯量乘法满足\begin{equation*}
%@see: 《高等代数(第三版 下册)》(丘维声) P111 (11)
	(\forall k\in F)
	(\forall\vb{A},\vb{B}\in\Hom(V,V))
	[
		k(\vb{A}\vb{B})
		=(k\vb{A})\vb{B}
		=\vb{A}(k\vb{B})
	].
\end{equation*}

\begin{definition}
%@see: 《Linear Algebra Done Right (Fourth Eidition)》(Sheldon Axler) P82 3.59
%@see: 《Linear Algebra Done Right (Fourth Eidition)》(Sheldon Axler) P82 3.61
%@see: 《线性代数》(李炯生 查建国) P184
%@see: 《线性代数》(李炯生 查建国) P196
%@see: 《高等代数学(第四版)》(谢启鸿 姚慕生 吴泉水) P184
给定\(\vb{A}\in\Hom(V,W)\).
如果存在\(\vb{B}\in\Hom(W,V)\),
使得\(\vb{B}\vb{A}=\vb{I}_V\)且\(\vb{A}\vb{B}=\vb{I}_W\),
其中\(\vb{I}_V\)是\(V\)上的恒等变换,
\(\vb{I}_W\)是\(W\)上的恒等变换,
则称“\(\vb{A}\) \DefineConcept{可逆}(invertible)”
或“\(\vb{A}\)有\DefineConcept{可逆性}(invertibility)”,
称“\(\vb{B}\)是\(\vb{A}\)一个的\DefineConcept{逆}(\(\vb{B}\) is the \emph{inverse} of \(\vb{A}\))”,
记作\(\vb{A}^{-1}\).
%@see: https://mathworld.wolfram.com/InvertibleLinearMap.html
\end{definition}
\begin{proposition}\label{theorem:线性映射.可逆线性映射有唯一逆}
%@see: 《Linear Algebra Done Right (Fourth Eidition)》(Sheldon Axler) P82 3.60
可逆线性映射有唯一逆.
\begin{proof}
假设\(\vb{B}_1,\vb{B}_2\in\Hom(W,V)\)都是可逆线性映射\(\vb{A}\in\Hom(V,W)\)的逆,
那么\begin{equation*}
	\vb{B}_1 = \vb{B}_1 \vb{I}
	= \vb{B}_1 (\vb{A} \vb{B}_2)
	= (\vb{B}_1 \vb{A}) \vb{B}_2
	= \vb{I} \vb{B}_2
	= \vb{B}_2.
	\qedhere
\end{equation*}
\end{proof}
\end{proposition}

\begin{proposition}\label{theorem:线性映射.零映射不可逆}
零映射不是可逆线性映射.
\begin{proof}
由\cref{theorem:线性映射.零映射与任意线性映射之积是零映射} 可知,
零映射与任意线性映射之积是零映射,而不是恒等变换,
所以零映射不可逆.
\end{proof}
\end{proposition}

\begin{proposition}\label{theorem:线性映射.可逆线性映射是同构}
%@see: 《Linear Algebra Done Right (Fourth Eidition)》(Sheldon Axler) P83 3.63
设\(V,V'\)都是域\(F\)上的线性空间,
映射\(\sigma\colon V \to V'\),
则\[
	\text{$\sigma$是可逆线性映射}
	\iff
	\text{$\sigma$是同构}.
\]
\begin{proof}
假设\(\sigma\)是可逆线性映射,
则存在线性映射\(\rho\colon V' \to V\)使得\(\rho\sigma = \vb{I}_V\).
由\cref{example:映射.可逆映射是双射} 可知
\(\sigma\)是双射,
于是\(\sigma\)是同构.

假设\(\sigma\)是同构,
由于\hyperref[theorem:线性空间.线性空间的同构的逆是同构]{线性空间的同构的逆映射是同构},
所以\(\rho \defeq \sigma^{-1}\)是同构,
%\cref{example.线性映射.同构是线性映射}
从而\(\rho\)是线性映射,
因此\(\sigma\)是可逆线性映射.
\end{proof}
\end{proposition}

\begin{proposition}%\label{theorem:线性映射.可逆线性映射的存在条件}
%@see: 《高等代数(第三版 下册)》(丘维声) P110
%@see: 《线性代数》(李炯生 查建国) P196 定理5.3.2
设\(V,V'\)都是域\(F\)上有限维线性空间.
\(V\)到\(V'\)的可逆线性映射存在的充分必要条件是
\(\dim V=\dim V'\).
\begin{proof}
由\cref{theorem:线性映射.可逆线性映射是同构} 可知,
\(V\)到\(V'\)的可逆线性映射存在,
当且仅当\(V\)与\(V'\)同构.
由\cref{theorem:线性空间的同构.线性空间同构的充分必要条件} 可知,
\(V\)与\(V'\)同构的充分必要条件是它们的维数相同.
\end{proof}
\end{proposition}

\begin{proposition}%\label{theorem:线性映射.可逆线性映射的逆是可逆线性映射}
%@see: 《高等代数(第三版 下册)》(丘维声) P110 命题4
%@see: 《高等代数(第三版 下册)》(丘维声) P107 例6
%@see: 《线性代数》(李炯生 查建国) P196 定理5.3.3
设\(V,W\)都是域\(F\)上的线性空间,
\(\vb{A}\)是从\(V\)到\(W\)的一个线性映射.
如果\(\vb{A}\)可逆,
则\(\vb{A}\)的逆\(\vb{A}^{-1}\)是从\(W\)到\(V\)的一个线性映射.
\begin{proof}
直接有\begin{align*}
	&\text{$\vb{A}$是从$V$到$W$的可逆线性映射} \\
	&\iff \text{$\vb{A}$是从$V$到$W$的同构}
		\tag{\cref{theorem:线性映射.可逆线性映射是同构}} \\
	&\implies \text{$\vb{A}^{-1}$是从$W$到$V$的同构}
		\tag{\cref{theorem:线性空间.线性空间的同构的逆是同构}} \\
	&\implies \text{$\vb{A}^{-1}$是从$W$到$V$的可逆线性映射}.
	\qedhere
\end{align*}
\end{proof}
\end{proposition}

\begin{proposition}%\label{theorem:线性映射.可逆线性映射之积是可逆线性映射}
%@see: 《线性代数》(李炯生 查建国) P196 定理5.3.4
设\(U,V,W\)都是域\(F\)上的线性空间,
\(\vb{A}\colon U \to V,
\vb{B}\colon V \to W\)都是可逆线性映射,
则\(\vb{B}\vb{A}\)可逆.
\begin{proof}
由\cref{theorem:线性空间.线性空间的同构的复合是同构} 立即可得.
\end{proof}
\end{proposition}

\begin{proposition}\label{theorem:线性映射.非零数与可逆线性映射之积是可逆线性映射}
设\(V,W\)都是域\(F\)上的线性空间,
\(\lambda \in F\)且\(\lambda\neq0\).
如果\(\vb{A}\colon V \to W\)是可逆线性映射,
那么\(\lambda\vb{A}\)也是可逆线性映射.
\begin{proof}
设\(\vb{B}\colon W \to V\)是\(\vb{A}\)的逆,
则\(\vb{B}\vb{A} = \vb{I}_V\)且\(\vb{A}\vb{B} = \vb{I}_W\),
易知\((\lambda^{-1}\vb{B})(\lambda\vb{A}) = \vb{I}_V\)且\((\lambda\vb{A})(\lambda^{-1}\vb{B}) = \vb{I}_W\),
因此\(\lambda^{-1}\vb{B}\)是\(\lambda\vb{A}\)的逆,
\(\lambda\vb{A}\)可逆.
\end{proof}
\end{proposition}

\begin{example}
举例说明:两个可逆线性映射之和可能不是可逆线性映射.
\begin{solution}
设\(\vb{A}\)是一个可逆线性映射,
那么,由\cref{theorem:线性映射.非零数与可逆线性映射之积是可逆线性映射} 可知,
\((-\vb{A})\)也是可逆线性映射,
%\cref{theorem:线性映射.零映射不可逆}
但是,\(\vb{A}\)和\((-\vb{A})\)之和\(\vb{A}+(-\vb{A}) = \vb0\)是零映射,而不是可逆线性映射.
\end{solution}
\end{example}

%@see: 《高等代数(第三版 下册)》(丘维声) P111
%@see: 《Linear Algebra Done Right (Fourth Eidition)》(Sheldon Axler) P137 5.13
%@see: 《Linear Algebra Done Right (Fourth Eidition)》(Sheldon Axler) P137 5.14
由于线性变换的乘法满足结合律,
因此可以定义线性变换\(\vb{A}\)的正整数指数幂:\begin{equation*}
%@see: 《高等代数(第三版 下册)》(丘维声) P111 (13)
	\vb{A}^m
	\defeq
	\underbrace{\vb{A}\circ\vb{A}\circ\dotsm\circ\vb{A}}_{\text{$m$个}}.
\end{equation*}
还可以定义\(\vb{A}\)的零次幂:\begin{equation*}
%@see: 《高等代数(第三版 下册)》(丘维声) P111 (14)
	\vb{A}^0
	\defeq
	\vb{I}.
\end{equation*}
容易验证:
对于\(\forall m,n\in\mathbb{N}\),
有\begin{gather*}
%@see: 《高等代数(第三版 下册)》(丘维声) P111 (15)
	\vb{A}^m\vb{A}^n=\vb{A}^{m+n}, \\
	(\vb{A}^m)^n=\vb{A}^{mn}.
\end{gather*}
当\(\vb{A}\)可逆时,可以定义:\begin{equation*}
%@see: 《高等代数(第三版 下册)》(丘维声) P111 (16)
	\vb{A}^{-m}
	\defeq
	(\vb{A}^{-1})^m,
	\quad m\in\mathbb{N}.
\end{equation*}

设\(f(x)=a_0+a_1 x+\dotsb+a_m x^m\)是域\(F\)上的一元多项式,
\(x\)用\(V\)上的线性变换\(\vb{A}\)代入,
得\begin{equation*}
%@see: 《高等代数(第三版 下册)》(丘维声) P111 (17)
	f(\vb{A}) \defeq a_0\vb{I}+a_1\vb{A}+\dotsb+a_m\vb{A}^m.
\end{equation*}
显然,\(f(\vb{A})\)仍是\(V\)上的一个线性变换,
称“\(f(\vb{A})\)是\(\vb{A}\)的一个多项式”.
容易验证:线性变换\(\vb{A}\)的任意两个多项式
\(f(\vb{A})\)与\(g(\vb{B})\)是可交换的,
即\begin{equation*}
%@see: 《高等代数(第三版 下册)》(丘维声) P111 (18)
	f(\vb{A}) g(\vb{A})
	=g(\vb{A}) f(\vb{A}).
\end{equation*}
把\(V\)上线性变换\(\vb{A}\)的所有多项式组成的集合记作\(F[\vb{A}]\).
容易验证:
\begin{itemize}
	\item \(F[\vb{A}]\)对于线性变换的减法、乘法都封闭,
	从而\(F[\vb{A}]\)是环\(\Hom(V,V)\)的一个子环;

	\item \(F[\vb{A}]\)是交换环;

	\item \(\vb{I}\in F[\vb{A}]\).
\end{itemize}
\(F[\vb{A}]\)中所有数乘变换组成的集合是\(F[\vb{A}]\)的一个子环,
并且域\(F\)与这个子环同构,
从而\(F[\vb{A}]\)可看成是\(F\)的一个扩环.
于是根据一元多项式环\(F[x]\)的通用性质,
\(x\)可用\(F[\vb{A}]\)的任一元素代入,
从\(F[x]\)的有关加法和乘法的等式
得到\(F[\vb{A}]\)中有关加法和乘法的相应等式.

\subsection{代数,代数的维数}
\begin{definition}
%@see: 《高等代数(第三版 下册)》(丘维声) P111 定义2
%@see: 《高等代数学(第四版)》(谢启鸿 姚慕生 吴泉水) P189 定义4.2.2
设\(A\)是域\(F\)上的线性空间,
\(A\)对加法和乘法成为一个有单位元的环,
且\[
	% 乘法与数乘的相容性
	(\forall k\in F)
	(\forall\alpha,\beta\in A)
	[
		k(\alpha\beta)
		=(k\alpha)\beta
		=\alpha(k\beta)
	],
\]
则称“\(A\)是域\(F\)上的一个\DefineConcept{代数}”,
把\(A\)的乘法单位元称为“\(A\)的\DefineConcept{恒等元}”,
把\(A\)的维数\(\dim A\)称为“代数\(A\)的\DefineConcept{维数}”.
\end{definition}
\begin{remark}
我们需要注意区分两个概念:代数\(A\)的乘法单位元,域\(F\)的乘法单位元.
\end{remark}

\begin{example}
域\(F\)上线性空间\(V\)上的所有线性变换组成的集合\(\Hom(V,V)\),
对于线性变换的加法、乘法与纯量乘法,
成为域\(F\)上的一个代数.
\end{example}

\begin{example}
域\(F\)上所有\(n\)阶矩阵组成的集合\(M_n(F)\),
对于矩阵的加法、乘法与数量乘法,
成为域\(F\)上的一个代数.
\end{example}

\subsection{线性变换的性质}
利用线性变换的运算,
我们可以研究线性变换的性质.

\begin{proposition}
%@see: 《高等代数(第三版 下册)》(丘维声) P109
设\(V\)是域\(F\)上的一个线性空间,
\(U,W\)是\(V\)的两个子空间,
且\(V=U \DirectSum W\),
则\(V\)平行于\(W\)在\(U\)上的投影\(\vb{P}_U\)
和\(V\)平行于\(U\)在\(W\)上的投影\(\vb{P}_W\)
满足\begin{gather*}
	%@see: 《高等代数(第三版 下册)》(丘维声) P109 (8)
	\vb{P}_U^2
	=\vb{P}_U, \\
	\vb{P}_U \vb{P}_W
	=\vb0, \\
	\vb{P}_W \vb{P}_U
	=\vb0, \\
	\vb{P}_W^2
	=\vb{P}_W.
\end{gather*}
\begin{proof}
\(V\)在子空间\(U\)上的投影\(\vb{P}_U\)
和\(V\)在子空间\(W\)上的投影\(\vb{P}_W\)
都是从\(V\)到\(V\)的映射,
对于\(\forall\alpha_1\in U,
\forall\alpha_2\in W\),
记\(\alpha=\alpha_1+\alpha_2\),
有\begin{gather*}
	\vb{P}_U(\vb{P}_U(\alpha))
	=\vb{P}_U(\alpha_1)
	=\alpha_1
	=\vb{P}_U(\alpha), \\
	\vb{P}_U(\vb{P}_W(\alpha))
	=\vb{P}_U(\alpha_2)
	=0, \\
	\vb{P}_W(\vb{P}_U(\alpha))
	=\vb{P}_W(\alpha_1)
	=0, \\
	\vb{P}_W(\vb{P}_W(\alpha))
	=\vb{P}_W(\alpha_2)
	=\alpha_2
	=\vb{P}_W(\alpha).
	\qedhere
\end{gather*}
\end{proof}
\end{proposition}
\begin{remark}
%@see: 《高等代数(第三版 下册)》(丘维声) P109
这就说明:
% 前提
如果\(V = U \DirectSum W\),
% 结论1
那么\(\vb{P}_U,\vb{P}_W\)都是幂等变换,
% 结论2
而且\(\vb{P}_U\)与\(\vb{P}_W\)是正交的.
\end{remark}

\begin{proposition}
%@see: 《高等代数(第三版 下册)》(丘维声) P112
设\(V\)是域\(F\)上的一个线性空间,
\(U,W\)是\(V\)的两个子空间,
且\(V=U \DirectSum W\),
则\(V\)平行于\(W\)在\(U\)上的投影\(\vb{P}_U\)
和\(V\)平行于\(U\)在\(W\)上的投影\(\vb{P}_W\)
满足\[
%@see: 《高等代数(第三版 下册)》(丘维声) P111 (19)
	\vb{P}_U+\vb{P}_W=\vb{I}.
\]
\begin{proof}
对于\(\forall\alpha_1\in U,
\alpha_2\in W\),
令\(\alpha=\alpha_1+\alpha_2\),
则\[
	(\vb{P}_U+\vb{P}_W)\alpha
	=\vb{P}_U\alpha+\vb{P}_W\alpha
	=\alpha_1+\alpha_2
	=\vb{I}\alpha,
\]
因此\(\vb{P}_U+\vb{P}_W=\vb{I}\).
\end{proof}
\end{proposition}
\begin{remark}
这就说明:
如果\(V=U \DirectSum W\),
则投影\(\vb{P}_U\)与\(\vb{P}_W\)的和等于恒等变换\(\vb{I}\).
\end{remark}

\begin{example}
%@see: 《高等代数(第三版 下册)》(丘维声) P113 习题9.1 8.
%@see: 《高等代数(大学高等代数课程创新教材 第二版 下册)》(丘维声) P238 例7
设\(V\)是域\(F\)上的一个线性空间,
\(\AutoTuple{\alpha}{n}\)是\(V\)的一个基,
\(\vb{A}\)是\(V\)上的一个线性变换.
证明:\(\vb{A}\)可逆,
当且仅当\(\AutoTuple{\vb{A}\alpha}{n}\)是\(V\)的一个基.
\begin{proof}
必要性.
假设\(\vb{A}\)可逆,
那么由\cref{theorem:线性映射.可逆线性映射是同构} 可知\(\vb{A}\)是\(V\)上的自同构.
因为\(\AutoTuple{\alpha}{n}\)是线性空间\(V\)的一个基,
所以由\cref{theorem:线性空间的同构.同构线性空间的性质5} 可知,
向量组\(\AutoTuple{\vb{A}\alpha}{n}\)也是线性空间\(V\)的一个基.

充分性.
假设\(\AutoTuple{\vb{A}\alpha}{n}\)是\(V\)的一个基,
那么由\cref{theorem:线性空间.任一向量可由给定基唯一线性表出} 可知,
\(V\)中任意一个向量\(\beta\)都有唯一的分解式\begin{equation*}
	\beta
	= x_1 \vb{A}\alpha_1 + \dotsb + x_n \vb{A}\alpha_n
	= \vb{A}(x_1 \alpha_1 + \dotsb + x_n \alpha_n),
	\quad \AutoTuple{x}{n} \in F,
\end{equation*}
% 构造逆映射
于是我们可以定义映射\(\vb{B}\colon V \to V\),使之满足\begin{equation*}
	\vb{B}(x_1 \vb{A}\alpha_1 + \dotsb + x_n \vb{A}\alpha_n)
	= x_1 \alpha_1 + \dotsb + x_n \alpha_n,
\end{equation*}
显然\(\vb{B}\)是映射.
由于\begin{align*}
	&\hspace{-20pt}
	\vb{B}(
		(x_1 \vb{A}\alpha_1 + \dotsb + x_n \vb{A}\alpha_n)
		+ (y_1 \vb{A}\alpha_1 + \dotsb + y_n \vb{A}\alpha_n)
	) \\
	&= (x_1 \alpha_1 + \dotsb + x_n \alpha_n)
		+ (y_1 \alpha_1 + \dotsb + y_n \alpha_n) \\
	&= \vb{B}(x_1 \vb{A}\alpha_1 + \dotsb + x_n \vb{A}\alpha_n)
		+ \vb{B}(y_1 \vb{A}\alpha_1 + \dotsb + y_n \vb{A}\alpha_n), \\
	&\hspace{-20pt}
	\vb{B}(k(x_1 \vb{A}\alpha_1 + \dotsb + x_n \vb{A}\alpha_n))
	= \vb{B}(k x_1 \vb{A}\alpha_1 + \dotsb + k x_n \vb{A}\alpha_n) \\
	&= k x_1 \alpha_1 + \dotsb + k x_n \alpha_n
	= k (x_1 \alpha_1 + \dotsb + x_n \alpha_n) \\
	&= k \vb{B}(x_1 \vb{A}\alpha_1 + \dotsb + x_n \vb{A}\alpha_n),
\end{align*}
所以\(\vb{B}\)是线性映射.
又因为\begin{align*}
	(\vb{B}\vb{A})(x_1 \alpha_1 + \dotsb + x_n \alpha_n)
	&= \vb{B}(x_1 \vb{A}\alpha_1 + \dotsb + x_n \vb{A}\alpha_n) \\
	&= x_1 \alpha_1 + \dotsb + x_n \alpha_n, \\
	(\vb{A}\vb{B})(x_1 \vb{A}\alpha_1 + \dotsb + x_n \vb{A}\alpha_n)
	&= \vb{A}(x_1 \alpha_1 + \dotsb + x_n \alpha_n) \\
	&= x_1 \vb{A}\alpha_1 + \dotsb + x_n \vb{A}\alpha_n,
\end{align*}
所以\(\vb{B}\vb{A} = \vb{A}\vb{B} = \vb{I}_V\),
这就说明\(\vb{A}\)可逆,且\(\vb{B}\)就是\(\vb{A}\)的逆.
\end{proof}
\end{example}

\begin{example}\label{example:线性映射.强循环向量组}
%@see: 《高等代数(第三版 下册)》(丘维声) P113 习题9.1 9.
%@see: 《高等代数(大学高等代数课程创新教材 第二版 下册)》(丘维声) P239 例8
设\(V\)是域\(F\)上的线性空间,
\(\vb{A}\)是\(V\)上的一个线性变换,
\(\alpha \in V\).
证明:如果存在正整数\(m\)使得\[
	\vb{A}^{m-1} \alpha \neq 0,
	\qquad
	\vb{A}^m \alpha = 0,
\]
则\(\alpha,\vb{A}\alpha,\vb{A}^2\alpha,\dotsc,\vb{A}^{m-1}\alpha\)线性无关.
\begin{proof}
假设存在正整数\(m\)使得\[
	\vb{A}^{m-1} \alpha \neq 0,
	\qquad
	\vb{A}^m \alpha = 0,
\]
令\begin{equation*}
	x_0 \alpha
	+ x_1 \vb{A}\alpha
	+ x_2 \vb{A}^2\alpha
	+ \dotsb
	+ x_{m-2} \vb{A}^{m-2}\alpha
	+ x_{m-1} \vb{A}^{m-1}\alpha
	= 0.
	\eqno(1)
\end{equation*}
由\(\vb{A}^m \alpha = 0\)可得\begin{equation*}
	\vb{A}^{m-1} (
		x_0 \alpha
		+ x_1 \vb{A}\alpha
		+ x_2 \vb{A}^2\alpha
		+ \dotsb
		+ x_{m-2} \vb{A}^{m-2}\alpha
		+ x_{m-1} \vb{A}^{m-1}\alpha
	)
	= x_0 \vb{A}^{m-1}\alpha
	= 0.
\end{equation*}
由于\(\vb{A}^{m-1}\alpha \neq 0\),
所以\(x_0 = 0\).
将\(x_0 = 0\)代入(1)式,得\begin{equation*}
	x_1 \vb{A}\alpha
	+ x_2 \vb{A}^2\alpha
	+ \dotsb
	+ x_{m-2} \vb{A}^{m-2}\alpha
	+ x_{m-1} \vb{A}^{m-1}\alpha
	= 0.
	\eqno(2)
\end{equation*}
再由\(\vb{A}^m \alpha = 0\)可得\begin{equation*}
	\vb{A}^{m-2} (
		x_1 \vb{A}\alpha
		+ x_2 \vb{A}^2\alpha
		+ \dotsb
		+ x_{m-2} \vb{A}^{m-2}\alpha
		+ x_{m-1} \vb{A}^{m-1}\alpha
	)
	= x_1 \vb{A}^{m-1}\alpha
	= 0.
\end{equation*}
由于\(\vb{A}^{m-1}\alpha \neq 0\),
所以\(x_1 = 0\).
以此类推,可证\[
	x_2 = x_3 = \dotsb = x_{m-1} = 0,
\]
因此\(\alpha,\vb{A}\alpha,\vb{A}^2\alpha,\dotsc,\vb{A}^{m-1}\alpha\)线性无关.
\end{proof}
\end{example}

\begin{example}
%@see: 《高等代数(第三版 下册)》(丘维声) P113 习题9.1 10.
%@see: 《高等代数(大学高等代数课程创新教材 第二版 下册)》(丘维声) P239 例9
设\(\vb{A},\vb{B}\)是\(V\)上的线性变换,
且\(\vb{A}\vb{B}-\vb{B}\vb{A}=\vb{I}\).
证明:存在正整数\(k\),使得\begin{equation*}
	\vb{A}^k\vb{B}-\vb{B}\vb{A}^k = k\vb{A}^{k-1}.
\end{equation*}
%TODO proof
\end{example}

\begin{example}
%@see: 《高等代数(第三版 下册)》(丘维声) P113 习题9.1 11.
%@see: 《高等代数(大学高等代数课程创新教材 第二版 下册)》(丘维声) P239 例10
设\(V\)是域\(F\)上的一个线性空间,\(\FieldChar F \neq 2\),
\(\vb{A},\vb{B}\)是\(V\)上的幂等变换.
证明:\begin{itemize}
	\item \(\vb{A}+\vb{B}\)是幂等变换,当且仅当\(\vb{A}\vb{B}=\vb{B}\vb{A}=\vb0\);
	\item 如果\(\vb{A}\vb{B}=\vb{B}\vb{A}\),则\(\vb{A}+\vb{B}-\vb{A}\vb{B}\)也是幂等变换.
\end{itemize}
%TODO proof
\end{example}

\section{线性映射的核与像}
\subsection{线性映射的核与像}
\begin{definition}
%@see: 《高等代数(第三版 下册)》(丘维声) P113 定义1
%@see: 《Linear Algebra Done Right (Fourth Edition)》(Sheldon Axler) P59 3.11
%@see: 《Linear Algebra Done Right (Fourth Edition)》(Sheldon Axler) P61 3.16
设\(V\)和\(V'\)都是域\(F\)上的线性空间,
\(\vb{A}\)是\(V\)到\(V'\)的一个线性映射.
我们把\(V'\)中零向量\(0'\)在\(\vb{A}\)下的原像集
\(\Set{
	\alpha\in V
	\given
	\vb{A}\alpha=0'
}\)
称为“\(\vb{A}\)的\DefineConcept{核}(kernel, null space)”,
记作\(\Ker\vb{A}\)或\(\operatorname{null}\vb{A}\).
把映射\(\vb{A}\)的值域
\(\Set{
	\beta \in V'
	\given
	\beta = \vb{A}\alpha,
	\alpha \in V
}\)
称为“\(\vb{A}\)的\DefineConcept{像}(image)”,
记作\(\Img\vb{A}\)或\(\vb{A}V\).
\end{definition}
%“核”的概念在群同态中也有定义
%考虑零元是加法群的单位元,
%因此线性映射的核的定义只是群同态的特殊情况
\begin{remark}
线性映射的核有可能含有非零元.
例如,在几何空间\(V\)中,
\(V\)在\(xOy\)平面上的正投影\(\vb{P}_W\)
把\(z\)轴上的每一个向量\(\alpha=(0,0,z)\ (z\in\mathbb{R})\)
映射为零向量\(0=(0,0,0)\);
显然,当\(z\neq0\)时,\(\alpha\)就是一个非零元,
但它又是线性映射\(\vb{P}_W\)的核的一个元素.
\end{remark}

\begin{example}
%@see: 《Linear Algebra Done Right (Fourth Edition)》(Sheldon Axler) P59 3.12
%@see: 《Linear Algebra Done Right (Fourth Edition)》(Sheldon Axler) P61 3.17
设\(V,W\)都是域\(F\)上的线性空间.
从\(V\)到\(W\)的零映射\(\vb0\)的核和像分别为\(\Ker\vb0 = V\)和\(\Img\vb0 = 0\).
\end{example}

\begin{example}
%@see: 《高等代数(第三版 下册)》(丘维声) P116 习题9.2 1.
设\(V\)是域\(F\)上的一个线性空间,
\(U\)和\(W\)都是\(V\)的子空间,
且\(V = U \DirectSum W\),
\(\vb{P}_U\)是\(V\)平行于\(W\)在\(U\)上的投影,
\(\vb{P}_W\)是\(V\)平行于\(U\)在\(W\)上的投影.
求\(\Ker\vb{P}_U,
\Img\vb{P}_U,
\Ker\vb{P}_W,
\Img\vb{P}_W\).
%TODO \(\Ker\vb{P}_U = W\)
%TODO \(\Img\vb{P}_U = U\)
\end{example}

\begin{proposition}\label{theorem:线性映射.线性映射的核空间和像空间分别是定义域和陪域的子空间}
%@see: 《高等代数(第三版 下册)》(丘维声) P113 命题1
%@see: 《Linear Algebra Done Right (Fourth Edition)》(Sheldon Axler) P59 3.13
%@see: 《Linear Algebra Done Right (Fourth Edition)》(Sheldon Axler) P61 3.18
设\(\vb{A}\)是域\(F\)上线性空间\(V\)到\(V'\)的一个线性映射,
则\(\Ker\vb{A}\)是\(V\)的一个子空间,
\(\Img\vb{A}\)是\(V'\)的一个子空间.
\begin{proof}
由于\(\vb{A}(0)=0\),
因此\(0 \in \Ker\vb{A}\).
任取\(\alpha,\beta \in \Ker\vb{A},
k \in F\),
则有\begin{gather*}
	\vb{A}(\alpha+\beta)
	= \vb{A}\alpha + \vb{A}\beta
	= 0 + 0
	= 0, \\
	\vb{A}(k\alpha)
	= k \vb{A}\alpha
	= k0
	= 0,
\end{gather*}
因此\(\Ker\vb{A}\)是\(V\)的一个子空间.

显然\(\Img\vb{A}\)是\(V'\)的一个非空子集.
在\(\Img\vb{A}\)中任取两个元素\(\gamma_1,\gamma_2\),
则存在\(\alpha_1,\alpha_2 \in V\),
使得\begin{gather*}
	\gamma_1 = \vb{A}\alpha_1, \\
	\gamma_2 = \vb{A}\alpha_2,
\end{gather*}
从而\begin{gather*}
	\gamma_1 + \gamma_2
	= \vb{A}\alpha_1 + \vb{A}\alpha_2
	= \vb{A}(\alpha_1 + \alpha_2)
	\in \Img\vb{A}, \\
	k \gamma_1
	= k \vb{A}\alpha_1
	= \vb{A}(k \alpha_1)
	\in \Img\vb{A},
	\quad k \in F,
\end{gather*}
因此\(\Img\vb{A}\)是\(V'\)的一个子空间.
\end{proof}
\end{proposition}

\begin{proposition}
设\(V,V'\)都是域\(F\)上的有限维线性空间,
\(\AutoTuple{\alpha}{n}\)是\(V\)的一个基,
\(\vb{A}\)是\(V\)到\(V'\)的一个线性映射,
则\begin{equation*}
	\Img\vb{A}
	= \Span\{\AutoTuple{\vb{A}\alpha}{n}\}.
\end{equation*}
%TODO proof
\end{proposition}

\subsection{线性映射的单射性与满射性}
\begin{proposition}\label{theorem:线性映射.线性映射是单射或满射的充分必要条件}
%@see: 《高等代数(第三版 下册)》(丘维声) P114 命题2
%@see: 《Linear Algebra Done Right (Fourth Edition)》(Sheldon Axler) P60 3.15
设\(\vb{A}\)是域\(F\)上线性空间\(V\)到\(V'\)的一个线性映射,
则\begin{gather*}
	\text{$\vb{A}$是单射}
	\iff
	\Ker\vb{A}=0, \\
	\text{$\vb{A}$是满射}
	\iff
	\Img\vb{A}=V'.
\end{gather*}
\begin{proof}
首先证明
\(\text{$\vb{A}$是单射}
\iff
\Ker\vb{A}=0\).
\begin{itemize}
	\item 设\(\vb{A}\)是单射.
	任取\(\alpha \in \Ker\vb{A}\),
	则\begin{equation*}
		% 第一个等号是由“核”的定义保证的
		\vb{A}\alpha = 0 = \vb{A}0,
	\end{equation*}
	% 由于\(\vb{A}\)是单射
	从而有\(\alpha = 0\),
	因此\(\Ker\vb{A} = 0\).

	\item 设\(\Ker\vb{A} = 0\).
	设\(\alpha_1,\alpha_2 \in V\)满足\(\vb{A}\alpha_1 = \vb{A}\alpha_2\),
	则\begin{equation*}
		0 = \vb{A}\alpha_2 - \vb{A}\alpha_1 = \vb{A}(\alpha_2 - \alpha_1),
	\end{equation*}
	从而\(\alpha_2 - \alpha_1 \in \Ker\vb{A}\).
	由\(\Ker\vb{A} = 0\)
	可知\(\alpha_2 - \alpha_1 = 0\),
	即\(\alpha_1 = \alpha_2\).
	这就说明\(\vb{A}\)是单射.
\end{itemize}

由满射的定义立即可得
\(\text{$\vb{A}$是满射}
\iff
\Img\vb{A}=V'\).
\end{proof}
\end{proposition}

\begin{example}\label{example:线性空间.笛卡尔和与直和之间的关系}
%@see: 《Linear Algebra Done Right (Fourth Edition)》(Sheldon Axler) P98 3.93
设\(V\)是域\(F\)上的线性空间,
\(\AutoTuple{V}{m}\)都是\(V\)的子空间.
证明:\(\AutoTuple{V}{m}[+]\)是直和,
当且仅当\(\Gamma\colon \AutoTuple{V}{m}[\CartesianSum] \to \AutoTuple{V}{m}[+],
(\AutoTuple{\alpha}{m}) \mapsto \AutoTuple{\alpha}{m}[+]\)是单射.
\begin{proof}
直接有\begin{align*}
	\text{$\Gamma$是单射}
	&\iff \Ker\Gamma = 0
		\tag{\cref{theorem:线性映射.线性映射是单射或满射的充分必要条件}} \\
	&\iff \text{$\AutoTuple{V}{m}[+]$中零向量的表示法唯一} \\
	&\iff \text{$\AutoTuple{V}{m}[+]$是直和}
		\tag{\cref{theorem:线性空间.子空间.直和的等价命题}}.
\end{align*}
\end{proof}
\end{example}
\begin{remark}
根据\hyperref[definition:线性空间.子空间.子空间的和]{子空间的和的定义},
映射\(\Gamma\)必定是满射,
因此\cref{example:线性空间.笛卡尔和与直和之间的关系} 可以改写为:
\(\AutoTuple{V}{m}[+]\)是直和,
当且仅当\(\Gamma\)可逆.
\end{remark}

\begin{definition}
设\(V\)和\(V'\)都是域\(F\)上的线性空间,
且\(V\)是有限维的,
\(\vb{A}\)是\(V\)到\(V'\)的一个线性映射.
我们把\(\vb{A}\)的核\(\Ker\vb{A}\)的维数\(\dim(\Ker\vb{A})\)
称为“\(\vb{A}\)的\DefineConcept{零度}(nullity)”,
%@see: https://mathworld.wolfram.com/Nullity.html
把\(\vb{A}\)的像\(\Img\vb{A}\)的维数\(\dim(\Img\vb{A})\)
称为“\(\vb{A}\)的\DefineConcept{秩}(rank)”.
\end{definition}

\subsection{线性映射基本定理}
\begin{theorem}\label{theorem:线性映射.线性映射基本定理}
%@see: 《高等代数(第三版 下册)》(丘维声) P114 定理3
%@see: 《Linear Algebra Done Right (Fourth Edition)》(Sheldon Axler) P62 3.21 fundamental theorem of linear maps
设\(V\)和\(V'\)都是域\(F\)上的线性空间,
且\(V\)是有限维的,
\(\vb{A}\)是\(V\)到\(V'\)的一个线性映射,
则\(\Ker\vb{A}\)和\(\Img\vb{A}\)都是有限维的,
且\begin{equation*}
%@see: 《高等代数(第三版 下册)》(丘维声) P114 (2)
	\dim(\Ker\vb{A})
	+\dim(\Img\vb{A})
	=\dim V.
\end{equation*}
\begin{proof}
因为\(V\)是有限维的,
%\cref{theorem:线性空间.子空间.有限维线性空间的子空间是有限维的}
%\cref{theorem:线性映射.线性映射的核空间和像空间分别是定义域和陪域的子空间}
所以它的子空间\(\Ker\vb{A}\)是有限维的.

设\(\dim(\Ker\vb{A}) = m,
\dim V = n\).
取\(\Ker\vb{A}\)的一个基\(\AutoTuple{\alpha}{m}\),
把它扩充成\(V\)的一个基\begin{equation*}
	\AutoTuple{\alpha}{m},
	\AutoTuple{\alpha}[m+1]{n}.
\end{equation*}
由核的定义可知,
对于任意一个向量\(\alpha \in V\),
有\begin{equation*}
	% 核空间\(\Ker\vb{A}\)中的任意一个向量\(\alpha\)在线性映射\(\vb{A}\)下的像\(\vb{A}\alpha\)必定等于\(0\)
	\alpha \in \Ker\vb{A}
	\implies
	\vb{A}\alpha = 0,
\end{equation*}
那么\begin{equation*}
	% 核空间\(\Ker\vb{A}\)的每一个基向量都是\(\Ker\vb{A}\)中的向量,自然它在线性映射\(\vb{A}\)下的像\(\vb{A}\alpha\)必定等于\(0\)
	\vb{A}\alpha_i = 0,
	\quad i=1,2,\dotsc,m,
\end{equation*}
于是,
对于任意\(\AutoTuple{x}{m} \in F\),
有\begin{equation*}
	\vb{A} \sum_{i=1}^m x_i \alpha_i
	= \sum_{i=1}^m x_i \vb{A}\alpha_i
	= \sum_{i=1}^m x_i 0
	= 0.
\end{equation*}
在\(\Img\vb{A}\)中任取一个向量\(\vb{A}\alpha\),
其中\(\alpha \in V\).
% \(\alpha\)作为\(V\)中的向量,自然可以由\(V\)的基\(\alpha_1,\dotsc,\alpha_n\)线性表出
设\(\alpha = \sum_{i=1}^n x_i \alpha_i\),
其中\(\AutoTuple{x}{m} \in F\),
则\begin{equation*}
%@see: 《高等代数(第三版 下册)》(丘维声) P114 (3)
	\vb{A}\alpha
	% 在\(\alpha = \sum_{i=1}^n x_i \alpha_i\)等号左右两边同时乘以\(\vb{A}\)便得
	= \sum_{i=1}^n x_i \vb{A} \alpha_i
	% 由上可知\(\sum_{i=1}^m x_i \vb{A} \alpha_i = 0\)
	= \sum_{i=m+1}^n x_i \vb{A} \alpha_i,
\end{equation*}
因此\begin{equation*}
%@see: 《高等代数(第三版 下册)》(丘维声) P114 (4)
	\Img\vb{A} = \Span\{\AutoTuple{\vb{A}\alpha}[m+1]{n}\},
\end{equation*}
从而\(\Img\vb{A}\)是有限维的.

接下来证明\(\AutoTuple{\vb{A}\alpha}[m+1]{n}\)线性无关.
令\begin{equation*}
	y_{m+1} \vb{A} \alpha_{m+1} + \dotsb + y_n \vb{A} \alpha_n = 0,
\end{equation*}
则\begin{equation*}
	\vb{A} (y_{m+1} \alpha_{m+1} + \dotsb + y_n \alpha_n) = 0,
\end{equation*}
于是\begin{equation*}
	\beta
	\defeq
	y_{m+1} \alpha_{m+1} + \dotsb + y_n \alpha_n \in \Ker\vb{A},
\end{equation*}
% 既然\(\beta\)是核空间\(\Ker\vb{A}\)中的向量,
% 那么\(\beta\)肯定可以由\(\Ker\vb{A}\)的基\(\AutoTuple{\alpha}{m}\)线性表出
所以\begin{equation*}
	y_{m+1} \alpha_{m+1} + \dotsb + y_n \alpha_n
	= z_1 \alpha_1 + \dotsb + z_m \alpha_m,
\end{equation*}
即\begin{equation*}
	-z_1 \alpha_1 - \dotsb - z_m \alpha_m + y_{m+1} \alpha_{m+1} + \dotsb + y_n \alpha_n = 0,
\end{equation*}
% 既然\(\AutoTuple{\alpha}{n}\)是\(V\)的一个基,
% 那么\(\AutoTuple{\alpha}{n}\)一定是线性无关的向量组,
% 于是上面这个关于\(z_1,\dotsc,z_m,y_{m+1},\dotsc,y_n\)的线性方程只有零解
从而有\begin{equation*}
	z_1 = \dotsb = z_m = y_{m+1} = \dotsb = y_n = 0.
\end{equation*}
这就说明\(\AutoTuple{\vb{A}\alpha}[m+1]{n}\)线性无关,
于是它是\(\Img\vb{A}\)的一个基.
因此\begin{equation*}
	\dim(\Img\vb{A})
	= \card\{\AutoTuple{\vb{A}\alpha}[m+1]{n}\}
	= n - m
	= \dim V - \dim(\Ker\vb{A}),
\end{equation*}
移项得\(\dim V = \dim(\Ker\vb{A}) + \dim(\Img\vb{A})\).
\end{proof}
\end{theorem}

\begin{corollary}\label{theorem:线性映射.定义域维数大于陪域维数的线性映射不是单射}
%@see: 《Linear Algebra Done Right (Fourth Edition)》(Sheldon Axler) P63 3.22
设\(V,W\)都是域\(F\)上的有限维线性空间,
且\(\dim V > \dim W\),
则\(\Hom(V,W)\)中的每一个线性映射都不是单射.
\begin{proof}
任取线性映射\(\vb{A}\in\Hom(V,W)\),
则\begin{align*}
	\dim(\Ker\vb{A})
	&= \dim V - \dim(\Img\vb{A})
		\tag{\hyperref[theorem:线性映射.线性映射基本定理]{线性映射基本定理}} \\
	&\geq \dim V - \dim W
		\tag{\cref{theorem:线性映射.线性映射的核空间和像空间分别是定义域和陪域的子空间,theorem:线性空间.线性空间及其子空间的维数序关系}} \\
	&> 0.
\end{align*}
既然\(\dim(\Ker\vb{A}) > 0\),
那么由\cref{theorem:线性映射.线性映射是单射或满射的充分必要条件} 可知,
\(\vb{A}\)不是单射.
\end{proof}
\end{corollary}

\begin{corollary}\label{theorem:线性映射.定义域维数小于陪域维数的线性映射不是满射}
%@see: 《Linear Algebra Done Right (Fourth Edition)》(Sheldon Axler) P64 3.24
设\(V,W\)都是域\(F\)上的有限维线性空间,
且\(\dim V < \dim W\),
则\(\Hom(V,W)\)中的每一个线性映射都不是满射.
\begin{proof}
任取线性映射\(\vb{A}\in\Hom(V,W)\),
则\begin{align*}
	\dim(\Img\vb{A})
	&= \dim V - \dim(\Ker\vb{A})
		\tag{\hyperref[theorem:线性映射.线性映射基本定理]{线性映射基本定理}} \\
	&\leq \dim V \\
	&< \dim W,
\end{align*}
这就说明\(\Img\vb{A} \neq W\),
\(\vb{A}\)不是满射.
\end{proof}
\end{corollary}

%@see: 《Linear Algebra Done Right (Fourth Edition)》(Sheldon Axler) P64
\cref{theorem:线性映射.定义域维数大于陪域维数的线性映射不是单射,theorem:线性映射.定义域维数小于陪域维数的线性映射不是满射}
对于线性方程论有很重要的意义.

%@see: 《Linear Algebra Done Right (Fourth Edition)》(Sheldon Axler) P64
给定正整数\(m,n\),
矩阵\(A \in M_{m \times n}(F)\),
考虑关于\(X \in F^n\)的齐次线性方程组\(AX=0\),
显然\(X=0\)是它的一个解,
接下来我们想要知道它还有没有其他解.
定义映射\begin{equation*}
	\vb{A}\colon F^n \to F^m,
	X \mapsto AX.
\end{equation*}
那么\(\vb{A}X = 0\)等价于上述齐次线性方程组\(AX=0\).
可以看出,方程\(AX=0\)有非零解,
当且仅当\(\vb{A}\)的核\(\Ker\vb{A}\)比零子空间“大”(即\(0 \subset \Ker\vb{A}\)),
这又相当于说\(\vb{A}\)不是单射(根据\cref{theorem:线性映射.线性映射是单射或满射的充分必要条件}),
于是可以得到如下结论:
\begin{theorem}\label{theorem:线性映射.线性映射的单射性决定齐次线性方程组是否具有非零解}
%@see: 《Linear Algebra Done Right (Fourth Edition)》(Sheldon Axler) P64
设\(F\)是一个域,
矩阵\(A \in M_{m \times n}(F)\),
线性映射\(\vb{A}\colon F^n \to F^m, X \mapsto AX\),
则\begin{equation*}
	\text{关于$X \in F^n$的齐次线性方程组$AX=0$有非零解}
	\iff
	\text{$\vb{A}$不是单射}.
\end{equation*}
\end{theorem}
\begin{corollary}\label{theorem:线性映射.方程个数少于未知量个数的齐次线性方程组必有非零解}
%@see: 《Linear Algebra Done Right (Fourth Edition)》(Sheldon Axler) P65 3.26
% 原话是: A homogeneous system of linear equations with more variables than equations has nonzero solutions.
方程个数少于未知量个数的齐次线性方程组必有非零解.
\begin{proof}
设\(F\)是一个域,
\(\vb{A}\)是从\(F^n\)到\(F^m\)的一个线性映射.
由\cref{theorem:线性映射.定义域维数大于陪域维数的线性映射不是单射} 可知,
当\(n > m\)时,\(\vb{A}\)不是单射.
再由\cref{theorem:线性映射.线性映射的单射性决定齐次线性方程组是否具有非零解} 可知,
齐次线性方程组\(\LinearMapMatrix(\vb{A})X=0\)有非零解.
\end{proof}
%\cref{theorem:线性方程组.方程个数少于未知量个数的齐次线性方程组必有非零解}
\end{corollary}

%@see: 《Linear Algebra Done Right (Fourth Edition)》(Sheldon Axler) P65
给定正整数\(m,n\),
矩阵\(A \in M_{m \times n}(F)\),
我们想要知道是否存在向量\(B \in F^m\),
使得关于\(X \in F^n\)的非齐次线性方程组\(AX=B\)无解.
定义映射\begin{equation*}
	\vb{A}\colon F^n \to F^m,
	X \mapsto AX.
\end{equation*}
那么\(\vb{A}X=B\)等价于上述非齐次线性方程组\(AX=B\).
可以看出,方程\(AX=B\)无解,
当且仅当\(B \notin \Img\vb{A}\).
于是\(B\)的存在与否,取决于\(\Img\vb{A}\)是不是不等于\(\vb{A}\)的陪域\(F^m\)
(当\(\Img\vb{A} \neq F^m\)时,差集\(F^m - \Img\vb{A}\)中的每一个向量\(B\)都能使方程\(AX=B\)无解).
换言之,\(B\)的存在与否,取决于\(\vb{A}\)是不是不是满射.
于是可以得到如下结论:
\begin{theorem}
%@see: 《Linear Algebra Done Right (Fourth Edition)》(Sheldon Axler) P65
设\(F\)是一个域,
矩阵\(A \in M_{m \times n}(F)\),
线性映射\(\vb{A}\colon F^n \to F^m, X \mapsto AX\),
则\begin{equation*}
	\text{存在向量$B \in F^m$,使得关于$X \in F^n$的齐次线性方程组$AX=B$无解}
	\iff
	\text{$\vb{A}$不是满射}.
\end{equation*}
\end{theorem}
\begin{corollary}
%@see: 《Linear Algebra Done Right (Fourth Edition)》(Sheldon Axler) P65 3.28
% 原话是: An inhomogeneous system of linear equations with more equations than variables has no solution for some choice of the constant terms.
方程个数多于未知量个数的非齐次线性方程组可能无解(具体取决于常数项).
\begin{proof}
设\(F\)是一个域,
\(\vb{A}\)是从\(F^n\)到\(F^m\)的一个线性映射.
由\cref{theorem:线性映射.定义域维数小于陪域维数的线性映射不是满射} 可知,
当\(n < m\)时,\(\vb{A}\)不是满射.
\end{proof}
\end{corollary}

\begin{corollary}
%@see: 《高等代数(第三版 下册)》(丘维声) P115
设\(V\)和\(V'\)都是域\(F\)上的线性空间,
且\(V\)是有限维的,
\(\vb{A}\)是\(V\)到\(V'\)的一个线性映射.
若\(\AutoTuple{\alpha}{n}\)是\(V\)的一个基,
则\begin{equation*}
	\Img\vb{A}=\Span\{\vb{A}\alpha_1,\dotsc,\vb{A}\alpha_n\}.
\end{equation*}
\end{corollary}

\begin{example}\label{example:线性映射.无限维线性空间.单有单射或满射推不出线性映射可逆1}
\def\MyPolynomialRing{\mathbb{R}[x]}%
\def\MyLinearMapSpace{\Hom(\MyPolynomialRing,\MyPolynomialRing)}%
设\(\vb{T}\in\MyLinearMapSpace\)表示给多项式乘以\(x^2\)因式.
虽然\(\vb{T}\)是单射,但它不是满射(这是因为零次多项式\(1\)不属于\(\Img\vb{T}\)),
所以\(\vb{T}\)不是可逆线性映射.
\end{example}
\begin{example}\label{example:线性映射.无限维线性空间.单有单射或满射推不出线性映射可逆2}
\def\MyVectorSpace{F^\infty}
设\(F\)是一个域,
线性变换\(\vb{T}\colon F^\infty \to F^\infty, (x_1,x_2,x_3,\dotsc) \mapsto (x_2,x_3,\dotsc)\)
虽是满射却不是单射(这是因为向量\((1,0,0,0,\dotsc)\)属于\(\Ker\vb{T}\)),
所以\(\vb{T}\)不是可逆线性映射.
\end{example}

\begin{corollary}\label{theorem:线性映射.有限维线性空间.单线性映射是满线性映射}
%@see: 《高等代数(第三版 下册)》(丘维声) P115 推论4
%@see: 《Linear Algebra Done Right (Fourth Edition)》(Sheldon Axler) P84 3.65
设\(V\)和\(V'\)都是域\(F\)上的\(n\)维线性空间,
\(\vb{A}\)是\(V\)到\(V'\)的一个线性映射,
则\begin{equation*}
	\text{$\vb{A}$是单射}
	\iff
	\text{$\vb{A}$是满射}.
\end{equation*}
\begin{proof}
直接有\begin{align*}
	\text{$\vb{A}$是单射}
	&\iff
	\Ker\vb{A} = 0 \\
	&\iff
	\dim(\Img\vb{A})
	= \dim V
	= \dim V' \\
	&\iff
	\Img\vb{A} = V' \\
	&\iff
	\text{$\vb{A}$是满射}.
	\qedhere
\end{align*}
\end{proof}
\end{corollary}
\begin{corollary}
%@see: 《高等代数(第三版 下册)》(丘维声) P115 推论5
设\(\vb{A}\)是域\(F\)上的有限维线性空间\(V\)上的线性变换,
则\begin{equation*}
	\text{$\vb{A}$是单射}
	\iff
	\text{$\vb{A}$是满射}.
\end{equation*}
\end{corollary}
\begin{remark}
\cref{theorem:线性映射.有限维线性空间.单线性映射是满线性映射} 说明:
在有限维线性空间中,
任意一个线性映射只要单有单射性或满射性,便可推出它有可逆性,
而不必像\cref{theorem:线性映射.可逆线性映射是同构} 所要求的那样,
只有双射才能保证可逆性.
\end{remark}

\begin{proposition}
%@see: 《Linear Algebra Done Right (Fourth Edition)》(Sheldon Axler) P85 3.68
设\(V,W\)都是有限维线性空间,且\(\dim V = \dim W\),
\(\vb{A}\in\Hom(V,W),
\vb{B}\in\Hom(W,V)\),
则\begin{equation*}
	\vb{A}\vb{B} = \vb{I}
	\iff
	\vb{B}\vb{A} = \vb{I}.
\end{equation*}
%TODO proof
\end{proposition}

\begin{remark}
对于有限维线性空间\(V\)上的线性变换\(\vb{A}\),
虽然子空间\(\Ker\vb{A}\)与\(\Img\vb{A}\)的维数之和等于\(\dim V\),
但是\(\Ker\vb{A}+\Img\vb{A}\)并不一定是整个空间\(V\).

例如,在线性空间\(K[x]_n\)中,
求导数\(\vb{D}\)的像为
\(\Img\vb{D}=K[x]_{n-1}\),
它的核为\(\Ker\vb{D}=K\).
显然\(K+K[x]_{n-1}\neq K[x]_n\).
\end{remark}

\begin{example}
%@see: 《高等代数(第三版 下册)》(丘维声) P115
%@see: 《Linear Algebra Done Right (Fourth Edition)》(Sheldon Axler) P90 3.78
设矩阵\(A \in M_{s \times n}(F)\).
令\begin{equation*}
	\vb{A}\colon F^n \to F^s, \alpha \mapsto A\alpha.
\end{equation*}
% 由\cref{example:线性映射.左乘矩阵是线性映射} 可知
则\(\vb{A}\)是从\(F^n\)到\(F^s\)的一个线性映射,
% 参考:第4章第7节
且\(\Ker\vb{A}\)是齐次线性方程组\(AX=0\)的解空间,
而\(\Img\vb{A}\)是矩阵\(A\)的列空间.
%TODO proof
\end{example}

\begin{example}
%@see: 《高等代数(第三版 下册)》(丘维声) P116 习题9.2 2.
设\(V\)是域\(F\)上的一个线性空间,
\(\vb{A}\)是\(V\)上的一个线性变换.
证明:如果\(\vb{A}\)是\(V\)上的幂等变换,
则\(V = \Ker\vb{A} \DirectSum \Img\vb{A}\),
且\(\vb{A}\)是\(V\)平行于\(\Ker\vb{A}\)在\(\Img\vb{A}\)上的投影.
%TODO proof
\end{example}

\begin{example}
%@see: 《高等代数(第三版 下册)》(丘维声) P116 习题9.2 3.
设\(V,U,W\)都是域\(F\)上的线性空间,且\(V\)是有限维的,
\(\vb{A}\)是从\(V\)到\(U\)的一个线性映射,
\(\vb{B}\)是从\(U\)到\(W\)的一个线性映射.
证明:\begin{equation*}
	\dim(\Ker(\vb{B}\vb{A}))
	\leq
	\dim(\Ker\vb{A})
	+ \dim(\Ker\vb{B}).
\end{equation*}
%TODO proof
\end{example}

% 西尔维斯特不等式
\begin{example}
%@see: 《高等代数(第三版 下册)》(丘维声) P116 习题9.2 4.
设\(V,U,W\)都是域\(F\)上的线性空间,\(\dim V = n,\dim U = m\),
\(\vb{A}\)是从\(V\)到\(U\)的一个线性映射,
\(\vb{B}\)是从\(U\)到\(W\)的一个线性映射.
证明:\begin{equation*}
	\rank(\vb{B}\vb{A})
	\geq \rank\vb{A} + \rank\vb{B} - m.
\end{equation*}
%TODO proof
\end{example}

\begin{example}
%@see: 《高等代数(第三版 下册)》(丘维声) P116 习题9.2 5.(1)
设\(\vb{A},\vb{B}\)都是域\(F\)上线性空间\(V\)上的幂等变换.
证明:\begin{equation*}
	\Img\vb{A} = \Img\vb{B}
	\iff
	\vb{A}\vb{B} = \vb{B},
	\vb{B}\vb{A} = \vb{A}.
\end{equation*}
%TODO proof
\end{example}

\begin{example}
%@see: 《高等代数(第三版 下册)》(丘维声) P116 习题9.2 5.(2)
设\(\vb{A},\vb{B}\)都是域\(F\)上线性空间\(V\)上的幂等变换.
证明:\begin{equation*}
	\Ker\vb{A} = \Ker\vb{B}
	\iff
	\vb{A}\vb{B} = \vb{A},
	\vb{B}\vb{A} = \vb{B}.
\end{equation*}
%TODO proof
\end{example}

\begin{example}
%@see: 《高等代数(第三版 下册)》(丘维声) P116 习题9.2 6.
设\(V\)和\(V'\)都是域\(F\)上的有限维线性空间,
\(\vb{A}\)是从\(V\)到\(V'\)的一个线性映射.
证明:存在直和分解\(V = U \DirectSum W,
V' = M \DirectSum N\),
使得\(\Ker\vb{A} = U\)且\(W \Isomorphism M\).
%TODO proof
\end{example}

\begin{example}
%@see: 《高等代数(第三版 下册)》(丘维声) P116 习题9.2 7.
设\(V\)是域\(F\)上的一个线性空间,域\(F\)的特征为\(\FieldChar F = 0\).
证明:如果\(\AutoTuple{\vb{A}}{s}\)是\(V\)上\(s\)个两两不同的线性变换,
那么\(V\)中至少有一个向量\(\alpha\),
使得\(\vb{A}_1\alpha,\allowbreak\dotsc,\allowbreak\vb{A}_s\alpha\)两两不同.
%TODO proof
\end{example}

\begin{theorem}
%@see: 《Linear Algebra Done Right (Fourth Edition)》(Sheldon Axler) P102 3.107
\def\T{\vb{T}}%
\def\wT{\widetilde{\T}}
设\(V,W\)都是域\(F\)上的线性空间,
\(\T\)是从\(V\)到\(W\)的一个线性映射,
\(Y = \Ker\T\),
\(\vb\pi\)是从\(V\)到\(V/Y\)的标准映射,
定义\begin{equation*}
	\wT\colon V/Y \to W,
	\alpha+Y \mapsto \T\alpha,
\end{equation*}
则\begin{itemize}
	\item \(\wT \vb\pi = \T\),
	\item \(\wT\)是单射,
	\item \(\Img\wT = \Img\T\),
	\item \(Y \Isomorphism \Img\T\).
\end{itemize}
%TODO proof
\end{theorem}

\section{线性映射的矩阵表示}
在本节,我们学习如何利用矩阵研究线性映射.

\subsection{用矩阵表示一个有限维线性空间上的线性变换}
%@see: 《高等代数(第三版 下册)》(丘维声) P116
%@see: 《Linear Algebra Done Right (Fourth Edition)》(Sheldon Axler) P69 3.31
设\(V\)是域\(F\)上的\(n\)维线性空间,
\(\vb{A}\)是\(V\)上的一个线性变换.
我们知道,\(\vb{A}\)被它在\(V\)上的一个基上的作用所决定.
于是取\(V\)的一个基\(\AutoTuple{\alpha}{n}\).
由于\(\vb{A}\alpha_i\in V\),
因此\(\vb{A}\alpha_i\)可以被\(V\)的这个基唯一地线性表出:\begin{equation*}
%@see: 《高等代数(第三版 下册)》(丘维声) P117 (1)
	\left\{ \begin{array}{l}
		\vb{A}\alpha_1=a_{11}\alpha_1+a_{21}\alpha_2+\dotsb+a_{n1}\alpha_n, \\
		\vb{A}\alpha_2=a_{12}\alpha_1+a_{22}\alpha_2+\dotsb+a_{n2}\alpha_n, \\
		\hdotsfor1 \\
		\vb{A}\alpha_n=a_{1n}\alpha_1+a_{2n}\alpha_2+\dotsb+a_{nn}\alpha_n.
	\end{array} \right.
\end{equation*}
我们可以在形式上把上式写成\begin{equation*}
%@see: 《高等代数(第三版 下册)》(丘维声) P117 (2)
	(\vb{A}\alpha_1,\vb{A}\alpha_2,\dotsc,\vb{A}\alpha_n)
	=(\AutoTuple{\alpha}{n})
	\begin{bmatrix}
		a_{11} & a_{12} & \dots & a_{1n} \\
		a_{21} & a_{22} & \dots & a_{2n} \\
		\vdots & \vdots && \vdots \\
		a_{n1} & a_{n2} & \dots & a_{nn}
	\end{bmatrix}.
\end{equation*}
我们把上式右端的\(n\)阶矩阵\begin{equation*}
	A \defeq \begin{bmatrix}
		a_{11} & a_{12} & \dots & a_{1n} \\
		a_{21} & a_{22} & \dots & a_{2n} \\
		\vdots & \vdots && \vdots \\
		a_{n1} & a_{n2} & \dots & a_{nn}
	\end{bmatrix}
\end{equation*}
称为“线性变换\(\vb{A}\)在基\(\AutoTuple{\alpha}{n}\)下的\DefineConcept{矩阵}”,
记作\(\LinearMapMatrix(\vb{A},(\AutoTuple{\alpha}{n}))\)\footnote{
	应该注意到,这里定义的\(\LinearMapMatrix\)记号,
	与之前在\cref{section:线性空间.向量的坐标}定义
	线性空间中的向量的坐标时
	所采用的记号完全相同.
	这是我们有意为之.
},
或在不致混淆的情况下简记为\(\LinearMapMatrix(\vb{A})\).

\(A\)的第\(j\ (j=1,2,\dotsc,n)\)列是
\(\vb{A}\alpha_j\)在基\(\AutoTuple{\alpha}{n}\)下的坐标.
% 线性变换和它的矩阵一一对应
因此\(A\)由线性变换\(\vb{A}\)唯一决定.

如果我们再把\((\vb{A}\alpha_1,\vb{A}\alpha_2,\dotsc,\vb{A}\alpha_n)\)
简记为\(\vb{A}(\AutoTuple{\alpha}{n})\),
那么上式可以化为\begin{equation*}
%@see: 《高等代数(第三版 下册)》(丘维声) P117 (3)
	\vb{A}(\AutoTuple{\alpha}{n})
	=(\AutoTuple{\alpha}{n})A.
\end{equation*}
这就是一个\(n\)阶矩阵\(A\)
是\(V\)上线性变换\(\vb{A}\)
在基\(\AutoTuple{\alpha}{n}\)下的矩阵的充分必要条件.

\begin{example}
%@see: 《高等代数(第三版 下册)》(丘维声) P117 例1
在\(\mathbb{R}^\mathbb{R}\)中,
设\(V=\Span\{1,\sin x,\cos x\}\),
证明:
导数\(\vb{D}\)是\(V\)上的线性变换,
写出\(\vb{D}\)在基\(1,\sin x,\cos x\)下的矩阵.
\begin{proof}
因为\begin{equation*}
	\vb{D}(k_1\cdot1+k_2\cdot\sin x+k_3\cos x)
	=-k_3\sin x+k_2\cos x
	\in V,
\end{equation*}
所以\(\vb{D}\)是\(V\)上的线性变换.
因为\begin{equation*}
	\left\{ \begin{array}{l}
		\vb{D}1
		=0
		=0\cdot1+0\cdot\sin x+0\cdot\cos x, \\
		\vb{D}\sin x
		=\cos x
		=0\cdot1+0\cdot\sin x+1\cdot\cos x, \\
		\vb{D}\cos x
		=-\sin x
		=0\cdot1+(-1)\cdot\sin x+0\cdot\cos x,
	\end{array} \right.
\end{equation*}
所以\(\vb{D}\)在基\(1,\sin x,\cos x\)下的矩阵是\begin{equation*}
	D=\begin{bmatrix}
		0 & 0 & 0 \\
		0 & 0 & -1 \\
		0 & 1 & 0
	\end{bmatrix}.
	\qedhere
\end{equation*}
\end{proof}
\end{example}

\subsection{用矩阵表示两个有限维线性空间之间的线性映射}
%@see: 《高等代数(第三版 下册)》(丘维声) P117
上例说明,\(n\)维线性空间\(V\)上的线性变换可以用矩阵来表示.
下面我们来讨论两个有限维线性空间之间的线性映射能不能用矩阵来表示.

设\(V\)和\(V'\)分别是域\(F\)上\(n\)维、\(s\)维线性空间,
\(\vb{A}\)是\(V\)到\(V'\)的一个线性映射.
在\(V\)中取一个基\(\AutoTuple{\alpha}{n}\),
在\(V'\)中取一个基\(\AutoTuple{\beta}{s}\),
由于\(\vb{A}\alpha_i\in V'\),
因此\(\vb{A}\alpha_i\)可以
由\(V'\)的基\(\AutoTuple{\beta}{s}\)唯一地线性表出:\begin{equation*}
%@see: 《高等代数(第三版 下册)》(丘维声) P118 (4)
	\left\{ \begin{array}{l}
		\vb{A}\alpha_1=a_{11}\beta_1+a_{21}\beta_2+\dotsb+a_{s1}\beta_s, \\
		\vb{A}\alpha_2=a_{12}\beta_1+a_{22}\beta_2+\dotsb+a_{s2}\beta_s, \\
		\hdotsfor1 \\
		\vb{A}\alpha_n=a_{1n}\beta_1+a_{2n}\beta_2+\dotsb+a_{sn}\beta_s.
	\end{array} \right.
\end{equation*}
我们可以在形式上把上式写成\begin{equation*}
%@see: 《高等代数(第三版 下册)》(丘维声) P118 (5)
	(\vb{A}\alpha_1,\vb{A}\alpha_2,\dotsc,\vb{A}\alpha_n)
	=(\AutoTuple{\beta}{s})
	\begin{bmatrix}
		a_{11} & a_{12} & \dots & a_{1n} \\
		a_{21} & a_{22} & \dots & a_{2n} \\
		\vdots & \vdots && \vdots \\
		a_{s1} & a_{s2} & \dots & a_{sn}
	\end{bmatrix}.
\end{equation*}
我们把上式右端的\(s\times n\)阶矩阵\begin{equation*}
	A \defeq \begin{bmatrix}
		a_{11} & a_{12} & \dots & a_{1n} \\
		a_{21} & a_{22} & \dots & a_{2n} \\
		\vdots & \vdots && \vdots \\
		a_{s1} & a_{s2} & \dots & a_{sn}
	\end{bmatrix}
\end{equation*}
称为“线性映射\(\vb{A}\)
在\(V\)的基\(\AutoTuple{\alpha}{n}\)
和\(V'\)的基\(\AutoTuple{\beta}{s}\)
下的\DefineConcept{矩阵}”,
记作\(\LinearMapMatrix(\vb{A},(\AutoTuple{\alpha}{n}),(\AutoTuple{\beta}{n}))\),
或在不致混淆的情况下简记为\(\LinearMapMatrix(\vb{A})\).

\(A\)的第\(j\ (j=1,2,\dotsc,n)\)列是
\(\vb{A}\alpha_j\)在基\(\AutoTuple{\beta}{s}\)下的坐标.
% 线性映射和它的矩阵一一对应
因此\(A\)由线性映射\(\vb{A}\)唯一决定.
那么上式可以化为\begin{equation*}
%@see: 《高等代数(第三版 下册)》(丘维声) P118 (6)
	\vb{A}(\AutoTuple{\alpha}{n})
	=(\AutoTuple{\beta}{s})A.
\end{equation*}
这就是一个\(s\times n\)矩阵\(A\)
是\(V\)到\(V'\)的线性映射\(\vb{A}\)
在\(V\)的基\(\AutoTuple{\alpha}{n}\)
和\(V'\)的基\(\AutoTuple{\beta}{s}\)下的矩阵的充分必要条件.

\begin{proposition}
%@see: 《Linear Algebra Done Right (Fourth Edition)》(Sheldon Axler) P89 3.75
设\(V,W\)都是域\(F\)上的有限维线性空间,
\(\AutoTuple{\alpha}{n}\)和\(\AutoTuple{\beta}{m}\)分别是\(V\)和\(W\)的基,
\(\vb{T}\in\Hom(V,W)\),
则线性映射\(\vb{T}\)的矩阵的第\(k\)列,
等于向量\(\vb{T}\alpha_k\)的坐标,
即\begin{equation*}
	\MatrixEntry{
		\LinearMapMatrix(\vb{T})
	}{*,k}
	= \VectorMatrix(\vb{T} \alpha_k),
	\quad k=1,2,\dotsc,n.
\end{equation*}
\end{proposition}

\begin{proposition}
设\(V_1\)、\(V_2\)分别是域\(F\)上\(n\)维、\(m\)维线性空间,
\(\vb{A} \in \Hom(V_1,V_2)\),
\(A\)是\(\vb{A}\)在\(V_1\)的基\(\AutoTuple{\alpha}{n}\)和\(V_2\)的基\(\AutoTuple{\beta}{m}\)下的矩阵,
则对于\(V_1\)中任一向量\(\alpha \defeq k_1 \alpha_1 + \dotsb + k_n \alpha_n\),
有\begin{equation*}
	\vb{A}(\alpha)
	= (\alpha_1,\dotsc,\alpha_n) A (k_1,\dotsc,k_n)^T.
\end{equation*}
\begin{proof}
直接计算得\begin{align*}
	\vb{A}(\alpha)
	&= \vb{A}(k_1 \alpha_1 + \dotsb + k_n \alpha_n) \\
	&= k_1 \vb{A}(\alpha_1) + \dotsb + k_n \vb{A}(\alpha_n)
		\tag{线性映射的定义} \\
	&= (\vb{A}(\alpha_1),\dotsc,\vb{A}(\alpha_n)) (k_1,\dotsc,k_n)^T \\
	&= \vb{A}(\alpha_1,\dotsc,\alpha_n) (k_1,\dotsc,k_n)^T \\
	&= (\alpha_1,\dotsc,\alpha_n) A (k_1,\dotsc,k_n)^T.
	\tag*{\qedhere}
\end{align*}
\end{proof}
\end{proposition}

\begin{proposition}
设\(V,V'\)都是域\(F\)上的有限维线性空间,
\(\vb{A}\)是\(V\)到\(V'\)的一个线性映射,
\(A\)是\(\vb{A}\)在\(V\)的基\(\AutoTuple{\alpha}{n}\)和\(V'\)的基\(\AutoTuple{\beta}{m}\)下的矩阵,
记\begin{equation*}
	\LinearMapMatrix(\Img\vb{A},(\AutoTuple{\beta}{m}))
	\defeq
	\Set{
		w' \in F^m
		\given
		w' = \LinearMapMatrix(v',(\AutoTuple{\beta}{m})),
		v' \in \Img\vb{A}
	},
\end{equation*}
则\begin{equation*}
	\LinearMapMatrix(\Img\vb{A},(\AutoTuple{\beta}{m}))
	= \Img A.
\end{equation*}
\begin{proof}
因为\(
	\vb{A}(k_1 \alpha_1 + \dotsb + k_n \alpha_n)
	= (\alpha_1,\dotsc,\alpha_n) A (k_1,\dotsc,k_n)^T
\),
所以\begin{align*}
	&\LinearMapMatrix(\Img\vb{A},(\AutoTuple{\beta}{m})) \\
	&= \Set{
		w' \in F^m
		\given
		w' = \LinearMapMatrix(v',(\AutoTuple{\beta}{m})),
		v' = \vb{A} v,
		v \in V
	} \\
	&= \Set{
		w' \in F^m
		\given
		w' = \LinearMapMatrix(v',(\AutoTuple{\beta}{m})),
		v' = \vb{A}(k_1 \alpha_1 + \dotsb + k_n \alpha_n),
		k_1,\dotsc,k_n \in F
	} \\
	&= \Set{
		w' \in F^m
		\given
		w' = \LinearMapMatrix(v',(\AutoTuple{\beta}{m})),
		v' = (\alpha_1,\dotsc,\alpha_n) A (k_1,\dotsc,k_n)^T,
		k_1,\dotsc,k_n \in F
	} \\
	&= \Set{
		w' \in F^m
		\given
		w' = A (k_1,\dotsc,k_n)^T,
		k_1,\dotsc,k_n \in F
	} \\
	&= \Set{
		w' \in F^m
		\given
		w' = A w,
		w \in F^n
	}
	= \Img A.
	\qedhere
\end{align*}
\end{proof}
\end{proposition}

\begin{proposition}
设\(V,V'\)都是域\(F\)上的有限维线性空间,
\(\vb{A}\)是\(V\)到\(V'\)的一个线性映射,
\(A\)是\(\vb{A}\)在\(V\)的基\(\AutoTuple{\alpha}{n}\)和\(V'\)的基\(\AutoTuple{\beta}{m}\)下的矩阵,
记\begin{equation*}
	\LinearMapMatrix(\Ker\vb{A},(\AutoTuple{\alpha}{n}))
	\defeq
	\Set{
		w \in F^n
		\given
		w = \LinearMapMatrix(v,(\AutoTuple{\alpha}{n})),
		v \in \Ker\vb{A}
	},
\end{equation*}
则\begin{equation*}
	\LinearMapMatrix(\Ker\vb{A},(\AutoTuple{\alpha}{n}))
	= \Ker A.
\end{equation*}
%TODO proof
\end{proposition}

\begin{example}
设线性映射\(T\colon \mathbb{R}^3 \to \mathbb{R}^2\)
在基\(
	\alpha_1 \defeq (-1,1,1)^T,
	\alpha_2 \defeq (1,0,-1)^T,
	\alpha_3 \defeq (0,1,1)^T
\)与\(
	\beta_1 \defeq (1,1)^T,
	\beta_2 \defeq (0,2)^T
\)下的矩阵表示为\begin{equation*}
	A \defeq \begin{bmatrix}
		1 & 1 & -1 \\
		0 & 1 & 2
	\end{bmatrix}
\end{equation*}
求\(T\)的核空间\(\Ker T\)与像空间\(\Img T\).
\begin{solution}
作初等行变换,得\begin{equation*}
	A
	= \begin{bmatrix}
		1 & 1 & -1 \\
		0 & 1 & 2
	\end{bmatrix}
	\to \begin{bmatrix}
		1 & 0 & -3 \\
		0 & 1 & 2
	\end{bmatrix},
\end{equation*}
那么\(\rank A = 2\),
线性方程组\(A (x_1,x_2,x_3)^T = (0,0)^T\)的基础解系是\((3,-2,1)^T\),
所以\(T\)的核空间为\(
	\Ker T
	= \Span\{3 \alpha_1 - 2 \alpha_2 + 1 \alpha_3\}
\),
\(T\)的像空间为\(
	\Img T
	= \Span\{
		\beta_1,
		\beta_1 + \beta_2
	\}
	= \mathbb{R}^2
\).
\end{solution}
\end{example}

\subsection{线性映射空间与矩阵空间同构}
从上面看到,
域\(F\)上\(n\)维线性空间\(V\)到\(s\)维线性空间\(V'\)的
每一个线性映射\(\vb{A}\)可以用一个\(s\times n\)矩阵\(A\)表示.
我们已经知道,\(V\)到\(V'\)的所有线性映射组成的集合\(\Hom(V,V')\)
是域\(F\)上的一个线性空间.
我们又知道,\(F\)上所有\(s\times n\)矩阵组成的集合\(M_{s\times n}(F)\)
也是域\(F\)上的一个线性空间.
容易证明,\(\Hom(V,V')\)与\(M_{s\times n}(F)\)同构.

\begin{theorem}\label{theorem:线性映射.线性映射空间与矩阵空间同构1}
%@see: 《高等代数(第三版 下册)》(丘维声) P119 定理1
%@see: 《Linear Algebra Done Right (Fourth Edition)》(Sheldon Axler) P87 3.71
%@see: 《Linear Algebra Done Right (Fourth Edition)》(Sheldon Axler) P87 3.32
设\(V\)和\(V'\)分别是域\(F\)上\(n\)维、\(s\)维线性空间,
则\begin{gather}
	\Hom(V,V') \Isomorphism M_{s\times n}(F), \\  % 同构
	\dim\Hom(V,V')
	=\dim M_{s\times n}(F)
	=sn.
\end{gather}
\begin{proof}
在\(V\)中取一个基\(\AutoTuple{\alpha}{n}\),
在\(V'\)中取一个基\(\AutoTuple{\beta}{s}\),
令\begin{equation*}
	\sigma \defeq \Set{
		(\vb{A},A)
		\given
		\vb{A} \in \Hom(V,V'),
		A \in M_{s \times n}(F),
		\vb{A}(\AutoTuple{\alpha}{n}) = (\AutoTuple{\beta}{s}) A
	}.
\end{equation*}
显然\(\sigma\)是从\(\Hom(V,V')\)到\(M_{s \times n}(F)\)的一个映射.

任给\(C \in M_{s \times n}(F)\),
令\begin{equation*}
%@see: 《高等代数(第三版 下册)》(丘维声) P118 (7)
	(\AutoTuple{\gamma}{n}) = (\AutoTuple{\beta}{s}) C,
\end{equation*}
显然\(\AutoTuple{\gamma}{n} \in V'\).
由\cref{theorem:线性映射.线性映射的存在性} 可知,
存在从\(V\)到\(V'\)的唯一一个线性映射\(\vb{C}\),
使得\begin{equation*}
	\vb{C} \alpha_j = \gamma_j,
	\quad j=1,2,\dotsc,n,
\end{equation*}
于是\begin{equation*}
%@see: 《高等代数(第三版 下册)》(丘维声) P118 (8)
	\vb{C} (\AutoTuple{\alpha}{n})
	= (\AutoTuple{\gamma}{n})
	= (\AutoTuple{\beta}{s}) C.
\end{equation*}
上式表明,\(C\)是线性映射\(\vb{C}\)的矩阵,
因此\(\sigma(\vb{C}) = C\).
这说明\(\sigma\)既是满射,也是单射.
换言之,\(\sigma\)是双射.

下面验证\(\sigma\)可以保持加法与纯量乘法运算.

设\(\vb{A},\vb{B} \in \Hom(V,V'),
k \in F,
\sigma(\vb{A}) = A,
\sigma(\vb{B}) = B\).
由于\begin{align*}
	(\vb{A}+\vb{B}) (\AutoTuple{\alpha}{n})
	&= (\vb{A}\alpha_1+\vb{B}\alpha_1,\dotsc,\vb{A}\alpha_n+\vb{B}\alpha_n) \\
	&= (\AutoTuple{\vb{A}\alpha}{n}) + (\AutoTuple{\vb{B}\alpha}{n}) \\
	&= (\AutoTuple{\beta}{s}) A + (\AutoTuple{\beta}{s}) B \\
	&= (\AutoTuple{\beta}{s}) (A+B),
\end{align*}
所以线性映射\(\vb{A}+\vb{B}\)的矩阵是\(A+B\),
从而\begin{equation*}
	\sigma(\vb{A}+\vb{B})
	= A+B
	= \sigma(\vb{A}) + \sigma(\vb{B}),
\end{equation*}
这表明\(\sigma\)保持加法运算.
由于\begin{align*}
	(k\vb{A}) (\AutoTuple{\alpha}{n})
	&= (\AutoTuple{k\vb{A}\alpha}{n}) \\
	&= k (\AutoTuple{\vb{A}\alpha}{n}) \\
	&= k ((\AutoTuple{\beta}{s}) A) \\
	&= (\AutoTuple{\beta}{s}) (kA),
\end{align*}
所以线性映射\(k\vb{A}\)的矩阵是\(kA\),
从而\begin{equation*}
	\sigma(k\vb{A})
	= kA
	= k \sigma(\vb{A}),
\end{equation*}
这表明\(\sigma\)保持纯量乘法运算.

综上所述,
\(\sigma\)是从\(\Hom(V,V')\)到\(M_{s \times n}(F)\)的一个同构.
\end{proof}
\end{theorem}

\begin{corollary}\label{theorem:线性映射.线性映射空间与矩阵空间同构2}
%@see: 《高等代数(第三版 下册)》(丘维声) P119 推论2
设\(V\)是域\(F\)上的\(n\)维线性空间,
则\begin{gather}
%@see: 《高等代数(第三版 下册)》(丘维声) P119 (12)
	\Hom(V,V) \Isomorphism M_n(F), \\  % 同构
%@see: 《高等代数(第三版 下册)》(丘维声) P119 (13)
	\dim\Hom(V,V) = \left(\dim V\right)^2.
\end{gather}
\end{corollary}

%@see: 《高等代数(第三版 下册)》(丘维声) P119
%@see: 《Linear Algebra Done Right (Fourth Edition)》(Sheldon Axler) P91 3.81
在\(\Hom(V,V)\)与\(M_n(F)\)中,都有乘法运算.
我们可以进一步证明:
把线性变换\(\vb{A}\)对应到它在\(V\)的基\(\AutoTuple{\alpha}{n}\)下的矩阵\(A\)的映射\(\sigma\)还保持乘法运算.

设线性变换\(\vb{B}\)在\(V\)的基\(\AutoTuple{\alpha}{n}\)下的矩阵是\(B\).
由于\begin{align*}
%@see: 《高等代数(第三版 下册)》(丘维声) P120 (14)
	&\hspace{-20pt}
	(\vb{A}\vb{B})(\AutoTuple{\alpha}{n}) \\
	&=\vb{A}(\vb{B}\alpha_1,\dotsc,\vb{B}\alpha_n) \\
	&=\vb{A}[(\AutoTuple{\alpha}{n})B] \\
	&=\vb{A}(b_{11}\alpha_1+\dotsb+b_{n1}\alpha_n,\dotsc,b_{1n}\alpha_1+\dotsb+b_{nn}\alpha_n) \\
	&=(b_{11}\vb{A}\alpha_1+\dotsb+b_{n1}\vb{A}\alpha_n,\dotsc,b_{1n}\vb{A}\alpha_1+\dotsb+b_{nn}\vb{A}\alpha_n) \\
	&=(\vb{A}\alpha_1,\dotsc,\vb{A}\alpha_n)
		\begin{bmatrix}
			b_{11} & \dots & b_{1n} \\
			\vdots & & \vdots \\
			b_{n1} & \dots & b_{nn}
		\end{bmatrix} \\
	&=[\vb{A}(\AutoTuple{\alpha}{n})]B \\
	&=[(\AutoTuple{\alpha}{n})A]B \\
	&=((\AutoTuple{\alpha}{n}))(AB),
\end{align*}
所以\(\vb{A}\vb{B}\)在基\(\AutoTuple{\alpha}{n}\)下的矩阵是\(AB\).
那么\begin{equation*}
%@see: 《高等代数(第三版 下册)》(丘维声) P120 (15)
	\sigma(\vb{A}\vb{B}) = AB = \sigma(\vb{A}) \sigma(\vb{B}).
\end{equation*}
这表明\(\sigma\)保持乘法运算.

从上述推导过程还可看到:\begin{equation*}
%@see: 《高等代数(第三版 下册)》(丘维声) P120 (16)
	\vb{A}[(\AutoTuple{\alpha}{n})B]
	= [\vb{A}(\AutoTuple{\alpha}{n})]B.
\end{equation*}

显然,\(V\)上的恒等变换\(\vb{I}\)在基\(\AutoTuple{\alpha}{n}\)下的矩阵是单位矩阵\(I\),
因此\(\sigma(\vb{I})=I\).

设\(V\)上线性变换\(\vb{A}\)在基\(\AutoTuple{\alpha}{n}\)下的矩阵是\(A\).
由于\begin{align*}
	&\text{线性变换$\vb{A}$可逆} \\
	&\iff \text{存在$V$上的线性变换$\vb{B}$使得$\vb{A}\vb{B}=\vb{B}\vb{A}=\vb{I}$} \\
	&\iff \text{存在$V$上的线性变换$\vb{B}$使得$\sigma(\vb{A}) \sigma(\vb{B}) = \sigma(\vb{B}) \sigma(\vb{A}) = \sigma(\vb{I})$} \\
	&\iff \text{存在域$F$上$n$阶矩阵$B$使得$AB=BA=I$} \\
	&\iff \text{矩阵$A$可逆},
\end{align*}
所以,\(V\)上线性变换\(\vb{A}\)可逆,
当且仅当它在\(V\)的一个基的矩阵\(A\)可逆.
从上述推导过程还可看到,
对于线性变换\(\vb{A},\vb{B}\),
假设它们在\(V\)的一个基下的矩阵分别是\(A,B\),
则\(\vb{B}\)是可逆线性变换\(\vb{A}\)的逆变换,
当且仅当\(B\)是可逆矩阵\(A\)的逆矩阵.

设\(\vb{A}\)是域\(F\)上\(n\)维线性空间\(V\)上的一个线性变换,
且\(\vb{A}\)在\(V\)的一个基\(\AutoTuple{\alpha}{n}\)下的矩阵是\(A\).
\(V\)中任一向量\(\alpha\)在基\(\AutoTuple{\alpha}{n}\)下的坐标记作\(X\).
由于\(\alpha=(\AutoTuple{\alpha}{n})X\),
所以\begin{align*}
	\vb{A}\alpha
	&= \vb{A}[(\AutoTuple{\alpha}{n})X]
	= [\vb{A}(\AutoTuple{\alpha}{n})]X \\
	&= [(\AutoTuple{\alpha}{n})A]X
	= (\AutoTuple{\alpha}{n})(AX).
\end{align*}
这表明\(\vb{A}\alpha\)在基\(\AutoTuple{\alpha}{n}\)下的坐标是\(AX\),
即\begin{equation*}
%@see: 《Linear Algebra Done Right (Fourth Edition)》(Sheldon Axler) P89 3.76
	\LinearMapMatrix(\vb{T}\alpha)
	= \LinearMapMatrix(\vb{T}) \VectorMatrix(\alpha).
\end{equation*}

由于\(V\)中两个向量相等,
当且仅当它们在\(V\)的一个基下的坐标相等,
因此,如果向量\(\gamma\)在基\(\AutoTuple{\alpha}{n}\)下的坐标是\(Y\),
则\begin{equation*}
%@see: 《高等代数(第三版 下册)》(丘维声) P120 (17)
	\vb{A}\alpha=\gamma
	\iff
	AX=Y.
\end{equation*}

\begin{example}
%@see: 《高等代数(第三版 下册)》(丘维声) P121 习题9.3 1.
设\(\vb{A}\)是\(K^3\)上的一个线性变换:\begin{equation*}
	\vb{A}
	\begin{bmatrix}
		x_1 \\ x_2 \\ x_3
	\end{bmatrix}
	= \begin{bmatrix}
		x_1 + 2x_2 \\
		x_3 - x_2 \\
		x_2 - x_3
	\end{bmatrix}.
\end{equation*}
求\(\vb{A}\)在标准基\(\AutoTuple{\epsilon}{3}\)下的矩阵.
\begin{solution}
注意到\begin{equation*}
	\begin{bmatrix}
		x_1 + 2x_2 \\
		x_3 - x_2 \\
		x_2 - x_3
	\end{bmatrix}
	= \begin{bmatrix}
		1 & 2 & 0 \\
		0 & -1 & 1 \\
		0 & 1 & -1
	\end{bmatrix}
	\begin{bmatrix}
		x_1 \\ x_2 \\ x_3
	\end{bmatrix},
\end{equation*}
于是\(\vb{A}\)在标准基\(\AutoTuple{\epsilon}{3}\)下的矩阵为\begin{equation*}
	A = \begin{bmatrix}
		1 & 2 & 0 \\
		0 & -1 & 1 \\
		0 & 1 & -1
	\end{bmatrix}.
\end{equation*}
\end{solution}
\end{example}

\begin{example}
%@see: 《高等代数(第三版 下册)》(丘维声) P121 习题9.3 2.
在映射空间\(\mathbb{R}^\mathbb{R}\)中,
令\(V \defeq \Span\{f_1,f_2\}\),
其中\(f_1(x) = e^{ax} \cos bx,
f_2(x) = e^{ax} \sin bx\).
%@Mathematica: f1[x_] := Exp[a x] Cos[b x]
%@Mathematica: f2[x_] := Exp[a x] Sin[b x]
证明求导数\(\vb{D}\)是\(V\)上的一个线性变换,
求\(\vb{D}\)在基\(f_1,f_2\)下的矩阵.
\begin{solution}
任取\(u,v \in V\),
其中\(u = p_1 f_1 + p_2 f_2,
v = q_1 f_1 + q_2 f_2\),
任取\(k \in \mathbb{R}\),
则\begin{align*}
%@Mathematica: D[u[x] + v[x], x]
%@Mathematica: D[k u[x], x]
	\vb{D}(u + v)
	&= a e^{a x} p_1 \cos b x
		+ b e^{a x} p_2 \cos b x
		+ a e^{a x} q_1 \cos b x
		+ b e^{a x} q_2 \cos b x \\
	&\hspace{20pt}
		- b e^{a x} p_1 \sin b x
		+ a e^{a x} p_2 \sin b x
		- b e^{a x} q_1 \sin b x
		+ a e^{a x} q_2 \sin b x \\
	&= \vb{D}u + \vb{D}v, \\
	\vb{D}(k u)
	&= k (a e^{a x} p_1 \cos b x + b e^{a x} p_2 \cos b x -
		b e^{a x} p_1 \sin b x + a e^{a x} p_2 \sin b x) \\
	&= k \vb{D}u,
\end{align*}
这就说明\(\vb{D}\)是\(V\)上的线性变换.

因为\begin{gather*}
	\vb{D}f_1 = e^{ax} (a \cos bx - b \sin bx)
	= a f_1 - b f_2, \\
	\vb{D}f_2 = e^{ax} (a \sin bx + b \cos bx)
	= b f_1 + a f_2,
%@Mathematica: D[f1[x], x] // Factor
%@Mathematica: D[f2[x], x] // Factor
\end{gather*}
所以\(\vb{A}\)在基\(f_1,f_2\)下的矩阵为\begin{equation*}
	A = \begin{bmatrix}
		a & b \\
		-b & a
	\end{bmatrix}.
\end{equation*}
\end{solution}
\end{example}

\subsection{线性变换在不同基下的矩阵的关系}
域\(F\)上\(n\)维线性空间\(V\)上的一个线性变换\(\vb{A}\)在\(V\)的不同基下的矩阵有什么关系?

\begin{proposition}
%@see: 《Linear Algebra Done Right (Fourth Edition)》(Sheldon Axler) P92 3.82
设\(V\)是域\(F\)上\(n\)维线性空间,
\(V\)上的恒等变换\(\vb{I}\)在\(V\)的两个基
\(\AutoTuple{\alpha}{n}\)与\(\AutoTuple{\beta}{n}\)下的矩阵分别为\(A,B\),
则\(A,B\)都是可逆矩阵,且\(A,B\)互为逆矩阵.
\end{proposition}

\begin{theorem}\label{theorem:线性映射的矩阵表示.线性变换在不同基下的矩阵相似}
%@see: 《高等代数(第三版 下册)》(丘维声) P120 定理3
%@see: 《Linear Algebra Done Right (Fourth Edition)》(Sheldon Axler) P93 3.84
设\(V\)是域\(F\)上\(n\)维线性空间,
\(V\)上的一个线性变换\(\vb{A}\)在\(V\)的两个基
\(\AutoTuple{\alpha}{n}\)与\(\AutoTuple{\beta}{n}\)下的矩阵分别为\(A,B\).
从基\(\AutoTuple{\alpha}{n}\)到基\(\AutoTuple{\beta}{n}\)的过渡矩阵是\(S\),
%@see: 《高等代数(第三版 下册)》(丘维声) P120 (18)
则\(B = S^{-1} A S\).
\begin{proof}
由已知条件有\begin{gather*}
	\vb{A}(\AutoTuple{\alpha}{n})
	= (\AutoTuple{\alpha}{n}) A, \\
	\vb{A}(\AutoTuple{\beta}{n})
	= (\AutoTuple{\beta}{n}) B, \\
	(\AutoTuple{\beta}{n})
	= (\AutoTuple{\alpha}{n}) S,
\end{gather*}
于是\begin{align*}
	(\AutoTuple{\beta}{n}) S^{-1}
	&= ((\AutoTuple{\alpha}{n}) S) S^{-1} \\
	&= (\AutoTuple{\alpha}{n}) (S S^{-1}) \\
	&= (\AutoTuple{\alpha}{n}),
\end{align*}
从而\begin{align*}
%@see: 《高等代数(第三版 下册)》(丘维声) P120 (19)
	\vb{A}(\AutoTuple{\beta}{n})
	&= \vb{A}((\AutoTuple{\alpha}{n}) S) \\
	&= (\vb{A}(\AutoTuple{\alpha}{n})) S \\
	&= ((\AutoTuple{\alpha}{n}) A) S \\
	&= (\AutoTuple{\alpha}{n}) (A S) \\
	&= ((\AutoTuple{\beta}{n}) S^{-1}) (AS) \\
	&= (\AutoTuple{\beta}{n}) (S^{-1} A S).
\end{align*}
上式表明,\(\vb{A}\)在基\(\AutoTuple{\beta}{n}\)下的矩阵是\(S^{-1} A S\).
由于\(\vb{A}\)在基\(\AutoTuple{\beta}{n}\)下的矩阵是唯一的,
因此\(B = S^{-1} A S\).
\end{proof}
\end{theorem}
可以看出,同一个线性变换\(\vb{A}\)在\(V\)的不同基下的矩阵是相似的.
反之,我们有如下命题:
\begin{proposition}
%@see: 《高等代数(第三版 下册)》(丘维声) P121 命题4
如果域\(F\)上\(n\)阶矩阵\(A\)与\(B\)相似,
那么\(A\)与\(B\)可以看成是域\(F\)上\(n\)维线性空间\(V\)上的
一个线性变换\(\vb{A}\)在\(V\)的不同基下的矩阵.
\begin{proof}
由于\(A\)与\(B\)相似,
因此有可逆矩阵\(S\),
使得\(B = S^{-1} A S\).
设\(V\)是域\(F\)上的\(n\)维线性空间,
在\(V\)中取一个基\(\AutoTuple{\alpha}{n}\).
从\cref{theorem:线性映射.线性映射空间与矩阵空间同构1} 的证明过程中可以看出,
存在\(V\)上唯一一个线性变换\(\vb{A}\),
使得\(\vb{A}\)在\(V\)的基\(\AutoTuple{\alpha}{n}\)下的矩阵为\(A\).
令\begin{equation*}
	(\AutoTuple{\beta}{n}) = (\AutoTuple{\alpha}{n}) S.
\end{equation*}
由\cref{theorem:线性空间.命题14} 可知
\(\AutoTuple{\beta}{n}\)是\(V\)的一个基.
设\(\vb{A}\)在基\(\AutoTuple{\beta}{n}\)下的矩阵为\(C\),
则根据\cref{theorem:线性映射的矩阵表示.线性变换在不同基下的矩阵相似} 得
\(C = S^{-1} A S\),
从而\(C = B\).
因此\(\vb{A}\)在基\(\vb{A}\)在基\(\AutoTuple{\beta}{n}\)下的矩阵为\(B\).
\end{proof}
\end{proposition}

\begin{theorem}\label{theorem:线性映射的矩阵表示.线性映射在不同基下的矩阵等价}
设\(V_1\)、\(V_2\)分别是域\(F\)上\(n\)维、\(m\)维线性空间,
\(\vb{A}\)是\(V_1\)到\(V_2\)的一个线性映射,
\(A\)是\(\vb{A}\)在\(V_1\)的基\(\AutoTuple{\alpha}{n}\)和\(V_2\)的基\(\AutoTuple{\beta}{m}\)下的矩阵,
\(B\)是\(\vb{A}\)在\(V_1\)的基\(\AutoTuple{\alpha'}{n}\)和\(V_2\)的基\(\AutoTuple{\beta'}{m}\)下的矩阵,
从基\(\AutoTuple{\alpha}{n}\)到基\(\AutoTuple{\alpha'}{n}\)的过渡矩阵是\(P\),
从基\(\AutoTuple{\beta}{m}\)到基\(\AutoTuple{\beta'}{m}\)的过渡矩阵是\(Q\),
则\begin{equation*}
	B = Q^{-1} A P.
\end{equation*}
\begin{proof}
由已知条件有\begin{gather*}
	\vb{A}(\AutoTuple{\alpha}{n})
	= (\AutoTuple{\beta}{m}) A, \\
	\vb{A}(\AutoTuple{\alpha'}{n})
	= (\AutoTuple{\beta'}{m}) B, \\
	(\AutoTuple{\alpha'}{n})
	= (\AutoTuple{\alpha}{n}) P, \\
	(\AutoTuple{\beta'}{m})
	= (\AutoTuple{\beta}{m}) Q,
\end{gather*}
于是\begin{align*}
	\vb{A}(\AutoTuple{\alpha'}{n})
	&= \vb{A}((\AutoTuple{\alpha}{n}) P) \\
	&= (\vb{A}(\AutoTuple{\alpha}{n})) P \\
	&= ((\AutoTuple{\beta}{m}) A) P \\
	&= (\AutoTuple{\beta}{m}) (A P) \\
	&= ((\AutoTuple{\beta'}{m}) Q^{-1}) (A P) \\
	&= (\AutoTuple{\beta'}{m}) (Q^{-1} A P),
\end{align*}
上式表明,\(\vb{A}\)在\(V_1\)的基\(\AutoTuple{\alpha'}{n}\)和\(V_2\)的基\(\AutoTuple{\beta'}{m}\)下的矩阵是\(Q^{-1} A P\).
由于\(\vb{A}\)在\(V_1\)的基\(\AutoTuple{\alpha'}{n}\)和\(V_2\)的基\(\AutoTuple{\beta'}{m}\)下的矩阵是唯一的,
因此\(B = Q^{-1} A P\).
\end{proof}
\end{theorem}
\begin{remark}
\cref{theorem:线性映射的矩阵表示.线性映射在不同基下的矩阵等价} 说明:
同一个线性映射\(\vb{A}\)在\(V_1\)、\(V_2\)的不同的两对基下的矩阵是等价的.
\end{remark}

\begin{proposition}
%@see: 《Linear Algebra Done Right (Fourth Edition)》(Sheldon Axler) P93 3.86
设\(V\)是域\(F\)上\(n\)维线性空间,
\(\vb{A}\)是\(V\)上的一个可逆线性变换,
\(\AutoTuple{\alpha}{n}\)的一个基,
则\(\vb{A}\)的逆\(\vb{A}^{-1}\)的矩阵
就是\(\vb{A}\)的矩阵的逆,
即\begin{equation*}
	\LinearMapMatrix(\vb{A}^{-1})
	= (\LinearMapMatrix(\vb{A}))^{-1}.
\end{equation*}
\end{proposition}

\section{线性函数与对偶空间}
%TODO 对偶空间与傅里叶级数、傅里叶变换有密切的联系,详见以下视频:
%@see: https://www.bilibili.com/video/BV1xxivYzEh8/
%@see: https://mathvideos.org/2021/richard-borcherds-rings-and-modules/

本节着重讨论一种特殊的线性映射 --- \hyperref[definition:线性映射.线性函数]{线性函数}.
\subsection{线性函数}
除了\cref{example:线性映射.给定区间上的定积分是线性函数} 中举出的
给定区间上的定积分是线性函数以外,
下面我们额外举几个例子.

\begin{example}
%@see: 《高等代数(第三版 下册)》(丘维声) P160
矩阵的迹,
是\(M_n(F)\)上的一个线性函数,
它把域\(F\)上每一个\(n\)阶矩阵,
对应到\(F\)中的一个元素,
并且保持加法与数量乘法.
\end{example}

\begin{example}
%@see: 《高等代数(第三版 下册)》(丘维声) P161 例1
设\(F\)是一个域,\(\AutoTuple{a}{n} \in F\),
令\begin{equation*}
	f\colon F^n \to F,
	(\AutoTuple{x}{n}) \mapsto a_1 x_1 + \dotsb + a_n x_n,
\end{equation*}
容易验证\(f\)是\(F^n\)上的一个线性函数.
\end{example}

\begin{example}
%@see: 《高等代数(第三版 下册)》(丘维声) P161 例2
\def\Z{\mathbb{Z}_2}
设\(\Z\)是模\(2\)剩余类域.
令\begin{equation*}
	f\colon \Z^2 \to \Z,
	(x_1,x_2) \mapsto x_1^2+x_2^2.
\end{equation*}
试判断\(f\)是不是\(\Z^2\)上的一个线性函数.
\begin{solution}
任取\((x_1,x_2),(y_1,y_2)\in\Z^2\),
有\begin{align*}
	f((x_1,x_2)+(y_1,y_2))
	&= f(x_1+y_1,x_2+y_2) \\  % 剩余类域上线性空间的加法
	&= (x_1+y_1)^2+(x_2+y_2)^2 \\  % 题设
	&= x_1^2+y_1^2+x_2^2+y_2^2
		\tag{\cref{example:域.域上的特征恒等式}} \\
	&= f(x_1,x_2)+f(y_1,y_2), \\  % 题设
	f(1\cdot(x_1,x_2))
	&= f(x_1,x_2)  % 剩余类域上线性空间的纯量乘法
	= 1\cdot f(x_1,x_2), \\  % 剩余类域上的乘法
	f(0\cdot(x_1,x_2))
	&= f(0,0)  % 剩余类域上线性空间的纯量乘法
	= 0 = 0\cdot f(x_1,x_2).
\end{align*}
因此\(f\)是\(\Z^2\)上的一个线性函数.
\end{solution}
\end{example}

\subsection{对偶空间}
\begin{definition}
%@see: 《高等代数(第三版 下册)》(丘维声) P162
%@see: 《Linear Algebra Done Right (Fourth Eidition)》(Sheldon Axler) P105 3.110
设\(V\)是域\(F\)上的一个线性空间,
把\(\Hom(V,F)\)称为“\(V\)上的\DefineConcept{线性函数空间}”
或“\(V\)的\DefineConcept{对偶空间}(dual space)”,
简记为\(V^*\).
\end{definition}

\begin{proposition}\label{theorem:对偶空间.对偶空间的维数}
%@see: 《高等代数(第三版 下册)》(丘维声) P162
%@see: 《Linear Algebra Done Right (Fourth Eidition)》(Sheldon Axler) P105 3.111
设\(V\)是域\(F\)上的\(n\)维线性空间,
\(V^*\)是\(V\)的对偶空间,
%@see: 《高等代数(第三版 下册)》(丘维声) P162 (4)
则\(\dim V^* = n\),
%@see: 《高等代数(第三版 下册)》(丘维声) P162 (5)
且\(V^* \Isomorphism V\).
\begin{proof}
因为\(\dim F = 1\),
所以由\cref{theorem:线性映射.线性映射空间与矩阵空间同构1} 可知
\(\dim V^*
= (\dim V)(\dim F)
= n \cdot 1
= n\).
再由\cref{theorem:线性空间的同构.线性空间同构的充分必要条件} 可知
\(V^* \Isomorphism V\).
\end{proof}
\end{proposition}
\begin{remark}
我们可以在\(V\)与\(V^*\)之间构造一个同构:
在\(V\)中取一个基\(\AutoTuple{\alpha}{n}\),
假设它的对偶基是\(\AutoTuple{\phi}{n}\),
那么映射\begin{equation}\label{equation:对偶空间.线性空间及其对偶空间之间的一个同构}
%@see: 《高等代数(第三版 下册)》(丘维声) P164 (15)
	\sigma\colon V \to V^*,
	\alpha = \sum_{i=1}^n x_i \alpha_i \mapsto \sum_{i=1}^n x_i \phi_i
\end{equation}
就是\(\sigma\)是从\(V\)到\(V^*\)的一个同构.
\end{remark}

\subsection{对偶基}
%\cref{theorem:线性映射.线性映射的存在性}
\begin{definition}
%@see: 《高等代数(第三版 下册)》(丘维声) P162
%@see: 《Linear Algebra Done Right (Fourth Eidition)》(Sheldon Axler) P106 3.112
设\(V\)是域\(F\)上\(n\)维线性空间,
\(\AutoTuple{\alpha}{n}\)是\(V\)的一个基.
定义映射:\begin{gather}
%@see: 《高等代数(第三版 下册)》(丘维声) P162 (6)
	\phi_i(\alpha_j)
	\defeq \left\{ \begin{array}{cl}
		1, & j = i, \\
		0, & j \neq i,
	\end{array} \right.
	\quad i=1,2,\dotsc,n,
		\label{equation:对偶空间.对偶基1} \\
%@see: 《高等代数(第三版 下册)》(丘维声) P162 (7)
	\phi_i(\alpha_j+\alpha_k)
	\defeq \phi_i(\alpha_j) + \phi_i(\alpha_k),
	\quad i=1,2,\dotsc,n,
		\label{equation:对偶空间.对偶基2} \\
	\phi_i(\lambda \alpha_j)
	\defeq \lambda \phi_i(\alpha_j),
	\quad i=1,2,\dotsc,n.
		\label{equation:对偶空间.对偶基3}
\end{gather}
把\(\AutoTuple{\phi}{n}\)
称为“\(\AutoTuple{\alpha}{n}\)的\DefineConcept{对偶基}%
(the \emph{dual basis} of \(\AutoTuple{\alpha}{n}\))”.
\end{definition}

\begin{proposition}\label{theorem:对偶空间.对偶基的性质}
%@see: 《高等代数(第三版 下册)》(丘维声) P162
%@see: 《Linear Algebra Done Right (Fourth Eidition)》(Sheldon Axler) P106 3.112
%@see: 《Linear Algebra Done Right (Fourth Eidition)》(Sheldon Axler) P106 3.114
%@see: 《Linear Algebra Done Right (Fourth Eidition)》(Sheldon Axler) P107 3.116
设\(V\)是域\(F\)上\(n\)维线性空间,
\(V^*\)是\(V\)的对偶空间,
\(\AutoTuple{\alpha}{n}\)是\(V\)的一个基,
\(\AutoTuple{\phi}{n}\)是\(\AutoTuple{\alpha}{n}\)的对偶基,
则\begin{itemize}
	\item \(\AutoTuple{\phi}{n}\)都是\(V\)上的线性函数,
	从而对于\(\forall \AutoTuple{x}{n} \in F\),
	有\begin{equation*}
	%@see: 《高等代数(第三版 下册)》(丘维声) P162 (7)
		\phi_i\left( \sum_{j=1}^n x_j \alpha_j \right)
		= \sum_{j=1}^n x_j \phi_i(\alpha_j)
		= x_i,
		\quad i=1,2,\dotsc,n;
	\end{equation*}

	\item \(\AutoTuple{\phi}{n}\)是\(V^*\)的一个基;

	\item \(V\)中任意一个向量\(\alpha\)
	在基\(\AutoTuple{\alpha}{n}\)下的坐标的第\(i\)个分量,
	等于\(\AutoTuple{\alpha}{n}\)的对偶基的第\(i\)个线性函数\(\phi_i\)
	在\(\alpha\)的值\(\phi_i(\alpha)\),
	即\begin{equation*}
	%@see: 《高等代数(第三版 下册)》(丘维声) P163 (9)
		(\forall \alpha \in V)
		[\alpha = \phi_1(\alpha) \alpha_1 + \dotsb + \phi_n(\alpha) \alpha_n];
	\end{equation*}

	\item \(V^*\)中任意一个线性函数\(\phi\)
	在基\(\AutoTuple{\phi}{n}\)下的坐标的第\(i\)个分量,
	等于\(\phi\)在\(\alpha_i\)的值\(\phi(\alpha_i)\),
	即\begin{equation*}
	%@see: 《高等代数(第三版 下册)》(丘维声) P163 (11)
		(\forall \phi \in V^*)
		[\phi = \phi(\alpha_1) \phi_1 + \dotsb + \phi(\alpha_n) \phi_n].
	\end{equation*}
\end{itemize}
\begin{proof}
由\cref{equation:对偶空间.对偶基2} 可知\(\phi_i\)适合可加性,
由\cref{equation:对偶空间.对偶基3} 可知\(\phi_i\)适合齐次性,
因此\(\phi_i\)是\(V\)上的线性函数.
再由\cref{theorem:线性映射.线性映射的性质} 可知,
对于\(\forall \AutoTuple{x}{n} \in F\),
有\begin{equation*}
%@see: 《高等代数(第三版 下册)》(丘维声) P162 (7)
	\phi_i\left( \sum_{j=1}^n x_j \alpha_j \right)
	= \sum_{j=1}^n x_j \phi_i(\alpha_j)
	= x_i,
	\quad i=1,2,\dotsc,n;
\end{equation*}

令\begin{equation*}
	x_1 \phi_1 + \dotsb + x_n \phi_n = 0,
\end{equation*}
则\begin{eqnarray}
	(x_1 \phi_1 + \dotsb + x_n \phi_n) \alpha_j = 0,
	\quad j=1,2,\dotsc,n.
\end{eqnarray}
根据线性映射的加法、纯量乘法的定义得\begin{equation*}
	x_1 \phi_1(\alpha_j) + \dotsb + x_j \phi_j(\alpha_j) + \dotsb + x_n \phi_n(\alpha_j) = 0,
	\quad j=1,2,\dotsc,n.
\end{equation*}
于是\(x_j = 0\ (j=1,2,\dotsc,n)\),
这就说明\(\AutoTuple{\phi}{n}\)线性无关.
再由\cref{theorem:对偶空间.对偶空间的维数,theorem:线性空间.线性相关性3} 可知
\(\AutoTuple{\phi}{n}\)是\(V^*\)的一个基.

对于任意\(\alpha \in V\),
% 由\cref{theorem:线性空间.任一向量可由给定基唯一线性表出} 可知,
存在\(\AutoTuple{k}{n} \in F\),
使得\begin{equation*}
	\alpha = \sum_{j=1}^n k_j \alpha_j.
\end{equation*}
那么有\begin{equation*}
	\phi_i(\alpha)
	= \phi_i\left( \sum_{j=1}^n k_j \alpha_j \right)
	= k_i,
\end{equation*}
于是\begin{equation*}
	\alpha = \sum_{j=1}^n \phi_j(\alpha) \alpha_j.
\end{equation*}

对于任意\(\phi \in V^*\),
存在\(\AutoTuple{k}{n} \in F\),
使得\begin{equation*}
	\phi = \sum_{j=1}^n k_j \phi_j.
\end{equation*}
那么有\begin{equation*}
	\phi(\alpha_i)
	= \left( \sum_{j=1}^n k_j \phi_j \right)(\alpha_i)
	= \sum_{j=1}^n k_j \phi_j(\alpha_i)
	= k_i,
\end{equation*}
于是\begin{equation*}
	\phi = \sum_{j=1}^n \phi(\alpha_j) \phi_j.
	\qedhere
\end{equation*}
\end{proof}
\end{proposition}

\subsection{不同对偶基之间的过渡矩阵}
\begin{theorem}
%@see: 《高等代数(第三版 下册)》(丘维声) P163 定理1
设\(V\)是域\(F\)上一个\(n\)维线性空间,
在\(V\)中取两个基\(\AutoTuple{\alpha}{n}\)与\(\AutoTuple{\beta}{n}\),
它们的对偶基分别为\(\AutoTuple{\phi}{n}\)与\(\AutoTuple{\psi}{n}\).
如果基\(\AutoTuple{\alpha}{n}\)到基\(\AutoTuple{\beta}{n}\)的过渡矩阵是\(A\),
%@see: 《高等代数(第三版 下册)》(丘维声) P163 (12)
则基\(\AutoTuple{\phi}{n}\)到基\(\AutoTuple{\psi}{n}\)的过渡矩阵为\((A^{-1})^T\).
\begin{proof}
由\((\AutoTuple{\beta}{n}) = (\AutoTuple{\alpha}{n}) A\)得
%@see: 《高等代数(第三版 下册)》(丘维声) P163 (13)
\((\AutoTuple{\alpha}{n}) = (\AutoTuple{\beta}{n}) A^{-1}\),
这就说明\(\alpha_j\)在基\(\AutoTuple{\beta}{n}\)下的坐标是\(A^{-1}\)的第\(j\)列,
从而坐标的第\(i\)个分量为\(\MatrixEntry{A^{-1}}{i,j}\).
由于\(\AutoTuple{\beta}{n}\)的对偶基是\(\AutoTuple{\psi}{n}\),
可知\(\alpha_j\)在基\(\AutoTuple{\beta}{n}\)下的坐标的第\(i\)个分量等于\(\psi_i(\alpha_j)\).
因此\(\MatrixEntry{A^{-1}}{i,j} = \psi_i(\alpha_j)\).

由于\(\AutoTuple{\alpha}{n}\)的对偶基是\(\AutoTuple{\phi}{n}\),
可知\(\psi_i(\alpha_j)\)等于\(\psi_i\)在\(\AutoTuple{\phi}{n}\)下的坐标的第\(j\)个分量.
%@see: 《高等代数(第三版 下册)》(丘维声) P163 (14)
由已知条件可知\((\AutoTuple{\psi}{n}) = (\AutoTuple{\phi}{n}) B\),
这就说明\(\psi_i\)在基\(\AutoTuple{\phi}{n}\)下的坐标的第\(j\)个分量等于\(\MatrixEntry{B}{j,i}\).
因此\(\psi_i(\alpha_j) = \MatrixEntry{B}{j,i}\).

综上所述,有\(\MatrixEntry{A^{-1}}{i,j}
= \psi_i(\alpha_j)
= \MatrixEntry{B}{j,i}
= \MatrixEntry{B^T}{i,j}\),
其中\(i,j=1,2,\dotsc,n\).
因此\(A^{-1} = B^T\),
换言之\(B = (A^{-1})^T\).
\end{proof}
\end{theorem}

\subsection{双重对偶空间}
给定域\(F\)上的一个\(n\)维线性空间\(V\),
我们知道\(V\)的对偶空间\(V^*\)也是域\(F\)上的一个\(n\)维线性空间,
那么我们可以考虑\(V^*\)的对偶空间\((V^*)^*\).

\begin{definition}
%@see: 《高等代数(第三版 下册)》(丘维声) P164
设\(V\)是域\(F\)上的一个线性空间,
\(V^*\)是\(V\)的对偶空间.
把\(V^*\)的对偶空间\((V^*)^*\)
称为“\(V\)的\DefineConcept{双重对偶空间}”,
简记为\(V^{**}\).
\end{definition}

\begin{corollary}
%@see: 《高等代数(第三版 下册)》(丘维声) P164
设\(V\)是域\(F\)上的一个\(n\)维线性空间,
\(V^{**}\)是\(V\)的双重对偶空间,
则\(V \Isomorphism V^{**}\).
\begin{proof}
由\cref{theorem:对偶空间.对偶空间的维数}
可知\(V \Isomorphism V^*,
V^* \Isomorphism V^{**}\),
利用传递性可得\(V \Isomorphism V^{**}\).
\end{proof}
\end{corollary}

给定线性空间\(V\)及其对偶空间\(V^*\)之间的同构\(\sigma\)
(见\cref{equation:对偶空间.线性空间及其对偶空间之间的一个同构}),
对于\(\forall \beta = \sum_{i=1}^n y_i \alpha_i \in V\),
有\begin{equation*}
%@see: 《高等代数(第三版 下册)》(丘维声) P164 (16)
	\sigma(\alpha)(\beta)
	= \left( \sum_{i=1}^n x_i \phi_i \right)(\beta)
	= \sum_{i=1}^n x_i \phi_i(\beta)
	= \sum_{i=1}^n x_i y_i.
\end{equation*}
上式表明,\(\alpha\)在\(\sigma\)下的像\(\sigma(\alpha)\)
在\(\beta\)的值\(\sigma(\alpha)(\beta)\)
等于\(\alpha\)与\(\beta\)的坐标的对应分量乘积之和.

假设\(\rho\)是\(V^*\)及其对偶空间\(V^{**} = \Hom(V^*,F)\)之间的一个同构,
记\(\alpha^{**} \defeq \rho(\sigma(\alpha))\),
那么对于\(V^*\)中任意一个线性函数\(\phi\),
由上述讨论可知,
\(\alpha^{**}(\phi)\)
等于\(\sigma(\alpha)\)与\(\phi\)在基\(\AutoTuple{\phi}{n}\)下的坐标的对应分量乘积之和,
具体地说,鉴于\begin{equation*}
%@see: 《高等代数(第三版 下册)》(丘维声) P164 (18)
	\sigma(\alpha)
	= \sum_{i=1}^n x_i \phi_i,  %\cref{equation:对偶空间.线性空间及其对偶空间之间的一个同构}
	\qquad
	\phi = \sum_{i=1}^n \phi(\alpha_i) \phi_i,  %\cref{theorem:对偶空间.对偶基的性质}
\end{equation*}
便有\begin{equation*}
%@see: 《高等代数(第三版 下册)》(丘维声) P164 (19)
	\alpha^{**}(\phi)
	= \sum_{i=1}^n x_i \phi(\alpha_i)
	= \phi\left( \sum_{i=1}^n x_i \alpha_i \right)
	= \phi(\alpha).
\end{equation*}

记\(\tau \defeq \rho \circ \sigma\),
显然\(\tau\)是从\(V\)到\(V^{**}\)的一个同构.
由\begin{equation*}
%@see: 《高等代数(第三版 下册)》(丘维声) P164 (20)
%@see: 《高等代数(第三版 下册)》(丘维声) P164 (21)
	(\forall \alpha \in V)
	(\forall \phi \in V^*)
	[
		\tau(\alpha)(\phi)
		= \phi(\alpha)
	]
\end{equation*}
可以看出,\(\alpha\)在\(\tau\)下的像\(\tau(\alpha) = \alpha^{**}\)不依赖于\(V\)中基的选择.

\begin{definition}
%@see: 《高等代数(第三版 下册)》(丘维声) P164
设\(V\)是域\(F\)上的一个线性空间,
\(V^*\)是\(V\)的对偶空间,
\(V^{**}\)是\(V\)的双重对偶空间.
如果同构\(\tau\colon V \to V^{**}\)
满足\begin{equation*}
	(\forall \alpha \in V)
	(\forall \phi \in V^*)
	[
		\tau(\alpha)(\phi)
		= \phi(\alpha)
	],
\end{equation*}
则称“\(\tau\)是从\(V\)到\(V^{**}\)的\DefineConcept{标准同构}”
或“\(\tau\)是从\(V\)到\(V^{**}\)的\DefineConcept{自然同构}”.
\end{definition}

由于\(V\)与\(V^{**}\)之间存在自然同构,
因此可以把\(V\)视同\(V^{**}\),
从而把\(V\)看成是\(V^*\)的对偶空间,
这样\(V\)与\(V^*\)就互为对偶空间.
这就是我们把\(V^*\)称为\(V\)的对偶空间的原因.

\subsection{对偶映射}
\begin{definition}
%@see: 《Linear Algebra Done Right (Fourth Eidition)》(Sheldon Axler) P107 3.118
设\(V,W\)都是域\(F\)上线性空间,
\(V^*,W^*\)分别是\(V,W\)的对偶空间,
\(T\)是从\(V\)到\(W\)的一个线性映射,
\(T^*\)是从\(W^*\)到\(V^*\)的映射.
如果对于\(\forall \phi \in W^*\)
都有\begin{equation*}
	T^*(\phi) = \phi \circ T,
\end{equation*}
则称“\(T^*\)是\(T\)的\DefineConcept{对偶映射}(the \emph{dual map} of \(T\))”.
\end{definition}

\begin{proposition}
%@see: 《Linear Algebra Done Right (Fourth Eidition)》(Sheldon Axler) P107
设\(V,W\)都是域\(F\)上线性空间,
\(V^*,W^*\)分别是\(V,W\)的对偶空间,
\(T\)是从\(V\)到\(W\)的一个线性映射,
\(T^*\)是\(T\)的对偶映射,
则\(T^*\)是从\(W^*\)到\(V^*\)的一个线性映射.
\begin{proof}
对于\(\forall \phi,\psi \in W^*\)和\(\forall \lambda \in F\),
有\begin{gather*}
	T^*(\phi+\psi)
	= (\phi+\psi) \circ T
	= \phi \circ T + \psi \circ T
	= T^*(\phi) + T^*(\psi), \\
	T^*(\lambda\phi)
	= (\lambda\phi) \circ T
	= \lambda (\phi \circ T)
	= \lambda T^*(\phi).
	\qedhere
\end{gather*}
\end{proof}
\end{proposition}

\begin{property}
%@see: 《Linear Algebra Done Right (Fourth Eidition)》(Sheldon Axler) P108 3.120
设\(V,W\)都是域\(F\)上线性空间,
\(T\)是从\(V\)到\(W\)的一个线性映射,
则\begin{itemize}
	\item 对于从\(V\)到\(W\)的任意一个线性映射\(S\),有\begin{equation*}
		(S+T)^* = S^* + T^*;
	\end{equation*}
	\item 对于任意\(\lambda \in F\),有\begin{equation*}
		(\lambda T)^* = \lambda T^*;
	\end{equation*}
	\item 对于从\(W\)到\(U\)的任意一个线性映射\(S\),有\begin{equation*}
		(S T)^* = T^* S^*.
	\end{equation*}
\end{itemize}
%TODO proof
\end{property}


\chapter{线性变换的特征值与特征向量}
\begingroup
\NewDocumentCommand\LambdaExp{mo}{(\lambda-\lambda_{#1})\IfValueTF{#2}{^{#2}}{}}%
\NewDocumentCommand\EigenPoly{mmo}{(#1-\lambda_{#2}\vb{I})\IfValueTF{#3}{^{#3}}{}}%
\NewDocumentCommand\RestrictedEigenPoly{mmo}{\EigenPoly{#1 \SetRestrict W_{#2}}{#2}[#3]}%
\NewDocumentCommand\LambdaExpL{m}{\LambdaExp{#1}[l_{#1}]}%
\NewDocumentCommand\EpAj{o}{\EigenPoly{\vb{A}}{j}[#1]}%
\NewDocumentCommand\RepAj{o}{\RestrictedEigenPoly{\vb{A}}{j}[#1]}%
\section{线性变换的特征值与特征向量,线性变换可相似对角化的条件}
\subsection{线性变换的特征值与特征向量}
\cref{theorem:线性映射的矩阵表示.线性变换在不同基下的矩阵相似} 表明,
域\(F\)上\(n\)维线性空间\(V\)上的线性变换\(\vb{A}\)在\(V\)的不同基下的矩阵是相似的.
由于相似的矩阵有相同的行列式、秩、迹、特征多项式、特征值,
因此我们可以把线性变换\(\vb{A}\)在\(V\)的某一个基下的矩阵\(A\)的行列式、秩、迹、特征多项式、特征值,
分别叫做线性变换\(\vb{A}\)的行列式、秩、迹、特征多项式、特征值.

为了更好地理解线性变换的特征值的几何意义,以及对无限维线性空间上的线性变换也考虑它的特征值,
我们给出如下的定义:
\begin{definition}\label{definition:线性变换的特征值和特征向量.线性变换的特征值和特征向量}
%@see: 《高等代数(第三版 下册)》(丘维声) P127 定义1
%@see: 《Linear Algebra Done Right (Fourth Edition)》(Sheldon Axler) P134 5.5
%@see: 《Linear Algebra Done Right (Fourth Edition)》(Sheldon Axler) P135 5.8
设\(\vb{A}\)是域\(F\)上线性空间\(V\)上的一个线性变换.
如果\(V\)中存在一个非零向量\(\xi\),
使得\begin{equation*}
%@see: 《高等代数(第三版 下册)》(丘维声) P127 (1)
	\vb{A}\xi=\lambda_0\xi,
	\quad \lambda_0\in F,
\end{equation*}
则称“\(\lambda_0\)是\(\vb{A}\)的一个\DefineConcept{特征值}%
(\(\lambda_0\) is an \emph{eigenvalue} of \(\vb{A}\))”
“\(\xi\)是\(\vb{A}\)的属于特征值\(\lambda_0\)的一个\DefineConcept{特征向量}%
(\(\xi\) is an \emph{eigenvector} of \(\vb{A}\) corresponding to \(\lambda_0\))”.
\end{definition}
从\cref{definition:线性变换的特征值和特征向量.线性变换的特征值和特征向量} 看出,
线性变换\(\vb{A}\)的特征向量\(\xi\)有这样的“几何意义”:
\(\vb{A}\)对\(\xi\)的作用是把\(\xi\)“拉伸”或“压缩”\(\lambda_0\)倍.
这个倍数\(\lambda_0\)就是\(\vb{A}\)的一个特征值.

现在设\(V\)是域\(F\)上\(n\)维线性空间,
\(V\)中取定一个基\(\AutoTuple{\alpha}{n}\).
\(V\)上的一个线性变换\(\vb{A}\)在基\(\AutoTuple{\alpha}{n}\)下的矩阵是\(A\),
向量\(\xi\)在基\(\AutoTuple{\alpha}{n}\)下的坐标是\(X\),
\(\lambda_0\in F\).
于是\begin{equation}\label{equation:线性变换的特征值和特征向量.与矩阵的特征值和特征向量的联系}
%@see: 《高等代数(第三版 下册)》(丘维声) P127 (2)
	\vb{A}\xi=\lambda_0\xi
	\iff
	AX=\lambda_0X.
\end{equation}
由此得出\begin{align*}
%@see: 《高等代数(第三版 下册)》(丘维声) P127 (3)
	&\text{$\lambda_0$是$\vb{A}$的一个特征值} \\
	&\iff \text{$\lambda_0$是$A$的一个特征值} \\
%@see: 《高等代数(第三版 下册)》(丘维声) P127 (4)
	&\text{$\xi$是$\vb{A}$的属于特征值$\lambda_0$的一个特征向量} \\
	&\iff \text{$\xi$的坐标$X$是$A$的属于特征值$\lambda_0$的一个特征向量}.
\end{align*}
可以看出,对于有限维线性空间,
用线性变换的矩阵的特征值定义线性变换的特征值,与上述定义是一致的.
同时,我们还得到了求有限维线性空间上线性变换\(\vb{A}\)的全部特征值和特征向量的方法:
只要取求\(\vb{A}\)在\(V\)的一个基下的矩阵\(A\)的全部特征值和特征向量.
但是要注意:
矩阵\(A\)的特征向量\(X\)是线性变换\(\vb{A}\)的特征向量\(\xi\)在基\(\AutoTuple{\alpha}{n}\)下的坐标.

\begin{proposition}
%@see: 《Linear Algebra Done Right (Fourth Edition)》(Sheldon Axler) P135 5.7
设\(V\)是域\(F\)上的有限维线性空间,
\(\vb{A}\)是\(V\)上的一个线性变换,
\(\vb{I}\)是\(V\)上的恒等变换,
则下列命题互相等价:\begin{itemize}
	\item \(\lambda\)是\(\vb{A}\)的一个特征值;
	\item \(\lambda\vb{I}-\vb{A}\)不是单射;
	\item \(\lambda\vb{I}-\vb{A}\)不是满射;
	\item \(\lambda\vb{I}-\vb{A}\)不可逆.
\end{itemize}
\begin{proof}
根据定义,
\(\lambda\)是\(\vb{A}\)的一个特征值,
当且仅当关于\(\xi \in V\)的方程\(\vb{A}\xi=\lambda\xi\)或\((\lambda\vb{I}-\vb{A})\xi=\vb0\)有非零解.
根据\cref{theorem:线性映射.线性映射的单射性决定齐次线性方程组是否具有非零解},
\((\lambda\vb{I}-\vb{A})\xi=\vb0\)有非零解,
当且仅当\(\lambda\vb{I}-\vb{A}\)不是单射.
因为\hyperref[theorem:线性映射.有限维线性空间.单线性映射是满线性映射]{在有限维线性空间中单线性映射就是满线性映射},
所以\((\lambda\vb{I}-\vb{A})\xi=\vb0\)有非零解,
当且仅当\(\lambda\vb{I}-\vb{A}\)不是满射.
\end{proof}
\end{proposition}

\begin{example}
%@see: 《高等代数(第三版 下册)》(丘维声) P130 习题9.4 4.
%@see: 《Linear Algebra Done Right (Fourth Edition)》(Sheldon Axler) P136 5.11
设\(V\)是域\(F\)上任意一个线性空间,
\(\vb{A}\)是\(V\)上的一个线性变换.
证明:\(\vb{A}\)的属于不同特征值的特征向量是线性无关的.
\begin{proof}
设\(\lambda_1,\lambda_2\)是\(\vb{A}\)的两个不同特征值,
\(\xi_1\)是\(\vb{A}\)的属于\(\lambda_1\)的一个特征向量,
\(\xi_2\)是\(\vb{A}\)的属于\(\lambda_2\)的一个特征向量.
令\begin{equation*}
	k_1 \xi_1 + k_2 \xi_2 = 0,
	\quad k_1,k_2 \in F.
\end{equation*}
%\(\vb{A} \xi_1 = \lambda_1 \xi_1\)
%\(\vb{A} \xi_2 = \lambda_2 \xi_2\)
那么\begin{gather*}
	\vb{A}(k_1 \xi_1 + k_2 \xi_2)
	= k_1 \lambda_1 \xi_1 + k_2 \lambda_2 \xi_2
	= 0, \tag1 \\
	\lambda_1(k_1 \xi_1 + k_2 \xi_2)
	= k_1 \lambda_1 \xi_1 + k_2 \lambda_1 \xi_2
	= 0, \tag2
\end{gather*}
将(1)(2)两式相减,得\begin{equation*}
	k_2 (\lambda_2 - \lambda_1) \xi_2 = 0.
\end{equation*}
由于\(\xi_2\)是非零向量,
所以\(k_2 (\lambda_2 - \lambda_1) = 0\).
由于\(\lambda_1 \neq \lambda_2\),
所以\(k_2 = 0\).
同理可证\(k_1 = 0\).
因此向量组\(\xi_1,\xi_2\)线性无关.
\end{proof}
%\cref{theorem:矩阵相似对角化.不同特征值的特征向量线性无关}
\end{example}

\begin{example}
%@see: 《高等代数(第三版 下册)》(丘维声) P130 习题9.4 5.
设\(V\)是域\(F\)上任意一个线性空间,
\(\vb{A}\)是\(V\)上的一个线性变换,
\(\lambda_1,\lambda_2\)是\(\vb{A}\)的两个不同特征值,
\(\xi_1,\xi_2\)分别是\(\vb{A}\)的属于\(\lambda_1,\lambda_2\)的特征向量.
证明:\(\xi_1+\xi_2\)不是\(\vb{A}\)的特征向量.
\begin{proof}
由题意有\(\vb{A} \xi_1 = \lambda_1 \xi_1,
\vb{A} \xi_2 = \lambda_2 \xi_2\).
假设\(\xi_1+\xi_2\)是\(\vb{A}\)的属于特征值\(\lambda\)的特征向量,
即\begin{equation*}
	\vb{A} (\xi_1 + \xi_2) = \lambda (\xi_1 + \xi_2),
\end{equation*}
则\begin{gather*}
	\vb{A} \xi_1 + \vb{A} \xi_2
	= \lambda_1 \xi_1 + \lambda_2 \xi_2
	= \lambda \xi_1 + \lambda \xi_2, \\
	(\lambda - \lambda_1) \xi_1 + (\lambda - \lambda_2) \xi_2
	= 0.
\end{gather*}
由于\(\vb{A}\)的属于不同特征值的特征向量是线性无关的,
所以\(\lambda = \lambda_1 = \lambda_2\),矛盾!
这说明\(\xi_1+\xi_2\)不是\(\vb{A}\)的特征向量.
\end{proof}
%@see: 《线性代数》(张慎语、周厚隆) P95 例5
\end{example}

\begin{example}
%@see: 《高等代数(第三版 下册)》(丘维声) P130 习题9.4 6.
设\(V\)是域\(F\)上任意一个线性空间,
\(\vb{A}\)是\(V\)上的一个线性变换.
证明:如果\(V\)中每一个非零向量都是\(\vb{A}\)的特征向量,则\(\vb{A}\)是数乘变换.
\begin{proof}
设\(V\)中每一个非零向量都是\(\vb{A}\)的特征向量.
假设\(\lambda_1,\lambda_2\)是\(\vb{A}\)的两个不同特征值,
\(\xi_1\)是\(\vb{A}\)的属于\(\lambda_1\)的一个特征向量,
\(\xi_2\)是\(\vb{A}\)的属于\(\lambda_2\)的一个特征向量,
那么\(\xi_1+\xi_2\)不是\(\vb{A}\)的特征向量,矛盾!
这说明\(\vb{A}\)有且仅有1个特征值,
\(\vb{A}\)是数乘变换.
\end{proof}
\end{example}

\begin{example}
%@see: 《高等代数(第三版 下册)》(丘维声) P130 习题9.4 7.(1)
设\(V\)是域\(F\)上任意一个线性空间,
\(\vb{A}\)是\(V\)上的一个可逆线性变换.
证明:\(\vb{A}\)的特征值一定不等于\(0\).
\begin{proof}
\begin{proof}[证法一]
用反证法.
假设\(0\)是可逆线性变换\(\vb{A}\)的一个特征值,
那么存在一个非零向量\(\xi\)使得
\(\vb{A}\xi = 0\xi = 0\),
从而有\(\vb{A}(2\xi) = 0(2\xi) = 0\),
显然\(\xi \neq 2\xi\),
这就说明\(\vb{A}\)不是单射.
但是,由\(\vb{A}\)是可逆线性变换可知\(\vb{A}\)是双射,矛盾!
因此\(\vb{A}\)的特征值一定不等于\(0\).
\end{proof}
\begin{proof}[证法二]
设\(A\)是\(\vb{A}\)在\(V\)的某一个基下的矩阵.
既然\(\vb{A}\)是可逆线性变换,那么\(A\)是可逆矩阵.
由\cref{example:矩阵的特征值与特征向量.零不是非奇异矩阵的特征值} 可知,
可逆矩阵\(A\)的特征值一定不等于\(0\),
那么\(\vb{A}\)的特征值也一定不等于\(0\).
\end{proof}\let\qed\relax
\end{proof}
\end{example}

\begin{example}
%@see: 《高等代数(第三版 下册)》(丘维声) P130 习题9.4 7.(2)
设\(V\)是域\(F\)上任意一个线性空间,
\(\vb{A}\)是\(V\)上的一个可逆线性变换.
证明:如果\(\lambda\)是\(\vb{A}\)的特征值,则\(\lambda^{-1}\)是\(\vb{A}^{-1}\)的特征值.
\begin{proof}
\begin{proof}[证法一]
设\(\lambda\neq0\)是\(\vb{A}\)的一个特征值,
\(\xi\)是\(\vb{A}\)的属于\(\lambda\)的一个特征向量,
即\(\vb{A}\xi=\lambda\xi\),
那么\begin{equation*}
	\vb{A}^{-1}(\vb{A}\xi)
	= (\vb{A}^{-1}\vb{A})\xi
	= \vb{I}\xi
	= \xi,
	\qquad
	\vb{A}^{-1}(\lambda\xi)
	= \lambda(\vb{A}^{-1}\xi),
\end{equation*}
于是\(\lambda(\vb{A}^{-1}\xi)=\xi\),
即\(\vb{A}^{-1}\xi=\lambda^{-1}\xi\).
\end{proof}
\begin{proof}[证法二]
设\(A\)是\(\vb{A}\)在\(V\)的某一个基下的矩阵.
既然\(\vb{A}\)是可逆线性变换,那么\(A\)是可逆矩阵.
由\cref{example:矩阵的特征值与特征向量.矩阵的多项式的特征值3} 可知,
如果\(\lambda\)是\(A\)的一个特征值,
那么\(\lambda^{-1}\)就是\(A\)的逆矩阵\(A^{-1}\)的一个特征值,
于是\(\lambda^{-1}\)是\(\vb{A}^{-1}\)的特征值.
\end{proof}\let\qed\relax
\end{proof}
\end{example}

\begin{proposition}
%@see: 《Linear Algebra Done Right (Fourth Edition)》(Sheldon Axler) P136 5.11
设\(V\)是域\(F\)上任意一个线性空间,
\(\vb{A}\)是\(V\)上的一个可逆线性变换,
则\begin{equation*}
	\card\Set{
		\lambda
		\given
		\text{$\lambda$是$\vb{A}$的特征值}
	}
	\leq \dim V.
\end{equation*}
%TODO proof
\end{proposition}

\subsection{线性变换的特征子空间}
设\(\vb{A}\)是域\(F\)上线性空间\(V\)上的一个线性变换,
\(\lambda_0\)是\(\vb{A}\)的一个特征值.
令\begin{equation}\label{equation:线性变换的特征值和特征向量.特征子空间}
%@see: 《高等代数(第三版 下册)》(丘维声) P128 (5)
	V_{\lambda_0}
	\defeq
	\Set{ \alpha \in V \given \vb{A}\alpha=\lambda_0\alpha }.
\end{equation}
可以验证\(V_{\lambda_0}\)是\(V\)的一个子空间,
因此称“\(V_{\lambda_0}\)是\(\vb{A}\)的属于特征值\(\lambda_0\)的\DefineConcept{特征子空间}”.

\(V_{\lambda_0}-\{0\}\)中的全部向量就是\(\vb{A}\)的属于\(\lambda_0\)的全部特征向量.

易知\begin{equation*}
	V_{\lambda_0}
	= \Ker(\lambda_0 \vb{I} - \vb{A}).
\end{equation*}

\subsection{线性变换的根子空间}
%@see: 《高等代数(第三版 下册)》(丘维声) P139 习题9.6 1.
设\(\vb{A}\)是域\(F\)上线性空间\(V\)上的一个线性变换,
\(\lambda_0\)是\(\vb{A}\)的一个特征值.
定义:\begin{equation}
	R_{\lambda_0} \defeq \Set{
		\alpha \in V
		\given
		(\exists r>0)
		[
			(\lambda_0 \vb{I} - \vb{A})^r \alpha = 0
		]
	}.
\end{equation}
可以验证\(R_{\lambda_0}\)是一个线性空间,
因此称“\(R_{\lambda_0}\)是\(\vb{A}\)的属于特征值\(\lambda_0\)的\DefineConcept{根子空间}”.

\subsection{线性变换的几何重数、代数重数}
设\(V\)是域\(F\)上\(n\)维线性空间,
\(V\)上线性变换\(\vb{A}\)在\(V\)的一个基\(\AutoTuple{\alpha}{n}\)下的矩阵为\(A\),
\(\lambda_0\)是\(\vb{A}\)的一个特征值,
由\cref{equation:线性变换的特征值和特征向量.与矩阵的特征值和特征向量的联系,%
equation:线性变换的特征值和特征向量.特征子空间} 可得,
线性变换\(\vb{A}\)的属于\(\lambda_0\)的特征子空间\(V_{\lambda_0}\)的
所有向量的坐标组成的集合是矩阵\(A\)的属于\(\lambda_0\)的特征子空间,
后者就是齐次线性方程组\((\lambda_0 E - A) X = 0\)的解空间,
因此\begin{equation}
	\dim V_{\lambda_0}
	= n - \rank(\lambda_0 E - A).
\end{equation}
\(V_{\lambda_0}\)的维数称为
“\(\vb{A}\)的特征值\(\lambda_0\)的\DefineConcept{几何重数}”,
\(\lambda_0\)作为\(\vb{A}\)的特征多项式的根的重数称为
“\(\lambda_0\)的\DefineConcept{代数重数}”.

\subsection{线性变换可相似对角化的条件}
无论是在理论上,还是在实际应用上,
我们都希望能在\(V\)中找到一个适当的基,
使得预先给定的线性变换\(\vb{A}\)在这个基下的矩阵具有最简单的形式.
由于\(\vb{A}\)在\(V\)的不同基下的矩阵是相似的,
因此这个问题也就转化为,
求\(\vb{A}\)在\(V\)的一个基下的矩阵\(A\)的相似标准型.
之前我们曾经讨论过\(n\)阶矩阵的相似标准型问题,
给出了\(A\)的相似标准型为对角矩阵的充分必要条件.
但是,对于不可以对角化的矩阵\(A\),它的相似标准型是什么?
我们尚未讨论.
接下来,我们就开始研究如何寻找\(V\)的一个恰当的基,
使得线性变换\(\vb{A}\)在这个基下的矩阵具有最简形式.

如果\(V\)中存在一个基,使得线性变换\(\vb{A}\)在这个基下的矩阵是对角矩阵,
则称线性变换\(\vb{A}\)可对角化.

设线性变换\(\vb{A}\)在\(V\)的一个基\(\AutoTuple{\alpha}{n}\)下的矩阵为\(A\),
则\(\vb{A}\)可对角化,当且仅当\(A\)可对角化.
于是从\(n\)阶矩阵\(A\)可对角化的充分必要条件,
以及\(V\)与\(F^n\)同构的性质,
可以得出线性变换\(\vb{A}\)可对角化的充分必要条件:
\begin{theorem}
%@see: 《高等代数(第三版 下册)》(丘维声) P129 定理1
%@see: 《高等代数(第三版 下册)》(丘维声) P131 习题9.4 8.
设\(\vb{A}\)是域\(F\)上\(n\)维线性空间\(V\)上的一个线性变换,
则\begin{align*}
	&\text{$\vb{A}$可对角化} \\
	&\iff \text{$\vb{A}$有$n$个线性无关的特征向量} \\
	&\iff \text{$V$中存在由$\vb{A}$的特征向量组成的一个基} \\
	&\iff \text{$\vb{A}$的属于不同特征值的特征子空间的维数之和等于$n$} \\
	&\iff V = \BigDirectSum_{i=1}^s V_{\lambda_i},
\end{align*}
其中\(\AutoTuple{\lambda}{s}\)是\(\vb{A}\)的所有不同特征值.
\end{theorem}

如果\(\vb{A}\)可对角化,
则\(\vb{A}\)有\(n\)个线性无关的特征向量\(\AutoTuple{\xi}{n}\),
从而有\begin{equation*}
%@see: 《高等代数(第三版 下册)》(丘维声) P129 (7)
	\vb{A} (\AutoTuple{\xi}{n})
	= (\AutoTuple{\xi}{n})
	\diag(\AutoTuple{\lambda}{n}),
\end{equation*}
其中\(\vb{A} \xi_i = \lambda_i \xi_i\ (i=1,2,\dotsc,n)\).
从上式可以看出,等号右端的对角矩阵的主对角元,恰好是\(\vb{A}\)的全部特征值(重根按重数计算).
因此这个对角矩阵除了主对角线上元素的排列次序外,是由线性变换\(\vb{A}\)唯一决定的.
我们把这个对角矩阵称为线性变换\(\vb{A}\)的\DefineConcept{标准型}.

\begin{example}
%@see: 《高等代数(第三版 下册)》(丘维声) P131 习题9.4 9.
设\(\vb{A}\)是域\(F\)上\(n\)维线性空间\(V\)上的一个线性变换.
证明:\(\vb{A}\)可对角化,
当且仅当\(\vb{A}\)的特征多项式\(f(\lambda)\)在\(F[\lambda]\)中的标准分解式为\begin{equation*}
	f(\lambda)
	= (\lambda-\lambda_1)^{r_1} \dotsm (\lambda-\lambda_s)^{r_s},
\end{equation*}
并且\(\vb{A}\)的每一个特征值\(\lambda_i\)的几何重数等于它的代数重数.
%TODO proof
%\cref{theorem:矩阵可以相似对角化的充分必要条件.定理2}
\end{example}

\section{线性变换的不变子空间}
在上一节,我们讨论了可对角化的线性变换的标准型.
本节讨论不可以对角化的线性变换的标准型.

我们首先注意到,
\(\vb{A}\)是可对角化的线性变换,
当且仅当空间\(V\)可以分解成\(\vb{A}\)的特征子空间的直和.
受此启发,我们在研究不可对角化的线性变换\(\vb{B}\)的结构时,
也可以从这里入手,研究如何将空间\(V\)分解成与\(\vb{B}\)有关的某种特殊类型的子空间的直和.

\subsection{线性变换的不变子空间}
\begin{definition}
%@see: 《高等代数(第三版 下册)》(丘维声) P131 定义1
%@see: 《Linear Algebra Done Right (Fourth Edition)》(Sheldon Axler) P133 5.2
设\(\vb{A}\)是域\(F\)上线性空间\(V\)上的线性变换,\(W\)是\(V\)的子空间.
如果\(W\)中的向量在\(\vb{A}\)下的像仍在\(W\)中,
即\((\forall\alpha \in W)[\vb{A}\alpha \in W]\),
或\(\vb{A}W \subseteq W\),
则称“\(W\)是\(\vb{A}\)的一个\DefineConcept{不变子空间}%
(a \emph{invariant} under \(\vb{A}\))”,
简称 \DefineConcept{\(\vb{A}\) - 子空间}.
\end{definition}

显然,对于\(V\)上每一个线性变换\(\vb{A}\)来说,
整个空间\(V\)和零子空间\(0\),
都是\(\vb{A}\)的不变子空间,
因此将它们称为“\(\vb{A}\)的\DefineConcept{平凡不变子空间}”.
如果\(W\)是\(\vb{A}\)的不变子空间,
且\(W\)不是\(\vb{A}\)的平凡不变子空间,
则称“\(W\)是\(\vb{A}\)的一个\DefineConcept{非平凡不变子空间}”.

\begin{definition}
设\(\vb{A}\)是域\(F\)上线性空间\(V\)上的线性变换.
\begin{itemize}
	\item 如果\begin{equation*}
		(\exists W \AlgebraSubstructure V)
		[\text{$W$是$\vb{A}$的非平凡不变子空间}],
	\end{equation*}
	则称“\(\vb{A}\)有非平凡不变子空间”.

	\item 如果\begin{equation*}
		(\forall W \AlgebraSubstructure V)
		[\text{$W$不是$\vb{A}$的非平凡不变子空间}],
	\end{equation*}
	则称“\(\vb{A}\)没有非平凡不变子空间”.
\end{itemize}
\end{definition}

\begin{example}
设\(\vb{I}\)是线性空间\(V\)上的恒等变换,
则\(V\)的每一个子空间\(W\)都是\(\vb{I}\)的一个不变子空间,
并且成立\(\vb{I}W = W\).
\end{example}
\begin{example}
设\(\vb0\)是线性空间\(V\)上的零映射,
则\(V\)的每一个非零子空间\(W\)都是\(\vb0\)的一个不变子空间,
% 本来说\(V\)的每一个子空间(不限定为“非零子空间”)都是零映射的一个不变子空间,
% 但是那样说的话就不能得到“\(\vb0W\)真包含于\(W\)”这个结论了.
并且成立\(\vb0W = 0 \subset W\).
\end{example}

\begin{proposition}%\label{theorem:线性映射.线性变换的不变子空间1}
%@see: 《高等代数(第三版 下册)》(丘维声) P131 命题1
\(V\)上线性变换\(\vb{A}\)的核\(\Ker\vb{A}\)与像\(\Img\vb{A}\),
以及\(\vb{A}\)的所有特征子空间,
都是\(\vb{A}\)的不变子空间.
\begin{proof}
任取\(\alpha \in \Ker\vb{A}\),
因为\(\vb{A}\alpha = 0 \in \Ker\vb{A}\),
所以\(\Ker\vb{A}\)是\(\vb{A}\)的不变子空间.

任取\(\alpha \in \Img\vb{A}\),
因为\(\vb{A}\alpha \in \Img\vb{A}\),
所以\(\Img\vb{A}\)也是\(\vb{A}\)的不变子空间.

任取\(\alpha \in V_\lambda\),
因为\(\vb{A}\alpha = \lambda \alpha \in V_\lambda\),
所以\(V_\lambda\)还是\(\vb{A}\)的不变子空间.
\end{proof}
\end{proposition}

\begin{example}
设\(W\)是线性变换\(\vb{A}\)的一个不变子空间.
证明:\(\vb{A}W\)是\(\vb{A}\)的一个不变子空间.
\begin{proof}
因为\(W\)是线性变换\(\vb{A}\)的一个不变子空间,
所以对于任意\(\alpha \in W\),
有\(\vb{A}\alpha \in W\);
同理,既然\(\vb{A}\alpha \in W\),
那么必有\(\vb{A}(\vb{A}\alpha) \in W\),
也就是说\begin{equation*}
	(\forall \alpha \in W)
	[
		\vb{A}(\vb{A}\alpha) \in W
	].
	\eqno(1)
\end{equation*}
现在要证\(
	\vb{A}W
	= \Set{
		\beta
		\given
		\beta = \vb{A}\alpha,
		\alpha \in W
	}
\)是\(\vb{A}\)的一个不变子空间,
只需证对于任意\(\beta \in \vb{A}W\),
成立\(\vb{A}\beta \in \vb{A}W\).
因为对于任意\(\beta \in \vb{A}W\),
存在\(\alpha \in W\),
使得\(\beta = \vb{A}\alpha\),
所以由命题(1)可知\begin{equation*}
	\vb{A}\beta
	= \vb{A}(\vb{A}\alpha)
	\in W;
\end{equation*}
这就是说\begin{equation*}
	(\forall \beta \in \vb{A}W)
	[
		\vb{A}\beta
		\in W
	].
	\eqno(2)
\end{equation*}
因此\(\vb{A}W\)是\(\vb{A}\)的一个不变子空间.
\end{proof}
\end{example}

\begin{example}
设\(V\)是域\(F\)上的一个线性空间,
\(\vb{A}\)是\(V\)上的线性变换,
取\(\alpha \in V - \{0\}\).
证明:\(\Span\{\alpha\}\)是\(\vb{A}\)的一个不变子空间的充分必要条件是
\(\alpha\)是\(\vb{A}\)的一个特征向量.
\begin{proof}
充分性.
设\(\alpha\)是\(\vb{A}\)的属于特征值\(\lambda\)的一个特征向量,
即\(\vb{A} \alpha = \lambda \alpha\),
所以\(\vb{A}\alpha \in \Span\{\alpha\}\),
这就说明\(\Span\{\alpha\}\)是线性变换\(\vb{A}\)的一个不变子空间.

必要性.
设\(\Span\{\alpha\}\)是线性变换\(\vb{A}\)的一个不变子空间,
那么对于任意\(k \in F - \{0\}\),
有\begin{equation*}
	\vb{A}(k\alpha)
	= k(\vb{A}\alpha)
	\in \Span\{\alpha\},
\end{equation*}
即存在\(l \in F\)使得\begin{equation*}
	\vb{A}(k\alpha)
	= k(\vb{A}\alpha)
	= l \alpha,
	\quad\text{即}\quad
	\vb{A}\alpha
	= k^{-1} l \alpha,
\end{equation*}
若记\(\lambda \defeq k^{-1} l\),
便有\(\vb{A} \alpha = \lambda \alpha\),
这就说明\(\alpha\)是\(\vb{A}\)的属于特征值\(\lambda\)的一个特征向量.
\end{proof}
\end{example}

\begin{proposition}%\label{theorem:线性映射.线性变换的不变子空间2}
%@see: 《高等代数(第三版 下册)》(丘维声) P131 命题2
如果线性变换\(\vb{A},\vb{B}\)可交换,即\(\vb{A}\vb{B} = \vb{B}\vb{A}\),
则\(\Ker\vb{B},
\Img\vb{B}\)
以及\(\vb{B}\)的特征子空间
都是\(\vb{A}\)的不变子空间.
\begin{proof}
任取\(\alpha \in \Ker\vb{B}\),
则\(\vb{B}\alpha = 0\).
于是\begin{equation*}
	\vb{B}(\vb{A}\alpha)
	= (\vb{B}\vb{A})\alpha
	= (\vb{A}\vb{B})\alpha
	= \vb{A}(\vb{B}\alpha)
	= \vb{A}0
	= 0.
\end{equation*}
因此\(\vb{A}\alpha \in \Ker\vb{B}\),
从而\(\Ker\vb{B}\)是\(\vb{A}\)的不变子空间.

同理可证\(\Img\vb{B}\)也是\(\vb{A}\)的不变子空间.

在\(\vb{B}\)的特征子空间\(V_\lambda\)中,任取一个向量\(\alpha\),
则\(\vb{B}\alpha = \lambda \alpha\),
从而\begin{equation*}
	\vb{B}(\vb{A}\alpha)
	= \vb{A}(\vb{B}\alpha)
	= \vb{A}(\lambda\alpha)
	= \lambda(\vb{A}\alpha),
\end{equation*}
因此\(\vb{A}\alpha \in V_\lambda\),
从而\(V_\lambda\)是\(A\)的不变子空间.
\end{proof}
\end{proposition}

\begin{corollary}\label{theorem:线性映射.线性变换的不变子空间3}
%@see: 《高等代数(第三版 下册)》(丘维声) P132 推论3
设\(\vb{A}\)是域\(F\)上线性空间\(V\)上的线性变换,
则对于域\(F\)上任意一个一元多项式\(f(x) \in F[x]\),
都有\(\Ker f(\vb{A}),
\Img f(\vb{A})\)
以及\(f(\vb{A})\)的特征子空间
都是\(\vb{A}\)的不变子空间.
\end{corollary}

\begin{proposition}%\label{theorem:线性映射.线性变换的不变子空间4}
%@see: 《高等代数(第三版 下册)》(丘维声) P132
设\(U_1,U_2\)是域\(F\)上线性空间\(V\)上的线性变换\(\vb{A}\)的两个不变子空间,
则\(U_1 \cap U_2\)和\(U_1 + U_2\)都是\(\vb{A}\)的不变子空间.
\end{proposition}

\begin{proposition}\label{theorem:线性映射.线性变换的不变子空间的基}
%@see: 《高等代数(第三版 下册)》(丘维声) P132 命题4
设\(\vb{A}\)是域\(F\)上线性空间\(V\)上的线性变换,
\(W = \Span\{\AutoTuple{\alpha}{m}\}\)是\(V\)的一个子空间,
则\(W\)是\(\vb{A}\)的不变子空间,
当且仅当\(\vb{A}\alpha_i \in W\ (i=1,2,\dotsc,m)\).
\begin{proof}
由于\(W\)中任意一个向量\(\alpha\)均可表示成\(\AutoTuple{\alpha}{m}\)的一个线性组合,
不妨设\(\alpha = k_1 \alpha_1 + \dotsb + k_m \alpha_m\),
其中\(\AutoTuple{k}{m} \in F\),
那么\begin{align*}
	&\text{$W$是$\vb{A}$的不变子空间} \\
	&\iff \alpha \in W \implies \vb{A}\alpha \in W \\
	&\iff (\forall \AutoTuple{k}{m} \in F)[\vb{A}(k_1 \alpha_1 + \dotsb + k_m \alpha_m) \in W] \\
	&\iff \vb{A}\alpha_i \in W\ (i=1,2,\dotsc,m).
	\qedhere
\end{align*}
\end{proof}
\end{proposition}

\begin{example}
%@see: 《高等代数(第三版 下册)》(丘维声) P136 习题9.5 3.
设\(V\)是复数域上的\(n\)维线性空间,
\(\vb{A},\vb{B}\)都是\(V\)上的线性变换,
且\(\vb{A}\vb{B}=\vb{B}\vb{A}\).
证明:\(\vb{A}\)与\(\vb{B}\)至少有一个公共的特征向量.
%TODO proof
% 取\(\vb{A}\)的一个特征值\(\lambda\),则\(V_\lambda\)是\(\vb{B}\)的一个不变子空间
% 取\(\vb{B} \SetRestrict V_\lambda\)的一个特征值\(\mu\),
% 则存在\(\xi \in V_\lambda\),
% 使得\(\vb{B} \xi = (\vb{B} \SetRestrict V_\lambda) \xi = \mu \xi\)
\end{example}

\begin{example}
%@see: 《高等代数(第三版 下册)》(丘维声) P136 习题9.5 5.
设\(V\)是实数域上\(n\)维线性空间.
证明:\(V\)上的任意一个线性变换\(\vb{A}\)
必有一个1维不变子空间
或者2维不变子空间.
%TODO proof
\end{example}

\begin{example}
%@see: 《高等代数(第三版 下册)》(丘维声) P136 习题9.5 6.
设\(V\)是实数域上的2维线性空间,
\(\vb{A}\)是\(V\)上的一个线性变换,
\(A\)是\(\vb{A}\)在基\(\AutoTuple{\epsilon}{2}\)下的矩阵,
其中\begin{equation*}
	A = \begin{bmatrix}
		\cos\theta & -\sin\theta \\
		\sin\theta & \cos\theta
	\end{bmatrix},
	\quad
	\theta \neq k\pi,
	\quad
	k \in \mathbb{Z}.
\end{equation*}
证明:\(\vb{A}\)没有非平凡不变子空间.
%TODO proof
\end{example}

\begin{example}
%@see: 《高等代数(第三版 下册)》(丘维声) P136 习题9.5 7.
设\(\vb{A}\)是复数域\(\mathbb{C}\)上\(n\)维线性空间\(V\)上的线性变换,
且\(\vb{A}\)有\(n\)个不同的特征值\(\AutoTuple{\lambda}{n}\).
求\(\vb{A}\)的所有不变子空间,
并且求出\(\vb{A}\)的不变子空间的个数.
%TODO
\end{example}

\subsection{线性变换的非平凡不变子空间的存在性}
线性变换不一定有非平凡不变子空间.
例如,取有理数域\(\mathbb{Q}\)上2维线性空间\(V\)上的一个线性变换\(\vb{A}\),
使得它的矩阵为\(A = \begin{bmatrix}
	2 & 1 \\
	3 & 2
\end{bmatrix}\);
考虑到即便\(\vb{A}\)有非平凡不变子空间,
那么\(\vb{A}\)的非平凡不变子空间也一定是1维的,
那么不妨设\(W = \Span\{\alpha_0\}\)是\(\vb{A}\)的一个非平凡不变子空间,
其中向量\(\alpha_0=(x_1,x_2)\neq0\);
%@Mathematica: A = {{2, 1}, {3, 2}}
%@Mathematica: a = {x1, x2}
%@Mathematica: a.A
于是\(\alpha_0\)在\(\vb{A}\)下的像为\begin{equation*}
	\beta
	\defeq \vb{A} \alpha_0
	= \alpha_0 A
	= (2x_1+3x_2,x_1+2x_2);
\end{equation*}
既然\(W\)是不变子空间,
% 不变子空间的定义
自然有\(\beta \in W\),
% \(\beta\)可以由\(\alpha_0\)线性表出
那么必定存在一个有理数\(k\)使得\(\beta = k \alpha_0\),
即\begin{equation*}
	(2x_1+3x_2,x_1+2x_2)=(k x_1,k x_2),
\end{equation*}
将上式看作一个关于\(k\)的线性方程组,
要保证它有解,
必须要有\begin{equation*}
	\rank\begin{bmatrix}
		x_1 \\ x_2
	\end{bmatrix}
	= \rank\begin{bmatrix}
		x_1 & 2x_1+3x_2 \\
		x_2 & x_1+2x_2
	\end{bmatrix}
	% 向量\((x_1,x_2)\)不等于0,因此这里秩一定等于1
	= 1;
\end{equation*}
由于\begin{equation*}
	\rank\begin{bmatrix}
		x_1 & 2x_1+3x_2 \\
		x_2 & x_1+2x_2
	\end{bmatrix}
	< 2,
\end{equation*}
所以\begin{equation*}
	\begin{vmatrix}
		x_1 & 2x_1+3x_2 \\
		x_2 & x_1+2x_2
	\end{vmatrix}
	= x_1^2 - 3x_2^2
	= 0,
\end{equation*}
鉴于\(x_1,x_2 \in \mathbb{Q}\),
上式若要成立,
则必定有\(x_1 = x_2 = 0\),
矛盾!
由此可见\(\beta \notin W\),
根据\cref{theorem:线性映射.线性变换的不变子空间的基},
\(W\)不是\(\vb{A}\)的不变子空间,
因此\(\vb{A}\)没有非平凡不变子空间.

对于域\(F\)上有限维线性空间\(V\)上的线性变换\(\vb{A}\),
它有没有非平凡不变子空间,与它的矩阵表示的形式,有密切关系.
\begin{theorem}
%@see: 《高等代数(第三版 下册)》(丘维声) P132 定理5
设\(\vb{A}\)是域\(F\)上\(n\)维线性空间\(V\)上的线性变换,
则\(\vb{A}\)有非平凡不变子空间,
当且仅当\(V\)中存在一个基,
使得\(\vb{A}\)在这个基下的矩阵是一个分块上三角矩阵.
\begin{proof}
必要性.
设\(W\)是\(\vb{A}\)的非平凡不变子空间,
在\(W\)中取一个基\(\AutoTuple{\alpha}{r}\),
把它扩充成\(V\)的一个基:\begin{equation*}
	\AutoTuple{\alpha}{r},
	\AutoTuple{\alpha}[r+1]{n}.
\end{equation*}
那么\begin{align*}
%@see: 《高等代数(第三版 下册)》(丘维声) P132 (3)
	&\vb{A} (\AutoTuple{\alpha}{r},\AutoTuple{\alpha}[r+1]{n}) \\
	&= (\AutoTuple{\alpha}{r},\AutoTuple{\alpha}[r+1]{n})
	\begin{bmatrix}
		a_{11} & \dots & a_{1r} & a_{1,r+1} & \dots & a_{1n} \\
		\vdots & & \vdots & \vdots & & \vdots \\
		a_{r1} & \dots & a_{rr} & a_{r,r+1} & \dots & a_{rn} \\
		0 & \dots & 0 & a_{r+1,r+1} & \dots & a_{r+1,n} \\
		\vdots & & \vdots & \vdots & & \vdots \\
		0 & \dots & 0 & a_{n,r+1} & \dots & a_{nn}
	\end{bmatrix}.
\end{align*}
因此\(\vb{A}\)在基\(\AutoTuple{\alpha}{r},\AutoTuple{\alpha}[r+1]{n}\)下的矩阵是
一个分块上三角矩阵\(\begin{bmatrix}
	A_1 & A_3 \\
	0 & A_2
\end{bmatrix}\),
其中\(A_1\)是\(A \SetRestrict W\)在\(W\)的一个基\(\AutoTuple{\alpha}{r}\)下的矩阵.

充分性.
设\(\vb{A}\)在\(V\)的一个基\(\AutoTuple{\alpha}{n}\)下的矩阵是
一个分块上三角矩阵\(\begin{bmatrix}
	A_1 & A_3 \\
	0 & A_2
\end{bmatrix}\),
其中\(A_1\)是\(r\)阶矩阵,
且\(0 < r < n\).
%@see: 《高等代数(第三版 下册)》(丘维声) P133 (4)
令\(W = \Span\{\AutoTuple{\alpha}{r}\}\),
则\(\vb{A}\alpha_i \in W\ (i=1,2,\dotsc,r)\).
因此\(W\)是\(\vb{A}\)的不变子空间.
显然\(W\)是非平凡的.
此时\(A_1\)是\(A \SetRestrict W\)在\(W\)的基\(\AutoTuple{\alpha}{r}\)下的矩阵.
\end{proof}
\end{theorem}

\subsection{线性空间的直和分解式}
\begin{theorem}\label{theorem:线性映射.线性空间可以分解为线性变换的一些非平凡不变子空间的直和的充分必要条件}
%@see: 《高等代数(第三版 下册)》(丘维声) P133 定理6
设\(\vb{A}\)是域\(F\)上\(n\)维线性空间\(V\)上的线性变换,
则\(V\)能分解成\(\vb{A}\)的一些非平凡不变子空间的直和,
当且仅当\(V\)中存在一个基,
使得\(\vb{A}\)在这个基下的矩阵是一个分块对角矩阵.
\def\BasisV{\alpha_{11},\dotsc,\alpha_{1 r_1},\dotsc,\alpha_{s1},\dotsc,\alpha_{s r_s}}
\def\BasisWi{\alpha_{i1},\dotsc,\alpha_{ir_i}}
\begin{proof}
必要性.
设\(V\)是\(\vb{A}\)的一些非平凡不变子空间的直和:\begin{equation*}
%@see: 《高等代数(第三版 下册)》(丘维声) P133 (6)
	V = W_1 \DirectSum \dotsb \DirectSum W_s.
\end{equation*}
在每个\(W_i\ (i=1,2,\dotsc,s)\)中取一个基\(\BasisWi\),
由上式可知\begin{equation*}
%@see: 《高等代数(第三版 下册)》(丘维声) P133 (7)
	\BasisV
\end{equation*}是\(V\)的一个基.
由于\(W_i\)是\(\vb{A}\)的不变子空间,
因此\begin{equation*}
%@see: 《高等代数(第三版 下册)》(丘维声) P133 (8)
	\vb{A} (\BasisWi)
	= (\BasisWi) A_i,
	\quad i=1,2,\dotsc,s.
\end{equation*}
从而\(\vb{A}\)在基\(\BasisV\)下的矩阵为\begin{equation*}
%@see: 《高等代数(第三版 下册)》(丘维声) P133 (9)
	\begin{bmatrix}
		A_1 \\
		& A_2 \\
		& & \ddots \\
		& & & A_s
	\end{bmatrix}.
\end{equation*}

充分性.
设\(\vb{A}\)在\(V\)一个基\begin{equation*}
	\BasisV
\end{equation*}下的矩阵\(\vb{A} = \diag(\AutoTuple{A}{s})\),
其中\(A_i\)是\(r_i\)阶方阵,
而\(i=1,2,\dotsc,s\).
令\begin{equation*}
	W_i = \Span\{\BasisWi\}
	\quad(i=1,2,\dotsc,s).
\end{equation*}
由于\begin{equation*}
%@see: 《高等代数(第三版 下册)》(丘维声) P133 (10)
	\vb{A} (\BasisWi) = (\BasisWi) A_i,
	\quad i=1,2,\dotsc,s,
\end{equation*}
所以\(\vb{A}\alpha_{i1},\dotsc,\vb{A}\alpha_{i r_i} \in W_i\),
从而\(W_i\)是\(\vb{A}\)的不变子空间.
显然\(W_i\)是非平凡的.
由于\(W_i\)的一个基\(\BasisWi\ (i=1,2,\dotsc,s)\)合起来\(V\)的一个基,
所以\(V = W_1 \DirectSum \dotsb \DirectSum W_s\).
\end{proof}
\begin{remark}
从\cref{theorem:线性映射.线性空间可以分解为线性变换的一些非平凡不变子空间的直和的充分必要条件} 的证明过程中可以看出,
在线性变换\(\vb{A}\)的矩阵\begin{equation*}
	\begin{bmatrix}
		A_1 \\
		& A_2 \\
		& & \ddots \\
		& & & A_s
	\end{bmatrix}
\end{equation*}中,
\(A_i\)就是\(\vb{A}\)在它的不变子空间\(W_i\)上的限制\(\vb{A} \SetRestrict W_i\)
在\(W_i\)的一个基\(\BasisWi\)下的矩阵,
其中\(i=1,2,\dotsc,s\).
\end{remark}
\end{theorem}

\begin{example}\label{example:线性映射.可逆线性变换在其有限维不变子空间上的限制是可逆线性变换}
%@see: 《高等代数(第三版 下册)》(丘维声) P136 习题9.5 2.(1)
设\(W\)是线性空间\(V\)上可逆线性变换\(\vb{A}\)的有限维不变子空间.
证明:\(\vb{A} \SetRestrict W\)是\(W\)上的可逆线性变换.
%TODO proof
% 先说明\(\vb{A} \SetRestrict W\)是单射
% 再据此结论说明\(\vb{A} \SetRestrict W\)是满射
\end{example}

\begin{example}\label{example:线性映射.可逆线性变换的逆变换的不变子空间}
%@see: 《高等代数(第三版 下册)》(丘维声) P136 习题9.5 2.(2)
设\(W\)是线性空间\(V\)上可逆线性变换\(\vb{A}\)的有限维不变子空间.
证明:\(W\)是\(\vb{A}^{-1}\)的不变子空间,
且\((\vb{A} \SetRestrict W)^{-1}
= \vb{A}^{-1} \SetRestrict W\).
%TODO proof
% 利用【《高等代数(第三版 下册)》(丘维声) P131 习题9.5 2.(1)】的结论
% 根据定义去证\(W\)是\(\vb{A}^{-1}\)的不变子空间
% 任取\(\beta \in W\),
% 令\((\vb{A} \SetRestrict W)^{-1} \beta = \gamma\)
% 则\(\gamma \in W\)
% 最后证明\(\gamma = (\vb{A}^{-1} \SetRestrict W) \beta\)
\end{example}

\subsection{求解非平凡子空间的基本步骤}
虽然我们从\cref{theorem:线性映射.线性空间可以分解为线性变换的一些非平凡不变子空间的直和的充分必要条件} 可以看出,
如果\(n\)维线性空间\(V\)能分解成线性变换\(\vb{A}\)的一些非平凡不变子空间的直和,
那么\(V\)中存在一个基,
使得\(\vb{A}\)在这个基下的矩阵是一个分块对角矩阵,
但是,要如何找出\(\vb{A}\)的所有非平凡不不变子空间呢?
根据\cref{theorem:线性映射.线性变换的不变子空间3},
对于任意一个\(f(x) \in F[x]\)都有\(\Ker f(\vb{A})\)是\(\vb{A}\)的不变子空间.
受此启发,我们希望找到一些多项式\(f_1(x),\dotsc,f_s(x) \in F[x]\),
使得\begin{equation*}
%@see: 《高等代数(第三版 下册)》(丘维声) P134 (11)
	V = \Ker f_1(\vb{A}) \DirectSum \dotsb \DirectSum \Ker f_s(\vb{A}).
\end{equation*}
为此,我们尚需研究:
对于不同的一元多项式\(f_1(x)\)和\(f_2(x)\),
不定元\(x\)用\(\vb{A}\)代入,
得到的\(f_1(\vb{A})\)与\(f_2(\vb{A})\)的核,
\(\Ker f_1(\vb{A})\)与\(\Ker f_2(\vb{A})\)之间,
有什么关系.
\begin{theorem}%\label{theorem:线性映射.线性映射多项式的核空间的直和分解式1}
%@see: 《高等代数(第三版 下册)》(丘维声) P134 定理7
设\(\vb{A}\)是域\(F\)上线性空间\(V\)上的线性变换,
而\(f(x),f_1(x),f_2(x)\)都是域\(F\)上的一元多项式.
如果\begin{equation*}
	f(x) = f_1(x) f_2(x)
	\quad\text{且}\quad
	(f_1(x),f_2(x)) = 1,
\end{equation*}
则\begin{equation*}
%@see: 《高等代数(第三版 下册)》(丘维声) P134 (12)
	\Ker f(\vb{A})
	= \Ker f_1(\vb{A})
	\DirectSum \Ker f_2(\vb{A}).
\end{equation*}
%TODO proof
\end{theorem}

用数学归纳法可以将上述定理推广.
\begin{corollary}\label{theorem:线性映射.线性映射多项式的核空间的直和分解式2}
%@see: 《高等代数(第三版 下册)》(丘维声) P135 推论8
设\(\vb{A}\)是域\(F\)上线性空间\(V\)上的线性变换,
\(f(x)\)以及\(f_1(x),\dotsc,f_s(x)\)都是域\(F\)上的一元多项式.
如果\begin{equation*}
	f(x) = f_1(x) \dotsm f_s(x)
	\quad\text{且}\quad
	\text{$f_1(x),\dotsc,f_s(x)$两两互素},
\end{equation*}
则\begin{equation*}
%@see: 《高等代数(第三版 下册)》(丘维声) P135 (15)
	\Ker f(\vb{A})
	= \Ker f_1(\vb{A})
	\DirectSum
	\dotsb
	\DirectSum \Ker f_s(\vb{A}).
\end{equation*}
\end{corollary}

\subsection{线性变换的零化多项式}
由于\(\Ker\vb0 = V\),
结合\cref{theorem:线性映射.线性映射多项式的核空间的直和分解式2} 给出的暗示,
不难想到,
如果能够找到一个多项式\(f(x)\),
使得\(f(\vb{A}) = \vb0\),
那么空间\(V\)就能分解成\begin{equation*}
%@see: 《高等代数(第三版 下册)》(丘维声) P135 (16)
	V = \Ker f_1(\vb{A})
	\DirectSum
	\dotsb
	\DirectSum \Ker f_s(\vb{A}),
\end{equation*}
其中\(f(x) = f_1(x) \dotsm f_s(x)\),
且\(f_1(x),\dotsc,f_s(x)\)两两互素.

\begin{definition}
%@see: 《高等代数(第三版 下册)》(丘维声) P135 定义2
设\(\vb{A}\)是\(V\)上的线性变换.
如果域\(F\)上的一个一元多项式\(f(x)\)满足\(f(\vb{A}) = \vb0\),
则称“\(f(x)\)是\(\vb{A}\)的一个\DefineConcept{零化多项式}”.
\end{definition}

容易看出,零多项式是任意一个线性变换的零化多项式.
因此,我们通常讨论的零化多项式是“非零的零化多项式”.

\begin{proposition}\label{theorem:线性映射.有限维线性空间上的线性变换都有非零的零化多项式}
%@see: 《高等代数(第三版 下册)》(丘维声) P135
有限维线性空间\(V\)上任意一个线性变换都有非零的零化多项式.
\begin{proof}
设\(\dim V = n\),
则\(\dim\Hom(V,V) = n^2\),
从而\(\vb{I},\vb{A},\vb{A}^2,\dotsc,\vb{A}^{n^2}\)一定线性相关,
于是\(F\)中存在不全为零的元素\(k_0,k_1,k_2,\dotsc,k_{n^2}\),
使得\begin{equation*}
%@see: 《高等代数(第三版 下册)》(丘维声) P135 (17)
	k_0 \vb{I}
	+ k_1 \vb{A}
	+ k_2 \vb{A}^2
	+ \dotsb
	+ k_{n^2} \vb{A}^{n^2}
	= \vb0.
\end{equation*}
令\begin{equation*}
	f(x) \defeq k_0
	+ k_1 x
	+ k_2 x^2
	+ \dotsb
	+ k_{n^2} x^{n^2},
\end{equation*}
则有\(f(\vb{A}) = \vb0\),
即\(f(x)\)是\(\vb{A}\)的一个非零的零化多项式.
\end{proof}
\end{proposition}

\begin{definition}
%@see: 《高等代数(第三版 下册)》(丘维声) P135 定义3
设\(A\)是域\(F\)上的一个\(n\)阶矩阵.
如果域\(F\)上的一个一元多项式\(f(x)\)满足\(f(A) = 0\),
则称“\(f(x)\)是矩阵\(A\)的一个\DefineConcept{零化多项式}”.
\end{definition}

\begin{proposition}
%@see: 《高等代数(第三版 下册)》(丘维声) P135
设\(\vb{A}\)是域\(F\)上\(n\)维线性空间\(V\)上的线性变换,
\(A\)是\(\vb{A}\)在\(V\)的一个基下的矩阵,
则\begin{equation*}
	\text{$f(x)$是$\vb{A}$的零化多项式}
	\iff
	\text{$f(x)$是$A$的零化多项式}.
\end{equation*}
%TODO proof
\end{proposition}

\section{哈密顿--凯莱定理}
设数域\(K\)上的2阶矩阵\(A\)为\begin{equation*}
	A = \begin{bmatrix}
		1 & 2 \\
		0 & -1
	\end{bmatrix},
\end{equation*}
则\begin{equation*}
	A^2
	= \begin{bmatrix}
		1 & 0 \\
		0 & 1
	\end{bmatrix}
	= I,
\end{equation*}
可见\(A^2 - I = 0\),
因此\(f(\lambda) = \lambda^2 - 1\)就是\(A\)的一个零化多项式.
由于\begin{equation*}
	\abs{\lambda I - A}
	= \begin{vmatrix}
		\lambda-1 & -2 \\
		0 & \lambda+1
	\end{vmatrix}
	= (\lambda-1)(\lambda+1)
	= \lambda^2-1,
\end{equation*}
因此\(A\)的特征多项式\(f(\lambda) = \lambda^2-1\)就是\(A\)的一个零化多项式.
我们不禁好奇,是不是域\(F\)上的任意一个\(n\)阶矩阵的特征多项式都是它的零化多项式.
为了讨论这个问题,我们需要首先把域上的矩阵的概念推广维整环上的矩阵.

\subsection{\texorpdfstring{$\lambda$}{\textlambda} - 矩阵}
与域\(F\)上的矩阵类似,
我们也可以为
整环\(R\)上的矩阵
定义加法、纯量乘法、乘法,
而且这三种运算满足与域\(F\)上矩阵一样的运算法则.
类似地,我们也可以定义整环\(R\)上\(n\)阶矩阵的行列式.
而且我们在之前介绍的行列式的性质、行列式展开定理,
对于整环\(R\)上的\(n\)阶矩阵的行列式也成立.
对于整环\(R\)上的\(n\)阶矩阵\(A\),
有\begin{equation*}
	A A^* = A^* A = \abs{A} I,
\end{equation*}
其中\(A^*\)是\(A\)的伴随矩阵.

域\(F\)上的一元多项式环\(F[\lambda]\)是一个整环,
因此可以考虑环\(F[\lambda]\)上的\(n\)阶矩阵.
我们把环\(F[\lambda]\)上的\(n\)阶矩阵
称为 \DefineConcept{\(\lambda\) - 矩阵}.

\begin{example}
给定\(\lambda\) - 矩阵\[
%@see: 《高等代数(第三版 下册)》(丘维声) P138 (1)
	A(\lambda) = \begin{bmatrix}
		2 \lambda^3 + \lambda^2 + 1 & \lambda^2 - 3 \\
		\lambda^3 - 1 & 2 \lambda + 5
	\end{bmatrix},
\]
我们可以按照整环上矩阵的加法、纯量乘法,
将它改写成\begin{align*}
%@see: 《高等代数(第三版 下册)》(丘维声) P138 (2)
	A(\lambda)
	&= \begin{bmatrix}
		2 \lambda^3 & 0 \\
		\lambda^3 & 0
	\end{bmatrix}
	+ \begin{bmatrix}
		\lambda^2 & \lambda^2 \\
		0 & 0
	\end{bmatrix}
	+ \begin{bmatrix}
		0 & 0 \\
		0 & 2 \lambda
	\end{bmatrix}
	+ \begin{bmatrix}
		1 & -3 \\
		-1 & 5
	\end{bmatrix} \\
	&= \lambda^3
	\begin{bmatrix}
		2 & 0 \\
		1 & 0
	\end{bmatrix}
	+ \lambda^2
	\begin{bmatrix}
		1 & 1 \\
		0 & 0
	\end{bmatrix}
	+ \lambda
	\begin{bmatrix}
		0 & 0 \\
		0 & 2
	\end{bmatrix}
	+ \begin{bmatrix}
		1 & -3 \\
		-1 & 5
	\end{bmatrix},
\end{align*}
其中\(\lambda^k\)的系数矩阵\[
	\begin{bmatrix}
		2 & 0 \\
		1 & 0
	\end{bmatrix},
	\qquad
	\begin{bmatrix}
		1 & 1 \\
		0 & 0
	\end{bmatrix},
	\qquad
	\begin{bmatrix}
		0 & 0 \\
		0 & 2
	\end{bmatrix},
	\qquad
	\begin{bmatrix}
		1 & -3 \\
		-1 & 5
	\end{bmatrix}
\]都是域\(F\)上的矩阵.
\end{example}

假设我们把两个\(\lambda\) - 矩阵\(A(\lambda)\)和\(B(\lambda)\)
都展开成\begin{equation*}
	A(\lambda)
	= \sum_{i=0}^m \lambda^i \alpha_i,
	\qquad
	B(\lambda)
	= \sum_{i=0}^m \lambda^i \beta_i,
\end{equation*}
其中\(\alpha_i,\beta_i \in M_n(F)\),
那么根据
两个一元多项式相等的定义
以及两个\(\lambda\) - 矩阵相等的定义,
可以推出,
\(A(\lambda)\)与\(B(\lambda)\)相等,
当且仅当它们系数矩阵对应相等.

\subsection{哈密顿--凯莱定理}
有了上述准备知识以后,
我们就可以着手证明下面的哈密顿--凯莱定理了.
\begin{theorem}
%@see: 《高等代数(第三版 下册)》(丘维声) P138 Hamilton-Cayley定理
设\(A\)是域\(F\)上的\(n\)阶矩阵,
则\(A\)的特征多项式\(f(\lambda)\)是\(A\)的一个零化多项式.
%TODO proof
\end{theorem}

\begin{corollary}
%@see: 《高等代数(第三版 下册)》(丘维声) P139 Hamilton-Cayley定理
设\(\vb{A}\)是域\(F\)上\(n\)维线性空间\(V\)上的一个线性变换,
则\(\vb{A}\)的特征多项式\(f(\lambda)\)是\(\vb{A}\)的一个零化多项式.
\end{corollary}

\begin{example}
%@see: 《高等代数(第三版 下册)》(丘维声) P139 习题9.6 1.
设\(\vb{A}\)是域\(F\)上\(n\)维线性空间\(V\)上的线性变换.
证明:如果\(\vb{A}\)的特征多项式\(f(\lambda)\)在\(F[\lambda]\)中可以分解成\[
	f(\lambda) = (\lambda-\lambda_1)^{r_1} \dotsm (\lambda-\lambda_s)^{r_s},
\]
则\[
	V = \Ker(\lambda_1 \vb{I} - \vb{A})^{r_1} \DirectSum \dotsb \DirectSum \Ker(\lambda_s \vb{I} - \vb{A})^{r_s}.
\]
%TODO proof
\end{example}

\begin{example}
%@see: 《高等代数(第三版 下册)》(丘维声) P139 习题9.6 2.
设\(A\)是域\(F\)上\(n\)阶可逆矩阵,
\(I\)是域\(F\)上\(n\)阶单位矩阵.
证明:存在\(F\)中元素\(k_0,k_1,\dotsc,k_{n-1}\),
使得\[
	A^{-1}
	= k_{n-1} A^{n-1} + k_{n-2} A^{n-2}
	+ \dotsb + k_2 A^2 + k_1 A + k_0 I.
\]
%TODO proof
\end{example}

\begin{example}
%@see: 《高等代数(第三版 下册)》(丘维声) P139 习题9.6 3.
设\(A\)是域\(F\)上的\(n\)阶矩阵,
\(B\)是域\(F\)上的\(m\)阶矩阵.
证明:矩阵方程\(AX-XB=0\)只有零解的充分必要条件是
\(A\)与\(B\)没有公共特征值.
%TODO proof
\end{example}

\section{线性变换的最小多项式}
由哈密顿凯莱定理可知,
有限维线性空间\(V\)上的线性变换\(\vb{A}\)的特征多项式\(f(\lambda)\)是\(\vb{A}\)的一个零化多项式.
在本节我们来讨论\(\vb{A}\)还有没有其他零化多项式.

\subsection{线性变换的最小多项式}
\begin{definition}
%@see: 《高等代数(第三版 下册)》(丘维声) P140 定义1
设\(\vb{A}\)是域\(F\)上线性空间\(V\)上的一个线性变换.
在\(\vb{A}\)的所有非零的零化多项式中,
次数最低的首项系数为\(1\)的多项式,
称为“\(\vb{A}\)的\DefineConcept{最小多项式}”.
\end{definition}

\begin{proposition}
%@see: 《高等代数(第三版 下册)》(丘维声) P140 命题1
线性空间\(V\)上的线性变换\(\vb{A}\)的最小多项式是唯一的.
%TODO proof
\end{proposition}

\begin{proposition}
%@see: 《高等代数(第三版 下册)》(丘维声) P140 命题2
线性空间\(V\)上的线性变换\(\vb{A}\)的
任一零化多项式\(g(\lambda)\)是
\(\vb{A}\)的最小多项式\(m(\lambda)\)的倍式.
%TODO proof
\end{proposition}

\begin{proposition}
%@see: 《高等代数(第三版 下册)》(丘维声) P140 命题3
域\(F\)上有限维线性空间\(V\)上的线性变换\(\vb{A}\)的最小多项式\(m(\lambda)\)
与特征多项式\(f(\lambda)\)
在\(F\)中有相同的根(重数可以不同).
%TODO proof
\end{proposition}

\begin{definition}
%@see: 《高等代数(第三版 下册)》(丘维声) P140 定义2
设\(A\)是域\(F\)上的\(n\)阶矩阵.
在\(A\)的所有非零的零化多项式中,
次数最低的首项系数为\(1\)的多项式,
称为“\(A\)的\DefineConcept{最小多项式}”.
\end{definition}

%@see: 《高等代数(第三版 下册)》(丘维声) P141
设\(n\)维线性空间\(V\)上的线性变换\(\vb{A}\)在\(V\)的一个基下的矩阵是\(A\).
之前我们已经指出,\(g(\lambda)\)是\(\vb{A}\)的零化多项式,
当且仅当\(g(\lambda)\)是\(A\)的零化多项式.
由此可见,\(m(\lambda)\)是\(\vb{A}\)的最小多项式,
当且仅当\(m(\lambda)\)是\(A\)的做小多项式.
于是得出:
\begin{corollary}
%@see: 《高等代数(第三版 下册)》(丘维声) P141 推论4
域\(F\)上\(n\)阶矩阵\(A\)的最小多项式\(m(\lambda)\)
与特征多项式\(f(x)\)
在\(F\)中有相同的根(重数可以不同).
\end{corollary}

%@see: 《高等代数(第三版 下册)》(丘维声) P141
由于相似的矩阵可以看作\(V\)上同一个线性变换\(\vb{A}\)在\(V\)的不同基下的矩阵,
因此有如下结论:
\begin{corollary}
%@see: 《高等代数(第三版 下册)》(丘维声) P141 推论5
相似的矩阵有相同的最小多项式.
\end{corollary}

\subsection{求解最小多项式的基本步骤}
如何求解线性变换或矩阵的最小多项式呢?
一种方法是:
先找出\(\vb{A}\)的一个零化多项式\(g(\lambda)\),
然后分析\(g(\lambda)\)的哪个因式是\(\vb{A}\)的最小多项式.
\begin{definition}%\label{definition:线性映射.幂零变换}
%@see: 《高等代数(第三版 下册)》(丘维声) P141 定义3
设\(\vb{A}\)是域\(F\)上线性空间\(V\)上的线性变换.
如果存在一个正整数\(k\),
使得\(\vb{A}^k = \vb0\),
则称\(\vb{A}\)是\DefineConcept{幂零变换}.
使得\(\vb{A}^k = \vb0\)成立的最小正整数
称为“\(\vb{A}\)的\DefineConcept{幂零指数}”.
\end{definition}

\begin{proposition}%\label{theorem:线性映射.幂零变换的等价命题}
%@see: 《高等代数(第三版 下册)》(丘维声) P141
设\(\vb{A}\)是域\(F\)上线性空间\(V\)上的线性变换,
则\begin{align*}
	&\text{$\vb{A}$是幂零指数为$k$的幂零变换} \\
	&\iff \vb{A}^k = \vb0,
		(\forall r<k)[\vb{A}^r \neq \vb0] \\
	&\iff \text{$\lambda^k$是$\vb{A}$的一个零化多项式},
		(\forall r<k)[\text{$\lambda^k$不是$\vb{A}$的一个零化多项式}] \\
	&\iff \text{$\vb{A}$的最小多项式是$\lambda^k$}.
\end{align*}
如果域\(F\)上\(n\)阶矩阵\(A\)是\(\vb{A}\)在\(V\)的一个基下的矩阵,
则\begin{equation*}
	\text{$A$是幂零指数为$k$的幂零矩阵}
	\iff \text{$A$的最小多项式是$\lambda^k$}.
\end{equation*}
\end{proposition}

\begin{proposition}%\label{theorem:线性映射.幂等变换的等价命题}
%@see: 《高等代数(第三版 下册)》(丘维声) P141
设\(\vb{A}\)是域\(F\)上线性空间\(V\)上的线性变换,
则\begin{align*}
	&\text{$\vb{A}$是幂等变换} \\
	&\iff \vb{A}^2 = \vb{A} \\
	&\iff \text{$\lambda^2 - \lambda$是$\vb{A}$的一个零化多项式} \\
	&\iff \text{$\vb{A}$的最小多项式是$\lambda^2-\lambda$或$\lambda$或$\lambda-1$}.
\end{align*}
如果域\(F\)上\(n\)阶矩阵\(A\)是\(\vb{A}\)在\(V\)的一个基下的矩阵,
则\begin{equation*}
	\text{$A$是幂等变换}
	\iff \text{$A$的最小多项式是$\lambda^2-\lambda$或$\lambda$或$\lambda-1$}.
\end{equation*}
\end{proposition}

\begin{proposition}\label{theorem:线性映射.任意线性变换的最小多项式}
%@see: 《高等代数(第三版 下册)》(丘维声) P141 命题6
设\(V\)是域\(F\)上的一个线性空间,
\(\vb{A}\)是\(V\)上的线性变换,
\(\vb{B}\)是\(V\)上幂零指数为\(k\)的幂零变换,
则\begin{gather*}
	\vb{A} = a \vb{I} + \vb{B}
	\iff \text{$\vb{A}$的最小多项式是$(\lambda-a)^k$}, \\
	\vb{A} = a \vb{I}
	\iff \text{$\vb{A}$的最小多项式是$\lambda-a$}.
\end{gather*}
%TODO proof
\end{proposition}

\begin{definition}%\label{definition:线性映射.若尔当块}
设\(F\)是一个域.
对于\(\forall a \in F\)和\(\forall k \in \mathbb{N}^+\),
定义:\begin{equation}
%@see: 《高等代数(第三版 下册)》(丘维声) P142 (1)
	J_k(a)
	\defeq
	\begin{bmatrix}
		a & 1 & 0 & \dots & 0 & 0 \\
		0 & a & 1 & \dots & 0 & 0 \\
		0 & 0 & a & \dots & 0 & 0 \\
		\vdots & \vdots & \vdots & & \vdots & \vdots \\
		0 & 0 & 0 & \dots & a & 1 \\
		0 & 0 & 0 & \dots & 0 & a
	\end{bmatrix}.
\end{equation}
把\(J_k(a)\)称为一个\(k\)阶\DefineConcept{若尔当块}.
\end{definition}

\begin{example}
%@see: 《高等代数(第三版 下册)》(丘维声) P136 习题9.5 4.
设\(\vb{A}\)是域\(F\)上\(n\)维线性空间\(V\)上的一个线性变换,
\(\vb{A}\)在基\(\AutoTuple{\alpha}{n}\)下的矩阵为\(n\)阶若尔当块\(J_n(a)\).
证明:\begin{itemize}
	\item 如果\(\alpha_n\)属于\(\vb{A}\)的不变子空间\(W\),则\(W = V\);
	\item 基向量\(\alpha_1\)属于\(\vb{A}\)的任意一个非零不变子空间;
	\item \(V\)不能分解成\(\vb{A}\)的两个非平凡不变子空间的直和.
\end{itemize}
%TODO proof
\end{example}

\begin{proposition}
%@see: 《高等代数(第三版 下册)》(丘维声) P142 命题7
设\(\vb{A}\)是域\(F\)上\(k\)维线性空间\(W\)上的线性变换.
若\(\vb{A} = a \vb{I} + \vb{B}\),
其中\(\vb{B}\)是\(W\)上幂零指数为\(k\)的幂零变换,
则\(W\)中有一个基\[
	\vb{B}^{k-1}\alpha,\vb{B}^{k-2}\alpha,\dotsc,\vb{B}\alpha,\alpha,
\]
使得\(\vb{A}\)在这个基下的矩阵为\(J_k(a)\).
%TODO proof
\end{proposition}

\begin{proposition}\label{theorem:线性映射.矩阵的最小多项式}
%@see: 《高等代数(第三版 下册)》(丘维声) P142 命题8
设\(A\)是域\(F\)上的一个\(k\)阶矩阵,
则\(A \sim J_k(a)\)
当且仅当\(A\)的最小多项式是\((\lambda-a)^k\).
%TODO proof
\end{proposition}

%@see: 《高等代数(第三版 下册)》(丘维声) P143
从幂零变换和幂等变换的最小多项式,
以及\cref{theorem:线性映射.任意线性变换的最小多项式},
我们可以看出:
线性变换的最小多项式能够决定这个线性变换是什么样子.
从\cref{theorem:线性映射.矩阵的最小多项式}
可以看出:
矩阵的最小多项式能够决定这个矩阵相似于什么样的形式最简单的矩阵(即若尔当块).
这促使我们利用线性变换的最小多项式来研究线性变换的形式最简的矩阵表示.

\subsection{线性映射在非平凡子空间上的限制的最小多项式}
设\(\vb{A}\)是域\(F\)上\(n\)维线性空间\(V\)上的线性变换.
把\(\vb{A}\)的最小多项式\(m(\lambda)\)在\(F[\lambda]\)中分解成
两两不等的首项系数为\(1\)的不可约多项式方幂的乘积,
则根据\cref{theorem:线性映射.线性映射多项式的核空间的直和分解式2} 可知,
\(V\)可以分解成\(\vb{A}\)的非平凡不变子空间\(\AutoTuple{W}{s}\)的直和:\[
	V = W_1 \DirectSum \dotsb \DirectSum W_s.
\]
在这些非平凡不变子空间中各取一个基,
把它们合起来得到\(V\)的一个基,
则\(\vb{A}\)在\(V\)的这个基下的矩阵\(A\)是一个分块对角矩阵
\(A = \diag\{\AutoTuple{A}{s}\}\),
其中\(A_j\)是\(\vb{A} \setrestrict W_j\)在\(W_j\)的上述基下的矩阵.
为了使\(\vb{A}\)有形式最简单的矩阵表示,
我们自然需要在\(W_j\)中取一个合适的基,
使得\(\vb{A} \setrestrict W_j\)在\(W_j\)的这个基下的矩阵具有最简单的形式.
为此产生一个问题:
\(\vb{A} \setrestrict W_j\)的最小多项式\(m_j(\lambda)\)
与\(\vb{A}\)的最小多项式\(m(\lambda)\)有什么关系?
下面的定义回答了这个问题.
\begin{theorem}
%@see: 《高等代数(第三版 下册)》(丘维声) P143 定理9
设\(\vb{A}\)是域\(F\)上线性空间\(V\)上线性变换.
如果\(V\)能分解成\(\vb{A}\)的一些非平凡不变子空间\(\AutoTuple{W}{s}\)的直和,
即\[
	V = W_1 \DirectSum \dotsb \DirectSum W_s,
\]
则\(\vb{A}\)的最小多项式为\[
	m(\lambda)
	= [m_1(\lambda),\dotsc,m_s(\lambda)],
\]
其中\(m_j(\lambda)\)是\(W_j\)上的线性变换\(\vb{A} \setrestrict W_j\)的最小多项式,
而\([m_1(\lambda),\dotsc,m_s(\lambda)]\)是
\(m_1(\lambda),\dotsc,m_s(\lambda)\)的最小公倍式.
%TODO proof
\end{theorem}

\begin{corollary}
%@see: 《高等代数(第三版 下册)》(丘维声) P144 推论10
设\(A\)是域\(F\)上一个\(n\)阶分块对角矩阵,
即\(A = \diag\{\AutoTuple{A}{s}\}\),
则\(A\)的最小多项式为\[
	m(\lambda)
	= [m_1(\lambda),\dotsc,m_s(\lambda)],
\]
其中\(m_j(\lambda)\)是矩阵\(A_j\)的最小多项式,
而\([m_1(\lambda),\dotsc,m_s(\lambda)]\)是
\(m_1(\lambda),\dotsc,m_s(\lambda)\)的最小公倍式.
%TODO proof
\end{corollary}

\begin{definition}
%@see: 《高等代数(第三版 下册)》(丘维声) P144 定义4
由若干个若尔当块组成的分块对角矩阵
称为\DefineConcept{若尔当形矩阵}.
\end{definition}
\begin{remark}
对角矩阵可以看成是由若干个1阶若尔当块组成的若尔当形矩阵.
\end{remark}

线性变换的最小多项式在研究线性变换的结构中起着十分重要的作用.
现在先利用最小多项式给出线性变换可对角化的一个充分必要条件,
然后用最小多项式研究不可以对角化的线性变换的结构.

\begin{theorem}
%@see: 《高等代数(第三版 下册)》(丘维声) P145 定理11
设\(\vb{A}\)是域\(F\)上\(n\)维线性空间\(V\)上的线性变换,
则\(\vb{A}\)可对角化的充分必要条件是,
\(\vb{A}\)的最小多项式\(m(\lambda)\)在\(F[\lambda]\)中
能分解成不同的一次因式的乘积.
%TODO proof
\end{theorem}

\begin{corollary}
%@see: 《高等代数(第三版 下册)》(丘维声) P145 推论12
域\(F\)上\(n\)阶矩阵\(A\)可对角化的充分必要条件是,
\(A\)的最小多项式在\(F[\lambda]\)中
能分解成不同的一次因式的乘积.
\end{corollary}
\begin{remark}
在一些情形下,
利用最小多项式来判别线性变换或矩阵是否可以对角化,
论证过程往往很简洁.
\end{remark}

利用最小多项式还可以研究不可对角化的线性变换的结构.

设\(\vb{A}\)是域\(F\)上\(n\)维线性空间\(V\)上的线性变换.
如果\(\vb{A}\)的最小多项式\(m(\lambda)\)在\(F[\lambda]\)中
能够分解成一次因式的乘积:\[
%@see: 《高等代数(第三版 下册)》(丘维声) P146 (5)
	m(\lambda)
	= (\lambda-\lambda_1)^{k_1}
	(\lambda-\lambda_2)^{k_2}
	\dotsm
	(\lambda-\lambda_s)^{k_s},
\]
其中\(\AutoTuple{\lambda}{s}\)是\(F\)中两两不同的元素,
则\[
%@see: 《高等代数(第三版 下册)》(丘维声) P146 (6)
	V
	= \Ker(\vb{A} - \lambda_1 \vb{I})^{k_1}
	\DirectSum
	\Ker(\vb{A} - \lambda_2 \vb{I})^{k_2}
	\DirectSum
	\dotsb
	\DirectSum
	\Ker(\vb{A} - \lambda_s \vb{I})^{k_s}.
\]
记\(W_j = \Ker(\vb{A} - \lambda_j \vb{I})^{k_j}\),
那么由\cref{theorem:线性映射.线性变换的不变子空间3} 可知,
\(W_j\)是\(\vb{A}\)的不变子空间.
由上式可知\[
%@see: 《高等代数(第三版 下册)》(丘维声) P146 (7)
	V = W_1 \DirectSum W_2 \DirectSum \dotsb \DirectSum W_s.
\]
在\(W_j\)中取一个基,
把它们合起来得到\(V\)的一个基,
\(\vb{A}\)在\(V\)的这个基下的矩阵\(A\)是一个分块对角矩阵
\(A = \diag\{\AutoTuple{A}{s}\}\),
其中\(A_j\)是\(\vb{A} \setrestrict W_j\)在\(W_j\)的上述基下的矩阵.
既然要最简化矩阵\(A\)的形式,
就应当最简化矩阵\(A_j\)的形式.
为此我们需要研究\(W_j\)上的线性变换\(\vb{A} \setrestrict W_j\).
我们断言\(\vb{A} \setrestrict W_j\)的最小多项式是\((\lambda-\lambda_j)^{k_j}\),
理由如下.

任取\(\alpha_j \in W_j\),
由于\(W_j = \Ker(\vb{A} - \lambda_j \vb{I})^{k_j}\),
所以\[
	(\vb{A} \setrestrict W_j - \lambda_j \vb{I})^{k_j} \alpha_j
	= (\vb{A} - \lambda_j \vb{I})^{k_j} \alpha_j
	= 0,
\]
从而\((\vb{A} \setrestrict W_j - \lambda_j \vb{I})^{k_j} = \vb0\),
于是\((\lambda-\lambda_j)^{k_j}\)是\(\vb{A} \setrestrict W_j\)的一个零化多项式,
因此\(\vb{A} \setrestrict W_j\)的最小多项式为
\(m_j(\lambda) = (\lambda-\lambda_j)^{t_j}\),
其中\(t_j \leq k_j\).
于是\(\vb{A}\)的最小多项式\(m(\lambda)\)为\begin{align*}
%@see: 《高等代数(第三版 下册)》(丘维声) P146 (8)
	m(\lambda)
	&= [(\lambda-\lambda_1)^{t_1},\dotsc,(\lambda-\lambda_s)^{t_s}] \\
	&= (\lambda-\lambda_1)^{t_1} \dotsm (\lambda-\lambda_s)^{t_s}.
\end{align*}
根据\hyperref[theorem:多项式.唯一因式分解定理]{唯一因式分解定理},
立即可得\[
	t_1 = k_1,
	t_2 = k_2,
	\dotsc,
	t_s = k_s.
\]
因此\(\vb{A} \setrestrict W_j\)的最小多项式\(m_j(\lambda) = (\lambda-\lambda_j)^{k_j}\).
根据\cref{theorem:线性映射.任意线性变换的最小多项式} 可知,
\(\vb{A} \setrestrict W_j = \lambda_j \vb{I} + \vb{B}_j\),
其中\(\vb{B}_j\)是\(W_j\)上的幂零变换,且幂零指数为\(k_j\).
由于\(\vb{A} \setrestrict W_j\)在\(W_j\)的上述基下的矩阵是\(A_j\),
因此\(\vb{B}_j = \vb{A} \setrestrict W_j - \lambda_j \vb{I}\)
在\(W_j\)的上述基下的矩阵\(B_j = A_j - \lambda_j I\).
于是为了最简化矩阵\(A_j\),
就应当最简化矩阵\(B_j\).
这里就产生一个问题:
如果\(\vb{B}\)是域\(F\)上\(r\)维线性空间\(W\)上幂零指数为\(k\)的幂零变换,
那么能否在\(W\)中找到一个基,使得\(\vb{B}\)在这个基下的矩阵最简单?
我们将在下一节详细讨论这个问题.

\section{幂零变换的结构}
在上一节的最后,我们指出,研究不可对角化的线性变换\(\vb{A}\)的结构的问题,
只要\(\vb{A}\)的最小多项式\(m(\lambda)\)在\(F[\lambda]\)中可分解成一次因式的乘积,
那么该问题就可以归结为研究幂零变换的结构的问题,
也就是研究域\(F\)上\(r\)维线性空间\(W\)上的幂零变换\(\vb{B}\),
能否在\(W\)中找到一个基,使得\(\vb{B}\)在此基下的矩阵最简单.

\begin{proposition}\label{theorem:线性映射.由幂零变换衍生的线性无关向量组}
%@see: 《高等代数(第三版 下册)》(丘维声) P148 命题1
设\(\vb{B}\)是域\(F\)上\(r\)维线性空间\(W\)上的幂零变换,且幂零指数为\(l\),
则存在\(\xi \in W\),
使得向量组\[
	\vb{B}^{l-1}\xi,
	\vb{B}^{l-2}\xi,
	\dotsc,
	\vb{B}^2\xi,
	\vb{B}\xi,
	\xi
\]线性无关,
从而\(l \leq \dim W\).
%TODO proof
\end{proposition}

\def\BasisL{\vb{B}^{l-1}\xi,\vb{B}^{l-2}\xi,\dotsc,\vb{B}^2\xi,\vb{B}\xi,\xi}
首先讨论幂零指数\(l = \dim W\)的情形.
此时根据\cref{theorem:线性映射.由幂零变换衍生的线性无关向量组} 得,
存在\(\xi \in W\),使得向量组\[
	\BasisL
\]线性无关.
由于\(l = \dim W\),
因此它们就是\(W\)的一个基.
根据\cref{theorem:线性映射.可以JC分解的线性变换在某个基下的矩阵是若尔当块} 得,
\(\vb{B}\)在\(W\)的基\[
	\BasisL
\]下的矩阵为\(J_l(0)\).

接着讨论幂零指数\(l < \dim W\)的情形.
此时,仍由\cref{theorem:线性映射.由幂零变换衍生的线性无关向量组} 得,
存在\(\xi \in W\),使得向量组\[
	\BasisL
\]线性无关.
由于\(l < \dim W\),上述向量组不是\(W\)的一个基.
注意到它们生成的子空间\[
	U \defeq \Span\{\BasisL\}
\]是\(\vb{B}\)的不变子空间,
并且\(\vb{B}\)在这个子空间\(U\)上的限制\(\vb{B} \setrestrict U\)
在基\[
	\BasisL
\]下的矩阵是一个若尔当块\(J_l(0)\).
为此,我们抽象出一个概念:
\begin{definition}\label{definition:线性映射.生成强循环子空间}
%@see: 《高等代数(第三版 下册)》(丘维声) P148 定义1
设\(\vb{B}\)是域\(F\)上\(r\)维线性空间\(W\)上的线性变换,
向量\(\xi \in W\).
记\[
	U \defeq \Span\{\BasisL\}.
\]
如果存在正整数\(t\),
使得\[
	\vb{B}^{t-1} \xi \neq 0,
	\qquad
	\vb{B}^t \xi = 0,
\]
则称“子空间\(U\)是由\(\xi\)生成的 \DefineConcept{\(\vb{B}\) - 强循环子空间}”,
记作\(Z_t(\xi;\vb{B})\).
\end{definition}

显然,\(Z_t(\xi;\vb{B})\)是\(\vb{B}\)的不变子空间.
由\cref{example:线性映射.强循环向量组} 可知,
向量组\begin{equation*}
	\BasisL
\end{equation*}线性无关,
从而它们是\(Z_t(\xi;\vb{B})\)的一个基,
于是\(l\)是\(Z_t(\xi;\vb{B})\)的维数.
由上述讨论可知,
\(\vb{B}\)在\(Z_t(\xi;\vb{B})\)上的限制
\(\vb{B} \setrestrict Z_t(\xi;\vb{B})\)
在这个基下的矩阵是一个\(l\)阶若尔当块\(J_l(0)\).
于是,只要我们能把\(W\)分解成若干个\(\vb{B}\) - 强循环子空间的直和,
那么在每个\(\vb{B}\) - 强循环子空间上选择上述这样的基,
合起来成为\(W\)的一个基,
\(\vb{B}\)在\(W\)的这个基下的矩阵
就是由若干个主对角元为\(0\)的若尔当块组成的分块对角矩阵,即若尔当形矩阵.
我们来探索这个问题.

对于任意\(\alpha \in W\)且\(\alpha\neq0\),
由于\(\vb{B}^l=\vb0\),
因此\(\vb{B}^l\alpha=0\).
设正整数\(t\)使得\(\vb{B}^{t-1}\alpha\neq0\),而\(\vb{B}^t\alpha=0\),
则有一个\(\vb{B}\) - 强循环子空间\(\Span\{\BasisL\}\).
由于\begin{equation*}
	\vb{B}(\vb{B}^{t-1}\alpha)
	= \vb{B}^t\alpha
	= 0
	= 0(\vb{B}^{t-1}\alpha),
\end{equation*}
因此\(\vb{B}^{t-1}\alpha\)是\(\vb{B}\)的属于特征值\(0\)的一个特征向量.
这表明任意一个\(\vb{B}\) - 强循环子空间\(\Span\{\BasisL\}\)的
第一个基向量\(\vb{B}^{t-1}\alpha\)是\(\vb{B}\)的属于特征值\(0\)的一个特征向量.

\(\vb{B}\)是幂零变换,
它的特征值只有\(0\).
我们把\(\vb{B}\)的属于特征值\(0\)的特征子空间记作\(W_0\).
对于任意\(\eta \in W_0\),
由于\(\vb{B}\eta = 0\eta = 0\),
因此\(\Span\{\eta\}\)是一个\(\vb{B}\) - 强循环子空间.
根据上面一段描述,
有可能\(\eta\)是某一个\(\vb{B}\) - 强循环子空间的第一个基向量\(\vb{B}^{t-1}\alpha\).
当\(t=1\)时,这个\(\vb{B}\) - 强循环子空间就是\(\Span\{\eta\}\).

假如\(W\)能分解成若干个\(\vb{B}\) - 强循环子空间的直和:\begin{equation*}
	W = \Span\{\vb{B}^{t_1-1}\alpha_1,\dotsc,\alpha_1\}
		\DirectSum \dotsb
		\DirectSum \Span\{\vb{B}^{t_s-1}\alpha_s,\dotsc,\alpha_s\}
		\DirectSum \Span\{\eta_1\}
		\DirectSum \dotsb
		\DirectSum \Span\{\eta_q\}.
\end{equation*}
于是向量组\begin{equation*}
	\vb{B}^{t_1-1}\alpha_1,
	\vb{B}^{t_2-1}\alpha_2,
	\dotsc,
	\vb{B}^{t_s-1}\alpha_s,
	\eta_1,
	\dotsc,
	\eta_q
\end{equation*}线性无关.
根据上面两段论述可知,它们都属于\(W_0\).
又由于对于任意\(\eta \in W_0\),
\(\eta\)都属于某一个\(\vb{B}\) - 强循环子空间,
因此我们猜测上述向量组是\(W_0\)的一个基.
从而猜测有下述结论:
\begin{theorem}
%@see: 《高等代数(第三版 下册)》(丘维声) P149 定理2
设\(\vb{B}\)是域\(F\)上\(r\)维线性空间\(W\)上的幂零变换,其幂零指数为\(l\),
把\(\vb{B}\)的属于特征值\(0\)的特征子空间记为\(W_0\),
则\(W\)可以分解成\(\dim W_0\)个\(\vb{B}\) - 强循环子空间的直和.
%TODO proof
\end{theorem}

有了上述定理,我们就可以了解幂零变换的结构了.
\begin{theorem}
%@see: 《高等代数(第三版 下册)》(丘维声) P151 定理3
设\(\vb{B}\)是域\(F\)上\(r\)维线性空间\(W\)上的幂零变换,其幂零指数为\(l\),
则\(W\)中存在一个基,使得\(\vb{B}\)在这个基下的矩阵\(B\)为若尔当形矩阵,
其中每个若尔当块的主对角元都是\(0\),且阶数不超过\(l\),
若尔当块的总数等于\(r - \rank\vb{B}\),
\(t\)阶若尔当块的个数为\begin{equation}
%@see: 《高等代数(第三版 下册)》(丘维声) P151 (1)
	N(t) = \rank\vb{B}^{t+1} + \rank\vb{B}^{t-1} - 2 \rank\vb{B}^t.
\end{equation}
\rm
这个若尔当形矩阵\(B\)称为幂零变换\(\vb{B}\)的\DefineConcept{若尔当标准型}.
除去若尔当块的排序次序外,\(\vb{B}\)的若尔当标准型是唯一的.
%TODO proof
\end{theorem}

\begin{corollary}
%@see: 《高等代数(第三版 下册)》(丘维声) P152 推论4
设\(B\)是域\(F\)上的\(r\)阶幂零矩阵,其幂零指数为\(l\),
则\(B\)相似于一个若尔当形矩阵,
其中每个若尔当块的主对角元都是\(0\),且阶数不超过\(l\),
若尔当块的总数等于\(r - \rank b\),
\(t\)阶若尔当块的个数为\begin{equation}
%@see: 《高等代数(第三版 下册)》(丘维声) P152 (10)
	N(t) = \rank B^{t+1} + \rank B^{t-1} - 2 \rank B^t.
\end{equation}
\end{corollary}

\section{线性变换的若尔当标准型}
现在我们利用幂零变换的结构来研究最小多项式可以分解成一次因式乘积的线性变换的结构.
\begin{theorem}
%@see: 《高等代数(第三版 下册)》(丘维声) P153 定理1
设\(\vb{A}\)是域\(F\)上\(n\)维线性空间\(V\)上的线性变换.
如果\(\vb{A}\)的最小多项式\(m(\lambda)\)
在\(F[\lambda]\)中能分解成一次因式的乘积\begin{equation*}
	m(\lambda)
	= (\lambda-\lambda_1)^{l_1}
	(\lambda-\lambda_2)^{l_2}
	\dotsm
	(\lambda-\lambda_s)^{l_s},
\end{equation*}
则\(V\)中存在一个基,
使得\(\vb{A}\)在这个基下的矩阵\(A\)是若尔当形矩阵,其主对角元是\(\vb{A}\)的全部特征值,
主对角元为\(\lambda_j\)的若尔当块的总数为\begin{equation*}
	N_j = \dim V - \rank(\vb{A}-\lambda_j\vb{I}),
	\quad j=1,2,\dotsc,s,
\end{equation*}
且其中每个若尔当块的阶数不超过\(l_j\),
\(t\)阶若尔当块\(J_t(\lambda_j)\)的个数为\begin{equation*}
	N_j(t)
	= \rank(\vb{A}-\lambda_j\vb{I})^{t+1}
	+ \rank(\vb{A}-\lambda_j\vb{I})^{t-1}
	- 2 \rank(\vb{A}-\lambda_j\vb{I})^t.
\end{equation*}
\rm
这个若尔当形矩阵\(A\)称为\(\vb{A}\)的若尔当标准型.
除去若尔当块的排序次序外,\(\vb{A}\)的若尔当标准型是唯一的.
%TODO proof
\end{theorem}

\endgroup

\chapter{内积空间}
迄今为止,我们对于线性空间和线性映射的研究
主要是围绕线性空间的加法、纯量乘法两种运算来进行的.
我们曾经对实数域上的\(n\)维向量空间\(\mathbb{R}^n\)定义了一个内积,
从而在\(\mathbb{R}^n\)中引进了长度、正交等度量概念.
受此启发,我们在本章讨论如何在实数域、复数域以至于任意域上的线性空间中引进度量概念,
然后研究具有度量的线性空间的结构,
并且研究与度量有关的线性变换的性质.

\section{双线性函数}

\section{欧几里得空间}
\subsection{实线性空间上的内积}
\begin{definition}\label{definition:欧几里得空间.实线性空间上的内积}
设\(V\)是实数域\(\mathbb{R}\)上的一个线性空间.
如果映射\(\rho\colon V \times V \to \mathbb{R}\)满足以下性质\begin{itemize}
	\item {\rm\bf 对称性}:\begin{equation*}
		(\forall \alpha,\beta \in V)
		[
			\rho(\alpha,\beta)
			= \rho(\beta,\alpha)
		];
	\end{equation*}

	\item {\rm\bf 线性性}:\begin{gather*}
		(\forall \alpha,\beta,\gamma \in V)
		[
			\rho(\alpha+\beta,\gamma)
			= \rho(\alpha,\gamma) + \rho(\beta,\gamma)
		], \\
		(\forall \alpha,\beta \in V)
		(\forall k \in \mathbb{C})
		[
			\rho(k\alpha,\beta)
			= k\rho(\alpha,\beta)
		];
	\end{gather*}

	\item {\rm\bf 正定性}:\begin{gather*}
		(\forall \alpha \in V)
		[
			\rho(\alpha,\alpha) \geq 0
		], \\
		(\forall \alpha \in V)
		[
			\rho(\alpha,\alpha) = 0
			\iff
			\alpha = 0
		],
	\end{gather*}
\end{itemize}
则称“\(f\)是实线性空间\(V\)上的一个\DefineConcept{内积}(inner product)”.
\end{definition}

虽然在\hyperref[definition:欧几里得空间.实线性空间上的内积]{定义}中,
我们只规定了内积对第一个自变量具有线性性,
但是我们容易证明,内积对第二个自变量也具有线性性:
\begin{property}\label{theorem:实线性空间.实线性空间上内积对第二个自变量具有线性性}
%@see: 《高等代数(第三版 下册)》(丘维声) P193
%@see: 《Linear Algebra Done Right (Fourth Eidition)》(Sheldon Axler) P185 6.6
设\(\rho\)是实线性空间\(V\)上的一个内积,
则内积\(\rho\)对第二个自变量具有线性性,
即\begin{equation}
	\rho(\alpha,k_1\beta_1+k_2\beta_2)
	= k_1 \rho(\alpha,\beta_1)
	+ k_2 \rho(\alpha,\beta_2).
\end{equation}
\begin{proof}
由内积的对称性和它对第一个自变量的线性性,有\begin{align*}
	\rho(\alpha,k_1\beta_1+k_2\beta_2)
	&= \rho(k_1\beta_1+k_2\beta_2,\alpha) \\
	&= k_1 \rho(\beta_1,\alpha)
		+ k_2 \rho(\beta_2,\alpha) \\
	&= k_1 \rho(\alpha,\beta_1)
		+ k_2 \rho(\alpha,\beta_2).
	\qedhere
\end{align*}
\end{proof}
\end{property}

\begin{proposition}\label{theorem:欧几里得空间.实内积空间上的内积的等价定义}
%@see: 《高等代数(第三版 下册)》(丘维声) P174 定义1
设\(V\)是实数域\(\mathbb{R}\)上的一个线性空间,
则“\(f\)是实线性空间\(V\)上的一个内积”的充分必要条件是
“\(f\)是\(V\)上的一个正定对称双线性函数”.
\end{proposition}

\begin{proposition}
%@see: 《高等代数(第三版 下册)》(丘维声) P174 命题1
设\(V\)是\(\mathbb{R}\)上的一个\(n\)维线性空间,
\(f\)是\(V\)上的一个双线性函数,
则\(f\)是正定对称的,
当且仅当\(f\)在\(V\)的某个基下的度量矩阵是正定对称的.
%TODO proof
\end{proposition}

\begin{example}
%@see: 《高等代数(第三版 下册)》(丘维声) P174 例1
在\(V = \mathbb{R}^3\)中,
令\(f(\alpha,\beta) \defeq a_1 b_1 + 2 a_2 b_2 + 3 a_3 b_3\),
其中\(\alpha=(\AutoTuple{a}{3})^T,
\beta=(\AutoTuple{b}{3})^T\).
容易验证,\(f\)是\(V\)上的一个内积.
\end{example}

\begin{example}
%@see: 《高等代数(大学高等代数课程创新教材 第二版 下册)》(丘维声) P461 例1
在\(V = \mathbb{R}^n\)中,
令\(f(\alpha,\beta) \defeq a_1 b_1 + a_2 b_2 + \dotsb + a_n b_n\),
其中\(\alpha=(\AutoTuple{a}{n})^T,
\beta=(\AutoTuple{b}{n})^T\).
容易验证,\(f\)是\(V\)上的一个内积.
我们把这个内积称为 \DefineConcept{\(\mathbb{R}^n\)上的标准内积}.
\end{example}

\begin{example}
%@see: 《高等代数(第三版 下册)》(丘维声) P174 例2
在\(V = M_n(\mathbb{R})\)中,
令\(f(A,B) \defeq \tr(AB^T)\).
容易验证,\(f\)是\(V\)上的一个内积.
\end{example}

\begin{example}
%@see: 《高等代数(第三版 下册)》(丘维声) P174 例3
在\(V = C[a,b]\)中,
令\(\rho(f,g) \defeq \int_a^b f(x) g(x) \dd{x}\).
容易验证,\(\rho\)是\(V\)上的一个内积.
\end{example}

\subsection{实内积空间,欧几里得空间}
\begin{definition}
%@see: 《高等代数(第三版 下册)》(丘维声) P174 定义2
%@see: 《Linear Algebra Done Right (Fourth Eidition)》(Sheldon Axler) P184 6.4
设\(V\)是一个实线性空间,
\(\rho\)是\(V\)上的一个内积,
则称“\((V,\rho)\)是一个\DefineConcept{实内积空间}(real inner product space)”.
\end{definition}

\begin{definition}
%@see: 《高等代数(第三版 下册)》(丘维声) P174 定义2
设\(V\)是一个实内积空间.
如果\(V\)是有限维的,
则称“\(V\)是一个\DefineConcept{欧几里得空间}”;
把线性空间\(V\)的维数称为“欧几里得空间\(V\)的\DefineConcept{维数}”.
\end{definition}

\begin{definition}
%@see: 《高等代数(第三版 下册)》(丘维声) P174 定义3
设\((V,\rho)\)是一个实内积空间,
\(\alpha \in V\).
把非负实数\(\sqrt{\rho(\alpha,\alpha)}\)
称为“向量\(\alpha\)的\DefineConcept{长度}”,
记作\(\VectorLengthA{\alpha}\)或\(\VectorLengthN{\alpha}\).
\end{definition}

\begin{property}\label{theorem:实内积空间.向量的长度具有非负性}
%@see: 《高等代数(第三版 下册)》(丘维声) P175
在实内积空间\(V\)中,
零向量的长度为\(0\),
非零向量的长度是正数.
\end{property}

\begin{property}\label{theorem:实内积空间.向量的长度具有齐次性}
%@see: 《高等代数(第三版 下册)》(丘维声) P175
在实内积空间\((V,\rho)\)中,
对于\(\forall \alpha \in V\)
和\(\forall k \in \mathbb{R}\),
有\(\VectorLengthA{k\alpha} = \abs{k} \VectorLengthA{\alpha}\).
\begin{proof}
\(\VectorLengthA{k\alpha}
= \sqrt{\rho(k\alpha,k\alpha)}
= \sqrt{k^2\rho(\alpha,\alpha)}
= \abs{k} \VectorLengthA{\alpha}\).
\end{proof}
\end{property}

\begin{definition}
%@see: 《高等代数(第三版 下册)》(丘维声) P175
设\((V,\rho)\)是实内积空间,
\(\alpha \in V\).
如果\(\VectorLengthA{\alpha} = 1\),
则称“\(\alpha\)是一个\DefineConcept{单位向量}”.
\end{definition}

\begin{property}
%@see: 《高等代数(第三版 下册)》(丘维声) P175
设\((V,\rho)\)是实内积空间,
\(\alpha \in V\).
如果\(\alpha\neq0\),
则\(\frac1{\VectorLengthA{\alpha}} \alpha\)是一个单位向量.
\end{property}
\begin{remark}
将一个非零向量\(\alpha\)变成单位向量\(\frac1{\VectorLengthA{\alpha}} \alpha\)的过程,
称为\DefineConcept{单位化}.
\end{remark}

\begin{theorem}
%@see: 《高等代数(第三版 下册)》(丘维声) P175 定理2(柯西-布尼亚科夫斯基不等式)
在实内积空间\((V,\rho)\)中,
对于\(\forall \alpha,\beta \in V\),
有\begin{equation}
	\abs{\rho(\alpha,\beta)} \leq \VectorLengthA{\alpha} \VectorLengthA{\beta}.
\end{equation}
当且仅当\(\{\alpha,\beta\}\)线性相关时,上式取“\(=\)”号.
%TODO proof
\end{theorem}

\begin{definition}
%@see: 《高等代数(第三版 下册)》(丘维声) P175 定义4
在实内积空间\((V,\rho)\)中,
\(\alpha,\beta\)是两个非零向量.
把\begin{equation}
%@see: 《高等代数(第三版 下册)》(丘维声) P175 (6)
	\arccos\frac{\rho(\alpha,\beta)}{\VectorLengthA{\alpha} \VectorLengthA{\beta}}
\end{equation}
称为“\(\alpha\)与\(\beta\)的\DefineConcept{夹角}”,
记为\(\VectorAngleA{\alpha}{\beta}\)或\(\VectorAngleP{\alpha}{\beta}\).
\end{definition}

\begin{property}
%@see: 《高等代数(第三版 下册)》(丘维声) P175
设\(\alpha,\beta\)是实内积空间\((V,\rho)\)中的两个非零向量,
则\(\alpha\)与\(\beta\)的夹角\(\theta = \VectorAngleA{\alpha}{\beta}\)
满足\(0 \leq \theta \leq \pi\).
%TODO proof
\end{property}

\begin{property}
%@see: 《高等代数(第三版 下册)》(丘维声) P175
设\(\alpha,\beta\)是实内积空间\((V,\rho)\)中的两个非零向量,
则\(\alpha\)与\(\beta\)的夹角\(\theta = \VectorAngleA{\alpha}{\beta}\)
满足\(\theta = \frac\pi2 \iff \rho(\alpha,\beta) = 0\).
%TODO proof
\end{property}

\begin{definition}
%@see: 《高等代数(第三版 下册)》(丘维声) P175 定义5
设\(\alpha,\beta\)是实内积空间\((V,\rho)\)中的两个非零向量.
如果\(\rho(\alpha,\beta) = 0\),
则称“\(\alpha\)与\(\beta\) \DefineConcept{正交}(orthogonal)”,
记为\(\alpha \perp \beta\).
\end{definition}

\begin{corollary}\label{theorem:实内积空间.三角不等式}
%@see: 《高等代数(第三版 下册)》(丘维声) P175 推论3
在实内积空间\((V,\rho)\)中,
\DefineConcept{三角不等式}(triangle inequality)成立,
即对于\(\forall \alpha,\beta \in V\),
有\begin{equation}
%@see: 《高等代数(第三版 下册)》(丘维声) P175 (7)
	\VectorLengthA{\alpha+\beta} \leq \VectorLengthA{\alpha} + \VectorLengthA{\beta}.
\end{equation}
当且仅当\(\alpha = k\beta\)或\(\beta = k\alpha\)(其中\(k\geq0\))时,上式取“\(=\)”号.
%TODO proof
\end{corollary}

\begin{corollary}\label{theorem:实内积空间.勾股定理}
%@see: 《高等代数(第三版 下册)》(丘维声) P176 推论4
%@see: 《Linear Algebra Done Right (Fourth Eidition)》(Sheldon Axler) P187 6.12
在实内积空间\((V,\rho)\)中,
\DefineConcept{勾股定理}成立,
即对于\(\forall \alpha,\beta \in V\),
如果\(\alpha\)与\(\beta\)正交,
则\begin{equation}
%@see: 《高等代数(第三版 下册)》(丘维声) P176 (8)
	\VectorLengthA{\alpha+\beta}^2 = \VectorLengthA{\alpha}^2 + \VectorLengthA{\beta}^2.
\end{equation}
\begin{proof}
假设\(\alpha\)与\(\beta\)正交,
那么\(\rho(\alpha,\beta) = 0\),
从而\(\VectorLengthA{\alpha+\beta}^2
= \rho(\alpha+\beta,\alpha+\beta)
= \rho(\alpha,\alpha) + \rho(\beta,\beta) + \rho(\alpha,\beta) + \rho(\beta,\alpha)
= \VectorLengthA{\alpha}^2 + \VectorLengthA{\beta}^2\).
\end{proof}
\end{corollary}

\begin{definition}
%@see: 《高等代数(第三版 下册)》(丘维声) P176 定义6
在实内积空间\((V,\rho)\)中,
\(\alpha,\beta \in V\).
把\(\VectorLengthA{\alpha-\beta}\)
称为“\(\alpha\)与\(\beta\)的\DefineConcept{距离}”,
记为\(d(\alpha,\beta)\).
\end{definition}

\begin{property}
%@see: 《高等代数(第三版 下册)》(丘维声) P176
在实内积空间\((V,\rho)\)中,
距离\(d\colon V \times V \to \mathbb{R},
(\alpha,\beta) \mapsto \VectorLengthA{\alpha-\beta}\)
满足以下性质:\begin{itemize}
	\item {\rm\bf 对称性}:\begin{equation*}
		(\forall \alpha,\beta \in V)
		[
			d(\alpha,\beta) = d(\beta,\alpha)
		];
	\end{equation*}

	\item {\rm\bf 正定性}:\begin{equation*}
		(\forall \alpha,\beta \in V)
		[
			d(\alpha,\beta) \geq 0
		];
	\end{equation*}
	当且仅当\(\alpha = \beta\)时,上式取“\(=\)”号;

	\item {\rm\bf 三角不等式}:\begin{equation*}
		(\forall \alpha,\beta,\gamma \in V)
		[
			d(\alpha,\gamma) \leq d(\alpha,\beta) + d(\beta,\gamma)
		].
	\end{equation*}
\end{itemize}
\end{property}

\subsection{欧几里得空间中的基}
我们希望在欧几里得空间\(V\)中找出一类基,
使得在这样的基下容易计算\(V\)中任意两个向量的内积,
从而易于计算长度、角度、距离等.

\begin{definition}
%@see: 《高等代数(第三版 下册)》(丘维声) P176
设\((V,\rho)\)是一个欧几里得空间,
\(A\)是\(V\)中的一个向量组,
如果\begin{equation*}
	(\forall \alpha \in A)
	[\alpha\neq0],
	\qquad
	(\forall \alpha,\beta \in A)
	[\alpha \perp \beta],
\end{equation*}
则称“\(A\)是\((V,\rho)\)中的一个\DefineConcept{正交向量组}”.
\end{definition}

\begin{definition}
%@see: 《高等代数(第三版 下册)》(丘维声) P176
设\((V,\rho)\)是一个欧几里得空间,
\(A\)是\(V\)中的一个向量组,
如果\begin{equation*}
	(\forall \alpha \in A)
	[\VectorLengthA{\alpha} = 1],
	\qquad
	(\forall \alpha,\beta \in A)
	[\alpha \perp \beta],
\end{equation*}
则称“\(A\)是\((V,\rho)\)中的一个\DefineConcept{正交单位向量组}”.
\end{definition}

\begin{proposition}
%@see: 《高等代数(第三版 下册)》(丘维声) P176 命题5
在欧几里得空间\((V,\rho)\)中,
任意一个正交向量组一定线性无关.
%TODO proof
\end{proposition}

\begin{definition}
%@see: 《高等代数(第三版 下册)》(丘维声) P176
%@see: 《Linear Algebra Done Right (Fourth Eidition)》(Sheldon Axler) P199 6.27
设\((V,\rho)\)是\(n\)维欧几里得空间,
\(A\)是\(V\)的一个基.
如果\(A\)是正交向量组,
则称“\(A\)是\((V,\rho)\)的一个\DefineConcept{正交基}(orthonormal basis)”.
\end{definition}

\begin{definition}
%@see: 《高等代数(第三版 下册)》(丘维声) P176
设\((V,\rho)\)是\(n\)维欧几里得空间,
\(A\)是\(V\)的一个基.
如果\(A\)是正交单位向量组,
则称“\(A\)是\((V,\rho)\)的一个\DefineConcept{标准正交基}”
或“\(A\)是\((V,\rho)\)的一个\DefineConcept{规范正交基}”.
\end{definition}

\begin{theorem}\label{theorem:欧几里得空间.欧几里得空间中向量组的施密特正交化}
%@see: 《高等代数(第三版 下册)》(丘维声) P176 定理6
%@see: 《Linear Algebra Done Right (Fourth Eidition)》(Sheldon Axler) P201 6.32
设\((V,\rho)\)是欧几里得空间,
\(\AutoTuple{\alpha}{s}\)是\(V\)中一个向量组,
令\begin{equation*}
%@see: 《高等代数(第三版 下册)》(丘维声) P176 (10)
	\beta_i
	= \alpha_i
		- \sum_{j=1}^{s-1} \frac{\rho(\alpha_i,\beta_j)}{\rho(\beta_j,\beta_j)} \beta_j,
	\quad i=1,2,\dotsc,s,
\end{equation*}
则\(\AutoTuple{\beta}{s}\)是正交向量组,
并且\(\AutoTuple{\alpha}{s}\)与\(\AutoTuple{\beta}{s}\)等价.
\end{theorem}
\begin{remark}
\cref{theorem:欧几里得空间.欧几里得空间中向量组的施密特正交化} 中,
把线性无关向量组\(\AutoTuple{\alpha}{s}\)
变成与它等价的正交向量组\(\AutoTuple{\beta}{s}\)的过程,
称为\DefineConcept{施密特正交化}.
只要再将\(\beta_i\)单位化,
就可以得到一个与\(\AutoTuple{\alpha}{s}\)等价的正交单位向量组\(\AutoTuple{\eta}{s}\).
因此,给定\(n\)维欧几里得空间\(V\)中一个基\(\AutoTuple{\alpha}{n}\),
经过施密特正交化、单位化,就可以得到\(V\)的一个标准正交基\(\AutoTuple{\eta}{n}\).
\end{remark}

在\(n\)维欧几里得空间\(V\)中,取一个基\(\AutoTuple{\alpha}{n}\),
经过施密特正交化把它变成正交基\(\AutoTuple{\beta}{n}\),
再经过单位化把它变成标准正交基\(\AutoTuple{\gamma}{n}\).
这就说明:
\begin{theorem}
%@see: 《Linear Algebra Done Right (Fourth Eidition)》(Sheldon Axler) P202 6.35
欧几里得空间中,一定存在标准正交基.
%TODO proof
\end{theorem}

\begin{proposition}
%@see: 《Linear Algebra Done Right (Fourth Eidition)》(Sheldon Axler) P203 6.36
欧几里得空间\(V\)中的任意一个正交单位向量组
均可以扩充成\(V\)的一个标准正交基.
%TODO proof
\end{proposition}

\subsection{欧几里得空间中向量的内积、傅里叶展开}
\begin{proposition}
%@see: 《高等代数(第三版 下册)》(丘维声) P177
设\((V,\rho)\)是\(n\)维欧几里得空间,
则向量组\(\AutoTuple{\eta}{n}\)是\((V,\rho)\)的一个标准正交基,
当且仅当\begin{equation*}
%@see: 《高等代数(第三版 下册)》(丘维声) P177 (11)
	\rho(\eta_i,\eta_j)
	= \delta(i,j),
	\quad i,j=1,2,\dotsc,n,
\end{equation*}
其中\(\delta\)是克罗内克\(\delta\)函数.
\end{proposition}

利用标准正交基,容易计算向量的内积.

设\((V,\rho)\)是\(n\)维欧几里得空间,
向量\(\alpha,\beta \in V\),
向量组\(\AutoTuple{\eta}{n}\)是\((V,\rho)\)的一个标准正交基,
\(\alpha,\beta\)在基\(\AutoTuple{\eta}{n}\)下的坐标
分别是\(X=(\AutoTuple{x}{n})^T,
Y=(\AutoTuple{y}{n})^T\),
则\begin{equation*}
%@see: 《高等代数(第三版 下册)》(丘维声) P177 (12)
	\rho(\alpha,\beta)
	= \rho\left( \sum_{i=1}^n x_i \eta_i, \sum_{j=1}^n y_j \eta_j \right)
	= \sum_{i=1}^n \sum_{j=1}^n x_i y_j \rho(\eta_i,\eta_j)
	= \sum_{i=1}^n x_i y_i
	= X^T Y.
\end{equation*}

利用标准正交基,可以借助内积,表达向量的坐标分量.

设\(\alpha\)在标准正交基\(\AutoTuple{\eta}{n}\)下的坐标为\(X=(\AutoTuple{x}{n})^T\),
则\begin{equation*}
	\alpha = \sum_{i=1}^n x_i \eta_i;
\end{equation*}
等号两边用\(\eta_j\)作内积,得\begin{equation*}
	\rho(\alpha,\eta_j)
	= \rho\left( \sum_{i=1}^n x_i \eta_i, \eta_j \right)
	= \sum_{i=1}^n x_i \rho(\eta_i,\eta_j)
	= x_j,
\end{equation*}
因此\begin{equation}\label{equation:欧几里得空间.向量的傅里叶展开}
%@see: 《高等代数(第三版 下册)》(丘维声) P177 (13)
	\alpha = \sum_{i=1}^n \rho(\alpha,\eta_i) \eta_i.
\end{equation}
我们把\cref{equation:欧几里得空间.向量的傅里叶展开}
称为“向量\(\alpha\)的\DefineConcept{傅里叶展开}”,
其中系数\(\rho(\alpha,\eta_i)\ (i=1,2,\dotsc,n)\)
称为“向量\(\alpha\)的\DefineConcept{傅里叶系数}”.

\begin{proposition}
%@see: 《高等代数(第三版 下册)》(丘维声) P177 命题7
在欧几里得空间\((V,\rho)\)中,
一个标准正交基到另一个标准正交基的过渡矩阵
一定是正交矩阵.
%TODO proof
\end{proposition}

\begin{proposition}
%@see: 《高等代数(第三版 下册)》(丘维声) P177 命题8
设\((V,\rho)\)是\(n\)维欧几里得空间,
向量组\(\AutoTuple{\eta}{n}\)是\((V,\rho)\)的一个标准正交基,
\(\AutoTuple{\beta}{n}\)是\(V\)中一个向量组.
如果存在正交矩阵\(P \in M_n(\mathbb{R})\),
使得\begin{equation*}
	(\AutoTuple{\beta}{n})
	= (\AutoTuple{\eta}{n}) P,
\end{equation*}
那么\(\AutoTuple{\beta}{n}\)是\(V\)中一个标准正交基.
%TODO proof
\end{proposition}

\begin{definition}
%@see: 《高等代数(第三版 下册)》(丘维声) P180 习题10.2 10.
设\(\AutoTuple{\alpha}{m}\)是\(n\)维欧几里得空间\((V,\rho)\)中一个向量组,
把\begin{equation}
	A \defeq \begin{bmatrix}
		\rho(\alpha_1,\alpha_1) & \rho(\alpha_1,\alpha_2) & \dots & \rho(\alpha_1,\alpha_m) \\
		\rho(\alpha_2,\alpha_1) & \rho(\alpha_2,\alpha_2) & \dots & \rho(\alpha_2,\alpha_m) \\
		\vdots & \vdots & & \vdots \\
		\rho(\alpha_m,\alpha_1) & \rho(\alpha_m,\alpha_2) & \dots & \rho(\alpha_m,\alpha_m) \\
	\end{bmatrix},
\end{equation}
称为“向量组\(\AutoTuple{\alpha}{m}\)的\DefineConcept{格拉姆矩阵}”,
记为\(G(\AutoTuple{\alpha}{m})\).
把\(A\)的行列式\(\abs{A}\)
称为“向量组\(\AutoTuple{\alpha}{m}\)的\DefineConcept{格拉姆行列式}”.
\end{definition}

\begin{example}
%@see: 《高等代数(第三版 下册)》(丘维声) P180 习题10.2 10.
设\(\AutoTuple{\alpha}{m}\)是\(n\)维欧几里得空间\((V,\rho)\)中一个向量组.
证明:\begin{equation*}
	\abs{G(\AutoTuple{\alpha}{m})} \geq 0.
\end{equation*}
当且仅当\(\AutoTuple{\alpha}{m}\)线性相关时,上式取“\(=\)”号.
\end{example}

\subsection{实内积空间之间的同构}
对于同一个实线性空间\(V\),
当指定不同的内积时,
\(V\)便成为不同的实内积空间,
这些实内积空间之间有什么关系?
不同的实线性空间,
各自指定了一个内积,成为实内积空间后,
它们之间又有什么关系?

\begin{definition}
%@see: 《高等代数(第三版 下册)》(丘维声) P178 定义7
设\((V_1,\rho_1),(V_2,\rho_2)\)都是实内积空间.
如果存在从\(V_1\)到\(V_2\)的一个双射\(\sigma\),
使得\begin{gather*}
	(\forall \alpha,\beta \in V_1)
	[
		\sigma(\alpha+\beta)
		= \sigma(\alpha) + \sigma(\beta)
	], \\
	(\forall \alpha \in V_1)
	(\forall k \in \mathbb{R})
	[
		\sigma(k\alpha)
		= k \sigma(\alpha)
	], \\
	(\forall \alpha,\beta \in V_1)
	[
		\rho_2(\sigma(\alpha),\sigma(\beta))
		= \rho_1(\alpha,\beta)
	],
\end{gather*}
则称“\(\sigma\)是从\(V_1\)到\(V_2\)的一个\DefineConcept{同构}”;
并称“\(V_1\)与\(V_2\) \DefineConcept{同构}”,
记为\(V_1 \Isomorphism V_2\).
\end{definition}
\begin{remark}
从上述定义可以看出,
实内积空间之间的一个同构\(\sigma\)
首先是实线性空间之间的一个同构,
其次\(\sigma\)还保持内积,
因此\(\sigma\)既具有线性空间的同构的性质,还具有与内积有关的性质.
\end{remark}

\begin{proposition}\label{theorem:欧几里得空间.实内积空间之间的同构的等价定义}
%@see: 《高等代数(第三版 下册)》(丘维声) P179
设\((V_1,\rho_1),(V_2,\rho_2)\)都是实内积空间.
如果存在从\(V_1\)到\(V_2\)的一个满射\(\sigma\),
使得\begin{gather*}
	(\forall \alpha,\beta \in V_1)
	[
		\sigma(\alpha+\beta)
		= \sigma(\alpha) + \sigma(\beta)
	], \\
	(\forall \alpha \in V_1)
	(\forall k \in \mathbb{R})
	[
		\sigma(k\alpha)
		= k \sigma(\alpha)
	], \\
	(\forall \alpha,\beta \in V_1)
	[
		\rho_2(\sigma(\alpha),\sigma(\beta))
		= \rho_1(\alpha,\beta)
	],
\end{gather*}
则\(\sigma\)是从\(V_1\)到\(V_2\)的一个同构.
\end{proposition}

\begin{property}
%@see: 《高等代数(第三版 下册)》(丘维声) P178
设\((V_1,\rho_1),(V_2,\rho_2)\)都是实内积空间,
\(\sigma\)是从\(V_1\)到\(V_2\)的一个同构,
\(\AutoTuple{\alpha}{n}\)是\(V_1\)的一个标准正交基,
则\(\sigma(\alpha_1),\dotsc,\sigma(\alpha_n)\)是\(V_2\)的一个标准正交基.
% \begin{proof}
% 因为\(\AutoTuple{\alpha}{n}\)是\(V_1\)的一个标准正交基,
% 所以\begin{equation*}
% 	\rho_1(\alpha_i,\alpha_j)
% 	= \delta(i,j),
% 	\quad i,j=1,2,\dotsc,n,
% \end{equation*}
% 其中\(\delta\)是克罗内克\(\delta\)函数;
% 从而有\begin{equation*}
% 	\rho_2(\sigma(\alpha_i),\sigma(\alpha_j))
% 	= \rho_1(\alpha_i,\alpha_j)
% 	= \delta(i,j),
% 	\quad i,j=1,2,\dotsc,n,
% \end{equation*}
% 说明\(\sigma(\alpha_1),\dotsc,\sigma(\alpha_n)\)是\(V_2\)的一个标准正交基.
% \end{proof}
\end{property}

\begin{theorem}\label{theorem:欧几里得空间.两个欧几里得空间同构的充分必要条件}
%@see: 《高等代数(第三版 下册)》(丘维声) P178 定理9
两个欧几里得空间同构的充分必要条件是它们的维数相同.
%TODO proof
\end{theorem}
\begin{remark}
从\cref{theorem:欧几里得空间.两个欧几里得空间同构的充分必要条件} 得出:
任意一个\(n\)维欧几里得空间\(V\)
都与装备了标准内积的欧几里得空间\(\mathbb{R}^n\)同构,
并且\begin{equation*}
	\sigma\colon V \to \mathbb{R}^n,
	\alpha = \sum_{i=1}^n x_i \eta_i \mapsto (\AutoTuple{x}{n})^T
\end{equation*}
就是从\(V\)到\(\mathbb{R}^n\)的一个同构,
其中\(\AutoTuple{\eta}{n}\)是\(V\)的一个标准正交基.
\end{remark}

\begin{property}
%@see: 《高等代数(第三版 下册)》(丘维声) P179
%@see: 《高等代数(第三版 下册)》(丘维声) P181 习题10.2 13.
实内积空间之间的同构关系,具有反身性、对称性、传递性,是一个等价关系.
\end{property}

\section{正交补,正交投影}
\subsection{正交补}
\begin{definition}\label{definition:正交补.利用内积构造的正交补}
%@see: 《高等代数(第三版 下册)》(丘维声) P181 定义1
设\((V,\rho)\)是实内积空间,
\(W\)是\(V\)的一个非空子集.
定义:\begin{equation*}
	W^\perp
	\defeq
	\Set{
		\alpha \in V
		\given
		(\forall \beta \in W)
		[\rho(\alpha,\beta) = 0]
	},
\end{equation*}
称之为“\(W\)的\DefineConcept{正交补}(the \emph{orthogonal complement} of \(W\))”.
\end{definition}
\begin{remark}
\cref{definition:正交补.利用内积构造的正交补} 中的“正交补”
与\cref{definition:双线性函数.利用双线性函数构造的正交补} 中的“正交补”稍微不同,
前者是后者的一个特例.
\end{remark}

\begin{property}
%@see: 《高等代数(第三版 下册)》(丘维声) P181
设\((V,\rho)\)是实内积空间,
\(W\)是\(V\)的一个非空子集,
则\(W\)的正交补\(W^\perp\)是\(V\)的一个子空间.
\begin{proof}
由\cref{theorem:双线性函数.双线性函数取值为零的条件2} 可知\(0 \in W^\perp\).
任取\(\alpha,\beta \in W^\perp\),
任取\(\gamma \in W\),
任取\(k \in \mathbb{R}\),
则\begin{gather*}
	\rho(\alpha+\beta,\gamma)
	= \rho(\alpha,\gamma) + \rho(\beta,\gamma)
	= 0 + 0 = 0, \\
	\rho(k\alpha,\gamma)
	= k\rho(\alpha,\gamma)
	= k0 = 0,
\end{gather*}
即\(W^\perp\)对加法、纯量乘法封闭.
因此\(W^\perp\)是\(V\)的子空间.
\end{proof}
\end{property}

\begin{property}\label{theorem:正交补.利用内积构造的正交补.正交补的对合律}
%@see: 《高等代数(第三版 下册)》(丘维声) P184 习题10.3 3.
%@see: 《高等代数(大学高等代数课程创新教材 第二版 下册)》(丘维声) P481 例2
设\((V,\rho)\)是\(n\)维欧几里得空间,
\(W\)是\(V\)的一个子空间,
则\(W\)的正交补\(W^\perp\)的正交补\((W^\perp)^\perp\)就是\(W\),
即\((W^\perp)^\perp = W\).
\begin{proof}
由于欧几里得空间的内积\(\rho\)是正定的对称双线性函数,
因此\(\rho\)是非退化的对称双线性函数,
由\cref{theorem:双线性函数.利用双线性函数构造的正交补.正交补的对合律} 可知
\((W^\perp)^\perp = W\).
\end{proof}
\end{property}

\begin{property}
%@see: 《高等代数(大学高等代数课程创新教材 第二版 下册)》(丘维声) P481 例3
设\(W_1,W_2\)是\(n\)维欧几里得空间\(V\)的两个子空间,
则\begin{gather}
%@see: 《高等代数(大学高等代数课程创新教材 第二版 下册)》(丘维声) P481 (14)
	(W_1 + W_2)^\perp
	= W_1^\perp \cap W_2^\perp, \\
	(W_1 \cap W_2)^\perp
	= W_1^\perp + W_2^\perp.
\end{gather}
%TODO proof
\end{property}

\begin{theorem}\label{theorem:正交补.实内积空间的正交直和分解}
%@see: 《高等代数(第三版 下册)》(丘维声) P181 定理1
设\(U\)是实内积空间\((V,\rho)\)的一个有限维子空间,
则\begin{equation*}
%@see: 《高等代数(第三版 下册)》(丘维声) P181 (2)
	V = U \DirectSum U^\perp.
\end{equation*}
\begin{proof}
先证\(V = U \DirectSum U^\perp\).
在\(U\)中取一个标准正交基\(\AutoTuple{\epsilon}{n}\).
任取\(\alpha \in V\),
令\begin{align*}
%@see: 《高等代数(第三版 下册)》(丘维声) P181 (3)
	\alpha_1
	&\defeq
	\rho(\alpha,\epsilon_1) \epsilon_1 + \dotsb + \rho(\alpha,\epsilon_m) \epsilon_m, \\
%@see: 《高等代数(第三版 下册)》(丘维声) P182 (4)
	\alpha_2
	&\defeq
	\alpha - \alpha_1.
\end{align*}
显然\(\alpha_1 \in U\)而\(\alpha_2 \in V\).
因为\begin{align*}
	\rho(\alpha_2,\epsilon_j)
	= \rho(\alpha-\alpha_1,\epsilon_j)
	= \rho(\alpha,\epsilon_j)
	- \rho(\alpha_1,\epsilon_j),
\end{align*}
其中\begin{align*}
	\rho(\alpha_1,\epsilon_j)
	&= \rho\left(
			\rho(\alpha,\epsilon_1) \epsilon_1 + \dotsb + \rho(\alpha,\epsilon_m) \epsilon_m,
			\epsilon_j
		\right) \\
	&= \rho(\alpha,\epsilon_1) \rho(\epsilon_1,\epsilon_j)
		+ \dotsb + \rho(\alpha,\epsilon_m) \rho(\epsilon_m,\epsilon_j) \\
	&= \rho(\alpha,\epsilon_j),
\end{align*}
所以\begin{equation*}
	\rho(\alpha_2,\epsilon_j)
	= \rho(\alpha,\epsilon_j) - \rho(\alpha,\epsilon_j)
	= 0,
\end{equation*}
这就说明\(\alpha_2 \in U^\perp\).
既然\(\alpha = \alpha_1 + \alpha_2\),
那么\(V = U + U^\perp\).

再证\(U \cap U^\perp = 0\).
假设\(\gamma \in U \cap U^\perp\),
则\(\gamma \in U\)且\(\gamma \in U^\perp\),
从而\(\rho(\gamma,\gamma) = 0\),
于是\(\gamma = 0\).

综上所述,\(V = U \DirectSum U^\perp\).
\end{proof}
\end{theorem}
\begin{remark}
从\cref{theorem:正交补.实内积空间的正交直和分解} 得到,
对于欧几里得空间\(V\)的任意一个非平凡子空间\(U\),
都有\(V = U \DirectSum U^\perp\),
这就说明\(U\)的一个标准正交基与\(U^\perp\)的一个标准正交基合起来就是\(V\)的一个标准正交基.
\end{remark}

\begin{example}
%@see: 《高等代数(第三版 下册)》(丘维声) P184 习题10.3 4.
证明:欧几里得空间\(\mathbb{R}^n\)(指定标准内积)的
任意一个子空间\(U\)是某一个齐次线性方程组的解空间.
%TODO proof
\end{example}

\subsection{正交投影的概念}
如果\(U\)是实内积空间\(V\)的一个子空间,
且\(V = U \DirectSum U^\perp\),
那么\(V\)中每个向量\(\alpha\)能唯一地表示成\begin{equation*}
%@see: 《高等代数(第三版 下册)》(丘维声) P182 (5)
	\alpha = \alpha_1 + \alpha_2,
	\quad \alpha_1 \in U, \alpha_2 \in U^\perp,
\end{equation*}
我们可以藉此构造一个\(V\)上的线性变换\begin{equation*}
	\vb{P}_U\colon V \to U,
	\alpha \mapsto \alpha_1,
\end{equation*}
并称之为“\(V\)在\(U\)上的\DefineConcept{正交投影}”,	% 这里是指线性变换
把\(\alpha_1\)称为“\(\alpha\)在\(U\)上的\DefineConcept{正交投影}”.	% 这里是指元素

这里可以看出:\(\beta_1 \in U\)是\(\beta\)在\(U\)上的正交投影,
当且仅当\(\beta - \beta_1 \in U^\perp\).

由\cref{theorem:正交补.实内积空间的正交直和分解} 可知,
只要\(U\)是有限维的,那么必定存在\(V\)在\(U\)上的正交投影.

\begin{example}
%@see: 《高等代数(第三版 下册)》(丘维声) P184 习题10.3 5.
设\(U\)是实内积空间\((V,\rho)\)的一个\(m\)维子空间,
在\(U\)中取一个标准正交基\(\AutoTuple{\eta}{m}\).
证明:向量\(\alpha \in V\)在\(U\)上的正交投影为\begin{equation*}
	\alpha_1 = \sum_{i=1}^m \rho(\alpha,\eta_i) \eta_i.
\end{equation*}
%TODO proof
\end{example}

\subsection{正交投影的性质}
向量\(\alpha\)在\(U\)上的正交投影\(\alpha_1\)具有什么性质呢?
从几何学中“垂线段最短”可以得到启发,猜测以下结论:
\begin{theorem}\label{theorem:正交补.垂线段最短}
%@see: 《高等代数(第三版 下册)》(丘维声) P182 定理2
设\(U\)是实内积空间\(V\)的一个有限维子空间,
向量\(\alpha \in V\),向量\(\alpha_1 \in U\),
则\(\alpha_1\)是\(\alpha\)在\(U\)上的正交投影,
当且仅当\begin{equation*}
%@see: 《高等代数(第三版 下册)》(丘维声) P182 (6)
	(\forall \gamma \in U)
	[d(\alpha,\alpha_1) \leq d(\alpha,\gamma)].
\end{equation*}
\begin{proof}
必要性.
设\(\alpha_1 \in U\)是\(\alpha\)在\(U\)上的正交投影,
则\(\alpha - \alpha_1 \in U^\perp\),
那么对于\(\forall \gamma \in U\),
有\begin{equation*}
	(\alpha - \alpha_1) \perp (\alpha_1 - \gamma);
\end{equation*}
由勾股定理有\begin{equation*}
	\VectorLengthA{\alpha - \alpha_1}^2
	+ \VectorLengthA{\alpha_1 - \gamma}^2
	= \VectorLengthA{(\alpha - \alpha_1) + (\alpha_1 - \gamma)}^2
	= \VectorLengthA{\alpha - \gamma}^2;
\end{equation*}
由此得出\(\VectorLengthA{\alpha - \alpha_1} \leq \VectorLengthA{\alpha - \gamma}\),
即\(d(\alpha,\alpha_1) \leq d(\alpha,\gamma)\).

充分性.
假设成立\((\forall \gamma \in U)[d(\alpha,\alpha_1) \leq d(\alpha,\gamma)]\).
同时假设向量\(\delta\)是\(\alpha\)在\(U\)上的正交投影,
从而必有\(\delta \in U\),
那么\(d(\alpha,\alpha_1) \leq d(\alpha,\delta)\).
再根据上述必要性证得的结论可知\(d(\alpha,\delta) \leq d(\alpha,\alpha_1)\).
于是由\(d(\alpha,\delta) \leq d(\alpha,\alpha_1) \leq d(\alpha,\delta)\)
有\(d(\alpha,\delta) = d(\alpha,\alpha_1)\).
由于\((\alpha-\delta) \in U^\perp,
(\delta-\alpha_1) \in U\),
所以\begin{equation*}
	\VectorLengthA{\alpha - \alpha_1}^2
	= \VectorLengthA{(\alpha - \delta) + (\delta - \alpha_1)}^2
	= \VectorLengthA{\alpha - \delta}^2 + \VectorLengthA{\delta - \alpha_1}^2.
\end{equation*}
由此得出\(\VectorLengthA{\delta - \alpha_1}^2 = 0\),
即\(\delta = \alpha_1\).
\end{proof}
\end{theorem}

\begin{example}
%@see: 《高等代数(第三版 下册)》(丘维声) P184 习题10.3 6.
设\(U\)是实内积空间\((V,\rho)\)的一个有限维子空间.
证明:\(V\)在\(U\)上的正交投影\(\vb{P}\)满足\begin{equation*}
	(\forall \alpha,\beta \in V)
	[
		\rho(\vb{P}\alpha,\beta)
		= \rho(\alpha,\vb{P}\beta)
	].
\end{equation*}
%TODO proof
\end{example}

\begin{example}
%@see: 《高等代数(第三版 下册)》(丘维声) P184 习题10.3 8.
%@see: 《Linear Algebra Done Right (Fourth Edition)》(Sheldon Axler) P198 6.26
设\(\AutoTuple{\epsilon}{m}\)是实内积空间\((V,\rho)\)中一个正交单位向量组.
证明:对于任意\(\alpha \in V\),
有\begin{equation}\label{equation:正交补.贝塞尔不等式}
	\sum_{i=1}^m \rho^2(\alpha,\epsilon_i)
	\leq \VectorLengthA{\alpha}^2.
\end{equation}
当且仅当\(\alpha = \sum_{i=1}^m \rho(\alpha,\epsilon_i) \epsilon_i\)时,上式取“\(=\)”号.
\begin{proof}
假设\(\alpha \in V\),
对它进行正交分解,得\(\alpha = \beta + \gamma\),
其中\(\beta = \sum_{i=1}^m \rho(\alpha,\epsilon_i) \epsilon_i,
\gamma = \alpha - \beta\).
注意到\(\rho(\gamma,\epsilon_k)
= \rho(\alpha,\epsilon_k) - \rho(\beta,\epsilon_k)
= \rho(\alpha,\epsilon_k) - \rho(\alpha,\epsilon_k) \rho(\epsilon_k,\epsilon_k)
= 0\),
从而有\(\rho(\gamma,\beta) = 0\),
于是由\hyperref[theorem:实内积空间.勾股定理]{勾股定理}可知
\(\VectorLengthA{\alpha}^2
= \VectorLengthA{\beta}^2 + \VectorLengthA{\gamma}^2
\geq \VectorLengthA{\beta}^2
= \sum_{i=1}^m \rho^2(\alpha,\epsilon_i)\).
\end{proof}
\end{example}
\begin{remark}
我们把\cref{equation:正交补.贝塞尔不等式} 称为\DefineConcept{贝塞尔不等式}(Bessel's inequality).
\end{remark}

\subsection{最佳逼近元}
从\cref{theorem:正交补.垂线段最短} 可以引出下述概念:
\begin{definition}
%@see: 《高等代数(第三版 下册)》(丘维声) P183 定义2
设\(U\)是实内积空间\(V\)的一个子空间,\(\alpha \in V\).
如果存在\(\delta \in U\),
使得对于任意\(\gamma \in U\),
都有\(d(\alpha,\delta) \leq d(\alpha,\gamma)\),
那么称“\(\delta\)是\(\alpha\)在\(U\)上的\DefineConcept{最佳逼近元}”.
\end{definition}

从\cref{theorem:正交补.垂线段最短} 立即可以得出:
\begin{proposition}
%@see: 《高等代数(第三版 下册)》(丘维声) P183
如果\(U\)是实内积空间\(V\)的有限维子空间,
那么\(V\)中任意一个向量\(\alpha\)在\(U\)上的最佳逼近元存在且唯一,
它就是\(\alpha\)在\(U\)上的正交投影.
\end{proposition}

\subsection{正交投影的应用 --- 最小二乘法}
%@see: 《高等代数(第三版 下册)》(丘维声) P183
%@see: 《高等代数》(丁南庆、刘公祥、纪庆忠、郭学军) P338 定义8.5.3
实际问题中,从观测数据列出的线性方程组\(\vb{A} \vb{x} = \vb\beta\)可能无解,
其中\(
	\vb{A} \in M_{s \times n}(\mathbb{R}),
	\vb\beta \in \mathbb{R}^s
\).
当\(\vb{A} \vb{x} = \vb\beta\)无解时,
我们还是想要找出一个向量\(\vb{x} \in \mathbb{R}^n\),
使得\(\vb{A} \vb{x}\)充分接近\(\vb\beta\),
即函数\begin{equation*}
	f(\vb{x})
	\defeq
	\VectorLengthA{\vb{A} \vb{x} - \vb\beta}^2
	=
	(\vb{A} \vb{x} - \vb\beta)^T (\vb{A} \vb{x} - \vb\beta)
\end{equation*}
取最小值,
我们把使\(f(\vb{x})\)取得最小值的向量\(\vb{x}\)
称为“线性方程组\(\vb{A} \vb{x} = \vb\beta\)的\DefineConcept{最小二乘解}(least squares solution)”.
将最小二乘解作为\(\vb{A} \vb{x} = \vb\beta\)的近似解的方法,
称为\DefineConcept{最小二乘法}(method of least squares).

容易看出:\begin{align*}
	&\text{$\vb\eta$是$\vb{A} \vb{x} = \vb\beta$的最小二乘解} \\
	&\iff
	(\forall \vb{x} \in \mathbb{R}^n)
	[f(\vb\eta) \leq f(\vb{x})] \\
	&\iff
	(\forall \vb\gamma \in \Im\vb{A})
	[\VectorLengthA{\vb{A} \vb\eta - \vb\beta} \leq \VectorLengthA{\vb\gamma - \vb\beta}] \\
	&\iff
	(\forall \vb\gamma \in \Im\vb{A})
	[d(\vb{A} \vb\eta,\vb\beta) \leq d(\vb\gamma,\vb\beta)] \\
	&\iff
	\text{$\vb{A} \vb\eta$是$\vb\beta$在$\Im\vb{A}$上的正交投影} \\
	&\iff
	\vb{A} \vb\eta - \vb\beta \in (\Im\vb{A})^\perp \\
	&\iff
	\vb{A} \vb\eta - \vb\beta \in \Ker\vb{A}^T \\
	&\iff
	\vb{A}^T (\vb{A} \vb\eta - \vb\beta) = \vb0 \\
	&\iff
	\vb{A}^T \vb{A} \vb\eta = \vb{A}^T \vb\beta \\
	&\iff
	\text{$\vb\eta$是$\vb{A}^T \vb{A} \vb{x} = \vb{A}^T \vb\beta$的解}.
\end{align*}

因为 \hyperref[example:线性方程组有解的充分必要条件.最小二乘解的存在性]{$\rank(\vb{A}^T \vb{A}, \vb{A}^T \vb\beta) = \rank(\vb{A}^T \vb{A})$},
所以线性方程组\(\vb{A}^T \vb{A} \vb{x} = \vb{A}^T \vb\beta\)有解.
下面来具体求出该方程组的解.

\begingroup  % 线性方程组\(\vb{A}^T \vb{A} \vb{x} = \vb{A}^T \vb\beta\)的解
%@see: 《高等代数》(丁南庆、刘公祥、纪庆忠、郭学军) P338 定理8.5.4
%\cref{theorem:线性方程组.齐次线性方程组的解的结构定理.推论1}
设\(\vb{A}^+\)是\(\vb{A}\)的穆尔--彭罗斯广义逆.
首先有\begin{equation*}
	\vb{A}^T \vb{A} (\vb{A}^+ \vb\beta)
	% 矩阵乘法的结合律
	= \vb{A}^T (\vb{A} \vb{A}^+) \vb\beta
	% \cref{theorem:线性方程组.广义逆的性质1}
	= \vb{A}^T (\vb{A} \vb{A}^+)^H \vb\beta
	% 矩阵乘积的转置
	= (\vb{A} \vb{A}^+ \vb{A})^T \vb\beta
	% \cref{theorem:线性方程组.广义逆的性质1}
	= \vb{A}^T \vb\beta,
\end{equation*}
这就是说\(\vb{A}^+ \vb\beta\)是\(\vb{A}^T \vb{A} \vb{x} = \vb{A}^T \vb\beta\)的一个特解.
其次,对于任意\(\vb{Z} \in \mathbb{R}^n\),有\begin{equation*}
	\vb{A}^T \vb{A} (\vb{E}_n - \vb{A}^+ \vb{A}) \vb{Z}
	= (\vb{A}^T \vb{A} - \vb{A}^T \vb{A} \vb{A}^+ \vb{A}) \vb{Z}
	= (\vb{A}^T \vb{A} - \vb{A}^T \vb{A}) \vb{Z}
	= \vb0,
\end{equation*}
其中\(\vb{E}_n\)是\(n\)阶单位矩阵,
这就说明\(
	\Im(\vb{E}_n - \vb{A}^+ \vb{A})
	\subseteq
	\Ker(\vb{A}^T \vb{A})
\);
再由\(\vb{A} = \vb{A} \vb{A}^+ \vb{A}\)
可知\(
	\vb{A}^+ \vb{A}
	= (\vb{A}^+ \vb{A})^2
\),
说明\(\vb{A}^+ \vb{A}\)是一个幂等矩阵,
并且由\cref{example:矩阵乘积的秩.矩阵相乘不变秩的特例1} 可知\(\rank(\vb{A}^+ \vb{A}) = \rank\vb{A}\),
所以\begin{align*}
	\dim\Im(\vb{E}_n - \vb{A}^+ \vb{A})
	&= \rank(\vb{E}_n - \vb{A}^+ \vb{A})
	% \cref{example:幂等矩阵.幂等矩阵的秩的性质1}
	% \(\rank\vb{A} + \rank(\vb{E} - \vb{A}) = n\)
	% \(\rank(\vb{A}^+ \vb{A}) + \rank(\vb{E}_n - \vb{A}^+ \vb{A}) = n\)
	= n - \rank(\vb{A}^+ \vb{A}) \\
	% \(\rank(\vb{A}^+ \vb{A}) = \rank\vb{A}\)
	&= n - \rank\vb{A}
	% \cref{equation:矩阵乘积的秩.实矩阵及其转置矩阵的乘积的秩}
	% \(\rank\vb{A} = \rank(\vb{A}^T \vb{A})\)
	= n - \rank(\vb{A}^T \vb{A})
	% \cref{theorem:线性方程组.齐次线性方程组的解向量个数}
	= \dim\Ker(\vb{A}^T \vb{A});
\end{align*}
% \cref{theorem:向量空间.两个非零子空间的关系2}
因此\(
	\Im(\vb{E}_n - \vb{A}^+ \vb{A})
	= \Ker(\vb{A}^T \vb{A})
\).
综上所述,\(\vb{A}^T \vb{A} \vb{x} = \vb{A}^T \vb\beta\)的解集是\begin{equation*}
	\Set*{
		\vb{A}^+ \vb\beta
		+ (\vb{E}_n - \vb{A}^+ \vb{A}) \vb{Z}
		\given
		\vb{Z} \in \mathbb{R}^n
	}.
\end{equation*}
\endgroup
容易看出,最小二乘解唯一,当且仅当\(\vb{A}^+ \vb{A} = \vb{E}_n\).
%TODO proof 尚未证明“最小二乘解唯一,当且仅当\(\vb{A}^+ \vb{A} = \vb{E}_n\)”

特别地,当\(\vb{A}\)是列满秩矩阵(即\(\rank\vb{A} = n\))
或\(\vb{A}^T \vb{A}\)是正定矩阵时,
\(\vb{A}^T \vb{A}\)可逆,
于是方程\(\vb{A}^T \vb{A} \vb{x} = \vb{A}^T \vb\beta\)有一个形式更简单的解:\begin{equation*}
	\{
		(\vb{A}^T \vb{A})^{-1} \vb{A}^T \vb\beta
	\}.
\end{equation*}
% 涉及正定矩阵的数值计算非常简便.
% 在计算时,无须交换行,也无须担心主元过小.
% 因此在此特别提及当\(\vb{A}^T \vb{A}\)是正定矩阵时的计算.

\section{正交变换}

\section{对称变换}

\section{酉空间}
%@see: 《高等代数(第三版 下册)》(丘维声) P193
在本节,我们研究在复线性空间中引进内积的概念,使之成为复内积空间.

\subsection{酉空间}
如何在复线性空间\(V\)中引进内积的概念呢?
如果我们照搬实线性空间内积的概念,
考虑复线性空间\(V\)上一个双线性函数\(f\),
那么对于\(V\)中任意一个非零向量\(\alpha\),
有\begin{equation*}
	f(\iu\alpha,\iu\alpha)
	= \iu^2 f(\alpha,\alpha)
	= -f(\alpha,\alpha).
\end{equation*}
若要求\((\forall \alpha \in V)[f(\alpha,\alpha) \in \mathbb{R}]\),
则\(f\)不满足正定性(因为当\(f(\alpha,\alpha) > 0\)时,必有\(f(\iu\alpha,\iu\alpha) < 0\)).
为了使复线性空间\(V\)上的内积仍具有正定性,
就不能要求它是双线性函数,
而只要求它对第一个自变量是线性的.
为了使内积在\(V\)的任意一个向量\(\alpha\)与自身组成的有序对上的函数值为实数,
需要让\(V\)上的内积\(\rho\)具有如下性质:
\begin{equation}\label{equation:酉空间.厄米性}
	(\forall \alpha,\beta \in V)
	[
		\rho(\alpha,\beta)
		= \ComplexConjugate{\rho(\beta,\alpha)}
	].
\end{equation}
我们把\cref{equation:酉空间.厄米性} 描述的性质
称为\DefineConcept{厄米性}或\DefineConcept{共轭对称性}(conjugate symmetry).
于是复线性空间上的内积的概念应当定义如下:
\begin{definition}\label{definition:酉空间.复线性空间上的内积}
%@see: 《高等代数(第三版 下册)》(丘维声) P193 定义1
%@see: 《Linear Algebra Done Right (Fourth Eidition)》(Sheldon Axler) P183 6.2
设\(V\)是复数域\(\mathbb{C}\)上的一个线性空间.
如果映射\(\rho\colon V \times V \to \mathbb{R}\)满足以下性质\begin{itemize}
	\item {\rm\bf 厄米性}:\begin{equation*}
		(\forall \alpha,\beta \in V)
		[
			\rho(\alpha,\beta)
			= \ComplexConjugate{\rho(\beta,\alpha)}
		];
	\end{equation*}

	\item {\rm\bf 线性性}:\begin{gather*}
		(\forall \alpha,\beta,\gamma \in V)
		[
			\rho(\alpha+\beta,\gamma)
			= \rho(\alpha,\gamma) + \rho(\beta,\gamma)
		], \\
		(\forall \alpha,\beta \in V)
		(\forall k \in \mathbb{C})
		[
			\rho(k\alpha,\beta)
			= k\rho(\alpha,\beta)
		];
	\end{gather*}

	\item {\rm\bf 正定性}:\begin{gather*}
		(\forall \alpha \in V)
		[
			\rho(\alpha,\alpha) \geq 0
		], \\
		(\forall \alpha \in V)
		[
			\rho(\alpha,\alpha) = 0
			\iff
			\alpha = 0
		],
	\end{gather*}
\end{itemize}
则称“\(\rho\)是复线性空间\(V\)上的一个\DefineConcept{内积}(inner product)”.
\end{definition}
\begin{remark}
\hyperref[definition:欧几里得空间.实线性空间上的内积]{实线性空间上的内积}%
与\hyperref[definition:酉空间.复线性空间上的内积]{复线性空间上的内积}的
最大“差别”在于前者要求内积具有对称性,后者要求内积具有厄米性.
但是,考虑到实数的共轭就是它本身,
所以复线性空间上的内积也可以用作实线性空间上的内积.
\end{remark}

\begin{definition}
%@see: 《高等代数(第三版 下册)》(丘维声) P193 定义1
%@see: 《Linear Algebra Done Right (Fourth Eidition)》(Sheldon Axler) P184 6.4
设\(V\)是一个复线性空间,\(\rho\)是\(V\)上的一个内积,
则称“\((V,\rho)\)是一个\DefineConcept{复内积空间}(complex inner product space)”
或“\((V,\rho)\)是一个\DefineConcept{酉空间}(unitary space)”.
\end{definition}

\begin{property}
%@see: 《Linear Algebra Done Right (Fourth Eidition)》(Sheldon Axler) P185 6.6
设\((V,\rho)\)是一个酉空间,
则对于任意\(\alpha \in V\),
映射\(x \mapsto \rho(x,\alpha)\)是\(V\)上的线性函数.
\end{property}

\begin{property}
%@see: 《Linear Algebra Done Right (Fourth Eidition)》(Sheldon Axler) P185 6.6
设\((V,\rho)\)是一个酉空间,
则对于任意\(\alpha \in V\),
有\begin{equation*}
	\rho(0,\alpha) = \rho(\alpha,0) = 0.
\end{equation*}
\end{property}

\begin{property}\label{theorem:酉空间.复线性空间上内积对第二个自变量具有半线性性}
%@see: 《高等代数(第三版 下册)》(丘维声) P193
%@see: 《Linear Algebra Done Right (Fourth Eidition)》(Sheldon Axler) P185 6.6
设\((V,\rho)\)是一个酉空间,
则内积\(\rho\)对第二个自变量具有\DefineConcept{半线性性},
即\begin{equation}
	\rho(\alpha,k_1\beta_1+k_2\beta_2)
	= \ComplexConjugate{k_1} \rho(\alpha,\beta_1)
	+ \ComplexConjugate{k_2} \rho(\alpha,\beta_2).
\end{equation}
\begin{proof}
由内积的厄米性和它对第一个自变量的线性性,有\begin{align*}
	\rho(\alpha,k_1\beta_1+k_2\beta_2)
	&= \ComplexConjugate{\rho(k_1\beta_1+k_2\beta_2,\alpha)} \\
	&= \ComplexConjugate{k_1 \rho(\beta_1,\alpha)}
		+ \ComplexConjugate{k_2 \rho(\beta_2,\alpha)} \\
	&= \ComplexConjugate{k_1} \rho(\alpha,\beta_1)
		+ \ComplexConjugate{k_2} \rho(\alpha,\beta_2).
	\qedhere
\end{align*}
\end{proof}
\end{property}
\begin{remark}
注意与\cref{theorem:实线性空间.实线性空间上内积对第二个自变量具有线性性} 进行对比.
\end{remark}

\begin{example}
%@see: 《高等代数(第三版 下册)》(丘维声) P193 例1
在\(V = \mathbb{C}^n\)中,
%@see: 《高等代数(第三版 下册)》(丘维声) P193 (1)
令\(f(X,Y) \defeq x_1 \ComplexConjugate{y_1} + \dotsb + x_n \ComplexConjugate{y_n}\),
其中\(X=(\AutoTuple{x}{n})^T,
Y=(\AutoTuple{y}{n})^T\).
容易验证,\(f\)是\(V\)上的一个内积.
我们把这个内积称为 \DefineConcept{\(\mathbb{C}^n\)上的标准内积}.
\end{example}

\begin{example}
%@see: 《高等代数(第三版 下册)》(丘维声) P194 例2
设\(V = \tilde{C}[a,b]\)表示区间\([a,b]\)上所有连续复值函数组成的线性空间.
%@see: 《高等代数(第三版 下册)》(丘维声) P194 (2)
令\(\rho(f,g) \defeq \int_a^b f(x) \ComplexConjugate{g(x)} \dd{x}\).
容易验证,\(\rho\)是\(V\)上的一个内积.
\end{example}

\begin{example}
%@see: 《高等代数(第三版 下册)》(丘维声) P194 例3
在\(V = M_n(\mathbb{C})\)中,
%@see: 《高等代数(第三版 下册)》(丘维声) P194 (3)
令\(f(A,B) \defeq \tr(A B^H)\).
容易验证,\(f\)是\(V\)上的一个内积.
\end{example}

\begin{definition}
%@see: 《高等代数(第三版 下册)》(丘维声) P194 定义2
%@see: 《Linear Algebra Done Right (Fourth Eidition)》(Sheldon Axler) P186 6.7
设\((V,\rho)\)是一个酉空间,
\(\alpha \in V\).
把非负实数\(\sqrt{\rho(\alpha,\alpha)}\)
称为“向量\(\alpha\)的\DefineConcept{长度}”,
记作\(\VectorLengthA{\alpha}\)或\(\VectorLengthN{\alpha}\).
\end{definition}

\begin{property}\label{theorem:酉空间.向量的长度具有非负性}
%@see: 《高等代数(第三版 下册)》(丘维声) P194
%@see: 《Linear Algebra Done Right (Fourth Eidition)》(Sheldon Axler) P186 6.9
在酉空间\(V\)中,
零向量的长度为\(0\),
非零向量的长度是正数.
\end{property}

\begin{property}\label{theorem:酉空间.向量的长度具有齐次性}
%@see: 《高等代数(第三版 下册)》(丘维声) P194
%@see: 《Linear Algebra Done Right (Fourth Eidition)》(Sheldon Axler) P186 6.9
在酉空间\((V,\rho)\)中,
对于\(\forall \alpha \in V\)
和\(\forall k \in \mathbb{C}\),
有\(\VectorLengthA{k\alpha} = \ComplexLengthA{k} \VectorLengthA{\alpha}\).
\begin{proof}
%@see: 《高等代数(第三版 下册)》(丘维声) P194 (4)
\(\VectorLengthA{k\alpha}
= \sqrt{\rho(k\alpha,k\alpha)}
= \sqrt{k \ComplexConjugate{k} \rho(\alpha,\alpha)}
= \ComplexLengthA{k} \VectorLengthA{\alpha}\).
\end{proof}
\end{property}

\begin{theorem}
%@see: 《高等代数(第三版 下册)》(丘维声) P194 定理1(柯西-布尼亚科夫斯基不等式)
%@see: 《Linear Algebra Done Right (Fourth Eidition)》(Sheldon Axler) P189 6.14
在酉空间\((V,\rho)\)中,
对于\(\forall \alpha,\beta \in V\),
有\begin{equation}
	\abs{\rho(\alpha,\beta)} \leq \VectorLengthA{\alpha} \VectorLengthA{\beta}.
\end{equation}
当且仅当\(\{\alpha,\beta\}\)线性相关时,上式取“\(=\)”号.
%TODO proof
\end{theorem}

\begin{definition}
%@see: 《高等代数(第三版 下册)》(丘维声) P194 定义3
设\(\alpha,\beta\)是酉空间\((V,\rho)\)中的两个非零向量.
把\begin{equation}
%@see: 《高等代数(第三版 下册)》(丘维声) P194 (7)
	\arccos\frac{\abs{\rho(\alpha,\beta)}}{\VectorLengthA{\alpha} \VectorLengthA{\beta}}
\end{equation}
称为“\(\alpha\)与\(\beta\)的\DefineConcept{夹角}”,
记为\(\VectorAngleA{\alpha}{\beta}\)或\(\VectorAngleP{\alpha}{\beta}\).
\end{definition}

\begin{property}
%@see: 《高等代数(第三版 下册)》(丘维声) P194
设\(\alpha,\beta\)是酉空间\((V,\rho)\)中的两个非零向量,
则\(\alpha\)与\(\beta\)的夹角\(\theta = \VectorAngleA{\alpha}{\beta}\)
满足\(0 \leq \theta \leq \pi/2\).
%TODO proof
\end{property}

\begin{property}
%@see: 《高等代数(第三版 下册)》(丘维声) P195
设\(\alpha,\beta\)是酉空间\((V,\rho)\)中的两个非零向量,
则\(\alpha\)与\(\beta\)的夹角\(\theta = \VectorAngleA{\alpha}{\beta}\)
满足\(\theta = \frac\pi2 \iff \rho(\alpha,\beta) = 0\).
%TODO proof
\end{property}

\begin{definition}
%@see: 《高等代数(第三版 下册)》(丘维声) P195 定义4
%@see: 《Linear Algebra Done Right (Fourth Eidition)》(Sheldon Axler) P187 6.10
设\(\alpha,\beta\)是酉空间\((V,\rho)\)中的两个非零向量.
如果\(\rho(\alpha,\beta) = 0\),
则称“\(\alpha\)与\(\beta\) \DefineConcept{正交}(orthogonal)”,
记为\(\alpha \perp \beta\).
\end{definition}

\begin{property}
%@see: 《Linear Algebra Done Right (Fourth Eidition)》(Sheldon Axler) P187 6.11
在酉空间\((V,\rho)\)中,零向量与任意一个向量正交.
\end{property}

\begin{property}\label{theorem:酉空间.酉空间中不存在非零迷向向量}
%@see: 《Linear Algebra Done Right (Fourth Eidition)》(Sheldon Axler) P187 6.11
在酉空间\((V,\rho)\)中,只有零向量与它本身正交.
\end{property}
\begin{remark}
\cref{theorem:酉空间.酉空间中不存在非零迷向向量} 说明:酉空间中不存在非零迷向向量.
\end{remark}

\begin{corollary}\label{theorem:酉空间.三角不等式}
%@see: 《高等代数(第三版 下册)》(丘维声) P195
%@see: 《Linear Algebra Done Right (Fourth Eidition)》(Sheldon Axler) P190 6.17
在酉空间\((V,\rho)\)中,
\DefineConcept{三角不等式}(triangle inequality)成立,
即对于\(\forall \alpha,\beta \in V\),
有\begin{equation}
	\VectorLengthA{\alpha+\beta} \leq \VectorLengthA{\alpha} + \VectorLengthA{\beta}.
\end{equation}
当且仅当\(\alpha = k\beta\)或\(\beta = k\alpha\)(其中\(k\geq0\))时,上式取“\(=\)”号.
%TODO proof
\end{corollary}

\begin{corollary}\label{theorem:酉空间.勾股定理}
%@see: 《高等代数(第三版 下册)》(丘维声) P195
%@see: 《Linear Algebra Done Right (Fourth Eidition)》(Sheldon Axler) P187 6.12
在酉空间\((V,\rho)\)中,
\DefineConcept{勾股定理}成立,
即对于\(\forall \alpha,\beta \in V\),
如果\(\alpha\)与\(\beta\)正交,
则\begin{equation}
	\VectorLengthA{\alpha+\beta}^2 = \VectorLengthA{\alpha}^2 + \VectorLengthA{\beta}^2.
\end{equation}
\begin{proof}
证明过程与\cref{theorem:实内积空间.勾股定理} 相同.
\end{proof}
\end{corollary}

\begin{proposition}\label{theorem:酉空间.平行四边形等式}
%@see: 《Linear Algebra Done Right (Fourth Eidition)》(Sheldon Axler) P191 6.21
在酉空间\((V,\rho)\)中,
\(\alpha,\beta \in V\),
则\begin{equation*}
	\VectorLengthA{\alpha+\beta}^2
	+ \VectorLengthA{\alpha-\beta}^2
	= 2(\VectorLengthA{\alpha}^2+\VectorLengthA{\beta}^2).
\end{equation*}
\begin{proof}
直接有\begin{align*}
	\VectorLengthA{\alpha+\beta}^2
	+ \VectorLengthA{\alpha-\beta}^2
	&= \rho(\alpha+\beta,\alpha+\beta) + \rho(\alpha-\beta,\alpha-\beta) \\
	&= (\VectorLengthA{\alpha}^2+\VectorLengthA{\beta}^2+\rho(\alpha,\beta)+\rho(\beta,\alpha)) \\
	&\hspace{20pt}+(\VectorLengthA{\alpha}^2+\VectorLengthA{\beta}^2-\rho(\alpha,\beta)-\rho(\beta,\alpha)) \\
	&= 2(\VectorLengthA{\alpha}^2+\VectorLengthA{\beta}^2).
	\qedhere
\end{align*}
\end{proof}
\end{proposition}

\begin{proposition}\label{theorem:酉空间.向量的正交分解}
%@see: 《Linear Algebra Done Right (Fourth Eidition)》(Sheldon Axler) P188 6.13
在酉空间\((V,\rho)\)中,
\(\alpha,\beta \in V\),
且\(\beta \neq 0\).
令\begin{equation*}
	k \defeq \frac{\rho(\alpha,\beta)}{\rho(\beta,\beta)},
	\qquad
	\gamma \defeq \alpha - k \beta,
\end{equation*}
则\(\alpha = k\beta + \gamma\)且\(\rho(\beta,\gamma) = 0\).
\end{proposition}
\begin{remark}
将一个向量\(\alpha\)化为两个正交向量之和\(k\beta + \gamma\)的过程,
称为\DefineConcept{正交分解}(orthogonal decomposition).
\end{remark}

\begin{definition}
%@see: 《高等代数(第三版 下册)》(丘维声) P195
在酉空间\((V,\rho)\)中,
\(\alpha,\beta \in V\).
把\(\VectorLengthA{\alpha-\beta}\)
称为“\(\alpha\)与\(\beta\)的\DefineConcept{距离}”,
记为\(d(\alpha,\beta)\).
\end{definition}

\subsection{有限维酉空间中的基}
我们希望在有限维酉空间\(V\)中找出一类基,
使得在这样的基下容易计算\(V\)中任意两个向量的内积,
从而易于计算长度、角度、距离等.

\begin{definition}
%@see: 《高等代数(第三版 下册)》(丘维声) P195
设\((V,\rho)\)是一个有限维酉空间,
\(A\)是\(V\)中的一个向量组,
如果\begin{equation*}
	(\forall \alpha \in A)
	[\alpha\neq0],
	\qquad
	(\forall \alpha,\beta \in A)
	[\alpha \perp \beta],
\end{equation*}
则称“\(A\)是\((V,\rho)\)中的一个\DefineConcept{正交向量组}(orthogonal list)”.
\end{definition}

\begin{definition}
%@see: 《高等代数(第三版 下册)》(丘维声) P195
%@see: 《Linear Algebra Done Right (Fourth Eidition)》(Sheldon Axler) P197 6.22
设\((V,\rho)\)是一个有限维酉空间,
\(A\)是\(V\)中的一个向量组,
如果\begin{equation*}
	(\forall \alpha \in A)
	[\VectorLengthA{\alpha} = 1],
	\qquad
	(\forall \alpha,\beta \in A)
	[\alpha \perp \beta],
\end{equation*}
则称“\(A\)是\((V,\rho)\)中的一个\DefineConcept{正交单位向量组}(orthonormal list)”.
\end{definition}

\begin{proposition}\label{theorem:酉空间.正交向量组的线性组合的长度}
%@see: 《Linear Algebra Done Right (Fourth Eidition)》(Sheldon Axler) P198 6.24
设\(\AutoTuple{\alpha}{s}\)是酉空间\((V,\rho)\)中一个正交向量组,
那么对于\(\forall \AutoTuple{k}{s} \in F\),
有\begin{equation*}
	\VectorLengthA{
		k_1 \alpha_1 + \dotsb + k_s \alpha_s
	}^2
	= \abs{k_1}^2 \VectorLengthA{\alpha_1}^2 + \dotsb + \abs{k_s}^2 \VectorLengthA{\alpha_s}^2.
\end{equation*}
\begin{proof}
由\hyperref[theorem:酉空间.勾股定理]{勾股定理}和\cref{theorem:酉空间.向量的长度具有齐次性} 立即可得.
\end{proof}
\end{proposition}

\begin{proposition}
%@see: 《Linear Algebra Done Right (Fourth Eidition)》(Sheldon Axler) P198 6.25
在酉空间\((V,\rho)\)中,
任意一个正交向量组一定线性无关.
%TODO proof
\end{proposition}

\begin{definition}
%@see: 《高等代数(第三版 下册)》(丘维声) P195
%@see: 《Linear Algebra Done Right (Fourth Eidition)》(Sheldon Axler) P199 6.27
设\((V,\rho)\)是\(n\)维酉空间,
\(A\)是\(V\)的一个基.
如果\(A\)是正交向量组,
则称“\(A\)是\((V,\rho)\)的一个\DefineConcept{正交基}(orthonormal basis)”.
\end{definition}

\begin{definition}
%@see: 《高等代数(第三版 下册)》(丘维声) P195
设\((V,\rho)\)是\(n\)维酉空间,
\(A\)是\(V\)的一个基.
如果\(A\)是正交单位向量组,
则称“\(A\)是\((V,\rho)\)的一个\DefineConcept{标准正交基}”
或“\(A\)是\((V,\rho)\)的一个\DefineConcept{规范正交基}”.
\end{definition}

% 与《高等代数(第三版 上册)》第4章第6节定理4的证明方法完全一样,
在酉空间\(V\)中也有施密特正交化过程,
它可以把一个线性无关向量组变成与之等价的正交向量组.

在\(n\)维酉空间\(V\)中,取一个基\(\AutoTuple{\alpha}{n}\),
经过施密特正交化把它变成正交基\(\AutoTuple{\beta}{n}\),
再经过单位化把它变成标准正交基\(\AutoTuple{\gamma}{n}\).
这就说明:
\begin{theorem}
%@see: 《Linear Algebra Done Right (Fourth Eidition)》(Sheldon Axler) P202 6.35
有限维酉空间中,一定存在标准正交基.
%TODO proof
\end{theorem}

\begin{proposition}
%@see: 《Linear Algebra Done Right (Fourth Eidition)》(Sheldon Axler) P203 6.36
有限维酉空间\(V\)中的任意一个正交单位向量组
均可以扩充成\(V\)的一个标准正交基.
%TODO proof
\end{proposition}

\subsection{酉空间中向量的内积、傅里叶展开}
\begin{proposition}
%@see: 《高等代数(第三版 下册)》(丘维声) P195
设\((V,\rho)\)是\(n\)维酉空间,
则向量组\(\AutoTuple{\eta}{n}\)是\((V,\rho)\)的一个标准正交基,
当且仅当\begin{equation*}
%@see: 《高等代数(第三版 下册)》(丘维声) P195 (8)
	\rho(\eta_i,\eta_j)
	= \delta(i,j),
	\quad i,j=1,2,\dotsc,n,
\end{equation*}
其中\(\delta\)是克罗内克\(\delta\)函数.
\end{proposition}

利用标准正交基,容易计算向量的内积.

设\((V,\rho)\)是\(n\)维酉空间,
向量\(\alpha,\beta \in V\),
向量组\(\AutoTuple{\eta}{n}\)是\((V,\rho)\)的一个标准正交基,
\(\alpha,\beta\)在基\(\AutoTuple{\eta}{n}\)下的坐标
分别是\(X=(\AutoTuple{x}{n})^T,
Y=(\AutoTuple{y}{n})^T\),
则\begin{equation*}
%@see: 《高等代数(第三版 下册)》(丘维声) P195 (9)
	\rho(\alpha,\beta)
	= \rho\left( \sum_{i=1}^n x_i \eta_i, \sum_{j=1}^n y_j \eta_j \right)
	= \sum_{i=1}^n \sum_{j=1}^n x_i \ComplexConjugate{y_j} \rho(\eta_i,\eta_j)
	= \sum_{i=1}^n x_i \ComplexConjugate{y_i}
	= Y^H X.
\end{equation*}
%@see: 《Linear Algebra Done Right (Fourth Eidition)》(Sheldon Axler) P200 6.30(c)
上式还可以写成\begin{equation}
	\rho(\alpha,\beta)
	= \sum_{i=1}^n \rho(\alpha,\eta_i) \ComplexConjugate{\rho(\beta,\eta_i)}.
\end{equation}

利用标准正交基,可以借助内积,表达向量的坐标分量.

设\(\alpha\)在标准正交基\(\AutoTuple{\eta}{n}\)下的坐标为\(X=(\AutoTuple{x}{n})^T\),
则\begin{equation*}
	\alpha = \sum_{i=1}^n x_i \eta_i;
\end{equation*}
等号两边用\(\eta_j\)作内积,得\begin{equation*}
	\rho(\alpha,\eta_j)
	= \rho\left( \sum_{i=1}^n x_i \eta_i, \eta_j \right)
	= \sum_{i=1}^n x_i \rho(\eta_i,\eta_j)
	= x_j,
\end{equation*}
因此\begin{equation}\label{equation:酉空间.向量的傅里叶展开}
%@see: 《高等代数(第三版 下册)》(丘维声) P196 (10)
%@see: 《Linear Algebra Done Right (Fourth Eidition)》(Sheldon Axler) P200 6.30(a)
	\alpha = \sum_{i=1}^n \rho(\alpha,\eta_i) \eta_i.
\end{equation}
我们把\cref{equation:酉空间.向量的傅里叶展开}
称为“向量\(\alpha\)的\DefineConcept{傅里叶展开}”,
其中系数\(\rho(\alpha,\eta_i)\ (i=1,2,\dotsc,n)\)
称为“向量\(\alpha\)的\DefineConcept{傅里叶系数}”.

由\cref{equation:酉空间.向量的傅里叶展开,theorem:酉空间.正交向量组的线性组合的长度} 可得\begin{equation}
%@see: 《Linear Algebra Done Right (Fourth Eidition)》(Sheldon Axler) P200 6.30(b)
	\VectorLengthA{\alpha}^2
	= \sum_{i=1}^n \rho^2(\alpha,\eta_i).
\end{equation}

\begin{proposition}
%@see: 《高等代数(第三版 下册)》(丘维声) P196
在有限维酉空间\((V,\rho)\)中,
一个标准正交基到另一个标准正交基的过渡矩阵
一定是酉矩阵.
%TODO proof
\end{proposition}

\begin{proposition}
%@see: 《高等代数(第三版 下册)》(丘维声) P196
设\((V,\rho)\)是\(n\)维酉空间,
向量组\(\AutoTuple{\eta}{n}\)是\((V,\rho)\)的一个标准正交基,
\(\AutoTuple{\beta}{n}\)是\(V\)中一个向量组.
如果存在酉矩阵\(P \in M_n(\mathbb{C})\),
使得\begin{equation*}
	(\AutoTuple{\beta}{n})
	= (\AutoTuple{\eta}{n}) P,
\end{equation*}
那么\(\AutoTuple{\beta}{n}\)是\(V\)中一个标准正交基.
%TODO proof
\end{proposition}

\subsection{酉空间之间的同构}
\begin{definition}
%@see: 《高等代数(第三版 下册)》(丘维声) P196
设\((V_1,\rho_1),(V_2,\rho_2)\)都是酉空间.
如果存在从\(V_1\)到\(V_2\)的一个双射\(\sigma\),
使得\begin{gather*}
	(\forall \alpha,\beta \in V_1)
	[
		\sigma(\alpha+\beta)
		= \sigma(\alpha) + \sigma(\beta)
	], \\
	(\forall \alpha \in V_1)
	(\forall k \in \mathbb{C})
	[
		\sigma(k\alpha)
		= k \sigma(\alpha)
	], \\
	(\forall \alpha,\beta \in V_1)
	[
		\rho_2(\sigma(\alpha),\sigma(\beta))
		= \rho_1(\alpha,\beta)
	],
\end{gather*}
则称“\(\sigma\)是从\(V_1\)到\(V_2\)的一个\DefineConcept{同构}”;
并称“\(V_1\)与\(V_2\) \DefineConcept{同构}”,
记为\(V_1 \Isomorphism V_2\).
\end{definition}

\begin{theorem}\label{theorem:酉空间.两个酉空间同构的充分必要条件}
%@see: 《高等代数(第三版 下册)》(丘维声) P196
两个酉空间同构的充分必要条件是它们的维数相同.
%TODO proof
\end{theorem}

\subsection{酉变换}
\begin{definition}
%@see: 《高等代数(第三版 下册)》(丘维声) P197 定义6
设\((V,\rho)\)是一个酉空间,
\(\vb{A}\)是一个从\(V\)到\(V\)的满射.
如果\(\vb{A}\)满足\begin{equation}
	(\forall \alpha,\beta \in V)
	[
		\rho(\vb{A}\alpha,\vb{A}\beta)
		= \rho(\alpha,\beta)
	],
\end{equation}
则称“\(\vb{A}\)保持向量的内积不变”
“\(\vb{A}\)是\(V\)上的一个\DefineConcept{酉变换}”.
\end{definition}

\begin{proposition}
%@see: 《高等代数(第三版 下册)》(丘维声) P197 命题2
酉空间上的酉变换一定是可逆线性变换.
%TODO proof
\end{proposition}

\begin{proposition}
%@see: 《高等代数(第三版 下册)》(丘维声) P197 命题3
设\(\vb{A}\)是\(n\)维酉空间\(V\)上的一个线性变换,
则\begin{align*}
	&\text{$\vb{A}$是酉变换} \\
	&\iff \text{$\vb{A}$把$V$的标准正交基映成标准正交基} \\
	&\iff \text{$\vb{A}$在$V$的标准正交基下的矩阵是酉矩阵}.
\end{align*}
\end{proposition}

\begin{proposition}
%@see: 《高等代数(第三版 下册)》(丘维声) P197 命题4
有限维酉空间上的酉变换的特征值的模等于\(1\).
\begin{proof}
由\cref{equation:幺正矩阵.幺正矩阵的行列式} 立即可得.
\end{proof}
\end{proposition}

\begin{proposition}
%@see: 《高等代数(第三版 下册)》(丘维声) P197 命题5
设\(\vb{A}\)是酉空间\((V,\rho)\)上的一个酉变换.
如果\(W\)是\(\vb{A}\)的有限维不变子空间,
则\(W\)的正交补\(W^\perp\)也是\(\vb{A}\)的不变子空间.
\begin{proof}
任取\(\beta \in W^\perp\).
要证\(\vb{A}\beta \in W^\perp\).
任取\(\alpha \in W\),
由于酉变换\(\vb{A}\)是可逆的,
因此由\cref{example:线性映射.可逆线性变换的逆变换的不变子空间} 可知
\(W\)也是\(\vb{A}^{-1}\)的不变子空间,
从而\(\vb{A}^{-1}\alpha \in W\).
于是\begin{equation*}
	\rho(\vb{A}\beta,\alpha)
	= \rho(\vb{A}\beta,\vb{A}\vb{A}^{-1}\alpha)
	= \rho(\beta,\vb{A}^{-1}\alpha)
	= 0,
\end{equation*}
即\(\vb{A}\beta \in W^\perp\),
说明\(W^\perp\)是\(\vb{A}\)的不变子空间.
\end{proof}
\end{proposition}

\begin{theorem}
%@see: 《高等代数(第三版 下册)》(丘维声) P197 定理6
设\(\vb{A}\)是\(n\)维酉空间\(V\)上的酉变换,
则\(V\)中存在一个标准正交基\(S\),
使得\(\vb{A}\)在基\(S\)下的矩阵是对角矩阵,
且主对角元都是模为\(1\)的复数.
%TODO proof
\end{theorem}

\begin{definition}
%@see: 《高等代数(第三版 下册)》(丘维声) P198 推论7
设矩阵\(A,B \in M_n(\mathbb{C})\).
若存在可逆矩阵\(P \in M_n(\mathbb{C})\),
使得\begin{equation}
	P^{-1} A P = B,
\end{equation}
则称“\(A\)与\(B\) \DefineConcept{酉相似}”
或“\(A\) \DefineConcept{酉相似于} \(B\)”.
\end{definition}

\begin{corollary}
%@see: 《高等代数(第三版 下册)》(丘维声) P198 推论7
任意一个\(n\)阶酉矩阵一定酉相似于某个主对角元都是模为\(1\)的复数的对角矩阵.
%TODO proof
\end{corollary}

\subsection{厄米变换}
类比于实内积空间\(V\)上的对称变换,引出酉空间上的下述变换:
\begin{definition}
%@see: 《高等代数(第三版 下册)》(丘维声) P198 定义7
设\(\vb{A}\)是酉空间\((V,\rho)\)上的一个线性变换.
如果\begin{equation*}
	(\forall \alpha,\beta \in V)
	[
		\rho(\vb{A}\alpha,\beta)
		= \rho(\alpha,\vb{A}\beta)
	],
\end{equation*}
则称“\(\vb{A}\)是\(V\)上的一个\DefineConcept{厄米变换}”
或“\(\vb{A}\)是\(V\)上的一个\DefineConcept{自伴随变换}”.
\end{definition}

\begin{proposition}
%@see: 《高等代数(第三版 下册)》(丘维声) P198 命题8
酉空间上的厄米变换一定是线性变换.
%TODO proof
\end{proposition}

\begin{proposition}
%@see: 《高等代数(第三版 下册)》(丘维声) P198 命题9
设\(\vb{A}\)是\(n\)维酉空间\(V\)上的一个线性变换,
则\(\vb{A}\)是厄米变换,
当且仅当\(\vb{A}\)在\(V\)的某个标准正交基下的矩阵是厄米矩阵.
%TODO proof
\end{proposition}

我们来探索\(n\)维酉空间上的厄米变换的最简单形式的矩阵表示.
\begin{proposition}
%@see: 《高等代数(第三版 下册)》(丘维声) P199 命题10
酉空间上的厄米变换的特征值只要存在就一定是实数.
%TODO proof
\end{proposition}

\begin{proposition}
%@see: 《高等代数(第三版 下册)》(丘维声) P199 命题11
设\(\vb{A}\)是酉空间\(V\)上的一个厄米变换.
如果\(W\)是\(\vb{A}\)的一个不变子空间,
则\(W\)的正交补\(W^\perp\)也是\(\vb{A}\)的一个不变子空间.
\end{proposition}

\begin{theorem}
%@see: 《高等代数(第三版 下册)》(丘维声) P199 定理12
设\(\vb{A}\)是酉空间\(V\)上的一个厄米变换,
则\(V\)中存在一个标准正交基\(S\),
使得\(\vb{A}\)在基\(S\)下的矩阵是对角矩阵,
且主对角元都是实数.
%TODO proof
\end{theorem}

\begin{corollary}
%@see: 《高等代数(第三版 下册)》(丘维声) P199 推论13
任意一个\(n\)阶厄米矩阵一定酉相似于某个实对角矩阵.
\end{corollary}

\section{辛空间}


\chapter{赋范线性空间}
\section{范数}
尽管我们通常出于几何(特别是欧氏几何)的考量,
将向量\(\alpha\)的模(或范数)定义为\(\sqrt{\VectorInnerProductDot{\alpha}{\alpha}}\),
不过我们还可以定义其他形式的模(或范数).

\subsection{范数的定义}
\begin{definition}
%@see: 《矩阵分析与应用(第2版)》(张贤达) P23 定义1.3.7(范数和赋范向量空间)
设\(V\)是域\(F\)上的一个线性空间.
如果映射\(p\colon V \to \mathbb{R}\)
满足\begin{itemize}
	\item {\rm\bf 非负性}:\begin{equation*}
		(\forall \alpha \in V)
		[p(\alpha) \geq 0],
		\qquad
		(\forall \alpha \in V)
		[
			p(\alpha) = 0
			\iff
			\alpha = 0
		];
	\end{equation*}

	\item {\rm\bf 齐次性}:\begin{equation*}
		(\forall \alpha \in V)
		(\forall k \in F)
		[
			p(k \alpha) = \abs{k} p(\alpha)
		];
	\end{equation*}

	\item {\rm\bf 三角不等式}:\begin{equation*}
		(\forall \alpha,\beta \in V)
		[
			p(\alpha+\beta) \leq p(\alpha) + p(\beta)
		],
	\end{equation*}
\end{itemize}
则称“\(f\)是线性空间\(V\)的一个\DefineConcept{范数}(norm)”
“\((V,f)\)是域\(F\)上的一个\DefineConcept{赋范线性空间}”,
在不致混淆的情况下简称“\(V\)是一个\DefineConcept{赋范线性空间}”;
对于任意向量\(\alpha \in V\),
把\(\alpha\)在\(f\)下的像
称为“向量\(\alpha\)的\(f\) \DefineConcept{范数}”,
记作\(\VectorLengthN{\alpha}\),
即\begin{equation*}
	\VectorLengthN{\alpha}
	\defeq
	f(\alpha).
\end{equation*}
\end{definition}

\subsection{半范数}
\begin{definition}
%@see: 《矩阵分析与应用(第2版)》(张贤达) P24 定义1.3.8
设\(V\)是域\(F\)上的一个线性空间.
如果映射\(p\colon V \to \mathbb{R}\)
满足\begin{itemize}
	\item {\rm\bf 非负性}:\begin{equation*}
		(\forall \alpha \in V)
		[p(\alpha) \geq 0];
	\end{equation*}

	\item {\rm\bf 齐次性}:\begin{equation*}
		(\forall \alpha \in V)
		(\forall k \in F)
		[
			p(k \alpha) = \abs{k} p(\alpha)
		];
	\end{equation*}

	\item {\rm\bf 三角不等式}:\begin{equation*}
		(\forall \alpha,\beta \in V)
		[
			p(\alpha+\beta) \leq p(\alpha) + p(\beta)
		],
	\end{equation*}
\end{itemize}
则称“\(f\)是线性空间\(V\)的一个\DefineConcept{半范数}(seminorm)”
或者“\(f\)是线性空间\(V\)的一个\DefineConcept{伪范数}(seminorm)”.
%@see: https://mathworld.wolfram.com/Seminorm.html
\end{definition}

半范数与范数的唯一区别是:
半范数不完全满足范数的非负性公理,
有可能当\(\alpha\neq0\)时成立\(p(\alpha) = 0\).
\begin{proposition}
设\(f\)是线性空间\(V\)的一个半范数.
如果\begin{equation*}
	(\forall \alpha \in V)
	[
		p(\alpha) = 0
		\iff
		\alpha = 0
	],
\end{equation*}
则\(f\)是线性空间\(V\)的一个范数.
\end{proposition}

\begin{example}
对于任意\(n\)维向量\(\alpha = (x_1,\dotsc,x_n)^T\),
只要满足\begin{equation*}
	x_1 + \dotsb + x_n = 0,
	% 即\(\alpha\)是“零均值向量”
\end{equation*}
则映射\begin{equation*}
	p(\alpha)
	\defeq
	x_1 + \dotsb + x_n
\end{equation*}
就是向量\(\alpha\)的一个半范数.
但是,显然\(p(\alpha) = 0\)并不意味着\(\alpha = 0\).
\end{example}

\begin{definition}% 向量的范数
%@see: 《数值分析(第5版)》(李庆扬、王能超、易大义) P53
设\(\vb{x} \defeq (x_1,\dotsc,x_n)^T \in \mathbb{R}^n\).
定义:\begin{align*}
	\norm{\vb{x}}_\infty
	&\defeq
	\max_{1 \leq i \leq n} \abs{x_i}, \\
	\norm{\vb{x}}_1
	&\defeq
	\sum_{i=1}^n \abs{x_i}, \\
	\norm{\vb{x}}_2
	&\defeq
	\left( \sum_{i=1}^n x_i^2 \right)^{\frac12}.
\end{align*}
\end{definition}

\begin{definition}% 连续函数的范数
%@see: 《数值分析(第5版)》(李庆扬、王能超、易大义) P53
设\(f \in C[a,b]\).
定义:\begin{align*}
	\norm{f}_\infty
	&\defeq
	\max_{a \leq x \leq b} \abs{f(x)}, \\
	\norm{f}_1
	&\defeq
	\int_a^b \abs{f(x)} \dd{x}, \\
	\norm{f}_2
	&\defeq
	\left( \int_a^b f^2(x) \right)^{\frac12}.
\end{align*}
\end{definition}

\subsection{拟范数}
\begin{definition}
%@see: 《矩阵分析与应用(第2版)》(张贤达) P24 定义1.3.9
设\(V\)是有序域\(F\)上的一个线性空间.
如果映射\(p\colon V \to \mathbb{R}\)
满足\begin{itemize}
	\item {\rm\bf 非负性}:\begin{equation*}
		(\forall \alpha \in V)
		[p(\alpha) \geq 0],
		\qquad
		(\forall \alpha \in V)
		[
			p(\alpha) = 0
			\iff
			\alpha = 0
		];
	\end{equation*}

	\item {\rm\bf 齐次性}:\begin{equation*}
		(\forall \alpha \in V)
		(\forall k \in F)
		[
			p(k \alpha) = \abs{k} p(\alpha)
		];
	\end{equation*}

	\item {\rm\bf 三角不等式}:\begin{equation*}
		(\forall \alpha,\beta \in V)
		(\exists c \in C)
		[
			p(\alpha+\beta) \leq c [p(\alpha) + p(\beta)]
		],
	\end{equation*}
	其中\(
		C
		\defeq
		\Set{
			x \in F
			\given
			x > 0,
			x \neq 1
		}
	\),
\end{itemize}
则称“\(f\)是线性空间\(V\)的一个\DefineConcept{拟范数}(quasinorm)”.
\end{definition}

拟范数与范数的唯一区别是:
拟范数不严格满足范数的三角不等式公理.

\begin{example}
%@see: 《矩阵分析与应用(第2版)》(张贤达) P24
同一个定义公式,有时给出拟范数,有时给出范数,取决于参数的变化.
例如,容易验证\begin{equation*}
%@see: 《矩阵分析与应用(第2版)》(张贤达) P24 (1.3.24)
	\norm{\alpha}_p
	\defeq
	\left( \sum_{i=1}^m x_i^p \right)^{1/p}
\end{equation*}
当\(0 < p < 1\)时给出的是拟范数,
当\(p \geq 1\)时给出的是范数.
\end{example}

\subsection{向量范数}
\begin{definition}
%@see: 《矩阵论》(詹兴致) P5
设\(\alpha \in \mathbb{C}^n\).
定义:\begin{equation}
	\MatrixNorm{A}_p
	\defeq
	\left( \sum_{i=1}^n \ComplexLengthA{a_i}^p \right)^{1/p},
\end{equation}
称之为“向量\(\alpha\)的 \DefineConcept{\(L_p\)范数}”,
其中\(\alpha = (\AutoTuple{a}{n})^T\).
\end{definition}

易见\begin{gather}
	\norm{\alpha}_1 = \VectorLengthA{x_1} + \VectorLengthA{x_2} + \dotsb + \VectorLengthA{x_n}, \\
	\norm{\alpha}_2 = \sqrt{x_1^2 + x_2^2 + \dotsb + x_n^2}, \\
	\norm{\alpha}_\infty = \max\{\VectorLengthA{x_1},\VectorLengthA{x_2},\dotsc,\VectorLengthA{x_n}\}.
\end{gather}

\subsection{矩阵范数}
由所有形状相同的矩阵组成的集合,对加法和标量乘法,成为一个线性空间.
于是我们可以在这个线性空间\(M_{m \times n}(K)\)上定义范数.

\begin{definition}
%@see: 《矩阵论》(詹兴致) P5
设\(A \in M_{m \times n}(\mathbb{C})\),
\(A^H\)是\(A\)的共轭转置.
定义:\begin{equation}
	\MatrixNorm{A}_F
	\defeq
	\sqrt{
		\tr(A^H A)
	},
\end{equation}
称之为“矩阵\(A\)的\DefineConcept{弗罗贝尼乌斯范数}”.
\end{definition}

\begin{definition}
%@see: 《矩阵论》(詹兴致) P5
设\(A \in M_{m \times n}(\mathbb{C})\).
定义:\begin{equation}
	\MatrixNorm{A}_r
	\defeq
	\max_{1 \leq i \leq m} \sum_{j=1}^n \abs{a_{ij}},
\end{equation}
称之为“矩阵\(A\)的\DefineConcept{行和范数}”.
\end{definition}

\begin{definition}
%@see: 《矩阵论》(詹兴致) P5
设\(A \in M_{m \times n}(\mathbb{C})\).
定义:\begin{equation}
	\MatrixNorm{A}_c
	\defeq
	\max_{1 \leq j \leq n} \sum_{i=1}^n \abs{a_{ij}},
\end{equation}
称之为“矩阵\(A\)的\DefineConcept{列和范数}”.
\end{definition}

\begin{definition}
%@see: 《矩阵论》(詹兴致) P5
设\(A \in M_{m \times n}(\mathbb{C})\),
\(f\)是\(\mathbb{C}^m\)上的一个范数,
\(g\)是\(\mathbb{C}^n\)上的一个范数.
定义:\begin{equation}
	\MatrixNorm{A}
	\defeq
	\max_{0 \neq x \in \mathbb{C}^n} \frac{f(Ax)}{g(x)},
\end{equation}
称之为“矩阵\(A\)由\(f\)和\(g\)诱导的\DefineConcept{算子范数}”.
\end{definition}

\begin{definition}
%@see: 《矩阵论》(詹兴致) P5
如果由\(f\)和\(g\)诱导的算子范数
\(N\colon M_n(\mathbb{C}) \to \mathbb{R}, A \mapsto \MatrixNorm{A}\)
满足\begin{equation*}
	(\forall A,B \in M_n(\mathbb{C}))
	[
		\MatrixNorm{AB}
		\leq \MatrixNorm{A} \MatrixNorm{B}
	],
\end{equation*}
则称“由\(f\)和\(g\)诱导的算子范数\(N\)是\DefineConcept{次可乘的}”.
\end{definition}

\begin{definition}
%@see: 《矩阵论》(詹兴致) P5
设\(A \in M_{m \times n}(\mathbb{C})\).
定义:\begin{equation}
	\MatrixNorm{A}_\infty
	\defeq
	\max_{x\neq0} \frac{\MatrixNorm{Ax}_2}{\MatrixNorm{x}_2},
\end{equation}
称之为“矩阵\(A\)的\DefineConcept{谱范数}”.
\end{definition}

\begin{property}
%@see: 《矩阵论》(詹兴致) P5
如果\(B\)是\(A \in M_{m \times n}(\mathbb{C})\)的一个子矩阵,
则\(\MatrixNorm{B}_\infty \leq \MatrixNorm{A}_\infty\).
\end{property}

\begin{property}
%@see: 《矩阵论》(詹兴致) P5
设\(A \in M_{m \times n}(\mathbb{C})\),
\(X\)是一个\(m\)阶酉矩阵,
\(Y\)是一个\(n\)阶酉矩阵,
则\begin{equation}
	\MatrixNorm{A}_\infty
	= \MatrixNorm{AY}_\infty
	= \MatrixNorm{XA}_\infty.
\end{equation}
\end{property}


\chapter{仿射几何与射影几何}
\section{仿射空间}
本节利用公理法描述仿射几何中的几何元素.

\subsection{仿射空间}
\begin{definition}
%@see: 《基础代数(第二卷)》(席南华) P161 定义4.1
%@see: 《代数学讲义(下册)》(李文威)
设\(A\)是一个非空集合,
\(V\)是域\(F\)上的一个线性空间.
若映射\(f\colon A \times V \to A\)满足\begin{gather*}
	(\forall p \in A)
	(\forall x,y \in V)
	[
		f(f(p,x),y)
		= f(p,x+y)
	], \\
	(\forall p \in A)
	[
		0 \in V
		\implies
		f(p,0) = p
	], \\
	(\forall p,q \in A)
	(\exists! x \in V)
	[
		f(p,x) = q
	],
\end{gather*}
则称“\((A,f)\)是与线性空间\(V\)关联的\DefineConcept{仿射空间}”.
把\(f\)称为“仿射空间\(A\)的\DefineConcept{加法}”,
在不致混淆的情况下把\(f(p,x)\)记为\(p + x\).
把\(A\)中的每一个元素称为“仿射空间\(A\)中的一个\DefineConcept{点}”.
对于\(A\)中任意两点\(p,q\),把满足\(f(p,x) = q\)的向量\(x \in V\)
称为“连接点\(p\)和点\(q\)的\DefineConcept{向量}”,记作\(\vec{pq}\)或\(q - p\).
把线性空间\(V\)的维数\(\dim V\)称为“仿射空间\(A\)的\DefineConcept{维数}”,记为\(\dim A\).
\end{definition}

\begin{property}%\label{theorem:仿射空间.仿射空间的减法1}
设\((A,f)\)是与线性空间\(V\)关联的仿射空间,
则\(x = \vec{pq}\)当且仅当\(f(p,x) = q\).
\end{property}

\begin{property}%\label{theorem:仿射空间.仿射空间的减法2}
设\((A,f)\)是与线性空间\(V\)关联的仿射空间,
则\(f(p,v_1) = f(p,v_2)\)当且仅当\(v_1 = v_2\).
\end{property}

\begin{proposition}\label{theorem:仿射空间.仿射空间与线性空间等势}
设\(A\)是与线性空间\(V\)关联的仿射空间,
则\(A\)与\(V\)等势.
\begin{proof}
由于仿射空间\(A\)是非空的,我们可以从中任意取定一个点\(o\).
然后构造一个从\(A\)到\(V\)的映射\(\sigma\colon A \to V, p \mapsto \vec{op}\).
根据仿射空间的定义,
对于\(A\)中任意一点\(p\),
满足\(f(o,v) = p\)的向量\(v \in V\)存在且唯一,
由此保证\(\sigma\)是单值的,且\(\ran\sigma \subseteq V\).
由\begin{equation*}
	(\forall p_1,p_2 \in A)
	[
		\sigma(p_1) = \sigma(p_2) = v
		\iff
		\vec{op_1} = \vec{op_2} = v
		\iff
		f(o,v) = p_1 = p_2
	]
\end{equation*}
可知\(\sigma\)是单射.
因为\begin{equation*}
	\sigma(q) = v
	\iff  % 映射\(\sigma\)的定义
	\vec{oq} = v
	\iff  % “连接两点的向量”的定义,\cref{theorem:仿射空间.仿射空间的减法1}
	f(o,v) = q,
\end{equation*}
% 又因为\(f(o,v) = f(o,v)\),
所以\(
	(\forall v \in V)
	[
		\sigma(f(o,v)) = v
	]
\),
可知\(\sigma\)是满射.
因此\(\sigma\)是从\(A\)到\(V\)的双射,
\(A\)与\(V\)等势.
\end{proof}
\end{proposition}

\begin{example}
%@see: 《基础代数(第二卷)》(席南华) P162 例4.2
设\(V\)是域\(F\)上的一个线性空间.
证明:\(V\)是一个仿射空间.
\begin{proof}
任意取定\(p,q \in V\).
对于任意\(x,y \in V\),
利用向量加法的结合律可得\begin{equation*}
	(p + x) + y
	= p + (x + y),
	\qquad
	p + 0
	= p.
\end{equation*}
同时,我们还有\begin{equation*}
	p + x = q
	\iff
	x = q - p \in V,
\end{equation*}
这就说明\(
	(\exists! x \in V)
	[
		f(p,x) = q
	]
\).
综上所述,\((V,+)\)是与\(V\)关联的仿射空间.
\end{proof}
\end{example}

\begin{example}\label{example:仿射空间.仿射空间与仿射几何的联系1}
%@see: 《基础代数(第二卷)》(席南华) P162 例4.3
设\(V\)是域\(F\)上的一个线性空间,\(W\)是\(V\)的一个子空间,
\(S\)是以向量\(\alpha \in V\)为代表、子空间\(W\)的一个陪集.
证明:\((S,+)\)是与\(W\)关联的仿射空间.
\begin{proof}
由题意有\(S = \alpha + W\).
任意取定\(p,q \in S\),
则存在\(w_1,w_2 \in W\)
使得\(p = \alpha + w_1, q = \alpha + w_2\).
由向量加法的封闭性可知\(p,q \in V\).
对于任意\(x,y \in W\),
利用向量加法的结合律可得\begin{equation*}
	(p + x) + y
	= p + (x + y),
	\qquad
	p + 0
	= p.
\end{equation*}
同时,我们还有\begin{equation*}
	p + x = q
	\iff
	x = q - p
	\iff
	x = (\alpha + w_2) - (\alpha + w_1)
	\iff
	x = w_2 - w_1 \in W,
\end{equation*}
这就说明\(
	(\exists! x \in W)
	[
		f(p,x) = q
	]
\).
综上所述,\((S,+)\)是与\(W\)关联的仿射空间.
\end{proof}
\end{example}

\subsection{仿射空间中的仿射映射与仿射同构}
\begin{definition}%\label{definition:仿射空间.仿射映射}
%@see: 《基础代数(第二卷)》(席南华) P163 定义4.4
设\(V,V'\)都是域\(F\)上的线性空间,
\(A,A'\)分别是与线性空间\(V,V'\)关联的仿射空间.
如果存在线性映射\(g \in \Hom(V,V')\)
使得映射\(f\colon A \to A'\)满足\begin{equation*}
%@see: 《基础代数(第二卷)》(席南华) P163 (4.1.1)
	(\forall p \in A)
	(\forall v \in V)
	[
		f(p + v)
		= f(p) + g(v)
	],
\end{equation*}
则称“\(f\)是从\(A\)到\(A'\)的一个\DefineConcept{仿射映射}”;
把\(g\)称为“仿射映射\(f\)的\DefineConcept{线性部分}”
或“仿射映射\(f\)的\DefineConcept{微分}”,
记作\(\dd{f}\).
\end{definition}

\begin{proposition}\label{theorem:仿射空间.仿射映射的微分的唯一性}
%@see: 《基础代数(第二卷)》(席南华) P163
设\(A,A'\)都是域\(F\)上的仿射空间,
\(p,q\)是\(A\)中两点,
\(f\)是从\(A\)到\(A'\)的一个仿射映射,
记\(p' \defeq f(p), q' \defeq f(q)\),
则\(
%@see: 《基础代数(第二卷)》(席南华) P163 (4.1.2)
	\vec{p'q'}
	= \dd{f}(\vec{pq})
\).
\end{proposition}
\begin{remark}
由\cref{theorem:仿射空间.仿射映射的微分的唯一性} 可以看出,
仿射映射的微分存在且唯一.
\end{remark}

\begin{corollary}
%@see: 《基础代数(第二卷)》(席南华) P164 命题4.7
设\(A,A'\)都是域\(F\)上的仿射空间,
\(o,p,q\)是\(A\)中三点,
\(f\)是从\(A\)到\(A'\)的一个仿射映射,
记\(o' \defeq f(o), p' \defeq f(p), q' \defeq f(q)\).
如果\(p \neq q\),
则\(
	\vec{op} \neq \vec{oq},
	\vec{o'p'} \neq \vec{o'q'}
\).
%TODO proof
\end{corollary}

\begin{definition}
设\(V,V'\)都是域\(F\)上的线性空间,
\(A,A'\)分别是与线性空间\(V,V'\)关联的仿射空间.
如果映射\(f\colon A \to A\)是从\(A\)到\(A\)的一个仿射映射,
则称“\(f\)是\(A\)上的一个\DefineConcept{仿射变换}”.
\end{definition}

\begin{definition}%\label{definition:仿射空间.平移变换}
%@see: 《基础代数(第二卷)》(席南华) P162
设\(A\)是与线性空间\(V\)关联的仿射空间,
向量\(\alpha \in V\),
把映射\begin{equation*}
	\tau_\alpha\colon A \to A,
	p \mapsto p + \alpha
\end{equation*}
称为“(仿射空间\(A\)上)由向量\(\alpha\)决定的\DefineConcept{平移变换}”.
\end{definition}

\begin{theorem}%\label{theorem:仿射空间.平移变换对乘法成群}
%@see: 《基础代数(第二卷)》(席南华) P162
设\(V\)是域\(F\)上的一个线性空间,
\(A\)是与线性空间\(V\)关联的仿射空间,
\(\alpha,\beta \in V\),
\(\tau_\alpha,\tau_\beta\)分别是\(A\)上由\(\alpha,\beta\)决定的平移变换,
\(I\)是\(A\)上的恒等映射,
则\begin{gather*}
	\tau_\alpha \tau_\beta = \tau_{\alpha + \beta}, \\
	\tau_{-\alpha} \tau_\alpha = I.
\end{gather*}
\end{theorem}
\begin{remark}
%@see: 《基础代数(第二卷)》(席南华) P162
上述定理说明:
两个平移的复合还是平移,
平移的逆映射也是平移.
全体平移对复合成群,它同构于线性空间\(V\)的加法群.
不难验证:
对于任意\(x,y \in F\),
如果令\begin{equation*}
	x \tau_\alpha + y \tau_\beta
	\defeq \tau_{x \alpha + y \beta},
\end{equation*}
则\(A\)的全体平移就成为一个线性空间(记作\(A^\sharp\)),它与\(V\)同构.
\end{remark}

\begin{definition}%\label{definition:仿射空间.伸缩变换}
%@see: 《基础代数(第二卷)》(席南华) P163 例4.6
设\(V\)是域\(F\)上的一个线性空间,
\(A\)是与线性空间\(V\)关联的仿射空间,
\(o\)是\(A\)中的一个点,
数\(\lambda \in F\),
把映射\begin{equation*}
	\delta_\lambda\colon A \to A,
	p \mapsto o + \lambda\vec{op}
\end{equation*}
称为“(仿射空间\(A\)上)以点\(o\)为中心、\(\lambda\)为伸缩率的\DefineConcept{伸缩变换}(dilatation)”.
\end{definition}

\begin{definition}%\label{definition:仿射空间.常值变换}
%@see: 《基础代数(第二卷)》(席南华) P163 例4.5(1)
设\(A\)是一个仿射空间,
点\(a \in A\).
把映射\(f\colon A \to A, x \mapsto a\)
称为“仿射空间\(A\)上的\DefineConcept{常值变换}”.
\end{definition}

\begin{proposition}%\label{theorem:仿射空间.常值变换的微分是零映射}
%@see: 《基础代数(第二卷)》(席南华) P163 例4.5(1)
仿射空间\(A\)上的常值变换的微分是零映射.
\end{proposition}

\begin{definition}%\label{definition:仿射空间.仿射同构}
%@see: 《基础代数(第二卷)》(席南华) P163 定义4.4
设\(V,V'\)都是域\(F\)上的线性空间,
\(A,A'\)分别是与线性空间\(V,V'\)关联的仿射空间.
如果从\(A\)到\(A'\)的一个仿射映射\(f\)是双射,
则称\(f\)是“从\(A\)到\(A'\)的一个\DefineConcept{仿射同构}”.
如果存在从\(A\)到\(A'\)的一个仿射同构,
则称“仿射空间\(A\)与\(A'\) \DefineConcept{同构}”,
记为\(A \Isomorphism A'\).
\end{definition}

\begin{definition}%\label{definition:仿射空间.仿射自同构}
%@see: 《基础代数(第二卷)》(席南华) P163 定义4.4
设\(V\)都是域\(F\)上的线性空间,
\(A\)分别是与线性空间\(V\)关联的仿射空间,
\(f\)是\(A\)上的一个仿射变换.
如果\(f\)是双射,
则称\(f\)是“\(A\)上的一个\DefineConcept{仿射自同构}”.
\end{definition}

\begin{proposition}%\label{theorem:仿射空间.仿射映射是双射当且仅当它的微分是双射}
%@see: 《基础代数(第二卷)》(席南华) P164 命题4.7
设\(V,V'\)都是域\(F\)上的线性空间,
\(A,A'\)分别是与线性空间\(V,V'\)关联的仿射空间,
映射\(f\colon A \to A'\),
则仿射映射\(f\)是双射,当且仅当\(f\)的微分\(\dd{f}\)是双射.
%TODO proof
\end{proposition}

\begin{proposition}\label{theorem:仿射空间.仿射空间与线性空间同构}
设\(V\)是域\(F\)上的线性空间,
\(A\)是与线性空间\(V\)关联的仿射空间,
则\(A\)与\(V\)同构.
\begin{proof}
在\cref{theorem:仿射空间.仿射空间与线性空间等势} 证明过程中,
我们已经给出了一个双射\(\sigma\colon A \to V, p \mapsto \vec{op}\),
故\(A \Isomorphism V\).
\end{proof}
\end{proposition}

\begin{proposition}%\label{theorem:仿射空间.两个仿射空间同构当且仅当它们的维数相同}
%@see: 《基础代数(第二卷)》(席南华) P164 定理4.8
设\(A\)和\(A'\)是域\(F\)上的两个仿射空间,
则\(A \Isomorphism A'\)当且仅当\(\dim A = \dim A'\).
\begin{proof}
设\(V,V'\)都是域\(F\)上的线性空间,
\(A,A'\)分别是与线性空间\(V,V'\)关联的仿射空间.
由\cref{theorem:仿射空间.仿射空间与线性空间同构} 可知,
\(A\)与\(V\)同构,\(A'\)与\(V'\)同构,
于是\(A\)与\(A'\)同构,当且仅当\(V\)与\(V'\)同构.
由\cref{theorem:线性空间的同构.线性空间同构的充分必要条件} 可知,
\(V\)与\(V'\)同构,当且仅当\(\dim V = \dim V'\).
因此\(A\)与\(A'\)同构,当且仅当\(\dim A = \dim A'\).
\end{proof}
\end{proposition}

\subsection{仿射空间中的坐标}
%@see: 《基础代数(第二卷)》(席南华) P164
把一个点\(o \in A\)与\(V\)的一个基\(\AutoTuple{e}{n}\)合在一起
称为“仿射空间\(A\)的一个\DefineConcept{坐标系}”
或“仿射空间\(A\)的一个\DefineConcept{标架}”,
记作\([o;\AutoTuple{e}{n}]\).
把\(o\)称为“标架\([o;\AutoTuple{e}{n}]\)的\DefineConcept{原点}”.
把向量\(\vec{op}\)在基\(\AutoTuple{e}{n}\)下的坐标
称为“点\(p\)在标架\([o;\AutoTuple{e}{n}]\)下的\DefineConcept{坐标}”.

只要在仿射空间\(A\)中取定一个点\(o\)(称之为原点),
就可以把点\(p\)与向量\(\vec{op}\)等同起来
(正如我们在\cref{theorem:仿射空间.仿射空间与线性空间等势} 证明过程中给出的双射\(\sigma\)那样),
然后把点与向量的加法看作向量的加法.
这种把点视同向量的做法,称为仿射空间的\DefineConcept{向量化}.
当然,这依赖原点的选取.
不过,只要取定\(n\)维仿射空间\(A\)的一个标架,
点与它的坐标的对应就是从仿射空间\(A\)到向量空间\(F^n\)的同构.

\begin{proposition}
%@see: 《基础代数(第二卷)》(席南华) P164 命题4.9
给定仿射空间\(A\)的一个标架\([o;\AutoTuple{e}{n}]\).
\begin{itemize}
	\item 如果\((\AutoTuple{x}{n})\)是点\(p\)的坐标,
	\((\AutoTuple{y}{n})\)是点\(q\)的坐标,
	那么向量\(\vec{pq}\)的坐标就是\((y_1-x_1,\dotsc,y_n-x_n)\).

	\item 给定向量\(v = a_1 e_1 + \dotsb + a_n e_n\),
	如果\((\AutoTuple{x}{n})\)是点\(p\)的坐标,
	那么点\(p + v\)的坐标是\((x_1+a_1,\dotsc,x_n+a_n)\).
\end{itemize}
\end{proposition}

\section{仿射几何}
\subsection{陪集的交与联}
%@see: 《高等代数与解析几何(第三版 下册)》(孟道骥) P435
设\(V\)是域\(F\)上的一个线性空间,
\(\{S_i\}_{i \in I}\)是一个有标集族.
如果\begin{equation*}
	(\forall i \in I)
	(\exists W \AlgebraSubstructure V)
	[
		\text{$S_i$是$W$的一个陪集}
	],
\end{equation*}
则称“\(\{S_i\}_{i \in I}\)是\(V\)的一个\DefineConcept{陪集族}”.

\begin{proposition}
设\(\{S_i\}_{i \in I}\)是域\(F\)上线性空间\(V\)的一个陪集族,
则\(V\)是包含\(\bigcup_{i \in I} S_i\)的一个陪集.
\begin{proof}
因为\(V\)是一个陪集,
并且\((\forall i \in I)[V \supseteq S_i]\),
所以\(V \supseteq \bigcup_{i \in I} S_i\).
\end{proof}
\end{proposition}

\begin{lemma}\label{theorem:线性空间.商空间.陪集.线性空间中全体陪集对联运算封闭}
%@see: 《高等代数与解析几何(第三版 下册)》(孟道骥) P438 定理10.2.1(1)
设\(V\)是域\(F\)上的一个线性空间,
\(W_1,W_2\)是\(V\)的两个子空间,
向量\(\alpha_1,\alpha_2 \in V\),
则\begin{equation*}
	\Gamma
	\defeq
	\bigcap\Set{
		X
		\given
		X \supseteq (\alpha_1 + W_1) \cup (\alpha_2 + W_2),
		\text{$X$是一个陪集}
	}
\end{equation*}
是陪集.
\begin{proof}
设陪集\(\beta + U\)满足\(\beta + U \supseteq (\alpha_1 + W_1) \cup (\alpha_2 + W_2)\).
由\(\beta + U \supseteq \alpha_1 + W_1\)
可得\(\alpha_1 - \beta \in U\)和\(W_1 \subseteq U\),
即\(\Span\{\alpha_1 - \beta\} + W_1 \subseteq U\).
由\(\beta + U \supseteq \alpha_2 + W_2\)
可得\(\alpha_2 - \beta \in U\)和\(W_2 \subseteq U\),
即\(\Span\{\alpha_2 - \beta\} + W_2 \subseteq U\).
%\cref{equation:集合论.集合代数公式6-8}
于是有\begin{equation*}
	\Span\{\alpha_1 - \beta,\alpha_2 - \beta\} + W_1 + W_2 \subseteq U.
\end{equation*}
令\(U_0 \defeq \Span\{\alpha_1 - \alpha_2\} + W_1 + W_2\).
由于\(
	\alpha_1 - \alpha_2
	= (\alpha_1 - \beta) - (\alpha_2 - \beta)
	\in \Span\{\alpha_1 - \beta,\alpha_2 - \beta\}
\),
所以\(
	\Span\{\alpha_1 - \alpha_2\}
	\subseteq
	\Span\{\alpha_1 - \beta,\alpha_2 - \beta\}
\),
从而有\(U \supseteq U_0\).
因此\(
	\alpha_1 + U_0
	\subseteq
	\alpha_1 + U
	= \beta + U
\),
这就说明\(\alpha_1 + U_0\)是包含\((\alpha_1 + W_1) \cup (\alpha_2 + W_2)\)的每一个陪集的子集,
于是由\cref{equation:集合论.集合代数公式6-9}
可知\begin{equation*}
	\alpha_1 + U_0
	\subseteq
	\Gamma.
\end{equation*}

由\(U_0 \supseteq W_1\)
可得\(\alpha_1 + W_1 \subseteq \alpha_1 + U_0\).
由\(U_0 \supseteq \Span\{\alpha_1 - \alpha_2\}\)
可得\(
	\alpha_2
	= \alpha_1 - (\alpha_1 - \alpha_2)
	\in \alpha_1 + \Span\{\alpha_1 - \alpha_2\}
	\subseteq \alpha_1 + U_0
\).
由\(\alpha_2 \in \alpha_1 + U_0\)
可得\(\alpha_2 + U_0 = \alpha_1 + U_0\).
由\(U_0 \supseteq W_2\)
可得\(
	\alpha_2 + W_2
	\subseteq
	\alpha_2 + U_0
	= \alpha_1 + U_0
\).
因此\begin{equation*}
	(\alpha_1 + W_1) \cup (\alpha_2 + W_2)
	\subseteq
	\alpha_1 + U_0.
\end{equation*}
由上可知\(\alpha_1 + U_0\)是包含\((\alpha_1 + W_1) \cup (\alpha_2 + W_2)\)的一个陪集,
即\begin{equation*}
	\alpha_1 + U_0
	\in
	\Set{
		X
		\given
		X \supseteq (\alpha_1 + W_1) \cup (\alpha_2 + W_2),
		\text{$X$是一个陪集}
	},
\end{equation*}
于是由\cref{equation:集合论.集合代数公式6-5}
可知\begin{equation*}
	\Gamma
	\subseteq
	\bigcap\{\alpha_1 + U_0\}
	=
	\alpha_1 + U_0.
\end{equation*}

综上所述,\(\Gamma = \alpha_1 + U_0\).
\end{proof}
\end{lemma}

\begin{definition}
%@see: 《高等代数与解析几何(第三版 下册)》(孟道骥) P436
设\(S_1\)和\(S_2\)是\(V\)中两个陪集.
把包含\(\mathscr{S} \defeq S_1 \cup S_2\)的所有陪集之交\begin{equation*}
	\bigcap\Set{
		X
		\given
		X \supseteq \mathscr{S},
		\text{$X$是一个陪集}
	}
\end{equation*}
称为“陪集\(S_1\)和\(S_2\)的\DefineConcept{联}”,
记作\(S_1 \vee S_2\).
\end{definition}

\begin{definition}
%@see: 《高等代数与解析几何(第三版 下册)》(孟道骥) P436 定义10.1.1
设\(\{S_i\}_{i \in I}\)是\(V\)中一个陪集族.
把包含\(\mathscr{S} \defeq \bigcup_{i \in I} S_i\)的所有陪集之交\begin{equation*}
	\bigcap\Set{
		X
		\given
		X \supseteq \mathscr{S},
		\text{$X$是一个陪集}
	}
\end{equation*}
称为“陪集族\(\{S_i\}_{i \in I}\)的\DefineConcept{联}”,
记作\(\bigvee_{i \in I} S_i\).
\end{definition}

\cref{theorem:线性空间.商空间.陪集.线性空间中全体陪集对联运算封闭} 说明:
线性空间\(V\)中全体陪集\(
	\mathscr{C}
	\defeq
	\Set{
		\alpha + W
		\given
		\alpha \in V,
		W \AlgebraSubstructure V
	}
\)对“陪集的联”运算封闭.

\cref{theorem:线性空间.商空间.陪集.线性空间中全体陪集对联运算封闭} 还说明:\begin{align}
	\bigvee_{i=1}^2 (\alpha_i + W_i)
	&= \alpha_1 + \Span\{\alpha_1 - \alpha_2\} + W_1 + W_2,
		\label{equation:仿射几何.两个陪集的联}
		\\
	\bigvee_{i=1}^3 (\alpha_i + W_i)
	&= \alpha_1 + \Span\{\alpha_1 - \alpha_2,\alpha_1 - \alpha_3\} + W_1 + W_2 + W_3,
		% \label{equation:仿射几何.三个陪集的联}
		\\
	\bigvee_{i=1}^n (\alpha_i + W_i)
	&= \alpha_1 + \Span\{\alpha_1 - \alpha_2,\dotsc,\alpha_1 - \alpha_n\}
		+ \sum_{i=1}^n W_i.
		% \label{equation:仿射几何.n个陪集的联}
\end{align}

\begin{property}
陪集的联运算适合交换律,即对于任意陪集\(x,y\),成立\begin{equation}
	x \vee y = y \vee z.
\end{equation}
\begin{proof}
显然\(
	x \vee y
	= \bigcap\Set{
		c \in \mathscr{C}
		\given
		c \supseteq x \cup y
	}
	= \bigcap\Set{
		c \in \mathscr{C}
		\given
		c \supseteq y \cup x
	}
	= y \vee x
\).
\end{proof}
\end{property}

\begin{property}
%@see: 《高等代数与解析几何(第三版 下册)》(孟道骥) P436
陪集的联运算适合结合律,即对于任意陪集\(x,y,z\),成立\begin{equation}
	(x \vee y) \vee z = x \vee (y \vee z).
\end{equation}
\begin{proof}
任取\(x,y,z \in \mathscr{C}\).
由定义可知\(
	x \vee y
	\supseteq
	x \cup y
\),
于是,对于任意\(c \in \mathscr{C}\),
当\(c \supseteq (x \vee y) \cup z\)时,
必有\(c \supseteq (x \cup y) \cup z\).
因此\begin{equation*}
	(x \vee y) \vee z
	= \bigcap\Set{
		c \in \mathscr{C}
		\given
		c \supseteq (x \vee y) \cup z
	}
	\supseteq
	\bigcap\Set{
		c \in \mathscr{C}
		\given
		c \supseteq (x \cup y) \cup z
	}.
\end{equation*}
与此同时,由\(x \cup y \subseteq (x \cup y) \cup z\)可知,
包含\((x \cup y) \cup z\)的任意一个陪集\(c\)必定包含\(x \cup y\),
由定义可知这样的\(c\)必定包含\(x \vee y\).
于是当\(c \supseteq (x \cup y) \cup z\)时,
必有\(c \supseteq (x \vee y) \cup z\).
因此\begin{equation*}
	(x \vee y) \vee z
	= \bigcap\Set{
		c \in \mathscr{C}
		\given
		c \supseteq (x \vee y) \cup z
	}
	\subseteq
	\bigcap\Set{
		c \in \mathscr{C}
		\given
		c \supseteq (x \cup y) \cup z
	}.
\end{equation*}
综上所述\begin{equation*}
	(x \vee y) \vee z
	= \bigcap\Set{
		c \in \mathscr{C}
		\given
		c \supseteq (x \cup y) \cup z
	}.
\end{equation*}
利用对称性可得\begin{equation*}
	x \vee (y \vee z)
	= \bigcap\Set{
		c \in \mathscr{C}
		\given
		c \supseteq x \cup (y \cup z)
	},
\end{equation*}
于是\((x \vee y) \vee z = x \vee (y \vee z)\).
\end{proof}
\end{property}

通过下面的例子可以看出“陪集的联”这个运算的几何意义.
\begin{example}
%@see: 《高等代数与解析几何(第三版 下册)》(孟道骥) P436 例10.1
设\(P,Q\)是几何空间\(V \defeq \mathbb{R}^3\)中的两个点,它们的坐标向量分别是\(\alpha,\beta\).
显然\(\{\alpha\}\)是以\(\alpha\)为代表的零空间的陪集,
\(\{\beta\}\)是以\(\beta\)为代表的零空间的陪集.
假设\(V\)的子空间\(W\)满足\(
	\{\alpha\} \vee \{\beta\}
	= \alpha + W
	= \beta + W
\),
那么\(\alpha - \beta \in W\).
显然\(U \defeq \Span\{\alpha - \beta\}\)是\(V\)的一个子空间,
并且\(\alpha + U = \beta + U\)就是包含\(\alpha\)和\(\beta\)的一个陪集,
因此\(W = U\).
这表明\(\{\alpha\} \vee \{\beta\}\)是过\(P,Q\)两点的直线.
\end{example}

\begin{example}
%@see: 《高等代数与解析几何(第三版 下册)》(孟道骥) P436 例10.2
%@see: https://math.stackexchange.com/q/5086322/591741
在几何空间\(V \defeq \mathbb{R}^3\)中,
取两个陪集\(
	S_1 \defeq \alpha + 0,
	S_2 \defeq \beta + U
\),
其中\(U\)是\(V\)的一个1维子空间,
且向量\(\alpha,\beta \in V\),
但\(\alpha \notin S_2\).
显然\(S_1\)是一个点,\(S_2\)是一条直线.
假设\(V\)的子空间\(W_1\)满足\(
	S_1 \vee S_2
	= \alpha + W_1
	= \beta + W_1
\),
那么\(\alpha - \beta \in W_1\)
且\(U \subseteq W_1\),
于是\(
	W_2
	\allowbreak
	\defeq \Span\{\alpha - \beta\} + U
	\allowbreak
	\subseteq W_1
\).
显然\(\alpha + W_2 = \beta + W_2\)是包含\(S_1\)和\(S_2\)的一个陪集,
因此\(W_2 = W_1\).
由于\(\alpha \notin S_2\),
所以\(\alpha - \beta \notin U\),
因此\(\dim W_1 = 2\).
这表明\(S_1 \vee S_2\)是过点\(S_1\)与直线\(S_2\)的平面.
\end{example}

\begin{example}
%@see: 《高等代数与解析几何(第三版 下册)》(孟道骥) P437 习题 2.
设\(V\)是域\(F\)上的一个线性空间,\(W\)是\(V\)的一个子空间,
向量\(\alpha \in W\).
证明:\((\alpha + W) \vee \{0\} = \Span\{\alpha\} + W\).
%TODO proof
\end{example}

\subsection{陪集的维数}
\begin{definition}
%@see: 《高等代数与解析几何(第三版 下册)》(孟道骥) P437 定义10.1.2
设\(V\)是域\(F\)上的一个线性空间,\(W\)是\(V\)的一个子空间,
向量\(\alpha \in V\).
定义:\begin{equation}
	\dim(\alpha + W)
	\defeq
	\dim W,
\end{equation}
称之为“陪集\(\alpha + W\)的\DefineConcept{维数}(the \emph{dimension} of \(\alpha + W\))”.
\end{definition}

\begin{example}
零陪集(即域\(F\)上线性空间\(V\)中以零向量\(0\)为代表的零子空间\(0\)的陪集\(0 \defeq 0+0\))的
维数为\(\dim 0 = 0\).
\end{example}

\subsection{仿射几何}
\begin{definition}
%@see: 《高等代数与解析几何(第三版 下册)》(孟道骥) P437 定义10.1.3
设\(V\)是域\(F\)上的一个线性空间,
\(S\)是\(V\)中一个陪集.
把\(S\)中全体陪集\begin{equation*}
	\Set{
		\alpha + W
		\given
		\alpha \in V,
		W \AlgebraSubstructure V,
		\alpha + W \subseteq S
	}
\end{equation*}
称为“(域\(F\)上线性空间\(V\)中)陪集\(S\)上的\DefineConcept{仿射几何}(affine geometry)”,
记作\(\mathcal{A}(S)\);
把陪集\(S\)的维数\(\dim S\)
称为“\(\mathcal{A}(S)\)的\DefineConcept{维数}”,
记作\(\dim\mathcal{A}(S)\);
将\(\mathcal{A}(S)\)中\(0\)维元素称为\DefineConcept{点};
将\(\mathcal{A}(S)\)中\(1\)维元素称为\DefineConcept{直线};
将\(\mathcal{A}(S)\)中\(2\)维元素称为\DefineConcept{平面};
将\(\mathcal{A}(S)\)中\(\dim S-1\)维元素称为\DefineConcept{超平面}.
\end{definition}
\begin{remark}
应该注意到,“线性空间\(V\)中某个陪集的仿射几何”
与“线性空间\(V\)对于某个子空间的商空间”是完全不同的两个概念.
\end{remark}

\begin{definition}
%@see: 《高等代数与解析几何(第三版 下册)》(孟道骥) P437 定义10.1.3
设\(V\)是域\(F\)上的一个线性空间,
\(R\)和\(S\)是\(V\)中两个陪集.
如果\(R \subseteq S\),
则称“\(\mathcal{A}(R)\)是\(\mathcal{A}(S)\)的\DefineConcept{子几何}”.
\end{definition}

\subsection{陪集之间的平行关系}
\begin{definition}
%@see: 《高等代数与解析几何(第三版 下册)》(孟道骥) P437 定义10.2.1
设\(V\)是域\(F\)上的一个线性空间,
\(U\)和\(W\)是\(V\)的两个子空间,
向量\(\alpha,\beta \in V\).
如果\(U \subseteq W\)或\(W \subseteq U\),
则称“陪集\(\alpha + U\)与\(\beta + W\) \DefineConcept{平行}”,
记作\((\alpha + U) \parallel (\beta + W)\);
反之,称“陪集\(\alpha + U\)与\(\beta + W\) \DefineConcept{不平行}”,
记作\((\alpha + U) \nparallel (\beta + W)\).
\end{definition}

\begin{property}
%@see: 《高等代数与解析几何(第三版 下册)》(孟道骥) P437
陪集之间的平行关系具有自反性,
即对于线性空间\(V\)中任意一个陪集\(x\),
有\(x \parallel x\).
\end{property}

\begin{property}
%@see: 《高等代数与解析几何(第三版 下册)》(孟道骥) P437
陪集之间的平行关系具有对称性,
即对于线性空间\(V\)中任意两个陪集\(x\)和\(y\),
当\(x \parallel y\)时,必有\(y \parallel x\).
\end{property}

\begin{property}
%@see: 《高等代数与解析几何(第三版 下册)》(孟道骥) P438
陪集之间的平行关系不具有传递性.
\begin{proof}
设\(V \defeq \mathbb{R}^3\).
取\begin{gather*}
	\alpha_1 \defeq (0,0,1)^T,
	\qquad
	\alpha_2 \defeq (0,0,2)^T, \\
	W_1 \defeq \Span\{(1,0,0)^T\},
	\qquad
	W_2 \defeq \Span\{(0,1,0)^T\},
	\qquad
	W \defeq W_1 + W_2, \\
	S_1 \defeq \alpha_1 + W_1,
	\qquad
	S_2 \defeq \alpha_2 + W_2,
	\qquad
	S_3 \defeq W,
\end{gather*}
则有\(
	S_1 \parallel S_3,
	S_2 \parallel S_3
\),
但是\(S_1 \nparallel S_2\).
\end{proof}
\end{property}

\subsection{仿射性质}
\begin{theorem}\label{theorem:仿射几何.陪集的交非空的充分必要条件1}
%@see: 《高等代数与解析几何(第三版 下册)》(孟道骥) P438 定理10.2.1(2)
设\(V\)是域\(F\)上的一个线性空间,
\(U\)和\(W\)是\(V\)的两个子空间,
向量\(\alpha,\beta \in V\),
则\((\alpha + U) \cap (\beta + W) \neq \emptyset\)
当且仅当\(\alpha - \beta \in U + W\).
\begin{proof}
\((\alpha + U) \cap (\beta + W) \neq \emptyset\)
当且仅当存在\(\gamma \in U\)和\(\delta \in W\),
使得\(\alpha + \gamma = \beta + \delta\),
即\(\beta - \alpha = \gamma - \delta \in U + W\).
\end{proof}
\end{theorem}

\begin{theorem}\label{theorem:仿射几何.陪集的交为空的充分必要条件1}
%@see: 《高等代数与解析几何(第三版 下册)》(孟道骥) P438 定理10.2.1(3)
设\(V\)是域\(F\)上的一个线性空间,
\(U\)和\(W\)是\(V\)的两个子空间,
向量\(\alpha,\beta \in V\),
则\((\alpha + U) \cap (\beta + W) = \emptyset\)
当且仅当\(\dim((\alpha + U) \vee (\beta + W)) = \dim(U + W) + 1\).
\begin{proof}
由\cref{theorem:仿射几何.陪集的交非空的充分必要条件1} 可知
\((\alpha + U) \cap (\beta + W) = \emptyset\)
当且仅当\(\alpha - \beta \notin U + W\),
即\(\Span\{\alpha - \beta\}\)与\(U + W\)的交集是零子空间.
再由 \hyperref[equation:仿射几何.两个陪集的联]{\(
	(\alpha + U) \vee (\beta + W)
	= \alpha + \Span\{\alpha - \beta\} + U + W
\)}、
陪集的维数的定义\(
	\dim(\alpha + \Span\{\alpha - \beta\} + U + W)
	= \dim(\Span\{\alpha - \beta\} + U + W)
\)
以及\hyperref[theorem:线性空间.子空间.子空间的维数公式]{子空间的维数公式}\begin{align*}
	\dim(\Span\{\alpha - \beta\} + U + W)
	&= \dim\Span\{\alpha - \beta\} + \dim(U + W) \\
	&\hspace{20pt}
		- \dim(\Span\{\alpha - \beta\} \cap (U + W)) \\
	&= \dim\Span\{\alpha - \beta\} + \dim(U + W) \\
	&= 1 + \dim(U + W)
\end{align*}
可知\(
	\dim((\alpha + U) \vee (\beta + W))
	= \dim(U + W) + 1
\).
\end{proof}
\end{theorem}

\begin{theorem}\label{theorem:商空间.陪集之间成立包含关系的必要条件2}
%@see: 《高等代数与解析几何(第三版 下册)》(孟道骥) P439 定理10.2.2(1)
设\(V\)是域\(F\)上的一个线性空间,
\(U\)和\(W\)是\(V\)的两个子空间,
向量\(\alpha,\beta \in V\).
如果\(\alpha + U \subseteq \beta + W\),
则\(\dim(\alpha + U) \leq \dim(\beta + W)\).
\begin{proof}
假设\(\alpha + U \subseteq \beta + W\).
由\cref{theorem:商空间.陪集之间成立包含关系的必要条件1}
可知\(U \subseteq W\).
由\cref{theorem:子空间.同一线性空间的两个子空间成立包含关系的必要条件}
可知\(U\)是\(W\)的一个子空间.
由\cref{theorem:线性空间.线性空间及其子空间的维数序关系}
可知\(\dim U \leq \dim W\).
于是\(
	\dim(\alpha + U)
	= \dim U
	\leq \dim W
	= \dim(\beta + W)
\).
\end{proof}
%\cref{theorem:向量空间.两个非零子空间的关系1}
\end{theorem}

\begin{theorem}
%@see: 《高等代数与解析几何(第三版 下册)》(孟道骥) P439 定理10.2.2(1)
设\(V\)是域\(F\)上的一个线性空间,
\(U\)和\(W\)是\(V\)的两个子空间,
向量\(\alpha,\beta \in V\).
如果\(\alpha + U \subseteq \beta + W\),
且\(\dim(\alpha + U) = \dim(\beta + W)\),
则\(\alpha + U = \beta + W\).
\begin{proof}
假设\(\alpha + U \subseteq \beta + W\),
那么从\cref{theorem:商空间.陪集之间成立包含关系的必要条件2} 的证明过程中可以看出
\(U\)是\(W\)的一个子空间.
假设\(\dim(\alpha + U) = \dim(\beta + W)\),
那么由\(
	\dim(\alpha + U) = \dim U,
	\dim(\beta + W) = \dim W
\)
可知\(\dim U = \dim W\).
再由\cref{theorem:子空间.与线性空间维数相等的子空间}
可知\(U = W\).
于是\(\alpha + U = \alpha + W \subseteq \beta + W\).
由\cref{theorem:商空间.陪集之间成立包含关系的必要条件1}
可知\(\alpha - \beta \in W\).
最后由\cref{theorem:商空间.陪集之间成立包含关系的等价条件1}
可知\(\alpha + U = \beta + W\).
\end{proof}
%\cref{theorem:向量空间.两个非零子空间的关系2}
\end{theorem}

\begin{theorem}
%@see: 《高等代数与解析几何(第三版 下册)》(孟道骥) P439 定理10.2.2(2)
设\(V\)是域\(F\)上的一个线性空间,
\(U\)和\(W\)是\(V\)的两个子空间,
向量\(\alpha,\beta \in V\).
如果\((\alpha + U) \cap (\beta + W) \neq \emptyset\),
则\begin{equation*}
	\dim((\alpha + U) \vee (\beta + W))
	+ \dim((\alpha + U) \cap (\beta + W))
	= \dim(\alpha + U)
	+ \dim(\beta + W).
\end{equation*}
\begin{proof}
假设\((\alpha + U) \cap (\beta + W) \neq \emptyset\).
任意取定\(\gamma \in (\alpha + U) \cap (\beta + W)\),
那么有\(\gamma \in \alpha + U\)和\(\gamma \in \beta + W\)同时成立.
由\cref{theorem:商空间.陪集的元素的性质,theorem:商空间.子空间相同的两个陪集相等的等价条件}
可知\(\alpha + U = \gamma + U\)且\(\beta + W = \gamma + W\).
由\cref{theorem:商空间.代表相同的两个陪集的交}
可知\begin{equation*}
	(\alpha + U) \cap (\beta + W)
	= (\gamma + U) \cap (\gamma + W) \\
	= \gamma + (U \cap W).
\end{equation*}
由\cref{equation:仿射几何.两个陪集的联}
可知\begin{equation*}
	(\alpha + U) \vee (\beta + W)
	= (\gamma + U) \vee (\gamma + W)
	= \gamma + (U + W).
\end{equation*}
最后由\hyperref[theorem:线性空间.子空间.子空间的维数公式]{子空间的维数公式}可知\begin{align*}
	&\hspace{-20pt}
	\dim((\alpha + U) \vee (\beta + W))
		+ \dim((\alpha + U) \cap (\beta + W)) \\
	&= \dim(\gamma + (U \cap W))
		+ \dim(\gamma + (U + W)) \\
	&= \dim(U + W)
		+ \dim(U \cap W)
	= \dim U + \dim W \\
	&= \dim(\alpha + U) + \dim(\beta + W).
	\qedhere
\end{align*}
\end{proof}
\end{theorem}

\begin{theorem}
%@see: 《高等代数与解析几何(第三版 下册)》(孟道骥) P439 定理10.2.2(3)
设\(V\)是域\(F\)上的一个线性空间,
\(U\)和\(W\)是\(V\)的两个子空间,
向量\(\alpha,\beta \in V\).
如果\((\alpha + U) \cap (\beta + W) \neq \emptyset\),
则\((\alpha + U) \parallel (\beta + W)\)
当且仅当\(\alpha + U \subseteq \beta + W\)
或\(\alpha + U \supseteq \beta + W\).
\begin{proof}
假设\((\alpha + U) \cap (\beta + W) \neq \emptyset\).
任意取定\(\gamma \in (\alpha + U) \cap (\beta + W)\),
那么有\(\gamma \in \alpha + U\)和\(\gamma \in \beta + W\)同时成立.
由\cref{theorem:商空间.陪集的元素的性质,theorem:商空间.子空间相同的两个陪集相等的等价条件}
可知\(\alpha + U = \gamma + U\)且\(\beta + W = \gamma + W\).
这时\begin{align*}
	&\hspace{-20pt}
	(\alpha + U) \parallel (\beta + W)
	\iff (U \subseteq W) \lor (U \supseteq W)
		\tag{陪集平行的定义} \\
	&\iff (\gamma + U \subseteq \gamma + W) \lor (\gamma + U \supseteq \gamma + W)
		\tag{\cref{theorem:商空间.陪集之间成立包含关系的等价条件1}} \\
	&\iff (\alpha + U \subseteq \beta + W) \lor (\alpha + U \supseteq \beta + W).
	\tag*\qedhere
\end{align*}
\end{proof}
\end{theorem}

\begin{theorem}
%@see: 《高等代数与解析几何(第三版 下册)》(孟道骥) P439 定理10.2.2(3)
设\(V\)是域\(F\)上的一个线性空间,
\(U\)和\(W\)是\(V\)的两个子空间,
向量\(\alpha,\beta \in V\).
如果\((\alpha + U) \cap (\beta + W) = \emptyset\),
则\((\alpha + U) \parallel (\beta + W)\)
当且仅当\begin{equation*}
	\dim((\alpha + U) \vee (\beta + W))
	= \max\{\dim(\alpha + U),\dim(\beta + W)\} + 1.
\end{equation*}
\begin{proof}
假设\((\alpha + U) \cap (\beta + W) = \emptyset\),
那么由\cref{theorem:仿射几何.陪集的交为空的充分必要条件1}
可知\begin{equation*}
	\dim((\alpha + U) \vee (\beta + W)) = \dim(U + W) + 1.
\end{equation*}
由\cref{theorem:线性空间.子空间.子空间的维数公式}
可知\begin{equation*}
	\dim(U + W)
	=  \dim U + \dim W - \dim(U \cap W).
\end{equation*}
于是\begin{align*}
	&\hspace{-20pt}
	\dim((\alpha + U) \vee (\beta + W))
	= \max\{\dim(\alpha + U),\dim(\beta + W)\} + 1 \\
	&\iff
	\dim(U + W)
	% 比较上式和待证的等式可得
	= \max\{\dim(\alpha + U),\dim(\beta + W)\} \\
	&\iff
	\dim(U + W)
	= \max\{\dim U,\dim W\}
		\tag{陪集的维数的定义} \\
	&\iff
	(U \subseteq W) \lor (U \supseteq W)
		\tag{\cref{theorem:子空间.子空间的和的维数等于最大子空间维数的等价条件}} \\
	&\iff	% 陪集平行的定义
	(\alpha + U) \parallel (\beta + W).
	\tag*\qedhere
\end{align*}
\end{proof}
\end{theorem}
\begin{remark}
%@see: 《高等代数与解析几何(第三版 下册)》(孟道骥) P440 推论1
%@see: 《高等代数与解析几何(第三版 下册)》(孟道骥) P440 推论2
从上述定理可以看出:
在2维仿射几何中,
两个不同点的联是一条直线,
两条非平行直线的交是一个点;
在3维仿射几何中,
两个不同点的联是一条直线,
两个非平行平面的交是一条直线,
交于一点的两条直线的联是一个平面,
两条共面的非平行直线的交是一个点,
两条不同的平行直线的联是一个平面,
一个点与不包含它的一条直线的联是一个平面,
一个平面与一个不平行于它的直线的交是一个点.
\end{remark}

\begin{example}
%@see: 《高等代数与解析几何(第三版 下册)》(孟道骥) P441 习题 2.(1)
设\(x\)和\(y\)是3维仿射几何中两条异面直线.
试证:存在唯一的平面\(p\)使得\(x \subseteq p\)且\(y \parallel p\).
%TODO proof
\end{example}

\begin{example}
%@see: 《高等代数与解析几何(第三版 下册)》(孟道骥) P441 习题 2.(2)
设\(x\)和\(y\)是3维仿射几何中两条异面直线.
试证:存在唯一的平面\(p\)使得\(y \subseteq p\)且\(x \parallel p\).
%TODO proof
\end{example}

\begin{example}
%@see: 《高等代数与解析几何(第三版 下册)》(孟道骥) P441 习题 2.(3)
设\(x\)和\(y\)是3维仿射几何中两条异面直线,
平面\(p_x\)满足\(x \subseteq p_x\)且\(y \parallel p_x\),
平面\(p_y\)满足\(y \subseteq p_y\)且\(x \parallel p_y\).
试证:\(p_x \parallel p_y\).
%TODO proof
\end{example}

\subsection{仿射同构}
如同线性空间一样,在仿射几何之间也有同构关系.
两个同构的仿射几何具有相同的性质与结构,
在某种程度上,可以视为同一个仿射几何.

\begin{definition}
%@see: 《高等代数与解析几何(第三版 下册)》(孟道骥) P441 定义10.3.1
设\(A\)与\(A'\)都是仿射几何,
\(\sigma\)是从\(A\)到\(A'\)的一个双射.
如果\begin{equation*}
	(\forall S_1,S_2 \in A)
	[
		\sigma(S_1) \subseteq \sigma(S_2)
		\iff
		S_1 \subseteq S_2
	],
\end{equation*}
那么称“\(\sigma\)是从\(A\)到\(A'\)的一个\DefineConcept{同构}(isomorphism)”
“\(A\)到\(A'\)同构(\(A\) is \emph{isomorphic} to \(A'\))”,
记作\(A \Isomorphism A'\).
\end{definition}

\begin{definition}
设\(A\)都是仿射几何,
\(\sigma\)是从\(A\)到\(A\)的一个双射.
如果\begin{equation*}
	(\forall S_1,S_2 \in A)
	[
		\sigma(S_1) \subseteq \sigma(S_2)
		\iff
		S_1 \subseteq S_2
	],
\end{equation*}
那么称“\(\sigma\)是\(A\)上的一个\DefineConcept{自同构}(automorphism)”.
\end{definition}

%@see: 《高等代数与解析几何(第三版 下册)》(孟道骥) P441
显然,仿射几何的同构关系是一个等价关系.

\begin{property}
%@see: 《高等代数与解析几何(第三版 下册)》(孟道骥) P441 性质1
仿射几何的同构保持交的运算,即\begin{equation*}
	\sigma\left( \bigcap_i S_i \right)
	= \bigcap_i \sigma(S_i).
\end{equation*}
\end{property}

\begin{property}
%@see: 《高等代数与解析几何(第三版 下册)》(孟道骥) P441 性质2
仿射几何的同构保持联的运算,即\begin{equation*}
	\sigma\left( \bigvee_i S_i \right)
	= \bigcap_i \sigma(S_i).
\end{equation*}
\end{property}

\begin{property}
%@see: 《高等代数与解析几何(第三版 下册)》(孟道骥) P441 性质3
仿射几何的同构保持维数不变,即\begin{equation*}
	\dim\sigma(S) = \dim S.
\end{equation*}
\begin{proof}
给定陪集\(S\),
我们可以构造一个陪集族\(\{S_n\}\),
使得\begin{equation}
	S \supseteq S_1 \supseteq S_2 \supseteq \dotsb \supseteq S_k,
\end{equation}
其中\(\dim S_i = \dim S - i\).
显然\(k = \dim S\),并且\begin{equation*}
	\sigma(S) \supseteq \sigma(S_1) \supseteq \sigma(S_2) \supseteq \dotsb \supseteq \sigma(S_k).
\end{equation*}
因而由\(\dim \sigma(S_i) > \dim \sigma(S_{i+1})\)可知\(\dim S \leq \dim \sigma(S)\).
反过来,由于\(\sigma^{-1}\)是从\(A'\)到\(A\)的同构,
所以\(
	\dim \sigma(S) \leq \dim \sigma^{-1}(\sigma(S)) = \dim S.
\)
\end{proof}
\end{property}

\begin{property}
%@see: 《高等代数与解析几何(第三版 下册)》(孟道骥) P441 性质4
仿射几何的同构保持平行关系,即\begin{equation*}
	\sigma(S_1) \parallel \sigma(S_2)
	\iff
	S_1 \parallel S_2.
\end{equation*}
\end{property}

\begin{theorem}
%@see: 《高等代数与解析几何(第三版 下册)》(孟道骥) P442 定理10.3.1
设\(V_1\)和\(V_2\)是域\(F\)上的两个线性空间,
\(S_1\)是\(V_1\)中一个陪集,
\(S_2\)是\(V_2\)中一个陪集,
\(\mathcal{A}(S_1)\)是\(S_1\)上的仿射几何,
\(\mathcal{A}(S_2)\)是\(S_2\)上的仿射几何,
则\(\mathcal{A}(S_1) \Isomorphism \mathcal{A}(S_2)\)的充分必要条件是
\(\dim\mathcal{A}(S_1) = \dim\mathcal{A}(S_2)\).
%TODO proof
\end{theorem}

\begin{example}
%@see: 《高等代数与解析几何(第三版 下册)》(孟道骥) P443 例10.4
设\(V\)是域\(F\)上的一个线性空间,
向量\(\alpha \in V\),
变换\(
	\tau_\alpha\colon \mathcal{A}(V) \to \mathcal{A}(V),
	x \mapsto \alpha + x
\)是由\(\alpha\)决定的平移.
%TODO 定义“由某个向量决定的平移”
证明:\(\tau_\alpha\)是\(\mathcal{A}(V)\)上的自同构.
\begin{proof}
设\(W_1,W_2\)是\(V\)的两个子空间,
向量\(\beta_1,\beta_2 \in V\),
则\begin{align*}
	\beta_1 + W_1 \subseteq \beta_2 + W_2
	&\iff \beta_1 - \beta_2 \in W_2, W_1 \subseteq W_2
		\tag{\cref{theorem:商空间.陪集之间成立包含关系的等价条件1}} \\
	&\iff (\alpha + \beta_1) - (\alpha + \beta_2) \in W_2, W_1 \subseteq W_2 \\
	&\iff \alpha + \beta_1 + W_1 \subseteq \alpha + \beta_2 + W_2 \\
	&\iff \tau_\alpha(\beta_1 + W_1) \subseteq \tau_\alpha(\beta_2 + W_2).
		\tag{变换\(\tau_\alpha\)的定义}
\end{align*}
因此\(\tau_\alpha\)是\(\mathcal{A}(V)\)上的自同构.
\end{proof}
\end{example}

\begin{example}
%@see: 《高等代数与解析几何(第三版 下册)》(孟道骥) P443 例10.5
设\(V\)是域\(F\)上的一个线性空间.
证明:对于\(V\)上的任意一个可逆线性变换\(f\),
变换\begin{equation*}
	\hat{f}\colon \mathcal{A}(V) \to \mathcal{A}(V),
	\alpha + W \mapsto f(\alpha) + f(W)
\end{equation*}
一定是\(\mathcal{A}(V)\)上的自同构.
\begin{proof}
设\(W_1,W_2\)是\(V\)的两个子空间,
向量\(\alpha_1,\alpha_2 \in V\),
则\begin{align*}
	&\hspace{-20pt}
		\alpha_1 + W_1 \subseteq \alpha_2 + W_2 \\
	&\iff \alpha_1 - \alpha_2 \in W_2, W_1 \subseteq W_2
		\tag{\cref{theorem:商空间.陪集之间成立包含关系的等价条件1}} \\
	&\iff f(\alpha_1 - \alpha_2) \in f(W_2),
			f(W_1) \subseteq f(W_2)
		\tag{\cref{example:线性映射.可逆线性变换保持向量与子空间的所属关系,example:线性映射.可逆线性变换保持子空间之间的包含关系}} \\
	&\iff f(\alpha_1) - f(\alpha_2) \in f(W_2),
			f(W_1) \subseteq f(W_2)
		\tag{\hyperref[definition:线性映射.线性映射]{线性映射的可加性}} \\
	&\iff f(\alpha_1) + f(W_1) \subseteq f(\alpha_2) + f(W_2)
		\tag{\cref{theorem:商空间.陪集之间成立包含关系的等价条件1}} \\
	&\iff \hat{f}(\alpha_1 + W_1) \subseteq \hat{f}(\alpha_2 + W_2).
		\tag{变换\(\hat{f}\)的定义}
\end{align*}
因此\(\hat{f}\)是\(\mathcal{A}(V)\)上的自同构.
\end{proof}
\end{example}

\subsection{仿射变换}
\begin{definition}
%@see: 《高等代数与解析几何(第三版 下册)》(孟道骥) P443 定义10.3.2
设\(V,V'\)都是域\(F\)上的线性空间.
对于\(V\)的任意一个子空间\(W\),
\(V'\)的任意一个子空间\(W'\),
任意向量\(\alpha \in V\),
任意向量\(\alpha' \in V'\),
以及从\(W\)到\(W'\)的任意一个同构\(\sigma\),
把\begin{equation*}
	\tau_{\alpha'} ~ \sigma ~ \tau_{-\alpha}
\end{equation*}
称为“从仿射几何\(\mathcal{A}(\alpha + W)\)
到仿射几何\(\mathcal{A}(\alpha' + W')\)的一个\DefineConcept{仿射变换}”.
\end{definition}

\begin{definition}
设\(V\)都是域\(F\)上的线性空间.
对于\(V\)的任意一个子空间\(W\),
任意向量\(\alpha \in V\),
以及\(W\)上的任意一个自同构\(\sigma\),
把\begin{equation*}
	\tau_\alpha ~ \sigma ~ \tau_{-\alpha}
\end{equation*}
称为“仿射几何\(\mathcal{A}(\alpha + W)\)上的一个\DefineConcept{仿射变换}”.
\end{definition}

\begin{property}
%@see: 《高等代数与解析几何(第三版 下册)》(孟道骥) P443
设\(V,V'\)都是域\(F\)上的线性空间,
\(W\)是\(V\)的一个子空间,
\(W'\)是\(V'\)的一个子空间,
向量\(\alpha \in V\),
向量\(\alpha' \in V'\),
\(\sigma\)是从\(W\)到\(W'\)的一个同构,
记\(\hat{\sigma} \defeq \tau_{\alpha'} \sigma \tau_{-\alpha}\),
那么对于任意向量\(\beta \in \alpha + W\),
有\(
	\sigma = \tau_{-\hat{\sigma}(\beta)} \hat{\sigma} \tau_\beta
\).
%TODO proof
\end{property}

\begin{property}
%@see: 《高等代数与解析几何(第三版 下册)》(孟道骥) P443
设\(V,V'\)都是域\(F\)上的线性空间,
\(W\)是\(V\)的一个子空间,
\(W'\)是\(V'\)的一个子空间,
向量\(\alpha \in V\),
向量\(\alpha' \in V'\),
\(\sigma\)是从\(W\)到\(W'\)的一个同构,
记\(\hat{\sigma} \defeq \tau_{\alpha'} \sigma \tau_{-\alpha}\),
那么\((\hat{\sigma})^{-1}\)是
从\(\mathcal{A}(\alpha' + W')\)到\(\mathcal{A}(\alpha + W)\)的一个仿射变换.
%TODO proof
\end{property}

\begin{property}
%@see: 《高等代数与解析几何(第三版 下册)》(孟道骥) P444
设\(\hat{\sigma}_1\colon \mathcal{A}(S_1) \to \mathcal{A}(S_2)\)
和\(\hat{\sigma}_2\colon \mathcal{A}(S_2) \to \mathcal{A}(S_3)\)都是仿射变换,
则\(\hat{\sigma}_2 \hat{\sigma}_1\)是
从\(\mathcal{A}(S_1)\)到\(\mathcal{A}(S_3)\)的一个仿射变换.
%TODO proof
\end{property}

\begin{definition}
%@see: 《高等代数与解析几何(第三版 下册)》(孟道骥) P444 定义10.3.3
设\(A\)是\(n\ (n\geq1)\)维仿射几何,
\(O,\AutoTuple{P}{n}\)是\(n+1\)个点,
且\begin{equation*}
	\dim\left( \bigvee_{i=1}^n P_i \vee O \right) = n,
\end{equation*}
则称\([O;\AutoTuple{P}{n}]\)是
“仿射几何\(A\)的(以\(O\)为原点的)\DefineConcept{标架}”.
\end{definition}
%TODO \DefineConcept{标准标架}

\begin{theorem}
%@see: 《高等代数与解析几何(第三版 下册)》(孟道骥) P444 定理10.3.2
设\([O;\AutoTuple{P}{n}]\)和\([O';\AutoTuple{P'}{n}]\)分别是仿射几何\(A\)和\(A'\)的标架,
那么存在从\(A\)到\(A'\)的唯一的仿射变换\(f\)使得\begin{equation*}
	f(P_i) = P'_i
	\ (i=1,2,\dotsc,n);
	\qquad
	f(O) = O'.
\end{equation*}
%TODO proof
\end{theorem}

\begin{definition}
%@see: 《高等代数与解析几何(第三版 下册)》(孟道骥) P445 定义10.3.4
设\(A\)是域\(F\)上的\(n\)维仿射几何,
把从\(A\)到\(\mathcal{A}(F^n)\)的每一个仿射变换
称为“仿射几何\(A\)的一个\DefineConcept{仿射坐标系}”.
\end{definition}

\begin{example}
%@see: 《高等代数与解析几何(第三版 下册)》(孟道骥) P445 习题 2.
设\(f\colon \mathcal{A}(S) \to \mathcal{A}(S')\)是一个仿射变换,
陪集\(T \subseteq S\).
证明:\(f\)在\(\mathcal{A}(T)\)上的限制\(f \SetRestrict \mathcal{A}(T)\)也是一个仿射变换.
%TODO proof
\end{example}

\begin{example}
%@see: 《高等代数与解析几何(第三版 下册)》(孟道骥) P445 习题 3.
设\(V\)都是域\(F\)上的线性空间.
\(W\)是\(V\)的一个子空间,
向量\(\alpha \in V\),
\(\sigma\)是\(W\)上的一个自同构,
记\(\hat{\sigma} \defeq \tau_\alpha \sigma \tau_{-\alpha}\).
证明:\begin{equation*}
	(\forall S \in \mathcal{A}(\alpha + W))
	[
		f(S) \parallel S
	]
	\iff
	(\forall U \AlgebraSubstructure W)
	[
		f(U) = U
	].
\end{equation*}
%TODO proof
\end{example}

\subsection{仿射几何基本定理}
%@see: 《高等代数与解析几何(第三版 下册)》(孟道骥) P445
设\(\mathcal{A}(V)\)是域\(F\)上的一个\(n\ (n\geq3)\)维仿射几何.
对于\(\mathcal{A}(V)\)中任意一个不含零向量的平面\(S\),
\(S\)中的每一点对应\(V\)的某个非零向量.
设\([O;\AutoTuple{P}{n}]\)是\(\mathcal{A}(V)\)的标架.
对于\(\mathcal{A}(V)\)中任意一点\(P\),
直线\(OP \defeq O \vee P\)是\(V\)的一个1维子空间.
我们把\(OP\)中的每一个非零向量\(p\)
称为“对于点\(P\)的一个\DefineConcept{齐性向量}”.

\begin{property}
%@see: 《高等代数与解析几何(第三版 下册)》(孟道骥) P445
向量\(p\)是对于点\(P\)的一个齐性向量,
当且仅当直线\(OP\)等于由\(p\)生成的子空间.
\end{property}

\section{射影几何}
本节利用构造法描述射影几何中的几何元素.

\subsection{射影几何}
\begin{definition}
%@see: 《高等代数与解析几何(第三版 下册)》(孟道骥) P452 定义10.5.1
设\(V\)是域\(F\)上的一个线性空间.
把\(V\)的全体子空间\begin{equation*}
	\Set{
		W
		\given
		W \AlgebraSubstructure V
	}
\end{equation*}
称为“域\(F\)上线性空间\(V\)上的\DefineConcept{射影几何}(projective geometry)”,
记为\(\mathcal{P}(V)\).
%@see: https://mathworld.wolfram.com/ProjectiveGeometry.html
\end{definition}

\begin{definition}
%@see: 《高等代数与解析几何(第三版 下册)》(孟道骥) P452 定义10.5.2
设\(W\)是线性空间\(V\)的一个子空间.
把\((\dim W - 1)\)称为“\(W\)的\DefineConcept{射影维数}”,
记为\(\pdim W\),
即\begin{equation*}
	\pdim W \defeq \dim W - 1.
\end{equation*}

将\(\mathcal{P}(V)\)中\(0\)维元素称为\DefineConcept{射影点}.

将\(\mathcal{P}(V)\)中\(1\)维元素称为\DefineConcept{射影直线}.

将\(\mathcal{P}(V)\)中\(2\)维元素称为\DefineConcept{射影平面}.

将\(\mathcal{P}(V)\)中\((\pdim V-1)\)维元素称为\DefineConcept{射影超平面}.
\end{definition}

\begin{definition}
%@see: 《高等代数与解析几何(第三版 下册)》(孟道骥) P452 定义10.5.2
设\(W\)是线性空间\(V\)的一个子空间.
如果\(W\)是射影点,
则把\(W\)中的每一个非零向量称为“射影点\(W\)的一个\DefineConcept{齐性向量}”.
\end{definition}

\begin{definition}
%@see: 《高等代数与解析几何(第三版 下册)》(孟道骥) P452 定义10.5.2
设\(V\)是域\(F\)上的一个线性空间.
把射影几何\(\mathcal{P}(V)\)中全体射影点\begin{equation*}
	\Set{
		W \AlgebraSubstructure V
		\given
		\pdim W = 0
	}
\end{equation*}
称为“由射影几何\(\mathcal{P}(V)\)决定的\DefineConcept{射影空间}”.
\end{definition}

\begin{property}
%@see: 《高等代数与解析几何(第三版 下册)》(孟道骥) P452
线性空间\(V\)的零子空间\(0\)不在由\(\mathcal{P}(V)\)决定的射影空间中.
%TODO proof
\end{property}

\begin{definition}
%@see: 《高等代数与解析几何(第三版 下册)》(孟道骥) P452
把线性空间\(V\)的零子空间\(0\)
称为“(由\(\mathcal{P}(V)\)决定的)射影空间的\DefineConcept{空子集}”.
\end{definition}

\begin{proposition}
%@see: 《高等代数与解析几何(第三版 下册)》(孟道骥) P452
设\(M,N \in \mathcal{P}(V)\),
则\(M \cap N = 0\)
当且仅当\(\pdim(M \cap N) = -1\).
\end{proposition}

\begin{definition}
%@see: 《高等代数与解析几何(第三版 下册)》(孟道骥) P452
设\(M,N \in \mathcal{P}(V)\).
如果\(M \cap N = 0\),
则称“\(M\)与\(N\)是\DefineConcept{交错的}”.
\end{definition}

%@see: 《高等代数与解析几何(第三版 下册)》(孟道骥) P452
在2维射影几何中,
两个不同点的联是一条直线,
两条不同直线的交是一个点.
在3维射影几何中,
两个不同点的联是一条直线,
两个不同平面的交是一条直线,
两条不同的相交直线的联是一个平面,
两条不同的共面直线的交是一个点,
一个点与不包含该点的一条直线的联是一个平面,
一个平面与不在该平面中的一条直线的交是一个点.

\begin{theorem}
%@see: 《高等代数与解析几何(第三版 下册)》(孟道骥) P453
设\(U\)和\(W\)是线性空间\(V\)的两个子空间,
则\begin{equation*}
	\pdim(U + W)
	+ \pdim (U \cap W)
	= \pdim U
	+ \pdim W.
\end{equation*}
\end{theorem}

