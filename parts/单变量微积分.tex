\part{单变量微积分}

\chapter{数列极限}
\input{微积分/数列极限/实数公理}
\section{数列极限的概念}
\begin{definition}
%@see: 《高等数学(第六版 上册)》 P26 定义1
%@see: 《数学分析(第二版 上册)》(陈纪修) P34 定义2.2.1
设\(\{x_n\}\)为一数列.
如果存在常数\(a\),
对于任意给定的正数\(\epsilon\)(不论它多么小),
总存在正整数\(N\),
使得当\(n > N\)时,
不等式\(\abs{x_n - a} < \epsilon\)都成立,
那么就称“数列\(\{x_n\}\)是\DefineConcept{收敛的}(convergent)”
“数列\(\{x_n\}\)~\DefineConcept{收敛}(converge)”
“数列\(\{x_n\}\)的极限存在且有限”
“极限\(\lim_{n\to\infty} x_n\)存在且有限”
“常数\(a\)是数列\(\{x_n\}\)的\DefineConcept{极限}(limit)”
“数列\(\{x_n\}\)的极限是\(a\)”
“数列\(\{x_n\}\)~\DefineConcept{收敛于}~\(a\)”,
记为\begin{equation*}
	\lim_{n\to\infty} x_n = a
	\quad\text{或}\quad
	x_n\to a\ (n\to\infty).
\end{equation*}
否则,称“数列\(\{x_n\}\)没有极限”
“数列\(\{x_n\}\)是\DefineConcept{发散的}(divergent)”
“数列\(\{x_n\}\)~\DefineConcept{发散}(diverge)”
“数列\(\{x_n\}\)的极限不存在”
“极限\(\lim_{n\to\infty} x_n\)不存在”.
%@see: https://mathworld.wolfram.com/Limit.html
\end{definition}

%@see: https://mathworld.wolfram.com/Epsilon-DeltaDefinition.html
%@see: https://mathworld.wolfram.com/Epsilon-DeltaProof.html
利用形式逻辑的语言,我们可以将上述定义简化为:
\begin{align*}
	\text{数列\(\{x_n\}\)收敛于\(a\)}
	&\defiff
	\lim_{n\to\infty} x_n = a \\
	&\defiff
	(\forall \epsilon > 0)
	(\exists N\in\mathbb{N})
	(\forall n\in\mathbb{N})
	[
		n > N
		\implies
		\abs{x_n - a} < \epsilon
	]. \\
	\text{数列\(\{x_n\}\)收敛}
	&\defiff
	(\exists a\in\mathbb{R})
	\left[
		\lim_{n\to\infty} x_n = a
	\right]. \\
	\text{数列\(\{x_n\}\)发散}
	&\iff
	(\forall a \in \mathbb{R})
	(\exists\epsilon>0)
	(\forall N\in\mathbb{N})
	(\exists n > N)
	[
		\abs{a_n - a} > \epsilon
	].
\end{align*}

\begin{definition}
%@see: 《数学分析(第五版 上册)》(华东师范大学) P25 定义2
若数列\(\{x_n\}\)满足\(\lim_{n\to\infty} x_n = 0\),
则称“数列\(\{x_n\}\)是\DefineConcept{无穷小}”.
\end{definition}

\begin{definition}
%@see: 《数学分析(第二版 上册)》(陈纪修) P46 定义2.3.1
%@see: 《数学分析(第五版 上册)》(华东师范大学) P25 定义3
若数列\(\{x_n\}\)满足\begin{equation*}
	(\forall G>0)
	(\exists N\in\mathbb{N})
	(\forall n\in\mathbb{N})
	[n > N \implies \abs{x_n} > G],
\end{equation*}
则称“数列\(\{x_n\}\)是\DefineConcept{无穷大}”
“数列\(\{x_n\}\)趋向于极限\(\infty\)”
“数列\(\{x_n\}\)发散到\(\infty\)”,
记作\begin{equation*}
	\lim_{n\to\infty} x_n = \infty
	\quad\text{或}\quad
	x_n\to\infty\ (n\to\infty).
\end{equation*}

若数列\(\{x_n\}\)满足\begin{equation*}
	(\forall G>0)
	(\exists N\in\mathbb{N})
	(\forall n\in\mathbb{N})
	[n > N \implies x_n > G],
\end{equation*}
则称“数列\(\{x_n\}\)是\DefineConcept{正无穷大}”
“数列\(\{x_n\}\)趋向于极限\(+\infty\)”
“数列\(\{x_n\}\)发散到\(+\infty\)”,
记作\begin{equation*}
	\lim_{n\to\infty} x_n = +\infty
	\quad\text{或}\quad
	x_n\to+\infty\ (n\to\infty).
\end{equation*}

若数列\(\{x_n\}\)满足\begin{equation*}
	(\forall G>0)
	(\exists N\in\mathbb{N})
	(\forall n\in\mathbb{N})
	[n > N \implies x_n < -G],
\end{equation*}
则称“数列\(\{x_n\}\)是\DefineConcept{负无穷大}”
“数列\(\{x_n\}\)趋向于极限\(-\infty\)”
“数列\(\{x_n\}\)发散到\(-\infty\)”,
记作\begin{equation*}
	\lim_{n\to\infty} x_n = -\infty
	\quad\text{或}\quad
	x_n\to-\infty\ (n\to\infty).
\end{equation*}

我们把正无穷大和负无穷大统称为\DefineConcept{定号无穷大},
把既非正无穷大又非负无穷大的无穷大称为\DefineConcept{不定号无穷大}.
\end{definition}

\begin{example}
证明数列\begin{equation*}
	2,\frac{1}{2},\frac{4}{3},\frac{3}{4},\dotsc,\frac{n+(-1)^{n-1}}{n},\dotsc
\end{equation*}的极限是1.
\begin{proof}
由于\begin{equation*}
	\abs{x_n - 1}
	= \abs{\frac{n+(-1)^{n-1}}{n}-1}
	= \abs{\frac{(-1)^{n-1}}{n}}
	= \frac{1}{n},
\end{equation*}
所以为使\(\abs{x_n - 1} < \epsilon\),
须取\(\frac{1}{n} < \epsilon\)或\(\frac{1}{\epsilon} < n\).
也就是说,对于\(\forall \epsilon > 0\),
取\(N = \floor*{\frac{1}{\epsilon}}\),
则当\(n > N\)时,就有\(\abs{x_n - a} < \epsilon\),
即\(\lim_{n\to\infty}\frac{n+(-1)^{n-1}}{n}=1\).
\end{proof}
\end{example}

\begin{example}
已知\(x_n = \frac{(-1)^n}{(n+1)^2}\),
证明数列\(\Set{x_n}\)的极限是\(0\).
\begin{proof}
因为\(\abs{x_n - a}
= \abs{\frac{(-1)^n}{(n+1)^2}-0}
= \frac{1}{(n+1)^2}
< \frac{1}{n+1}\),
所以对于\(\forall\epsilon>0\)(设\(\epsilon<1\)),
只要\(\frac{1}{n+1}<\epsilon\)或\(n>\frac{1}{\epsilon}-1\),
不等式\(\abs{x_n-a}<\epsilon\)必定成立.
所以,取\(N=\floor*{\frac{1}{\epsilon}-1}\),
则当\(n>N\)时就有\(\abs{x_n - a}<\epsilon\),
即\(\lim_{n\to\infty}\frac{(-1)^n}{(n+1)^2}=0\).
\end{proof}
\end{example}

在利用数列极限的定义来论证某个数\(a\)是数列\(\{x_n\}\)的极限时,
重要的是对于任意给定的正数\(\epsilon\),要能够指出定义中所说的这种正整数\(N\)确实存在,
但没有必要去求最小的\(N\).
如果知道\(\abs{x_n-a}\)小于某个量(这个量是\(n\)的一个函数),
那么当这个量小于\(\epsilon\)时,\(\abs{x_n-a}<\epsilon\)当然也成立.
若令这个量小于\(\epsilon\)来定出\(N\)比较方便的话,就可采用这种方法.

\begin{example}
%@see: 《数学分析(第二版 上册)》(陈纪修) P36 例2.2.2
设\(\abs{q}<1\),
证明等比数列\begin{equation*}
	1,q,q^2,\dotsc,q^{n-1},\dotsc
\end{equation*}的极限是\(0\).
\begin{proof}
对于\(\forall\epsilon>0\)(设\(\epsilon<1\)),
因为\(\abs{x_n-0}=\abs{q^{n-1}-0}=\abs{q}^{n-1}\),
要使\(\abs{x_n-0}<\epsilon\),
只要\(\abs{q}^{n-1}<\epsilon\).
取自然对数得\((n-1)\ln\abs{q}<\ln\epsilon\).
因为\(\abs{q}<1\),
\(\ln\abs{q}<0\),
故\(n>1+\frac{\ln\epsilon}{\ln\abs{q}}\).
取\(N=\floor*{1+\frac{\ln\epsilon}{\ln\abs{q}}}\),
则当\(n>N\)时,
就有\(\abs{q^{n-1}-0}<\epsilon\),
即
\begin{equation}\label{equation:数列极限.重要极限1}
	\lim_{n\to\infty}q^{n-1}=0
	\quad(\abs{q}<1).
\end{equation}
由此可知,当\(\abs{q}<1\)时,
等比数列\(1,q,q^2,\dotsc,q^{n-1},\dotsc\)的极限是\(0\).
\end{proof}
\end{example}

\begin{example}
%@see: 《数学分析(第二版 上册)》(陈纪修) P46 例2.3.1
设\(\abs{q}>1\),证明\(\{q^n\}\)是无穷大.
\begin{proof}
对\(\forall G>1\),
取\(N=\ceil*{\frac{\ln G}{\ln\abs{q}}}\),
于是对\(\forall n>N\)
有\(\abs{q}^n > \abs{q}^N \geq \abs{q}^{\frac{\ln G}{\ln\abs{q}}} = G\).
因此\(\{q^n\}\)是无穷大.
\end{proof}
\end{example}

\begin{example}\label{example:极限.常数的方根的极限1}
%@see: 《数学分析(第二版 上册)》(陈纪修) P36 例2.2.3
设\(a>1\),证明:\(\lim_{n\to\infty} \sqrt[n]{a} = 1\).
\begin{proof}
记\(y_n=\sqrt[n]{a}-1\).
因为\(a>1\),所以对于\(n\geq1\)总有\(\sqrt[n]{a}>1\),即\(y_n>0\).
应用二项式定理,有\begin{equation*}
	a = (1+y_n)^n
	= 1 + n y_n + \frac{n(n-1)}2 y_n^2 + \dotsb + y_n^n
	> 1 + n y_n,
\end{equation*}
于是\(\abs{\sqrt[n]{a}-1} = \abs{y_n} < \frac{a-1}{n}\).
那么对于\(\forall\epsilon>0\),只要取\(N=\ceil*{\frac{a-1}{\epsilon}}\),
当\(n>N\)时就有\(\abs{\sqrt[n]{a}-1} < \frac{a-1}{n} < \epsilon\)成立.
同理可证当\(0<a<1\)时,也有\(\lim_{n\to\infty} \sqrt[n]{a} = 1\).
\end{proof}
\end{example}
\begin{remark}
在\cref{example:极限.常数的方根的极限2} 中,
我们会把\cref{example:极限.常数的方根的极限1} 的条件减弱为\(a>0\).
\end{remark}

\begin{example}
%@see: 《数学分析(第二版 上册)》(陈纪修) P37 例2.2.4
证明:\begin{equation}\label{equation:数列极限.重要极限2}
	\lim_{n\to\infty} \sqrt[n]{n} = 1.
\end{equation}
\begin{proof}
要证\(\lim_{n\to\infty} \sqrt[n]{n} = 1\),
只需证对于\(\forall\epsilon>0\),
\(\exists N > 0\),
使得当\(n > N\)时,
有\begin{equation*}
	\abs{\sqrt[n]{n} - 1} < \epsilon.
	\eqno(1)
\end{equation*}

因为\(n \geq 1\),
\(\sqrt[n]{n} \geq 1\),
所以(1)式等价于\begin{equation*}
	\abs{\sqrt[n]{n} - 1}
	= \sqrt[n]{n} - 1
	< \epsilon,
\end{equation*}
也即\(\sqrt[n]{n} < 1 + \epsilon\),
取对数得\begin{equation*}
	\frac{1}{n} \ln n < \ln(1+\epsilon);
	\eqno(2)
\end{equation*}
又因为\(\ln n < \sqrt{n}\),
所以只要有\begin{equation*}
	\frac{1}{n} \ln n
	< \frac{1}{n} \sqrt{n}
	= \frac{1}{\sqrt{n}}
	\leq \ln(1+\epsilon)
	\quad\text{或}\quad
	n \geq \left[ \frac1{\ln(1+\epsilon)} \right]^2
	\eqno(3)
\end{equation*}
成立即有(2)式成立,
那么取\begin{equation*}
	N = \ceil*{\frac1{\ln^2(1+\epsilon)}},
\end{equation*}
就对\(\forall\epsilon>0\),
\(\exists N > 0\),
使得当\(n > N\)时,
有\(\abs{\sqrt[n]{n} - 1} < \epsilon\)成立.
\end{proof}
\end{example}

\begin{example}\label{example:极限.数列的算术平均的极限}
%@see: 《数学分析(第二版 上册)》(陈纪修) P37 例2.2.6
设\(\lim_{n\to\infty} a_n = a\).
证明:\(\lim_{n\to\infty} \frac{a_1+a_2+\dotsb+a_n}{n} = a\).
\begin{proof}
下面按\(a\)的不同取值,分两种情况讨论:
\begin{itemize}
	\item 当\(a=0\)时,
	有\begin{equation*}
		\lim_{n\to\infty} a_n = 0
		\iff
		(\forall\epsilon>0)
		(\exists N_1\in\mathbb{N})
		(\forall n\in\mathbb{N})
		\left[n>N_1 \implies \abs{a_n}<\frac\epsilon2\right].
	\end{equation*}
	由于\(a_1 + a_2 + \dotsb + a_{N_1}\)是与\(n\)无关的常量,
	因此\begin{equation*}
		(\forall\epsilon>0)
		(\exists N_2\in\mathbb{N})
		(\forall n\in\mathbb{N})
		\left[
			n>N_2
			\implies
			\abs{\frac{a_1 + a_2 + \dotsb + a_{N_1}}{n}} < \frac\epsilon2
		\right].
	\end{equation*}
	于是利用三角不等式(\cref{theorem:不等式.三角不等式1,theorem:不等式.三角不等式1.推论1})可得,
	对于\(\forall\epsilon>0\),
	当\(n>N=\max\{N_1,N_2\}\)时,
	有\begin{align*}
		\abs{\frac{a_1 + a_2 + \dotsb + a_n}{n}}
		&= \abs{
			\frac{a_1 + a_2 + \dotsb + a_{N_1}}{n}
			+ \frac{a_{N_1+1} + a_{N_1+2} + \dotsb + a_n}{n}
		} \\
		&\leq \abs{\frac{a_1 + a_2 + \dotsb + a_{N_1}}{n}}
		+ \abs{\frac{a_{N_1+1} + a_{N_1+2} + \dotsb + a_n}{n}} \\
		&< \frac\epsilon2 + \frac1n (n-N_1) \frac\epsilon2
		< \frac\epsilon2 + \frac\epsilon2
		= \epsilon.
	\end{align*}

	\item 当\(a\neq0\)时,
	显然\(\lim_{n\to\infty} (a_n - a) = 0\),
	于是\begin{equation*}
		\lim_{n\to\infty} \left(\frac{a_1+a_2+\dotsb+a_n}{n}-a\right)
		= \lim_{n\to\infty} \frac{(a_1-a)+(a_2-a)+\dotsb+(a_n-a)}{n}
		= 0,
	\end{equation*}
	也就是说\(\lim_{n\to\infty} \frac{a_1+a_2+\dotsb+a_n}{n} = a\).
	\qedhere
\end{itemize}
\end{proof}
\end{example}
\begin{remark}
反过来,若已知\(\lim_{n\to\infty} \frac{a_1+a_2+\dotsb+a_n}{n} = a\),
却不一定有\(\lim_{n\to\infty} a_n = a\).
\end{remark}

\begin{proposition}\label{theorem:极限.数列的绝对值的极限}
设数列\(\{x_n\}\)的极限\(\lim_{n\to\infty} x_n\)存在且有限,
则\begin{equation*}
	\lim_{n\to\infty} \abs{x_n} = \abs{\lim_{n\to\infty} x_n}.
\end{equation*}
\begin{proof}
假设\(\lim_{n\to\infty} x_n = a\),
那么\begin{equation*}
	(\forall\epsilon>0)
	(\exists N\in\mathbb{N})
	(\forall n\in\mathbb{N})
	[n>N \implies \abs{x_n - a} < \epsilon].
\end{equation*}
由\hyperref[theorem:不等式.三角不等式2]{三角不等式}有\begin{equation*}
	\abs{\abs{x_n} - \abs{a}} \leq \abs{x_n - a}.
\end{equation*}
于是有\begin{equation*}
	(\forall\epsilon>0)
	(\exists N\in\mathbb{N})
	(\forall n\in\mathbb{N})
	[n>N \implies \abs{\abs{x_n} - \abs{a}} < \epsilon],
\end{equation*}
也就是说\(\lim_{n\to\infty} \abs{x_n} = \abs{a}\).
\end{proof}
\end{proposition}

\begin{example}\label{example:极限.指标变化时数列极限不变}
%@see: 《数学分析(第二版 上册)》(陈纪修) P45 习题 4.
证明:\begin{equation*}
	(\forall k\in\mathbb{N}^+)
	\left[ \lim_{n\to\infty} x_n = a \iff \lim_{n\to\infty} x_{n+k} = a \right].
\end{equation*}
\begin{proof}
假设\(\lim_{n\to\infty} x_n = a\).
根据定义有\begin{equation*}
	(\forall\epsilon>0)
	(\exists N\in\mathbb{N})
	(\forall n\in\mathbb{N})
	[
		n>N
		\implies
		\abs{x_n-a}<\epsilon
	].
\end{equation*}
因为\(n+k>n>N\),所以\(\abs{x_{n+k}-a}<\epsilon\)必然成立,
从而有\(\lim_{n\to\infty} x_{n+k} = a\).

假设\(\lim_{n\to\infty} x_{n+k} = a\).
根据定义有\begin{equation*}
	(\forall\epsilon>0)
	(\exists N\in\mathbb{N})
	(\forall n\in\mathbb{N})
	[
		n>N
		\implies
		m \defeq n+k > N
		\implies
		\abs{x_m-a}<\epsilon
	],
\end{equation*}
于是\(\lim_{m\to\infty} x_m = a\).
\end{proof}
\end{example}

\begin{example}\label{example:数列极限.奇偶子列收敛则数列收敛}
%@see: 《数学分析(第二版 上册)》(陈纪修) P45 习题 5.
%@see: 《高等数学(第六版 上册)》 P31 习题1-2 6.
设\(\lim_{k\to\infty} x_{2k} = \lim_{k\to\infty} x_{2k+1} = a\),
证明:\(\lim_{n\to\infty} x_n = a\).
\begin{proof}
由于\begin{gather*}
	\lim_{k\to\infty} x_{2k} = a
	\iff
	(\forall\epsilon>0)
	(\exists N_1\in\mathbb{N})
	(\forall k\in\mathbb{N})
	[
		k>N_1
		\implies
		\abs{x_{2k}-a}<\epsilon
	], \\
	\lim_{k\to\infty} x_{2k+1} = a
	\iff
	(\forall\epsilon>0)
	(\exists N_2\in\mathbb{N})
	(\forall k\in\mathbb{N})
	[
		k>N_2
		\implies
		\abs{x_{2k+1}-a}<\epsilon
	],
\end{gather*}
所以,只要\(k>N=\max\{N_1,N_2\}\),
就有\(\abs{x_{2k}-a}<\epsilon\)和\(\abs{x_{2k+1}-a}<\epsilon\)同时成立;
这就是说,只要\(n>2N+1\),
就有\(\abs{x_n-a}<\epsilon\)成立;
因此\(\lim_{n\to\infty} x_n = a\).
\end{proof}
\end{example}

\begin{example}\label{example:数列极限的概念.柯西列的等价性}
设数列\(\{x_n\},\{y_n\}\)的极限都存在,
且\(\lim_{n\to\infty} \abs{x_n - y_n} = 0\).
证明:\(\lim_{n\to\infty} x_n = \lim_{n\to\infty} y_n\).
\begin{proof}
%@credit: {腾讯元宝}
设\(\lim_{n\to\infty} x_n = a < \infty\),
那么对于任意\(\epsilon>0\),
存在正整数\(N_1\),
使得当\(n > N_1\)时,
有\(\abs{x_n - a} < \frac\epsilon2\)成立.
同时,由\(\lim_{n\to\infty} \abs{x_n - y_n} = 0\)可知,
对于上述\(\epsilon\),
存在正整数\(N_2\),
使得当\(n > N_2\)时,
有\(\abs{x_n - y_n} < \frac\epsilon2\)成立.
取\(N = \max\{N_1,N_2\}\),
则当\(n > N\)时,
有\begin{equation*}
	\abs{y_n - a}
	= \abs{(y_n - x_n) + (x_n - a)}
	\leq \abs{y_n - x_n} + \abs{x_n - a}
	< \frac\epsilon2 + \frac\epsilon2
	= \epsilon,
\end{equation*}
这就说明\(\lim_{n\to\infty} y_n = a\).

设\(\lim_{n\to\infty} x_n = \infty\),
那么对于任意\(G>0\),
存在正整数\(N_3\),
使得当\(n > N_3\)时,
有\(\abs{x_n} > G\)成立.
同时,由\(\lim_{n\to\infty} \abs{x_n - y_n} = 0\)可知,
对于任意\(\epsilon<0\),
存在正整数\(N_4\),
使得当\(n > N_4\)时,
有\(\abs{x_n - y_n} < \epsilon\)成立.
不妨设\(G > \epsilon\),
取\(N' = \max\{N_3,N_4\}\),
则当\(n > N'\)时,
有\begin{equation*}
	\abs{y_n}
	= \abs{(y_n - x_n) + x_n}
	\geq \abs{
		\abs{x_n} - \abs{y_n - x_n}
	}
	> G - \epsilon,
\end{equation*}
这就说明\(\lim_{n\to\infty} y_n = \infty\).
\end{proof}
\end{example}

\section{数列极限的性质}
\subsection{唯一性}
\begin{theorem}[唯一性]\label{theorem:极限.收敛数列的唯一性}
%@see: 《高等数学(第六版 上册)》 P28 定理1
%@see: 《数学分析(第二版 上册)》(陈纪修) P39 定理2.2.1
如果数列\(\{x_n\}\)收敛,那么它的极限唯一.
\begin{proof}
用反证法.
假设当\(n\to\infty\)时,同时有\(x_n \to a\)及\(x_n \to b\),且\(a < b\).
取\(\epsilon = \frac{b-a}{2}\).
因为\(\lim_{n\to\infty} x_n = a\),所以有
\begin{equation*}
	(\exists N_1\in\mathbb{N})
	(\forall n\in\mathbb{N})
	\left[n > N_1 \implies \abs{x_n - a} < \frac{b-a}{2}\right]
	\eqno(1)
\end{equation*}成立.

同理,因为\(\lim_{n\to\infty} x_n = b\),所以有
\begin{equation*}
	(\exists N_2 \in \mathbb{N})
	(\forall n\in\mathbb{N})
	\left[n > N_2 \implies \abs{x_n - b} < \frac{b-a}{2}\right]
	\eqno(2)
\end{equation*}成立.

取\(N = \max\{N_1,N_2\}\),则当\(n > N\)时,上述两个不等式应同时成立.
但由(1)式有\(x_n<\frac{a+b}{2}\),由(2)式有\(x_n>\frac{a+b}{2}\),
矛盾,故收敛数列的极限必定唯一.
\end{proof}
\end{theorem}

\begin{example}\label{example:极限.振荡数列不存在极限}
证明数列\(x_n=(-1)^{n+1}\ (n=1,2,\dotsc)\)是发散的.
\begin{proof}
假设这级数收敛,则它具有唯一的极限\(\lim_{n\to\infty} x_n = a\).
按数列极限的定义,对于\(\epsilon=1/2\),\(\exists N \in \mathbb{N}^+\),当\(n > N\)时,\(\abs{x_n-a}<1/2\)或\(x_n\in\left(a-\frac{1}{2},a+\frac{1}{2}\right)\)成立.
但这是不可能的,因为\(n\to\infty\)时,\(x_n\)无休止地一再重复取得\(1\)和\(-1\)这两个数,而这两个数不可能同时属于长度为\(1\)的开区间\(\left(a-\frac{1}{2},a+\frac{1}{2}\right)\)内,因此这数列发散.
\end{proof}
\end{example}

\subsection{有界性}
\begin{theorem}[有界性]\label{theorem:极限.收敛数列的有界性}
%@see: 《高等数学(第六版 上册)》 P29 定理2
%@see: 《数学分析(第二版 上册)》(陈纪修) P39 定理2.2.2
如果数列\(\{x_n\}\)收敛,
那么数列\(\{x_n\}\)一定有界.
\begin{proof}
既然数列\(\{x_n\}\)收敛,
不妨设\(\lim_{n\to\infty} x_n = a\).
根据数列极限的定义,
对于\(\epsilon = 1\),
\(\exists N \in \mathbb{N}^+\),
当\(n > N\)时,
不等式\(\abs{x_n - a} < 1\)都成立.
于是,
当\(n > N\)时,
\begin{equation*}
	\abs{x_n} = \abs{(x_n - a) + a} \leq \abs{x_n - a} + \abs{a} < 1 + \abs{a}.
\end{equation*}
取\(M = \max\{\abs{x_1},\abs{x_2},\dotsc,\abs{x_N},1+\abs{a}\}\),
那么数列\(\{x_n\}\)中的一切\(x_n\)都满足不等式\begin{equation*}
	\abs{x_n} \leq M.
\end{equation*}
这就证明了数列\(\{x_n\}\)是有界的.
\end{proof}
\end{theorem}

根据\cref{theorem:极限.收敛数列的有界性} 立即有以下推论.
\begin{corollary}
如果数列\(\{x_n\}\)是无界的,那么数列\(\{x_n\}\)一定发散.
\end{corollary}
但是,如果数列\(\{x_n\}\)有界,却不能断定数列\(\{x_n\}\)一定收敛.
例如,在\cref{example:极限.振荡数列不存在极限} 中,
数列\begin{equation*}
	1,-1,1,\dotsc,(-1)^{n+1},\dotsc
\end{equation*}有界,
但它是发散的.
于是我们可以说:
数列有界是数列收敛的必要不充分条件.

\subsection{保序性}
\begin{theorem}[保序性]\label{theorem:极限.收敛数列的保序性1}
%@see: 《数学分析(第二版 上册)》(陈纪修) P39 定理2.2.3
设数列\(\{x_n\},\{y_n\}\)均收敛.
若\(\lim_{n\to\infty} x_n = a,
\lim_{n\to\infty} y_n = b\),
且\(a < b\),
则存在正整数\(N\),当\(n>N\)时,
有\(x_n < y_n\).
\begin{proof}
取\(\epsilon=\frac{b-a}2>0\),
则\begin{equation*}
	\lim_{n\to\infty} x_n = a
	\implies
	(\exists N_1\in\mathbb{N})(\forall n\in\mathbb{N})
	\left[
		\begin{array}{rl}
			n>N_1
			&\implies
			\abs{x_n - a} < \epsilon = \frac{b-a}2 \\
			&\implies
			x_n < a + \frac{b-a}2 = \frac{a+b}2
		\end{array}
	\right],
\end{equation*}\begin{equation*}
	\lim_{n\to\infty} y_n = b
	\implies
	(\exists N_2\in\mathbb{N})(\forall n\in\mathbb{N})
	\left[
		\begin{array}{rl}
			n>N_2
			&\implies
			\abs{y_n - b} < \epsilon = \frac{b-a}2 \\
			&\implies
			y_n > b - \frac{b-a}2 = \frac{a+b}2
		\end{array}
	\right],
\end{equation*}
于是,取\(N = \max\{N_1,N_2\}\),
则对\(\forall n\in\mathbb{N}\),
只要\(n>N\),
就有\(x_n < \frac{a+b}2 < y_n\).
\end{proof}
\end{theorem}

需要注意到,\cref{theorem:极限.收敛数列的保序性1} 的逆命题不成立,
也就是说,由\begin{equation*}
	\lim_{n\to\infty} x_n = a, \qquad
	\lim_{n\to\infty} y_n = b, \qquad
	(\exists N\in\mathbb{N})
	(\forall n\in\mathbb{N})
	[n>N \implies x_n < y_n]
\end{equation*}这三个条件,
无法推出\(a<b\)的结论.
数列\begin{equation*}
	\{x_n = 0\}, \qquad
	\{y_n = 1/n\},
	\quad\text{和}\quad
	\{z_n = 2/n\}
\end{equation*}就是例子.
因此,我们只能得到如下结论:
\begin{theorem}\label{theorem:极限.收敛数列的保序性2}
%@see: 《数学分析(第二版 上册)》(陈纪修) P41
设\(\lim_{n\to\infty} x_n = a,
\lim_{n\to\infty} y_n = b\).
若\((\exists N\in\mathbb{N})
(\forall n\in\mathbb{N})
[n>N \implies x_n < y_n]\),
则\(a \leq b\).
\end{theorem}

\begin{corollary}[保号性]\label{theorem:极限.收敛数列的保号性1}
%@see: 《高等数学(第六版 上册)》 P29 定理3
%@see: 《数学分析(第二版 上册)》(陈纪修) P40 推论
设\(\lim_{n\to\infty} x_n = a\).
\begin{itemize}
	\item 若\(a > 0\),
	那么\((\exists N\in\mathbb{N})
	(\forall n\in\mathbb{N})
	[n>N \implies x_n > 0]\).

	\item 若\(a < 0\),
	那么\((\exists N\in\mathbb{N})
	(\forall n\in\mathbb{N})
	[n>N \implies x_n < 0]\).
\end{itemize}
\begin{proof}
当\(a > 0\)时,
由数列极限的定义,
对\(\epsilon = \frac{a}{2} > 0\),
\(\exists N \in \mathbb{N}^+\),
当\(n > N\)时,
有\(\abs{x_n - a} < \frac{a}{2}\),
从而\(x_n > a - \frac{a}{2} = \frac{a}{2} > 0\).

同样地,
当\(a < 0\)时,
对\(\epsilon = -\frac{a}{2} > 0\),
\(\exists N \in \mathbb{N}^+\),
当\(n > N\)时,
有\(\abs{x_n - a} < -\frac{a}{2}\),
从而\(x_n < a - \frac{a}{2} = \frac{a}{2} < 0\).
\end{proof}
\end{corollary}

\begin{corollary}\label{theorem:极限.收敛数列的保号性2}
%@see: 《高等数学(第六版 上册)》 P30 推论
设\(\lim_{n\to\infty} x_n = a\).
\begin{itemize}
	\item 若\((\exists N\in\mathbb{N})
	(\forall n\in\mathbb{N})
	[n > N \implies x_n \geq 0]\),
	那么\(a \geq 0\).

	\item 若\((\exists N\in\mathbb{N})
	(\forall n\in\mathbb{N})
	[n > N \implies x_n \leq 0]\),
	那么\(a \leq 0\).
\end{itemize}
\end{corollary}

\begin{example}\label{example:数列极限.有界函数与无穷小的乘积是无穷小}
%@see: 《高等数学(第六版 上册)》 P31 习题1-2 5.
设数列\(\{x_n\}\)有界,又\(\lim_{n\to\infty} y_n = 0\).
证明:\(\lim_{n\to\infty} x_n y_n = 0\).
\begin{proof}
因为\begin{equation*}
	\text{数列\(\{x_n\}\)有界}
	\iff
	(\exists M>0)
	(\forall n\in\mathbb{N})
	[\abs{x_n} \leq M],
\end{equation*}
又因为\(\lim_{n\to\infty} y_n = 0\),
所以\begin{equation*}
	(\forall \epsilon>0)
	(\exists N\in\mathbb{N})
	(\forall n\in\mathbb{N})
	\left[
		\begin{array}{l}
			n>N
			\implies
			\abs{y_n - 0}
				= \abs{y_n}
				< \frac{\epsilon}{M} \\
			\implies
			\abs{x_n y_n - 0}
			= \abs{x_n y_n}
			= \abs{x_n} \abs{y_n}
			< M \cdot \frac{\epsilon}{M}
			= \epsilon
		\end{array}
	\right].
\end{equation*}
因此\(\lim_{n\to\infty} x_n y_n = 0\).
\end{proof}
%@see: https://www.bilibili.com/video/BV1TyszeCE3w/
\end{example}

\begin{example}
%@see: 《数学分析(第二版 上册)》(陈纪修) P45 习题 6.
设\(x_n\geq0\)且\(\lim_{n\to\infty} x_n = a \geq 0\),
证明:\(\lim_{n\to\infty} \sqrt{x_n} = \sqrt{a}\).
\begin{proof}
因为\(\lim_{n\to\infty} x_n = a\),
所以对于任意\(\epsilon>0\),存在正整数\(N\),
使得当\(n > N\)时,成立\begin{equation*}
	\abs{x_n - a} < \epsilon^2.
\end{equation*}
于是由\cref{equation:不等式.根式差的绝对值的不等式} 可知\begin{equation*}
	\abs{\sqrt{x_n} - \sqrt{a}}
	\leq \sqrt{\abs{x_n - a}}
	< \epsilon.
	\qedhere
\end{equation*}
\end{proof}
\end{example}

\subsection{线性性}
\begin{theorem}[线性性]
设\(a,b\)是实常数,对于数列\(\{x_n\}\),
有\begin{equation*}
	\lim_{n\to\infty} x_n = a
	\iff
	\lim_{n\to\infty} (k x_n + b) = k a + b.
\end{equation*}
\begin{proof}
根据数列极限的定义有\begin{align*}
	&\lim_{n\to\infty} x_n = a \\
	&\iff
	(\forall \epsilon > 0)
	(\exists N\in\mathbb{N})
	(\forall n\in\mathbb{N})
	[
		n > N
		\implies
		\abs{k x_n - k a} < k \epsilon
	] \\
	&\iff
	(\forall \epsilon > 0)
	(\exists N\in\mathbb{N})
	(\forall n\in\mathbb{N})
	[
		n > N
		\implies
		\abs{(k x_n + b) - (k a + b)} < k \epsilon
	] \\
	&\iff
	\lim_{n\to\infty} (k x_n + b) = k a + b.
	\qedhere
\end{align*}
\end{proof}
\end{theorem}

\section{夹逼准则}
\begin{theorem}\label{theorem:数列极限.夹逼准则}
%@see: 《数学分析(第二版 上册)》(陈纪修) P41 定理2.2.4
%@see: 《高等数学(第六版 上册)》 P50 准则I
如果数列\(\{x_n\},\{y_n\},\{z_n\}\)满足\begin{itemize}
	\item \((\exists n_0\in\mathbb{N})
	(\forall n\in\mathbb{N})
	[y_n \leq x_n \leq z_n]\),
	\item \(\lim_{n\to\infty} y_n = \lim_{n\to\infty} z_n = a\),
\end{itemize}
那么\(\lim_{n\to\infty} x_n = a\).
\begin{proof}
因为\(\lim_{n\to\infty} y_n = a\),
\(\lim_{n\to\infty} z_n = a\),
根据数列极限的定义,
有\begin{equation*}
	(\forall\epsilon>0)
	(\exists N_1,N_2\in\mathbb{N})
	(\forall n\in\mathbb{N})
	\left[
		\begin{array}{l}
			n > N_1 \implies \abs{y_n - a} < \epsilon, \\
			n > N_2 \implies \abs{z_n - a} < \epsilon
		\end{array}
	\right].
\end{equation*}

现在取\(N = \max\{n_0,N_1,N_2\}\),
那么,当\(n > N\)时,有\begin{equation*}
	\left\{ \begin{array}{l}
		\abs{y_n - a} < \epsilon, \\
		\abs{z_n - a} < \epsilon,
	\end{array} \right.
	\quad\text{或}\quad
	\left\{ \begin{array}{l}
		a - \epsilon < y_n, \\
		z_n < a + \epsilon
	\end{array} \right.
\end{equation*}同时成立.

又因当\(n > N\)时,有\begin{equation*}
	a - \epsilon < y_n \leq x_n \leq z_n < a + \epsilon,
\end{equation*}即\begin{equation*}
	\abs{x_n - a} < \epsilon
\end{equation*}成立.

综上所述,\begin{equation*}
	(\forall\epsilon>0)
	(\exists N\in\mathbb{N})
	(\forall n\in\mathbb{N})
	[
		n > N
		\implies
		\abs{x_n - a} < \epsilon
	];
\end{equation*}
这就证明了\(\lim_{n\to\infty} x_n = a\).
\end{proof}
\end{theorem}

\begin{example}
%@see: 《数学分析(第二版 上册)》(陈纪修) P41 例2.2.7
求数列\(\{\sqrt{n+1}-\sqrt{n}\}\)的极限.
\begin{solution}
首先我们有\begin{equation*}
	\sqrt{n+1}-\sqrt{n}
	= \frac{(\sqrt{n+1}-\sqrt{n})(\sqrt{n+1}+\sqrt{n})}{\sqrt{n+1}+\sqrt{n}}
	= \frac1{\sqrt{n+1}+\sqrt{n}}.
\end{equation*}
由于\begin{equation*}
	0 < \frac1{\sqrt{n+1}+\sqrt{n}} < \frac1{\sqrt{n}}, \qquad
	\lim_{n\to\infty} 0 = \lim_{n\to\infty} \frac1{\sqrt{n}} = 0,
\end{equation*}
所以利用\hyperref[theorem:数列极限.夹逼准则]{夹逼准则}可得
\(\lim_{n\to\infty} (\sqrt{n+1}-\sqrt{n}) = 0\).
\end{solution}
\end{example}

\begin{example}\label{equation:数列极限.重要极限6}
%@see: 《数学分析(第二版 上册)》(陈纪修) P42 例2.2.8
证明:\begin{equation}
	\lim_{n\to\infty} (a_1^n + a_2^n + \dotsb + a_p^n)^{\frac1n}
	= \max_{1\leq i\leq p} \{a_i\},
\end{equation}
其中\(a_i\geq0\ (i=1,2,\dotsc,p)\).
\begin{proof}
不失一般性,设\(a_1 = \max_{1\leq i\leq p} \{a_i\}\),
于是\begin{equation*}
	a_1 \leq (a_1^n + a_2^n + \dotsb + a_p^n)^{\frac1n} \leq a_1 \sqrt[n]{p}.
\end{equation*}
因为~\hyperref[example:极限.常数的方根的极限2]{\(\lim_{n\to\infty} \sqrt[n]{p} = 1\)},
所以利用\hyperref[theorem:数列极限.夹逼准则]{夹逼准则}可得
\(\lim_{n\to\infty} (a_1^n + a_2^n + \dotsb + a_p^n)^{\frac1n} = a_1\).
\end{proof}
\end{example}
\begin{example}
设\(0<a<b\).
求\(\lim_{n\to\infty} \left( \frac1{a^n} + \frac1{b^n} \right)^{\frac1n}\).
\begin{solution}
因为\(0<a<b\),\(0<\frac1b<\frac1a\),
所以由\cref{equation:数列极限.重要极限6} 可知
\(\lim_{n\to\infty} \left( \frac1{a^n} + \frac1{b^n} \right)^{\frac1n} = \frac1a\).
\end{solution}
\end{example}

\begin{example}
证明:\begin{equation}\label{equation:数列极限.重要极限3}
	\lim_{n\to\infty} \sqrt[n]{k n} = 1
	\quad(k>0).
\end{equation}
\begin{proof}
当\(n \geq 3\)时,
将\(\sqrt[n]{k n}\)看作一个\(k\)、两个\(\sqrt{n}\)与\(n-3\)个\(1\)的几何平均值,
则有\begin{equation*}
	1 \leq \sqrt[n]{k n} = (k \cdot \sqrt{n}^2 \cdot 1^{n-3})^{1/n}
	< \frac{k + 2\sqrt{n} + n-3}{n}
	= 1 + \frac{2}{\sqrt{n}} + \frac{k-3}{n}.
\end{equation*}
因为\begin{equation*}
	\lim_{n\to\infty} 1
	= \lim_{n\to\infty} \left(1 + \frac{2}{\sqrt{n}} + \frac{k-3}{n}\right) = 1,
\end{equation*}
由夹逼定理可得\(\lim_{n\to\infty} \sqrt[n]{k n} = 1\).
\end{proof}
\end{example}

\begin{example}
%@see: 《数学分析(第二版 上册)》(陈纪修) P45 习题 8.(4)
求:\(\lim_{n\to\infty} \frac{1 \cdot 3 \cdot 5 \dotsm (2n-1)}{2 \cdot 4 \cdot 6 \dotsm (2n)}\).
\begin{solution}
因为\((2n)^2 = 4n^2 > 4n^2-1 = (2n-1)(2n+1)\),\(2n > \sqrt{(2n-1)(2n+1)}\),
所以\begin{equation*}
	2 > \sqrt{1 \cdot 3},
	4 > \sqrt{3 \cdot 5},
	6 > \sqrt{5 \cdot 7},
	\dotsc,
\end{equation*}
故\begin{equation*}
	0 < \frac{1 \cdot 3 \cdot 5 \dotsm (2n-1)}{2 \cdot 4 \cdot 6 \dotsm (2n)}
	< \frac{1 \cdot 3 \cdot 5 \dotsm (2n-1)}{\sqrt{1 \cdot 3} \sqrt{3 \cdot 5} \sqrt{5 \cdot 7} \dotsm \sqrt{(2n-1)(2n+1)}}
	= \frac{1}{\sqrt{2n+1}}.
\end{equation*}
又因为\(\lim_{n\to\infty} \frac{1}{\sqrt{2n+1}} = 0\),
所以\begin{equation}\label{equation:数列极限.重要极限4}
	\lim_{n\to\infty} \frac{1 \cdot 3 \cdot 5 \dotsm (2n-1)}{2 \cdot 4 \cdot 6 \dotsm (2n)} = 0.
\end{equation}
\end{solution}
\end{example}

\begin{example}
求:\(\lim_{n\to\infty} \frac{2 \cdot 4 \cdot 6 \dotsm (2n)}{1 \cdot 3 \cdot 5 \dotsm (2n+1)}\).
\begin{solution}
因为\begin{equation*}
	2 \cdot 4 < 3^2,
	4 \cdot 6 < 5^2,
	\dotsc,
	(2n-2)(2n) < (2n-1)^2,
	(2n)(2n+2) < (2n+1)^2,
\end{equation*}
所以\begin{gather*}
	2 \cdot [4 \cdot 6 \dotsm \cdot (2n)]^2 \cdot (2n+2)
	< [3 \cdot 5 \dotsm (2n+1)]^2, \\
	[2 \cdot 4 \cdot 6 \dotsm (2n)]^2 \cdot (2n+2)
	< 2 \cdot [3 \cdot 5 \dotsm (2n+1)]^2, \\
	2 \cdot 4 \cdot 6 \dotsm (2n) \cdot \sqrt{2n+2}
	< \sqrt2 \cdot [3 \cdot 5 \dotsm (2n+1)], \\
	\frac{2 \cdot 4 \cdot 6 \dotsm (2n)}{1 \cdot 3 \cdot 5 \dotsm (2n+1)}
	< \frac{\sqrt2}{\sqrt{2n+2}}
	= \frac1{\sqrt{n+1}}.
\end{gather*}
又因为\(\lim_{n\to\infty} \frac1{\sqrt{n+1}} = 0\),
所以\begin{equation}\label{equation:数列极限.重要极限5}
	\lim_{n\to\infty} \frac{2 \cdot 4 \cdot 6 \dotsm (2n)}{1 \cdot 3 \cdot 5 \dotsm (2n+1)} = 0.
\end{equation}
\end{solution}
\end{example}

\begin{proposition}
设数列\(\{x_n\}\)满足\begin{equation*}
	\lim_{n\to\infty} \frac{x_{n+1}}{x_n} = \rho \in (-1,1),
\end{equation*}
那么\(\lim_{n\to\infty} x_n = 0\).
\begin{proof}
因为\(\lim_{n\to\infty} \frac{x_{n+1}}{x_n} = \rho\),
所以根据数列极限的定义,
\(\forall\epsilon>0\),
\(\exists N\in\mathbb{N}\),
\(\forall n\in\mathbb{N}\),
只要\(n > N\),
就有\(\rho-\epsilon < \frac{x_{n+1}}{x_n} < \rho+\epsilon\).
取\(r=\max\{\abs{\rho-\epsilon},\abs{\rho+\epsilon}\}\),
那么有\(\abs{\frac{x_{n+1}}{x_n}} < r\),
即\(\abs{x_{n+1}} < r \abs{x_n}\),
于是\(0 \leq \abs{x_{n+k}} < r^k \abs{x_n}\ (k=1,2,\dotsc)\);
又因为当\(\epsilon\)足够小时,必有\(0<r<1\),
所以\begin{equation*}
	\lim_{k\to\infty} r^k \abs{x_n} = 0;
\end{equation*}
那么由\hyperref[theorem:数列极限.夹逼准则]{夹逼准则}可知\begin{equation*}
	\lim_{k\to\infty} \abs{x_{n+k}} = 0;
\end{equation*}
再由\cref{theorem:极限.数列的绝对值的极限} 可知\begin{equation*}
	\lim_{k\to\infty} x_{n+k} = 0;
\end{equation*}
最后由\cref{example:极限.指标变化时数列极限不变} 可知\(\lim_{n\to\infty} x_n = 0\).
\end{proof}
\end{proposition}

\section{数列极限的四则运算}
\begin{theorem}\label{theorem:极限.数列极限的四则运算法则}
%@see: 《高等数学(第六版 上册)》 P45 定理4
%@see: 《数学分析(第二版 上册)》(陈纪修) P42 定理2.2.5
设数列\(\{x_n\}\)和\(\{y_n\}\)
满足\(\lim_{n\to\infty} x_n = A\)
和\(\lim_{n\to\infty} y_n = B\),
那么\begin{enumerate}
	\item \(\lim_{n\to\infty} (x_n \pm y_n) = A \pm B\);
	\item \(\lim_{n\to\infty} (x_n \cdot y_n) = A \cdot B\);
	\item 当\(y_n \neq 0\ (n=1,2,\dotsc,)\)且\(B \neq 0\)时,
	\(\lim_{n\to\infty}{\frac{x_n}{y_n}}=\frac{A}{B}\).
\end{enumerate}
\end{theorem}

\begin{corollary}
%@see: 《数学分析(第二版 上册)》(陈纪修) P42 定理2.2.5
设\(\lim_{n\to\infty} x_n = A,
\lim_{n\to\infty} y_n = B\),
\(a,b\)是常数,
则\(\lim_{n\to\infty} (a x_n + b y_n) = a A + b B\).
\end{corollary}

\begin{example}\label{example:极限.常数的方根的极限2}
%@see: 《数学分析(第二版 上册)》(陈纪修) P43 例2.2.10
设\(a>0\),证明:\(\lim_{n\to\infty} \sqrt[n]{a} = 1\).
\begin{proof}
在\cref{example:极限.常数的方根的极限1} 中我们已经证明
当\(a>1\)时\(\lim_{n\to\infty} \sqrt[n]{a} = 1\).
当\(a=1\)时,\(\sqrt[n]{a}\)恒等于\(1\),从而也有\(\lim_{n\to\infty} \sqrt[n]{a} = 1\).
现在考虑\(0<a<1\),这时\(\frac1a>1\),从而\(\lim_{n\to\infty} \sqrt[n]{\frac1a} = 1\),
那么利用\hyperref[theorem:极限.数列极限的四则运算法则]{极限的四则运算法则}可得
\(\lim_{n\to\infty} \sqrt[n]{a}
= \lim_{n\to\infty} \frac1{\sqrt[n]{1/a}} = 1\).
\end{proof}
\end{example}

\begin{example}
%@see: 《数学分析(第二版 上册)》(陈纪修) P43 例2.2.11
求极限\(\lim_{n\to\infty} n(\sqrt{n^2+1}-\sqrt{n^2-1})\).
\begin{solution}
直接计算得\begin{align*}
	&\hspace{-20pt}
	\lim_{n\to\infty} n(\sqrt{n^2+1}-\sqrt{n^2-1})
	= \lim_{n\to\infty} \frac{2n}{\sqrt{n^2+1}+\sqrt{n^2-1}} \\ % 分子有理化
	&= \lim_{n\to\infty} \frac2{\sqrt{1+(1/n)^2}+\sqrt{1-(1/n)^2}} % 分子分母同时除以\(n\)
	= 1.
\end{align*}
\end{solution}
\end{example}

\begin{corollary}
%@see: 《数学分析(第二版 上册)》(陈纪修) P43
设\(\lim_{n\to\infty} x_{in} = A_i\ (i=1,2,\dotsc,m)\),
则\begin{equation*}
	\lim_{n\to\infty} \sum_{i=1}^m x_{in} = \sum_{i=1}^m A_i,
\end{equation*}\begin{equation*}
	\lim_{n\to\infty} \prod_{i=1}^m x_{in} = \prod_{i=1}^m A_i.
\end{equation*}
\end{corollary}

\begin{remark}
%@see: 《数学分析(第二版 上册)》(陈纪修) P43
数列极限的四则运算法则只能推广到有限个数列相加或相乘的情况,
不能随意推广到无限个数列上去.
例如,若将\cref{theorem:极限.数列极限的四则运算法则} 随意推广,
可能会得出极限\begin{equation*}
	\lim_{n\to\infty} \left(
		\frac1{\sqrt{n^2+1}}
		+ \frac1{\sqrt{n^2+2}}
		+ \dotsb + \frac1{\sqrt{n^2+n}}
	\right)
\end{equation*}为\(0\)的错误结论;
但是,由于\begin{equation*}
	\frac{n}{\sqrt{n^2+n}}
	< \frac1{\sqrt{n^2+1}}
	+ \frac1{\sqrt{n^2+2}}
	+ \dotsb + \frac1{\sqrt{n^2+n}}
	< \frac{n}{\sqrt{n^2+1}},
\end{equation*}
利用\hyperref[theorem:数列极限.夹逼准则]{夹逼准则}可知该极限其实为\(1\).
\end{remark}

\begin{example}\label{example:极限.收敛数列前n项积的n次方根}
%@see: 《数学分析(第二版 上册)》(陈纪修) P44 例2.2.12
设\(a_n>0\),\(\lim_{n\to\infty} a_n = a\),证明:
\(\lim_{n\to\infty} \sqrt[n]{a_1 a_2 \dotsm a_n} = a\).
\begin{proof}
应用\hyperref[theorem:不等式.均值不等式]{均值不等式}可得\begin{equation*}
	\frac{a_1+a_2+\dotsb+a_n}n
	\geq \sqrt[n]{a_1 a_2 \dotsm a_n}
	\geq n\left(\frac1{a_1}+\frac1{a_2}+\dotsb+\frac1{a_n}\right)^{-1}
	> 0.
\end{equation*}
由\cref{example:极限.数列的算术平均的极限} 我们已经知道
\(\lim_{n\to\infty} \frac{a_1+a_2+\dotsb+a_n}{n} = a\).
要想利用\hyperref[theorem:数列极限.夹逼准则]{夹逼准则}证明上述问题,
就必须证明上面不等式中\(\sqrt[n]{a_1 a_2 \dotsm a_n}\)右边的数列极限也等于\(a\).

下面按\(a\)的不同取值,分两种情况讨论:
\begin{itemize}
	\item 当\(a=0\)时,
	有\(\sqrt[n]{a_1 a_2 \dotsm a_n} > 0 = a\).

	\item 当\(a>0\)时,
	由\hyperref[theorem:极限.数列极限的四则运算法则]{数列极限的四则运算法则}可知
	\(\lim_{n\to\infty} \frac1{a_n} = \frac1a\),
	故\begin{equation*}
		\lim_{n\to\infty} \frac1n \left(\frac1{a_1}+\frac1{a_2}+\dotsb+\frac1{a_n}\right) = \frac1a,
	\end{equation*}
	从而
	\(\lim_{n\to\infty} n\left(\frac1{a_1}+\frac1{a_2}+\dotsb+\frac1{a_n}\right)^{-1} = a\).
\end{itemize}
在上述两种情况下,\(\sqrt[n]{a_1 a_2 \dotsm a_n}\)右边的数列极限都等于\(a\),
所以\(\lim_{n\to\infty} \sqrt[n]{a_1 a_2 \dotsm a_n} = a\).
\end{proof}
\end{example}

\begin{example}
%@see: 《数学分析(第二版 上册)》(陈纪修) P45 习题 8.(1)
求极限\(\lim_{n\to\infty} \sqrt[n]{1+\frac12+\dotsb+\frac1n}\).
\begin{solution}
首先我们有\(1 \leq 1+\frac12+\dotsb+\frac1n \leq n\),
因为\(\lim_{n\to\infty} \sqrt[n]{1} = \lim_{n\to\infty} \sqrt[n]{n} = 1\),
所以\begin{equation*}
	\lim_{n\to\infty} \sqrt[n]{1+\frac12+\dotsb+\frac1n} = 1.
\end{equation*}
\end{solution}
\end{example}

\begin{example}
%@see: 《数学分析(第二版 上册)》(陈纪修) P45 习题 8.(2)
求极限\(\lim_{n\to\infty} \left(\frac1{n+\sqrt1}+\frac1{n+\sqrt2}+\dotsb+\frac1{n+\sqrt{n}}\right)\).
\begin{solution}
首先有\(\frac{n}{n+\sqrt{n}}
\leq \frac1{n+\sqrt1}+\frac1{n+\sqrt2}+\dotsb+\frac1{n+\sqrt{n}}
\leq \frac{n}{n+1}\).
因为\begin{equation*}
	\lim_{n\to\infty} \frac{n}{n+\sqrt{n}}
	= \lim_{n\to\infty} \frac{n}{n+1} = 1,
\end{equation*}
所以\(\lim_{n\to\infty} \left(\frac1{n+\sqrt1}+\frac1{n+\sqrt2}+\dotsb+\frac1{n+\sqrt{n}}\right) = 1\).
\end{solution}
\end{example}

\begin{example}
%@see: 《数学分析(第二版 上册)》(陈纪修) P45 习题 9.(7)
求极限\(\lim_{n\to\infty} \sqrt[n]{\frac1{n!}}\).
\begin{solution}\let\qed\relax
\begin{proof}[解法一]
因为\(\lim_{n\to\infty} \frac1n = 0\),
所以由\cref{example:极限.收敛数列前n项积的n次方根} 可知
\(\lim_{n\to\infty} \sqrt[n]{\frac1{n!}}
= \lim_{n\to\infty} \sqrt[n]{\frac11 \cdot \frac12 \dotsm \frac1n}
= 0\).
\end{proof}
\begin{proof}[解法二]
注意到\begin{equation*}
	\left(\frac1{n!}\right)^2
	= \prod_{k=1}^n \frac1{(n-k+1)k},%首尾相乘
\end{equation*}
而\(\frac1{(n-k+1)k} \leq \frac1n\),
\(\prod_{k=1}^n \frac1{(n-k+1)k} \leq \frac1{n^n}\),
\(\frac1{n!} \leq \sqrt{\frac1{n^n}}\),
\(0 < \sqrt[n]{\frac1{n!}} \leq \frac1{\sqrt{n}}\),
加之\(\lim_{n\to\infty} \frac1{\sqrt{n}} = 0\),
因此根据\hyperref[theorem:数列极限.夹逼准则]{夹逼准则}可得
\(\lim_{n\to\infty} \sqrt[n]{\frac1{n!}} = 0\).
\end{proof}
\end{solution}
\end{example}

\begin{example}
%@see: 《数学分析(第二版 上册)》(陈纪修) P45 习题 9.(8)
求极限\(\lim_{n\to\infty} \left(1-\frac1{2^2}\right) \left(1-\frac1{3^2}\right) \dotsm \left(1-\frac1{n^2}\right)\).
\begin{solution}
直接计算得\begin{align*}
	&\hspace{-20pt}
	\lim_{n\to\infty} \left(1-\frac1{2^2}\right) \left(1-\frac1{3^2}\right) \dotsm \left(1-\frac1{n^2}\right) \\
	&= \lim_{n\to\infty} \frac{1\cdot3}{2^2} \cdot \frac{2\cdot4}{3^2} \cdot \frac{3\cdot5}{4^2} \dotsm \frac{(n-2)n}{(n-1)^2} \cdot \frac{(n-1)(n+1)}{n^2} \\
	&= \lim_{n\to\infty} \frac{2n(n+1)}{2^2 n^2}
	= \lim_{n\to\infty} \frac{n+1}{2n}
	= \frac12.
\end{align*}
\end{solution}
\end{example}

\begin{example}
%@see: 《数学分析(第二版 上册)》(陈纪修) P45 习题 9.(9)
求极限\(\lim_{n\to\infty} \sqrt[n]{n \ln n}\).
\begin{solution}
因为\(n < n \ln n < n^2\ (n\geq3)\),
\(\lim_{n\to\infty} \sqrt[n]{n}
= \lim_{n\to\infty} \sqrt[n]{n^2} = 1\),
所以\(\lim_{n\to\infty} \sqrt[n]{n \ln n} = 1\).
\end{solution}
\end{example}

\begin{theorem}
%@see: 《数学分析(第二版 上册)》(陈纪修) P47 定理2.3.1
设\(x_n\neq0\),
则\(\{x_n\}\)是无穷大的充分必要条件是\(\{1/x_n\}\)是无穷小.
\begin{proof}
设\(\{x_n\}\)是无穷大.
对\(\forall\epsilon>0\),
取\(G = \frac1\epsilon\),
那么\begin{equation*}
	\text{\(\{x_n\}\)是无穷大}
	\implies
	(\exists N\in\mathbb{N})
	(\forall n\in\mathbb{N})
	\left[
		n>N
		\implies
		\abs{x_n} > G = \frac1\epsilon
		\implies
		\abs{\frac1{x_n}} < \epsilon
	\right],
\end{equation*}
即\(\{1/x_n\}\)是无穷小.

反过来,设\(\{1/x_n\}\)是无穷小.
对\(\forall G>0\),
取\(\epsilon = \frac1G\),
那么\begin{equation*}
	\text{\(\{1/x_n\}\)是无穷小}
	\implies
	(\exists N\in\mathbb{N})
	(\forall n\in\mathbb{N})
	\left[
		n>N
		\implies
		\abs{\frac1{x_n}} < \epsilon = \frac1G
		\implies
		\abs{x_n} > G
	\right],
\end{equation*}
即\(\{x_n\}\)是无穷大.
\end{proof}
\end{theorem}

关于无穷大的运算,如下的性质是显然的:
同号无穷大之和仍然是该符号的无穷大,
而异号无穷大之差是与被减无穷大的符号相同的无穷大,
无穷大与有界量之和或之差仍然是无穷大,
同号无穷大之积是正无穷大,
异号无穷大之积是负无穷大,
\begin{gather*}
	(+\infty) + (+\infty) = +\infty, \\
	(-\infty) + (-\infty) = -\infty, \\
	(+\infty) - (-\infty) = +\infty, \\
	(-\infty) - (+\infty) = -\infty, \\
	(+\infty) \cdot (+\infty) = +\infty, \\
	(-\infty) \cdot (-\infty) = +\infty, \\
	(+\infty) \cdot (-\infty) = -\infty, \\
	(-\infty) \cdot (+\infty) = -\infty.
\end{gather*}

\begin{theorem}%无穷大与无界量之积还是无穷大
%@see: 《数学分析(第二版 上册)》(陈纪修) P47 定理2.3.2
设数列\(\{x_n\}\)是无穷大,若当\(n>N\)时\(\abs{y_n}\geq\delta>0\)成立,
则\(\{x_n y_n\}\)是无穷大.
\end{theorem}

\begin{corollary}%无穷大与收敛于非零数的数列之积还是无穷大
%@see: 《数学分析(第二版 上册)》(陈纪修) P48 推论
设\(\{x_n\}\)是无穷大,
\(\lim_{n\to\infty} y_n = b \neq 0\),
则\(\{x_n y_n\}\)与\(\{x_n/y_n\}\)都是无穷大.
\end{corollary}

\section{施托尔茨定理}
\begin{theorem}[施托尔茨定理I]\label{theorem:极限.施托尔茨定理1}
%@see: 《数学分析(第二版 上册)》(陈纪修) P49 定理2.3.3
设数列\(\{x_n\},\{y_n\}\).
若\(\{y_n\}\)是严格单调增加的正无穷大,
且\begin{equation*}
	\lim_{n\to\infty} \frac{x_{n+1}-x_n}{y_{n+1}-y_n}
	= C
	\in[-\infty,+\infty],
\end{equation*}
那么有\begin{equation*}
	\lim_{n\to\infty} \frac{x_n}{y_n}
	= C.
\end{equation*}
%TODO proof
% \begin{proof}
% 设\(\lim_{n\to\infty} \frac{x_{n+1}-x_n}{y_{n+1}-y_n} = C\in(-\infty,+\infty)\),
% 故对\(\forall\epsilon>0\),\(\exists N>0\),当\(n > N\)时,有\begin{equation*}
% \abs{\frac{x_{n+1}-x_n}{y_{n+1}-y_n} - C} < \epsilon
% \end{equation*}成立,即\begin{equation*}
% C - \epsilon < \frac{x_{n+1}-x_n}{y_{n+1}-y_n} < C + \epsilon;
% \end{equation*}
% 由于\begin{equation*}
% \frac{
% (x_{N+1}-x_N)
% + (x_{N+2}-x_{N+1})
% + \dotsb
% + (x_{n+1}-x_n)
% }{
% (y_{N+1}-y_N)
% + (y_{N+2}-y_{N+1})
% + \dotsb
% + (y_{n+1}-y_n)
% }
% = \frac{x_{n+1}-x_N}{y_{n+1}-y_N},
% \end{equation*}再根据\cref{example:不等式.不同浓度的溶液的混合},于是有\begin{equation*}
% C - \epsilon <
% \frac{x_{n+1}-x_N}{y_{n+1}-y_N}
% < C + \epsilon;
% \end{equation*}进而有\begin{equation*}
% \lim_{n\to\infty} \frac{x_{n+1}-x_N}{y_{n+1}-y_N}
% = \lim_{n\to\infty} \frac{\frac{x_{n+1}}{y_{n+1}}-\frac{x_N}{y_{n+1}}}{1-\frac{y_N}{y_{n+1}}}
% = C;
% \end{equation*}又因为\(\lim_{n\to\infty} y_n = +\infty\),
% 注意到\(x_N\)、\(y_N\)是不随\(n\)变化的常数,
% 于是有\(\lim_{n\to\infty} \frac{x_N}{y_{n+1}}
% = \lim_{n\to\infty} \frac{y_N}{y_{n+1}}
% = 0\),从而有\(\lim_{n\to\infty} \frac{x_{n+1}}{y_{n+1}} = C\).
% \end{proof}
\end{theorem}

\begin{theorem}[施托尔茨定理II]\label{theorem:极限.施托尔茨定理2}
设数列\(\{x_n\},\{y_n\}\).
若\(\{x_n\}\)是无穷小,
\(\{y_n\}\)是严格单调减少的无穷小,且\begin{equation*}
	\lim_{n\to\infty} \frac{x_{n+1}-x_n}{y_{n+1}-y_n}
	= C
	\in[-\infty,+\infty],
\end{equation*}
那么有\begin{equation*}
	\lim_{n\to\infty} \frac{x_n}{y_n}
	= C.
\end{equation*}
%TODO proof
\end{theorem}

\begin{remark}
% 在以后我们会觉察到,施托尔茨定理实际上是离散形式的洛必达法则.
可以证明:如果把上述定理中的极限值\(C\)改为不定号无穷大,则结论不成立.
\end{remark}

\begin{example}
%@see: 《数学分析(第二版 上册)》(陈纪修) P50
利用施托尔茨定理重新证明\cref{example:极限.数列的算术平均的极限}.
\begin{proof}
令\(x_n=a_1+a_2+\dotsb+a_n,y_n=n\),
由\cref{theorem:极限.施托尔茨定理1} 有\begin{equation*}
	\lim_{n\to\infty} \frac{x_n}{y_n}
	= \lim_{n\to\infty} \frac{x_{n+1}-x_n}{y_{n+1}-y_n}
	= \lim_{n\to\infty} \frac{a_{n+1}}{1}
	= a.
	\qedhere
\end{equation*}
\end{proof}
\end{example}

\begin{example}
%@see: 《数学分析(第二版 上册)》(陈纪修) P50 例2.3.4
设\(p>-1\).
证明:\begin{equation}\label{equation:数列极限.重要极限X}
	\lim_{n\to\infty} \frac{1^p+2^p+\dotsb+n^p}{n^{p+1}}
	= \frac{1}{p+1}.
\end{equation}
\begin{proof}
令\(x_n=1^p+2^p+\dotsb+n^p,
y_n=n^{p+1}\),
由\cref{theorem:极限.施托尔茨定理1} 有\begin{align*}
	\lim_{n\to\infty} \frac{x_n}{y_n}
	&= \lim_{n\to\infty} \frac{x_{n+1}-x_n}{y_{n+1}-y_n}
	= \lim_{n\to\infty} \frac{(n+1)^p}{(n+1)^{p+1} - n^{p+1}} \\
	&= \lim_{n\to\infty} \frac{n^p + p n^{p-1} + \dotsb + 1}
		{(p+1) n^p + \frac{1}{2} (p+1) p n^{p-1} + \dotsb + 1} \\
	&= \frac{1}{p+1}.
	\qedhere
\end{align*}
\end{proof}
\end{example}

\begin{example}
%@see: 《数学分析(第二版 上册)》(陈纪修) P50 例2.3.5
设\(\lim_{n\to\infty} a_n = a\),
求极限\(\lim_{n\to\infty} \frac{a_1+2a_2+\dotsb+na_n}{n^2}\).
\begin{solution}
令\(x_n=a_1+2a_2+\dotsb+na_n,
y_n=n^2\),
由\cref{theorem:极限.施托尔茨定理1} 有\begin{equation*}
	\lim_{n\to\infty} \frac{x_n}{y_n}
	= \lim_{n\to\infty} \frac{x_{n+1}-x_n}{y_{n+1}-y_n}
	= \lim_{n\to\infty} \frac{(n+1)a_{n+1}}{(n+1)^2-n^2}
	= \lim_{n\to\infty} \frac{(n+1)a_{n+1}}{2n+1}
	= \frac12 a.
\end{equation*}
\end{solution}
\end{example}

\section{收敛准则}
\subsection{单调有界数列收敛定理}
由\cref{theorem:极限.收敛数列的有界性} 可知,收敛的数列一定有界.
但是,我们也知道,有界的数列不一定收敛,例如\begin{equation*}
	\{ x_n = \sin n \}, \qquad
	\{ y_n = (-1)^n \}.
\end{equation*}
于是我们有这样两个问题:
对有界数列加上什么条件,就可以保证它必定收敛?
不对有界数列加任何条件,我们可以得到怎样的(比收敛稍弱一些的)结论?

我们先来回答第一个问题:
如果数列不仅有界,而且是单调的,那么这数列一定收敛,
其极限就是它的值域的上确界或下确界.

\begin{theorem}\label{theorem:极限.数列的单调有界定理}
%@see: 《高等数学(第六版 上册)》 P52 准则II
%@see: 《数学分析(第二版 上册)》(陈纪修) P52 定理2.4.1
%@see: 《数学分析:原理与方法》(胡适耕) P33 4.1.3定理
单调有界数列必有极限.
\begin{proof}
不妨设数列\(\{a_n\}\)是单调增加的,
即\begin{equation*}
	a_n \leq a_{n+1},
	\quad n=1,2,\dotsc;
	\eqno(1)
\end{equation*}
又设\(\{a_n\}\)有界,
且\begin{equation*}
	\abs{a_n} < c,
	\quad n=1,2,\dotsc.
	\eqno(2)
\end{equation*}

现在我们把连续统分成两个集合\(A\)和\(B\),
把大于所有\(a_n\)的任何实数(例如数\(c\))放入集合\(B\),
而把其余的所有实数放入\(A\),即取\begin{equation*}
	B = \Set{ x \in \mathbb{R} \given x > a_n\ (n=1,2,\dotsc) },
	\eqno(3)
\end{equation*}\begin{equation*}
	A = \mathbb{R} - B.
	\eqno(4)
\end{equation*}
显然\(\Set{A,B}\)是\(\mathbb{R}\)的一个分割.
设\(\alpha\)是这个分割的界限,
那么必有\begin{equation*}
	a_n \leq \alpha,
	\quad n=1,2,\dotsc;
	\eqno(5)
\end{equation*}
这是因为假设这个数列的某一项\(a_m\)满足\(a_m > \alpha\),
依照界限的定义会有\(a_m \in B\),而这与\(B\)的定义式(3)矛盾.

假设“\(\alpha\)不是\(\{a_n\}\)的极限”,
根据数列发散的定义,\begin{equation*}
	(\exists\epsilon>0)
	(\forall n\in\mathbb{N}^+)
	(\exists n_0>n)
	[\abs{a_{n_0} - \alpha} > \epsilon]
\end{equation*}成立;
由(5)可知,\(\abs{a_{n_0} - \alpha} = \alpha - a_{n_0}\);
又因为\(\{a_n\}\)是单调增加的,
所以\(a_{n_0} \geq a_n\),
\(-a_{n_0} \leq -a_n\),
\(\alpha - a_{n_0} \leq \alpha - a_n\).
因此,\(\exists\epsilon>0\),对\(\forall n\in\mathbb{N}^+\),都有\begin{equation*}
	\alpha - a_n > \epsilon
	\quad\text{或}\quad
	a_n < \alpha - \epsilon.
	\eqno(6)
\end{equation*}
结合集合\(B\)的定义(3),
由(6)便得\(\alpha - \epsilon \in B\);
但由\(\alpha - \epsilon < \alpha\)可知,应该有\((\alpha - \epsilon) \in A\);
矛盾!因此假设不成立,\(\alpha\)就是数列\(\{a_n\}\)的极限,
即\(\lim_{n\to\infty} a_n = \alpha\).
\end{proof}
\end{theorem}
\begin{remark}
在\cref{theorem:极限.数列的单调有界定理} 的证明中,
我们有\(-c < a_n < c\ (n=1,2,\dotsc)\),
那么根据\cref{theorem:极限.收敛数列的保序性2}
不难得知\begin{equation*}
	-c \leq \lim_{n\to\infty} a_n \leq c.
\end{equation*}
\end{remark}

按极限的定义证明一个数列收敛必须提前知道它的极限是什么.
这个要求对于许多实际情况来说并不现实,
我们往往无法事先得知收敛数列的极限.
\cref{theorem:极限.数列的单调有界定理} 使得我们可以从数列本身出发去研究其敛散性,
在确定数列数列收敛以后,再利用极限运算的性质去求出相应的极限.

\begin{example}
%@see: 《计算机组成原理(第3版)》(唐朔飞) P11
设\(a\geq0\),
数列\(\{x_n\}\)满足\(x_{n+1} = \frac12\left(x_n+\frac{a}{x_n}\right)\ (n=1,2,\dotsc)\).
证明:只要首项\(x_1>0\),数列\(\{x_n\}\)总是收敛,并求它的极限.
\begin{solution}
先设\(x_1=\sqrt{a}\).
于是\begin{equation*}
	x_2 = \frac12 \left(x_1+\frac{a}{x_1}\right)
	= \sqrt{a},
\end{equation*}
利用数学归纳法易证\(x_n = \sqrt{a}\ (n=1,2,\dotsc)\),
于是\(\{x_n\}\)是常数列,
\(\lim_{n\to\infty} x_n = \sqrt{a}\).

再设\(x_1\neq\sqrt{a}\)且\(x_1>0\).
既然\(x_1>0\),利用数学归纳法容易证明\(x_n>0\ (n=1,2,\dotsc)\),
于是可以利用\hyperref[theorem:不等式.基本不等式2推论1]{均值不等式}得到\begin{equation*}
	x_2
	= \frac12\left(x_1+\frac{a}{x_1}\right)
	> \sqrt{x_1 \cdot \frac{a}{x_1}}
	= \sqrt{a},
\end{equation*}
利用数学归纳法易证\(x_n > \sqrt{a}\ (n=2,3,\dotsc)\),
数列\(\{x_n\}\)有界.
又因为\(x_n^2 > a\ (n=2,3,\dotsc)\),
所以\begin{align*}
	x_{n+1}-x_n
	&= \frac12 \left(\frac{a}{x_n}-x_n\right)
	= \frac{a-x_n^2}{2x_n}
	< 0
	\quad(n=2,3,\dotsc).
\end{align*}
数列\(\{x_n\}\)从第二项开始严格单调减少.
由\hyperref[theorem:极限.数列的单调有界定理]{单调有界定理}可知数列\(\{x_n\}\)收敛.
假设\(\lim_{n\to\infty} x_n = x\).
在等式\(x_{n+1} = \frac12\left(x_n+\frac{a}{x_n}\right)\)
两边同时求极限,
得到方程\(x = \frac12\left(x+\frac{a}{x}\right)\),
解得\(x = \sqrt{a}\).
%@credit: {5f4d2f8a-fc8b-4798-85d6-98670f6761e7},{ce603838-a24d-4616-9395-d7b223e8cb72},{6c964576-9569-472e-969e-54699e35974b}
\end{solution}
\end{example}

\begin{example}
%@see: 《数学分析(第二版 上册)》(陈纪修) P53 例2.4.1
设数列\(\{x_n\}\)满足\(x_1>0\),且有递推公式\(x_{n+1}=1+\frac{x_n}{1+x_n}\ (n=1,2,\dotsc)\).
证明:\(\{x_n\}\)收敛,并求它的极限.
\begin{solution}
首先,应用数学归纳法可以直接得到\(1<x_n<2\ (n\geq2)\).
然后由递推公式可得\begin{equation*}
	x_{n+1}-x_n = \frac{x_n-x_{n-1}}{(1+x_n)(1+x_{n-1})}.
\end{equation*}
这说明,对于\(\forall n\geq2\),\(x_{n+1}-x_n\)具有相同的符号,从而\(\{x_n\}\)是单调数列.
由\cref{theorem:极限.数列的单调有界定理} 可知,\(\{x_n\}\)收敛.

设\(\{x_n\}\)的极限是\(a\).
在递推公式等号两边同时求极限,
得到方程\begin{equation*}
	a = 1 + \frac{a}{1+a},
\end{equation*}
解得\(a = \frac{1\pm\sqrt5}2\).
由\(x_n>0\),舍去负值,即有\(\lim_{n\to\infty} x_n = \frac{1+\sqrt5}2\).
\end{solution}
\end{example}

\begin{example}
%@see: 《数学分析(第二版 上册)》(陈纪修) P53 例2.4.2
设数列\(\{x_n\}\)满足\(0<x_1<1\),且有递推公式\(x_{n+1}=x_n(1-x_n)\ (n=1,2,\dotsc)\).
证明:\(\{x_n\}\)收敛,并求它的极限.
\begin{solution}
应用数学归纳法,可以得到\(0<x_n<1\ (n=1,2,\dotsc)\).
然后由递推公式可得\begin{equation*}
	x_{n+1}-x_n = -x_n^2 < 0.
\end{equation*}
这说明\(\{x_n\}\)单调减少有下界.
由\cref{theorem:极限.数列的单调有界定理} 可知,\(\{x_n\}\)收敛.

设\(\{x_n\}\)的极限是\(a\).
在递推公式等号两边同时求极限,
得到方程\begin{equation*}
	a=a(1-a),
\end{equation*}
解得\(a=0\).
于是\(\lim_{n\to\infty} x_n = 0\).
\end{solution}
\end{example}

\begin{example}
%@see: 《数学分析(第二版 上册)》(陈纪修) P54 例2.4.3
设数列\(\{x_n\}\)满足\(x_1=\sqrt2\),且有递推公式\(x_{n+1}=\sqrt{3+2x_n}\ (n=1,2,\dotsc)\).
证明:\(\{x_n\}\)收敛,并求它的极限.
\begin{solution}
对\(0<x_1=\sqrt2<3\)应用数学归纳法可得\(0<x_n<3\ (n=1,2,\dotsc)\).
然后由递推公式可得\begin{equation*}
	x_{n+1}-x_n = \sqrt{3+2x_n} - x_n
	= \frac{(3-x_n)(1+x_n)}{\sqrt{3+2x_n}+x_n}
	> 0,
\end{equation*}
这说明\(\{x_n\}\)单调增加且有上界.
由\cref{theorem:极限.数列的单调有界定理} 可知,\(\{x_n\}\)收敛.
设\(\{x_n\}\)的极限是\(a\).
在递推公式等号两边同时求极限,
得到方程\begin{equation*}
	a=\sqrt{3+2a},
\end{equation*}
解得\(a\in\{3,-1\}\).
由\(x_n>0\),舍去负值,即有\(\lim_{n\to\infty} x_n = 3\).
\end{solution}
\end{example}

\subsection{无理数\texorpdfstring{$\pi$}{\textpi}}\label{section:极限.无理数pi}
设单位圆内接正\(n\)边形的半周长为\(L_n\),
则\(L_n = n \sin\frac{180^\circ}{n}\).
%@see: 《数学分析(第二版 上册)》(陈纪修) P56 例2.4.5
令\(t=\frac{180^\circ}{n(n+1)}\).
当\(n\geq3\)时,\(nt\leq45^\circ\).
于是\begin{equation*}
	\tan nt
	= \frac{\tan(n-1)t + \tan t}{1 - \tan(n-1)t \tan t}
	\geq \tan(n-1)t + \tan t
	\geq \dotsb \geq n \tan t,
\end{equation*}
从而\begin{equation*}
	\sin(n+1)t = \sin nt \cos t + \cos nt \sin t
	= \sin nt \cos t \left(1 + \frac{\tan t}{\tan nt}\right)
	\leq \frac{n+1}n \sin nt.
\end{equation*}
所以,当\(n\geq3\)时,
\(L_n = n \sin\frac{180^\circ}{n}
\leq (n+1) \sin\frac{180^\circ}{n+1} = L_{n+1}\).

另一方面,单位圆内接正\(n\)边形的面积\begin{equation*}
	S_n = n \sin\frac{180^\circ}{n} \cos\frac{180^\circ}{n} < 4.
\end{equation*}
因此,当\(n\geq3\)时,
\(L_n = n \sin\frac{180^\circ}{n}
< 4 \left(\cos\frac{180^\circ}{n}\right)^{-1}
\leq \frac4{\cos60^\circ}
= 8\).

综上所述,数列\(\{L_n\}\)单调增加且有上界,因此收敛.
我们把\(\lim_{n\to\infty} L_n\)称为圆周率,记作\(\pi\),
即\begin{equation*}
	\pi \defeq \lim_{n\to\infty} n \sin\frac{180^\circ}{n}.
\end{equation*}

%证明\(\pi\)是无理数:
%@see: https://mathscholar.org/2018/09/simple-proofs-the-irrationality-of-pi/
%@see: https://math.stackexchange.com/questions/2588293/a-simple-proof-that-pi-is-irrational-by-ivan-niven
%@see: https://proofwiki.org/wiki/Pi_is_Irrational
%@see: https://www.bilibili.com/read/cv7988106/

\subsection{无理数\texorpdfstring{$e$}{e}}\label{section:极限.无理数e}
设\(x_n=\left(1+\frac1n\right)^n,
y_n=\left(1+\frac1n\right)^{n+1}\).
我们接下来证明数列\(\{x_n\}\)单调增加,
数列\(\{y_n\}\)单调减少,
且两者收敛于同一个极限.

由\hyperref[theorem:不等式.基本不等式n几何平均数与算术平均数]{基本不等式}可得\begin{equation*}
	x_n = \left(1+\frac1n\right)^n \cdot 1
	\leq \left[\frac{n\left(1+\frac1n\right)+1}{n+1}\right]^{n+1} = x_{n+1},
\end{equation*}
和\begin{equation*}
	\frac1{y_n} = \left(\frac{n}{n+1}\right)^{n+1} \cdot 1
	\leq \left[\frac{(n+1)\frac{n}{n+1}+1}{n+2}\right]^{n+2} = \frac1{y_{n+1}}.
\end{equation*}
可见\(\{x_n\}\)单调增加,\(\{y_n\}\)单调减少.
又因为\begin{equation*}
	2 = x_1 \leq x_n < y_n \leq y_1 = 4,
\end{equation*}
所以\(\{x_n\},\{y_n\}\)都收敛.
因为\(y_n=x_n\left(1+\frac1n\right)\),
所以\(\{x_n\},\{y_n\}\)具有相同的极限.

我们习惯上用字母\(e\)表记极限\(\lim_{n\to\infty} \left(1+\frac1n\right)^n\),
把它称为\DefineConcept{纳皮尔--欧拉常数}(Napier--Euler Constant).
它在数值上近似等于{2.718~282}.
%@Mathematica: N[E, 7]
%@see: https://mathworld.wolfram.com/e.html

可以验证:\begin{gather}
	e^x = \lim_{n\to\infty} \left(1+\frac{x}{n}\right)^n,
	\label{equation:特殊函数.以e为底的指数函数} \\
	\ln x = \lim_{n\to\infty} n \left( \sqrt[n]{x} - 1 \right).
	\label{equation:特殊函数.以e为底的对数函数}
\end{gather}

\subsection{闭区间套定理}
\begin{definition}\label{definition:极限.闭区间套的定义}
%@see: 《数学分析(第二版 上册)》(陈纪修) P60 定义2.4.1
闭区间序列\(\{[a_n,b_n]\}\)如果满足\begin{itemize}
	\item \([a_{n+1},b_{n+1}] \subseteq [a_n,b_n]\ (n=1,2,\dotsc)\);
	\item \(\lim_{n\to\infty} (b_n - a_n) = 0\),
\end{itemize}
就称该序列为\DefineConcept{闭区间套}(nested intervals).
\end{definition}

\begin{theorem}[闭区间套定理]\label{definition:极限.闭区间套定理}
%@see: 《数学分析(第二版 上册)》(陈纪修) P60 定理2.4.2
如果序列\(\{[a_n,b_n]\}\)是闭区间套,
那么\begin{equation*}
	(\exists!\xi\in\mathbb{R})
	\left[
		\xi \in \bigcap_{i=1}^\infty [a_i,b_i]
		\land
		\lim_{n\to\infty} a_n = \lim_{n\to\infty} b_n = \xi
	\right].
\end{equation*}
\begin{proof}
根据闭区间套的定义,我们有\begin{equation*}
	a_1 \leq a_2 \leq \dotsb \leq a_{n-1} \leq a_n
	< b_n \leq b_{n-1} \leq \dotsb \leq b_2 \leq b_1.
\end{equation*}
显然数列\(\{a_n\}\)单调增加且有上界\(b_1\),
数列\(\{b_n\}\)单调减少而有下界\(\{a_1\}\).
根据\hyperref[theorem:极限.数列的单调有界定理]{数列的单调有界定理},
\(\{a_n\}\)与\(\{b_n\}\)都收敛.

设\(\lim_{n\to\infty} a_n = \xi\),
则\begin{equation*}
	\lim_{n\to\infty} b_n
	= \lim_{n\to\infty} [(b_n - a_n) + a_n]
	= \lim_{n\to\infty} (b_n - a_n) + \lim_{n\to\infty} a_n
	= \xi.
\end{equation*}
由于\(\xi\)是\(\{a_n\}\)的值域的上确界,
也是\(\{b_n\}\)的值域的下确界,
于是有\(a_n \leq \xi \leq b_n\ (n=1,2,\dotsc)\),
即\(\xi\)属于所有的闭区间\([a_n,b_n]\).
假设另有实数\(\eta\)属于所有的闭区间\([a_n,b_n]\),
则也有\(a_n \leq \eta \leq b_n\ (n=1,2,\dotsc)\).
令\(n\to\infty\),
由\hyperref[theorem:数列极限.夹逼准则]{夹逼准则}得到\(\eta=\lim_{n\to\infty} a_n=\xi\).
这就说明满足结论的实数\(\xi\)是唯一的.
\end{proof}
\end{theorem}

需要指出,如果将\cref{definition:极限.闭区间套定理} 条件中的闭区间套改为开区间套,
则数列\(\{a_n\},\{b_n\}\)依然收敛于同一个极限\(\xi\),
但是这个\(\xi\)可能不属于任何一个开区间\((a_n,b_n)\).
例如,令\(a_n = 0, b_n = \frac1n\),
易见\(\lim_{n\to\infty} a_n = \lim_{n\to\infty} b_n = 0\),
但是\(0\)不属于任何一个开区间\((a_n,b_n)\).

\subsection{子列极限}
上、下极限是数列极限的必要组成部分,
它们各有三种等价的描述方式,
或者说三种等价的定义.
给定一种定义后,
其余两种定义的内容可以命题或定理的形式得到证明.

\begin{definition}
若数列\(\{x_n\}\)的子列\(\{x_{n_k}\}\)满足\begin{equation*}
	\lim_{k\to\infty} x_{n_k} = \xi,
\end{equation*}
则称数\(\xi\)为“数列\(\{x_n\}\)的\DefineConcept{子列极限}%
(或\DefineConcept{极限点}(limit point)、\DefineConcept{聚点})”.
\end{definition}

\begin{theorem}\label{theorem:子列极限.数列收敛的充分必要条件}
%@see: 《高等数学(第六版 上册)》 P30 定理4
%@see: 《数学分析(第二版 上册)》(陈纪修) P62 定理2.4.4
如果数列\(\{x_n\}\)收敛于\(a\),
那么它的任一子列也收敛于\(a\).
\begin{proof}
设数列\(\{x_{n_k}\}\)是数列\(\{x_n\}\)的任一子数列.
由于\(\lim_{n\to\infty}x_n = a\),
故\(\forall \epsilon > 0\),
\(\exists N \in \mathbb{N}^+\),
当\(n > N\)时,
\(\abs{x_n - a} < \epsilon\)成立.
取\(K = N\),
则当\(k > K\)时,
由\(n_k > n_K \geq N\)
得\(\abs{x_{n_k} - a} < \epsilon\),
也就是说\(\lim_{k\to\infty}x_{n_k} = a\).
\end{proof}
\end{theorem}

\cref{theorem:子列极限.数列收敛的充分必要条件} 经常被用来判断一个数列是否发散.
\begin{corollary}\label{theorem:子列极限.具有不同子列极限的数列必定发散}
%@see: 《高等数学(第六版 上册)》 P30
%@see: 《数学分析(第二版 上册)》(陈纪修) P62 推论
如果数列\(\{x_n\}\)的两个子列收敛于不同的极限,那么数列\(\{x_n\}\)是发散的.
\end{corollary}

\begin{example}\label{example:极限.负1的次幂组成的数列发散}
%@see: 《高等数学(第六版 上册)》 P30
数列\(\{x_n=(-1)^{n+1}\}\)的子数列\(\{x_{2k-1}\}\)收敛于\(1\),
而其子数列\(\{x_{2k}\}\)收敛于\(-1\),因此数列\(\{x_n\}\)是发散的.
同时这个例子也说明,一个发散的数列也可能有收敛的子数列.
\end{example}

\subsection{上极限与下极限}
我们介绍两种特殊的子列极限.
\begin{definition}
数列\(\{x_n\}\)的最小子列极限称为“数列\(x_n\)的\DefineConcept{下极限}”,
记作\begin{equation*}
	\varliminf_{n\to\infty} x_n
	\quad\text{或}\quad
	\liminf_{n\to\infty} x_n.
\end{equation*}
\end{definition}

\begin{definition}
数列\(\{x_n\}\)的最大子列极限称为“数列\(x_n\)的\DefineConcept{上极限}”,
记作\begin{equation*}
	\varlimsup_{n\to\infty} x_n
	\quad\text{或}\quad
	\limsup_{n\to\infty} x_n.
\end{equation*}
\end{definition}
上面对上下极限的定义可以利用“\(\epsilon-N\)语言”更加简洁精确地重新定义为\begin{align*}
	\varliminf_{n\to\infty} x_n = \alpha
	&\defiff
	(\forall\epsilon>0)
	(\exists N\in\mathbb{N})
	(\forall n\in\mathbb{N})
	[
		n>N
		\implies
		\alpha-\epsilon < x_n
	]; \\
	\varlimsup_{n\to\infty} x_n = \beta
	&\defiff
	(\forall\epsilon>0)
	(\exists N\in\mathbb{N})
	(\forall n\in\mathbb{N})
	[
		n>N
		\implies
		x_n < \beta+\epsilon
	].
\end{align*}

\begin{theorem}\label{theorem:极限.上下极限的等价定义1}
数列\(\{x_n\}\)的上下极限满足:\begin{equation*}
	\varliminf_{n\to\infty} x_n
	= \lim_{n\to\infty} \inf_{k\geq n} x_k
	= \lim_{n\to\infty} \inf\{x_n,x_{n+1},\dotsc\},
\end{equation*}\begin{equation*}
	\varlimsup_{n\to\infty} x_n
	= \lim_{n\to\infty} \sup_{k\geq n} x_k
	= \lim_{n\to\infty} \sup\{x_n,x_{n+1},\dotsc\}.
\end{equation*}
\end{theorem}
\cref{theorem:极限.上下极限的等价定义1}
也是数列的上下极限的一种等价定义.

由\cref{theorem:子列极限.数列收敛的充分必要条件} 立即可得如下结论.
\begin{corollary}
%@see: 《数学分析(第二版 下册)》(陈纪修) P10 定理9.2.2'
对于数列\(\{x_n\}\),
总有\begin{equation*}
	\lim_{n\to\infty} x_n = a
	\iff
	\varliminf_{n\to\infty} x_n
	= \varlimsup_{n\to\infty} x_n
	= a.
\end{equation*}
\end{corollary}

\begin{example}
%@see: 《数学分析(第二版 下册)》(陈纪修) P12 定理9.2.4 (1)
设\(\{a_n\},\{b_n\}\)都是数列.
证明:\begin{gather}
	\varliminf_{n\to\infty} a_n
	+ \varliminf_{n\to\infty} b_n
	\leq \varliminf_{n\to\infty} (a_n + b_n)
	\leq \varliminf_{n\to\infty} a_n
	+ \varlimsup_{n\to\infty} b_n. \\
	\varliminf_{n\to\infty} a_n
	+ \varlimsup_{n\to\infty} b_n
	\leq \varlimsup_{n\to\infty} (a_n + b_n)
	\leq \varlimsup_{n\to\infty} a_n
	+ \varlimsup_{n\to\infty} b_n.
\end{gather}
%TODO proof
%\cref{example:实数.确界的序}
\end{example}

\begin{example}
%@see: 《数学分析(第二版 下册)》(陈纪修) P12 定理9.2.4 (2)
设\(\{a_n\},\{b_n\}\)都是数列,
极限\(\lim_{n\to\infty} a_n\)存在,
则\begin{gather*}
	\varlimsup_{n\to\infty} (a_n + b_n)
	= \lim_{n\to\infty} a_n + \varlimsup_{n\to\infty} b_n, \\
	\varliminf_{n\to\infty} (a_n + b_n)
	= \lim_{n\to\infty} a_n + \varliminf_{n\to\infty} b_n.
\end{gather*}
%TODO proof
\end{example}

\begin{example}
%@see: 《数学分析(第二版 下册)》(陈纪修) P13 定理9.2.5 (1)
设\(\{a_n\},\{b_n\}\)都是数列.
若\(a_n,b_n\geq0\),
则\begin{gather*}
	\varlimsup_{n\to\infty} (a_n b_n)
	\leq \varlimsup_{n\to\infty} a_n \cdot \varlimsup_{n\to\infty} b_n, \\
	\varliminf_{n\to\infty} (a_n b_n)
	\geq \varliminf_{n\to\infty} a_n \cdot \varliminf_{n\to\infty} b_n.
\end{gather*}
%TODO proof
\end{example}

\begin{example}
%@see: 《数学分析(第二版 下册)》(陈纪修) P13 定理9.2.5 (2)
设\(\{a_n\},\{b_n\}\)都是数列.
若\(\lim_{n\to\infty} a_n = a \in (0,+\infty)\),
则\begin{gather*}
	\varlimsup_{n\to\infty} (a_n b_n)
	= \lim_{n\to\infty} a_n \cdot \varlimsup_{n\to\infty} b_n, \\
	\varliminf_{n\to\infty} (a_n b_n)
	= \lim_{n\to\infty} a_n \cdot \varliminf_{n\to\infty} b_n.
\end{gather*}
%TODO proof
\end{example}

\begin{example}
%@see: 《高等数学(第六版 上册)》 P31 习题1-2 4.
若\(\lim_{n\to\infty} u_n = a\),
证明\(\lim_{n\to\infty} \abs{u_n} = \abs{a}\).
并举例说明:
即便数列\(\{\abs{x_n}\}\)有极限,
数列\(\{x_n\}\)也未必有极限.
\begin{proof}
因为\(\lim_{n\to\infty} u_n = a\),
所以\begin{equation*}
	(\forall\epsilon>0)
	(\exists N\in\mathbb{N})
	(\forall n\in\mathbb{N})
	[n>N \implies \abs{u_n-a}<\epsilon].
\end{equation*}
由\hyperref[theorem:不等式.三角不等式2]{三角不等式}有
\(\abs{\abs{u_n}-\abs{a}}\leq\abs{u_n-a}\).
于是当\(n>N\)时,
也有\(\abs{\abs{u_n}-\abs{a}}<\epsilon\)成立,
从而有\(\lim_{n\to\infty} \abs{u_n} = \abs{a}\)成立.

根据\cref{example:极限.负1的次幂组成的数列发散},
数列\(x_n = (-1)^{n+1}\)发散,
但\(\abs{x_n} = \abs{(-1)^{n+1}} = 1\)收敛,
说明如果数列\(\{\abs{x_n}\}\)有极限,
但数列\(\{x_n\}\)未必有极限.
\end{proof}
\end{example}
\begin{remark}
可以证明:当\(a=0\)时,成立\begin{equation*}
	\lim_{n\to\infty} \abs{u_n} = \abs{a}
	\iff
	\lim_{n\to\infty} u_n = a.
\end{equation*}
这是因为,
只要有\(a=0\)且\(\lim_{n\to\infty} \abs{u_n} = 0\),
就有\begin{equation*}
	(\forall\epsilon>0)
	(\exists N\in\mathbb{N})
	(\forall n\in\mathbb{N})
	[n>N \implies \abs{u_n-0}=\abs{\abs{u_n}-0}<\epsilon],
\end{equation*}
所以\(\lim_{n\to\infty} a_n = 0\).
\end{remark}

\subsection{波尔查诺--魏尔斯特拉斯定理}
现在我们来回答本节提出的第二个问题:
如果把单调性这一条件去掉,只考虑数列是有界的,
则只能得到下面这个稍弱的结论.
\begin{theorem}[波尔查诺--魏尔斯特拉斯定理]\label{theorem:极限.波尔查诺--魏尔斯特拉斯定理}
%@see: 《数学分析(第二版 上册)》(陈纪修) P63 定理2.4.5
任何有界数列至少有一个有限的子列极限.
\begin{proof}
设数列\(\{x_n\}\)有界,
那么存在实数\(a_1,b_1\)使得\((\forall n\in\mathbb{N}^+)[a_1 \leq x_n \leq b_1]\).
将闭区间\([a_1,b_1]\)等分为两个小区间\([a_1,c_1]\)和\([c_1,b_1]\),
这里\(c_1=\frac{a_1+b_1}2\),
则其中至少有一个含有数列\(\{x_n\}\)中的无穷多项,把它记为\([a_2,b_2]\).
再把闭区间\([a_2,b_2]\)等分为两个小区间\([a_2,c_2]\)和\([c_2,b_2]\),
这里\(c_2=\frac{a_2+b_2}2\),
同样其中至少有一个含有输了\(\{x_n\}\)中的无穷多项,把它记为\([a_3,b_3]\).
这样的步骤可以一直做下去,于是得到一个闭区间套\(\{[a_k,b_k]\}\),
其中每一个闭区间\([a_k,b_k]\)中都含有数列\(\{x_n\}\)中的无穷多项.

根据\hyperref[definition:极限.闭区间套定理]{闭区间套定理},
存在实数\(\xi\)满足\(\lim_{k\to\infty} a_k = \lim_{k\to\infty} b_k = \xi\).
现在证明数列\(\{x_n\}\)必有一个子列收敛于实数\(\xi\).

首先在\([a_1,b_1]\)中选取\(\{x_n\}\)中的某一项,记为\(x_{n_1}\).
然后在\([a_2,b_2]\)中选取位于\(x_{n_1}\)后的某一项,记为\(x_{n_2}\).
以此类推,我们就得到了数列\(\{x_n\}\)的一个子列\(\{x_{n_k}\}\),
它满足\((\forall k\in\mathbb{N}^+)[a_k \leq x_{n_k} \leq b_k]\).
利用\hyperref[theorem:数列极限.夹逼准则]{夹逼准则}得到\(\lim_{k\to\infty} x_{n_k} = \xi\).
\end{proof}
\end{theorem}

当数列无界时,也有与\cref{theorem:极限.波尔查诺--魏尔斯特拉斯定理} 相对应的结论.
\begin{theorem}
%@see: 《数学分析(第二版 上册)》(陈纪修) P63 定理2.4.6
若\(\{x_n\}\)是一个无界数列,
则存在子列\(\{x_{n_k}\}\),
使得\(\lim_{k\to\infty} x_{n_k} = \infty\).
\begin{proof}
由于\(\{x_n\}\)无界,
因此对于\(\forall M>0\),
\(\{x_n\}\)中必定存在无穷多个\(x_n\),
满足\(\abs{x_n}>M\),
否则可以得出\(\{x_n\}\)有界的结论.

令\(M_1=1\),则存在\(x_{n_1}\),使得\(\abs{x_{n_1}}>1\).
再令\(M_2=1\),因为在\(\{x_n\}\)中有无穷多项满足\(\abs{x_n}>2\),
可以取到排在\(x_{n_1}\)之后的\(x_{n_2}\),使得\(\abs{x_{n_2}}>2\).
以此类推,便可得到\(\{x_n\}\)的一个子列\(\{x_{n_k}\}\),
满足\((\forall k\in\mathbb{N}^+)[\abs{x_{n_k}}>k]\).
由定义可知\(\lim_{k\to\infty} x_{n_k} = \infty\).
\end{proof}
\end{theorem}

\subsection{柯西极限存在准则}
前面已经指出,从数列\(\{x_n\}\)本身的特征直接判断它是否收敛是一个很有意义的重要问题.
但是\hyperref[theorem:极限.数列的单调有界定理]{单调有界定理}只是给出了判断数列收敛的一个充分不必要条件.
但是许多收敛数列并非单调的,所以我们有必要从数列本身出发来寻找其收敛的充分必要条件.

\begin{definition}
%@see: 《数学分析(第二版 上册)》(陈纪修) P64 定义2.4.3
如果数列\(\{x_n\}\)满足\begin{equation*}
	(\forall\epsilon>0)
	(\exists N\in\mathbb{N})
	(\forall n,m\in\mathbb{N})
	[n > N \land m > N \implies \abs{x_n - x_m} < \epsilon],
\end{equation*}
则称“数列\(\{x_n\}\)是一个\DefineConcept{基本数列}”.
\end{definition}

\begin{theorem}[柯西极限存在准则]\label{theorem:极限.数列的柯西极限存在准则}
%@see: 《高等数学(第六版 上册)》 P55 柯西极限存在准则
%@see: 《数学分析(第二版 上册)》(陈纪修) P65 定理2.4.7
数列\(\{x_n\}\)收敛的充分必要条件是:
\(\{x_n\}\)是基本数列.
\begin{proof}
必要性.
设\(\lim_{n\to\infty} x_n = a\).
由数列极限的定义有\begin{equation*}
	(\forall\epsilon>0)
	(\exists N\in\mathbb{N})
	(\forall n,m\in\mathbb{N})
	\left[\def\arraystretch{1.2}
		\begin{array}{c}
			n > N \implies \abs{x_n - a} < \frac\epsilon2 \\
			m > N \implies \abs{x_m - a} < \frac\epsilon2
		\end{array}
	\right],
\end{equation*}
于是\begin{equation*}
	(\forall\epsilon>0)
	(\exists N\in\mathbb{N})
	(\forall n,m\in\mathbb{N})
	[
		n > N \land m > N
		\implies
		\abs{x_m - x_n}
		% = \abs{(x_m - a) - (x_n - a)}
		\leq \abs{x_m - a} + \abs{x_n - a}
		< \epsilon
	].
\end{equation*}

充分性.
取\(\epsilon_0 = 1\),
因为\(\{x_n\}\)是基本数列,
所以\begin{equation*}
	(\exists N_0\in\mathbb{N})
	(\forall n\in\mathbb{N})
	[
		n > N_0
		\implies
		\abs{x_n - x_{N_0+1}} < 1.
	]
\end{equation*}
令\(M = \max\{\abs{x_1},\abs{x_2},\dotsc,\abs{x_{N_0}},\abs{x_{N_0+1}}+1\}\),
则\begin{equation*}
	(\forall n\in\mathbb{N})
	[\abs{x_n} \leq M],
\end{equation*}
这就说明基本数列必定有界.
那么由\hyperref[theorem:极限.波尔查诺--魏尔斯特拉斯定理]{波尔查诺--魏尔斯特拉斯定理},
在数列\(\{x_n\}\)中必有收敛子列\(\{x_{n_k}\}\),使得\begin{equation*}
	\lim_{k\to\infty} x_{n_k} = \xi.
\end{equation*}
因为\(\{x_n\}\)是基本数列,
所以\begin{equation*}
	(\forall\epsilon>0)
	(\exists N\in\mathbb{N})
	(\forall n,m\in\mathbb{N})
	\left[
		n > N \land m > N
		\implies
		\abs{x_n - x_m} < \frac\epsilon2
	\right].
\end{equation*}
在上式中取\(x_m = x_{n_k}\),其中\(k\)充分大,满足\(n_k > N\),
并且令\(k\to\infty\),
于是由\cref{theorem:极限.收敛数列的保序性2} 得到\begin{equation*}
	\abs{x_n - \xi}
	\leq \frac\epsilon2
	< \epsilon,
\end{equation*}
这就说明\(\{x_n\}\)收敛.
\end{proof}
\end{theorem}

\hyperref[theorem:极限.数列的柯西极限存在准则]{柯西极限存在准则}表明,
由实数构成的基本数列\(\{x_n\}\)必存在实数极限,
这一性质称为“实数系的\DefineConcept{完备性}”.
值得注意的是有理数集\(\mathbb{Q}\)不具有完备性.
例如,数列\(\left\{\left(1+\frac1n\right)^n\right\}\)是由有理数构成的基本数列,
但是它的极限\(e\)并不是有理数(我们将在以后给出严格证明).%TODO: 证明e不是有理数

\begin{example}\label{example:收敛准则.压缩映射原理1}
设数列\(\{x_n\}\)从某一项开始满足压缩性条件\begin{equation*}
	\abs{x_{n+1} - a}
	\leq k \abs{x_n - a},
	\quad 0<k<1,
\end{equation*}
证明:\(\lim_{n\to\infty} x_n = a\).
%TODO proof
\end{example}

\begin{example}\label{example:收敛准则.压缩映射原理2}
%@see: 《数学分析(第二版 上册)》(陈纪修) P66 例2.4.14
设数列\(\{x_n\}\)满足压缩性条件\begin{equation*}
	\abs{x_{n+1} - x_n}
	\leq k \abs{x_n - x_{n-1}},
	\quad 0<k<1, n=2,3,\dotsc,
\end{equation*}
证明:\(\{x_n\}\)收敛.
\begin{proof}
对于\(\forall n\in\mathbb{N}^+\),有\begin{equation*}
	\abs{x_{n+1} - x_n}
	\leq k \abs{x_n - x_{n-1}}
	\leq k^2 \abs{x_{n-1} - x_{n-2}}
	\leq \dotsb
	\leq k^{n-1} \abs{x_2 - x_1}.
\end{equation*}
设\(m > n\),
则\begin{align*}
	\abs{x_m - x_n}
	&\leq \abs{x_m - x_{m-1}}
	+ \abs{x_{m-1} - x_{m-2}}
	+ \dotsb + \abs{x_{n+1} - x_n} \\
	&\leq k^{m-2} \abs{x_2 - x_1}
	+ k^{m-3} \abs{x_2 - x_1}
	+ \dotsb + k^{n-1} \abs{x_2 - x_1} \\
	&< \frac{k^{n-1}}{1-k} \abs{x_2 - x_1},
\end{align*}
因为\(0<k<1\),
所以\(\lim_{n\to\infty} \frac{k^{n-1}}{1-k} \abs{x_2 - x_1} = 0\),
于是\(\{x_n\}\)是基本数列,必定收敛.
\end{proof}
\end{example}


\chapter{函数极限}
\input{微积分/函数极限/函数极限的概念}
\section{函数极限与数列极限的关系,海涅定理}
\subsection{海涅定理}
\begin{theorem}[海涅定理]\label{theorem:极限.海涅定理}
%@see: 《数学分析(第二版 上册)》(陈纪修) P78 定理3.1.5
%@see: 《数学分析(第二版 上册)》(陈纪修) P80 定理3.1.5'
%@see: 《数学分析教程(第3版 上册)》(史济怀) P70 定理2.4.1
%@see: 《数学分析(第7版 第一卷)》(卓里奇) P92 命题1
设函数\(f\colon D\to\mathbb{R}\),
则\(\lim_{x \to x_0} f(x) = A\)的充分必要条件是:
任何一个值域为\(D-\{x_0\}\)且收敛于\(x_0\)的数列\(\{x_n\}_{n\geq1}\)
总满足\(\lim_{n\to\infty} f(x_n) = A\).
\begin{proof}
必要性.
由\(\lim_{x \to x_0} f(x) = A\)可知\begin{equation*}
	(\forall\epsilon>0)
	(\exists\delta>0)
	(\forall x\in D)
	[
		0<\abs{x-x_0}<\delta
		\implies
		\abs{f(x)-A}<\epsilon
	].
\end{equation*}
因为\(\lim_{n\to\infty} x_n = x_0\),
且\(x_n \neq x_0\ (n=1,2,\dotsc)\),
所以\begin{equation*}
	(\forall\delta>0)
	(\exists N\in\mathbb{N})
	(\forall n\in\mathbb{N})
	[
		n>N
		\implies
		0<\abs{x_n-x_0}<\delta
		\implies
		\abs{f(x_n)-A}<\epsilon
	],
\end{equation*}
即\(\lim_{n\to\infty} f(x_n) = A\).

充分性.
用反证法.
我们知道,命题“函数\(f\)在点\(x_0\)的极限是\(A\)”可以表述为\begin{equation*}
	(\forall\epsilon>0)
	(\exists\delta>0)
	(\forall x\in D)
	[
		0<\abs{x-x_0}<\delta
		\implies
		\abs{f(x)-A}<\epsilon
	],
\end{equation*}
那么它的否命题“函数\(f\)在点\(x_0\)的极限不是\(A\)”可以表述为\begin{equation*}
	(\exists\epsilon_0>0)
	(\forall\delta>0)
	(\exists x\in D)
	[
		0<\abs{x-x_0}<\delta
		\implies
		\abs{f(x)-A}\geq\epsilon_0
	].
\end{equation*}
于是\begin{equation*}
	(\forall n\in\mathbb{N}^+)
	(\exists x_n\in D)
	\left[
		0<\abs{x_n-x_0}<\frac1n
		\implies
		\abs{f(x_n)-A}\geq\epsilon_0
	\right],
\end{equation*}
如此,我们就找到了一个数列\(\{x_n\}_{n\geq1}\)
满足\(x_n\neq x_0\ (n=1,2,\dotsc)\)且\(\lim_{n\to\infty} x_n = x_0\),
但是\(\{f(x_n)\}\)不可能满足\(\lim_{n\to\infty} f(x_n) = A\).
由此推翻假设,得到\(\lim_{x\to x_0} f(x) = A\).
\end{proof}
\end{theorem}
海涅定理也称作\DefineConcept{归结原则}.

这里要注意定理的条件\(x_n \neq x_0\),
因为只要没有这个条件就不能保证海涅定理的结果成立,下面我们举例说明.
设\begin{equation*}
	f(x) = \left\{ \begin{array}{cl}
		2, & x\neq0, \\
		1, & x=0.
	\end{array} \right.
\end{equation*}
又设点\(x_0\)与数列\(\{x_n\}\)满足\(x_0=x_n=0\ (n=1,2,\dotsc)\).
那么有\begin{equation*}
	\lim_{n\to\infty} x_n = 0,
	\quad\text{和}\quad
	\lim_{x\to0} f(x) = 2,
\end{equation*}
但是\begin{equation*}
	\lim_{n\to\infty} f(x_n) = \lim_{n\to\infty} f(0) = 1.
\end{equation*}这就说明不总有\begin{equation*}
	\lim_{n\to\infty} f(x_n)
	= \lim_{x \to x_0} f(x)
\end{equation*}成立.

\begin{example}
%@see: 《数学分析教程(第3版 上册)》(史济怀) P71 例4
%@see: 《数学分析(第二版 上册)》(陈纪修) P80 例3.1.6
证明:\(\lim_{x\to0} \sin\frac1x\)不存在.
\begin{proof}
令\begin{equation*}
	x_n = \frac1{(2n+1/2)\pi},
	\qquad
	y_n = \frac1{2n\pi}.
\end{equation*}
显然\begin{equation*}
	x_n>0,
	\qquad
	y_n>0,
	\qquad
	\lim_{n\to\infty} x_n = 0,
	\qquad
	\lim_{n\to\infty} y_n = 0.
\end{equation*}
但是\begin{equation*}
	\sin\frac1{x_n}
	= \sin\left(2n+\frac12\right)\pi
	= 1,
	\qquad
	\sin\frac1{y_n}
	= \sin 2n\pi
	= 0,
\end{equation*}
于是\begin{equation*}
	\lim_{n\to\infty} \sin\frac1{x_n} = 1,
	\qquad
	\lim_{n\to\infty} \sin\frac1{y_n} = 0.
\end{equation*}
由\hyperref[theorem:极限.海涅定理]{海涅定理}可知,该极限不存在.
\end{proof}
\end{example}
\begin{example}\label{example:海涅定理.狄利克雷函数在任意一点的极限都不存在}
%@see: 《数学分析教程(第3版 上册)》(史济怀) P71 例5
证明:狄利克雷函数\begin{equation*}
	D(x) = \left\{ \begin{array}{ll}
		1, & x \in \mathbb{Q}, \\
		0, & x \in \mathbb{R}-\mathbb{Q}
	\end{array} \right.
\end{equation*}在任意一点的极限都不存在.
\begin{proof}
对于任意一点\(x_0\in\mathbb{R}\),一定存在全由有理数组成的数列\(\{s_n\}\)
和全由无理数组成的数列\(\{t_n\}\),
使它们都趋向于\(x_0\),
这样就有\begin{equation*}
	\lim_{n\to\infty} D(s_n) = 1,
	\qquad
	\lim_{n\to\infty} D(t_n) = 0.
\end{equation*}
由\hyperref[theorem:极限.海涅定理]{海涅定理}可知
\(\lim_{x \to x_0} D(x)\)不存在.
\end{proof}
\end{example}

\begin{example}
函数\(f\)在\((-\infty,+\infty)\)内单调有界,
数列\(\{x_n\}\)收敛,
但函数列\(\{f(x_n)\}\)不一定收敛.
例如,取函数\begin{equation*}
	f(x) = \left\{ \begin{array}{rl}
		1 & x\geq0, \\
		-1, & x<0,
	\end{array} \right.
\end{equation*}
和数列\(x_n = \frac{(-1)^n}{n}\),
容易看出\(\varlimsup_{n\to\infty} f(x_n) = 1\)
和\(\varliminf_{n\to\infty} f(x_n) = -1\),
于是\(\lim_{n\to\infty} f(x_n)\)发散.
\end{example}

\begin{example}
函数\(f\)在\((-\infty,+\infty)\)内单调有界,
函数列\(\{f(x_n)\}\)单调或收敛,
但数列\(\{x_n\}\)不一定收敛.
例如,取\begin{equation*}
	f(x) = \arctan x,
\end{equation*}
和数列\(x_n = n\),
容易看出\(\{f(x_n)\}\)单调且收敛,
但是\(\{x_n\}\)不收敛.
\end{example}

\begin{example}
证明:函数\(\sin\frac1x\)在点\(x=0\)没有极限.
\begin{proof}
取\(x_n = \frac1{n\pi}\ (n=1,2,\dotsc)\).
显然有\(x_n\neq0\)且\(\lim_{n\to\infty} x_n = 0\).

再取\(y_n = \frac1{2n\pi+\pi/2}\ (n=1,2,\dotsc)\).
显然有\(y_n\neq0\)且\(\lim_{n\to\infty} y_n = 0\).

但是由于\begin{equation*}
	\lim_{n\to\infty} \sin\frac1{x_n} = 0
	\neq 1 = \lim_{n\to\infty} \sin\frac1{y_n},
\end{equation*}
所以根据\cref{theorem:极限.海涅定理} 可知,
函数\(\sin\frac1x\)在点\(x=0\)没有极限.
\end{proof}
\end{example}

\begin{theorem}
%@see: 《数学分析(第二版 上册)》(陈纪修) P88 习题12
设函数\(f\colon (a,+\infty)\to\mathbb{R}\),
则\(\lim_{x\to+\infty} f(x) = -\infty\)的充分必要条件是:
对于任意正无穷大\(\{x_n\}\),成立\(\lim_{n\to\infty} f(x_n) = -\infty\).
%TODO proof
\end{theorem}

\subsection{上极限与下极限}
\begin{definition}
设函数\(f\colon D\to\mathbb{R}\)在点\(x_0\)的某个去心邻域中有定义.
\begin{itemize}
	\item 把\begin{equation*}
		\lim_{\delta\to0^+} \sup_{0<\abs{x-x_0}<\delta} f(x)
	\end{equation*}称为“函数\(f\)在点\(x_0\)的\DefineConcept{上极限}(upper limit)”,
	记为\(\varlimsup_{x \to x_0} f(x)\).
	%@see: https://mathworld.wolfram.com/UpperLimit.html

	\item 把\begin{equation*}
		\lim_{\delta\to0^+} \inf_{0<\abs{x-x_0}<\delta} f(x)
	\end{equation*}称为“函数\(f\)在点\(x_0\)的\DefineConcept{下极限}(lower limit)”,
	记为\(\varliminf_{x \to x_0} f(x)\).
	%@see: https://mathworld.wolfram.com/LowerLimit.html
\end{itemize}
\end{definition}

\begin{property}
设函数\(f\colon D\to\mathbb{R}\)在点\(x_0\)的某个去心邻域中有定义,
则\begin{equation*}
	\varliminf_{x \to x_0} f(x) \leq \varlimsup_{x \to x_0} f(x).
\end{equation*}
\end{property}

\begin{theorem}
设函数\(f\colon D\to\mathbb{R}\)在点\(x_0\)的某个去心邻域中有定义,
则函数\(f\)在点\(x_0\)时极限存在的充分必要条件是:
\(f\)在点\(x_0\)的上、下极限相等,
即\begin{equation*}
	\varlimsup_{x \to x_0} f(x)
	= \varliminf_{x \to x_0} f(x)
	= A
	\iff
	\lim_{x \to x_0} f(x) = A.
\end{equation*}
\end{theorem}

\section{函数极限的性质}
与收敛数列的性质相比较,可得函数极限的一些相应的性质.
它们都可以根据函数极限的定义,运用类似于证明收敛数列性质的方法加以证明.
由于函数极限的定义按自变量的变化过程不同有各种形式,
下面仅以“\(\lim_{x \to x_0}f(x)\)”这种形式为代表给出关于函数极限性质的一些定理,
并就其中的几个给出证明.
至于其他形式的极限的性质及其证明,只要相应地做一些修改即可得出.

\subsection{唯一性}
\begin{theorem}[唯一性]\label{theorem:极限.函数极限的唯一性}
%@see: 《高等数学(第六版 上册)》 P36 定理1
%@see: 《数学分析(第二版 上册)》(陈纪修) P74 定理3.1.1
%@see: 《数学分析教程(第3版 上册)》(史济怀) P71 定理2.4.2(函数极限的唯一性)
如果\(\lim_{x \to x_0} f(x)\)存在,那么这极限唯一.
\end{theorem}

\subsection{局部保序性}
\begin{theorem}[局部保序性]\label{theorem:极限.函数极限的局部保序性1}
%@see: 《数学分析(第二版 上册)》(陈纪修) P74 定理3.1.2
设\(f,g\in\mathbb{R}^D\).
若\(\lim_{x\to x_0} f(x) = A,
\lim_{x\to x_0} g(x) = B\),
且\(A>B\),
则\begin{equation*}
	(\exists\delta>0)
	(\forall x\in D)
	[0<\abs{x-x_0}<\delta \implies f(x)>g(x)].
\end{equation*}
\begin{proof}
取\(\epsilon_0=\frac{A-B}2>0\).
由\(\lim_{x\to x_0} f(x) = A\)有\begin{equation*}
	(\exists\delta_1>0)
	(\forall x\in D)
	\left[
		0<\abs{x - x_0}<\delta_1
		\implies
		\abs{f(x) - A}<\epsilon_0
		\implies
		\frac{A+B}2 < f(x)
	\right];
\end{equation*}
由\(\lim_{x\to x_0} g(x) = B\)有\begin{equation*}
	(\exists\delta_2>0)
	(\forall x\in D)
	\left[
		0<\abs{x - x_0}<\delta_2
		\implies
		\abs{g(x) - B}<\epsilon_0
		\implies
		g(x) < \frac{A+B}2
	\right].
\end{equation*}
取\(\delta=\min\{\delta_1,\delta_2\}\),
当\(0<\abs{x-x_0}<\delta\)时,
有\(g(x) < \frac{A+B}2 < f(x)\).
\end{proof}
\end{theorem}

\begin{remark}
%@see: 《数学分析(第二版 上册)》(陈纪修) P83
当函数极限是不定号无穷大时,局部保序性不成立.
\end{remark}

\begin{corollary}\label{theorem:极限.函数极限的局部保序性1.推论1}
%@see: 《高等数学(第六版 上册)》 P37 定理3'
%@see: 《数学分析(第二版 上册)》(陈纪修) P75 推论1
设\(f\in\mathbb{R}^D\).
若\(\lim_{x\to x_0} f(x) = A \neq 0\),
那么\begin{equation*}
	(\exists \delta>0)
	(\forall x\in D)
	\left[
		0<\abs{x-x_0}<\delta
		\implies
		\abs{f(x)}>\frac12\abs{A}.
	\right].
\end{equation*}
\end{corollary}

\begin{corollary}\label{theorem:极限.函数极限的局部保序性1.推论2}
%@see: 《数学分析(第二版 上册)》(陈纪修) P75 推论2
%@see: 《高等数学(第六版 上册)》 P46 定理5
%@see: 《数学分析教程(第3版 上册)》(史济怀) P73 定理2.4.6
设\(f,g\in\mathbb{R}^D\).
若\(\lim_{x\to x_0} f(x) = A,
\lim_{x\to x_0} g(x) = B\),
且\begin{equation*}
	(\exists\rho>0)[0<\abs{x-x_0}<\rho \implies f(x) \geq g(x)],
\end{equation*}
则\(A \geq B\).
\end{corollary}
\begin{remark}
即使把\cref{theorem:极限.函数极限的局部保序性1.推论2} 的条件
加强为\((\exists\rho>0)[0<\abs{x-x_0}<\rho \implies f(x) > g(x)]\),
也只能得到\(A \geq B\)的结论,而不能得到\(A > B\)的结论.
例如,对于\(\forall\rho>0\),
函数\(f(x) = x^4\)和\(g(x) = x^2\)当\(0<\abs{x}<\rho\)时总满足\(f(x) > g(x)\),
但是\(\lim_{x\to0} f(x) = \lim_{x\to0} g(x) = 0\).
\end{remark}

%TODO: 来源请求
% \begin{theorem}\label{theorem:极限.函数极限的局部保序性2}
% 若函数\(f(x)\)和\(g(x)\)在区间\(I\)上满足\(f(x) \leq g(x)\),那么有\begin{equation*}
% 	\varlimsup_{x \to a} f(x) \leq \varlimsup_{x \to a} g(x),
% \end{equation*}\begin{equation*}
% 	\varliminf_{x \to a} f(x) \leq \varliminf_{x \to a} g(x).
% \end{equation*}
% \end{theorem}

\begin{corollary}[局部保号性]\label{theorem:极限.函数极限的局部保号性1}
%@see: 《高等数学(第六版 上册)》 P37 定理3
设\(\lim_{x \to x_0} f(x) = A\).
\begin{itemize}
	\item 若\(A>0\),
	则\((\exists\delta>0)
	(\forall x\in\mathbb{R})
	[0<\abs{x-x_0}<\delta \implies f(x)>0]\).
	\item 若\(A<0\),
	则\((\exists\delta>0)
	(\forall x\in\mathbb{R})
	[0<\abs{x-x_0}<\delta \implies f(x)<0]\).
\end{itemize}
\end{corollary}

\begin{corollary}\label{theorem:极限.函数极限的局部保号性3}
%@see: 《高等数学(第六版 上册)》 P37 推论
设\(\lim_{x \to x_0} f(x) = A\).
\begin{itemize}
	\item 若\((\forall x\in\mathbb{R})
	[x\in\mathring{U}(x_0) \implies f(x) \geq 0]\),
	则\(A \geq 0\).
	\item 若\((\forall x\in\mathbb{R})
	[x\in\mathring{U}(x_0) \implies f(x) \leq 0]\),
	则\(A \leq 0\).
\end{itemize}
\end{corollary}

\subsection{局部有界性}
\begin{corollary}[局部有界性]\label{theorem:极限.函数极限的局部有界性}
%@see: 《数学分析(第二版 上册)》(陈纪修) P75 推论3
%@see: 《数学分析教程(第3版 上册)》(史济怀) P72 定理2.4.3
如果\(\lim_{x \to x_0} f(x) = A\),
那么\((\exists\delta>0)[\text{函数$f$在$\mathring{U}(x_0,\delta)$中有界}]\).
\end{corollary}

\section{夹逼准则}
\begin{theorem}\label{theorem:函数极限.夹逼准则}
%@see: 《数学分析(第二版 上册)》(陈纪修) P76 定理3.1.3
%@see: 《高等数学(第六版 上册)》 P50 准则I'
%@see: 《数学分析教程(第3版 上册)》(史济怀) P72 定理2.4.5(夹逼原理)
设函数\(f,g,h\in\mathbb{R}^D\)满足\begin{itemize}
	\item \((\exists\rho>0)
	[0<\abs{x-x_0}<\rho \implies g(x) \leq f(x) \leq h(x)]\),
	\item \(\lim_{x \to x_0} g(x) = \lim_{x \to x_0} h(x) = A\),
\end{itemize}
则\(\lim_{x \to x_0} f(x) = A\).
\begin{proof}
对于\(\forall\epsilon>0\),
有\begin{gather*}
	\lim_{x \to x_0} h(x) = A
	\implies
	(\exists\delta_1>0)
	(\forall x\in\mathbb{R})
	\left[
		\begin{array}{rl}
			0<\abs{x-x_0}<\delta_1
			&\implies
			\abs{h(x)-A}<\epsilon \\
			&\implies
			h(x)<A+\epsilon
		\end{array}
	\right], \\
	\lim_{x \to x_0} g(x) = A
	\implies
	(\exists\delta_2>0)
	(\forall x\in\mathbb{R})
	\left[
		\begin{array}{rl}
			0<\abs{x-x_0}<\delta_2
			&\implies
			\abs{h(x)-A}<\epsilon \\
			&\implies
			A-\epsilon<g(x)
		\end{array}
	\right].
\end{gather*}
取\(\delta=\min\{\delta_1,\delta_2,\rho\}\),
则\begin{equation*}
	(\forall x\in\mathbb{R})
	[
		0<\abs{x-x_0}<\delta
		\implies
		A-\epsilon < g(x) \leq f(x) \leq h(x) < A+\epsilon
	],
\end{equation*}
即\(\lim_{x \to x_0} f(x) = A\).
\end{proof}
\end{theorem}

\begin{remark}
%@see: 《数学分析(第二版 上册)》(陈纪修) P83
当函数极限是不定号无穷大时,夹逼定理不成立,即\begin{equation*}
	g(x) \leq f(x) \leq h(x)
	\land
	\lim g(x) = \lim h(x) = \infty
	\notimplies
	\lim f(x) = \infty.
\end{equation*}
\end{remark}

\begin{example}[重要极限I]
%@see: 《高等数学(第六版 上册)》 P51
%@see: 《数学分析(第二版 上册)》(陈纪修) P76 例3.1.4
%@see: 《数学分析教程(第3版 上册)》(史济怀) P76 例6
试证:\begin{equation}\label{equation:函数极限.重要极限1}
	\lim_{x\to0} \frac{\sin x}{x} = 1.
\end{equation}
\begin{proof}
如图所示,
\begin{center}
	\begin{tikzpicture}
		\pgfmathsetmacro{\r}{4}
		\pgfmathsetmacro{\cx}{\r/sqrt(2)}
		\coordinate(O)at(0,0);
		\coordinate(A)at(\r,0);
		\coordinate(B)at(\cx,\cx);
		\coordinate(C)at(\cx,0);
		\coordinate(D)at(\r,\r);
		\draw (O)node[left]{\(O\)}
			--(C)node[below]{\(C\)}node[midway,below]{\(\cos x\)}
			--(A)node[below]{\(A\)}arc[start angle=0,end angle=90,radius=\r] (C)
			--(B)node[above]{\(B\)}node[midway,left]{\(\sin x\)}
			--(O)node[midway,above left]{\(1\)}
			--(0,\r)
			(B)
			--(D)node[right]{\(D\)}
			--(A)node[midway,right]{\(\tan x\)};
	\end{tikzpicture}
\end{center}
由于在\(0 < x < \pi/2\)时,\begin{equation*}
	0 < \sin x < x < \tan x
	\implies
	1 < \frac{x}{\sin x} < \frac{1}{\cos x}
	\implies
	\cos x < \frac{\sin x}{x} < 1.
\end{equation*}
因为\(\lim_{x\to0}\cos x = 1\),%FIXME 这个依据是无根之木,需要加以论述
所以由\hyperref[theorem:数列极限.夹逼准则]{夹逼准则}可知,
\(\lim_{x\to0} \frac{\sin x}{x} = 1\).
\end{proof}
\end{example}

\begin{example}
求\(\lim_{x\to\infty} x \sin\frac1x\).
\begin{solution}
直接计算得\begin{equation*}
	\lim_{x\to\infty} x \sin\frac1x
	= \lim_{x\to\infty} \frac{\sin(1/x)}{1/x}
	\xlongequal{t=1/x} \lim_{t\to0} \frac{\sin t}{t}
	= 1.
\end{equation*}
\end{solution}
\end{example}

\begin{example}\label{example:夹逼准则.不能应用等价无穷小1}
计算极限\begin{equation*}
	\lim_{x\to0} \frac1x \sin\left(x^2 \sin\frac1x\right).
\end{equation*}
\begin{solution}
%@see: https://www.bilibili.com/video/BV1FA4m1N7uv/
因为\(0\leq\abs{\sin x}\leq\abs{x}\ (-\infty<x<+\infty)\),
所以\begin{equation*}
	0 \leq \abs{\sin\left(x^2 \sin\frac1x\right)}
	\leq \abs{x^2 \sin\frac1x}
	\leq \abs{x^2},
\end{equation*}
从而\begin{equation*}
	0 \leq \abs{\frac1x \sin\left(x^2 \sin\frac1x\right)} \leq \abs{x}.
\end{equation*}
因为\(\lim_{x\to0} \abs{x} = 0\),
所以由\hyperref[theorem:数列极限.夹逼准则]{夹逼准则}可知
\(\lim_{x\to0} \abs{\frac1x \sin\left(x^2 \sin\frac1x\right)} = 0\).
\end{solution}
\end{example}
\begin{remark}
在\cref{example:夹逼准则.不能应用等价无穷小1} 中,
虽然\(\lim_{x\to0} x^2 \sin\frac1x = 0\),
但是不可以利用“等价无穷小”
把\(\sin\left(x^2 \sin\frac1x\right)\)替换为\(x^2 \sin\frac1x\).
记\(f(x) = x^2 \sin\frac1x\),
在点\(x=0\)的任意邻域内,
总存在无数个点\(\xi\)使得\(f(\xi) = 0\),
于是分式\begin{equation*}
	\frac{\sin f(x)}{f(x)}
\end{equation*}在点\(x=0\)的任意邻域内有无数个点没有定义,
根本不满足\hyperref[definition:极限.函数极限的定义1]{函数极限的定义}对于
极限\(\lim_{x \to x_0} g(x)\)存在所要求的前提条件
“函数\(g\)在点\(x_0\)的某个去心邻域内有定义”,
自然\(\sin f(x)\)与\(f(x)\)不是等价无穷小.
因此我们在运用等价替换规则时,
必须严格按照\cref{theorem:无穷小与无穷大.等价无穷小的换元法} 的要求考察函数的定义域.
\end{remark}

\begin{example}
%@see: 《高等数学(第六版 上册)》 P56 习题1-6 4. (2)
证明:\(\lim_{n\to\infty} n \left(\frac1{n^2+\pi}+\frac1{n^2+2\pi}+\dotsb+\frac1{n^2+n\pi}\right)=1\).
\begin{proof}
易见\begin{equation*}
	\frac1{n^2+\pi}
	\geq \frac1{n^2+2\pi}
	\geq \dotsb
	\geq \frac1{n^2+n\pi},
\end{equation*}
于是\begin{equation*}
	\frac{n}{n+\pi}
	\leq
	n \left(\frac1{n^2+\pi}+\frac1{n^2+2\pi}+\dotsb+\frac1{n^2+n\pi}\right)
	\leq
	\frac{n^2}{n^2+\pi}.
\end{equation*}
因为\begin{equation*}
	\lim_{n\to\infty} \frac{n}{n+\pi}
	= \lim_{n\to\infty} \frac{n^2}{n^2+\pi}
	= 1,
\end{equation*}
所以\(\lim_{n\to\infty} n \left(\frac1{n^2+\pi}+\frac1{n^2+2\pi}+\dotsb+\frac1{n^2+n\pi}\right)=1\).
\end{proof}
\end{example}

\begin{example}
%@see: 《高等数学(第六版 上册)》 P57 习题1-6 4. (4)
证明:\(\lim_{x\to0} \sqrt[n]{1+x} = 1\).
\begin{proof}
函数\(\sqrt[n]{1+x}\)的定义域是\(\Set{ x \given 1+x\geq 0 } = [-1,+\infty)\).

当\(x > 0\)时,
因为\(1 < \sqrt[n]{1+x} < 1+x\),
且\(\lim_{x\to0^+} 1 = \lim_{x\to0^+}(1+x) = 1\),
所以\begin{equation*}
	\lim_{x\to0^+} \sqrt[n]{1+x} = 1.
\end{equation*}

当\(-1 < x < 0\)时,
因为\(1+x < \sqrt[n]{1+x} < 1\),
且\(\lim_{x\to0^-} 1 = \lim_{x\to0^-}(1+x) = 1\),
所以\begin{equation*}
	\lim_{x\to0^-} \sqrt[n]{1+x} = 1.
\end{equation*}

综上,\(\lim_{x\to0} \sqrt[n]{1+x} = 1\).
\end{proof}
\end{example}

\begin{example}
%@see: 《高等数学(第六版 上册)》 P57 习题1-6 4. (5)
证明:\(\lim_{x\to0^+} x \floor*{\frac1x} = 1\).
\begin{proof}
令\(t=1/x\),
那么\begin{equation*}
	x \floor*{\frac1x} = \frac{1}{t} \floor{t}.
\end{equation*}
又因为\begin{equation*}
	t - 1 < \floor{t} \leq t,
\end{equation*}\begin{equation*}
	1 - \frac{1}{t} < \frac{1}{t} \floor{t} \leq 1;
\end{equation*}
而\begin{equation*}
	\lim_{t\to+\infty} 1 - \frac{1}{t} = 1,
	\quad
	\lim_{t\to+\infty} 1 = 1,
\end{equation*}
所以\begin{equation*}
	\lim_{x\to0^+} x \floor*{\frac1x} = \lim_{t\to+\infty} \frac{1}{t} \floor{t} = 1.
	\qedhere
\end{equation*}
\end{proof}
\end{example}

%TODO: 函数\(\frac{\sin x}{x}\)的图像
% \begin{figure}[htb]
% 	\centering
% 	\begin{tikzpicture}
% 		\begin{axis}[
% 			xmin=-4*pi,xmax=4*pi,
% 			ymin=-.5,ymax=1.2,
% 			width=\textwidth,
% 			height=.3\textheight,
% 			grid=both,
% 			xlabel=$x$,
% 			ylabel=$y$,
% 			enlargelimits,
% 			axis lines=middle,
% 			xtick={3.14,6.28,9.42,-3.14,-6.28,-9.42},
% 			xticklabels={$\pi$,$2\pi$,$3\pi$,$-\pi$,$-2\pi$,$-3\pi$},
% 		]
% 			\begin{scope}[samples=50,thick,red]
% 				\addplot[domain=.01:4*pi]{sin(deg(x))/x};
% 				\addplot[domain=-4*pi:-.01]{sin(deg(x))/x};
% 			\end{scope}
% 			\filldraw[draw=black,fill=white](0,1)circle(1pt);
% 		\end{axis}
% 	\end{tikzpicture}
% 	\caption{函数\(y=\frac{\sin x}{x}\)的图像}
% 	\label{figure:极限.函数[y=sin(x)/x]的图像}
% \end{figure}

\begin{example}[重要极限II]
%@see: 《高等数学(第六版 上册)》 P53
%@see: 《数学分析(第二版 上册)》(陈纪修) P84 例3.1.13
%@see: 《数学分析教程(第3版 上册)》(史济怀) P82 例2
试证:\begin{equation}\label{equation:函数极限.重要极限2}
	\lim_{x\to\infty} \left(1+\frac1x\right)^x = e.
\end{equation}
\begin{figure}[htb]
%@Mathematica: Plot[{E,(1+1/x)^x},{x,-10,10},PlotRange->{0,10},AspectRatio->Automatic,PlotStyle->{{Dashed,Thin},Thin}]
	\centering
	\begin{tikzpicture}[scale=.5]
		\begin{axis}[
			xmin=-10,xmax=10,ymin=0,ymax=10,
			grid=both,width=\textwidth,height=\textwidth,
			xlabel=$x$,
			ylabel=$y$,
			enlarge x limits=0.1,
			enlarge y limits=0.1,
			axis lines=middle,
			xtick={-9,-7,...,9},
			ytick={1,2.718,10},
			yticklabels={$1$,$e$,$10$},
		]
			\begin{scope}[samples=50,thick,red]
				\addplot[domain=-10:-0]{(1+1/x)^x};
				\addplot[domain=+0:+10]{(1+1/x)^x};
			\end{scope}
		\end{axis}
	\end{tikzpicture}
	\caption{函数\(y=\left(1+\tfrac1x\right)^x\)的图像}
\end{figure}
\begin{proof}
在\cref{section:极限.无理数e}我们已经知道,
数列\(x_n = \left(1+\frac1n\right)^n\)单调增加且有上界,收敛于\(e\).

当\(n\geq1\)时,
对于\(\forall x\in[n,n+1)\),
必有\begin{equation*}
	\left(1+\frac1{n+1}\right)^n
	< \left(1+\frac1x\right)^x
	< \left(1+\frac1n\right)^{n+1},
\end{equation*}
且当\(n\to\infty\)时,
\(x\to+\infty\),
而\begin{equation*}
	\lim_{n\to\infty} \left(1+\frac1{n+1}\right)^n
	= \lim_{n\to\infty} \frac{\left(1+\dfrac1{n+1}\right)^{n+1}}{1+\dfrac1{n+1}}
	= \frac{\lim\limits_{n\to\infty} \left(1+\dfrac1{n+1}\right)^{n+1}}{\lim\limits_{n\to\infty} \left(1+\dfrac1{n+1}\right)}
	= \frac{e}1
	= e,
\end{equation*}\begin{equation*}
	\lim_{n\to\infty} \left(1+\frac1n\right)^{n+1}
	= \lim_{n\to\infty} \left[\left(1+\frac1n\right)^n\cdot\left(1+\frac1n\right)\right]
	= \lim_{n\to\infty} \left(1+\frac1n\right)^n \cdot \lim_{n\to\infty} \left(1+\frac1n\right)
	= e \cdot 1
	= e,
\end{equation*}
应用\hyperref[theorem:函数极限.夹逼准则]{夹逼准则}可得\begin{equation*}
	\lim_{x\to+\infty} \left(1+\frac1x\right)^x = e.
\end{equation*}
令\(x=-(t+1)\),
则\(x\to-\infty\)时,
\(t\to+\infty\),
从而\begin{align*}
	\lim_{x\to-\infty} \left(1+\frac1x\right)^x
	&= \lim_{t\to+\infty} \left(1-\frac{1}{t+1}\right)^{-(t+1)}
	= \lim_{t\to+\infty} \left(\frac{t}{t+1}\right)^{-(t+1)} \\
	&= \lim_{t\to+\infty} \left(1+\frac1t\right)^{t+1}
	= \lim_{t\to+\infty} \left[\left(1+\frac1t\right)^t\cdot\left(1+\frac1t\right)\right]
	= e.
\end{align*}

综上所述,根据\cref{theorem:函数极限.极限与单侧极限的关系2},
由于\begin{equation*}
	\lim_{x\to+\infty}\left(1+\frac1x\right)^x
	= \lim_{x\to-\infty}\left(1+\frac1x\right)^x
	= e,
\end{equation*}
所以\begin{equation*}
	\lim_{x\to\infty} \left(1+\frac1x\right)^x = e.
	\qedhere
\end{equation*}
\end{proof}
\end{example}

\section{函数极限的运算}
在下面的讨论中,记号“\(\lim\)”下面没有标明自变量的变化过程,
实际上,下面的定理对\(x \to x_0\)及\(x \to \infty\)都是成立的.
在论证时,我们只证明了\(x \to x_0\)的情形,
只要把\(\delta\)改成\(X\),把\(0<\abs{x-x_0}<\delta\)改成\(\abs{x}>X\),
就可得\(x\to\infty\)情形的证明.

\subsection{函数极限的四则运算}
\begin{theorem}\label{theorem:极限.极限的四则运算法则}
%@see: 《数学分析(第二版 上册)》(陈纪修) P77 定理3.1.4
%@see: 《高等数学(第六版 上册)》 P44 定理3
%@see: 《数学分析教程(第3版 上册)》(史济怀) P72 定理2.4.4
设\(f,g\in\mathbb{R}^X\),\(\mathcal{B}\)是\(X\)中的基.
如果\(\lim_\mathcal{B} f(x) = u,
\lim_\mathcal{B} g(x) = v\),
那么\begin{itemize}
	\item \(\lim_\mathcal{B} [f(x) \pm g(x)]
	= \lim_\mathcal{B} f(x) \pm \lim_\mathcal{B} g(x)
	= u \pm v\);

	\item \(\lim_\mathcal{B} [f(x) \cdot g(x)]
	= \lim_\mathcal{B} f(x) \cdot \lim_\mathcal{B} g(x)
	= u \cdot v\);

	\item 若\(v\neq0\),
	则\(\lim_\mathcal{B} \frac{f(x)}{g(x)} = \frac{u}{v}\).
\end{itemize}
%TODO proof
\end{theorem}
\begin{remark}
如果已知\(\lim_\mathcal{B} \frac{f(x)}{g(x)} = A < \infty\)
且\(\lim_\mathcal{B} g(x) = 0\),
那么一定成立\(\lim_\mathcal{B} f(x) = 0\).
这是因为\begin{equation*}
	\lim_\mathcal{B} f(x)
	= \lim_\mathcal{B} \left[
			g(x) \cdot \frac{f(x)}{g(x)}
		\right]
	= \lim_\mathcal{B} g(x) \cdot \lim_\mathcal{B} \frac{f(x)}{g(x)}
	= 0 \cdot A = 0.
\end{equation*}
在下一节我们会认识到,满足上述条件的\(f\)和\(g\)是同阶无穷小.
\end{remark}
\begin{example}
举例说明:即便\(\lim_\mathcal{B} f(x) \cdot g(x)\)和\(\lim_\mathcal{B} f(x)\)均存在,但是\(\lim_\mathcal{B} g(x)\)不存在.
\begin{solution}
取\(f(x) = 0,
g(x) = \sin x\),
则\(\lim_{x\to\infty} f(x) \cdot g(x)
= \lim_{x\to\infty} f(x)
= 0\),
但是\(\lim_{x\to\infty} g(x)\)不存在.
\end{solution}
\end{example}
\begin{example}
举例说明:纵使\(\lim_\mathcal{B} f(x)\)和\(\lim_\mathcal{B} g(x)\)均不存在,但是\(\lim_\mathcal{B} f(x) \cdot g(x)\)存在.
\begin{solution}
取\(f(x) = \sin x,
g(x) = \frac1{\sin x}\),
则\(\lim_\mathcal{B} f(x)\)和\(\lim_\mathcal{B} g(x)\)均不存在,
但是\(\lim_\mathcal{B} f(x) \cdot g(x) = 1\).
\end{solution}
\end{example}

\begin{corollary}
%@see: 《高等数学(第六版 上册)》 P45 推论1
设\(f\in\mathbb{R}^X\),\(\mathcal{B}\)是\(X\)中的基.
如果\(\lim_\mathcal{B} f(x)\)存在,\(c\)是常数,
则\begin{equation*}
	\lim_\mathcal{B} [c f(x)] = c \lim_\mathcal{B} f(x).
\end{equation*}
\end{corollary}

\begin{corollary}
%@see: 《高等数学(第六版 上册)》 P45 推论2
设\(f\in\mathbb{R}^X\),\(\mathcal{B}\)是\(X\)中的基.
如果\(\lim_\mathcal{B} f(x)\)存在,而\(n\)是正整数,
则\begin{equation*}\lim_\mathcal{B} [f(x)]^n = [\lim_\mathcal{B} f(x)]^n.\end{equation*}
\end{corollary}

\cref{theorem:极限.极限的四则运算法则} 的第1条、第2条可以推广到有限个函数的情形.
\begin{corollary}
%@see: 《高等数学(第六版 上册)》 P45
设\(\lim f_i(x) = A_i\ (i=1,2,\dotsc,n)\),
则\begin{equation*}
	\lim \sum_{i=1}^n c_i f_i(x) = \sum_{i=1}^n c_i A_i
	\quad(c_i\in\mathbb{R}),
\end{equation*}\begin{equation*}
	\lim \prod_{i=1}^n f_i(x) = \prod_{i=1}^n A_i.
\end{equation*}
\end{corollary}
例如,如果\(\lim f(x)\)、\(\lim g(x)\)和\(\lim h(x)\)都存在,
则有\begin{equation*}
	\lim[f(x) + g(x) - h(x)] = \lim f(x) + \lim g(x) - \lim h(x),
\end{equation*}\begin{equation*}
	\lim[f(x) \cdot g(x) \cdot h(x)] = \lim f(x) \cdot \lim g(x) \cdot \lim h(x).
\end{equation*}

\begin{example}
%@see: 《高等数学(第六版 上册)》 P46
设有理整函数\begin{equation*}
	P_n(x) = a_n x^n + a_{n-1} x^{n-1} + \dotsb + a_1 x + a_0.
\end{equation*}
求有理整函数\(P_n\)当\(x\to\xi\)时的极限\(\lim_{x\to\xi} P_n(x)\)时,
只要用\(\xi\)代入\(x\)就行了,
也就是说\begin{equation}\label{equation:函数极限.重要极限3}
	\lim_{x \to \xi} (a_0 x^n + a_1 x^{n-1} + \dotsb + a_{n-1} x + a_n)
	= a_0 \xi^n + a_1 \xi^{n-1} + \dotsb + a_{n-1} \xi + a_n.
\end{equation}
\end{example}

\begin{example}
%@see: 《高等数学(第六版 上册)》 P48
%@see: 《数学分析(第二版 上册)》(陈纪修) P83 例3.1.12
设有理分式函数\begin{equation*}
	F(x)
	= \frac{a_n x^n + a_{n-1} x^{n-1} + \dotsb + a_1 x + a_0}
	{b_n x^m + b_{m-1} x^{m-1} + \dotsb + b_1 x + b_0}.
\end{equation*}
求有理分式函数\(F\)当\(x\to\xi\)时的极限\(\lim_{x\to\xi} F(x)\)时,
只要分母在点\(\xi\)不为零,即\begin{equation*}
	b_n \xi^m + b_{m-1} \xi^{m-1} + \dotsb + b_1 \xi + b_0 \neq 0,
\end{equation*}
就可以直接用\(\xi\)代入\(x\),
得到\begin{equation}\label{equation:函数极限.重要极限4}
	\lim_{x\to\xi} F(x)
	= \frac{a_n \xi^n + a_{n-1} \xi^{n-1} + \dotsb + a_1 \xi + a_0}
	{b_n \xi^m + b_{m-1} \xi^{m-1} + \dotsb + b_1 \xi + b_0}.
\end{equation}
特别地,我们有\begin{equation}\label{equation:函数极限.重要极限5}
	\lim_{x\to0} \frac{a_n x^n + a_{n-1} x^{n-1} + \dotsb + a_p x^p}
	{b_m x^m + b_{m-1} x^{m-1} + \dotsb + a_q x^q}
	= \left\{ \begin{array}{cl}
		a_p/b_p, & p=q, \\
		0, & p>q, \\
		\infty, & p<q.
	\end{array} \right.
\end{equation}

假设\(a_n,b_n\neq0\),
有理分式函数\(F\)当\(x\to\infty\)时的极限为\begin{equation}\label{equation:函数极限.重要极限6}
	\lim_{x\to\infty} F(x)
	= \left\{ \begin{array}{cl}
		a_n/b_n, & m=n, \\
		0, & n<m, \\
		\infty, & n>m.
	\end{array} \right.
\end{equation}
\end{example}

\begin{example}
%@see: 《高等数学(第六版 上册)》 P52 例1
求:\(\lim_{x\to0} \frac{\tan x}{x}\).
\begin{solution}
直接计算得\begin{align*}
	\lim_{x\to0} \frac{\tan x}{x}
	&= \lim_{x\to0} \left(\frac{\sin x}{x} \cdot \frac{1}{\cos x}\right)
		\tag{正切函数的定义} \\
	&= \lim_{x\to0} \frac{\sin x}{x} \cdot \lim_{x\to0} \frac{1}{\cos x},
		\tag{\hyperref[theorem:极限.极限的四则运算法则]{四则运算法则}}
\end{align*}
其中,由\cref{equation:函数极限.重要极限1} 可知\(\lim_{x\to0} \frac{\sin x}{x} = 1\);
又因为\(\lim_{x\to0} \cos x = 1\),
所以\begin{align*}
	\lim_{x\to0} \frac{1}{\cos x}
	&= \left(\lim_{x\to0} \cos x\right)^{-1}
		\tag{\hyperref[theorem:极限.极限的四则运算法则]{四则运算法则}} \\
	&= 1;
\end{align*}
因此\begin{equation}\label{equation:函数极限.重要极限7}
	\lim_{x\to0} \frac{\tan x}{x} = 1.
\end{equation}
\end{solution}
\end{example}

\subsection{复合函数的极限运算法则}
\begin{theorem}\label{theorem:极限.复合函数的极限运算法则1}
%@see: 《高等数学(第六版 上册)》 P48 定理6(复合函数的极限运算法则)
%@see: 《数学分析教程(第3版 上册)》(史济怀) P74 定理2.4.8
设\(f\colon X\to Y\)与\(g\colon Y\to\mathbb{R}\)复合而成的函数\(g \circ f\)
在点\(x_0\)的某个去心邻域内有定义,
即\begin{equation*}
	(\exists\rho>0)[\mathring{U}(x_0,\rho) \subseteq \dom(g \circ f)].
\end{equation*}
如果\(\lim_{x \to x_0} f(x) = u_0\),
\(\lim_{u \to u_0} g(u) = A\),
且\begin{equation*}
	(\exists\delta>0)(\forall x)[x\in\mathring{U}(x_0,\delta) \implies f(x)\neq u_0],
\end{equation*}
则\begin{equation*}
	\lim_{x \to x_0} g(f(x))
	= \lim_{u \to u_0} g(u)
	= A.
\end{equation*}
%TODO proof
\end{theorem}

\begin{remark}
%@see: 《数学分析(第二版 上册)》(陈纪修) P96
应该注意到,
条件“\((\exists\delta>0)(\forall x)[x\in\mathring{U}(x_0,\delta) \implies f(x)\neq u_0]\)”
是不可或缺的!
取\begin{equation*}
	f(x) = \abs{\sgn x}, \qquad
	g(x) = x \sin\frac1x,
\end{equation*}
显然有\begin{equation*}
	\lim_{x\to0} g(x) = 0, \qquad
	\lim_{u\to0} f(u) = 1.
\end{equation*}
但是复合函数\(f \circ g\)在\(x=0\)没有极限.
这是因为当\(x\neq0\)时\begin{equation*}
	g(x) = 0
	\iff
	\sin\frac1x = 0
	\iff
	\frac1x = n\pi
	\iff
	x = \frac1{n\pi},
\end{equation*}
其中\(n\in\mathbb{Z}\),
因此在点\(x=0\)的任意去心邻域内,总有\(g\)的零点,无法满足上述条件.
\end{remark}

\begin{example}
%@see: 《高等数学(第六版 上册)》 P52 例2
%@see: 《数学分析教程(第3版 上册)》(史济怀) P77 例7
求:\(\lim_{x\to0} \frac{1-\cos x}{x^2}\).
\begin{solution}
由\hyperref[equation:三角函数.正弦的半倍角公式]{半倍角公式}可知
\(1 - \cos\alpha = 2\sin^2\frac\alpha2\ (-\infty<\alpha<+\infty)\),
于是\begin{equation*}
	\lim_{x\to0} \frac{1-\cos x}{x^2}
	= 2 \lim_{x\to0} \frac1{x^2} \sin^2\frac{x}2.
\end{equation*}
令\(t=\frac{x}2\),
则\(x=2t\),
且\(t\to0\ (x\to0)\),
于是由复合函数的极限运算法则得\begin{equation*}
	\lim_{x\to0} \frac{1-\cos x}{x^2}
	= 2 \lim_{t\to0} \frac{\sin^2 t}{(2t)^2}
	= \frac12 \lim_{t\to0} \left(\frac{\sin t}{t}\right)^2
	= \frac12 \left(\lim_{t\to0} \frac{\sin t}{t}\right)^2.
\end{equation*}
由\cref{equation:函数极限.重要极限1} 可知\(\lim_{t\to0} \frac{\sin t}{t} = 1\),
因此\begin{equation}\label{equation:函数极限.重要极限8}
	\lim_{x\to0} \frac{1-\cos x}{x^2} = \frac12.
\end{equation}
\end{solution}
\end{example}

\begin{example}
%@see: 《高等数学(第六版 上册)》 P60 习题1-7 3. (2)
求:\(\lim_{x\to0} \frac{\sec x-1}{x^2}\).
\begin{solution}
因为\(\sec x = \frac1{\cos x}\),
所以\begin{equation*}
	\lim_{x\to0} \frac{\sec x-1}{x^2}
	= \lim_{x\to0} \frac{1-\cos x}{x^2} \cdot \frac1{\cos x}
	= \lim_{x\to0} \frac{1-\cos x}{x^2} \cdot \lim_{x\to0} \frac1{\cos x}.
\end{equation*}
又因为\(\lim_{x\to0} \cos x = 1\),
并且由\cref{equation:函数极限.重要极限8} 有
\(\lim_{x\to0} \frac{1-\cos x}{x^2} = \frac12\),
因此\begin{equation}\label{equation:函数极限.重要极限15}
	\lim_{x\to0} \frac{\sec x-1}{x^2} = \frac12.
\end{equation}
\end{solution}
\end{example}

\begin{example}
%@see: 《高等数学(第六版 上册)》 P52 例3
求:\(\lim_{x\to0} \frac{\arcsin x}{x}\).
\begin{solution}
令\(t = \arcsin x\),
则\(x = \sin t\).
当\(x\to0\)时,
有\(t\to0\).
于是由复合函数的极限运算法则得\begin{equation*}
	\lim_{x\to0} \frac{\arcsin x}{x}
	= \lim_{t\to0} \frac{t}{\sin t}
	= \left(\lim_{t\to0} \frac{\sin t}{t}\right)^{-1}.
\end{equation*}
因此\begin{equation}\label{equation:函数极限.重要极限9}
	\lim_{x\to0} \frac{\arcsin x}{x} = 1.
\end{equation}
\end{solution}
\end{example}

\begin{example}
%@see: 《高等数学(第六版 上册)》 P59 习题1-7 3. (1)
计算极限\(\lim_{x\to0} \frac{\arctan x}{x}\).
\begin{solution}
令\(t = \arctan x\),
则\(x = \tan t\),
且\(t\to0(x\to0)\).
于是\begin{equation*}
	\lim_{x\to0} \frac{\arctan x}{x}
	= \lim_{t\to0} \frac{t}{\tan t}.
\end{equation*}
由\cref{equation:函数极限.重要极限7} 可知
\(\lim_{t\to0} \frac{t}{\tan t} = 1\),
因此\begin{equation}\label{equation:函数极限.重要极限10}
	\lim_{x\to0} \frac{\arctan x}{x} = 1.
\end{equation}
\end{solution}
\end{example}

\begin{example}
%@see: 《高等数学(第六版 上册)》 P55
计算极限\(\lim_{x\to0} (1+x)^{\frac1x}\).
\begin{solution}
令\(z = \frac1x\),
则\(z\to\infty\ (x\to0)\),
于是由复合函数的极限运算法则得\begin{equation*}
	\lim_{x\to0} (1+x)^{\frac1x}
	= \lim_{z\to\infty} \left(1+\frac1z\right)^z.
\end{equation*}
由\cref{equation:函数极限.重要极限2} 可知
\(\lim_{z\to\infty} \left(1+\frac1z\right)^z = e\).
因此\begin{equation}\label{equation:函数极限.重要极限11}
	\lim_{x\to0} (1+x)^{\frac1x} = e.
\end{equation}
\end{solution}
\end{example}

\begin{example}
计算极限\(\lim_{x\to0} \frac{\ln(1+x)}{x}\).
\begin{solution}
由\hyperref[equation:函数.对数的基本运算法则3]{对数的运算法则}有
\(\frac{\ln(1+x)}{x} = \ln(1+x)^{\frac1x}\).
记\(u(x) = (1+x)^{\frac1x}\ (-1<x<1,x\neq0)\).
那么显然有\(u(x)>0\),
从而复合函数\(\ln u(x)\)在去心邻域\(\mathring{U}(0,1)\)内有定义.
因为由\cref{equation:函数极限.重要极限11} 可知\(\lim_{x\to0} u(x) = e\),
由\cref{equation:函数极限.重要极限13} 可知\(\lim_{u\to e} \ln u = 1\),
并且在去心邻域\(\mathring{U}(0,1)\)内成立\(u(x)\neq e\),
于是由复合函数的极限运算法则得\begin{equation}\label{equation:函数极限.重要极限12}
	\lim_{x\to0} \frac{\ln(1+x)}{x} = 1.
\end{equation}
\end{solution}
\end{example}

\begin{example}
计算极限\(\lim_{x\to0} \frac{e^x-1}{x}\).
\begin{solution}
令\(x=\ln(1+t)\),
则\(e^x-1=t\),
且\(t\to0\ (x\to0)\),
于是\begin{equation*}
	\lim_{x\to0} \frac{e^x-1}{x}
	= \lim_{t\to0} \frac{t}{\ln(1+t)}.
\end{equation*}
由\cref{equation:函数极限.重要极限12} 可知\(\lim_{t\to0} \frac{\ln(1+t)}{t} = 1\),
因此\begin{equation}\label{equation:函数极限.重要极限14}
	\lim_{x\to0} \frac{e^x-1}{x} = 1.
\end{equation}
\end{solution}
\end{example}

\begin{example}
计算极限\(\lim_{x\to0} \frac{a^x-1}{x}\).
\begin{solution}
令\(t=a^x-1\),
则\(x=\log_a(t+1)\),
且\(t\to0\ (x\to0)\),
于是\begin{equation*}
	\lim_{x\to0} \frac{a^x-1}{x}
	= \lim_{t\to0} \frac{t}{\log_a(t+1)}.
\end{equation*}
利用\hyperref[equation:函数.换底公式]{换底公式}可得
\(\log_a(t+1) = \frac{\ln(t+1)}{\ln a}\),
因此\begin{equation*}
	\lim_{x\to0} \frac{a^x-1}{x}
	= \ln a \cdot \lim_{t\to0} \frac{t}{\ln(t+1)}.
\end{equation*}
由\cref{equation:函数极限.重要极限12} 可知
\(\lim_{t\to0} \frac{\ln(t+1)}{t} = 1\),
因此\begin{equation}\label{equation:函数极限.重要极限17}
	\lim_{x\to0} \frac{a^x-1}{x} = \ln a.
\end{equation}
\end{solution}
\end{example}

\begin{example}
%@see: 《高等数学(第六版 上册)》 P55 例4
计算极限\(\lim_{x\to\infty} \left(1-\frac1x\right)^x\).
\begin{figure}[htb]
%@Mathematica: Plot[{1/E, (1 - 1/x)^x}, {x, -10, 10}, PlotRange -> {0, 1}, PlotStyle -> {{Dashed, Thin}, Thin}]
	\centering
	\begin{tikzpicture}[scale=.5]
		\begin{axis}[
			xmin=-10,xmax=10,
			ymin=0,ymax=1,
			grid=both,
			width=\textwidth,height=\textwidth,
			xlabel=$x$,
			ylabel=$y$,
			axis lines=middle,
			xtick={-9,-7,...,10},
			ytick={.3679,1},
			yticklabels={$\frac1e$,$1$},
		]
			\begin{scope}[samples=50,thick,red]
				\addplot[domain=-10:-0]{(1-1/x)^x};
				\addplot[domain=+1:+10]{(1-1/x)^x};
			\end{scope}
		\end{axis}
	\end{tikzpicture}
	\caption{函数\(y=\left(1-\tfrac1x\right)^x\)的图像}
\end{figure}
\begin{solution}
令\(t = -x\),
则当\(x \to +\infty\)时,
\(t \to -\infty\),
于是\begin{equation*}
	\lim_{x\to+\infty} \left(1-\frac1x\right)^x
	= \lim_{t\to-\infty} \left(1+\frac1t\right)^{-t}
	= \left[\lim_{t\to-\infty} \left(1+\frac1t\right)^t\right]^{-1}
	= \frac1e.
\end{equation*}
\end{solution}
\end{example}

\begin{example}
计算极限\(\lim_{x\to0^-} \left(1-\frac1x\right)^x\).
\begin{solution}
直接计算得\begin{equation*}
	\lim_{x\to0^-} \left(1-\frac1x\right)^x
	\xlongequal{t=-1/x} \lim_{t\to+\infty} \frac{1}{\sqrt[t]{1+t}}
	= \left(\lim_{t\to+\infty} \sqrt[t]{1+t}\right)^{-1}
	= 1.
\end{equation*}
\end{solution}
\end{example}

\begin{theorem}
%@see: 《数学分析(第7版 第一卷)》(卓里奇) P110 定理5(复合函数极限定理)
设\(\mathcal{B}_Y\)是集合\(Y\)中的基,
\(\mathcal{B}_X\)是集合\(X\)中的基,
映射\(g\colon Y\to\mathbb{R}\)在基\(\mathcal{B}_Y\)上有极限,
映射\(f\colon X\to Y\)满足
\((\forall B_Y\in\mathcal{B}_Y)
(\exists B_X\in\mathcal{B}_X)
[f(B_X) \subseteq B_Y]\),
则复合映射\(g \circ f\colon X\to\mathbb{R}\)在基\(\mathcal{B}_X\)上有极限,
且\begin{equation*}
	\lim_{\mathcal{B}_X} (g \circ f)(x)
	= \lim_{\mathcal{B}_X} g(f(x))
	= \lim_{\mathcal{B}_Y} g(y).
\end{equation*}
\begin{proof}
设\(\lim_{\mathcal{B}_Y} g(y) = A\).
我们来证明\(\lim_{\mathcal{B}_X} g(f(x)) = A\).
按照点\(A\)的给定邻域\(V(A)\)求出基\(\mathcal{B}_Y\)的元素\(B_Y\),
使得\(g(B_Y) \subseteq V(A)\).
根据条件,可以求出基\(\mathcal{B}_X\)的元素\(B_X\),
使得\(f(B_X) \subseteq B_Y\).
但此时\((g \circ f)(B_X) = g(f(B_X)) \subseteq g(B_Y) \subseteq V(A)\),
这就说明\(A\)是\(g \circ f\)在基\(\mathcal{B}_X\)上的极限.
\end{proof}
\end{theorem}

\section{无穷小与无穷大}
\subsection{无穷小的概念}
\begin{definition}
%@see: 《高等数学(第六版 上册)》 P39 定义1
%@see: 《数学分析(第二版 上册)》(陈纪修) P100 定义3.3.1
若\(\lim_{x \to x_0} f(x) = 0\),
则称“函数\(f\)是当\(x \to x_0\)时的\DefineConcept{无穷小}(infinitesimal)”.
\end{definition}
这里的极限过程\(x \to x_0\)可以扩充到\(x \to x_0^+\)、\(x \to x_0^-\)、\(x \to \infty\)、\(x \to +\infty\)、\(x \to -\infty\)等情况.

\begin{definition}
设\(f\in\mathbb{R}^X\),\(\mathcal{B}\)是\(X\)中的基.
若\(\lim_\mathcal{B} f(x) = 0\),
则称“函数\(f\)是在基\(\mathcal{B}\)上的\DefineConcept{无穷小}”.
\end{definition}

\begin{theorem}
%@see: 《高等数学(第六版 上册)》 P39 定理1
设\(f\in\mathbb{R}^X\),\(\mathcal{B}\)是\(X\)中的基.
\(\lim_\mathcal{B} f(x) = A \in \mathbb{R}\)的充分必要条件是:\begin{equation*}
	(\exists\alpha\in\mathbb{R}^X)
	\left[
		\lim_\mathcal{B} \alpha(x) = 0
		\land
		f(x) = A + \alpha(x)
	\right].
\end{equation*}
\begin{proof}
这里假设\(f\)在点\(x_0\)的某个去心邻域内有定义,
即\((\exists\rho>0)[\mathring{U}(x_0,\rho) \subseteq X]\),
在此前提下证明\begin{equation*}
	\lim_{x \to x_0} f(x) = A
	\iff
	(\exists\alpha\in\mathbb{R}^X)
	\left[
		\lim_{x \to x_0} \alpha(x) = 0
		\land
		f(x) = A + \alpha(x)
	\right].
\end{equation*}

必要性.
设\(\lim_{x \to x_0} f(x) = A\),
由定义有\begin{equation*}
	(\forall\epsilon>0)
	(\exists\delta>0)
	(\forall x\in X)
	[
		0<\abs{x-x_0}<\delta
		\implies
		\abs{f(x)-A}<\epsilon
	].
\end{equation*}
令\(\alpha(x)=f(x)-A\),
则\begin{equation*}
	(\forall\epsilon>0)
	(\exists\delta>0)
	(\forall x\in X)
	[
		0<\abs{x-x_0}<\delta
		\implies
		\abs{\alpha(x)-0}<\epsilon
	],
\end{equation*}
也就是说函数\(\alpha\)是当\(x \to x_0\)时的无穷小.

充分性.
设\(f(x)=A+\alpha(x)\),其中\(A\)是常数,\(\alpha\)是当\(x \to x_0\)时的无穷小.
由定义有\begin{equation*}
	(\forall\epsilon>0)
	(\exists\delta>0)
	(\forall x\in X)
	[
		0<\abs{x-x_0}<\delta
		\implies
		\abs{\alpha(x)-0}<\epsilon
		\implies
		\abs{f(x)-A}<\epsilon
	],
\end{equation*}
这就说明\(A\)是函数\(f\)当\(x \to x_0\)时的极限.
\end{proof}
\end{theorem}

\subsection{无穷大的概念}
\begin{definition}
%@see: 《数学分析(第二版 上册)》(陈纪修) P103 定义3.3.2
设\(f\in\mathbb{R}^X\).
\begin{itemize}
	\item 若\(\lim_{x\to\infty} f(x) = \infty\),
	则称“函数\(f\)是当\(x\to\infty\)时的\DefineConcept{无穷大}”.

	\item 若\(\lim_{x\to\infty} f(x) = +\infty\),
	则称“函数\(f\)是当\(x\to\infty\)时的\DefineConcept{正无穷大}”.

	\item 若\(\lim_{x\to\infty} f(x) = -\infty\),
	则称“函数\(f\)是当\(x\to\infty\)时的\DefineConcept{负无穷大}”.
\end{itemize}
我们把正无穷大和负无穷大统称为\DefineConcept{定号无穷大},
把既非正无穷大又非负无穷大的无穷大称为\DefineConcept{不定号无穷大}.
\end{definition}
这里的极限过程\(x \to \infty\)可以扩充到\(x \to x_0\)、\(x \to x_0^+\)、\(x \to x_0^-\)、\(x \to +\infty\)、\(x \to -\infty\)等情况.

\begin{definition}
设\(f\in\mathbb{R}^X\),\(\mathcal{B}\)是\(X\)中的基.
\begin{itemize}
	\item 若\(\lim_\mathcal{B} f(x) = \infty\),
	则称“函数\(f\)是在基\(\mathcal{B}\)上的\DefineConcept{无穷大}”.

	\item 若\(\lim_\mathcal{B} f(x) = +\infty\),
	则称“函数\(f\)是在基\(\mathcal{B}\)上的\DefineConcept{正无穷大}”.

	\item 若\(\lim_\mathcal{B} f(x) = -\infty\),
	则称“函数\(f\)是在基\(\mathcal{B}\)上的\DefineConcept{负无穷大}”.
\end{itemize}
\end{definition}

\subsection{无穷小的比较}
现在我们已经知道,
两个无穷小的和、差、积仍旧是无穷小.
但是,两个无穷小的商,却会出现不同的情况.
例如,当\(x\to0\)时,
\(3x\)、\(x^2\)、\(\sin x\)都是无穷小,
而\begin{equation*}
	\lim_{x\to0}\frac{x^2}{3x}=0, \qquad
	\lim_{x\to0}\frac{3x}{x^2}=\infty, \qquad
	\lim_{x\to0}\frac{\sin x}{3x}=\frac{1}{3}.
\end{equation*}
两个无穷小之比的极限的各种不同情况,
反映了不同的无穷小趋于零的“快慢”程度.
就上面几个例子来说,
在\(x\to0\)的过程中,
\(x^2\to0\)比\(3x\to0\)要“快些”,
反过来说\(3x\to0\)比\(x^2\to0\)要“慢些”,
而\(\sin x\to0\)与\(x\to0\)“快慢相仿”.

\begin{definition}\label{definition:无穷小和无穷大.无穷小的比较}
%@see: 《高等数学(第六版 上册)》 P57 定义
设\(\alpha,\beta\in\mathbb{R}^X\),\(\mathcal{B}\)是\(X\)中的基,
\(\alpha\)和\(\beta\)都是在基\(\mathcal{B}\)上的无穷小,
且\(\alpha(x)\neq0\).
\newcommand{\lf}[1][]{\lim_\mathcal{B} \frac{\beta(x)}{\alpha^{#1}(x)}}
\begin{itemize}
	\item 如果\(\lf=0\),
	就说“在基\(\mathcal{B}\)上,\(\beta\)是比\(\alpha\)~\DefineConcept{高阶}的无穷小
	(\(\beta\) is an infinitesimal of higher order than \(\alpha\))”,
	记作\(\beta=o(\alpha)\).

	\item 如果\(\lf=\infty\),
	就说“在基\(\mathcal{B}\)上,\(\beta\)是比\(\alpha\)~\DefineConcept{低阶}的无穷小
	(\(\beta\) is an infinitesimal of lower order than \(\alpha\))”.

	%@see: 《数学分析(第二版 上册)》(陈纪修) P101 (2)
	\item 如果\begin{equation*}
		(\exists A>0)
		(\exists B\in\mathcal{B})
		(\forall x\in B)
		\left[
			\abs{\frac{\beta(x)}{\alpha(x)}} \leq A
		\right],
	\end{equation*}
	就说“\(\frac\beta\alpha\)是在基\(\mathcal{B}\)上的\DefineConcept{有界量}”,
	记为\(\beta = O(\alpha)\).

	\item 如果\begin{equation*}
		(\exists A>0)
		(\exists a>0)
		(\exists B\in\mathcal{B})
		(\forall x\in B)
		\left[
			a \leq \abs{\frac{\beta(x)}{\alpha(x)}} \leq A
		\right],
	\end{equation*}
	就说“\(\alpha\)与\(\beta\)是在基\(\mathcal{B}\)上的\DefineConcept{同阶}无穷小”,
	记作\(\beta = \Theta(\alpha)\).

	\item 如果\(\lf[k]=c\ (\text{$c$是非零常数},\text{$k$是正常数})\),
	就说“\(\beta\)是在基\(\mathcal{B}\)上的关于\(\alpha\)的\(k\)~\DefineConcept{阶}无穷小”.

	\item 如果\(\lf=1\),
	就说“\(\beta\)与\(\alpha\)是在基\(\mathcal{B}\)上的\DefineConcept{等价无穷小}”,
	记作\(\alpha\sim\beta\).
\end{itemize}
%@see: https://encyclopediaofmath.org/wiki/Order
\end{definition}
\begin{remark}
只要我们把无穷小和高阶无穷小两者的定义联系起来,
再结合\begin{equation*}
	\lim_{x\to0} \frac{x}{1} = 0,
	\qquad
	\lim_{n\to\infty} \frac{1/n}{1} = 0
\end{equation*}这两个例子,可以看出:
任意一个无穷小都是常数\(1\)的高阶无穷小.
因此,在不需要强调具体的阶数时,我们常用\(o(1)\)表示某个函数是无穷小.
同理,在不需要强调具体的阶数时,用\(O(1)\)表示某个函数是有界量.
最后,我们用\(\Theta(1)\)表示某个函数是常数函数.
\end{remark}

\begin{proposition}
%@see: 《数学分析(第二版 上册)》(陈纪修) P101 (2)
设\(\alpha,\beta\in\mathbb{R}^X\),\(\mathcal{B}\)是\(X\)中的基,
\(\alpha\)和\(\beta\)都是在基\(\mathcal{B}\)上的无穷小,
且\(\alpha(x)\neq0\).
若\begin{equation*}
	\lim_\mathcal{B} \frac{\beta(x)}{\alpha(x)} = c\ (\text{$c$是非零常数}),
\end{equation*}
则\(\beta\)与\(\alpha\)是同阶无穷小.
\end{proposition}

%@see: 《高等数学(第六版 上册)》 P57
\begin{example}
因为\(\lim_{x\to0} \frac{3x^2}{x} = 0\),
所以当\(x\to0\)时,
\(3x^2\)是比\(x\)高阶的无穷小,
即\(3x^2 = o(x)\ (x\to0)\).
\end{example}

%@see: 《高等数学(第六版 上册)》 P58
\begin{example}
因为\(\lim_{n\to\infty} \frac{1/n}{1/n^2} = \infty\),
所以当\(n\to\infty\)时,
\(\frac{1}{n}\)是比\(\frac{1}{n^2}\)低阶的无穷小.
\end{example}

\begin{example}
因为\(\lim_{x\to3} \frac{x^2-9}{x-3} = 6\),
所以当\(x\to3\)时,
\(x^2-9\)与\(x-3\)是同阶无穷小.
\end{example}

\begin{example}
由\cref{equation:函数极限.重要极限8} 可知
\(\lim_{x\to0} \frac{1-\cos x}{x^2} = \frac12\),
所以当\(x\to0\)时,
\(1-\cos x\)是关于\(x\)的二阶无穷小,
也是关于\(x^2\)的同阶无穷小,
还是关于\(\frac12 x^2\)的等价无穷小.
\end{example}

\begin{example}
因为\(\lim_{x\to0} \frac{\sin x}{x} = 1\),
所以当\(x\to0\)时,
\(\sin x\)与\(x\)是等价无穷小,
即\(\sin x \sim x\ (x\to0)\).
\end{example}

\begin{remark}
记号“\(o\)”、“\(O\)”和“\(\sim\)”都是相对于一定的极限过程的,
一般来说,在使用这些记号时,我们应该附上表示极限过程的记号,例如“\((x \to x_0)\)”.
只有在意义明确,不会发生误解的前提下,才可以省略.
\end{remark}
\begin{remark}
应该注意到,
记号\(o(\alpha)\)实际上是满足\(\lim_\mathcal{B} \frac{\beta(x)}{\alpha(x)} = 0\)的全体函数,
即\begin{equation*}
	o(\alpha) = \Set*{ \beta\in\mathbb{R}^X \given \lim_\mathcal{B} \frac{\beta(x)}{\alpha(x)} = 0 }.
\end{equation*}
当我们说\(\beta = o(\alpha)\)时,
意思其实是“函数\(\beta\)是函数族\(o(\alpha)\)的一个元素”.
当我们说\(\gamma = \beta + o(\alpha)\)时,
意思其实是“存在\(\phi\in o(\alpha)\),使得\(\gamma = \beta + \phi\)”.
由此可知:当\(x\to0\)时,有\begin{equation*}
	o(1)
	\supseteq o(x)
	\supseteq o(x^2)
	\supseteq \dotsb
	\supseteq o(x^n)
	\supseteq \dotsb.
\end{equation*}
换句话说,高阶无穷小是低阶无穷小的子集.

另一方面,我们可以注意到:“\(\beta = \Theta(\alpha)\)”蕴含了“\(\beta = O(\alpha)\)”,
即\(\Theta(\alpha) \subseteq O(\alpha)\).
\end{remark}

\begin{proposition}
设\(\alpha,\beta\in\mathbb{R}^X\),\(\mathcal{B}\)是\(X\)中的基,
\(\alpha\)和\(\beta\)都是在基\(\mathcal{B}\)上的无穷小,
且\(\alpha(x)\neq0\),
则\begin{equation*}
	\text{\(\beta\)是比\(\alpha\)高阶的无穷小}
	\iff
	\text{\(\alpha\)是比\(\beta\)低阶的无穷小}.
\end{equation*}
\end{proposition}

\begin{remark}
显然,等价无穷小是同阶无穷小的特殊情形.
\end{remark}

\begin{definition}
%@see: 《数学分析(第二版 上册)》(陈纪修) P102
设\(f\colon\mathbb{R}\to\mathbb{R}\)是无穷小.
定义\DefineConcept{无穷小的阶}:\begin{gather*}
	\ord_{x \to x_0} f(x) = k
	\defiff
	\lim_{x \to x_0} \frac{f(x)}{(x-x_0)^k} = c < \infty, \\
	\ord_{x \to x_0^+} f(x) = k
	\defiff
	\lim_{x \to x_0^+} \frac{f(x)}{(x-x_0)^k} = c < \infty, \\
	\ord_{x \to x_0^-} f(x) = k
	\defiff
	\lim_{x \to x_0^-} \frac{f(x)}{(x-x_0)^k} = c < \infty, \\
	\ord_{x \to \infty} f(x) = k
	\defiff
	\lim_{x \to \infty} x^k f(x) = c < \infty, \\
	\ord_{x \to +\infty} f(x) = k
	\defiff
	\lim_{x \to +\infty} x^k f(x) = c < \infty, \\
	\ord_{x \to -\infty} f(x) = k
	\defiff
	\lim_{x \to -\infty} x^k f(x) = c < \infty.
\end{gather*}
\end{definition}
\begin{remark}
可以我们可以利用同阶无穷小记号把上述定义简化为:\begin{gather*}
	\ord_{x \to x_0} f(x) = k
	\defiff
	f(x) = \Theta((x-x_0)^k)\ (x \to x_0), \\
	\ord_{x \to x_0^+} f(x) = k
	\defiff
	f(x) = \Theta((x-x_0)^k)\ (x \to x_0^+), \\
	\ord_{x \to x_0^-} f(x) = k
	\defiff
	f(x) = \Theta((x-x_0)^k)\ (x \to x_0^-), \\
	\ord_{x \to \infty} f(x) = k
	\defiff
	f(x) = \Theta(1/x^k)\ (x \to \infty), \\
	\ord_{x \to +\infty} f(x) = k
	\defiff
	f(x) = \Theta(1/x^k)\ (x \to +\infty), \\
	\ord_{x \to -\infty} f(x) = k
	\defiff
	f(x) = \Theta(1/x^k)\ (x \to -\infty).
\end{gather*}
\end{remark}
\begin{example}
幂函数\(f(x) = a x^\mu\ (a\neq0)\)是当\(x\to0\)时的无穷小,
它的阶为\begin{equation*}
	\ord_{x\to0} f(x) = \mu.
\end{equation*}
\end{example}

\subsection{无穷大的比较}
\begin{definition}\label{definition:无穷小和无穷大.无穷大的比较}
设\(\alpha,\beta\in\mathbb{R}^X\),\(\mathcal{B}\)是\(X\)中的基,
\(\alpha\)和\(\beta\)都是在基\(\mathcal{B}\)上的无穷大.
\newcommand{\lf}[1][]{\lim_\mathcal{B} \frac{\beta(x)}{\alpha^{#1}(x)}}
\begin{itemize}
	%@see: 《数学分析(第二版 上册)》(陈纪修) P103 (1)
	\item 如果\(\lf=\infty\),
	就说“\(\beta\)是比\(\alpha\)~\DefineConcept{高阶}的无穷大”.

	\item 如果\(\lf=0\),
	就说“\(\beta\)是比\(\alpha\)~\DefineConcept{低阶}的无穷大”,
	记作\(\beta = o(\alpha)\).

	%@see: 《数学分析(第二版 上册)》(陈纪修) P103 (2)
	\item 如果\begin{equation*}
		(\exists A>0)
		(\exists B\in\mathcal{B})
		(\forall x\in B)
		\left[
			\abs{\frac{\beta(x)}{\alpha(x)}} \leq A
		\right],
	\end{equation*}
	就说“\(\frac\beta\alpha\)是在基\(\mathcal{B}\)上的\DefineConcept{有界量}”,
	记为\(\beta = O(\alpha)\).

	\item 如果\begin{equation*}
		(\exists A>0)
		(\exists a>0)
		(\exists B\in\mathcal{B})
		(\forall x\in B)
		\left[
			a \leq \abs{\frac{\beta(x)}{\alpha(x)}} \leq A
		\right],
	\end{equation*}
	就说“\(\alpha\)与\(\beta\)是\DefineConcept{同阶}无穷大\footnote{在研究算法时,
	我们会说“\(\alpha\)是\(\beta\)的一个\DefineConcept{渐进紧确界}%
	(asymptotically tight bound)”.}”,
	记作\(\beta = \Theta(\alpha)\).

	\item 如果\(\lf=1\),
	就说“\(\beta\)与\(\alpha\)是\DefineConcept{等价无穷大}”,
	记作\(\alpha\sim\beta\).
\end{itemize}
\end{definition}
\begin{remark}
根据\cref{definition:无穷小和无穷大.无穷小的比较,definition:无穷小和无穷大.无穷大的比较},
只要\(\alpha\)是无穷小或无穷大,
就成立\(\lim_\mathcal{B} \frac{o(\alpha)}{\alpha} = 0\).
\end{remark}

\begin{proposition}
设\(\alpha,\beta\in\mathbb{R}^X\),\(\mathcal{B}\)是\(X\)中的基,
\(\alpha\)和\(\beta\)都是在基\(\mathcal{B}\)上的无穷大,
则\begin{equation*}
	\text{\(\beta\)是比\(\alpha\)高阶的无穷大}
	\iff
	\text{\(\alpha\)是比\(\beta\)低阶的无穷大}.
\end{equation*}
\end{proposition}

\begin{proposition}
设\(\alpha,\beta\in\mathbb{R}^X\),\(\mathcal{B}\)是\(X\)中的基,
\(\alpha\)和\(\beta\)都是在基\(\mathcal{B}\)上的无穷大.
若\begin{equation*}
	\lim_\mathcal{B} \frac{\beta(x)}{\alpha(x)} = c\ (\text{$c$是非零常数}),
\end{equation*}
则\(\beta\)与\(\alpha\)是同阶无穷大.
\end{proposition}

\subsection{无穷小与无穷大的关系}
无穷大与无穷小之间有一种简单的关系,即:
\begin{theorem}\label{theorem:极限.无穷大与无穷小的关系}
%@see: 《高等数学(第六版 上册)》 P41 定理2
设\(f\in\mathbb{R}^X\),\(\mathcal{B}\)是\(X\)中的基.
令\(g(x) = \frac1{f(x)}\).
\begin{itemize}
	\item 若\(f\)是在基\(\mathcal{B}\)上的无穷大,
	则\(g\)为无穷小.

	\item 若\(f\)是在基\(\mathcal{B}\)上的无穷小,
	且\((\exists B\in\mathcal{B})(\forall x\in B)[f(x) \neq 0]\),
	则\(g\)为无穷大.
\end{itemize}
\end{theorem}

\begin{proposition}%渐进比较
%@see: 《算法导论(原书第3版)》 P29
%@see: 《阶的估计基础》(潘承洞、于秀源) P7 法则2
设\(\alpha,\beta,\gamma\in\mathbb{R}^X\),\(\mathcal{B}\)是\(X\)中的基,
\(\alpha,\beta,\gamma\)都是在基\(\mathcal{B}\)上的无穷小,
则\begin{itemize}
	\item {\rm\bf 传递性}\begin{gather*}
		\alpha = \Theta(\beta)
		\land
		\beta = \Theta(\gamma)
		\implies
		\alpha = \Theta(\gamma). \\
		\alpha = O(\beta)
		\land
		\beta = O(\gamma)
		\implies
		\alpha = O(\gamma). \\
		\alpha = o(\beta)
		\land
		\beta = o(\gamma)
		\implies
		\alpha = o(\gamma).
	\end{gather*}

	\item {\rm\bf 自反性}\begin{gather*}
		\alpha = \Theta(\alpha). \\
		\alpha = O(\alpha).
	\end{gather*}

	\item {\rm\bf 对称性}\begin{gather*}
		\alpha = \Theta(\beta)
		\iff
		\beta = \Theta(\alpha).
	\end{gather*}
\end{itemize}
\end{proposition}

显然,当一个函数是无穷大时,它必定无界;
但当一个函数无界时,它却不一定是无穷大.
\begin{example}
证明:函数\(f(x) = \frac1x \sin\frac1x\)在区间\((0,1]\)上无界,
但该函数不是\(x\to0^+\)时的无穷大.
\begin{proof}
取数列\(u_n = \frac{\pi}{2} + n\pi\ (n=0,1,2,\dotsc)\),
那么恒有\(\abs{\sin u_n} = 1\)和\begin{equation*}
	0 < \dotsb < \frac1{u_n} < \dotsb < \frac1{u_1} < \frac1{u_0} < 1
\end{equation*}成立.
易见\((\forall M > 0)(\forall n\in\mathbb{N})[n > M/\pi \implies u_n > M]\),
数列\(\{u_n\}\)无界.
由于\begin{equation*}
	\abs{f(x)} = \abs{\frac1x \sin\frac1x}
	= \abs{\frac1x} \abs{\sin\frac1x}
	= \frac1x \abs{\sin\frac1x},
\end{equation*}\begin{equation*}
	\abs{f(1/u_n)}
	= u_n \abs{\sin u_n}
	= u_n,
\end{equation*}
所以函数列\(\{\abs{f(1/u_n)}\}\)也无界,
自然地,函数\(f\)也无界.

用反证法.
假设\(f\)是当\(x\to0^+\)时的无穷大,
即\begin{equation*}
	(\forall M>0)
	(\exists\delta>0)
	(\forall x)
	[
		0<x<\delta
		\implies
		\abs{f(x)}>M
	].
\end{equation*}
取数列\(v_n = n\pi\ (n=1,2,\dotsc)\),
那么\(\sin v_n = 0\)和\(v_n>0\)恒成立,
且\begin{equation*}
	(\forall\delta>0)
	(\forall n\in\mathbb{N})
	\left[n>\ceil*{\frac1{\pi\delta}} \implies 0<\frac1{v_n}<\delta\right].
\end{equation*}
由于\begin{equation*}
	\abs{f\left(\frac1{v_n}\right)}
	= v_n \sin v_n
	= 0,
\end{equation*}
所以函数列\(\{\abs{f(1/v_n)}\}\)恒为零,
与假设矛盾,说明\(f(x)\)不是\(x\to0^+\)时的无穷大.
\end{proof}
\end{example}

\subsection{无穷小与无穷大的运算}
我们不难看出:当\(x\to0\)时,有\begin{gather*}
	x \cdot o(1) = o(x), \qquad
	x \cdot o(x) = o(x^2), \\
	x^{-1} \cdot o(x) = o(1), \qquad
	x^{-1} \cdot o(x^2) = o(x), \qquad
	x^{-2} \cdot o(x^2) = o(1),
\end{gather*}
利用数学归纳法可得以下结论.
\begin{proposition}
当\(x\to0\)时,对于任意整数\(k\),有\begin{gather}
	x^k \cdot o(x^n) = o(x^{n+k})
	\quad(k \geq 0), \\
	x^{-k} \cdot o(x^n) = o(x^{n-k})
	\quad(n \geq k > 0).
\end{gather}
\end{proposition}

\begin{proposition}\label{theorem:函数的无穷小与无穷大.和差取大规则}
设\(\alpha,\beta\in\mathbb{R}^X\),\(\mathcal{B}\)是\(X\)中的基,
\(\alpha\)和\(\beta\)都是在基\(\mathcal{B}\)上的无穷小.
如果\(\beta = o(\alpha)\),
则\begin{equation*}
	\alpha \pm \beta \sim \alpha.
\end{equation*}
%TODO proof
\end{proposition}
\begin{remark}
\cref{theorem:函数的无穷小与无穷大.和差取大规则} 说明,
当高阶无穷小\(\beta\)和低阶无穷小\(\alpha\)相加或相减时,
它们的等价无穷小就是低阶无穷小\(\alpha\).
如\cref{figure:无穷小与无穷大.次数越大的幂函数越贴近横轴} 所示,
因为当自变量变化时,高阶无穷小比低阶无穷小更快地趋于零,
相对而言,低阶无穷小就显得更“大”一些,
因此我们常常把\cref{theorem:函数的无穷小与无穷大.和差取大规则} 称为“和差取大规则”.
\end{remark}
\begin{figure}[htb]
	\centering
	\begin{tikzpicture}[scale=.6]
		\begin{axis}[
			name=Infinitesimal,
			xmin=0,xmax=.8,
			ymin=0,ymax=.8,
			grid=both,width=\textwidth,height=\textwidth,
			axis lines=middle,
			xlabel=$x$,
			ylabel=$y$,
			x label style={at={(ticklabel* cs:1.00)}, inner sep=5pt, anchor=west},
			y label style={at={(ticklabel* cs:1.00)}, inner sep=2pt, anchor=south},
			xtick={.1,.2,...,.8},
			ytick={.1,.2,...,.8},
		]
			\addplot[color=blue,samples=50,smooth,domain=0:.8]
				{x};\label{pgfplots:无穷小与无穷大.幂函数1}
			\addplot[color=orange,samples=50,smooth,domain=0:.8]
				{x^2};\label{pgfplots:无穷小与无穷大.幂函数2}
			\addplot[color=green,samples=50,smooth,domain=0:.8]
				{x^4};\label{pgfplots:无穷小与无穷大.幂函数4}
			\addplot[color=red,samples=50,smooth,domain=0:.8]
				{x^8};\label{pgfplots:无穷小与无穷大.幂函数8}
		\end{axis}
		\node[draw,fill=white,inner sep=0pt,below right=1em]
		at(Infinitesimal.north west){\small\begin{tblr}{cl}
			\ref{pgfplots:无穷小与无穷大.幂函数1} & \(x\) \\
			\ref{pgfplots:无穷小与无穷大.幂函数2} & \(x^2\) \\
			\ref{pgfplots:无穷小与无穷大.幂函数4} & \(x^4\) \\
			\ref{pgfplots:无穷小与无穷大.幂函数8} & \(x^8\) \\
		\end{tblr}};
	\end{tikzpicture}
	\caption{}
	\label{figure:无穷小与无穷大.次数越大的幂函数越贴近横轴}
\end{figure}

特别地,我们可得如下结论.
\begin{corollary}
当\(x\to0\)时,对于非负整数\(m,n\),如果\(m>n\),则\begin{equation*}
	o(x^m) + o(x^n) = o(x^n).
\end{equation*}
\end{corollary}
\begin{remark}
对于上述命题,我们还可以从集合包含关系的角度进行理解:
前面我们提到,高阶无穷小是低阶无穷小的子集,
而当\(x\to0\)时,如果\(m>n\),则\(o(x^n) \supseteq o(x^m)\),
于是\begin{equation*}
	o(x^m) \cup o(x^n) = o(x^n).
\end{equation*}
\end{remark}

\begin{proposition}
%@see: 《阶的估计基础》(潘承洞、于秀源) P7 法则9
设\(\alpha,\beta\in\mathbb{R}^X\),
则\begin{equation*}
	o(\alpha) + o(\beta) = o(\abs{\alpha} + \abs{\beta}).
\end{equation*}
%TODO proof
\end{proposition}
\begin{theorem}
%@see: 《高等数学(第六版 上册)》 P43 定理1
有限个无穷小的和也是无穷小.
\begin{proof}
对\cref{theorem:函数的无穷小与无穷大.和差取大规则} 运用数学归纳法可得.
\end{proof}
\end{theorem}

\begin{proposition}
%@see: 《阶的估计基础》(潘承洞、于秀源) P7 法则1
设\(\alpha,\beta\in\mathbb{R}^X\),
\(\mathcal{B}\)是\(X\)中的基,
\(\alpha\)是基\(\mathcal{B}\)上的无穷大,
\(\beta=O(1)\),
则\begin{equation*}
	\beta = o(\alpha).
\end{equation*}
%TODO proof
\end{proposition}
\begin{proposition}
%@see: 《阶的估计基础》(潘承洞、于秀源) P7 法则6
设\(\alpha\in\mathbb{R}^X\),
则\begin{equation*}
	o(1)~O(\alpha) = o(\alpha).
\end{equation*}
%TODO proof
\end{proposition}
\begin{proposition}
%@see: 《阶的估计基础》(潘承洞、于秀源) P7 法则7
设\(\alpha\in\mathbb{R}^X\),
则\begin{equation*}
	O(1)~o(\alpha) = o(\alpha).
\end{equation*}
%TODO proof
\end{proposition}
\begin{theorem}\label{theorem:函数极限.无穷小.有界函数与无穷小的乘积是无穷小}
%@see: 《高等数学(第六版 上册)》 P43 定理2
有界量与无穷小的乘积是无穷小.
\end{theorem}
%\cref{example:数列极限.有界函数与无穷小的乘积是无穷小}

\begin{corollary}
%@see: 《高等数学(第六版 上册)》 P44 推论1
常数与无穷小的乘积是无穷小.
%TODO proof
\end{corollary}

\begin{proposition}
%@see: 《阶的估计基础》(潘承洞、于秀源) P7 法则10
设\(\alpha,\beta\in\mathbb{R}^X\),
则\begin{equation*}
	o(\alpha)~o(\beta) = o(\alpha \beta).
\end{equation*}
%TODO proof
\end{proposition}
\begin{corollary}
%@see: 《高等数学(第六版 上册)》 P44 推论2
有限个无穷小的乘积也是无穷小.
%TODO proof
\end{corollary}

\begin{proposition}
%@see: 《阶的估计基础》(潘承洞、于秀源) P7 法则3
设\(\alpha,\beta,\gamma\in\mathbb{R}^X\).
如果\(\alpha = O(\beta),
\beta = o(\gamma)\)
则\begin{equation*}
	\alpha = o(\gamma).
\end{equation*}
%TODO proof
\end{proposition}
\begin{proposition}
%@see: 《阶的估计基础》(潘承洞、于秀源) P7 法则4
设\(\alpha,\beta\in\mathbb{R}^X\),
则\begin{equation*}
	O(\alpha) + O(\beta) = O(\alpha + \beta).
\end{equation*}
%TODO proof
\end{proposition}
\begin{proposition}
%@see: 《阶的估计基础》(潘承洞、于秀源) P7 法则5
设\(\alpha,\beta\in\mathbb{R}^X\),
则\begin{equation*}
	O(\alpha)~O(\beta) = O(\alpha \beta).
\end{equation*}
%TODO proof
\end{proposition}
\begin{proposition}
%@see: 《阶的估计基础》(潘承洞、于秀源) P7 法则8
设\(\alpha\in\mathbb{R}^X\),
则\begin{equation*}
	O(\alpha) + o(\alpha) = O(\alpha).
\end{equation*}
%TODO proof
\end{proposition}
\begin{proposition}
%@see: 《阶的估计基础》(潘承洞、于秀源) P7 法则12
设\(\alpha\in\mathbb{R}^X\),
则\begin{equation*}
	(o(\alpha))^k = o(\alpha^k).
\end{equation*}
%TODO proof
\end{proposition}

\begin{example}
%@see: 《高等数学(第六版 上册)》 P48 例8
求\(\lim_{x\to\infty} \frac{\sin x}{x}\).
\begin{solution}
当\(x\to\infty\)时,分子的极限\(\lim_{x\to\infty} \sin x\)不存在,分母\(x\to\infty\),
故\hyperref[theorem:极限.极限的四则运算法则]{关于商的极限的运算法则}不能应用.
但是我们可以把\(\frac{\sin x}{x}\)看作\(\sin x\)与\(\frac1x\)的乘积.
由于\(\frac1x\)是当\(x\to\infty\)时的无穷小,
而\(\sin x\)是有界函数,
那么根据\cref{theorem:函数极限.无穷小.有界函数与无穷小的乘积是无穷小}
有\begin{equation}
	\lim_{x\to\infty} \frac{\sin x}{x} = 0.
\end{equation}
\end{solution}
\end{example}

\begin{example}
求\(\lim_{x\to0} x \sin\frac1x\).
\begin{solution}
当\(x\to0\)时,\(\sin\frac1x\)是有界函数,而\(x\)是无穷小,
那么根据\cref{theorem:函数极限.无穷小.有界函数与无穷小的乘积是无穷小}
有\begin{equation}
	\lim_{x\to0} x \sin\frac1x = 0.
\end{equation}
\end{solution}
\end{example}

\begin{example}
%@see: 《高等数学(第六版 上册)》 P49 习题1-5 3. (2)
计算极限\(\lim_{x\to\infty} \frac{\arctan x}{x}\).
\begin{solution}
令\(t = \arctan x\),
则\(x = \tan t\),
于是根据\cref{theorem:函数极限.无穷小.有界函数与无穷小的乘积是无穷小} 可得\begin{equation*}
	\lim_{x\to+\infty} \frac{\arctan x}{x}
	= \lim_{t\to\frac\pi2^-} \frac{t}{\tan t}
	= 0,
\end{equation*}\begin{equation*}
	\lim_{x\to-\infty} \frac{\arctan x}{x}
	= \lim_{t\to-\frac\pi2^+} \frac{t}{\tan t}
	= 0.
\end{equation*}
因此,根据\cref{theorem:函数极限.极限与单侧极限的关系2} 有\begin{equation}
	\lim_{x\to\infty} \frac{\arctan x}{x} = 0.
\end{equation}
\end{solution}
\end{example}

\subsection{等价替换}
必须指出的是,等价无穷小、等价无穷大都是等价关系.
\begin{property}
%@see: 《高等数学(第六版 上册)》 P60 习题1-7 5.
%@see: 《阶的估计基础》(潘承洞、于秀源) P7 法则13
设\(\mathcal{B}\)是\(X\)中的基,
\(\alpha,\beta,\gamma\in\mathbb{R}^X\)都是在基\(\mathcal{B}\)上的无穷小(或无穷大),
那么\begin{itemize}
	\item {\rm\bf 自反性}:
	\(\alpha \sim \alpha\);

	\item {\rm\bf 对称性}:
	\(\alpha \sim \beta \implies \beta \sim \alpha\);

	\item {\rm\bf 传递性}:
	\(\alpha \sim \beta \land \beta \sim \gamma \implies \alpha \sim \gamma\).
\end{itemize}
\end{property}

与其他等价关系类似,在计算极限时,
我们也可以用一个等价无穷小(或无穷大)去替换另一个无穷小(或无穷大).

\begin{theorem}\label{theorem:极限.无穷小的比较1}
%@see: 《高等数学(第六版 上册)》 P58 定理1
%@see: 《阶的估计基础》(潘承洞、于秀源) P7 法则14
设\(\alpha,\beta\in\mathbb{R}^X\),\(\mathcal{B}\)是\(X\)中的基,
\(\alpha\)和\(\beta\)都是在基\(\mathcal{B}\)上的无穷小,
则\(\beta\)与\(\alpha\)是等价无穷小的充分必要条件是
\(\beta = \alpha + o(\alpha)\).
\begin{proof}
必要性.
设\(\alpha\sim\beta\),
由定义有\(\lim_\mathcal{B} \frac{\beta(x)}{\alpha(x)} = 1\),
则\begin{equation*}
	\lim_\mathcal{B} \frac{\beta(x)-\alpha(x)}{\alpha(x)}
	= \lim_\mathcal{B} \left(\frac{\beta(x)}{\alpha(x)}-1\right)
	= \lim_\mathcal{B} \frac{\beta(x)}{\alpha(x)}-1 = 0,
\end{equation*}
因此\(\beta-\alpha=o(\alpha)\),
即\(\beta=\alpha+o(\alpha)\).

充分性.
设\(\beta=\alpha+o(\alpha)\),
则\begin{equation*}
	\lim_\mathcal{B} \frac{\beta(x)}{\alpha(x)}
	= \lim_\mathcal{B} \frac{\alpha(x)+o(\alpha)}{\alpha(x)}
	= \lim_\mathcal{B} \left(1+\frac{o(\alpha)}{\alpha(x)}\right)
	= 1 + \lim_\mathcal{B} \frac{o(\alpha)}{\alpha(x)}
	= 1,
\end{equation*}
因此\(\alpha\sim\beta\).
\end{proof}
\end{theorem}
\begin{corollary}
设\(\alpha,\beta\in\mathbb{R}^X\),\(\mathcal{B}\)是\(X\)中的基,
\(\alpha\)和\(\beta\)都是在基\(\mathcal{B}\)上的无穷小,
则\(\beta\)与\(\alpha\)是同阶无穷小的充分必要条件是
存在非零实数\(c\)使得
\(\beta = c \alpha + o(\alpha)\).
\end{corollary}

\begin{theorem}\label{theorem:极限.无穷小的比较2}
%@see: 《数学分析(第二版 上册)》(陈纪修) P106 定理3.3.1
%@see: 《高等数学(第六版 上册)》 P59 定理2
设\(\mathcal{B}\)是\(X\)中的基,
\(u,v,w\in\mathbb{R}^X\)在\(x_0 \in X\)的某个去心邻域上有定义,
且\begin{equation*}
	\lim_\mathcal{B} \frac{v(x)}{w(x)} = 1.
\end{equation*}
\begin{itemize}
	\item 当\(\lim_\mathcal{B} u(x) \cdot w(x) = A\)时,
	有\(\lim_\mathcal{B} u(x) \cdot v(x) = A\).

	\item 当\(\lim_\mathcal{B} \frac{u(x)}{w(x)} = A\)时,
	有\(\lim_\mathcal{B} \frac{u(x)}{v(x)} = A\).
\end{itemize}
\begin{proof}
由\hyperref[theorem:极限.极限的四则运算法则]{极限的四则运算法则}直接可得.
\end{proof}
\end{theorem}

\cref{theorem:极限.无穷小的比较2} 表明,
求两个无穷小之比的极限时,
分子及分母都可用等价无穷小来代替.
因此,如果用来代替的无穷小选得适当的话,
可以使计算简化.

\begin{example}
%@see: 《高等数学(第六版 上册)》 P58 例1
证明:当\(x\to0\)时,\begin{equation}
	\sqrt[n]{1+x} - 1 \sim \frac1n x.
\end{equation}
\begin{proof}
因为\begin{align*}
	\frac{\sqrt[n]{1+x} - 1}{\frac1n x}
	&= \frac{(\sqrt[n]{1+x})^n - 1}{\frac1n x \left[ \sqrt[n]{(1+x)^{n-1}} + \sqrt[n]{(1+x)^{n-2}} + \dotsb + 1 \right]} \\
	&= \frac{n}{\sqrt[n]{(1+x)^{n-1}} + \sqrt[n]{(1+x)^{n-2}} + \dotsb + 1},
\end{align*}
而\begin{equation*}
	\lim_{x\to0} \sqrt[n]{(1+x)^m} = 1,
\end{equation*}
所以\begin{equation*}
	\lim_{x\to0} \frac{\sqrt[n]{1+x} - 1}{\frac1n x} = \lim_{x\to0} \frac{n}{1 \cdot n} = 1,
\end{equation*}
也就是说\(\sqrt[n]{1+x} - 1 \sim \frac1n x \quad(x\to0)\).
\end{proof}
\end{example}

\begin{example}
设\(a\neq0\).
计算极限\(\lim_{x\to0} \frac{(1+x)^a-1}x\).
\begin{solution}
%@see: https://www.bilibili.com/video/BV1mG411Z7mV/?t=622
令\(t = (1+x)^a-1\),
则\begin{equation*}
	a \ln(1+x) = \ln(1+t),
	\qquad
	1+x=\exp[\frac1a \ln(1+t)],
	\qquad
	x=\exp[\frac1a \ln(1+t)]-1,
\end{equation*}
且\(t\to0\ (x\to0)\),
于是\begin{equation*}
	\lim_{x\to0} \frac{(1+x)^a-1}x
	= \lim_{t\to0} \frac{t}{\exp[\frac1a \ln(1+t)]-1}.
\end{equation*}
因为\(\ln(1+t)\to0\ (t\to0)\)和\(e^x-1 \sim x\ (x\to0)\),
所以\begin{equation*}
	\lim_{x\to0} \frac{(1+x)^a-1}x
	= \lim_{t\to0} \frac{t}{\frac1a \ln(1+t)}
	= a \lim_{t\to0} \frac{t}{\ln(1+t)}.
\end{equation*}
由\cref{equation:函数极限.重要极限12} 可知
\(\lim_{t\to0} \frac{\ln(1+t)}{t} = 1\),
因此\begin{equation}\label{equation:函数极限.重要极限16}
	\lim_{x\to0} \frac{(1+x)^a-1}x
	= a.
\end{equation}
\end{solution}
\end{example}
\begin{remark}
从\cref{equation:函数极限.重要极限16} 可以看出:
只要\(a\neq0\),就有\((1+x)^a-1 \sim ax\ (x\to0)\).
\end{remark}

\begin{example}
%@see: 《数学分析(第二版 上册)》(陈纪修) P106 例3.3.9
计算极限\(\lim_{x\to0} \frac{\ln(1+x^2)}{(e^{2x}-1) \tan x}\).
\begin{solution}
由于当\(x\to0\)时,
\(\tan x \sim x,
e^{2x}-1 \sim 2x,
\ln(1+x^2) \sim x^2\),
所以\begin{equation*}
	\lim_{x\to0} \frac{\ln(1+x^2)}{(e^{2x}-1) \tan x}
	= \lim_{x\to0} \frac{x^2}{2x \cdot x}
	= \frac12.
\end{equation*}
\end{solution}
\end{example}

\begin{proposition}[等价无穷小的换元法]\label{theorem:无穷小与无穷大.等价无穷小的换元法}
设\(\alpha,\beta,\gamma\)都是当\(x\to0\)时的无穷小,
其中\(\alpha \sim \beta\),
并且有\begin{equation*}
	(\exists\delta>0)
	(\forall x)
	[
		x\in\mathring{U}(0,\delta)
		\implies
		\gamma(x) \neq 0
	],
\end{equation*}
则\begin{equation*}
	\alpha\circ\gamma \sim \beta\circ\gamma.
\end{equation*}
\begin{proof}
记\(f(x) = \frac{\alpha(x)}{\beta(x)}\),
利用\hyperref[theorem:极限.复合函数的极限运算法则1]{复合函数的极限运算法则}立即可得\begin{equation*}
	\lim_{x\to0} f(\gamma(x))
	= \lim_{x\to0} f(x)
	= 1.
	\qedhere
\end{equation*}
\end{proof}
\end{proposition}

\begin{proposition}[和差代替规则]\label{theorem:极限.无穷小的比较4}
设\(\alpha\sim\alpha'\),\(\beta\sim\beta'\),\(\beta\)与\(\alpha\)不是等价无穷小,则\begin{equation*}
	\alpha\pm\beta\sim\alpha'\pm\beta'.
\end{equation*}
\end{proposition}

\begin{proposition}[因式代替规则]\label{theorem:极限.无穷小的比较5}
设\(\alpha\sim\beta\),且函数\(\phi\)有界或\(\lim\phi\)存在,则\begin{equation*}
	\alpha \phi \sim \beta \phi.
\end{equation*}
\end{proposition}

\begin{example}
%@see: https://www.bilibili.com/video/BV1E4iPetEzN/
计算极限\(\lim_{n\to\infty} \left(1+\frac1{n^2}\right)\left(1+\frac2{n^2}\right)\dotsm\left(1+\frac{n}{n^2}\vphantom{\frac1{n^2}}\right)\).
\begin{solution}
%@see: {47583c12-05b9-4815-9975-7962eeb50ede}
因为\begin{equation*}
	\ln\left(1+\frac{k}{n^2}\right)
	\sim \frac{k}{n^2}
	\quad(n\to\infty),
\end{equation*}
所以\begin{align*}
	&\hspace{-20pt}
	\left(1+\frac1{n^2}\right)\left(1+\frac2{n^2}\right)\dotsm\left(1+\frac{n}{n^2}\vphantom{\frac1{n^2}}\right) \\
	&= \exp\sum_{k=1}^n \ln\left(1+\frac{k}{n^2}\right)
	= \exp\left[ \sum_{k=1}^n \left(\frac{k}{n^2} + o\left(\frac1{n^2}\right)\right) \right] \\
	&= \exp\left( \frac{n+1}{2n} + o\left(\frac1n\right) \right)
	\to e^{\frac12}
	\quad(n\to\infty).
\end{align*}
\end{solution}
\end{example}

\section{收敛准则}
\subsection{单调有界函数收敛定理}
相应于\hyperref[theorem:极限.数列的单调有界定理]{单调有界数列收敛定理},函数极限也有类似的收敛准则.
\begin{theorem}\label{theorem:极限.函数的单调有界定理}
%@see: 《高等数学(第六版 上册)》 P55 准则II'
设函数\(f\)在点\(x_0\)的某个左邻域\((x_0-\delta,x_0)\)内单调并且有界,
则\(f\)在\(x_0\)的左极限\(f(x_0^-)\)必定存在.
\end{theorem}

\subsection{柯西极限存在准则}
我们首先研究函数极限存在的一个特殊情况.

\begin{theorem}\label{theorem:极限.函数的柯西极限存在准则}
%@see: 《数学分析(第二版 上册)》(陈纪修) P85 定理3.1.6
%@see: 《数学分析教程 (第3版 上册)》(史济怀) P73 定理2.4.7
设函数\(f\colon D\to\mathbb{R}\).
函数极限\(\lim_{x\to+\infty} f(x)\)存在且有限的充分必要条件是:\begin{equation*}
	(\forall\epsilon>0)
	(\exists X>0)
	(\forall x_1,x_2\in D)\\ \relax
	[
		x_1 > X \land x_2 > X
		\implies
		\abs{f(x_1) - f(x_2)} < \epsilon
	].
\end{equation*}
\begin{proof}
必要性.
假设\(\lim_{x\to+\infty} f(x) = A\),
则\begin{equation*}
	(\forall\epsilon>0)
	(\exists X>0)
	(\forall x_1,x_2\in D)
	\left[
		\begin{array}{l}
			x_1 > X \land x_2 > X \\
			\implies
			\abs{f(x_1) - A} < \frac\epsilon2
			\land
			\abs{f(x_2) - A} < \frac\epsilon2 \\
			\implies
			\abs{f(x_1) - f(x_2)} < \epsilon
		\end{array}
	\right].
\end{equation*}

充分性.
假设\begin{equation*}
	(\forall\epsilon>0)
	(\exists X>0)
	(\forall x_1,x_2\in D)
	[
		x_1 > X \land x_2 > X
		\implies
		\abs{f(x_1) - f(x_2)} < \epsilon
	].
\end{equation*}
任意选取数列\(\{x_n\}\),使得\(\lim_{n\to+\infty} x_n = +\infty\),
则对于上述\(X\)有\begin{equation*}
	(\exists N\in\mathbb{N})
	(\forall n\in\mathbb{N})
	[
		n > N
		\implies
		x_n > X
	].
\end{equation*}
于是根据假设可知,对于上述\(X\)有\begin{equation*}
	(\exists N\in\mathbb{N})
	(\forall m,n\in\mathbb{N})
	\left[
		n > N \land m > N
		\implies
		x_n > X \land x_m > X
		\implies
		\abs{f(x_n) - f(x_m)} < \epsilon
	\right],
\end{equation*}
这说明\(\{f(x_n)\}\)是基本数列,必定收敛.
根据海涅定理,可知\(\lim_{x\to+\infty} f(x)\)存在且有限.
\end{proof}
\end{theorem}

接下来我们研究更一般的滤子极限存在的充分必要条件.

\begin{definition}\label{definition:极限.函数在集合上的振幅}
%@see: 《数学分析(第7版 第一卷)》(卓里奇) P109 定义16
设\(f\in\mathbb{R}^X\),
集合\(E \subseteq X\).
把\begin{equation*}
	\sup_{x_1,x_2 \in E}\abs{f(x_1)-f(x_2)},
\end{equation*}
称为“函数\(f\)在集合\(E\)上的\DefineConcept{振幅}”,
记作\(\amp(f;E)\).
\end{definition}

\begin{theorem}
%@see: 《数学分析(第7版 第一卷)》(卓里奇) P109 定理4(函数极限存在的柯西准则)
设\(\mathcal{B}\)是\(X\)中的基.
函数\(f\in\mathbb{R}^X\)在基\(\mathcal{B}\)上的极限存在且有限 的充分必要条件是:\begin{equation*}
	(\forall\epsilon>0)
	(\exists B\in\mathcal{B})
	[\amp(f;B)<\epsilon].
\end{equation*}
\begin{proof}
必要性.
假设\(\lim_\mathcal{B} f(x) = A \in \mathbb{R}\),
则\begin{equation*}
	(\forall\epsilon>0)
	(\exists B\in\mathcal{B})
	(\forall x\in B)
	[
		\abs{f(x)-A}<\epsilon/3
	].
\end{equation*}
那么\begin{equation*}
	(\forall x_1,x_2\in\mathcal{B})
	\left[
		\abs{f(x_1)-f(x_2)}
		< \abs{f(x_1)-A}+\abs{f(x_2)-A}
		< 2\epsilon/3
	\right],
\end{equation*}
于是\(\amp(f;B)<\epsilon\).

充分性.
假设\begin{equation*}
	(\forall\epsilon>0)
	(\exists B\in\mathcal{B})
	[\amp(f;B)<\epsilon].
\end{equation*}
依次取\(\epsilon=1,\frac12,\dotsc,\frac1n,\dotsc\),
我们得到基\(\mathcal{B}\)的一系列元素\(B_1,B_2,\dotsc,B_n,\dotsc\),
它们满足\begin{equation*}
	\amp(f;B_n)
	=\sup_{b_{n1},b_{n2} \in B_n} \abs{f(b_{n1})-f(b_{n2})}
	<\epsilon=\frac1n.
	\eqno(1)
\end{equation*}
因为由\hyperref[definition:函数极限.滤子基的定义]{基的定义}有
\(B_n\neq\emptyset\),
所以在每个\(B_n\)中可以取一个点\(x_n\),
我们就得到了数列\(\{x_n\}\).
又因为由\hyperref[definition:函数极限.滤子基的定义]{基的定义}有
\(B_n \cap B_m \neq \emptyset\),
所以只要取辅助点\(\xi \in B_n \cap B_m\),
根据\hyperref[theorem:不等式.三角不等式1]{三角不等式},
就有\begin{equation*}
	\abs{f(x_n) - f(x_m)}
	\leq \abs{f(x_n) - f(\xi)} + \abs{f(\xi) - f(x_m)}
	< \frac1n + \frac1m,
	\eqno(2)
\end{equation*}
这就说明数列\(\{f(x_n)\}\)是基本数列.
根据\hyperref[theorem:极限.数列的柯西极限存在准则]{数列的柯西极限存在准则},
数列\(\{f(x_n)\}\)存在极限.
假设\(\lim_{n\to\infty} f(x_n) = A\).
那么根据\cref{theorem:极限.收敛数列的保序性2},
当\(m\to\infty\)时,
从(2)式推出\begin{equation*}
	\abs{f(x_n)-A}\leq\frac1n.
	\eqno(3)
\end{equation*}
于是对于\(\forall x\in B_n\)有\begin{align*}
	\abs{f(x)-A}
	&\leq \abs{f(x)-f(x_n)}+\abs{f(x_n)-A} \\
	&\leq \amp(f;B_n)+\abs{f(x_n)-A} \\
	&< \frac1n+\frac1n
	= \frac2n.
	\tag4
\end{align*}
要证\(\lim_\mathcal{B} f(x) = A\),
需证\begin{equation*}
	(\forall V(A))
	(\exists B\in\mathcal{B})
	[f(B) \subseteq V(A)],
\end{equation*}
或\begin{equation*}
	(\forall\delta>0)
	(\exists B\in\mathcal{B})
	(\forall x\in B)
	[\abs{f(x)-A}<\delta],
\end{equation*}
或\begin{equation*}
	(\forall\delta>0)
	(\exists N\in\mathbb{N})
	(\forall n\in\mathbb{N})
	[
		n>N
		\implies
		(\forall x\in B_n)
		[\abs{f(x)-A}<\delta]
	].
\end{equation*}
现在要使\(\abs{f(x)-A}<\delta\)成立,
由(4)式可知只需\(\frac2n<\delta\)或\(n>\frac2\delta\)成立,
因此,当\(n>N=\ceil*{\frac2\delta}\)时,
就有\((\forall x\in B_n)[\abs{f(x)-A}<\delta]\).
\end{proof}
\end{theorem}


\chapter{连续函数}
\section{连续函数}\label{section:连续函数.函数的连续性与间断点}
\subsection{连续点}
\begin{definition}\label{definition:极限.函数在一点的连续性}
%@see: 《数学分析(第二版 上册)》(陈纪修) P88 定义3.2.1
%@see: 《高等数学(第六版 上册)》 P61 定义
设函数\(f\)在点\(x_0\)的某一邻域内有定义.
如果\begin{equation*}
	\lim_{x \to x_0} f(x) = f(x_0),
\end{equation*}
那么就称“函数\(f\)在点\(x_0\)~\DefineConcept{连续}
(\(f\) is continuous at \(x_0\))”,
称“点\(x_0\)是函数\(f\)的\DefineConcept{连续点}(point of continuity)”.
\end{definition}

上述对函数连续的定义可以简化为:\begin{equation*}
	\text{\(f\)在点\(x_0\)连续}
	\defiff
	(\forall\epsilon>0)
	(\exists\delta>0)
	(\forall x)
	[
		\abs{x - x_0} < \delta
		\implies
		\abs{f(x) - f(x_0)} < \epsilon
	].
\end{equation*}

\begin{proposition}%函数在一点连续的等价定义
设函数\(f\)在点\(x_0\)的某一邻域内有定义.
如果\begin{equation*}
	\lim_{h\to0} (f(x_0+h) - f(x_0)) = 0,
\end{equation*}
则函数\(f\)在点\(x_0\)连续.
%TODO proof
\end{proposition}
\begin{proposition}\label{theorem:连续函数.函数连续点与海涅定理的关系}
%@see: 《数学分析习题课讲义(第2版 上册)》(谢惠民、恽自求、易法槐、钱定边) P49
函数\(f\)在点\(x_0\)连续的充分必要条件是:
对于每个收敛于\(x_0\)的数列\(\{x_n\}_{n\geq1}\),
% 与海涅定理有一点差别,这里不再要求数列\(\{x_n\}_{n\geq1}\)在\(x_0\)的去心邻域内
成立\begin{equation*}
	\lim_{n\to\infty} f(x_n) = f(x_0).
\end{equation*}
\begin{proof}
由\hyperref[theorem:极限.海涅定理]{海涅定理}立即可得.
\end{proof}
\end{proposition}

\begin{definition}\label{definition:极限.函数在一点的左连续性}
%@see: 《数学分析(第二版 上册)》(陈纪修) P89 定义3.2.3
如果函数\(f\)在点\(x_0\)的某一左邻域内有定义,
极限\(f(x_0^-) \defeq \lim_{x \to x_0^-} f(x)\)存在,
且\begin{equation*}
	f(x_0^-) = f(x_0),
\end{equation*}
则称“函数\(f\)在点\(x_0\)~\DefineConcept{左连续}
(\(f\) is left-continuous at \(x_0\))”.
\end{definition}
上述对函数左连续的定义可以简化为:\begin{equation*}
%@see: 《数学分析(第二版 上册)》(陈纪修) P89
	\text{\(f\)在点\(x_0\)左连续}
	\defiff
	(\forall\epsilon>0)
	(\exists\delta>0)
	(\forall x)
	[
		-\delta < x - x_0 \leq 0
		\implies
		\abs{f(x) - f(x_0)} < \epsilon
	].
\end{equation*}

\begin{definition}\label{definition:极限.函数在一点的右连续性}
%@see: 《数学分析(第二版 上册)》(陈纪修) P89 定义3.2.3
如果函数\(f\)在点\(x_0\)的某一右邻域内有定义,
极限\(f(x_0^+) \defeq \lim_{x \to x_0^+} f(x)\)存在,
且\begin{equation*}
	f(x_0^+) = f(x_0),
\end{equation*}
则称“函数\(f\)在点\(x_0\)~\DefineConcept{右连续}
(\(f\) is right-continuous at \(x_0\))”.
\end{definition}
上述对函数右连续的定义可以简化为:\begin{equation*}
%@see: 《数学分析(第二版 上册)》(陈纪修) P89
	\text{\(f\)在点\(x_0\)右连续}
	\defiff
	(\forall\epsilon>0)
	(\exists\delta>0)
	(\forall x)
	[
		0 \leq x - x_0 < \delta
		\implies
		\abs{f(x) - f(x_0)} < \epsilon
	].
\end{equation*}

我们把左连续和右连续这两个概念统称为\DefineConcept{单侧连续}.

\begin{proposition}\label{theorem:极限.函数在一点的连续性及其单侧连续性的关系}
%@see: 《数学分析(第二版 上册)》(陈纪修) P92
函数\(f\)在点\(x_0\)连续的充分必要条件是\begin{equation*}
	f(x_0^-) = f(x_0^+) = f(x_0).
\end{equation*}
\end{proposition}

\subsection{连续区间}
\begin{definition}
%@see: 《数学分析(第二版 上册)》(陈纪修) P89 定义3.2.2
如果函数\(f\)满足\begin{equation*}
	(\forall x_0\in(a,b))
	[\text{\(f\)在点\(x_0\)连续}],
\end{equation*}
那么称“函数\(f\)在开区间\((a,b)\)内连续”.
\end{definition}

\begin{example}
设函数\(f\colon(a,b)\to\mathbb{R}\)在开区间\((a,b)\)内连续,
举例说明:\(f(a^+)\)和\(f(b^-)\)均不存在.
\begin{solution}
%@credit: {61d1026b-642e-438a-9506-08e3e7865f96}
取\(f(x) = \tan x\),
它在\(\left( -\frac\pi2,\frac\pi2 \right)\)内连续,
但\begin{equation*}
	\lim_{x\to-\frac\pi2^-} f(x) = -\infty,
	\qquad
	\lim_{x\to\frac\pi2^+} f(x) = +\infty.
\end{equation*}
\end{solution}
\end{example}
\begin{example}
设函数\(f\)在点\(x_0\)的某个去心邻域内连续,
举例说明:\(f\)在点\(x_0\)的极限不存在.
\begin{solution}
取\(f(x) = \frac1x\),
它在点\(x=0\)的任意去心邻域内连续,
但是\begin{equation*}
	\lim_{x\to0} f(x) = \infty.
\end{equation*}
也可取\(g(x) = \sgn x\),
它也在点\(x=0\)的任意去心邻域内连续,
但是\begin{equation*}
	\lim_{x\to0^-} f(x) = -1,
	\neq
	\lim_{x\to0^+} f(x) = 1,
\end{equation*}
\(g\)在点\(x=0\)的极限不存在.
\end{solution}
\end{example}

\begin{example}\label{example:连续函数.狄利克雷函数处处不连续}
%@see: 《数学分析教程(第3版 上册)》(史济怀) P91 例4(1)
证明:狄利克雷函数\begin{equation*}
	D(x) = \left\{ \begin{array}{ll}
		1, & x \in \mathbb{Q}, \\
		0, & x \in \mathbb{R}-\mathbb{Q}
	\end{array} \right.
\end{equation*}在\((-\infty,\infty)\)上的每一个点都不连续.
\begin{proof}
由\cref{example:海涅定理.狄利克雷函数在任意一点的极限都不存在}
可知狄利克雷函数在任意一点的极限都不存在,
那么它当然在任意一点不连续.
\end{proof}
\end{example}
\begin{example}\label{example:连续函数.狄利克雷函数改1只在一点连续}
%@see: 《数学分析教程(第3版 上册)》(史济怀) P91 例4(2)
证明:函数\(f(x) = x~D(x)\)除了在点\(x=0\)连续之外,在其他各点均不连续.
\begin{proof}
当\(x\to0\)时,\(D(x)\)是一个有界量,
于是由\cref{theorem:函数极限.无穷小.有界函数与无穷小的乘积是无穷小} 可知
\(\lim_{x\to0} f(x) = f(0) = 0\).

取\(x_0\neq0\),有\begin{equation*}
	\lim_{\substack{x \to x_0 \\ x \in \mathbb{Q}}} f(x)
	= x_0
	\neq
	\lim_{\substack{x \to x_0 \\ x \in \mathbb{R}-\mathbb{Q}}} f(x)
	= 0,
\end{equation*}
所以由\hyperref[theorem:极限.海涅定理]{海涅定理}可知\(\lim_{x \to x_0} f(x)\)不存在.
\end{proof}
\end{example}
\begin{remark}
%@see: https://www.bilibili.com/video/BV1Rr421M7T8/
\cref{example:连续函数.狄利克雷函数处处不连续} 说明:
狄利克雷函数处处不连续.
\cref{example:连续函数.狄利克雷函数改1只在一点连续} 说明:
函数在某一点连续是函数在该点的邻域内连续的必要不充分条件.
\end{remark}

\begin{definition}
%@see: 《数学分析(第二版 上册)》(陈纪修) P90 定义3.2.4
如果函数\(f\)不仅在开区间\((a,b)\)内连续,
还在点\(a\)处右连续,且在\(b\)处左连续,
那么称“函数\(f\)在闭区间\([a,b]\)上连续”.
\end{definition}

\begin{remark}
%@see: 《数学分析(第二版 上册)》(陈纪修) P90 注
上述定义可以统一地表示为如下形式:
设函数\(f\)在某区间\(X\)上有定义.
如果\begin{equation*}
	(\forall x_0\in X)
	(\forall\epsilon>0)
	(\exists\delta>0)
	(\forall x\in X)
	[
		\abs{x-x_0}<\delta
		\implies
		\abs{f(x)-f(x_0)}<\epsilon
	],
\end{equation*}
则称“函数\(f\)在区间\(X\)上连续”.
\end{remark}

\begin{example}
根式函数\(\sqrt{x}\)在\([0,+\infty)\)上连续.
%\cref{example:极限.根式函数在某一点的极限}
\end{example}

\begin{example}
有理整函数\begin{equation*}
	P_n(x) = a_0 x^n + a_1 x^{n-1} + \dotsb + a_n
\end{equation*}在\((-\infty,+\infty)\)上连续.
%\cref{equation:函数极限.重要极限3}
\end{example}

\begin{example}
有理分式函数\begin{equation*}
	F(x) = \frac{P_n(x)}{P_m(x)}
\end{equation*}在其定义域\(\Set{ x\in\mathbb{R} \given P_m(x)\neq0 }\)上连续.
\end{example}

\begin{example}\label{example:极限.正弦函数在实数域上连续}
%@see: 《数学分析(第二版 上册)》(陈纪修) P90 例3.2.3
证明:函数\(f(x) = \sin x\)在\((-\infty,+\infty)\)上连续.
\begin{proof}
任取\(x_0\in(-\infty,+\infty)\).
由\hyperref[equation:函数.三角函数.和积互化公式12]{和积互化公式}有\begin{equation*}
	\abs{\sin x - \sin x_0}
	= 2 \abs{\cos\frac{x+x_0}2 \sin\frac{x-x_0}2}
	= 2 \abs{\cos\frac{x+x_0}2} \abs{\sin\frac{x-x_0}2}.
\end{equation*}
因为\((\forall\alpha\in\mathbb{R})[\abs{\cos\alpha}\leq1]\),
所以\begin{equation*}
	\abs{\sin x - \sin x_0} \leq 2 \abs{\sin\frac{x-x_0}2}.
\end{equation*}
又因为当\(\alpha=0\)时有\(0=\sin\alpha=\alpha\),
而当\(\alpha\neq0\)时有\(0\leq\abs{\sin\alpha}<\abs{\alpha}\),
所以\((\forall\alpha\in\mathbb{R})[\abs{\sin\alpha}\leq\abs{\alpha}]\),
于是\begin{equation*}
	\abs{\sin x - \sin x_0}
	\leq 2 \abs{\frac{x-x_0}2}
	= \abs{x-x_0}.
\end{equation*}
对于\(\forall\epsilon>0\),
取\(\delta=\epsilon\),
当\(\abs{x-x_0}<\delta\)时,
就有\(\abs{\sin x-\sin x_0}<\epsilon\),
所以\(\sin x\)在\((-\infty,+\infty)\)上连续.
\end{proof}
\end{example}

类似地可以证明,函数\(f(x) = \cos x\)在区间\((-\infty,+\infty)\)内是连续的.

% \begin{example}
% %@see: 《数学分析(第二版 上册)》(陈纪修) P91 例3.2.4
% 证明:函数\(f(x) = a^x\ (a>0,a\neq1)\)在\((-\infty,+\infty)\)上连续.
% \begin{proof}
% 首先有\begin{equation*}
% 	(\forall x_0\in\mathbb{R})
% 	[a^x-a^{x_0} = a^{x_0}(a^{x-x_0}-1)].
% \end{equation*}
% 因此,证\(\lim_{x\to x_0} a^x = a^{x_0}\)就归结为证\(\lim_{t\to0} a^t = 1\).
% \end{proof}
% \end{example}

\subsection{连续函数族}
\begin{definition}\label{definition:函数族.连续函数族}
由区间\(I\)上全部的连续函数组成的集合,称作\DefineConcept{连续函数族},
记作\(C(I)\),
即\begin{equation*}
	C(I)
	\defeq
	\Set*{
		f\in\mathbb{R}^I
		\given
		(\forall x \in I)
		[\text{\(f\)在点\(x\)连续}]
	}.
\end{equation*}
%@see: https://mathworld.wolfram.com/ContinuousFunction.html
\end{definition}

\subsection{间断点}
\begin{definition}
%@see: 《数学分析(第二版 上册)》(陈纪修) P92
设函数\(f\)在点\(x_0\)的某去心邻域内有定义.
如果函数\(f\)有下列三种情形之一:
\begin{itemize}
	\item 在\(x=x_0\)没有定义;
	\item 虽在\(x=x_0\)有定义,
	但\(\lim_{x \to x_0} f(x)\)不存在;
	\item 虽在\(x=x_0\)有定义,
	且\(\lim_{x \to x_0} f(x)\)存在,
	但\(\lim_{x \to x_0} f(x) \neq f(x_0)\),
\end{itemize}
则称“函数\(f\)在点\(x_0\)不连续”
“点\(x_0\)是函数\(f\)的\DefineConcept{不连续点}”
或“点\(x_0\)是函数\(f\)的\DefineConcept{间断点}(discontinuity)”.
\end{definition}

%@see: 《数学分析教程(第3版 上册)》(史济怀) P94 定义2.7.4
如果\(\lim_{x \to x_0} f(x) = \infty\),
则称点\(x_0\)为“函数\(f\)的\DefineConcept{无穷间断点}(infinite discontinuity)”.
例如,点\(x=0\)是函数\(y=\frac{1}{x}\)的无穷间断点.
%@see: https://mathworld.wolfram.com/InfiniteDiscontinuity.html

如果\(f\)在点\(x_0\)的某一邻域是有界的,
但其左、右极限均不存在,
则称点\(x_0\)为“函数\(f\)的\DefineConcept{振荡间断点}(oscillating discontinuity)”.
% 在 wolfram 网站上没有找到与“振荡间断点”对应的概念,这里的英文单词是我翻译的
例如,点\(x=0\)是函数\(y=\sin\frac{1}{x}\)的振荡间断点.

如果\(\lim_{x \to x_0} f(x) = A < \infty\),
但是\(f\)在点\(x_0\)没有定义,或者\(f(x_0) \neq A\),
% 有的教科书在定义可去间断点时,暗示\(f\)在点\(x_0\)有定义,例如:
% 《数学分析教程(第3版 上册)》(史济怀)
则称点\(x_0\)为“函数\(f\)的\DefineConcept{可去间断点}(removable discontinuity)”.
例如,点\(x=0\)是函数\(y=\frac{\sin x}{x}\)的可去间断点.
%@see: https://mathworld.wolfram.com/RemovableDiscontinuity.html

如果\(f\)在点\(x_0\)的左、右极限均存在且有限但不相等,
即\(\lim_{x \to x_0^-} f(x) \neq \lim_{x \to x_0^+} f(x)\),
则称点\(x_0\)为“函数\(f\)的\DefineConcept{跳跃间断点}(jump discontinuity)”,
把\(\abs{f(x_0^+)-f(x_0^-)}\)称为
“函数\(f\)在点\(x_0\)的\DefineConcept{跳跃}”.
例如,点\(x=0\)是函数\(y=\sgn x\)的跳跃间断点.
%@see: https://mathworld.wolfram.com/JumpDiscontinuity.html

如果\(x_0\)是函数\(f\)的间断点,
但左极限\(f(x_0^-)\)及右极限\(f(x_0^+)\)都存在,
那么\(x_0\)称为“函数\(f\)的\DefineConcept{第一类间断点}(discontinuity of the first kind)”.
不是第一类间断点的间断点,称为\DefineConcept{第二类间断点}(discontinuity of the second kind).
% 有的教科书在定义第二类间断点时,说“如果\(f(x_0^+)\)与\(f(x_0^-)\)二者中至少有一个不存在或者不是有限的数,那么\(x_0\)叫作\(f\)的第二类间断点”,例如:
% 《数学分析教程(第3版 上册)》(史济怀)

显然,可去间断点、跳跃间断点是第一类间断点,
无穷间断点、振荡间断点是第二类间断点.

\begin{example}
%@see: 《数学分析(第二版 上册)》(陈纪修) P93 例3.2.7
\DefineConcept{黎曼函数}\begin{equation*}
	R(x) = \left\{ \begin{array}{cl}
		\frac1p, & \text{$x=\frac{q}{p}$是既约分数}, \\
		1, & x=0, \\
		0, & \text{$x$是无理数}
	\end{array} \right.
\end{equation*}是以\(1\)为周期的周期函数,
它在任意一点\(x_0\)的极限存在且等于\(0\),
一切无理点是\(R\)的连续点,
一切有理点是\(R\)的可去间断点.
%TODO proof
\end{example}

\begin{theorem}
%@see: 《数学分析教程(第3版 上册)》(史济怀) P95 定理2.7.5
%@see: 《数学分析(第二版 上册)》(陈纪修) P94 例3.2.8
设\(f\)是开区间\((a,b)\)上的单调函数,
则\(f\)的间断点一定是跳跃间断点,它的跳跃间断点集至多是可数的.
%TODO proof
\end{theorem}

\begin{example}
已知函数\(f(x) = \frac{x-x^3}{\sin \pi x}\),求该函数的可去间断点的个数.
\begin{solution}
当\(\sin \pi x = 0\)或\(x \in \mathbb{Z}\)时,函数\(f\)无定义;
也就是说,点\(x\in\mathbb{Z}\)都是\(f\)的间断点.
要使点\(x\)成为函数\(f\)的可去间断点,必有\(x-x^3=0\),解得\(x\in\{-1,0,1\}\).
又因为\begin{gather*}
	\lim_{x\to0} \frac{x-x^3}{\sin \pi x}
	= \lim_{x\to0} \frac{x(1-x^2)}{\pi x}
	= \frac1\pi, \\
	\lim_{x\to1} \frac{x-x^3}{\sin \pi x}
	= \lim_{x\to1} \frac{1-3x^2}{\pi \cos \pi x}
	= \frac2\pi, \\
	\lim_{x\to-1} \frac{x-x^3}{\sin \pi x}
	= \lim_{x\to-1} \frac{1-3x^2}{\pi \cos \pi x}
	= \frac2\pi,
\end{gather*}
综上,函数\(f\)共有3个可去间断点.
\end{solution}
\end{example}

\begin{example}
设函数\(f(x) = \frac{x^2-x}{x^2-1}\sqrt{1+\frac{1}{x^2}}\).
试计算\(f\)的间断点种类及其个数.
\begin{solution}
因为\begin{equation*}
	f(x) = \frac{x(x-1)}{(x-1)(x+1)} \frac{\sqrt{x^2+1}}{\abs{x}},
\end{equation*}
所以\(f\)的间断点为\(x\in\{-1,0,1\}\).

又因为\begin{align*}
	&\lim_{x\to1} f(x)
	= \lim_{x\to1} \frac{\sqrt{x^2+1}}{x+1}
	= \frac{\sqrt{2}}{2}, \\
	&\lim_{x\to0^+} f(x)
	= \lim_{x\to0^+} \frac{\sqrt{x^2+1}}{x+1}
	= 1, \\
	&\lim_{x\to0^-} f(x)
	= \lim_{x\to0^-} -\frac{\sqrt{x^2+1}}{x+1}
	= -1, \\
	&\lim_{x\to-1} f(x)
	= \lim_{x\to-1} \frac{\sqrt{x^2+1}}{x+1}
	= \infty,
\end{align*}
所以点\(x=1\)是可去间断点,
点\(x=0\)是跳跃间断点,
点\(x=-1\)是无穷间断点.
那么可去间断点、跳跃间断点和无穷间断点的个数均为1个.
\end{solution}
\end{example}

% \subsection{半连续性}
% \begin{definition}
% 设\(f\colon D\to\mathbb{R}\).
% 规定:\begin{equation*}
% 	\begin{split}
% 		\text{\(f\)在点\(x_0\) \DefineConcept{上半连续}}
% 		\defiff
% 			&\text{\(f\)在\(D\)内有上界} \\
% 			&\land
% 			(\forall\epsilon>0)
% 			(\exists\delta>0)
% 			(\forall x \in D)
% 			[
% 				\abs{x-x_0}<\delta
% 				\implies
% 				f(x)<f(x_0)+\epsilon
% 			]; \\
% 		\text{\(f\)在点\(x_0\) \DefineConcept{下半连续}}
% 		\defiff
% 			&\text{\(f\)在\(D\)内有下界} \\
% 			&\land
% 			(\forall\epsilon>0)
% 			(\exists\delta>0)
% 			(\forall x \in D)
% 			[
% 				\abs{x-x_0}<\delta
% 				\implies
% 				f(x)>f(x_0)-\epsilon
% 			].
% 	\end{split}
% \end{equation*}
% %@see: https://healy.econ.ohio-state.edu/kcb/Ec181/Lecture13.pdf
% \end{definition}

\section{连续函数的运算}
\subsection{连续函数的四则运算}
\begin{theorem}\label{theorem:极限.连续函数的极限1}
%@see: 《高等数学(第六版 上册)》 P66 定理1
%@see: 《数学分析(第二版 上册)》(陈纪修) P91
设函数\(f\)和\(g\)都在点\(x_0\)连续,
则\begin{itemize}
	\item \(f\)和\(g\)的和(差)\(f \pm g\)在点\(x_0\)连续,即\begin{equation*}
		\lim_{x \to x_0} (f(x) + g(x))
		= f(x_0) + g(x_0);
	\end{equation*}
	\item \(f\)和\(g\)的积\(f \cdot g\)在点\(x_0\)连续,即\begin{equation*}
		\lim_{x \to x_0} (f(x) \cdot g(x))
		= f(x_0) \cdot g(x_0);
	\end{equation*}
	\item 如果\(g(x_0)\neq0\),则\(f\)和\(g\)的商\(\frac{f}{g}\)在点\(x_0\)连续,即\begin{equation*}
		\lim_{x \to x_0} \frac{f(x)}{g(x)}
		= \frac{f(x_0)}{g(x_0)}.
	\end{equation*}
\end{itemize}
\end{theorem}

\begin{example}
%@see: 《高等数学(第六版 上册)》 P66 例1
因为\begin{equation*}
	\tan x=\frac{\sin x}{\cos x}, \qquad
	\cot x=\frac{\cos x}{\sin x},
\end{equation*}
而由\cref{example:极限.正弦函数在实数域上连续} 可知,
\(\sin x\)和\(\cos x\)都在区间\((-\infty,+\infty)\)内连续,
故由\cref{theorem:极限.连续函数的极限1} 可知,
\(\tan x\)和\(\cot x\)在它们的定义域内是连续的.
\end{example}

\begin{example}\label{example:连续函数的运算.开区间上的单调函数的间断点必为第一类间断点}
%@see: 《数学分析(第二版 上册)》(陈纪修) P94 例3.2.8
区间\((a,b)\)上单调函数的间断点必为第一类间断点.
\begin{proof}
不妨设\(f\)在\((a,b)\)单调增加.
任取\(x_0\in(a,b)\).
显然集合\(A=\Set{ f(x) \given x\in(a,x_0) }\)有上界.
由\hyperref[theorem:实数.确界原理]{确界存在定理}可知,
\(A\)必定存在上确界\(\alpha=\sup A\).
于是\begin{equation*}
	(\forall x\in(a,x_0))
	[f(x)\leq\alpha]
	\land
	(\forall\epsilon>0)
	(\exists x'\in(a,x_0))
	[f(x')>\alpha-\epsilon].
\end{equation*}
取\(\delta=x_0-x'>0\),
则当\(-\delta<x-x_0<0\)时,
有\(x'<x<x_0\),
于是又有\begin{equation*}
	-\epsilon
	< f(x') - \alpha
	\leq f(x) - \alpha
	\leq 0,
\end{equation*}
这就说明\(\lim_{x \to x_0^-} f(x) = \alpha\).
同理可证\(\lim_{x \to x_0^+} f(x) = \beta\),
其中\(\beta=\inf\Set{ f(x) \given x\in(x_0,b) }\).
\end{proof}
\end{example}

\subsection{反函数的连续性}
\begin{theorem}\label{theorem:极限.连续函数的极限2}
%@see: 《高等数学(第六版 上册)》 P66 定理2
%@see: 《数学分析(第二版 上册)》(陈纪修) P95 定理3.2.2(反函数连续性定理)
如果函数\(f\colon[a,b]\to\mathbb{R}\)在区间\([a,b]\)上连续且严格单调增加,
那么它的反函数\(f^{-1}\)在\([\alpha,\beta]\)上连续且严格单调增加,
其中\(\alpha=f(a),\beta=f(b)\).
\begin{proof}
首先我们利用\cref{example:连续函数的运算.开区间上的单调函数的间断点必为第一类间断点} 的结论来证明\begin{equation*}
	\ran f
	= \dom f^{-1}
	= f([a,b])
	= [\alpha,\beta].
\end{equation*}

显然\(\alpha,\beta \in f([a,b])\).
任取\(\gamma\in(\alpha,\beta)\).
记\begin{equation*}
	S \defeq \Set{ x\in[a,b] \given f(x)<\gamma }.
\end{equation*}
那么集合\(S\)非空且有上界.
由\hyperref[theorem:实数.确界原理]{确界存在定理}可知,
\(S\)必有上确界.
记\(x_0 = \sup S\),
则\(x_0\in(a,b)\).
由于\(f\)严格单调增加,
当\(x<x_0\)时有\(f(x)<\gamma\),
当\(x>x_0\)时有\(f(x)>\gamma\).
于是由\cref{example:连续函数的运算.开区间上的单调函数的间断点必为第一类间断点} 得到\begin{equation*}
	f(x_0^-) \leq \gamma \leq f(x_0^+).
\end{equation*}
由于\(f\)在点\(x_0\)连续,
所以\(f(x_0)
= f(x_0^+)
= f(x_0^-)
= \gamma\).
这就说明\(\ran f = [\alpha,\beta]\).

由于\(f\)严格单调增加,
所以它的反函数\(f^{-1}\)必定存在且同样严格单调增加.

现在只需要证明\(f^{-1}\)在\([\alpha,\beta]\)上连续.

设\(y_0\in(\alpha,\beta)\),
相应地有\(f^{-1}(y_0)=x_0\in(a,b)\).
对于任意给定的\(\epsilon>0\),
要找出\(\delta>0\),
使得当\(\abs{y-y_0}<\delta\)时,
成立\begin{equation*}
	\abs{f^{-1}(y) - f^{-1}(y_0)}
	= \abs{f^{-1}(y) - x_0}
	< \epsilon,
\end{equation*}
即\begin{equation*}
	x_0 - \epsilon < f^{-1}(y) < x_0 + \epsilon.
\end{equation*}
令\(y_1 = f(x_0 - \epsilon),
y_2 = f(x_0 + \epsilon)\).
取\(\delta = \min\{y_0 - y_1, y_2 - y_0\} > 0\).
显然当\(\abs{y - y_0} < \delta\)时,
成立\begin{equation*}
	\abs{f^{-1}(y) - f^{-1}(y_0)} < \epsilon.
\end{equation*}
如果\(y_0 = \alpha\),
则只要证明右连续性.
如果\(y_0 = \beta\),
则只要证明左连续性.
\end{proof}
\end{theorem}

\begin{example}
%@see: 《高等数学(第六版 上册)》 P66 例2
由于\(\sin x\)在闭区间\(\left[-\frac{\pi}{2},\frac{\pi}{2}\right]\)上单调增加且连续,
所以它的反函数\(\arcsin x\)在闭区间\([-1,1]\)上也是单调增加且连续的.
同理可证:
\begin{itemize}
	\item \(\arccos x\)在闭区间\([-1,1]\)上单调减少且连续.
	\item \(\arctan x\)在区间\((-\infty,+\infty)\)内单调增加且连续.
	\item \(\arccot x\)在区间\((-\infty,+\infty)\)内单调减少且连续.
\end{itemize}
总之,反三角函数\(\arcsin x,\arccos x,\arctan x,\arccot x\)在它们的定义域内都是连续的.
\end{example}

\subsection{复合函数的连续性}
\begin{theorem}\label{theorem:极限.连续函数的极限3}
%@see: 《高等数学(第六版 上册)》 P66 定理3
%@see: 《数学分析(第二版 上册)》(陈纪修) P96
设\(f\colon X \to Y\),\(g\colon Y\to\mathbb{R}\),
\((\exists\rho>0)[\mathring{U}(x_0,\rho) \subseteq \dom(g \circ f)]\),
\(\lim_{x \to x_0} f(x) = u_0\),
函数\(g\)在点\(u_0\)连续,
则\begin{equation*}
	\lim_{x \to x_0} g(f(x))
	= \lim_{u \to u_0} g(u)
	= g(u_0).
\end{equation*}
\begin{proof}
在\cref{theorem:极限.复合函数的极限运算法则1} 中,
令\(A = g(u_0)\),
并取消前提条件\footnote{前提条件可以取消的理由是:\begin{equation*}
	(\forall\epsilon>0)
	(\forall x)
	[
		f(x) = u_0
		\implies
		\abs{g(f(x)) - g(u_0)} < \epsilon
	].
\end{equation*}
因此附加\(f(x) \neq u_0\)这个条件就没有必要了.}\begin{equation*}
	(\exists\delta>0)(\forall x)[x\in\mathring{U}(x_0,\delta) \implies f(x)\neq u_0],
\end{equation*}
便得此定理.
\end{proof}
\end{theorem}
\begin{remark}
\cref{theorem:极限.连续函数的极限3} 说明,
在其给定条件下,求复合函数\(g \circ f\)的极限时,
外层函数符号\(g\)与极限号\(\lim\)可以交换次序,即\begin{equation*}
	\lim_{x \to x_0} g(f(x))
	= g\left(\lim_{x \to x_0} f(x)\right).
\end{equation*}
\end{remark}

\begin{theorem}\label{theorem:极限.连续函数的极限4}
%@see: 《高等数学(第六版 上册)》 P66 定理4
%@see: 《数学分析(第二版 上册)》(陈纪修) P96 定理3.2.3
设\(f\colon X \to Y\),\(g\colon Y\to\mathbb{R}\),
\((\exists\rho>0)[U(x_0,\rho) \subseteq \dom(g \circ f)]\),
函数\(f\)在点\(x_0\)连续,
函数\(g\)在点\(u_0 = f(x_0)\)连续,
则函数\(g \circ f\)在点\(x_0\)也连续.
%TODO proof
\end{theorem}

\subsection{初等函数的连续性}
\begin{theorem}
%@see: 《高等数学(第六版 上册)》 P68
基本初等函数在其定义域内都是连续的.
\end{theorem}

\begin{corollary}
%@see: 《高等数学(第六版 上册)》 P68
%@see: 《数学分析教程(第3版 上册)》(史济怀) P94 定理2.7.4
一切初等函数在其定义区间(即包含在定义域内的区间)内都是连续的.
\end{corollary}

\begin{example}
%@see: 《高等数学(第六版 上册)》 P69 例6
求:\(\lim_{x\to0} \frac{\log_a (1+x)}{x}\ (a>0,a\neq1)\).
\begin{solution}
\(\lim_{x\to0} \frac{\log_a (1+x)}{x}
= \lim_{x\to0} \log_a (1+x)^{\frac{1}{x}}
= \log_a \lim_{x\to0} (1+x)^{\frac{1}{x}}
= \log_a e
= \frac{1}{\ln a}\).
\end{solution}
\end{example}

\begin{example}
%@see: 《高等数学(第六版 上册)》 P69 例7
求:\(\lim_{x\to0} \frac{a^x - 1}{x}\ (a>0,a\neq1)\).
\begin{solution}
\(\lim_{x\to0} \frac{a^x - 1}{x}
\xlongequal{t=a^x-1} \lim_{t\to0}\frac{t}{\log_a (1+t)}
= \ln a\).
\end{solution}
\end{example}

\begin{example}
计算极限\(\lim_{x\to0^+} \left(1+\frac1x\right)^x\).
\begin{solution}
直接计算得\begin{equation*}
	\lim_{x\to0^+} \left(1+\frac1x\right)^x
	= \exp\lim_{x\to0^+} x\ln(1+\frac1x) % 连续函数的复合函数
	\xlongequal{t=1/x} \exp\lim_{t\to+\infty} \frac{\ln(1+t)}{t} % 换元
	= \exp\lim_{t\to+\infty} \frac1{1+t} % 洛必达法则
	= 1.
\end{equation*}
\end{solution}
\end{example}

\begin{example}
%@see: https://www.bilibili.com/video/BV1MF4m1L7Jw
计算极限\(\lim_{x\to+\infty} \left[ \frac1e \left(1+\frac1x\right)^x \right]^x\).
\begin{solution}
直接计算得\begin{align*}
	&\lim_{x\to+\infty} \left[ \frac1e \left(1+\frac1x\right)^x \right]^x
	= \lim_{x\to+\infty} \exp[x^2 \ln(1+\frac1x) - x] \\
	&\xlongequal{t=1/x} \exp\lim_{t\to0^+} \left[ \frac1{t^2} \ln(1+t) - \frac1t \right]
	= \exp\lim_{t\to0^+} \frac{\ln(1+t)-t}{t^2} \\
	&= \exp\lim_{t\to0^+} \frac{t-\frac12t^2+o(t^2)-t}{t^2}
	= \exp(-\frac12)
	= \frac1{\sqrt{e}}.
\end{align*}
\end{solution}
%@Mathematica: Limit[((1 + 1/x)^x/E)^x, x -> Infinity]
\end{example}

\begin{example}
设\(x>0\).
证明:\(\lim_{n\to\infty} n \left( \sqrt[n]{x} - 1 \right) = \ln x\).
%@see: \cref{equation:特殊函数.以e为底的对数函数}
\begin{proof}
%@see: https://math.stackexchange.com/a/2021180/591741
直接计算得\begin{align*}
	\lim_{n\to\infty} n \left( \sqrt[n]{x} - 1 \right)
	&\xlongequal{x = e^t} t \lim_{n\to\infty} \frac{e^{t/n} - 1}{t/n} \\
	&\xlongequal{u=t/n} t \lim_{u\to0} \frac{e^u - 1}u \\
	&= t
	= \ln x.
	\qedhere
\end{align*}
\end{proof}
\end{example}

\begin{proposition}
%@see: 《高等数学(第六版 上册)》 P69 习题1-9 2.
设函数\(f\)和\(g\)在点\(x_0\)连续,
则函数\begin{equation*}
	\phi(x) = \max\{f(x),g(x)\}, \qquad
	\psi(x) = \min\{f(x),g(x)\}
\end{equation*}在点\(x_0\)也连续.
\begin{proof}
因为\begin{equation*}
	\phi(x)
	= \frac{f(x)+g(x)}{2}+\frac{\abs{f(x)-g(x)}}{2}, \qquad
	\psi(x)
	= \frac{f(x)+g(x)}{2}-\frac{\abs{f(x)-g(x)}}{2},
\end{equation*}
所以\(\phi(x)\)和\(\psi(x)\)在点\(x_0\)也连续.
\end{proof}
\end{proposition}

\begin{proposition}[幂指函数的极限]\label{theorem:幂指函数.幂指函数的极限}
%@see: 《高等数学(第六版 上册)》 P69
%@see: 《数学分析教程(第3版 上册)》(史济怀) P100
设幂指函数\(f\colon X\to\mathbb{R}\)
满足\(f(x) = u(x)^{v(x)}\),
其中\(u,v\)都是连续函数,
且\begin{equation*}
	(\exists B\in\mathcal{B})
	(\forall x \in B)
	[u(x)>0],
\end{equation*}
而\(\mathcal{B}\)是\(X\)中的基.
\begin{itemize}
	\item 如果\(u(x) \equiv 1\),
	那么\begin{equation*}
		\lim_\mathcal{B} u(x)^{v(x)} = 1.
	\end{equation*}

	\item 如果\begin{equation*}
		(\exists B\in\mathcal{B})
		(\forall x \in B)
		[u(x)\neq1],
	\end{equation*}且\begin{equation*}
		\lim_\mathcal{B} u(x) = a \in (0,+\infty), \qquad
		\lim_\mathcal{B} v(x) = b \in (-\infty,+\infty),
	\end{equation*}
	那么\begin{equation*}
		\lim_\mathcal{B} u(x)^{v(x)} = a^b.
	\end{equation*}

	\item 如果\begin{equation*}
		(\exists B\in\mathcal{B})
		(\forall x \in B)
		[u(x)\neq1],
	\end{equation*}且\begin{equation*}
		\lim_\mathcal{B} u(x) = 1, \qquad
		\lim_\mathcal{B} v(x) = \infty,
	\end{equation*}
	那么\begin{equation*}
		\lim_\mathcal{B} u(x)^{v(x)}
		= \exp\lim_\mathcal{B} (u(x)-1) v(x).
	\end{equation*}
\end{itemize}
\begin{proof}
既然\(u,v\)都是连续函数,
那么由\cref{theorem:极限.连续函数的极限4} 可知
幂指函数\begin{equation*}
	u(x)^{v(x)} = e^{u(x) \ln v(x)}
\end{equation*}也是连续函数.

假设\(u(x) \equiv 1\),
那么\(u(x)^{v(x)} \equiv 1\),
于是\begin{equation*}
	\lim_\mathcal{B} u(x)^{v(x)} = 1.
\end{equation*}

假设\begin{equation*}
	(\exists B\in\mathcal{B})
	(\forall x \in B)
	[u(x)\neq1],
\end{equation*}且\begin{equation*}
	\lim_\mathcal{B} u(x) = a \in (0,+\infty), \qquad
	\lim_\mathcal{B} v(x) = b \in (-\infty,+\infty),
\end{equation*}
那么由\cref{theorem:极限.连续函数的极限3} 有\begin{equation*}
	\lim_\mathcal{B} u(x)^{v(x)} = a^b.
\end{equation*}

假设\begin{equation*}
	(\exists B\in\mathcal{B})
	(\forall x \in B)
	[u(x)\neq1],
\end{equation*}且\begin{equation*}
	\lim_\mathcal{B} u(x) = 1, \qquad
	\lim_\mathcal{B} v(x) = \infty.
\end{equation*}
把幂指函数\(u^v\)改写为\begin{equation*}
	u^v = \left((1+(u-1))^{\frac1{u-1}}\right)^{(u-1)v}.
\end{equation*}
记\(A = u - 1\).
当\(u \to 1\)时,\(A\)是无穷小.
因为\begin{equation*}
	\lim_{A\to0} (1+A)^{\frac1A} = e,
\end{equation*}
故只要极限\(\lim_\mathcal{B} (u-1) v = \lambda\)存在且有限,
那么立即得出\begin{equation*}
	\lim_\mathcal{B} u^v = e^\lambda.
	\qedhere
\end{equation*}
\end{proof}
\end{proposition}

\begin{example}
%@see: 《1996年全国硕士研究生入学统一考试(数学一)》一填空题/第1题
设\(\lim_{x\to\infty} \left( \frac{x+2a}{x-a} \right)^x = 8\),
求\(a\)的取值范围.
\begin{solution}
由于当\(x\to\infty\)时,
\(\frac{x+2a}{x-a} \to 1\),
于是\begin{equation*}
	\lim_{x\to\infty} \left( \frac{x+2a}{x-a} \right)^x
	= \exp\lim_{x\to\infty} \frac{3a}{x-a} \cdot x
	= e^{3a}
	= 8 = e^{\ln8} = e^{3\ln2},
\end{equation*}
易知\(a = \ln2\).
\end{solution}
\end{example}

\begin{example}
%@see: 《数学分析(第二版 上册)》(陈纪修) P107 例3.3.12
计算极限\(\lim_{x\to0} (\cos x)^{\frac1{x^2}}\).
\begin{solution}
由于当\(x\to0\)时,
\(\cos x \to 1\)而\(\frac1{x^2} \to \infty\),
于是\begin{equation*}
	\lim_{x\to0} (\cos x)^{\frac1{x^2}}
	= \exp\lim_{x\to0} (\cos x - 1) \cdot \frac1{x^2}
	%\cref{equation:函数极限.重要极限8}
	= \exp\lim_{x\to0} \left( -\frac12 x^2 \right) \frac1{x^2}
	= e^{-\frac12}.
\end{equation*}
% 直接计算得\begin{equation*}
% 	\lim_{x\to0} (\cos x)^{\frac1{x^2}}
% 	= \lim_{x\to0} [1-(1-\cos x)]^{\frac1{x^2}}
% 	= \lim_{x\to0} \left(1-\frac{x^2}2\right)^{\frac1{x^2}}
% 	= \frac1{\sqrt{e}}.
% \end{equation*}
\end{solution}
\end{example}

\begin{example}
%@see: 《2022年全国硕士研究生入学统一考试(数学二)》二填空题/第11题
计算极限\(\lim_{x\to0} \left( \frac{1+e^x}2 \right)^{\cot x}\).
\begin{solution}
由于当\(x\to0\)时,
\(\frac{1+e^x}2 \to 1\)而\(\cot x \to \infty\),
于是\begin{equation*}
	\lim_{x\to0} \left( \frac{1+e^x}2 \right)^{\cot x}
	= \exp\lim_{x\to0} \frac{e^x-1}2 \cdot \frac1{\tan x}
	= \exp\lim_{x\to0} \frac{x}2 \cdot \frac1x
	= e^{\frac12}.
\end{equation*}
\end{solution}
\end{example}

\begin{example}
%@see: https://www.bilibili.com/video/BV1RGz2YZENV/
计算极限\(\lim_{x\to\infty} \left[ \frac{x^n}{(x+1)(x+2)\dotsm(x+n)} \right]^x\).
\begin{solution}
记\(A \defeq (x+1)(x+2)\dotsm(x+n)\).
当\(x\to\infty\)时,\(\frac{A}{x^n}
= \left( 1+\frac1x \right) \left( 1+\frac2x \right) \dotsm \left( 1+\frac{n}{x} \vphantom{\frac1x} \right)
\to 1\),
于是\begin{align*}
	\lim_{x\to\infty} \left( \frac{x^n}{A} \right)^x
	&= \exp\lim_{x\to\infty} x \left( \frac{x^n}{A} - 1 \right) \\
	&= \exp\lim_{x\to\infty} x \left( -1 \cdot \frac1{x+1} - 2 \cdot \frac1{x+2} - \dotsb - n \frac1{x+n} \right) \\
	&= \exp\left( -1-2-\dotsb-n \right)
	= e^{-\frac12 n(n+1)}.
\end{align*}
\end{solution}
%@Mathematica: Limit[(x^n/Product[x + k, {k, 1, n}])^x, x -> Infinity]
\end{example}
\begin{example}
%@see: https://www.bilibili.com/video/BV1RGz2YZELr/
计算极限\(\lim_{x\to+\infty} \left( \sin\frac1x + \cos\frac1x \right)^x\).
\begin{solution}
由于当\(x\to+\infty\)时,
\(\sin\frac1x \sim \frac1x,
1-\cos\frac1x \sim \frac12 \cdot \frac1{x^2}\),
所以\begin{equation*}
	\lim_{x\to+\infty} \frac{\sin\frac1x + \cos\frac1x - 1}{\frac1x}
	= \lim_{x\to+\infty} \frac{\sin\frac1x}{\frac1x}
	+ \lim_{x\to+\infty} \frac{\cos\frac1x - 1}{\frac1x}
	= 1 + 0
	= 1,
\end{equation*}
于是\(\lim_{x\to+\infty} \left( \sin\frac1x + \cos\frac1x \right)^x
= e\).
\end{solution}
%@Mathematica: Limit[(Sin[1/x] + Cos[1/x])^x, x -> Infinity]
\end{example}

\begin{proposition}\label{theorem:幂指函数.与幂指函数有关的等价无穷小}
设幂指函数\(f\colon X\to\mathbb{R}\)
满足\(f(x) = (1+\alpha(x))^{\beta(x)}\),
其中\(\alpha(x)>-1\ (x \in X)\),
而\(\mathcal{B}\)是\(X\)中的基.
如果\begin{equation*}
	(\exists B\in\mathcal{B})
	(\forall x \in B)
	[\alpha(x)\neq0],
\end{equation*}
且\begin{equation*}
	\lim_\mathcal{B} \alpha(x) = 0,
	\qquad
	\lim_\mathcal{B} \alpha(x) \beta(x) = 0,
\end{equation*}
则\begin{equation*}
	\lim_\mathcal{B} \frac{(1+\alpha(x))^{\beta(x)}-1}{\alpha(x) \beta(x)} = 1.
\end{equation*}
\begin{proof}
由于\(\alpha\to0\),
所以\(\ln(1+\alpha) = \alpha + o(\alpha)\),
从而有\begin{equation*}
	(1+\alpha)^{\beta}
	= \exp[\beta \ln(1+\alpha)]
	= \exp[\beta (\alpha + o(\alpha))].
\end{equation*}
又因为\(\alpha \beta \to 0\),
所以\(o(\alpha) \beta \to 0\),
故有\begin{equation*}
	\exp[\beta (\alpha + o(\alpha))] - 1
	\sim \beta (\alpha + o(\alpha)),
\end{equation*}
于是\begin{align*}
	\lim_\mathcal{B} \frac{(1+\alpha)^{\beta}-1}{\alpha \beta}
	&= \lim_\mathcal{B} \frac{\exp[\beta (\alpha + o(\alpha))]-1}{\alpha \beta} \\
	&= \lim_\mathcal{B} \frac{\beta (\alpha + o(\alpha))}{\alpha \beta} \\
	&= \lim_\mathcal{B} \frac{\alpha + o(\alpha)}{\alpha}
	= 1.
	\qedhere
\end{align*}
\end{proof}
\end{proposition}

\section{闭区间上的连续函数}
在\cref{section:连续函数.函数的连续性与间断点}中已说明了函数在区间上连续的概念,
在闭区间上连续的函数有几个重要的性质,下面予以叙述.

\subsection{有界性与最大值最小值定理}
\begin{definition}
函数\(f\)在区间\(I\)上有定义.

如果有\(x_0 \in I\),使得对于任一\(x \in I\)都有\begin{equation*}
	f(x) \leq f(x_0),
\end{equation*}
则称“\(f(x_0)\)是函数\(f\)在区间\(I\)上的\DefineConcept{最大值}(global maximum)”,
记作\(\max_{x \in I} f(x) = f(x_0)\).

如果有\(x_0 \in I\),使得对于任一\(x \in I\)都有\begin{equation*}
	f(x) \geq f(x_0),
\end{equation*}
则称“\(f(x_0)\)是函数\(f\)在区间\(I\)上的\DefineConcept{最小值}(global minimum)”,
记作\(\min_{x \in I} f(x) = f(x_0)\).
\end{definition}
函数的最大值、最小值,统称为函数的\DefineConcept{最值}(global extremum).
%@see: https://mathworld.wolfram.com/Extremum.html

\begin{theorem}[有界性与最大值最小值定理]\label{theorem:极限.最值定理}
%@see: 《高等数学(第六版 上册)》 P71 定理1
%@see: 《数学分析(第二版 上册)》(陈纪修) P109 定理3.4.1
%@see: 《数学分析(第二版 上册)》(陈纪修) P110 定理3.4.2
在闭区间上连续的函数在该区间上有界且一定能取得它的最大值和最小值.
\begin{proof}
用反证法.
假设连续函数\(f\)在闭区间\([a,b]\)上无界,
那么根据\hyperref[definition:函数.函数的有界性]{有界函数的定义}可知\begin{equation*}
	(\forall M>0)
	(\exists x_n\in[a,b])
	[\abs{f(x_n)} > M],
\end{equation*}
换言之,函数列\(\{f(x_n)\}\)满足\begin{equation*}
	\lim_{n\to\infty} f(x_n) = \infty;
\end{equation*}
但是数列\(\{x_n\}\)有界(即\(a \leq x_n \leq b\)),
因此在闭区间\([a,b]\)内存在收敛子列\(\{x_{n_k}\}\),
不妨设\begin{equation*}
	\lim_{k\to\infty} x_{n_k} = c \in [a,b];
\end{equation*}
然而根据函数\(f\)的连续性可知,\begin{equation*}
	f(c) = \lim_{k\to\infty} f(x_{n_k}),
\end{equation*}
这就和\(f(x_n)\to\infty\ (n\to\infty)\)矛盾,因此函数\(f\)在\([a,b]\)上有界.
\end{proof}
\end{theorem}
我们也把\hyperref[theorem:极限.最值定理]{有界性与最大值最小值定理}称为%
\DefineConcept{魏尔斯特拉斯最值定理}.

注意:如果函数在开区间内连续,或函数在闭区间上有间断点,
那么函数在该区间上不一定有界,也不一定有最大值或最小值.

如果函数在开区间内连续,
或者函数在闭区间上有间断点,
那么这个函数在该区间上不一定有解,
也不一定有最大值或最小值.
例如,函数\(y=\tan x\)在开区间\(\left(-\frac{\pi}{2},\frac{\pi}{2}\right)\)内是连续的,
但它在\(\left(-\frac{\pi}{2},\frac{\pi}{2}\right)\)内是无界的,
且既无最大值又无最小值;
又如,函数\begin{equation*}
	y=\left\{ \begin{array}{ll}
		1-x, & 0\leq x<1, \\
		1, & x=1, \\
		3-x, & 1<x\leq2
	\end{array} \right.
\end{equation*}在闭区间\([0,2]\)上间断点\(x=1\),
它在闭区间\([0,2]\)上虽然有界,
但是既无最大值又无最小值.

\subsection{零点定理与介值定理}
\begin{definition}
如果\(x_0\)使\(f(x_0) = 0\),则\(x_0\)称为函数\(f\)的\DefineConcept{零点}.
\end{definition}

\begin{theorem}[零点定理]\label{theorem:极限.零点定理}
%@see: 《高等数学(第六版 上册)》 P71 定理2
%@see: 《数学分析(第二版 上册)》(陈纪修) P111 定理3.4.3
设函数\(f\)在闭区间\([a,b]\)上连续,
且\(f(a) \cdot f(b)<0\),
那么\begin{equation*}
	(\exists\xi\in(a,b))[f(\xi) = 0].
\end{equation*}
\begin{proof}
用反证法.
假设对\(\forall x\in(a,b)\)都有\(f(x) \neq 0\).

因为\(f(a) \cdot f(b)<0\),
不妨设\(f(a) < 0 < f(b)\).
若\(f\left(\frac{a+b}{2}\right)>0\),取\(a_1=a\)、\(b_1=\frac{a+b}{2}\);
若\(f\left(\frac{a+b}{2}\right)<0\),取\(a_1=\frac{a+b}{2}\)、\(b_1=b\);
总之,在上述两种情形下均有不等式\(f(a_1) < 0 < f(b_1)\)成立.
以此类推,建立闭区间序列\(\{[a_n,b_n]\}\),
可知\begin{equation*}
	[a_{n+1},b_{n+1}] \subseteq [a_n,b_n],
	f(a_n) < 0 < f(b_n),
	\lim_{n\to\infty} (b_n - a_n)
	= \lim_{n\to\infty} \frac{b-a}{2^n}
	= 0.
\end{equation*}
根据\cref{definition:极限.闭区间套定理},
\(\exists\xi\in[a,b]\)
满足\(\lim_{n\to\infty} a_n
= \xi
= \lim_{n\to\infty} b_n\).

由\hyperref[theorem:极限.函数极限的局部保号性3]{函数极限的局部保号性}有\begin{equation*}
	\lim_{n\to\infty} f(a_n) \leq 0,
	\qquad
	\lim_{n\to\infty} f(b_n) \geq 0.
\end{equation*}
又因为\(f \in C[a,b]\),由连续函数的定义可知,\(\lim_{x\to\xi} f(x) = f(\xi)\);
那么由\hyperref[theorem:极限.海涅定理]{海涅定理}可知\begin{equation*}
	\lim_{n\to\infty} f(a_n)
	= \lim_{n\to\infty} f(b_n)
	= f(\xi).
\end{equation*}
再由\hyperref[theorem:函数极限.夹逼准则]{夹逼准则}有\(f(\xi)=0\).
\end{proof}
\end{theorem}
我们也把\hyperref[theorem:极限.零点定理]{零点定理}称为%
\DefineConcept{波尔查诺--柯西中值定理}.

从几何上看,零点定理表示:
如果连续曲线弧\(y = f(x)\)的两个端点位于\(x\)轴的不同侧,
那么这段曲线弧与\(x\)轴至少有一个交点.

\begin{theorem}[介值定理]\label{theorem:极限.闭区间上连续函数的性质.介值定理1}
%@see: 《高等数学(第六版 上册)》 P72 定理3
%@see: 《数学分析(第二版 上册)》(陈纪修) P112 定理3.4.4
设函数\(f\)在闭区间\([a,b]\)上连续,且在这区间的端点取不同的函数值\begin{equation*}
	f(a) = A
	\quad\text{及}\quad
	f(b) = B.
\end{equation*}
那么,对于\(A\)与\(B\)之间的任意一个数\(C\),
在开区间\((a,b)\)内至少有一个点\(\xi\),使得\begin{equation*}
	f(\xi)=C
	\quad(a<\xi<b).
\end{equation*}
\begin{proof}
设\(\phi(x)=f(x)-C\),
则\(\phi(x)\)在闭区间\([a,b]\)上连续,
且\(\phi(a)=A-C\)与\(\phi(b)=B-C\)异号.
根据零点定理,开区间\((a,b)\)内至少有一点\(\xi\)使得\begin{equation*}
	\phi(\xi)=0
	\quad(a<\xi<b).
\end{equation*}
又\(\phi(\xi)=f(\xi)-C\),
因此由上式即得\begin{equation*}
	f(\xi)=C
	\quad(a<\xi<b).
	\qedhere
\end{equation*}
\end{proof}
\end{theorem}
\cref{theorem:极限.闭区间上连续函数的性质.介值定理1} 的几何意义是:
连续曲线弧\(y=f(x)\)与水平直线\(y=C\)至少相交于一点.

\begin{corollary}\label{theorem:极限.闭区间上连续函数的性质.介值定理2}
%@see: 《高等数学(第六版 上册)》 P72 推论
在闭区间上连续的函数必取得介于最大值\(M\)与最小值\(m\)之间的任何值.
\begin{proof}
设函数\(f \in C[a,b]\),取\begin{equation*}
	M=\max_{a \leq x \leq b} f(x), \qquad
	m=\min_{a \leq x \leq b} f(x),
\end{equation*}
假设\(m=f(x_1)\),\(M=f(x_2)\),且\(m \neq M\),\(x_1 \neq x_2\).
在闭区间\([\min\{x_1,x_2\},\max\{x_1,x_2\}]\)上应用介值定理,
于是有\begin{equation*}
	(\forall\mu\in(m,M))(\exists\xi\in(a,b))
	[f(\xi)=\mu].
	\qedhere
\end{equation*}
\end{proof}
\end{corollary}

\begin{corollary}
如果\(I\)是区间,且\(f\in C(I)\),则\(f\ImageOfSetUnderRelation{I}\)也是区间.
\end{corollary}

\begin{corollary}
设函数\(f\)是定义在区间\(I\)上的严格单调函数,
则\(f\)连续当且仅当\(f\ImageOfSetUnderRelation{I}\)也是区间.
\end{corollary}

\begin{corollary}
定义在区间\(I\)上的严格单调连续函数必然可逆,且逆函数也是严格单调连续的.
\end{corollary}

\begin{corollary}
如果\(f \in C(I)\),则\(f\)严格单调当且仅当\(f^{-1}\)存在.
\end{corollary}

\begin{corollary}
如果\(f\colon[a,b]\to[a,b]\)连续,则\(\exists\xi\in[a,b]: f(\xi)=\xi\).
\end{corollary}

\begin{example}
设\(f \in C[a,b]\).
证明:函数\begin{equation*}
	m(x) = \inf_{a \leq y \leq x} f(y), \qquad
	M(x) = \sup_{a \leq y \leq x} f(y)
\end{equation*}都在\([a,b]\)上连续.
\begin{proof}
这里只证\(m\)是连续的.
\begin{enumerate}
	\item 首先证明\(m\)在点\(x=a\)处右连续.
	注意到\(m(a) = f(a)\).
	对\(\forall\epsilon>0\),
	由于\(f\)在\(x=a\)处右连续,
	所以\(\exists\delta>0\)
	使得对\(\forall x\in(a,a+\delta)\)
	都有\(\abs{f(x)-f(a)}<\epsilon\)成立;
	因此\begin{equation*}
	f(x)
	> f(a) - \epsilon
	= m(a) - \epsilon.
	\end{equation*}
	于是,当\(x\in[a,a+\delta)\)时,
	对\(\forall y\in[a,x]\)总有\begin{equation*}
		m(a) - \epsilon < f(y);
	\end{equation*}
	再根据\(m\)的定义有\begin{equation*}
		m(a) - \epsilon \leq m(x) \leq m(a);
	\end{equation*}
	这样就有不等式\(\abs{m(x)-m(a)}<\epsilon\)成立.

	\item 然后证明\(m\)在点\(x=b\)处左连续.
	因为\(f \in C[a,b]\),
	所以由\cref{theorem:极限.闭区间上连续函数的性质.介值定理1} 可知,
	\(\exists\xi\in[a,b]\)
	满足\(f(\xi) = \min_{a \leq x \leq b} f = m(b)\).
		\begin{enumerate}
			\item 首先假设\(\xi=b\),
			即有\(f(b)=m(b)\).
			对\(\forall\epsilon>0\),
			由于\(f\)在\(x=b\)处左连续,
			所以\(\exists\delta>0\)
			使得对\(\forall x\in(b-\delta,b)\)
			都有\(\abs{f(x)-f(b)}<\epsilon\)成立;
			因此\begin{equation*}
				f(x)
				< f(b) + \epsilon
				= m(b) + \epsilon.
			\end{equation*}
			于是,当\(x\in(b-\delta,b)\)时,
			根据\(m\)的定义有\begin{equation*}
				m(x) \leq m(b) + \epsilon,
			\end{equation*}
			即\(\abs{m(x)-m(b)}<\epsilon\),
			也即\(\lim_{x \to b^-} m(x) = m(b)\).

			\item 然后假设\(\xi\in[a,b)\).
			那么对\(\forall x\in(a,b)\)
			有\begin{equation*}
				m(x) = \inf_{a \leq y \leq x} f(y)
				\leq f(\xi)
				= m(b)
				\geq \inf_{a \leq y \leq b} f(y)
				= m(b).
			\end{equation*}
		\end{enumerate}
		由上可知\(m(x) = m(b)\),
		也就是说\(m\)在点\(x=b\)处左连续.

	\item 最后证明\(m\)在\((a,b)\)内连续.
	证明已经蕴含在上述证明\(m\)在点\(x=a\)处右连续和在点\(x=b\)处左连续的过程中.
	\qedhere
\end{enumerate}
\end{proof}
\end{example}

\section{函数的一致连续性}
\subsection{一致连续性的概念}
在\cref{section:连续函数.函数的连续性与间断点}中,
我们已经指出,函数\(f\)在某个区间\(X\)上连续,
是指它在区间\(X\)上的每一点连续(对区间端点是指左连续与右连续).
我们只要观察函数在一点的连续性的定义\begin{equation*}
	(\forall\epsilon>0)
	(\exists\delta>0)
	(\forall x)
	[
		\abs{x - x_0} < \delta
		\implies
		\abs{f(x) - f(x_0)} < \epsilon
	],
\end{equation*}
就可以发现,这里的\(\delta\)与两个因素有关:
它既依赖于\(\epsilon\),也依赖于\(x_0\).
也就是说,\(\delta\)是由\(x_0\)和\(\epsilon\)两个变量决定的函数\(\delta(x_0,\epsilon)\).

这样就产生一个问题:
对于任意给定的\(\epsilon>0\),
能否找到一个只与\(\epsilon\)有关,
而对区间\(X\)上一切点都适用的\(\delta=\delta(\epsilon)\).

这个问题的答案是不一定的.
它不仅与所讨论的函数\(f\)有关,也与所讨论的区间\(X\)有关.

\begin{definition}\label{definition:极限.函数的一致连续性}
%@see: 《数学分析(第二版 上册)》(陈纪修) P113 定理3.4.1
%@see: 《高等数学(第六版 上册)》 P73 定义
设函数\(f\)在区间\(I\)上有定义.
如果对于任意给定的正数\(\epsilon\),
总存在着正数\(\delta\),
使得对于区间\(I\)上的任意两点\(x_1\)、\(x_2\),
当\(\abs{x_1 - x_2} < \delta\)时,就有\begin{equation*}
	\abs{f(x_1) - f(x_2)} < \epsilon,
\end{equation*}
那么称“函数\(f\)在区间\(I\)上是\DefineConcept{一致连续的}(uniformly continuous)”.
\end{definition}
函数的一致连续性表示:
不论在区间\(I\)的任何部分,
只要自变量的两个数值接近到一定程度,
就可使对应的两个函数值达到所指定的接近程度.

需要注意的是,讲述函数的一致连续性时,一定要讲明它是在哪个区间上一致连续.
一个函数虽说在区间\(I_1\)上不是一致连续的,但可以在不同的区间\(I_2\)上一致连续.

\begin{example}
%@see: 《数学分析(第二版 上册)》(陈纪修) P113 例3.4.3
证明:函数\(f(x)=\sin x\)在区间\((-\infty,+\infty)\)上是一致连续的.
\begin{proof}
因为\begin{align*}
	\abs{f(x_1)-f(x_2)}
	&= \abs{\sin x_1 - \sin x_2}
	= 2\abs{\cos\frac{x_1+x_2}{2} \sin\frac{x_1-x_2}{2}} \\
	&\leq 2\abs{\sin\frac{x_1-x_2}{2}},
\end{align*}
而当\(\abs{x_1-x_2} \leq \pi\)时有\begin{equation*}
	\abs{\sin\frac{x_1-x_2}{2}} \leq \abs{\frac{x_1-x_2}{2}}
\end{equation*}成立,
所以\(\forall\epsilon>0\)(设\(\epsilon<\pi\)),
总可取\(\delta=\epsilon\),
对于\(\forall x_1,x_2\in\mathbb{R}\),
当\(0<\abs{x_1-x_2}<\delta=\epsilon\)时,
能使不等式\begin{equation*}
	\abs{f(x_1)-f(x_2)} \leq \abs{x_1-x_2} < \epsilon
\end{equation*}成立,
也就是说\(f(x)\)是一致连续的.
\end{proof}
\end{example}

\begin{example}\label{example:极限.无界函数可以是一致连续的}
证明:函数\(f(x)=x\)在区间\((-\infty,+\infty)\)上是一致连续的.
\begin{proof}
因为\(\abs{f(x_1)-f(x_2)}=\abs{x_1-x_2}\),
所以\(\forall\epsilon>0\),
总可取\(\delta=\epsilon\),
对于\(\forall x_1,x_2\in\mathbb{R}\),
当\(0<\abs{x_1-x_2}<\delta=\epsilon\)时,
能使不等式\begin{equation*}
	\abs{f(x_1)-f(x_2)} = \abs{x_1-x_2} < \epsilon
\end{equation*}成立,
也就是说\(f\)是一致连续的.
\end{proof}
\end{example}
\begin{remark}
\cref{example:极限.无界函数可以是一致连续的} 说明:
无界函数可以是一致连续的.
\end{remark}

\begin{example}\label{example:极限.在半开区间连续的函数不一定在该区间上一致连续}
%@see: 《高等数学(第六版 上册)》 P73 例2
%@see: 《数学分析(第二版 上册)》(陈纪修) P114 例3.4.4
试证:函数\(f(x) = \frac{1}{x}\)在区间\((0,1]\)上是连续的,但不是一致连续的.
\begin{proof}
因为函数\(f(x) = 1/x\)是初等函数,
它在区间\((0,1]\)上有定义,
所以在\((0,1]\)上是连续的.
假定\(f(x) = 1/x\)是\((0,1]\)上一致连续,
那么\(\forall \epsilon \in (0,1)\),
\(\exists \delta > 0\),
使得对于\(\forall x_1,x_2 \in (0,1]\),
当\(\abs{x_1 - x_2} < \delta\)时,
就有\(\abs{f(x_1) - f(x_2)} < \epsilon\).

现在取原点附近的两点\begin{equation*}
	x_1 = \frac{1}{n}, \quad
	x_2 = \frac{1}{n+1},
\end{equation*}
其中\(n\in\mathbb{N}^+\),
显然\(x_1,x_2 \in (0,1]\)上.
因\begin{equation*}
	\abs{x_1 - x_2} = \abs{\frac{1}{n} - \frac{1}{n+1}}
	= \frac{1}{n(n+1)},
\end{equation*}
故只要\(n\)取得足够大,
总能使\(\abs{x_1 - x_2} < \delta\).
但这时有\begin{equation*}
	\abs{f(x_1) - f(x_2)}
	= \abs{\frac{1}{1/n} - \frac{1}{1/(n+1)}}
	= \abs{n - (n+1)}
	= 1 > \epsilon,
\end{equation*}
不符合一致连续的定义,
所以\(f(x) = \frac{1}{x}\)在\((0,1]\)上不是一致连续的.
\end{proof}
\end{example}
\begin{remark}
\cref{example:极限.在半开区间连续的函数不一定在该区间上一致连续} 说明:
在半开区间连续的函数不一定在该区间上一致连续.
\end{remark}

\begin{example}\label{example:极限.在开区间上有界且连续的函数不一定在该区间上一致连续}
证明:函数\(\sin\frac{1}{x}\)在\((0,1)\)上是不一致连续的.
\begin{proof}
取\begin{equation*}
	s_n = \frac{1}{2n\pi+\pi/2},
	\qquad
	t_n = \frac{1}{2n\pi},
\end{equation*}
故\begin{equation*}
	s_n,t_n\in(0,1),
	\quad n\in\mathbb{N}^*.
\end{equation*}我们有\begin{equation*}
	0 < t_n - s_n = \frac{\pi/2}{2n\pi(2n\pi+\pi/2)} < \frac{1}{2n\pi} < \frac{1}{n},
\end{equation*}
但是\begin{equation*}
	\abs{\sin\frac{1}{t_n} - \sin\frac{1}{s_n}}
	= \abs{\sin(2n\pi) - \sin(2n\pi+\frac{\pi}{2})}
	= 1.
\end{equation*}
这就证明了函数\(\sin\frac{1}{x}\)在\((0,1)\)上不是一致连续的.
\end{proof}
\end{example}
\begin{remark}
\cref{example:极限.在开区间上有界且连续的函数不一定在该区间上一致连续} 说明:
在开区间上有界且连续的函数不一定在该区间上一致连续.
\end{remark}

\begin{example}\label{example:极限.两个一致连续函数的乘积不一定一致连续}
证明:函数\(h(x) = x \sin x\)在\((-\infty,+\infty)\)上不是一致连续的.
\end{example}
\begin{remark}
\cref{example:极限.两个一致连续函数的乘积不一定一致连续} 说明:
两个一致连续函数的乘积不一定一致连续.
\end{remark}

\subsection{一致连续性的判别法}
\begin{theorem}
%@see: 《数学分析(第二版 上册)》(陈纪修) P114 定理3.4.5
设函数\(f\)在区间\(X\)上有定义,
则\(f\)在\(X\)上一致连续的充分必要条件是:
对任何点列\(\{a_n\}\ (a_n \in X)\)和\(\{b_n\}\ (b_n \in X)\),
只要满足\(\lim_{n\to\infty} (a_n-b_n) = 0\),
就成立\(\lim_{n\to\infty} (f(a_n)-f(b_n)) = 0\).
\end{theorem}

\begin{example}
%@see: 《数学分析(第二版 上册)》(陈纪修) P115 例3.4.5
证明:函数\(f(x) = x^2\)在\([0,+\infty)\)上不是一致连续的,
但是在\([0,A]\)上一致连续(\(A\)是任意有限正数).
\begin{proof}
取\(a_n = \sqrt{n+1},
b_n = \sqrt{n}
\ (n=1,2,\dotsc)\),
于是\begin{equation*}
	\lim_{n\to\infty} (a_n-b_n)
	= \lim_{n\to\infty} (\sqrt{n+1}-\sqrt{n})
	= \lim_{n\to\infty} \frac1{\sqrt{n+1}+\sqrt{n}}
	= 0,
\end{equation*}
但是\(\lim_{n\to\infty} (f(a_n)-f(b_n)) = 1\),
所以\(f\)在\([0,+\infty)\)上不是一致连续的.

当区间限制在\([0,A]\)时,
有\begin{equation*}
	\abs{x_1^2-x_2^2}
	= \abs{(x_1+x_2)(x_1-x_2)}
	\leq 2A \abs{x_1-x_2}.
\end{equation*}
对于任意给定\(\epsilon>0\),
可以取\(\delta=\frac\epsilon{2A}\),
对任意\(x_1,x_2\in[0,A]\),
只要\(\abs{x_1-x_2}<\delta\),
就成立\(\abs{x_1^2-x_2^2}<\epsilon\),
即\(f\)在\([0,A]\)上一致连续.
\end{proof}
\end{example}

\subsection{一致连续函数族}
\begin{definition}\label{definition:函数族.一致连续函数族}
定义:区间\(I\)上一致连续函数族\begin{equation*}
	C_U(I)
	\defeq
	\Set{
		f\in\mathbb{R}^I
		\given
		\text{\(f\)在区间\(I\)上是一致连续的}
	}.
\end{equation*}
\end{definition}

已知函数\(f\colon I\to\mathbb{R}\).
上述对函数一致连续性的定义可以简化为:\begin{equation*}
	f \in C_U(I)
	\iff
	(\forall\epsilon>0)
	(\exists\delta>0)
	(\forall x_1,x_2 \in I)
	[
		\abs{x_1 - x_2} < \delta
		\implies
		\abs{f(x_1) - f(x_2)} < \epsilon
	];
\end{equation*}
相反地,有\begin{equation*}
	f \notin C_U(I)
	\iff
	(\exists\epsilon_0>0)
	(\forall\delta>0)
	(\exists x_1,x_2 \in I)
	[
		\abs{x_1-x_2}<\delta
		\land
		\abs{f(x_1)-f(x_2)}\geq\epsilon_0
	].
\end{equation*}

\begin{theorem}\label{theorem:极限.闭区间上连续函数的性质.一致连续函数一定连续}
如果函数\(f\)在区间\(I\)上一致连续,那么\(f\)在区间\(I\)上连续.
\begin{proof}
因为函数\(f\)在区间\(I\)上一致连续,
只要任意取定一点\(x_0 \in I\),
就有\begin{equation*}
	(\forall \epsilon > 0)
	(\exists \delta > 0)
	(\forall x,x_0 \in I)
	[
		\abs{x-x_0} < \delta
		\implies
		\abs{f(x)-f(x_0)} < \epsilon
	],
\end{equation*}
这就是说,函数\(f\)在区间\(I\)内任意一点\(x_0\)连续,
因此\(f\)在区间\(I\)上连续.
\end{proof}
\end{theorem}

\begin{theorem}[一致连续函数的四则运算法则]\label{theorem:极限.闭区间上连续函数的性质.一致连续函数的四则运算法则}
设函数\(f,g \in C_U(I)\),则
\begin{enumerate}
	\item 两个一致连续函数的线性组合也是一致连续的,
	即\begin{equation*}
		(\forall\alpha,\beta\in\mathbb{R})
		[\alpha f + \beta g \in C_U(I)].
	\end{equation*}

	\item 如果\(f,g\)在\(I\)上有界,
	则\(f \cdot g \in C_U(I)\).

	\item 如果\(f\)在\(I\)上有界,
	且\((\exists\epsilon_0)
	[x \in I \implies g \geq \epsilon_0]\),
	则\begin{equation*}
		\frac{f}{g} \in C_U(I).
	\end{equation*}
\end{enumerate}
\end{theorem}

\begin{theorem}[一致连续性定理]\label{theorem:极限.一致连续性定理}
如果函数\(f\)在闭区间\([a,b]\)上连续,
那么它在该区间上一致连续,
即\begin{equation*}
	f \in C[a,b]
	\implies
	f \in C_U[a,b].
\end{equation*}
\begin{proof}
用反证法.
假设当\(f \in C[a,b]\)时\(f \notin C_U[a,b]\),那么\begin{equation*}
	(\exists\delta_0>0)
	(\exists\{x_n\},\{y_n\}\subseteq[a,b])
	[
		x_n-y_n\to0
		\land
		\abs{f(x_n)-f(y_n)}\geq\epsilon_0
	].
\end{equation*}

由于\(\{y_n\}\)有界,可以找到收敛子列\(\{y_{n_k}\}\)满足\begin{equation*}
	y_{n_k} \to y_0\in[a,b].
\end{equation*}

取\(x_{n_k} = y_{n_k} + (x_{n_k} - y_{n_k})
\to y_0 + 0 = y_0\),
得到\begin{equation*}
	0 < \epsilon_0 \leq \abs{f(x_{n_k})-f(y_{n_k})}
	\to \abs{f(y_0)-f(y_0)} = 0.
\end{equation*}矛盾,
说明函数\(f\)在\([a,b]\)上定是一致连续的.
\end{proof}
\end{theorem}
我们也把\hyperref[theorem:极限.一致连续性定理]{一致连续性定理}称为\DefineConcept{康托--海涅定理}.

\begin{example}
设函数\(f\)对于\(\forall x,y \in [a,b]\),
恒有\(\abs{f(x) - f(y)} \leq L \abs{x - y}\),
其中常数\(L > 0\),且\(f(a) \cdot f(b) < 0\).
证明:至少有一点\(\xi \in (a,b)\),使得\(f(\xi) = 0\).
\begin{proof}
因为\(\forall x,y \in [a,b] \implies \abs{f(x) - f(y)} \leq L \abs{x - y}\),
对于\(\forall \epsilon > 0\),
为了使当\(\abs{x - y} < \delta\)时不等式\begin{equation*}
\abs{f(x) - f(y)} \leq L \abs{x - y} < \epsilon
\end{equation*}成立,
只要\(L \delta = \epsilon\)或\(\delta = \epsilon / L\)即可.
这说明\(f\)在区间\([a,b]\)上一致连续.
再由零点定理可知命题成立.
\end{proof}
\end{example}

\begin{example}
试证:若\(f\)在\([a,b]\)上连续,
\(a < x_1 < x_2 < \dotsb < x_n < b \ (n \geq 3)\),
则在\((x_1,x_n)\)内至少有一点\(\xi\),使\begin{equation*}
f(\xi) = \frac{1}{n} \bigl[
	f(x_1) + f(x_2) + \dotsb + f(x_n)
\bigr].
\end{equation*}
\begin{proof}
根据有界性与最大值最小值定理,因为\(f\)在\([a,b]\)上连续,所以\begin{equation*}
\forall x \in [a,b] :
	\alpha \leq f(x) \leq \beta,
\end{equation*}其中\(\alpha = \min f(x)\),\(\beta = \max f(x)\);那么\begin{equation*}
n \alpha \leq f(x_1) + f(x_2) + \dotsb + f(x_n) \leq n \beta,
\end{equation*}\begin{equation*}
\alpha \leq \frac{1}{n} \bigl[f(x_1) + f(x_2) + \dotsb + f(x_n)\bigr] \leq \beta.
\end{equation*}根据介值定理,在\([a,b]\)上必存在一点\(\xi\)使得\begin{equation*}
f(\xi) = \frac{1}{n} \bigl[ f(x_1) + f(x_2) + \dotsb + f(x_n) \bigr]
\end{equation*}成立.
\end{proof}
\end{example}

\begin{theorem}\label{theorem:极限.闭区间上连续函数的性质.开区间上的连续函数一致连续的充分必要条件1}
设\(f \in C(a,b)\),
则\(f \in C_U(a,b)\)的充分必要条件是:
极限\(f(a^+)\)和\(f(b^-)\)都存在.
\end{theorem}

\begin{theorem}\label{theorem:极限.闭区间上连续函数的性质.开区间上的连续函数一致连续的充分必要条件2}
设\(f \in C(-\infty,+\infty)\),
则\(f \in C_U(-\infty,+\infty)\)的充分必要条件是:
极限\(f(-\infty)\)和\(f(+\infty)\)都存在.
\end{theorem}

\begin{definition}
对于任意一个函数\(f\colon X\to\mathbb{R}\),
集合\(E \subseteq X\),
我们把\begin{equation*}
	\sup_{\substack{
		\abs{x_1-x_2}<\delta \\
		x_1,x_2 \in E
	}}\abs{f(x_1)-f(x_2)},
\end{equation*}
称为“函数\(f\)的\DefineConcept{连续模}”,
记为\(\amp(f;E;\delta)\).
\end{definition}

我们可以用连续模来度量一个函数的一致连续性.

\section{迭代数列的敛散性}
\begin{definition}\label{definition:迭代数列.不动点}
%@see: 《数学分析习题课讲义(第2版 上册)》(谢惠民、恽自求、易法槐、钱定边) P49 命题2.6.1(第一律)
设函数\(f\colon D\to\mathbb{R}\).
若点\(x_0 \in D\)满足\begin{equation*}
	f(x_0) = x_0,
\end{equation*}
则称“\(x_0\)是\(f\)的\DefineConcept{不动点}”.
\end{definition}
\begin{proposition}\label{theorem:数列极限.迭代数列.第一律}
%@see: 《数学分析习题课讲义(第2版 上册)》(谢惠民、恽自求、易法槐、钱定边) P49 命题2.6.1(第一律)
设数列\(\{x_n\}_{n\geq1}\)满足递推公式\(x_{n+1} = f(x_n)\ (n\in\mathbb{N}^+)\).
若有\begin{equation*}%equation:数列极限.迭代数列.第一律.条件1
	\lim_{n\to\infty} x_n = x_0,
\end{equation*}和\begin{equation}\label{equation:数列极限.迭代数列.第一律.条件2}
	\lim_{n\to\infty} f(x_n) = f(x_0),
\end{equation}
则\(x_0\)是\(f\)的不动点.
\begin{proof}
在递推公式等号两边同时求极限,令\(n\to\infty\),
得\begin{equation*}
	x_0 = \lim_{n\to\infty} x_{n+1} = \lim_{n\to\infty} f(x_n) = f(x_0).
	\qedhere
\end{equation*}
\end{proof}
\end{proposition}
\begin{remark}
\cref{equation:数列极限.迭代数列.第一律.条件2} 可以加强为
“\(f\)在点\(x_0\)~\emph{连续}”
(参考\cref{theorem:连续函数.函数连续点与海涅定理的关系,theorem:连续函数.单调迭代数列收敛定理}).
\cref{theorem:数列极限.迭代数列.第一律} 用处在于:
即便我们不知道数列是否收敛,也可以先去求解方程\(f(x) = x\).
而求出方程的根对于判定数列\(\{x_n\}_{n\geq1}\)的收敛性往往是有帮助的.
例如,如果方程\(f(x) = x\)在实数范围内无解,
就可以直接断定:满足\(x_{n+1} = f(x_n)\)的数列\(\{x_n\}_{n\geq1}\)一定是发散的.
\end{remark}
\begin{proposition}\label{theorem:数列极限.迭代数列.第二律}
%@see: 《数学分析习题课讲义(第2版 上册)》(谢惠民、恽自求、易法槐、钱定边) P49 命题2.6.2(第二律)
设数列\(\{x_n\}_{n\geq1}\)满足\(x_{n+1} = f(x_n)\ (n\in\mathbb{N}^+)\),
其中\(f\colon D\to\mathbb{R}\)是单调函数,
且\(\{x_n\}_{n\geq1}\)的值域为\(D\).
\begin{itemize}
	\item 当\(f\)单调增加时,\(\{x_n\}_{n\geq1}\)是单调数列.
	\item 当\(f\)单调减少时,\(\{x_n\}_{n\geq1}\)不是单调数列,
	但它的奇子列\(\{x_{2k-1}\}_{k\geq1}\)和偶子列\(\{x_{2k}\}_{k\geq1}\)都是单调数列,
	且具有相反的单调性.
\end{itemize}
\begin{proof}
假设\(f\)单调增加.
由给定条件有\(x_n \in D\ (n\geq1)\).
如果\(x_2 \geq x_1\),
那么\begin{equation*}
	x_3 = f(x_2) \geq f(x_1) = x_2,
\end{equation*}
利用数学归纳法可证\(x_{n+1} \geq x_n\)对\(n\geq1\)成立,
数列\(\{x_n\}_{n\geq1}\)单调增加.
同理,如果\(x_2 \leq x_1\),
那么数列\(\{x_n\}_{n\geq1}\)单调减少.

假设\(f\)单调减少.
由\cref{theorem:函数.两个严格单调减少函数的复合严格单调增加} 可知,
复合函数\(f \circ f\)单调增加,
这就是说,只要有\begin{equation*}
	a,b,f(a),f(b) \in D,
	\quad\text{且}\quad
	a<b,
\end{equation*}
就有\begin{equation*}
	f(f(a)) \leq f(f(b)).
\end{equation*}
如果\(x_1 = x_3\),则奇子列\(\{x_{2k-1}\}_{k\geq1}\)是常数列.
如果\(x_1 < x_3\),
由\(f\)单调减少可知\begin{equation*}
	x_2 = f(x_1) \geq f(x_3) = x_4,
\end{equation*}
然后推出\begin{equation*}
	x_3 = f(x_2) \leq f(x_4) = x_5.
\end{equation*}
用数学归纳法可证奇子列\(\{x_{2k-1}\}_{k\geq1}\)单调增加,
偶子列\(\{x_{2k}\}_{k\geq1}\)单调减少.
对于\(x_1>x_3\)的讨论完全类似,从略.
\end{proof}
\end{proposition}
\begin{remark}
从\cref{theorem:数列极限.迭代数列.第二律} 的证明过程中不难看出,
当\(f\)单调增加时,若\(\{x_n\}_{n\geq1}\)单调增加,
只有两种可能情形:
要么\(\{x_n\}_{n\geq1}\)从某项起成为常数列(每一项都是\(f\)的不动点),
要么\(\{x_n\}_{n\geq1}\)是严格单调增加数列.
\(\{x_n\}_{n\geq1}\)单调减少的情况类似.

另外,我们还可以得出结论:
如果数列\(\{x_n\}_{n\geq1}\)的值域包含于函数\(f\)的单调区间\(D\),
即\(x_n \in D\ (n=1,2,\dotsc)\),
且\(\{x_n\}_{n\geq1}\)有界(\(D\)可以是无界的),
那么当\(f\)单调增加时,数列\(\{x_n\}_{n\geq1}\)必定收敛;
当\(f\)单调减少时,数列\(\{x_n\}_{n\geq1}\)可能收敛,也可能发散,
但它的两个子列\(\{x_{2k-1}\}_{k\geq1}\)和\(\{x_{2k}\}_{k\geq1}\)必定收敛,
于是\(\{x_n\}_{n\geq1}\)是否收敛取决于
\(\{x_{2k-1}\}_{k\geq1}\)和\(\{x_{2k}\}_{k\geq1}\)的极限是否相等.
对于数列\(\{x_n\}_{n\geq1}\)无界的情况,可以作出类似的讨论.
%TODO 类似的讨论是什么意思?
\end{remark}
\begin{figure}[htb]
%@see: 《数学分析习题课讲义(第2版 上册)》(谢惠民、恽自求、易法槐、钱定边) P50 图2.4
	\centering
	\pgfplotsset{ticks=none}
	\begin{tikzpicture}
		\begin{axis}[
			xmin=0,xmax=1.5,
			ymin=0,ymax=1.5,
			axis lines=middle,
			axis equal=true,
			xlabel=$x$,
			ylabel=$y$,
			enlarge x limits=0.1,
			enlarge y limits=0.1,
			x label style={at={(ticklabel* cs:1.00)}, inner sep=5pt, anchor=south},
			y label style={at={(ticklabel* cs:1.00)}, inner sep=2pt, anchor=west},
		]
			\addplot[color=black!30,dashed,samples=2,smooth,domain=0:1.5]{x};
			\addplot[color=blue,samples=50,smooth,domain=0:1.5]{ln(x+2)};
			\begin{scope}
				\def\DrawArrow#1#2#3{
					\draw[black!30,dashed](#1,#2)--(#1,0);
					\draw[>=Stealth,->](#1,#2)--(#1,{ln(#1+2)});
					\draw[>=Stealth,->](#1,{ln(#1+2)})--({ln(#1+2)},{ln(#1+2)});
					\draw(#1,0)node[below]{#3};
				}
				\DrawArrow{.1}{0}{$x_1$}
				\DrawArrow{.74}{.74}{$x_2$}
				\DrawArrow{1}{1}{$x_3$}
			\end{scope}
		\end{axis}
	\end{tikzpicture}~\begin{tikzpicture}
		\begin{axis}[
			xmin=0,xmax=1.5,
			ymin=0,ymax=1.5,
			axis lines=middle,
			axis equal=true,
			xlabel=$x$,
			ylabel=$y$,
			enlarge x limits=0.1,
			enlarge y limits=0.1,
			x label style={at={(ticklabel* cs:1.00)}, inner sep=5pt, anchor=south},
			y label style={at={(ticklabel* cs:1.00)}, inner sep=2pt, anchor=west},
		]
			\addplot[color=black!30,dashed,samples=2,smooth,domain=0:1.5]{x};
			\addplot[color=blue,samples=50,smooth,domain=0:1.5]{exp(-x)};
			\begin{scope}
				\def\DrawArrow#1#2#3{
					\draw[black!30,dashed](#1,#2)--(#1,0);
					\draw[>=Stealth,->](#1,#2)--(#1,{exp(-#1)});
					\draw[>=Stealth,->](#1,{exp(-#1)})--({exp(-#1)},{exp(-#1)});
					\draw(#1,0)node[below]{#3};
				}
				\DrawArrow{.1}{0}{$x_1$}
				\DrawArrow{.9048}{.9048}{$x_2$}
				\DrawArrow{.4046}{.4046}{$x_3$}
				\DrawArrow{.6672}{.6672}{$x_4$}
			\end{scope}
		\end{axis}
	\end{tikzpicture}
	\caption{}
\end{figure}
\begin{proposition}\label{theorem:连续函数.单调迭代数列收敛定理}
%@see: 《数学分析习题课讲义(第2版 上册)》(谢惠民、恽自求、易法槐、钱定边) P51 命题2.6.3
%@see: 《数学分析:原理与方法》(胡适耕) P35 4.1.6命题
设\(a\)是函数\(f\)的不动点,
\(f\)在区间\((a-r,a+r)\)内严格单调增加且连续,
并且在区间\((a-r,a)\)内有\(f(x)>x\),
而在区间\((a,a+r)\)内有\(f(x)<x\),
那么只要数列\(\{x_n\}_{n\geq1}\)满足\begin{equation*}
	x_1 \in (a-r,a)\cup(a,a+r),
	\qquad
	x_{n+1} = f(x_n)\ (n=1,2,\dotsc),
\end{equation*}
则\begin{itemize}
	% 数列\(\{x_n\}_{n\geq1}\)以后不会越出区间\((a-r,a+r)\),
	\item \(x_n \in (a-r,a+r)\ (n=1,2,\dotsc)\),
	\item \(x_n \to a\ (n\to\infty)\),
	\item \(\{x_n\}_{n\geq1}\)是严格单调数列.
\end{itemize}
\begin{proof}
从条件可知,\(f\)在点\(a\)的两侧均有\(f(x) \neq x\),
因此\(f\)在区间\((a-r,a+r)\)内只可能有唯一的不动点\(a\).

不妨设\(x_1 \in (a-r,a)\).
从\(f\)的严格单调性和\(a_1 < a\)
得到\(a_2 = f(a_1) < f(a) = a\).
又因为在区间\((a-r,a)\)上满足条件\(f(x) > x\),
就有\(a_2 = f(a_1) > a_1\).
合起来就有\(a_1 < a_2 < a\).

用数学归纳法可以证明数列\(\{x_n\}_{n\geq1}\)完全落在区间\((a-r,a)\)内,
且严格单调增加,
即\begin{equation*}
	x_n < x_{n+1} < a
	\quad(n=1,2,\dotsc).
\end{equation*}
由于它以\(a\)为上界,
因此由\hyperref[theorem:极限.数列的单调有界定理]{单调有界定理}可知收敛.
不妨设\(\lim_{n\to\infty} x_n = b\).
由\cref{theorem:极限.收敛数列的保序性2} 可知\begin{equation*}
	b \in [a_1,a] \subseteq (a-r,a].
\end{equation*}
由\cref{theorem:连续函数.函数连续点与海涅定理的关系} 可知\begin{equation*}
	\lim_{n\to\infty} f(x_n) = f(b).
\end{equation*}
于是\begin{equation*}
	b = \lim_{n\to\infty} x_{n+1}
	= \lim_{n\to\infty} f(x_n)
	= f(b),
\end{equation*}
也就是说\(b\)是\(f\)的不动点.
因为\(a\)是唯一的不动点,
所以\(b = a\),
即\begin{equation*}
	\lim_{n\to\infty} x_n = a.
\end{equation*}

同理可证当\(x_1 \in (a,a+r)\)时,
数列\(\{x_n\}_{n\geq1}\)是以\(a\)为极限的严格单调减少数列.
\end{proof}
\end{proposition}
\begin{remark}
\cref{theorem:连续函数.单调迭代数列收敛定理} 的
条件“\(f\)在区间\((a-r,a+r)\)内连续”
不能弱化为“\(f\)在点\(a\)连续”,
否则无法保证在区间\((a-r,a+r)\)内只有一个不动点.
%@credit: 反例是由 {5f4d2f8a-fc8b-4798-85d6-98670f6761e7} 给出的
取\begin{equation*}
	f(x) = \left\{ \begin{array}{cc}
		\left( x + \frac12 \right)^3 - \frac12, & -1 < x < -\frac12, \\
		x^3, & -\frac12 \leq x < 1.
	\end{array} \right.
\end{equation*}
%@Mathematica: f[x_] := Piecewise[{{(x + 1/2)^3 - 1/2, -1 < x < -1/2}, {x^3, -1/2 <= x < 1}}]
%@Mathematica: Plot[{x, f[x]}, {x, -1, 1}]
%@Mathematica: Limit[f[x], x -> -1/2, Direction -> "FromBelow"]
%@Mathematica: Plot[f[x] - x, {x, -1, 1}]
%@Mathematica: Plot[f'[x] - 1, {x, -1, 1}]
点\(a=0\)是函数\(f\)的不动点,
\(f\)在点\(a\)连续,在\((-1,1)\)上严格单调增加,
并且在区间\((-1,0)\)上有\(f(x)>x\),
在区间\(0,1\)上有\(f(x)<x\).
可以看出点\(a\)是唯一的不动点.
但是由于\(f\)在点\(x=-1/2\)的左极限是\(-1/2\),
也具有类似不动点的性质:
当\(x_1 \in (-1,-1/2)\)时,
由\(x_{n+1} = f(x_n)\)迭代生成的
数列\(\{x_n\}_{n\geq1}\)的极限
就是\(-\frac12\),而不是\(a\).
\end{remark}

\begin{example}
%@see: https://www.bilibili.com/video/BV1hhtHegEnf
设数列\(\{x_n\}\)满足\(x_1=1\),且有递推公式\(x_n=x_{n+1}+2\ln(1+x_{n+1})\ (n=1,2,\dotsc)\).
\begin{itemize}
	\item 证明\(\lim_{n\to\infty} x_n\)存在,并计算它的值.
	\item 证明\(\lim_{n\to\infty} 2^n x_n\)存在,并计算它的值.
\end{itemize}
\begin{solution}
首先利用数学归纳法证明\(\{x_n\}\)有下界.
注意到\(x_1=1>0\).
假设\(x_k>0\ (k=1,2,\dotsc)\).
用反证法,假设\(x_{k+1}\leq0\),则\(\ln(1+x_{k+1})\leq0\),
从而有\(x_k=x_{k+1}+2\ln(1+x_{k+1})\leq0\),矛盾,于是\(x_{k+1}>0\).
由上可知,\(0\)是数列\(\{x_n\}\)的一个下界.
又因为\begin{equation*}
	x_n-x_{n+1}=2\ln(1+x_{n+1})>0
	\quad(n=1,2,\dotsc),
\end{equation*}
数列\(\{x_n\}\)严格单调减少,
所以由单调有界定理可知数列\(\{x_n\}\)收敛.
不妨设\(\lim_{n\to\infty} x_n = x\),
再对递推公式令\(n\to\infty\)
得\(x=x+2\ln(1+x)\),
解得\(x = \lim_{n\to\infty} x_n = 0\).

因为\begin{equation*}
	\frac{2^{n+1} x_{n+1}}{2^n x_n}
	= 2~\frac{x_{n+1}}{x_n} % 代入递推公式
	= 2~\frac{x_{n+1}}{x_{n+1}+2\ln(1+x_{n+1})} % \(\{x_n\}\)是无穷小,可以进行等价替换
	\to \frac23 < 1
	\quad(n\to\infty),
\end{equation*}
所以由\cref{theorem:无穷级数.正项级数的比值审敛法}
可知正项级数\(\sum_{n=1}^\infty 2^n x_n\)收敛,
再由\cref{theorem:无穷级数.级数收敛的必要条件}
可知\(\lim_{n\to\infty} 2^n x_n\)收敛.
\end{solution}
\end{example}

\begin{example}
%@see: https://www.bilibili.com/video/BV1cg2rY3E6L/
设数列\(\{x_n\}\)满足\(\ln x_n + \frac1{x_{n+1}} \leq 1\).
证明:\(\lim_{n\to\infty} x_n\)存在,并计算它的值.
\begin{solution}
记\(f(x) = \ln x + \frac1x\ (x>0)\),
求导得\(f'(x) = \frac1x - \frac1{x^2} = \frac{x-1}{x^2}\),
可见\(f\)在\((0,1)\)上单调减少,在\([1,+\infty)\)上单调增加,
在\(x=1\)取得最小值\(f(1) = 1\),
故对\(\forall x>0\)成立\begin{equation*}
	f(x) \geq 1.
	\eqno(1)
\end{equation*}
在(1)式中用\(x_n\)代\(x\),并由题设可知\begin{equation*}
	\ln x_n + \frac1{x_{n+1}}
	\leq 1
	\leq f(x_n)
	= \ln x_n + \frac1{x_n},
\end{equation*}
在不等号两边消去\(\ln x_n\)得\begin{equation*}
	\frac1{x_{n+1}} \leq \frac1{x_n},
	\quad\text{即}\quad
	x_n \leq x_{n+1},
\end{equation*}
数列\(\{x_n\}\)单调增加.
又因为\begin{equation*}
	\ln x_n
	\leq \ln x_n + \frac1{x_{n+1}}
	\leq 1,
\end{equation*}
所以\(x_n \leq e\),
数列\(\{x_n\}\)有上界.
由单调有界定理可知数列\(\{x_n\}\)收敛.
不妨设\(\lim_{n\to\infty} x_n = A\).
对不等式\(\ln x_n + \frac1{x_{n+1}} \leq 1\)
令\(n\to\infty\)得\begin{equation*}
	1 \leq f(A) = \ln A + \frac1A \leq 1,
\end{equation*}
于是\(f(A) = 1\),而\(A\)只能等于\(1\),
即\(\lim_{n\to\infty} x_n = 1\).
\end{solution}
\end{example}

% \begin{landscape}
% \eject \pdfpagewidth=297mm \pdfpageheight=420mm %把页面设置为A3纸大小
\section{本章总结}
现在总结一下本章介绍的解极限常用公式、方法:
\begin{itemize}
	\item 根式有理化
	\item 计算非零因子
	\item 拆分极限存在的项
	\item 提取公因子
	\item 利用等价无穷小或泰勒公式进行等价替换
	\item \hyperref[theorem:幂指函数.幂指函数的极限]{幂指函数的指数化}
	\item 换元法(如倒代换等)
	\item 洛必达法则
\end{itemize}

重要不等式(可以用于放缩法):
\begin{itemize}
	\item \(\frac{x}{1+x} < \ln(1+x) < x \quad(x>-1)\).%\cref{example:微分中值定理.拉格朗日中值定理.重要不等式1}
	\item \(1+x \leq e^x \quad(-\infty<x<+\infty)\),当且仅当\(x=0\)时取“\(=\)”号.
	\item \(\frac1{n+1} < \ln(1+\frac1n) < \frac1n\).
	\item \(\ln(1+n) < \sum_{k=1}^n \frac1{k} < 1 + \ln n\).
	%\cref{equation:微分中值定理.若尔当不等式}
	%\cref{equation:单调性.正切不等式}
	\item \(\frac2\pi x < \sin x < x < \tan x \quad(0<x<\frac\pi2)\).
\end{itemize}

重要极限公式有:
\begin{itemize}
	\item \(\lim_{n\to\infty} q^n=0\ (\abs{q}<1)\).%\cref{equation:数列极限.重要极限1}
	\item \(\lim_{n\to\infty} \sqrt[n]{n}=1\).%\cref{equation:数列极限.重要极限2}
	\item \(\lim_{n\to\infty} \sqrt[n]{k n} = 1\ (k>0)\).%\cref{equation:数列极限.重要极限3}
	\item \(\lim_{n\to\infty} \frac{1 \cdot 3 \cdot 5 \dotsm (2n-1)}{2 \cdot 4 \cdot 6 \dotsm (2n)} = 0\).%\cref{equation:数列极限.重要极限4}
	\item \(\lim_{n\to\infty} \frac{2 \cdot 4 \cdot 6 \dotsm (2n)}{1 \cdot 3 \cdot 5 \dotsm (2n+1)} = 0\).%\cref{equation:数列极限.重要极限5}
	\item \(\lim_{n\to\infty} (a_1^n + a_2^n + \dotsb + a_p^n)^{\frac1n} = \max_{1\leq i\leq p} \{a_i\}\).%\cref{equation:数列极限.重要极限6}
	\item \(\lim_{x\to0} \frac{\sin x}{x} = 1\).%\cref{equation:函数极限.重要极限1}
	\item \(\lim_{x\to0} \frac{\sin \mu x}{x}=\mu\).
	\item \(\lim_{x\to\infty} \frac{\sin x}{x} = 0\).
	\item \(\lim_{x\to\infty} \left(1+\frac1x\right)^x = e\).%\cref{equation:函数极限.重要极限2}
	\item \(\lim_{n\to\infty} \left(1+\frac{x}{n}\right)^n=e^x\ (x\in\mathbb{R})\).%\cref{equation:特殊函数.以e为底的指数函数}
	\item \(\lim_{n\to\infty} n\left(\sqrt[n]{x}-1\right)=\ln x\ (x>0)\).%\cref{equation:特殊函数.以e为底的对数函数}
	\item \(\lim_{x\to0^+} x^\alpha \ln^\beta x = 0\ (\alpha,\beta > 0)\).%\cref{example:微分中值定理.洛必达法则.零乘无穷大型2}
	\item \(\lim_{x\to+\infty} \frac{\ln^\alpha x}{x^\beta} = 0\ (\alpha,\beta>0)\).%\cref{example:微分中值定理.洛必达法则.无穷大比无穷大型1}
	\item \(\lim_{x\to+\infty} \frac{x^n}{e^{\lambda x}}=0\ (n>0,\lambda>0)\).%\cref{example:微分中值定理.洛必达法则.无穷大比无穷大型2}
	\item \(\lim_{x\to0^+} x^x = 1\).%\cref{example:微分中值定理.洛必达法则.零次方零型1}
\end{itemize}

常见的等价无穷小有:
\begin{itemize}
	\item \(\sin x%\cref{equation:函数极限.重要极限1}
		\sim \tan x%\cref{equation:函数极限.重要极限7}
		\sim \arcsin x%\cref{equation:函数极限.重要极限9}
		\sim \arctan x%\cref{equation:函数极限.重要极限10}
		\sim \ln(1+x)%\cref{equation:函数极限.重要极限12}
		\sim e^x-1%\cref{equation:函数极限.重要极限14}
		\sim x\ (x\to0)\).
	\item \(\sqrt[n]{1+x} - 1 \sim \frac1n x\ (x\to0)\).
	\item 对任意\(a\neq0\)总有\((1+x)^a-1 \sim ax\ (x\to0)\).%\cref{equation:函数极限.重要极限16}
	\item 如果\(\alpha\neq0,\alpha\to0,\alpha\beta\to0\),则\((1+\alpha)^\beta-1 \sim \alpha\beta\).%\cref{theorem:幂指函数.与幂指函数有关的等价无穷小}
	\item \(1-\cos x%\cref{equation:函数极限.重要极限8}
		\sim \sec x-1%\cref{equation:函数极限.重要极限15}
		\sim \frac12 x^2\ (x\to0)\).
	\item \(a^x-1 \sim x \ln a\ (x\to0)\).%\cref{equation:函数极限.重要极限17}
	\item \(x^x-1 \sim x \ln x\ (x\to1)\).%证明:在\cref{equation:函数极限.重要极限14} 中用\(x \ln x\)代\(x\)便得
	%@Mathematica: Series[Tan[x] - Sin[x], {x, 0, 3}]
	\item \(\tan x - \sin x \sim \frac12 x^3\ (x\to0)\).
	%@Mathematica: Series[x - Sin[x], {x, 0, 3}]
	\item \(x - \sin x \sim \frac16 x^3\ (x\to0)\).
	%@Mathematica: Series[Tan[x] - x, {x, 0, 3}]
	\item \(\tan x - x \sim \frac13 x^3\ (x\to0)\).
	%@Mathematica: Series[x - ArcTan[x], {x, 0, 3}]
	\item \(x - \arctan x \sim \frac13 x^3\ (x\to0)\).
	%@Mathematica: Series[ArcSin[x] - x, {x, 0, 3}]
	\item \(\arcsin x - x \sim \frac16 x^3\ (x\to0)\).
	%@Mathematica: Series[Exp[x] - x - 1, {x, 0, 3}]
	\item \(e^x - x - 1 \sim \frac12 x^2\ (x\to0)\).
	%@Mathematica: Series[ArcSin[x] - Sin[x], {x, 0, 3}]
	\item \(\arcsin x - \sin x \sim \frac13 x^3\ (x\to0)\).
	%@Mathematica: Series[Tan[x] - ArcTan[x], {x, 0, 3}]
	\item \(\tan x - \arctan x \sim \frac23 x^3\ (x\to0)\).
	%@Mathematica: Series[Tan[x] - ArcSin[x], {x, 0, 3}]
	\item \(\tan x - \arcsin x \sim \frac16 x^3\ (x\to0)\).
	%@Mathematica: Series[Sin[x] - ArcTan[x], {x, 0, 3}]
	\item \(\sin x - \arctan x \sim \frac16 x^3\ (x\to0)\).
	%@Mathematica: Series[x - Log[1 + x], {x, 0, 5}]
	\item \(x - \ln(1+x) \sim \frac12 x^2\ (x\to0)\).
	\item \(x - \ln(1+x) - \frac12 x^2 \sim -\frac13 x^3\ (x\to0)\).
	%@Mathematica: Series[ArcSin[x] - ArcTan[x], {x, 0, 3}]
	\item \(\arcsin x - \arctan x \sim \frac12 x^3\ (x\to0)\).
\end{itemize}

常见的等价无穷大有:
\begin{itemize}
	%\cref{equation:定积分.伽马函数的斯特林近似}
	\item \(n! \sim \sqrt{2 \pi n} \left( \frac{n}{e} \right)^n\ (n\to\infty)\).
	%@Mathematica: Limit[Sum[1/k, {k, 1, n}]/Log[n], n -> Infinity]
	\item \(1+\frac12+\frac13+\dotsb+\frac1n \sim \ln n\ (n\to\infty)\).
	\item 若\(p>-1\),则\(1^p+2^p+3^p+\dotsb+n^p \sim \frac{n^{p+1}}{p+1}\ (n\to\infty)\).%\cref{equation:数列极限.重要极限X}
	\item \(1+\frac12+\frac13+\dotsb+\frac1n \sim \ln n + \gamma\ (n\to\infty)\),其中\(\gamma\)是欧拉--马歇罗尼常数.%\cref{example:微分中值定理.拉格朗日中值定理.欧拉--马歇罗尼常数}
\end{itemize}

与连续函数有关的问题:
\begin{itemize}
	\item 函数在一点连续、左连续、右连续的定义.
	\item 函数在区间上连续的定义.
	\item \hyperref[example:连续函数.狄利克雷函数处处不连续]{处处不连续的狄利克雷函数}.
	\item 间断点的类型:\begin{enumerate}
		\item 第一类间断点:\begin{enumerate}
			\item 可去间断点
			\item 跳跃间断点
		\end{enumerate}
		\item 第二类间断点:\begin{enumerate}
			\item 无穷间断点
			\item 振荡间断点
		\end{enumerate}
	\end{enumerate}
	\item \hyperref[theorem:极限.连续函数的极限1]{连续函数的四则运算}
	\item \hyperref[theorem:极限.连续函数的极限2]{反函数的连续性}
	\item 复合函数的连续性:\cref{theorem:极限.连续函数的极限3,theorem:极限.连续函数的极限4}
	\item 闭区间上的连续函数的性质:\begin{enumerate}
		\item \hyperref[theorem:极限.最值定理]{最值定理}
		\item \hyperref[theorem:极限.零点定理]{零点定理}
		\item \hyperref[theorem:极限.闭区间上连续函数的性质.介值定理1]{介值定理}
	\end{enumerate}
	\item \hyperref[definition:极限.函数的一致连续性]{一致连续性}
\end{itemize}

% \end{landscape}


% \section{极限存在准则,两个重要极限}
\subsection{单调有界定理}

另外,易证\begin{equation*}
\lim_{x\to1^-} \left(1+\frac1x\right)^x = +\infty.
\end{equation*}

类似地,函数极限也有自己的柯西审敛原理.

% 设变量\(u\)从它的一个初值\(u_1\)变到终值\(u_2\),
终值与初值的差\(u_2 - u_1\)就叫做变量\(u\)的\DefineConcept{增量},
记作\(\increment u\),即\begin{equation*}
	\increment u = u_2 - u_1.
\end{equation*}

在实数域中,增量\(\increment u\)既可以是正的,也可以是负的.
当\(\increment u > 0\)时,变量\(u\)从初值变到终值时是增大的;
当\(\increment u < 0\)时,变量\(u\)从初值变到终值时是减小的.

应该注意到:
记号\(\increment u\)并不表示某个量\(\Delta\)与变量\(u\)的乘积,
而是一个不可分割的符号.

%\subsection{连续曲线}
%\begin{definition}
%设平面曲线\(L\)的参数方程为\begin{equation*}
%	\left\{ \begin{array}{l}
%		x = \phi(t) \\
%		y = \psi(t)
%	\end{array} \right.,
%	\quad
%	t \in [\alpha,\beta].
%\end{equation*}
%如果\(\phi(t)\)、\(\psi(t)\)在\([\alpha,\beta]\)上连续,
%则称曲线\(L\)为\DefineConcept{连续曲线}.
%点\(\opair{\phi(\alpha),\psi(\alpha)}\)称为曲线的\DefineConcept{起点},
%点\(\opair{\phi(\beta),\psi(\beta)}\)称为曲线的\DefineConcept{终点}.
%
%如果存在\(t_1\)、\(t_2\)满足\(\alpha \leq t_1 < t_2 \leq \beta\)
%且\((t_1-\alpha)^2+(t_2-\beta)^2 \neq 0\),使得对应的两点重合,
%即\(\opair{\phi(t_1),\psi(t_1)}=\opair{\phi(t_2),\psi(t_2)}\)成立,
%则称该点为曲线\(L\)的\DefineConcept{重点}.
%
%无重点的连续曲线称为\DefineConcept{若尔当曲线}或\DefineConcept{简单曲线}.
%
%仅起点和终点重合
%(即\(\opair{\phi(\alpha),\psi(\alpha)}
%=\opair{\phi(\beta),\psi(\beta)}\))
%的简单曲线称作\DefineConcept{若尔当闭曲线}或者\DefineConcept{简单闭曲线}.
%\end{definition}


\chapter{导数与微分}\label{chapter:导数}
\section{导数的基本概念}
\subsection{导数的定义}
\begin{definition}\label{definition:导数.函数在一点的可导性}
%@see: 《高等数学(第六版 上册)》 P79 定义
%@see: 《数学分析(第二版 上册)》(陈纪修) P122 定义4.1.2
设\(X\subseteq\mathbb{R}\),
函数\(f\in\mathbb{R}^X\)在点\(x_0\)的某个邻域\(U(x_0)\)内有定义,
\(x_0 + \increment x \in U(x_0)\).
如果函数增量\(\increment y = f(x_0 + \increment x) - f(x_0)\)
与\(\increment x\)之比\(\frac{\increment y}{\increment x}\)
当\(\increment x\to0\)时的极限\[
	\lim_{\increment x \to 0} \frac{\increment y}{\increment x}
\]存在,
则称“函数\(f\)在点\(x_0\)~\DefineConcept{可导}%
(\(f\) is \emph{differentiable} at \(x_0\))”
“\(f\)在点\(x_0\)具有导数”
“\(f\)在点\(x_0\)的导数存在”
“点\(x_0\)是\(f\)的\DefineConcept{可导点}”,
把这个极限称为“函数\(f\)在点\(x_0\)的\DefineConcept{导数}%
(the \emph{derivative} of \(f\) at \(x_0\))”,
记为\[
	f'(x_0), \qquad
	\eval{y'}_{x=x_0}, \qquad
	\eval{\dv{y}{x}}_{x=x_0}, \qquad
	\eval{\dv{f(x)}{x}}_{x=x_0},
	\quad\text{或}\quad
	\dv{f(x_0)}{x},
\]
即\begin{equation}
	f'(x_0)
	\defeq
	\lim_{\increment x\to0} \frac{f(x_0 + \increment x)-f(x_0)}{\increment x}.
\end{equation}

如果极限\[
	\lim_{\increment x \to 0} \frac{\increment y}{\increment x}
\]不存在,
则称“函数\(f\)在点\(x_0\)~\DefineConcept{不可导}”
“点\(x_0\)是\(f\)的\DefineConcept{不可导点}”.
%@see: https://mathworld.wolfram.com/Derivative.html
\end{definition}

\begin{example}\label{example:导数.导数定义式的变形}
%@see: 《数学分析(第二版 上册)》(陈纪修) P131 习题 1.
设函数\(f\)在点\(x_0\)可导.
计算极限\begin{itemize}
	\item \(\lim_{h\to0} \frac{f(x_0 - h) - f(x_0)}{h}\);
	\item \(\lim_{x \to x_0} \frac{f(x) - f(x_0)}{x - x_0}\);
	\item \(\lim_{h\to0} \frac{f(x_0+h) - f(x_0-h)}{h}\).
\end{itemize}
\begin{solution}
直接计算得\begin{gather*}
	\lim_{h\to0} \frac{f(x_0 - h) - f(x_0)}{h}
	= - \lim_{h\to0} \frac{f(x_0 + (-h)) - f(x_0)}{(-h)}
	= - f'(x_0), \\
	\lim_{x \to x_0} \frac{f(x) - f(x_0)}{x - x_0}
	\xlongequal{t = x - x_0}
	\lim_{t \to 0} \frac{f(x_0 + (x - x_0)) - f(x_0)}{t}
	= f'(x_0), \\
	\lim_{h\to0} \frac{f(x_0+h) - f(x_0-h)}{h}
	= \lim_{h\to0} \frac{f(x_0+h) - f(x_0)}{h}
	- \lim_{h\to0} \frac{f(x_0-h) - f(x_0)}{h}
	= 2 f'(x_0).
\end{gather*}
\end{solution}
\end{example}

\begin{proposition}
设\(f\colon\mathbb{R}\to\mathbb{R}\)在点\(x\)可导,
\(\lambda>0\)是常数,
则\[
	\lim_{h\to0} \frac{f(x+\lambda h)-f(x)}{h}
	= \lambda f'(x).
\]
\begin{proof}
直接计算得\begin{align*}
	\lim_{h\to0} \frac{f(x+\lambda h)-f(x)}{h}
	&= \lambda \lim_{h\to0} \frac{f(x+\lambda h)-f(x)}{\lambda h} \\
	&\xlongequal{t=\lambda h}
		\lambda \lim_{t\to0} \frac{f(x+t)-f(x)}{t}
	= \lambda f'(x).
	\qedhere
\end{align*}
\end{proof}
\end{proposition}

\subsection{单侧导数和狄尼导数的定义}
\begin{definition}
%@see: 《数学分析教程(第3版 上册)》(史济怀) P125 定义3.1.2
设\(X\subseteq\mathbb{R}\),
函数\(f\in\mathbb{R}^X\).
\begin{itemize}
	\item 如果\(f\)在点\(x_0\)的某个左邻域内有定义,且极限\[
		\lim_{h\to0^-} \frac{f(x_0+h)-f(x_0)}{h}
	\]存在且有限,
	那么把这个极限称为
	“函数\(f\)在点\(x_0\)的\DefineConcept{左导数}(left-sided derivative)”,
	记作\(f'_-(x_0)\).
	\item 如果\(f\)在点\(x_0\)的某个右邻域内有定义,且极限\[
		\lim_{h\to0^+} \frac{f(x_0+h)-f(x_0)}{h}
	\]存在且有限,
	那么把这个极限称为
	“函数\(f\)在点\(x_0\)的\DefineConcept{右导数}(right-sided derivative)”,
	记作\(f'_+(x_0)\).
\end{itemize}
左导数和右导数统称\DefineConcept{单侧导数}(one-sided derivative).
\end{definition}

\begin{definition}
设函数\(f\colon D\to\mathbb{R}\)在点\(x_0\)的某个去心邻域中有定义.
\begin{itemize}
	\item 把\[
		\lim_{\delta\to0^+} \sup_{0<x-x_0<\delta} f(x)
	\]称为“函数\(f\)在点\(x_0\)的\DefineConcept{右上导数}(upper right Dini derivative)”.
	\item 把\[
		\lim_{\delta\to0^+} \inf_{0<x-x_0<\delta} f(x)
	\]称为“函数\(f\)在点\(x_0\)的\DefineConcept{右下导数}(lower right Dini derivative)”.
	\item 把\[
		\lim_{\delta\to0^+} \sup_{-\delta<x-x_0<0} f(x)
	\]称为“函数\(f\)在点\(x_0\)的\DefineConcept{左上导数}(upper left Dini derivative)”.
	\item 把\[
		\lim_{\delta\to0^+} \inf_{-\delta<x-x_0<0} f(x)
	\]称为“函数\(f\)在点\(x_0\)的\DefineConcept{左下导数}(lower left Dini derivative)”.
\end{itemize}
右上导数、右下导数、左上导数和左下导数统称为\DefineConcept{狄尼导数}(Dini derivative).
%@see: https://mathworld.wolfram.com/DiniDerivative.html
%@see: https://people.math.sc.edu/schep/diffmonotone.pdf
\end{definition}

\begin{theorem}[导数存在的充分必要条件]\label{theorem:导数.函数在一点的可导性及其单侧可导性的关系}
%@see: 《数学分析教程(第3版 上册)》(史济怀) P125
函数\(f\)在点\(x_0\)可导的充分必要条件是:
其左导数\(f'_-(x_0)\)和右导数\(f'_+(x_0)\)都存在且相等.
%TODO proof
%\cref{theorem:函数极限.极限与单侧极限的关系1}
\end{theorem}

\begin{definition}\label{definition:导数.函数在开区间内可导}
%@see: 《数学分析教程(第3版 上册)》(史济怀) P127 定义3.1.3
如果函数\(f\colon(a,b)\to\mathbb{R}\)在开区间\((a,b)\)内的每一个点可导,
就称“函数\(f\)在开区间\((a,b)\)内可导”.
\end{definition}

\begin{definition}\label{definition:导数.函数在闭区间上可导}
%@see: 《数学分析教程(第3版 上册)》(史济怀) P127 定义3.1.3
如果函数\(f\colon[a,b]\to\mathbb{R}\)在开区间\((a,b)\)内可导,
且在点\(a\)的右导数\(f'_+(a)\)及在点\(b\)的左导数\(f'_-(b)\)都存在,
就说“函数\(f\)在闭区间\([a,b]\)上可导”.
\end{definition}

类似地,可以定义\(f\)在\([a,b)\)与\((a,b]\)上可导.

\begin{example}
%@see: 《2016年全国硕士研究生入学统一考试(数学一)》一选择题/第4题
设函数\[
	f(x) = \left\{ \begin{array}{cl}
		x, & x\leq0, \\
		1/n, & 1/(n+1)<x\leq1/n,\,n=1,2,\dotsc.
	\end{array} \right.
\]
考察函数\(f\)在点\(x=0\)的连续性和可导性.
\begin{solution}
显然\(f\)在点\(x=0\)左连续.
对于任意\(\epsilon>0\),存在正整数\(n_0\)满足\(\frac1{n_0}<\epsilon\),
取\(\delta=\frac1{n_0}\),
当\(0 \leq x < \delta\)时,
成立\[
	\abs{f(x) - f(0)}
	= f(x)
	\leq \frac1{n_0}
	< \epsilon,
\]
这就说明函数\(f\)在点\(x=0\)右连续.
由\cref{theorem:极限.函数在一点的连续性及其单侧连续性的关系} 可知\(f\)在点\(x=0\)连续.

显然\(f\)在点\(x=0\)的左导数存在:\[
	f'_-(0) = \lim_{x\to0^-} \frac{f(x) - f(0)}{x - 0}
	= \lim_{x\to0^-} \frac{x}{x}
	= 1.
\]
当\(\frac1{n+1} < x \leq \frac1n\)时,
有\[
	f(x) = \frac1n,
	\qquad
	n \leq \frac1x < n+1,
\]
从而有\[
	1 \leq \frac{f(x)}{x} < \frac{n+1}{n}.
\]
因为当\(n\to\infty\)时有\(x\to0^+\)和\(\frac{n+1}{n}\to1\),
所以由\cref{theorem:数列极限.夹逼准则} 可得\[
	\frac{f(x)}{x} \to 1
	\quad(n\to\infty).
\]
于是\(f\)在点\(x=0\)的右导数为\[
	f'_+(0) = \lim_{x\to0^+} \frac{f(x) - f(0)}{x - 0}
	= \lim_{x\to0^+} \frac{f(x)}{x}
	= 1.
\]
由\cref{theorem:导数.函数在一点的可导性及其单侧可导性的关系} 可知\(f\)在点\(x=0\)可导.
\end{solution}
%@Mathematica: Plot[Piecewise[{{1/Floor[1/x], 1/(Floor[1/x] + 1) < x <= 1/Floor[1/x] && x > 0}, {x, x <= 0}}], {x, -.5, .5}]
\end{example}

\subsection{导函数的定义}
设函数\(f\colon I\to\mathbb{R}\)在区间\(I\)内的每一个点可导,
那么,对于\(\forall x_0 \in I\),
都对应着\(f(x_0)\)的一个确定的导数值\(f'(x_0)\).
这样就构成一个新的函数.

\begin{definition}
设函数\(f\colon I\to\mathbb{R}\)在区间\(I\)内的每一个点可导,
则\(I\)中每一点\(x_0\)与其相应的\(f\)在点\(x_0\)的导数\(f'(x_0)\)的关系\[
	\Set{ (x_0,f'(x_0)) \given x_0 \in I }
\]称为“函数\(f\)的\DefineConcept{导函数}(derivative function)”,
简称\DefineConcept{导数},
记作\(f'\)或\(\dv{f}{x}\)或\(\dv{x} f\).
\end{definition}

\begin{definition}\label{definition:函数族.可导函数族}
由区间\(I\)上全部的可导函数组成的集合,称作\DefineConcept{可导函数族},
记作\(D(I)\)\footnote{当\(I=(a,b)\)时,可将\(D(I)\)改写为\(D(a,b)\).
当\(I=[a,b]\)时,可将\(D(I)\)改写为\(D[a,b]\).
以此类推.},
即\[
	D(I)
	\defeq
	\Set*{
		f\in\mathbb{R}^I
		\given
		(\forall x \in I)
		[\text{\(f\)在点\(x\)可导}]
	}.
\]
\end{definition}

\begin{example}%\label{example:导数.常数函数的导数}
%@see: 《高等数学(第六版 上册)》 P81 例1
求函数\(f(x) = C\)的导数,其中\(C\)为常数.
\begin{solution}
\(f'(x)
= \lim_{h\to0} \frac{f(x+h)-f(x)}{h}
= \lim_{h\to0} \frac{C-C}{h}
= 0\).
\end{solution}
\end{example}

\begin{example}%\label{example:导数.幂函数的导数}
%@see: 《高等数学(第六版 上册)》 P81 例2
求函数\(f(x) = x^n\)在\(x=a\)处的导数,
其中\(a\in\mathbb{R},
n\in\mathbb{N}^+\).
\begin{solution}
\(f'(a)
= \lim_{x \to a} \frac{x^n-a^n}{x-a}
= \lim_{x \to a} (x^{n-1}+ax^{n-2}+\dotsb+a^{n-1})
= na^{n-1}\).
\end{solution}
\end{example}

更一般地,对于幂函数\(y=x^{\mu}\ (\mu\in\mathbb{R})\),
有\begin{equation}
	(x^{\mu})' = \mu x^{\mu-1}.
\end{equation}

\begin{example}%\label{example:导数.正弦函数的导数}
%@see: 《高等数学(第六版 上册)》 P81 例3
求函数\(f(x) = \sin x\)的导数.
\begin{solution}
\(f'(x) = \lim_{h\to0} \frac{\sin(x+h)-\sin x}{h}
= \lim_{h\to0} \cos(x+\frac{h}{2}) \frac{\sin(h/2)}{h/2}
= \cos x\).
\end{solution}
\end{example}

\begin{example}%\label{example:导数.余弦函数的导数}
求函数\(f(x) = \cos x\)的导数.
\begin{solution}
\(f'(x) = \lim_{h\to0} \frac{\cos(x+h)-\cos x}{h}
= - \lim_{h\to0} \sin\left(x+\frac{h}2\right) \frac{\sin(h/2)}{h/2}
= - \sin x\).
\end{solution}
\end{example}

\begin{example}%\label{example:导数.指数函数的导数}
%@see: 《高等数学(第六版 上册)》 P82 例4
求函数\(f(x) = a^x\ (a>0,a\neq1)\)的导数.
\begin{solution}
\(f'(x)
= \lim_{h\to0}\frac{a^{x+h}-a^x}{h}
= a^x \lim_{h\to0}\frac{a^h-1}{h}
= a^x \ln a\).
\end{solution}
\end{example}
\begin{remark}
特别地,当\(a=e\)时,因\(\ln e = 1\),故有\[
	(e^x)' = e^x.
\]
\end{remark}

\begin{example}%\label{example:导数.对数函数的导数}
%@see: 《高等数学(第六版 上册)》 P82 例5
求函数\(f(x) = \log_a x\ (a>0,a\neq1)\)的导数.
\begin{solution}
\(f'(x)
= \lim_{h\to0}\frac{\log_a(x+h)-\log_a x}{h}
= \lim_{h\to0}{\frac{1}{h} \log_a\frac{x+h}{x}}
= \frac{1}{x} \lim_{h\to0}\frac{\log_a(1+h/x)}{h/x}
= \frac{1}{x \ln a}\).
\end{solution}
\end{example}

\begin{example}
%@see: 《高等数学(第六版 上册)》 P82 例6
求函数\(f(x) = \abs{x}\)在\(x=0\)处的导数.
\begin{solution}
\(\frac{f(0+h)-f(0)}{h} = \frac{\abs{h}-0}{h} = \frac{\abs{h}}{h}\).

当\(h < 0\)时,\(\frac{\abs{h}}{h} = -1\),
故\(\lim_{h\to0^-}\frac{f(0+h)-f(0)}{h}
= \lim_{h\to0^-}\frac{\abs{h}}{h} = -1\).

当\(h > 0\)时,\(\frac{\abs{h}}{h} = 1\),
故\(\lim_{h\to0^+}\frac{f(0+h)-f(0)}{h}
= \lim_{h\to0^+}\frac{\abs{h}}{h} = 1\).

综上,\(\lim_{h\to0}\frac{f(0+h)-f(0)}{h}\)不存在,即函数\(f(x) = \abs{x}\)在\(x = 0\)处不可导.
\end{solution}
\end{example}

\subsection{导数的几何意义}
曲线\(y=f(x)\)在点\(M(x_0,y_0)\)处的\DefineConcept{切线方程}为\[
	y-y_0=f'(x_0)(x-x_0).
\]

过切点\(M(x_0,y_0)\)且与切线垂直的直线叫做曲线\(y=f(x)\)在点\(M\)处的\DefineConcept{法线}.
如果\(f'(x_0) \neq 0\),则法线的斜率为\(-\frac{1}{f'(x_0)}\),从而法线方程为\[
	y-y_0=-\frac{1}{f'(x_0)}(x-x_0);
\]
而如果\(f'(x_0) = 0\),则法线方程为\(x = x_0\).

\begin{proposition}\label{theorem:导数与微分.导函数的奇偶性}
设\(R>0\),\(f'\)是函数\(f\colon(-R,R)\to\mathbb{R}\)的导函数.
\begin{itemize}
	\item 如果\(f\)是奇函数,那么\(f'\)是偶函数.
	\item 如果\(f\)是偶函数,那么\(f'\)是奇函数.
\end{itemize}
\begin{proof}
由导数的定义有\begin{equation*}
	f'(-x)
	= \lim_{h\to0} \frac{f(-x+h)-f(-x)}{h}.
	\eqno(1)
\end{equation*}
由\cref{example:导数.导数定义式的变形} 有\begin{equation*}
	-f'(x)
	= \lim_{h\to0} \frac{f(x-h)-f(x)}{h}.
	\eqno(2)
\end{equation*}

假设\(f\)是奇函数,
即\begin{equation*}
	(\forall x)
	[
		-R < x < R
		\implies
		f(-x) = -f(x)
	],
\end{equation*}
那么\(f(-x+h) = -f(x-h)\),
再由(1)(2)两式可知\begin{equation*}
	f'(-x)
	= \lim_{h\to0} \frac{-f(x-h)+f(x)}{h}
	= -\lim_{h\to0} \frac{f(x-h)-f(x)}{h}
	= f'(x).
\end{equation*}
这就说明\(f'\)是偶函数.

假设\(f\)是偶函数,
即\begin{equation*}
	(\forall x)
	[
		-R < x < R
		\implies
		f(-x) = f(x)
	],
\end{equation*}
那么\(f(-x+h) = f(x-h)\),
再由(1)(2)两式可知\[
	f'(-x)
	= \lim_{h\to0} \frac{f(x-h)-f(x)}{h}
	= -f'(x).
\]
这就说明\(f'\)是奇函数.
\end{proof}
\end{proposition}
\begin{proposition}\label{theorem:导数与微分.导函数的周期性}%周期函数的导数也是周期函数且周期相同
以\(T\)为周期的可导函数\(f\colon\mathbb{R}\to\mathbb{R}\)的导函数也是以\(T\)为周期的函数.
\begin{proof}
%@see: https://www.bilibili.com/video/BV1sg4y1i7Ak/
由给定条件有\[
	f(x+T) = f(x),
\]
那么对于\(\forall h\in\mathbb{R}\)成立\[
	f(x+T+h) = f(x+h),
\]
于是\begin{equation*}
	f'(x+T)
	= \lim_{h\to0} \frac{f(x+T+h) - f(x+T)}{h}
	= \lim_{h\to0} \frac{f(x+h) - f(x)}{h}
	= f'(x).
	\qedhere
\end{equation*}
\end{proof}
%\cref{theorem:定积分.周期函数的积分}
\end{proposition}

\subsection{函数可导性与连续性的关系}
\begin{theorem}\label{theorem:导数与微分.函数可导性与连续性的关系}
%@see: 《数学分析教程(第3版 上册)》(史济怀) P125 定理3.1.1
如果函数\(f\)在点\(x_0\)可导,
则\(f\)必定在点\(x_0\)连续.
\begin{proof}
记\(f\)在点\(x_0\)的导数为\(f'(x_0)\),
于是由\begin{align*}
	\lim_{x \to x_0} (f(x) - f(x_0))
	&= \lim_{x \to x_0} \frac{f(x) - f(x_0)}{x - x_0} \cdot (x - x_0) \\
	&= \lim_{x \to x_0} \frac{f(x) - f(x_0)}{x - x_0} \cdot \lim_{x \to x_0} (x - x_0) \\
	&= f'(x_0) \cdot 0
	= 0,
\end{align*}
可知\(\lim_{x \to x_0} f(x) = f(x_0)\),
这就说明\(f\)在点\(x_0\)连续.
\end{proof}
\end{theorem}
借用\cref{definition:函数族.连续函数族} 和\cref{definition:函数族.可导函数族} 的记号,
可以将\cref{theorem:导数与微分.函数可导性与连续性的关系} 描述为:\begin{equation*}
	% f \in D(I) \implies f \in C(I)
	D(I) \subseteq C(I).
\end{equation*}

\begin{example}
设函数\(f\)在点\(x=0\)可导,
且\(f\left( \frac1n \right) = \frac2n\ (n=1,2,\dotsc)\).
求\(f'(0)\).
\begin{solution}
由\cref{theorem:导数与微分.函数可导性与连续性的关系} 可知\(f\)在点\(x=0\)连续,
而\(\lim_{x\to0^+} f(x)
= \lim_{n\to\infty} f\left( \frac1n \right)
= 0\),
%\cref{definition:极限.函数在一点的连续性}
所以\(f(0) = 0\),
%\cref{definition:导数.函数在一点的可导性}
%\cref{theorem:极限.海涅定理}
那么\begin{equation*}
	f'(0) = \lim_{x\to0} \frac{f(x) - f(0)}{x - 0}
	= \lim_{n\to\infty} \frac{f(1/n)}{1/n}
	= \lim_{n\to\infty} \frac{2/n}{1/n}
	= 2.
\end{equation*}
\end{solution}
\end{example}

\begin{example}\label{example:导数.函数在一点的导数是无穷大}
函数\(y=f(x)=\sqrt[3]x\)
在区间\((-\infty,+\infty)\)内连续,
但是在点\(x=0\)处不可导.
这是因为在点\(x=0\)处有\[
	\frac{f(0+h)-f(0)}{h}
	=\frac{\sqrt[3]{h}-0}{h}
	=\frac{1}{h^{2/3}}>0,
\]
因而,
\(\lim_{h\to0} \frac{f(0+h)-f(0)}{h}
=\lim_{h\to0} \frac{1}{h^{2/3}}
=\infty\),
即导数为无穷大(导数不存在).
这事实在图形中表现为:
曲线\(y=\sqrt[3]x\)在原点具有垂直于\(x\)轴的切线\(x=0\).
\end{example}

\begin{example}\label{example:导数.函数在一点的左右导数不相等}
函数\(y=\sqrt{x^2}\)(即\(y=\abs{x}\))
在\((-\infty,+\infty)\)内连续,
但是在\(x=0\)处不可导,
曲线\(y=\sqrt{x^2}\)在原点没有切线.
这是因为\[
	\lim_{x\to0^+} \frac{f(x)-f(0)}{x-0}
	= \lim_{x\to0^+} \frac{x-0}{x-0}
	= 1,
\]
而\[
	\lim_{x\to0^-} \frac{f(x)-f(0)}{x-0}
	= \lim_{x\to0^-} \frac{(-x)-0}{x-0}
	= -1.
\]
\end{example}

\begin{theorem}
“函数\(f\)在点\(x_0\)连续”是“\(f\)在点\(x_0\)可导”的必要不充分条件.
\begin{proof}
由\cref{theorem:导数与微分.函数可导性与连续性的关系} 已知:
如果函数\(f\)在点\(x_0\)可导,则\(f\)必定在点\(x_0\)连续.
由\cref{example:导数.函数在一点的导数是无穷大,example:导数.函数在一点的左右导数不相等} 可知:
纵使函数\(f\)在点\(x_0\)连续,\(f\)在点\(x_0\)也可能不可导.
\end{proof}
\end{theorem}

\begin{example}\label{example:连续函数.狄利克雷函数改2只在一点可导}
%@see: https://www.bilibili.com/video/BV1Rr421M7T8/
证明:函数\(f(x) = x^2 D(x)\)在点\(x=0\)可导,在其他各点既不可导也不连续.
\begin{proof}
由定义有\[
	f'(0) = \lim_{x\to0} \frac{f(x) - f(0)}{x - 0}
	= \lim_{x\to0} \frac{x^2 D(x)}{x}
	= \lim_{x\to0} x D(x)
	= 0,%\cref{example:连续函数.狄利克雷函数改1只在一点连续}
\]
函数\(f\)在点\(x=0\)可导.

当\(x\neq0\)时,有\(D(x) = f(x) / x^2\).
用反证法.
假设\(f\)在点\(x_0\neq0\)连续,那么\[
	\lim_{x \to x_0} D(x)
	= \lim_{x \to x_0} \frac{f(x)}{x^2}
	% 连续性、极限的四则运算
	= \frac{f(x_0)}{x_0^2}
	= D(x_0),
\]
即\(D\)在点\(x_0\)连续,
这与\cref{example:连续函数.狄利克雷函数处处不连续} 的结论矛盾,
从而说明\(f\)在任意一点\(x_0\neq0\)不连续,自然也不可导.
\end{proof}
\end{example}
\begin{remark}
\cref{example:连续函数.狄利克雷函数改2只在一点可导} 说明:
单靠“函数\(f\)在点\(x_0\)可导”这个条件,
既不能推出“\(f\)在点\(x_0\)的邻域可导”,
也不能推出“\(f\)在点\(x_0\)的邻域连续”.
%@credit: {de3029b8-10a6-4ae5-8f64-108dae1c10a9}
由于函数\(f\)在非零点处处不可导,
所以\(f\)不具有二阶导数.
同理可知,对于任意整数\(n\geq3\),
函数\(x \mapsto x^n D(x)\)
只有一阶导数,没有二阶以上导数.
因此得出结论:
即便有\(f(x) = o(x^n)\ (x\to0)\)成立,
也不能推出“\(f\)在点\(x=0\)具有\(n\)阶导数”.
\end{remark}

\begin{example}
\DefineConcept{魏尔斯特拉斯函数}\[
	W(x) = \lim_{n\to\infty} \sum_{k=0}^n a^k \cos(b^k \pi x),
\]在定义域上处处连续而又处处不可导,
其中,参数\(a\)和\(b\)满足\[
	0<a<1,
	\qquad
	b\in\Set{ p \given p = 2q+1, q\in\mathbb{N} },
	\qquad
	ab > 1+\frac{3}{2}\pi.
\]
%TODO proof
\end{example}

\begin{example}
举例说明:可导函数\(f\)满足\(\lim_{x\to+\infty} f(x) < \infty\)和\(\lim_{x\to+\infty} f'(x) = 0\).
\begin{solution}
取\(f(x) = \frac{\sin x}{x}\),
则\(\lim_{x\to+\infty} f(x) = \lim_{x\to+\infty} f'(x) = 0\).
\end{solution}
%Mathematica: Limit[Sin[x]/x, x -> +Infinity]
%Mathematica: Limit[D[Sin[x]/x, x], x -> +Infinity]
\end{example}
\begin{example}
%@see: https://www.bilibili.com/video/BV1My411q7aE/
举例说明:可导函数\(f\)满足\(\lim_{x\to+\infty} f(x) < \infty\),但是不满足\(\lim_{x\to+\infty} f'(x) = 0\).
\begin{solution}
取\(f(x) = \frac{\sin x^2}{x}\),
则\(\lim_{x\to+\infty} f(x) = 0\),
而\[
	f'(x) = 2 \cos x^2 - \frac{\sin x^2}{x^2}.
\]
当\(x\to+\infty\)时,虽然\(\frac{\sin x^2}{x^2} \to 0\),
但是\(\cos x^2\)的极限不存在,
也就是说\(\lim_{x\to+\infty} f'(x)\)不存在.
\end{solution}
%Mathematica: Limit[Sin[x^2]/x, x -> +Infinity]
%Mathematica: D[Sin[x^2]/x, x]
\end{example}

% 考研数学经常考察抽象函数在点\(x=0\)的连续性、可导性
\begin{example}
设函数\(f\)在点\(x=0\)的某个邻域内连续,
函数\(g\)是当\(x\to0\)时的无穷小,
且\(f\)是\(g\)的同阶无穷小或高阶无穷小.
要使\(f\)在点\(x=0\)可导,
\(g\)应该满足什么条件?
\begin{solution}
由于\(f\)是无穷小,
% 即\(\lim_{x\to0} f(x) = 0\),
且\(f\)在点\(x=0\)连续,
% 即\(f(0) = \lim_{x\to0} f(x)\),
所以\(f(0) = 0\).
根据导数的定义,有\[
	f'(0)
	= \lim_{x\to0} \frac{f(x) - f(0)}{x-0} % 函数\(f\)在点\(x=0\)的导数的定义
	= \lim_{x\to0} \frac{f(x)}{x} % 代入\(f\)在点\(x=0\)的函数值\(f(0) = 0\)
	= \lim_{x\to0} \frac{f(x)}{g(x)} \cdot \frac{g(x)}{x}. % 极限的四则运算法则
\]
由此可见,要使\(f\)在点\(x=0\)可导,
只需要极限\(\lim_{x\to0} \frac{g(x)}{x}\)存在且有限,
即\(g\)是\(x\)的同阶无穷小或高阶无穷小.
\end{solution}
\end{example}
\begin{example}
%@see: 《2020年全国硕士研究生入学统一考试(数学一)》一选择题/第2题
设函数\(f\)在点\(x=0\)可导,
且\(\lim_{x\to0} f(x) = 0\).
证明:\(\lim_{x\to0} \frac{f(x)}{\sqrt{\abs{x}}}\)存在且有限.
\begin{proof}
因为函数\(f\)在点\(x=0\)可导,
所以\(f\)在点\(x=0\)连续,
从而\(f(0) = \lim_{x\to0} f(x) = 0\),
于是函数\(f\)在点\(x=0\)的导数为\begin{equation*}
	f'(0)
	= \lim_{x\to0} \frac{f(x) - f(0)}{x - 0}
	= \lim_{x\to0} \frac{f(x)}{x},
\end{equation*}
所以\begin{gather*}
	\lim_{x\to0^+} \frac{f(x)}{\sqrt{\abs{x}}}
	% 当\(x>0\)时,\(\abs{x} = x\)
	= \lim_{x\to0^+} \frac{f(x)}{\sqrt{x}}
	% 当\(x>0\)时,\(\sqrt{x}^2 = x\)
	= \lim_{x\to0^+} \frac{f(x)}{x} \cdot \sqrt{x}
	%\cref{theorem:极限.极限的四则运算法则}
	= \lim_{x\to0^+} \frac{f(x)}{x} \cdot \lim_{x\to0^+} \sqrt{x}
	= 0, \\
	\lim_{x\to0^-} \frac{f(x)}{\sqrt{\abs{x}}}
	% 当\(x<0\)时,\(\abs{x} = -x\)
	= \lim_{x\to0^-} \frac{f(x)}{\sqrt{-x}}
	% 当\(x<0\)时,\(\sqrt{-x}^2 = -x\)
	= \lim_{x\to0^-} \frac{f(x)}{-x} \cdot \sqrt{-x}
	%\cref{theorem:极限.极限的四则运算法则}
	= \lim_{x\to0^-} \frac{f(x)}{-x} \cdot \lim_{x\to0^-} \sqrt{-x}
	= 0,
\end{gather*}
于是由\cref{theorem:函数极限.极限与单侧极限的关系1} 可知
\(\lim_{x\to0} \frac{f(x)}{\sqrt{\abs{x}}} = 0\).
\end{proof}
\end{example}
\begin{example}
%@see: 《2024年全国硕士研究生入学统一考试(数学一)》一选择题/第4题/选项(A)
%@see: 《2024年全国硕士研究生入学统一考试(数学一)》一选择题/第4题/选项(B)
函数\(f\)在区间\((-1,1)\)上有定义,且\(\lim_{x\to0} f(x) = 0\).
举例说明:即便成立\(\lim_{x\to0} \frac{f(x)}{x} = m\)和\(\lim_{x\to0} f'(x) = m\),也不成立\(f'(0) = m\).
\begin{solution}
取\begin{equation*}
	f(x) = \left\{ \begin{array}{cl}
		mx, & x\neq0, \\
		1, & x=0.
	\end{array} \right.
\end{equation*}
易见\(f\)在点\(x=0\)不连续,遑论可导,即\(f'(0)\)不存在.
但是当\(x\neq0\)时成立\(f'(x) = m\),
从而有\(\lim_{x\to0} \frac{f(x)}{x} = m\)和\(\lim_{x\to0} f'(x) = m\).
\end{solution}
\end{example}
\begin{example}
%@see: 《2024年全国硕士研究生入学统一考试(数学一)》一选择题/第4题/选项(D)
函数\(f\)在区间\((-1,1)\)上有定义,且\(\lim_{x\to0} f(x) = 0\).
举例说明:即便成立\(f'(0) = m\),但是不成立\(\lim_{x\to0} f'(x) = m\).
\begin{solution}
取\begin{equation*}
	f(x) = \left\{ \begin{array}{cl}
		mx + x^2 \sin\frac1x, & x\neq0, \\
		0, & x=0.
	\end{array} \right.
\end{equation*}
那么\(f'(0) = \lim_{x\to0} \frac{f(x) - f(0)}{x - 0}
= \lim_{x\to0} \frac{mx + x^2 \sin\frac1x}{x}
= \lim_{x\to0} \left( x + x \sin\frac1x \right)
= m\).
但是当\(x\neq0\)时,
\(f'(x) = m + 2x \sin\frac1x - \cos\frac1x\),
点\(x=0\)是它的一个振荡间断点.
\end{solution}
\end{example}
\begin{remark}
函数\(f\)在点\(x_0\)可导,
无法说明它的导函数\(f'\)在点\(x_0\)的极限\(\lim_{x \to x_0} f'(x)\)存在,
更不能说明\(f'\)在点\(x_0\)连续.
\end{remark}

\subsection{导函数的连续性与间断点}
\begin{definition}
设函数\(f\)在点\(x_0\)的某一邻域内可导.
如果\(f\)的导函数\(f'\)在点\(x_0\)连续,
则称“函数\(f\)在点\(x_0\)~\DefineConcept{连续可导}(continuously differentiable)”
%@see: https://mathworld.wolfram.com/ContinuouslyDifferentiableFunction.html
“函数\(f\)在点\(x_0\)具有\DefineConcept{连续导数}(continuous derivative)”.
\end{definition}

%@see: https://zhuanlan.zhihu.com/p/666265696
让我们首先介绍几个例子,回顾导函数可能具有的间断点的类型.

\begin{example}
对函数\[
	f(x) = \left\{ \begin{array}{cl}
		x+1, & x>0, \\
		x-1, & x\leq0
	\end{array} \right.
\]求导得\[
	f'(x) = 1
	\quad(x\neq0).
\]
显然点\(x=0\)是导函数\(f'\)的可去间断点.
\end{example}

\begin{example}
对函数\[
	f(x) = \abs{x}
\]求导得\[
	f'(x) = \left\{ \begin{array}{cl}
		1, & x>0, \\
		-1, & x<0.
	\end{array} \right.
\]
显然点\(x=0\)是导函数\(f'\)的跳跃间断点.
\end{example}

\begin{example}
对函数\[
	f(x) = \frac1x
\]求导得\[
	f'(x) = -\frac1{x^2}.
\]
显然点\(x=0\)是导函数\(f'\)的无穷间断点.
\end{example}

\begin{example}
对函数\[
	f(x) = \left\{ \begin{array}{cl}
		x^2 \sin\frac1x, & x \neq 0, \\
		0, & x = 0
	\end{array} \right.
\]求导得\[
	f'(x) = \left\{ \begin{array}{cl}
		2x \sin\frac1x - \cos\frac1x, & x \neq 0, \\
		0, & x = 0.
	\end{array} \right.
\]
这就是说,函数\(f\)在\((-\infty,+\infty)\)内处处可导.
但是由于点\(x=0\)是函数\(x \mapsto \cos\frac1x\)的振荡间断点,
而\(\lim_{x\to0} x \sin\frac1x = 0\),
所以点\(x=0\)是导函数\(f'\)的振荡间断点,
自然\(f'\)在点\(x=0\)的极限不存在.
%@Mathematica: Plot[Piecewise[{{x^2 Sin[1/x], x != 0}, {0, x == 0}}], {x, -.5, .5}, PlotRange -> {-.1, .1}]
%@Mathematica: Plot[Piecewise[{{2 x Sin[1/x] - Cos[1/x], x != 0}, {0, x == 0}}], {x, -.5, .5}]
\end{example}

以上四个例子说明:函数的不可导点可能是它的导函数的任意一种类型的间断点.

\section{函数的求导法则}
\subsection{函数的和、差、积、商的求导法则}
\begin{theorem}
%@see: 《高等数学(第六版 上册)》 P88 定理1
%@see: 《数学分析(第二版 上册)》(陈纪修) P134 定理4.3.1
%@see: 《数学分析(第二版 上册)》(陈纪修) P135 定理4.3.2
%@see: 《数学分析(第二版 上册)》(陈纪修) P136 定理4.3.3
%@see: 《数学分析(第二版 上册)》(陈纪修) P137 推论
如果函数\(u\)和\(v\)都在点\(x\)具有导数,
那么它们的和、差、积、商(除分母为零的点外)都在点\(x\)具有导数,
且\begin{itemize}
	\item \((u \pm v)' = u' \pm v'\);
	\item \((uv)' = u'v + uv'\);
	\item \(\left(\frac{u}{v}\right)' = \frac{u'v - uv'}{v^2}\ (v \neq 0)\).
\end{itemize}
\begin{proof}
显然有
\begin{itemize}
\item 函数\(u\)和\(v\)都在\(x\)可导,
也就是说\(u'(x)\)和\(v'(x)\)都存在,
所以利用\cref{theorem:极限.极限的四则运算法则} 便得
\begin{align*}
	&[u(x) \pm v(x)]'
	=\lim_{\increment x\to0}
		\frac{[u(x+\increment x) \pm v(x+\increment x)]-[u(x) \pm v(x)]}{\increment x} \\
	&=\lim_{\increment x\to0} \frac{u(x+\increment x)-u(x)}{\increment x}
		\pm \lim_{\increment x\to0} \frac{v(x+\increment x)-v(x)}{\increment x} \\
	&=u'(x) \pm v'(x).
\end{align*}

\item 因为\(v\)在点\(x\)可导,所以\(v\)在点\(x\)连续,于是\begin{equation*}
	\lim_{\increment x\to0} v(x+\increment x)
	= v\left(x+\lim_{\increment x\to0} \increment x\right)
	= v(x).
\end{equation*}
因此
\begin{align*}
	&[u(x) v(x)]'
	= \lim_{\increment x\to0}
		\frac{u(x+\increment x) v(x+\increment x) - u(x) v(x)}{\increment x} \\
	&= \lim_{\increment x\to0} \left[
		\frac{u(x+\increment x) - u(x)}{\increment x} v(x+\increment x)
		+ u(x) \frac{v(x+\increment x) - v(x)}{\increment x}
	\right] \\
	&= \lim_{\increment x\to0} \frac{u(x+\increment x) - u(x)}{\increment x}
		\lim_{\increment x\to0} v(x+\increment x)
		+ u(x) \lim_{\increment x\to0} \frac{v(x+\increment x)-v(x)}{\increment x} \\
	&= u'(x) v(x) + u(x) v'(x).
\end{align*}

\item 因为函数\(u\)和\(v\)都在\(x\)连续,
所以利用\cref{theorem:极限.连续函数的极限1} 便得
\begin{align*}
	&\left[ \frac{u(x)}{v(x)} \right]'
	= \lim_{\increment x\to0} \frac1{\increment x} \left[
		\frac{u(x+\increment x)}{v(x+\increment x)}
		- \frac{u(x)}{v(x)}
	\right] \\
	&= \lim_{\increment x\to0}
		\frac{u(x+\increment x) v(x) - u(x) v(x+\increment x)}
			{v(x+\increment x) v(x) \increment x} \\
	&= \lim_{\increment x\to0}
		\frac{[u(x+\increment x) - u(x)] v(x) - u(x) [v(x+\increment x) - v(x)]}
			{v(x+\increment x) v(x) \increment x} \\
	&= \lim_{\increment x\to0}
		\frac1{v(x+\increment x) v(x)} \left[
		\frac{u(x+\increment x) - u(x)}{\increment x} v(x)
		- u(x) \frac{v(x+\increment x) - v(x)}{\increment x}
	\right] \\
	&= \frac{u'(x) v(x) - u(x) v'(x)}{v^2(x)}.
	\qedhere
\end{align*}
\end{itemize}
\end{proof}
\end{theorem}

\begin{corollary}
如果函数\(u=u(x)\)在点\(x\)具有导数,
那么\begin{equation*}
	(C u)' = C u'.
\end{equation*}
\end{corollary}

\begin{corollary}
如果函数\(u=u(x)\)、\(v=v(x)\)和\(w=w(x)\)都在点\(x\)具有导数,
那么\begin{equation*}
	(uvw)'
	= [(uv)w]'
	= (uv)'w + (uv)w'
	= u'vw + uv'w + uvw'.
\end{equation*}
\end{corollary}

\begin{example}%\label{example:导数.正切函数的导数}
求正切函数\(y=\tan x\)的导数.
\begin{solution}
\((\tan x)'
= \dv{x}(\frac{\sin x}{\cos x})
= \frac{(\sin x)' \cos x - \sin x (\cos x)'}{(\cos x)^2}
= \frac{\cos^2 x + \sin^2 x}{\cos^2 x}
= \frac1{\cos^2 x}
= \sec^2 x\).
\end{solution}
\end{example}

\begin{example}%\label{example:导数.余切函数的导数}
求余切函数\(y = \cot x\)的导数.
\begin{solution}
\((\cot x)'
= \dv{x}(\frac{\cos x}{\sin x})
= \frac{(\cos x)' \sin x - \cos x (\sin x)'}{(\sin x)^2}
= \frac{-\sin^2 x-\cos^2 x}{\sin^2 x}
= -\frac1{\sin^2 x}
= -\csc^2 x\).
\end{solution}
\end{example}

\begin{example}%\label{example:导数.正割函数的导数}
求正割函数\(y=\sec x\)的导数.
\begin{solution}
\((\sec x)'
= \dv{x}(\frac1{\cos x})
= \frac{-(\cos x)'}{(\cos x)^2}
= \frac{\sin x}{\cos^2 x}
= \sec x \tan x\).
\end{solution}
\end{example}

\begin{example}%\label{example:导数.余割函数的导数}
求余割函数\(y=\csc x\)的导数.
\begin{solution}
\((\csc x)'
= \dv{x}(\frac1{\sin x})
= \frac{-(\sin x)'}{(\sin x)^2}
= \frac{-\cos x}{\sin^2 x}
= -\csc x \cot x\).
\end{solution}
\end{example}

\begin{example}%\label{example:导数.双曲正弦函数的导数}
求双曲正弦函数\(y = \sinh x\)的导数.
\begin{solution}
\((\sinh x)'
= \dv{x}(\frac{e^x - e^{-x}}2)
= \frac{(e^x)' - (e^{-x})'}2
= \frac{e^x + e^{-x}}2
= \cosh x\).
\end{solution}
\end{example}

\begin{example}%\label{example:导数.双曲余弦函数的导数}
求双曲余弦函数\(y = \cosh x\)的导数.
\begin{solution}
\((\cosh x)'
= \dv{x}(\frac{e^x + e^{-x}}2)
= \frac{(e^x)' + (e^{-x})'}2
= \frac{e^x - e^{-x}}2
= \sinh x\).
\end{solution}
\end{example}

\begin{example}%\label{example:导数.双曲正切函数的导数}
求双曲正切函数\(y = \tanh x\)的导数.
\begin{solution}
\((\tanh x)'
= \dv{x}(\frac{\sinh x}{\cosh x})
= \frac{(\sinh x)' \cosh x - \sinh x (\cosh x)'}{(\cosh x)^2}
= \frac{\cosh^2 x - \sinh^2 x}{\cosh^2 x}
= \sech^2 x\).
\end{solution}
\end{example}

\begin{example}%\label{example:导数.双曲余切函数的导数}
求双曲余切函数\(y = \coth x\)的导数.
\begin{solution}
\((\coth x)'
= \dv{x}(\frac{\cosh x}{\sinh x})
= \frac{(\cosh x)' \sinh x - \cosh x (\sinh x)'}{(\sinh x)^2}
= \frac{\sinh^2 x - \cosh^2 x}{\sinh^2 x}
= -\csch^2 x\).
\end{solution}
\end{example}

\begin{example}%\label{example:导数.双曲正割函数的导数}
求双曲正割函数\(y = \sech x\)的导数.
\begin{solution}
\((\sech x)'
= \dv{x}(\frac1{\cosh x})
= -\frac{(\cosh x)'}{(\cosh x)^2}
= -\frac{\sinh x}{\cosh^2 x}
= -\tanh x \sech x\).
\end{solution}
\end{example}

\begin{example}%\label{example:导数.双曲余割函数的导数}
求双曲余割函数\(y = \csch x\)的导数.
\begin{solution}
\((\csch x)'
= \dv{x}(\frac1{\sinh x})
= -\frac{(\sinh x)'}{(\sinh x)^2}
= -\frac{\cosh x}{\sinh^2 x}
= -\coth x \csch x\).
\end{solution}
\end{example}

\subsection{反函数的求导法则}
\begin{theorem}
%@see: 《高等数学(第六版 上册)》 P90 定理2
%@see: 《数学分析(第二版 上册)》(陈纪修) P137 定理4.3.4
如果函数\(f\)在区间\((a,b)\)内连续、严格单调、可导并且\(f'(x)\neq0\),
则它的反函数\(f^{-1}\)在区间\((\alpha,\beta)\)内可导,
且\begin{equation*}
	[f^{-1}(y)]'=\frac1{f'(x)},
\end{equation*}
其中\(\alpha=\min\{f(a^+),f(b^-)\},
\beta=\max\{f(a^+),f(b^-)\}\).
\begin{proof}
因为函数\(f\)在\((a,b)\)上连续且严格单调,
由\hyperref[theorem:极限.连续函数的极限2]{反函数连续性定理}可知,
它的反函数\(f^{-1}\)在\((\alpha,\beta)\)上存在、连续且严格单调.
这时\begin{equation*}
	\increment y = f(x + \increment x) - f(x) \neq 0
	\iff
	\increment x = f^{-1}(y + \increment y) - f^{-1}(y) \neq 0,
\end{equation*}
并且当\(\increment y \to 0\)时有\(\increment x \to 0\).
因此\begin{align*}
	[f^{-1}(y)]'
	&= \lim_{\increment y \to 0}
		\frac{f^{-1}(y + \increment y) - f^{-1}(y)}{\increment y} \\
	&= \lim_{\increment x \to 0}
		\frac{\increment x}{f(x + \increment x) - f(x)} \\
	&= \left[
		\lim_{\increment x \to 0}
		\frac{f(x + \increment x) - f(x)}{\increment x}
	\right]^{-1}
	= \frac1{f'(x)}.
	\qedhere
\end{align*}
\end{proof}
\end{theorem}

简单地说,反函数\(y=f^{-1}(x)\)的导数等于直接函数\(x=f(y)\)导数的倒数.

\begin{example}
%@see: 《高等数学(第六版 上册)》 P91 例6
求\(y=\arcsin x\)的导数.
\begin{solution}
由直接函数\(x=\sin y\),
\(x\)的取值范围是\(-1 \leq x \leq 1\),
求导得\begin{equation*}
	\dv{x}{y}
	= \dv{y} \sin y
	= \cos y,
\end{equation*}
则\begin{equation*}
	\dv{y}{x}
	= \left(\dv{x}{y}\right)^{-1}
	= \frac1{\cos y}.
\end{equation*}
利用三角恒等式得到\begin{equation*}
	\cos^2 y = 1 - \sin^2 y,
\end{equation*}
开方得\begin{equation*}
	\cos y = \pm\sqrt{1 - \sin^2 y}.
\end{equation*}
现在我们要决定\(\cos y\)的符号.
因为反正弦函数\(y = \arcsin x\)的值域为\(-\frac\pi2 \leq y \leq \frac\pi2\),
所以\(0 \leq \cos y \leq 1\),
于是\(\cos y\)是非负的,
即\begin{equation*}
	\cos y
	= \sqrt{1 - \sin^2 y}
	= \sqrt{1 - x^2}.
\end{equation*}
最后,我们还要使\(\cos y\)的倒数有意义,
这就要求\(\sqrt{1-x^2}\neq0\),
即\(x\neq\pm1\)必须成立.
考虑到\(-1 \leq x \leq 1\),
\(x\)的取值范围就只能是\(-1 < x < 1\).
那么\begin{equation*}
	(\arcsin x)' = \frac1{\sqrt{1 - x^2}}
	\quad(-1<x<1).
\end{equation*}
\end{solution}
\end{example}

类似地,可得\begin{equation*}
	(\arccos x)' = \frac{-1}{\sqrt{1 - x^2}}
	\quad(-1<x<1).
\end{equation*}

\begin{example}
%@see: 《高等数学(第六版 上册)》 P91 例7
求反正切函数\(y=\arctan x\)的导数.
\begin{solution}
由直接函数\(x=\tan y\),有\begin{equation*}
	\dv{x}{y}
	= \dv{y} \tan y
	= \sec^2 y
	= 1 + \tan^2 y
	= 1 + x^2,
\end{equation*}
那么\begin{equation*}
	(\arctan x)' = \frac1{1+x^2}.
\end{equation*}
\end{solution}
\end{example}

类似地,可得反余切函数的导数为\begin{equation*}
	(\arccot x)' = -\frac1{1+x^2}.
\end{equation*}

\begin{example}
求反正割函数\(y=\arcsec x\)的导数.
\begin{solution}
由直接函数\(x=\sec y\),有\begin{equation*}
	\dv{x}{y}
	= \sec y \tan y
	= x \sqrt{x^2-1},
\end{equation*}
那么\begin{equation*}
	(\arcsec x)' = \frac1{x \sqrt{x^2-1}}.
\end{equation*}
\end{solution}
\end{example}

类似地,可得反余割函数的导数为\begin{equation*}
	(\arccsc x)' = -\frac1{x \sqrt{x^2-1}}.
\end{equation*}

\begin{example}
%@see: 《高等数学(第六版 上册)》 P92 例8
利用指数函数的导数公式\((a^x)' = a^x \ln a\)
计算对数函数\(y=\log_a x\)的导数,
其中\(a>0,a\neq1\).
\begin{solution}
由直接函数\(x=a^y\),有\begin{equation*}
	\dv{x}{y} = \dv{y} a^y = a^y \ln a \neq 0,
\end{equation*}
那么\begin{equation*}
	(\log_a x)' = \frac1{a^y \ln a} = \frac1{x \ln a}.
\end{equation*}
\end{solution}
\end{example}

\begin{example}
求\(y = \arcsec x\)的导数.
\begin{solution}
由直接函数\(x=\sec y\),有\begin{equation*}
	\dv{x}{y}
	= \sec y \tan y
	= \sec y \sqrt{\sec^2 y-1}
	= x \sqrt{x^2-1},
\end{equation*}
那么\begin{equation*}
	(\arcsec x)'
	= \frac1{x \sqrt{x^2-1}}.
\end{equation*}
\end{solution}
\end{example}

类似地,可得\begin{equation*}
	(\arccsc x)'
	= -\frac1{x \sqrt{x^2-1}}.
\end{equation*}

\begin{example}\label{example:导数.反双曲正弦函数的导数}
求反双曲正弦函数\(y = \arsinh x\)的导数.
\begin{solution}
由直接函数\(x = \sinh y = \frac{e^y - e^{-y}}2\),有\begin{equation*}
	\dv{x}{y}
	= \cosh y
	= \sqrt{1 + \sinh^2 y}
	= \sqrt{1 + x^2}.
\end{equation*}
那么\begin{equation*}
	(\arsinh x)' = \frac1{\sqrt{1+x^2}}.
\end{equation*}
\end{solution}
\end{example}
\begin{remark}
由于\(\arsinh x = \ln(x+\sqrt{x^2+1})\),
所以我们也可以直接求导得\begin{equation*}
	(\arsinh x)' = \frac{1+\frac{2x}{2\sqrt{x^2+1}}}{x+\sqrt{x^2+1}}
	= \frac1{\sqrt{1+x^2}}.
\end{equation*}
%@Mathematica: D[Log[x + Sqrt[1 + x^2]], x] // Simplify
\end{remark}

\begin{example}
求反双曲余弦函数\(y = \arcosh x\)的导数.
\begin{solution}
由直接函数\(x = \cosh y\),有\begin{equation*}
	\dv{x}{y}
	= \sinh y
	= \sqrt{\cosh^2 y - 1}
	= \sqrt{x^2 - 1}
	\quad(x\geq1).
\end{equation*}
那么\begin{equation*}
	(\arcosh x)' = \frac1{\sqrt{x^2-1}}
	\quad(x>1).
\end{equation*}
\end{solution}
\end{example}

\begin{example}
求反双曲正切函数\(y = \artanh x\)的导数.
\begin{solution}
由直接函数\(x = \tanh y\),有\begin{equation*}
	\dv{x}{y}
	= \sech^2 y
	= 1 - \tanh^2 y
	= 1 - x^2.
\end{equation*}
那么\begin{equation*}
	(\artanh x)' = \frac1{1-x^2}
	\quad(\abs{x}<1).
\end{equation*}
\end{solution}
\end{example}

\begin{example}
求反双曲余切函数\(y = \arcoth x\)的导数.
\begin{solution}
由直接函数\(x = \coth y\),有\begin{equation*}
	\dv{x}{y} = -\csch^2 y
	= 1 - \coth^2 y
	= 1 - x^2.
\end{equation*}
那么\begin{equation*}
	(\arcoth x)' = \frac1{1-x^2}
	\quad(\abs{x}>1).
\end{equation*}
\end{solution}
\end{example}

\begin{example}%野题
设函数\begin{equation*}
	f(x) = \frac{\cos x}{3+\sin x}
	\quad(0<x<\pi).
\end{equation*}
求\(f\)的反函数\(f^{-1}\)在点\(x=0\)的导数.
\begin{solution}
首先计算\(f^{-1}(0)\).
解方程\(f(x) = 0\ (0<x<\pi)\)得\(x=\pi/2\).
于是\(f^{-1}\)在点\(x=0\)的导数
就是\(f\)在点\(x=\pi/2\)的导数的倒数,
即\begin{equation*}
	\eval{\dv{f^{-1}}{x}}_{x=0}
	= \left( \eval{\dv{f}{x}}_{x=\pi/2} \right)^{-1},
\end{equation*}
其中\begin{equation*}
	\eval{\dv{f}{x}}_{x=\pi/2}
	= \lim_{x\to\pi/2} \frac{f(x) - f(\pi/2)}{x - \pi/2}
	= \lim_{x\to\pi/2} \frac1{3+\sin x} \cdot \frac{\cos x}{x-\pi/2}
	= \frac14 \lim_{x\to\pi/2} \frac{-\sin x}{1}
	= -\frac14,
\end{equation*}
于是\begin{equation*}
	\eval{\dv{f^{-1}}{x}}_{x=0}
	= -4.
\end{equation*}
\end{solution}
\end{example}

\subsection{复合函数的求导法则}
\begin{theorem}
%@see: 《高等数学(第六版 上册)》 P92 定理3
%@see: 《数学分析(第二版 上册)》(陈纪修) P142 定理4.4.1(复合函数求导法则)
如果函数\(u=g(x)\)在点\(x\)可导,
而函数\(y=f(u)\)在点\(u=g(x)\)可导,
则复合函数\(y=f[g(x)]\)在点\(x\)可导,
且其导数为\begin{equation*}
	\dv{y}{x} = f'(u) \cdot g'(x)
	\quad\text{或}\quad
	\dv{y}{x} = \dv{y}{u} \cdot \dv{u}{x}.
\end{equation*}
\end{theorem}

复合函数的求导法则可以推广到多个中间变量的情形.
设\(y=f(u)\),\(u=\phi(v)\),\(v=\psi(x)\),
则复合函数\(y=f\{\phi[\psi(x)]\}\)的导数为\begin{equation*}
	\dv{y}{x} = \dv{y}{u} \cdot \dv{u}{v} \cdot \dv{v}{x}.
\end{equation*}

上述复合函数的求导公式也称作\DefineConcept{链式法则}(chain rule).

\subsection{行列式函数的求导法则}
\begin{theorem}
\def\f#1{f_{#1}(x)}%
\def\g#1{f_{#1}'(x)}%
设函数\begin{equation*}
	f(x) = \begin{vmatrix}
		\f{11} & \f{12} & \dots & \f{1n} \\
		\vdots & \vdots & & \vdots \\
		\f{i1} & \f{i2} & \dots & \f{in} \\
		\vdots & \vdots & & \vdots \\
		\f{n1} & \f{n2} & \dots & \f{nn}
	\end{vmatrix}
\end{equation*}的任意分量函数都可导(即\(\f{ij}\ (i,j=1,2,\dotsc,n)\)可导),
那么\(f(x)\)可导,
且\begin{equation*}
	\dv{x} f(x)
	= \sum_{i=1}^n \begin{vmatrix}
		\f{11} & \f{12} & \dots & \f{1n} \\
		\vdots & \vdots & & \vdots \\
		\g{i1} & \g{i2} & \dots & \g{in} \\
		\vdots & \vdots & & \vdots \\
		\f{n1} & \f{n2} & \dots & \f{nn}
	\end{vmatrix}.
\end{equation*}
\end{theorem}

\subsection{对数导数}
\begin{definition}
设函数\(f\colon I \to \mathbb{R}, f \in D(I)\),
我们把\begin{equation*}
	\dv{x} \ln f(x)
\end{equation*}称为“函数\(f\)的\DefineConcept{对数导数}(logarithmic derivative)”.
%@see: https://mathworld.wolfram.com/LogarithmicDerivative.html
\end{definition}

\section{高阶导数}
\subsection{函数的\texorpdfstring{\(n\)}{n}阶导数}
%@see: 《高等数学(第六版 上册)》 P99
%@see: 《数学分析(第二版 上册)》(陈纪修) P154
设\(X\subseteq\mathbb{R}\),
函数\(f\colon X\to\mathbb{R}\)在点\(x_0\)可导,
且\(f\)的导数\(f'\)也在\(x_0\)可导,
即极限\[
	\lim_{h\to0} \frac{f'(x_0+h) - f'(x_0)}{h}
\]存在,
则称“函数\(f\)在点\(x_0\)~\DefineConcept{二阶可导}(twice differentiable)”
“\(f\)在点\(x_0\)的二阶导数存在”
“\(f\)在点\(x_0\)具有二阶导数”,
把\(f'\)的导数称为“函数\(f\)的\DefineConcept{二阶导数}(second derivative)”,
记作\[
	y''
	\quad\text{或}\quad
	\dv[2]{y}{x},
\]
即\[
	\dv[2]{y}{x}
	\defeq
	\dv{x}(\dv{f}{x}).
\]
相对地,把\(f\)的导数\(f'\)称为
“函数\(f\)的\DefineConcept{一阶导数}(first derivative)”.

类似地,
如果函数\(f\)的二阶导数\(f''\)在点\(x_0\)可导,
则称“函数\(f\)在点\(x_0\)~\DefineConcept{三阶可导}”
“\(f\)在点\(x_0\)的三阶导数存在”
“\(f\)在点\(x_0\)具有三阶导数”,
把\(f''\)的导数称为“函数\(f\)的\DefineConcept{三阶导数}(third derivative)”,
记作\[
	y'''
	\quad\text{或}\quad
	\dv[3]{y}{x}.
\]

如果函数\(f\)的三阶导数\(f'''\)在点\(x_0\)可导,
则称“函数\(f\)在点\(x_0\)~\DefineConcept{四阶可导}”
“\(f\)在点\(x_0\)的四阶导数存在”
“\(f\)在点\(x_0\)具有四阶导数”,
把\(f'''\)的导数称为“函数\(f\)的\DefineConcept{四阶导数}(fourth derivative)”,
记作\[
	y^{(4)}
	\quad\text{或}\quad
	\dv[4]{y}{x}.
\]

我们把二阶以上的导数统称\DefineConcept{高阶导数}(higher order derivative).

%@see: 《数学分析(第二版 上册)》(陈纪修) P155 定义4.5.1
如果函数\(f\)的\(n-1\)阶导数\(f^{(n-1)}\)在点\(x_0\)可导,
则称“函数\(f\)在点\(x_0\)~\DefineConcept{\(n\)阶可导}”
“\(f\)在点\(x_0\)的\(n\)阶导数存在”
“\(f\)在点\(x_0\)具有\(n\)阶导数”,
把\(f^{(n-1)}\)的导数称为
“函数\(f\)的~\DefineConcept{\(n\)阶导数}(n-th derivative)”,
记作\[
	y^{(n)}
	\quad\text{或}\quad
	\dv[n]{y}{x},
\]
即\begin{equation}\label{equation:高阶导数.高阶导数的递归定义}
	\dv[n]{y}{x}
	\defeq
	\dv[n-1]{y}{x}
	\quad(n\geq2).
\end{equation}
我们把\cref{equation:高阶导数.高阶导数的递归定义} 称为~\DefineConcept{\(n\)阶导数的递归定义}.

由于当函数\(f\)在点\(x_0\)具有\(n\)阶导数时,
它的小于\(n\)阶的导数都存在,
于是称“\(f\)具有直到\(n\)阶导数”.

特别地,规定:\begin{equation}
	f^{(0)}(x) = f(x).
\end{equation}

和\cref{definition:导数.函数在开区间内可导,definition:导数.函数在闭区间上可导} 一样,
我们可以定义函数在某个区间上的可导性:
\begin{definition}
如果函数\(f\colon(a,b)\to\mathbb{R}\)在开区间\((a,b)\)内的每一个点\(n\)阶可导,
就称“函数\(f\)在开区间\((a,b)\)内\(n\)阶可导”.
\end{definition}
\begin{definition}
如果函数\(f\colon[a,b]\to\mathbb{R}\)在开区间\((a,b)\)内\(n\)阶可导,
且在点\(a\)的右\(n\)阶导数\(f^{(n)}_+(a)\)及在点\(b\)的左\(n\)阶导数\(f^{(n)}_-(b)\)都存在,
就说“函数\(f\)在闭区间\([a,b]\)上\(n\)阶可导”.
\end{definition}

\begin{proposition}\label{theorem:高阶导数.在一点高阶可导是在该点某邻域内低阶可导的充分条件}
如果函数\(f\)在点\(x_0\)具有\(n\ (n\geq2)\)阶导数,
则\(f\)在点\(x_0\)的某个邻域内\(n-1\)阶可导.
\begin{proof}
根据定义,函数\(f\)在点\(x_0\)具有\(n\)阶导数,
当且仅当函数\(f\)的\(n-1\)阶导数在点\(x_0\)可导,
即极限\[
	\lim_{h\to0} \frac{f^{(n-1)}(x_0+h) - f^{(n-1)}(x_0)}{h}
\]存在且有限,
这就要求在\(x_0\)的某个邻域内表达式\(f^{(n-1)}(x_0+h)\)是有定义的,
因此\(f\)必定在点\(x_0\)的某个邻域内\(n-1\)阶可导.
\end{proof}
\end{proposition}

\begin{example}
%@see: 《数学分析(第二版 上册)》(陈纪修) P155 例4.5.1
求\(y = e^x\)的\(n\)阶导函数.
\begin{solution}
由于\[
	(e^x)' = e^x,
\]
显然\[
	(e^x)'
	= (e^x)''
	= \dotsb
	= (e^x)^{(n)}
	= e^x.
\]
\end{solution}
\end{example}
类似可以得到\[
	(a^x)^{(n)}
	= a^x \ln^n a.
\]

\begin{example}
%@see: 《数学分析(第二版 上册)》(陈纪修) P155 例4.5.2
求\(y = \sin x\)和\(y = \cos x\)的\(n\)阶导函数.
\begin{solution}
因为\[
	(\sin x)' = \cos x
	= \sin\left(x + \frac\pi2\right),
\]
利用复合函数的求导法则\[
	(\sin x)'' = \dv{x} \sin\left(x+\frac\pi2\right)
	= \cos\left(x+\frac\pi2\right)
	= \sin\left(x+\frac{2\pi}2\right),
\]
以此类推,用数学归纳法容易证明\[
	(\sin x)^{(n)}
	= \sin\left(x+\frac{n\pi}2\right).
\]
\end{solution}
\end{example}
同理,\(y = \cos x\)的\(n\)阶导数为\[
	(\cos x)^{(n)}
	= \cos\left(x+\frac{n\pi}2\right).
\]

\begin{example}
%@see: 《数学分析(第二版 上册)》(陈纪修) P156 例4.5.3
求幂函数\(y = x^m\ (\text{$m$是正整数})\)的\(n\)阶导函数.
\begin{solution}
由幂函数的求导公式可得\[
	(x^m)' = m x^{m-1},
	\qquad
	(x^m)'' = m(m-1) x^{m-2},
\]
递归可得\begin{equation}
	(x^m)^{(n)} = \left\{ \begin{array}{cl}
		m(m-1)\dotsm(m-n+1) x^{m-n}, & n \leq m, \\
		0, & n > m.
	\end{array} \right.
\end{equation}
特别地,成立\begin{equation}
	(x^m)^{(m)} = m!.
\end{equation}
\end{solution}
\end{example}

\begin{example}
%@see: 《数学分析(第二版 上册)》(陈纪修) P156 例4.5.4
求\(y = \ln x\)的\(n\)阶导函数.
\begin{solution}
因为\[
	(\ln x)' = \frac1x = x^{-1},
\]
于是\begin{gather*}
	(\ln x)'' = (x^{-1})' = -x^{-2}, \\
	(\ln x)''' = (-x^{-2})' = 2x^{-3},
\end{gather*}
递归可得\begin{gather}
	(\ln x)^{(n)}
	= (-1)^{n-1} \frac{(n-1)!}{x^n}, \\
	\left(\frac1x\right)^{(n)}
	= (\ln x)^{(n+1)}
	= (-1)^n \frac{n!}{x^{n+1}}.
\end{gather}
\end{solution}
\end{example}

\begin{example}
设函数\(f(x) = \frac1{2^x+1}\).
求\(f^{(4)}(0)\).
\begin{solution}
记\(g(x) = \frac1{2^x+1} - \frac12\),
则\(f(x) = g(x) + \frac12\).
因为\begin{align*}
	g(-x) &= \frac1{2^{-x}+1} - \frac12
	% 通分
	= \frac{2-(2^{-x}+1)}{2(2^{-x}+1)}
	% 减法
	= \frac{1-2^{-x}}{2(1+2^{-x})} \\
	% 分子分母同乘以\(2^x\)
	&= \frac{2^x-1}{2(2^x+1)}
	% 除法
	= \frac{2^x+1-2}{2(2^x+1)}
	= \frac12 - \frac1{2^x+1}
	= g(x),
\end{align*}
也就是说\(g\)是奇函数,
所以由\cref{theorem:导数与微分.导函数的奇偶性} 可知
\(g'\)是偶函数,\(g''\)是奇函数,
\(g'''\)是偶函数,\(g^{(4)}\)是奇函数,
显然\(g\)的各阶导数都在点\(x=0\)连续,
于是\(f^{(4)}(0) = g^{(4)}(0) = 0\).
\end{solution}
\end{example}

\begin{example}
%@credit: {8b6edada-f2fd-4ae5-9020-eb533149a54c},{ce603838-a24d-4616-9395-d7b223e8cb72}
设函数\(f\)在点\(x=0\)具有\(n\)阶导数.
证明:\begin{equation}\label{equation:高阶导数.高阶导数与差分的联系}
	f^{(n)}(0)
	= \lim_{h\to0} \frac1{h^n} \sum_{k=0}^n (-1)^k C_n^k f((n-k)h).
\end{equation}
%TODO proof 狗狗说可以用归纳法证明
% \begin{proof}
% 当\(n=1\)时,\cref{equation:高阶导数.高阶导数与差分的联系} 成为\begin{equation*}
% 	f'(0) = \lim_{h\to0} \frac{f(h) - f(0)}h,
% \end{equation*}
% 它正是函数\(f\)在点\(x=0\)的一阶导数的定义式,自然成立.
% 假设\begin{equation*}
% 	f^{(n-1)}(0)
% 	= \lim_{h\to0} \frac1{h^{n-1}} \sum_{k=0}^{n-1} (-1)^k C_{n-1}^k f((n-1-k)h)
% \end{equation*}成立,
% 那么\begin{align*}
% 	f^{(n)}(0)
% 	% = \lim_{h\to0} \frac1{h^n} \sum_{k=0}^n (-1)^k C_n^k f((n-k)h).
% 	&= \lim_{x\to0} \frac{f^{(n-1)}(x) - f^{(n-1)}(0)}{x}
% \end{align*}
% \end{proof}
\end{example}

\subsection{高阶导数的运算规则}
\begin{theorem}
%@see: 《数学分析(第二版 上册)》(陈纪修) P156 定理4.5.1
设\(f\)和\(g\)都是\(n\)阶可导函数,
则对任意常数\(c_1\)和\(c_2\),
它们的线性组合\(c_1 f + c_2 g\)也是\(n\)阶可导的,
且满足\begin{equation}
	(c_1 f + c_2 g)^{(n)}
	= c_1 f^{(n)} + c_2 f^{(n)}.
\end{equation}
\end{theorem}

这个结论可以推广到有限个函数线性组合的情况:\begin{equation}
	\left[ \sum_{i=1}^n c_i f_i \right]^{(n)}
	= \sum_{i=1}^n c_i f_i^{(n)}.
\end{equation}

\begin{theorem}
%@see: 《数学分析(第二版 上册)》(陈纪修) P157 定理4.5.2(Leibniz公式)
设\(f\)和\(g\)都是\(n\)阶可导函数,
则\begin{equation}\label{equation:导数与微分.莱布尼茨公式}
	(f \cdot g)^{(n)}
	= \sum_{k=0}^n C_n^k f^{(n-k)} g^{(k)},
\end{equation}
其中\(C_n^k \equiv \frac{n!}{k! (n-k)!}\)是组合系数.
%TODO proof
\end{theorem}
\cref{equation:导数与微分.莱布尼茨公式} 称为\DefineConcept{莱布尼茨公式}.

\begin{example}
设\(f(x) = x^2 \ln(2-x)\).
求\(f^{(n)}(0)\ (n\geq3)\).
\begin{solution}
由\cref{equation:导数与微分.莱布尼茨公式} 可知
\(f^{(n)}(0) = -\frac{n!}{2^{n-2} (n-2)}\ (n\geq3)\).
\end{solution}
\end{example}

\subsection{高阶可导函数族}
\begin{definition}\label{definition:函数族.n阶可导函数族}
由区间\(I\)上全部的\(n\)阶可导函数组成的集合,
称作\(n\)阶\DefineConcept{可导函数族},
记作\(D^n(I)\),
即\begin{equation*}
	D^n(I)
	\defeq
	\Set*{
		f\in\mathbb{R}^I
		\given
		f^{(n-1)} \in D(I)
	}.
\end{equation*}
\end{definition}

\begin{definition}\label{definition:函数族.n阶连续可导函数族}
由区间\(I\)上全部的\(n\)阶连续可导函数组成的集合,
称作\(n\)阶\DefineConcept{连续可导函数族},
记作\(C^n(I)\),
即\begin{equation*}
	C^n(I)
	\defeq
	\Set*{
		f\in\mathbb{R}^I
		\given
		[f\in D^n(I)]
		\land
		[f^{(n)}\in C(I)]
	}.
\end{equation*}
%@see: https://mathworld.wolfram.com/C-kFunction.html
\end{definition}

\begin{theorem}
如果函数\(f(x)\)在点\(x\)处具有\(n\)阶导数,
那么\(f(x)\)在点\(x\)的某一邻域内必定具有一切低于\(n\)阶的导数.
\end{theorem}
换句话说,\(C(I) \supseteq
D(I) = D^1(I) \supseteq
C^1(I) \supseteq
D^2(I) \supseteq
C^2(I) \supseteq
D^3(I) \supseteq
\dotsb\).

\subsection{光滑函数}
\begin{definition}\label{definition:函数族.光滑函数族}
定义:\begin{gather*}
	D^\infty (I) \defeq \bigcap_{n\geq1} D^n(I), \\
	C^\infty (I) \defeq \bigcap_{n\geq1} C^n(I).
\end{gather*}

称函数\(f \in C^\infty (I)\)为\(I\)上的\DefineConcept{光滑函数}(smooth function).
\end{definition}

\begin{property}\label{theorem:函数族.光滑函数族的性质1}
\(D^\infty (I) = C^\infty (I)\).
\end{property}

\section{隐函数及由参数方程所确定的函数的导数}\label{section:导数与微分.隐函数及由参数方程所确定的函数的导数}
\subsection{隐函数的导数}
\begin{definition}
形如\(y=f(x)\)的函数,在等号左端是因变量的符号,而右端是含有自变量的式子,当自变量取定义域内任一值时,由这个式子能确定对应的函数.用这种方式表达的函数叫做\DefineConcept{显函数}.
相对地,形如\(g(x,y)=0\)的函数,称为\DefineConcept{隐函数}.
把一个隐函数化为显函数,叫做\DefineConcept{隐函数的显化}.
\end{definition}

\begin{example}
例如从方程\(x+y^3-1=0\)解出\(y=\sqrt[3]{1-x}\),就把隐函数化成了显函数.
\end{example}

\begin{example}
%@see: 《高等数学(第六版 上册)》 P104 例1
求由方程\(e^y + xy - e = 0\)所确定的隐函数的导数\(\displaystyle\dv{y}{x}\).
\begin{solution}
把方程两边分别对\(x\)求导得\begin{equation*}
	\dv{x}(e^y+xy-e) = e^y \dv{y}{x} + y + x \dv{y}{x} = 0,
\end{equation*}
解得\begin{equation*}
	\dv{y}{x} = -\frac{y}{x+e^y},
	\quad x+e^y \neq 0.
\end{equation*}
在这个结果中,分式中的\(y=y(x)\)是由方程\(e^y + xy - e = 0\)所确定的隐函数.
\end{solution}
%@Mathematica: eqn = Exp[y[x]] + x y[x] - E == 0
%@Mathematica: Solve[D[eqn, x], y'[x]]
\end{example}

%@see: 《数学分析(第二版 上册)》(陈纪修) P126 例4.2.1
\begin{example}
%@see: 《高等数学(第六版 上册)》 P105 例3
%@see: 《数学分析(第二版 上册)》(陈纪修) P128 例4.2.2
求椭圆\(\frac{x^2}{16}+\frac{y^2}{9}=1\)在点\(\left(2,\frac32\sqrt3\right)\)处的切线方程.
\begin{solution}
由导数的几何意义可知,所求切线的斜率为\begin{equation*}
	k = \eval{y'}_{x=2}.
\end{equation*}

椭圆方程的两边分别对\(x\)求导,得\begin{equation*}
	\frac{x}{8}+\frac{2}{9}y\dv{y}{x}=0,
\end{equation*}\begin{equation*}
	\dv{y}{x}=-\frac{9x}{16y}.
	\eqno(1)
\end{equation*}

当\(x=2\)时,\(y=\frac{3}{2}\sqrt3\),代入(1)式可得\begin{equation*}
	\eval{\dv{y}{x}}_{x=2}=-\frac{\sqrt3}{4}.
\end{equation*}
于是所求的切线方程为\begin{equation*}
	y-\frac{3}{2}\sqrt3 = -\frac{\sqrt3}{4}(x-2),
\end{equation*}
即\begin{equation*}
	\sqrt3 x + 4 y - 8\sqrt3 = 0.
\end{equation*}
\end{solution}
\end{example}

% \begin{lemma}
% 设函数\(y=\ln\abs{x}\).
% 当\(x\neq0\)时,\(y'=1/x\).
% \end{lemma}

\begin{example}
已知双曲线\(C: y^2 = 2px\ (p>0)\),
过它的焦点弦\(PQ\)的两个端点作它的两条切线,求这两条切线的交点的坐标.
\begin{figure}[htb]
	\centering
	\pgfmathsetmacro{\p}{1}%参数p
	\pgfmathsetmacro{\f}{\p/2}%焦点横坐标
	\pgfmathsetmacro{\k}{3}%焦点弦斜率k
	\pgfmathsetmacro{\a}{\p/(2*\k^2)}
	\pgfmathsetmacro{\b}{\p/\k}
	\pgfmathsetmacro{\c}{sqrt(\k^2+1)}
	\pgfmathsetmacro{\pa}{\a*(\c+1)^2}% x_1
	\pgfmathsetmacro{\pb}{\b*(\c+1)}% y_1
	\pgfmathsetmacro{\qa}{\a*(\c-1)^2}% x_2
	\pgfmathsetmacro{\qb}{-\b*(\c-1)}% y_2
	\pgfmathsetmacro{\sa}{-\p/2}% x_0
	\pgfmathsetmacro{\sb}{\p/\k}% y_0
	\begin{tikzpicture}
		\begin{axis}[
			xmin=-2,xmax=2,
			axis lines=middle,
			xlabel=$x$,
			ylabel=$y$,
		]
			\coordinate(F)at(\f,0);
			\coordinate(X)at(10,0);
			\coordinate(P)at(\pa,\pb);
			\coordinate(Q)at(\qa,\qb);
			\coordinate(S)at(\sa,\sb);
			\draw[purple](-\f,-4)--(-\f,4);%准线
			\draw(2,-3)node[above right]{$C$};
			\begin{scope}[color=orange,samples=100,smooth]
				\addplot[domain=0:10]{sqrt(2*\p*x)};
				\addplot[domain=0:10]{-sqrt(2*\p*x)};
			\end{scope}
			\draw[blue](P)--(F)--(Q);
			\draw(P)--(S)--(Q);
			\draw pic["$\theta$",draw=gray,-,angle eccentricity=1.7,angle radius=5mm]{angle=X--F--P};
			\fill(P)circle(2pt)node[below]{$P$}
				(F)circle(2pt)node[below]{$F$}
				(Q)circle(2pt)node[right]{$Q$}
				(S)circle(2pt)node[left]{$S$};
		\end{axis}
	\end{tikzpicture}
	\caption{}
	\label{figure:导数与微分.抛物线[y^2=2px]的焦点弦与切线}
\end{figure}
\begin{solution}
如\cref{figure:导数与微分.抛物线[y^2=2px]的焦点弦与切线},
设直线\(l: y=k\left(x-\frac{p}{2}\right)\)(不妨设\(k>0\))
与双曲线\(C\)相交于\(P(x_1,y_1)\)和\(Q(x_2,y_2)\)两点.
那么\begin{equation*}
	x_1 = \frac{p}{2 k^2} (\sqrt{k^2+1} + 1)^2,
	\qquad
	y_1 = \frac{p}{k} (1+\sqrt{k^2+1}),
\end{equation*}\begin{equation*}
	x_2 = \frac{p}{2 k^2} (\sqrt{k^2+1} - 1)^2,
	\qquad
	y_2 = \frac{p}{k} (1-\sqrt{k^2+1}).
\end{equation*}

隐函数\(y^2 = 2px\)对\(x\)求导,得\begin{equation*}
	2 y \dv{y}{x} = 2 p,
\end{equation*}
整理得\begin{equation*}
	\dv{y}{x} = \frac{p}{y},
\end{equation*}
所以\begin{equation*}
	k_1 = \eval{\dv{y}{x}}_P
	= \frac{k}{1+\sqrt{k^2+1}},
	\qquad
	k_2 = \eval{\dv{y}{x}}_Q
	= \frac{k}{1-\sqrt{k^2+1}}.
\end{equation*}

那么抛物线过点\(P\)和点\(Q\)的切线的方程为\begin{equation*}
	l_1: y-y_1 = k_1 (x-x_1),
\end{equation*}\begin{equation*}
	l_2: y-y_2 = k_2 (x-x_2).
\end{equation*}
联立切线方程,消去\(y\),得\begin{equation*}
	y_1 + k_1(x - x_1) = y_2 + k_2(x - x_2),
\end{equation*}
移项得\begin{equation*}
	(k_1 - k_2) x = y_2 - y_1 + k_1 x_1 - k_2 x_2,
\end{equation*}
解得\begin{equation*}
	x_0 = \frac{y_2 - y_1 + k_1 x_1 - k_2 x_2}{k_1 - k_2}
	= -\frac{p}{2},
\end{equation*}
代入得\begin{equation*}
	y_0 = y_1 + k_1(x_0 - x_1)
	= \frac{p}{k}.
\end{equation*}
综上所述,所求的交点坐标为\begin{equation*}
	\left(-\frac{p}{2},\frac{p}{k}\right).
\end{equation*}
\end{solution}
\end{example}

\subsection{对数求导法}
在某些场合,利用所谓\emph{对数求导法}求导数比用通常的方法简便些,
它适用于求解函数的积、商,根式形式、幂形式、指数形式或幂指形式的函数.
这种方法是先在\(y=f(x)\)的两边取对数,然后再求出\(y\)的导数\(y'\).
但要注意,因为对数运算要求其真数必须大于零(即\(\log_a x \implies x>0\)),
所以在使用对数求导法时,一定要注意\(f(x)\)的定义域.

\begin{example}
%@see: 《高等数学(第六版 上册)》 P106 例5
求\(y=x^{\sin x}\ (x>0)\)的导数.
\begin{solution}
因为\(x > 0\),
所以\(y=x^{\sin x} > 0\).
对函数等式两边取对数得\begin{equation*}
	\ln y = \sin x \ln x,
\end{equation*}
对上式两边求导得\begin{equation*}
	\frac{y'}{y} = \cos x \ln x + \frac{\sin x}{x},
\end{equation*}
于是\begin{equation*}
	y' = x^{\sin x} \left( \cos x \ln x + \frac{\sin x}{x} \right).
\end{equation*}
\end{solution}
\end{example}

\begin{theorem}
%@see: 《高等数学(第六版 上册)》 P106
对于一般形式的幂指函数\begin{equation*}
	y = u^v
	\quad(u > 0),
\end{equation*}
如果\(u=u(x),v=v(x)\)都可导,
则\begin{equation}
	y' = \dv{x} e^{v \ln u}
	= u^v \left( v' \ln u + \frac{u'v}{u} \right).
\end{equation}
\end{theorem}
这里一定要注意\(u>0\)这个条件,它是函数\(\ln u\)对定义域的要求.

\begin{example}
求\(y=\sqrt{\frac{(x-1)(x-2)}{(x-3)(x-4)}}\)的导数.
\begin{solution}
由\(\frac{(x-1)(x-2)}{(x-3)(x-4)}\geq0\)
得\(x \in (-\infty,1]\cup[2,3)\cup(4,+\infty)\).
这就是函数\(y=y(x)\)的定义域.

当\(x>4\)时,对函数式两边取对数,得\begin{equation*}
	\ln y = \frac{1}{2} \bigl[
		\ln(x-1)+\ln(x-2)-\ln(x-3)-\ln(x-4)
	\bigr],
\end{equation*}
上式两边对\(x\)求导,得\begin{equation*}
	\frac{y'}{y} = \frac{1}{2} \left(
		\frac{1}{x-1} + \frac{1}{x-2} - \frac{1}{x-3} - \frac{1}{x-4}
	\right),
\end{equation*}
于是\begin{equation*}
	y' = \frac{y}{2} \left(
		\frac{1}{x-1} + \frac{1}{x-2} - \frac{1}{x-3} - \frac{1}{x-4}
	\right).
\end{equation*}

当\(x<1\)时,对原函数变形得\(y=\sqrt{\frac{(1-x)(2-x)}{(3-x)(4-x)}}\);
当\(2<x<3\)时,对原函数变形得\(y=\sqrt{\frac{(x-1)(x-2)}{(3-x)(4-x)}}\);
用同样的方法可得与上面相同的结果.
\end{solution}
\end{example}

\subsection{由参数方程所确定的函数的导数}
对于参数方程\begin{equation*}
%@see: 《高等数学(第六版 上册)》 P107 (3)
	\left\{ \begin{array}{l}
		x = \phi(t), \\
		y = \psi(t),
	\end{array} \right.
\end{equation*}
若函数\(x = \phi(t)\)具有单调连续反函数\(t=\phi^{-1}(x)\),
且该反函数可与函数\(y = \psi(t)\)构成复合函数,
那么由上述参数方程所确定的函数可以看成是
由函数\(y=\psi(t)\)、\(t=\phi^{-1}(x)\)复合而成的函数\(y=\psi(\phi^{-1}(x))\).
现在,要计算这个复合函数的导数.
为此再假定函数\(x = \phi(t)\)和\(y = \psi(t)\)都可导,且\(\phi'(t) \neq 0\).
于是根据复合函数的求导法则与反函数的求导法则,就有\begin{equation*}
	\dv{y}{x}
	= \dv{y}{t} \cdot \dv{t}{x},
\end{equation*}
即\begin{equation}\label{equation:导数.参数方程确定的函数的一阶导数}
%@see: 《高等数学(第六版 上册)》 P108 (4)
	\dv{y}{x}
	= \frac{\psi'(t)}{\phi'(t)}.
\end{equation}
这就是上述参数方程所确定的\(x\)的函数的导数公式.

如果函数\(x = \phi(t)\)和\(y = \psi(t)\)还都是二阶可导的,
那么有二阶导数公式\begin{equation*}
	\dv[2]{y}{x}
	= \dv{x}(\dv{y}{x})
	= \dv{t}(\dv{y}{x}) \cdot \dv{t}{x},
\end{equation*}
即\begin{equation}\label{equation:导数.参数方程确定的函数的二阶导数}
%@see: 《高等数学(第六版 上册)》 P108 (5)
	\dv[2]{y}{x}
	= \frac{\psi''(t) \phi'(t) - \psi'(t) \phi''(t)}{(\phi'(t))^3}.
\end{equation}

\begin{example}
%@see: 《数学分析(第二版 上册)》(陈纪修) P149 例4.4.8
计算由摆线的参数方程\begin{equation*}
	\left\{ \begin{array}{l}
		x=a(t-\sin t), \\
		y=a(1-\cos t)
	\end{array} \right.
\end{equation*}所确定的函数\(y=y(x)\)的二阶导数.
\begin{solution}
求导得\begin{equation*}
	\dv{x}{t}=a(1-\cos t), \qquad
	\dv{y}{t}=a\sin t.
\end{equation*}
令\(\dv{x}{t}\neq0\),
得\(a(1-\cos t)\neq0\),
\(\cos t\neq1\),
即\begin{equation*}
	t\neq 2n\pi
	\quad(n\in\mathbb{Z}).
\end{equation*}
因此\begin{equation*}
	\dv{y}{x}
	=\frac{y'(t)}{x'(t)}
	=\frac{a\sin t}{a(1-\cos t)}
	=\frac{\sin t}{1-\cos t}
	=\cot\frac{t}{2}
	\quad(t\neq 2n\pi,n\in\mathbb{Z}).
\end{equation*}
再求导,得\begin{align*}
	\dv[2]{y}{x}
	&=\dv{t}(\cot\frac{t}{2}) \cdot \dv{t}{x}
	=\dv{t}(\cot\frac{t}{2}) \cdot \frac{1}{x'(t)}
	=\frac{-1}{2\sin^2\frac{t}{2}}
	\cdot\frac{1}{a(1-\cos t)} \\
	&=-\frac{1}{a(1-\cos t)^2}
	\quad(t\neq 2n\pi,n\in\mathbb{Z}).
\end{align*}
\end{solution}
\end{example}

\begin{example}
%@see: 《2009年全国硕士研究生入学统一考试(数学二)》二填空题/第9题
求曲线\begin{equation*}
	\left\{ \begin{array}{l}
		x = \int_0^{1-t} e^{-u^2} \dd{u}, \\
		y = t^2 \ln(2-t^2)
	\end{array} \right.
\end{equation*}在点\((0,0)\)处的切线方程.
\begin{solution}
令\(x = 0, y = 0\)得\(t = 1\).
求导得\begin{equation*}
	\dv{x}{t}
	= -e^{-(1-t)^2},
	\qquad
	\dv{y}{t}
	= 2t \ln(2-t^2) - 2\frac{t^3}{2-t^2},
\end{equation*}
于是\begin{equation*}
	\eval{\dv{y}{x}}_{(0,0)}
	= \eval{\frac{y'_t}{x'_t}}_{t=1}
	= \frac{-2}{-1}
	= 2,
\end{equation*}
那么所求切线方程为\(y = 2 x\).
\end{solution}
\end{example}

\subsection{相关变化率}
假设\(x=x(t)\)和\(y=y(t)\)都是可导函数,
而变量\(x\)与\(y\)之间存在某种关系,
从而变化率\(\dv{x}{t}\)与\(\dv{y}{t}\)之间也存在一定关系.
这两个相互依赖的变化率称为\DefineConcept{相关变化率}.
相关变化率问题就是研究这两个变化率之间的关系,
以便从其中一个变化率求出另一个变化率.

\begin{example}
%@see: 《高等数学(第六版 上册)》 P111 例10
一个气球从离开观察员 $\qty{500}{\meter}$ 处离地铅直上升,
当气球高度为 $\qty{500}{\meter}$ 时,
其速率为 $\qty[per-mode=symbol]{140}{\meter\per\minute}$.
求此时观察员视线的仰角增加的速率是多少?
\begin{solution}
设气球上升\(t\)~\unit{\second} 后,其高度为\(h\),观察员的仰角为\(\alpha\),
则\begin{equation*}
	\tan\alpha=\frac{h}{500},
\end{equation*}
其中\(\alpha\)及\(h\)都与\(t\)存在可导的函数关系.
上式两边对\(t\)求导,得\begin{equation*}
	\sec^2\alpha \cdot \dv{\alpha}{t}
	= \frac1{500} \cdot \dv{h}{t}.
\end{equation*}

由已知条件,在某个时刻(\(t=t_0\)时),
有\(\eval{h}_{t=t_0}=\qty{500}{\meter}\),
\(\eval{\dv{h}{t}}=\qty[per-mode=symbol]{140}{\meter\per\minute}\).
可以计算出\(\eval{\tan\alpha}_{t=t_0}=1\),
从而\(\eval{\sec^2\alpha}_{t=t_0}=2\).
代入上式得\begin{equation*}
	2 \eval{\dv{\alpha}{t}}_{t=t_0} = \frac{140}{500},
\end{equation*}
所以\begin{equation*}
	\eval{\dv{\alpha}{t}}_{t=t_0} = \frac{70}{500}
	= \qty[per-mode=symbol]{0.14}{\radian\per\minute},
\end{equation*}
此时观察员视线的仰角增加的速率是 \qty[per-mode=symbol]{0.14}{\radian\per\minute}.
\end{solution}
\end{example}

\section{函数的微分}
\subsection{微分的定义}
\begin{definition}
%@see: 《高等数学(第六版 上册)》 P113 定义
%@see: 《数学分析(第二版 上册)》(陈纪修) P121 定义4.1.1
设函数\(f\in\mathbb{R}^X\)在点\(x_0\)的某个邻域\(U(x_0)\)内有定义,
\(x_0 + \increment x \in U(x_0)\).
如果存在一个只与\(x_0\)有关,但不依赖于\(\increment x\)的常数\(A\),
使得函数增量\begin{equation*}
	\increment y
	= f(x_0 + \increment x) - f(x_0)
\end{equation*}
满足关系式\begin{equation*}
	\increment y
	= A \increment x + o(\increment x),
\end{equation*}
其中\(o(\increment x)\)是当\(\increment x\to0\)时的无穷小,
那么称“函数\(f\)在点\(x_0\)~\DefineConcept{可微}”,
或称“函数\(f\)在\(x_0\)处的微分存在”;
而把\(A\increment x\)叫做
“函数\(f\)在点\(x_0\)相应于自变量增量\(\increment x\)的\DefineConcept{微分}”,
记作\(\dd{y}\),
即\begin{equation*}
	\dd{y}
	\defeq
	A \increment x;
\end{equation*}
称“\(\dd{y}\)是\(\increment y\)的\DefineConcept{主部}”.
又由于\(\dd{y} = f'(x_0)\increment x\)是\(\increment x\)的线性函数,
所以在\(f'(x_0) \neq 0\)的条件下,
称\(\dd{y}\)为“\(\increment y\)的\DefineConcept{线性主部}”.
\end{definition}

\begin{property}
当\(\increment x\to0\)时,
函数增量\(\increment y\)与函数微分\(\dd{y}\)是等价无穷小,
即\begin{equation*}
	\lim_{\increment x\to0} \frac{\increment y}{\dd{y}}
	= \lim_{\increment x\to0} \frac{\increment y}{f'(x_0) \increment x}
	= \frac{1}{f'(x_0)} \lim_{\increment x\to0} \frac{\increment y}{\increment x}
	= 1.
\end{equation*}
\end{property}

\begin{theorem}
函数\(f(x)\)在点\(x_0\)可微的充分必要条件是:
函数\(f(x)\)在点\(x_0\)可导.
当\(f(x)\)在点\(x_0\)可微时,
其微分一定是\begin{equation*}
	\dd{y}=f'(x_0)\increment x.
\end{equation*}
\end{theorem}

\subsection{微分的运算法则}
由函数的微分的表达式\(\dd{y} = f'(x) \dd{x}\)可知,
要计算函数的微分,
只要计算函数的导数,
再乘以自变量的微分\(\dd{x}\)即可.

\begin{theorem}[函数和、差、积、商的微分法则]
\begin{gather}
	\dd(u \pm v) = \dd{u}\pm\dd{v}, \\
	\dd(C u) = C \dd{u}, \\
	\dd(u v) = v \dd{u} + u \dd{v}, \\
	\dd(\frac{u}{v}) = \frac{v \dd{u} - u \dd{v}}{v^2} \quad (v \neq 0).
\end{gather}
\end{theorem}

\begin{theorem}[复合函数的微分法则]
设函数\(y=f(u)\)及\(u=g(x)\)都可导,则复合函数\(y=f[g(x)]\)的微分为\begin{equation*}
\dd{y}=y'_x\dd{x}=f'(u)g'(x)\dd{x}=\dv{y}{u}\dv{u}{x}\dd{x}.
\end{equation*}

由于\(g'(x)\dd{x}=\dd{u}\),所以,复合函数\(y=f[g(x)]\)的微分公式也可以写成\begin{equation*}
\dd{y}=y'_u\dd{u}=f'(u)\dd{u}=\dv{y}{u}\dd{u}.
\end{equation*}由此可见,无论\(u\)是自变量还是中间变量,微分形式\(\dd{y}=f'(u)\dd{u}\)保持不变.这一性质称为\DefineConcept{微分形式不变性}.这性质表明,当变换自变量时,微分形式\(\dd{y}=f'(u)\dd{u}\)并不改变.
\end{theorem}

\subsection{微分在近似计算中的应用}
\subsubsection{函数的近似计算}
在工程问题中,经常会遇到一些复杂的计算公式.如果直接用这些公式进行计算,那是很费力的.利用微分往往可以把一些复杂的计算公式用简单的近似公式来代替.

前面说过,如果\(y=f(x)\)在点\(x_0\)处的导数\(f'(x_0)\neq0\),且\(\abs{\increment x}\)很小时,我们有\begin{equation*}
\increment y \approx\dd{y} = f'(x_0) \increment x.
\end{equation*}这个式子也可以写成\begin{gather}
\increment y = f(x_0 + \increment x) - f(x_0) \approx f'(x_0) \increment x, \tag1
\end{gather}或\begin{gather}
f(x_0 + \increment x) \approx f(x_0) + f'(x_0) \increment x. \tag2
\end{gather}

在上式令\(x = x_0 + \increment x\),即\(\increment x = x - x_0\),那么上式可改写为\begin{gather}
f(x) \approx f(x_0) + f'(x_0) (x - x_0). \tag3
\end{gather}

如果\(f(x_0)\)和\(f'(x_0)\)都容易计算,那么可利用(1)式来近似计算\(\increment y\),利用(2)式来近似计算\(f(x_0 + \increment x)\),或利用(3)式来近似计算\(f(x)\).这种近似计算的实质就是用\(x\)的线性函数\(f(x_0) + f'(x_0) (x - x_0)\)来近似表达函数\(f(x)\).从导数的几何意义可知,这也就是用曲线\(y=f(x)\)在点\(\opair{x_0,f(x_0)}\)处的切线来近似代替该曲线(就切点邻近部分来说).

如果在(3)式中取\(x_0 = 0\),可得\begin{gather}
f(x) \approx f(0) + f'(0) x. \tag4
\end{gather}

运用(4)式可以推得以下几个在工程上常用的近似公式(下面都假定\(\abs{x}\)是较小的数值):
\begin{gather}
	\sqrt[n]{1+x} \approx 1 + \frac{x}{n}, \\
	\sin x \approx x, \\
	\tan x \approx x, \\
	e^x \approx 1 + x, \\
	\ln(1+x) \approx x.
\end{gather}

\subsubsection{误差估计}
在生产实践中,经常要测量各种数据.
但是有的数据不易直接,
这时我们就通过测量其他有关数据后,根据某种公式算出所要的数据.
例如,要计算圆形钢柱的截面积\(A\),可先用卡尺测量其截面的直径\(D\),
然后根据公式\(A = \frac{\pi}{4} D^2\)算出\(A\).

由于测量仪器的精度、测量的条件和测量的方法等各种因素的影响,
测得的数据往往带有误差,
而根据带有误差的数据计算所得的结果也会有误差,我们把它叫做\DefineConcept{间接测量误差}.

下面就讨论怎样利用微分来估计间接测量误差.
先说明什么叫绝对误差、相对误差.

如果某个量的精确值为\(A\),它的测量值(或近似值)为\(a\),那么\(\abs{A-a}\)叫做\(a\)的\DefineConcept{绝对误差},而绝对误差与\(\abs{a}\)的比值\(\frac{\abs{A-a}}{\abs{a}}\)叫做\(a\)的\DefineConcept{相对误差}.

在实际工作中,某个量的精确值往往是无法知道的,于是绝对误差和相对误差也就无法求得.
但是根据测量仪器的精度等因素,有时能够确定误差在某一范围内.
如果某个量的精确值是\(A\),它的测量值(或近似值)是\(a\),又知道它的误差不超过\(\delta_A\),即\begin{equation*}
\abs{A-a} \leq \delta_A,
\end{equation*}那么\(\delta_A\)叫做测量\(A\)的\DefineConcept{绝对误差限},而\(\frac{\delta_A}{\abs{a}}\)叫做测量\(A\)的\DefineConcept{相对误差限}.

\begin{example}
设测得圆钢截面的直径\(D = 60.03\ mm\),测量\(D\)的绝对误差限\(\delta_D = 0.05\ mm\).利用公式\begin{equation*}
A = \frac{\pi}{4} D^2
\end{equation*}计算圆钢的截面积时,试估计面积的误差.
\begin{solution}
我们把测量\(D\)时所产生的误差当做自变量\(D\)的增量\(\increment D\),
那么,利用公式\(A = \frac{\pi}{4} D^2\)来计算\(A\)时所产生的误差
就是函数\(A\)的对应增量\(\increment A\).当\(\abs{\increment D}\)很小时,
可以利用微分\(\dd{A}\)近似地代替增量\(\increment A\),即\begin{equation*}
\increment A \approx \dd{A} = A' \cdot \increment D = \frac{\pi}{2} D \cdot \increment D.
\end{equation*}由于\(D\)的绝对误差限为\(\delta_D = 0.05\ mm\),所以\begin{equation*}
\abs{\increment D} \leq \delta_D = 0.05,
\end{equation*}而\begin{equation*}
\abs{\increment A} \approx \abs{\dd{A}} = \frac{\pi}{2} D \cdot \abs{\increment D} \leq \frac{\pi}{2} D \cdot \delta_D,
\end{equation*}因此得出\(A\)的绝对误差限约为\begin{equation*}
\delta_A = \frac{\pi}{2} D \cdot \delta_D = \frac{\pi}{2} \times 60.03 \times 0.05 \approx 4.715\ (\mathrm{mm}^2);
\end{equation*}\(A\)的相对误差限约为\begin{equation*}
\frac{\delta_A}{A} = \frac{\frac{\pi}{2} D \cdot \delta_D}{\frac{\pi}{4} D^2}
= 2 \frac{\delta_D}{D} = 2 \times \frac{0.05}{60.03} \approx 0.17\%.
\end{equation*}
\end{solution}
\end{example}

一般地,根据直接测量的\(x\)值按公式\(y = f(x)\)计算\(y\)值时,如果已知测量\(x\)的绝对误差限是\(\delta_x\),即\begin{equation*}
\abs{\increment x} \leq \delta_x,
\end{equation*}那么,当\(y' \neq 0\)时,\(y\)的绝对误差\begin{equation*}
\abs{\increment y} \approx \abs{\dd{y}} = \abs{y'} \cdot \abs{\increment x} \leq \abs{y'} \cdot \delta_x,
\end{equation*}即\(y\)的绝对误差限约为\begin{equation*}
\delta_y = \abs{y'} \cdot \delta_x;
\end{equation*}\(y\)的相对误差限约为\begin{equation*}
\frac{\delta_y}{\abs{y}} = \abs{\frac{y'}{y}} \cdot \delta_x.
\end{equation*}

\section{本章总结}
\subsection*{常用结论}
奇函数的导函数是偶函数.

偶函数的导函数是奇函数.

周期函数的导函数是周期函数.

可导一定连续,
连续不一定可导.


\subsection*{显函数的导数公式}
\begin{table}[htb]
	\centering
	\begin{tblr}{*2{p{8cm}}}
		幂函数 \\ \hline
		%\cref{example:导数.常数函数的导数}
		\(C' = 0\ (\text{$C$是常数})\)
		%\cref{example:导数.幂函数的导数}
		& \((x^\mu)'=\mu x^{\mu-1}\) \\
		\((\sqrt{x})' = \frac1{2\sqrt{x}}\)
		& \((\sqrt[n]{x})' = \frac{\sqrt[n]{x}}{n x}\) \\
		\(\left(\frac1x\right)' = -\frac1{x^2}\)
		& \(\left(\frac1{x^n}\right)' = -\frac{n}{x^{n+1}}\) \\
		\SetCell[c=2]{l}
		\((x^\mu)^{(n)}
		= \mu(\mu-1)(\mu-2)\dotsm(\mu-n+1) \cdot x^{\mu-n}
		= x^{\mu-n}\prod_{k=0}^{n-1} {(\mu - k)}\) \\
		\((x^n)^{(n)} = n!\)
		& \((x^n)^{(n+1)} = 0\) \\
	\end{tblr}
\end{table}

\begin{table}[htb]
	\centering
	\begin{tblr}{*2{p{8cm}}}
		指数函数 \\ \hline
		%\cref{example:导数.指数函数的导数}
		\((e^x)' = e^x\)
		& \((a^x)' = a^x \ln a\) \\
		\((e^x)^{(n)} = e^x\)
		& \((a^x)^{(n)} = a^x \ln^n a\) \\
	\end{tblr}
\end{table}

\begin{table}[htb]
	\centering
	\begin{tblr}{*2{p{8cm}}}
		对数函数 \\ \hline
		%\cref{example:导数.对数函数的导数}
		\((\ln x)' = \frac1x\ (x>0)\)
		& \((\ln\abs{x})' = \frac1x\ (x\neq0)\) \\
		\((\log_a x)' = \frac1{x \ln a}\ (x>0)\) \\
		\([\ln(1+x)]^{(n)} = (-1)^{n-1} \frac{(n-1)!}{(1+x)^n}\) \\
	\end{tblr}
\end{table}

\begin{table}[htb]
	\centering
	\begin{tblr}{*2{p{8cm}}}
		三角函数 \\ \hline
		%\cref{example:导数.正弦函数的导数}
		\((\sin x)' = \cos x\)
		%\cref{example:导数.余弦函数的导数}
		& \((\cos x)' = - \sin x\) \\
		%\cref{example:导数.正切函数的导数}
		\((\tan x)' = \sec^2 x\)
		%\cref{example:导数.余切函数的导数}
		& \((\cot x)' = - \csc^2 x\) \\
		%\cref{example:导数.正割函数的导数}
		\((\sec x)' = \sec x \tan x\)
		%\cref{example:导数.余割函数的导数}
		& \((\csc x)' = - \csc x \cot x\) \\
		\((\sin x)^{(n)} = \sin\left(x+\frac{n\pi}2\right)\)
		& \((\cos x)^{(n)} = \cos\left(x+\frac{n\pi}2\right)\) \\
	\end{tblr}
\end{table}

\begin{table}[htb]
	\centering
	\begin{tblr}{*2{p{8cm}}}
		反三角函数 \\ \hline
		\((\arcsin x)' = \frac1{\sqrt{1 - x^2}} \quad (-1<x<1)\)
		& \((\arccos x)' = - \frac1{\sqrt{1 - x^2}} \quad (-1<x<1)\) \\
		\((\arctan x)' = \frac1{1 + x^2}\)
		& \((\arccot x)' = - \frac1{1 + x^2}\) \\
		\((\arcsec x)' = \frac1{\abs{x} \sqrt{x^2-1}}\)
		& \((\arccsc x)' = -\frac1{\abs{x} \sqrt{x^2-1}}\) \\
	\end{tblr}
\end{table}

\begin{table}[htb]
	\centering
	\begin{tblr}{*2{p{8cm}}}
		双曲函数 \\ \hline
		%\cref{example:导数.双曲正弦函数的导数}
		\((\sinh x)' = \cosh x\)
		%\cref{example:导数.双曲余弦函数的导数}
		& \((\cosh x)' = \sinh x\) \\
		%\cref{example:导数.双曲正切函数的导数}
		\((\tanh x)' = \sech^2 x\)
		%\cref{example:导数.双曲余切函数的导数}
		& \((\coth x)' = -\csch^2 x\) \\
		%\cref{example:导数.双曲正割函数的导数}
		\((\sech x)' = -\sech x \tanh x\)
		%\cref{example:导数.双曲余割函数的导数}
		& \((\csch x)' = -\csch x \coth x\) \\
	\end{tblr}
\end{table}

\begin{table}[htb]
	\centering
	\begin{tblr}{*2{p{8cm}}}
		反双曲函数 \\ \hline
		\((\arsinh x)' = \frac1{\sqrt{1 + x^2}}\)
		& \((\arcosh x)' = \frac1{\sqrt{x^2 - 1}} \quad (x > 1)\) \\
		\((\artanh x)' = \frac1{1 - x^2} \quad(\abs{x}<1)\)
		& \((\arcoth x)' = \frac1{1-x^2} \quad(\abs{x}>1)\) \\
	\end{tblr}
\end{table}

\clearpage
\subsection*{隐函数的导数}
幂指函数\[
	y = u^v
	\quad(u > 0),
\]的导数是\[
	\dv{x} e^{v \ln u}
	= u^v \left( v' \ln u + \frac{u'v}{u} \right).
\]

对于参数方程\[
	\left\{ \begin{array}{l}
		x = \phi(t), \\
		y = \psi(t),
	\end{array} \right.
\]
若函数\(x = \phi(t)\)具有单调连续反函数\(t=\phi^{-1}(x)\),
且\(x = \phi(t)\)和\(y = \psi(t)\)都可导,
且\(\phi'(t) \neq 0\),
则\[
	\dv{y}{x}
	= \frac{\psi'(t)}{\phi'(t)};
\]
若\(x = \phi(t)\)和\(y = \psi(t)\)都二阶可导,
则\[
	\dv[2]{y}{x}
	= \frac{\psi''(t)~\phi'(t) - \psi'(t)~\phi''(t)}{(\phi'(t))^3}.
\]


\chapter{微分中值定理}
\section{微分中值定理}
\subsection{费马引理}
\begin{lemma}[费马引理]\label{theorem:微分中值定理.费马引理}
%@see: 《高等数学(第六版 上册)》 P128 费马引理
%@see: 《数学分析(第二版 上册)》(陈纪修) P167 定理5.1.1(Fermat引理)
%@see: 《数学分析教程(第3版 上册)》(史济怀) P144 定理3.4.1(Fermat)
设函数\(f\colon D\to\mathbb{R}\)在点\(x_0\)的某邻域\(U(x_0)\)内有定义,
并且在\(x_0\)可导.
如果\(\exists\delta>0\),
\(\forall x \in U(x_0,\delta)\),
有\(f(x) \leq f(x_0)\)
或\(f(x) \geq f(x_0)\),
那么\(f'(x_0) = 0\).
\begin{proof}
不妨设\(x \in U(x_0)\)时,
\(f(x) \leq f(x_0)\),
于是,对于\(x_0 + \increment x \in U(x_0)\),有\begin{equation*}
    f(x_0 + \increment x) \leq f(x_0),
\end{equation*}
从而当\(\increment x > 0\)时,\begin{equation*}
    \frac{f(x_0 + \increment x) - f(x_0)}{\increment x} \leq 0;
\end{equation*}
当\(\increment x < 0\)时,\begin{equation*}
    \frac{f(x_0 + \increment x) - f(x_0)}{\increment x} \geq 0.
\end{equation*}
根据函数\(f(x)\)在\(x_0\)可导的条件及极限的保号性,便得到\begin{equation*}
    f'(x_0) = f'_+(x_0)
    = \lim_{\increment x\to0^+}
    \frac{f(x_0 + \increment x) - f(x_0)}{\increment x} \leq 0,
\end{equation*}\begin{equation*}
    f'(x_0) = f'_-(x_0)
    = \lim_{\increment x\to0^-}
    \frac{f(x_0 + \increment x) - f(x_0)}{\increment x} \geq 0.
\end{equation*}
所以,\(f'(x_0) = 0\).

同理,对于当\(x \in U(x_0)\)时,
\(f(x) \geq f(x_0)\)的情形,可以类似地证明.
\end{proof}
\end{lemma}

\subsection{罗尔定理}
\begin{figure}[htb]
	\centering
	\begin{tikzpicture}
		\begin{axis}[
			xmin=0,xmax=8,
			ymin=0,ymax=3,
			axis lines=middle,
			xlabel=$x$,
			ylabel=$y$,
			enlarge x limits=0.1,
			enlarge y limits=0.1,
			ticks=none,
		]
			\addplot[domain=1:7.283,color=blue]{.5*sin(deg(x-1))+2};
			\draw(1,2)coordinate(A)node[left]{$A$}
				(1,0)coordinate(a)node[below]{$a$}
				(2.571,2.5)coordinate(C)node[above]{$C$}
				(2.571,0)coordinate(c)node[below]{$\xi$}
				(5.713,1.5)node[below]{$D$}
				(7.283,2)coordinate(B)node[right]{$B$}
				(7.283,0)coordinate(b)node[below]{$b$};
			\draw[dashed,black!30](1,2)--(7.283,2)
				(A)--(a) (C)--(c) (B)--(b);
		\end{axis}
	\end{tikzpicture}
	\caption{}
	\label{figure:微分中值定理.罗尔定理的几何意义}
\end{figure}

观察\cref{figure:微分中值定理.罗尔定理的几何意义},
设曲线弧\(\Arc{AB}\)是函数\(y=f(x)\ (a\leq x\leq b)\)的图形.
这是一条连续的曲线弧,除端点外,处处有不垂直于\(x\)的切线,
且两个端点的纵坐标相等,即\(f(a)=f(b)\).
可以发现在曲线弧的最高点\(C\)处或最低点\(D\)处,曲线有水平的切线.
如果记\(C\)的横坐标为\(\xi\),那么就有\(f'(\xi)=0\).
现在用分析语言把这个几何现象描述出来,就可得下面的罗尔定理.

\begin{theorem}[罗尔定理]\label{theorem:微分中值定理.罗尔定理}
%@see: 《高等数学(第六版 上册)》 P129 罗尔定理
%@see: 《数学分析(第二版 上册)》(陈纪修) P168 定理5.1.2(Rolle定理)
%@see: 《数学分析教程(第3版 上册)》(史济怀) P144 定理3.4.2(Rolle)
如果函数\(f\colon [a,b]\to\mathbb{R}\)
在闭区间\([a,b]\)上连续,在开区间\((a,b)\)内可导,
即\(f \in C[a,b] \cap D(a,b)\),
并且\(f(a)=f(b)\),
那么\(\exists \xi \in (a,b)\),
使得\(f'(\xi) = 0\).
\begin{proof}
由于\(f\)在闭区间\([a,b]\)上连续,
根据闭区间上连续函数的最大值最小值定理,
\(f\)在闭区间\([a,b]\)上必定取得它的最大值\(M\)和最小值\(m\).
这样,只有两种可能情形:
\begin{enumerate}
	\item[情况一] \(M=m\).
		这时\(f\)在区间\([a,b]\)上是常数函数,即\(f(x)=M\).
		由此,\(\forall x\in(a,b)\),有\(f'(x)=0\).
		因此,任取\(\xi\in(a,b)\),有\(f'(\xi)=0\).

	\item[情况二] \(M>m\).
		因为\(f(a)=f(b)\),
		所以\(M\)和\(m\)这两个数中,
		至少有一个不等于\(f(x)\)在区间\([a,b]\)的端点处的函数值.
		不妨设\(M \neq f(a)\),
		那么必定在开区间\((a,b)\)内有一点\(\xi\)使\(f(\xi)=M\).
		因此,\(\forall x\in[a,b]\),
		有\(f(x) \leq f(\xi)\),从而由费马引理可知\(f'(\xi)=0\).
		\qedhere
\end{enumerate}
\end{proof}
\end{theorem}

\begin{example}
% \hyperref[theorem:微分中值定理.罗尔定理]{罗尔定理}的第三个条件是必需的.
设函数\(f\colon[a,b]\to\mathbb{R}\)
满足\(f\in C[a,b]\cap D(a,b)\)
且\((\exists\xi\in(a,b))[f'(\xi)=0]\),
却不一定有\((\exists x_1,x_2\in[a,b])[f(x_1)=f(x_2)]\).
例如,取\(f(x)=x^5\),
它满足\(f\in C[-1,1]\cap D(-1,1)\)和\(f'(0)=0\),
但是它在\([-1,1]\)上函数值处处不相等.
\end{example}

\begin{example}
设函数\(f\colon[a,b]\to\mathbb{R}\)
满足\(f\in C[a,b]\),
\(f(a)=f(b)\),
以及\(f\)在\((a,b)\)内除一点外处处可导,
但是\((\nexists\xi\in(a,b))[f'(\xi)=0]\).
例如,取\begin{equation*}
	f(x) = \left\{ \begin{array}{cl}
		x, & 0\leq x<1, \\
		2-x, & 1\leq x\leq 2.
	\end{array} \right.
\end{equation*}
它满足\(f\in C[0,2]\),\(f(0)=f(2)=0\).
显然除点\(x=1\)以外,\(f\)处处可导,
且\begin{equation*}
	f'(x) = \left\{ \begin{array}{rl}
		1, & 0<x<1, \\
		-1, & 1<x<2.
	\end{array} \right.
\end{equation*}
所以在\((0,2)\)内不存在导数为零的点.
\end{example}

\begin{example}
%@see: 《数学分析(第五版 上册)》(华东师范大学) P112 例1
设\(f \in D(-\infty,+\infty)\).
证明:若方程\(f'(x) = 0\)没有实根,
则方程\(f(x) = 0\)至多只有一个实根.
\begin{proof}
用反证法.
假设方程\(f(x) = 0\)由两个相异实根\(x_1,x_2\)(不妨设\(x_1 < x_2\)),
则函数\(f\)在\([x_1,x_2]\)上满足\hyperref[theorem:微分中值定理.罗尔定理]{罗尔定理}的三个条件,
从而存在\(\xi\in(x_1,x_2)\),使得\(f'(\xi) = 0\),
这与\(f'(x) \neq 0\)的题设矛盾,
因此方程\(f(x) = 0\)至多只有一个实根.
\end{proof}
\end{example}

\begin{corollary}
设函数\(f \in D(a,b)\),且\begin{equation*}
	\lim_{x \to a^+} f(x)
	= \lim_{x \to b^-} f(x),
\end{equation*}
则\(\exists\xi\in(a,b)\),
使得\(f'(\xi) = 0\).
\begin{proof}
令\begin{equation*}
	g(x) = \left\{ \begin{array}{cl}
		f(x), & a < x < b, \\
		A, & x = a \lor x = b,
	\end{array} \right.
\end{equation*}
其中\(A = \lim_{x \to a^+} f(x)\),
那么函数\(g\)在\([a,b]\)上连续,在\((a,b)\)内可导.

由\hyperref[theorem:微分中值定理.罗尔定理]{罗尔定理}可知,
存在\(\xi\in(a,b)\)使得\(g'(\xi) = f'(\xi) = 0\).
\end{proof}
\end{corollary}

\begin{example}\label{example:微分中值定理.一元高次方程的根的存在性}
%@see: 《高等数学(第六版 上册)》 P134 习题3-1 7.
若方程\(a_0 x^n + a_1 x^{n-1} + \dotsb + a_{n-1} x = 0\)有一个正根\(x = x_0\).
证明:方程\begin{equation*}
	a_0 n x^{n-1} + a_1 (n-1) x^{n-2} + \dotsb + a_{n-1} = 0
\end{equation*}必有一个小于\(x_0\)的正根.
\begin{proof}
设\(f(x) = a_0 x^n + a_1 x^{n-1} + \dotsb + a_{n-1} x\),
则\(f(0) = f(x_0) = 0\),
而\begin{equation*}
	f'(x) = a_0 n x^{n-1} + a_1 (n-1) x^{n-2} + \dotsb + a_{n-1}.
\end{equation*}
根据\hyperref[theorem:微分中值定理.罗尔定理]{罗尔定理},
\(\exists \xi \in (0,x_0)\)使得\(f'(\xi) = 0\),
即\(\xi\)就是小于\(x_0\)的正根.
\end{proof}
\end{example}
\begin{example}
%@see: 《高等数学(第六版 上册)》 P134 习题3-1 8.
设函数\(f \in D^2(a,b)\),且\(f(x_1) = f(x_2) = f(x_3)\),其中\(a < x_1 < x_2 < x_3 < b\).
证明:在\(x_1,x_3\)内至少有一点\(\xi\),使得\(f''(\xi) = 0\).
\begin{proof}
由\hyperref[theorem:微分中值定理.罗尔定理]{罗尔定理}可知,
\(\exists y_1\in(x_1,x_2)\)使得\(f'(y_1) = 0\),
\(\exists y_2\in(x_2,x_3)\)使得\(f'(y_2) = 0\).
再次应用\hyperref[theorem:微分中值定理.罗尔定理]{罗尔定理}可知
\(\exists \xi\in(y_1,y_2)\subseteq(x_1,x_3)\)
使得\(f''(\xi) = 0\).
\end{proof}
\end{example}
\begin{example}
%@see: 《数学分析(第二版 上册)》(陈纪修) P183 习题 14.
证明对于每个正整数\(n\ (n\geq2)\),方程\begin{equation*}
	x^n + x^{n-1} + \dotsb + x^2 + x = 1
\end{equation*}在\((0,1)\)内必有唯一的实根\(x_n\),
并计算极限\(\lim_{n\to\infty} x_n\).
%TODO proof
\end{example}

\begin{example}
%@see: 《数学分析(第二版 上册)》(陈纪修) P169 例5.1.1
证明:\(n\)次\DefineConcept{拉格朗日多项式}\begin{equation*}
	p_n(x) = \frac1{2^n n!} \dv[n]{x} (x^2-1)^n
	\quad(n=0,1,2,\dotsc)
\end{equation*}在\((-1,1)\)上恰有\(n\)个不同的根.
%TODO proof
% \begin{proof}
% 由高阶导数的莱布尼茨公式,容易知道
% \end{proof}
\end{example}

\begin{example}
%@see: https://www.bilibili.com/video/BV1XWxreZEGP/
设函数\(f \in C[0,3] \cap D(0,3)\),
且\begin{equation*}
	f(0) + f(1) + f(2) = 3,
	\qquad
	f(3) = 1.
\end{equation*}
证明:存在\(\xi\in(0,3)\)使得\(f'(\xi) = 0\).
\begin{proof}
因为\(f \in C[0,2]\),
所以由\hyperref[theorem:极限.最值定理]{最值定理}可知
函数\(f\)在区间\([0,2]\)上有界,
即\begin{equation*}
	(\forall x\in[0,2])
	[m \leq f(x) \leq M],
\end{equation*}
其中\begin{equation*}
	m = \min\Set{ f(x) \given x\in[0,2] },
	\qquad
	M = \max\Set{ f(x) \given x\in[0,2] },
\end{equation*}
于是\begin{equation*}
	m = \frac{m+m+m}3 \leq \frac{f(0) + f(1) + f(2)}3 = 1 \leq \frac{M+M+M}3 = M,
\end{equation*}
再由\hyperref[theorem:极限.闭区间上连续函数的性质.介值定理2]{介值定理}可知
存在\(c\in[0,2]\)使得\(f(c) = 1\),
那么由\hyperref[theorem:微分中值定理.罗尔定理]{罗尔定理}可知
存在\(\xi\in(c,3)\subseteq(0,3)\)使得\(f'(\xi) = 0\).
\end{proof}
\end{example}

\begin{example}
%@see: https://www.bilibili.com/video/BV1ty4y1w74x/
%@see: https://www.bilibili.com/video/BV1GX4y1R7p9/
在函数\(f\)在\([0,1]\)上二阶可导,
且\(f(0) = f'(0) = 0\).
证明:存在\(\xi\in(0,1)\),
使得\begin{equation*}
	f''(\xi) = \frac{2 f(\xi)}{(1-\xi)^2}.
\end{equation*}
\begin{proof}
注意到\begin{align*}
	&f''(x) = \frac{2 f(x)}{(1-x)^2} \\
	&\iff
	(x-1)^2 f''(x) - 2 f(x) = 0 \\
	&\iff
	(x-1)^2 f''(x) + 2 (x-1) f'(x) - 2 (x-1) f'(x) - 2 f(x) = 0 \\
	&\iff
	[(x-1)^2 f'(x)]' - [2 (x-1) f(x)]' = 0,
\end{align*}
构造辅助函数\(F(x) = (x-1)^2 f'(x) - 2 (x-1) f(x)\),
那么\begin{equation*}
	F(0) = F(1) = 0.
\end{equation*}
由\hyperref[theorem:微分中值定理.罗尔定理]{罗尔定理}可知
\(\exists\xi\in(0,1)\)使得\(F'(\xi) = 0\).
\end{proof}
\end{example}
\begin{remark}
这个例子启发我们,当遇到形如\begin{equation*}
	f(x) g''(x) - f''(x) g(x) = 0
\end{equation*}的关系式时,
可以构造辅助函数\begin{equation*}
	F(x) = f(x) g'(x) - f'(x) g(x).
\end{equation*}
\end{remark}

\begin{example}
%@see: 《2007年全国硕士研究生入学统一考试(数学一)》三解答题/第19题
%@see: https://www.bilibili.com/video/BV1bmyGYvEBc/
设\(f,g \in C[a,b] \cap D^2(a,b)\)且在\((a,b)\)内具有相等的最大值,
而\begin{equation*}
	f(a) = g(a),
	\qquad
	f(b) = g(b).
\end{equation*}
证明:存在\(\xi\in(a,b)\)使得\(f''(\xi) = g''(\xi)\).
\begin{proof}
设\(x_1,x_2\in(a,b)\)满足\begin{equation*}
	f(x_1) = \max_{a<x<b} f(x) = M = \max_{a<x<b} g(x) = g(x_2).
\end{equation*}
令\(F(x) = f(x) - g(x)\),
由题意有\(F(a) = F(b) = 0\).

当\(x_1 = x_2\)时,有\begin{equation*}
	F(x_1) = f(x_1) - g(x_1) = M - M = 0.
	\eqno(1)
\end{equation*}
当\(x_1 \neq x_2\)时,有\begin{equation*}
	F(x_1) = f(x_1) - g(x_1)
	= M - g(x_1)
	\geq 0,
	\qquad
	F(x_2) = f(x_2) - g(x_2)
	= f(x_2) - M
	\leq 0,
\end{equation*}
记\(x_1' = \min\{x_1,x_2\},
x_2' = \max\{x_1,x_2\}\),
那么由\hyperref[theorem:极限.零点定理]{零点定理}可知,
存在\(x_3\in[x_1',x_2']\)使得\begin{equation*}
	F(x_3) = 0.
	\eqno(2)
\end{equation*}
记\begin{equation*}
	c = \left\{ \begin{array}{cl}
		x_1, & x_1 = x_2, \\
		x_3, & x_1 \neq x_2.
	\end{array} \right.
\end{equation*}
由\hyperref[theorem:微分中值定理.罗尔定理]{罗尔定理}可知,
存在\(\xi_1\in(a,c)\),存在\(\xi_2\in(c,b)\),
使得\begin{equation*}
	F'(\xi_1) = F'(\xi_2) = 0.
\end{equation*}
再由\hyperref[theorem:微分中值定理.罗尔定理]{罗尔定理}可知,
存在\(\xi\in(\xi_1,\xi_2)\subseteq(a,b)\)使得\begin{equation*}
	F''(\xi) = f''(\xi) - g''(\xi) = 0.
	\qedhere
\end{equation*}
\end{proof}
\end{example}
\begin{example}
%@see: 《2017年全国硕士研究生入学统一考试(数学一)》三解答题/第18题
设函数\(f\)在区间\([0,1]\)上具有\(2\)阶导数,
且\(f(1) > 0,
\lim_{x\to0^+} \frac{f(x)}{x} < 0\).
证明:\begin{itemize}
	\item 方程\(f(x) = 0\)在区间\((0,1)\)内至少存在一个实根;
	\item 方程\(f(x) f''(x) + (f'(x))^2 = 0\)在区间\((0,1)\)内至少存在两个不同实根.
\end{itemize}
\begin{solution}
因为\(\lim_{x\to0^+} \frac{f(x)}{x} < 0\),
所以\begin{equation*}
	\lim_{x\to0^+} f(x)
	= \lim_{x\to0^+} \frac{f(x)}{x} \cdot x
	= \lim_{x\to0^+} \frac{f(x)}{x} \cdot \lim_{x\to0^+} x
	= 0,
\end{equation*}
从而有\begin{equation*}
	f(0) = \lim_{x\to0^+} f(x) = 0;
	\eqno(1)
\end{equation*}
另外由\hyperref[theorem:极限.函数极限的局部保号性1]{函数极限的局部保号性}可知,
存在\(\delta\in(0,1)\),
当\(0<x<\delta\)时,
成立\begin{equation*}
	\frac{f(x)}{x} < 0.
\end{equation*}
那么,只要任意取定\(x_1\in(0,\delta)\subset(0,1)\),
就有\begin{equation*}
	f(x_1) < 0.
	\eqno(2)
\end{equation*}
因为\(f\)在区间\([0,1]\)上具有\(2\)阶导数,
所以\(f\)在\([0,1]\)上连续.
又因为\(f(1) > 0\),
所以由\hyperref[theorem:极限.零点定理]{零点定理}可知,
存在\(x_2\in(x_1,1)\subset(0,1)\),
使得\begin{equation*}
	f(x_2) = 0.
	\eqno(3)
\end{equation*}
这就是说,点\(x_2\)是方程\(f(x) = 0\)在区间\((0,1)\)内的一个实根.

令\(F(x) = f(x) f'(x)\),
则\(F'(x) = f(x) f''(x) + (f'(x))^2\).
由于\(f(0) = f(x_2) = 0\),
所以由\hyperref[theorem:微分中值定理.罗尔定理]{罗尔定理}可知,
存在\(x_3\in(0,x_2)\),
使得\begin{equation*}
	f'(x_3) = 0.
	\eqno(4)
\end{equation*}
于是\begin{equation*}
	F(0) = F(x_3) = F(x_2) = 0.
	\eqno(5)
\end{equation*}
又由\hyperref[theorem:微分中值定理.罗尔定理]{罗尔定理}可知,
存在\(\xi_1\in(0,x_3)\subset(0,1)\),
存在\(\xi_2\in(x_3,x_2)\subset(0,1)\),
使得\begin{equation*}
	F'(\xi_1) = F'(\xi_2) = 0.
	\qedhere
\end{equation*}
\end{solution}
\end{example}

\subsection{拉格朗日中值定理}
\begin{theorem}[拉格朗日中值定理]\label{theorem:微分中值定理.拉格朗日中值定理}
%@see: 《高等数学(第六版 上册)》 P129 拉格朗日中值定理
%@see: 《数学分析(第二版 上册)》(陈纪修) P170 定理5.1.3(Lagrange中值定理)
%@see: 《数学分析教程(第3版 上册)》(史济怀) P146 定理3.4.3(Lagrange)
如果函数\(f\colon[a,b]\to\mathbb{R}\)
在闭区间\([a,b]\)上连续,在开区间\((a,b)\)内可导,
即\(f \in C[a,b] \cap D(a,b)\),
那么\(\exists \xi \in (a,b)\),
使得\begin{equation}\label{equation:微分中值定理.拉格朗日中值公式}
	f(b) - f(a) = f'(\xi) \cdot (b-a).
\end{equation}
\begin{proof}
令\begin{equation*}
	\phi(x)=f(x)-f(a)-\frac{f(b)-f(a)}{b-a}(x-a).
\end{equation*}
不难得\(\phi(a)=\phi(b)=0\),
\(\phi\in C[a,b]\cap D(a,b)\),
且\begin{equation*}
	\phi'(x)=f'(x)-\frac{f(b)-f(a)}{b-a}.
\end{equation*}
根据\hyperref[theorem:微分中值定理.罗尔定理]{罗尔定理}可知,
在\(\exists\xi\in(a,b)\)使得\(\phi'(\xi)=0\),
即\begin{equation*}
	f'(\xi)-\frac{f(b)-f(a)}{b-a}=0,
	\quad\text{或}\quad
	\frac{f(b)-f(a)}{b-a}=f'(\xi),
\end{equation*}
亦即\begin{equation*}
	f(b)-f(a)=f'(\xi)(b-a).
	\qedhere
\end{equation*}
\end{proof}
\end{theorem}
\cref{equation:微分中值定理.拉格朗日中值公式} 叫做\DefineConcept{拉格朗日中值公式}.

设\(x\)为区间\([a,b]\)内一点,
\(x+\increment x\)为这区间内的另一点(\(\increment x \gtrless 0\)),
则\cref{equation:微分中值定理.拉格朗日中值公式}
在区间\([x,x+\increment x]\)(当\(\increment x>0\)时)
或在区间\([x+\increment x,x]\)(当\(\increment x<0\)时)上
就成为\begin{equation}
	f(x+\increment x) - f(x)
	= f'(x+\theta \increment x) \cdot \increment x
	\quad(0<\theta<1).
\end{equation}

如果记\(f(x)\)为\(y\),\(\increment y = f(x+\increment x) - f(x)\),
则上式又可写成
\begin{equation}\label{equation:微分中值定理.有限增量公式}
	\increment y = f'(x+\theta \increment x) \cdot \increment x
	\quad(0<\theta<1).
\end{equation}
我们知道,函数的微分\(\dd{y} = f'(x) \cdot \increment x\)是
函数的增量\(\increment y\)的近似表达式.
一般说来,以\(\dd{y}\)近似代替\(\increment y\)时所产生的误差
只有当\(\increment x\to0\)时才趋于零;
而\cref{equation:微分中值定理.有限增量公式}
却给出了自变量取得有限增量\(\increment x\)(\(\abs{\increment x}\)不一定很小)时,
函数增量\(\increment y\)的准确表达式.
因此,这个定理也叫做\DefineConcept{有限增量定理},
\cref{equation:微分中值定理.有限增量公式} 称为\DefineConcept{有限增量公式}.

\begin{remark}
对于\hyperref[theorem:微分中值定理.拉格朗日中值定理]{拉格朗日中值定理},
我们需要注意到:
通常来说\(\xi\)不是唯一的.
只有当函数\(f\)是严格单调函数时,\(\xi\)是\(x\)的函数,
例如:\begin{itemize}
	\item 由\(e^x - e^0 = x e^\xi\)可得\(\xi = \ln\frac{e^x-1}{x}\),
	而\(\lim_{x\to0} \frac{\xi}{x} = \frac12\).
	\item 由\(\ln x - \ln1 = (x-1) \frac1\xi\)可得\(\xi = \frac{x-1}{\ln x}\),
	而\(\lim_{x\to0} \frac{\xi}{x} = \infty\).
\end{itemize}
\end{remark}

\begin{example}\label{example:微分中值定理.拉格朗日中值定理.重要不等式1}
证明:当\(x>0\)时,成立不等式\begin{equation*}
	\frac{x}{1+x} < \ln(1+x) < x.
\end{equation*}
\begin{proof}
设\(f(t) = \ln(1+t)\),
显然\(f(t)\)在区间\([0,x]\)上满足拉格朗日中值定理的条件,
那么存在\(\xi\in(0,x)\)
使得\begin{equation*}
	f(x)-f(0)=f'(\xi)\cdot(x-0).
\end{equation*}
由于\(f(0)=0\),
\(f'(\xi)=\frac1{1+\xi}\),
故\begin{equation*}
	\ln(1+x) = \frac{x}{1+\xi}.
\end{equation*}
又由\(0<\xi<x\),
有\begin{equation*}
	\frac{x}{1+x}<\frac{x}{1+\xi}<x,
\end{equation*}
所以\begin{equation*}
	\frac{x}{1+x}<\ln(1+x)<x, \quad x > 0.
	\qedhere
\end{equation*}
\end{proof}
\end{example}
\begin{remark}
同理可证:当\(-1<x<0\)时,
仍有\(\frac{x}{1+x}<\ln(1+x)<x\)成立.
于是\cref{example:微分中值定理.拉格朗日中值定理.重要不等式1} 中
不等式的成立条件可以扩大为\(x>-1\).
\end{remark}

\begin{example}\label{example:微分中值定理.拉格朗日中值定理.欧拉--马歇罗尼常数}
证明:极限\(\lim_{n\to\infty} \left(\sum_{k=1}^n \frac{1}{k} - \ln n\right)\)收敛.
\begin{proof}
记\(x_n = \sum_{k=1}^n \frac{1}{k} - \ln n\).

利用\cref{example:微分中值定理.拉格朗日中值定理.重要不等式1} 的结论,
任取\(n\in\mathbb{N}^+\),
令\(x=\frac{1}{n}\),
则有\begin{equation*}
	\frac{1}{n+1} < \ln(1+\frac{1}{n}) < \frac{1}{n}.
\end{equation*}
因此\begin{equation*}
	x_{n+1} - x_n = \frac{1}{n+1} - \ln\frac{n+1}{n}
	= \frac{1}{n+1} - \ln(1+\frac{1}{n}) < 0,
\end{equation*}
可知\(\{x_n\}\)是单调减少数列.
又因为\begin{align*}
	x_n &= 1 + \frac{1}{2} + \dotsb + \frac{1}{n} - \ln n \\
	&> \ln(1+1) + \ln(1+\frac{1}{2}) + \dotsb + \ln(1+\frac{1}{n}) - \ln n \\
	&= (\ln2-\ln1)+(\ln3-\ln2)+\dotsb+[\ln(n+1)-\ln n] - \ln n \\
	&= \ln(n+1) - \ln n
	= \ln(1+\frac{1}{n})
	> \frac{1}{n+1} > 0,
\end{align*}
可知\(\{x_n\}\)有界.
综上,根据\hyperref[theorem:极限.函数的单调有界定理]{单调有界定理}可知,
数列\(\{x_n\}\)收敛于有限值.
\end{proof}
\end{example}
\begin{remark}
我们把极限\(\lim_{n\to\infty} \left(\sum_{k=1}^n \frac{1}{k} - \ln n\right)\)
称为\DefineConcept{欧拉--马歇罗尼常数}(Euler--Mascheroni Constant),
记作\(\gamma\),
其在数值上近似等于{0.577~216}.
%@Mathematica: N[EulerGamma, 6]
%@see: https://mathworld.wolfram.com/Euler-MascheroniConstant.html
\end{remark}

我们知道,如果函数\(f\)在某一区间上是一个常数,
那么\(f\)在该区间上的导数恒为零.作为拉格朗日中值定理的一个应用,
可以推出以上命题的逆命题也是成立的,即:
\begin{theorem}
%@see: 《高等数学(第六版 上册)》 P131 定理
%@see: 《数学分析(第二版 上册)》(陈纪修) P171 定理5.1.4
如果函数\(f\colon(a,b)\to\mathbb{R}\)在\((a,b)\)上的导数恒为零,
那么\(f\)在\((a,b)\)上恒为常数.
\begin{proof}
设\(x_1,x_2\in(a,b)\)且\(x_1<x_2\),
在\([x_1,x_2]\)上应用\hyperref[theorem:微分中值定理.拉格朗日中值定理]{拉格朗日中值定理},
便知\begin{equation*}
	(\exists\xi\in(x_1,x_2)\subseteq(a,b))
	[f(x_2) - f(x_1) = f'(\xi) (x_2 - x_1)].
\end{equation*}
因为\(f'(\xi) = 0\),
所以\(f(x_1) = f(x_2)\).
考虑到\(x_1,x_2\)的任意性,就知道\(f\)在\((a,b)\)上恒为常数.
\end{proof}
\end{theorem}

\begin{example}
证明恒等式:\begin{equation*}
	\arcsin x + \arccos x = \frac{\pi}{2},
	\quad -1 \leq x \leq 1.
\end{equation*}
\begin{proof}
设\(f(t) = \arcsin t + \arccos t\ (-1 \leq t \leq 1)\),求导得\begin{equation*}
	f'(t) = \dv{t} \arcsin t + \dv{t} \arccos t
	= \frac{1}{\sqrt{1-t^2}} - \frac{1}{\sqrt{1-t^2}} = 0.
\end{equation*}
这说明\(f(t)\)在区间\([-1,1]\)上是常数.

代入\(x=0\)得\(\arcsin 0 = 0\),\(\arccos 0 = \pi/2\),那么\begin{equation*}
	f(x) = f(0) \equiv \arcsin 0 + \arccos 0 = \frac{\pi}{2},
	\quad -1 \leq x \leq 1.
	\qedhere
\end{equation*}
\end{proof}
\end{example}

\begin{example}
计算极限\(\lim_{x\to0} \left[\sin x - \sin(\sin x)\right]\).
\begin{solution}
由\hyperref[theorem:微分中值定理.拉格朗日中值定理]{拉格朗日中值定理},
% 当\(x>0\)时,有\(0 < \sin x < x\);
% 当\(x<0\)是,有\(x < \sin x < 0\).
存在\(\xi\)在\(x\)和\(\sin x\)之间,
使得\begin{equation*}
	\sin x - \sin(\sin x)
	= \cos\xi (x-\sin x).
\end{equation*}
当\(x\to0\)时,\(\sin x\to0\),
故\(\xi\to0\),\(\cos\xi\to1\),那么\begin{equation*}
	\lim_{x\to0} \left[\sin x - \sin(\sin x)\right]
	= \lim_{x\to0} \cos\xi \cdot
		\left(\lim_{x\to0} x - \lim_{x\to0} \sin x\right)
	= 1 \cdot (0-0) = 0.
\end{equation*}
\end{solution}
\end{example}
\begin{remark}
上例也可直接根据\cref{theorem:极限.连续函数的极限3} 得到,
即直接将\(x=0\)代入函数\(f(x) = \sin x - \sin(\sin x)\)中.
\end{remark}
\begin{example}%武忠祥
计算极限\(\lim_{x\to+\infty} x \left( \frac\pi4 - \arctan\frac{x}{x+1} \right)\).
\begin{solution}
注意到\(\arctan1 = \frac\pi4\),
并且当\(x>0\)时,\(\frac{x}{x+1} < 1\)恒成立.
因为\(\arctan\)在\((-\infty,+\infty)\)连续且可导,
所以由\hyperref[theorem:微分中值定理.拉格朗日中值定理]{拉格朗日中值定理}可知
存在\(\xi\in\left( \frac{x}{x+1},1 \right)\)使得\begin{equation*}
	\arctan1 - \arctan\frac{x}{x+1}
	= \eval{(\arctan x)'}_{x=\xi} \cdot \left( 1 - \frac{x}{x+1} \right)
	= \frac1{1+\xi^2} \cdot \left( 1 - \frac{x}{x+1} \right).
\end{equation*}
因为当\(x\to+\infty\)时,有\(\frac{x}{x+1} \to 1\),所以\(\xi \to 1\).
于是\begin{align*}
	\lim_{x\to+\infty} x \left( \frac\pi4 - \arctan\frac{x}{x+1} \right)
	&= \lim_{x\to+\infty} x
		\cdot \frac1{1+\xi^2}
		\cdot \left( 1 - \frac{x}{x+1} \right) \\
	&= \lim_{x\to+\infty} \frac1{1+\xi^2}
		\cdot \lim_{x\to+\infty} \frac{x}{x+1}
	= \frac12.
\end{align*}
\end{solution}
\end{example}
\begin{example}
%@see: https://www.bilibili.com/video/BV1DZ421K7XU/
计算极限\(\lim_{x\to0^+} \frac{(1+x)^{\frac1x} - e}{x}\).
\begin{solution}
我们已经知道\begin{equation*}
	(1+x)^{\frac1x}
	= e^{\frac1x \ln(1+x)}
	\to e^1
	\quad(x\to0^+).
\end{equation*}
因为函数\(x \mapsto e^x\)在\((-\infty,+\infty)\)连续且可导,
所以由\hyperref[theorem:微分中值定理.拉格朗日中值定理]{拉格朗日中值定理}可知,
在\(1\)与\(\frac1x \ln(1+x)\)之间存在\(\xi\),
使得\begin{equation*}
	\frac{(1+x)^{\frac1x} - e}{x}
	= \frac{e^\xi}{x} \left[ \frac{\ln(1+x)}{x} - 1 \right]
	= e^\xi \frac{\ln(1+x) - x}{x^2}.
\end{equation*}
当\(x\to0^+\)时,\(\xi\to1\),从而\(e^\xi \to e\),
另外有\begin{equation*}
	\lim_{x\to0^+} \frac{\ln(1+x) - x}{x^2}
	= \lim_{x\to0^+} \frac{-\frac12 x^2}{x^2}
	= -\frac12.
\end{equation*}
\end{solution}
\end{example}

\begin{example}
%@see: 《高等数学(第六版 上册)》 P134 习题3-1 9.
设\(a>b>0,n>1\).
证明:\begin{equation*}
	n b^{n-1} (a-b) < a^n-b^n < n a^{n-1} (a-b).
\end{equation*}
\begin{proof}
\begin{proof}[证法一]
注意到\begin{equation*}
	a^n-b^n
	= (a-b)(a^{n-1} + a^{n-2} b + \dotsb + a b^{n-2} + b^{n-1}).
\end{equation*}
由于\(a>b>0\),
所以\begin{equation*}
	n b^{n-1} < a^{n-1} + a^{n-2} b + \dotsb + a b^{n-2} + b^{n-1} < n a^{n-1},
\end{equation*}
于是\begin{equation*}
	n b^{n-1} (a-b) < a^n-b^n < n a^{n-1} (a-b).
	\qedhere
\end{equation*}
\end{proof}
\begin{proof}[证法二]
令\(f(x) = x^n\),
显然\(f\)在\((-\infty,+\infty)\)内具有任意阶导数,
并且\begin{equation*}
	f'(x) = n x^{n-1},
	\qquad
	f''(x) = n(n-1) x^{n-2},
\end{equation*}
由拉格朗日中值定理可知\(\exists\xi\in(b,a)\)使得\begin{equation*}
	\frac{a^n-b^n}{a-b}
	= \frac{f(a)-f(b)}{a-b}
	= f'(\xi).
\end{equation*}
注意到当\(x>0\)时\(f''(x)>0\),即\(f'\)在\((0,+\infty)\)上严格单调增加,
于是\begin{equation*}
	n b^{n-1} = f'(b) < f'(\xi) < f'(a) = n a^{n-1}.
\end{equation*}
因此\begin{equation*}
	n b^{n-1} (a-b) < a^n-b^n < n a^{n-1} (a-b).
	\qedhere
\end{equation*}
\end{proof}\let\qed\relax
\end{proof}
\end{example}
\begin{example}
%@see: 《高等数学(第六版 上册)》 P134 习题3-1 10.
设\(a>b>0\).
证明:\begin{equation*}
	\frac{a-b}{a} < \ln\frac{a}{b} < \frac{a-b}{b}.
\end{equation*}
\begin{proof}
由拉格朗日中值定理可知\(\exists\xi\in(b,a)\)使得\begin{equation*}
	\ln\frac{a}{b}
	= \ln a - \ln b
	= \frac1\xi (a-b).
\end{equation*}
又因为函数\(x \mapsto \frac1x\)在\((0,+\infty)\)上严格单调减少,
所以\(\frac1a < \frac1\xi < \frac1b\),
因此\begin{equation*}
	\frac{a-b}{a} < \ln\frac{a}{b} < \frac{a-b}{b}.
	\qedhere
\end{equation*}
\end{proof}
\end{example}
\begin{example}
%@see: 《高等数学(第六版 上册)》 P134 习题3-1 13.
设\(f,g \in C[a,b] \cap D(a,b)\).
证明:\(\exists\xi\in(a,b)\)使得\begin{equation*}
	\begin{vmatrix}
		f(a) & f(b) \\
		g(a) & g(b)
	\end{vmatrix}
	= (b-a) \begin{vmatrix}
		f(a) & f'(\xi) \\
		g(a) & g'(\xi)
	\end{vmatrix}.
\end{equation*}
\begin{proof}
令\begin{equation*}
	F(x) = \begin{vmatrix}
		f(a) & f(x) \\
		g(a) & g(x)
	\end{vmatrix}.
\end{equation*}
显然\(F(a) = 0\).
求导得\begin{equation*}
	F'(x) = \begin{vmatrix}
		f(a) & f'(x) \\
		g(a) & g'(x)
	\end{vmatrix}.
\end{equation*}
由拉格朗日中值定理可知\begin{equation*}
	F(b) - F(a) = F'(\xi) (b-a)
	\quad(a<\xi<b).
	\qedhere
\end{equation*}
\end{proof}
\end{example}
\begin{example}
%@see: 《高等数学(第六版 上册)》 P134 习题3-1 14.
证明:若函数\(f\)在\((-\infty,+\infty)\)内满足关系式\(f'(x) = f(x)\),
且\(f(0) = 1\),则\(f(x) = e^x\).
\begin{proof}
记\(F(x) = e^{-x} f(x)\),
求导得\begin{equation*}
	F'(x) = e^{-x} (f'(x) - f(x))
	= 0.
\end{equation*}
于是由拉格朗日中值定理可知,
如果\(-\infty<a<b<+\infty\),那么\begin{equation*}
	F(b) - F(a)
	= (b-a) F'(\xi)
	= 0
	\quad(a<\xi<b).
\end{equation*}
由此可见\(F\)是常数函数,
那么\begin{equation*}
	f(x) = F(x)~e^x = C~e^x,
\end{equation*}
其中\(C\)是常数.
又因为\(f(0) = C = 1\),
所以\(f(x) = e^x\).
\end{proof}
\end{example}

\begin{example}
%@see: 《2020年全国硕士研究生入学统一考试(数学一)》三解答题/第19题
设函数\(f\)在区间\([0,2]\)上具有连续导数,
\(f(0) = f(2) = 0,
M = \max_{0 \leq x \leq 2} \abs{f(x)}\).
证明:\begin{itemize}
	\item 存在\(\xi\in(0,2)\)使得\(\abs{f'(\xi)} \geq M\);
	\item 若对任意\(x\in(0,2)\)有\(\abs{f'(x)} \leq M\),则\(M = 0\).
\end{itemize}
\begin{proof}
假设\(M = 0\),
则\(f(x)\)恒等于\(0\),\(f'(x)\)也恒等于\(0\),
于是对于任意\(\xi\in(0,2)\),总是成立\(\abs{f'(\xi)} \geq M\).

假设\(M > 0\),
且点\(c\in(0,2)\)满足\(\abs{f(c)} = M\).
% 这里之所以假设\(c\in(0,2)\),是因为\(f(0) = f(2) = 0 < M\)
由拉格朗日中值定理,
存在\(\xi_1\in(0,c)\),
存在\(\xi_2\in(c,2)\),
使得\begin{equation*}
	\abs{f(c)}
	% \(f(0) = f(2) = 0\)
	= \abs{f(c) - f(0)}
	= \abs{f'(\xi_1) \cdot (c - 0)},
	\qquad
	\abs{f(c)}
	% \(f(0) = f(2) = 0\)
	= \abs{f(2) - f(c)}
	= \abs{f'(\xi_2) \cdot (2 - c)},
\end{equation*}
于是\begin{equation*}
	% 代入\(\abs{f(c)} = M\)
	\abs{f'(\xi_1)} = \frac{M}{c},
	\qquad
	\abs{f'(\xi_2)} = \frac{M}{2-c}.
\end{equation*}
当\(0<c<1\)时,\(\abs{f'(\xi_1)} > M\),取\(\xi=\xi_1\);
当\(c=1\)时,\(\abs{f'(\xi_1)} = \abs{f'(\xi_2)} = M\),取\(\xi=\xi_1\);
当\(1<c<2\)时,\(\abs{f'(\xi_2)} > M\),取\(\xi=\xi_2\);
总而言之,\(\abs{f'(\xi)} \geq M\).

假设对任意\(x\in(0,2)\)有\(\abs{f'(x)} \leq M\).
因为当\(c\in(0,1)\cup(1,2)\)时,由上可知,
存在\(\xi\in(0,1)\)使得\(\abs{f'(\xi)} > M\),与假设矛盾,
所以\(c\)只可能等于\(1\),即有\(\abs{f(1)} = M\),
或者说\(x=1\)是\(f\)的极值点,
那么由费马引理可知\(f'(1) = 0\).
\begin{itemize}
	\item 假设\(f(1) = M\).
	令\(g(x) = f(x) - M x\),
	则\begin{equation*}
		g(0) = f(0) - 0 = 0,
		\qquad
		g(1) = f(1) - M = 0,
		\qquad
		g'(x) = f'(x) - M \leq 0,
	\end{equation*}
	所以\(g\)在\((0,1)\)内单调不增,
	可以用反证法证明\(g\)在\((0,1)\)内只能恒等于\(0\).
	于是\(f(x) = M x\),
	那么\(f'_-(1) = M = 0\).

	\item 假设\(f(1) = -M\).
	令\(g(x) = f(x) + M x\),
	则\begin{equation*}
		g(0) = f(0) + 0 = 0,
		\qquad
		g(1) = f(1) + M = 0,
		\qquad
		g'(x) = f'(x) + M \geq 0,
	\end{equation*}
	所以\(g\)在\((0,1)\)内单调不减,
	可以用反证法证明\(g\)在\((0,1)\)内只能恒等于\(0\).
	于是\(f(x) = -M x\),
	那么\(f'_-(1) = -M = 0\),
	即\(M = 0\).
\end{itemize}
综上所述,若对任意\(x\in(0,2)\)有\(\abs{f'(x)} \leq M\),则\(M = 0\).
\end{proof}
\end{example}

\subsection{柯西中值定理}
\begin{theorem}[柯西中值定理]\label{theorem:微分中值定理.柯西中值定理}
%@see: 《高等数学(第六版 上册)》 P133 柯西中值定理
%@see: 《数学分析(第二版 上册)》(陈纪修) P179 定理5.1.9(Cauchy中值定理)
%@see: 《数学分析教程(第3版 上册)》(史济怀) P149 定理3.4.4(Cauchy)
如果函数\(f\colon[a,b]\to\mathbb{R}\)和\(g\colon[a,b]\to\mathbb{R}\)
都在闭区间\([a,b]\)上连续,都在开区间\((a,b)\)内可导,
即\(f,g \in C[a,b] \cap D(a,b)\),
并且\((\forall x\in(a,b))[g'(x) \neq 0]\),
那么\(\exists\xi\in(a,b)\),
使得\begin{equation}
	\frac{f(b)-f(a)}{g(b)-g(a)}=\frac{f'(\xi)}{g'(\xi)}.
\end{equation}
\begin{proof}
因为\(g(b)-g(a)=g'(\eta)(b-a)\ (a<\eta<b)\),
根据假定\(g'(\eta)\neq0\),
又\(b-a\neq0\),
所以\(g(b)-g(a)\neq0\).
令\begin{equation*}
	\phi(x)=f(x)-f(a)-\frac{f(b)-f(a)}{g(b)-g(a)}[g(x)-g(a)].
\end{equation*}
不难得\(\phi(a)=\phi(b)=0\),
\(\phi\in C[a,b]\cap D(a,b)\),
且\begin{equation*}
	\phi'(x)=f'(x)-\frac{f(b)-f(a)}{g(b)-g(a)}\cdot g'(x).
\end{equation*}
根据\hyperref[theorem:微分中值定理.罗尔定理]{罗尔定理}可知,
在\(\exists\xi\in(a,b)\)使得\(\phi'(\xi)=0\),
即\begin{equation*}
	f'(\xi)-\frac{f(b)-f(a)}{g(b)-g(a)}\cdot g'(\xi)=0,
\end{equation*}
亦即\begin{equation*}
	\frac{f(b)-f(a)}{g(b)-g(a)}=\frac{f'(\xi)}{g'(\xi)}.
	\qedhere
\end{equation*}
\end{proof}
\end{theorem}

我们把\hyperref[theorem:微分中值定理.罗尔定理]{罗尔定理}、
\hyperref[theorem:微分中值定理.拉格朗日中值定理]{拉格朗日中值定理}%
和\hyperref[theorem:微分中值定理.柯西中值定理]{柯西中值定理}%
统称为\DefineConcept{微分中值定理}(mean-value theorem).
%@see: https://mathworld.wolfram.com/Mean-ValueTheorem.html
%@see: https://mathworld.wolfram.com/ExtendedMean-ValueTheorem.html

\begin{example}
%@see: 《高等数学(第六版 上册)》 P134 习题3-1 15.
设函数\(f\)在点\(x=0\)的某邻域内具有\(n\)阶导数,
且\begin{equation*}
	f(0) = f'(0) = \dotsb = f^{(n-1)}(0) = 0.
\end{equation*}
利用柯西中值定理证明:\begin{equation*}
	\frac{f(x)}{x^n} = \frac{f^{(n)}(\theta x)}{n!}
	\quad(0<\theta<1).
\end{equation*}
\begin{proof}
记\(g(x) = x^n\).
由柯西中值定理可知\begin{equation*}
	\frac{f(x) - f(0)}{g(x) - g(0)}
	= \frac{f(x)}{g(x)}
	= \frac{f'(\xi_1)}{g'(\xi_1)}
	\quad(\text{$\xi_1$在$0$与$x$之间}),
\end{equation*}
归纳可知\begin{equation*}
	\frac{f^{(k)}(\xi_k)}{g^{(k)}(\xi_k)}
	= \frac{f^{(k+1)}(\xi_{k+1})}{g^{(k+1)}(\xi_{k+1})}
	\quad(\text{$\xi_{k+1}$在$0$与$\xi_k$之间};k=0,1,2,\dotsc,n-1).
\end{equation*}
因此\begin{equation*}
	\frac{f(x)}{g(x)}
	= \frac{f'(\xi_1)}{g'(\xi_1)}
	= \dotsb
	= \frac{f^{(n)}(\xi_n)}{g^{(n)}(\xi_n)},
\end{equation*}
即\begin{equation*}
	\frac{f(x)}{x^n}
	= \frac{f^{(n)}(\xi_n)}{n!}.
	\qedhere
\end{equation*}
\end{proof}
\end{example}

\subsection{达布定理}
一般来说,一个可微函数的导数并不一定连续,
但是导函数却像闭区间上的连续函数一样,
服从自己的“零点定理”和“介值定理”.

\begin{theorem}[达布零点定理]\label{theorem:微分中值定理.达布定理1}
%@see: 《数学分析教程(第3版 上册)》(史济怀) P150 定理3.4.5(1)
设函数\(f \in D(a,b)\),\(a<x_1<x_2<b\).
如果\(f'(x_1) \cdot f'(x_2) < 0\),
那么\begin{equation*}
	(\exists\xi\in(x_1,x_2))
	[f'(\xi) = 0].
\end{equation*}
\begin{proof}
不妨设\(f'(x_1)<0,f'(x_2)>0\).
那么根据\hyperref[theorem:极限.函数极限的局部保号性1]{函数极限的局部保号性},
存在正数\(\delta_1,\delta_2\),
使得\(x_1\)的\(\delta_1\)邻域\(U(x_1,\delta_1)\)
和\(x_2\)的\(\delta_2\)邻域\(U(x_2,\delta_2)\)
都是区间\((a,b)\)的真子集,
且对任意\(x \in U(x_1,\delta_1)\)成立\(\frac{f(x)-f(x_1)}{x-x_1}<0\)
以及对任意\(x \in U(x_2,\delta_2)\)成立\(\frac{f(x)-f(x_2)}{x-x_2}>0\).

任取实数\(\alpha \in U(x_1,\delta_1)\)和\(\beta \in U(x_2,\delta_2)\),
使得\(x_1<\alpha<\beta<x_2\),
由上可知\begin{equation*}
	\begin{split}
		\frac{f(\alpha)-f(x_1)}{\alpha-x_1}<0
		\iff
		f(\alpha)-f(x_1)<0
		\iff
		f(\alpha)<f(x_1), \\
		\frac{f(\beta)-f(x_2)}{\beta-x_2}>0
		\iff
		f(\beta)-f(x_2)<0
		\iff
		f(\beta)<f(x_2).
	\end{split}
\end{equation*}
这就是说\(f(x_1)\)和\(f(x_2)\)都不是\(f\)在\([x_1,x_2]\)上的最小值.

由于\(f \in D(a,b)\)而\([x_1,x_2]\subset(a,b)\),
所以\(f \in D[x_1,x_2]\),
那么根据\cref{theorem:导数与微分.函数可导性与连续性的关系}
有\(f \in C[x_1,x_2]\).
根据\hyperref[theorem:极限.最值定理]{魏尔斯特拉斯最值定理},
有\begin{equation*}
	(\exists\xi\in(x_1,x_2))
	(\forall x\in(x_1,x_2))
	[f(\xi) \leq f(x)].
\end{equation*}
那么利用\hyperref[theorem:微分中值定理.费马引理]{费马引理}可知\(f'(\xi)=0\).
\end{proof}
\end{theorem}

我们可以将\cref{theorem:微分中值定理.达布定理1} 作一番推广.
\begin{theorem}[达布介值定理]\label{theorem:微分中值定理.达布定理2}
%@see: 《数学分析教程(第3版 上册)》(史济怀) P150 定理3.4.5(1)
%@see: 《数学分析(第五版 上册)》(华东师范大学) P115 定理6.5(达布定理,Darboux定理)
设函数\(f \in D[a,b]\),且\(f'(a) < f'(b)\).
那么\(\forall\lambda\in(f'(a),f'(b))\),\(\exists\xi\in(a,b)\),
使得\begin{equation*}
	f'(\xi) = \lambda.
\end{equation*}
\begin{proof}
令\(F(x) = f(x) - \lambda x\ (a \leq x \leq b)\).
那么\(F \in D[a,b]\),且\begin{equation*}
	F'(a) = f'(a) - \lambda < 0, \qquad
	F'(b) = f'(b) - \lambda > 0.
\end{equation*}
根据\cref{theorem:微分中值定理.达布定理1} 就有\(\exists\xi\in[a,b]\)使得\begin{equation*}
	F'(\xi)=0,
\end{equation*}
即\(f'(x) = \lambda\).
\end{proof}
\end{theorem}

\begin{theorem}
%@see: 《数学分析教程(第3版 上册)》(史济怀) P150 定理3.4.5(2)
设\(f \in D[a,b]\),那么导函数\(f'\)没有第一类间断点.
\begin{proof}
用反证法.
设\(x_0\)是\(f'\)的一个第一类间断点,
那么\(f'(x_0^+)\)和\(f'(x_0^-)\)都存在.

因为\(f \in D[a,b]\),
由\hyperref[theorem:微分中值定理.拉格朗日中值定理]{拉格朗日中值定理},
可得\begin{equation*}
	f'(x_0)
	= f'_+(x_0)
	= \lim_{x \to x_0^+} \frac{f(x)-f(x_0)}{x-x_0}
	= \lim_{x \to x_0^+} \frac{f'(\xi) (x-x_0)}{x-x_0}
	= \lim_{x \to x_0^+} f'(\xi)
	\quad(x_0<\xi<x).
\end{equation*}
由于当\(x \to x_0^+\)时,\(\xi \to x_0^+\),
且已知\(f'(x_0^+)\)存在,
所以有\begin{equation*}
	f'(x_0)=f'(x_0^+).
\end{equation*}
同理可证\(f'(x_0)=f'(x_0^-)\).
由此可知\(f'\)在点\(x_0\)连续,
而这与\(x_0\)是\(f'\)的间断点矛盾!
\end{proof}
\end{theorem}
\begin{remark}
%@see: 《数学分析教程(第3版 上册)》(史济怀) P151
如果\(f'\)是\([a,b]\)上的连续函数,
它当然具有\hyperref[theorem:极限.闭区间上连续函数的性质.介值定理1]{介值性}.
达布介值定理的意义在于,即使\(f'\)在\([a,b]\)上不连续,\(f'\)仍然具有介值性,
这是导函数所特有的性质.
从这个性质出发,马上可以断言,不存在可导函数\(f\),使得\begin{equation*}
	f'(x) = D(x)
	\quad\text{或}\quad
	f'(x) = R(x),
\end{equation*}
其中\(D\)是狄利克雷函数,\(R\)是黎曼函数.
\end{remark}

\subsection{导数极限定理}
\begin{theorem}[导数极限定理]\label{theorem:微分中值定理.导数极限定理}
%@see: 《数学分析讲义(第1册)》(程艺) P102 定理3.19
%@see: 《数学分析(第五版 上册)》(华东师范大学) P113 推论3(导数极限定理)
设函数\(f \in C[a,b] \cap D(a,b)\),\(f'\)是\(f\)的导函数.
\begin{itemize}
	\item 若\(f'\)在点\(a\)的右极限\(f'(a^+)\)存在,
	则\(f\)在点\(a\)的右导数\(f'_+(a)\)也存在,
	且\begin{equation*}
		f'_+(a) = f'(a^+).
	\end{equation*}
	\item 若\(f'\)在点\(b\)的左极限\(f'(b^-)\)存在,
	则\(f\)在点\(b\)的左导数\(f'_-(b)\)也存在,
	且\begin{equation*}
		f'_-(b) = f'(b^-).
	\end{equation*}
\end{itemize}
\begin{proof}
对\(\forall x\in(a,b)\),
由\hyperref[theorem:微分中值定理.拉格朗日中值定理]{拉格朗日中值定理}可知,
\(\exists\xi\in(a,x)\),
使得\begin{equation*}
	f'(\xi) = \frac{f(x)-f(a)}{x-a}.
\end{equation*}
令\(x \to a^+\),则\(\xi \to a^+\),
于是\begin{equation*}
	f'_+(a)
	= \lim_{x \to a^+} \frac{f(x)-f(a)}{x-a}
	= \lim_{\xi \to a^+} f'(\xi)
	= f'(a^+).
\end{equation*}

同理,\(\exists\zeta\in(x,b)\),
使得\begin{equation*}
	f'(\zeta) = \frac{f(b)-f(x)}{b-x}.
\end{equation*}
令\(x \to b^-\),则\(\xi \to b^-\),
于是\begin{equation*}
	f'_-(b)
	= \lim_{x \to b^-} \frac{f(x)-f(b)}{x-b}
	= \lim_{\xi \to b^-} f'(\xi)
	= f'(b^-).
	\qedhere
\end{equation*}
\end{proof}
\end{theorem}

%TODO 需要证明、例子
% \cref{theorem:微分中值定理.导数极限定理} 说明:
% 如果函数在区间内处处可导,
% 则在区间内的每一点处,导函数\(f'\)要么连续,要么有第二类间断点,绝不可能有第一类间断点.
% 由此推出,具有第一类间断点的函数(如符号函数\(\sgn\))不能作为某个函数的导函数.

%TODO 需要证明、例子
% 还应该指出:
% 如果函数\(f\)在区间\((a,b)\)内可导,
% 那么\(f\)的导函数\(f'\)在\((a,b)\)内
% 不可能存在可去间断点、跳跃间断点和无穷间断点,
% 只可能存在振荡间断点.

\section{洛必达法则}
如果当\(x \to a\)(或\(x \to \infty\))时,
\(f\)与\(g\)这两个函数都趋于零或都趋于无穷大,
那么极限\(\lim\frac{f(x)}{g(x)}\)可能存在,也可能不存在.
通常把这种极限叫做\DefineConcept{未定式}(indeterminate form),
并简记为\(\frac{0}{0}\)或\(\frac{\infty}{\infty}\).
对于这类极限,即是它存在也不能用“商的极限等于极限的商”这一法则.
下面我们将根据柯西中值定理来推出求这类极限的一种简便且重要的方法.
%@see: https://mathworld.wolfram.com/Indeterminate.html

我们着重讨论\(x \to a^+\)时的未定式\(\frac{0}{0}\)的情形,关于这情形有以下定理:
\begin{theorem}[洛必达法则]\label{theorem:微分中值定理.洛必达法则1}
%@see: 《高等数学(第六版 上册)》 P135 定理1
%@see: 《高等数学(第六版 上册)》 P136 定理2
%@see: 《数学分析教程(第3版 上册)》(史济怀) P173 定理3.6.1
%@see: 《数学分析(第二版 上册)》(陈纪修) P185 定理5.2.1(L'Hospital法则)
%@see: 《数学分析简明教程(第二版 下册)》(邓东皋、尹小玲) P133 定理5.6
%@see: 《数学分析简明教程(第二版 下册)》(邓东皋、尹小玲) P134 定理5.7
%@see: 《数学分析简明教程(第二版 下册)》(邓东皋、尹小玲) P134 定理5.8
设函数\(f\colon(a,b)\to\mathbb{R}\)和\(g\colon(a,b)\to\mathbb{R}\)都在\((a,b)\)内可导.
如果\begin{itemize}
	\item \(\lim_{x \to a^+} f(x) = \lim_{x \to a^+} g(x) = 0\),
	或\(\lim_{x \to a^+} g(x) = \infty\);
	\item \((\forall x \in (a,b))[g'(x) \neq 0]\);
	\item \(\lim_{x \to a^+} \frac{f'(x)}{g'(x)}\)存在且有限,
	或\(\lim_{x \to a^+} \frac{f'(x)}{g'(x)} = \infty\),
\end{itemize}
那么\begin{equation*}
	\lim_{x \to a^+} \frac{f(x)}{g(x)}
	= \lim_{x \to a^+} \frac{f'(x)}{g'(x)}.
\end{equation*}
\begin{proof}
假设\(\lim_{x \to a^+} f(x) = \lim_{x \to a^+} g(x) = 0\),
且\(\lim_{x \to a^+} \frac{f'(x)}{g'(x)}\)存在且有限.
由于函数在点\(x=a\)的值与当\(x \to a^+\)时的极限无关,
因此可以补充定义:\begin{equation*}
	f(a) = g(a) = 0,
\end{equation*}
使得\(f\)和\(g\)在区间\([a,b)\)上连续.
取\(c\in(a,b)\),
则经补充定义后的函数\(f\)和\(g\)在\([a,c]\subset[a,b)\)上
满足\hyperref[theorem:微分中值定理.柯西中值定理]{柯西中值定理}的条件,
因而对于任意\(x\in(a,c)\),存在\(\xi\in(a,c)\),满足\begin{equation*}
	\frac{f(x)}{g(x)}
	= \frac{f(x)-f(a)}{g(x)-g(a)}
	= \frac{f'(\xi)}{g'(\xi)}.
\end{equation*}
当\(x \to a^+\)时显然有\(\xi \to a^+\),
于是\begin{equation*}
	\lim_{x \to a^+} \frac{f(x)}{g(x)}
	= \lim_{\xi \to a^+} \frac{f'(\xi)}{g'(\xi)}
	= \lim_{x \to a^+} \frac{f'(x)}{g'(x)}.
\end{equation*}

假设\(\lim_{x \to a^+} g(x) = \infty\),
且\(\lim_{x \to a^+} \frac{f'(x)}{g'(x)} = A\)存在且有限.
取\(c\in(a,b)\),
则当\(x \neq c\)时,商式\(\frac{f(x)}{g(x)}\)可以改写为\begin{align*}
	\frac{f(x)}{g(x)}
	&= \frac{f(x) - f(c) + f(c)}{g(x)}
	= \frac{f(x) - f(c)}{g(x)} + \frac{f(c)}{g(x)} \\
	&= \frac{g(x) - g(c)}{g(x)} \cdot \frac{f(x) - f(c)}{g(x) - g(c)} + \frac{f(c)}{g(x)} \\
	&= \left[1 - \frac{g(c)}{g(x)}\right] \frac{f(x) - f(c)}{g(x) - g(c)} + \frac{f(c)}{g(x)}.
\end{align*}
于是\begin{align*}
	\abs{\frac{f(x)}{g(x)} - A}
	&= \abs{
		\left[1 - \frac{g(c)}{g(x)}\right] \frac{f(x) - f(c)}{g(x) - g(c)}
		+ \frac{f(c)}{g(x)} - A
	} \\
	&= \abs{
		\left[1 - \frac{g(c)}{g(x)}\right]
		\left[\frac{f(x) - f(c)}{g(x) - g(c)} - A\right]
		+ \left[1 - \frac{g(c)}{g(x)}\right] A
		+ \frac{f(c)}{g(x)} - A
	} \\
	&= \abs{
		\left[1 - \frac{g(c)}{g(x)}\right]
		\left[\frac{f(x) - f(c)}{g(x) - g(c)} - A\right]
		+ \frac{f(c) - A g(c)}{g(x)}
	} \\
	&\leq \abs{1 - \frac{g(c)}{g(x)}}
	\abs{\frac{f(x) - f(c)}{g(x) - g(c)} - A}
	+ \abs{\frac{f(c) - A g(c)}{g(x)}}.
\end{align*}
因为\(\lim_{x \to a^+} \frac{f'(x)}{g'(x)} = A\),
所以对于\(\forall\epsilon>0\),
\(\exists\rho\in(0,b-a)\),
当\(0<x-a<\rho\)时,
成立\begin{equation*}
	\abs{\frac{f'(x)}{g'(x)} - A} < \epsilon.
\end{equation*}
取\(c=a+\rho\),
由\hyperref[theorem:微分中值定理.柯西中值定理]{柯西中值定理},
对于\(\forall x \in (a,c)\),
\(\exists\xi\in(x,c)\subset(a,b)\),
满足\begin{equation*}
	\frac{f(x) - f(c)}{g(x) - g(c)} = \frac{f'(\xi)}{g'(\xi)}.
\end{equation*}
于是得到\begin{equation*}
	\abs{\frac{f(x) - f(c)}{g(x) - g(c)} - A}
	= \abs{\frac{f'(\xi)}{g'(\xi)} - A}
	< \epsilon.
\end{equation*}
又因为\(\lim_{x \to a^+} g(x) = \infty\),
所以\begin{equation*}
	\lim_{x \to a^+} \abs{1 - \frac{g(c)}{g(x)}} = 1,
	\qquad
	\lim_{x \to a^+} \abs{\frac{f(c) - A g(c)}{g(x)}} = 0,
\end{equation*}
也就是说\(\exists\delta\in(0,\rho)\),
当\(0<x-a<\delta\)时,
成立\begin{equation*}
	\abs{1-\frac{g(c)}{g(x)}} < 2, \qquad
	\abs{\frac{f(c) - A g(c)}{g(x)}} < \epsilon.
\end{equation*}
综上所述,对于\(\forall\epsilon>0\),
\(\exists\delta>0\),
当\(0<x-a<\delta\)时,
成立\begin{align*}
	\abs{\frac{f(x)}{g(x)} - A}
	&\leq \abs{1 - \frac{g(c)}{g(x)}}
	\abs{\frac{f(x) - f(c)}{g(x) - g(c)} - A}
	+ \abs{\frac{f(c) - A g(c)}{g(x)}} \\
	&< 2\epsilon + \epsilon = 3\epsilon,
\end{align*}
由定义有\begin{equation*}
	\lim_{x \to a^+} \frac{f(x)}{g(x)}
	= \lim_{x \to a^+} \frac{f'(x)}{g'(x)}
	= A.
	\qedhere
\end{equation*}
\end{proof}
\end{theorem}
像这样,在一定条件下,
通过分子分母分别求导再求极限来确定未定式的值的方法,
称为\DefineConcept{洛必达法则}(L'Hospital's rule).

如果\(\frac{f'(x)}{g'(x)}\)当\(x \to a^+\)时仍属\(\frac{0}{0}\)型,
且这时\(f'(x)\)和\(g'(x)\)能满足定理中\(f(x)\)和\(g(x)\)所要满足的条件,
那么可以继续施用洛必达法则先确定\(\lim_{x \to a^+} \frac{f'(x)}{g'(x)}\),
从而确定\(\lim_{x \to a^+} \frac{f(x)}{g(x)}\),即\begin{equation*}
	\lim_{x \to a^+} \frac{f(x)}{g(x)}
	= \lim_{x \to a^+} \frac{f'(x)}{g'(x)}
	= \lim_{x \to a^+} \frac{f''(x)}{g''(x)};
\end{equation*}
且可以此类推.

洛必达法则只是未定式极限存在的充分不必要条件.
例如,对于极限\begin{equation*}
	\lim_{x\to\infty} \frac{x+\sin x}{x},
\end{equation*}
易见\begin{equation*}
	\lim_{x\to\infty} (x+\sin x)
	= \lim_{x\to\infty} x
	= \infty,
	\qquad
	(x+\sin x)' = 1+\cos x, \qquad
	x' = 1.
\end{equation*}
虽然极限\begin{equation*}
	\lim_{x\to\infty}\frac{(x+\sin x)'}{x'} = \lim_{x\to\infty}(1+\cos x)
\end{equation*}不存在,
但原极限存在:\begin{equation*}
	\lim_{x\to\infty} \frac{x+\sin x}{x}
	= 1 + \lim_{x\to\infty} \frac{\sin x}{x}
	= 1 + 0 = 1.
\end{equation*}

\begin{example}
%@see: https://www.bilibili.com/video/BV1k24y1q7aN/
设函数\(f\)在点\(x=0\)二阶可导,
且\(\lim_{x\to0} \frac{f(x)}{x^\mu} = 1\ (\mu>1)\).
证明:\begin{itemize}
	\item \(f'(0) = 0\);
	\item 当\(\mu=2\)时,成立\(\lim_{x\to0} \frac{f(x)}{x^\mu}
	= \lim_{x\to0} \frac{f'(x)}{\mu x^{\mu-1}}\);
	\item 说明为什么不可以对\(\lim_{x\to0} \frac{f(x)}{x^\mu}\)运用洛必达法则求解极限.
\end{itemize}
\begin{solution}
因为函数\(f\)在点\(x=0\)二阶可导,
所以由\cref{theorem:高阶导数.在一点高阶可导是在该点某邻域内低阶可导的充分条件}
可知\(f\)在点\(x=0\)的某个邻域内连续且可导,
从而有\(f(0) = 0\),
于是\begin{equation*}
	f'(0) = \lim_{x\to0} \frac{f(x) - f(0)}{x}
	= \lim_{x\to0} \frac{f(x)}{x^\mu} \cdot \frac{x^\mu}{x}
	%\cref{theorem:极限.极限的四则运算法则}
	= \lim_{x\to0} \frac{f(x)}{x^\mu} \cdot \lim_{x\to0} \frac{x^\mu}{x}
	= 1 \cdot 0
	= 0.
\end{equation*}

又因为\begin{equation*}
	f''(0) = \lim_{x\to0} \frac{f'(x) - f'(0)}{x}
	= \lim_{x\to0} \frac{f'(x)}{x}
	< \infty,
\end{equation*}
所以当\(\mu=2\)时,
成立\begin{equation*}
	\lim_{x\to0} \frac{f'(x)}{2x}
	= \frac12 f''(0).
\end{equation*}

当\(\mu>2\)时,
由于\(\lim_{x\to0} \frac{x}{\mu x^{\mu-1}}
= \frac1\mu \lim_{x\to0} \frac1{x^{\mu-2}}
= \infty\),
所以不能利用\hyperref[theorem:极限.极限的四则运算法则]{极限的四则运算法则}保证\begin{equation*}
	\lim_{x\to0} \frac{f'(x)}{\mu x^{\mu-1}}
	= \lim_{x\to0} \frac{f'(x)}{x} \cdot \frac{x}{\mu x^{\mu-1}}
	< \infty,
\end{equation*}
当然就不能运用\hyperref[theorem:微分中值定理.洛必达法则1]{洛必达法则}求解\(\lim_{x\to0} \frac{f(x)}{x^\mu}\).
%@credit: {ce603838-a24d-4616-9395-d7b223e8cb72} 指出\(f'\)不一定是\(x\to0\)时的无穷小,所以不能求出\(f''(0)\)
特别地,取\begin{equation*}
	f(x) = \left\{ \begin{array}{cl}
		x^\mu + x^{\mu+1} \sin\frac1x, & x\neq0, \\
		0, & x=0,
	\end{array} \right.
\end{equation*}
则\begin{equation*}
	f'(x) = \left\{ \begin{array}{cl}
		\mu x^{\mu-1} + (\mu+1) x^\mu \sin\frac1x - x^{\mu-1} \cos\frac1x, & x\neq0, \\
		0, & x=0,
	\end{array} \right.
\end{equation*}
那么当\(x\neq0\)时,有\begin{equation*}
	\frac{f'(x)}{\mu x^{\mu-1}}
	= 1 + \frac{\mu+1}\mu x \sin\frac1x - \frac1\mu \cos\frac1x,
\end{equation*}
这里\(\cos\frac1x\)当\(x\to0\)时的极限不存在,
那么\(\frac{f'(x)}{\mu x^{\mu-1}}\)在点\(x=0\)的极限也不存在.
\end{solution}
\end{example}

\begin{example}
%@see: 《高等数学(第六版 上册)》 P136 例1
求\(\lim_{x\to0} \frac{\sin ax}{\sin bx}\ (b \neq 0)\).
\begin{solution}
\(\lim_{x\to0} \frac{\sin ax}{\sin bx}
= \lim_{x\to0} \frac{a \cos ax}{b \cos bx}
= \frac{a}{b}\).
\end{solution}
\end{example}

\begin{example}\label{example:微分中值定理.洛必达法则.零比零型2}
%@see: 《高等数学(第六版 上册)》 P136 例2
求\(\lim_{x\to1} \frac{x^3-3x+2}{x^3-x^2-x+1}\).
\begin{solution}
\(\lim_{x\to1} \frac{x^3-3x+2}{x^3-x^2-x+1}
= \lim_{x\to1} \frac{3x^2-3}{3x^2-2x-1}
= \lim_{x\to1} \frac{6x}{6x-2}
= \frac32\).
\end{solution}
\end{example}
\begin{remark}
在\cref{example:微分中值定理.洛必达法则.零比零型2} 中,
\(\lim_{x\to1} \frac{6x}{6x-2}\)已经不是未定式,
不能再对它应用洛必达法则,否则会导致错误结果.
以后在使用洛必达法则时,一定要时刻注意极限是否还是未定式,
如果不是未定式,就不能应用洛必达法则.
\end{remark}

\begin{example}
%@see: 《高等数学(第六版 上册)》 P136 例3
求\(\lim_{x\to0} \frac{x-\sin x}{x^3}\).
\begin{solution}
\(\lim_{x\to0} \frac{x-\sin x}{x^3}
= \lim_{x\to0} \frac{1-\cos x}{3x^2}
= \lim_{x\to0} \frac{\sin x}{6x}
= \frac16\).
\end{solution}
\end{example}

\begin{example}
求\(\lim_{x\to\pi} (\csc x+\cot x)\).
\begin{solution}
\(\lim_{x\to\pi} (\csc x+\cot x)
= \lim_{x\to\pi} \frac{1+\cos x}{\sin x}
= \lim_{x\to\pi} \frac{-\sin x}{\cos x}
= 0\).
\end{solution}
\end{example}

\begin{example}
%@see: 《高等数学(第六版 上册)》 P136 例4
求\(\lim_{x\to+\infty} \frac{\frac{\pi}{2} - \arctan x}{\frac{1}{x}}\).
\begin{solution}
\(\lim_{x\to+\infty} \frac{\frac{\pi}{2} - \arctan x}{\frac{1}{x}}
= \lim_{x\to+\infty} \frac{-\frac{1}{1+x^2}}{-\frac{1}{x^2}}
= \lim_{x\to+\infty} \frac{x^2}{1+x^2} = 1\).
\end{solution}
\end{example}

\begin{example}\label{example:微分中值定理.洛必达法则.无穷大比无穷大型1}
%@see: 《高等数学(第六版 上册)》 P136 例5
求\(\lim_{x\to+\infty} \frac{\ln x}{x^n}\ (n>0)\).
\begin{solution}
\(\lim_{x\to+\infty} \frac{\ln x}{x^n}
= \lim_{x\to+\infty} \frac{\frac{1}{x}}{n x^{n-1}} % 应用洛必达法则,分子分母同时求导
= \lim_{x\to+\infty} \frac{1}{n x^n} = 0\).
\end{solution}
\end{example}
\begin{example}
求\(\lim_{x\to+\infty} \frac{\ln^\alpha x}{x^\beta}\ (\alpha,\beta>0)\).
\begin{solution}
应用洛必达法则,得\begin{align*}
	\lim_{x\to+\infty} \frac{\ln^\alpha x}{x^\beta}
	&= \lim_{x\to+\infty} \frac{\alpha \ln^{\alpha-1} x}{\beta x^\beta}
	= \lim_{x\to+\infty} \frac{\alpha(\alpha-1) \ln^{\alpha-2} x}{\beta^2 x^\beta} \\
	&= \dotsb
	= \lim_{x\to+\infty} \frac{\alpha(\alpha-1)\dotsm(\alpha-n+1) \ln^{\alpha-n} x}{\beta^n x^\beta},
\end{align*}
当\(n\)足够大时,\(\alpha-n<0\),而\(\lim_{x\to+\infty} \ln^{\alpha-n} x = 0\),
于是\(\lim_{x\to+\infty} \frac{\ln^\alpha x}{x^\beta} = 0\).
\end{solution}
%@Mathematica: Limit[Log[x]^a/x^b, x -> Infinity, Assumptions -> {a > 0, b > 0}]
\end{example}

\begin{example}\label{example:微分中值定理.洛必达法则.无穷大比无穷大型2}
%@see: 《高等数学(第六版 上册)》 P137 例6
求\(\lim_{x\to+\infty} \frac{x^n}{e^{\lambda x}}\ (n>0,\lambda>0)\).
\begin{solution}
相继应用洛必达法则\(n\)次,
得\begin{align*}
	\lim_{x\to+\infty} \frac{x^n}{e^{\lambda x}}
	&= \lim_{x\to+\infty} \frac{n x^{n-1}}{\lambda e^{\lambda x}} % 应用洛必达法则,分子分母同时求导,但是分子求导有限次以后会化为零
	= \lim_{x\to+\infty} \frac{n(n-1) x^{n-2}}{\lambda^2 e^{\lambda x}} \\
	&= \dotsb = \lim_{x\to+\infty} \frac{n!}{\lambda^n e^{\lambda x}}
	= 0.
\end{align*}
\end{solution}
\end{example}

其他还有一些\(0 \cdot \infty\)、\(\infty - \infty\)、\(0^0\)、\(1^\infty\)、\(\infty^0\)型的未定式,也可通过\(\frac{0}{0}\)或\(\frac{\infty}{\infty}\)型的未定式来计算.

形如\(\infty - \infty\)的未定式,
通分为\(\frac{0}{0}\)或\(\frac{\infty}{\infty}\)型.
形如\(0 \cdot \infty\)的未定式,
将其中一个因子取倒数作为分母,
化为\(\frac{0}{0}\)或\(\frac{\infty}{\infty}\)型.
形如\(0^0\)、\(1^\infty\)、\(\infty^0\)的未定式,
先取对数,化为\(0 \cdot \infty\)型.

\begin{example}\label{example:微分中值定理.洛必达法则.零乘无穷大型1}
%@see: 《高等数学(第六版 上册)》 P137 例7
求\(\lim_{x\to0^+} x^n \ln x\ (n > 0)\).
\begin{solution}
这是未定式\(0\cdot\infty\).
因为\(x^n \ln x = \frac{\ln x}{x^{-n}}\),
当\(x\to0^+\)时,上式右端是未定式\(\frac{\infty}{\infty}\),
应用洛必达法则,
得\(\lim_{x\to0^+} x^n \ln x
= \lim_{x\to0^+} \frac{\ln x}{x^{-n}}
= \lim_{x\to0^+} \frac{x^{-1}}{-nx^{-n-1}}
= \lim_{x\to0^+} \frac{-x^n}{n}
= 0\).
\end{solution}
\end{example}
\begin{example}\label{example:微分中值定理.洛必达法则.零乘无穷大型2}
求\(\lim_{x\to0^+} x^\alpha \ln^\beta x\ (\alpha,\beta>0)\).
\begin{solution}
这同样是未定式\(0\cdot\infty\),
应用洛必达法则,得\begin{equation*}
	\lim_{x\to0^+} x^\alpha \ln^\beta x
	= \lim_{x\to0^+} \frac{\beta x^{-1} \ln^{\beta-1} x}{-\alpha x^{-\alpha-1}}
	= \lim_{x\to0^+} \frac{\beta \ln^{\beta-1} x}{-\alpha x^{-\alpha}}.
\end{equation*}
归纳可知,只要继续应用洛必达法则,分子中\(\ln x\)的次数会逐步减\(1\).
如果\(\beta\)是正整数,则一定可以将\(\ln x\)的次数减为\(1\),
把问题化归为\cref{example:微分中值定理.洛必达法则.零乘无穷大型1} 的情形.
如果\(\beta\)不是正整数,
则必能使\(\ln x\)的次数减为某个负数\(\gamma\),
而\(\ln^\gamma x \to 0\ (x\to0^+)\),
%@Mathematica: Limit[Log[x]^c, x -> 0, Assumptions -> {c < 0}]
从而有\(\lim_{x\to0^+} \frac{\ln^\gamma x}{x^{-\alpha}} = 0\).
于是\begin{equation*}
	\lim_{x\to0^+} x^\alpha \ln^\beta x = 0.
\end{equation*}
\end{solution}
%@Mathematica: Limit[x^a Log[x]^b, x -> 0, Direction -> "FromAbove", Assumptions -> {a > 0, b > 0}]
\end{example}

\begin{example}
%@see: 《2025年全国硕士研究生入学统一考试(数学一)》二填空题/第11题
计算极限\(\lim_{x\to0^+} \frac{x^x-1}{\ln x \cdot \ln(1-x)}\).
\begin{solution}
因为\(x \ln x \to 0\ (x\to0^+)\),
所以\(x^x - 1
= e^{x \ln x} - 1
\sim x \ln x
\ (x\to0^+)\),
加之\(\ln(1-x) \sim -x\ (x\to0^+)\),
于是\begin{equation*}
	\lim_{x\to0^+} \frac{x^x-1}{\ln x \cdot \ln(1-x)}
	= \lim_{x\to0^+} \frac{x \ln x}{-x \ln x}
	= \lim_{x\to0^+} \frac{x}{-x}
	= -1.
\end{equation*}
\end{solution}
%@Mathematica: Limit[(x^x - 1)/(Log[x] Log[1 - x]), x -> 0, Direction -> "FromAbove"]
\end{example}

\begin{example}
%@see: 《高等数学(第六版 上册)》 P138 例8
求\(\lim_{x\to\frac\pi2} (\sec x-\tan x)\).
\begin{solution}
这是未定式\(\infty-\infty\).
因为\(\sec x-\tan x = \frac{1-\sin x}{\cos x}\),
当\(x\to\frac\pi2\)时,上式右边是未定式\(\frac00\),
应用洛必达法则,
得\(\lim_{x\to\frac\pi2} (\sec x-\tan x)
= \lim_{x\to\frac\pi2} \frac{1-\sin x}{\cos x}
= \lim_{x\to\frac\pi2} \frac{-\cos x}{-\sin x}
= 0\).
\end{solution}
\end{example}

\begin{example}
求\(\lim_{x\to-\frac\pi2} (\sec x+\tan x)\).
\begin{solution}
\(\lim_{x\to-\frac\pi2} (\sec x+\tan x)
= \lim_{x\to-\frac\pi2} \frac{1+\sin x}{\cos x}
= \lim_{x\to-\frac\pi2} \frac{\cos x}{-\sin x}
= 0\).
\end{solution}
\end{example}

\begin{example}\label{example:微分中值定理.洛必达法则.零次方零型1}
%@see: 《高等数学(第六版 上册)》 P138 例9
求\(\lim_{x\to0^+} x^x\).
\begin{solution}
这是未定式\(0^0\).
设\(y = x^x\),取对数得\(\ln y = x \ln x\).
当\(x\to0^+\)时,上式右端是未定式\(0\cdot\infty\).
应用\cref{example:微分中值定理.洛必达法则.零乘无穷大型1} 的结果,
得\begin{equation*}
	\lim_{x\to0^+} \ln y
	= \lim_{x\to0^+} (x \ln x) = 0.
\end{equation*}
因为\(y = e^{\ln y}\),
而\(\lim_{x\to0^+} y
%\cref{equation:函数.对数恒等式}
= \lim_{x\to0^+} e^{\ln y}
%\cref{theorem:极限.连续函数的极限3}
= \exp\lim_{x\to0^+} \ln y\),
所以\begin{equation*}
	\lim_{x\to0^+} x^x
	= \lim_{x\to0^+} y
	= e^0
	= 1.
\end{equation*}
\end{solution}
\end{example}

洛必达法则是求未定式的一种有效方法,但最好能与其他求极限的方法结合使用.
例如能化简时应尽量先化简,可以应用等价无穷小替代或重要极限时应尽可能应用,这样可以使运算简便.

\begin{example}
%@see: 《高等数学(第六版 上册)》 P138 例10
求\(\lim_{x\to0} \frac{\tan x - x}{x^2 \sin x}\).
\begin{solution}
如果直接用洛必达法则,那么分母的导数(尤其是高阶导数)比较复杂.
如果作一个等价无穷小替代,那么运算就方便得多:
\begin{align*}
	\lim_{x\to0} \frac{\tan x - x}{x^2 \sin x}
	&= \lim_{x\to0} \frac{\tan x - x}{x^3} \cdot \frac{x}{\sin x}
	= \lim_{x\to0} \frac{\tan x - x}{x^3} \\
	&= \lim_{x\to0} \frac{\sec^2 x - 1}{3x^2} \\
	&= \lim_{x\to0} \frac{2\sec^2 x \tan x}{6x} \\
	&= \frac13 \lim_{x\to0} \frac{\tan x}{x} = \frac13.
\end{align*}
\end{solution}
\end{example}

\begin{example}
求\(\lim_{x\to0} \left[ \frac1x - \frac1{\ln(x+\sqrt{1+x^2})} \right]\).
\begin{solution}
直接计算得\begin{align*}
	\lim_{x\to0} \left[ \frac1x - \frac1{\ln(x+\sqrt{1+x^2})} \right]
	% 通分
	&= \lim_{x\to0} \frac{\ln(x+\sqrt{1+x^2}) - x}{x \ln(x+\sqrt{1+x^2})} \\
	% 等价无穷小\(\ln(x+\sqrt{1+x^2}) \sim x\)
	&= \lim_{x\to0} \frac{\ln(x+\sqrt{1+x^2}) - x}{x^2} \\
	% 洛必达法则 \cref{example:导数.反双曲正弦函数的导数}
	&= \lim_{x\to0} \frac{(1+x^2)^{-\frac12}-1}{2x} \\
	% 等价无穷小\((1+x^2)^{-\frac12}-1 \sim -\frac12 x^2\)
	&= \lim_{x\to0} \frac{-\frac12 x^2}{2x}
	= 0.
\end{align*}
\end{solution}
\end{example}
\begin{example}
求\(\lim_{x\to1} \left( \frac{x}{x-1} - \frac1{\ln x} \right)\).
\begin{solution}
直接计算得\begin{align*}
	\lim_{x\to1} \left( \frac{x}{x-1} - \frac1{\ln x} \right)
	% 通分
	&= \lim_{x\to1} \frac{x \ln x - (x-1)}{(x-1) \ln x} \\
	% 等价无穷小\(\ln x = \ln(1 + x-1) \sim x-1\ (x\to1)\)
	&= \lim_{x\to1} \frac{x \ln x - (x-1)}{(x-1)^2} \\
	% 洛必达法则
	&= \lim_{x\to1} \frac{\ln x}{2(x-1)}
	= \lim_{x\to1} \frac1{2x}
	= \frac12.
\end{align*}
\end{solution}
\end{example}

\begin{example}
求\(\lim_{x\to0} \frac{a^{x^2}-b^{x^2}}{(a^x-b^x)^2}\ (0<a<b)\).
\begin{solution}
这里我们先化简,再利用洛必达法则:\begin{align*}
	\lim_{x\to0} \frac{a^{x^2}-b^{x^2}}{(a^x-b^x)^2}
	&= \lim_{x\to0} \frac{a^{x^2}-b^{x^2}}{x^2}
		\cdot \lim_{x\to0} \left(\frac{x}{a^x-b^x}\right)^2 \\
	&= \lim_{x\to0^+} \frac{a^x-b^x}{x}
		\cdot \lim_{x\to0} \left(\frac{x}{a^x-b^x}\right)^2 \\
	&= \lim_{x\to0} \frac{x}{a^x-b^x} \\
	&= \lim_{x\to0} \frac{1}{a^x \ln a - b^x \ln b}
	= \frac{1}{\ln(a/b)}.
\end{align*}
\end{solution}
\end{example}

\section{泰勒公式}\label{section:微分中值定理.泰勒公式}
对于一些较复杂的函数,为了便于研究,往往希望用一些简单的函数来近似表达.
由于用多项式表示的函数,只要对自变量进行有限次加、减、乘三种算术运算,
便能求出它的函数值来,因此我们经常用多项式来近似表达函数.

在微分的应用中已经知道,当\(\abs{x}\)很小时,有如下的近似等式:\begin{equation*}
	e^x \approx 1+x, \qquad
	\ln(1+x) \approx x.
\end{equation*}
这些都是用一次多项式来近似表达函数的例子.
显然,在\(x=0\)处这些一次多项式及其一阶导数的值,
分别等于被近似表达的函数及其导数的相应值.

但是这种近似表达式还存在着不足之处:
首先是精确度不高,它所产生的误差仅是关于\(x\)的高阶无穷小;
其次是用它来作近似计算时,不能具体估算出误差大小.
因此,对于精确度要求较高且需要估计误差的时候,
就必须用高次多项式来近似表达函数,同时给出误差公式.

于是提出如下的问题:
设函数\(f\)在含有\(x_0\)的开区间内具有直到\((n+1)\)阶导数,
试找出一个关于\((x-x_0)\)的\(n\)次多项式\begin{equation*}
%@see: 《高等数学(第六版 上册)》 P140 (1)
	p_n(x)
	= a_0 + a_1 (x-x_0) + a_2 (x-x_0)^2 + \dotsb + a_n (x-x_0)^n
\end{equation*}来近似表达\(f\),
要求\(p_n(x)\)与\(f(x)\)之差是比\((x-x_0)^n\)高阶的无穷小,
并给出误差\(\abs{f(x) - p_n(x)}\)的具体表达式.

下面我们来讨论这个问题.
假设\(p_n\)在\(x_0\)处的函数值及它的直到\(n\)阶导数在\(x_0\)处的值
依次与\(f(x_0),f'(x_0),\dotsc,f^{(n)}(x_0)\)相等,
即满足\begin{equation*}
	p_n(x) = f(x_0),
	p_n'(x) = f'(x_0),
	p_n''(x) = f''(x_0),
	\dotsc,
	p_n^{(n)}(x_0) = f^{(n)}(x_0),
\end{equation*}
按这些等式来确定多项式函数\(p_n\)的系数\(a_0,a_1,a_2,\dotsc,a_n\).
为此,对\(p_n\)求各阶导数,然后分别代入以上等式,得\begin{equation*}
	0! a_0 = f(x_0),
	1! a_1 = f'(x_0),
	2! a_2 = f''(x_0),
	\dotsc,
	n! a_n = f^{(n)}(x_0),
\end{equation*}
即得\begin{equation*}
	a_0 = \frac1{0!} f(x_0),
	a_1 = \frac1{1!} f'(x_0),
	a_2 = \frac1{2!} f''(x_0),
	\dotsc,
	a_n = \frac1{n!} f^{(n)}(x_0).
\end{equation*}

将求得的系数\(a_0,a_1,a_2,\dotsc,a_n\)代回多项式\(p_n\),有\begin{equation*}
%@see: 《高等数学(第六版 上册)》 P140 (2)
	p_n(x) = f(x_0) + f'(x_0) (x-x_0)
	+ \frac{f''(x_0)}{2!} (x-x_0)^2
	+ \dotsb
	+ \frac{f^{(n)}(x_0)}{n!} (x-x_0)^n.
\end{equation*}
下面的定理表明,上述多项式的确是所要找的\(n\)次多项式.

\begin{theorem}[泰勒中值定理]\label{theorem:微分中值定理.泰勒中值定理}
%@see: 《数学分析(第7版 第一卷)》(卓里奇) P184 定理2
%@see: 《高等数学(第六版 上册)》 P140 泰勒中值定理
%@see: 《数学分析教程(第3版 上册)》(史济怀) P199 定理4.3.1
\def\dyy{I}%定义域
\def\Xc{x_0,x}%
\def\Xa{\min\{\Xc\}}%
\def\Xb{\max\{\Xc\}}%
\def\X{\Xa,\Xb}%
设\([a,b] \subseteq \dyy \subseteq \mathbb{R}\),
函数\(f\colon \dyy\to\mathbb{R}\)满足\begin{equation*}
	f \in C^n[a,b] \cap D^{n+1}(a,b).
\end{equation*}
那么对于\(\forall\Xc\in[a,b]\),
记\begin{equation*}
	\alpha=\Xa, \qquad
	\beta=\Xb, \qquad
	X = (\alpha,\beta), \qquad
	\overline{X} = [\alpha,\beta],
\end{equation*}
对于\(\forall\phi \in C(\overline{X}) \cap D(X)
\cap \Set{u \given (\exists x \in X)[u'(x)\neq0]}\),
\(\exists\xi \in X\),
使得
\begin{equation}\label{equation:微分中值定理.泰勒公式1}
%@see: 《数学分析(第7版 第一卷)》(卓里奇) P183 (7)
%@see: 《高等数学(第六版 上册)》 P140 (3)
	f(x) = p_n(x) + R_n(x),
\end{equation}
其中\begin{gather}
%@see: 《数学分析(第7版 第一卷)》(卓里奇) P183 (5)
	p_n(x) = \sum_{k=0}^n \frac{f^{(k)}(x_0)}{k!} (x-x_0)^k,
		\label{equation:微分中值定理.泰勒公式.多项式1} \\
%@see: 《数学分析(第7版 第一卷)》(卓里奇) P184 (8)
	R_n(x) = \frac{\phi(x)-\phi(x_0)}{\phi'(\xi) n!} f^{(n+1)}(\xi) (x-\xi)^n.
		\label{equation:微分中值定理.泰勒公式.余项0}
\end{gather}
\begin{proof}
考虑关于\(t\)的函数\begin{equation*}
%@see: 《数学分析(第7版 第一卷)》(卓里奇) P184 (10)
	F(t) = f(x) - \left[
		\frac{f(t)}{0!} + \frac{f'(t)}{1!} (x-t) + \frac{f''(t)}{2!} (x-t)^2
		+ \dotsb + \frac{f^{(n)}(t)}{n!} (x-t)^n
	\right].
	\eqno(1)
\end{equation*}
可知\(F \in C(\overline{X}) \cap D(X)\),
且\begin{align*}
	\dv{x}\kern-3pt\left[ \frac{f(t)}{0!} \right]
	&= \frac{f'(t)}{0!}, \\
	\dv{x}\kern-3pt\left[ \frac{f^{(k)}(t)}{k!} (x-t)^k \right]
	&= -\frac{f^{(k)}(t)}{(k-1)!} (x-t)^{k-1}
	+ \frac{f^{(k+1)}(t)}{k!} (x-t)^k,
	\quad k=2,3,\dotsc,n,
\end{align*}
从而有\begin{align*}
	F'(t)
	&= -\biggl[
	\frac{f'(t)}{0!} - \frac{f'(t)}{1!} + \frac{f''(t)}{1!} (x-t) - \frac{f''(t)}{1!} (x-t) \\
	&\hspace{25pt}+ \frac{f'''(t)}{2!} (x-t)^2 - \dotsb + \frac{f^{(n+1)}(t)}{n!} (x-t)^n
	\biggr] \\
	&= -\frac{f^{(n+1)}(t)}{n!} (x-t)^n.
\end{align*}
应用\hyperref[theorem:微分中值定理.柯西中值定理]{柯西中值定理},
可知\(\exists\xi\in X\),
使得\begin{equation*}
	\frac{F(\beta) - F(\alpha)}{\phi(\beta) - \phi(\alpha)}
	= \frac{F'(\xi)}{\phi'(\xi)}.
	\eqno(2)
\end{equation*}

把\(F'(\xi)\)的表达式\begin{equation*}
	F'(\xi) = -\frac{f^{(n+1)}(\xi)}{n!} (x-\xi)^n
	\eqno(3)
\end{equation*}代入(2)式,
得\begin{equation*}
	F(\beta) - F(\alpha)
	= -\frac{\phi(\beta) - \phi(\alpha)}{\phi'(\xi) n!} f^{(n+1)}(\xi) (x-\xi)^n.
\end{equation*}
于是\begin{equation*}
	F(x) - F(x_0)
	= -\frac{\phi(x) - \phi(x_0)}{\phi'(\xi) n!} f^{(n+1)}(\xi) (x-\xi)^n.
	\eqno(4)
\end{equation*}
最后,把\(F(x) = 0\)和\(F(x_0) = f(x) - p_n(x)\)代入(4)式,
就可得到\cref{equation:微分中值定理.泰勒公式.余项0}.
\end{proof}
\end{theorem}
多项式 \labelcref{equation:微分中值定理.泰勒公式.多项式1}
称为“函数\(f(x)\)按\((x-x_0)\)的幂展开的\(n\)次\DefineConcept{泰勒多项式}”.

在\cref{equation:微分中值定理.泰勒公式.余项0} 中取\(\phi(t) = (x-t)^{n+1}\),就得到
\begin{equation}\label{equation:微分中值定理.泰勒公式.余项1}
%@see: 《数学分析教程(第3版 上册)》(史济怀) P200 (2)
%@see: 《数学分析(第7版 第一卷)》(卓里奇) P184 (12)
%@see: 《高等数学(第六版 上册)》 P140 (4)
	R_n(x) = \frac{f^{(n+1)}(\xi)}{(n+1)!} (x-x_0)^{n+1}.
\end{equation}
像这样的\(R_n(x)\)的表达式 \labelcref{equation:微分中值定理.泰勒公式.余项1}
称为\DefineConcept{拉格朗日余项}(the Lagrange form of the remainder),
继而公式 \labelcref{equation:微分中值定理.泰勒公式1}
称为“\(f(x)\)按\((x-x_0)\)的幂展开的带有拉格朗日余项的\(n\)阶\DefineConcept{泰勒公式}
(Taylor's formula with the Lagrange form of the remainder)”.

在\cref{equation:微分中值定理.泰勒公式.余项0} 中取\(\phi(t) = x-t\)就得到
\begin{equation}\label{equation:微分中值定理.泰勒公式.余项4}
%@see: 《数学分析教程(第3版 上册)》(史济怀) P200 (3)
%@see: 《数学分析(第7版 第一卷)》(卓里奇) P184 (11)
	R_n(x) = \frac{f^{(n+1)}(\xi)}{n!} (x-\xi)^n (x-x_0).
\end{equation}
像这样的\(R_n(x)\)的表达式 \labelcref{equation:微分中值定理.泰勒公式.余项4}
称为\DefineConcept{柯西余项}(the Cauchy form of the remainder).

当\(n=0\)时,泰勒公式变成\hyperref[equation:微分中值定理.拉格朗日中值公式]{拉格朗日中值公式}:\begin{equation*}
	f(x) = f(x_0) + f'(\xi) (x-x_0), \quad x_0 < \xi < x.
\end{equation*}
因此,泰勒中值定理是拉格朗日中值定理的推广.

由泰勒中值定理可知,以多项式\(p_n(x)\)近似表达函数\(f(x)\)时,
其误差为\(\abs{R_n(x)}\).
如果对于某个固定的\(n\),
当\(x\in(a,b)\)时,
\(\abs{f^{(n+1)}(x)} \leq M\),
则有估计式\begin{equation}\label{equation:微分中值定理.泰勒公式.误差1}
	\abs{R_n(x)}
	= \abs{\frac{f^{(n+1)}(\xi)}{(n+1)!} (x-x_0)^{n+1}}
	\leq \frac{M}{(n+1)!} \abs{x-x_0}^{n+1}
\end{equation}
及\begin{equation*}
	\lim_{x \to x_0} \frac{R_n(x)}{(x-x_0)^n} = 0
\end{equation*}
由此可见,当\(x \to x_0\)时,
误差\(\abs{R_n(x)}\)是比\((x-x_0)^n\)高阶的无穷小,
即\begin{equation}\label{equation:微分中值定理.泰勒公式.余项2}
	R_n(x) = o[(x-x_0)^n].
\end{equation}

在不需要余项的精确表达式时,\(n\)阶泰勒公式也可以写成
\begin{equation}\label{equation:微分中值定理.泰勒公式2}
	f(x) = p_n(x) + o[(x - x_0)^n].
\end{equation}
\(R_n(x)\)的表达式 \labelcref{equation:微分中值定理.泰勒公式.余项2} 称为\DefineConcept{皮亚诺余项}.
\cref{equation:微分中值定理.泰勒公式2} 称为
“\(f(x)\)按\((x-x_0)\)的幂展开的带有皮亚诺余项的\(n\)阶泰勒公式
(Taylor's formula with the Peano form of the remainder)”.

在泰勒公式 \labelcref{equation:微分中值定理.泰勒公式1} 中,
如果取\(x_0 = 0\),则\(\xi\)在\(0\)与\(x\)之间.
因此可以令\(\xi = \theta x\ (0 < \theta < 1)\),从而泰勒公式变成较简单的形式,
即所谓“带有拉格朗日余项的\DefineConcept{麦克劳林公式}
(Maclaurin's formula with the Lagrange form of the remainder)”:
\begin{equation}\label{equation:微分中值定理.泰勒公式3}
	f(x)=\sum_{k=0}^n \frac{f^{(k)}(0)}{k!} x^k
		+ \frac{f^{(n+1)}(\theta x)}{(n+1)!} x^{n+1},
	\quad 0 < \theta < 1.
\end{equation}

在泰勒公式 \labelcref{equation:微分中值定理.泰勒公式2} 中,
如果取\(x_0 = 0\),则有“带有皮亚诺余项的麦克劳林公式
(Maclaurin's formula with the Peano form of the remainder)”:
\begin{equation}\label{equation:微分中值定理.泰勒公式4}
	f(x)=\sum_{k=0}^n \frac{f^{(k)}(0)}{k!} x^k + o(x^n).
\end{equation}

误差估计式 \labelcref{equation:微分中值定理.泰勒公式.误差1} 相应地变成:
\begin{equation}\label{equation:微分中值定理.泰勒公式.误差2}
	\abs{R_n(x)} \leq \frac{M}{(n+1)!} \abs{x}^{n+1}.
\end{equation}

\begin{example}
%@see: 《高等数学(第六版 上册)》 P142 例1
写出函数\(f(x) = e^x\)的带有拉格朗日余项的\(n\)阶麦克劳林公式.
\begin{solution}
因为\(f(x)=f'(x)=f''(x)=\dotsb=f^{(n)}(x)=f^{(n+1)}(x)=e^x\),所以\begin{equation*}
	f(0)=f'(0)=f''(0)=\dotsb=f^{(n)}(0)=1.
\end{equation*}
将这些值代入带有拉格朗日余项的麦克劳林公式,
并注意到\(f^{(n+1)}(\theta x) = e^{\theta x}\)
便得\begin{equation*}%\(e^x\)的泰勒公式
	e^x = 1 + x + \frac{1}{2!} x^2 + \dotsb
	+ \frac{1}{n!} x^n + \frac{e^{\theta x}}{(n+1)!} x^{n+1},
	\quad 0 < \theta < 1.
\end{equation*}

由这个公式可知,若把\(e^x\)用它的\(n\)次泰勒多项式表达为\begin{equation*}
	e^x \approx 1 + x + \frac{x^2}{2!} + \dotsb + \frac{x^n}{n!},
\end{equation*}
这时所产生的误差为\begin{equation*}
	\abs{R_n(x)} = \abs{\frac{e^{\theta x}}{(n+1)!} x^{n+1}}
	< \frac{e^{\abs{x}}}{(n+1)!} \abs{x}^{n+1},
	\quad 0 < \theta < 1.
\end{equation*}

如果取\(x = 1\),则得无理数\(e\)的近似式为\begin{equation*}
	e \approx 1 + 1 + \frac{1}{2!} + \dotsb + \frac{1}{n!},
\end{equation*}
其误差\(\abs{R_n} < \frac{e}{(n+1)!} < \frac{3}{(n+1)!}\).
当\(n=10\)时,可算出\(e \approx 2.718\ 282\),其误差不超过\(10^{-6}\).
\end{solution}
\end{example}

\begin{example}
%@see: 《高等数学(第六版 上册)》 P143 例2
求\(f(x)=\sin x\)的带有拉格朗日余项的\(n\)阶麦克劳林公式.
\begin{solution}
因为\begin{align*}
	f'(x)=\cos x,
	f''(x)=-\sin x,
	f'''(x)=-\cos x, \\
	f^{(4)}(x)=\sin x,
	\dotsc,
	f^{(n)}(x)=\sin\left(x+\frac{n\pi}{2}\right),
\end{align*}
所以\begin{equation*}
	f(0)=0,f'(0)=1,f''(0)=0,f'''(0)=-1,f^{(4)}(0)=0
\end{equation*}等等,
它们依次循环地取四个数\(0,1,0,-1\),
于是按带有拉格朗日余项的麦克劳林公式(令\(n=2m\))得\begin{equation*}
	\sin x = x - \frac{x^3}{3!} + \frac{x^5}{5!} - \dotsb + (-1)^{m-1} \frac{x^{2m-1}}{(2m-1)!} + R_{2m},
\end{equation*}
其中\begin{align*}
	R_{2m}
	&= \frac{1}{(2m+1)!} \sin\left[\theta x + (2m+1)\frac{\pi}{2}\right] x^{2m+1} \\
	&= (-1)^m \frac{\cos \theta x}{(2m+1)!} x^{2m+1},
	\quad 0<\theta<1.
\end{align*}

如果取\(m=1\),则得近似公式\begin{equation*}
	\sin x \approx x,
\end{equation*}
这时误差为\begin{equation*}
	\abs{R_2} = \abs{-\frac{\cos \theta x}{3!} x^3}
	\leq \frac{\abs{x}^3}{6},
	\quad 0<\theta<1.
\end{equation*}
\end{solution}
\end{example}

类似地,还可以得到\begin{equation*}%\(\cos x\)的泰勒公式
	\cos x
	= 1 - \frac{1}{2!} x^2
		+ \frac{1}{4!} x^4 - \dotsb
		+ (-1)^m \frac{1}{(2m)!} x^{2m}
		+ R_{2m+1}(x),
\end{equation*}
其中\begin{align*}%\(\cos x\)的泰勒公式的拉格朗日余项
	R_{2m+1}(x)
	&= \frac{x^{2m+2}}{(2m+2)!} \cos\left[\theta x + (m+1)\pi\right] \\
	&= (-1)^{m+1} \frac{\cos \theta x}{(2m+2)!} x^{2m+2},
	\quad 0<\theta<1;
\end{align*}
以及\begin{equation*}%\(\ln(1+x)\)的泰勒公式
	\ln(1+x)
	= x - \frac{1}{2} x^2 + \frac{1}{3} x^3 - \dotsb
		+ (-1)^{n-1} \frac{1}{n} x^n + R_n(x),
\end{equation*}
其中\begin{equation*}%\(\ln(1+x)\)的泰勒公式的拉格朗日余项
	R_n(x) = \frac{\alpha(\alpha-1)\dotsm(\alpha-n+1)(\alpha-n)}{(n+1)!}
	(1+\theta x)^{\alpha-n-1} x^{n+1}
	\quad(0<\theta<1)
\end{equation*}

\begin{example}
%@see: 《高等数学(第六版 上册)》 P144 例3
求极限\(\lim_{x\to0}\frac{\sin x - x \cos x}{\sin^3 x}\).
\begin{solution}
由于分式的分母\(\sin^3 x \sim x^3\ (x\to0)\),
我们只需将分子中的\(\sin x\)和\(x \cos x\)分别用带有皮亚诺余项的三阶麦克劳林公式表示,
即\begin{equation*}
	\sin x = x - \frac{x^3}{3!} + o(x^3),
	\qquad
	x \cos x = x - \frac{x^3}{2!} + o(x^3).
\end{equation*}
于是\begin{equation*}
	\sin x - x \cos x = \frac{1}{3} x^3 + o(x^3),
\end{equation*}
故\begin{equation*}
	\lim_{x\to0}\frac{\sin x - x \cos x}{\sin^3 x}
	= \lim_{x\to0}\frac{\frac{1}{3} x^3 + o(x^3)}{x^3} = \frac{1}{3}.
\end{equation*}
\end{solution}
\end{example}

\begin{example}
设函数\(f\)具有二阶连续导数,\(\lim_{x\to0} \frac{f(x)}{x} = 0\),\(f''(0)\neq0\),
若\begin{equation*}
	\lim_{x\to0} \frac{e^{f(x)}-ax-b}{x^2} = c \neq 0,
\end{equation*}
求\(a,b,c\).
\begin{solution}
因为\(\lim_{x\to0} \frac{f(x)}{x} = 0\),
所以\(f(x) = o(x)\),
\(\lim_{x\to0} f(x) = 0\).
又因为\(f\)是连续的,
所以\(f(0) = \lim_{x\to0} f(x) = 0\),
从而\begin{equation*}
	f'(0) = \lim_{x\to0} \frac{f(x) - f(0)}{x - 0} = \frac{f(x)}{x} = 0.
\end{equation*}
因此\begin{equation*}
	f(x) = \frac12 f''(0) x^2 + o(x^2),
	\qquad
	[f(x)]^2 = o(x^2).
\end{equation*}

因为\(e^x = 1 + x + \frac{1}{2!} x^2 + o(x^2)\),
所以\begin{equation*}
	e^{f(x)} = 1 + f(x) + \frac{1}{2} [f(x)]^2 + o(x^2)
	= 1 + \frac{1}{2} f''(0) x^2 + o(x^2).
\end{equation*}

因为\begin{equation*}
	\lim_{x\to0} \frac{e^{f(x)} - ax - b}{c x^2} = 1,
\end{equation*}
所以\(e^{f(x)} - ax - b = c x^2 + o(x^2)\),
即\begin{equation*}
	\left[ 1 + \frac{1}{2} f''(0) x^2 + o(x^2) \right] - ax - b = c x^2 + o(x^2).
\end{equation*}
对比得\(a = 0, b = 1, c = \frac{1}{2} f''(0)\).
\end{solution}
\end{example}

\begin{example}
计算极限\begin{equation*}
	\lim_{x\to1} \left(\frac{m}{1-x^m} - \frac{n}{1-x^n}\right).
\end{equation*}
\begin{solution}
直接计算得
\begin{align*}
	&\hspace{-10pt}
	\lim_{x\to1} \left(\frac{m}{1-x^m} - \frac{n}{1-x^n}\right) \\
	&\xlongequal{x-1=t}
	\lim_{t\to0} \left[\frac{m}{1-(1+t)^m}-\frac{n}{1-(1+t)^n}\right] \\
	&=
	\lim_{t\to0} \frac{m[1-(1+t)^n]-n[1-(1+t)^m]}{[1-(1+t)^m][1-(1+t)^n]} \\
	&=
	\lim_{t\to0} \frac{m\left[-nt-\frac{1}{2}n(n-1)t^2+o(t^2)\right]
		-n\left[-mt-\frac{1}{2}m(m-1)t^2+o(t^2)\right]}{[-mt+o(t)][-nt+o(t)]} \\
	&=
	\lim_{t\to0} \frac{\frac{1}{2}mn(m-n)t^2+o(t^2)}{mnt^2+o(t^2)}
	= \frac{m-n}{2}.
\end{align*}
\end{solution}
\end{example}

\begin{example}
计算极限\begin{equation*}
	\lim_{n\to\infty} \frac{1}{e^n} \left(1+\frac{1}{n}\right)^{n^2}.
\end{equation*}
\begin{solution}
注意到当\(n\to\infty\)时,
\(1+\frac1n\to1\)而\(n^2\to\infty\),
可见\(\left(1+\frac{1}{n}\right)^{n^2}\)是\(1^\infty\)型未定式.
于是利用\hyperref[equation:函数.对数恒等式]{对数恒等式}变形得到\begin{equation*}
	\frac{1}{e^n} \left(1+\frac{1}{n}\right)^{n^2}
	= \exp\ln\frac{1}{e^n} \left(1+\frac{1}{n}\right)^{n^2}
	= \exp[ n^2 \ln(1+\frac{1}{n}) - n ].
\end{equation*}
利用\(\ln(1+x)\)的泰勒公式\begin{equation*}
	\ln(1+x)
	= x - \frac{1}{2} x^2 + \frac{1}{3} x^3 - \dotsb
		+ (-1)^{n-1} \frac{1}{n} x^n + o(x^n)
\end{equation*}得到\begin{equation*}
	\ln(1+\frac1n)
	= \frac1n - \frac12\cdot\frac1{n^2} + \frac13\cdot\frac1{n^3}
	+ o\left( \frac1{n^3} \right),
\end{equation*}
那么\begin{equation*}
	\exp[ n^2 \ln(1+\frac{1}{n}) - n ]
	= \exp[
		n - \frac12 + \frac13\cdot\frac1n + o\left( \frac1n \right)
		- n
	]
	\to e^{-\frac12}
	\quad(n\to\infty).
\end{equation*}
\end{solution}
\end{example}
\begin{example}
求\(f(x) = (1+x)^{\frac1x}\)的带有皮亚诺余项的\(n\)阶麦克劳林公式.
\begin{solution}
%@see: https://math.stackexchange.com/a/1975754/591741
利用\hyperref[equation:函数.对数恒等式]{对数恒等式}变形得到\begin{equation*}
	f(x) = (1+x)^{\frac1x}
	= \exp\frac{\ln(1+x)}x.
\end{equation*}
由\(\ln(1+x)\)的泰勒公式\begin{equation*}
	\ln(1+x)
	= x - \frac12 x^2 + \frac13 x^3 - \dotsb
		+ (-1)^{n-1} \frac1n x^n + o(x^n)
\end{equation*}得\begin{align*}
	f(x) &= \exp\left(
		1 - \frac12 x + \frac13 x^2 - \dotsb
		+ (-1)^{n-1} \frac1n x^{n-1} + o(x^n)
	\right) \\
	&= e \cdot \exp(-\frac12 x + \frac13 x^2 - \dotsb + (-1)^{n-1} \frac1n x^{n-1} + o(x^n)).
\end{align*}
记\(g(x) = -\frac12 x + \frac13 x^2 - \dotsb + (-1)^{n-1} \frac1n x^{n-1}\),
则\(f(x) = e \cdot \exp(g(x) + o(x^n))\).
再由\(e^x\)的泰勒公式\begin{equation*}
	e^x = 1 + x + \frac1{2!} x^2 + \dotsb
	+ \frac1{n!} x^n + o(x^n)
\end{equation*}得\begin{equation*}
	f(x)
	= e \left[ 1 + g(x) + \frac1{2!} g^2(x) + \dotsb + \frac1{n!} g^n(x) + o(x^n) \right].
\end{equation*}
特别地,当\(n=4\)时,有\begin{equation*}
	f(x) = e \left( 1 - \frac12 x + \frac{11}{24} x^2 - \frac7{16} x^3 \right) + o(x^4).
\end{equation*}
\end{solution}
%@Mathematica: Series[(1 + x)^(1/x), {x, 0, 3}]
\end{example}

\begin{example}
%@see: 《2023年全国硕士研究生入学统一考试(数学一)》三解答题/第20题
设函数\(f(x)\)在\([-a,a]\)上具有二阶连续导数,证明:
\begin{enumerate}
	\item 若\(f(0)=0\),则存在\(\xi\in(-a,a)\),
	使得\(f''(\xi) = \frac{1}{a^2} [f(a) + f(-a)]\);

	\item 若\(f(x)\)在\((-a,a)\)内取得极值,则存在\(\eta\in(-a,a)\),
	使得\begin{equation*}
		\abs{f''(\eta)}
		\geq
		\frac{1}{2a^2} \abs{f(a) - f(-a)}.
	\end{equation*}
\end{enumerate}
\begin{proof}
假设\(f(0) = 0\).
因为函数\(f\)在\([-a,a]\)上具有二阶连续导数,
由泰勒中值定理可知,
对\(\forall x\in(-a,a)\),有
\begin{align*}
	f(x) &= f(0) + f'(0) x + \frac{1}{2} f''(\xi) x^2 \\
	&= f'(0) x + \frac{1}{2} f''(\xi) x^2,
\end{align*}
其中\(\xi\)是\(0\)与\(x\)之间的某个值.
因此\begin{equation*}
	f(a) = f'(0) a + \frac{1}{2} f''(\xi_1) a^2,
	\quad \xi_1\in(0,a),
	\eqno(1)
\end{equation*}\begin{equation*}
	f(-a) = f'(0) (-a) + \frac{1}{2} f''(\xi_2) (-a)^2,
	\quad \xi_2\in(-a,0).
	\eqno(2)
\end{equation*}
(1)、(2)两式相加得\begin{equation*}
	f(a) + f(-a) = \frac{a^2}{2} [f''(\xi_1) + f''(\xi_2)].
	\eqno(3)
\end{equation*}
又因为\(f''(x)\)在闭区间\([\xi_2,\xi_1]\)上连续,
必有最大值\(M\)和最小值\(m\),即\begin{equation*}
	m \leq f''(\xi_1) \leq M,
	\qquad
	m \leq f''(\xi_2) \leq M,
\end{equation*}
从而\begin{equation*}
	m \leq \frac{f''(\xi_1) + f''(\xi_2)}{2} \leq M.
\end{equation*}
由介值定理得,\(\exists\xi\in[\xi_2,\xi_1]\subseteq(-a,a)\),使得\begin{equation*}
	\frac{f''(\xi_1) + f''(\xi_2)}{2} = f''(\xi).
	\eqno(4)
\end{equation*}
将(4)式代入(3)式,命题1得证.

假设\(f(x)\)在\((-a,a)\)内取得极值.
设\(f\)在点\(x=x_0\in(-a,a)\)处取得极值,由费马引理可知\(f'(x_0)=0\).
于是有函数\(f\)按\(x-x_0\)的幂展开的带有拉格朗日余项的1阶泰勒公式:
\begin{align*}
	f(x) &= f(x_0) + f'(x_0) (x-x_0) + \frac{f''(\gamma)}{2!} (x-x_0)^2 \\
	&= f(x_0) + \frac{f''(\gamma)}{2!} (x-x_0)^2
	\quad(\text{\(\gamma\)介于\(x_0\)与\(x\)之间}),
\end{align*}
则\begin{equation*}
	f(-a) = f(x_0) + \frac{f''(\gamma_1)}{2!}(-a-x_0)^2,
	\quad\gamma_1\in(-a,x_0),
\end{equation*}\begin{equation*}
	f(a) = f(x_0) + \frac{f''(\gamma_2)}{2!} (a-x_0)^2,
	\quad\gamma_2\in(x_0,a),
\end{equation*}
从而
\begin{align*}
	\abs{f(a)-f(-a)}
	&= \abs{\frac{1}{2} (a-x_0)^2 f''(\gamma_2) - \frac{1}{2} (a+x_0)^2 f''(\gamma_1)} \\
	&\leq \frac{1}{2} \abs{(a-x_0)^2 f''(\gamma_2)} + \frac{1}{2} \abs{(a+x_0)^2 f''(\gamma_1)}.
\end{align*}
因为\(\abs{f''(x)}\)连续,
设\(M = \max\{ \abs{f''(\gamma_1)}, \abs{f''(\gamma_2)} \}\),
则\begin{equation*}
	\abs{f(a) - f(-a)}
	\leq \frac{1}{2} M(a+x_0)^2 + \frac{1}{2} M(a-x_0)^2
	= M(a^2 + x_0^2).
\end{equation*}
因为\(x_0\in(-a,a)\),
则\begin{equation*}
	\abs{f(a) - f(-a)} \leq M(a^2+x_0^2) \leq 2 M a^2,
\end{equation*}
则\(M \geq \frac{1}{2 a^2} \abs{f(a) - f(-a)}\),
即存在\(\eta\in\{\gamma_1,\gamma_2\}\subseteq(-a,a)\),
使得\(\abs{f''(\eta)} \geq \frac{1}{2 a^2} \abs{f(a) - f(-a)}\).
命题2得证.
\end{proof}
\end{example}

%积分余项的表达式为\begin{equation}\label{equation:微分中值定理.泰勒公式.余项3}
%R_n = \int_{x_0}^x f^{(n+1)}(t) \frac{(x-t)^n}{n!} \dd{t}.
%\end{equation}
%积分余项的应用条件是:\(f \in C^n\),即\(n\)阶可导且导函数均连续.

\begin{example}
%@see: https://math.stackexchange.com/questions/4829692/what-is-the-limit-of-x-n-x-n1n
%@credit: {8b6edada-f2fd-4ae5-9020-eb533149a54c},{5a781423-ba4e-4629-ac1a-eac743a4d445}
设数列\(\{x_n\}\)满足\(0<x_1<\pi\),且有递推公式\(x_{n+1}=\sin x_n\ (n=1,2,\dotsc)\).
证明以下极限收敛,并进行计算:\begin{equation*}
	\lim_{n\to\infty} x_n,
	\qquad
	\lim_{n\to\infty} \left(\frac{x_{n+1}}{x_n}\right)^{1/x_n^2},
	\qquad
	\lim_{n\to\infty} \left(\frac{x_n}{x_{n+1}}\right)^n.
\end{equation*}
\begin{solution}
令\(f(x) = x - \sin x\ (0<x<\pi)\),
求导得\(f'(x) = 1 - \cos x\).
因为\(-1 < \cos x < 1\ (0<x<\pi)\),
所以\begin{equation*}
	0 < f'(x) < 2
	\quad(0<x<\pi)
\end{equation*}
恒成立,
这就说明函数\(f\)在\((0,\pi)\)严格单调增加.
又因为\(f(0) = 0\),
所以恒成立\begin{equation*}
	f(x) = x - \sin x > 0
	\quad(0<x<\pi).
\end{equation*}
因为\(0<x_1<\pi\),
所以\begin{equation*}
	0<x_2=\sin x_1\leq1<\frac\pi2.
\end{equation*}
利用数学归纳法易证:当\(n\geq2\)时,成立\(0<x_n\leq1\),数列\(\{x_n\}\)有界.
进一步有\begin{equation*}
	0<x_3=\sin x_2<x_2<1.
\end{equation*}
利用数学归纳法易证:当\(n\geq3\)时,成立\(0<x_{n+1}<x_n<1\),数列\(\{x_n\}\)严格单调减少.
根据\hyperref[theorem:极限.数列的单调有界定理]{单调有界定理}可知,数列\(\{x_n\}\)收敛.

假设\(\lim_{n\to\infty} x_n = x\).
对递推公式\(x_{n+1}=\sin x_n\)令\(n\to\infty\)得\(x=\sin x\).
由上可知,方程\(x=\sin x\ (0 \leq x \leq 1)\)有唯一解\(x=0\),
因此\(\lim_{n\to\infty} x_n = 0\).

直接计算\begin{align*}
	\lim_{n\to\infty} \left(\frac{x_{n+1}}{x_n}\right)^{1/x_n^2}
	&= \lim_{x\to0^+} \left(\frac{\sin x}{x}\right)^{1/x^2}
	= \lim_{x\to0^+} \exp\left(\frac1{x^2}\ln\frac{\sin x}{x}\right) \\
	&= \exp\lim_{x\to0^+} \frac1{x^2} \ln\left(1+\frac{\sin x-x}{x}\right) \\
	&= \exp\lim_{x\to0^+} \frac{\sin x-x}{x^3} \\
	&= \exp\lim_{x\to0^+} \frac{-x^3/6}{x^3}
	= e^{-\frac16}.
\end{align*}

令\begin{equation*}
	f(x) = \left\{ \begin{array}{cl}
		\frac{x}{\sin x}, & x\neq0, \\
		1, & x=0.
	\end{array} \right.
\end{equation*}
当\(x\neq0\)时,求导得\begin{equation*}
	f'(x) = \frac{\sin x - x \cos x}{\sin^2 x},
	\qquad
	f''(x) = \frac{x \sin^3 x - 2 \sin^2 x \cos x + 2 x \sin x \cos^2 x}{\sin^4 x}.
\end{equation*}
于是\(f(0) = 1,
f'(0) = 0,% 这里用到了\hyperref[theorem:微分中值定理.导数极限定理]{导数极限定理}
f''(0) = 1/3\),
从而有\begin{equation*}
	f(x) = 1 + \frac16 x^2 + o(x^2).
\end{equation*}
因此\begin{align*}
	\lim_{n\to\infty} \left(\frac{x_n}{x_{n+1}}\right)^n
	&= \lim_{n\to\infty} \left(\frac{x_n}{\sin x_n}\right)^n
	= \lim_{n\to\infty} \left(1+\frac16 x_n^2+o(x_n^2)\right)^n \\
	&= \exp\lim_{n\to\infty} n\ln\left(1+\frac16 x_n^2+o(x_n^2)\right)
	= \exp\lim_{n\to\infty} \frac{n x_n^2}6,
\end{align*}
这里\begin{align*}
	\lim_{n\to\infty} n x_n^2
	&= \lim_{n\to\infty} \frac{n}{x_n^{-2}}
	= \lim_{n\to\infty} \frac1{x_{n+1}^{-2}-x_n^{-2}}
		\tag{\hyperref[theorem:极限.施托尔茨定理1]{施托尔茨定理}} \\
	&= \lim_{n\to\infty} \frac{x_n^2 x_{n+1}^2}{x_n^2-x_{n+1}^2}
	= \lim_{n\to\infty} \frac{x_n^2 \sin^2 x_n}{(x_n-\sin x_n)(x_n+\sin x_n)} \\
	&= \lim_{n\to\infty} \frac{x_n^4}{\frac16 x_n^3 \cdot 2 x_n}
	= 3,
\end{align*}
可知\begin{equation*}
	\lim_{n\to\infty} \left(\frac{x_n}{x_{n+1}}\right)^n
	= e^{\frac12}.
\end{equation*}
\end{solution}
% 利用递推公式定义数列
%@Mathematica: f[1] := 1
%@Mathematica: f[n_] := Sin[f[n - 1]]
% 在指定区间上绘制数列的散点图
%@Mathematica: DiscretePlot[f[n], {n, 1, 20}]
% 定义数列
%@Mathematica: g[n_] := (f[n]/f[n + 1])^n
% 在指定区间上绘制数列的散点图
%@Mathematica: DiscretePlot[g[n], {n, 1, 20}]
\end{example}

\begin{example}
%@see: 《2016年全国硕士研究生入学统一考试(数学一)》二填空题/第12题
设函数\(f(x) = \arctan x - \frac{x}{1 + a x^2}\),
且\(f'''(0) = 1\),
求\(a\)的取值.
\begin{solution}
由于\begin{equation*}
	\arctan x = x - \frac13 x^3 + o(x^3),
	\qquad
	\frac1{1+x} = 1 - x + o(x),
\end{equation*}
而\begin{equation*}
	\frac1{1 + a x^2}
	= 1 - a x^2 + o(x^2),
	\qquad
	\frac{x}{1 + a x^2}
	= x - a x^3 + o(x^3),
\end{equation*}
所以\begin{equation*}
	f(x) = \left( x - \frac13 x^3 + o(x^3) \right)
	- \left( x - a x^3 + o(x^3) \right)
	= \left( a - \frac13 \right) x^3 + o(x^3).
\end{equation*}
又因为\begin{equation*}
	f(x) = f(0) + f'(0) x + \frac{f''(0)}{2!} x^2 + \frac{f'''(0)}{3!} x^3 + o(x^3),
\end{equation*}
所以由泰勒公式的唯一性可知\(\frac{f'''(0)}{3!} = a - \frac13\),
因此\(a = \frac13 + \frac1{3!}
= \frac12\).
\end{solution}
\end{example}

\section{线性插值}
%@see: https://mathworld.wolfram.com/Interpolation.html
%@see: https://math.libretexts.org/Bookshelves/Applied_Mathematics/Numerical_Methods_(Chasnov)/05%3A_Interpolation
在做力学实验时,我们经常会得到一系列实验数据.
要想从杂乱无章的数据中,提取出物理规律,
就需要我们有能力将一个点列\(\{(x_n,y_n)\}\)转化为一个函数关系,
利用这个函数给出预测结果,再用实验证实预测是否正确.
我们把这种能力称为“函数拟合”.
最常见的一种方法,是找出一个多项式函数\(l\),
使得\(l(x_i) = y_i\ (i=0,1,2,\dotsc,n)\)成立.

\subsection{范德蒙德插值法}
我们可以直接假设\begin{equation*}
	P_n(x) = c_0 x^n + c_1 x^{n-1} + \dotsb + c_n.
\end{equation*}
由此得到线性方程组\begin{equation*}
	\left\{ \begin{array}{l}
		y_0 = c_0 x_0^n + c_1 x_0^{n-1} + \dotsb + c_{n-1} x_0 + c_n, \\
		y_1 = c_0 x_1^n + c_1 x_1^{n-1} + \dotsb + c_{n-1} x_1 + c_n, \\
		\hdotsfor{1}, \\
		y_n = c_0 x_n^n + c_1 x_n^{n-1} + \dotsb + c_{n-1} x_n + c_n
	\end{array} \right.
\end{equation*}
或\begin{equation*}
	\begin{bmatrix}
		x_0^n & x_0^{n-1} & \dots & x_0 & 1 \\
		x_1^n & x_1^{n-1} & \dots & x_1 & 1 \\
		\vdots & \vdots && \vdots \\
		x_n^n & x_n^{n-1} & \dots & x_n & 1 \\
	\end{bmatrix}
	\begin{bmatrix}
		c_0 \\ c_1 \\ \vdots \\ c_n
	\end{bmatrix}
	= \begin{bmatrix}
		y_0 \\ y_1 \\ \vdots \\ y_n
	\end{bmatrix}.
\end{equation*}

\subsection{拉格朗日插值法}
\begin{definition}
%@see: 《数学分析教程(第3版 上册)》(史济怀) P201
设函数\(f\colon[a,b]\to\mathbb{R}\).
由两点\((a,f(a))\)和\((b,f(b))\)所决定的线性函数\begin{equation*}
	l(x) = \frac{b-x}{b-a} f(a) + \frac{x-a}{b-a} f(b)
\end{equation*}称为“\(f\)在区间\([a,b]\)上的\DefineConcept{线性插值}”.
\end{definition}

\begin{theorem}\label{theorem:拉格朗日插值.误差估计}
%@see: 《数学分析教程(第3版 上册)》(史济怀) P202 定理4.3.2
设\(f\in C[a,b]\cap D^2(a,b)\),
\(l\)是\(f\)在区间\([a,b]\)上的线性插值.
如果\(\abs{f''}\)在\((a,b)\)上的上界为\(M\),
那么对任意的\(x\in[a,b]\),
有\begin{equation*}
	\abs{f(x)-l(x)}
	\leq \frac{M}8 (b-a)^2.
\end{equation*}
\begin{proof}
由\begin{gather*}
	l(x) = \frac{b-x}{b-a} f(a) + \frac{x-a}{b-a} f(b), \\
	f(x) = \frac{b-x}{b-a} f(x) + \frac{x-a}{b-a} f(x)
\end{gather*}
可得\begin{equation*}
	l(x) - f(x)
	= \frac{b-x}{b-a} (f(a) - f(x))
	+ \frac{x-a}{b-a} (f(b) - f(x)).
\end{equation*}
由\hyperref[equation:微分中值定理.泰勒公式.余项1]{带有拉格朗日余项的泰勒公式}有\begin{gather*}
	f(a) - f(x)
	= (a-x) f'(x) + \frac12 (a-x)^2 f''(\xi)
	\quad(a<\xi<x), \\
	f(b) - f(x)
	= (b-x) f'(x) + \frac12 (b-x)^2 f''(\eta)
	\quad(x<\eta<b).
\end{gather*}
于是\begin{equation*}
	l(x) - f(x)
	= \frac{(b-x)(x-a)}2
	\left(
		\frac{x-a}{b-a} f''(\xi)
		+ \frac{b-x}{b-a} f''(\eta)
	\right).
\end{equation*}
由于\(\frac{x-a}{b-a},\frac{b-x}{b-a}>0\)
且\(\frac{x-a}{b-a}+\frac{b-x}{b-a}=1\),
所以\begin{align*}
	\abs{l(x) - f(x)}
	&\leq
	\frac{(b-x)(x-a)}2
	\left(
		\frac{x-a}{b-a} \abs{f''(\xi)}
		+ \frac{b-x}{b-a} \abs{f''(\eta)}
	\right) \\
	&\leq
	\frac{M}2 (b-x)(x-a)
	\leq
	\frac{M}2 \left(
		\frac{b-x+x-a}2
	\right)^2
	= \frac{M}8 (b-a)^2.
	\qedhere
\end{align*}
\end{proof}
\end{theorem}
\begin{remark}
\cref{theorem:拉格朗日插值.误差估计} 说明,
如果\(M\)越小,那么线性插值的逼近效果就越好.
当\(M\)很小时,曲线\(y=f(x)\)的切线改变得不剧烈.
这也是符合几何直观的.
\end{remark}

我们可以推广到已知\(m\)个点的情形.
\begin{equation}\label{equation:拉格朗日插值.拉格朗日插值公式2}
	\begin{aligned}
		l(x)
		&= y_0 \cdot \frac{(x-x_1)\dotsm(x-x_m)}{(x_0-x_1)\dotsm(x_0-x_m)}
		+ \dotsb \\
		&\hspace{20pt}
		+ y_i \cdot \frac{(x-x_0)\dotsm(x-x_{i-1})(x-x_{i+1})\dotsm(x-x_m)}
		{(x_i-x_0)\dotsm(x_i-x_{i-1})(x_i-x_{i+1})\dotsm(x_i-x_m)} \\
		&\hspace{20pt}
		+ \dotsb
		+ y_m \cdot \frac{(x-x_0)\dotsm(x-x_{m-1})}{(x_m-x_0)\dotsm(x_m-x_{m-1})}.
	\end{aligned}
\end{equation}
若记\begin{equation}
	l_i(x) = \frac{(x-x_0)\dotsm(x-x_{i-1})(x-x_{i+1})\dotsm(x-x_m)}
	{(x_i-x_0)\dotsm(x_i-x_{i-1})(x_i-x_{i+1})\dotsm(x_i-x_m)},
\end{equation}
则\cref{equation:拉格朗日插值.拉格朗日插值公式2} 又可写为\begin{equation}
	l(x) = \sum_{k=0}^m y_k l_k(x).
\end{equation}

插值多项式不仅存在而且唯一.
\begin{theorem}
设\(x_0,\dotsc,x_m\)是任意\(m+1\)个两两不等的实数,
则函数组\begin{equation*}
	l_0(x), l_1(x), \dotsc, l_m(x)
\end{equation*}是多项式函数空间\(\mathbb{P}_m\)的一组基,
即对任意函数\(P\in\mathbb{P}_m\),
总有\(P(x) = \sum_{k=0}^m P(x_k) l_k(x)\).
\end{theorem}

\subsection{牛顿插值法}
牛顿插值法采用的插值多项式是\begin{equation*}
	P_n(x) = c_0 + c_1 (x-x_0) + c_2 (x-x_0)(x-x_1)
	+ \dotsb + c_n (x-x_0)\dotsm(x-x_{n-1}).
\end{equation*}

\subsection{行列式插值法}
%@see: https://www.bilibili.com/video/BV1HE421F7Ew/
如果已知函数\(f \in C[a,b] \cap D^n(a,b)\),
且\(f(x_i) = y_i\ (i=0,1,2,\dotsc,n)\),
其中\(i \neq j \implies x_i \neq x_j\ (i,j=0,1,2,\dotsc,n)\).
可令\begin{equation*}
	F(x) = \begin{vmatrix}
		f(x) & x^n & x^{n-1} & \dots & x & 1 \\
		f(x_0) & x_0^n & x_0^{n-1} & \dots & x_0 & 1 \\
		f(x_1) & x_1^n & x_1^{n-1} & \dots & x_1 & 1 \\
		f(x_2) & x_2^n & x_2^{n-1} & \dots & x_2 & 1 \\
		\vdots & \vdots & \vdots & & \vdots & \vdots \\
		f(x_n) & x_n^n & x_n^{n-1} & \dots & x_n & 1 \\
	\end{vmatrix}_{n+2}.
\end{equation*}
容易验证\(F(x_i) = 0\ (i=0,1,2,\dotsc,n)\).
求导可得\begin{equation*}
	\dv[m]{x} F(x)
	= \dv[m]{x} \left(
		f(x)~A_{11}
		+ x^n A_{12}
		+ x^{n-1} A_{13}
		+ \dotsb
		+ x~A_{1\,n+1}
		+ A_{1\,n+2}
	\right),
	\quad m \leq n,
\end{equation*}
其中\(A_{1j}\ (j=1,2,\dotsc,n+2)\)是行列式\(F(x)\)的\((1,j)\)元素的代数余子式.
由\hyperref[theorem:微分中值定理.罗尔定理]{罗尔定理}可知\begin{equation*}
	(\exists\xi_{1i}\in(x_i,x_{i+1}))
	[F'(\xi_{1i}) = 0],
	\quad i=0,1,2,\dotsc,n-1.
\end{equation*}
继续运用罗尔定理可知\begin{equation*}
	(\exists\xi_{2i}\in(\xi_{1i},\xi_{1\,i+1}))
	[F''(\xi_{2i}) = 0],
	\quad i=0,1,2,\dotsc,n-2.
\end{equation*}
以此类推,成立\begin{equation*}
	(\exists\xi_n\in(\xi_{n-1\,0},\xi_{n-1\,1}))
	[F^{(n)}(\xi_n) = 0].
\end{equation*}

\section{本章总结}
\subsection*{边缘分布,随机变量的独立性}
%\cref{theorem:多维随机变量及其分布.联合密度、边缘密度的关系}
设\(F(x,y)\)为二维随机变量\((X,Y)\)的二维分布函数,
则\(X\)与\(Y\)的边缘分布函数分别为\begin{gather*}
	F_X(x) = F(x,+\infty)
	\quad(-\infty < x < +\infty), \\
	F_Y(y) = F(+\infty,y)
	\quad(-\infty < x < +\infty).
\end{gather*}

%\cref{definition:多维随机变量及其分布.随机变量的独立性}
设\(\AutoTuple{X}{n}\)是\(n\)维随机变量.
若对任意\(n\)个实数\(\AutoTuple{x}{n}\),
\(n\)个事件\((X_1 \leq x_1),\allowbreak\dotsc,\allowbreak(X_n \leq x_n)\)相互独立,
即有\begin{equation*}
	P(X_1 \leq x_1,\dotsc,X_n \leq x_n)
	= \prod_{i=1}^n P(X_i \leq x_i)
	= P(X_1 \leq x_1) \dotsm P(X_n \leq x_n)
\end{equation*}
或\begin{equation*}
	F(x_1,\dotsc,x_n)
	= \prod_{i=1}^n F_i(x_i)
	= F_1(x_1) \dotsm F_n(x_n),
\end{equation*}
其中\(F\)是\(n\)维随机变量\(\AutoTuple{X}{n}\)的联合分布函数,
而\(F_1,\dotsc,F_n\)分别是\(X_1,\dotsc,X_n\)的边缘分布函数,
则称“\(n\)个随机变量\(\AutoTuple{X}{n}\)相互独立”;
否则称“\(n\)个随机变量\(\AutoTuple{X}{n}\)不相互独立”
或“\(n\)个随机变量\(\AutoTuple{X}{n}\)相依”.

\(n\)个事件两两独立是它们相互独立的必要不充分条件.

%\cref{theorem:多维随机变量及其分布.两个离散型随机变量相互独立的充分必要条件}
设\((X,Y)\)是二维离散型随机变量,有二维概率分布\begin{equation*}
	p_{ij} = P(X=x_i,Y=y_j), \quad i,j=1,2,\dotsc,
\end{equation*}
和边缘分布\begin{gather*}
	p_{i*} = \sum_j p_{ij},
	\quad i=1,2,\dotsc; \\
	p_{*j} = \sum_i p_{ij},
	\quad j=1,2,\dotsc,
\end{gather*}
则随机变量\(X\)与\(Y\)相互独立的充分必要条件是:\begin{equation*}
	p_{ij} = p_{i*} p_{*j}, \quad i,j=1,2,\dotsc.
\end{equation*}

%\cref{theorem:多维随机变量及其分布.两个连续型随机变量相互独立的充分必要条件}
设二维连续型随机变量\((X,Y)\)的二维密度为\(f(x,y)\),
\(X\)与\(Y\)的边缘密度分别为\(f_X(x)\)和\(f_Y(y)\),
则\begin{align*}
	f_X(x) = \int_{-\infty}^{+\infty} f(x,y) \dd{y}, \\
	f_Y(y) = \int_{-\infty}^{+\infty} f(x,y) \dd{x}.
\end{align*}
而\(X\)与\(Y\)相互独立的充分必要条件是:\begin{equation*}
	f(x,y) = f_X(x) f_Y(y).
\end{equation*}在三个密度函数的公共连续点上成立.

\subsection*{联合分布、边缘分布与条件分布的联系}
设二维连续型随机变量\((X,Y)\)有二维密度\(f(x,y)\),
从而\(X\)及\(Y\)有边缘密度\(f_X(x),f_Y(y)\),
则\begin{gather*}
	F_{X \vert Y}(x \vert y)
	= \int_{-\infty}^x \frac{f(u,y)}{f_Y(y)}\dd{u}
	\quad(-\infty < x < +\infty), \\
	F_{Y \vert X}(y \vert x)
	= \int_{-\infty}^y \frac{f(x,v)}{f_X(x)}\dd{v}
	(\quad y \in \mathbb{R}).
\end{gather*}
\(X\)关于\(Y\)的条件密度函数为\begin{equation*}
	%\cref{equation:多维随机变量及其分布.条件密度、联合密度、边缘密度的关系1}
	f_{X \vert Y}(x \vert y)
	= \frac{f(x,y)}{f_Y(y)}.
\end{equation*}
\(Y\)关于\(X\)的条件密度函数为\begin{equation*}
	%\cref{equation:多维随机变量及其分布.条件密度、联合密度、边缘密度的关系2}
	f_{Y \vert X}(y \vert x)
	= \frac{f(x,y)}{f_X(x)}.
\end{equation*}
反过来,可以利用边缘密度函数和条件密度函数计算联合密度函数:\begin{equation*}
	f(x,y) = f_Y(y) \cdot f_{X \vert Y}(x \vert y)
	= f_X(x) \cdot f_{Y \vert X}(y \vert x).
\end{equation*}

\subsection*{分布的可加性}
%\cref{theorem:多维随机变量及其分布.二项分布的可加性1}
设\(X \sim B(n,p)\),
\(Y \sim B(m,p)\),
且\(X\)与\(Y\)相互独立,
则\begin{equation*}
	X+Y \sim B(n+m,p).
\end{equation*}

%\cref{theorem:多维随机变量及其分布.泊松分布的可加性1}
设\(X \sim P(\lambda_1)\),
\(Y \sim P(\lambda_2)\),
且\(X\)与\(Y\)相互独立,
则\begin{equation*}
	X+Y \sim P(\lambda_1 + \lambda_2).
\end{equation*}

%\cref{theorem:正态分布与自然指数分布族.正态分布的可加性1}
设\(X \sim N(\mu_1,\sigma_1^2)\),
\(Y \sim N(\mu_2,\sigma_2^2)\),
且\(X\)与\(Y\)相互独立,
则\begin{equation*}
	X+Y \sim N(\mu_1+\mu_2,\sigma_1^2+\sigma_2^2).
\end{equation*}

%\cref{theorem:正态分布与自然指数分布族.正态分布的可加性2}
设随机变量\(\AutoTuple{X}{n}\)相互独立,
且\begin{equation*}
	X_i \sim N(\mu_i,\sigma_i^2),
	\quad i=1,2,\dotsc,n,
\end{equation*}
且\(C_1,C_2,\dotsc,C_n\)为常数,
则\begin{equation*}
	\sum_{i=1}^n {C_i X_i}
	\sim N\left(
	\sum_{i=1}^n {C_i \mu_i},
	\sum_{i=1}^n {C_i^2 \sigma_i^2}
	\right).
\end{equation*}


\chapter{函数图形的绘制}
\section{平面曲线的渐近线}
%@see: https://mathworld.wolfram.com/Asymptote.html
\subsection{水平渐近线、铅直渐近线}
%@see: 《高等数学(第六版 上册)》 P35
如果当\(x\to\infty\)、\(x\to+\infty\)或\(x\to-\infty\)时,有\(f(x) \to A\),
则称“直线\(y = A\)是
函数\(f\)的图形的\DefineConcept{水平渐近线}(horizontal asymptote line)”.

%@see: 《高等数学(第六版 上册)》 P41
如果当\(x \to x_0\)、\(x \to x_0^+\)或\(x \to x_0^-\)时,有\(f(x) \to \infty\),
则称“直线\(x = x_0\)是
函数\(f\)的图形的\DefineConcept{铅直渐近线}(vertical asymptote line)”.

\begin{example}
%@see: 《高等数学(第六版 上册)》 P42 习题1-4 8.
求函数\(f(x)=\frac4{2-x^2}\)的图形的渐近线.
\begin{solution}
因为当\(x\to\sqrt2\)时,\(2-x^2\to0\),\(f(x)\to\infty\),
所以\(x=\sqrt2\)是函数\(f(x)=\frac4{2-x^2}\)的图形的铅直渐近线.
\end{solution}
\end{example}

\subsection{斜渐近线}
如果直线\(L: y = kx+b\ (k \neq 0)\)满足:
当\(x\to\infty\)、\(x\to+\infty\)或\(x\to-\infty\)时,
曲线\(y = f(x)\)上的动点\(M(x,f(x))\)到直线\(L\)的欧氏距离\(d(M,L)\to0\),
则称“直线\(L\)是
曲线\(y = f(x)\)的\DefineConcept{斜渐近线}”.

显然,\(\vb{\nu}=(1,k,0)\)是直线\(L: y=kx+b\)的一个方向向量.
取直线\(L\)上一点\(M_0(0,b,0)\).
根据\cref{equation:解析几何.点到直线的距离},
点\(M(x,f(x),0)\)到直线\(L\)的距离为\[
	d(M,L)=\frac{\abs{\vec{M_0M}\times\vb{\nu}}}{\abs{\vb{\nu}}}
	=\frac{
		\abs{kx-f(x)+b}
	}{
		\sqrt{1+k^2}
	}.
\]
为了使得\[
	\lim_{x\to\infty} \frac{\abs{kx-f(x)+b}}{\sqrt{1+k^2}} = 0
\]成立,
考虑到\(k\)是常数,必有\(\abs{kx-f(x)+b}\)是\(x\to\infty\)时的无穷小,
从而有\[
	\lim_{x\to\infty} (f(x)-kx-b) = 0,
\]
即\[
	\lim_{x\to\infty} (f(x)-kx) = b.
\]
因为\hyperref[theorem:函数极限.无穷小.有界函数与无穷小的乘积是无穷小]{有界函数与无穷小的乘积是无穷小},
所以\[
	\lim_{x\to\infty} \frac{f(x)-kx}{x}
	= \lim_{x\to\infty} \left( \frac{f(x)}x-k \right)
	= 0,
\]
于是\[
	\lim_{x\to\infty} \frac{f(x)}{x} = k.
\]

于是我们得到以下命题.
\begin{proposition}
直线\(L: y = kx+b\)为曲线\(y = f(x)\)的渐近线的充分必要条件是:\[
	k = \lim_{x\to\infty} \frac{f(x)}{x},
	\quad\text{且}\quad
	b = \lim_{x\to\infty} \left( f(x) - kx \right),
\]
或\[
	k = \lim_{x\to+\infty} \frac{f(x)}{x},
	\quad\text{且}\quad
	b = \lim_{x\to+\infty} \left( f(x) - kx \right),
\]
或\[
	k = \lim_{x\to-\infty} \frac{f(x)}{x},
	\quad\text{且}\quad
	b = \lim_{x\to-\infty} \left( f(x) - kx \right).
\]
\end{proposition}

\begin{remark}
在计算函数\(f\)的图形的渐近线方程时,应该注意函数\(f\)的定义域.
例如,当计算定义在\((0,+\infty)\)上的函数\(f\)的水平渐近线时,
极限过程就不可以是\(x\to\infty\)和\(x\to-\infty\);
当计算定义在\((a,b)\)上的函数\(f\)的铅直渐近线时,
极限过程就不可以是\(x \to a\)和\(x \to a^-\)!
\end{remark}

% \begin{example}
% 求出曲线\(C: y = x \ln\left(e+\frac{1}{x-1}\right)\)的渐近线方程.
% \begin{solution}
% 设直线\(L: y = kx+b\)为曲线\(C\)的渐近线,则\begin{align*}
% 	k &= \lim_{x\to\infty} \frac{x \ln\left(e+\frac{1}{x-1}\right)}{x}
% 	= \lim_{x\to\infty} \ln\left(e+\frac{1}{x-1}\right)
% 	= 1, \\
% 	b &= \lim_{x\to\infty} \left[ x \ln\left(e+\frac{1}{x-1}\right) - kx \right]
% 	= \lim_{x\to\infty} x \left[ \ln\left(e+\frac{1}{x-1}\right) - 1 \right] \\
% 	&= \lim_{x\to\infty} x \ln\left[1+\frac{1}{e(x-1)}\right]
% 	= \lim_{x\to\infty} \frac{x}{e(x-1)}
% 	= \frac1e.
% \end{align*}
% 因此,曲线\(C\)的渐近线方程为\(y = x + \frac1e\).
% \end{solution}
% \end{example}

\begin{example}
%@see: 《1998年全国硕士研究生入学统一考试(数学二)》一填空题/第5题
求曲线\(C: y = x \ln(e+\frac1x)\ (x>0)\)的渐近线方程.
\begin{solution}
因为\begin{align*}
	k &= \lim_{x\to\infty} \frac{y}{x}
	= \lim_{x\to\infty} \ln(e+\frac1x)
	= 1, \\
	b &= \lim_{x\to\infty} (y-x)
	= \lim_{x\to\infty} x\left[ \ln(e+\frac1x) - 1 \right] \\
	&= \lim_{t\to0} \frac{\ln(e+t)-1}{t}
	= \lim_{t\to0} \frac1{e+t}
	= \frac1e,
\end{align*}
所以曲线\(C\)的渐近线方程为\(y = x + \frac1e\).
\end{solution}
\end{example}

我们可以观察到,
求解一条曲线\(y = f(x)\)的斜渐近线的本质,
就是将\(f\)展开成\[
	f(x) = x \left( k + \frac{b}{x} + o\left(\frac1x\right) \right)
\]或\[
	f(x) = k x + b + o(1),
\]
其中\(o(1)\)是当\(x\to\infty\)时的无穷小.
于是,我们可以利用泰勒展式求解斜渐近线.

\begin{example}
%@see: https://www.bilibili.com/video/BV1sV4y1h7Q8/
求曲线\(y = e^{\frac1x} \sqrt{1+x^2}\)的斜渐近线.
\begin{solution}
因为当\(x\to\infty\)时\begin{align*}
	y &= e^{\frac1x} \sqrt{1+x^2} \\
	&= \abs{x} \cdot e^{\frac1x} \sqrt{1+\frac1{x^2}} \\
	&= \abs{x} \cdot \left( 1 + \frac1x + o\left(\frac1x\right) \right)
	\left( 1 + \frac12 \frac1{x^2} + o\left(\frac1{x^2}\right) \right) \\
	&= \abs{x} \cdot \left( 1 + \frac1x + o\left(\frac1x\right) \right),
\end{align*}
所以\(y = \pm(x + 1)\)是\(y = e^{\frac1x} \sqrt{1+x^2}\)的斜渐近线.
\end{solution}
\end{example}
\begin{example}
求曲线\(y = \sqrt{4x^2+x} \ln(2+\frac1x)\)的斜渐近线.
\begin{solution}
因为当\(x\to\infty\)时\begin{align*}
	y &= 2 \abs{x} \cdot \sqrt{1 + \frac1{4x}} \left[ \ln2 + \ln(1+\frac1{2x}) \right] \\
	&= 2 \abs{x} \cdot \left( 1 + \frac12 \frac1{4x} + o\left(\frac1x\right) \right)
	\left( \ln 2 + \frac1{2x} + o\left(\frac1x\right) \right) \\
	&= 2 \abs{x} \cdot \left[ \ln2 + \left( \frac{\ln2}8 + \frac12 \right) \frac1x + o\left(\frac1x\right) \right],
\end{align*}
所以\(y = \pm\left( 2x\ln2 + \frac14 \ln2 + 1 \right)\)
是\(y = \sqrt{4x^2+x} \ln(2+\frac1x)\)的斜渐近线.
\end{solution}
\end{example}
\begin{example}
%@see: 《2020年全国硕士研究生入学统一考试(数学二)》三解答题/第15题
求曲线\(y = \frac{x^{1+x}}{(1+x)^x}\ (x>0)\)的斜渐近线.
\begin{solution}
因为当\(x\to+\infty\)时\begin{align*}
	y &= \frac{x^{1+x}}{(1+x)^x}
	= x \cdot \left( 1 + \frac1x \right)^{-x} \\
	&= x \cdot \exp[-x \ln(1 + \frac1x)] \\
	&= x \cdot \exp[-x \left( \frac1x - \frac12 \frac1{x^2} + o\left(\frac1{x^2}\right) \right)] \\
	&= x \cdot \exp\left( -1 + \frac12 \frac1x + o\left(\frac1x\right) \right) \\
	&= x \cdot e^{-1} \left[ 1 + \frac12 \frac1x + o\left(\frac1x\right) \right],
\end{align*}
所以\(y = \frac{x}{e} + \frac1{2e}\)是\(y = \frac{x^{1+x}}{(1+x)^x}\ (x>0)\)的斜渐近线.
\end{solution}
\end{example}

\begin{example}
%@see: https://www.bilibili.com/video/BV1hQ4y1z7TT
求笛卡尔叶形线\(C: x^3+y^3-3axy=0\)的斜渐近线方程.
\begin{solution}
令\(t \defeq \frac{y}{x}\),
则\(y = tx\),
代入笛卡尔叶形线的方程得\[
	x^3+(tx)^3-3ax \cdot tx = 0,
\]
整理得\[
	x^3(1+t^3)-3atx^2=0,
\]
化简得\[
	x = \frac{3at}{1+t^3},
	\qquad
	y = \frac{3at^2}{1+t^3}.
\]
因为当\(t\to-1\)时\(x\to\infty\),
所以笛卡尔叶形线的斜渐近线的斜率\(k\)和截距\(b\)分别为\begin{gather*}
	k = \lim_{x\to\infty} \frac{y}{x}
	= \lim_{t\to-1} t
	= -1, \\
	b = \lim_{x\to\infty} (y-kx)
	= 3a \lim_{t\to-1} \frac{t}{1-t+t^2}
	= -a.
\end{gather*}
因此笛卡尔叶形线的斜渐近线方程为\(x+y+a=0\).
\end{solution}
\end{example}

\begin{example}
%@see: https://www.bilibili.com/video/BV1VF411D7Td
求参数方程\[
	\left\{ \begin{array}{l}
		x = t \ln t, \\
		y = \frac{\ln t}t
	\end{array} \right.
\]确定的曲线\(C\)的渐近线.
%@Mathematica: ParametricPlot[{t Log[t], Log[t]/t}, {t, .1, 2}]
\begin{solution}
首先注意到参数\(t\)的取值范围是\(t>0\).
对参数\(t\)求导得\[
	\left\{ \begin{array}{l}
		x'(t) = \ln t + 1, \\
		y'(t) = \frac{1-\ln t}{t^2}.
	\end{array} \right.
\]
令\(x'(t) \geq 0\)得\(t \geq e^{-1}\),
可知\(x(t)\)在\((0,e^{-1})\)内单调减少,
在\((e^{-1},+\infty)\)内单调增加.
当\(t = e^{-1}\)时,有\(x(t) = -e^{-1}\),
于是\[
	\lim_{t\to0^+} t \ln t
	= \lim_{t\to0^+} \frac{1/t}{-1/t^2}
	= 0,
	\qquad
	\lim_{t\to+\infty} t \ln t
	= +\infty.
\]
令\(y'(t) \geq 0\)得\(t \leq e\),
可知\(y(t)\)在\((0,e)\)内单调增加,
在\((e,+\infty)\)内单调减少.
当\(t = e\)时,有\(y(t) = e^{-1}\),
于是\[
	\lim_{t\to0^+} \frac{\ln t}{t}
	= -\infty,
	\qquad
	\lim_{t\to+\infty} \frac{\ln t}{t}
	= \lim_{t\to+\infty} \frac1t
	= 0.
\]
综上所述,由该参数方程确定的隐函数\(y(x)\)的
定义域是\((-e^{-1},+\infty)\),
值域是\((-\infty,e^{-1})\).

由于\[
	\lim_{x\to+\infty} \frac{y(x)}{x}
	= \lim_{t\to+\infty} \frac{y(t)}{x(t)}
	= \lim_{t\to+\infty} \frac1{t^2}
	= 0,
	\qquad
	\lim_{x\to+\infty} y(x)
	= \lim_{t\to+\infty} y(t)
	= 0,
\]
所以\(y=0\)是曲线\(C\)的一条水平渐近线.

由于\[
	\lim_{y\to-\infty} \frac{x(y)}{y}
	= \lim_{t\to0^+} \frac{x(t)}{y(t)}
	= \lim_{t\to0^+} t^2
	= 0,
	\qquad
	\lim_{y\to-\infty} x(y)
	= \lim_{t\to0^+} x(t)
	= 0,
\]
所以\(x=0\)是曲线\(C\)的一条铅直渐近线.
\end{solution}
\end{example}

\begin{example}
%@see: https://www.bilibili.com/video/BV1mF411f7Zo
求曲线\(C: x^5+x^2y^3=y^4\)的斜渐近线方程.
\end{example}
\begin{example}
%@see: https://www.bilibili.com/video/BV1mF411f7Zo
求曲线\(C: x^5+y^5=y^4\)的斜渐近线方程.
\end{example}

\begin{example}
%@credit: {4dc9721f-7c76-4700-adc0-75d458e212ba}
求曲线\(C: x^3+y^3=y^2\)的斜渐近线方程.
%TODO
% \begin{solution}
% 移项得\(x^3=y^2-y^3\),
% 开方得\(x=\sqrt[3]{y^2-y^3}\),
% 对\(y\)求导得\[
% 	\dv{x}{y} = \frac13 (y^2-y^3)^{-\frac23} (2y-3y^2).
% \]
% 那么\[
% 	\lim_{y\to\infty} \dv{x}{y}
% 	= \lim_{y\to\infty} \frac{2y-3y^2}{3(y^2-y^3)^{\frac23}}
% 	= \lim_{y\to\infty} \frac{(2-3y)y}{3y^2(1/y-1)^{\frac23}}
% 	= \lim_{y\to\infty} \frac{2-3y}{3y(1/y-1)^{\frac23}}
% \]
% \end{solution}
\end{example}

\section{函数的单调性与曲线的凹凸性}
\subsection{函数单调性的判定法}
\begin{theorem}[函数的单调性]\label{theorem:图形绘制.单调性.利用导数判定函数单调性}
%@see: 《数学分析(第二版 上册)》(陈纪修) P172 定理5.1.5(一阶导数与单调性的关系)
%@see: 《高等数学(第六版 上册)》 P146 定理1
设函数\(f\colon X\to\mathbb{R}\)在区间\(X\)上可导.
\begin{itemize}
	\item 函数\(f\)在\(X\)上单调增加的充分必要条件是:
	对于任意\(x \in X\)有\(f'(x) \geq 0\).

	\item 如果对于任意\(x \in X\)有\(f'(x) > 0\),
	则\(f\)在\(X\)上严格单调增加.

	\item 函数\(f\)在\(X\)上单调减少的充分必要条件是:
	对于任意\(x \in X\)有\(f'(x) \leq 0\).

	\item 如果对于任意\(x \in X\)有\(f'(x) < 0\),
	则\(f\)在\(X\)上严格单调减少.
\end{itemize}
%TODO proof
\end{theorem}
\begin{remark}
%@see: 《数学分析(第二版 上册)》(陈纪修) P174
%@see: 《数学分析(第二版 上册)》(陈纪修) P182 习题 11.
可以证明:若将\cref{theorem:图形绘制.单调性.利用导数判定函数单调性} 的条件
“对于任意\(x \in X\)有\(f'(x) > 0\)”
减弱为“在\(X\)中除了有限个点外,都有\(f'(x)>0\)”,
结论“\(f\)在\(X\)上严格单调增加”依然成立.
因此“对于任意\(x \in X\)有\(f'(x) > 0\)”只是“\(f\)在\(X\)上严格单调增加”的充分不必要条件.
\end{remark}
\begin{example}\label{example:图形绘制.单调性.函数在定义域上不单调}
%@see: 《高等数学(第六版 上册)》 P146 例2
讨论函数\(f(x) = e^x - x - 1\)的单调性.
\begin{solution}
函数\(f\)的定义域为\(\dom f = (-\infty,+\infty)\).
因为在\((-\infty,0)\)内\(f'(x)<0\),
所以\(f\)在\((-\infty,0)\)上单调减少.
因为在\((0,+\infty)\)内\(f'(x)>0\),
所以\(f\)在\((0,+\infty)\)上单调增加.
\end{solution}
\end{example}
\begin{example}\label{example:图形绘制.单调性.函数在个别点不可导}
%@see: 《高等数学(第六版 上册)》 P147 例3
讨论函数\(f(x) = \sqrt[3]{x^2}\)的单调性.
\begin{solution}
函数\(f\)的定义域为\(\dom f = (-\infty,+\infty)\).
当\(x\neq0\)时,它的导数为\[
	f'(x) = \frac23 x^{-\frac13}.
\]
当\(x=0\)时,它的导数不存在.
在\((-\infty,0)\)内,\(f'(x)<0\),
因此\(f\)在\((-\infty,0)\)上单调减少.
在\((0,+\infty)\)内,\(f'(x)>0\),
因此\(f\)在\((0,+\infty)\)上单调增加.
\end{solution}
\end{example}
\begin{remark}
%@see: 《高等数学(第六版 上册)》 P147
从\cref{example:图形绘制.单调性.函数在定义域上不单调} 看出,
有些函数在它的定义区间上不是单调的.
从\cref{example:图形绘制.单调性.函数在个别点不可导} 看出,
有些函数在有限个点不可导.
因此,我们可以得出如下结论:
如果函数在定义区间上连续,
除去有限个导数不存在的点外导数存在且连续,
那么只要用方程\(f'(x) = 0\)的根
及\(f'(x)\)不存在的点
来划分函数\(f\)的定义区间,
就能保证\(f'\)在各个部分区间内保持固定符号,
因而函数\(f\)在每个部分区间上单调.
\end{remark}
\begin{example}\label{example:图形绘制.单调性.函数在个别点导数为零}
%@see: 《高等数学(第六版 上册)》 P148 例5
讨论函数\(f(x) = x^3\)的单调性.
\begin{solution}
函数\(f\)的定义域为\(\dom f = (-\infty,+\infty)\).
它的导数为\(f'(x) = 3x^2\).
显然,除了点\(x=0\)使得\(y'(x) = 0\)以外,在其余各点均有\(f'(x)>0\).
因此,函数\(f\)在\((-\infty,0]\)及\([0,\infty)\)上都是单调增加的,
从而它在整个定义域\((-\infty,+\infty)\)内是单调增加的.
\end{solution}
\end{example}
\begin{remark}
%@see: 《高等数学(第六版 上册)》 P148
从\cref{example:图形绘制.单调性.函数在个别点导数为零} 可以看出:
如果\(f'\)在某区间内的有限个点处为零,
在其余各点处均为正(或负)时,
那么\(f\)在该区间上仍旧是单调增加(或单调减少)的.
\end{remark}

\begin{example}
%@see: 《高等数学(第六版 上册)》 P148 例6
证明:当\(x > 1\)时,\(2 \sqrt{x} > 3 - \frac{1}{x}\).
\begin{proof}
令\(f(x) = 2 \sqrt{x} - \left(3 - \frac{1}{x}\right)\),
则\[
	f'(x) = \frac{1}{\sqrt{x}} - \frac{1}{x^2}
	= \frac{1}{x^2} (x \sqrt{x} - 1).
\]

\(f\)在\([1,+\infty)\)上连续,在\((1,+\infty)\)内\(f'(x) > 0\),
因此在\([1,+\infty)\)上\(f\)单调增加,
从而当\(x > 1\)时,\(f(x) > f(1) = 0\),
即\(2 \sqrt{x} - \left(3 - \frac{1}{x}\right) > 0\),
\(2 \sqrt{x} > 3 - \frac{1}{x}\).
\end{proof}
\end{example}

\begin{example}
%@see: 《数学分析(第二版 上册)》(陈纪修) P182 习题 12. (1)
证明\DefineConcept{若尔当不等式}:
当\(0<x<\frac\pi2\)时,
有\begin{equation}\label{equation:微分中值定理.若尔当不等式}
	\frac2\pi < \frac{\sin x}{x} < 1.
\end{equation}
\begin{proof}
设\(f(x) = \frac{\sin x}{x}\ (0<x<\frac\pi2)\),
那么\(f'(x) = \frac{x \cos x - \sin x}{x^2}\).
又设\(g(x) = x \cos x - \sin x\),
那么\[
	g'(x) = \cos x - x \sin x - \cos x = -x \sin x < 0\ (0<x<\frac\pi2),
\]
说明\(g\)是\((0,\frac\pi2)\)上的单调减少函数.
又因为\(g(0) = 0\),
从而\(g(x) < 0\ (0<x<\frac\pi2)\),
所以\(f'(x) < 0\),
\(f\)也是\((0,\frac\pi2)\)上的单调减少函数.
应用洛必达法则,得\[
	\lim_{x\to0^+} f(x)
	= \lim_{x\to0^+} \cos x
	= 1.
\]
再因为\[
	\lim_{x\to\frac\pi2^-} f(x)
	= \frac{\sin(\pi/2)}{\pi/2}
	= \frac2\pi,
\]
所以\[
	\frac2\pi < \frac{\sin x}{x} < 1.
	\qedhere
\]
\end{proof}
\end{example}

\begin{example}
证明:
当\(0<x<\frac\pi2\)时,
有\begin{equation}\label{equation:单调性.正切不等式}
	\tan x > x.
\end{equation}
\begin{proof}
设\(f(x) = \tan x - x\ (0<x<\frac\pi2)\).
求导得\(f'(x) = \sec^2 x - 1 = \tan^2 x\).
显然\(f'(x) > 0\ (0<x<\frac\pi2)\),
\(f\)是\((0,\frac\pi2)\)上的严格单调增加函数.
又因为\(f(0) = \tan0 - 0 = 0\),
所以\(f(x) = \tan x - x > 0\ (0<x<\frac\pi2)\),
即\(\tan x > x\ (0<x<\frac\pi2)\).
\end{proof}
\end{example}

\begin{example}
%@see: 《数学分析(第二版 上册)》(陈纪修) P182 习题 12. (5)
证明不等式\[
	\frac1{2^{p-1}}
	\leq x^p + (1-x)^p
	\leq 1
	\quad(0 \leq x \leq 1, p>1).
\]
%TODO proof
\end{example}
\begin{example}
%@see: 《数学分析(第二版 上册)》(陈纪修) P183 习题 12. (6)
证明不等式\[
	\frac{\tan x}{x} > \frac{x}{\sin x}
	\quad(0 < x < \pi/2).
\]
%TODO proof
\end{example}

\begin{example}
%@see: https://www.bilibili.com/video/BV19YpbepEGF
设数列\(\{x_n\}\)有递推公式\(x_{n+1} = f(x_n)\ (n=1,2,\dotsc)\).
证明:\begin{itemize}
	\item 如果\(f'(x) > 0\),则\(\{x_n\}\)是单调的.
	\item 如果\(f'(x) < 0\),则\(\{x_n\}\)不是单调的.
\end{itemize}
\begin{proof}
假设\(f'(x) > 0\)且\(x_2 \geq x_1\),
那么\(x_3 = f(x_2) \geq f(x_1) = x_2\),
利用数学归纳法可证\(x_{n+1} \geq x_n\)对\(n=1,2,\dotsc\)成立.
因此,当\(f'(x) > 0\)时,\(\{x_n\}\)是单调的.

假设\(f'(x) < 0\)且\(x_2 \geq x_1\),
那么\(x_3 = f(x_2) \leq f(x_1) = x_2\),
显然,当\(f'(x) < 0\)时,\(\{x_n\}\)不是单调的.
\end{proof}
\end{example}

\begin{example}
%@see: 《数学分析(第二版 上册)》(陈纪修) P177 例5.1.5
判别\(e^\pi\)与\(\pi^e\)的大小关系.
\begin{solution}
对于正实数\(a,b\),有\begin{align*}
	a^b > b^a
	&\iff
	b \ln a = \ln a^b > \ln b^a = a \ln b \\
	&\iff
	\frac{\ln a}{a} > \frac{\ln b}{b}.
\end{align*}
下面考察函数\(f(x) = \frac{\ln x}{x}\)的单调性.
求导得\[
	f'(x) = \frac{1 - \ln x}{x^2}.
\]
令\(f'(x) > 0\)得\(0 < x < e\).
由\cref{theorem:图形绘制.单调性.利用导数判定函数单调性} 可知,
函数\(f\)在\((0,e)\)上严格单调增加,在\((e,+\infty)\)上严格单调减少.
由于\(\pi > e\),所以\[
	\frac{\ln e}{e} > \frac{\ln \pi}{\pi},
\]
那么\(e^\pi > \pi^e\).
\end{solution}
\end{example}

\subsection{曲线的凹凸性与拐点}
\begin{definition}\label{definition:函数图像的绘制.凹凸性的定义}
%@see: 《数学分析(第7版 第一卷)》(卓里奇) P202 定义1
%@see: 《数学分析(第7版 第一卷)》(卓里奇) P203 定义2
%@see: 《数学分析(第二版 上册)》(陈纪修) P173 定义5.1.2
设函数\(f\colon D\to\mathbb{R}\),其中\(D=(a,b)\subset\mathbb{R}\).

若\[
	(\forall x_1,x_2 \in D)
	(\forall \lambda \in [0,1])
	[
		f(\lambda x_1 + (1-\lambda) x_2)
		\leq
		\lambda f(x_1) + (1-\lambda) f(x_2)
	],
\]
则称“\(f\)是\(D\)上的\DefineConcept{凹函数}(convex function)”.

若\[
	(\forall x_1,x_2 \in D)
	(\forall \lambda \in [0,1])
	[
		f(\lambda x_1 + (1-\lambda) x_2)
		\geq
		\lambda f(x_1) + (1-\lambda) f(x_2)
	],
\]
则称“\(f\)是\(D\)上的\DefineConcept{凸函数}(concave function)
\footnote{有的书把凹函数称为\DefineConcept{下凸函数},
把凸函数称为\DefineConcept{上凸函数}.
% 例如:
% 《数学分析(第二版 上册)》(陈纪修)
有的书反过来把凹函数称为凸函数,
把凸函数称为凹函数.
% 例如:
% 《数学分析(第7版 第一卷)》(卓里奇)
}”.

若\[
	(\forall x_1,x_2 \in D)
	[
		x_1 \neq x_2
		\implies
		(\forall \lambda \in (0,1))
		[
			f(\lambda x_1 + (1-\lambda) x_2)
			<
			\lambda f(x_1) + (1-\lambda) f(x_2)
		]
	],
\]
则称“\(f\)是\(D\)上的\DefineConcept{严格凹函数}%
(strictly convex function)”.

若\[
	(\forall x_1,x_2 \in D)
	[
		x_1 \neq x_2
		\implies
		(\forall \lambda \in (0,1))
		[
			f(\lambda x_1 + (1-\lambda) x_2)
			>
			\lambda f(x_1) + (1-\lambda) f(x_2)
		]
	],
\]
则称“\(f\)是\(D\)上的\DefineConcept{严格凸函数}%
(strictly concave function)”.
%@see: https://mathworld.wolfram.com/ConvexFunction.html
%@see: https://mathworld.wolfram.com/ConcaveFunction.html
%@see: https://doi.org/10.1007/978-3-030-41804-5
\end{definition}

\begin{definition}
设函数\(f\colon D\to\mathbb{R}\).

若\[
	(\exists\alpha>0)
	[\text{函数\(g(x)=f(x)-\alpha\abs{x}^2\)是凹函数}],
\]
则称“\(f(x)\)是\(D\)上的\DefineConcept{强凹函数}
(strongly convex function)”.

若\[
	(\exists\alpha>0)
	[\text{函数\(g(x)=f(x)-\alpha\abs{x}^2\)是凸函数}],
\]
则称“\(f(x)\)是\(D\)上的\DefineConcept{强凸函数}
(strongly concave function)”.
%@see: https://www.princeton.edu/~aaa/Public/Teaching/ORF523/S16/ORF523_S16_Lec7_gh.pdf
\end{definition}

\begin{proposition}
强凹(凸)函数必定严格凹(凸).
\end{proposition}

\begin{proposition}[延森不等式]
%@see: 《数学分析(第7版 第一卷)》(卓里奇) P207 命题7(延森不等式)
%@see: 《数学分析(第二版 上册)》(陈纪修) P176 定理5.1.8
%@see: 《数学分析(第二版 上册)》(陈纪修) P184 习题 24.(Jensen不等式)
设函数\(f\colon(a,b)\to\mathbb{R}\),
\(\AutoTuple{x}{n}\in(a,b)\),
\(\AutoTuple{k}{n}\in(0,1)\)
且\(\AutoTuple{k}{n}[+]=1\).
\begin{itemize}
	\item 如果\(f\)是凸函数,
	则\[
		%@see: 《数学分析(第7版 第一卷)》(卓里奇) P207 (15)
		f(k_1 x_1 + \dotsb + k_n x_n)
		\geq
		k_1 f(x_1) + \dotsb + k_n f(x_n).
	\]
	\item 如果\(f\)是凹函数,
	则\[
		%@see: 《数学分析(第7版 第一卷)》(卓里奇) P207 (14)
		f(k_1 x_1 + \dotsb + k_n x_n)
		\leq
		k_1 f(x_1) + \dotsb + k_n f(x_n).
	\]
\end{itemize}
如果\(\AutoTuple{k}{n}\)均不为零,
那么上述两条不等式的取等条件就是\(x_1=x_2=\dotsb=x_n\).
\begin{proof}
这里只证“\(f\)是凹函数”的情形.

假设\(f\)是凹函数.
当\(n=2\)时,条件、结论与\cref{definition:函数图像的绘制.凹凸性的定义} 相同,自然成立.

利用数学归纳法.
假设当\(n=m-1\ (m\geq3)\)时结论成立.
当\(n=m\)时,
记\(p = k_2 + \dotsb + k_n\),
则\[
	\frac{k_2}{p} + \dotsb + \frac{k_n}{p} = 1,
\]
于是\begin{align*}
	f(k_1 x_1 + \dotsb + k_n x_n)
	&= f\left( k_1 x_1 + p \left( \frac{k_2}{p} x_2 + \dotsb + \frac{k_n}{p} x_n \right) \right) \\
	&\leq k_1 f(x_1) + p f\left( \frac{k_2}{p} x_2 + \dotsb + \frac{k_n}{p} x_n \right),
\end{align*}
这里\(k_1 + p = 1\)
且\(\frac{k_2}{p} x_2 + \dotsb + \frac{k_n}{p} x_n \in (a,b)\).
再由归纳假设可知\[
	f\left( \frac{k_2}{p} x_2 + \dotsb + \frac{k_n}{p} x_n \right)
	\leq \frac{k_2}{p} f(x_2) + \dotsb + \frac{k_n}{p} f(x_n),
\]
所以\begin{align*}
	f(k_1 x_1 + \dotsb + k_n x_n)
	&\leq k_1 f(x_1) + p f\left( \frac{k_2}{p} x_2 + \dotsb + \frac{k_n}{p} x_n \right) \\
	&\leq k_1 f(x_1) + k_2 f(x_2) + \dotsb + k_n f(x_n).
	\qedhere
\end{align*}
\end{proof}
\end{proposition}

\begin{theorem}[曲线凹凸的判定]\label{theorem:微分中值定理.曲线凹凸的判定}
%@see: 《数学分析(第二版 上册)》(陈纪修) P173 定理5.1.6(二阶导数与凹凸性的关系)
设\(f\colon X\to\mathbb{R}\)在区间\(X\)内具有二阶导数.
\begin{itemize}
	\item 若在\(X\)内\(f''(x)>0\),
	则\(f\)是\(X\)上的严格凹函数.
	\item 若在\(X\)内\(f''(x)<0\),
	则\(f\)是\(X\)上的严格凸函数.
\end{itemize}
\begin{proof}
在情形1,设\(x_1\)和\(x_2\)为\(X\)内任意两点,且\(x_1 < x_2\),
记\(\frac{x_1 + x_2}{2} = x_0\),
并记\(x_2 - x_0 = x_0 - x_1 = h\),
则\(x_1 = x_0 - h\),\(x_2 = x_0 + h\),
由拉格朗日中值公式,
得\begin{gather*}
	f(x_0 + h) - f(x_0) = f'(x_0 + \theta_1 h) h, \\
	f(x_0) - f(x_0 - h) = f'(x_0 - \theta_2 h) h,
\end{gather*}
其中\(0 < \theta_1 < 1\),\(0 < \theta_2 < 1\).
两式相减,即得\[
	f(x_0 + h) + f(x_0 - h) - 2 f(x_0)
	= [ f'(x_0 + \theta_1 h) - f'(x_0 - \theta_2 h) ] h.
\]
对\(f'(x)\)在区间\([x_0 - \theta_2 h,x_0 + \theta_1 h]\)上再利用拉格朗日中值公式,
得\[
	[ f'(x_0 + \theta_1 h) - f'(x_0 - \theta_2 h) ] h
	= f''(\xi) (\theta_1 + \theta_2) h^2,
\]
其中\(x_0 - \theta_2 h < \xi < x_0 + \theta_1 h\).
按情形1的假设,\(f''(\xi) > 0\),
故有\[
	f(x_0 + h) + f(x_0 - h) - 2 f(x_0) > 0,
\]
即\[
	\frac{f(x_0 + h) + f(x_0 - h)}{2} > f(x_0),
\]
亦即\[
	\frac{f(x_1) + f(x_2)}{2} > f\left(\frac{x_1 + x_2}{2}\right),
\]
所以\(f(x)\)在\(X\)上的图形是凹的.

类似地可证情形2.
\end{proof}
\end{theorem}
\begin{remark}
%@see: 《数学分析(第二版 上册)》(陈纪修) P174
可以证明:若将\cref{theorem:微分中值定理.曲线凹凸的判定} 的条件
“在\(X\)内\(f''(x)>0\)”
减弱为“在区间\(X\)中除了有限个点外,都有\(f''(x)>0\)”,
结论“\(f\)是\(X\)上的严格凹函数”依然成立.
因此“在\(X\)内\(f''(x)>0\)”只是“\(f\)是\(X\)上的严格凹函数”的充分不必要条件.
\end{remark}

\begin{definition}\label{definition:单调性和凹凸性.拐点}
%@see: 《数学分析(第二版 上册)》(陈纪修) P175
如果函数\(f\)的凹凸性在点\(x_0\)左邻域、右邻域相反,
那么就称“点\(x_0\)是函数\(f\)的\DefineConcept{拐点}(inflection point)”
%@see: https://mathworld.wolfram.com/InflectionPoint.html
“点\((x_0,f(x_0))\)是曲线\(y=f(x)\)的拐点”.
\end{definition}

\begin{example}
函数\(f(x) = x^3\)以\(x=0\)为拐点.
函数\(g(x) = \abs{x(x-1)}\)以\(x=0\)和\(x=1\)为拐点.
\end{example}

\begin{theorem}\label{theorem:单调性和凹凸性.拐点的判定}
%@see: 《数学分析(第二版 上册)》(陈纪修) P175 定理5.1.7
设\(f\colon X\to\mathbb{R}\)在区间\(X\)上连续,
存在\(\delta>0\)使得\((x_0-\delta,x_0+\delta) \subseteq X\).
\begin{itemize}
	\item 设\(f\)在\((x_0-\delta,x_0)\cup(x_0,x_0+\delta)\)上二阶可导.
	\begin{itemize}
		\item 若\(f''\)在\((x_0-\delta,x_0)\)与\((x_0,x_0+\delta)\)上的符号相反,
		则点\(x_0\)是函数\(f\)的拐点.
		\item 若\(f''\)在\((x_0-\delta,x_0)\)与\((x_0,x_0+\delta)\)上的符号相同,
		则点\(x_0\)不是函数\(f\)的拐点.
	\end{itemize}

	\item 设\(f\)在\((x_0-\delta,x_0+\delta)\)上二阶可导.
	若点\(x_0\)是函数\(f\)的拐点,
	则\(f''(x_0) = 0\).
\end{itemize}
\begin{proof}
假设\(f\)在\((x_0-\delta,x_0+\delta)\)上二阶可导,
且点\(x_0\)是函数\(f\)的拐点.
不妨设\(f\)是\((x_0-\delta,x_0)\)上的凹函数,\((x_0,x_0+\delta)\)上的凸函数.
由\cref{theorem:微分中值定理.曲线凹凸的判定} 可知
在\((x_0-\delta,x_0)\)上\(f''(x)\geq0\),
在\((x_0,x_0+\delta)\)上\(f''(x)\leq0\);
换言之,\(f'\)在\((x_0-\delta,x_0)\)上单调增加,在\((x_0,x_0+\delta)\)上单调减少;
因此点\(x_0\)是\(f'\)的极大值点.
因为\(f''\)在点\(x_0\)有定义,
那么由\hyperref[theorem:微分中值定理.费马引理]{费马引理}可知\(f''(x_0)=0\).
\end{proof}
\end{theorem}
\begin{remark}
“\(f''(x_0) = 0\)”是二阶可导函数曲线拐点的必要不充分条件.
例如,曲线\(y = x^4\)上的点\((0,0)\)就满足条件\(f''(0) = 0\),但它不是拐点.
又例如,曲线\(y = x^{1/3}\)在点\(x=0\)不可导,但是点\((0,0)\)是它的拐点.
\end{remark}

由\cref{theorem:微分中值定理.曲线凹凸的判定} 可知,由\(f''(x)\)的符号可以判定曲线的凹凸性.
因此,如果\(f''(x)\)在\(x_0\)的左右两侧邻近异号,那么点\((x_0,f(x_0))\)就是曲线的一个拐点,
所以,要寻找拐点,只要找出\(f''(x)\)符号发生变化的分界点即可.
如果\(f(x)\)在区间\((a,b)\)内具有二阶连续导数,那么在这样的分界点处必然有\(f''(x)=0\);
除此以外,\(f(x)\)的二阶导数不存在的点,也有可能是\(f''(x)\)的符号发生变化的分界点.

{\color{red}
综合以上分析,我们可以按下列步骤来判定区间\(I\)上的连续曲线\(y=f(x)\)的拐点:
\begin{enumerate}
	\item 求\(f''(x)\);

	\item 令\(f''(x) = 0\),解出这个方程在区间\(I\)内的实根,
	并求出在区间\(I\)内\(f''(x)\)不存在的点;

	\item 对于上一步中求出的每一个实根或二阶导数不存在的点\(x_0\),
	检查\(f''(x)\)在\(x_0\)左右两侧邻近的符号,
	那么当两侧的符号相反时,点\((x_0,f(x_0))\)就是拐点;
	当两侧的符号相同时,点\((x_0,f(x_0))\)不是拐点.
\end{enumerate}}

\begin{example}
证明:当\(x>0\),\(y>0\),\(x \neq y\),且\(n>1\)时,
成立\begin{equation}\label{equation:微分中值定理.平均数的比较1}
	\frac{1}{2} (x^n+y^n) > \left(\frac{x+y}{2}\right)^n.
\end{equation}
\begin{proof}
设\(f(x) = x^n\),那么\(f'(x) = n x^{n-1}\),\(f''(x) = n(n-1) x^{n-2}\).
由指数函数的性质可知,
当\(x > 0\)时,\(\forall \mu \in \mathbb{R}\)都有\(x^{\mu} > 0\)成立,
故当\(n > 1\)时,\(n(n-1)>0\),\(f''(x) = n(n-1) x^{n-1} > 0\)成立,
即\(f(x)\)在\((0,+\infty)\)上是凹函数,
那么\(\forall x,y>0\),
只要\(x \neq y\),
就有\[
	\frac{f(x)+f(y)}{2} = \frac{x^n+y^n}{2}
	> \left(\frac{x+y}{2}\right)^n = f\left(\frac{x+y}{2}\right)
\]成立.
\end{proof}
\end{example}

\begin{example}
证明:当\(x \neq y\)时,有\begin{equation}
	\frac{e^x + e^y}{2} > e^{\frac{x+y}{2}}.
\end{equation}
\begin{proof}
设\(f(x) = e^x\),那么\(f'(x) = f''(x) = e^x > 0\ (x\in\mathbb{R})\),可知\(f(x)\)是凹函数.
因此\[
	f\left(\frac{x+y}{2}\right) = e^{\frac{x+y}{2}}
	< \frac{e^x+e^y}{2} = \frac{f(x)+f(y)}{2}.
	\qedhere
\]
\end{proof}
\end{example}

\begin{example}[杨格不等式]\label{example:函数图像的绘制.杨格不等式}
%@see: 《数学分析(第二版 上册)》(陈纪修) P179 例5.1.8
设\(a,b\geq0\),正数\(p,q\)满足\(\frac1p+\frac1q=1\).
证明:\[
	a b \leq \frac1p a^p + \frac1q b^q.
\]
当且仅当\(a^p = b^q\)时上式取“\(=\)”号.
\begin{proof}
当\(a=0\)或\(b=0\)时,上式显然成立.

下面考虑\(a,b>0\)的情形.
由于函数\(f(x) = \ln x\ (x>0)\)的一阶导数\[
	f'(x) = \frac1x > 0,
\]
二阶导数\[
	f''(x) = -\frac1{x^2} < 0,
\]
所以\(f\)是\((0,+\infty)\)上的严格单调增加的严格凸函数.
那么由\cref{definition:函数图像的绘制.凹凸性的定义} 得\[
	\ln(ab) = \frac1p \ln a^p + \frac1q \ln b^q
	\leq \ln\left( \frac1p a^p + \frac1q b^q \right),
\]
因此\[
	ab \leq \frac1p a^p + \frac1q b^q.
	\qedhere
\]
\end{proof}
\end{example}
\begin{remark}
\cref{example:函数图像的绘制.杨格不等式}
利用微分学重新证明了\hyperref[theorem:不等式.杨格不等式]{杨格不等式}.
\end{remark}
\begin{remark}
类似地可以证明对于任意正数\(x_1,\dotsc,x_n\),成立\[
	\frac{\ln x_1 + \dotsb + \ln x_n}{n}
	\leq \ln\left( \frac{x_1 + \dotsb + x_n}{n} \right),
\]
从而有\[
	\sqrt[n]{x_1 \dotsm x_n}
	\leq \frac{x_1 + \dotsb + x_n}{n}.
\]
\end{remark}

\begin{example}[赫尔德不等式]\label{example:不等式.赫尔德不等式}
%@see: 《数学分析教程 (第3版 上册)》(史济怀) P306 例8
设\(\AutoTuple{x}{n},\AutoTuple{y}{n}\geq0\),\(p,q>1\),
且满足\(\frac{1}{p}+\frac{1}{q}=1\),
则
\def\s{\sum_{i=1}^n}%
\def\sp#1#2#3{\left( \s #1^#2 \right)^{#3/#2}}%
\begin{equation}
	\s x_i y_i
	\leq
	\sp{x_i}{p}{1} \sp{y_i}{q}{1},
\end{equation}
当且仅当存在常数\(\lambda\)使得\(a_i^p = \lambda b_i^q\ (i=1,2,\dotsc,n)\)时取“\(=\)”号.
\begin{proof}
令\[
	a_j = x_j \sp{x_i}{p}{-1}, \qquad
	b_j = y_j \sp{y_i}{q}{-1}.
\]
那么根据\cref{example:函数图像的绘制.杨格不等式} 得到\[
	\s a_i b_i \leq \s \left( \frac{a_i^p}{p} + \frac{b_i}{q} \right)
	\leq \frac{1}{p} + \frac{1}{q} = 1.
	\qedhere
\]
\end{proof}
\end{example}

\begin{example}[闵可夫斯基不等式]\label{example:不等式.闵可夫斯基不等式}
设\(\AutoTuple{x}{n},\AutoTuple{y}{n}\in\mathbb{R}\),\(p\geq1\),则
\def\s{\sum_{i=1}^n}%
\def\sumonly#1{\s \abs{#1}^p}%
\newcommand\sumpower[2][1]{\left( \sumonly{#2} \right)^{\frac{#1}{p}}}%
\begin{equation}
	\sumpower{x_i+y_i} \leq \sumpower{x_i} + \sumpower{y_i}.
\end{equation}
\begin{proof}
利用\cref{example:不等式.赫尔德不等式} 就有\begin{align*}
	&\hspace{-20pt}\sumonly{x_i+y_i}
	= \s \abs{x_i+y_i} \abs{x_i+y_i}^{p-1}
	\leq \s (\abs{x_i}+\abs{y_i}) \abs{x_i+y_i}^{p-1} \\
	&\leq \sumpower{x_i} \sumpower[p-1]{x_i+y_i}
	+ \sumpower{y_i} \sumpower[p-1]{x_i+y_i}.
\end{align*}
整理即得欲证不等式.
\end{proof}
\end{example}

\section{函数的极值与最值}
\subsection{函数的极值及其求法}
\begin{definition}\label{definition:函数的极值与最值.驻点}
设函数\(f\colon[a,b]\to\mathbb{R}\),
\(f \in C[a,b] \cap D(a,b)\).
如果\(f'(x_0)=0\),
则称“\((x_0,f(x_0))\)是\(f\)的\DefineConcept{驻点}(stationary point)”.
%@see: https://mathworld.wolfram.com/StationaryPoint.html
\end{definition}

\begin{figure}[thb]
	\centering
	\begin{tikzpicture}[scale=.7]
		\begin{axis}[
			xmin=0,xmax=2,
			ymin=0,ymax=2,
			axis lines=middle,
			xlabel=$x$,
			ylabel=$y$,
			enlarge x limits=0.1,
			enlarge y limits=0.1,
			x label style={at={(ticklabel* cs:1.00)}, inner sep=5pt, anchor=south},
			y label style={at={(ticklabel* cs:1.00)}, inner sep=2pt, anchor=west},
		]
			\addplot[color=blue,samples=50,smooth,domain=0:2]{(x-1)^2+1};
			\fill(1,1)circle(2pt)node[below]{$f(x)=(x-1)^2+1$};
		\end{axis}
	\end{tikzpicture}~\begin{tikzpicture}[scale=.7]
		\begin{axis}[
			xmin=0,xmax=2,
			ymin=0,ymax=2,
			axis lines=middle,
			xlabel=$x$,
			ylabel=$y$,
			enlarge x limits=0.1,
			enlarge y limits=0.1,
			x label style={at={(ticklabel* cs:1.00)}, inner sep=5pt, anchor=south},
			y label style={at={(ticklabel* cs:1.00)}, inner sep=2pt, anchor=west},
		]
			\addplot[color=blue,samples=50,smooth,domain=0:2]{-(x-1)^2+1};
			\fill(1,1)circle(2pt)node[above]{$f(x)=1-(x-1)^2$};
		\end{axis}
	\end{tikzpicture}~\begin{tikzpicture}[scale=.7]
		\begin{axis}[
			xmin=0,xmax=2,
			ymin=0,ymax=2,
			axis lines=middle,
			xlabel=$x$,
			ylabel=$y$,
			enlarge x limits=0.1,
			enlarge y limits=0.1,
			x label style={at={(ticklabel* cs:1.00)}, inner sep=5pt, anchor=south},
			y label style={at={(ticklabel* cs:1.00)}, inner sep=2pt, anchor=west},
		]
			\addplot[color=blue,samples=50,smooth,domain=0:2]{(x-1)^3+1};
			\fill(1,1)circle(2pt)node[below]{$f(x)=(x-1)^3+1$};
		\end{axis}
	\end{tikzpicture}
	\caption{驻点}
\end{figure}

\begin{definition}\label{definition:函数的极值与最值.极值点}
%@see: 《高等数学(第六版 上册)》 P154 定义
设函数\(f\colon D\to\mathbb{R}\),点\(x_0 \in D\).
\begin{itemize}
	\item 如果\begin{equation*}
		(\exists\delta>0)
		(\forall x \in D)
		\left[
			0<\abs{x-x_0}<\delta
			\implies
			f(x)<f(x_0)
		\right],
	\end{equation*}
	那么称
	“\(x_0\)是函数\(f\)的一个\DefineConcept{极大值点}”
	“\(f(x_0)\)是函数\(f\)的一个\DefineConcept{极大值}(local maximum)”.

	\item 如果\begin{equation*}
		(\exists\delta>0)
		(\forall x \in D)
		\left[
			0<\abs{x-x_0}<\delta
			\implies
			f(x)>f(x_0)
		\right],
	\end{equation*}
	那么称
	“\(x_0\)是函数\(f\)的一个\DefineConcept{极小值点}”
	“\(f(x_0)\)是函数\(f\)的一个\DefineConcept{极小值}(local minimum)”.
\end{itemize}

函数的极大值点、极小值点,统称为\DefineConcept{极值点}.
函数的极大值、极小值,统称为函数的\DefineConcept{极值}(local extremum).
%@see: https://mathworld.wolfram.com/LocalExtremum.html
\end{definition}

\begin{theorem}[函数存在极值的必要条件]\label{theorem:微分中值定理.函数存在极值的必要条件}
设函数\(f\)在点\(x_0\)可导,且在\(x_0\)取得极值,
那么\begin{equation*}
	f'(x_0) = 0.
\end{equation*}
\end{theorem}
由此可见,可导函数\(f\)的极值点必定是它的驻点,然而函数的驻点不一定是极值点.
此外,函数在它的导数不存在的点处也可能取得极值.

\begin{theorem}[函数存在极值的第一充分条件]\label{theorem:微分中值定理.函数存在极值的第一充分条件}
设函数\(f\)在\(x_0\)连续,且在\(x_0\)的某去心邻域内可导.
\begin{itemize}
	\item 若\(x \in (x_0-\delta,x_0)\)时\(f'(x)>0\),
	而\(x \in (x_0,x_0+\delta)\)时\(f'(x)<0\),
	则\(f(x)\)在\(x_0\)处取得极大值.

	\item 若\(x \in (x_0-\delta,x_0)\)时\(f'(x)<0\),
	而\(x \in (x_0,x_0+\delta)\)时\(f'(x)>0\),
	则\(f(x)\)在\(x_0\)处取得极小值.

	\item 若\(x \in \mathring{U}(x_0,\,\delta)\)时,
	\(f'(x)\)的符号保持不变,
	则\(f(x)\)在\(x_0\)处没有极值.
\end{itemize}
\begin{proof}
事实上,就情形1来说,
根据函数单调性的判定法,
函数\(f\)在\((x_0 - \delta,x_0)\)内单调增加,
而在\((x_0,x_0 + \delta)\)内单调减少.
又由于函数\(f\)在\(x_0\)处是连续的,
故当\(x\in\mathring{U}(x_0,\delta)\)时,
总有\(f(x) < f(x_0)\),
所以\(f(x_0)\)是\(f(x)\)的一个极大值.
\begin{figure}[htb]
    \def\subwidth{.5\linewidth}
    \begin{subfigure}[b]{\subwidth}
    \centering
        \begin{tikzpicture}
        \draw[thick,->] (0,0) -> (4,0)node[above]{\(x\)};
        \draw[thick,->] (0,0) -> (0,4)node[right]{\(y\)};
        \draw (0,0)node[below left]{\(O\)};
        \draw (2,2)node[above]{\(y=f(x)\)};
        \draw (2,.5)node[left]{\(f'(x)>0\)}node[right]{\(f'(x)<0\)};
        \draw[thick] (2,2)[rotate=90]arc[start angle=0,end angle=-60,radius=2];
        \draw[thick] (2,2)[rotate=90]arc[start angle=0,end angle=20,radius=5];
        \draw (.5,2)--(3.5,2);
        \draw[dashed] (2,2)--(2,0)node[below]{\(x_0\)};
        \end{tikzpicture}
    \subcaption{}
    \end{subfigure}%
    \begin{subfigure}[b]{\subwidth}
    \centering
        \begin{tikzpicture}
        \draw[thick,->] (0,0) -> (4,0)node[above]{\(x\)};
        \draw[thick,->] (0,0) -> (0,4)node[right]{\(y\)};
        \draw (0,0)node[below left]{\(O\)};
        \draw (2,2)node[above]{\(y=f(x)\)};
        \draw (2,.5)node[left]{\(f'(x)<0\)}node[right]{\(f'(x)>0\)};
        \draw[thick] (2,2)[rotate=-90]arc[start angle=0,end angle=60,radius=2];
        \draw[thick] (2,2)[rotate=-90]arc[start angle=0,end angle=-20,radius=5];
        \draw (.5,2)--(3.5,2);
        \draw[dashed] (2,2)--(2,0)node[below]{\(x_0\)};
        \end{tikzpicture}
    \subcaption{}
    \end{subfigure}%
    \\
    \begin{subfigure}[b]{\subwidth}
    \centering
        \begin{tikzpicture}
        \draw[thick,->] (0,0) -> (4,0)node[above]{\(x\)};
        \draw[thick,->] (0,0) -> (0,4)node[right]{\(y\)};
        \draw (0,0)node[below left]{\(O\)};
        \draw (2,2)node[above]{\(y=f(x)\)};
        \draw (2,.5)node[left]{\(f'(x)>0\)}node[right]{\(f'(x)>0\)};
        \draw[thick] (2,2)[rotate=-90]arc[start angle=0,end angle=60,radius=2];
        \draw[thick] (2,2)[rotate=90]arc[start angle=0,end angle=20,radius=5];
        \draw (.5,2)--(3.5,2);
        \draw[dashed] (2,2)--(2,0)node[below]{\(x_0\)};
        \end{tikzpicture}
    \subcaption{}
    \end{subfigure}%
    \begin{subfigure}[b]{\subwidth}
    \centering
        \begin{tikzpicture}
        \draw[thick,->] (0,0) -> (4,0)node[above]{\(x\)};
        \draw[thick,->] (0,0) -> (0,4)node[right]{\(y\)};
        \draw (0,0)node[below left]{\(O\)};
        \draw (2,2)node[above]{\(y=f(x)\)};
        \draw (2,.5)node[left]{\(f'(x)<0\)}node[right]{\(f'(x)<0\)};
        \draw[thick] (2,2)[rotate=90]arc[start angle=0,end angle=-60,radius=2];
        \draw[thick] (2,2)[rotate=-90]arc[start angle=0,end angle=-20,radius=5];
        \draw (.5,2)--(3.5,2);
        \draw[dashed] (2,2)--(2,0)node[below]{\(x_0\)};
        \end{tikzpicture}
    \subcaption{}
    \end{subfigure}%
    \caption{}
\end{figure}

类似地,可证情形2及情形3.
\end{proof}
%@see: https://mathworld.wolfram.com/FirstDerivativeTest.html
\end{theorem}
上述定理也可简单地表述为:
当\(x\)在\(x_0\)的邻域内由小及大经过\(x_0\)时,
如果\(f'(x)\)的符号由正变负(即\(f'_-(x_0)>0\)且\(f'_+(x_0)<0\)),
那么\(f(x)\)在\(x_0\)处取得极大值;
如果\(f'(x)\)的符号由负变正(即\(f'_-(x_0)<0\)且\(f'_+(x_0)>0\)),
那么\(f(x)\)在\(x_0\)处取得极小值;
如果\(f'(x)\)的符号并不改变,
那么\(f(x)\)在\(x_0\)处没有极值.

根据\cref{theorem:微分中值定理.函数存在极值的必要条件,theorem:微分中值定理.函数存在极值的第一充分条件},
如果函数\(f\)在所讨论的区间内连续,除个别点歪处处可导,
那么就可以按下列步骤来求\(f(x)\)在该区间内的极值点和相应的极值:
\begin{enumerate}
	\item 求出导数\(f'(x)\);
	\item 求出\(f(x)\)的全部驻点与不可导点;
	\item 考察\(f'(x)\)的符号在每个驻点或不可导点的左右邻域的情形,
	以确定该点是否为极值点;
	如果是极值点,
	进一步确定是极大值点还是极小值点;
	\item 求出各极值点的函数值,就得函数\(f\)的全部极值.
\end{enumerate}

\begin{theorem}[函数存在极值的第二充分条件]\label{theorem:微分中值定理.函数存在极值的第二充分条件}
设函数\(f\)在\(x_0\)处具有二阶导数,
且\(f'(x_0)=0\).
\begin{itemize}
	\item 当\(f''(x_0)<0\)时,函数\(f\)在\(x_0\)处取得极大值.
	\item 当\(f''(x_0)>0\)时,函数\(f\)在\(x_0\)处取得极小值.
\end{itemize}
%TODO proof
%@see: https://mathworld.wolfram.com/SecondDerivativeTest.html
\end{theorem}
上述定理表明,如果函数\(f\)在驻点\(x_0\)处的二阶导数\(f''(x_0)\neq0\),
那么该驻点\(x_0\)一定是极值点,
并且可以按二阶导数\(f''(x_0)\)的符号来判定\(f(x_0)\)是极大值还是极小值.
但如果\(f''(x_0)=0\),上述定理就不能应用.
事实上,当\(f'(x_0)=0\)且\(f''(x_0)=0\)时,
\(f(x)\)在\(x_0\)处可能有极大值,也可能有极小值,也可能没有极值;
例如,\(f_1(x) = -x^4\),\(f_2(x) = x^4\)和\(f_3(x) = x^3\)
这三个函数在\(x=0\)处就分别属于这三种情况.
因此,如果函数在驻点处的二阶导数为零,
那么还得用一阶导数在驻点左右邻域的符号来判定.

\begin{example}
设函数\(f(x),g(x)\)都具有二阶导数,
且\(g''(x)<0\),\(g(x_0)=a\)是\(g(x)\)的极值,
证明:“\(f'(a)>0\)”是“\(f[g(x)]\)在\(x_0\)取极大值”的充分条件.
\begin{solution}
因为\(g(x_0)=a\)是\(g(x)\)的极值,
所以根据\cref{theorem:微分中值定理.函数存在极值的必要条件}
必有\(g'(x_0)=0\).

记\(F(x) = f[g(x)]\),则\(F'(x) = \eval{f'(u)}_{u=g(x)} \cdot g'(x)\),
\(F''(x) = \eval{f''(u)}_{u=g(x)} \cdot g'(x) \cdot g'(x)
+ \eval{f'(u)}_{u=g(x)} \cdot g''(x)\).
根据\cref{theorem:微分中值定理.函数存在极值的第二充分条件},
要使“\(F(x)=f[g(x)]\)在\(x_0\)取极大值”,只需令\(F''(x_0) < 0\),
那么有\begin{equation*}
	\eval{f'(u_0)}_{u_0=g(x_0)} \cdot g''(x_0) < 0.
\end{equation*}
由题可知\(g''(x)<0\),
故\(\eval{f'(u_0)}_{u_0=g(x_0)} = f'(a) > 0\),
也就是说“\(f'(a)>0\)”是“\(f[g(x)]\)在\(x_0\)取极大值”的充分条件.
\end{solution}
\end{example}

\begin{example}
%@see: 《2023年全国硕士研究生入学统一考试(数学二)》一选择题/第7题
设函数\(f(x) = (x^2 + a) e^x\),
\(f\)没有极值点,但曲线\(y = f(x)\)有拐点.
求\(a\)的取值范围.
\begin{solution}
%@see: https://www.bilibili.com/video/BV1VGxWeVEy4/
求导得\begin{gather*}
	f'(x) = (x^2 + 2x + a) e^x, \\
	f''(x) = (x^2 + 4x + a + 2) e^x.
\end{gather*}

因为\(f\)没有极值点,
所以\(f'\)不变号,
那么恒成立\(x^2 + 2x + a \geq 0\),
故\(x^2 + 2x + a\)的判别式\(4 - 4a \leq 0\),
即\(a \geq 1\).

因为\(f\)有拐点,
所以\(f''\)在拐点变号,
那么\(x^2 + 4x + a + 2\)必有两个零点,
故\(x^2 + 4x + a + 2\)的判别式\(16 - 4(a+2) > 0\),
即\(a < 2\).

综上所述,\(1 \leq a < 2\).
\end{solution}
\end{example}

\subsection{最大值最小值问题}
假定函数\(f\)在闭区间\([a,b]\)上连续,
在开区间\((a,b)\)内除有限个点外可导,
且至多有有限个驻点.
在上述条件下,我们来讨论\(f(x)\)在\([a,b]\)上的最大值和最小值的求法.

首先由闭区间上连续函数的性质,
可知\(f(x)\)在\([a,b]\)上的最大值和最小值一定存在.

其次,如果最大值(或最小值)\(f(x_0)\)在开区间\((a,b)\)内的点\(x_0\)处取得,
那么,按\(f(x)\)在开区间内除有限个点外可导且至多有有限个驻点的假定,
可知\(f(x_0)\)一定也是\(f(x)\)的极大值(或极小值),
从而\(x_0\)一定是\(f(x)\)的驻点或不可导点.
又\(f(x)\)的最大值和最小值也可能在区间的端点处取得.

可用如下的方法求\(f(x)\)在\([a,b]\)上的最大值和最小值:
\begin{enumerate}
	\item 求出\(f(x)\)在\((a,b)\)内的
	驻点\(\AutoTuple{x}{m}\)和不可导点\(x'_1,x'_2,\dotsc,x'_n\);
	\item 计算\(f(x_i)\ (i=1,2,\dotsc,m)\)
	和\(f(x'_j)\ (j=1,2,\dotsc,n)\),
	以及\(f(a)\)和\(f(b)\);
	\item 比较上一步中求出的各个函数值,
	其中最大的就是\(f(x)\)在\([a,b]\)上的最大值,
	最小的就是\(f(x)\)在\([a,b]\)上的最小值.
\end{enumerate}

在求函数的最大值(或最小值)时,特别值得指出的是下述情形:
\(f(x)\)在一个区间(不论有限区间还是无限区间,开区间或闭区间)内可导,
且只有一个驻点\(x_0\),并且这个驻点\(x_0\)是函数\(f\)的极值点,
那么,当\(f(x_0)\)是极大值时,\(f(x_0)\)就是\(f(x)\)在该区间上的最大值;
当\(f(x_0)\)是极小值时,\(f(x_0)\)就是\(f(x)\)在该区间上的最小值.

\section{曲率}\label{section:微分中值定理.曲率}
\subsection{弧微分}
作为曲率的预备知识,先介绍弧微分的概念.

设函数\(f(x)\)在区间\((a,b)\)内具有连续导数.
在曲线\(y=f(x)\)上取固定点\(M_0(x_0,y_0)\)作为度量弧长的基点,
并规定依\(x\)增大的方向作为曲线的正向.
对曲线上任一点\(M(x,y)\),
规定有向弧段\(\Arc{M_0 M}\)的值\(s\)(简称为弧\(s\))如下:
\(s\)的绝对值等于这弧段的长度,
当有向弧段\(\Arc{M_0 M}\)的方向与曲线的正向一致时\(s>0\),相反时\(s<0\).
显然,弧\(s\)与\(x\)存在函数关系:
\(s = s(x)\),而且\(s(x)\)是\(x\)的单调增加函数.
下面来求\(s(x)\)的导数及微分.

设\(x\)、\(x+\increment x\)为\((a,b)\)内两个邻近的点,
它们在曲线\(y=f(x)\)上的对应点为\(M\)、\(M'\),
并设对应于\(x\)的增量\(\increment x\),
弧\(s\)的增量为\(\increment s\),
那么\begin{equation*}
	\increment s = \Arc{M_0 M'} - \Arc{M_0 M} = \Arc{M M'}.
\end{equation*}
于是\begin{align*}
	\left(\frac{\increment s}{\increment x}\right)^2
	&= \left(\frac{\Arc{M M'}}{\increment x}\right)^2
	= \left(\frac{\Arc{M M'}}{\abs{M M'}}\right)^2
		\cdot \frac{\abs{M M'}^2}{(\increment x)^2} \\
	&= \left(\frac{\Arc{M M'}}{\abs{M M'}}\right)^2
		\cdot \frac{(\increment x)^2 + (\increment y)^2}{(\increment x)^2} \\
	&= \left(\frac{\Arc{M M'}}{\abs{M M'}}\right)^2
		\cdot \left[ 1 + \left(\frac{\increment y}{\increment x}\right)^2 \right],
\end{align*}\begin{equation*}
	\frac{\increment s}{\increment x}
	= \pm \sqrt{\left(\frac{\Arc{M M'}}{\abs{M M'}}\right)^2
	\cdot \left[ 1 + \left(\frac{\increment y}{\increment x}\right)^2 \right]}.
\end{equation*}
令\(\increment x\to0\)取极限,
由于\(\increment x\to0\)时,
\(M' \to M\),
这时弧的长度与弦的长度之比的极限等于1,
即\begin{equation*}
	\lim_{M' \to M} \frac{\abs{\Arc{M M'}}}{\abs{M M'}} = 1,
\end{equation*}
又\begin{equation*}
	\lim_{\increment x\to0} \frac{\increment y}{\increment x} = y',
\end{equation*}
因此得\begin{equation*}
	\dv{s}{x} = \pm \sqrt{1 + (y')^2}.
\end{equation*}
由于\(s = s(x)\)是单调增加函数,从而上式根号前应取正号,
于是有\begin{equation}
	\dd{s} = \sqrt{1 + (y')^2} \dd{x},
\end{equation}
这就是\DefineConcept{弧微分公式}.

弧微分公式也可写作\begin{equation}
	\dd{s} = \sqrt{(\dd{x})^2 + (\dd{y})^2}.
\end{equation}

\subsection{曲率及其计算公式}
设曲线\(C\)是光滑的(即曲线上每一点处都具有切线,且切线随切点的移动而连续转动),
在曲线\(C\)上选定一点\(M_0\)作为度量弧\(s\)的基点.
设曲线上点\(M\)对应于弧\(s\),
在点\(M\)处切线的倾角为\(\alpha\)
(这里假定曲线\(C\)所在的平面上已设立了\(xOy\)坐标系),
曲线上另外一点\(M'\)对应于弧\(s+\increment s\),
在点\(M'\)处切线的倾角为\(\alpha + \increment \alpha\).
那么,弧段\(\Arc{MM'}\)的长度为\(\abs{\increment s}\).
当动点从\(M\)移动到\(M'\)时切线转过的角度为\(\abs{\increment \alpha}\).

我们用比值\(\frac{\abs{\increment\alpha}}{\abs{\increment s}}\),
即单位弧段上切线转过的角度的大小来表达弧段\(\Arc{MM'}\)的平均弯曲程度,
把这比值叫做弧段\(\Arc{MM'}\)的\DefineConcept{平均曲率},
并记作\(\overline{K}\),即\begin{equation*}
	\overline{K} = \abs{\frac{\increment\alpha}{\increment s}}.
\end{equation*}

类似于从平均速度引进瞬时速度的方法,
当\(\increment s\to0\)(即\(M' \to M\))时,
上述平均曲率的极限叫做曲线\(C\)在点\(M\)处的\DefineConcept{曲率}(curvature),
%@see: https://mathworld.wolfram.com/Curvature.html
记作\(K\),
即\begin{equation*}
	K \defeq \lim_{\increment s\to0} \abs{\frac{\increment\alpha}{\increment s}}.
\end{equation*}
在\(\displaystyle \lim_{\increment s\to0} \frac{\increment\alpha}{\increment s}
= \dv{\alpha}{s}\)存在的条件下,
\(K\)也可以表示为\begin{equation*}
	K = \abs{\dv{\alpha}{s}}.
\end{equation*}

对于直线来说,切线与直线本身重合,
当点沿直线移动时,切线的倾角不变,
\(\increment\alpha = 0\),
\(\frac{\increment\alpha}{\increment s} = 0\),
从而\(K = \abs{\displaystyle\dv{\alpha}{s}} = 0\).
这就是说,直线上任意点\(M\)处的曲率都等于零,这与我们直觉认识到的“直线不弯曲”一致.

\begin{figure}%曲率圆
	\centering
	\begin{tikzpicture}
		\draw[thick,->] (0,0) -> (9,0)node[above]{\(x\)};
		\draw[thick,->] (0,0) -> (0,8)node[right]{\(y\)};
		\pgfmathsetmacro{\cx}{4}
		\pgfmathsetmacro{\cy}{4}
		\pgfmathsetmacro{\cr}{3}
		\coordinate(M)at(\cx,\cy);
		\draw (M)circle(\cr);
		\pgfmathsetmacro{\ta}{30}
		\pgfmathsetmacro{\tb}{80}
		\pgfmathsetmacro{\pax}{\cx+\cr*sin(\ta)}
		\pgfmathsetmacro{\pay}{\cy-\cr*cos(\ta)}
		\pgfmathsetmacro{\pbx}{\cx+\cr*sin(\tb)}
		\pgfmathsetmacro{\pby}{\cy-\cr*cos(\tb)}
		\coordinate(P1)at(\pax,\pay);
		\coordinate(P2)at(\pbx,\pby);
		\draw (M)node[left]{\(D\)}--(P1)node[below]{\(M\)}
			(M)--(P2)node[right]{\(M'\)}node[midway,above]{\(a\)};
		\pgfmathsetmacro{\paz}{\pax+\pay*(\cy-\pay)/(\cx-\pax)}
		\pgfmathsetmacro{\pbz}{\pbx+\pby*(\cy-\pby)/(\cx-\pbx)}
		\coordinate(Q1)at(\paz,0);
		\coordinate(Q2)at(\pbz,0);
		\coordinate(X)at(100,0);
		\draw (Q1)--(P1) (Q2)--(P2);

		\draw pic["\(\increment\alpha\)",draw=orange,-,below right]{angle=P1--M--P2};
		\draw pic["\(\alpha\)",draw=orange,-,angle eccentricity=1.5,angle radius=5mm]{angle=X--Q1--P1};
		\draw pic["\(\alpha+\increment\alpha\)",draw=orange,-,angle eccentricity=1,angle radius=3mm,above right]{angle=X--Q2--P2};
		\draw pic[draw=gray,-,angle radius=0.3cm]{right angle=M--P1--Q1};
		\draw pic[draw=gray,-,angle radius=0.3cm]{right angle=M--P2--Q2};
	\end{tikzpicture}
	\caption{曲率圆}
	\label{figure:微分中值定理.曲率圆}
\end{figure}

设圆的半径为\(a\),由\cref{figure:微分中值定理.曲率圆} 可见,
圆在点\(M\)、\(M'\)处的切线所夹的角\(\increment\alpha\)等于中心角,
即\begin{equation*}
	\angle{M D M'} = \increment\alpha.
\end{equation*}
但是\begin{equation*}
	\angle{M D M'} = \frac{\increment s}{a},
\end{equation*}
于是\begin{equation*}
	\frac{\increment\alpha}{\increment s}
	= \frac{\increment s / a}{\increment s}
	= \frac{1}{a},
\end{equation*}
从而\begin{equation*}
	K = \abs{\dv{\alpha}{s}} = \frac{1}{a}.
\end{equation*}
因为点\(M\)是圆上任意取定的一点,
上述结论表示,
圆上各点处的曲率都等于半径\(a\)的倒数\(\frac{1}{a}\),
这就是说,圆的弯曲程度到处一样;
且半径越小的圆,曲率越大,即圆弯曲得越厉害.

设曲线的直角坐标方程为\(y=f(x)\),
且\(f(x)\)具有二阶导数(这时\(f'(x)\)连续,从而曲线是光滑的).
因为\(\tan\alpha = y'\),
所以再在等号两边同时对\(x\)求导便得\begin{equation*}
	\sec^2\alpha \dv{\alpha}{x} = y'',
\end{equation*}
即\begin{equation*}
	\dv{\alpha}{x}
	= \frac{y''}{\sec^2\alpha}
	= \frac{y''}{1 + \tan^2\alpha}
	= \frac{y''}{1 + (y')^2},
\end{equation*}
于是\begin{equation*}
	\dd{\alpha}
	= \frac{y''}{1 + (y')^2} \dd{x}.
\end{equation*}
又因为\(\dd{s} = \sqrt{1+(y')^2} \dd{x}\),
从而有\begin{equation}
	K
	= \abs{\dv{\alpha}{s}}
	= \frac{1}{\sqrt{1+(y')^2}}
	\abs{\frac{y''}{1 + (y')^2}}
	= \frac{\abs{y''}}{(1+(y')^2)^{\frac32}}.
\end{equation}

设曲线由参数方程\(\left\{ \begin{array}{c}
	x = \phi(t) \\
	y = \psi(t)
\end{array} \right.\)给出,
由\cref{equation:导数.参数方程确定的函数的一阶导数,equation:导数.参数方程确定的函数的二阶导数}
可知\begin{equation*}
	y' = \frac{\psi'(t)}{\phi'(t)}, \qquad
	y'' = \frac{\psi''(t) \phi'(t) - \psi'(t) \phi''(t)}{(\phi'(t))^3},
\end{equation*}
那么有\begin{equation}
	K = \frac{
		\abs{\phi'(t)\psi''(t)-\phi''(t)\psi'(t)}
	}{
		[(\phi'(t))^2+(\psi'(t))^2]^{\frac32}
	}.
\end{equation}

在某些实际问题中,
\(\abs{y'}\)同\(1\)比较起来是很小的(即\(\abs{y'} \ll 1\)),可以忽略不计,
这时,由\(1 + (y')^2 \approx 1\),而有曲率的近似计算公式为\begin{equation*}
	K \approx \abs{y''}.
\end{equation*}
这就是说,当\(\abs{y'} \ll 1\)时,
曲率\(K\)近似于\(\abs{y''}\).
经过这样的简化之后,对一些复杂问题的计算和讨论就方便多了.

\subsection{曲率圆与曲率半径}
设曲线\(y=f(x)\)在点\(M(x,y)\)处的曲率为\(K\ (K\neq0)\).
在点\(M\)处的曲线的发现上,在凹的一侧取一点\(D\),使\(\abs{DM} = \frac{1}{K} = \rho\).
以\(D\)为圆心,\(\rho\)为半径作圆,这个圆叫做曲线在点\(M\)处的\DefineConcept{曲率圆},
曲率圆的圆心\(D\)叫做曲线在点\(M\)处的\DefineConcept{曲率中心},
曲率圆的半径\(\rho\)叫做曲线在点\(M\)处的\DefineConcept{曲率半径}.

按上述规定可知,曲率圆与曲线在点\(M\)有相同的切线和曲率,且在点\(M\)邻近有相同的凹向.
因此,在实际问题中,常常用曲率圆在点\(M\)邻近的一段圆弧来近似代替曲线弧,以使问题简化.

按上述规定,曲线在点\(M\)处的曲率\(K\ (K\neq0)\)
与曲线在点\(M\)处的曲率半径\(\rho\)有如下的关系:\begin{equation*}
	\rho = \frac{1}{K}, \qquad
	K = \frac{1}{\rho}.
\end{equation*}
这就是说:曲线上一点处的曲率半径与曲线在该点处的曲率互为倒数.

\begin{example}
求出对数曲线\(y = \ln x\)上曲率半径最小的点.
\begin{solution}
显然有\(y' = \frac{1}{x}\),\(y'' = -\frac{1}{x^2}\),那么曲率为\begin{equation*}
	K = \frac{\abs{y''}}{(1+(y')^2)^{\frac32}}
	= \frac{1/x^2}{(1+1/x^2)^{\frac32}}
	= \frac{x}{(1+x^2)^{\frac32}}.
\end{equation*}
曲率\(K\)对\(x\)求导得\begin{equation*}
	K' = \frac{(1+x^2)^{\frac32} - x \frac32 (1+x^2)^{\frac12} 2x}{(1+x^2)^3}
	= \frac{1 - 2x^2}{(1+x^2)^{\frac52}}.
\end{equation*}
令\(K' = 0\),考虑\(x>0\),
解得\(x_0 = \frac{1}{\sqrt{2}}\).
当\(0<x<x_0\)时,\(K'>0\);
当\(x>x_0\)时,\(K'<0\);
说明\(K'\)在\(x=x_0\)时取得极大值.
而曲率半径最小的点就是曲率最大的点,
即\(x = \frac{1}{\sqrt{2}}\)时,
曲率半径最小值为\(\frac{3\sqrt{3}}{2}\).
\end{solution}
\end{example}

\subsection{曲率中心的计算公式以及渐屈线、渐伸线}
设已知曲线的方程是\(y=f(x)\),
且其二阶导数\(y''\)在点\(x\)不为零.
又设曲线\(y=f(x)\)在点\(M(x,y)\)的曲率为\(K\),
曲率中心为\(D(\alpha,\beta)\),
曲率半径为\(\rho=K^{-1}\),
那么曲率圆的方程为\begin{equation*}
	(\xi-\alpha)^2+(\eta-\beta)^2=\rho^2,
	\quad(\xi,\eta)\in\mathbb{R}^2.
\end{equation*}

因为\(M(x,y)\)在这个曲率圆上,所以满足曲率圆的方程,
有\begin{equation}\label{equation:曲率圆.曲率中心的推导1}
	(x-\alpha)^2+(y-\beta)^2=\rho^2.
\end{equation}
又因为曲线在点\(M\)的切线与曲率圆的半径\(DM\)垂直,
两者的斜率分别为\begin{equation*}
	y'
	\quad\text{和}\quad
	\frac{y-\beta}{x-\alpha},
\end{equation*}
所以\begin{equation*}
	y' \cdot \frac{y-\beta}{x-\alpha} = -1,
\end{equation*}
即\begin{equation}\label{equation:曲率圆.曲率中心的推导2}
	y' = -\frac{x-\alpha}{y-\beta}.
\end{equation}
由\cref{equation:曲率圆.曲率中心的推导1,equation:曲率圆.曲率中心的推导2}
消去\(x-\alpha\),解出\begin{equation*}
	(y-\beta)^2
	=\frac{\rho^2}{1+(y')^2}
	=\frac{(1+(y')^2)^2}{(y'')^2}.
\end{equation*}

由于当\(y''>0\)时,曲线\(y=f(x)\)是凹弧,\(y-\beta<0\);
当\(y''<0\)时,曲线是凸弧,\(y-\beta>0\).
总之,\(y''\)与\(y-\beta\)异号.
因此取上式两边的平方根,得\begin{equation*}
	y-\beta
	=-\frac{1+(y')^2}{y''};
\end{equation*}
于是\begin{equation*}
	x-\alpha
	=-y'(y-\beta)
	=\frac{y'(1+(y')^2)}{y''}.
\end{equation*}

因此,曲线在对应点\(M(x,y)\)的曲率中心\(D(\alpha,\beta)\)的坐标为
\begin{equation}
	\left\{ \def\arraystretch{1.5} \begin{array}{l}
		\alpha = x - y' \frac{1 + (y')^2}{y''}, \\
		\beta = y + \frac{1 + (y')^2}{y''}.
	\end{array} \right.
\end{equation}

当点\((x,f(x))\)沿曲线\(C\)移动时,
相应的曲率中心\(D\)的轨迹曲线\(G\)称为曲线\(C\)的\DefineConcept{渐屈线}(evolute).
%@see: https://mathworld.wolfram.com/Evolute.html
相对地,曲线\(C\)称为曲线\(G\)的\DefineConcept{渐伸线}(involute).
%@see: https://mathworld.wolfram.com/Involute.html

\section{本章总结}
借助一阶导数的符号,可以确定函数图形在哪个区间上上升,在哪个区间上下降,在什么地方有极值点;
借助二阶导数的符号,可以确定函数图形在哪个区间上为凹,在哪个区间上为凸,在什么地方有拐点.
知道了函数图形的升降、凹凸以及极值点和拐点后,也就可以掌握函数的性态,并把函数的图形画得比较准确.

现在,随着现代计算机技术的发展,借助计算机和许多数学软件,可以方便地画出各种函数的图形.
但是,如何识别机器作图中的误差,如何掌握图形上的关键点,如何选择作图的范围等,
从而进行人工干预,仍然需要我们有运用微分学的方法描绘图形的基本知识.

描绘函数图形的一般步骤如下:\begin{enumerate}
	\item 确定函数\(y=f(x)\)的定义域,特别是要找出函数的奇点,
	例如\begin{itemize}
		\item 函数\(x \mapsto 1/x\)的定义域是\(\Set{ x \given x\neq0 }\),
		\item 函数\(x \mapsto \ln x\)的定义域是\(\Set{ x \given x>0 }\),
		\item 函数\(x \mapsto \sqrt{x}\)的定义域是\(\Set{ x \given x\geq0 }\);
	\end{itemize}

	\item 确定函数\(y=f(x)\)的值域,判断该函数是单值的还是多值的;

	\item 发现函数所具有的某些特性,如有界性、奇偶性、周期性等;
	特别是要利用以下几点结论:\begin{itemize}
		\item \hyperref[theorem:导数与微分.导函数的奇偶性]{可导奇函数的导函数是偶函数,可导偶函数的导函数是奇函数};
		\item \hyperref[theorem:导数与微分.导函数的周期性]{周期函数的导数也是周期函数,且周期相同};
		\item \hyperref[theorem:定积分.利用对称性简化计算1]{奇函数在对称区间上的积分等于零,偶函数在对称区间上的积分等于在正半部分上的积分的两倍};
		\item \hyperref[theorem:定积分.积分上限函数的奇偶性]{被积函数是奇函数的积分上限函数是偶函数,被积函数是偶函数的积分上限函数是奇函数};
		\item \hyperref[theorem:定积分.周期函数的积分]{周期函数在长度是周期的整倍数的区间上的积分具有平移对称性};
		\item \hyperref[example:定积分.积分上限函数是周期函数的充分必要条件]{积分上限函数是以\(T\)周期的函数,当且仅当被积函数在任意一个长度为\(T\)的区间上的积分等于零};
	\end{itemize}

	\item 求出函数的一阶导数\(f'(x)\)和二阶导数\(f''(x)\);

	\item 求出函数\(f(x)\)及其一阶导数\(f'(x)\)和二阶导数\(f''(x)\)在函数定义域内的全部零点,
	并求出函数\(f(x)\)的间断点及\(f'(x)\)和\(f''(x)\)不存在的点,
	用这些点把函数的定义域划分成几个部分区间,
	确定在这些区间内\(f'(x)\)和\(f''(x)\)的符号,
	并利用\cref{theorem:图形绘制.单调性.利用导数判定函数单调性,theorem:微分中值定理.曲线凹凸的判定,theorem:单调性和凹凸性.拐点的判定}
	确定函数图形的%
	\hyperref[definition:函数的性质.单调性]{单调性}、%
	\hyperref[definition:函数图像的绘制.凹凸性的定义]{凹凸性}、%
	\hyperref[definition:函数的极值与最值.驻点]{驻点}(导数为零的点)、%
	\hyperref[definition:函数的极值与最值.极值点]{极值点}(函数值严格大于或严格小于其邻域内任意一点函数值的点)、%
	\hyperref[definition:单调性和凹凸性.拐点]{拐点}(凹凸性发生改变的点);

	\item 利用公式\begin{equation*}
		k = \lim_{x\to\infty} \frac{y}{x},
		\quad\text{和}\quad
		b = \lim_{x\to\infty} \left[y - kx\right],
	\end{equation*}确定函数图形的渐近线的斜率\(k\)和截距\(b\);

	\item 利用公式\begin{equation*}
		K = \frac{\abs{y''}}{(1+(y')^2)^{\frac32}}
	\end{equation*}求出函数图形在某一点处的曲率\(K\),
	并进一步确定函数图形在该点的曲率圆、曲率中心;
	确定函数图形的渐屈线和渐伸线,以及其他变化趋势;

	\item 算出\(f(x)\)、\(f'(x)\)和\(f''(x)\)的零点、间断点、驻点、极值点、拐点等
	关键点所对应的函数值或者单侧极限,定出图形上的相应的点;
	要注意一个点可能具有多重身份,也就是说一个点可能既是零点也是驻点,既是极值点也是拐点;
	为了把图形描绘得准确些,有时还需要补充一些点;
	然后结合前两步中得到的结果,联结这些点画出函数\(y=f(x)\)的图形.
\end{enumerate}
我们常常把上述分析过程称为\emph{函数性态分析}.


\chapter{不定积分}\label{chapter:不定积分}
\section{不定积分的概念与性质}
在\cref{chapter:导数}我们已经介绍了求导运算.
现在我们来探索求导运算的逆运算.

\subsection{原函数与不定积分的概念}
\begin{definition}\label{definition:不定积分.原函数}
%@see: 《高等数学(第六版 上册)》 P184 定义1
%@see: 《数学分析(第二版 上册)》(陈纪修) P241 定义6.1.1
如果在区间\(I\)上,
可导函数\(F\)的导函数为\(f\),
即对任一\(x \in I\),
都有\begin{equation*}
	F'(x)=f(x)
	\quad\text{或}\quad
	\dd{F(x)}=f(x) \dd{x},
\end{equation*}
那么称
“\(F\)是\(f\)在区间\(I\)上的一个\DefineConcept{原函数}(primitive)”
“\(f\)在区间\(I\)上具有原函数\(F\)”.
\end{definition}
\begin{remark}
由\cref{definition:不定积分.原函数} 可知:
任一函数只要具有原函数,它的原函数一定可导.
反过来说,如果一个函数在某个区间上不可导,它一定不是定义在这个区间上的任意函数的原函数.
\end{remark}

关于原函数,我们首先要问:
一个函数具备什么条件,才能保证它的原函数一定存在?
这个问题将在下一章中讨论,这里先介绍一个结论.
\begin{theorem}[原函数存在定理]%提前叙述该定理.参见\cref{theorem:定积分.原函数存在定理}
如果函数\(f(x)\)在区间\(I\)上连续,
那么在区间\(I\)上存在可导函数\(F(x)\),使对任一\(x \in I\)都有\begin{equation*}
	F'(x)=f(x).
\end{equation*}
换言之,连续函数一定有原函数.
\end{theorem}
下面还要说明两点.

第一,如果\(f(x)\)在区间\(I\)上有原函数,
即有一个函数\(F(x)\),
使对任一\(x \in I\),
都有\(F'(x) = f(x)\),
那么,对任何常数\(C\),显然也有\begin{equation*}
	[F(x) + C]' = f(x),
\end{equation*}
即对任何常数\(C\),函数\(F(x) + C\)也是\(f(x)\)的原函数.
这说明,如果\(f(x)\)有一个原函数,那么\(f(x)\)就有无限多个原函数.

第二,如果在区间\(I\)上\(F(x)\)是\(f(x)\)的一个原函数,
那么\(f(x)\)的其他原函数与\(F(x)\)有什么关系?

设\(\Phi(x)\)是\(f(x)\)的另一个原函数,
即对任一\(x \in I\)有\begin{equation*}
	\Phi'(x) = f(x),
\end{equation*}
于是\begin{equation*}
	[\Phi(x) - F(x)]' = \Phi'(x) - F'(x) = f(x) - f(x) = 0.
\end{equation*}
在前面章节已经知道,
在一个区间上导数恒为零的函数必为常数,所以\begin{equation*}
	\Phi(x) - F(x) = C_0,
\end{equation*}
其中\(C_0\)是某个常数.这就表明\(\Phi(x)\)与\(F(x)\)只差一个常数.
因此,当\(C\)为任意的常数时,
表达式\(F(x) + C\)就可表示\(f(x)\)的任意一个原函数.
也就是说,\(f(x)\)的全体原函数所组成的集合,就是函数族\begin{equation*}
	\Set{ F(x) + C \given C \in (-\infty,\infty) }.
\end{equation*}

\begin{definition}
%@see: 《高等数学(第六版 上册)》 P185 定义2
%@see: 《数学分析(第二版 上册)》(陈纪修) P242 定义6.1.2
在区间\(I\)上,
函数\(f\)的全体原函数,
称为“\(f\)在区间\(I\)上的\DefineConcept{不定积分}(indefinite integral)”,
记作\begin{equation*}
	\int f(x) \dd{x},
\end{equation*}
即\begin{equation*}
	\int f(x) \dd{x}
	\defeq
	\Set{ F \in \mathbb{R}^I \given (\forall x \in I)[F'(x) = f(x)] },
\end{equation*}
其中记号\(\int\)称为\DefineConcept{积分号},
\(f(x)\)称为\DefineConcept{被积函数},
\(f(x) \dd{x}\)称为\DefineConcept{被积表达式},
\(x\)称为\DefineConcept{积分变量}.
\end{definition}
由此定义及前面的说明可知,
如果\(F(x)\)是\(f(x)\)在区间\(I\)上的一个原函数,
那么\(F(x) + C\)就是\(f(x)\)的不定积分,
即\begin{equation*}
	\int f(x) \dd{x} = F(x) + C.
\end{equation*}
因而不定积分\(\int f(x) \dd{x}\)可以表示\(f(x)\)的任意一个原函数.

对初等函数来说,在其定义区间上,它的原函数一定存在,但原函数不一定都是初等函数,如\begin{equation*}
	\int e^{-x^2} \dd{x}, \qquad
	\int \frac{\sin x}{x} \dd{x}, \qquad
	\int \frac{\dd{x}}{\ln{x}}, \qquad
	\int \frac{\dd{x}}{\sqrt{1+x^4}}
\end{equation*}
等等,它们的原函数就都不是初等函数.

\begin{example}
求\(\int \frac{\dd{x}}{x}\).
\begin{solution}
当\(x > 0\)时,由于\((\ln x)' = \frac{1}{x}\),
所以\(\ln x\)是\(\frac{1}{x}\)在区间\((0,+\infty)\)内的一个原函数.
因此,在\((0,+\infty)\)内,\begin{equation*}
	\int \frac{\dd{x}}{x} = \ln x + C_1.
\end{equation*}

当\(x < 0\)时,
由于\([\ln(-x)]' = \frac{1}{-x} \cdot (-1) = \frac{1}{x}\),
所以\(\ln(-x)\)是\(\frac{1}{x}\)在\((-\infty,0)\)内的一个原函数.
因此,在\((-\infty,0)\)内,\begin{equation*}
	\int \frac{\dd{x}}{x} = \ln(-x) + C_2.
\end{equation*}

把在\(x > 0\)及\(x < 0\)内的结果合起来,
可写作\begin{equation}
	\int \frac{\dd{x}}{x} = \left\{ \begin{array}{lc}
		\ln x + C_1, & x>0, \\
		\ln(-x) + C_2, & x<0.
	\end{array} \right.
\end{equation}
虽然常数\(C_1\)和\(C_2\)的取值可以是独立的,
但在不严谨的情况下,为方便记忆,上式也可写作\begin{equation}
	\int \frac{\dd{x}}{x} = \ln\abs{x} + C.
\end{equation}
\end{solution}
\end{example}

\begin{definition}
函数\(f(x)\)的原函数的图形称为\(f(x)\)的积分曲线.
\end{definition}

从不定积分的定义,即可知以下关系:
\begin{align*}
	\text{\(\int f(x) \dd{x}\)是\(f(x)\)的原函数}
	&\iff
	\dv{x}\relax\left[ \int f(x) \dd{x} \right] = f(x) \\
	&\iff
	\dd\relax\left[\int f(x) \dd{x}\right] = f(x) \dd{x}. \\
	\text{\(F(x)\)是\(F'(x)\)的原函数}
	&\iff
	\int F'(x) \dd{x} = F(x) + C \\
	&\iff
	\int \dd{F(x)} = F(x) + C.
\end{align*}

由此可见,微分运算(以记号\(\dd\relax\)表示)
与求不定积分的运算(简称积分运算,以记号\(\int\)表示)是互逆的;
当记号\(\int\)与\(\dd\relax\)连在一起时,或者抵消,或者抵消后差一个常数.

\subsection{基本积分表}
既然积分运算是微分运算的逆运算,
那么很自然地可以从导数公式得到相应的积分公式.

\subsection{不定积分的性质}
\begin{property}\label{theorem:不定积分.性质1}
%@see: 《高等数学(第六版 上册)》 P189 性质1
%@see: 《数学分析(第二版 上册)》(陈纪修) P243 定理6.1.1(线性性)
设函数\(f\)及\(g\)的原函数存在,
则\begin{equation*}
	\int [f(x) + g(x)] \dd{x}
	= \int f(x) \dd{x}
	+ \int g(x) \dd{x}.
\end{equation*}
\end{property}
\begin{remark}
\cref{theorem:不定积分.性质1} 对于有限个函数都是成立的.
\end{remark}

\begin{property}\label{theorem:不定积分.性质2}
%@see: 《高等数学(第六版 上册)》 P190 性质2
%@see: 《数学分析(第二版 上册)》(陈纪修) P243 定理6.1.1(线性性)
设函数\(f\)的原函数存在,
\(k\)为非零常数,
则\begin{equation*}
	\int k f(x) \dd{x} = k \int f(x) \dd{x}.
\end{equation*}
\end{property}
\begin{remark}
我们在\cref{theorem:不定积分.性质2} 中强调\(k\neq0\),
是因为如果\(k=0\),那么上式右边为\(0 \int f(x) \dd{x} = 0\),
而上式左边\(\int 0 f(x) \dd{x}\)却是全体常数函数的集合,
这就造成了定义上的混乱.
\end{remark}

\begin{example}
%@see: 《高等数学(第六版 上册)》 P191 例11
求\(\int 2^x e^x \dd{x}\).
\begin{solution}
应为\(2^x e^x = (2e)^x\),
所以由基本积分表有\begin{equation*}
	\int 2^x e^x \dd{x}
	= \int (2e)^x \dd{x}
	= \frac{(2e)^x}{\ln(2e)} + C
	= \frac{2^x e^x}{1+\ln2} + C.
\end{equation*}
\end{solution}
\end{example}
\begin{example}
求\(\int \sin^2 x \dd{x}\)和\(\int \cos^2 x \dd{x}\).
\begin{solution}
记\begin{equation*}
	I_1 \defeq \int \sin^2 x \dd{x},
	\qquad
	I_2 \defeq \int \cos^2 x \dd{x}.
\end{equation*}
那么\begin{gather*}
	I_1 + I_2
	= \int (\sin^2 x + \cos^2 x) \dd{x}
	= \int \dd{x}
	= x + C_1, \\
	I_2 - I_1
	= \int (\cos^2 x - \sin^2 x) \dd{x}
	= \int \cos 2x \dd{x}
	= \frac12 \sin 2x + C_2,
\end{gather*}
解得\begin{gather*}
	\int \sin^2 x \dd{x} = \frac{x}2 - \frac14 \sin 2x + C_3, \\
	\int \cos^2 x \dd{x} = \frac{x}2 + \frac14 \sin 2x + C_4.
\end{gather*}
\end{solution}
\end{example}
\begin{example}
%@see: 《高等数学(第六版 上册)》 P191 例12
求\(\int \tan^2 x \dd{x}\).
\begin{solution}
根据\hyperref[equation:三角函数.毕达哥拉斯三角恒等式2]{三角恒等式}
\(\tan^2 x + 1 = \sec^2 x\),
有\begin{equation*}
	\int \tan^2 x \dd{x}
	= \int (\sec^2 x - 1) \dd{x}
	= \int \sec^2 x \dd{x} - \int \dd{x}
	= \tan x - x + C.
\end{equation*}
\end{solution}
\end{example}
\begin{example}
求\(\int \cot^2 x \dd{x}\).
\begin{solution}
根据\hyperref[equation:三角函数.毕达哥拉斯三角恒等式3]{三角恒等式}
\(1 + \cot^2 x = \csc^2 x\),
有\begin{equation*}
	\int \cot^2 x \dd{x}
	= \int (\csc^2 x - 1) \dd{x}
	= -\cot x - x + C.
\end{equation*}
\end{solution}
\end{example}

\section{换元积分法}
利用基本积分表与积分的性质,我们可以计算的不定积分是非常有限的.
因此,有必要进一步来研究不定积分的求法.
本节把复合函数的微分法反过来用于求解不定积分,
利用中间变量的代换,得到复合函数的积分法,
称为\DefineConcept{换元积分法},简称\DefineConcept{换元法}.

\subsection{第一类换元法}
\begin{theorem}
设\(f(u)\)具有原函数,\(u=\phi(x)\)可导,则有换元公式\begin{equation*}
	\int f\bigl[\phi(x)\bigr] \phi'(x) \dd{x}
	= \left[ \int f(u) \dd{u} \right]_{u=\phi(x)}.
\end{equation*}
\begin{proof}
设\(f(u)\)的原函数是\(F(u)\),即\begin{equation*}
	F'(u) = f(u),
	\qquad
	\int f(u) \dd{u} = F(u) + C.
\end{equation*}
因为\(u = \phi(x)\)可导,
那么,根据复合函数微分法,
有\begin{equation*}
	\dd{F\left[\phi(x)\right]} = f\left[\phi(x)\right] \phi'(x) \dd{x},
\end{equation*}
从而根据不定积分的定义有\begin{equation*}
	\int f\bigl[\phi(x)\bigr] \phi'(x) \dd{x}
	= F\bigl[\phi(x)\bigr] + C
	= \left[ \int f(u) \dd{u} \right]_{u=\phi(x)}.
	\qedhere
\end{equation*}
\end{proof}
\end{theorem}

由此定理可见,虽然\(\int f[\phi(x)] \phi'(x) \dd{x}\)是一个整体的记号,
但是从形式上看,被积表达式中的\(\dd{x}\)也可当作\(x\)的微分对待,
从而微分等式\(\phi'(x) \dd{x} = \dd{u}\)可以方便地应用到被积表达式中来,
就像我们把\(\int F'(x) \dd{x}\)记作\(\int \dd{F(x)}\)一样,
就是按微分\(F'(x) \dd{x} = \dd{F(x)}\),
把被积表达式\(F'(x) \dd{x}\)记作\(\dd{F(x)}\).

\begin{example}
求\(\int \frac{\dd{x}}{3+2x}\).
\begin{solution}
令\(u = 3+2x\),
那么被积函数\(\frac{1}{3+2x} = \frac{1}{u}\).
这里缺少\(\displaystyle\dv{u}{x}=2\)这样一个因子,
但由于\(\displaystyle\dv{u}{x}\)是个常数,
故可改变系数凑出这个因子:\begin{equation*}
	\frac{1}{3+2x}
	= \frac{1}{2} \cdot \frac{1}{3+2x} \cdot 2
	= \frac{1}{2} \cdot \frac{1}{3+2x} (3+2x)',
\end{equation*}
从而\begin{align*}
	\int \frac{\dd{x}}{3+2x}
	&= \frac{1}{2} \int \frac{1}{3+2x} (3+2x)' \dd{x}
	= \frac{1}{2} \int \frac{\dd{u}}{u} \\
	&= \frac{1}{2} \ln\abs{u} + C
	= \frac{1}{2} \ln\abs{3+2x} + C.
\end{align*}
\end{solution}
\end{example}

\begin{remark}
%@see: 《高等数学(第六版 上册)》 P195
一般地,对于积分\(\int f(ax+b) \dd{x}\),
总可作变换\(u=ax+b\),
把它化为\begin{align*}
	\int f(ax+b) \dd{x}
	&= \int \frac{1}{a} f(ax+b) \dd(ax+b) \\
	&= \frac{1}{a} \left[ \int f(u) \dd{u} \right]_{u=ax+b}.
\end{align*}
\end{remark}

\begin{example}
求\(\int 2x e^{x^2} \dd{x}\).
\begin{solution}
令\(u=x^2\),
则被积函数\(2x e^{x^2} = e^{x^2} (x^2)' = e^u u'\),
于是\begin{equation*}
	\int 2x e^{x^2} \dd{x}
	= \int e^u \dd{u}
	= e^u + C
	= e^{x^2} + C.
\end{equation*}
\end{solution}
\end{example}

\begin{example}
求\(\int x \sqrt{1-x^2} \dd{x}\).
\begin{solution}
令\(u=1-x^2\),
则\(\dd{u} = -2x\dd{x}\),
\(-\frac{1}{2}\dd{u} = x\dd{x}\),
因此\begin{align*}
	\int x \sqrt{1-x^2} \dd{x}
	&= \int u^{\frac{1}{2}} \left(-\frac{1}{2}\right) \dd{u}
	= -\frac{1}{2} \frac{u^{\frac{3}{2}}}{\frac{3}{2}} + C \\
	&= -\frac{1}{3} u^{\frac{3}{2}} + C
	= -\frac{1}{3} (1-x^2)^{\frac{3}{2}} + C.
\end{align*}
\end{solution}
\end{example}

\begin{example}
求\(\int \frac{\dd{x}}{a^2+x^2}\).
\begin{solution}
令\(u=\frac{x}{a}\),
\(\dd{x}=a\dd{u}\),
\(\frac{1}{a^2+x^2}
= \frac{1}{a^2(1+u^2)}\),
于是\begin{align}
	\int \frac{\dd{x}}{a^2+x^2}
	&= \int \frac{1}{a^2(1+u^2)} \cdot a\dd{u}
		\nonumber \\
	&= \frac{1}{a} \int \frac{\dd{u}}{1+u^2}
		\nonumber \\
	&= \frac{1}{a} \arctan u + C
		\nonumber \\
	&= \frac{1}{a} \arctan\frac{x}{a} + C.
\end{align}
\end{solution}
\end{example}

\begin{example}
求\(\int \frac{\dd{x}}{\sqrt{a^2-x^2}}\ (a>0)\).
\begin{solution}
令\(u=\frac{x}{a}\),
\(\dd{x}=a\dd{u}\),
\(\frac{1}{\sqrt{a^2-x^2}}
= \frac{1}{a\sqrt{1-u^2}}\),
于是\begin{align}
	\int \frac{\dd{x}}{\sqrt{a^2-x^2}}
	&= \int \frac{1}{a\sqrt{1-u^2}} \cdot a\dd{u}
		\nonumber \\
	&= \int \frac{\dd{u}}{\sqrt{1-u^2}}
		\nonumber \\
	&= \arcsin u + C
		\nonumber \\
	&= \arcsin\frac{x}{a} + C.
\end{align}
\end{solution}
\end{example}

\begin{example}
求\(\int \frac{\dd{x}}{x^2 - a^2}\).
\begin{solution}
因为\begin{equation*}
	\frac{1}{x^2 - a^2}
	= \frac{1}{2a} \left(\frac{1}{x-a} - \frac{1}{x+a}\right),
\end{equation*}
所以\begin{align}
	\int \frac{\dd{x}}{x^2 - a^2}
	&= \frac{1}{2a} \int \left(\frac{1}{x-a} - \frac{1}{x+a}\right) \dd{x}
		\nonumber \\
	&= \frac{1}{2a} \left[ \int \frac{\dd(x-a)}{x-a} - \int \frac{\dd(x+a)}{x+a} \right]
		\nonumber \\
	&= \frac{1}{2a} ( \ln\abs{x-a} - \ln\abs{x+a} ) + C
		\nonumber \\
	&= \frac{1}{2a} \ln\abs{\frac{x-a}{x+a}} + C.
\end{align}
\end{solution}
\end{example}

\begin{example}
求\(\int \sin mx \cos nx \dd{x}\ (m \neq n)\).
\begin{solution}
由\cref{equation:函数.三角函数.和积互化公式7},
\begin{equation*}
	\sin mx \cos nx
	= \frac12 [\sin(m+n)x + \sin(m-n)x],
\end{equation*}
于是\begin{align}
	\int \sin mx \cos nx \dd{x}
	&= \frac12 \left[
		\int \sin(m+n)x \dd{x}
		+ \int \sin(m-n)x \dd{x}
	\right]
	\nonumber \\
	&= -\frac{\cos(m+n)x}{2(m+n)}
		- \frac{\cos(m-n)x}{2(m-n)}
		+ C.
\end{align}
\end{solution}
\end{example}

\begin{example}
求\(\int \sin mx \sin nx \dd{x}\ (m \neq n)\).
\begin{solution}
由\cref{equation:函数.三角函数.和积互化公式10},
\begin{equation*}
	\sin mx \sin nx
	= -\frac12 [\cos(m+n)x - \cos(m-n)x],
\end{equation*}
于是\begin{align}
	\int \sin mx \sin nx \dd{x}
	&= -\frac12 \left[
		\int \cos(m+n)x \dd{x}
		- \int \cos(m-n)x \dd{x}
	\right]
	\nonumber \\
	&= \frac{\sin(m-n)x}{2(m-n)}
		- \frac{\sin(m+n)x}{2(m+n)}
		+ C.
\end{align}
\end{solution}
\end{example}

\begin{example}
求\(\int \cos mx \cos nx \dd{x}\ (m \neq n)\).
\begin{solution}
由\cref{equation:函数.三角函数.和积互化公式9},
\begin{equation*}
	\cos mx \cos nx
	= \frac12 [\cos(m+n)x + \cos(m-n)x],
\end{equation*}
于是\begin{align}
	\int \cos mx \cos nx \dd{x}
	&= \frac12 \left[
		\int \cos(m+n)x \dd{x}
		+ \int \cos(m-n)x \dd{x}
	\right]
	\nonumber \\
	&= \frac{\sin(m+n)x}{2(m+n)}
		+ \frac{\sin(m-n)x}{2(m-n)}
		+ C.
\end{align}
\end{solution}
\end{example}

\begin{example}
求\(\int \sin^3 x \dd{x}\).
\begin{solution}
\(\int \sin^3 x \dd{x}
= \int \sin^2 x \cdot \sin x \dd{x}
= -\int (1 - \cos^2 x) \dd(\cos x)
= -\cos x + \frac{1}{3} \cos^3 x + C\).
\end{solution}
\end{example}

\begin{remark}
%@see: 《高等数学(第六版 上册)》 P197
一般地,对于\(\sin^{2k+1} x \cos^n x\)或\(\sin^n x \cos^{2k+1} x\)
(其中\(k\in\mathbb{N}\))型函数的积分,
总可依次作变换\(u = \cos x\)或\(u = \sin x\),求得结果.
\end{remark}

\begin{example}
%@see: 《高等数学(第六版 上册)》 P197 例13
求\(\int \tan x \dd{x}\).
\begin{solution}
由\cref{equation:三角函数.正切与正余弦的关系}
有\(\tan x = \frac{\sin x}{\cos x}\),
所以\begin{align*}
	\int \tan x \dd{x}
	&= \int \frac{\sin x}{\cos x} \dd{x}
	= - \int \frac{1}{\cos x} \dd(\cos x) \\
	&= - \ln\abs{\cos x} + C.
\end{align*}
\end{solution}
\end{example}

%@see: 《高等数学(第六版 上册)》 P198
类似地可得\(\int \cot x \dd{x} = \ln\abs{\sin x} + C\).

\begin{example}
%@see: 《高等数学(第六版 上册)》 P198 例14
求\(\int \cos^2 x \dd{x}\).
\begin{solution}
\(\begin{aligned}[t]
	\int \cos^2 x \dd{x}
	&= \int \frac{1 + \cos 2x}{2} \dd{x}
	= \frac{1}{2} \left( \int \dd{x} + \int \cos 2x \dd{x} \right) \\
	&= \frac{1}{2} \int \dd{x} + \frac{1}{4} \int \cos 2x \dd(2x) \\
	&= \frac{x}{2} + \frac{\sin 2x}{4} + C.
\end{aligned}\)
\end{solution}
\end{example}

\begin{example}
%@see: 《高等数学(第六版 上册)》 P198 例15
求\(\int \sin^2 x \cos^4 x \dd{x}\).
\begin{solution}
\(\begin{aligned}[t]
	\int \sin^2 x \cos^4 x \dd{x}
	&= \frac{1}{8} \int (1 - \cos 2x) (1 + \cos 2x)^2 \dd{x} \\
	&= \frac{1}{8} \int (1 + \cos 2x - \cos^2 2x - \cos^3 2x) \dd{x} \\
	&= \frac{1}{8} \int (\cos 2x - \cos^3 2x) \dd{x}
		+ \frac{1}{8} \int (1 - \cos^2 2x) \dd{x} \\
	&= \frac{1}{8} \int \sin^2 2x \frac{1}{2} \dd(\sin 2x)
		+ \frac{1}{8} \int \frac{1}{2} (1 - \cos 4x) \dd{x} \\
	&= \frac{1}{48} \sin^3 2x + \frac{1}{16} x - \frac{1}{64} \sin 4x + C.
\end{aligned}\)
\end{solution}
\end{example}

\begin{remark}
%@see: 《高等数学(第六版 上册)》 P198
一般地,对于\(\sin^{2k} x \cos^{2l} x\ (k,l\in\mathbb{N})\)型函数,
总可利用\hyperref[equation:三角函数.余弦的二倍角公式1]{三角恒等式}\begin{equation*}
	\sin^2 x = \frac{1}{2} (1 - \cos 2x)
	\quad\text{和}\quad
	\cos^2 x = \frac{1}{2} (1 + \cos 2x)
\end{equation*}把积分函数化为\(\cos 2x\)的多项式.
\end{remark}

\begin{example}
%@see: 《高等数学(第六版 上册)》 P198 例16
求\(\int \sec^6x \dd{x}\).
\begin{solution}
\(\begin{aligned}[t]
	\int \sec^6x \dd{x}
	&= \int (\sec^2x)^2 \sec^2x \dd{x} \\
	&= \int (1+\tan^2x)^2 \dd(\tan x) \\
	&= \int (1+2\tan^2x+\tan^4x) \dd(\tan x) \\
	&= \tan x + \frac23 \tan^3x + \frac15 \tan^5x + C.
\end{aligned}\)
\end{solution}
\end{example}

\begin{remark}
一般的,对于\(\tan^n x \sec^{2k} x\)
或\(\tan^{2k+1} x \sec^n x\ (k \in \mathbb{N}^+)\)型函数的积分,
可依次作变换\(u=\tan x\)或\(u=\sec x\),
利用\hyperref[equation:三角函数.毕达哥拉斯三角恒等式2]{三角恒等式}
\(\sec^2 x = \tan^2 x + 1\)
和微分公式\(\dd(\tan x) = \sec^2 x \dd{x}\),
\(\dd(\sec x) = \sec x \tan x \dd{x}\),
求得结果.
\end{remark}

\begin{example}
%@see: 《高等数学(第六版 上册)》 P199 例18
求\(\int \csc x \dd{x}\).
\begin{solution}
直接计算得
\begin{align*}
	\int \csc x \dd{x}
	&= \int \frac{\dd{x}}{\sin x}
	= \int \frac{\dd{x}}{2 \sin\frac{x}{2} \cos\frac{x}{2}} \\
	&= \int \frac{\dd(\frac{x}{2})}{\tan\frac{x}{2} \cos^2\frac{x}{2}}
	= \int \frac{\dd(\tan\frac{x}{2})}{\tan\frac{x}{2}} \\
	&= \ln\abs{\tan\frac{x}{2}} + C.
\end{align*}
\end{solution}
\end{example}

\begin{example}
%@see: 《高等数学(第六版 上册)》 P199 例19
求\(\int \sec x \dd{x}\).
\begin{solution}
直接计算得\begin{align*}
	\int \sec x \dd{x}
	&= \int \csc\left(x+\frac\pi2\right) \dd(x+\frac\pi2) \\
	&= \ln\abs{\csc\left(x+\frac\pi2\right)-\cot\left(x+\frac\pi2\right)} + C \\
	&= \ln\abs{\sec x+\tan x} + C.
\end{align*}
\end{solution}
\end{example}
\begin{remark}
不定积分\(\int \sec x \dd{x}\)还有其他计算方式:
\begin{itemize}
	\item \(\begin{aligned}[t]
		\int \sec x \dd{x}
		= \int \frac{\sec x (\sec x + \tan x)}{\sec x + \tan x} \dd{x}
		= \int \frac{\dd(\sec x + \tan x)}{\sec x + \tan x}
		= \ln\abs{\sec x + \tan x} + C.
	\end{aligned}\)

	\item \(\begin{aligned}[t]
		\int \sec x \dd{x}
		= \int \frac{\dd{x}}{\cos x}
		= \int \frac{\cos x \dd{x}}{\cos^2x}
		= \int \frac{\dd(\sin x)}{1 - \sin^2x}
		= \frac12 \ln\abs{\frac{1+\sin x}{1-\sin x}} + C.
	\end{aligned}\)
\end{itemize}
\end{remark}

\subsection{第二类换元法}
\begin{theorem}
%@see: 《高等数学(第六版 上册)》 P201 定理2
设\(x = \psi(t)\)是单调的、可导的函数,
并且\(\psi'(t) \neq 0\).又设\(f[\psi(t)] \psi'(t)\)具有原函数,
则有换元公式\begin{equation*}
	\int f(x) \dd{x}
	= \left[ \int f[\psi(t)] \psi'(t) \dd{t} \right]_{t=\psi^{-1}(x)},
\end{equation*}
其中\(t=\psi^{-1}(x)\)是\(x=\psi(t)\)的反函数.
\begin{proof}
设\(f[\psi(t)] \psi'(t)\)的原函数为\(\Phi(t)\),
记\(\Phi[\psi^{-1}(x)] = F(x)\),
利用复合函数及反函数的求导法则,
得到\begin{equation*}
	F'(x) = \dv{\Phi}{t} \cdot \dv{t}{x}
	= f[\psi(t)] \psi'(t) \cdot \frac{1}{\psi'(t)}
	= f[\psi(t)] = f(x),
\end{equation*}
即\(F(x)\)是\(f(x)\)的原函数,
所以有\begin{align*}
	\int f(x) \dd{x} &= F(x) + C
	= \Phi[\psi^{-1}(x)] + C \\
	&= \left[ \int f[\psi(t)] \psi'(t) \dd{t} \right]_{t=\psi^{-1}(x)}.
	\qedhere
\end{align*}
\end{proof}
\end{theorem}

\begin{example}
%@see: 《高等数学(第六版 上册)》 P201 例21
求\(\int \sqrt{a^2 - x^2} \dd{x}\).
\begin{solution}
求这个积分的困难在于根式\(\sqrt{a^2-x^2}\),
但我们可以利用\hyperref[equation:三角函数.毕达哥拉斯三角恒等式1]{三角恒等式}\begin{equation*}
	\sin^2 t + \cos^2 t = 1
\end{equation*}来化去根式.
设\(x = a \sin t\ (-\frac\pi2 < x < \frac\pi2)\),
那么\begin{equation*}
	\sqrt{a^2 - x^2}
	= \sqrt{a^2 - a^2 \sin^2 t}
	= a \cos t,
\end{equation*}\begin{equation*}
	\dd{x}
	= a \cos t \dd{t},
\end{equation*}
于是根式化成了三角式,
所求积分化为\begin{align*}
	\int \sqrt{a^2 - x^2} \dd{x}
	&= \int a \cos t \cdot a \cos t \dd{t}
	= a^2 \int \cos^2 t \dd{t} \\
	&= a^2 \left( \frac{t}{2} + \frac{\sin 2t}{4} \right) + C \\
	&= \frac{1}{2} a^2 t + \frac{1}{2} a^2 \sin t \cos t + C.
\end{align*}
因为\(t = \arcsin\frac{x}{a}\),
\(\cos t
= \sqrt{1-\sin^2 t}
= \sqrt{1-\left(\frac{x}{a}\right)^2}
= \frac{\sqrt{a^2-x^2}}{a}\),
于是所求积分为\begin{equation*}
	\int \sqrt{a^2 - x^2} \dd{x}
	= \frac{1}{2} a^2 \arcsin\frac{x}{a} + \frac{1}{2} x \sqrt{a^2 - x^2} + C.
\end{equation*}
\end{solution}
\end{example}

\begin{example}
%@see: 《高等数学(第六版 上册)》 P202 例22
求\(\int \frac{\dd{x}}{\sqrt{x^2 + a^2}}\ (a>0)\).
\begin{solution}
和上例类似,这里我们可以利用\hyperref[equation:三角函数.毕达哥拉斯三角恒等式2]{三角恒等式}\begin{equation*}
	\tan^2 t + 1 = \sec^2 t
\end{equation*}来化去根式.
注意到被积函数的定义域为\((-\infty,+\infty)\),
故可设\(x = a \tan t\ (-\frac\pi2 < t < \frac\pi2)\),
这时\(\sec t > 0\),
那么\begin{equation*}
	\sqrt{x^2 + a^2}
	= \sqrt{a^2 \tan^2 t + a^2}
	= a \sqrt{\tan^2 t + 1}
	= a \sec t,
\end{equation*}\begin{equation*}
	\dd{x} = a \sec^2 t \dd{t},
\end{equation*}
于是\begin{equation*}
	\int \frac{\dd{x}}{\sqrt{x^2 + a^2}}
	= \int \frac{a \sec^2 t}{a \sec t} \dd{t}
	= \int \sec t \dd{t}
	= \ln\abs{\sec t + \tan t} + C_1.
\end{equation*}
将\(\tan t = \frac{x}{a}\)和\(\sec t = \frac{\sqrt{x^2+a^2}}{a}\)代回上式,
考虑到当\(-\frac\pi2 < t < \frac\pi2\)时有\(\sec t + \tan t > 0\),
所以\begin{align}
	\int \frac{\dd{x}}{\sqrt{x^2 + a^2}}
	&= \ln( \frac{x}{a} + \frac{\sqrt{x^2 + a^2}}{a} ) + C_1 \notag \\
	&= \ln(x + \sqrt{x^2 + a^2}) + C,
\end{align}
其中\(C = C_1 - \ln a\).
\end{solution}
\end{example}
\begin{remark}
%@see: 《高等数学(第六版 上册)》 P204
不定积分\(\int \frac{\dd{x}}{\sqrt{x^2 + a^2}}\ (a>0)\)还有其他计算方式:
设\(x = a \sinh t\),
那么\(\sqrt{x^2+a^2}
=\sqrt{a^2\sinh^2t+a^2}
=a\cosh t\),
\(\dd{x}=a\cosh t\dd{t}\),
于是\begin{align*}
	\int\frac{\dd{x}}{\sqrt{x^2+a^2}}
	&= \int\frac{a\cosh t}{a\cosh t}\dd{t}
	= \int\dd{t}
	= t+C_1 \\
	&= \arsinh\frac{x}{a} + C_1
	= \ln\left[\frac{x}{a}+\sqrt{\left(\frac{x}{a}\right)^2+1}\right] + C_1 \\
	&= \ln(x+\sqrt{x^2+a^2})+C,
\end{align*}
其中\(C=C_1-\ln a\).
\end{remark}

\begin{example}
%@see: 《高等数学(第六版 上册)》 P202 例23
求\(\int \frac{\dd{x}}{\sqrt{x^2 - a^2}}\ (a>0)\).
\begin{solution}
首先要注意被积函数的定义域是\(\Set{ x \given x^2-a^2\geq0 }=(-\infty,-a)\cup(a,+\infty)\),
故须分别在这两个区间求解不定积分.

\begin{itemize}
	\item 当\(x > a\)时,
	设\(x = a \sec t\ (0 < t < \frac\pi2)\),
	那么\begin{equation*}
		\sqrt{x^2 - a^2} = \sqrt{a^2 \sec^2 t - a^2} = a \sqrt{\sec^2 t - 1} = a \tan t,
	\end{equation*}\begin{equation*}
		\dd{x} = a \sec t \tan t \dd{t},
	\end{equation*}
	于是\begin{equation*}
		\int \frac{\dd{x}}{\sqrt{x^2 - a^2}}
		= \int \frac{a \sec t \tan t \dd{t}}{a \tan t}
		= \int \sec t \dd{t}
		= \ln\abs{\sec t + \tan t} + C_3.
	\end{equation*}
	将\(\sec t = \frac{x}{a}\)和\(\tan t = \frac{\sqrt{x^2-a^2}}{a}\)代回上式,
	考虑到当\(0<t<\frac\pi2\)时有\(\sec t + \tan t > 0\),
	所以\begin{equation*}
		\int \frac{\dd{x}}{\sqrt{x^2 - a^2}}
		= \ln( \frac{x}{a} + \frac{\sqrt{x^2 - a^2}}{a} ) + C_3
		= \ln( x + \sqrt{x^2 - a^2} ) + C_1,
	\end{equation*}
	其中\(C_1 = C_3 - \ln a\).

	\item 当\(x < -a\)时,
	令\(x = -u\),
	可得\(u > a\),
	如此便把问题归结为上述第一种情形:\begin{align*}
		\int \frac{\dd{x}}{\sqrt{x^2 - a^2}}
		&= -\int \frac{\dd{u}}{\sqrt{u^2 - a^2}}
		= -\ln(u + \sqrt{u^2 - a^2}) + C_4 \\
		&= -\ln(-x + \sqrt{x^2 - a^2}) + C_4
		= \ln\frac1{-x + \sqrt{x^2 - a^2}} + C_4 \\
		&= \ln\frac{-x - \sqrt{x^2 - a^2}}{a^2} + C_4
		= \ln(-x - \sqrt{x^2 - a^2}) + C_2,
	\end{align*}
	其中\(C_2 = C_4 - 2 \ln a\).
\end{itemize}
综上所述,\begin{equation}
	\int \frac{\dd{x}}{\sqrt{x^2 - a^2}}
	= \left\{ \def\arraystretch{1.2} \begin{array}{cl}
		\ln(x+\sqrt{x^2-a^2}) + C_1, & x>a, \\
		\ln(-x-\sqrt{x^2-a^2}) + C_2, & x<-a.
	\end{array} \right.
\end{equation}
为了方便记忆,我们通常把上面的分段函数简记为\begin{equation}
	\int \frac{\dd{x}}{\sqrt{x^2 - a^2}}
	= \ln\abs{x + \sqrt{x^2 - a^2}} + C.
\end{equation}
\end{solution}
\end{example}

\begin{remark}
%@see: 《高等数学(第六版 上册)》 P202
如果被积函数含有\(\sqrt{a^2 - x^2}\),可以作代换\(x = a \sin t\)化去根式;
如果被积函数含有\(\sqrt{x^2 + a^2}\),可以作代换\(x=a \tan t\)化去根式;
如果被积函数含有\(\sqrt{x^2 - a^2}\),可以作代换\(x=\pm a \sec t\)化去根式.
\end{remark}

\begin{example}
求\(\int \frac{\sqrt{a^2 - x^2}}{x^4} \dd{x}\).
\begin{solution}
令\(x = \frac{1}{t}\),
那么\(\dd{x} = -\frac{\dd{t}}{t^2}\),
于是\begin{align*}
	\int \frac{\sqrt{a^2 - x^2}}{x^4} \dd{x}
	&= \int \frac{\sqrt{a^2 - \frac{1}{t^2}}}{\frac{1}{t^4}}
		\left( -\frac{\dd{t}}{t^2} \right) \\
	&= -\int t^2 \sqrt{a^2 - \frac{1}{t^2}} \dd{t} \\
	&= -\int \abs{t} \sqrt{a^2 t^2 - 1} \dd{t}.
\end{align*}
当\(x > 0\)时,\(t > 0\),
那么\begin{align*}
	\int \frac{\sqrt{a^2 - x^2}}{x^4} \dd{x}
	&= -\frac{1}{2a^2} \int \sqrt{a^2 t^2 - 1} \dd(a^2 t^2 - 1) \\
	&= -\frac{(a^2 t^2 - 1)^{\frac32}}{3 a^2} + C \\
	&= -\frac{(a^2 - x^2)^{\frac32}}{3 a^2 x^3} + C,
\end{align*}
同理,当\(x < 0\)时,有相同的结果.
\end{solution}
\end{example}

\section{分部积分法}
\begin{theorem}[分部积分公式]
设函数\(u=u(x)\)及\(v=v(x)\)具有连续导数,那么\begin{equation*}
	\int u \dd{v} = uv - \int u' \dd{v}.
\end{equation*}
\begin{proof}
因为函数\(u=u(x)\)及\(v=v(x)\)具有连续导数,
那么两个函数乘积的导数公式为\begin{equation*}
	(uv)' = u'v + uv',
\end{equation*}
移项,得\begin{equation*}
	uv' = (uv)' - u'v.
\end{equation*}

对这个等式两边求不定积分,得\begin{equation*}
	\int u v' \dd{x} = \int (uv)' \dd{x} - \int u' v \dd{x}
	= uv - \int u' v \dd{x}.
	\qedhere
\end{equation*}
\end{proof}
\end{theorem}
如果直接求\(\int u v' \dd{x}\)有困难,
而求\(\int u' v \dd{x}\)时比较容易时,
分部积分公式就可以发挥作用了.

\begin{example}
求\(\int x \cos x \dd{x}\).
\begin{solution}
设\(u = x, \dd{v} = \cos x \dd{x}\),
那么\(\dd{u} = \dd{x}, v = \sin x\),得\begin{equation*}
	\int x \cos x \dd{x}
	= x \sin x - \int \sin x \dd{x},
\end{equation*}
而\(\int v \dd{u} = \int \sin x \dd{x}\)容易积出,所以\begin{equation*}
	\int x \cos x \dd{x}
	= x \sin x + \cos x + C.
\end{equation*}

求这个积分的时候,如果设\(u = \cos x, \dd{v} = x \dd{x}\),那么\begin{equation*}
\dd{u} = -\sin x \dd{x}, \qquad v = \frac{x^2}{2}.
\end{equation*}于是\begin{equation*}
\int x \cos x \dd{x} = \frac{x^2}{2} \cos x + \int \frac{x^2}{2} \sin x \dd{x}.
\end{equation*}上式右端的积分比原积分更不容易求出.
\end{solution}
\end{example}
由此可见,如果\(u\)和\(\dd{v}\)选取不当,就求不出结果,所以应用分部积分法时,恰当选取\(u\)和\(\dd{v}\)是一个关键.
选取\(u\)和\(\dd{v}\)一般要考虑下面两点:\begin{enumerate}
\item \(v\)要容易求得;
\item \(\int v \dd{u}\)要比\(\int u \dd{v}\)容易积出.
\end{enumerate}

\begin{example}
求\(\int x e^x \dd{x}\).
\begin{solution}
设\(u = x\),\(\dd{v} = e^x \dd{x}\),那么\(\dd{u} = \dd{x}\),\(v = e^x\),于是\begin{equation*}
\int x e^x \dd{x}
= \int x \dd(e^x)
= x e^x - \int e^x \dd{x}
= x e^x - e^x + C
= e^x (x - 1) + C.
\end{equation*}
\end{solution}
\end{example}

\begin{example}
求\(\int x^2 e^x \dd{x}\).
\begin{solution}
设\(u = x^2\),\(\dd{v} = e^x \dd{x}\),那么\begin{equation*}
\int x^2 e^x \dd{x}
= \int x^2 \dd(e^x)
= x^2 e^x - \int e^x \dd{x^2}
= x^2 e^x - 2 \int x e^x \dd{x}.
\end{equation*}

这里\(\int x e^x \dd{x}\)比\(\int x^2 e^x \dd{x}\)更容易积出,因为被积函数中\(x\)的幂次前者比后者降低了一次.由上例可知,对\(\int x e^x \dd{x}\)再使用一次分部积分就可以了,于是\begin{equation*}
\int x^2 e^x \dd{x} = e^x (x^2 -2x + 2) + C.
\end{equation*}
\end{solution}
\end{example}

\begin{example}
求\(\int x \sin a x \dd{x}\).
\begin{solution}
直接计算得\begin{align*}
	\int x \sin a x \dd{x}
	&= -\frac1a \int x \dd(\cos a x)
	= -\frac1a (x \cos a x - \int \cos a x \dd{x}) \\
	&= -\frac1a x \cos a x + \frac1{a^2} \sin a x + C.
\end{align*}
\end{solution}
%@Mathematica: Integrate[x Sin[a x], x]
\end{example}

\begin{example}
求\(\int x \cos a x \dd{x}\).
\begin{solution}
直接计算得\begin{align*}
	\int x \cos a x \dd{x}
	&= \frac1a \int x \dd(\sin a x)
	= \frac1a (x \sin a x - \int \sin a x \dd{x}) \\
	&= \frac1a x \sin a x + \frac1{a^2} \cos a x.
\end{align*}
\end{solution}
%@Mathematica: Integrate[x Cos[a x], x]
\end{example}

\begin{example}
求\(\int x \ln x \dd{x}\).
\begin{solution}
设\(u=\ln x\),\(\dd{v} = x \dd{x}\),那么\begin{align*}
\int x \ln x \dd{x}
&= \int \ln x \dd(\frac{x^2}{2})
= \frac{x^2}{2} \ln x - \int \frac{x^2}{2} \dd(\ln x) \\
&= \frac{x^2}{2} \ln x - \frac{1}{2} \int x \dd{x}
= \frac{x^2}{2} \ln x - \frac{x^2}{4} + C.
\end{align*}
\end{solution}
\end{example}

\begin{example}
求\(\int \arccos x \dd{x}\).
\begin{solution}
设\(u = \arccos x\),\(\dd{v} = \dd{x}\),那么\begin{equation*}
\int \arccos x \dd{x} = x \arccos x - \int x \dd(\arccos x),
\end{equation*}其中\begin{align*}
\int x \dd(\arccos x)
&= -\int \frac{x}{\sqrt{1-x^2}} \dd{x}
= \frac{1}{2} \int \frac{\dd(1-x^2)}{(1-x^2)^{1/2}} \\
&= \sqrt{1-x^2} + C,
\end{align*}所以\begin{equation*}
\int \arccos x \dd{x} = x \arccos x - \sqrt{1-x^2} + C.
\end{equation*}
\end{solution}
\end{example}

\begin{example}
求\(\int x \arctan x \dd{x}\).
\begin{solution}
设\(u = \arctan x\),\(\dd{v} = x \dd{x}\),那么\begin{equation*}
\int x \arctan x \dd{x}
= \frac{1}{2} \int \arctan x \dd(x^2)
= \frac{1}{2} \left( x^2 \arctan x
	- \int \frac{x^2}{1+x^2} \dd{x} \right),
\end{equation*}其中\begin{equation*}
\int \frac{x^2}{1+x^2} \dd{x}
= \int \left(1-\frac{1}{1+x^2}\right) \dd{x}
= x - \arctan x + C,
\end{equation*}所以\begin{align*}
\int x \arctan x \dd{x}
&= \frac{1}{2} \left[ x^2 \arctan x
	- (x - \arctan x + C) \right] \\
&= \frac{1}{2} (x^2+1) \arctan x - \frac{1}{2} x + C_1.
\end{align*}
\end{solution}
\end{example}

\begin{example}
计算\(I_1 = \int e^x \cos x\dd{x}\)和\(I_2 = \int e^x \sin x\dd{x}\).
\begin{solution}
因为\begin{align*}
I_1 &= \int e^x \cos x\dd{x}
= \int e^x \dd(\sin x) \\
&= e^x \sin x - \int \sin x \dd(e^x)
= e^x \sin x - I_2, \\
I_2 &= \int e^x \sin x\dd{x}
= -\int e^x \dd(\cos x) \\
&= -\left[ e^x \cos x - \int \cos x \dd(e^x) \right]
= I_1 - e^x \cos x.
\end{align*}解得\begin{equation*}
I_1 = \frac{1}{2} e^x (\sin x + \cos x) + C,
\qquad
I_2 = \frac{1}{2} e^x (\sin x - \cos x) + C.
\end{equation*}
\end{solution}
\end{example}

\begin{example}
求\(\int \sec^3 x \dd{x}\).
\begin{solution}
由题有\begin{align*}
\int \sec^3 x \dd{x}
&= \int \sec x \dd(\tan x) \\
&= \sec x \tan x - \int \sec x \tan^2 x \dd{x} \\
&= \sec x \tan x - \int \sec x (\sec^2 x - 1) \dd{x} \\
&= \sec x \tan x - \int \sec^3 x \dd{x} + \int \sec x \dd{x} \\
&= \sec x \tan x + \ln\abs{\sec x + \tan x} - \int \sec^3 x \dd{x},
\end{align*}解得\begin{equation*}
\int \sec^3 x \dd{x}
= \frac{1}{2} \left(
	\sec x \tan x
	+ \ln\abs{\sec x + \tan x}
\right) + C.
\end{equation*}
\end{solution}
\end{example}

\begin{example}
求\(\int e^{\sqrt{x}} \dd{x}\).
\begin{solution}
令\(t = \sqrt{x}\),则\(x = t^2\),\(\dd{x} = 2t\dd{t}\),于是\begin{equation*}
\int e^{\sqrt{x}} \dd{x}
= 2 \int t e^t \dd{t}
= 2 e^t (t-1) + C
= 2 e^{\sqrt{x}} (\sqrt{x}-1) + C.
\end{equation*}
\end{solution}
\end{example}

\begin{example}
计算\(I_1 = \int \sin{\ln{x}} \dd{x}\)和\(I_2 = \int \cos{\ln{x}} \dd{x}\).
\begin{solution}
\begin{align*}
I_1
&= \int \sin{\ln x}\dd{x}
\xlongequal{u = \ln x} \int \sin u \dd(e^u) \\
&= \frac{1}{2} e^u (\sin u - \cos u) + C
= \frac{1}{2} x (\sin{\ln x} - \cos{\ln x}) + C, \\
I_2
&= \int \cos{\ln x}\dd{x}
\xlongequal{u = \ln x} \int \cos u \dd(e^u) \\
&= \frac{1}{2} e^u (\sin u + \cos u) + C
= \frac{1}{2} x (\sin{\ln x} + \cos{\ln x}) + C.
\end{align*}
\end{solution}
\end{example}

\begin{example}
求\(\displaystyle\int \frac{x \cos x}{(x + \cos x)^2}\dd{x}\).
\begin{solution}
因为\begin{align*}
\int \frac{x \cos x}{(x + \cos x)^2}\dd{x}
&= \int \frac{x \cos x}{(x + \cos x)^2} \frac{1 - \sin x}{1 - \sin x}\dd{x} \\
&= \int \frac{x \cos x}{(x + \cos x)^2} \frac{1}{1 - \sin x} \dd(x + \cos x) \\
&= -\int \frac{x \cos x}{1 - \sin x} \dd(\frac{1}{x + \cos x}) \\
&= -\frac{x \cos x}{1 - \sin x} \frac{1}{x + \cos x}
	+\int \frac{1}{x + \cos x} \dd(\frac{x \cos x}{1 - \sin x}),
\end{align*}
其中
\begin{align*}
&\int \frac{1}{x + \cos x} \dd(\frac{x \cos x}{1 - \sin x}) \\
&\qquad= \int \frac{1}{x + \cos x}
	\frac{(\cos x  - x \sin x)(1 - \sin x) - x \cos x (-\cos x)}{(1 - \sin x)^2}\dd{x} \\
&\qquad= \int \frac{1}{x + \cos x}
	\frac{\cos x - \sin x \cos x - x \sin x + x \sin^2 x + x \cos^2 x}{(1 - \sin x)^2}\dd{x} \\
&\qquad= \int \frac{1}{x + \cos x}
	\frac{\cos x - \sin x \cos x - x \sin x + x}{(1 - \sin x)^2}\dd{x} \\
&\qquad= \int \frac{1}{x + \cos x}
	\frac{(\cos x + x)(1 - \sin x)}{(1 - \sin x)^2}\dd{x} \\
&\qquad= \int \frac{1}{1 - \sin x}\dd{x} \\
&\qquad\xlongequal{u=\tan(x/2)}
	\int \frac{1}{1 - \frac{2u}{u^2 + 1}} \frac{2\dd{u}}{u^2 + 1}
= \int \frac{u^2 + 1}{(u - 1)^2} \frac{2\dd{u}}{u^2 + 1} \\
&\qquad= \int \frac{2}{(u - 1)^2}\dd{u}
\xlongequal{v=u-1} 2 \int v^{-2}\dd{v}
= 2 \cdot (-1) v^{-1} + C \\
&\qquad= \frac{2}{-v} + C
= \frac{2}{1 - u} + C
= \frac{2}{1 - \tan(x/2)} + C,
\end{align*}
所以\begin{equation*}
\int \frac{x \cos x}{(x + \cos x)^2}\dd{x}
= -\frac{x \cos x}{1 - \sin x} \frac{1}{x + \cos x}
	+\frac{2}{1 - \tan(x/2)} + C.
\end{equation*}
\end{solution}
\end{example}

\section{有理函数的积分}
两个(不含公因式的)多项式\(P(x)\)和\(Q(x)\)的商\(\frac{P(x)}{Q(x)}\)
称为\DefineConcept{有理函数}(rational function),
又称\DefineConcept{有理分式}.
%@see: https://mathworld.wolfram.com/PolynomialQuotient.html
%@see: https://mathworld.wolfram.com/RationalFunction.html
%@Mathematica: PolynomialQuotient[x^4 + x^3 + x^2 + x + 1, x^2 - 1, x]
%@Mathematica: PolynomialQuotientRemainder[x^4 + x^3 + x^2 + x + 1, x^2 - 1, x]
当分子多项式\(P(x)\)的次数小于分母多项式\(Q(x)\)的次数时,
称这有理函数为\DefineConcept{真分式};
否则称为\DefineConcept{假分式}.

利用多项式的除法,总可以将一个假分式化为一个多项式与一个真分式之和的形式.
例如\[
	\frac{2x^4+x^2+3}{x^2+1}
	= 2x^2-1+\frac{4}{x^2+1}.
%@Mathematica: Apart[(2 x^4 + x^2 + 3)/(x^2 + 1)]
\]

对于真分式\(\frac{P(x)}{Q(x)}\),
如果分母可分解为两个多项式的乘积\[
	Q(x) = Q_1(x)~Q_2(x),
\]
且\(Q_1(x)\)与\(Q_2(x)\)没有公因式,
那么它可分拆成两个真分式之和\[
	\frac{P(x)}{Q(x)} = \frac{P_1(x)}{Q_1(x)} + \frac{P_2(x)}{Q_2(x)},
\]
我们把上述
将真分式化为\DefineConcept{部分分式}(partial fraction)之和的
步骤称为\DefineConcept{部分分式分解}(partial fraction decomposition).
如果部分分式的分母还能再分解成两个没有公因式的多项式的乘积,
那么就可再分拆成更简单的部分分式.
最后,有理函数的分解式中只出现三类函数:
\begin{enumerate}
	\item 多项式\(F(x)\);
	\item 部分分式\(\frac{P_1(x)}{(x-a)^k}\);
	\item 部分分式\(\frac{P_2(x)}{(x^2+px+q)^l}\),
\end{enumerate}
其中\(p^2-4q<0\),
\(P_1(x)\)为小于\(k\)次的多项式,
\(P_2(x)\)为小于\(2l\)次的多项式.

对于部分分式\(\frac{P_1(x)}{(x-a)^k}\),
我们总可运用以下公式:\[
	\int \frac{\dd{x}}{x-a} = \ln\abs{x-a} + C,
\]\[
	\int \frac{\dd{x}}{(x-a)^k} = \frac{(x-a)^{1-k}}{1-k} + C,
	\quad k>1.
\]

现在我们研究如何计算\[
	\int \frac{Ax+B}{(x^2+px+q)^l} \dd{x}.
\]
经过配方,得\[
	x^2+px+q = \left(x+\frac{p}{2}\right)^2 + q-\frac{p^2}{4}.
\]
令\(a^2=q-p^2/4\),利用\(u=x+p/2\)换元,得\[
	\int \frac{Ax+B}{(x^2+px+q)^l} \dd{x}
	= A \int \frac{u \dd{u}}{(a^2+u^2)^l}
	+ \left(B - \frac{Ap}{2}\right) \int \frac{\dd{u}}{(a^2+u^2)^l}.
\]

我们先求上式右边第一个不定积分\[
	J_l = \int \frac{u \dd{u}}{(a^2+u^2)^l}.
\]
当\(l=1\)时,\[
	J_1
	= \int \frac{u \dd{u}}{a^2+u^2}
	= \frac{1}{2} \ln(a^2+u^2) + C;
\]
当\(l>1\)时,\[
	J_l
	= \int \frac{u \dd{u}}{(a^2+u^2)^l}
	= \frac{1}{2(1-l)} (a^2+u^2)^{1-l} + C.
\]

我们再求右边第二个不定积分\[
	I_l = \int \frac{1}{(a^2+u^2)^l} \dd{u}.
\]
作分部积分,得\begin{align*}
	I_l &= \frac{u}{(a^2+u^2)^l} + 2l \int \frac{u^2}{(a^2+u^2)^{l+1}} \dd{u} \\
	&= \frac{u}{(a^2+u^2)^l} + 2l \int \frac{a^2+u^2-a^2}{(a^2+u^2)^{l+1}} \dd{u} \\
	&= \frac{u}{(a^2+u^2)^l} + 2l I_l - 2la^2 I_{l+1},
\end{align*}
由此推出\[
	I_{l+1} = \frac{1}{2la^2} \frac{u}{(a^2+u^2)^l} + \frac{2k-1}{2ka^2} I_l.
\]
这是一个递推公式.
反复利用这个公式可以把指标\(l\)降低,
最后归结为已知的不定积分.
最初的几个是\begin{align*}
	I_1 &= \int \frac{1}{a^2+u^2} \dd{u} = \frac{1}{a} \arctan\frac{u}{a} + C, \\
	I_2 &= \frac{1}{2a^2} \left( \frac{u}{a^2+u^2} + I_1 \right), \\
	I_3 &= \frac{1}{4a^2} \left[ \frac{u}{(a^2+u^2)^2} + 3 I_2 \right].
\end{align*}

\begin{example}
求\(\int \frac{x+1}{x^2-5x+6} \dd{x}\).
\begin{solution}
被积函数的分母可分解成\((x-3)(x-2)\),故可设\[
	\frac{x+1}{x^2-5x+6}
	= \frac{A}{x-3} + \frac{B}{x-2},
\]
其中\(A,B\)为待定系数.上式两端通分后,得\[
	x+1 = A(x-2)+B(x-3),
\]
即\[
	x+1 = (A+B)x -(2A+3B).
\]
比较上式两端同次幂的系数,即有\[
	\left\{ \begin{array}{l}
		A+B = 1, \\
		2A+3B = -1,
	\end{array} \right.
\]
从而解得\(A=4\),\(B=-3\).
于是\[
	\int \frac{x+1}{x^2-5x+6} \dd{x}
	= \int \left(\frac{4}{x-3} - \frac{3}{x-2}\right) \dd{x}
	= 4\ln\abs{x-3} - 3\ln\abs{x-2} + C.
%@Mathematica: Apart[(x + 1)/(x^2 - 5 x + 6)]
%@Mathematica: Integrate[(x + 1)/(x^2 - 5 x + 6), x]
\]
\end{solution}
\end{example}

\begin{example}
计算不定积分\(\int \frac1{(x+1)(x^2-2x+2)} \dd{x}\).
\begin{solution}
设\begin{align*}
	\frac1{(x+1)(x^2-2x+2)}
	&= \frac{A}{x+1} + \frac{Bx+C}{x^2-2x+2} \\
	&= \frac{(A+B)x^2+(B+C-2A)x+(2A+C)}{(x+1)(x^2-2x+2)},
\end{align*}
比较可得\begin{equation*}
	\left\{ \begin{array}{*7r}
		A &+& B && &=& 0, \\
		-2A &+& B &+& C &=& 0, \\
		2A && &+& C &=& 1.
	\end{array} \right.
\end{equation*}
解得\(A=\frac15,
B=-\frac15,
C=\frac35\).
于是\begin{equation*}
	\frac1{(x+1)(x^2-2x+2)}
	= \frac15 \left( \frac1{x+1} - \frac{x-3}{x^2-2x+2} \right).
\end{equation*}
而\begin{align*}
	\int \frac1{x+1} \dd{x}
	&= \ln(x+1) + C_1, \\
	\int \frac{x-3}{x^2-2x+2} \dd{x}
	&= \int \frac{x-1}{(x-1)^2+1} \dd{x}
	- \int \frac{2}{(x-1)^2+1} \dd{x} \\
	&= \frac12 \int \frac{\dd{[(x-1)^2+1]}}{(x-1)^2+1}
	- 2 \int \frac{\dd{(x-1)}}{(x-1)^2+1} \\
	&= \frac12 \ln(x-1)^2
	- 2 \arctan(x-1) + C_2.
\end{align*}
因此\begin{equation*}
	\int \frac1{(x+1)(x^2-2x+2)} \dd{x}
	= \frac15 \left[
		\ln(x+1) + \frac12 \ln(x-1)^2 - 2 \arctan(x-1)
	\right] + C.
\end{equation*}
\end{solution}
\end{example}

\section{本章总结}
\subsection{常见积分表}
\begin{gather*}
	% 幂函数
	\int k \dd{x}
	= kx + C \\
	\int x^\mu \dd{x}
	= \frac{x^{\mu + 1}}{\mu + 1} + C \quad (\mu \neq -1) \\
	\int \frac{\dd{x}}{x}
	= \ln\abs{x} + C \\
	% 分式函数
	\int \frac{\dd{x}}{1 + x^2}
	= \arctan x + C \\
	\int \frac{\dd{x}}{\sqrt{1 - x^2}}
	= \arcsin x + C \\
	\int \frac{\dd{x}}{a^2 + x^2}
	= \frac{1}{a} \arctan\frac{x}{a} + C \\
	\int \frac{\dd{x}}{\sqrt{a^2 - x^2}}
	= \arcsin\frac{x}{a} + C \\
	\int \frac{\dd{x}}{x^2 - a^2}
	= \frac{1}{2a} \ln\abs{\frac{x - a}{x + a}} + C \\
	\int \frac{\dd{x}}{\sqrt{x^2 + a^2}}
	= \ln(x + \sqrt{x^2 + a^2}) + C \\
	\int \frac{\dd{x}}{\sqrt{x^2 - a^2}}
	= \ln\abs{x + \sqrt{x^2 - a^2}} + C \\
	% 根式
	\int \sqrt{a^2 - x^2} \dd{x}
	= \frac12 a^2 \arcsin\frac{x}{a} + \frac12 x \sqrt{a^2 - x^2} + C \\
	% 三角函数1次
	\int \cos x \dd{x}
	= \sin x + C \\
	\int \sin x \dd{x}
	= -\cos x + C \\
	\int \tan x \dd{x}
	= -\ln\abs{\cos x} + C \\
	\int \cot x \dd{x}
	= \ln\abs{\sin x} + C \\
	\int \sec x \dd{x}
	= \ln\abs{\sec x + \tan x} + C \\
	\int \csc x \dd{x}
	= \ln\abs{\csc x - \cot x} + C \\
	% 三角函数2次
	%@Mathematica: Integrate[Sin[x]^2, x]
	\int \sin^2 x \dd{x}
	= \frac{x}2 - \frac14 \sin 2x + C \\
	%@Mathematica: Integrate[Cos[x]^2, x]
	\int \cos^2 x \dd{x}
	= \frac{x}2 + \frac14 \sin 2x + C \\
	%@Mathematica: Integrate[Tan[x]^2, x]
	\int \tan^2 x \dd{x}
	= \tan x - x + C \\
	%@Mathematica: Integrate[Cot[x]^2, x]
	\int \cot^2 x \dd{x}
	= -\cot x - x + C \\
	%@Mathematica: Integrate[Sec[x]^2, x]
	\int \sec^2 x \dd{x}
	= \tan x + C \\
	%@Mathematica: Integrate[Csc[x]^2, x]
	\int \csc^2 x \dd{x}
	= -\cot x + C \\
	%@Mathematica: Integrate[Sec[x] Tan[x], x]
	\int \sec x \tan x \dd{x}
	= \sec x + C \\
	%@Mathematica: Integrate[Csc[x] Cot[x], x]
	\int \csc x \cot x \dd{x}
	= -\csc x + C \\
	% 三角函数3次
	%@Mathematica: Integrate[Sin[x]^3, x] // TrigExpand
	\int \sin^3 x \dd{x}
	= -\cos x + \frac{1}{3} \cos^3 x + C \\
	% 正余弦函数 正交函数列
	\int \sin mx \cos nx \dd{x}
	= -\frac{\cos(m+n)x}{2(m+n)}
		- \frac{\cos(m-n)x}{2(m-n)}
		+ C \\
	\int \sin mx \sin nx \dd{x}
	= \frac{\sin(m-n)x}{2(m-n)}
		- \frac{\sin(m+n)x}{2(m+n)}
		+ C \\
	\int \cos mx \cos nx \dd{x}
	= \frac{\sin(m+n)x}{2(m+n)}
		+ \frac{\sin(m-n)x}{2(m-n)}
		+ C \\
	% 指数函数
	\int e^x \dd{x}
	= e^x + C \\
	\int a^x \dd{x}
	= \frac{a^x}{\ln a} + C \\
	% 对数函数
	%@Mathematica: Integrate[Log[x], x]
	\int \ln x \dd{x}
	= x \ln x - x + C \\
	% 双曲函数
	\int \sinh x \dd{x}
	= \cosh x + C \\
	\int \cosh x \dd{x}
	= \sinh x + C
\end{gather*}

\subsection{积分技巧}
对于积分\(\int f(ax+b) \dd{x}\),总可作变换\(u=ax+b\),把它作为\begin{equation*}
	\int f(ax+b) \dd{x}
	= \int \frac{1}{a} f(ax+b) \dd{(ax+b)} \\
	= \frac{1}{a} \left[ \int f(u) \dd{u} \right]_{u=ax+b}.
\end{equation*}

对于\(\sin^{2k+1} x \cos^n x\)
或\(\sin^n x \cos^{2k+1} x\ (k \in \mathbb{N})\)型函数的积分,
总可依次作变换\(u=\cos x\)或\(u=\sin x\),
利用恒等式\(\sin^2 x + \cos^2 x \equiv 1\)求得结果.
\begin{align*}
	\int \sin^{2k+1} x \cos^n x \dd{x}
	&= - \int \sin^{2k} x \cos^n x \cdot \sin x \dd{x} \\
	&= - \int (1-\cos^2 x)^k \cos^n x \dd(\cos x), \\
	\int \sin^n x \cos^{2k+1} x \dd{x}
	&= \int \sin^n x \cos^{2k} x \cdot \cos x \dd{x} \\
	&= \int \sin^n x (1-\sin^2 x)^k \dd(\sin x).
\end{align*}

对于\(\sin^{2k} x \cos^{2l} x\ (k,l \in \mathbb{N})\)型函数,
总可利用三角恒等式\(\sin^2 x = \frac12(1-\cos 2x)\),
\(\cos^2 x = \frac12(1+\cos 2x)\)化成\(\cos 2x\)的多项式,求得结果.
例如:\begin{equation*}
	\int \sin^{2k} x \cos^{2l} x \dd{x}
	= \int \left[\frac12(1-\cos 2x)\right]^k
		\left[\frac12(1+\cos 2x)\right]^l \dd{x}
	= \int f(\cos 2x) \dd{x}.
\end{equation*}

对于\(\tan^n x \sec^{2k} x\)
或\(\tan^{2k+1} x \sec^n x\ (k \in \mathbb{N}^+)\)型函数的积分,
可依次作变换\(u=\tan x\)或\(u=\sec x\),
利用三角恒等式\(\sec^2 x = \tan^2 x + 1\)
和微分公式\(\dd(\tan x) = \sec^2 x \dd{x}\),
\(\dd(\sec x) = \sec x \tan x \dd{x}\),
求得结果.
\begin{align*}
	\int \tan^n x \sec^{2k} x \dd{x}
	&=\int \tan^n x\sec^{2k-2} x \cdot \sec^2 x \dd{x} \\
	&=\int \tan^n x(1+\tan^2 x)^{k-1} \dd(\tan x), \\
	\int \tan^{2k+1} x \sec^n x \dd{x}
	&=\int \tan^{2k} x \sec^{n-1} x \cdot \sec x\tan x\dd{x} \\
	&=\int (\sec^2 x - 1)^k \sec^{n-1} x \dd(\sec x).
\end{align*}

如果被积函数含有\(\sqrt{a^2 - b^2 x^2}\),可以作代换\(x = \frac{a}{b} \sin t\)化去根式.

如果被积函数含有\(\sqrt{a^2 + b^2 x^2}\),可以作代换\(x = \frac{a}{b} \tan t\)化去根式.

如果被积函数含有\(\sqrt{b^2 x^2 - a^2}\),可以作代换\(x = \frac{a}{b} \sec t\)化去根式.

如果被积函数有高次多项式,可以首先利用平方差公式和平方和公式对其配方.

当被积函数含有\(\sqrt{x^2 \pm a^2}\)时,为了化去根式,
除采用三角代换\(x = a \tan t\)或\(x = \pm a \sec t\)外,
还可利用公式\(\cosh^2 t - \sinh^2 t = 1\),
采用双曲代换\(x = a \sinh t\)和\(x = \pm a \cosh t\)来化去根式.

如果被积函数是分式,
且积分变量在分子中的最高幂次比其在分母中的最高幂次要低2次或3次,
则可利用倒代换技巧,
即令\(t=\frac{1}{x}\).

如果被积函数是幂函数和正(余)弦函数或幂函数和指数函数的乘积,
就可以考虑用分部积分法,
用正(余)弦函数或指数函数凑微分,
即\begin{equation*}
	\sin x \dd{x} = -\dd(\cos x),
	\qquad
	\cos x \dd{x} = \dd(\sin x),
	\qquad
	e^x \dd{x} = \dd(e^x).
\end{equation*}
这样每用一次分部积分法就可使幂次降低一次.

如果被积函数是幂函数和对数函数或幂函数和反三角函数的乘积,
也可以考虑用分部积分法,
用幂函数凑微分,
即\begin{equation*}
	x^\mu \dd{x} = \frac1{\mu+1} \dd(x^{\mu+1}).
\end{equation*}

% 有的老师编了这样一段顺口溜:反对幂指三.

如果被积函数中含有简单根式\(\sqrt[n]{ax+b}\)或\(\sqrt[n]{\frac{ax+b}{cx+d}}\),
可以令这个简单根式为\(u\),
即\begin{align*}
	u=\sqrt[n]{ax+b} &\implies x=\frac{1}{a}(u^n-b) \\
	u=\sqrt[n]{\frac{ax+b}{cx+d}} &\implies x=\frac{u^nd-b}{a-u^nc}
\end{align*}
由于这样的变换具有反函数,且反函数是\(u\)的有理函数,
因此原积分可以化为有理函数的积分.


\chapter{定积分}
\section{定积分的概念与性质}
\subsection{定积分的概念}
\begin{definition}
%@see: 《高等数学(第六版 上册)》 P225 定义
%@see: 《数学分析(第二版 上册)》(陈纪修) P275 定义7.1.1
设函数\(f\colon[a,b]\to\mathbb{R}\)在闭区间\([a,b]\)上有界.
在\([a,b]\)中任意插入若干个分点\begin{equation*}
	a=x_0 < x_1 < x_2 < \dotsb < x_{n-1} < x_n = b,
\end{equation*}
把区间\([a,b]\)分成\(n\)个小区间\begin{equation*}
	[x_0,x_1],[x_1,x_2],\dotsc,[x_{n-1},x_n],
\end{equation*}
各个小区间的长度依次为\begin{equation*}
	\increment x_1=x_1-x_0,
	\increment x_2=x_2-x_1,
	\dotsc,
	\increment x_n=x_n-x_{n-1},
\end{equation*}
在每个小区间\([x_{i-1},x_i]\)上任取一点\(\xi_i\ (x_{i-1} \leq \xi_i \leq x_i)\),
作函数值\(f(\xi_i)\)与小区间长度\(\increment x_i\)的乘积
\(f(\xi_i)\increment x_i\ (i=1,2,\dotsc,n)\),
并求和\begin{equation*}
	S = \sum_{i=1}^n f(\xi_i) \increment x_i,
\end{equation*}
和\(S\)通常称为\DefineConcept{积分和}.
记\(\lambda=\max\{\increment x_1,\increment x_2,\dotsc,\increment x_n\}\).
如果不论对\([a,b]\)怎么划分,
也不论小区间\([x_{i-1},x_i]\)上点\(\xi_i\)怎样选取,
只要当\(\lambda\to0\)时,
和\(S\)总趋于确定的极限\(I\),
那么称这个极限\(I\)为
“函数\(f\)在区间\([a,b]\)上的\DefineConcept{定积分}”,
记作\(\int_a^b f(x) \dd{x}\),
即\begin{equation}
	\int_a^b f(x) \dd{x}
	\defeq
	\lim_{\lambda\to0} \sum_{i=1}^n f(\xi_i)\increment x_i,
\end{equation}
其中\(f(x)\)称为\DefineConcept{被积函数},
\(f(x)\dd{x}\)称为\DefineConcept{被积表达式},
\(x\)称为\DefineConcept{积分变量},
\(a\)称为\DefineConcept{积分下限},
\(b\)称为\DefineConcept{积分上限},
\([a,b]\)称为\DefineConcept{积分区间}.
我们称“函数\(f\)在\([a,b]\)上\DefineConcept{黎曼可积}(integrable)”.
\end{definition}
以上积分定义又称为\DefineConcept{黎曼积分}.

注意:
当和\(\sum_{i=1}^n f(\xi_i) \increment x_i\)的极限
\(\lim_{\lambda\to0} \sum_{i=1}^n f(\xi_i) \increment x_i\)存在时,
其极限\(I\)仅与被积函数\(f\)及积分区间\([a,b]\)有关.
如果既不改变被积函数\(f\),也不改变积分区间\([a,b]\),
而只把积分变量\(x\)改写成其他字母,则和的极限\(I\)不变,也就是定积分的值不变,
即\begin{equation*}
	\int_a^b f(x) \dd{x}
	= \int_a^b f(t) \dd{t}
	= \int_a^b f(u) \dd{u}.
\end{equation*}
换句话说,定积分的值只与被积函数及积分区间有关,而与积分变量的记法无关.

在上面对定积分的定义中,由于\([a,b]\)是区间,所以蕴含了\(a<b\)这一条件,
但是我们有时候需要计算的定积分\(\int_a^b f(x) \dd{x}\)满足\(a \geq b\).
因此,为了以后计算及应用方便起见,对定积分作以下补充规定:
交换定积分的上下限,定积分的绝对值不变而符号相反,即
\begin{equation}\label{equation:定积分.交换上下限改变定积分的符号}
	\int_a^b f(x) \dd{x}
	= - \int_b^a f(x) \dd{x}.
\end{equation}
由此我们得到一个结论:
定积分上下限相等时,定积分为零,即
\begin{equation}\label{equation:定积分.上下限相等的定积分为零}
    \int_a^a f(x) \dd{x} = 0.
\end{equation}

根据定积分定义判定定积分不存在的方法有两种:
\begin{itemize}
	\item 证明存在一种分法使得积分和极限不存在.
	\item 证明存在两种分法使得两个极限不相等.
\end{itemize}

\begin{example}
%@see: 《高等数学(第六版 上册)》 P227 例1
利用定义计算定积分\(\int_0^1 x^2 \dd{x}\).
\begin{solution}
因为被积函数\(f(x) = x^2\)在积分区间\([0,1]\)上连续,而连续函数是可积的,
所以积分与区间\([0,1]\)的分法及点\(\xi_i\)的取法无关.
因此,为了便于计算,不妨把区间\([0,1]\)分成\(n\)等份,
分点为\(x_i = i/n,\,i=1,2,\dotsc,n-1\);
这样,每个小区间\([x_{i-1},x_i]\)的长度\(\increment x_i = 1/n,\,i=1,2,\dotsc,n\);
取\(\xi_i=x_i,\,i=1,2,\dotsc,n\).
于是积分和为\begin{align*}
	\sum_{i=1}^n f(\xi_i) \increment x_i
	&= \sum_{i=1}^n \xi_i^2 \increment x_i
	= \sum_{i=1}^n \left(\frac{i}{n}\right)^2 \frac{1}{n}
	= \frac{1}{n^3} \sum_{i=1}^n i^2 \\
	&= \frac{1}{n^3} \cdot \frac{1}{6} n(n+1)(2n+1) \\
	&= \frac{1}{6} \left(1+\frac{1}{n}\right) \left(2+\frac{1}{n}\right).
\end{align*}
当\(\lambda\to0\)时,\(n\to\infty\),那么有\begin{equation*}
	\int_0^1 x^2 \dd{x}
	= \lim_{\lambda\to0} \sum_{i=1}^n f(\xi_i) \increment x_i
	= \lim_{n\to\infty}
		\frac{1}{6} \left(1+\frac{1}{n}\right) \left(2+\frac{1}{n}\right)
	= \frac{1}{3}.
\end{equation*}
\end{solution}
\end{example}

\subsection{黎曼可积条件}
不是所有的有界函数都是可积的.
考虑狄利克雷函数\begin{equation*}
	D(x) = \left\{ \begin{array}{ll}
		1, & x \in \mathbb{Q}, \\
		0, & x \in \mathbb{R}-\mathbb{Q}.
	\end{array} \right.
\end{equation*}
显然狄利克雷函数是有界的.
由于有理数和无理数在实数域中都是稠密的,
因此不论用区间\([0,1]\)的哪一个划分,在每一个小区间中一定是既有有理数又有无理数.
于是,当\(\xi_i\)全部取为有理数时,积分和等于\(1\);
而当\(\xi_i\)全部取为无理数时,积分和等于\(0\).
两种情况下的积分和的极限虽然存在但是不相同,所以狄利克雷函数不是黎曼可积的.

下面我们来探求有界函数在\([a,b]\)上黎曼可积的充分必要条件.

记\begin{equation*}
	M \defeq \sup\Set{ f(x) \given x\in[a,b] }, \qquad
	m \defeq \inf\Set{ f(x) \given x\in[a,b] }.
\end{equation*}
再记\begin{equation*}
	M_i \defeq \sup\Set{ f(x) \given x\in[x_{i-1},x_i] }, \qquad
	m_i \defeq \inf\Set{ f(x) \given x\in[x_{i-1},x_i] },
	\quad i=1,2,\dotsc,n.
\end{equation*}
设\(P\)是\([a,b]\)的一个划分.
定义:\begin{equation*}
	\overline{S}(P) \defeq \sum_{i=1}^n M_i \increment x_i, \qquad
	\underline{S}(P) \defeq \sum_{i=1}^n m_i \increment x_i.
\end{equation*}
把\(\overline{S}(P)\)与\(\underline{S}(P)\)分别称为
“相应于划分\(P\)的\DefineConcept{达布大和}”
与“相应于划分\(P\)的\DefineConcept{达布小和}”.
那么显然有\begin{equation*}
	\underline{S}(P)
	\leq \sum_{i=1}^n f(\xi_i) \increment x_i
	\leq \overline{S}(P).
\end{equation*}
如果对于任意一种划分\(P\),
当\(\lambda = \max\{\increment x_1,\dotsc,\increment x_n\} \to 0\)时,
\(\overline{S}(P)\)和\(\underline{S}(P)\)的极限都存在并且相等,
那么\(f\)是可积的,反之亦然.
下面我们来严格证明这一结论.

\begin{lemma}
%@see: 《数学分析(第二版 上册)》(陈纪修) P277 引理7.1.1
若在原有划分中加入分点形成新的划分,则大和不增,小和不减.
\end{lemma}

以下用\(\overline{\mathbb{S}}\)表示一切可能的划分所得到的大和的集合,
而\(\underline{\mathbb{S}}\)表示一切可能的划分所得到的小和的集合.
\begin{lemma}
%@see: 《数学分析(第二版 上册)》(陈纪修) P278 引理7.1.2
对于任意\(\overline{S}(P_1) \in \overline{\mathbb{S}}\)
和\(\underline{S}(P_2) \in \underline{\mathbb{S}}\),
总有\begin{equation*}
	m(b-a)
	\leq \underline{S}(P_2)
	\leq \overline{S}(P_1)
	\leq M(b-a).
\end{equation*}
\end{lemma}
可以看出,\(\overline{\mathbb{S}}\)和\(\underline{\mathbb{S}}\)都是有界的,
因此分别有下确界和上确界.
记\begin{equation*}
	L \defeq \inf\Set{ s \given s \in \overline{\mathbb{S}} }, \qquad
	l \defeq \sup\Set{ s \given s \in \underline{\mathbb{S}} },
\end{equation*}
则对任意\(\overline{S}(P_1) \in \overline{\mathbb{S}}\)
和\(\underline{S}(P_2) \in \underline{\mathbb{S}}\),\begin{equation*}
	\underline{S}(P_2)
	\leq l \leq L
	\leq \overline{S}(P_1).
\end{equation*}
下面我们来证明,当\(\lambda\to0\)时,
达布大和与达布小和的极限确实存在,
且分别等于它们各自的下确界和上确界.
\begin{lemma}
%@see: 《数学分析(第二版 上册)》(陈纪修) P279 引理7.1.3(Darboux定理)
对任意在\([a,b]\)上有界的函数\(f\),
恒有\begin{equation*}
	\lim_{\lambda\to0} \overline{S}(P) = L, \qquad
	\lim_{\lambda\to0} \underline{S}(P) = l.
\end{equation*}
\end{lemma}

\begin{theorem}
%@see: 《数学分析(第二版 上册)》(陈纪修) P280 定理7.1.1
有界函数\(f\)在\([a,b]\)上黎曼可积的充分必要条件是:
对于任意划分\(P\),当\(\lambda\to0\)时,
达布大和与达布小和的极限相等.
\end{theorem}

若记\begin{equation*}
	\omega_i \defeq M_i - m_i
\end{equation*}为\(f\)在\([x_{i-1},x_i]\)上的振幅,
则上述定理也可等价地表述为下面的定理.

\begin{theorem}
%@see: 《数学分析(第二版 上册)》(陈纪修) P281 定理7.1.2
有界函数\(f\)在\([a,b]\)上黎曼可积的充分必要条件是:
对于任意划分,当\(\lambda\to0\)时,
成立\begin{equation*}
	\lim_{\lambda\to0} \sum_{i=1}^n \omega_i \increment x_i = 0.
\end{equation*}
\end{theorem}

\begin{corollary}\label{theorem:定积分.黎曼可积条件.闭区间上的连续函数必定可积}
%@see: 《高等数学(第六版 上册)》 P226 定理1
%@see: 《数学分析(第二版 上册)》(陈纪修) P281 推论1
闭区间上的连续函数必定可积.
\end{corollary}

\begin{corollary}
%@see: 《数学分析(第二版 上册)》(陈纪修) P282 推论2
闭区间上的单调函数必定可积.
\end{corollary}

\begin{theorem}
%@see: 《数学分析(第二版 上册)》(陈纪修) P283 定理7.1.3
有界函数\(f\)在\([a,b]\)可积的充分必要条件是:
对于任意给定的\(\epsilon>0\),存在一种划分,使得相应的振幅满足\begin{equation*}
	\sum_{i=1}^n \omega_i \increment x_i < \epsilon.
\end{equation*}
\end{theorem}

\begin{corollary}\label{theorem:定积分.黎曼可积条件.闭区间上只有有限个间断点的有界函数必定可积}
%@see: 《高等数学(第六版 上册)》 P227 定理2
%@see: 《数学分析(第二版 上册)》(陈纪修) P283 推论3
闭区间上只有有限个间断点的有界函数必定可积.
\end{corollary}

\begin{example}
%@see: 《数学分析(第二版 上册)》(陈纪修) P285 习题 6.
设函数\(f\)在区间\([a,b]\)上可积,
且在\([a,b]\)上满足\(\abs{f(x)} \geq m > 0\ (\text{$m$是常数})\).
证明:函数\(x \mapsto [f(x)]^{-1}\)在\([a,b]\)上也可积.
%TODO proof
\end{example}

% \begin{remark}
% 有界函数\(f\)在\([a,b]\)上黎曼可积的充分必要条件是:
% \(f\)的全体间断点是零测度的.
% %TODO
% \end{remark}

\begin{definition}\label{definition:函数族.黎曼可积函数族}
由闭区间\([a,b]\)上全部的黎曼可积函数组成的集合,
称作\DefineConcept{黎曼可积函数族}\footnote{%
不作特别强调时,可以将其简称为\DefineConcept{黎曼可积函数族}.%
},记作\(R[a,b]\).
\end{definition}

\subsection{定积分的几何意义}
在区间\([a,b]\)上\(f(x) \geq 0\)时,
定积分\(\int_a^b f(x) \dd{x}\)在几何上表示
由曲线\(y=f(x)\)、直线\(x=a\)、直线\(x=b\)与\(x\)轴所围成的曲边梯形的面积.
在区间\([a,b]\)上\(f(x) \leq 0\)时,
由曲线\(y=f(x)\)、直线\(x=a\)、直线\(x=b\)与\(x\)轴所围成的曲边梯形在\(x\)轴的下方,
定积分\(\int_a^b f(x) \dd{x}\)在几何上表示上述曲边梯形的面积的负值.

实际上,对于任意黎曼可积函数\(f\colon \mathbb{R} \to \mathbb{R}\),
定积分\(\int_a^b f(x) \dd{x}\)还可以理解为非空集\(R[a,b]\)上的泛函.

\subsection{定积分的性质}
下面讨论定积分的性质.
下列各性质中积分上下限的大小,
如不特别指明,均不加限制;
并假定各性质中所列出的定积分都是存在的.
\begin{property}\label{theorem:定积分.定积分性质1}
%@see: 《高等数学(第六版 上册)》 P231 性质1
%@see: 《数学分析(第二版 上册)》(陈纪修) P286 性质1(线性性质)
\(\int_a^b [f(x) \pm g(x)] \dd{x}
= \int_a^b f(x) \dd{x} \pm \int_a^b g(x) \dd{x}\).
\begin{proof}
显然有\begin{align*}
	\int_a^b [f(x) \pm g(x)] \dd{x}
	&= \lim_{\lambda\to0}
		\sum_{i=1}^n [f(\xi_i) \pm g(\xi_i)] \increment x_i \\
	&= \lim_{\lambda\to0}
		\sum_{i=1}^n f(\xi_i) \increment x_i
		\pm
		\lim_{\lambda\to0}
		\sum_{i=1}^n g(\xi_i) \increment x_i \\
	&= \int_a^b f(x) \dd{x} \pm \int_a^b g(x) \dd{x}.
	\qedhere
\end{align*}
\end{proof}
\end{property}
上述性质对任意有限个函数都是成立的.

\begin{property}\label{theorem:定积分.定积分性质2}
%@see: 《高等数学(第六版 上册)》 P231 性质2
%@see: 《数学分析(第二版 上册)》(陈纪修) P286 性质1(线性性质)
对于任意实数\(k\),
总有\(\int_a^b k f(x) \dd{x}
=k\int_a^b f(x) \dd{x}\).
\begin{proof}
显然有\begin{align*}
	\int_a^b k f(x) \dd{x}
	&= \lim_{\lambda\to0}
		\sum_{i=1}^n k f(\xi_i) \increment x_i \\
	&= k \lim_{\lambda\to0}
		\sum_{i=1}^n f(\xi_i) \increment x_i \\
	&= k \int_a^b f(x) \dd{x}.
	\qedhere
\end{align*}
\end{proof}
\end{property}

\begin{corollary}\label{theorem:定积分.定积分性质2推论1}
\(\int_a^b 0 \dd{x} = 0\).
\begin{proof}
在\cref{theorem:定积分.定积分性质2} 中,令\(k=0\),便得.
\end{proof}
\end{corollary}

\begin{corollary}\label{theorem:定积分.定积分性质2推论2}
%@see: 《数学分析(第二版 上册)》(陈纪修) P286 推论
设\(f\)在\([a,b]\)上黎曼可积,
而\(g\)只在有限个点上与\(f\)的取值不相同,
则\(g\)也在\([a,b]\)上黎曼可积,
并且有\begin{equation*}
	\int_a^b f(x) \dd{x}
	= \int_a^b g(x) \dd{x}.
\end{equation*}
\begin{proof}
取\(h(x) = f(x) - g(x)\),
则\(h\)只在有限个点上不为零,除此以外处处为零.
因为\hyperref[theorem:定积分.黎曼可积条件.闭区间上只有有限个间断点的有界函数必定可积]{闭区间上只有有限个间断点的有界函数必定可积},
所以\(h\)在\([a,b]\)上可积,
且\(\int_a^b h(x) \dd{x} = 0\).
再由定积分的线性性质 \labelcref{theorem:定积分.定积分性质1,theorem:定积分.定积分性质2}
可知\(g\)也在\([a,b]\)上可积,
且\begin{equation*}
	\int_a^b g(x) \dd{x}
	= \int_a^b (f(x) - h(x)) \dd{x}
	= \int_a^b f(x) \dd{x}.
	\qedhere
\end{equation*}
\end{proof}
\end{corollary}
\begin{remark}
\cref{theorem:定积分.定积分性质2推论2} 说明:
在有限个点上改变一个可积函数的函数值,不会改变它的可积性和积分值.
\end{remark}

\begin{property}\label{theorem:定积分.定积分性质4}
%@see: 《高等数学(第六版 上册)》 P232 性质4
\(\int_a^b 1 \dd{x}
= \int_a^b \dd{x}
= b-a\).
%TODO proof
\end{property}

\begin{property}\label{theorem:定积分.乘积可积性}
%@see: 《数学分析(第二版 上册)》(陈纪修) P287 性质2(乘积可积性)
设\(f\)和\(g\)都在\([a,b]\)上黎曼可积,
则\(f \cdot g\)在\([a,b]\)上黎曼可积.
%TODO proof
% \begin{proof}
% 由于\(f\)和\(g\)都在\([a,b]\)上可积,
% 所以它们在\([a,b]\)上有界.
% 因此存在常数\(M>0\),
% 满足\begin{equation*}
% 	\abs{f(x)} \leq M
% 	\quad\text{和}\quad
% 	\abs{g(x)} \leq M,
% 	x\in[a,b].
% \end{equation*}
% 对\([a,b]\)的任意划分\begin{equation*}
% 	a = x_0 < x_1 < x_2 < \dotsb < x_n = b,
% \end{equation*}
% 设
% \end{proof}
\end{property}
\begin{remark}
虽然“\(f\)和\(g\)都在\([a,b]\)上黎曼可积”蕴含“\(f \cdot g\)在\([a,b]\)上黎曼可积”,
但是,一般说来有\begin{equation*}
	\int_a^b f(x) g(x) \dd{x}
	\neq
	\left(
		\int_a^b f(x) \dd{x}
	\right) \cdot \left(
		\int_a^b g(x) \dd{x}
	\right).
\end{equation*}
例如,虽然\begin{equation*}
	\int_a^b 1 \dd{x} = b-a, \qquad
	\int_a^b 2 \dd{x} = 2(b-a),
\end{equation*}
但是\begin{equation*}
	\int_a^b 1\cdot2 \dd{x} = 2(b-a)
	\neq
	2(b-a)^2 = \left(\int_a^b 1 \dd{x}\right) \left(\int_a^b 2 \dd{x}\right).
\end{equation*}
\end{remark}

\begin{property}\label{theorem:定积分.定积分性质3}
%@see: 《高等数学(第六版 上册)》 P231 性质3
%@see: 《数学分析(第二版 上册)》(陈纪修) P288 性质5(区间可加性)
设\(f\)在\([a,b]\)上黎曼可积,
则对任意一点\(c\in(a,b)\)成立\(f\)在\([a,c]\)和\([c,b]\)上都黎曼可积,
反过来\(f\)在\([a,c]\)和\([c,b]\)上都黎曼可积蕴含\(f\)在\([a,b]\)上黎曼可积,
并且\begin{equation}
	\int_a^b f(x) \dd{x}
	= \int_a^c f(x) \dd{x}
	+ \int_c^b f(x) \dd{x}.
\end{equation}
%TODO proof
\end{property}
上述性质表明定积分对于积分区间具有“可加性”.

\begin{example}
%@see: 《2023年全国硕士研究生入学统一考试(数学一)》二填空题/第14题
设连续函数\(f\)满足\(f(x+2)-f(x)=x\),
且\(\int_0^2 f(x) \dd{x} = 0\),
求\(\int_1^3 f(x) \dd{x}\).
\begin{solution}
由\cref{theorem:定积分.定积分性质3} 有\begin{align*}
	\int_1^3 f(x) \dd{x}
	&= \int_0^3 f(x) \dd{x}
	- \int_0^1 f(x) \dd{x} \\
	&= \int_0^2 f(x) \dd{x}
	+ \int_2^3 f(x) \dd{x}
	- \int_0^1 f(x) \dd{x} \\
	% 代入\(\int_0^2 f(x) \dd{x} = 0\)
	&= \int_2^3 f(x) \dd{x}
	- \int_0^1 f(x) \dd{x} \\
	% 代入\(\int_0^1 f(x+2) \dd{x} = \int_2^3 f(x) \dd{x}\)
	&= \int_0^1 f(x+2) \dd{x}
	- \int_0^1 f(x) \dd{x} \\
	%= \int_0^1 (f(x+2)-f(x)) \dd{x} \\
	&= \int_0^1 x \dd{x}
	= \eval{\frac12 x^2}_0^1
	= \frac12.
\end{align*}
\end{solution}
\end{example}

\begin{property}\label{theorem:定积分.定积分性质5}
%@see: 《高等数学(第六版 上册)》 P232 性质5
设\(f\)在\([a,b]\)上黎曼可积.
如果在区间\([a,b]\)上,\(f(x) \geq 0\),
则\(\int_a^b f(x) \dd{x} \geq 0\).
\begin{proof}
因为\(f(x) \geq 0\),
所以在区间\([a,b]\)上任取\(n\)个分点\(\AutoTuple{x}{n}\)使之满足\begin{equation*}
	a = x_0 \leq x_1 \leq x_2 \leq \dotsb \leq x_n \leq x_{n+1} = b,
\end{equation*}
在各个小区间上任取一点\(\xi_i\in[x_i,x_{i+1}]\ (i=0,1,2,\dotsc,n)\),
就有\begin{equation*}
	f(\xi_i)\geq0.
\end{equation*}

又因为小区间\([x_i,x_{i+1}]\)的长度
\(\increment x_i = x_{i+1}-x_i \geq 0\ (i=0,1,2,\dotsc,n)\),
因此\begin{equation*}
	\sum_{i=0}^n f(\xi_i) \increment x_i \geq 0,
\end{equation*}
令\(\lambda = \max\{\increment x_0, \increment x_1, \dotsc, \increment x_n\}\),
则由极限的保号性可得\begin{equation*}
	\int_a^b f(x) \dd{x}
	= \lim_{\lambda\to0} \sum_{i=0}^n f(\xi_i) \increment x_i \geq 0.
	\qedhere
\end{equation*}
\end{proof}
\end{property}
\begin{remark}
\cref{theorem:定积分.定积分性质5} 表明定积分具有“保号性”.
同理可证:如果在区间\([a,b]\)上,黎曼可积函数\(f\)严格大于零,
则\(\int_a^b f(x) \dd{x} > 0\).
\end{remark}

\begin{example}
比较\(\int_0^1 \abs{\ln t} \ln^n(1+t) \dd{t}\)
与\(\int_0^1 t^n \abs{\ln t} \dd{t}\)的大小(\(n=1,2,\dotsc\)).
\begin{solution}
直接相减得
\begin{align*}
	&\hspace{-20pt}
	\int_0^1 \abs{\ln t} \ln^n(1+t) \dd{t} - \int_0^1 t^n \abs{\ln t} \dd{t} \\
	&= \int_0^1 [\abs{\ln t} \ln^n(1+t) - t^n \abs{\ln t}] \dd{t} \\
	&= \int_0^1 \abs{\ln t} [\ln^n(1+t) - t^n] \dd{t}.
\end{align*}
注意到\(\ln t\)的自然定义域为\(t > 0\).
当\(0 < t \leq 1\)时,
\(\ln(1+t) < t\),
\(\ln^n(1+t) < t^n\),
\begin{equation*}
	\abs{\ln t} [\ln^n(1+t) - t^n] < 0,
\end{equation*}\begin{equation*}
	\int_0^1 \abs{\ln t} [\ln^n(1+t) - t^n] \dd{t} < 0,
\end{equation*}\begin{equation*}
	\int_0^1 \abs{\ln t} \ln^n(1+t) \dd{t} < \int_0^1 t^n \abs{\ln t} \dd{t}.
\end{equation*}
\end{solution}
\end{example}

\begin{corollary}\label{theorem:定积分.定积分性质5推论1}
%@see: 《高等数学(第六版 上册)》 P232 推论1
%@see: 《数学分析(第二版 上册)》(陈纪修) P287 性质3(保序性)
设\(f\)和\(g\)都在\([a,b]\)上黎曼可积,
且在\([a,b]\)上恒有\(f(x) \leq g(x)\),
则\begin{equation*}
	\int_a^b f(x) \dd{x} \leq \int_a^b g(x) \dd{x}.
\end{equation*}
\begin{proof}
令\(\phi(x) = g(x) - f(x)\).
因为\(f(x) \leq g(x)\),所以\(\phi(x) \geq 0\).
由\cref{theorem:定积分.定积分性质5} 得\begin{equation*}
	\int_a^b \phi(x) \dd{x} \geq 0,
\end{equation*}
又由\cref{theorem:定积分.定积分性质1} 得\begin{equation*}
	\int_a^b \phi(x) \dd{x}
	= \int_a^b [g(x) - f(x)] \dd{x}
	= \int_a^b g(x) \dd{x} - \int_a^b f(x) \dd{x},
\end{equation*}
所以\begin{equation*}
	\int_a^b g(x) \dd{x} - \int_a^b f(x) \dd{x} \geq 0,
\end{equation*}
从而有\begin{equation*}
	\int_a^b g(x) \dd{x} \geq \int_a^b f(x) \dd{x}.
	\qedhere
\end{equation*}
\end{proof}
\end{corollary}
\begin{remark}
容易看出,\cref{theorem:定积分.定积分性质5} 与\cref{theorem:定积分.定积分性质5推论1} 是等价命题.
而且同样可证:如果在区间\([a,b]\)上\(f(x)\)严格小于\(g(x)\),
则\(\int_a^b f(x) \dd{x} < \int_a^b g(x) \dd{x}\).
\end{remark}

\begin{corollary}\label{theorem:定积分.定积分性质5推论2}
%@see: 《高等数学(第六版 上册)》 P233 推论2
%@see: 《数学分析(第二版 上册)》(陈纪修) P288 性质4(绝对可积性)
设\(f\)在\([a,b]\)上黎曼可积,
则\begin{equation}
	\abs{\int_a^b f(x) \dd{x}} \leq \int_a^b \abs{f(x)} \dd{x}.
\end{equation}
\begin{proof}
因为\begin{equation*}
	-\abs{f(x)} \leq f(x) \leq \abs{f(x)},
\end{equation*}
那么由\cref{theorem:定积分.定积分性质5推论1,theorem:定积分.定积分性质2} 可得\begin{equation*}
	-\int_a^b \abs{f(x)} \dd{x}
	\leq
	\int_a^b f(x) \dd{x}
	\leq
	\int_a^b \abs{f(x)} \dd{x},
\end{equation*}
即\begin{equation*}
	\abs{\int_a^b f(x) \dd{x}} \leq \int_a^b \abs{f(x)} \dd{x}.
	\qedhere
\end{equation*}
\end{proof}
\end{corollary}

\begin{example}
%@see: 《高等数学(第六版 上册)》 P236 习题5-1 12. (1)
设\(f \in C[a,b]\),
在\([a,b]\)上\(f(x)\geq0\),
且\(\int_a^b f(x) \dd{x} = 0\).
证明:在\([a,b]\)上\(f(x)\equiv0\).
\begin{proof}
用反证法.
假设在\([a,b]\)上\(f(x)\not\equiv0\),
不妨设\(f(c)>0\ (a<c<b)\).
由于\(f\)连续,所以由连续性的定义可知\(\lim_{x \to c} f(x) = f(c)\).
再由\cref{theorem:极限.函数极限的局部保序性1.推论1} 可知,
存在\(\delta>0\),使得当\(0<\abs{x-c}<\delta\)时,成立\begin{equation*}
	f(x) > \frac12 f(c).
\end{equation*}
因为\(f(x)\geq0\),
所以由\cref{theorem:定积分.定积分性质5} 可知\begin{equation*}
	\int_a^{c-\delta} f(x) \dd{x} \geq 0,
	\qquad
	\int_{c+\delta}^b f(x) \dd{x} \geq 0.
\end{equation*}
于是\begin{align*}
	\int_a^b f(x) \dd{x}
	&= \left( \int_a^{c-\delta} + \int_{c-\delta}^{c+\delta} + \int_{c+\delta}^b \right) f(x) \dd{x}
	\geq \int_{c-\delta}^{c+\delta} f(x) \dd{x} \\
	&> \frac12 f(c) \int_{c-\delta}^{c+\delta} \dd{x}
	= \frac12 f(c) \cdot 2\delta
	> 0,
\end{align*}
与题设\(\int_a^b f(x) \dd{x} = 0\)矛盾!因此\(f(x)\equiv0\).
\end{proof}
\end{example}
\begin{example}
%@see: 《数学分析(第二版 上册)》(陈纪修) P293 习题 5.
设\(f\)在\([a,b]\)上连续.
如果\(f\)在区间\([a,b]\)上非负且不恒为零,
则\(\int_a^b f(x) \dd{x} > 0\).
\begin{proof}
由于\(f\)在区间\([a,b]\)上非负且不恒为零,
不妨设\(c \in (a,b)\)且\(f(c)>0\),
于是由\hyperref[definition:极限.函数在一点的连续性]{连续函数的定义}有\begin{equation*}
	\lim_{x \to c} f(x) = f(c) > 0.
\end{equation*}
那么由\cref{theorem:极限.函数极限的局部保序性1.推论1} 可知,
存在\(\delta>0\),当\(0 < \abs{x-c} < \delta\)时,满足\begin{equation*}
	f(x) > \frac12 f(c),
\end{equation*}
于是由\cref{theorem:定积分.定积分性质5推论1,theorem:定积分.定积分性质5推论3} 得\begin{equation*}
	\int_a^b f(x) \dd{x}
	\geq \int_{c-\delta}^{c+\delta} f(x) \dd{x}
	\geq \int_{c-\delta}^{c+\delta} \frac12 f(c) \dd{x}
	= \frac12 f(c) \cdot 2 \delta
	> 0.
	\qedhere
\end{equation*}
\end{proof}
\end{example}

\begin{corollary}\label{theorem:定积分.定积分性质5推论3}
设函数\(f\)是区间\([a,b]\)上的非负的黎曼可积函数,
那么对于\(\forall c,d\in(a,b)\),
有\begin{equation*}
	c<d
	\implies
	\int_a^c f(x) \dd{x} \leq \int_a^d f(x) \dd{x}.
\end{equation*}
\begin{proof}
根据\cref{theorem:定积分.定积分性质5},
有\begin{equation*}
	\int_c^d f(x) \dd{x} \geq 0.
\end{equation*}
再根据\cref{theorem:定积分.定积分性质3},
有\begin{equation*}
	\int_a^c f(x) \dd{x}
	+ \int_c^d f(x) \dd{x}
	= \int_a^d f(x) \dd{x}.
\end{equation*}
因此\begin{equation*}
	\int_a^c f(x) \dd{x}
	\leq
	\int_a^d f(x) \dd{x}.
	\qedhere
\end{equation*}
\end{proof}
\end{corollary}

\begin{property}\label{theorem:定积分.定积分性质6}
%@see: 《高等数学(第六版 上册)》 P233 性质6
设\(M\)及\(m\)分别是函数\(f\)在区间\([a,b]\)上的最大值及最小值,
则\begin{equation*}
	m(b-a) \leq \int_a^b f(x) \dd{x} \leq M(b-a).
\end{equation*}
\begin{proof}
因为\(m \leq f(x) \leq M\),
所以由上述推论,
得\begin{equation*}
	m(b-a)
	= \int_a^b m \dd{x}
	\leq \int_a^b f(x) \dd{x}
	\leq \int_a^b M \dd{x}
	= M(b-a).
	\qedhere
\end{equation*}
\end{proof}
\end{property}
上述性质说明,由被积函数在积分区间上的最大值及最小值,可以估计积分值的大致范围.

\subsection{积分中值定理}
\begin{theorem}[积分第一中值定理]\label{theorem:定积分.积分中值定理1}
%@see: 《数学分析(第二版 上册)》(陈纪修) P290 性质6(积分第一中值定理)
设\(f,g \in R[a,b]\),
且\(g\)在\([a,b]\)上不变号,
记\begin{equation*}
	m \defeq \inf\Set{ f(x) \given x \in [a,b] }, \qquad
	M \defeq \sup\Set{ f(x) \given x \in [a,b] },
\end{equation*}
那么\(\exists\mu\in[m,M]\),
使得\begin{equation}
	\int_a^b f(x) g(x) \dd{x} = \mu \int_a^b g(x) \dd{x}.
\end{equation}
\begin{proof}
因为\(g\)在\([a,b]\)上不变号,
不妨设\(g(x)\geq0\).
于是有\begin{equation*}
	m g(x) \leq f(x) g(x) \leq M g(x).
\end{equation*}
由\cref{theorem:定积分.定积分性质5推论1} 可得\begin{equation*}
	m \int_a^b g(x) \dd{x}
	\leq \int_a^b f(x) g(x) \dd{x}
	\leq M \int_a^b g(x) \dd{x}.
\end{equation*}
由于\(\int_a^b f(x) g(x) \dd{x}\)
和\(\int_a^b g(x) \dd{x}\)都是常数,
因而必有某个\(\mu\in[m,M]\),
使得\begin{equation*}
	\int_a^b f(x) g(x) \dd{x}
	= \mu \int_a^b g(x) \dd{x}.
	\qedhere
\end{equation*}
\end{proof}
\end{theorem}

\begin{corollary}\label{theorem:定积分.积分中值定理1推论1}
%@see: 《数学分析(第二版 上册)》(陈纪修) P290 性质6(积分第一中值定理)
设\(f \in C[a,b]\),
\(g \in R[a,b]\),
且\(g\)在\([a,b]\)上不变号,
那么\(\exists\xi\in[a,b]\),
使得\begin{equation}
	\int_a^b f(x) g(x) \dd{x}
	= f(\xi) \int_a^b g(x) \dd{x}.
\end{equation}
\begin{proof}
由\cref{theorem:定积分.积分中值定理1} 可知,
存在\(\mu\in[m,M]\),
使得\begin{equation*}
	\int_a^b f(x) g(x) \dd{x}
	= \mu \int_a^b g(x) \dd{x}.
\end{equation*}
因为\(f\)在\([a,b]\)上连续,
由\hyperref[theorem:极限.闭区间上连续函数的性质.介值定理2]{闭区间上连续函数的介值定理}可知,
存在\(\xi\in[a,b]\)使得\(f(\xi) = \mu\),
因此\begin{equation*}
	\int_a^b f(x) g(x) \dd{x}
	= f(\xi) \int_a^b g(x) \dd{x}.
	\qedhere
\end{equation*}
\end{proof}
\end{corollary}

\begin{figure}[htb]
	\centering
	\begin{tikzpicture}
		\begin{axis}[
			axis lines=middle,
			xlabel=$x$,
			ylabel=$y$,
			xmin=0,xmax=3,
			ymin=0,ymax=3,
			xtick={.5,1.5,2.5},
			xticklabels={$a$,$\xi$,$b$},
			ytick={1.5},
			yticklabels={$f(\xi)$},
		]
			\addplot[domain=.5:2.5]{(x-1.5)^3+1.5};
			\begin{scope}[dashed,black!30]
				\draw(.5,1.5)--(.5,0);
				\draw(2.5,2.5)--(2.5,0);
				\draw(1.5,1.5)--(1.5,0);
				\draw(2.5,1.5)--(0,1.5);
			\end{scope}
			\draw(2.5,2.5)node[left]{$y=f(x)$};
		\end{axis}
	\end{tikzpicture}
	\caption{}
	\label{figure:定积分.积分中值定理的几何意义}
\end{figure}

\begin{corollary}\label{theorem:定积分.积分中值定理1推论2}
%@see: 《高等数学(第六版 上册)》 P233 性质7(定积分中值定理)
%@see: 《数学分析(第二版 上册)》(陈纪修) P290
设函数\(f\)在\([a,b]\)上连续,
则\(\exists\xi\in[a,b]\),
使\begin{equation}
	\int_a^b f(x) \dd{x} = f(\xi) (b-a).
\end{equation}
\begin{proof}
令\(g(x) \equiv 1\),
由\cref{theorem:定积分.积分中值定理1推论1} 立即可得.
\end{proof}
\end{corollary}
\begin{remark}
%@see: 《高等数学(第六版 上册)》 P234
如\cref{figure:定积分.积分中值定理的几何意义} 所示,
\cref{theorem:定积分.积分中值定理1推论2} 的几何意义十分明确:
在区间\([a,b]\)上至少存在一点\(\xi\),
使得以区间\([a,b]\)为底边、以曲线\(y=f(x)\)为曲边的曲边梯形的面积
等于同一底边而高为\(f(\xi)\)的一个矩形的面积.
\end{remark}
\begin{remark}
注意到\cref{theorem:定积分.积分中值定理1推论2} 中\(\xi\)的取值范围是\([a,b]\).
实际上,我们可以舍去端点,让\(\xi\)的取值范围变成\((a,b)\).
在下一节,我们就可以证明这个更强的结论 --- \cref{theorem:定积分.积分中值定理1推论2改进}.
\end{remark}
\begin{remark}
对于\cref{theorem:定积分.积分中值定理1推论2},
我们需要注意到:
通常来说\(\xi\)不是唯一的.
只有当被积函数是严格单调函数时,\(\xi\)是\(x\)的函数,
例如:\begin{itemize}
	\item 由\(\int_0^x e^t \dd{t} = x e^\xi\)可得\(\xi = \ln\frac{e^x-1}{x}\),
	而\(\lim_{x\to0} \frac{\xi}{x} = \frac12\).
	\item 由\(\int_0^x \ln t \dd{t} = x \ln\xi\)可得\(\xi = \frac{x}{e}\),
	而\(\lim_{x\to0} \frac{\xi}{x} = \frac1e\).
\end{itemize}

一般地,假设\(f\)在\([0,x]\)上单调连续,
则\begin{equation*}
	\xi = f^{-1}\kern-2pt\left[ \frac1x \int_0^x f(t) \dd{t} \right].
\end{equation*}
\end{remark}
\begin{definition}
把\begin{equation}
	\frac{1}{b-a} \int_a^b f(x) \dd{x}
\end{equation}
称为“函数\(f\)在区间\([a,b]\)上的\DefineConcept{平均值}”.
\end{definition}

\begin{example}
%@see: https://www.bilibili.com/video/BV1R627YaEom/
设函数\(f \in C[0,1]\).
证明:\(\lim_{n\to\infty} \int_0^1 f(\sqrt[n]{x}) \dd{x} = f(1)\).
\begin{proof}
因为\(f \in C[0,1]\),
所以由\hyperref[theorem:极限.最值定理]{闭区间上的连续函数的有界性与最大值最小值定理}可知\begin{equation*}
	(\exists M>0)
	(\forall x\in[0,1])
	[\abs{f(x)} \leq M].
\end{equation*}
由\hyperref[theorem:定积分.定积分性质5推论2]{绝对可积性}可知\begin{equation*}
	0 \leq \abs{\int_0^{\frac1n} f(\sqrt[n]{x}) \dd{x}}
	\leq \int_0^{\frac1n} \abs{f(\sqrt[n]{x})} \dd{x}
	\leq \int_0^{\frac1n} M \dd{x}
	= \frac{M}{n}
	\to 0\ (n\to\infty),
\end{equation*}
所以由\hyperref[theorem:数列极限.夹逼准则]{夹逼准则}可知
\begin{equation*}
	\lim_{n\to\infty} \int_0^{\frac1n} f(\sqrt[n]{x}) \dd{x} = 0.
\end{equation*}

由\hyperref[theorem:定积分.积分中值定理1推论2]{积分中值定理}可知\begin{equation*}
	\int_{\frac1n}^1 f(\sqrt[n]{x}) \dd{x}
	= f(\sqrt[n]{\xi}) \cdot \left( 1-\frac1n \right),
\end{equation*}
其中\(1/n \leq \xi \leq 1\),
从而有\(\sqrt[n]{1/n} \leq \sqrt[n]{\xi} \leq 1\),
由\cref{equation:数列极限.重要极限2} 有\(\sqrt[n]{1/n} \to 1\ (n\to\infty)\),
所以由\hyperref[theorem:数列极限.夹逼准则]{夹逼准则}可得
\(\sqrt[n]{\xi} \to 1\ (n\to\infty)\),
再由\hyperref[theorem:极限.连续函数的极限3]{连续性}可得
\(f(\sqrt[n]{\xi}) \to f(1)\),
那么\begin{equation*}
	\lim_{n\to\infty} \int_{\frac1n}^1 f(\sqrt[n]{x}) \dd{x}
	= \lim_{n\to\infty} f(\sqrt[n]{\xi})
		\cdot \lim_{n\to\infty} \left( 1-\frac1n \right)
	= f(1).
\end{equation*}

综上所述\begin{equation*}
	\lim_{n\to\infty} \int_0^1 f(\sqrt[n]{x}) \dd{x}
	= \lim_{n\to\infty} \left(
		\int_0^{\frac1n} f(\sqrt[n]{x}) \dd{x}
		+ \int_{\frac1n}^1 f(\sqrt[n]{x}) \dd{x}
	\right)
	= f(1).
	\qedhere
\end{equation*}
\end{proof}
\end{example}

\begin{example}
%@see: https://www.bilibili.com/video/BV11FqSY2Eub/
设函数\(f \in C[0,1]\),且\(\int_0^1 f(x) \dd{x} = 0\).
证明:存在\(\xi\in(0,1)\),使得\(f(\xi) = 2\xi \int_0^\xi f(t) \dd{t}\).
\begin{proof}
记\(F(x) = \int_0^x f(t) \dd{t}\).
%@Mathematica: DSolve[F'[x] == 2 x F[x], F[x], x]
%@Mathematica: DSolve[F'[x] == 2 x F[x] && F[1] == 0, F[x], x]
% 注意到所求结论是一个一阶线性微分方程\(F'(x) = 2x F(x)\).
% 因此可以构造一个辅助函数.
令\(\phi(x) = e^{-x^2} F(x)\).
因为\(\phi(0) = \phi(1) = 0\),
所以由\hyperref[theorem:微分中值定理.罗尔定理]{罗尔定理}可知,
存在\(\xi\in(0,1)\),
使得\(\phi'(\xi)
= e^{-\xi^2} (f(\xi) - 2\xi F(\xi))
= 0\),
即\(f(\xi) = 2\xi \int_0^\xi f(t) \dd{t}\).
\end{proof}
\end{example}

\begin{theorem}[积分第二中值定理]\label{theorem:定积分.积分中值定理2}
%@see: 《数学分析(第二版 上册)》(陈纪修) P373 定理8.2.4(积分第二中值定理)
设\(f,g \in R[a,b]\),
则\begin{itemize}
	\item 若\(f\)在\((a,b)\)上单调,
	则\(\exists \xi \in [a,b]\),
	使得\begin{equation*}
		\int_a^b f(x) g(x) \dd{x}
		= f(a^+) \int_a^{\xi} g(x) \dd{x} + f(b^-) \int_{\xi}^b g(x) \dd{x}.
	\end{equation*}
	\item 若\(f\)在\((a,b)\)上单调递减且\(f(x) \geq 0\),
	则\(\exists \xi \in [a,b]\),
	使得\begin{equation*}
		\int_a^b f(x) g(x) \dd{x}
		= f(a^+) \int_a^{\xi} g(x) \dd{x}.
	\end{equation*}
	\item 若\(f\)在\((a,b)\)上单调递增且\(f(x) \geq 0\),
	则\(\exists \xi \in [a,b]\),
	使得\begin{equation*}
		\int_a^b f(x) g(x) \dd{x}
		= f(b^-) \int_{\xi}^b g(x) \dd{x}.
	\end{equation*}
\end{itemize}
%TODO proof
\end{theorem}

\begin{example}
%@see: 《数学分析(第二版 上册)》(陈纪修) P294 习题 13.
设\(f \in R[a,b]\).
证明:等式\begin{equation*}
	\int_a^b f^2(x) \dd{x} = 0
\end{equation*}成立的充分必要条件是:
对于函数\(f\)在\([a,b]\)上的所有连续点都有\(f(x)=0\).
%TODO proof
\end{example}

\begin{example}
%@see: 《数学分析教程 (第3版 上册)》(史济怀) P248 练习题6.2 8.
设\(f \in C[0,1]\)且\(f(x)>0\).
证明:\begin{equation*}
	\int_0^1 f(x) \dd{x}
	\int_0^1 \frac1{f(x)} \dd{x}
	\geq 1.
\end{equation*}
%TODO proof
\end{example}

\begin{example}
%@see: 《数学分析(第二版 上册)》(陈纪修) P294 习题 8.
设\(\phi\)在\([0,a]\)上连续,\(f\)在\((-\infty,+\infty)\)上二阶可导,且\(f''(x)\geq0\).
证明:\begin{equation*}
	f\left(\frac1a \int_0^a \phi(t) \dd{t}\right)
	\leq \frac1a \int_0^a f(\phi(t)) \dd{t}.
\end{equation*}
%TODO proof
\end{example}

\begin{example}
%@see: 《数学分析(第二版 上册)》(陈纪修) P294 习题 11.
设函数\(f\)在\([a,b]\)上可积,
对于任意给定\(h\in\mathbb{R}\)有\(f_h(x) = f(x+h)\)在\([a,b]\)上可积.
证明:\begin{equation*}
	\lim_{h\to0} \int_a^b \abs{f_h(x)-f(x)} \dd{x} = 0.
\end{equation*}
%TODO proof
\end{example}

\section{微积分基本公式}
\subsection{积分上限的函数及其导数}
\begin{theorem}\label{theorem:定积分.变限积分定理}
%@see: 《高等数学(第六版 上册)》 P237 定理1
%@see: 《数学分析(第二版 上册)》(陈纪修) P296 定理7.3.1
设函数\(f\colon[a,b]\to\mathbb{R}\),
令\(\Phi(x)
= \int_a^x f(t) \dd{t}
\ (a \leq x \leq b)\).
\begin{itemize}
	\item 如果\(f\)在\([a,b]\)上黎曼可积,
	则\(\Phi\)在\([a,b]\)上连续.

	\item 如果\(f\)在\([a,b]\)上连续,
	则\(\Phi\)在\([a,b]\)上可导,
	并且它的导数为\begin{equation*}
		\Phi'(x)
		= \dv{x} \int_a^x f(t) \dd{t}
		= f(x),
		\quad a \leq x \leq b.
	\end{equation*}
\end{itemize}
\begin{proof}
易知\(\Phi\)在整个闭区间\([a,b]\)上有定义.
我们按\(x\)的取值,分为\(x\in(a,b)\)、\(x=a\)、\(x=b\)三种情况讨论.

若\(x\in(a,b)\),
取\(x + \increment x \in (a,b)\),
则\begin{equation*}
	\Phi(x + \increment x) = \int_a^{x+\increment x} f(t) \dd{t},
\end{equation*}
于是\begin{align*}
	\increment\Phi
	&= \Phi(x + \increment x) - \Phi(x) \\
	&= \int_a^{x+\increment x} f(t) \dd{t} - \int_a^x f(t) \dd{t} \\
	&= \int_a^x f(t) \dd{t} + \int_x^{x+\increment x} f(t) \dd{t} - \int_a^x f(t) \dd{t}
		\tag{\cref{theorem:定积分.定积分性质3}} \\
	&= \int_x^{x+\increment x} f(t) \dd{t}.
\end{align*}
记\(m \defeq \min_{a \leq x \leq b} f(x),
M \defeq \max_{a \leq x \leq b} f(x),
\alpha \defeq \min\{x,x+\increment x\},
\beta \defeq \max\{x,x+\increment x\}\).
由\hyperref[theorem:定积分.积分中值定理1]{定积分第一中值定理}可知:
如果\(f\)在\([a,b]\)上黎曼可积,
则\(\exists\mu\in[m,M]\),
使得\(\increment\Phi = \mu \cdot \increment x\);
如果\(f\)在\([a,b]\)上连续,
则\(\exists\xi\in(\alpha,\beta)\),
使得\(\increment\Phi = f(\xi) \cdot \increment x\).
显然,不管在哪种情况下,当\(\increment x\to0\)时,
都有\(\increment\Phi\to0\),
即\(\Phi\)在\([a,b]\)上连续.

如果\(f\)在\([a,b]\)上连续,
当\(\increment x\to0\)时\(\xi \to x\),
因而\(f(\xi) \to f(x)\),
于是当\(x\in(a,b)\)时,成立\begin{equation*}
	\Phi'(x)
	= \lim_{\increment x\to0} \frac{\increment\Phi}{\increment x}
	= \lim_{\increment x\to0} f(\xi)
	= \lim_{\xi \to x} f(\xi)
	= f(x).
\end{equation*}
当\(x = a\)时,取\(\increment x > 0\),则同理可证\begin{equation*}
	\Phi'_+(a) = f(a).
\end{equation*}
当\(x = b\)时,取\(\increment x < 0\),则同理可证\begin{equation*}
	\Phi'_-(b) = f(b).
	\qedhere
\end{equation*}
\end{proof}
\end{theorem}
\cref{theorem:定积分.变限积分定理} 指出了一个重要结论:
连续函数\(f\)取变上限\(x\)的定积分然后求导,其结果还原为\(f\)本身.
联想到原函数的定义,就可以从该定理推知\(\Phi\)是连续函数\(f\)的一个原函数.
因此,我们引出如下的原函数的存在定理:
\begin{theorem}[原函数存在定理]\label{theorem:定积分.原函数存在定理}
%@see: 《高等数学(第六版 上册)》 P238 定理2
设函数\(f\)在区间\([a,b]\)上连续,
则函数\begin{equation*}
	\Phi(x) = \int_a^x f(t) \dd{t}
\end{equation*}就是\(f\)在\([a,b]\)上的一个原函数.
%TODO proof
\end{theorem}
\begin{remark}
\cref{theorem:定积分.原函数存在定理} 为判定一个函数的原函数是否存在,提供了一个充分不必要条件.
例如,虽然函数\(f(x) = \frac1x\)在点\(x=0\)不连续,
但是函数\(g(x) = \ln\abs{x}\)确实是它的一个原函数.
\end{remark}
\begin{remark}
\cref{theorem:定积分.原函数存在定理} 的重要意义是:
一方面肯定了连续函数的原函数是存在的,
另一方面初步地揭示了积分学中的定积分与原函数之间的联系.
因此,我们就有可能通过原函数来计算定积分.
\end{remark}

\begin{example}%野题
设\begin{equation*}
%@Mathematica: f[x_] := Piecewise[{{Sin[x], 0 <= x < \[Pi]}, {2, \[Pi] <= x <= 2 \[Pi]}}]
%@Mathematica: F[x_] := Integrate[f[t], {t, 0, x}, Assumptions -> {x \[Element] Reals}]
%@Mathematica: Plot[{f[x], F[x]}, {x, 0, 2 \[Pi]}, PlotLegends -> "Expressions"]
	f(x) = \left\{ \begin{array}{cl}
		\sin x, & 0 \leq x < \pi, \\
		2, & \pi \leq 2 \leq 2\pi,
	\end{array} \right.
	\qquad
	F(x) = \int_0^x f(t) \dd{t}.
\end{equation*}
考察函数\(F\)在点\(x=\pi\)的连续性和可导性.
\begin{solution}
当\(0 \leq x < \pi\)时,
有\begin{equation*}
	F(x) = \int_0^x f(t) \dd{t}
	= \int_0^x \sin t \dd{t}
	= \eval{\cos t}_x^0
	= 1 - \cos x,
\end{equation*}
%@Mathematica: Limit[F[x], x -> \[Pi], Direction -> "FromBelow"]
故\(\lim_{x\to\pi^-} F(x) = 2\).
当\(\pi \leq x \leq 2\pi\)时,
有\begin{equation*}
	F(x) = \int_0^\pi f(t) \dd{t} + \int_\pi^x f(t) \dd{t}
	= 2(1 + x - \pi),
\end{equation*}
%@Mathematica: Limit[F[x], x -> \[Pi], Direction -> "FromAbove"]
故\(\lim_{x\to\pi^+} F(x) = 2\).
又因为\(F(\pi) = 2\),
所以函数\(F\)在点\(x=\pi\)连续.

因为\begin{align*}
%@Mathematica: Limit[(F[x] - F[\[Pi]])/(x - \[Pi]), x -> \[Pi], Direction -> "FromBelow"]
	F'_-(\pi)
	&= \lim_{x\to\pi^-} \frac{F(x) - F(\pi)}{x - \pi}
	= \lim_{x\to\pi^-} \frac{1 - \cos x - 2}{x - \pi}
	% 洛必达法则
	= 0, \\
%@Mathematica: Limit[(F[x] - F[\[Pi]])/(x - \[Pi]), x -> \[Pi], Direction -> "FromAbove"]
	F'_+(\pi)
	&= \lim_{x\to\pi^+} \frac{F(x) - F(\pi)}{x - \pi}
	= \lim_{x\to\pi^+} \frac{2(1 + x - \pi) - 2}{x - \pi}
	= 2,
\end{align*}
左导数与右导数不相等,
所以函数\(F\)在点\(x=\pi\)不可导.
\end{solution}
\end{example}

\begin{example}
%@credit: {ce603838-a24d-4616-9395-d7b223e8cb72}
举例说明:函数可积但它的原函数不存在.
\begin{solution}
取\begin{equation*}
	f(x) = \sgn x
	\quad(-1 \leq x \leq 1).
\end{equation*}
显然\(f\)在\([-1,1]\)上可积,
且\begin{equation*}
	\int_{-1}^x f(t) \dd{t}
	= \abs{x} - 1.
\end{equation*}
假设\(F\colon[-1,1]\to\mathbb{R}\)是\(f\)的一个原函数.
由定义,对\(\forall x\in[-1,1]\),有\begin{equation*}
	F'(x) = f(x).
\end{equation*}
特别地,有\(F'(0) = f(0) = 0\).
%@credit: {c6cf8893-93b7-42d9-8607-d381c488d453}
现在来检验函数\(F\)在点\(x=0\)的右导数:\begin{equation*}
	F'_+(0)
	= \lim_{h\to0^+} \frac{F(0+h)-F(0)}{h}
	= \lim_{h\to0^+} F'(h)
	= \lim_{h\to0^+} f(h)
	= 1
	\neq F'(0),
\end{equation*}
矛盾!
因此\(f\)的原函数不存在.
\end{solution}
\end{example}

\begin{example}
%@credit: {ce603838-a24d-4616-9395-d7b223e8cb72}
举例说明:函数不可积但它的原函数存在.
\begin{solution}
取\begin{equation*}
	f(x) = 2 x \sin\frac1{x^2} - \frac2x \cos\frac1{x^2}
	\quad(x\neq0).
\end{equation*}
显然\(g(x) = x^2 \sin\frac1{x^2}\)是\(f\)的一个原函数.
但是\(f\)在点\(x=0\)的邻域内无界振荡,不可积.
\end{solution}
\end{example}

\begin{example}
设函数\(f\)在区间\([a,b]\)上连续,函数\(\phi\)和\(\psi\)可导.
证明:\begin{equation*}
	F(x) = \int_{\psi(x)}^{\phi(x)} f(t) \dd{t}
\end{equation*}在\([a,b]\)上可导,
且其导数为\begin{equation}\label{equation:定积分.含参变量积分的导数1}
	F'(x) = \dv{x} \int_{\psi(x)}^{\phi(x)} f(t) \dd{t}
	= f[\phi(x)] \phi'(x) - f[\psi(x)] \psi'(x).
\end{equation}
\begin{proof}
令\(F_1(x) = \int_{\xi}^{\phi(x)} f(t) \dd{t}\),\(u = \phi(x)\),
设函数\(G\)是被积函数\(f\)的一个原函数,
那么\begin{equation*}
	F_1(x) = G[\phi(x)] - G(\xi),
\end{equation*}
从而\begin{align*}
	F'_1(x) = \dv{x} F_1(x)
	&= \dv{x} G[\phi(x)] - \dv{x} G(\xi) = \dv{G}{u} \dv{u}{x} - 0 \\
	&= f(u) \phi'(x) = f[\phi(x)] \phi'(x).
\end{align*}
同样地,令\(F_2(x) = \int_{\psi(x)}^{\xi} f(t) \dd{t}\),\(v = \psi(x)\),
则有\begin{equation*}
	F'_2(x) = \dv{x} F_2(x) = -\dv{x} \int_{\xi}^{\psi(x)} f(x) \dd{t}
	= -f[\psi(x)] \psi'(x).
\end{equation*}
因为\(F(x) = F_1(x) + F_2(x)\),所以\begin{equation*}
	F'(x) = F'_1(x) + F'_2(x)
	= f[\phi(x)] \phi'(x) - f[\psi(x)] \psi'(x).
	\qedhere
\end{equation*}
\end{proof}
\end{example}

\begin{example}
计算极限\begin{equation*}
	\lim_{x \to a} \frac{x}{x-a} \int_a^x f(t) \dd{t}
	\quad(a\neq0),
\end{equation*}
其中\(f\)在点\(a\)的邻域内连续.
\begin{solution}
显然\begin{equation*}
	\frac{x}{x-a} \int_a^x f(t) \dd{t}
	= \frac1{1 - a/x} \int_a^x f(t) \dd{t}.
\end{equation*}

根据\cref{equation:定积分.上下限相等的定积分为零} 有\begin{equation*}
	\lim_{x \to a} \int_a^x f(t) \dd{t} = 0, \qquad
	\lim_{x \to a} \left( 1 - \frac{a}{x} \right) = 0,
\end{equation*}

根据\cref{theorem:定积分.变限积分定理},有\begin{equation*}
	\dv{x} \int_a^x f(t) \dd{t} = f(x);
\end{equation*}
又有\begin{equation*}
	\dv{x}(1 - \frac{a}{x}) = \frac{a}{x^2};
\end{equation*}
那么根据\cref{theorem:微分中值定理.洛必达法则1} 有\begin{equation*}
	\lim_{x \to a} \frac{x}{x-a} \int_a^x f(t) \dd{t}
	= \lim_{x \to a} \frac{f(x)}{a/x^2}
	= a f(a).
\end{equation*}
\end{solution}
\end{example}

\begin{example}
%@see: 《高等数学(第六版 上册)》 P244 习题5-2 10.
设\begin{equation*}
	f(x) = \left\{ \begin{array}{cl}
		x^2, & 0 \leq x < 1, \\
		x, & 1 \leq x \leq 2.
	\end{array} \right.
\end{equation*}
求\(\Phi(x) = \int_0^x f(t) \dd{t}\)
在\([0,2]\)上的表达式,
并讨论\(\Phi\)在\((0,2)\)内的连续性.
\begin{solution}
当\(0 \leq x < 1\)时,有\begin{equation*}
	\Phi(x) = \int_0^x f(t) \dd{t}
	= \int_0^x t^2 \dd{t}
	= \frac13 x^3.
\end{equation*}

当\(1 \leq x \leq 2\)时,有\begin{equation*}
	\Phi(x) = \int_0^x f(t) \dd{t}
	= \int_0^1 t^2 \dd{t}
	+ \int_1^x t \dd{t}
	= \frac13 + \frac12 (x^2-1).
\end{equation*}

因此\begin{equation*}
	\Phi(x) = \left\{ \def\arraystretch{1.5} \begin{array}{cl}
		\frac13 x^3, & 0 \leq x < 1, \\
		\frac13 + \frac12 (x^2-1), & 1 \leq x \leq 2.
	\end{array} \right.
\end{equation*}
易见\begin{equation*}
	\lim_{x\to1^-} \Phi(x)
	= \lim_{x\to1^+} \Phi(x)
	= \Phi(1)
	= \frac13,
\end{equation*}
所以\(\Phi\)在\((0,2)\)内连续.
\end{solution}
\end{example}

\begin{example}
%@see: 《高等数学(第六版 上册)》 P244 习题5-2 12.
设函数\(f \in C[a,b] \cap D(a,b)\),
而\begin{equation*}
	F(x) = \frac1{x-a} \int_a^x f(t) \dd{t}.
\end{equation*}
证明:若在\((a,b)\)内有\(f'(x) \leq 0\),
则在\((a,b)\)内有\(F'(x) \leq 0\).
\begin{proof}
求导得\begin{equation*}
	F'(x) = \frac1{(x-a)^2} \left[ (x-a)~f(x) - \int_a^x f(t) \dd{t} \right].
\end{equation*}
那么\(F'(x) \leq 0\)当且仅当\begin{equation*}
	(x-a)~f(x) - \int_a^x f(t) \dd{t} \leq 0.
\end{equation*}
记\(\phi(x) = (x-a)~f(x) - \int_a^x f(t) \dd{t}\).
显然\(\phi(a) = 0\).
%@credit: {cde036ea-2111-4ff3-b743-6d81e96b49cf},{957ff224-9b08-4ea6-97e8-f592d5fc610b}
求导得\begin{equation*}
	\phi'(x) = f(x) + (x-a)~f'(x) - f(x)
	= (x-a)~f'(x).
\end{equation*}
因为在\((a,b)\)内有\(f'(x) \leq 0\),
所以在\((a,b)\)内有\(\phi'(x) \leq 0\),
即\(\phi\)在\((a,b)\)上严格单调减少,
又因为\(\phi(a) = 0\),
所以在\((a,b)\)内有\(\phi(x) \leq 0\),
从而有\(F'(x) \leq 0\)成立.
\end{proof}
\end{example}
\begin{example}
%@see: 《高等数学(第六版 上册)》 P244 习题5-2 13.
设\(F(x) = \int_0^x \frac{\sin t}{t} \dd{t}\).
求\(F'(0)\).
\begin{solution}
显然\(F(0) = 0\).
注意到点\(x=0\)是\(f(x) = \frac{\sin x}{x}\)的间断点,
不能直接对\(F\)求导,
于是根据导数的定义可知\begin{equation*}
	F'(0) = \lim_{x\to0} \frac{F(x)-F(0)}{x-0}
	= \lim_{x\to0} \frac1x \int_0^x \frac{\sin t}{t} \dd{t}
	= \lim_{x\to0} \frac{\sin x}{x}
	= 1.
\end{equation*}
\end{solution}
\end{example}

\subsection{牛顿--莱布尼茨公式}
现在我们根据\hyperref[theorem:定积分.原函数存在定理]{原函数存在定理}%
来证明一个重要定理,
它给出了用原函数计算定积分的公式.
\begin{theorem}
%@see: 《高等数学(第六版 上册)》 P239 定理3
%@see: 《数学分析(第二版 上册)》(陈纪修) P298 定理7.3.2(微积分基本定理)
设函数\(F\)是连续函数\(f\)在区间\([a,b]\)上的一个原函数,
则\begin{equation}\label{equation:定积分.牛顿--莱布尼茨公式}
	\int_a^b f(x) \dd{x} = F(b) - F(a).
\end{equation}
\begin{proof}
已知函数\(F\)是连续函数\(f\)的一个原函数,
又根据\hyperref[theorem:定积分.原函数存在定理]{原函数存在定理}知道,
积分上限的函数\begin{equation*}
	\Phi(x) = \int_a^x f(t) \dd{t}
\end{equation*}也是\(f\)的一个原函数.
于是这两个原函数之差\(F(x) - \Phi(x)\)在\([a,b]\)上必定是某一个常数\(C\),
即\begin{equation*}
	F(x) - \Phi(x) = C. \eqno(1)
\end{equation*}

在上式中令\(x=a\),得\(F(a) - \Phi(a) = C\).
又由\(\Phi(x)\)的定义式及定积分的补充规定可知\(\Phi(a) = 0\),因此,\(C = F(a)\).
以\(F(a)\)代入(1)式中的\(C\),
以\(\int_a^x f(t) \dd{t}\)代入(1)式中的\(\Phi(x)\),
可得\begin{equation*}
	\int_a^x f(t) \dd{t} = F(x) - F(a).
\end{equation*}
在上式中令\(x=b\),
就得到所要证明的公式,
并称之为\DefineConcept{牛顿--莱布尼茨公式},
或\DefineConcept{微积分基本公式}.
\end{proof}
\end{theorem}
为求简便,可将\(F(b) - F(a)\)记成\(\eval{F(x)}_a^b\),
于是又有\begin{equation*}
	\int_a^b f(x) \dd{x} = \eval{F(x)}_a^b.
\end{equation*}

牛顿--莱布尼茨公式进一步揭示了定积分与被积函数的原函数或不定积分之间的联系.
它表明:一个连续函数在区间\([a,b]\)上的定积分等于
它的任一个原函数在区间\([a,b]\)上的增量.
这就给定积分提供了一个有效而简便的计算方法,大大简化了定积分的计算手续.

\begin{example}
%@see: 《高等数学(第六版 上册)》 P240 例1
计算\(\int_0^1 x^2 \dd{x}\).
\begin{solution}
由于\(F(x) = x^3/3\)是\(x^2\)的一个原函数,所以按牛顿--莱布尼茨公式,有\begin{equation*}
	\int_0^1 x^2 \dd{x} = \left[\frac{x^3}{3}\right]_0^1
	= \frac{1^3}{3} - \frac{0^3}{3} = \frac1{3} - 0 = \frac1{3}.
\end{equation*}
\end{solution}
\end{example}

\begin{example}
%@see: 《高等数学(第六版 上册)》 P240 例2
计算\(\int_{-2}^{-1} \frac{\dd{x}}{x}\).
\begin{solution}
当\(x<0\)时,\(1/x\)的一个原函数是\(\ln\abs{x}\),
现在积分区间是\([-2,-1]\),
这个原函数的形式可以化为\(\ln(-x)\),
所以按牛顿--莱布尼茨公式,有\begin{equation*}
	\int_{-2}^{-1} \frac{\dd{x}}{x}
	= [ \ln\abs{x} ]_{-2}^{-1}
	= \ln1 - \ln2
	= -\ln2.
\end{equation*}
\end{solution}
\end{example}

\begin{proposition}\label{theorem:定积分.积分中值定理1推论2改进}
%@see: 《高等数学(第六版 上册)》 P241 例6
设函数\(f\)在\([a,b]\)上连续,
则\(\exists\xi\in(a,b)\),
使\begin{equation*}
	\int_a^b f(x) \dd{x} = f(\xi) (b-a).
\end{equation*}
\begin{proof}
因\(f\)连续,故它的原函数存在,
设\(F\)为\(f\)的一个原函数,
即在\([a,b]\)上\(F'(x) = f(x)\).
根据\hyperref[equation:定积分.牛顿--莱布尼茨公式]{牛顿--莱布尼茨公式},
有\begin{equation*}
	\int_a^b f(x) \dd{x}
	= F(b) - F(a).
\end{equation*}

由\cref{theorem:定积分.变限积分定理} 可知,
函数\(F\)在区间\([a,b]\)上可导,
满足\hyperref[theorem:微分中值定理.拉格朗日中值定理]{拉格朗日中值定理}的条件,
因此存在实数\(\xi\in(a,b)\),使\begin{equation*}
	F(b) - F(a) = F'(\xi) (b-a),
\end{equation*}
也即\begin{equation*}
	\int_a^b f(x) \dd{x} = f(\xi) (b-a).
	\qedhere
\end{equation*}
\end{proof}
\end{proposition}
\begin{remark}
我们也把\cref{theorem:定积分.积分中值定理1推论2改进} 称为“积分中值定理”.
它正是对\cref{theorem:定积分.积分中值定理1推论2} 的改进.
从证明过程中不难看出积分中值定理
与\hyperref[theorem:微分中值定理.拉格朗日中值定理]{微分中值定理}的联系.
\end{remark}

\begin{example}
%@see: https://www.bilibili.com/video/BV19M411j7uq/
设函数\(f\)在\([0,1]\)上二阶连续可导,
\(f(x)\geq0,
f''(x)<0\),
令\begin{equation*}
	I_n = \int_0^1 f(x^n) \dd{x}
	\quad(n\in\mathbb{N}).
\end{equation*}
\begin{itemize}
	\item 证明:\(I_n \leq f\left( \frac1{n+1} \right)\).
	\item 设\(f(0) = 0\),求极限\(\lim_{n\to\infty} I_n\).
\end{itemize}
\begin{solution}
写出\(f\)的泰勒公式:\begin{equation*}
	f(x) = f(x_0) + f'(x_0) (x-x_0) + \frac1{2!} f''(\xi) (x-x_0)^2
	\quad(\text{$\xi$在$x$与$x_0$之间}).
\end{equation*}
因为\(f''(\xi)<0\),
所以\begin{equation*}
	f(x) \leq f(x_0) + f'(x_0) (x-x_0).
\end{equation*}
用\(x^n\)代\(x\),用\(\frac1{n+1}\)代\(x_0\),得\begin{equation*}
	f(x^n)
	\leq
	f\left(\frac1{n+1}\right) + f'\left(\frac1{n+1}\right) \left(x^n-\frac1{n+1}\right).
\end{equation*}
积分得\begin{equation*}
	I_n = \int_0^1 f(x^n) \dd{x}
	\leq f\left(\frac1{n+1}\right)
	+ f'\left(\frac1{n+1}\right) \int_0^1 \left(x^n-\frac1{n+1}\right) \dd{x}.
\end{equation*}
这里\begin{equation*}
	\int_0^1 \left(x^n-\frac1{n+1}\right) \dd{x} = 0,
\end{equation*}
故\begin{equation*}
	I_n \leq f\left(\frac1{n+1}\right).
\end{equation*}

接下来对不等式\(0 < I_n \leq f\left(\frac1{n+1}\right)\)
令\(n\to\infty\)得\begin{equation*}
	0 \leq \lim_{n\to\infty} I_n \leq f(0) = 0,
\end{equation*}
于是\(\lim_{n\to\infty} I_n = 0\).
\end{solution}
\end{example}

\begin{example}
%@see: 《高等数学(第六版 上册)》 P241 例7
%@see: 《数学分析(第二版 上册)》(陈纪修) P310 习题 3.
\def\fu{\displaystyle \int_0^x t f(t) \dd{t}}
\def\fv{\displaystyle \int_0^x f(t) \dd{t}}
\def\fvv{\left[ \fv \right]^2}
\def\fw{\displaystyle \int_0^x (x-t) f(t) \dd{t}}
设\(f\)在\([0,+\infty)\)内连续且\(f(x) > 0\).
证明:函数\begin{equation*}
	F(x) = \frac{\fu}{\fv}
\end{equation*}在\([0,+\infty)\)内为单调增加函数.
\begin{proof}
因为\begin{equation*}
	\dv{x} \fu = x f(x),
	\qquad
	\dv{x} \fv = f(x),
\end{equation*}
所以\begin{equation*}
	F'(x) = \frac{x f(x) \fv - f(x) \fu}{\fvv}
	= \frac{f(x) \fw}{\fvv}.
\end{equation*}
按假设,当\(0 < t < x\)时,\(f(t) > 0\),\((x-t) f(t) > 0\),
根据\hyperref[theorem:定积分.积分中值定理1推论2改进]{改进后的积分中值定理}可知\begin{equation*}
	\fv > 0, \qquad \fw > 0,
\end{equation*}
所以\(F'(x) > 0\ (x > 0)\),
从而\(F\)在\((0,+\infty)\)内为单调增加函数.
\end{proof}
\end{example}

\begin{example}
%@see: 《高等数学(第六版 上册)》 P241 例8
求\(\lim_{x\to0} \frac1{x^2} \int_{\cos x}^1 \exp(-t^2) \dd{t}\).
\begin{solution}
已知这是一个\(0/0\)型未定式,我们利用洛必达法则来计算.
分子可写成\begin{equation*}
	- \int_1^{\cos x} \exp(-t^2) \dd{t},
\end{equation*}
它是以\(\cos x\)为上限的积分,
作为\(x\)的函数可看成是以\(u = \cos x\)为中间变量的复合函数,
故\begin{align*}
	\dv{x} \int_{\cos x}^1 \exp(-t^2) \dd{t}
	&= -\dv{x} \int_1^{\cos x} \exp(-t^2) \dd{t} \\
	&= -\left[ \dv{u}\int_1^u \exp(-t^2) \dd{t} \right]_{u=\cos x} \cdot (\cos x)' \\
	&= -\exp(-\cos^2 x) \cdot (-\sin x) \\
	&= \sin x \exp(-\cos^2 x).
\end{align*}
因此\begin{equation*}
	\lim_{x\to0} \frac1{x^2} \int_{\cos x}^1 \exp(-t^2) \dd{t}
	= \lim_{x\to0} \frac{\sin x \exp(-\cos^2 x)}{2x}
	= \frac1{2e}.
\end{equation*}
\end{solution}
\end{example}

\begin{example}
%@see: 《数学分析(第二版 上册)》(陈纪修) P299 例7.3.5
计算\(\lim_{n\to\infty} \left(\frac1{n+1}+\frac1{n+2}+\dotsb+\frac1{2n}\right)\).
\begin{solution}
将和式改写为\begin{equation*}
	\frac1{n+1}+\frac1{n+2}+\dotsb+\frac1{2n}
	= \frac1n \left(\frac1{1+1/n}+\frac1{1+2/n}+\dotsb+\frac1{1+n/n}\right),
\end{equation*}
这相当于在\([0,1]\)中对函数\(f(x) = \frac1{1+x}\)作\(\increment x_i = \frac1n\)的等距分割后,
在小区间\([x_{i-1},x_i]\)上将\(\xi_i\)取为\(x_i\ (i=1,2,\dotsc,n)\)的积分和\begin{equation*}
	\sum_{i=1}^n f(\xi_i) \increment x_i.
\end{equation*}
于是\begin{align*}
	\lim_{n\to\infty} \left(\frac1{n+1}+\frac1{n+2}+\dotsb+\frac1{2n}\right)
	&= \lim_{\lambda\to0} \sum_{i=1}^n \frac1{1+\xi_i} \increment x_i \\
	&= \int_0^1 \frac{\dd{x}}{1+x}
	= \eval{\ln(1+x)}_0^1
	= \ln2.
\end{align*}
\end{solution}
\end{example}

\begin{example}
%@see: 《数学分析(第二版 上册)》(陈纪修) P313 习题 21.
设函数\(f\)的导数\(f'\)在\([a,b]\)上连续.
证明:\begin{equation*}
	\max_{a \leq x \leq b} \abs{f(x)}
	\leq \abs{\frac1{b-a} \int_a^b f(x) \dd{x}}
	+ \int_a^b \abs{f'(x)} \dd{x}.
\end{equation*}
\begin{proof}
因为\(f\)在\([a,b]\)上连续,
所以由\hyperref[theorem:极限.最值定理]{最值定理}可知,
\(f\)必能取得它的最大值和最小值,
不妨设\begin{gather*}
	\abs{f(\xi)} = \max_{a \leq x \leq b} \abs{f(x)},
	\quad\xi\in[a,b], \\
	\abs{f(\eta)} = \min_{a \leq x \leq b} \abs{f(x)},
	\quad\xi\in[a,b].
\end{gather*}
于是\begin{align*}
	\max_{a \leq x \leq b} \abs{f(x)} - \min_{a \leq x \leq b} \abs{f(x)}
	&= \abs{f(\xi)} - \abs{f(\eta)} \\
	&\leq \abs{f(\xi) - f(\eta)}
		\tag{\hyperref[theorem:不等式.三角不等式2]{三角不等式}} \\
	&= \abs{\int_\eta^\xi f'(x) \dd{x}}
		\tag{\hyperref[equation:定积分.牛顿--莱布尼茨公式]{牛顿--莱布尼茨公式}} \\
	&\leq \int_\eta^\xi \abs{f'(x)} \dd{x}
		\tag{\hyperref[theorem:定积分.定积分性质5推论2]{绝对可积性}} \\
	&\leq \int_a^b \abs{f'(x)} \dd{x}.
		\tag{\cref{theorem:定积分.定积分性质5推论3}}
\end{align*}

由\hyperref[theorem:定积分.积分中值定理1推论2]{积分中值定理}可知,
存在\(\zeta\in[a,b]\),使得\begin{equation*}
	f(\zeta) = \frac1{b-a} \int_a^b f(x) \dd{x}.
\end{equation*}
于是\begin{equation*}
	\min_{a \leq x \leq b} \abs{f(x)}
	\leq \abs{f(\zeta)}
	= \abs{\frac1{b-a} \int_a^b f(x) \dd{x}}.
\end{equation*}

综上所述,我们有\begin{align*}
	\max_{a \leq x \leq b} \abs{f(x)}
	&= \max_{a \leq x \leq b} \abs{f(x)} - \min_{a \leq x \leq b} \abs{f(x)}
		+ \min_{a \leq x \leq b} \abs{f(x)} \\
	&\leq \int_a^b \abs{f'(x)} \dd{x}
		+ \abs{\frac1{b-a} \int_a^b f(x) \dd{x}}.
	\qedhere
\end{align*}
\end{proof}
\end{example}

\section{定积分的换元积分法}
由上节结果知道,计算定积分\(\int_a^b f(x) \dd{x}\)的简便方法是
把它转化为求\(f(x)\)的原函数的增量.
在\cref{chapter:不定积分}中,
我们知道用换元积分法和分部积分法可以求出一些函数的原函数.
因此,在一定条件下,可以用换元积分法和分部积分法来计算定积分.
下面就来讨论定积分的这两种计算方法.

为了说明如何用换元法来计算定积分,先证明下面的定理.
\begin{theorem}\label{theorem:定积分.定积分的换元法}
%@see: 《数学分析(第二版 上册)》(陈纪修) P303 定理7.3.4
%@see: 《数学分析教程(第3版 上册)》(史济怀) P258 定理6.4.2
设函数\(f \in C[a,b]\),
函数\(\phi \in C^1[\alpha,\beta]\),
\(\ran\phi \subseteq [a,b]\),
\(\phi(\alpha)=a,
\phi(\beta)=b\),
则有\begin{equation}\label{equation:定积分.定积分的换元公式}
	\int_a^b f(x) \dd{x} = \int_\alpha^\beta f[\phi(t)] \phi'(t) \dd{t}.
\end{equation}
\begin{proof}
由假设可知,
被积函数\(f(x)\)和\(f[\phi(t)] \phi'(t)\)都是连续的,
因此不仅\cref{equation:定积分.定积分的换元公式} 两边的定积分都存在,
而且由\hyperref[theorem:定积分.原函数存在定理]{原函数存在定理}知道,
被积函数的原函数也都存在.
所以\cref{equation:定积分.定积分的换元公式} 两边的定积分
都可应用\hyperref[equation:定积分.牛顿--莱布尼茨公式]{牛顿--莱布尼茨公式}.
假设\(F(x)\)是\(f(x)\)的一个原函数,则\begin{equation*}
	\int_a^b f(x) \dd{x} = F(b) - F(a).
\end{equation*}
另一方面,记\(\Phi(t) = F[\phi(t)]\),
它是由\(F(x)\)与\(x=\phi(t)\)复合而成的函数.
由复合函数求导法则,得\begin{equation*}
	\Phi'(t) = \dv{F}{x} \dv{x}{t}
	= f(x) \phi'(t)
	= f[\phi(t)] \phi'(t).
\end{equation*}
这表明\(\Phi(t)\)是\(f[\phi(t)] \phi'(t)\)的一个原函数,
因此有\begin{equation*}
	\int_\alpha^\beta f[\phi(t)] \phi'(t) \dd{t}
	= \Phi(\beta) - \Phi(\alpha).
\end{equation*}
又由\(\Phi(t) = F[\phi(t)]\),
\(\phi(\alpha) = a\),
\(\phi(\beta) = b\),
可得\begin{equation*}
	\Phi(\beta) - \Phi(\alpha)
	= F[\phi(\beta)] - F[\phi(\alpha)]
	= F(b) - F(a).
\end{equation*}
所以\begin{equation*}
	\int_a^b f(x) \dd{x}
	= F(b) - F(a)
	= \Phi(\beta) - \Phi(\alpha)
	= \int_\alpha^\beta f[\phi(t)] \phi'(t) \dd{t}.
	\qedhere
\end{equation*}
\end{proof}
\end{theorem}
在定积分\(\int_a^b f(x) \dd{x}\)中的\(\dd{x}\),本来是整个定积分记号中不可分割的一部分,但由上述定理可知,在一定条件下,它确实可以作为微分记号来对待.
这就是说,应用换元公式时,如果把\(\int_a^b f(x) \dd{x}\)中的\(x\)换成\(\phi(t)\),则\(\dd{x}\)就换成\(\phi'(t) \dd{t}\),这正好是\(x = \phi(t)\)的微分\(\dd{x}\).

应用换元公式时,有两点值得注意:\begin{enumerate}
\item 用\(x = \phi(t)\)把原来变量\(x\)代换成新变量\(t\)时,积分限也要换成相应于新变量\(t\)的积分限;
\item 求出\(f[\phi(t)] \phi'(t)\)的一个原函数\(\Phi(t)\)后,不必像计算不定积分那样再要把\(\Phi(t)\)变换成原来变量\(x\)的函数,而只要把新变量\(t\)的上、下限分别代入\(\Phi(t)\)中然后相减就行了.
\end{enumerate}

\begin{example}
%@see: 《高等数学(第六版 上册)》 P246 例1
计算\(\int_0^a \sqrt{a^2-x^2} \dd{x}\ (a > 0)\).
\begin{solution}
令\(x = a \sin t\ (-\frac\pi2 \leq t \leq \frac\pi2)\),
则\begin{equation*}
	\dd{x} = a \cos t \dd{t}, \qquad
	\sqrt{a^2-x^2} = a \cos t.
\end{equation*}
当\(x = 0\)时,取\(t = 0\).
当\(x = a\)时,取\(t = \frac{\pi}{2}\).
那么有\begin{equation*}
	\int_0^a \sqrt{a^2-x^2} \dd{x}
	= \int_0^{\frac{\pi}{2}} a \cos t \cdot a \cos t \dd{t}
	= \frac{a^2}{2} \int_0^{\frac{\pi}{2}} (1+\cos 2 t) \dd{t}
	= \frac{\pi a^2}{4}.
\end{equation*}
\end{solution}
\end{example}

\begin{example}
%@see: 《高等数学(第六版 上册)》 P246 例2
计算\(\int_0^{\frac\pi2} \cos^5 x \sin x \dd{x}\).
\begin{solution}
设\(t = \cos x\),
则\(\dd{t} = -\sin x \dd{x}\).
当\(x=0\)时,有\(t=1\).
当\(x=\frac\pi2\)时,有\(t=0\).
于是\begin{equation*}
	\int_0^{\frac\pi2} \cos^5 x \sin x \dd{x}
	= -\int_1^0 t^5 \dd{t}
	= \int_0^1 t^5 \dd{t}
	= \eval{\frac{t^6}6}_0^1
	= \frac16.
\end{equation*}
\end{solution}
\end{example}

\begin{example}
%@see: 《高等数学(第六版 上册)》 P246 例3
计算\(\int_0^\pi \sqrt{\sin^3 x - \sin^5 x} \dd{x}\).
\begin{solution}
由于\begin{equation*}
	\sqrt{\sin^3 x - \sin^5 x}
	= \sqrt{\sin^3 x (1 - \sin^2 x)}
	= \sin^{\frac32} x \cdot \abs{\cos x},
\end{equation*}
其中\begin{equation*}
	\abs{\cos x} = \left\{ \begin{array}{rl}
		\cos x, & 0 \leq x \leq \frac\pi2, \\
		-\cos x, & \frac\pi2 < x \leq \pi,
	\end{array} \right.
\end{equation*}
所以\begin{align*}
	\int_0^\pi \sqrt{\sin^3 x - \sin^5 x} \dd{x}
	&= \int_0^{\frac\pi2} \sin^{\frac32} x \cdot \cos x \dd{x}
		+ \int_{\frac\pi2}^\pi \sin^{\frac32} x \cdot (-\cos x) \dd{x} \\
	&= \int_0^{\frac\pi2} \sin^{\frac32} x \dd(\sin x)
		- \int_{\frac\pi2}^\pi \sin^{\frac32} x \dd(\sin x) \\
	&= \frac25 \eval{\sin^{\frac52} x}_0^{\frac\pi2}
		- \frac25 \eval{\sin^{\frac52} x}_{\frac\pi2}^\pi
	= \frac45.
\end{align*}
\end{solution}
\end{example}

\begin{example}
%@see: 《高等数学(第六版 上册)》 P247 例4
计算\(\int_0^4 \frac{x+4}{\sqrt{2x+1}} \dd{x}\).
\begin{solution}
设\(\sqrt{2x+1}=t\),
则\(x=\frac{t^2-1}2\)且\(\dd{x}=t\dd{t}\).
当\(x=0\)时,有\(t=1\).
当\(x=4\)时,有\(t=3\).
于是\begin{align*}
	\int_0^4 \frac{x+2}{\sqrt{2x+1}} \dd{x}
	&= \int_1^3 \left(\frac{t^2-1}2+2\right) \dd{t} \\
	&= \frac12 \int_1^3 (t^2+3) \dd{t} \\
	&= \frac12 \eval{\left(\frac13 t^3 + 3t\right)}_1^3
	= \frac{22}3.
\end{align*}
\end{solution}
\end{example}

\begin{example}
%@see: https://www.bilibili.com/video/BV1TK411k7t6/
计算\(\int_0^1 \floor{e^x+1} \dd{x}\).
\begin{solution}
令\(t = e^x+1\),
得\(x = \ln(t-1),
\dd{x} = \frac{\dd{t}}{t-1}\).
当\(x=0\)时,\(t=2\);
当\(x=1\)时,\(t=e+1\).
于是\begin{equation*}
	I = \int_0^1 \floor{e^x+1} \dd{x}
	= \int_2^{e+1} \floor{t} \cdot \frac{\dd{t}}{t-1}.
\end{equation*}
由于\(2<e<3\),
所以\begin{align*}
	I &= \left( \int_2^3 + \int_3^{e+1} \right) \floor{t} \cdot \frac{\dd{t}}{t-1} \\
	&= \int_2^3 2 \cdot \frac{\dd{t}}{t-1}
	+ \int_3^{e+1} 3 \cdot \frac{\dd{t}}{t-1} \\
	&= 2 \eval{\ln(t-1)}_2^3 + 3 \eval{\ln(t-1)}_3^{e+1}
	= 3 - \ln2.
\end{align*}
\end{solution}
\end{example}
\begin{example}
%@see: https://www.bilibili.com/video/BV16M411V7Ni/
计算\(\int_{1/6}^6 \frac1x \floor*{\frac1{\sqrt{x}}} \dd{x}\).
\begin{solution}
令\(t = 1/\sqrt{x}\),
得\(x = t^{-2},
\dd{x} = -2 t^{-3} \dd{t}\).
当\(x=1/6\)时,\(t=\sqrt6\);
当\(x=6\)时,\(t=1/\sqrt6\).
于是\begin{equation*}
	I = \int_{1/6}^6 \frac1x \floor*{\frac1{\sqrt{x}}} \dd{x}
	= \int_{\sqrt6}^{1/\sqrt6} t^2 \floor{t} \cdot (-2) t^{-3} \dd{t}
	= 2 \int_{1/\sqrt6}^{\sqrt6} \frac{\floor{t}}{t} \dd{t}.
\end{equation*}
由于\(2 = \sqrt4 < \sqrt6 < \sqrt9 = 3,
0 < 1/\sqrt6 < 1/2 < 1\),
所以\begin{align*}
	I &= 2 \left( \int_{1/\sqrt6}^1 + \int_1^2 + \int_2^{\sqrt6} \right) \frac{\floor{t}}{t} \dd{t} \\
	&= 2 \left( \int_{1/\sqrt6}^1 \frac0t \dd{t} + \int_1^2 \frac1t \dd{t} + \int_2^{\sqrt6} \frac2t \dd{t} \right) \\
	&= 2 \left( 0 + \eval{\ln t}_1^2 + 2 \eval{\ln t}_2^{\sqrt6} \right)
	= 2 \ln3.
\end{align*}
\end{solution}
%@Mathematica: Integrate[1/x Floor[1/Sqrt[x]], {x, 1/6, 6}] // FullSimplify
\end{example}

\begin{example}
%@see: 《高等数学(第六版 上册)》 P254 习题5-3 6.
证明:\begin{itemize}
	\item 若\(f\)是连续的奇函数,则\(\int_0^x f(t) \dd{t}\)是偶函数.
	\item 若\(f\)是连续的偶函数,则\(\int_0^x f(t) \dd{t}\)是奇函数.
\end{itemize}
\begin{proof}
设\(F(x) = \int_0^x f(t) \dd{t}\),
那么\begin{equation*}
	F(-x) = \int_0^{-x} f(t) \dd{t}
	\xlongequal{t=-u} \int_0^x f(-u) \dd(-u).
\end{equation*}
\begin{itemize}
	\item 若\(f\)是连续的奇函数,那么\(f(-u) = -f(u)\),所以\begin{equation*}
		F(-x) = \int_0^x f(u) \dd{u} = F(x),
	\end{equation*}
	即\(F\)是偶函数;

	\item 若\(f\)是连续的偶函数,那么\(f(-u) = f(u)\),所以\begin{equation*}
		F(-x) = -\int_0^x f(u) \dd{u} = -F(x),
	\end{equation*}
	即\(F\)是奇函数.
	\qedhere
\end{itemize}
\end{proof}
\end{example}

\begin{proposition}\label{theorem:定积分.利用对称性简化计算0}
设\(f \in R[-a,a]\),
则\begin{equation}
	\int_{-a}^a f(x) \dd{x} = \int_0^a [f(x) + f(-x)] \dd{x}.
\end{equation}
\begin{proof}
由\cref{theorem:定积分.定积分性质3} 有\begin{equation*}
	\int_{-a}^a f(x) \dd{x}
	= \int_{-a}^0 f(x) \dd{x}
	+ \int_0^a f(x) \dd{x};
\end{equation*}
由\cref{equation:定积分.交换上下限改变定积分的符号} 有\begin{equation*}
	\int_{-a}^0 f(x) \dd{x}
	\xlongequal{x=-t} \int_a^0 f(-t) \dd(-t)
	= \int_0^a f(-t) \dd{t}
	= \int_0^a f(-x) \dd{x};
\end{equation*}
故由\cref{theorem:定积分.定积分性质1} 有\begin{equation*}
	\int_{-a}^a f(x) \dd{x}
	= \int_0^a f(-x) \dd{x} + \int_0^a f(x) \dd{x}
	= \int_0^a [f(-x) + f(x)] \dd{x}.
	\qedhere
\end{equation*}
\end{proof}
\end{proposition}

\begin{proposition}\label{theorem:定积分.利用对称性简化计算1}
%@see: 《高等数学(第六版 上册)》 P247 例5
%@see: 《数学分析(第二版 上册)》(陈纪修) P308 定理7.3.5
设\(f \in R[-a,a]\).
\begin{itemize}
	\item 若函数\(f\)在区间\([-a,a]\)上是偶函数,
	则\begin{equation*}
		\int_{-a}^a f(x) \dd{x} = 2 \int_0^a f(x) \dd{x}.
	\end{equation*}

	\item 若函数\(f\)在区间\([-a,a]\)上是奇函数,
	则\begin{equation*}
		\int_{-a}^a f(x) \dd{x} = 0.
	\end{equation*}
\end{itemize}
\begin{proof}
由\cref{theorem:定积分.利用对称性简化计算0} 有\begin{equation*}
	\int_{-a}^a f(x) \dd{x}
	= \int_0^a [f(-x) + f(x)] \dd{x}.
\end{equation*}

接下来我们对函数\(f\)分情况讨论.
\begin{itemize}
	\item 若\(f\)是偶函数,则\begin{equation*}
		f(x) + f(-x) = 2 f(x),
	\end{equation*}
	从而\begin{equation*}
		\int_{-a}^a f(x) \dd{x} = 2 \int_0^a f(x) \dd{x};
	\end{equation*}

	\item 若\(f\)是奇函数,则\begin{equation*}
		f(x) + f(-x) = 0,
	\end{equation*}
	从而\begin{equation*}
		\int_{-a}^a f(x) \dd{x} = 0.
		\qedhere
	\end{equation*}
\end{itemize}
\end{proof}
\end{proposition}
\begin{remark}
常用\cref{theorem:定积分.利用对称性简化计算1} 简化奇、偶函数在对称于原点的区间上的定积分.

当积分区间不是关于原点对称的区间时,
我们可以利用坐标变换公式,
对被积函数换元,
将积分区间平移,把它变成对称于原点的区间:\begin{equation}\label{equation:定积分.区间平移}
	\int_a^b f(x) \dd{x}
	\xlongequal{x=t+(a+b)/2}
	\int_{-\frac{b-a}2}^{\frac{b-a}2} f\left(t+\frac{a+b}2\right) \dd{t}.
\end{equation}

当我们处理非奇非偶函数函数\(f\)时,
可以利用我们从\cref{example:函数.任一函数可拆为奇偶函数之和} 得到的公式\begin{equation*}
	g(x) = \frac{f(x) + f(-x)}{2}, \qquad
	h(x) = \frac{f(x) - f(-x)}{2},
\end{equation*}
将\(f\)中的偶函数\(g\)和奇函数\(h\)分别抽离出,
再利用\cref{theorem:定积分.利用对称性简化计算1} 简化计算.
\end{remark}

\begin{example}
%@see: 《2018年全国硕士研究生入学统一考试(数学一)》一选择题/第4题
设\begin{equation*}
	M = \int_{-\frac\pi2}^{\frac\pi2} \frac{(1+x)^2}{1+x^2} \dd{x},
	\qquad
	N = \int_{-\frac\pi2}^{\frac\pi2} \frac{1+x}{e^x} \dd{x},
	\qquad
	K = \int_{-\frac\pi2}^{\frac\pi2} (1+\sqrt{\cos x}) \dd{x},
\end{equation*}
比较三个积分的大小.
\begin{solution}
%@see: https://www.bilibili.com/video/BV1no23Y4EFu/
注意到第一个定积分的被积函数\begin{equation*}
	\frac{(1+x)^2}{1+x^2}
	= \frac{1+2x+x^2}{1+x^2}
\end{equation*}含有一项关于\(x\)的奇函数\(\frac{x}{1+x^2}\),而积分区间是对称的,
由\cref{theorem:定积分.利用对称性简化计算1} 可知\begin{equation*}
	M = \int_a^b 1 \dd{x}.
\end{equation*}
接下来比较三个定积分的被积函数的大小,可得\begin{equation*}
	\frac{1+x}{e^x} \leq 1 \leq 1 + \sqrt{\cos x},
\end{equation*}
那么由定积分\cref{theorem:定积分.定积分性质5推论1} 可知\begin{equation*}
	N < M < K.
\end{equation*}
\end{solution}
\end{example}

\begin{corollary}\label{theorem:定积分.积分上限函数的奇偶性}
设\(f \in R(-\infty,+\infty)\),
积分上限函数\(g(x) = \int_0^x f(t) \dd{t}\).
\begin{itemize}
	\item 若\(f\)是偶函数,则\(g\)是奇函数;
	\item 若\(f\)是奇函数,则\(g\)是偶函数.
\end{itemize}
\begin{proof}
因为\begin{align*}
	g(-x) &= \int_0^{-x} f(t) \dd{t} \\
	&\xlongequal{t=-u}
	-\int_0^x f(-u) \dd{u},
\end{align*}
所以当\(f\)是偶函数,即\(f(x) = f(-x)\)时,
有\begin{equation*}
	g(-x) = -\int_0^x f(u) \dd{u} = -g(x),
\end{equation*}
即\(g\)是奇函数;
当\(f\)是奇函数,即\(f(-x) = -f(x)\)时,
有\begin{equation*}
	g(-x) = -\int_0^x (-f(u)) \dd{u} = g(x),
\end{equation*}
即\(g\)是偶函数.
\end{proof}
\end{corollary}

\begin{proposition}\label{theorem:定积分.具有轴对称图像的函数的积分}
%@see: 《数学分析(第二版 上册)》(陈纪修) P312 习题 12.
设函数\(f\)在\([a,b]\)上可积,图形\(y=f(x)\)关于\(x=T\)对称,则\begin{equation}
	\int_a^b f(x) \dd{x} = \int_a^{2T-b} f(x) \dd{x} + 2 \int_T^b f(x) \dd{x}.
\end{equation}
\begin{proof}
不妨设\(a < T < b\).
由\hyperref[theorem:定积分.定积分性质3]{区间可加性}有\begin{equation*}
	\int_a^b f(x) \dd{x}
	= \int_a^{2T-b} f(x) \dd{x}
	+ \int_{2T-b}^T f(x) \dd{x}
	+ \int_T^b f(x) \dd{x}.
\end{equation*}
因为\(y=f(x)\)关于\(x=T\)对称,
所以\(f(x) = f(2T-x)\),
那么\begin{equation*}
	\int_{2T-b}^T f(x) \dd{x}
	\xlongequal{y=2T-x}
	\int_b^T f(2T-y) \dd(2T-y)
	= \int_T^b f(y) \dd{y},
\end{equation*}
于是\(\int_a^b f(x) \dd{x} = \int_a^{2T-b} f(x) \dd{x} + 2 \int_T^b f(x) \dd{x}\).
\end{proof}
\end{proposition}
\begin{remark}
根据\cref{theorem:定积分.具有轴对称图像的函数的积分} 的结论,
结合\cref{figure:定积分.具有轴对称图像的函数的积分} 可以看出:
如果一个函数的图像关于直线\(x=T\)对称,
那么对于对称轴两侧的任意两个区间\([p_1,q_1],[p_2,q_2]\),
只要这两个区间也是关于对称轴\(x=T\)对称的,
即成立\(p_1 + q_2 = q_1 + p_2 = 2T\),
那么对应的定积分\begin{equation*}
	\int_{p_1}^{q_1} f(x) \dd{x}
	\quad\text{和}\quad
	\int_{p_2}^{q_2} f(x) \dd{x}
\end{equation*}必定相等.
\end{remark}
\begin{figure}[hbt]
	\centering
	\begin{tikzpicture}
		\begin{axis}[
			xscale=2,
			xmin=4,xmax={3*pi+1},
			domain=5:{3*pi},
			ymin=0,ymax=2.2,
			axis x line=middle,
			axis y line=none,
			xtick={5,2*pi,2.5*pi,3*pi},
			xticklabels={$a\vphantom{T}$,$2T-b$,$T$,$b$},
		]
			\addplot[color=blue,samples=20,smooth]{.5*sin(deg(x))+.7};
			\begin{scope}[dashed]
				\draw(2.5*pi,0)--(2.5*pi,{.5*sin(deg(2.5*pi))+.7});
				\draw(2*pi,0)--(2*pi,.7);
				\draw(3*pi,0)--(3*pi,.7);
				\draw(5,0)--(5,{.7+.5*sin(deg(5))});
			\end{scope}
		\end{axis}
	\end{tikzpicture}
	\caption{}
	\label{figure:定积分.具有轴对称图像的函数的积分}
\end{figure}

\begin{example}
设\(f \in R[0,2a]\).
证明:\begin{equation}\label{equation:定积分.区间折半}
	\int_0^{2a} f(x) \dd{x} = \int_0^a [f(x) + f(2a-x)] \dd{x}.
\end{equation}
\begin{proof}
由\cref{theorem:定积分.定积分性质3} 有\begin{equation*}
	\int_0^{2a} f(x) \dd{x}
	= \int_0^a f(x) \dd{x} + \int_a^{2a} f(x) \dd{x};
\end{equation*}
由\cref{equation:定积分.交换上下限改变定积分的符号} 有\begin{equation*}
	\int_a^{2a} f(x) \dd{x}
	\xlongequal{x=2a-t} \int_a^0 f(2a-t) \dd(2a-t)
	= \int_0^a f(2a-t) \dd{t}
	= \int_0^a f(2a-x) \dd{x};
\end{equation*}
故由\cref{theorem:定积分.定积分性质1} 有\begin{equation*}
	\int_0^{2a} f(x) \dd{x}
	= \int_0^a f(x) \dd{x} + \int_0^a f(2a-x) \dd{x}
	= \int_0^a [f(x) + f(2a-x)] \dd{x}.
	\qedhere
\end{equation*}
\end{proof}
\end{example}

\begin{proposition}\label{theorem:定积分.区间再现}
%@see: 《高等数学(第六版 上册)》 P254 习题5-3 2.
设\(f \in C[a,b]\).
证明:\begin{equation}
	\int_a^b f(x) \dd{x}
	= \int_a^b f(a+b-x) \dd{x}.
\end{equation}
\begin{proof}
令\(x=a+b-t\),
则\(\dd{x} = -\dd{t}\).
当\(x=a\)时,取\(t=b\).
当\(x=b\)时,取\(t=a\).
于是\begin{equation*}
	\int_a^b f(x) \dd{x}
	= \int_b^a f(a+b-t) \dd(-t)
	= \int_a^b f(a+b-t) \dd{t}.
	\qedhere
\end{equation*}
\end{proof}
\end{proposition}

\begin{example}\label{example:欧拉第一型积分.轮换对称性1}
%@see: 《高等数学(第六版 上册)》 P254 习题5-3 4.
证明:\begin{equation}
	\int_0^1 x^m (1-x)^n \dd{x}
	= \int_0^1 x^n (1-x)^m \dd{x}
	\quad(m,n\in\mathbb{N}).
\end{equation}
\begin{proof}
记\(f(x) = x^m (1-x)^n\),
则\(f(1+0-x) = x^n (1-x)^m\).
由\cref{theorem:定积分.区间再现} 可知,上式显然成立.
\end{proof}
\end{example}

\begin{remark}
利用\cref{theorem:定积分.区间再现},
我们可以便捷地换元,而不改变积分区间.
并且,如果换元前后的被积函数的线性组合\begin{equation*}
	k_1 f(x) + k_2 f(a+b-x)
\end{equation*}恒为一个常数,
即\begin{equation*}
	k_1 f(x) + k_2 f(a+b-x) = s,
\end{equation*}
其中\(s\)是常数,
就可以化简被积函数,
得到\begin{equation*}
	(k_1 + k_2) \int_a^b f(x) \dd{x}
	= \int_a^b [k_1 f(x) + k_2 f(a+b-x)] \dd{x}
	= s \int_a^b \dd{x} = s(b-a),
\end{equation*}
于是\begin{equation*}
	\int_a^b f(x) \dd{x}
	= \frac{s(b-a)}{k_1 + k_2}.
\end{equation*}
\end{remark}

\begin{example}
计算定积分\(\int_0^{\frac{\pi}{2}} \frac{\sin x}{\sin x + \cos x} \dd{x}\).
\begin{solution}
因为\begin{equation*}
	\frac{\sin x}{\sin x + \cos x}
	\xlongequal{x=0+\frac\pi2-t}
	\frac{\sin\left(\frac{\pi}{2}-t\right)}
		{\sin\left(\frac{\pi}{2}-t\right) + \cos\left(\frac{\pi}{2}-t\right)}
	= \frac{\cos t}{\cos t + \sin t},
\end{equation*}
而\begin{equation*}
	1 \cdot \frac{\sin x}{\sin x + \cos x}
	+ 1 \cdot \frac{\cos x}{\sin x + \cos x}
	= 1,
\end{equation*}
所以\begin{equation*}
	\int_0^{\frac{\pi}{2}} \frac{\sin x}{\sin x + \cos x} \dd{x}
	= \frac{1}{1+1}\left(\frac\pi2-0\right)
	= \frac\pi4.
\end{equation*}
\end{solution}
\end{example}

\begin{proposition}\label{theorem:定积分.正余弦函数的复合的积分1}
%@see: 《高等数学(第六版 上册)》 P248 例6
%@see: 《数学分析(第二版 上册)》(陈纪修) P311 习题 9. (1)
%@see: 《数学分析(第二版 上册)》(陈纪修) P312 习题 9. (2)
设\(f \in C[0,1]\),
则\begin{gather}
	\int_0^{\frac\pi2} f(\sin x) \dd{x}
	= \int_0^{\frac\pi2} f(\cos x) \dd{x}, \\
	\int_0^\pi x f(\sin x) \dd{x}
	= \frac{\pi}{2} \int_0^\pi f(\sin x) \dd{x}.
\end{gather}
\begin{proof}
显然有\begin{align*}
	\int_0^{\frac\pi2} f(\sin x) \dd{x}
	&\xlongequal{x = \pi/2-t}
		\int_{\frac\pi2}^0 f\left[\sin(\frac{\pi}{2}-t)\right] \dd(\frac{\pi}{2}-t) \\
	&= \int_0^{\frac\pi2} f(\cos x) \dd{x}.
		\tag{\cref{equation:函数.三角函数.诱导公式10}}
\end{align*}

因为\begin{align*}
	\int_0^\pi x f(\sin x) \dd{x}
	&\xlongequal{x = \pi-t}
		\int_{\pi}^0 (\pi-t) f[\sin(\pi-t)] \dd(\pi-t) \\
	&= \int_0^\pi (\pi-x) f(\sin x) \dd{x}
		\tag{\cref{equation:函数.三角函数.诱导公式5}} \\
	&= \pi \int_0^\pi f(\sin x) \dd{x}
		- \int_0^\pi x f(\sin x) \dd{x},
\end{align*}
所以\begin{equation*}
	\int_0^\pi x f(\sin x) \dd{x}
	= \frac{\pi}{2} \int_0^\pi f(\sin x) \dd{x}.
	\qedhere
\end{equation*}
\end{proof}
\end{proposition}
\begin{example}
%@see: https://www.bilibili.com/video/BV1xUimYGE1C/
设\(f \in C[0,1]\).
证明:\begin{equation}
	\int_0^\pi f(\sin x) \dd{x}
	= 2 \int_0^{\frac\pi2} f(\sin x) \dd{x}.
\end{equation}
\begin{proof}
% 区间可加性
因为\(\int_0^\pi f(\sin x) \dd{x}
= \int_0^{\frac\pi2} f(\sin x) \dd{x}
+ \int_{\frac\pi2}^\pi f(\sin x) \dd{x}\),
而\begin{equation*}
	\int_{\frac\pi2}^\pi f(\sin x) \dd{x}
	\xlongequal{x=\pi-t}% 换元
	\int_0^{\frac\pi2} f(\sin(\pi-t)) \dd{t},
\end{equation*}
%\cref{equation:函数.三角函数.诱导公式5}
并且\(\sin(\pi-t) = \sin t\),
所以\(\int_{\frac\pi2}^\pi f(\sin x) \dd{x}
= \int_0^{\frac\pi2} f(\sin t) \dd{t}\),
从而有\begin{equation*}
	\int_0^\pi f(\sin x) \dd{x}
	= 2 \int_0^{\frac\pi2} f(\sin x) \dd{x}.
\end{equation*}
\end{proof}
\end{example}
\begin{remark}
一般地,当\(f \in C[0,1]\)时,有\(\int_0^\pi f(\cos x) \dd{x}
\neq 2 \int_0^{\frac\pi2} f(\cos x) \dd{x}\).
\end{remark}

\begin{example}
%@see: 《高等数学(第六版 上册)》 P249
计算\(\int_0^\pi \frac{x \sin x}{1 + \cos^2 x} \dd{x}\).
\begin{solution}
利用\cref{theorem:定积分.正余弦函数的复合的积分1} 得\begin{align*}
	\int_0^\pi \frac{x \sin x}{1 + \cos^2 x} \dd{x}
	&= \frac\pi2 \int_0^\pi \frac{\sin x}{1 + \cos^2 x} \dd{x} \\
	&= -\frac\pi2 \int_0^\pi \frac{\dd(\cos x)}{1 + \cos^2 x} \\
	&= -\frac\pi2 \eval{[\arctan(\cos x)]}_0^\pi \\
	&= \frac{\pi^2}4.
\end{align*}
\end{solution}
\end{example}

\begin{proposition}\label{theorem:定积分.周期函数的积分}
%@see: 《高等数学(第六版 上册)》 P249 例7
%@see: 《数学分析(第二版 上册)》(陈纪修) P309 定理7.3.6
%@see: 《数学分析教程(第3版 上册)》(史济怀) P259 例4
设函数\(f\)是以\(T\)为周期的黎曼可积函数.
证明:\begin{gather}
	\int_a^{a+T} f(x) \dd{x} = \int_0^T f(x) \dd{x}, \\
	\int_a^{a+nT} f(x) \dd{x} = n\int_0^T f(x) \dd{x}
	\quad(n\in\mathbb{N}).
\end{gather}
\begin{proof}
记\(\Phi(a) = \int_a^{a+T} f(x) \dd{x}\).
利用\cref{theorem:定积分.变限积分定理},
对\(a\)求导得\begin{equation*}
	\dv{a} \Phi(a) = f(a+T) - f(a) = 0,
\end{equation*}
这就是说\(\Phi(a)\)与\(a\)无关,
于是\(\Phi(a) = \Phi(0)\),即\begin{equation*}
	\int_a^{a+T} f(x) \dd{x} = \int_0^T f(x) \dd{x};
\end{equation*}
由此可得\begin{equation*}
	\int_a^{a+nT} f(x) \dd{x}
	= \sum_{k=0}^{n-1} \int_{a+kT}^{(a+kT)+T} f(x) \dd{x}
	= \sum_{k=0}^{n-1} \int_0^T f(x) \dd{x}
	= n \int_0^T f(x) \dd{x}.
	\qedhere
\end{equation*}
\end{proof}
%\cref{theorem:导数与微分.导函数的周期性}
\end{proposition}
\begin{remark}
\cref{theorem:定积分.周期函数的积分} 说明:
周期函数在长度是周期的整倍数的区间上的积分具有平移对称性.
\end{remark}

\begin{example}
%@see: 《高等数学(第六版 上册)》 P249
计算\(\int_0^{n\pi} \sqrt{1 + \sin2x} \dd{x}\).
\begin{solution}
显然被积函数\(x \mapsto \sqrt{1 + \sin2x}\)是以\(\pi\)为周期的连续函数,
利用\cref{theorem:定积分.周期函数的积分} 得\begin{align*}
	\int_0^{n\pi} \sqrt{1 + \sin2x} \dd{x}
	&= n \int_0^\pi \sqrt{1 + \sin2x} \dd{x} \\
	&= n \int_0^\pi \abs{\sin x + \cos x} \dd{x} \\
	&= \sqrt2 n \int_0^\pi \abs{\sin\left(x + \frac\pi4\right)} \dd{x} \\
	&= \sqrt2 n \int_{\frac\pi4}^{\frac{5\pi}4} \abs{\sin t} \dd{t} \\
	&= \sqrt2 n \int_0^\pi \abs{\sin t} \dd{t} \\
	&= \sqrt2 n \int_0^\pi \sin t \dd{t} \\
	&= 2\sqrt2 n.
\end{align*}
\end{solution}
\end{example}

\begin{example}
%@see: 《数学分析(第二版 上册)》(陈纪修) P312 习题 17.
计算\(a_n = \int_0^{n\pi} x \abs{\sin x} \dd{x}\ (n=1,2,\dotsc)\).
\begin{solution}
%@see: https://www.bilibili.com/video/BV1eieAe1ERR
%@credit: https://space.bilibili.com/433361437
由\cref{theorem:定积分.区间再现} 可得\begin{align*}
	\int_0^{n\pi} x \abs{\sin x} \dd{x}
	&= \int_0^{n\pi} (n\pi-x) \abs{\sin(n\pi-x)} \dd{x} \\
	&= n\pi \int_0^{n\pi} \abs{\sin x} \dd{x}
		- \int_0^{n\pi} x \abs{\sin x} \dd{x}.
\end{align*}
因为\(x \mapsto \abs{\sin x}\)是以\(\pi\)为周期的连续函数,
所以由\cref{theorem:定积分.周期函数的积分} 可得\begin{equation*}
	\int_0^{n\pi} \abs{\sin x} \dd{x}
	= n \int_0^\pi \sin x \dd{x}
	= 2n.
\end{equation*}
于是\begin{equation*}
	a_n = \frac{n\pi}2 \int_0^{n\pi} \abs{\sin x} \dd{x}
	= \frac{n\pi}2 \cdot 2n
	= n^2\pi.
\end{equation*}
\end{solution}
\end{example}
\begin{example}
%@see: https://www.bilibili.com/video/BV1LRhSeMEFr/
计算\(\lim_{x\to+\infty} \frac1x \int_0^x \abs{\sin t} \dd{t}\).
\begin{solution}
对于\(\forall x\geq0\),\(\exists n\in\mathbb{N}\),
使得\(n\pi \leq x < (n+1)\pi\),
或者\(\frac1{(n+1)\pi} < \frac1x \leq \frac1{n\pi}\),
于是\begin{equation*}
	\frac1{(n+1)\pi} \int_0^{n\pi} \abs{\sin t} \dd{t}
	\leq \frac1x \int_0^x \abs{\sin t} \dd{t}
	\leq \frac1{n\pi} \int_0^{(n+1)\pi} \abs{\sin t} \dd{t}.
\end{equation*}
因为当\(n\to\infty\)时有\begin{gather*}
	\frac1{(n+1)\pi} \int_0^{n\pi} \abs{\sin t} \dd{t}
	= \frac{n}{(n+1)\pi} \int_0^\pi \abs{\sin t} \dd{t}
	= \frac{2n}{(n+1)\pi}
	\to \frac2\pi, \\
	\frac1{n\pi} \int_0^{(n+1)\pi} \abs{\sin t} \dd{t}
	= \frac{n+1}{n\pi} \int_0^\pi \abs{\sin t} \dd{t}
	= \frac{2(n+1)}{n\pi}
	\to \frac2\pi,
\end{gather*}
所以由\hyperref[theorem:函数极限.夹逼准则]{夹逼准则}可知\begin{equation*}
	\lim_{x\to+\infty} \frac1x \int_0^x \abs{\sin t} \dd{t}
	= \frac2\pi.
\end{equation*}
\end{solution}
\end{example}
\begin{example}
%@see: https://www.bilibili.com/video/BV1SC2wY5EtD/
计算\(\lim_{x\to+\infty} \frac1{x^2} \int_0^x t \abs{\sin t} \dd{t}\).
\begin{solution}
对于\(\forall x\geq0\),\(\exists n\in\mathbb{N}\),
使得\(n\pi \leq x < (n+1)\pi\),
或者\(\frac1{(n+1)\pi} < \frac1x \leq \frac1{n\pi}\),
于是\begin{equation*}
	\frac1{(n+1)^2\pi^2} \int_0^{n\pi} t \abs{\sin t} \dd{t}
	\leq \frac1{x^2} \int_0^x t \abs{\sin t} \dd{t}
	\leq \frac1{n^2\pi^2} \int_0^{(n+1)\pi} t \abs{\sin t} \dd{t}.
\end{equation*}
因为当\(n\to\infty\)时有\begin{gather*}
	\frac1{(n+1)^2\pi^2} \int_0^{n\pi} t \abs{\sin t} \dd{t}
	= \frac{n^2\pi}{(n+1)^2\pi^2}
	\to \frac1\pi, \\
	\frac1{n^2\pi^2} \int_0^{(n+1)\pi} t \abs{\sin t} \dd{t}
	= \frac{(n+1)^2\pi}{n^2\pi^2}
	\to \frac1\pi,
\end{gather*}
所以由\hyperref[theorem:函数极限.夹逼准则]{夹逼准则}可知\begin{equation*}
	\lim_{x\to+\infty} \frac1{x^2} \int_0^x t \abs{\sin t} \dd{t} = \frac1\pi.
\end{equation*}
\end{solution}
\end{example}

\begin{example}
%@see: 《数学分析中的典型问题与方法(第3版)》(裴礼文) P39 例1.3.20
设函数\(f\)是以\(T\)为周期的连续函数,
证明:\begin{equation}\label{equation:定积分.周期函数的平均值的极限}
	\lim_{x\to+\infty} \frac1x \int_0^x f(t) \dd{t}
	= \frac1T \int_0^T f(t) \dd{t}.
\end{equation}
\begin{proof}
首先假设\(f(x)\geq0\ (x\geq0)\),
那么对于\(\forall x\geq0\),\(\exists n\in\mathbb{N}\),
使得\(nT \leq x < (n+1)T\),
于是\begin{equation*}
	\frac1{(n+1)T} \int_0^{nT} f(t) \dd{t}
	\leq \frac1x \int_0^x f(t) \dd{t}
	\leq \frac1{nT} \int_0^{(n+1)T} f(t) \dd{t}.
\end{equation*}
因为当\(n\to\infty\)时有\begin{gather*}
	\frac1{(n+1)T} \int_0^{nT} f(t) \dd{t}
	= \frac{n}{(n+1)T} \int_0^T f(t) \dd{t}
	\to \frac1T \int_0^T f(t) \dd{t}, \\
	\frac1{nT} \int_0^{(n+1)T} f(t) \dd{t}
	= \frac{n+1}{nT} \int_0^T f(t) \dd{t}
	\to \frac1T \int_0^T f(t) \dd{t},
\end{gather*}
所以由\hyperref[theorem:函数极限.夹逼准则]{夹逼准则}可知\begin{equation*}
	\lim_{x\to+\infty} \frac1x \int_0^x f(t) \dd{t}
	= \frac1T \int_0^T f(t) \dd{t}.
\end{equation*}

对于一般情况(无法保证\(f(x)\geq0\ (x\geq0)\)成立),
由于连续周期函数必定有界,不妨设\(m\)是\(f\)在\([0,T]\)上的一个下界,
即\(f(x) \geq m\ (x\geq0)\),
令\(g(x) \defeq f(x) - m\),
则有\(g(x)\geq0\ (x\geq0)\),
于是\begin{equation*}
	\lim_{x\to+\infty} \frac1x \int_0^x g(t) \dd{t}
	= \frac1T \int_0^T g(t) \dd{t},
\end{equation*}
其中\begin{align*}
	\lim_{x\to+\infty} \frac1x \int_0^x g(t) \dd{t}
	&= \lim_{x\to+\infty} \frac1x \int_0^x (f(t) - m) \dd{t} \\
	&= \lim_{x\to+\infty} \left[
		\frac1x \int_0^x f(t) \dd{t}
		- m \cdot \frac1x \int_0^x \dd{t}
	\right] \\
	&= \left[ \lim_{x\to+\infty} \frac1x \int_0^x f(t) \dd{t} \right] - m, \\
	\frac1T \int_0^T g(t) \dd{t}
	&= \frac1T \int_0^T (f(t) - m) \dd{t} \\
	&= \frac1T \left[
		\int_0^T f(t) \dd{t}
		- m \cdot \int_0^T \dd{t}
	\right] \\
	&= \left[ \frac1T \int_0^T f(t) \dd{t} \right] - m,
\end{align*}
因此\begin{equation*}
	\lim_{x\to+\infty} \frac1x \int_0^x f(t) \dd{t}
	= \frac1T \int_0^T f(t) \dd{t}.
	\qedhere
\end{equation*}
%@credit: {fff74d84-2e79-4c04-bec3-8c318cc2293c}
\end{proof}
\end{example}

\begin{proposition}\label{theorem:定积分.正余弦函数的复合的积分2}
%@see: 《高等数学(第六版 上册)》 P254 习题5-3 5.
设\(f \in C[0,1]\),且\(n\in\mathbb{Z}\),则\begin{equation}
	\int_{\frac{n}{2} \pi}^{\frac{n+1}{2} \pi} f(\abs{\sin x}) \dd{x}
	= \int_{\frac{n}{2} \pi}^{\frac{n+1}{2} \pi} f(\abs{\cos x}) \dd{x}
	= \int_0^{\frac{\pi}{2}} f(\sin x) \dd{x}.
\end{equation}
\def\arraystretch{1.5}
\begin{proof}
令\(x = n\pi+\frac{\pi}{2}-t\),
则有\begin{align*}
	\int_{\frac{n}{2} \pi}^{\frac{n+1}{2} \pi} f(\abs{\sin x}) \dd{x}
	&= \int_{\frac{n+1}{2} \pi}^{\frac{n}{2} \pi} f\left[\abs{\pm\sin(\frac{\pi}{2}-t)}\right] \dd(\frac{2n+1}{2}\pi-t) \\
	&= \int_{\frac{n}{2} \pi}^{\frac{n+1}{2} \pi} f(\abs{\cos x}) \dd{x}.
\end{align*}

又令\(x=u+\frac{n}{2} \pi\),
则有\begin{align*}
	\int_{\frac{n}{2} \pi}^{\frac{n+1}{2} \pi} f(\abs{\cos x}) \dd{x}
	&= \int_0^{\frac{1}{2}\pi} f\left[ \abs{\cos(u+\frac{\pi}{2})} \right] \dd(u+\frac{n}{2}\pi) \\
	&= \int_0^{\frac{1}{2}\pi} f\left( \abs{-\sin u} \right) \dd{u} \\
	&= \int_0^{\frac{1}{2}\pi} f(\sin x) \dd{x}.
	\qedhere
\end{align*}
\end{proof}
\end{proposition}

%@see: https://www.bilibili.com/video/BV1fN411y7z1/
\begin{example}\label{example:定积分.积分上限函数是周期函数的充分必要条件}
证明:如果对于某个实数\(a\),
函数\(f\colon\mathbb{R}\to\mathbb{R}\)满足\begin{equation*}
	\int_a^{a+T} f(x) \dd{x} = 0,
\end{equation*}
则函数\(g(x) = \int_0^x f(t) \dd{t}\)是周期函数,且\(T\)是\(g\)的周期.
\begin{proof}
由\cref{theorem:定积分.周期函数的积分} 可知\begin{equation*}
	\int_a^{a+T} f(x) \dd{x} = \int_0^T f(x) \dd{x},
\end{equation*}
于是\begin{equation*}
	g(x+T)
	= \int_0^{x+T} f(t) \dd{t}
	= \int_0^x f(t) \dd{t} + \int_x^{x+T} f(t) \dd{t}
	= g(x) + \int_0^T f(x) \dd{x},
\end{equation*}
因此\begin{equation*}
	g(x+T) = g(x)
	\iff
	\int_0^T f(x) \dd{x} = 0
	\iff
	\int_a^{a+T} f(x) \dd{x} = 0.
	\qedhere
\end{equation*}
\end{proof}
\end{example}
\begin{example}
设函数\(f\colon\mathbb{R}\to\mathbb{R}\)是以\(T\)为周期的连续函数.
证明:函数\begin{equation*}
	g(x) = \int_0^x f(t) \dd{t} - \int_{-x}^0 f(t) \dd{t}
\end{equation*}也是以\(T\)为周期的函数.
\begin{proof}
令\(t=-u\),则\begin{equation*}
	\int_{-x}^0 f(t) \dd{t}
	= \int_0^x f(-u) \dd{u},
\end{equation*}
于是\begin{equation*}
	g(x) = \int_0^x f(t) \dd{t} - \int_0^x f(-u) \dd{u}
	= \int_0^x (f(t) - f(-t)) \dd{t}.
\end{equation*}
记\(\phi(x) \defeq f(x) - f(-x)\),
由于\begin{equation*}
	\phi(-x) + \phi(x)
	= f(-x) - f(x) + f(x) - f(-x)
	= 0,
\end{equation*}
即\(\phi\)是奇函数,
所以\begin{equation*}
	\int_{-T/2}^{T/2} (f(x) - f(-x)) \dd{x}
	= 0.
\end{equation*}
于是由\cref{example:定积分.积分上限函数是周期函数的充分必要条件} 可知
\(g\)是以\(T\)为周期的函数.
\end{proof}
\end{example}

\begin{example}
%@see: 《数学分析教程(第3版 上册)》(史济怀) P261 例7
设函数\(f \in C(a,b)\),\(\alpha,\beta\in(a,b)\).
证明:\begin{equation*}
	\lim_{h\to0} \frac1h \int_\alpha^\beta (f(x+h) - f(x)) \dd{x}
	= f(\beta) - (\alpha).
\end{equation*}
\begin{proof}
令\(t = x + h\),得\begin{equation*}
	\int_\alpha^\beta f(x+h) \dd{x}
	= \int_{\alpha + h}^{\beta + h} f(t) \dd{t}
	= \int_\alpha^\beta f(t) \dd{t}
	+ \int_\beta^{\beta + h} f(t) \dd{t}
	- \int_\alpha^{\alpha + h} f(t) \dd{t}.
\end{equation*}
由\hyperref[theorem:定积分.积分中值定理1推论2]{积分中值定理}得\begin{gather*}
	\int_\beta^{\beta + h} f(t) \dd{t}
	= f(\xi) h
	\quad(\text{$\xi$在$\beta$与$\beta+h$之间}), \\
	\int_\alpha^{\alpha + h} f(t) \dd{t}
	= f(\eta) h
	\quad(\text{$\eta$在$\alpha$与$\alpha+h$之间}).
\end{gather*}
因此\begin{equation*}
	\frac1h \int_\alpha^\beta (f(x + h) - f(x)) \dd{x}
	= f(\xi) - f(\eta).
\end{equation*}
令\(h\to0\),由\(f\)的连续性,可知\begin{equation*}
	\lim_{h\to0} f(\xi) = f(\beta),
	\qquad
	\lim_{h\to0} f(\eta) = \alpha.
	\qedhere
\end{equation*}
\end{proof}
\end{example}

\section{定积分的分部积分法}
\begin{theorem}\label{theorem:定积分.定积分的分部积分法}
%@see: 《高等数学(第六版 上册)》 P251
%@see: 《数学分析(第二版 上册)》(陈纪修) P300 定理7.3.3
若函数\(u,v \in C[a,b] \cap D(a,b)\),
则\begin{equation}
	\int_a^b u(x) v'(x) \dd{x}
	= \eval{[u(x) v(x)]}_a^b
	- \int_a^b v(x) u'(x) \dd{x},
\end{equation}
或\begin{equation}
	\int_a^b u(x) \dd{[v(x)]}
	= \eval{[u(x) v(x)]}_a^b
	- \int_a^b v(x) \dd{[u(x)]}.
\end{equation}
\end{theorem}
公式表明原函数已经积出的部分可以先用上、下限代入.

\begin{example}
%@see: 《高等数学(第六版 上册)》 P251 例10
计算\(\int_0^{\frac12} \arcsin x \dd{x}\).
\begin{solution}
直接计算得\begin{align*}
	\int_0^{\frac12} \arcsin x \dd{x}
	&= \eval{(x \arcsin x)}_0^{\frac12}
		- \int_0^{\frac12} \frac{x}{\sqrt{1-x^2}} \dd{x} \\
	&= \frac12 \cdot \frac\pi6 + \eval{\sqrt{1-x^2}}_0^{\frac12} \\
	&= \frac\pi{12} + \frac{\sqrt3}2 - 1.
\end{align*}
\end{solution}
\end{example}

\begin{example}
%@see: 《高等数学(第六版 上册)》 P251 例11
计算\(\int_0^1 e^{\sqrt{x}} \dd{x}\).
\begin{solution}
先用换元法.
令\(\sqrt{x}=t\),
则\(x=t^2\),
\(\dd{x} = 2t\dd{t}\),
那么\begin{equation*}
	\int_0^1 e^{\sqrt{x}} \dd{x}
	= 2 \int_0^1 t \dd(e^t)
	= 2 (t e^t)_0^1 - 2 \int_0^1 e^t \dd{t}
	= 2.
\end{equation*}
\end{solution}
\end{example}

\begin{example}
计算\(\int_0^1 \frac{\arctan x}{1+x} \dd{x}\).
\begin{solution}
首先有\begin{equation*}
	\int_0^1 \frac{\arctan x}{1+x} \dd{x}
	= \eval{\left[\ln(1+x) \arctan x\right]}_0^1
	- \int_0^1 \frac{\ln(1+x)}{1+x^2} \dd{x},
\end{equation*}
其中\begin{equation*}
	\eval{\left[\ln(1+x) \arctan x\right]}_0^1
	= \ln2 \cdot \frac\pi4 - 0
	= \frac\pi4 \ln2,
\end{equation*}
而\begin{align*}
	\int_0^1 \frac{\ln(1+x)}{1+x^2} \dd{x}
	&\xlongequal{x=\tan t}
	\int_0^{\pi/4} \frac{\ln(1+\tan t)}{1+\tan^2t} \cdot \sec^2t \dd{t} \\
	&= \int_0^{\pi/4} \ln(1+\tan t) \dd{t} \\
	&= \int_0^{\pi/4} \ln\left[1+\tan(\frac\pi4-t)\right] \dd{t}
		\tag{\cref{theorem:定积分.区间再现}} \\
	&= \int_0^{\pi/4} \ln\left[1+\frac{\tan(\pi/4)-\tan t}{1+\tan(\pi/4) \tan t}\right] \dd{t}
		\tag{\hyperref[equation:函数.三角函数.和积互化公式3]{和积互化公式}} \\
	&= \int_0^{\pi/4} \ln\frac2{1+\tan t} \dd{x} \\
	&= \int_0^{\pi/4} [\ln2 - \ln(1+\tan t)] \dd{x}
		\tag{\hyperref[equation:函数.对数的基本运算法则2]{对数运算法则}} \\
	&= \int_0^{\pi/4} \ln2 \dd{x} - \int_0^{\pi/4} \ln(1+\tan t) \dd{t},
		\tag{\hyperref[theorem:定积分.定积分性质1]{定积分的性质}}
\end{align*}
从而\begin{equation*}
	\int_0^1 \frac{\ln(1+x)}{1+x^2} \dd{x}
	= \int_0^{\pi/4} \ln(1+\tan t) \dd{t}
	= \frac12 \int_0^{\pi/4} \ln2 \dd{x}
	= \frac\pi8 \ln2.
\end{equation*}
于是\(\int_0^1 \frac{\arctan x}{1+x} \dd{x}
= \frac\pi4 \ln2 - \frac\pi8 \ln2
= \frac\pi8 \ln2\).
\end{solution}
%TODO 据说还有一种换元方法是令\(t = \frac{1-x}{1+x}\),值得一试.
\end{example}

多次运用分部积分公式可以进一步得到以下结果:
\begin{corollary}
设函数\(u,v \in C^{n+1}[a,b]\),
则\begin{equation*}
	\int_a^b u(x) v^{(n+1)}(x) \dd{x}
	= \eval{\left[
		\sum_{k=0}^n (-1)^k u^{(k)}(x) v^{(n-k)}(x)
	\right]}_a^b
	+ (-1)^{n+1} \int_a^b u^{(n+1)}(x) v(x) \dd{x}
\end{equation*}对\(n=1,2,\dotsc\)均成立.
\end{corollary}

\begin{theorem}\label{theorem:定积分.带有积分余项的泰勒中值定理}
%@see: 《数学分析教程(第3版 上册)》(史济怀) P257 定理6.4.1(Taylor公式的积分余项)
设函数\(f\)在\((a,b)\)上有直到\(n+1\)阶连续导函数,
那么对任意\(x_0\in(a,b)\),
有\begin{equation*}
%\cref{equation:微分中值定理.泰勒公式1}
	f(x) = p_n(x) + R_n(x),
\end{equation*}
其中\begin{gather}
%\cref{equation:微分中值定理.泰勒公式.多项式1}
	p_n(x) = \sum_{k=0}^n \frac{f^{(k)}(x_0)}{k!} (x-x_0)^k, \\
%\cref{equation:微分中值定理.泰勒公式.余项0}
	R_n(x) = \frac1{n!} \int_{x_0}^x (x-t)^n f^{(n+1)}(t) \dd{t}
	\quad(a<x<b).
\end{gather}
%TODO proof
\end{theorem}
\begin{remark}
与\cref{theorem:微分中值定理.泰勒中值定理} 相比,
\cref{theorem:定积分.带有积分余项的泰勒中值定理} 要求的条件稍强,
不仅要求\(f^{(n+1)}\)存在,而且要求它连续.
\end{remark}

\section{正交函数列}
\begin{definition}
%@see: 《数学分析(第二版 上册)》(陈纪修) P300 定义7.3.1
设\(\{g_n\}_{n\geq0}\)是定义在\([a,b]\)上的一列函数.
若对任意\(m,n\in\mathbb{N}\),
函数\(g_m\)和\(g_n\)都在\([a,b]\)上黎曼可积,
且有\begin{equation*}
	\int_a^b g_m(x) g_n(x) \dd{x} = \left\{ \begin{array}{ll}
		0, & m \neq n, \\
		\int_a^b g_n^2(x) \dd{x}, & m = n,
	\end{array} \right.
\end{equation*}
则称\(\{g_n\}_{n\geq0}\)是“区间\([a,b]\)上的\DefineConcept{正交函数列}”.

特别地,当\(g_n\)是\(n\)次多项式时,
称\(\{g_n\}_{n\geq0}\)是“区间\([a,b]\)上的\DefineConcept{正交多项式函数列}”.
\end{definition}

\begin{example}
%@see: 《数学分析(第二版 上册)》(陈纪修) P301 例7.3.7
证明:拉格朗日多项式\begin{equation*}
	p_n(x) = \frac1{2^n n!} \dv[n]{x} (x^2-1)^n
	\quad(n=0,1,2,\dotsc)
\end{equation*}组成的函数列\(\{p_n\}_{n\geq0}\)是\([-1,1]\)上的正交多项式函数列.
\begin{proof}
不妨设\(n \geq m\).
记\begin{align*}
	I(m,n) &\defeq m! ~ n! ~ 2^m ~ 2^n \int_{-1}^1 p_m(x) p_n(x) \dd{x} \\
	&= \int_{-1}^1 \dv[m]{x} (x^2-1)^m \cdot \dv[n]{x} (x^2-1)^n \dd{x}.
\end{align*}
只要我们把\(\dv[m]{x} (x^2-1)^m\)看作\(u(x)\),
把\(\dv[n]{x} (x^2-1)^n\)看作\(v'(x)\),
那么利用分部积分法可得\begin{align*}
	I(m,n) = \dv[m]{x} (x^2-1)^m
		\cdot \eval{\dv[n-1]{x} (x^2-1)^n}_{-1}^1
		- \int_{-1}^1 \dv[m+1]{x} (x^2-1)^m
			\cdot \dv[n-1]{x} (x^2-1)^n \dd{x}.
\end{align*}
我们知道,函数\begin{equation*}
	\dv[n-k]{x} (x^2-1)^n
	\quad(k=1,2,\dotsc,n-1)
\end{equation*}中都含有\((x^2-1)\)因子,
因此\begin{equation*}
	\eval{\dv[n-1]{x} (x^2-1)^n}_{x=1}
	= \eval{\dv[n-1]{x} (x^2-1)^n}_{x=-1}
	= 0,
\end{equation*}
所以\begin{equation*}
	I(m,n) = - \int_{-1}^1 \dv[m+1]{x} (x^2-1)^m
					\cdot \dv[n-1]{x} (x^2-1)^n \dd{x}.
\end{equation*}
反复执行上述过程,最后得到\begin{equation*}
	I(m,n) = (-1)^n \int_{-1}^1 \left[\dv[m+n]{x} (x^2-1)^m\right] \cdot (x^2-1)^n \dd{x}.
\end{equation*}

若\(n>m\),则有\(\dv[m+n]{x} (x^2-1)^m = 0\),
因此\(\int_{-1}^1 p_m(x) p_n(x) \dd{x} = 0\).

若\(n=m\),则有\(\dv[m+n]{x} (x^2-1)^m = (2n)!\),
再次利用分部积分法\begin{align*}
	I(m,n) &= (2n)! \int_{-1}^1 (1-x)^n (1+x)^n \dd{x} \\
	&= \frac{(2n)! n}{n+1} \int_{-1}^1 (1-x)^{n-1} (1+x)^{n+1} \dd{x} \\
	&= \frac{(2n)! n(n-1)}{(n+1)(n+2)} \int_{-1}^1 (1-x)^{n-2} (1+x)^{n+2} \dd{x} \\
	&= \dotsb \\
	&= \frac{(2n)! n(n-1)\dotsm1}{(n+1)(n+2)\dotsm(2n)} \int_{-1}^1 (1+x)^{2n} \dd{x} \\
	&= \frac{(n!)^2 2^{2n+1}}{2n+1}.
\end{align*}
于是便有\begin{equation*}
	\int_{-1}^1 p_n^2(x) \dd{x}
	= \frac1{(n!)^2 2^{2n}} I(m,n)
	= \frac2{2n+1}.
	\qedhere
\end{equation*}
\end{proof}
\end{example}

\begin{example}\label{example:定积分.点火公式}
%@see: 《高等数学(第六版 上册)》 P251 例12
%@see: 《数学分析(第二版 上册)》(陈纪修) P302 例7.3.8
%@see: 《数学分析教程(第3版 上册)》(史济怀) P259 例3
证明:\begin{equation}\label{equation:定积分.点火公式1}
	I_n = \int_0^{\frac{\pi}{2}} \sin^n x \dd{x}
	= \left\{ \def\arraystretch{1.5} \begin{array}{rl}
		\frac{\pi}{2}\frac{(n-1)!!}{n!!},
			& \text{$n$是偶数}, \\
		\frac{(n-1)!!}{n!!},
			& \text{$n$是奇数}.
	\end{array} \right.
\end{equation}
\begin{proof}
由于\begin{align*}
	I_n &= -\int_0^{\frac{\pi}{2}} \sin^{n-1} x \dd(\cos x) \\
	&= \eval{\left[-\cos x \sin^{n-1} x\right]}_0^{\frac\pi2}
		+ (n-1) \int_0^{\frac\pi2} \sin^{n-2} x \cos^2 x \dd{x} \\
	&= (n-1) \int_0^{\frac\pi2} \sin^{n-2} x (1-\sin^2 x) \dd{x} \\
	&= (n-1) I_{n-2} - (n-1) I_n,
	\end{align*}
所以\begin{equation*}
	I_n = \frac{n-1}{n} I_{n-2}.
\end{equation*}
于是对于\(k=1,2,\dotsc\)成立\begin{gather*}
	I_{2k}
	= \frac{2k-1}{2k} \cdot \frac{2k-3}{2k-2}
	\cdot \dotsm \cdot \frac{5}{6} \cdot \frac{3}{4} \cdot \frac{1}{2} \cdot I_0, \\
	I_{2k+1}
	= \frac{2k}{2k+1} \cdot \frac{2k-2}{2k-1}
		\cdot \dotsm \cdot \frac{6}{7} \cdot \frac{4}{5} \cdot \frac{2}{3} \cdot I_1.
\end{gather*}
而\begin{equation*}
	I_0 = \int_0^{\frac{\pi}{2}} \dd{x} = \frac{\pi}{2},
	\qquad
	I_1 = \int_0^{\frac{\pi}{2}} \sin x \dd{x} = 1,
\end{equation*}
因此\begin{gather*}
	I_{2k} = \frac{2k-1}{2k} \cdot \frac{2k-3}{2k-2}
	\cdot \dotsm \cdot \frac{5}{6} \cdot \frac{3}{4}
	\cdot \frac{1}{2} \cdot \frac{\pi}{2}, \\
	I_{2k+1} = \frac{2k}{2k+1} \cdot \frac{2k-2}{2k-1}
		\cdot \dotsm \cdot \frac{6}{7} \cdot \frac{4}{5} \cdot \frac{2}{3}.
	\qedhere
\end{gather*}
\end{proof}
\end{example}
\begin{remark}
由\cref{theorem:定积分.正余弦函数的复合的积分1}
可知\begin{equation}\label{equation:定积分.点火公式2}
	\int_0^{\frac\pi2} \sin^n x \dd{x}
	= \int_0^{\frac\pi2} \cos^n x \dd{x}.
\end{equation}
\end{remark}
\begin{remark}
由诱导公式 \labelcref{equation:函数.三角函数.诱导公式5}
可知\(\sin(\pi-x)=\sin x\),
于是正弦函数\(\sin x\)的正整数次幂的图形
都关于直线\(x=\frac\pi2\)对称
(如\cref{figure:定积分.点火公式}),
那么有\begin{equation*}
	\int_0^{\frac\pi2} \sin^n x \dd{x}
	= \int_{\frac\pi2}^\pi \sin^n x \dd{x},
\end{equation*}
即有\begin{equation}\label{equation:定积分.点火公式3}
	\int_0^\pi \sin^n x \dd{x}
	= 2 \int_0^{\frac\pi2} \sin^n x \dd{x}.
\end{equation}
%\cref{equation:定积分.区间折半}
\begin{figure}[htb]
	\centering
	\begin{tikzpicture}
		\begin{axis}[
			xmin=0,xmax=3.14,
			ymin=0,ymax=1,
			grid=both,
			axis lines=middle,
			xlabel=$x$,
			ylabel=$y$,
			x label style={at={(ticklabel* cs:1.00)}, inner sep=5pt, anchor=north},
			y label style={at={(ticklabel* cs:1.00)}, inner sep=2pt, anchor=east},
			ytick={1},
			yticklabels={\relax},
			xtick={1.5708},
			xticklabels={$\frac{\pi}{2}$},
		]
			\addplot[color=blue,samples=50,smooth,domain=0:3.14,variable=\x]{sin(\x r)};
			\addplot[color=blue,samples=50,smooth,domain=0:3.14,variable=\x]{(sin(\x r))^2};
			\addplot[color=blue,samples=50,smooth,domain=0:3.14,variable=\x]{(sin(\x r))^4};
			\addplot[color=blue,samples=50,smooth,domain=0:3.14,variable=\x]{(sin(\x r))^8};
			\addplot[color=blue,samples=50,smooth,domain=0:3.14,variable=\x]{(sin(\x r))^16};
		\end{axis}
	\end{tikzpicture}
	\caption{}
	\label{figure:定积分.点火公式}
\end{figure}

但是由诱导公式 \labelcref{equation:函数.三角函数.诱导公式6}
可知\(\cos(\pi-x)=-\cos x\),
于是只有余弦函数\(\cos x\)的正偶数次幂的图形
都关于直线\(x=\frac\pi2\)对称,
反之,它的正奇数次幂的图形
都关于点\((\frac\pi2,0)\)中心对称,
那么有\begin{equation}\label{equation:定积分.点火公式4}
	\int_0^\pi \cos^n x \dd{x}
	= \left\{ \begin{array}{cl}
		2 \int_0^{\frac\pi2} \sin^n x \dd{x},
		& \text{$n$是偶数}, \\
		0,
		& \text{$n$是奇数}.
	\end{array} \right.
\end{equation}
\end{remark}

下面特别就\cref{equation:定积分.点火公式1} 做一点点展开.
\begin{example}
证明:\begin{equation}
	\lim_{n\to\infty} \int_0^{\frac\pi2} \sin^n x \dd{x} = 0.
\end{equation}
\begin{proof}
因为\(\sin^n\frac{\pi}{2}=1\),
所以对于任意\(n\)总有函数\(\sin^n x\)在点\(\frac{\pi}{2}\)的左邻域取值接近\(1\).
另一方面,对于固定的\(x\)取值,只要\(x<\frac{\pi}{2}\),
则当\(n\)增加时函数值\(\sin^n x\)就很快趋于\(0\).
接下来我们就采用这样的“分而治之”的方法证明
\(\lim_{n\to\infty} \int_0^{\frac\pi2} \sin^n x \dd{x} = 0\).

当\(0<x<\frac{\pi}{2}\)时,\(0<\sin x<1\).
\(\forall\epsilon>0\)(不妨设\(\epsilon<\pi\)).
因为\(\sin x\)在\([0,\pi/2]\)上单调增加且大于零,
故根据\cref{theorem:定积分.定积分性质6},
有\begin{equation*}
	\int_0^{(\pi-\epsilon)/2} \sin^n x \dd{x}
	\leq
	\int_0^{(\pi-\epsilon)/2} \sin^n\frac{\pi-\epsilon}{2} \dd{x}
	\leq
	\frac{\pi}{2} \sin^n\frac{\pi-\epsilon}{2},
\end{equation*}
因此\begin{equation*}
	0 \leq \int_0^{\frac\pi2} \sin^n x \dd{x}
	= \int_0^{(\pi-\epsilon)/2} \sin^n x \dd{x}
	+ \int_{(\pi-\epsilon)/2}^{\frac\pi2} \sin^n x \dd{x}
	\leq \frac{\pi}{2} \sin^n\frac{\pi-\epsilon}{2} + \frac{\epsilon}{2}.
\end{equation*}
由\(0<\sin\frac{\pi-\epsilon}{2}<1\),
可见\(\lim_{n\to\infty} \sin^n\frac{\pi-\epsilon}{2} = 0\).
从而对于上述\(\epsilon\),有\begin{equation*}
	(\exists N\in\mathbb{N})
	(\forall n\in\mathbb{N})
	\left[
		\begin{array}{ll}
			n>N
			&\implies
			0<\frac{\pi}{2} \sin^n\frac{\pi-\epsilon}{2}<\frac{\epsilon}{2} \\
			&\implies
			0 \leq \int_0^{\frac\pi2} \sin^n x \dd{s} < \epsilon.
		\end{array}
	\right]
\end{equation*}
这就证明了\(\lim_{n\to\infty} \int_0^{\frac\pi2} \sin^n x \dd{x} = 0\).
\end{proof}
\end{example}

可以观察到\cref{example:定积分.点火公式} 中,
\(I_{2k}\)和\(I_{2k+1}\)几乎只差一个系数\(\pi/2\).
我们不禁猜想是否可以利用\cref{equation:定积分.点火公式1} 求出圆周率\(\pi\).
\begin{example}[沃利斯公式]
%@see: 《数学分析教程 (第3版 上册)》(史济怀) P309
证明:\begin{equation}\label{equation:定积分.沃利斯公式}
	\lim_{n\to\infty} \frac{1}{2n+1} \left[
		\frac{(2n)!!}{(2n-1)!!}
	\right]^2
	= \frac{\pi}{2}.
\end{equation}
\begin{proof}
因为当\(0<x<\frac{\pi}{2}\)时有\(0<\sin x<1\),
因此对\(\forall n\in\mathbb{N}^*\)就有\begin{equation*}
	\sin^{2n+2} x < \sin^{2n+1} x < \sin^{2n} x.
\end{equation*}
这样就成立积分不等式\(I_{2n+2} < I_{2n+1} < I_{2n}\),那么\begin{equation*}
	I_{2n+2} = \frac{2n+1}{2n+2} \cdot I_{2n}
	< I_{2n+1} < I_{2n}.
\end{equation*}
在上式两边同时除以\(I_{2n}\),并取极限,
根据\hyperref[theorem:函数极限.夹逼准则]{夹逼准则}有\begin{equation*}
	\lim_{n\to\infty} \frac{I_{2n+1}}{I_{2n}} = 1
\end{equation*}
或\begin{equation*}
	\lim_{n\to\infty} \frac{1}{2n+1} \cdot \left[
		\frac{(2n)!!}{(2n-1)!!}
	\right]^2 \cdot \frac{2}{\pi}
	= 1.
	\qedhere
\end{equation*}
\end{proof}
\end{example}

我们把\cref{equation:定积分.沃利斯公式}
称为\hyperref[equation:定积分.沃利斯公式]{沃利斯公式}(Wallis Formula).
%@see: https://mathworld.wolfram.com/WallisFormula.html
%@see: https://www.statisticshowto.com/wallis-formula-definition-examples/
我们还可以把沃利斯公式写成\begin{equation}
%@see: 《数学分析教程 (第3版 上册)》(史济怀) P309 (1)
	\sqrt\pi
	= \lim_{n\to\infty} \frac{(n!)^2 2^{2n}}{(2n)! \sqrt{n}}.
\end{equation}

\begin{example}
计算\begin{equation*}
	I_n = \int_0^\pi x \sin^n x \dd{x}
	\quad(n\in\mathbb{N}^+).
\end{equation*}
\begin{solution}
直接计算得\begin{align*}
	\int_0^\pi x \sin^n x \dd{x}
	&= \frac\pi2 \int_0^\pi \sin^n x \dd{x}
		\tag{\cref{theorem:定积分.正余弦函数的复合的积分1}} \\
	&= \pi \int_0^{\frac\pi2} \sin^n x \dd{x}.
		\tag{\cref{equation:定积分.点火公式3}}
\end{align*}
\end{solution}
%@Mathematica: Integrate[x Sin[x]^n, {x, 0, Pi}, Assumptions -> {n \[Element] PositiveIntegers}]
\end{example}

\begin{example}
计算:\(\int_0^{2\pi}\left(\int_x^{2\pi}\frac{\sin t}{t}\dd{t}\right)\dd{x}\).
\begin{solution}
这里我们把\(\int_x^{2\pi}\frac{\sin t}{t}\dd{t}\)看成一个整体,
即令\(u = \int_x^{2\pi}\frac{\sin t}{t}\dd{t}\),
那么我们就利用分部积分法,
得到\(\int_0^{2\pi} u \dd{x} = (x u)_0^{2\pi} - \int_0^{2\pi} x \dd{u}\).
按照这个思路,
我们有\begin{align*}
	\int_0^{2\pi}\left(\int_x^{2\pi}\frac{\sin t}{t}\dd{t}\right)\dd{x}
	&= \left(x \int_x^{2\pi}\frac{\sin t}{t}\dd{t}\right)_0^{2\pi}
	- \int_0^{2\pi} x \dd(\int_x^{2\pi}\frac{\sin t}{t}\dd{t}) \\
	&= 0 - \int_0^{2\pi} x \left(-\frac{\sin x}{x}\right) \dd{x} \\
	&= \int_0^{2\pi} \sin x \dd{x} = 0.
\end{align*}
\end{solution}
\end{example}

\section{定积分的应用}
\subsection{直角坐标下的图形面积}
我们已经知道,由曲线\(y=f(x)\ (f(x)\geq0)\)及直线\(x=a\)和\(x=b\ (a<b)\)与\(x\)轴
所围成的曲边梯形的面积\(A\)是定积分\begin{equation*}
	A = \int_a^b f(x) \dd{x},
\end{equation*}
其中被积表达式\(f(x) \dd{x}\)就是直角坐标系下的面积元素,
它表示高为\(f(x)\)、底为\(\dd{x}\)的一个矩形的面积.

应用定积分,不但可以计算曲边梯形面积,还可以计算一些比较复杂的平面图形的面积.

\begin{example}
%@see: 《高等数学(第六版 上册)》 P275 例2
计算抛物线\(y^2=2x\)与直线\(y=x-4\)所围成的图形的面积.
\begin{solution}
先求抛物线和直线的交点,解方程组\begin{equation*}
	\left\{ \begin{array}{l}
		y^2=2x, \\
		y=x-4
	\end{array} \right.
\end{equation*}得交点\((2,-2)\)和\((8,4)\),
从而知道这图形在直线\(y=-2\)和\(y=4\)之间.
现在,选取纵坐标\(y\)为积分变量,
对应的积分区间为\([-2,4]\).
相应于\([-2,4]\)上任一小区间\([y,y+\dd{y}]\)的窄条面积
近似于高为\(\dd{y}\)、底为\((y+4)-\frac12y^2\)的窄矩形的面积,
从而得到面积元素\begin{equation*}
	\dd{A} = \left(y+4-\frac12y^2\right) \dd{y}.
\end{equation*}
把这个面积元素作为被积表达式,在闭区间\([-2,4]\)上作定积分,
便得所求的面积为\begin{align*}
	A &= \int_{-2}^4 \left(y+4-\frac12y^2\right) \dd{y} \\
	&= \eval{\left[\frac12y^2+4y-\frac16y^3\right]}_{-2}^4 \\
	&= 18.
\end{align*}
\end{solution}
\end{example}

\begin{example}
%@see: 《高等数学(第六版 上册)》 P276 例3
%@see: 《数学分析(第二版 上册)》(陈纪修) P315 例7.4.3
求椭圆\(\frac{x^2}{a^2}+\frac{y^2}{b^2}=1\)所围成的图形的面积.
\begin{solution}
这椭圆关于两坐标轴都对称,所以椭圆所围成的图形的面积为\(A=4A_1\),
其中\(A_1\)是该椭圆在第一象限部分与两坐标轴所围图形的面积\begin{equation*}
	A_1 = \int_0^a y \dd{x}.
\end{equation*}
利用椭圆的参数方程\begin{equation*}
	\left\{ \begin{array}{l}
		x = a \cos t, \\
		y = a \sin t,
	\end{array} \right.
	\quad 0 \leq t \leq \frac\pi2,
\end{equation*}
应用定积分换元法,
令\(x = a \cos t\),
则\begin{equation*}
	y = b \sin t, \qquad
	\dd{x} = -a \sin t,
\end{equation*}
且\(t \to \frac\pi2\ (x\to0),
t \to 0\ (x \to a)\),
所以\begin{align*}
	A_1 &= \int_{\frac\pi2}^0 b \sin t (-a \sin t) \dd{t} \\
	&= -ab \int_{\frac\pi2}^0 \sin^2t \dd{t} \\
	&= ab \cdot \frac12 \cdot \frac\pi2
	= \frac14 \pi ab.
\end{align*}
于是\(A = 4 A_1 = \pi ab\).
\end{solution}
\end{example}

% 短摆线(curtate cycloid)
%@see: https://mathworld.wolfram.com/CurtateCycloid.html
% 参数方程:\begin{equation*}
% 	\left\{ \begin{array}{l}
% 		x = a \theta - b \sin\theta, \\
% 		y = a - b \cos\theta
% 	\end{array} \right.
%	\quad(b<a).
% \end{equation*}
% 长摆线(prolate cycloid)
%@see: https://mathworld.wolfram.com/ProlateCycloid.html
% 参数方程:\begin{equation*}
% 	\left\{ \begin{array}{l}
% 		x a \theta - b \sin\theta, \\
% 		y = a - b \cos\theta
% 	\end{array} \right.
% 	\quad(b>a).
% \end{equation*}
\begin{example}
%@see: 《数学分析(第二版 上册)》(陈纪修) P316 例7.4.4
求摆线的一拱\begin{equation*}
	\left\{ \begin{array}{l}
		x = a (t - \sin t), \\
		y = a (1 - \cos t)
	\end{array} \right.
	\quad(0 \leq t \leq 2\pi)
\end{equation*}与\(x\)轴所围区域的面积.
\begin{solution}
所求区域的面积为\begin{equation*}
	S = a^2 \int_0^{2\pi} (1-\cos t)^2 \dd{t}
	= a^2 \int_0^{2\pi} \left(
		1 - 2 \cos t + \frac{1 + \cos2t}2
	\right) \dd{t}
	= 3\pi a^2.
\end{equation*}
\end{solution}
\end{example}

%@see: 《数学分析(第二版 上册)》(陈纪修) P315
若曲线的方程是用参数形式\begin{equation*}
	\left\{ \begin{array}{l}
		x = x(t), \\
		y = y(t)
	\end{array} \right.
	\quad(\alpha \leq t \leq \beta)
\end{equation*}给出的,
\(x(t)\)在\([\alpha,\beta]\)上具有连续导数,
且\(x'(t)\neq0\).
那么曲线与\(x\)轴所围成的曲边梯形的面积为\begin{equation}
	S = \int_\alpha^\beta \abs{y(t) x'(t)} \dd{t}.
\end{equation}

\subsection{极坐标下的图形面积}
某些平面图形,用极坐标来计算它们的面积比较方便.

设由曲线\(\rho = \phi(\theta)\)
及射线\(\theta=\alpha\)、\(\theta=\beta\)围成一个平面图形(特别地,称其为“曲边扇形”),
现在要计算它的面积.
这里,\(\phi(\theta)\)在\([\alpha,\beta]\)上连续,且\(\phi(\theta)\geq0\).

由于当\(\theta\)在\([\alpha,\beta]\)上变动时,
极径\(\rho=\phi(\theta)\)也随之变动,
因此所求图形的面积不能直接利用扇形面积计算公式\begin{equation*}
    A = \frac{1}{2} R^2 \theta
\end{equation*}来计算.

取极角\(\theta\)为积分变量,它的变化区间为\([\alpha,\beta]\).
相应于任一小区间\([\theta,\theta+\dd{\theta}]\)的窄曲边扇形的面积可以用
半径为\(\rho=\phi(\theta)\)、中心角为\(\dd{\theta}\)的扇形的面积来近似代替,
从而得到这窄曲边扇形面积的近似值,即曲边扇形的面积元素\begin{equation}
    \dd{A}
    = \frac{1}{2} [\phi(\theta)]^2 \dd{\theta}.
\end{equation}
考虑到\(A = \int \dd{A}\),
只要以\(\frac{1}{2} [\phi(\theta)]^2 \dd{\theta}\)为被积表达式,
在闭区间\([\alpha,\beta]\)上作定积分,便得所求曲边扇形的面积为\begin{equation}
	A = \int_\alpha^\beta \frac{1}{2} [\phi(\theta)]^2 \dd{\theta}.
\end{equation}

\begin{example}
%@see: 《高等数学(第六版 上册)》 P277 例4
计算阿基米德螺线\begin{equation*}
	\rho = a \theta, \quad a>0
\end{equation*}上相应于\(\theta\)从\(0\)变到\(2\pi\)的一段弧与极轴所围成的图形的面积.
\begin{solution}
在指定的这段罗翔上,\(\theta\)的变化区间为\([0,2\pi]\).
相应于\([0,2\pi]\)上任一小区间\([\theta,\theta+\dd{\theta}]\)的窄曲边扇形的面积
近似于半径为\(a \theta\)、中心角为\(\dd{\theta}\)的圆扇形的面积,
从而得到面积元素\begin{equation*}
	\dd{A} = \frac12 (a \theta)^2 \dd{\theta}.
\end{equation*}
于是所求面积为\begin{equation*}
	A = \int_0^{2\pi} \frac12 (a \theta)^2 \dd{\theta}
	= \eval{\frac16 a^2 \theta^3}_0^{2\pi}
	= \frac43 a^2 \pi^3.
\end{equation*}
\end{solution}
\end{example}

\begin{example}
%@see: 《高等数学(第六版 上册)》 P277 例5
计算心形线\begin{equation*}
	\rho = a (1 + \cos\theta), \quad a>0
\end{equation*}所围成的图形的面积.
\begin{solution}
心形线所围成的图形对称于极轴,
因此所求图形的面积\(A\)是极轴以上部分图形面积\(A_1\)的两倍.
对于极轴以上部分的图形,\(\theta\)的变化区间为\([0,\pi]\).
相应于\([0,\pi]\)上任一小区间\([\theta,\theta+\dd{\theta}]\)的窄曲边扇形的面积
近似于半径为\(a (1 + \cos\theta)\)、中心角为\(\dd{\theta}\)的圆扇形的面积,
从而得到面积元素\begin{equation*}
	\dd{A} = \frac12 a^2 (1 + \cos\theta)^2 \dd{\theta}.
\end{equation*}
于是\begin{align*}
	A_1 &= \int_0^\pi \frac12 a^2 (1 + \cos\theta)^2 \dd{\theta} \\
	&= \frac12 a^2 \int_0^\pi (1 + 2\cos\theta + \cos^2\theta) \dd{\theta} \\
	&= \frac12 a^2 \int_0^\pi \left(\frac32 + 2 \cos\theta + \frac12 \cos2\theta\right) \dd{\theta} \\
	&= \eval{\frac12 a^2 \left[\frac32 \theta + 2 \sin\theta + \frac14 \sin2\theta\right]}_0^\pi
	= \frac34 \pi a^2,
\end{align*}
因而所求面积为\begin{equation*}
	A = 2 A_1 = \frac32 \pi a^2.
\end{equation*}
\end{solution}
\end{example}

\begin{example}
%@see: 《数学分析(第二版 上册)》(陈纪修) P317 例7.4.6
求三叶玫瑰线\(\rho = a \sin3\theta\ (0 \leq \theta \leq \pi)\)所围区域的面积.
\begin{solution}
利用对称性,我们只求半叶“玫瑰”的面积,
这时\(\theta\)的变化范围是\(0 \leq \theta \leq \frac\pi6\).
于是\begin{equation*}
	S = 6 \cdot \frac{a^2}2 \int_0^{\frac\pi6} \sin^23\theta \dd{\theta}
	\xlongequal{\phi=3\theta} a^2 \int_0^{\frac\pi2} \sin^2\phi \dd{\phi}
	= \frac\pi4 a^2.
\end{equation*}
\end{solution}
\end{example}

\begin{example}
求伯努利双纽线\((x^2+y^2)^2 = 2 a^2 x y\)所围成的图形的面积.
\begin{solution}
注意到\(F(x,y) = (x^2+y^2)^2 - 2 a^2 x y\)具有轮换对称性,
即伯努利双纽线\((x^2+y^2)^2 = 2 a^2 x y\)所围成的图形关于直线\(y=x\)对称,
所以它所围成的图形的面积
等于它与\(x\)轴、直线\(y=x\)所围成的图形的面积的4倍.
令\(x = \rho \cos\theta,
y = \rho \sin\theta\),
则曲线方程化为\begin{equation*}
	\rho^4 = 2 a^2 \rho^2 \cos\theta \sin\theta,
\end{equation*}
消去\(\rho^2\)得\begin{equation*}
	\rho^2 = a^2 \sin2\theta.
\end{equation*}
伯努利双纽线\((x^2+y^2)^2 = 2 a^2 x y\)所围成的图形在第一象限部分所占据的角度范围为\begin{equation*}
	\sin2\theta \geq 0,
	\qquad
	0 \leq \theta \leq \frac\pi4,
\end{equation*}
即\begin{equation*}
	0 \leq \theta \leq \frac\pi4.
\end{equation*}
因此伯努利双纽线\((x^2+y^2)^2 = 2 a^2 x y\)所围成的图形的面积为\begin{equation*}
	S = 4 \int_0^{\frac\pi4} \frac12 a^2 \sin2\theta \dd{\theta}
	= a^2.
\end{equation*}
\end{solution}
%@Mathematica: RegionPlot[(x^2 + y^2)^2 <= 2 x y, {x, -1, 1}, {y, -1, 1}]
\end{example}
\begin{example}
求伯努利双纽线\((x^2+y^2)^2 = a^2(x^2-y^2)\)所围成的图形的面积.
\begin{solution}
注意到\(F(x,y) = (x^2+y^2)^2 - a^2(x^2-y^2)\)
既是关于变量\(x\)的偶函数,
也是关于变量\(y\)的偶函数,
所以它所围成的图形的面积
等于它所围成的图形在第一象限部分的面积的4倍.
令\(x = \rho \cos\theta,
y = \rho \sin\theta\),
则曲线方程化为\begin{equation*}
	\rho^4 = a^2 (\rho^2 \cos^2\theta - \rho^2 \sin^2\theta),
\end{equation*}
消去\(\rho^2\)得\begin{equation*}
	\rho^2 = a^2 \cos2\theta.
\end{equation*}
伯努利双纽线\((x^2+y^2)^2 = a^2(x^2-y^2)\)所围成的图形在第一象限部分所占据的角度范围为\begin{equation*}
	\cos2\theta \geq 0,
	\qquad
	0 \leq \theta \leq \frac\pi2,
\end{equation*}
即\begin{equation*}
	0 \leq \theta \leq \frac\pi4.
\end{equation*}
因此伯努利双纽线\((x^2+y^2)^2 = a^2(x^2-y^2)\)所围成的图形的面积为\begin{equation*}
	S = 4 \int_0^{\frac\pi4} \frac12 a^2 \cos2\theta \dd{\theta}
	= a^2.
\end{equation*}
\end{solution}
%@Mathematica: RegionPlot[(x^2 + y^2)^2 <= x^2 - y^2, {x, -1, 1}, {y, -1, 1}]
\end{example}
\begin{example}
%@see: https://www.bilibili.com/video/BV1dbm6YwEBy/
求曲线\((x^2+y^2)^2 = a^2 x^2 - b^2 y^2\)所围成的图形的面积.
\begin{solution}
注意到\(F(x,y) = (x^2+y^2)^2 - a^2(x^2-y^2)\)
既是关于变量\(x\)的偶函数,
也是关于变量\(y\)的偶函数,
所以它所围成的图形的面积
等于它所围成的图形在第一象限部分的面积的4倍.
令\(x = \rho \cos\theta,
y = \rho \sin\theta\),
则曲线方程化为\begin{equation*}
	\rho^4 = a^2 \rho^2 \cos^2\theta - b^2 \rho^2 \sin^2\theta,
\end{equation*}
消去\(\rho^2\)得\begin{equation*}
	\rho^2 = a^2 \cos^2\theta - b^2 \sin^2\theta.
\end{equation*}
曲线\((x^2+y^2)^2 = a^2 x^2 - b^2 y^2\)所围成的图形在第一象限部分所占据的角度范围为\begin{equation*}
	a^2 \cos^2\theta - b^2 \sin^2\theta \geq 0,
	\qquad
	0 \leq \theta \leq \frac\pi2,
\end{equation*}
即\begin{equation*}
	\tan^2\theta \leq \frac{a^2}{b^2}
	\quad\text{或}\quad
	0 \leq \theta \leq c = \arctan\frac{a^2}{b^2}.
\end{equation*}
因此曲线\((x^2+y^2)^2 = a^2 x^2 - b^2 y^2\)所围成的图形的面积为\begin{align*}
	S &= 4 \int_0^c \frac12 (a^2 \cos^2\theta - b^2 \sin^2\theta) \dd{\theta} \\
	% \int \sin^2 x \dd{x} = \frac{x}2 - \frac14 \sin 2x + C_3
	% \int \cos^2 x \dd{x} = \frac{x}2 + \frac14 \sin 2x + C_4
	%@Mathematica: 4 Integrate[1/2 (a^2 Cos[\[Theta]]^2 - b^2 Sin[\[Theta]]^2), {\[Theta], 0, c}] // Simplify
	&= c (a^2 - b^2) + \frac12 \sin2c (a^2 + b^2) \\
	%@Mathematica: % /. c -> ArcTan[a^2/b^2] // Simplify
	&= \frac{a^2 b^2 (a^2 + b^2)}{a^4 + b^4} + (a^2 - b^2) \arctan\frac{a^2}{b^2}.
\end{align*}
\end{solution}
\end{example}

\subsection{旋转体的体积}
\DefineConcept{旋转体}就是由一个平面图形绕着这平面内一条直线旋转一周而成的立体.
这条直线叫做\DefineConcept{旋转轴}.
圆柱可以看成是由矩形绕它的一条边旋转一周而成的立体.
圆锥可以看成是由直角三角形绕它的直角边旋转一周而成的立体.
圆台可以看成是由直角梯形绕它的直角腰旋转一周而成的立体.
球体可以看成是由半圆绕它的直径旋转一周而成的立体.
所以圆柱、圆锥、圆台、球体都是旋转体.

上述旋转体都可以看作是
由连续曲线\(y=f(x)\)、直线\(x=a\)和\(x=b\)及\(x\)轴所围成的曲边梯形
绕\(x\)轴旋转一周而成的立体.
现在我们考虑用定积分来计算这种旋转体的体积.

取横坐标\(x\)为积分变量,它的变化区间为\([a,b]\).
相应于\([a,b]\)上的任一小区间\([x,x+\dd{x}]\)的窄曲边梯形绕\(x\)轴旋转而成的薄片的体积
近似于以\(f(x)\)为底半径、\(\dd{x}\)为高的扁圆柱体的体积,
即体积元素\begin{equation*}
	\dd{V} = \pi [f(x)]^2 \dd{x}.
\end{equation*}
以\(\pi [f(x)]^2 \dd{x}\)为被积表达式,
在闭区间\([a,b]\)上作定积分,
便得所求旋转体体积为\begin{equation}\label{equation:定积分.曲边梯形绕x轴旋转体的体积}
	V = \pi \int_a^b [f(x)]^2 \dd{x}.
\end{equation}

\begin{example}
%@see: 《高等数学(第六版 上册)》 P278 例6
连接坐标原点\(O\)及点\(P(h,r)\)的直线、直线\(x=h\)及\(x\)轴围成一个直角三角形.
将它绕\(x\)轴旋转一周构成一个底半径为\(r\)、高为\(h\)的圆锥体.
计算着圆锥体的体积.
\begin{solution}
过原点\(O\)及点\(P(h,r)\)的直线方程为\begin{equation*}
	y = \frac{r}{h} x.
\end{equation*}
取横坐标\(x\)为积分变量,它的变化区间为\([0,h]\).
圆锥体中相应于\([0,h]\)上任一小区间\([x,x+\dd{x}]\)的薄片的体积
近似于底半径为\(\frac{r}{h} x\)、高为\(\dd{x}\)的扁圆柱体的体积,
即体积元素\begin{equation*}
	\dd{V} = \pi \left(\frac{r}{h} x\right)^2 \dd{x}.
\end{equation*}
于是所求圆锥体的体积为\begin{equation*}
	V = \int_0^h \pi \left(\frac{r}{h} x\right)^2 \dd{x}
	= \eval{\frac{\pi r^2 x^3}{3 h^2}}_0^h
	= \frac\pi3 r^2 h.
\end{equation*}
\end{solution}
\end{example}

\begin{example}
%@see: 《高等数学(第六版 上册)》 P279 例7
计算由椭圆\begin{equation*}
	\frac{x^2}{a^2}+\frac{y^2}{b^2}=1
\end{equation*}所围成的图形绕\(x\)轴旋转一周而成的旋转体的体积.
\begin{solution}
这个旋转椭球体也可以看作是由半个椭圆\(y = \frac{b}{a} \sqrt{a^2-x^2}\)
及\(x\)轴围成的图形
绕\(x\)轴旋转一周而成的立体.

取\(x\)为积分变量,它的变化区间为\([-a,a]\).
旋转椭球体中相应于\([-a,a]\)上任一小区间\([x,x+\dd{x}]\)的薄片的体积,
近似于底半径为\(\frac{b}{a} \sqrt{a^2-x^2}\)、高为\(\dd{x}\)的扁圆柱体的体积,
即体积元素\begin{equation*}
	\dd{V} = \frac{\pi b^2}{a^2} (a^2-x^2) \dd{x}.
\end{equation*}
于是所求旋转椭球体的体积为\begin{equation*}
	V = \int_{-a}^a \pi \frac{b^2}{a^2} (a^2-x^2) \dd{x}
	= \eval{\pi \frac{b^2}{a^2} \left(a^2x-\frac13x^3\right)}_{-a}^a
	= \frac43 \pi a b^2.
\end{equation*}
\end{solution}
\end{example}

\begin{example}
%@see: 《2009年全国硕士研究生入学统一考试(数学一)》三解答题/第17题
椭球面\(S_1\)是椭圆\(\frac{x^2}{a^2} + \frac{y^2}{b^2} = 1\)绕\(x\)轴旋转而成,
圆锥面\(S_2\)是由过点\(M_0(p,0)\)且与上述椭圆相切的直线绕\(x\)轴旋转而成,
其中\(p>a>0,b>0\),
计算\(S_1\)与\(S_2\)之间的立体的体积.
\begin{solution}
椭圆上任意一点\((\xi,\eta)\)的切线方程为\begin{equation*}
	\frac{x\xi}{a^2} + \frac{y\eta}{b^2} = 1.
	\eqno(1)
\end{equation*}
代入点\(M_0\)的坐标得\(\frac{p\xi}{a^2} = 1\),
解得\begin{equation*}
%@Mathematica: Solve[{(x \[Xi])/a^2 + (y \[Eta])/b^2 == 1}, \[Xi]] /. {x -> p, y -> 0}
	\xi = \frac{a^2}{p}.
	\eqno(2)
\end{equation*}
由于\((\xi,\eta)\)同时满足旋转椭球面方程,即\begin{equation*}
	\frac{\xi^2}{a^2} + \frac{\eta^2}{b^2} = 1,
\end{equation*}
解出\(\eta\)的正根,得\begin{equation*}
%@Mathematica: Solve[\[Xi]^2/a^2 + \[Eta]^2/b^2 == 1, \[Eta]] /. %1[[1]]
	\eta = \frac{b}{p} \sqrt{p^2 - a^2}.
	\eqno(3)
\end{equation*}
于是连接\(M\left( \frac{a^2}{p}, \frac{b}{p} \sqrt{p^2 - a^2} \right)\)与\(M_0\)的直线
就是上述椭圆切线\begin{equation*}
	l: y = \frac{\eta}{\xi - p} (x - p),
\end{equation*}
而圆锥面\(S_2\)可以看作切线\(l\)绕\(x\)轴旋转所得.
因此\(S_1\)与\(S_2\)之间的立体的体积为\begin{align*}
%@Mathematica: V1 = Integrate[((\[Eta] (x - p))/(\[Xi] - p))^2, {x, \[Xi], p}]
%@Mathematica: V2 = Integrate[b^2 (1 - x^2/a^2), {x, \[Xi], a}]
	V &= \pi \int_\xi^p \left[ \frac{\eta}{\xi - p} (x - p) \right]^2 \dd{x}
	- \pi \int_\xi^a b^2 \left( 1 - \frac{x^2}{a^2} \right) \dd{x} \\
	&= \pi \left[ \frac{b^2 (a^2-p^2)^2}{3 p^3} - \frac{2 a b^2}{3} - \frac{a^4 b^2}{3 p^3} + \frac{a^2 b^2}{p} \right].
\end{align*}
\end{solution}
\end{example}

由连续曲线\(y=f(x)\)、直线\(x=a\)和\(x=b\)及\(x\)轴所围成的曲边梯形
绕\(y\)轴旋转一周而成的立体的体积为
\begin{equation}\label{equation:定积分.曲边梯形绕y轴旋转体的体积}
	V = 2\pi \int_a^b \abs{x f(x)} \dd{x}.
\end{equation}

\begin{theorem}[古尔丁体积定理]
%@see: http://www.shuxueji.com/w/1063
由平面图形绕与它共面的一条直线旋转一周而成的立体的体积为\begin{equation}
	V = S d,
\end{equation}
其中\(S\)是这个平面图形的面积,
\(d\)是这个平面图形的几何中心到旋转轴的距离.
%TODO proof
\end{theorem}

\subsection{平行截面面积为已知的立体的体积}
从计算旋转体体积的过程中可以看出:
如果一个立体不是旋转体,但我们知道该立体上垂直于某个轴的各个截面的面积,
那么,这个立体的体积也可以用定积分来计算.

取上述定轴为\(x\)轴,
并设该立体在过点\(x=a\)、\(x=b\)且垂直于\(x\)轴的两个平面之间.
以\(A(x)\)表示过点\(x\)且垂直于\(x\)轴的截面面积.
假定\(A(x)\)是\(x\)的连续函数.
这时,取\(x\)为积分变量,它的变化区间为\([a,b]\);
立体中相应于\([a,b]\)上任一小区间\([x,x+\dd{x}]\)的一薄片的体积,
近似于底面积为\(A(x)\)、高为\(\dd{x}\)的扁柱体的体积,
即体积元素\begin{equation*}
	\dd{V} = A(x) \dd{x}.
\end{equation*}
以\(A(x) \dd{x}\)为被积表达式,在闭区间\([a,b]\)上作定积分,
便得所求立体的体积\begin{equation*}
	V = \int_a^b A(x) \dd{x}.
\end{equation*}

\begin{example}
%@see: 《高等数学(第六版 上册)》 P281 例9
一平面经过半径为\(R\)的圆柱体的底圆中心,并与底圆交成角\(\alpha\).
计算这平面截圆柱体所得立体的体积.
\begin{solution}
取这平面与圆柱体的底面的交线为\(x\)轴,
底面上过圆中心且垂直于\(x\)轴的直线为\(y\)轴.
那么,底圆的方程为\(x^2+y^2=R^2\).
立体中过\(x\)轴上的点\(x\)且垂直于\(x\)轴的截面是一个直角三角形.
它的两条直角边的长分别为\(y\)及\(y \tan\alpha\),
即\(\sqrt{R^2-x^2}\)及\(\sqrt{R^2-x^2} \tan\alpha\).
因而截面积为\begin{equation*}
	A(x) = \frac12 (R^2-x^2) \tan\alpha,
\end{equation*}
于是所求立体体积为\begin{equation*}
	V = \int_{-R}^R \frac12 (R^2-x^2) \tan\alpha \dd{x}
	= \frac12 \tan\alpha \eval{\left(R^2x-\frac13x^3\right)}_{-R}^R
	= \frac23 R^3 \tan\alpha.
\end{equation*}
\end{solution}
\end{example}

\begin{example}
%@see: 《高等数学(第六版 上册)》 P281 例10
求以半径为\(R\)的圆为底、平行且等于底圆直径的线段为顶、高为\(h\)的正劈锥体的体积.
\begin{solution}
取底圆所在的平面为\(xOy\)平面,圆心\(O\)为原点,并使\(x\)轴与正劈锥体的顶平行.
底圆的方程为\(x^2+y^2=R^2\).
过\(x\)轴上的点\(x\ (-R \leq x \leq R)\)作垂直于\(x\)轴的平面,
截正劈锥体得等腰三角形,这个截面的面积为\begin{equation*}
	A(x) = h \cdot y = h \sqrt{R^2-x^2},
\end{equation*}
于是所求正劈锥体的体积为\begin{equation*}
	V = \int_{-R}^R A(x) \dd{x}
	= h \int_{-R}^R \sqrt{R^2-x^2} \dd{x}
	= 2 R^2 h \int_0^{\frac\pi2} \sin^2\theta \dd{\theta}
	= \frac{\pi R^2 h}2.
\end{equation*}
由此可见,正劈锥体的体积等于同底同高的圆柱体体积的一半.
\end{solution}
\end{example}

\begin{example}
%@see: https://www.bilibili.com/video/BV1xT42167ni
计算由曲线\(y=\sqrt{x}\)与直线\(y=x\)所围成的图形绕直线\(y=x\)旋转一周而成的旋转体的体积.
\begin{solution}
先求抛物线和直线的交点,解方程组\begin{equation*}
	\left\{ \begin{array}{l}
		y=\sqrt{x}, \\
		y=x
	\end{array} \right.
\end{equation*}得交点\((0,0)\)和\((1,1)\),
从而知道这图形在直线\(x=0\)和\(x=1\)之间.
现在,选取横坐标\(x\)为积分变量,
对应的积分区间为\([0,1]\).
取抛物线上一点\(P(x,\sqrt{x})\),
则点\(P\)到直线\(y=x\)的距离为\begin{equation*}
	r = \frac{\abs{x-\sqrt{x}}}{\sqrt2}.
\end{equation*}
邻近点\(P\)的抛物线弧与直线夹成的窄曲边梯形绕直线旋转而成的薄片的体积
近似于底半径为\(r\)、高为\(\sqrt2 \dd{x}\)的扁圆柱体的体积,
即体积元素\begin{equation*}
	\dd{V} = \frac\pi2 (\sqrt{x}-x)^2 \cdot \sqrt2 \dd{x}.
\end{equation*}
把这个体积元素作为被积表达式,在闭区间\([0,1]\)上作定积分,
便得所求的体积为\begin{equation*}
	V = \int_0^1 \frac\pi2 (\sqrt{x}-x)^2 \cdot \sqrt2 \dd{x}
	= \frac\pi{30\sqrt2}.
\end{equation*}
\end{solution}
\end{example}

\subsection{平面曲线的弧长}
我们知道,圆的周长可以利用圆的内接正多边形(或外切正多边形)的周长在其边数诬陷增多时的极限来确定.
类似地,我们建立平面曲线的弧长的概念,并运用定积分计算平面曲线的弧长.

设\(A\)、\(B\)是平面曲线弧的两个端点.
在弧\(\Arc{AB}\)上依次任取分点\begin{equation*}
	A=M_0,M_1,M_2,\dotsc,M_{n-1},M_n=B,
\end{equation*}并依次连接相邻的分点得一条折线.
当分点的数目无限增加且每个小段\(\Arc{M_{i-1}M_i}\)都缩向一点时,
如果此折线的长\begin{equation*}
	\sum_{i=1}^n \abs{M_{i-1} M_i}
\end{equation*}的极限存在,
则称“极限\(\lim_{n\to\infty} \sum_{i=1}^n \abs{M_{i-1} M_i}\)是
曲线弧\(\Arc{AB}\)的\DefineConcept{弧长}”,
并称“曲线弧\(\Arc{AB}\)是\DefineConcept{可求长的}”.

\begin{theorem}
%@see: 《高等数学(第六版 上册)》 P282 定理
光滑曲线弧是可求长的.
\end{theorem}

下面利用定积分的元素法来讨论平面光滑曲线弧长的计算公式.

设曲线弧由参数方程\begin{equation*}
	\left\{ \begin{array}{l}
	x = \phi(t), \\
	y = \psi(t)
	\end{array} \right.
	\quad(\alpha \leq t \leq \beta)
\end{equation*}给出,
其中\(\phi\)、\(\psi\)在\([\alpha,\beta]\)上具有连续导数,
且\(\phi'\)、\(\psi'\)不同时为零.
现在来计算这曲线弧的长度.

取参数\(t\)为积分变量,它的变化区间为\([\alpha,\beta]\).
相应于\([\alpha,\beta]\)上任一小区间\([t,t+\dd{t}]\)的小弧段的长度\(\increment s\)
近似等于对应的弦的长度\(\sqrt{(\increment x)^2+(\increment y)^2}\),
因为\begin{equation*}
	\increment x
	= \phi(t+\dd{t})-\phi(t)
	\approx \dd{x}
	= \phi'(t) \dd{t},
\end{equation*}\begin{equation*}
	\increment y
	= \psi(t+\dd{t})-\psi(t)
	\approx \dd{y}
	= \psi'(t) \dd{t},
\end{equation*}
所以,\(\increment s\)的近似值(弧微分)即弧长元素为\begin{equation*}
	\dd{s} = \sqrt{(\dd{x})^2 + (\dd{y})^2}
	= \sqrt{(\phi'(t))^2 + (\psi'(t))^2} \dd{t}.
\end{equation*}
于是所求弧长为\begin{equation}
	s = \int_\alpha^\beta \sqrt{(\phi'(t))^2 + (\psi'(t))^2} \dd{t}.
\end{equation}

当曲线弧由直角坐标方程\begin{equation*}
	y = f(x) \quad(a \leq x \leq b)
\end{equation*}给出,
其中\(f(x)\)在\([a,b]\)上具有一阶连续导数,
这时曲线弧有参数方程\begin{equation*}
	\left\{ \begin{array}{l}
		x = x, \\
		y = f(x)
	\end{array} \right.
	\quad(\alpha \leq t \leq \beta),
\end{equation*}
从而所求弧长为\begin{equation}
	s = \int_a^b \sqrt{1+(y')^2} \dd{x}.
\end{equation}

当曲线弧由极坐标方程\begin{equation*}
	\rho=\rho(\theta)
	\quad(\alpha \leq \theta \leq \beta)
\end{equation*}给出,
其中\(\rho(\theta)\)在\([\alpha,\beta]\)上具有连续导数,
则由直角坐标与极坐标的关系可得\begin{equation*}
	\left\{ \begin{array}{c}
		x = \rho(\theta) \cos\theta, \\
		y = \rho(\theta) \sin\theta
	\end{array} \right.
	\quad(\alpha \leq \theta \leq \beta).
\end{equation*}
这就是以极角\(\theta\)为参数的曲线弧的参数方程.
而\begin{gather*}
	x'(\theta) = \rho'(\theta) \cos\theta - \rho(\theta) \sin\theta, \\
	y'(\theta) = \rho'(\theta) \sin\theta + \rho(\theta) \cos\theta.
\end{gather*}
于是,弧长元素为\begin{align*}
	\dd{s}
	&= \sqrt{(x'(\theta))^2 + (y'(\theta))^2} \dd{\theta} \\
	&= \sqrt{(\rho' \cos\theta - \rho \sin\theta)^2
		+ (\rho' \sin\theta + \rho \cos\theta)^2} \dd{\theta} \\
	&= \sqrt{(\rho')^2\cos^2\theta-2\rho'\rho\cos\theta\sin\theta+\rho^2\sin^2\theta
		+ (\rho')^2\sin^2\theta+2\rho'\rho\sin\theta\cos\theta+\rho^2\cos^2\theta} \dd{\theta} \\
	&= \sqrt{\rho^2 + (\rho')^2} \dd{\theta}.
\end{align*}
从而所求弧长为\begin{equation}
	s = \int_\alpha^\beta \sqrt{\rho^2(\theta) + (\rho'(\theta))^2} \dd{\theta}.
\end{equation}

\begin{example}
%@see: 《高等数学(第六版 上册)》 P283 例12
计算摆线的一拱\begin{equation*}
	\left\{ \begin{array}{l}
		x = a (\theta - \sin\theta), \\
		y = a (1 - \cos\theta)
	\end{array} \right.
	\quad(0 \leq \theta \leq 2\pi)
\end{equation*}的长度.
\begin{solution}
弧长元素为\begin{align*}
	\dd{s} &= \sqrt{a^2 (1 - \cos\theta)^2 + a^2 \sin^2\theta} \dd{\theta} \\
	&= a \sqrt{2 (1 - \cos\theta)} \dd{\theta}
	= 2 a \sin\frac\theta2 \dd{\theta}.
\end{align*}
从而,所求弧长\begin{equation*}
	s = \int_0^{2\pi} 2a \sin\frac\theta2 \dd{\theta}
	= -4a \eval{\left(\cos\frac\theta2\right)}_0^{2\pi}
	= 8a.
\end{equation*}
\end{solution}
\end{example}

\begin{example}
%@see: 《高等数学(第六版 上册)》 P283 例13
求阿基米德螺线\(\rho=a\theta\ (a>0)\)相应于\(0\leq\theta\leq2\pi\)一段的弧长.
%@Mathematica: PolarPlot[t, {t, 0, 2 Pi}]
\begin{solution}
弧长元素为\begin{equation*}
	\dd{s} = \sqrt{(a\theta)^2 + a^2} \dd{\theta}
	= a\sqrt{\theta^2+1} \dd{\theta},
\end{equation*}
于是所求弧长为\begin{equation*}
	s = a \int_0^{2\pi} \sqrt{1+\theta^2} \dd{\theta}
	= \frac{a}{2} \left[
	2\pi\sqrt{1+4\pi^2} + \ln(2\pi+\sqrt{1+4\pi^2})
	\right].
\end{equation*}
\end{solution}
\end{example}

\begin{example}
%@see: 《高等数学(第六版 上册)》 P287 习题6-2 28.
求对数螺线\(\rho=e^{\theta}\)相应于\(0\leq\theta\leq\pi\)一段的弧长.
%@Mathematica: PolarPlot[Exp[t], {t, 0, Pi}]
\begin{solution}
弧长元素为\begin{equation*}
	\dd{s} = \sqrt{e^{2\theta}+e^{2\theta}} \dd{\theta}
	= \sqrt{2}e^{\theta} \dd{\theta},
\end{equation*}
于是所求弧长为\begin{equation*}
	s = \sqrt{2} \int_0^\pi e^{\theta} \dd{\theta}
	= \sqrt{2} (e^\pi-1).
\end{equation*}
\end{solution}
\end{example}

\begin{example}
%@see: 《高等数学(第六版 上册)》 P287 习题6-2 30.
求心形线\(\rho=a(1+\cos\theta)\)的全长.
%@Mathematica: PolarPlot[1 + Cos[t], {t, 0, 2 Pi}]
\begin{solution}
弧长元素为\begin{equation*}
	\dd{s} = \sqrt{a^2(1+\cos\theta)^2+a^2\sin^2\theta} \dd{\theta}
	= a \sqrt{2(1+\cos\theta)} \dd{\theta},
\end{equation*}
考虑到心形线\(\rho=a(1+\cos\theta)\)对称于极轴,
于是所求弧长为\begin{equation*}
	s = 2 \int_0^\pi a \sqrt{2(1+\cos\theta)} \dd{\theta}
	= 4a \int_0^\pi \cos\frac\theta2 \dd{\theta}
	= 8a.
\end{equation*}
\end{solution}
\end{example}

\subsection{旋转面的表面积}
%@see: 《数学分析(第二版 上册)》(陈纪修) P324
设参数方程\begin{equation*}
	\left\{ \begin{array}{l}
		x = x(t), \\
		y = y(t)
	\end{array} \right.
	\quad(\alpha \leq t \leq \beta)
\end{equation*}确定平面上一段光滑曲线\(C\),
且在\([\alpha,\beta]\)上\(y(t) \geq 0\),
它绕\(x\)轴旋转一周得到一个旋转曲面.

对区间\([\alpha,\beta]\)作划分:\begin{equation*}
	\alpha = t_0 < t_1 < t_2 < \dotsb < t_{n-1} < t_n = \beta.
\end{equation*}
由此得到曲线上顺次排列的\(n+1\)个点\(P_0,P_1,\dotsc,P_n\).
记\(\increment S_i\)是连接\(P_{i-1}\)和\(P_i\)的直线段
绕\(x\)轴旋转一周得到的圆台的侧面面积,
即\begin{equation*}
	\increment S_i
	\defeq
	\pi [y(t_{i-1} + y(t_i))] \cdot \overline{P_{i-1} P_i}.
\end{equation*}
若当\(\lambda=\max_{1 \leq i \leq n} \increment t_i \to 0\)时,
极限\begin{equation*}
	\lim_{\lambda\to0} \sum_{i=1}^n \increment S_i
\end{equation*}存在,
且极限值与区间\([\alpha,\beta]\)的划分无关,
则称极限值为“曲线\(C\)绕\(x\)轴旋转一周所得旋转曲面的面积”.
可以证明:曲线\(C\)绕\(x\)轴旋转一周所得旋转曲面的面积为\begin{equation}
	S = 2\pi \int_\alpha^\beta y(t) \sqrt{[x'(t)]^2 + [y'(t)]^2} \dd{x}.
\end{equation}

\begin{theorem}[古尔丁表面积定理]
%@see: http://www.shuxueji.com/w/1063
由连续曲线\(y=f(x)\ (a \leq x \leq b)\)
绕\(x\)轴旋转一周而成的旋转面的表面积\begin{equation}
	S = 2\pi \int_a^b y \sqrt{1+\left(\dv{y}{x}\right)^2} \dd{x}.
\end{equation}
%TODO proof
\end{theorem}

\begin{example}
%@see: 《数学分析(第二版 上册)》(陈纪修) P325 例7.4.13
求半径为\(a\)的球的表面积.
\begin{solution}
半径为\(a\)的球的表面积,
就是半径为\(a\)的圆的上半部分\(y=\sqrt{a^2-x^2}\)
绕\(x\)轴旋转一周所得的旋转曲面的面积.
于是球的表面积为\begin{equation*}
	S = 2\pi \int_{-a}^a y \sqrt{1+(y')^2} \dd{x}
	= 2\pi \int_{-a}^a \sqrt{a^2-x^2} \cdot
		\sqrt{1+\left(\frac{-x}{\sqrt{a^2-x^2}}\right)^2} \dd{x}
	= 4\pi a^2.
\end{equation*}
\end{solution}
\end{example}

\begin{example}
%@see: 《数学分析(第二版 上册)》(陈纪修) P325 例7.4.14
求摆线的一拱\begin{equation*}
	\left\{ \begin{array}{l}
		x = a (t - \sin t), \\
		y = a (1 - \cos t)
	\end{array} \right.
	\quad(0 \leq t \leq 2\pi)
\end{equation*}绕\(x\)轴旋转一周所得旋转曲面的面积.
\begin{solution}
将摆线的参数方程代入旋转曲面面积公式,得\begin{align*}
	S &= 2\pi a^2 \int_0^{2\pi} (1-\cos t) \sqrt{(1-\cos t)^2 + \sin^2 t} \dd{t} \\
	&= 2\sqrt2 \pi a^2 \int_0^{2\pi} (1-\cos t) \sqrt{1-\cos t} \dd{t} \\
	&= 16\pi a^2 \int_0^{2\pi} \sin^3 \frac{t}{2} \dd(\frac{t}{2})
	= \frac{64}3 \pi a^2.
\end{align*}
\end{solution}
\end{example}

\section{面积原理}
%@see: 《数学分析教程 (第3版 上册)》(史济怀) P300
在定义积分的时候,我们就把积分解释为曲边梯形的面积.
这样就把“数”与“形”结合了起来.
数与形的恰当的、巧妙的结合,往往给我们带来新的思想和新的发现.
在这一节中,我们研究怎么利用积分估计和式.

\begin{figure}[htb]
%@see: 《数学分析教程 (第3版 上册)》(史济怀) P301 图7.7
	\centering
	\begin{tikzpicture}
		\begin{axis}[
			xmin=.4,xmax=1,
			ymin=0,ymax=1,
			axis x line=middle,
			axis y line=none,
			xtick={.6,.8},
			xticklabels={$k-1$,$k$},
		]
			\addplot[color=blue,samples=50,smooth,domain=.5:.9]{x^2};
			\draw(.6,0)--(.6,.36);
			\draw(.8,0)--(.8,.64);
			\draw[dashed](.6,.36)|-(.8,.64);
			\draw[dashed](.6,.36)--(.8,.36);
		\end{axis}
	\end{tikzpicture}
	\caption{}
	\label{figure:面积原理.定积分与曲边梯形面积的联系}
\end{figure}

\begin{theorem}
%@see: 《数学分析教程 (第3版 上册)》(史济怀) P301 定理7.3.1
设\(m\)是正整数,
\(f\)是在\([m,+\infty)\)上非负的单调增加的函数,
那么当\(\xi \geq m\)时,
有\begin{equation*}
%@see: 《数学分析教程 (第3版 上册)》(史济怀) P301 (1)
	\abs{
		\sum_{k=m}^{\floor{\xi}} f(k)
		- \int_m^\xi f(x) \dd{x}
	} \leq f(\xi).
\end{equation*}
\begin{proof}
由\cref{theorem:定积分.定积分性质3} 可知\begin{equation*}
	\int_m^n f(x) \dd{x}
	= \sum_{k=m+1}^n \int_{k-1}^k f(x) \dd{x}.
\end{equation*}
由\cref{theorem:取整函数.性质1} 可知,
\(\floor{\xi} \leq \xi < \xi + 1\),
于是再次利用\cref{theorem:定积分.定积分性质3} 可得\begin{equation*}
	\int_m^\xi f(x) \dd{x}
	= \int_m^{\floor{\xi}} f(x) \dd{x}
	+ \int_{\floor{\xi}}^\xi f(x) \dd{x}.
\end{equation*}
如\cref{figure:面积原理.定积分与曲边梯形面积的联系},
利用定积分的几何意义,容易看出\begin{equation*}
	f(m) + f(m+1) + \dotsb + f(\floor{\xi}-1)
	\leq \int_m^{\floor{\xi}} f(x) \dd{x}
	\leq f(m+1) + f(m+2) + \dotsb + f(\floor{\xi}).
\end{equation*}
又因为\(f\)是单调增加的,
\(f(\xi)\)是\(f\)在\([m,\xi]\)上的最大值,
由\cref{theorem:定积分.定积分性质6} 有\begin{equation*}
	0 \leq \int_{\floor{\xi}} f(x) \dd{x}
	\leq f(\xi) (\xi - \floor{\xi})
	\leq f(\xi),
\end{equation*}
所以\begin{equation*}
	f(m) + f(m+1) + \dotsb + f(\floor{\xi}-1)
	\leq \int_m^\xi f(x) \dd{x}
	\leq f(m+1) + f(m+2) + \dotsb + f(\floor{\xi}) + f(\xi).
\end{equation*}
在上式等号两边同时减去\(\sum_{k=m}^{\floor{\xi}} f(k)
= f(m) + f(m+1) + \dotsb + f(\floor{\xi})\),
便得\begin{equation*}
	-f(\xi)
	\leq -f(\floor{\xi})
	\leq \int_m^\xi f(x) \dd{x} - \sum_{k=m}^{\floor{\xi}} f(k)
	\leq f(\xi) - f(m)
	\leq f(\xi).
\end{equation*}
于是\begin{equation*}
	\abs{\int_m^\xi f(x) \dd{x} - \sum_{k=m}^{\floor{\xi}} f(k)} \leq f(\xi).
	\qedhere
\end{equation*}
\end{proof}
\end{theorem}

\begin{example}
\(\abs{\sum_{k=1}^n k^p - \int_1^n x^p \dd{x}}
= \abs{1^p+\dotsb+n^p - \frac{n^{p+1}-1}{p+1}}
\leq n^p\).
%TODO 怎么丢掉常数\(\frac1{p+1}\),
%得到\(\abs{1^p+\dotsb+n^p-\frac{n^{p+1}}{p+1}}\)?
%从这里可以进一步得到
%\(\abs{\frac{1^p+\dotsb+n^p}{n^{p+1}}-\frac1{p+1}} \leq \frac1n\).
\end{example}

\begin{example}
\(\abs{\sum_{k=1}^n \ln k - \int_1^n \ln x \dd{x}}
= \abs{\ln n! - \ln\left[\left(\frac{n}{e}\right)^n e\right]}
\leq \ln n\).
\end{example}
\begin{remark}
打开绝对值符号,
立即可以得到\begin{equation*}
	\ln\left[\left(\frac{n}{e}\right)^n \frac{e}{n}\right]
	\leq \ln n!
	\leq \ln\left[\left(\frac{n}{e}\right)^n e n\right],
\end{equation*}
从而有\begin{equation*}
	\left(\frac{n}{e}\right)^n \frac{e}{n}
	\leq n!
	\leq \left(\frac{n}{e}\right)^n e n.
\end{equation*}
\end{remark}

\begin{theorem}
%@see: 《数学分析教程 (第3版 上册)》(史济怀) P302 定理7.3.2
设\(m\)是正整数,
\(f\)是在\([m,+\infty)\)上非负的单调减少的函数,
则极限\begin{equation*}
	\alpha
	\defeq
	\lim_{n\to\infty} \left[
		\sum_{k=m}^n f(k)
		- \int_m^n f(t) \dd{t}
	\right]
\end{equation*}存在,
且\(\alpha\in[0,f(m)]\).
\begin{proof}
记\(g(x)
= \sum_{k=m}^{\floor{x}} f(k)
- \int_m^x f(t) \dd{t}\),
那么\begin{equation*}
	g(n) - g(n+1)
	= -f(n+1)
	+ \int_n^{n+1} f(t) \dd{t}.
\end{equation*}
因为\begin{equation*}
	\int_n^{n+1} f(t) \dd{t}
	\geq
	1 \cdot f(n+1),
\end{equation*}
所以\(g(n) - g(n+1) \geq 0\),\(\{g(n)\}\)单调性得证.

又因为\begin{align*}
	g(n)
	&= \sum_{k=m}^{n-1} f(k) - \int_m^n f(t) \dd{t} \\
	&= \sum_{k=m}^{n-1} \left[
			f(k) - \int_k^{k+1} f(t) \dd{t}
		\right] + f(n) \\
	&\geq \sum_{k=m}^{n-1} [f(k) - f(k)] + f(n)
	= f(n) \geq 0,
\end{align*}
这就是说\(\{g(n)\}\)是非负的单调减少的数列,
从而\(\{g(n)\}\)有界性得证.

因此,根据单调有界定理,\(\alpha = \lim_{n\to\infty} g(n)\)存在.
又由\(0 \leq g(n) \leq g(m) = f(m)\)
可知\(0 \leq \alpha \leq f(m)\).
\end{proof}
%@see: https://encyclopediaofmath.org/wiki/Area_principle
\end{theorem}
\begin{remark}
取\(m=1\)和\(f(x) = \frac1x\),
立即可得\cref{example:微分中值定理.拉格朗日中值定理.欧拉--马歇罗尼常数} 的结论.
\end{remark}

\begin{corollary}
%@see: 《数学分析教程 (第3版 上册)》(史济怀) P302 定理7.3.2
设\(m\)是正整数,
\(f\)是在\([m,+\infty)\)上非负的单调减少的函数,
且\(\lim_{x\to+\infty} f(x) = 0\),
则\begin{equation*}
	\abs{
		\sum_{k=m}^{\floor{\xi}} f(k)
		- \int_m^\xi f(t) \dd{t}
		- \alpha
	}
	\leq
	f(\xi-1),
\end{equation*}
其中\(\alpha = \lim_{n\to\infty} \left[
	\sum_{k=m}^n f(k)
	- \int_m^n f(t) \dd{t}
\right]\).
%TODO proof
\end{corollary}

\begin{proposition}
%@see: 《数学分析教程 (第3版 上册)》(史济怀) P304 例6
设\(0<a<b\),则\begin{equation}
%@see: 《数学分析教程 (第3版 上册)》(史济怀) P304 (9)
	\frac2{a+b}
	< \frac{\ln b - \ln a}{b - a}
	< \frac12 \left(\frac1a + \frac1b\right).
\end{equation}
\begin{proof}
考察函数\(f(x) = \frac1x\).
当\(x>0\)时,它是凸函数.
连接\((a,1/a)\)与\((b,1/b)\)两点的弦必定在相应的曲线段\(y=1/x\ (a \leq x \leq b)\)的上方,
于是由这条弦与直线\(x=a\)、\(x=b\)以及横轴所围成的梯形的面积
大于曲线\(y = f(x)\)与直线\(x=a\)、\(x=b\)以及横轴所围成的曲边梯形的面积,
因此\begin{equation*}
	\int_a^b \frac{\dd{x}}{x}
	< \frac12 \left( \frac1a + \frac1b \right) (b-a),
\end{equation*}
从而有\begin{equation*}
	\frac{\ln b - \ln a}{b - a}
	< \frac12 \left( \frac1a + \frac1b \right).
\end{equation*}

接下来,过曲线\(y = f(x)\)上一点\(\left( \frac{a+b}{2}, \frac{2}{a+b} \right)\)作它的切线,
这条切线与直线\(x=a\)、\(x=b\)以及横轴所围成的梯形的面积正好等于\(2\frac{b-a}{a+b}\),
小于曲线\(y = f(x)\)与直线\(x=a\)、\(x=b\)以及横轴所围成的曲边梯形的面积,
因此\begin{equation*}
	\int_a^b \frac{\dd{x}}{x}
	> 2\frac{b-a}{b+a}.
	\qedhere
\end{equation*}
\end{proof}
\end{proposition}

\begin{proposition}
%@see: 《数学分析教程 (第3版 上册)》(史济怀) P305 例7
设连续函数\(\phi\)在\([0,+\infty)\)上严格单调增加,且\(\phi(0) = 0\),
则\begin{itemize}
	\item \(\phi\)的反函数\(\phi^{-1}\)存在,
	且在\([0,\phi(+\infty)]\)上严格单调增加,
	并且\(\phi^{-1}(0) = 0\).

	\item 对于\(\forall a>0\)和\(\forall b\in(0,\phi(+\infty))\),有\begin{equation}
	%@see: 《数学分析教程 (第3版 上册)》(史济怀) P305 (11)
		a b \leq \int_0^a \phi(x) \dd{x} + \int_0^b \phi^{-1}(y) \dd{y},
	\end{equation}
	当且仅当\(b = \phi(a)\)时取“\(=\)”号.
\end{itemize}
%TODO proof
\end{proposition}

\section{积分不等式}
%@see: https://loss.math.gatech.edu/10springtea/Lecture1.pdf
%@see: https://www2.math.upenn.edu/~brweber/Courses/2011/Math361/Notes/YMandH.pdf
%@see: https://pi.math.cornell.edu/~erin/analysis/lectures.pdf

\subsection{杨格不等式}
\begin{theorem}
%@see: 《数学分析(第二版 上册)》(陈纪修) P294 习题 10.
%@see: 《数学分析教程 (第3版 上册)》(史济怀) P305 例7
设\(f\)是在\([0,+\infty)\)上严格单调增加的连续函数,且\(f(0)=0\),\(f^{-1}\)是\(f\)的反函数,
则对于\(\forall a,b>0\)有\begin{equation}
	\int_0^a f(x) \dd{x} + \int_0^b f^{-1}(y) \dd{y} \geq ab.
\end{equation}
当且仅当\(b = f(a)\)时上式取“\(=\)”号.
\begin{proof}
易知\(f^{-1}\)是在\([0,f(+\infty))\)上严格单调增加的连续函数,且\(f^{-1}(0) = 0\).
\begin{figure}[htb]
	\centering
	\begin{tikzpicture}
		\begin{axis}[
			xmin=0,xmax=2,
			ymin=0,ymax=3,
			axis lines=middle,
			xlabel=$x$,
			ylabel=$y$,
			enlarge x limits=0.1,
			enlarge y limits=0.1,
			x label style={at={(ticklabel* cs:1.00)}, inner sep=5pt, anchor=south},
			y label style={at={(ticklabel* cs:1.00)}, inner sep=2pt, anchor=west},
		]
			\addplot[color=blue,samples=50,smooth,domain=0:2]{x^2};
			\draw(1.5,1)--(0,1);
			\draw[dashed](1,1)--(1,0);
			\draw(1.5,0)--(1.5,2.25);
		\end{axis}
	\end{tikzpicture}\hspace{1cm}\begin{tikzpicture}
		\begin{axis}[
			xmin=0,xmax=2,
			ymin=0,ymax=3,
			axis lines=middle,
			xlabel=$x$,
			ylabel=$y$,
			enlarge x limits=0.1,
			enlarge y limits=0.1,
			x label style={at={(ticklabel* cs:1.00)}, inner sep=5pt, anchor=south},
			y label style={at={(ticklabel* cs:1.00)}, inner sep=2pt, anchor=west},
		]
			\addplot[color=blue,samples=50,smooth,domain=0:2]{x^2};
			\draw[dashed](1,1)--(0,1);
			\draw(0,2)--({sqrt(2)},2);
			\draw(1,2)--(1,0);
		\end{axis}
	\end{tikzpicture}
	\caption{}
	\label{figure:积分不等式.杨格不等式}
\end{figure}
观察\cref{figure:积分不等式.杨格不等式} 中图形的面积关系即可确认不等式的正确性.
\end{proof}
\end{theorem}

\subsection{赫尔德不等式}
\begin{theorem}\label{theorem:定积分.赫尔德不等式}
%@see: 《数学分析(第二版 上册)》(陈纪修) P291 例7.2.2(Holder不等式)
%@see: 《数学分析教程 (第3版 上册)》(史济怀) P307 练习题7.3 3.
设\(f\)和\(g\)都在\([a,b]\)上连续,正数\(p,q\)满足\(\frac1p+\frac1q=1\),
则\begin{equation}
	\int_a^b \abs{f(x)~g(x)} \dd{x}
	\leq \left[\int_a^b \abs{f(x)}^p \dd{x}\right]^{\frac1p}
	\left[\int_a^b \abs{g(x)}^q \dd{x}\right]^{\frac1q},
\end{equation}
当且仅当\(\abs{f}^p = B \abs{g}^q\ (\text{$B$是常数})\)时取“\(=\)”号.
\begin{proof}
当\(f(x) \equiv 0\)或\(g(x) \equiv 0\)时,上式显然成立.

当\(f(x) \not\equiv 0\)且\(g(x) \not\equiv 0\)时,
令\begin{equation*}
	\phi(x) = \frac{\abs{f(x)}}{\left[\int_a^b \abs{f(x)}^p \dd{x}\right]^{\frac1p}},
	\qquad
	\psi(x) = \frac{\abs{g(x)}}{\left[\int_a^b \abs{g(x)}^q \dd{x}\right]^{\frac1q}},
	\quad
	x\in[a,b].
\end{equation*}
由\hyperref[example:函数图像的绘制.杨格不等式]{杨格不等式}可得\begin{equation*}
	\phi(x)~\psi(x)
	\leq \frac1p \phi^p(x) + \frac1q \psi^q(x),
\end{equation*}
即\begin{equation*}
	\frac{\abs{f(x)~g(x)}}{
		\left[\int_a^b \abs{f(x)}^p \dd{x}\right]^{\frac1p}
		\left[\int_a^b \abs{g(x)}^q \dd{x}\right]^{\frac1q}
	}
	\leq \frac{\abs{f(x)}^p}{p \int_a^b \abs{f(x)}^p \dd{x}}
	+ \frac{\abs{g(x)}^q}{q \int_a^b \abs{g(x)}^q \dd{x}},
	\quad x\in[a,b].
\end{equation*}
对上式两边在\([a,b]\)上求积分,利用定积分的性质得\begin{equation*}
	\frac{
		\int_a^b \abs{f(x)~g(x)} \dd{x}
	}{
		\left[\int_a^b \abs{f(x)}^p \dd{x}\right]^{\frac1p}
		\left[\int_a^b \abs{g(x)}^q \dd{x}\right]^{\frac1q}
	}
	\leq \frac{\int_a^b \abs{f(x)}^p \dd{x}}{p \int_a^b \abs{f(x)}^p \dd{x}}
	+ \frac{\int_a^b \abs{g(x)}^q \dd{x}}{q \int_a^b \abs{g(x)}^q \dd{x}}
	= \frac1p + \frac1q = 1.
\end{equation*}
在上式两边同乘以\(\left[\int_a^b \abs{f(x)}^p \dd{x}\right]^{\frac1p}
\left[\int_a^b \abs{g(x)}^q \dd{x}\right]^{\frac1q}\),
便得\begin{equation*}
	\int_a^b \abs{f(x)~g(x)} \dd{x}
	\leq \left[\int_a^b \abs{f(x)}^p \dd{x}\right]^{\frac1p}
	\left[\int_a^b \abs{g(x)}^q \dd{x}\right]^{\frac1q}.
	\qedhere
\end{equation*}
%TODO 没有证明取等条件
\end{proof}
\end{theorem}

\begin{example}
%@see: 《数学分析(第二版 上册)》(陈纪修) P292 例7.2.3
设函数\(f \in D^2[a,b]\),且\(f\left(\frac{a+b}2\right) = 0\).
记\(M = \sup_{a \leq x \leq b} \abs{f''(x)}\).
证明:\begin{equation*}
	\int_a^b f(x) \dd{x}
	\leq \frac{M(b-a)^3}{24}.
\end{equation*}
\begin{proof}
记\(c=\frac{a+b}2\).
函数\(f\)在\(x=c\)的带有拉格朗日余项的泰勒公式为\begin{align*}
	f(x) &= f(c) + f'(c) (x-c) + \frac12 f''(\xi) (x-c)^2 \\
	&= f'(c) (x-c) + \frac12 f''(\xi) (x-c)^2,
\end{align*}
其中\(a \leq \xi \leq b\).
对等式两边求积分,利用\(\int_a^b (x-c) \dd{x} = 0\),得到\begin{align*}
	\int_a^b f(x) \dd{x}
	&= f'(c) \int_a^b (x-c) \dd{x} + \frac12 \int_a^b f''(\xi) (x-c)^2 \dd{x} \\
	&= \frac12 \int_a^b f''(\xi) (x-c)^2 \dd{x},
\end{align*}
于是\begin{align*}
	\abs{\int_a^b f(x) \dd{x}}
	&\leq \frac12 \int_a^b \abs{f''(\xi) (x-c)^2} \dd{x}
		\tag{\cref{theorem:定积分.定积分性质5推论2}} \\
	&\leq \frac{M}2 \int_a^b (x-c)^2 \dd{x}
		\tag{\hyperref[theorem:定积分.赫尔德不等式]{赫尔德不等式}} \\
	&= \frac{M(b-a)^3}{24}.
	\qedhere
\end{align*}
\end{proof}
\end{example}

\subsection{柯西--施瓦茨不等式}
我们首先证明被积函数\(f,g\)都是连续函数的情形.
\begin{theorem}
设函数\(f,g \in C[a,b]\),
则\begin{equation}
	\left[ \int_a^b f(x)~g(x) \dd{x} \right]^2
	\leq
	\int_a^b f^2(x) \dd{x} \int_a^b g^2(x) \dd{x}.
\end{equation}
\begin{proof}
%@see: https://www.bilibili.com/video/BV1t2x4e2Eqw/
令\(F(t) = \int_a^t f^2(x) \dd{x} \int_a^t g^2(x) \dd{x}
- \left[ \int_a^t f(x)~g(x) \dd{x} \right]^2\),
易见\(F(a) = 0\),
且\begin{align*}
	F'(t) &= f^2(t) \int_a^t g^2(x) \dd{x}
	+ g^2(t) \int_a^t f^2(x) \dd{x}
	- 2 f(t)~g(t) \int_a^t f(x)~g(x) \dd{x} \\
	&= \int_a^t (f(t)~g(x)-f(x)~g(t))^2 \dd{x}
	\geq 0
	\quad(t \geq a),
\end{align*}
于是\(F\)在\([a,b]\)上单调增加,且\begin{equation*}
	F(b) = \int_a^b f^2(x) \dd{x} \int_a^b g^2(x) \dd{x}
	- \left[ \int_a^b f(x)~g(x) \dd{x} \right]^2
	\geq F(a) = 0.
	\qedhere
\end{equation*}
\end{proof}
\end{theorem}

接下来我们把条件弱化为被积函数\(f,g\)只是黎曼可积函数.
\begin{theorem}[柯西--施瓦茨不等式]\label{theorem:定积分.柯西--施瓦茨不等式}
%@see: 《数学分析习题课讲义(第2版 上册)》(谢惠民、恽自求、易法槐、钱定边) P346 命题11.2.2
%@see: 《数学分析(第二版 上册)》(陈纪修) P294 习题 12. (1)
%@see: 《数学分析教程 (第3版 上册)》(史济怀) P248 练习题6.2 10.
设函数\(f,g \in R[a,b]\),
则\begin{equation}\label{equation:定积分.柯西--施瓦茨不等式}
	\left[ \int_a^b f(x)~g(x) \dd{x} \right]^2
	\leq
	\int_a^b f^2(x) \dd{x} \int_a^b g^2(x) \dd{x}.
\end{equation}
若函数\(f,g \in C[a,b]\),
则上式“\(=\)”当且仅当“存在不全为零的常数\(\alpha\)和\(\beta\),
使得\(\alpha f(x) = \beta g(x)\)”成立.
\begin{proof}
因为对任意实数\(t\),
总有\begin{align*}
	&\hspace{-20pt}
	\int_a^b f^2(x) \dd{x}
	+ 2t \int_a^b f(x)~g(x) \dd{x}
	+ t^2 \int_a^b g^2(x) \dd{x} \\
	&= \int_a^b [ f(x) + t g(x) ]^2 \dd{x}
	\geq \int_a^b 0 \dd{x}
	= 0
	\tag1
\end{align*}
成立.
将(1)式视作关于\(t\)的一元二次多项式,
假设二次项系数\(\int_a^b g^2(x) \dd{x} > 0\),
由于该多项式是非负的,故其判别式非正,即\begin{equation*}
	\left[ 2 \int_a^b f(x)~g(x) \dd{x} \right]^2
	- 4 \int_a^b g^2(x) \dd{x} \int_a^b f^2(x) \dd{x}
	\leq 0,
\end{equation*}移项并化简,
可得\cref{equation:定积分.柯西--施瓦茨不等式}.
如果二次项系数\(\int_a^b g^2(x) \dd{x} = 0\),
那么\(g\)在其所有连续点上恒等于零.
由于可积函数的连续点稠密,
因此可以推出积分\(\int_a^b f(x)~g(x) \dd{x} = 0\),
从而有\cref{equation:定积分.柯西--施瓦茨不等式} 成立.
\end{proof}
%\cref{theorem:不等式.柯西不等式}
\end{theorem}
\begin{remark}
\cref{theorem:定积分.柯西--施瓦茨不等式}
是\cref{theorem:定积分.赫尔德不等式} 当\(p=q=2\)时的特殊情形.
\end{remark}

\subsection{闵可夫斯基不等式}
\begin{theorem}[闵可夫斯基不等式]\label{theorem:定积分.闵可夫斯基不等式}
%@see: 《数学分析(第二版 上册)》(陈纪修) P294 习题 12. (2)
设函数\(f,g \in C[a,b]\),
则\begin{equation}\label{equation:定积分.闵可夫斯基不等式}
	\sqrt{ \int_a^b [f(x)+g(x)]^2 \dd{x} }
	\leq \sqrt{ \int_a^b [f(x)]^2 \dd{x} }
			+ \sqrt{ \int_a^b [g(x)]^2 \dd{x} }.
\end{equation}
%TODO proof
\end{theorem}

\begin{example}
%@see: 《数学分析(第二版 上册)》(陈纪修) P294 习题 13.
设函数\(f,g\)都在\([a,b]\)上连续,
且\(f(x)\geq0,
g(x)>0\).
证明:\begin{equation*}
	\lim_{n\to\infty} \left[ \int_a^b f^n(x)~g(x) \dd{x} \right]^{\frac1n}
	= \max_{a \leq x \leq b} f(x).
\end{equation*}
%TODO proof
\end{example}

\subsection{欧庇尔不等式}
\begin{theorem}[欧庇尔不等式]\label{theorem:定积分.欧庇尔不等式}
%@see: https://www.bilibili.com/video/BV18a411h7JA/
设函数\(f(x)\)在\([a,b]\)上有连续导数,\(f(a)=f(b)=0\),
则\begin{equation}\label{equation:定积分.欧庇尔不等式}
	\int_a^b \abs{f(x) f'(x)} \dd{x}
	\leq \frac{b-a}{4}
	\int_a^b [f'(x)]^2 \dd{x}.
\end{equation}
\begin{proof}
由于\(f(a)=f(b)=0\),所以\begin{itemize}
	\item 当\(x \in \left[a,c\right]\)时,
	\(\abs{f(x)} = \abs{\int_a^x f'(t) \dd{t}} \leq \int_a^x \abs{f'(t)} \dd{t} = F(x)\);
	\item 当\(x \in \left[c,b\right]\)时,
	\(\abs{f(x)} = \abs{\int_b^x f'(t) \dd{t}} \leq \int_b^x \abs{f'(t)} \dd{t} = G(x)\),
\end{itemize}
其中\(c=\frac{a+b}{2}\).
从而,\begin{align*}
	\int_a^b \abs{f(x) f'(x)} \dd{x}
	&= \int_a^c \abs{f(x) f'(x)} \dd{x}
		+ \int_c^b \abs{f(x) f'(x)} \dd{x} \\
	&\leq \int_a^c F(x) F'(x) \dd{x}
		+ \int_c^b G(x) G'(x) \dd{x} \\
	&= \frac{1}{2} \left[ F^2(c) + G^2(c) \right].
\end{align*}
分别代入\(F(x)\)与\(G(x)\)的表达式,
可得\begin{equation*}
	F^2(c) = \left( \int_a^c \abs{f'(t)} \dd{t} \right)^2,
	\qquad
	G^2(c) = \left( \int_c^b \abs{f'(t)} \dd{t} \right)^2.
\end{equation*}
根据\hyperref[equation:定积分.柯西--施瓦茨不等式]{柯西--施瓦茨不等式},
有\begin{align*}
	&\hspace{-20pt}\int_a^b{\abs{f(x) f'(x)} \dd{x}}
	\leq \frac{1}{2} \left[
		F^2(c)
		+ G^2(c)
		\right] \\
	&= \frac{1}{2} \left[
		\left(\int_a^c{\abs{f'(t)}\dd{t}}\right)^2
		+\left(\int_c^b{\abs{f'(t)}\dd{t}}\right)^2
		\right] \\
	&\leq \frac{1}{2} \left\{
		\int_a^c{1^2 \dd{x}}
		\int_a^c{[f'(x)]^2 \dd{x}}
		+\int_c^b{1^2 \dd{x}}
		\int_c^b{[f'(x)]^2 \dd{x}}
		\right\} \\
	&= \frac{b-a}{4} \int_a^b{[f'(x)]^2 \dd{x}}.
\end{align*}
\cref{equation:定积分.欧庇尔不等式} 当且仅当\begin{equation*}
	f(x) = \left\{ \begin{array}{cl}
		c(x-a), & x\in\left[a,c\right], \\
		-c(x-b), & x\in\left[c,b\right],
	\end{array} \right.
\end{equation*}时取等号,其中\(c\neq0\).
\end{proof}
%@see: https://doi.org/10.1007/978-94-011-3562-7
\end{theorem}

\subsection{贝尔曼--格朗沃尔不等式}
\begin{theorem}[贝尔曼--格朗沃尔不等式]\label{theorem:定积分.贝尔曼--格朗沃尔不等式}
%https://www.sciencedirect.com/topics/engineering/gronwall-bellman-inequality
设\(f(x),g(x),\phi(x)\)为\([a,b]\)上的连续函数,
\(g(x)\)单调递增,
\(\phi(x)\geq0\),
且\begin{equation*}
	(\forall x \in [a,b])
	\left[
		f(x) \leq g(x) + \int_a^x \phi(t) f(t) \dd{t}
	\right],
\end{equation*}
那么\begin{equation*}
	(\forall x \in [a,b])
	\left[
		f(x) \leq g(x) e^{\int_a^x \phi(s) \dd{s}}
	\right].
\end{equation*}
\begin{proof}
设\(F(x) = \int_a^x \phi(t) f(t) \dd{t}\),
则\begin{equation*}
	F'(x) = \phi(x) f(x) \leq \phi(x) g(x) + \phi(x) F(x),
\end{equation*}
移项得\(F'(x) - \phi(x) F(x) \leq \phi(x) g(x)\),于是\begin{equation*}
	\dv{x}\left[F(x) e^{-\int_a^x \phi(t) \dd{t}}\right]
	\leq \phi(x) g(x) e^{-\int_a^x \phi(t) \dd{t}},
\end{equation*}\begin{equation*}
	F(x) \leq \int_a^x \phi(t) g(t) e^{\int_t^x \phi(s) \dd{s}} \dd{t}.
\end{equation*}
由于\(g(x)\)单调递增,
根据\hyperref[theorem:定积分.积分中值定理2]{积分第二中值定理},
有\begin{align*}
	f(x) &\leq g(x) + F(x)
		\leq g(x) + \int_a^x \phi(t) g(t) e^{\int_t^x \phi(s) \dd{s}} \dd{t} \\
	&\leq g(x) + g(x) \int_a^x \phi(t) e^{\int_t^x \phi(s) \dd{s}} \dd{t} \\
	&= g(x) + g(x) \left[-e^{\int_t^ \phi(s) \dd{s}}\right]_a^x \\
	&= g(x) e^{\int_a^x \phi(s) \dd{s}}.
	\qedhere
\end{align*}
\end{proof}
\end{theorem}

\section{本章总结}
\subsection*{边缘分布,随机变量的独立性}
%\cref{theorem:多维随机变量及其分布.联合密度、边缘密度的关系}
设\(F(x,y)\)为二维随机变量\((X,Y)\)的二维分布函数,
则\(X\)与\(Y\)的边缘分布函数分别为\begin{gather*}
	F_X(x) = F(x,+\infty)
	\quad(-\infty < x < +\infty), \\
	F_Y(y) = F(+\infty,y)
	\quad(-\infty < x < +\infty).
\end{gather*}

%\cref{definition:多维随机变量及其分布.随机变量的独立性}
设\(\AutoTuple{X}{n}\)是\(n\)维随机变量.
若对任意\(n\)个实数\(\AutoTuple{x}{n}\),
\(n\)个事件\((X_1 \leq x_1),\allowbreak\dotsc,\allowbreak(X_n \leq x_n)\)相互独立,
即有\begin{equation*}
	P(X_1 \leq x_1,\dotsc,X_n \leq x_n)
	= \prod_{i=1}^n P(X_i \leq x_i)
	= P(X_1 \leq x_1) \dotsm P(X_n \leq x_n)
\end{equation*}
或\begin{equation*}
	F(x_1,\dotsc,x_n)
	= \prod_{i=1}^n F_i(x_i)
	= F_1(x_1) \dotsm F_n(x_n),
\end{equation*}
其中\(F\)是\(n\)维随机变量\(\AutoTuple{X}{n}\)的联合分布函数,
而\(F_1,\dotsc,F_n\)分别是\(X_1,\dotsc,X_n\)的边缘分布函数,
则称“\(n\)个随机变量\(\AutoTuple{X}{n}\)相互独立”;
否则称“\(n\)个随机变量\(\AutoTuple{X}{n}\)不相互独立”
或“\(n\)个随机变量\(\AutoTuple{X}{n}\)相依”.

\(n\)个事件两两独立是它们相互独立的必要不充分条件.

%\cref{theorem:多维随机变量及其分布.两个离散型随机变量相互独立的充分必要条件}
设\((X,Y)\)是二维离散型随机变量,有二维概率分布\begin{equation*}
	p_{ij} = P(X=x_i,Y=y_j), \quad i,j=1,2,\dotsc,
\end{equation*}
和边缘分布\begin{gather*}
	p_{i*} = \sum_j p_{ij},
	\quad i=1,2,\dotsc; \\
	p_{*j} = \sum_i p_{ij},
	\quad j=1,2,\dotsc,
\end{gather*}
则随机变量\(X\)与\(Y\)相互独立的充分必要条件是:\begin{equation*}
	p_{ij} = p_{i*} p_{*j}, \quad i,j=1,2,\dotsc.
\end{equation*}

%\cref{theorem:多维随机变量及其分布.两个连续型随机变量相互独立的充分必要条件}
设二维连续型随机变量\((X,Y)\)的二维密度为\(f(x,y)\),
\(X\)与\(Y\)的边缘密度分别为\(f_X(x)\)和\(f_Y(y)\),
则\begin{align*}
	f_X(x) = \int_{-\infty}^{+\infty} f(x,y) \dd{y}, \\
	f_Y(y) = \int_{-\infty}^{+\infty} f(x,y) \dd{x}.
\end{align*}
而\(X\)与\(Y\)相互独立的充分必要条件是:\begin{equation*}
	f(x,y) = f_X(x) f_Y(y).
\end{equation*}在三个密度函数的公共连续点上成立.

\subsection*{联合分布、边缘分布与条件分布的联系}
设二维连续型随机变量\((X,Y)\)有二维密度\(f(x,y)\),
从而\(X\)及\(Y\)有边缘密度\(f_X(x),f_Y(y)\),
则\begin{gather*}
	F_{X \vert Y}(x \vert y)
	= \int_{-\infty}^x \frac{f(u,y)}{f_Y(y)}\dd{u}
	\quad(-\infty < x < +\infty), \\
	F_{Y \vert X}(y \vert x)
	= \int_{-\infty}^y \frac{f(x,v)}{f_X(x)}\dd{v}
	(\quad y \in \mathbb{R}).
\end{gather*}
\(X\)关于\(Y\)的条件密度函数为\begin{equation*}
	%\cref{equation:多维随机变量及其分布.条件密度、联合密度、边缘密度的关系1}
	f_{X \vert Y}(x \vert y)
	= \frac{f(x,y)}{f_Y(y)}.
\end{equation*}
\(Y\)关于\(X\)的条件密度函数为\begin{equation*}
	%\cref{equation:多维随机变量及其分布.条件密度、联合密度、边缘密度的关系2}
	f_{Y \vert X}(y \vert x)
	= \frac{f(x,y)}{f_X(x)}.
\end{equation*}
反过来,可以利用边缘密度函数和条件密度函数计算联合密度函数:\begin{equation*}
	f(x,y) = f_Y(y) \cdot f_{X \vert Y}(x \vert y)
	= f_X(x) \cdot f_{Y \vert X}(y \vert x).
\end{equation*}

\subsection*{分布的可加性}
%\cref{theorem:多维随机变量及其分布.二项分布的可加性1}
设\(X \sim B(n,p)\),
\(Y \sim B(m,p)\),
且\(X\)与\(Y\)相互独立,
则\begin{equation*}
	X+Y \sim B(n+m,p).
\end{equation*}

%\cref{theorem:多维随机变量及其分布.泊松分布的可加性1}
设\(X \sim P(\lambda_1)\),
\(Y \sim P(\lambda_2)\),
且\(X\)与\(Y\)相互独立,
则\begin{equation*}
	X+Y \sim P(\lambda_1 + \lambda_2).
\end{equation*}

%\cref{theorem:正态分布与自然指数分布族.正态分布的可加性1}
设\(X \sim N(\mu_1,\sigma_1^2)\),
\(Y \sim N(\mu_2,\sigma_2^2)\),
且\(X\)与\(Y\)相互独立,
则\begin{equation*}
	X+Y \sim N(\mu_1+\mu_2,\sigma_1^2+\sigma_2^2).
\end{equation*}

%\cref{theorem:正态分布与自然指数分布族.正态分布的可加性2}
设随机变量\(\AutoTuple{X}{n}\)相互独立,
且\begin{equation*}
	X_i \sim N(\mu_i,\sigma_i^2),
	\quad i=1,2,\dotsc,n,
\end{equation*}
且\(C_1,C_2,\dotsc,C_n\)为常数,
则\begin{equation*}
	\sum_{i=1}^n {C_i X_i}
	\sim N\left(
	\sum_{i=1}^n {C_i \mu_i},
	\sum_{i=1}^n {C_i^2 \sigma_i^2}
	\right).
\end{equation*}


\chapter{反常积分}
有一些实际问题中,常会遇到积分区间为无穷区间\((-\infty,+\infty)\),
或者被积函数为无界函数的积分,它们已经不属于前面所说的定积分了.
因此,我们对定积分作如下两种推广,从而形成反常积分(或广义积分)的概念.
相对地,原本的定积分又称作常义积分.

\section{无穷限的反常积分}
“反常积分(improper integral)”
可以分为“无穷限的反常积分”和“无界函数的反常积分”两大类,
本节讨论第一类反常积分,
即积分区间无界的积分.

\subsection{无穷限的反常积分的概念}
%@see: 《数学分析教程(第3版 上册)》(史济怀) P278
\begin{definition}\label{definition:定积分.无穷限的反常积分的定义1}
%@see: 《数学分析(第二版 上册)》(陈纪修) P358 定义8.1.1
设函数\(f\colon[a,+\infty)\to\mathbb{R}\)在任意有限闭区间上黎曼可积,
即对\(\forall b\in(a,+\infty)\)都有\(f \in R[a,b]\).

如果极限\begin{equation*}
	\lim_{t \to +\infty} \int_a^t f(x) \dd{x}
\end{equation*}存在且有限,
那么称“反常积分\(\int_a^{+\infty} f(x) \dd{x}\)~\DefineConcept{收敛}”
或“函数\(f\)在\([a,+\infty)\)上\DefineConcept{可积}”,
记作\(f \in R[a,+\infty)\);
并把这个极限称为
“函数\(f\)在\([a,+\infty)\)上的\DefineConcept{反常积分}”,
记作\begin{equation*}
%@see: 《数学分析教程(第3版 上册)》(史济怀) P278 (1)
	\int_a^{+\infty} f(x) \dd{x},
\end{equation*}
即\begin{equation*}
	\int_a^{+\infty} f(x) \dd{x}
	\defeq
	\lim_{t \to +\infty} \int_a^t f(x) \dd{x}.
\end{equation*}

如果极限\begin{equation*}
	\lim_{t \to +\infty} \int_a^t f(x) \dd{x}
\end{equation*}不存在,
则称“反常积分\(\int_a^{+\infty} f(x) \dd{x}\)~\DefineConcept{发散}”.
\end{definition}

类似地,
设\(f\colon(-\infty,b]\to\mathbb{R}\)
对\(\forall a\in(-\infty,b)\)都有\(f \in R[a,b]\),
则可以定义反常积分\begin{equation*}
	\int_{-\infty}^b f(x) \dd{x}
	\defeq
	\lim_{t \to -\infty} \int_t^b f(x) \dd{x}.
\end{equation*}

\begin{definition}\label{definition:定积分.无穷限的反常积分的定义3}
设函数\(f\colon\mathbb{R}\to\mathbb{R}\)
在\((-\infty,+\infty)\)内的任意有界闭区间上都是可积的,
即\begin{equation*}
	(\forall a,b\in\mathbb{R})
	[
		[a,b]\subset(-\infty,+\infty)
		\implies
		f \in R[a,b]
	].
\end{equation*}
任取\(a\in\mathbb{R}\),
如果反常积分\begin{equation*}
%@see: 《数学分析教程(第3版 上册)》(史济怀) P279 (2)
	\int_{-\infty}^a f(x) \dd{x}
	\quad\text{和}\quad
	\int_a^{+\infty} f(x) \dd{x}
\end{equation*}都收敛,
那么称“反常积分\(\int_{-\infty}^{+\infty} f(x) \dd{x}\)~\DefineConcept{收敛}”,
并且规定\begin{equation*}
%@see: 《数学分析教程(第3版 上册)》(史济怀) P279 (3)
	\int_{-\infty}^{+\infty} f(x) \dd{x}
	\defeq
	\int_{-\infty}^a f(x) \dd{x} + \int_a^{+\infty} f(x) \dd{x},
\end{equation*}
将其称作“函数\(f(x)\)在无穷区间\((-\infty,\infty)\)上的\DefineConcept{反常积分}”.

反之,如果反常积分\begin{equation*}
	\int_{-\infty}^a f(x) \dd{x}, \qquad
	\int_a^{+\infty} f(x) \dd{x}
\end{equation*}中至少有一个发散,
那么称“反常积分\(\int_{-\infty}^{+\infty} f(x) \dd{x}\)~\DefineConcept{发散}”.
\end{definition}

容易证明,\cref{definition:定积分.无穷限的反常积分的定义3} 实际上与点\(a\)的选取毫无关系.

\cref{definition:定积分.无穷限的反常积分的定义1,%
definition:定积分.无穷限的反常积分的定义3}
定义的反常积分统称为\DefineConcept{无穷限的反常积分}%
或\DefineConcept{第一类反常积分}.

\subsection{无穷限的反常积分的计算法}
由上述定义及牛顿--莱布尼茨公式,可得如下结果.
\begin{theorem}\label{theorem:定积分.利用牛顿莱布尼茨公式计算无穷限的反常积分}
%@see: 《数学分析教程(第3版 上册)》(史济怀) P280 定理6.7.1
如果函数\(f\)在\([a,+\infty)\)上可积,
且函数\(F\)为\(f\)在区间\([a,+\infty)\)上的一个原函数,
那么\begin{equation}\label{equation:定积分.利用牛顿莱布尼茨公式计算无穷限的反常积分1}
	\int_a^{+\infty} f(x) \dd{x}
	= F(+\infty) - F(a).
\end{equation}
%%其中\(F(+\infty) \defeq \lim_{x\to+\infty} F(x)\).
%FIX: 删去“其中...”,原因是我们已经在[函数的极限](definition:极限.函数极限的定义5)处定义过了.

如果函数\(f\)在\((-\infty,a]\)上可积,
且函数\(F\)为\(f\)在区间\((-\infty,a]\)上的一个原函数,
那么\begin{equation}\label{equation:定积分.利用牛顿莱布尼茨公式计算无穷限的反常积分2}
	\int_{-\infty}^a f(x) \dd{x}
	= F(a) - F(-\infty).
\end{equation}
%%其中\(F(-\infty) \defeq \lim_{x\to-\infty} F(x)\).
%FIX: 删去“其中...”,原因是我们已经在[函数的极限](definition:极限.函数极限的定义5)处定义过了.

如果函数\(f\)在\((-\infty,-\infty)\)上可积,
且函数\(F\)为\(f\)在区间\((-\infty,-\infty)\)上的一个原函数,
\begin{equation}\label{equation:定积分.利用牛顿莱布尼茨公式计算无穷限的反常积分3}
	\int_{-\infty}^{+\infty} f(x) \dd{x}
	= F(+\infty) - F(-\infty).
\end{equation}
\begin{proof}
先证\cref{equation:定积分.利用牛顿莱布尼茨公式计算无穷限的反常积分1},
根据定义,\begin{align*}
	\int_a^{+\infty} f(x) \dd{x}
	&= \lim_{b\to+\infty} \int_a^b f(x) \dd{x}
	= \lim_{b\to+\infty} (F(b) - F(a)) \\
	&= \lim_{b\to+\infty} F(b) - F(a)
	= F(+\infty) - F(a).
\end{align*}

同理可证\cref{equation:定积分.利用牛顿莱布尼茨公式计算无穷限的反常积分2,equation:定积分.利用牛顿莱布尼茨公式计算无穷限的反常积分3}.
\end{proof}
\end{theorem}

\begin{example}
%@see: 《高等数学(第六版 上册)》 P256 例1
%@see: 《数学分析(第二版 上册)》(陈纪修) P360 例8.1.4
计算反常积分\(\int_{-\infty}^{+\infty} \frac{\dd{x}}{1+x^2}\).
\begin{solution}
因为\(\arctan x\)是\(\frac{1}{1+x^2}\)的一个原函数,
所以\begin{align*}
	\int_{-\infty}^{+\infty} \frac{\dd{x}}{1+x^2}
	&= \eval{\arctan x}_{-\infty}^{+\infty} \\
	&= \lim_{x\to+\infty} \arctan x
		-\lim_{x\to-\infty} \arctan x \\
	&= \frac{\pi}{2} - \left(-\frac{\pi}{2}\right)
	= \pi.
\end{align*}
\end{solution}
\end{example}

\begin{example}\label{example:反常积分.重要反常积分公式1}
%@see: 《数学分析(第二版 上册)》(陈纪修) P360 例8.1.3
设常数\(a\in\mathbb{R}\),
讨论\(\int_0^{+\infty} e^{-ax} \dd{x}\)的敛散性.
\begin{solution}
当\(a=0\)时,有\(e^{-ax} = 1\),那么\begin{equation*}
	\int_0^t e^{-ax} \dd{x} = t,
	\qquad
	\int_0^{+\infty} e^{-ax} \dd{x}
	= \lim_{t\to+\infty} \int_0^t e^{-ax} \dd{x}
	= +\infty.
\end{equation*}

当\(a\neq0\)时,
\(x \mapsto -\frac1a e^{-ax}\)是\(x \mapsto e^{-ax}\)的一个原函数,
那么有\begin{equation*}
	\int_0^{+\infty} e^{-ax} \dd{x}
	= -\frac1a \eval{e^{-ax}}_0^{+\infty}
	= \left\{ \begin{array}{rl}
		1/a, & a>0, \\
		+\infty, & a < 0.
	\end{array} \right.
\end{equation*}

综上所述,当\(a>0\)时,反常积分\(\int_0^{+\infty} e^{-ax} \dd{x}\)收敛于\(1/a\);
当\(a\leq0\)时,它发散.
\end{solution}
\end{example}

\begin{example}
%@see: 《高等数学(第六版 上册)》 P256 例2
计算反常积分\(\int_0^{+\infty} t e^{-pt} \dd{t}\),其中常数\(p>0\).
\begin{solution}
直接计算得\begin{equation*}
	\int_0^{+\infty} t e^{-pt} \dd{t}
	= -\frac1p \int_0^{+\infty} t \dd(e^{-pt}) % 分部积分法
	= -\frac1p \left[
		\eval{t e^{-pt}}_0^{+\infty}
		- \int_0^{+\infty} e^{-pt} \dd{t}
	\right],
\end{equation*}
其中\(\lim_{t\to0} t e^{-pt} = 0\),% 无穷小与有界量的乘积还是无穷小
\(\lim_{t\to+\infty} t e^{-pt}
= \lim_{t\to+\infty} \frac1{p e^{pt}} % 洛必达法则
= 0\),
从而有\(\eval{t e^{-pt}}_0^{+\infty} = 0\),
那么\begin{equation*}
	\int_0^{+\infty} t e^{-pt} \dd{t}
	= \frac1p \int_0^{+\infty} e^{-pt} \dd{t}.
\end{equation*}
又由\cref{example:反常积分.重要反常积分公式1}
可知\(\int_0^{+\infty} e^{-pt} \dd{t} = \frac1p\),
所以\begin{equation*}
	\int_0^{+\infty} t e^{-pt} \dd{t}
	= \frac1{p^2}.
\end{equation*}
\end{solution}
\end{example}
\begin{remark}
从这个例子中我们可以归纳出一个递推公式:\begin{align*}
	\int_0^{+\infty} t^n e^{-pt} \dd{t}
	&= -\frac1p \int_0^{+\infty} t^n \dd(e^{-pt}) \\
	&= -\frac1p \left[\eval{(t^n e^{-pt})}_0^{+\infty} - \int_0^{+\infty} e^{-pt} \dd(t^n)\right] \\
	&= \frac{n}{p} \int_0^{+\infty} t^{n-1} e^{-pt} \dd{t},
\end{align*}
由此可得\begin{align*}
	\int_0^{+\infty} t^n e^{-pt} \dd{t}
	&= \frac{n}{p} \int_0^{+\infty} t^{n-1} e^{-pt} \dd{t}
	= \frac{n(n-1)}{p^2} \int_0^{+\infty} t^{n-2} e^{-pt} \dd{t} \\
	&= \dotsb = \frac{n!}{p^n} \int_0^{+\infty} e^{-pt} \dd{t}
	= \frac{n!}{p^{n+1}}.
\end{align*}
%@Mathematica: Integrate[t^n Exp[-p t], {t, 0, +Infinity}, Assumptions -> {p > 0, n >= 0}]
\end{remark}

\begin{example}\label{example:反常积分.指数函数与三角函数之积的反常积分}
%@see: 《数学分析教程(第3版 上册)》(史济怀) P280 例2
设\(a>0\).
计算反常积分\begin{equation*}
	\int_0^{+\infty} e^{-ax} \cos bx \dd{x}
	\quad\text{和}\quad
	\int_0^{+\infty} e^{-ax} \sin bx \dd{x}.
\end{equation*}
\begin{solution}
用分部积分法,得\begin{align*}
	\int_0^{+\infty} e^{-ax} \cos bx \dd{x}
	&= \frac1b \int_0^{+\infty} e^{-ax} \dd(\sin bx) \\
	&= \frac1b \left(
		\eval{e^{-ax} \sin bx}_0^{+\infty}
		+ a \int_0^{+\infty} e^{-ax} \sin bx \dd{x}
	\right).
\end{align*}
因为\(\lim_{x\to+\infty} e^{-ax} \sin bx = 0\),
所以\begin{equation*}
	\int_0^{+\infty} e^{-ax} \cos bx \dd{x}
	= \frac{a}{b} \int_0^{+\infty} e^{-ax} \sin bx \dd{x}.
	\eqno(1)
\end{equation*}
再用分部积分法,得\begin{align*}
	\int_0^{+\infty} e^{-ax} \sin bx \dd{x}
	&= -\frac1b \int_0^{+\infty} e^{-ax} \dd(\cos bx) \\
	&= -\frac1b \left(
		\eval{e^{-ax} \cos bx}_0^{+\infty}
		+ a \int_0^{+\infty} e^{-ax} \cos bx \dd{x}
	\right).
\end{align*}
因为\(\lim_{x\to+\infty} e^{-ax} \cos bx = 0\),
所以\begin{equation*}
	\int_0^{+\infty} e^{-ax} \sin bx \dd{x}
	= -\frac1b \left(
		-1
		+ a \int_0^{+\infty} e^{-ax} \cos bx \dd{x}
	\right).
	\eqno(2)
\end{equation*}
联立(1)(2)两式可以解得\begin{gather}
	\int_0^{+\infty} e^{-ax} \cos bx \dd{x} = \frac{a}{a^2+b^2}, \\
	\int_0^{+\infty} e^{-ax} \sin bx \dd{x} = \frac{b}{a^2+b^2}.
\end{gather}
\end{solution}
\end{example}

\begin{example}
%@see: 《数学分析教程(第3版 上册)》(史济怀) P281 例3
计算\(\int_0^{+\infty} \frac{\dd{x}}{(a^2+x^2)^{3/2}}\).
\begin{solution}
不妨设\(a>0\).
令\(x = a \tan t\),
则\(\dd{x} = a \sec^2t \dd{t}\).
于是\begin{equation*}
	\int_0^{+\infty} \frac{\dd{x}}{(a^2+x^2)^{3/2}}
	= \int_0^{\pi/2} \frac{a \sec^2t \dd{t}}{a^3 \sec^3t}
	= \frac1{a^2} \int_0^{\pi/2} \cos t \dd{t}
	= \frac1{a^2}.
\end{equation*}
\end{solution}
\end{example}
\begin{remark}
我们看到,一个反常积分经过换元之后变成了常义积分.
反过来,一个常义积分经过换元以后也可变为反常积分.
这种现象是经常发生的,不足为怪.
\end{remark}

\begin{example}
%@see: 《数学分析(第二版 上册)》(陈纪修) P364 例8.1.10
计算反常积分\(I_n = \int_0^{+\infty} \frac{\dd{x}}{(x^2+a^2)^n}\).
\begin{solution}
当\(n\geq2\)时,有\begin{align*}
	I_n &= \int_0^{+\infty} \frac{\dd{x}}{(x^2+a^2)^n} \\
	&= \frac1{a^2} \int_0^{+\infty} \frac{x^2+a^2-x^2}{(x^2+a^2)^n} \dd{x} \\
	&= \frac1{a^2} \int_0^{+\infty} \left[
		\frac1{(x^2+a^2)^{n-1}}
		+ \frac{-x^2}{(x^2+a^2)^n}
	\right] \dd{x} \\
	&= \frac1{a^2} I_{n-1}
		+ \frac1{a^2} \int_0^{+\infty} \frac{-x^2}{(x^2+a^2)^n} \dd{x},
\end{align*}
其中\begin{align*}
	\int_0^{+\infty} \frac{-x^2}{(x^2+a^2)^n} \dd{x}
	&= \frac1{2(n-1)} \int_0^{+\infty} x \dd{\left[\frac1{(x^2+a^2)^{n-1}}\right]} \\
	&= \frac1{2(n-1)} \left[
		\eval{\frac{x}{(x^2+a^2)^{n-1}}}_0^{+\infty}
		- \int_0^{+\infty} \frac{\dd{x}}{(x^2+a^2)^{n-1}}
	\right] \\
	&= -\frac1{2(n-1)} I_{n-1},
\end{align*}
%@Mathematica: Limit[x/(x^2 + a^2)^(n - 1), x -> Infinity, Assumptions -> {n >= 2, n \[Element] Integers}]
%@Mathematica: Limit[x/(x^2 + a^2)^(n - 1), x -> 0, Assumptions -> {n >= 2, n \[Element] Integers}]
从而有\begin{equation*}
	I_n = \frac1{a^2} I_{n-1} - \frac1{a^2} \cdot \frac1{2(n-1)} I_{n-1},
\end{equation*}
整理得\begin{equation*}
	I_n = \frac1{a^2} \cdot \frac{3-2n}{2-2n} \cdot I_{n-1}.
\end{equation*}
又因为\begin{equation*}
	I_1 = \int_0^{+\infty} \frac{\dd{x}}{x^2+a^2}
	= \eval{\frac1a \arctan\frac{x}{a}}_0^{+\infty}
	= \frac\pi{2a},
\end{equation*}
所以\begin{equation*}
	I_n = \frac\pi{2a^{2n-1}} \cdot \frac{(2n-3)!!}{(2n-2)!!}.
\end{equation*}
\end{solution}
\end{example}

\begin{proposition}[\(p\)积分]\label{example:定积分.p积分}
%@see: 《高等数学(第六版 上册)》 P257 例3
%@see: 《数学分析(第二版 上册)》(陈纪修) P359 例8.1.2
%@see: 《数学分析教程(第3版 上册)》(史济怀) P279 例1
反常积分\begin{equation*}
	\int_a^{+\infty} \frac{\dd{x}}{x^p}
	\quad(a>0)
\end{equation*}当\(p>1\)时收敛,当\(p\leq1\)时发散.
\begin{proof}
当\(p=1\)时,有\begin{equation*}
	\int_a^t \frac{\dd{x}}{x^p}
	= \int_a^t \frac{\dd{x}}{x}
	= \eval{\ln x}_a^t
	= \ln t - \ln a,
\end{equation*}
于是\begin{equation*}
	\int_a^{+\infty} \frac{\dd{x}}{x^p}
	= \lim_{t\to+\infty} \int_a^t \frac{\dd{x}}{x^p}
	= \lim_{t\to+\infty} (\ln t - \ln a)
	= +\infty.
\end{equation*}

当\(p\neq1\)时,有\begin{equation*}
	\int_a^t \frac{\dd{x}}{x^p}
	= \frac{1}{1-p} \int_a^t \dd(x^{1-p})
	= \frac{1}{1-p} \eval{x^{1-p}}_a^t
	= \frac{t^{1-p}-a^{1-p}}{1-p},
\end{equation*}
于是\begin{equation*}
	\int_a^{+\infty} \frac{\dd{x}}{x^p}
	= \lim_{t\to+\infty} \int_a^t \frac{\dd{x}}{x^p}
	= \left\{ \begin{array}{cl}
		\frac{a^{1-p}}{p-1}, & p>1, \\
		+\infty, & p<1.
	\end{array} \right.
\end{equation*}

综上所述,反常积分\begin{equation*}
	\int_a^{+\infty} \frac{\dd{x}}{x^p}
	\quad(a>0)
\end{equation*}当\(p>1\)时收敛于\(\frac{a^{1-p}}{p-1}\),当\(p\leq1\)时发散.
\end{proof}
\end{proposition}

\begin{example}
证明:反常积分\begin{equation*}
	\int_2^{+\infty} \frac{1}{x \ln^p x} \dd{x}
\end{equation*}
当\(p>1\)时收敛,当\(p\leq1\)时发散.
\begin{proof}
应用换元法,令\(t = \ln x\),
则\(t \to \ln2\ (x\to2)\),\(t \to +\infty\ (x\to+\infty)\),
\begin{equation*}
	\int_2^{+\infty} \frac{1}{x \ln^p x} \dd{x}
	= \int_{\ln2}^{+\infty} \frac{\dd{t}}{t^p};
\end{equation*}
与~\hyperref[example:定积分.p积分]{\(p\)积分}作比较即知,
反常积分\(\int_2^{+\infty} \frac{1}{x \ln^p x} \dd{x}\)%
当\(p>1\)时收敛,当\(p\leq1\)时发散.
\end{proof}
\end{example}

\begin{example}
设\(f\colon[0,+\infty)\to\mathbb{R}\)在任意有限闭区间上黎曼可积,
且\(\lim_{x\to+\infty} f(x) = A\).
证明:\begin{equation*}
	\lim_{x\to+\infty} \frac{1}{x} \int_0^x f(t) \dd{t} = A.
\end{equation*}
\begin{proof}
因为\(\frac{1}{x} \int_0^x A \dd{t} = A\),
所以,要证\(\lim_{x\to+\infty} \frac{1}{x} \int_0^x f(t) \dd{t} = A\),
只需证\begin{equation*}
	\lim_{x\to+\infty} \frac{1}{x} \int_0^x [f(t) - A] \dd{t} = 0,
\end{equation*}
即证对\(\forall\epsilon>0,
\exists X>0\),
当\(x>X\)时,成立\begin{equation*}
	\abs{\frac{1}{x} \int_0^x [f(t) - A] \dd{t} - 0} < \epsilon.
	\eqno(1)
\end{equation*}

当\(x>0\)时,
由\cref{theorem:定积分.定积分性质5推论2} 有,
\begin{align*}
	\abs{\frac{1}{x} \int_0^x [f(t) - A] \dd{t}}
	&\leq
		\frac{1}{x} \int_0^x \abs{f(t) - A} \dd{t} \\
	&=
		\frac{1}{x} \int_0^X \abs{f(t) - A} \dd{t}
		+ \frac{1}{x} \int_X^x \abs{f(t) - A} \dd{t}.
	\tag2
\end{align*}
因为\(\int_0^X \abs{f(t) - A} \dd{t}\)是一个与\(x\)无关的常数,
所以\(\lim_{x\to+\infty} \frac{1}{x} \int_0^X \abs{f(t) - A} \dd{t} = 0\),
也就是说,对于\(\forall\epsilon>0,
\exists Y>0\),
当\(x>Y\)时,成立\begin{equation*}
	\abs{\frac{1}{x} \int_0^X \abs{f(t) - A} \dd{t} - 0}
	= \frac{1}{x} \int_0^X \abs{f(t) - A} \dd{t}
	< \frac\epsilon2.
	\eqno(3)
\end{equation*}
由\(\lim_{x\to+\infty} f(x) = A\),故对\(\forall\epsilon>0,
\exists Z>0\),
当\(x>Z\)时,成立\(\abs{f(x) - A} < \frac\epsilon2\),
于是\begin{equation*}
	\frac{1}{x} \int_X^x \abs{f(t) - A} \dd{t}
	< \frac{1}{x} \cdot (x-X) \cdot \frac\epsilon2.
	\eqno(4)
\end{equation*}
因此,由(2)(3)(4)式可知,
对于\(\forall \epsilon>0\),
当\(x > \max\{Y,Z\}\)时,成立\begin{equation*}
	\abs{\frac{1}{x} \int_0^x [f(t) - A] \dd{t}}
	< \frac\epsilon2 + \frac{1}{x} \cdot (x-X) \cdot \frac\epsilon2
	= \frac\epsilon2 \left( 2 - \frac{X}{x} \right)
	< \epsilon.
\end{equation*}
于是,只要取\(X = \max\{Y,Z\}\),
就有(1)式成立.
\end{proof}
%@see: 《数学分析中的典型问题与方法(第3版)》(裴礼文) P39 例1.3.20
\end{example}

\begin{example}
%@see: https://www.bilibili.com/video/BV1VG4y1S7Ym/
计算极限\(\lim_{n\to\infty} \frac1n \left( \floor*{\frac{2n}{k}} - 2 \floor*{\frac{n}{k} \vphantom{\frac{2n}{k}}} \right)\).
\begin{solution}\let\qed\relax
\begin{proof}[解法一]
由积分的定义有\begin{equation*}
	I = \lim_{n\to\infty} \frac1n \left( \floor*{\frac{2n}{k}} - 2 \floor*{\frac{n}{k} \vphantom{\frac{2n}{k}}} \right)
	= \int_0^1 \left( \floor*{\frac2x} - 2 \floor*{\frac1x} \right) \dd{x}.
\end{equation*}
令\(t = 1/x\),
得\(\dd{x} = -\frac{\dd{t}}{t^2}\).
于是\begin{equation*}
	I = \int_1^{+\infty} \frac{\floor{2t} - 2\floor{t}}{t^2} \dd{t}.
\end{equation*}
应该注意到\begin{equation*}
	\int_1^{+\infty} \frac{\floor{2t} - 2\floor{t}}{t^2} \dd{t}
	\neq
	\int_1^{+\infty} \frac{\floor{2t}}{t^2} \dd{t}
	- \int_1^{+\infty} \frac{2\floor{t}}{t^2} \dd{t},
\end{equation*}
这是因为不等号右边的两个反常积分都发散.

对于任意正整数\(n\),
当\(n \leq t < n+1\)时,
有\(2n \leq 2t < 2(n+1) = 2n+2\),
从而有\(\floor{t} = n\),
但是不能确定\(\floor{2t}\)的取值,
这是因为当\(2n \leq 2t < 2n+1\)时\(\floor{2t} = 2n\),
而当\(2n+1 \leq 2t < 2n+2\)时\(\floor{2t} = 2n+1\),
所以需要进一步讨论.
由于\begin{gather*}
	2n \leq 2t < 2n+1
	\iff
	n \leq t < n+\frac12, \\
	2n+1 \leq 2t < 2n+2
	\iff
	n+\frac12 \leq t < n+1,
\end{gather*}
所以\begin{equation*}
	\floor{2t} - 2\floor{t}
	= \left\{ \begin{array}{cl}
		0, & \floor{t} \leq t < \floor{t}+1/2, \\
		1, & \floor{t}+1/2 \leq t < \floor{t}+1.
	\end{array} \right.
\end{equation*}
于是\begin{align*}
	I &= \sum_{n=1}^\infty \int_n^{n+1} \frac{\floor{2t}-2\floor{t}}{t^2} \dd{t} \\
	&= \sum_{n=1}^\infty \left(
		\int_n^{n+\frac12} \frac{0}{t^2} \dd{t}
		+ \int_{n+\frac12}^{n+1} \frac{1}{t^2} \dd{t}
	\right) \\
	&= \sum_{n=1}^\infty \int_{n+\frac12}^{n+1} \dd(-\frac1t) \\
	&= \sum_{n=1}^\infty \eval{\left( -\frac1t \right)}_{n+\frac12}^{n+1} \\
	&= \sum_{n=1}^\infty \left( \frac2{2n+1} - \frac2{2n+2} \right) \\
	&= 2 \left( \frac13 - \frac14 + \frac15 - \frac16 + \dotsb \right) \\
	%\cref{equation:无穷级数.幂级数展开式4}
	&= 2 \left( \ln2 - \frac12 \right)
	= 2\ln2 - 1.
\end{align*}
\end{proof}
\begin{proof}[解法二]
直接计算得\begin{align*}
	I &= \lim_{n\to\infty} \frac1n \left( \floor*{\frac{2n}{k}} - 2 \floor*{\frac{n}{k} \vphantom{\frac{2n}{k}}} \right)
	= \int_0^1 \left( \floor*{\frac2x} - 2 \floor*{\frac1x} \right) \dd{x} \\
	&= \int_0^1 \left( \floor*{\frac2x} - \frac2x + \frac2x - 2 \floor*{\frac1x} \right) \dd{x} \\
	&= \int_0^1 \left( \floor*{\frac2x} - \frac2x \right) \dd{x}
	+ 2 \int_0^1 \left( \frac1x - \floor*{\frac1x} \right) \dd{x} \\
	&\xlongequal{2/x=1/t}
	2 \int_0^1 \left( \frac1x - \floor*{\frac1x} \right) \dd{x}
	- 2 \int_0^{\frac12} \left( \frac1x - \floor*{\frac1x} \right) \dd{x} \\
	&= 2 \int_{\frac12}^1 \left( \frac1x - \floor*{\frac1x} \right) \dd{x} \\
	&= 2 \int_{\frac12}^1 \left( \frac1x - 1 \right) \dd{x} \\
	&= 2 \eval{\ln x}_{\frac12}^1 - 1
	= 2\ln2 - 1.
\end{align*}
\end{proof}
\end{solution}
\end{example}

\section{无界函数的反常积分}
本节我们讨论第二类反常积分 --- “无界函数的反常积分”.

%@see: 《数学分析教程(第3版 上册)》(史济怀) P281
先看一个具体的例子.
表达式\begin{equation*}
	\int_0^1 \frac{\dd{x}}{\sqrt{x}}
\end{equation*}在黎曼积分的意义下是没有意义的,
因为被积函数\(\frac{1}{\sqrt{x}}\)在\(0\)的右邻域无界,
于是我们把点\(x=0\)称为这个积分的“瑕点”.
但是,对于\(\forall\epsilon\in(0,1)\),积分\begin{equation*}
	\int_\epsilon^1 \frac{\dd{x}}{\sqrt{x}}
\end{equation*}是有意义的,
它是一个带有变动下限\(\epsilon\)的积分.
由于极限\begin{equation*}
	\lim_{\epsilon\to0^+} \int_\epsilon^1 \frac{\dd{x}}{\sqrt{x}}
	= \lim_{\epsilon\to0^+} 2\sqrt{x}\eval_\epsilon^1
	= 2 \lim_{\epsilon\to0^+} (1-\sqrt\epsilon)
	= 2
\end{equation*}存在且有限,
我们就定义\begin{equation*}
	\int_0^1 \frac{\dd{x}}{\sqrt{x}}=2.
\end{equation*}
接下来,我们把这个定义推广.

\subsection{无界函数的反常积分的概念}
\begin{definition}\label{definition:定积分.无界函数的反常积分的定义1}
%@see: 《数学分析(第二版 上册)》(陈纪修) P361 定义8.1.2
设函数\(f\colon(a,b]\to\mathbb{R}\).
点\(a\)是\(f\)的\DefineConcept{瑕点},
即\begin{equation*}
	\lim_{x \to a^+} f(x) = \infty.
\end{equation*}
设对于\(\forall\epsilon\in(0,b-a)\),
函数\(f\)在\([a+\epsilon,b]\)上可积.
如果极限\begin{equation*}
	\lim_{\epsilon\to0^+} \int_{a+\epsilon}^b f(x) \dd{x}
\end{equation*}存在且有限,
那么称“反常积分\(\int_a^b f(x) \dd{x}\)~\DefineConcept{收敛}”,
并把这个极限称为
“函数\(f\)在\((a,b]\)上的\DefineConcept{反常积分}”,
记作\begin{equation*}
%@see: 《数学分析教程(第3版 上册)》(史济怀) P282 (5)
	\int_a^b f(x) \dd{x},
\end{equation*}
即\begin{equation*}
	\int_a^b f(x) \dd{x}
	\defeq
	\lim_{\epsilon\to0^+} \int_{a+\epsilon}^b f(x) \dd{x}.
\end{equation*}

如果极限\begin{equation*}
	\lim_{\epsilon\to0^+} \int_{a+\epsilon}^b f(x) \dd{x}
\end{equation*}不存在,
则称“反常积分\(\int_a^b f(x) \dd{x}\)~\DefineConcept{发散}”.
\end{definition}

类似地,当\(b\)是瑕点时,
我们也可以定义反常积分\begin{equation*}
	\int_a^b f(x) \dd{x}
	\defeq
	\lim_{\epsilon\to0^+} \int_a^{b-\epsilon} f(x) \dd{x}.
\end{equation*}
当\(a\)和\(b\)都是瑕点时,
那么任取一点\(c\in(a,b)\),我们定义\begin{align*}
	\int_a^b f(x) \dd{x}
	&\defeq
	\int_a^c f(x) \dd{x}
	+ \int_c^b f(x) \dd{x} \\
	&= \lim_{\epsilon\to0^+} \int_{a+\epsilon}^c f(x) \dd{x}
	+ \lim_{\epsilon\to0^+} \int_c^{b-\epsilon} f(x) \dd{x}.
\end{align*}
当开区间\((a,b)\)内一点\(d\)是瑕点时,
我们定义\begin{align*}
	\int_a^b f(x) \dd{x}
	&\defeq \int_a^d f(x) \dd{x}
		+ \int_d^b f(x) \dd{x} \\
	&= \lim_{\epsilon\to0^+} \int_a^{d-\epsilon} f(x) \dd{x}
		+ \lim_{\epsilon\to0^+} \int_{d+\epsilon}^b f(x) \dd{x}.
\end{align*}

%@see: 《数学分析(第二版 上册)》(陈纪修) P362
容易看出,无穷限的反常积分与无界函数的反常积分是可以互相转换的.
假设\(a>0\),那么\begin{align*}
	\int_a^{+\infty} f(x) \dd{x}
	&\xlongequal{x=1/t}
	-\int_{1/a}^0 \frac1{t^2} f\left(\frac1t\right) \dd{t} \\
	&\xlongequal{g(t) \defeq \frac1{t^2} f\left(\frac1t\right)}
	\int_0^{1/a} g(t) \dd{t},
\end{align*}
这里我们就把无穷限的反常积分化成了无界函数的反常积分.

\subsection{无界函数的反常积分的计算法}
计算无界函数的反常积分,也可借助于牛顿--莱布尼茨公式.
\begin{theorem}\label{theorem:定积分.利用牛顿莱布尼茨公式计算无界函数的反常积分1}
设函数\(f \in C(a,b]\),点\(a\)是\(f\)的瑕点,
函数\(F\)是\(f\)在区间\((a,b]\)上的一个原函数.
若极限\(F(a^+) = \lim_{x \to a^+} F(x)\)存在,
则有
\begin{equation}\label{equation:定积分.利用牛顿莱布尼茨公式计算无界函数的反常积分1'}
	\int_a^b f(x) \dd{x}
	= F(b) - F(a^+);
\end{equation}
若极限\(F(a^+)\)不存在,则反常积分\(\int_a^b f(x) \dd{x}\)发散.
\end{theorem}
我们仍用记号\([F(x)]_a^b\)来表示\(F(b) - F(a^+)\),
这样就可以简化\cref{equation:定积分.利用牛顿莱布尼茨公式计算无界函数的反常积分1'},
从而形式上仍有
\begin{equation}\label{equation:定积分.利用牛顿莱布尼茨公式计算无界函数的反常积分1}
	\int_a^b f(x) \dd{x} = [F(x)]_a^b.
\end{equation}

对于\(f\)在\([a,b)\)上连续、\(b\)为瑕点的反常积分,也有类似的计算公式.
这里不再详述.

\begin{theorem}\label{theorem:定积分.利用牛顿莱布尼茨公式计算无界函数的反常积分2}
设函数\(f \in C[a,b]\),点\(c\in(a,b)\)是\(f\)的瑕点,
函数\(F\)是\(f\)在区间\((a,b]\)上的一个原函数.
若极限\(F(c^+),F(c^-)\)都存在,
则有
\begin{equation}\label{equation:定积分.利用牛顿莱布尼茨公式计算无界函数的反常积分2}
	\int_a^b f(x) \dd{x}
	= [F(b) - F(c^+)] + [F(c^-) - F(a)];
\end{equation}
否则,反常积分\(\int_a^b f(x) \dd{x}\)发散.
\end{theorem}
在\cref{theorem:定积分.利用牛顿莱布尼茨公式计算无界函数的反常积分2} 中,
由于\(F(c^+)\)与\(F(c^-)\)不一定相等(例如,点\(c\)可能是函数\(F\)的跳跃间断点),
所以\begin{equation*}
	\int_a^b f(x) \dd{x}
	\neq [F(x)]_a^b = F(b) - F(a).
\end{equation*}

\begin{example}
%@see: 《高等数学(第六版 上册)》 P258 例4
计算反常积分\begin{equation*}
	\int_0^a \frac{\dd{x}}{\sqrt{a^2-x^2}}
	\quad(a>0).
\end{equation*}
\begin{solution}
因为\begin{equation*}
	\lim_{x \to a^-} \frac{1}{\sqrt{a^2-x^2}} = +\infty,
\end{equation*}
所以点\(a\)是瑕点,于是\begin{equation*}
	\int_0^a \frac{\dd{x}}{\sqrt{a^2-x^2}}
	= \left[ \arcsin\frac{x}{a} \right]_0^a
	= \lim_{x \to a^-} \arcsin\frac{x}{a} - 0 = \frac{\pi}{2}.
\end{equation*}
\end{solution}
\end{example}

\begin{example}
%@see: 《高等数学(第六版 上册)》 P259 例5
讨论反常积分\(\int_{-1}^1 \frac{\dd{x}}{x^2}\)的收敛性.
\begin{solution}
被积函数\(f(x) = \frac{1}{x^2}\)在积分区间\([-1,1]\)上除\(x=0\)外连续,
且\begin{equation*}
	\lim_{x\to0} \frac{1}{x^2} = +\infty.
\end{equation*}

由于\begin{equation*}
	\int_{-1}^0 \frac{\dd{x}}{x^2}
	= \left[-\frac{1}{x}\right]_{-1}^0
	= \lim_{x\to0^-} \left(-\frac{1}{x}\right) - 1
	= +\infty,
\end{equation*}
即反常积分\(\int_{-1}^0 \frac{\dd{x}}{x^2}\)发散,
所以反常积分\(\int_{-1}^1 \frac{\dd{x}}{x^2}\)发散.
\end{solution}
\end{example}
\begin{remark}
如果疏忽了\(x=0\)是被积函数的瑕点,就可能得到以下错误结果:\begin{equation*}
	\int_{-1}^1 \frac{\dd{x}}{x^2}
	= \left[ -\frac{1}{x} \right]_{-1}^1
	= -1 - 1 = -2.
\end{equation*}
\end{remark}

\begin{example}
%@see: 《数学分析(第二版 上册)》(陈纪修) P361 例8.1.6
讨论反常积分\(\int_{-1}^1 \frac{e^{1/x}}{x^2} \dd{x}\)的敛散性.
\begin{solution}
注意到\(x=0\)是被积函数的瑕点,
由于\begin{equation*}
	\int_0^1 \frac{e^{1/x}}{x^2} \dd{x}
	= \eval{-e^{1/x}}_0^1
	= +\infty,
\end{equation*}
所以\(\int_{-1}^1 \frac{e^{1/x}}{x^2} \dd{x}\)发散.
\end{solution}
\end{example}

\begin{proposition}[\(q\)积分]\label{example:定积分.q积分}
%@see: 《高等数学(第六版 上册)》 P259 例6
%@see: 《数学分析(第二版 上册)》(陈纪修) P361 例8.1.5
%@see: 《数学分析教程(第3版 上册)》(史济怀) P282 例4
反常积分\begin{equation*}
	\int_a^b \frac{\dd{x}}{(x-a)^q}
	\quad(a \neq b)
\end{equation*}
当\(0 < q < 1\)时收敛;
当\(q \geq 1\)时发散.
\begin{proof}
当\(q=1\)时,\begin{equation*}
	\int_a^b \frac{\dd{x}}{(x-a)^q}
	= \int_a^b \frac{\dd{x}}{x-a}
	= \eval{\ln(x-a)}_a^b
	= \ln(b-a) - \lim_{x \to a^+} \ln(x-a)
	= +\infty.
\end{equation*}

当\(q\neq1\)时,\begin{equation*}
	\frac{\dd{x}}{(x-a)^q}
	= \frac{1}{1-q} \dd((x-a)^{1-q}),
\end{equation*}
于是,当\(0<q<1\)时,
\(\int_a^b \frac{\dd{x}}{(x-a)^q}
= \frac{(b-a)^{1-q}}{1-q}\);
当\(q>1\)时,
\(\int_a^b \frac{\dd{x}}{(x-a)^q}
= +\infty\).

综上,当\(0<q<1\)时,这反常积分收敛于\(\frac{(b-a)^{1-q}}{1-q}\);
当\(q\geq1\)时,这反常积分发散.
\end{proof}
\end{proposition}

如果收敛的反常积分\(\int_a^b f(x) \dd{x}\)的
被积函数\(f \in C(a,b)\)(\(a\)可以是\(-\infty\),
\(b\)可以是\(+\infty\),\(a\)、\(b\)也可以是\(f(x)\)的瑕点),
且应用的换元函数在\((a,b)\)内单调增加(或减少),
那么可以像定积分一样换元.

\begin{example}
%@see: 《高等数学(第六版 上册)》 P260 例7
求反常积分\(\int_0^{+\infty} \frac{\dd{x}}{\sqrt{x(x+1)^3}}\).
\begin{solution}\let\qed\relax
\begin{proof}[解法一]
这里,积分上限为\(+\infty\),且下限\(x=0\)为被积函数的瑕点.

令\(\sqrt{x} = t\),
则\(x = t^2\),
\(x\to0^+\)时\(t\to0\),
\(x\to+\infty\)时\(t\to+\infty\).
于是\begin{equation*}
	\int_0^{+\infty} \frac{\dd{x}}{\sqrt{x(x+1)^3}}
	= \int_0^{+\infty} \frac{2t\dd{t}}{t(t^2+1)^{\frac32}}
	= 2 \int_0^{+\infty} \frac{\dd{t}}{(t^2+1)^{\frac32}}.
\end{equation*}
再令\(t = \tan u\),
则\(u = \arctan t\),
\(t=0\)时\(u=0\),
\(t\to+\infty\)时,
\(u\to\frac\pi2\).
于是\begin{equation*}
	\int_0^{+\infty} \frac{\dd{x}}{\sqrt{x(x+1)^3}}
	= 2 \int_0^{\frac\pi2} \frac{\sec^2 u \dd{u}}{\sec^3 u}
	= 2 \int_0^{\frac\pi2} \cos u \dd{u}
	= 2.
\end{equation*}
\end{proof}
\begin{proof}[解法二]
令\(t = \frac1x\),则\(x=\frac1t\),\(\dd{x} = -\frac{\dd{t}}{t^2}\).
% 本例也可令\(t = \frac{1}{x+1}\).
当\(x\to0^+\)时\(t\to+\infty\).
当\(x\to+\infty\)时\(t\to0\).
于是\begin{align*}
	\int_0^{+\infty} \frac{\dd{x}}{\sqrt{x(x+1)^3}}
	&= \int_0^{+\infty} \frac1{\sqrt{(1/t)(1/t+1)^3}} \cdot \frac{\dd{t}}{t^2} \\
	&= \int_0^{+\infty} \frac{\dd{t}}{(1+t)^{3/2}} \\
	&= -2 \eval{(1+t)^{-1/2}}_0^{+\infty}
	= 2.
\end{align*}
\end{proof}
\end{solution}
\end{example}

\begin{example}
%@see: 《数学分析教程(第3版 上册)》(史济怀) P283 例5
%@see: 《数学分析(第二版 上册)》(陈纪修) P364 例8.1.9
计算\(\int_0^1 \ln x \dd{x}\).
\begin{solution}
因为\(\lim_{x\to0^+} \ln x = -\infty\),
所以\(x=0\)是一个瑕点.
用分部积分法,
同时利用\cref{example:微分中值定理.洛必达法则.零乘无穷大型1} 的结论\begin{equation*}
	\lim_{x\to0^+} x \ln x = 0,
\end{equation*}
得\begin{equation*}
	\int_0^1 \ln x \dd{x}
	= \eval{x \ln x}_0^1
	- \int_0^1 \dd{x}
	= -1.
\end{equation*}
\end{solution}
\end{example}
\begin{remark}
从这个例子中我们可以归纳出一个结论:\begin{equation}
	\int_0^1 x^n \ln x \dd{x}
	= -\frac1{(n+1)^2}
	\quad(n\geq0).
\end{equation}
%@Mathematica: Integrate[x^n Log[x], {x, 0, 1}, Assumptions -> {n >= 0}]
\end{remark}

\begin{example}
%@see: 《数学分析教程(第3版 上册)》(史济怀) P283 例6
计算\(\int_a^b \frac{\dd{x}}{\sqrt{(x-a)(b-x)}}\).
\begin{solution}
这里\(a\)与\(b\)都是瑕点.
容易看出,这个反常积分是收敛的.
我们用换元法,
考虑到当\(a<x<b\)时,分式\begin{equation*}
	\frac{x-a}{b-a}
	\quad\text{和}\quad
	\frac{b-x}{b-a}
\end{equation*}都是正数,
并且两者之和等于\(1\),
因此可令\begin{equation*}
	\frac{x-a}{b-a} = \sin^2\theta
	\quad(0\leq\theta\leq\pi/2),
\end{equation*}
则\begin{gather*}
	x = a + (b-a) \sin^2\theta
	= a \cos^2\theta + b \sin^2\theta, \\
	\dd{x} = 2(b-a) \cos\theta \sin\theta \dd{\theta}.
\end{gather*}
于是\begin{equation*}
	\int_a^b \frac{\dd{x}}{\sqrt{(x-a)(b-x)}}
	= \int_0^{\pi/2} \frac{2(b-a) \cos\theta \sin\theta \dd{\theta}}
		{\sqrt{(b-a) \sin^2\theta \cdot (b-a) \cos^2\theta}}
	= 2 \int_0^{\pi/2} \dd{\theta}
	= \pi.
\end{equation*}
\end{solution}
\end{example}

\begin{example}
%@see: 《数学分析教程(第3版 上册)》(史济怀) P284 例7
%@see: 《数学分析(第二版 上册)》(陈纪修) P365 例8.1.11
计算反常积分\begin{equation*}
	\int_0^{\pi/2} \ln\sin x \dd{x}
	\quad\text{和}\quad
	\int_0^{\pi/2} \ln\cos x \dd{x}.
\end{equation*}
\begin{solution}
易知点\(x=0\)是瑕点.
令\(x=\pi/2-t\),
那么\begin{equation*}
	\int_0^{\pi/2} \ln\sin x \dd{x}
	= -\int_{\pi/2}^0 \ln\sin\left(\frac\pi2-t\right) \dd{t}
	= \int_0^{\pi/2} \ln\cos t \dd{t}.
\end{equation*}
记\(I \defeq \int_0^{\pi/2} \ln\sin x \dd{x}\).
于是\begin{align*}
	2I &= \int_0^{\pi/2} (\ln\sin x + \ln\cos x) \dd{x} \\
	&= \int_0^{\pi/2} \ln(\sin x \cos x) \dd{x} \\
	&= \int_0^{\pi/2} \ln\left(\frac12 \sin2x\right) \dd{x} \\
	&= -\frac\pi2 \ln2 + \int_0^{\pi/2} \ln\sin2x \dd{x}.
\end{align*}
考虑\begin{align*}
	\int_0^{\pi/2} \ln\sin2x \dd{x}
	&\xlongequal{2x=t}
	\frac12 \int_0^\pi \ln\sin t \dd{t} \\
	&= \frac12 \left(
		\int_0^{\pi/2} \ln\sin t \dd{t}
		+ \int_{\pi/2}^\pi \ln\sin t \dd{t}
	\right) \\
	&= \frac12 I + \frac12 \int_{\pi/2}^\pi \ln\sin t \dd{t},
\end{align*}
其中\begin{align*}
	\int_{\pi/2}^\pi \ln\sin t \dd{t}
	&\xlongequal{t=u+\pi/2}
	\int_0^{\pi/2} \ln\sin\left(u+\frac\pi2\right) \dd{u} \\
	&= \int_0^{\pi/2} \ln\cos u \dd{u}
	= I,
\end{align*}
于是有\begin{equation*}
	2I = -\frac\pi2 \ln2 + \frac12 I + \frac12 I,
\end{equation*}
解得\(I = -\frac\pi2 \ln2\),
也就是说\begin{equation}
	\int_0^{\pi/2} \ln\sin x \dd{x}
	= \int_0^{\pi/2} \ln\cos x \dd{x}
	= -\frac\pi2 \ln2.
\end{equation}
\end{solution}
\end{example}

\section{反常积分的审敛法}
反常积分的收敛性,可以通过求被积函数的原函数,然后按定义取极限,根据极限的存在与否来判定.
但在本节中,我们可以建立不通过被积函数的原函数判定反常积分收敛性的判定法.

\subsection{柯西极限存在准则}
由于反常积分\(\int_a^{+\infty} f(x) \dd{x}\)收敛
即为极限\(\lim_{t\to+\infty} \int_a^t f(x) \dd{x}\)存在,
因此对反常积分收敛性的最本质的刻画,
就是极限论中的\hyperref[theorem:极限.函数的柯西极限存在准则]{柯西极限存在准则},
它可以表述为如下形式:
\begin{theorem}\label{theorem:反常积分.柯西极限存在准则}
%@see: 《数学分析(第二版 上册)》(陈纪修) P370 定理8.2.1(Cauchy收敛原理)
设函数\(f\colon[a,+\infty)\to\mathbb{R}\),
则反常积分\(\int_a^{+\infty} f(x) \dd{x}\)收敛的充分必要条件是:
对任意给定\(\epsilon>0\),存在\(X \geq a\),
使得对任意\(x_1,x_2 \geq X\),
有\begin{equation*}
	\abs{\int_{x_1}^{x_2} f(t) \dd{t}} < \epsilon.
\end{equation*}
\end{theorem}

同样地,由于反常积分\(\int_a^b f(x) \dd{x}\ (\text{$x=a$是瑕点})\)收敛
即为极限\(\lim_{\epsilon\to0^+} \int_{a+\epsilon}^b f(x) \dd{x}\)存在,
因此我们可以把\hyperref[theorem:极限.函数的柯西极限存在准则]{柯西极限存在准则}表述为如下形式:
\begin{theorem}
%@see: 《数学分析(第二版 上册)》(陈纪修) P376 定理8.2.1'(Cauchy收敛原理)
设函数\(f\colon[a,b)\to\mathbb{R}\),
点\(x=b\)是瑕点,
则反常积分\(\int_a^b f(x) \dd{x}\)收敛的充分必要条件是:
对任意给定\(\epsilon>0\),存在\(\delta>0\),
使得对任意\(d_1,d_2\in(0,\delta)\),
有\begin{equation*}
	\abs{\int_{b-d_1}^{b-d_2} f(x) \dd{x}} < \epsilon.
\end{equation*}
\end{theorem}

\subsection{绝对收敛,条件收敛}
\begin{definition}
%@see: 《数学分析(第二版 上册)》(陈纪修) P370 定义8.2.1
设函数\(f\colon[a,+\infty)\to\mathbb{R}\)在任意有限闭区间上黎曼可积.
\begin{itemize}
	\item 如果反常积分\begin{equation*}
		\int_a^{+\infty} \abs{f(x)} \dd{x}
	\end{equation*}收敛,
	则称“反常积分\(\int_a^{+\infty} f(x) \dd{x}\)~\DefineConcept{绝对收敛}”.

	\item 如果反常积分\begin{equation*}
		\int_a^{+\infty} f(x) \dd{x}
	\end{equation*}收敛,
	而\begin{equation*}
		\int_a^{+\infty} \abs{f(x)} \dd{x}
	\end{equation*}不收敛,
	则“反常积分\(\int_a^{+\infty} f(x) \dd{x}\)~\DefineConcept{条件收敛}”.
\end{itemize}
\end{definition}

\begin{theorem}\label{theorem:定积分.绝对收敛的无穷限反常积分必收敛}
%@see: 《高等数学(第六版 上册)》 P264 定理5
%@see: 《数学分析(第二版 上册)》(陈纪修) P370 推论
绝对收敛的反常积分必定收敛.
\begin{proof}
对于任意给定\(\epsilon>0\),
由于\(\int_a^{+\infty} \abs{f(x)} \dd{x}\)收敛,
所以存在\(X \geq a\),
使得对任意\(x_1,x_2 \geq X\),
成立\begin{equation*}
	\int_{x_1}^{x_2} \abs{f(x)} \dd{x} < \epsilon.
\end{equation*}
利用\cref{theorem:定积分.定积分性质5推论2} 可得\begin{equation*}
	\abs{\int_{x_1}^{x_2} f(x) \dd{x}}
	\leq \int_{x_1}^{x_2} f(x) \dd{x}
	< \epsilon,
\end{equation*}
再次运用\hyperref[theorem:反常积分.柯西极限存在准则]{柯西极限存在准则}便知
\(\int_a^{+\infty} f(x) \dd{x}\)收敛.
\end{proof}
\end{theorem}

\begin{example}
%@see: 《高等数学(第六版 上册)》 P268 习题5-5 2.
设反常积分\(\int_1^{+\infty} f^2(x) \dd{x}\)收敛.
证明:反常积分\(\int_1^{+\infty} \frac{f(x)}{x} \dd{x}\)绝对收敛.
\begin{proof}
由\hyperref[theorem:不等式.基本不等式2]{基本不等式}
\(a^2 + b^2 \geq 2ab\ (a,b\in\mathbb{R})\)可得,
在区间\([1,+\infty)\)上有\begin{equation*}
	\abs{\frac{f(x)}{x}}
	= \frac{\abs{f(x)}}{x}
	\leq \frac{1}{2} \left[
		\abs{f(x)}^2 + \frac{1}{x^2}
	\right]
	= \frac{1}{2} f^2(x) + \frac{1}{2x^2}
\end{equation*}成立,
且\(\int_1^{+\infty} f^2(x) \dd{x}\)和\(\int_1^{+\infty} \frac{1}{x^2} \dd{x}\)都收敛,
故\(\int_1^{+\infty} \abs{\frac{f(x)}{x}} \dd{x}\)收敛.
\end{proof}
\end{example}

\subsection{非负函数的反常积分的审敛法}
\begin{theorem}\label{theorem:定积分.无穷限反常积分的审敛法引理}
%@see: 《高等数学(第六版 上册)》 P261 定理1
%@see: 《数学分析教程 (第3版 下册)》(史济怀) P274 定理16.1.1
设函数\(f\colon[a,+\infty)\to\mathbb{R}\)非负,
则反常积分\(\int_a^{+\infty} f(x) \dd{x}\)收敛的充分必要条件是:
函数\begin{equation*}
	F\colon[a,+\infty)\to\mathbb{R},
	x \mapsto \int_a^x f(t) \dd{t}
\end{equation*}在\([a,+\infty)\)上有上界.
\begin{proof}
根据\hyperref[definition:定积分.无穷限的反常积分的定义1]{定义},
反常积分\(\int_a^{+\infty} f(x) \dd{x}\)收敛,
当且仅当极限\(F(+\infty) = \lim_{x\to+\infty} F(x)\)存在且有限.
由于\(f(x)\geq0\),
所以\(F(x) = \int_a^x f(t) \dd{t}\)是单调增加的,
于是由\hyperref[theorem:极限.函数的单调有界定理]{单调有界定理}%
和\hyperref[theorem:极限.函数极限的局部有界性]{局部有界性}可知,
极限\(F(+\infty)\)存在且有限的充分必要条件是\(F\)有界.
\end{proof}
\end{theorem}

根据\cref{theorem:定积分.无穷限反常积分的审敛法引理},
对于非负函数的无穷限的反常积分,有以下的比较审敛原理.
\begin{theorem}[比较审敛原理]\label{theorem:定积分.无穷限反常积分的比较审敛原理}
%@see: 《高等数学(第六版 上册)》 P261 定理2(比较审敛原理)
%@see: 《数学分析(第二版 上册)》(陈纪修) P371 定理8.2.2(比较判别法)
设函数\(f,g\)在区间\([a,+\infty)\)上连续.
\begin{itemize}
	\item 如果\begin{equation*}
		0 \leq f(x) \leq g(x)
		\quad(a \leq x < +\infty),
	\end{equation*}
	并且\(\int_a^{+\infty} g(x) \dd{x}\)收敛,
	则\(\int_a^{+\infty} f(x) \dd{x}\)也收敛.

	\item 如果\begin{equation*}
		0 \leq g(x) \leq f(x)
		\quad(a \leq x < +\infty),
	\end{equation*}
	并且\(\int_a^{+\infty} g(x) \dd{x}\)发散,
	则\(\int_a^{+\infty} f(x) \dd{x}\)也发散.
\end{itemize}
\begin{proof}
任意取定\(t \in (a,+\infty)\).

当有\(0 \leq f(x) \leq g(x)\)且\(\int_a^{+\infty} g(x) \dd{x}\)收敛时,
根据\cref{theorem:定积分.定积分性质5推论1}
得\begin{equation*}
	\int_a^t f(x) \dd{x}
	\leq
	\int_a^t g(x) \dd{x}
	\leq
	\int_a^{+\infty} g(x) \dd{x}.
\end{equation*}
这表明作为积分上限\(t\)的函数\begin{equation*}
	F(t) = \int_a^t f(x) \dd{x}
\end{equation*}
在区间\([a,+\infty)\)上有上界.
由\cref{theorem:定积分.无穷限反常积分的审敛法引理} 可知,
反常积分\(\int_a^{+\infty} f(x) \dd{x}\)也收敛.
\end{proof}
\end{theorem}

\begin{example}
%@credit: {3ba7e116-f533-4966-b9ee-3e553864ebec}
设函数\(f\)在\([0,+\infty)\)上可导,
且\(f'(x)>0,f(0)=1\).
证明:\(\int_0^{+\infty} \frac1{f(x)+f'(x)} \dd{x}\)收敛
是\(\int_0^{+\infty} \frac1{f(x)} \dd{x}\)收敛的充分必要条件.
\begin{proof}
因为\(f'(x)>0\)且\(f(0)=1\),
所以\(f\)在\([0,+\infty)\)上严格单调增加且恒大于或等于\(1\),
从而函数\(x \mapsto \frac1{f(x)}\)在\([0,+\infty)\)上严格单调减少且\(0 < \frac1{f(x)} \leq 1\),
那么由\hyperref[theorem:极限.函数的单调有界定理]{单调有界定理}可知\begin{equation*}
	\lim_{x\to+\infty} \frac1{f(x)} = A < \infty.
\end{equation*}
当\(x\geq0\)时,有\begin{equation*}
	0 < \frac1{f(x)} - \frac1{f(x)+f'(x)}
	= \frac{f'(x)}{f^2(x)+f(x) f'(x)}
	< \frac{f'(x)}{f^2(x)}.
\end{equation*}
又因为\begin{equation*}
	\int_0^{+\infty} \frac{f'(x)}{f^2(x)} \dd{x}
	= \eval{-\frac1{f(x)}}_0^{+\infty}
	= 1 - A,
\end{equation*}
所以由\hyperref[theorem:定积分.无穷限反常积分的比较审敛原理]{比较审敛原理}可知
反常积分\(\int_0^{+\infty} \left( \frac1{f(x)} - \frac1{f(x)+f'(x)} \right) \dd{x}\)收敛.

因此\(\int_0^{+\infty} \frac1{f(x)+f'(x)} \dd{x}\)收敛
是\(\int_0^{+\infty} \frac1{f(x)} \dd{x}\)收敛的充分必要条件.
\end{proof}
\end{example}

\begin{corollary}[比较审敛原理的极限形式]
%@see: 《数学分析(第二版 上册)》(陈纪修) P371 推论(比较判别法的极限形式)
设函数\(f,g\)在\([a,+\infty)\)上非负,
记\begin{equation*}
	\rho \defeq \lim_{x\to+\infty} \frac{f(x)}{g(x)}.
\end{equation*}
\begin{itemize}
	\item 若\(0 \leq \rho < +\infty\),
	则\begin{equation*}
		\text{\(\int_a^{+\infty} g(x) \dd{x}\)收敛}
		\implies
		\text{\(\int_a^{+\infty} f(x) \dd{x}\)收敛}.
	\end{equation*}
	\item 若\(0 < \rho \leq +\infty\),
	则\begin{equation*}
		\text{\(\int_a^{+\infty} g(x) \dd{x}\)发散}
		\implies
		\text{\(\int_a^{+\infty} f(x) \dd{x}\)发散}.
	\end{equation*}
\end{itemize}
\end{corollary}

\begin{theorem}[比较审敛法1]\label{theorem:定积分.无穷限反常积分的比较审敛法}
%@see: 《高等数学(第六版 上册)》 P262 定理3(比较审敛法1)
%@see: 《数学分析(第二版 上册)》(陈纪修) P372 定理8.2.3(Cauchy判别法)
设函数\(f\)在区间\([a,+\infty)\)上非负.
\begin{itemize}
	\item 如果存在常数\(M>0\)及\(p>1\),使得\begin{equation*}
		f(x) \leq \frac{M}{x^p}
		\quad(a \leq x < +\infty),
	\end{equation*}
	则反常积分\(\int_a^{+\infty} f(x) \dd{x}\)收敛.
	\item 如果存在常数\(N>0\),使得\begin{equation*}
		f(x) \geq \frac{N}{x}
		\quad(a \leq x < +\infty),
	\end{equation*}
	则反常积分\(\int_a^{+\infty} f(x) \dd{x}\)发散.
\end{itemize}
\end{theorem}

\begin{theorem}[比较审敛法2]\label{theorem:定积分.无界函数的反常积分的比较审敛法}
%@see: 《高等数学(第六版 上册)》 P265 定理6(比较审敛法2)
%@see: 《数学分析(第二版 上册)》(陈纪修) P376 定理8.2.3'(Cauchy判别法)
设函数\(f\)在区间\((a,b]\)上非负,
点\(x=a\)是瑕点.
\begin{itemize}
	\item 如果存在常数\(M > 0\)及\(q < 1\),使得\begin{equation*}
		f(x) \leq \frac{M}{(x-a)^q}
		\quad(a < x \leq b),
	\end{equation*}
	则反常积分\(\int_a^b f(x) \dd{x}\)收敛.
	\item 如果存在常数\(N > 0\)及\(q \geq 1\),使得\begin{equation*}
		f(x) \geq \frac{N}{(x-a)^q}
		\quad(a < x \leq b),
	\end{equation*}
	则反常积分\(\int_a^b f(x) \dd{x}\)发散.
\end{itemize}
\end{theorem}

\begin{theorem}[极限审敛法1]\label{theorem:定积分.无穷限反常积分的极限审敛法}
%@see: 《高等数学(第六版 上册)》 P262 定理4(极限审敛法1)
%@see: 《数学分析(第二版 上册)》(陈纪修) P373 推论(Cauchy判别法的极限形式)
设函数\(f\)在区间\([a,+\infty)\)上非负,
记\begin{equation*}
	\rho(k) \defeq \lim_{x\to+\infty} x^k f(x).
\end{equation*}
\begin{itemize}
	\item 如果存在常数\(k > 1\),
	使得\(0 \leq \rho(k) < +\infty\),
	则反常积分\(\int_a^{+\infty} f(x) \dd{x}\)收敛.
	\item 如果存在常数\(k \leq 1\),
	使得\(0 < \rho(k) \leq +\infty\),
	则反常积分\(\int_a^{+\infty} f(x) \dd{x}\)发散.
\end{itemize}
\end{theorem}
\begin{remark}
\cref{theorem:定积分.无穷限反常积分的极限审敛法} 说明:
如果\(f\)是当\(x\to+\infty\)时比\(1/x\)高阶的无穷小,
则反常积分\(\int_a^{+\infty} f(x) \dd{x}\)收敛.
如果\(f\)是当\(x\to+\infty\)时与\(1/x\)同阶的无穷小,
或\(f\)是当\(x\to+\infty\)时比\(1/x\)低阶的无穷小,
则反常积分\(\int_a^{+\infty} f(x) \dd{x}\)发散.
\end{remark}

\begin{theorem}[极限审敛法2]\label{theorem:定积分.无界函数的反常积分的极限审敛法}
%@see: 《高等数学(第六版 上册)》 P265 定理7(极限审敛法2)
%@see: 《数学分析(第二版 上册)》(陈纪修) P372 推论'(Cauchy判别法的极限形式)
设函数\(f\)在区间\((a,b]\)上非负,
点\(x=a\)是瑕点,
记\begin{equation*}
	\rho(k) \defeq \lim_{x\to a^+} (x-a)^k f(x).
\end{equation*}
\begin{itemize}
	\item 如果存在常数\(k < 1\),
	使得\(0 \leq \rho(k) < +\infty\),
	则反常积分\(\int_a^b f(x) \dd{x}\)收敛.

	\item 如果存在常数\(k \geq 1\),
	使得\(0 < \rho(k) \leq +\infty\),
	则反常积分\(\int_a^b f(x) \dd{x}\)发散.
\end{itemize}
\end{theorem}
\begin{remark}
\cref{theorem:定积分.无界函数的反常积分的极限审敛法} 说明:
如果\(f\)是当\(x \to a^+\)时比\(1/(x-a)\)低阶的无穷大,
则反常积分\(\int_a^b f(x) \dd{x}\)收敛.
如果\(f\)是当\(x \to a^+\)时与\(1/(x-a)\)同阶的无穷大,
或\(f\)是当\(x \to a^+\)时比\(1/(x-a)\)高阶的无穷大,
则反常积分\(\int_a^b f(x) \dd{x}\)发散.
\end{remark}

\begin{example}\label{example:定积分.高斯积分的收敛性}
证明:高斯积分\begin{equation*}
	\int_0^{+\infty} e^{-x^2} \dd{x}
\end{equation*}收敛.
\begin{proof}
因为在区间\([0,+\infty)\)上被积函数\(f(x) = e^{-x^2} > 0\),
且由\cref{example:微分中值定理.洛必达法则.无穷大比无穷大型2} 可知极限\begin{equation*}
	\lim_{x\to+\infty} x^2 e^{-x^2}
	= \lim_{x\to+\infty} \frac{x^2}{e^{x^2}}
	\xlongequal{u=x^2} \lim_{u\to+\infty} \frac{u}{e^u}
	= 0,
\end{equation*}
那么根据\cref{theorem:定积分.无穷限反常积分的极限审敛法},
高斯积分\(\int_0^{+\infty} e^{-x^2} \dd{x}\)收敛.
\end{proof}
\end{example}
\begin{remark}
同理可知\(\int_{-\infty}^0 e^{-x^2} \dd{x}\)也收敛,
于是高斯积分\begin{equation*}
	\int_{-\infty}^{+\infty} e^{-x^2} \dd{x}
\end{equation*}收敛.
%\cref{example:反常积分.利用伽马函数计算高斯积分}
%\cref{example:二重积分.利用二重积分计算高斯积分}
\end{remark}

\begin{example}
%@see: 《2010年全国硕士研究生入学统一考试(数学一)》一选择题/第3题
设\(m,n\)是正整数,
讨论反常积分\(\int_0^1 \frac{\sqrt[m]{\ln^2(1-x)}}{\sqrt[n]{x}} \dd{x}\)的敛散性.
\begin{proof}
点\(x=0\)和\(x=1\)都可能是被积函数\(\frac{\sqrt[m]{\ln^2(1-x)}}{\sqrt[n]{x}}\)的瑕点.

先考虑\(\int_0^{\frac12} \frac{\sqrt[m]{\ln^2(1-x)}}{\sqrt[n]{x}} \dd{x}\)的敛散性.

因为当\(x\to0^+\)时\(\ln(1-x) \sim -x\),
所以\begin{equation*}
	\lim_{x\to0^+} \frac{\sqrt[m]{\ln^2(1-x)}}{\sqrt[n]{x}}
	= \lim_{x\to0^+} x^{\frac2m-\frac1n}
	= \left\{ \def\arraystretch{1.5} \begin{array}{cl}
		0, & \frac2m-\frac1n>0, \\
		1, & \frac2m-\frac1n=0, \\
		\infty, & \frac2m-\frac1n<0.
	\end{array} \right.
\end{equation*}
由于\(m,n\)都是正整数,所以\(-1<\frac2m-\frac1n<2\).
于是当\(\frac2m-\frac1n\geq0\)时,
点\(x=0\)不是被积函数\(\frac{\sqrt[m]{\ln^2(1-x)}}{\sqrt[n]{x}}\)的瑕点,
定积分\(\int_0^{\frac12} \frac{\sqrt[m]{\ln^2(1-x)}}{\sqrt[n]{x}} \dd{x}\)收敛.
当\(-1<\frac2m-\frac1n<0\)时,
点\(x=0\)是被积函数\(\frac{\sqrt[m]{\ln^2(1-x)}}{\sqrt[n]{x}}\)的瑕点,
只要取\(q=\frac1n-\frac2m\in(0,1)\),
就有\begin{equation*}
	\lim_{x\to0^+} x^q \frac{\sqrt[m]{\ln^2(1-x)}}{\sqrt[n]{x}} = 1.
\end{equation*}
由\cref{theorem:定积分.无界函数的反常积分的极限审敛法} 可知
反常积分\(\int_0^{\frac12} \frac{\sqrt[m]{\ln^2(1-x)}}{\sqrt[n]{x}} \dd{x}\)收敛.
因此,不论\(m,n\)取什么正整数,
积分\(\int_0^{\frac12} \frac{\sqrt[m]{\ln^2(1-x)}}{\sqrt[n]{x}} \dd{x}\)总是收敛.

再考虑\(\int_{\frac12}^1 \frac{\sqrt[m]{\ln^2(1-x)}}{\sqrt[n]{x}} \dd{x}\)的敛散性.
因为\begin{align*}
	\lim_{x\to1^-} \sqrt{1-x} \frac{\sqrt[m]{\ln^2(1-x)}}{\sqrt[n]{x}}
	&= \lim_{x\to1^-} \sqrt{1-x} \ln^{\frac2m}(1-x) \\
	&\xlongequal{u=1-x}
	\lim_{u\to0^+} u^{\frac12} \ln^{\frac2m} u
	= 0, \tag{\cref{example:微分中值定理.洛必达法则.零乘无穷大型2}}
\end{align*}
所以由\cref{theorem:定积分.无界函数的反常积分的极限审敛法} 可知
不论\(m,n\)取什么正整数,
反常积分\(\int_{\frac12}^1 \frac{\sqrt[m]{\ln^2(1-x)}}{\sqrt[n]{x}} \dd{x}\)总是收敛.
\end{proof}
\end{example}
\begin{example}
%@see: 《2016年全国硕士研究生入学统一考试(数学一)》一选择题/第1题
设反常积分\(\int_0^{+\infty} \frac{\dd{x}}{x^a (1+x)^b}\)收敛,
判断\(a,b\)的取值范围.
\begin{solution}
令\(f(x) = \frac1{x^a (1+x)^b}\).
由\cref{theorem:定积分.定积分性质3} 可得\begin{equation*}
	\int_0^{+\infty} \frac{\dd{x}}{x^a (1+x)^b}
	= \int_0^1 \frac{\dd{x}}{x^a (1+x)^b}
	+ \int_1^{+\infty} \frac{\dd{x}}{x^a (1+x)^b}
	= I_1 + I_2.
\end{equation*}

先考虑积分\(I_1\).
\begin{itemize}
	\item 当\(a\leq0\)时,
	\(I_1\)是常义积分.
	\item 当\(a>0\)时,\(x=0\)是瑕点,有\begin{equation*}
		\lim_{x\to0^+} x^a f(x)
		= \lim_{x\to0^+} \frac1{(1+x)^b}
		= 1 > 0,
	\end{equation*}
	由\cref{theorem:定积分.无界函数的反常积分的极限审敛法} 可知,
	当\(0<a<1\)时,积分\(I_1\)收敛;
	当\(a\geq1\)时,积分\(I_1\)发散.
\end{itemize}
因此,要使\(I_1\)收敛,必有\(a < 1\).

再考虑积分\(I_2\).
注意到\begin{equation*}
	\lim_{x\to+\infty} x^{a+b} f(x)
	= \lim_{x\to+\infty} \frac{x^b}{(1+x)^b}
	= 1 > 0,
\end{equation*}
由\cref{theorem:定积分.无穷限反常积分的极限审敛法} 可知,
当\(a+b>1\)时,积分\(I_2\)收敛;
当\(a+b\leq1\)时,积分\(I_2\)发散.

综上所述,反常积分\(\int_0^{+\infty} \frac{\dd{x}}{x^a (1+x)^b}\)收敛,
当且仅当\(a<1\)且\(a+b>1\).
\end{solution}
\end{example}

\subsection{一般函数的反常积分的审敛法}
\begin{theorem}\label{theorem:反常积分.无穷限的反常积分的阿贝尔--狄利克雷审敛法}
%@see: 《数学分析(第二版 上册)》(陈纪修) P374 定理8.2.5
设函数\(f,g\colon[a,+\infty)\to\mathbb{R}\).

若下列两个条件之一满足,
则反常积分\(\int_a^{+\infty} f(x) ~ g(x) \dd{x}\)收敛:\begin{itemize}
	\item {\rm\bf 阿贝尔条件}
	反常积分\(\int_a^{+\infty} f(x) \dd{x}\)收敛,
	函数\(g\)在\([a,+\infty)\)上单调有界;

	\item {\rm\bf 狄利克雷条件}
	函数\(x \mapsto \int_a^x f(t) \dd{t}\)在\([a,+\infty)\)上有界,
	函数\(g\)在\([a,+\infty)\)上单调且\(\lim_{x\to+\infty} g(x) = 0\).
\end{itemize}
%TODO proof
%\cref{theorem:定积分.积分中值定理2}
%\cref{theorem:无穷级数.无穷级数的阿贝尔--狄利克雷审敛法}
\end{theorem}

\begin{example}
%@see: 《数学分析(第二版 上册)》(陈纪修) P375 例8.2.4
证明:反常积分\(\int_1^{+\infty} \frac{\sin x}{x} \dd{x}\)条件收敛.
% 与狄利克雷积分\(\int_0^{+\infty} \frac{\sin x}{x} \dd{x}\)的积分下限不同!
\begin{proof}
显然函数\(x \mapsto \int_1^x \sin t \dd{t}\)有界,
函数\(x \mapsto \frac1x\)在\([1,+\infty)\)上单调减少且\(\lim_{x\to+\infty} \frac1x = 0\),
满足\hyperref[theorem:反常积分.无穷限的反常积分的阿贝尔--狄利克雷审敛法]{狄利克雷条件},
因此反常积分\(\int_1^{+\infty} \frac{\sin x}{x} \dd{x}\)收敛.

但是在\([1,+\infty)\)上,有\begin{equation*}
	\abs{\frac{\sin x}{x}}
	\geq \frac{\sin^2x}{x}
	= \frac1{2x} - \frac{\cos2x}{2x},
\end{equation*}
因反常积分\(\int_1^{+\infty} \frac{\cos2x}{2x} \dd{x}\)收敛
(仿照上面对反常积分\(\int_1^{+\infty} \frac{\sin x}{x} \dd{x}\)的讨论),
而反常积分\(\int_1^{+\infty} \frac{\dd{x}}{2x}\)发散,
所以反常积分\(\int_1^{+\infty} \frac{\sin^2x}{x} \dd{x}\)发散.
再由\hyperref[theorem:定积分.无穷限反常积分的比较审敛原理]{比较审敛原理}可知,
\(\int_1^{+\infty} \abs{\frac{\sin x}{x}} \dd{x}\)发散.

综上所述,\(\int_1^{+\infty} \frac{\sin x}{x} \dd{x}\)条件收敛.
\end{proof}
\end{example}
\begin{example}
%@see: 《数学分析(第二版 上册)》(陈纪修) P375 例8.2.5
证明:反常积分\(\int_1^{+\infty} \frac{\sin x \arctan x}{x} \dd{x}\)条件收敛.
\begin{proof}
由上例可知\(\int_1^{+\infty} \frac{\sin x}{x} \dd{x}\)收敛,
而函数\(x \mapsto \arctan x\)在\([1,+\infty)\)上单调有界,
满足\hyperref[theorem:反常积分.无穷限的反常积分的阿贝尔--狄利克雷审敛法]{阿贝尔条件},
因此\(\int_1^{+\infty} \frac{\sin x \arctan x}{x} \dd{x}\)收敛.

当\(\sqrt3 \leq x < +\infty\)时,有\begin{equation*}
	\abs{\frac{\sin x \arctan x}{x}}
	\geq \abs{\frac{\sin x}{x}}.
\end{equation*}
因为\(\int_1^{+\infty} \abs{\frac{\sin x}{x}} \dd{x}\)发散,
所以由\hyperref[theorem:定积分.无穷限反常积分的比较审敛原理]{比较审敛原理}可知,
\(\int_1^{+\infty} \abs{\frac{\sin x \arctan x}{x}} \dd{x}\)发散.

综上所述,\(\int_1^{+\infty} \frac{\sin x \arctan x}{x} \dd{x}\)条件收敛.
\end{proof}
\end{example}

\begin{theorem}\label{theorem:反常积分.无界函数的反常积分的阿贝尔--狄利克雷审敛法}
%@see: 《数学分析(第二版 上册)》(陈纪修) P376 定理8.2.5'
设函数\(f,g\colon(a,b]\to\mathbb{R}\),
点\(x=a\)是瑕点.

若下列两个条件之一满足,
则反常积分\(\int_a^b f(x) ~ g(x) \dd{x}\)收敛:\begin{itemize}
	\item {\rm\bf 阿贝尔条件}
	反常积分\(\int_a^b f(x) \dd{x}\)收敛,
	函数\(g\)在\((a,b]\)上单调有界.

	\item {\rm\bf 狄利克雷条件}
	函数\(x \mapsto \int_{a+x}^b f(t) \dd{t}\)在\((0,b-a]\)上有界,
	函数\(g\)在\((a,b]\)上单调且\(\lim_{x \to a^+} g(x) = 0\).
\end{itemize}
%TODO proof
\end{theorem}

\subsection{反常积分的重要特性}
%@see: 《数学分析(第二版 上册)》(陈纪修) P362
设函数\(f\colon[a,+\infty)\to\mathbb{R}\).
由\cref{example:定积分.p积分} 可知,
\(\lim_{x\to+\infty} f(x) = 0\)
并不能保证\(\int_a^{+\infty} f(x) \dd{x}\)收敛.
% 反常积分\(\int_0^{+\infty} \sin x^2 \dd{x}\)收敛于\(\sqrt{\frac\pi8}\),但是函数\(x \mapsto \sin x^2\)不是当\(x\to+\infty\)时的无穷小
反过来,若\(\int_a^{+\infty} f(x) \dd{x}\)收敛,
能否保证\(\lim_{x\to+\infty} f(x) = 0\),
或者退一步,至少保证\(f\)在定义域上有界呢?
下面我们用一个例子说明:同样不能!
\begin{example}
%@see: 《数学分析(第二版 上册)》(陈纪修) P362 例8.1.7
设函数\(f\colon[1,+\infty)\to\mathbb{R}\)
满足\begin{equation*}
	f(x) = \left\{ \begin{array}{cl}
		n+1, & n \leq x \leq n+\frac1{n(n+1)^2}, \\
		0, & n+\frac1{n(n+1)^2} < x < n+1,
	\end{array} \right.
	\quad n=1,2,\dotsc.
\end{equation*}
那么对于任意\(t>1\),
总可以取自然数\(n\),
使得\(n \leq t < n+1\).
由于\(f(x)\geq0\),
所以\begin{equation*}
	\int_1^n f(x) \dd{x}
	\leq \int_1^t f(x) \dd{x}
	\leq \int_1^{n+1} f(x) \dd{x}.
\end{equation*}
考虑到\begin{equation*}
	\int_k^{k+1} f(x) \dd{x}
	= \int_k^{k+\frac1{k(k+1)^2}} (k+1) \dd{x}
	= (k+1) \cdot \frac1{k(k+1)^2}
	= \frac1{k(k+1)},
	\quad k=1,2,\dotsc,
\end{equation*}
令\(n\to\infty\),
得\begin{align*}
	\lim_{n\to\infty} \int_1^n f(x) \dd{x}
	&= \lim_{n\to\infty} \left[
		\int_1^2 f(x) \dd{x}
		+ \int_2^3 f(x) \dd{x}
		+ \dotsb
		+ \int_{n-1}^n f(x) \dd{x}
	\right] \\
	&= \lim_{n\to\infty} \left[
		\frac1{1\cdot2}
		+ \frac1{2\cdot3}
		+ \dotsb
		+ \frac1{(n-1)n}
	\right] \\
	&= \lim_{n\to\infty} \left[
		\left(1-\frac12\right)
		+ \left(\frac12-\frac13\right)
		+ \dotsb
		+ \left(\frac1{n-1}-\frac1n\right)
	\right] \\
	&= \lim_{n\to\infty} \left(1-\frac1n\right)
	= 1.
\end{align*}
同理可得\begin{equation*}
	\lim_{n\to\infty} \int_1^{n+1} f(x) \dd{x} = 1.
\end{equation*}
利用\hyperref[theorem:数列极限.夹逼准则]{夹逼准则}便知\begin{equation*}
	\int_1^{+\infty} f(x) \dd{x}
	= \lim_{t\to+\infty} \int_1^t f(x) \dd{x}
	= 1,
\end{equation*}
但是\(f\)显然是无界的.
\end{example}
我们还可以断言,即使\(f\)在\([a,+\infty)\)上连续(甚至\(n\)次可微),
也可仿照上例,构造出使\(\int_a^{+\infty} f(x) \dd{x}\)收敛,
而\(f\)在\([a,+\infty)\)无界的例子.

下面证明,只要把“\(f\)在\([a,+\infty)\)上连续”
或“\(f\)在\([a,+\infty)\)上\(n\)次可微”
改为“\(f\)在\([a,+\infty)\)上一致连续”,
就能得到\(\lim_{x\to+\infty} f(x) = 0\).

\begin{proposition}
%@see: 《数学分析(第二版 上册)》(陈纪修) P379 例8.2.9
设函数\(f\colon[a,+\infty)\to\mathbb{R}\)在\([a,+\infty)\)一致连续,
且反常积分\(\int_a^{+\infty} f(x) \dd{x}\)收敛,
则\(\lim_{x\to+\infty} f(x) = 0\).
%TODO proof
\end{proposition}

\section{柯西主值积分}
\subsection{第一类柯西主值积分}
%@see: https://mathworld.wolfram.com/CauchyPrincipalValue.html
\begin{definition}
%@see: 《数学分析(第二版 上册)》(陈纪修) P366 定义8.1.3
设函数\(f\colon\mathbb{R}\to\mathbb{R}\).
如果极限\begin{equation*}
	\lim_{R\to+\infty} \int_{-R}^R f(x) \dd{x}
\end{equation*}存在且有限,
则称“反常积分\(\int_{-\infty}^{+\infty} f(x) \dd{x}\)~\DefineConcept{在柯西主值意义下收敛}”,
并把这个极限称为
“反常积分\(\int_{-\infty}^{+\infty} f(x) \dd{x}\)
的\DefineConcept{柯西主值}(Cauchy principal value)”
或“函数\(f\)在区间\((-\infty,+\infty)\)上的积分的柯西主值
(the Cauchy principal value of the integral of \(f\) over \((-\infty,+\infty)\))”,
记为\begin{equation*}
	\pvint_{-\infty}^{+\infty} f(x) \dd{x},
\end{equation*}
即\begin{equation*}
	\pvint_{-\infty}^{+\infty} f(x) \dd{x}
	\defeq \lim_{R\to+\infty} \int_{-R}^R f(x) \dd{x}.
\end{equation*}
\end{definition}

若\(\int_{-\infty}^{+\infty} f(x) \dd{x}\)收敛,
则它在柯西主值意义下也收敛,
且\(\int_{-\infty}^{+\infty} f(x) \dd{x}\)的值
与它的柯西主值\(\pvint_{-\infty}^{+\infty} f(x) \dd{x}\)相等.
可是,当柯西主值\(\pvint_{-\infty}^{+\infty} f(x) \dd{x}\)收敛时,
\(\int_{-\infty}^{+\infty} f(x) \dd{x}\)本身未必一定收敛.
例如,\(\pvint_{-\infty}^{+\infty} x \dd{x} = 0\),
而\(\int_{-\infty}^{+\infty} x \dd{x}\)发散.
但在一般给出的问题中,要么只需要求柯西主值,
要么不难预先看出\(\int_{-\infty}^{+\infty} f(x) \dd{x}\)收敛,
因此,只要求出柯西主值\(\pvint_{-\infty}^{+\infty} f(x) \dd{x}\)的值,
也就求出了\(\int_{-\infty}^{+\infty} f(x) \dd{x}\)的值.

\subsection{第二类柯西主值积分}
\begin{definition}
设函数\(f\colon[a,b]\to\mathbb{R}\),点\(c\in(a,b)\)是一个瑕点.
如果极限\begin{equation*}
	\lim_{\epsilon\to0^+} \left[
		\int_a^{c-\epsilon} f(x) \dd{x}
		+ \int_{c+\epsilon}^b f(x) \dd{x}
	\right]
\end{equation*}存在且有限,
则称“反常积分\(\int_a^b f(x) \dd{x}\)~\DefineConcept{在柯西主值意义下收敛}”,
并把这个极限称为
“反常积分\(\int_a^b f(x) \dd{x}\)
的\DefineConcept{柯西主值}(Cauchy principal value)”,
记为\begin{equation*}
	\pvint_a^b f(x) \dd{x},
\end{equation*}
即\begin{equation*}
	\pvint_a^b f(x) \dd{x}
	\defeq \lim_{\epsilon\to0^+} \left[
		\int_a^{c-\epsilon} f(x) \dd{x}
		+ \int_{c+\epsilon}^b f(x) \dd{x}
	\right].
\end{equation*}
\end{definition}

%@Mathematica: Integrate[1/x, {x, -1, 2}, PrincipalValue -> True]

\input{微积分/反常积分/贝塔函数}
\section{伽马函数}
\subsection{伽马函数的历史渊源}
假设函数\(\Gamma\colon\mathbb{R}\to\mathbb{R}\)的值\(\Gamma(n+1)\)
恰好与\(n\)的阶乘\(n!\)满足相等关系\begin{equation*}
	\Gamma(n+1)
	= n!
	\equiv 1 \cdot 2 \cdot 3 \dotsm n
	\quad(n\in\mathbb{N}).
	\eqno(1)
\end{equation*}
又假设函数\(\Gamma\)满足递推公式\begin{equation*}
	\Gamma(z+1) = z \cdot \Gamma(z)
	\quad(z\in\mathbb{R}),
	\eqno(2)
\end{equation*}
那么有\begin{equation*}
	\Gamma(z+n+1)
	= z(z+1)\dotsm(z+n) \cdot \Gamma(z),
\end{equation*}
即有\begin{equation*}
	\Gamma(z) = \frac{\Gamma(z+n+1)}{z(z+1)\dotsm(z+n)}.
	\eqno(3)
\end{equation*}
假设当\(n\to\infty\)时,
\(\Gamma(z+n+1)\)是\((z+n)!\)的等价无穷大,
即\begin{equation*}
	\lim_{n\to\infty} \frac{(z+n)!}{\Gamma(z+n+1)}
	= \lim_{n\to\infty} \frac{1 \cdot 2 \cdot 3 \dotsm n \cdot (n+1) \dotsm (n+z)}{\Gamma(z+n+1)}
	= 1;
\end{equation*}
那么对(3)式取极限就有\begin{align*}
	\Gamma(z)
	\equiv \lim_{n\to\infty} \Gamma(z)
	&= \lim_{n\to\infty}
		\frac{\Gamma(z+n+1)}{z(z+1)\dotsm(z+n)} \\
	&= \lim_{n\to\infty}
		\frac{\Gamma(z+n+1)}{z(z+1)\dotsm(z+n)} \cdot \frac{(z+n)!}{\Gamma(z+n+1)} \\
	&= \lim_{n\to\infty}
		\frac{1 \cdot 2 \cdot 3 \dotsm n}{z(z+1)\dotsm(z+n)} \cdot (n+1) \dotsm (n+z).
	\tag4
\end{align*}
又因为\begin{equation*}
	\lim_{n\to\infty} \frac{(n+1)(n+2)\dotsm(n+z)}{n^z} = 1,
\end{equation*}
所以(4)式化为\begin{equation}\label{equation:特殊函数.伽马函数的极限定义}
	\Gamma(z)
	= \lim_{n\to\infty} \frac{n^z \cdot n!}{z(z+1)\dotsm(z+n)}.
\end{equation}
像这样,我们就得到\(\Gamma\)函数的极限定义.

我们可以验证,根据上述极限定义,有\(\Gamma(1) = 0!\),即\begin{equation*}
	\Gamma(1)
	= \lim_{n\to\infty} \frac{n^1 \cdot 1 \cdot 2 \dotsm n}{1 \cdot 2 \dotsm n \cdot (n+1)}
	= \lim_{n\to\infty} \frac{n}{n+1}
	= 1.
\end{equation*}
而且从\begin{align*}
	\frac{\Gamma(z+1)}{\Gamma(z)}
	&= \left[ \lim_{n\to\infty} \frac{n^{z+1} \cdot n!}{(z+1)(z+2)\dotsm(z+n+1)} \right] \bigg/ \left[ \lim_{n\to\infty} \frac{n^z \cdot n!}{z(z+1)\dotsm(z+n)} \right] \\
	&= \lim_{n\to\infty} \left[ \frac{n^{z+1} \cdot n!}{(z+1)(z+2)\dotsm(z+n+1)} \bigg/ \frac{n^z \cdot n!}{z(z+1)\dotsm(z+n)} \right] \\
	&= \lim_{n\to\infty} \frac{n \cdot z}{z+n+1}
	= z
\end{align*}
也可以看出,函数\(\Gamma\)也确实满足递推公式(2),进而也满足阶乘关系(1).

\subsection{伽马函数的收敛性}
现在我们来讨论含参反常积分\begin{equation*}
	\Gamma\colon(0,+\infty)\to\mathbb{R},
	s \mapsto \int_0^{+\infty} t^{s-1} e^{-t} \dd{t}
\end{equation*}的收敛性问题.
这个积分的积分区间为\([0,+\infty)\),又当\(s-1<0\)时,\(t=0\)是被积函数的瑕点.
为此,分别讨论下列两个积分\begin{equation*}
	I_1 = \int_0^1 t^{s-1} e^{-t} \dd{t},
	\qquad
	I_2 = \int_1^{+\infty} t^{s-1} e^{-t} \dd{t}
\end{equation*}的收敛性.

当\(s \geq 1\)时,
\(I_1\)是定积分;
当\(0 < s < 1\)时,
因为\begin{equation*}
	t^{s-1} \cdot e^{-t}
	= \frac{1}{t^{1-s}} \cdot \frac{1}{e^t} < \frac{1}{t^{1-s}},
\end{equation*}
而\(1-s < 1\),
根据\hyperref[theorem:定积分.无界函数的反常积分的比较审敛法]{无界函数反常积分的比较审敛法},
反常积分\(I_1\)收敛.

又因为\begin{equation*}
	\lim_{t\to+\infty} t^2 (e^{-t} t^{s-1})
	= \lim_{t\to+\infty} \frac{t^{s+1}}{e^t} = 0,
\end{equation*}
根据\hyperref[theorem:定积分.无穷限反常积分的极限审敛法]{无穷限反常积分的极限审敛法},
反常积分\(I_2\)也收敛.

综上所述,反常积分\(\int_0^{+\infty} e^{-t} t^{s-1} \dd{t}\)对\(s > 0\)均收敛.

\subsection{伽马函数的定义}
\begin{definition}
定义\(\Gamma\)函数:
\begin{equation}\label{equation:特殊函数.伽马函数的积分定义}
	\Gamma(s)
	\defeq
	\int_0^{+\infty} t^{s-1} e^{-t} \dd{t}
	\quad(s>0)
\end{equation}
\end{definition}
有时候我们把含参反常积分 \labelcref{equation:特殊函数.伽马函数的积分定义}
称为\DefineConcept{欧拉第二型积分}.

\(\Gamma\)函数的图形如\cref{figure:定积分.伽马函数} 所示.

\begin{figure}%伽马函数
	\centering
	\begin{tikzpicture}
	%@Mathematica: Plot[Gamma[s], {s, 0, 5}, PlotRange -> {0, 6}]
		\begin{axis}[
			xmin=0,xmax=5.1,
			restrict y to domain=-6:6,
			axis lines=middle,
			xlabel=$x$,
			ylabel=$y$,
			enlarge x limits=0.05,
			enlarge y limits=0.1,
			x label style={at={(ticklabel* cs:1.00)}, inner sep=5pt, anchor=north},
			y label style={at={(ticklabel* cs:1.00)}, inner sep=2pt, anchor=south east},
		]
			\addplot[color=blue,samples=222,smooth,domain=0:5]{sqrt(2*pi)*x^(x-0.5)*exp(-x)*exp(1/(12*x))};
		\end{axis}
	\end{tikzpicture}
	\caption{\(\Gamma\)函数的图形}
	\label{figure:定积分.伽马函数}
\end{figure}

特别地,函数\begin{equation*}
	G(s,z) = \int_z^{+\infty} t^{s-1} e^{-t} \dd{t}
\end{equation*}
称为\DefineConcept{不完全伽马函数}.
函数\begin{equation*}
	G(s,z_0,z_1) = \int_{z_0}^{z_1} t^{s-1} e^{-t} \dd{t}
\end{equation*}
称为\DefineConcept{广义不完全伽马函数}.

\subsection{伽马函数与贝塔函数的关系}
\begin{theorem}
%@see: 《概率论与数理统计》(茆诗松、周纪芗、张日权) P89
对任意的\(a,b>0\),
有\begin{equation}\label{equation:定积分.伽马函数与贝塔函数的关系}
	B(a,b) = \frac{\Gamma(a) \Gamma(b)}{\Gamma(a+b)}.
\end{equation}
\begin{proof}
由伽马函数的定义可知\begin{align*}
	\Gamma(a) \Gamma(b)
	&= \int_0^{+\infty} \int_0^{+\infty} x^{a-1} y^{b-1} e^{-(x+y)} \dd{x}\dd{y} \\
	&\xlongequal{x=uv,y=u(1-v)}
	\int_0^{+\infty} \int_0^1 (uv)^{a-1} [u(1-v)]^{b-1} e^{-u} u \dd{x}\dd{y} \\
	&=\int_0^{+\infty} u^{a+b-1} e^{-u} \dd{u}
		\int_0^1 v^{a-1} (1-v)^{b-1} \dd{v} \\
	&= \Gamma(a+b) B(a,b).
	\qedhere
\end{align*}
\end{proof}
\end{theorem}

\subsection{伽马函数的性质}
\begin{property}
对于任意\(s > 0\),总有
\begin{equation}\label{equation:伽马函数.递推公式}
	\Gamma(s+1) = s~\Gamma(s).
\end{equation}
\begin{proof}
应用分部积分法,
有\begin{align*}
	\Gamma(s+1)
	&= \int_0^{+\infty} t^s e^{-t} \dd{t} \\
	&= -\int_0^{+\infty} t^s \dd(e^{-t}) \\
	&= \left[-t^s e^{-t}\right]_0^{+\infty}
		+ s \int_0^{+\infty} t^{s-1} e^{-t} \dd{t} \\
	&= s~\Gamma(s).
	\qedhere
\end{align*}
\end{proof}
\end{property}

\begin{property}
\(\Gamma(1) = 1\).
\begin{proof}
\(\Gamma(1)
= \int_0^{+\infty} e^{-x} \dd{x}
= \left(-e^{-x}\right)_0^{+\infty}
= 1\).
\end{proof}
\end{property}

\begin{property}
对于\(n\in\mathbb{N}\),
有\begin{equation}\label{equation:定积分.伽马函数与阶乘的联系}
	\Gamma(n+1) = n!.
\end{equation}
\begin{proof}
对\hyperref[equation:伽马函数.递推公式]{递推公式}运用数学归纳法即可.
\end{proof}
\end{property}

\begin{proposition}
\(\Gamma\)函数在\(s > 0\)时连续,且有各阶连续导数.
\end{proposition}
\begin{proposition}
伽马函数的导数为\begin{equation}
	\dv{s} \Gamma(s)
	= \int_0^{+\infty} t^{s-1} e^{-t} \ln t \dd{t}.
\end{equation}
\begin{proof}
由导数的定义有\begin{align*}
	\Gamma'(s)
	&= \lim_{h\to0} \frac{\Gamma(s+h)-\Gamma(s)}{h} \\
	&= \lim_{h\to0} \frac1h
		\left[ \int_0^{+\infty} t^{(s+h)-1} e^{-t} \dd{t}
		- \int_0^{+\infty} t^{s-1} e^{-t} \dd{t} \right] \\
	&= \lim_{h\to0}
		\int_0^{+\infty} t^{s-1} \frac{t^h - 1}{h} e^{-t} \dd{t}
		\tag{\hyperref[theorem:定积分.定积分性质1]{线性性}} \\
	&= \int_0^{+\infty} t^{s-1} \lim_{h\to0}\frac{t^h - 1}{h} e^{-t} \dd{t} \\ %TODO 这里是不是运用了什么性质才能交换
	%@see: [Taking the limit inside the integral is not always allowed](https://math.stackexchange.com/a/253697/591741)
	&= \int_0^{+\infty} t^{s-1} e^{-t} \ln t \dd{t}. %\cref{equation:函数极限.重要极限17}
	\qedhere
\end{align*}
\end{proof}
\end{proposition}

\begin{proposition}
\(\Gamma(s) > 0\ (s>0)\).
\end{proposition}

\begin{property}
\(\Gamma(s)\to+\infty\ (s\to0^+)\).
\begin{proof}
因为\begin{equation*}
	\Gamma(s) = \frac{\Gamma(s+1)}{s},
	\qquad
	\Gamma(1) = 1,
\end{equation*}
所以当\(s\to0^+\)时,
\(\Gamma(s)\to+\infty\).
\end{proof}
\end{property}

\begin{property}[余元公式]
\begin{equation}\label{equation:定积分.余元公式}
	\Gamma(s) \cdot \Gamma(1-s)
	= \frac{\pi}{\sin{\pi s}} \quad (0 < s < 1).
\end{equation}
\begin{proof}
由\cref{equation:定积分.伽马函数与贝塔函数的关系} 有\begin{equation*}
	\Gamma(s) \cdot \Gamma(1-s)
	= B(s,1-s)
	= \int_0^1 t^{s-1} (1-t)^{-s} \dd{t}
	= \int_0^1 \frac1t \left(\frac{t}{1-t}\right)^s \dd{t}.
\end{equation*}
令\(x=\frac{t}{1-t}\),
则\(t=\frac{x}{1+x}\),
\(\dd{t}=\frac1{(1+x)^2} \dd{x}\),
于是\begin{equation*}
	\Gamma(s) \cdot \Gamma(1-s)
	= \int_0^{+\infty} \frac{1+x}{x} \cdot x^s \cdot \frac1{(1+x)^2} \dd{x}
	= \int_0^{+\infty} \frac{x^{s-1}}{1+x} \dd{x}.
\end{equation*}
记\(I_1 = \int_0^1 \frac{x^{s-1}}{1+x} \dd{x},
I_2 = \int_1^{+\infty} \frac{x^{s-1}}{1+x} \dd{x}\),
则\(\Gamma(s) \cdot \Gamma(1-s) = I_1 + I_2\).

对于\(I_2 = \int_1^{+\infty} \frac{x^{s-1}}{1+x} \dd{x}\),
可令\(x = \frac1t\),
则\(I_2
= \int_0^1 \frac{t^{-s}}{1+t} \dd{t}
= \int_0^1 \frac{x^{-s}}{1+x} \dd{x}\).
于是\begin{align*}
	I_1 + I_2
	&= \int_0^1 \frac{x^{s-1} + x^{-s}}{1+x} \dd{x}
	= \int_0^1 (x^{s-1} + x^{-s}) \sum_{k=0}^\infty (-x)^k \dd{x} \\%FIXME: 这里用到了函数项级数,但是函数项级数在后续章节,尚未介绍!
	&= \int_0^1 \sum_{k=0}^\infty (-1)^k x^{k+s-1} \dd{x}
	+ \int_0^1 \sum_{k=0}^\infty (-1)^k x^{k-s} \dd{x} \\
	&= \sum_{k=0}^\infty \frac{(-1)^k}{k+s}
	+ \sum_{k=0}^\infty \frac{(-1)^k}{k-s+1} \\
	&= \frac1s
	+ \sum_{k=1}^\infty \frac{(-1)^k}{k+s}
	+ \sum_{k=1}^\infty \frac{(-1)^{k-1}}{k-s} \\
	&= \frac1s + \sum_{k=1}^\infty \frac{(-1)^k \cdot 2s}{s^2-k^2}.
\end{align*}

记\(f(x)=\cos(sx)\).
将\(f\)展开,得\begin{equation*}
	\cos(sx) = \frac{a_0}2 + \sum_{n=1}^\infty (a_n \cos(nx) + b_n \sin(nx)),
\end{equation*}
其中\begin{gather*}
	a_n = \frac1\pi \int_{-\pi}^\pi \cos(sx) \cos(nx) \dd{x}
	= \frac{(-1)^n \cdot 2s \cdot \sin(\pi s)}{\pi(s^2-n^2)}, \\
	b_n = \frac1\pi \int_{-\pi}^\pi \cos(sx) \sin(nx) \dd{x} = 0.
\end{gather*}
于是\begin{equation*}
	\cos(sx) = \frac{\sin(\pi s)}{\pi s}
	+ \sum_{n=1}^\infty \frac{(-1)^n \cdot 2s \cdot \sin(\pi s)}{\pi(s^2-n^2)} \cdot \cos(nx).
\end{equation*}
令\(x=0\),
得\(1 = \frac{\sin(\pi s)}{\pi s}
+ \frac{\sin(\pi s)}{\pi} \sum_{n=1}^\infty \frac{(-1)^n \cdot 2s}{s^2-n^2}\),
即\(\sum_{n=1}^\infty \frac{(-1)^n \cdot 2s}{s^2-n^2} = \frac{\pi}{\sin(\pi s)} - \frac1s\).

由上可知\begin{equation*}
	\Gamma(s) \cdot \Gamma(1-s)
	= \frac{\pi}{\sin(\pi s)}.
	\qedhere
\end{equation*}
%@see: 【余元公式的证明推导过程】https://www.bilibili.com/video/BV17M4m1m7ZH
\end{proof}
\end{property}

\begin{corollary}
\(\Gamma\left(\frac{1}{2}\right) = \sqrt{\pi}\).
\begin{proof}
在\cref{equation:定积分.余元公式} 中取\(s = \frac12\)即得.
\end{proof}
\end{corollary}

\begin{example}\label{example:反常积分.利用伽马函数计算高斯积分}
利用伽马函数计算高斯积分\(\int_0^{+\infty} e^{-u^2} \dd{u}\).
\begin{solution}
在\(\Gamma(s) = \int_0^{+\infty} e^{-x} x^{s-1} \dd{x}\)中,
作代换\(x = u^2\),
有\begin{equation}\label{equation:定积分.伽马函数.中间步骤1}
	\Gamma(s) = 2 \int_0^{+\infty} e^{-u^2} u^{2s-1} \dd{u}.
\end{equation}
再令\(2s-1 = t\)或\(s = \frac{1+t}{2}\),
即有\begin{equation}
	\int_0^{+\infty} e^{-u^2} u^t \dd{u}
	= \frac{1}{2} \Gamma\left(\frac{1+t}{2}\right)
	\quad (t > -1).
\end{equation}

在 \labelcref{equation:定积分.伽马函数.中间步骤1} 中,
令\(s=\frac12\),
得\begin{equation*}
	2\int_0^{+\infty} e^{-u^2} \dd{u}
	= \Gamma\left(\frac{1}{2}\right)
	= \sqrt{\pi},
\end{equation*}
于是我们算得\cref{example:定积分.高斯积分的收敛性} 中的高斯积分的值为\begin{equation}
	\int_0^{+\infty} e^{-u^2} \dd{u} = \frac{\sqrt{\pi}}{2}.
\end{equation}
\end{solution}
%\cref{example:定积分.高斯积分的收敛性}
%\cref{example:二重积分.利用二重积分计算高斯积分}
\end{example}

\begin{example}\label{example:伽马函数.伽马分布服从规范性的依据}
证明:\begin{equation}
	\int_0^{+\infty} x^{\alpha-1} e^{-\beta x} \dd{x}
	= \frac{\Gamma(\alpha)}{\beta^\alpha}.
\end{equation}
\begin{proof}
直接计算得
\begin{align*}
	\int_0^{+\infty} x^{\alpha-1} e^{-\beta x} \dd{x}
	&\xlongequal{u=\beta x}
		\int_0^{+\infty}
			\left(\frac{u}{\beta}\right)^{\alpha-1}
			e^{-u} \dd(\frac{u}{\beta}) \\
	&= \frac{1}{\beta^\alpha} \int_0^{+\infty} u^{\alpha-1} e^{-u} \dd{u}
	= \frac{\Gamma(\alpha)}{\beta^\alpha}.
	\qedhere
\end{align*}
\end{proof}
\end{example}

\begin{example}
证明:\begin{equation*}
	\Gamma\left(\frac{2k+1}{2}\right)
	= \sqrt{\pi} \cdot \frac{(2k-1)!!}{2^k},
\end{equation*}
其中\(k\in\mathbb{N}^+\).
\begin{proof}
由\cref{equation:伽马函数.递推公式} 有\begin{equation*}
	\Gamma\left(\frac{2k+1}{2}\right)
	= \Gamma\left(\frac{2k-1}{2}+1\right)
	= \frac{2k-1}{2} \cdot \Gamma\left(\frac{2k-1}{2}\right),
\end{equation*}
又因为\(\Gamma\left(\frac12\right) = \sqrt{\pi}\),
所以\begin{equation*}
	\Gamma\left(\frac{2k+1}{2}\right)
	= \Gamma\left(\frac12\right) \cdot \prod_{i=1}^k \frac{2i-1}{2}
	= \sqrt{\pi} \cdot \frac{(2k-1)!!}{2^k}.
	\qedhere
\end{equation*}
\end{proof}
\end{example}

\begin{example}
试证:\begin{equation}\label{equation:定积分.伽马函数与双阶乘的联系1}
	(2n)!! = 2^n \cdot \Gamma(n+1).
\end{equation}
\begin{proof}
因为\begin{align*}
	(2n)!!
	&= \prod_{k=1}^n 2k
		\tag{双阶乘的定义} \\
	&= 2^n \cdot \prod_{k=1}^n k \\
	&= 2^n \cdot n!
		\tag{阶乘的定义} \\
	&= 2^n \cdot \Gamma(n+1),
		\tag{\cref{equation:定积分.伽马函数与阶乘的联系}}
\end{align*}
所以\cref{equation:定积分.伽马函数与双阶乘的联系1} 对任意整数\(n\)都成立.
\end{proof}
\end{example}

\begin{example}
试证:\begin{equation}\label{equation:定积分.伽马函数与双阶乘的联系2}
	(2n-1)!! = \frac{\Gamma(2n)}{2^{n-1} \cdot \Gamma(n)}.
\end{equation}
\begin{proof}
由\cref{equation:定积分.伽马函数与阶乘的联系} 有,\begin{equation*}
	(2n)! = \Gamma(2n+1).
\end{equation*}
又因为\begin{align*}
	(2n)!
	&= \prod_{k=1}^{2n} k \\
	&= \left(\prod_{k=1}^n 2k\right)
		\cdot \left[\prod_{k=1}^n (2k-1)\right] \\
	&= (2k)!! \cdot (2n-1)!!
		\tag{双阶乘的定义} \\
	&= 2^n \cdot \Gamma(n+1) \cdot (2n-1)!!,
		\tag{\cref{equation:定积分.伽马函数与双阶乘的联系1}}
\end{align*}
所以\begin{equation*}
	\Gamma(2n+1)
	= 2^n \cdot \Gamma(n+1) \cdot (2n-1)!!,
\end{equation*}
因此\begin{equation*}
	(2n-1)!!
	= \frac{\Gamma(2n+1)}{2^n \cdot \Gamma(n+1)}.
\end{equation*}
也就是说,\cref{equation:定积分.伽马函数与双阶乘的联系2} 对任意整数\(n\)都成立.
\end{proof}
\end{example}

\begin{example}
计算定积分\(\int_0^1 x^m \ln^n x \dd{x}\ (n>-1)\).
\begin{solution}
直接计算得
\begin{align*}
	\int_0^1 x^m \ln^n x \dd{x}
	&\xlongequal{t=\ln x}
	\int_{-\infty}^0 e^{mt} t^n \cdot e^t \dd{t} \\
	&\xlongequal{t=-u}
	\int_{+\infty}^0 e^{-(m+1)u} (-u)^n \cdot (-1) \dd{u} \\
	&\xlongequal{s=(m+1)u}
	\int_0^{+\infty} e^{-s} \left(-\frac{s}{m+1}\right)^n \cdot \frac{1}{m+1} \dd{s} \\
	&=
	\frac{(-1)^n}{(m+1)^{n+1}} \int_0^{+\infty} e^{-s} s^n \dd{s} \\
	&=
	\frac{(-1)^n}{(m+1)^{n+1}} \cdot \Gamma(n+1).
\end{align*}
\end{solution}
\end{example}
我们从这个例子得到以下重要公式
\begin{equation}
	\int_0^1 x^m \ln^n x \dd{x}
	= \frac{(-1)^n}{(m+1)^{n+1}} \cdot \Gamma(n+1)
	\quad(n>-1).
\end{equation}
特别地,有
\begin{equation}
	\int_0^1 \ln^n x \dd{x}
	= (-1)^n \cdot \Gamma(n+1)
	\quad(n>-1).
\end{equation}

\begin{example}
计算极限\(\lim_{n\to\infty} \int_0^{+\infty} e^{-x^n} \dd{x}\).
\begin{solution}
利用换元法,有\begin{align*}
	\int_0^{+\infty} e^{-x^n} \dd{x}
	&\xlongequal{t=x^n}
	\int_0^{+\infty} e^{-t} \cdot \frac{1}{n} t^{\frac{1}{n}-1} \dd{t} \\
	&= \frac{1}{n} \Gamma\left(\frac{1}{n}\right)
	= \Gamma\left(1+\frac{1}{n}\right).
\end{align*}
又因为伽马函数在其定义域上是连续的,所以\begin{equation*}
	\lim_{n\to\infty} \int_0^{+\infty} e^{-x^n} \dd{x}
	= \lim_{n\to\infty} \Gamma\left(1+\frac{1}{n}\right)
	= \Gamma\left(1+\lim_{n\to\infty} \frac{1}{n}\right)
	= \Gamma(1) = 1.
\end{equation*}
\end{solution}
\end{example}

\subsection{勒让德倍量公式}
\begin{theorem}[勒让德倍量公式]
对于\(\forall n\in\mathbb{N}^+\),
总有\begin{equation}\label{equation:定积分.勒让德倍量公式}
\sqrt{\pi} \cdot \Gamma(2n)
= 2^{2n-1} \cdot \Gamma(n) \cdot \Gamma\left(n+\frac{1}{2}\right).
\end{equation}
\end{theorem}

\subsection{伽马函数的斯特林近似}
\begin{lemma}\label{theorem:定积分.伽马函数的斯特灵近似.引理1}
%@see: 《数学分析(第3册)》(徐森林) P229 引理15.4.2
%@see: 《数学分析教程(第3版 下册)》(史济怀) P392 引理18.4.1
对任意的\(x>0\),
有不等式\begin{equation}\label{equation:定积分.伽马函数的斯特灵近似.引理1}
	0 < \left(x+\frac{1}{2}\right) \ln(1+\frac{1}{x}) - 1
	< \frac{1}{12}\left(\frac{1}{x}-\frac{1}{x+1}\right).
\end{equation}
\begin{proof}
由于\begin{equation*}
	\ln\frac{1+x}{1-x}
	= 2 \sum_{k=0}^\infty \frac{x^{2k+1}}{2k+1}
	\quad(-1<x<1),
\end{equation*}
所以\begin{align*}
	\left(x+\frac{1}{2}\right) \ln(1+\frac{1}{x})
	&= \frac{2x+1}{2} \ln\frac{
		1+\frac{1}{2x+1}
	}{
		1-\frac{1}{2x+1}
	} \\
	&= \frac{2x+1}{2} \cdot
	2 \sum_{k=0}^\infty \frac{1}{2k+1} \left(\frac{1}{2x+1}\right)^{2k+1} \\
	&= \sum_{k=0}^\infty \frac{1}{2k+1} \left(\frac{1}{2x+1}\right)^{2k},
\end{align*}
从而\begin{equation*}
	\left(x+\frac{1}{2}\right) \ln(1+\frac{1}{x}) - 1
	= \frac{1}{3} \left(\frac{1}{2x+1}\right)^2
	+ \frac{1}{5} \left(\frac{1}{2x+1}\right)^4
	+ \dotsb.
\end{equation*}
显然上式(当\(x>0\)时)恒大于零;
另外,上式右边小于\begin{align*}
	&\hspace{-20pt}
	\frac{1}{3} \left(\frac{1}{2x+1}\right)^2 \left[
	1 + \left(\frac{1}{2x+1}\right)^2 + \left(\frac{1}{2x+1}\right)^4 + \dotsb
	\right] \\
	&= \frac{1}{3} \left(\frac{1}{2x+1}\right)^2 \left[
	1 - \left(\frac{1}{2x+1}\right)^2
	\right]^{-1}
	= \frac{1}{3} \frac{1}{4x^2 + 4x}
	= \frac{1}{12} \left(\frac{1}{x} - \frac{1}{x+1}\right).
	\qedhere
\end{align*}
\end{proof}
\end{lemma}

\begin{lemma}\label{theorem:定积分.伽马函数的斯特灵近似.引理2}
%@see: 《数学分析教程(第3版 下册)》(史济怀) P393 引理18.4.2
对任意的\(x>0\),
有\begin{equation}\label{equation:定积分.伽马函数的斯特灵近似.引理2}
	0 < \int_0^{+\infty} \frac{\floor{t} - t + 1/2}{t + x} \dd{x} < \frac{1}{12 x}.
\end{equation}
\begin{proof}
计算得\begin{align*}
	\int_0^{+\infty} \frac{\floor{t} - t + 1/2}{t + x} \dd{x}
	&= \sum_{k=0}^\infty \int_k^{k+1} \frac{k - t + 1/2}{t + x} \dd{t} \\
	&= \sum_{k=0}^\infty \left[
	\left(k + \frac{1}{2} + x\right) \ln(1 + \frac{1}{k + x}) - 1
	\right].
\end{align*}
在 \labelcref{equation:定积分.伽马函数的斯特灵近似.引理1} 中,
用\((k+x)\)代替\(x\),
则得\begin{align*}
	0 &< \int_0^{+\infty} \frac{\floor{t} - t + 1/2}{t + x} \dd{x} \\
	&< \frac{1}{12} \sum_{k=0}^\infty \left(\frac{1}{k+x}
		- \frac{1}{k+x+1}\right) \\
	&= \frac{1}{12} \left[
		\left(\frac{1}{x} - \frac{1}{x+1}\right)
		+ \left(\frac{1}{x+1} - \frac{1}{x+2}\right) + \dotsb
	\right]
	= \frac{1}{12x}.
	\qedhere
\end{align*}
\end{proof}
\end{lemma}

\begin{lemma}\label{theorem:定积分.伽马函数的斯特灵近似.引理3}
%@see: 《数学分析教程(第3版 下册)》(史济怀) P394 引理18.4.3
对任意的\(n\in\mathbb{N}^+\)和\(x>0\),
有\begin{equation}\label{equation:定积分.伽马函数的斯特灵近似.引理3}
	\begin{split}
		\int_0^n \frac{\floor{t} - t + 1/2}{t + x} \dd{t}
		&= \left(n+x+\frac{1}{2}\right) \ln(n+x) \\
		&\hspace{20pt}
		- \left(\frac{1}{2}+x\right) \ln x
		- \sum_{k=1}^n \ln(k+x) - n.
	\end{split}
\end{equation}
\begin{proof}
直接计算得\begin{align*}
	&\hspace{-5pt}
		\int_0^n \frac{\floor{t} - t + 1/2}{t + x} \dd{t}
		= \sum_{k=0}^{n-1} \int_k^{k+1} \left(
			\frac{k + 1/2 + x}{t + x} - 1
		\right) \dd{t} \\
	&= \sum_{k=0}^{n-1} \left[
			\left(k + \frac{1}{2} + x\right)
			\left(\ln\overline{k+1+x} - \ln\overline{k+x}\right)
			- 1
		\right] \\
	&= \sum_{k=0}^{n-1} \left(k + \frac{1}{2} + x\right) \ln(k+1+x)
		- \sum_{k=0}^{n-1} \left(k + \frac{1}{2} + x\right) \ln(k+x)
		- n \\
	&= \sum_{k=1}^n \left(k - \frac{1}{2} + x\right) \ln(k+x)
		- \sum_{k=0}^{n-1} \left(k + \frac{1}{2} + x\right) \ln(k+x)
		- n \\
	&= \sum_{k=1}^n \left(k + \frac{1}{2} + x\right) \ln(k+x)
		- \sum_{k=1}^n \ln(k+x)
		- \sum_{k=0}^{n-1} \left(k + \frac{1}{2} + x\right) \ln(k+x)
		- n \\
	&= \left(n + \frac{1}{2} + x\right) \ln(n+x)
		- \left(\frac{1}{2} + x\right) \ln x
		- \sum_{k=1}^n \ln(k+x)
		- n.
	\qedhere
\end{align*}
\end{proof}
\end{lemma}

\begin{lemma}\label{theorem:定积分.伽马函数的斯特灵近似.引理4}
%@see: 《数学分析教程(第3版 下册)》(史济怀) P395 引理18.4.4
\begin{equation}\label{equation:定积分.伽马函数的斯特灵近似.引理4}
	\lim_{n\to\infty} \left[
		\ln n! + n - \left(n+\frac{1}{2}\right) \ln n
	\right]
	= \ln\sqrt{2\pi}.
\end{equation}
\begin{proof}
在斯特林公式\begin{equation*}
	\lim_{n\to\infty} \frac{n!}{(n/e)^n \sqrt{2\pi n}} = 1
\end{equation*}的两边取对数,
即得 \labelcref{equation:定积分.伽马函数的斯特灵近似.引理4}.
\end{proof}
\end{lemma}

\begin{theorem}[\(\Gamma\)函数的斯特林近似公式]
%@see: 《数学分析(第3册)》(徐森林) P231 定理15.4.9
%@see: 《数学分析教程(第3版 下册)》(史济怀) P392 定理18.4.10
对于任意的\(x>0\),
存在\(\theta(x)\in(0,1)\),
使得\begin{equation}\label{equation:定积分.伽马函数的斯特林近似}
	\Gamma(x+1) = \sqrt{2\pi x}
		\left(\frac{x}{e}\right)^x e^{\frac{\theta(x)}{12x}}.
\end{equation}
\begin{proof}
因为\(\Gamma(x+1) = x \Gamma(x)\),
故可证\begin{equation*}
	\Gamma(x) = \frac{1}{x} \sqrt{2\pi x} \left(\frac{x}{e}\right)^x e^{\frac{\theta(x)}{12x}},
\end{equation*}
再两边取对数,
得\begin{equation*}
	\ln \Gamma(x) = \ln\sqrt{2\pi} + \left(x - \frac{1}{2}\right) \ln x - x + \frac{\theta(x)}{12x}.
\end{equation*}

由等式\begin{equation*}
	\ln\frac{n^x n!}{x(x+1)\dotsm(x+n)}
	= \ln n! + x \ln n - \sum_{k=0}^n \ln(k+x)
\end{equation*}
和\cref{theorem:定积分.伽马函数的斯特灵近似.引理3},
可得\begin{align*}
	&\hspace{-5pt}
	\ln\frac{n^x n!}{x(x+1)\dotsm(x+n)}
	- \int_0^n \frac{\floor{t} - t + 1/2}{t + x} \dd{t} \\
	&= \ln n! + x \ln n - \ln x - \left(n+x+\frac{1}{2}\right) \ln(n+x) + \left(\frac{1}{2}+x\right) \ln x + n \\
	&= \ln n! + n + x \ln n - \ln x
	- \left(n+x+\frac{1}{2}\right) \left[\ln n + \ln(1+\frac{x}{n})\right]
	+ \left(\frac{1}{2}+x\right) \ln x \\
	&= \ln n! + n - \left(n+\frac{1}{2}\right) \ln n
	- \left(n+x+\frac{1}{2}\right) \ln(1+\frac{x}{n})
	+ \left(x-\frac{1}{2}\right) \ln x.
\end{align*}
由\cref{theorem:定积分.伽马函数的斯特灵近似.引理4} 可知,
上式右边的前三项之和当\(n\to\infty\)时趋于\(\ln\sqrt{2\pi}\),
上式左边的第一项当\(n\to\infty\)时趋于\(\ln\Gamma(x)\),
而{\def\xn{\left(1+\frac{x}{n}\right)}%
\begin{align*}
	\lim_{n\to\infty} \left(n+x+\frac{1}{2}\right) \ln\xn
	&= \lim_{n\to\infty} \ln\xn^{n+x+\frac{1}{2}} \\
	&= \lim_{n\to\infty} \ln\xn^n \xn^{x+\frac{1}{2}}
	= x.
\end{align*}}
那么只要令\(n\to\infty\),
即得\begin{equation*}
	\ln\Gamma(x) - \int_0^{+\infty} \frac{\floor{t} - t + 1/2}{t + x} \dd{t}
	= \ln\sqrt{2\pi} - x + \left(x - \frac{1}{2}\right) \ln x.
\end{equation*}
若记\begin{equation*}
	\theta(x) = 12 x \int_0^{+\infty} \frac{\floor{t} - t + 1/2}{t + x} \dd{t},
\end{equation*}
则由\cref{theorem:定积分.伽马函数的斯特灵近似.引理3} 可知,
必有\(0<\theta(x)<1\),
如此便得要证的 \labelcref{equation:定积分.伽马函数的斯特林近似}.
\end{proof}
\end{theorem}
\begin{remark}
\cref{equation:定积分.伽马函数的斯特林近似} 说明,
\(\Gamma(x+1)\)与\(\sqrt{2\pi x} \left(\frac{x}{e}\right)^x\)是\(x\to+\infty\)时的等价无穷大.
\end{remark}

\begin{corollary}
%@see: 《数学分析教程(第3版 下册)》(史济怀) P396 推论18.4.1
对任意的实数\(a\),
有\begin{equation}\label{equation:反常积分.伽马函数.极限1}
	\lim_{x\to+\infty} \frac{x^a \Gamma(x)}{\Gamma(x+a)} = 1.
\end{equation}
%TODO proof
\end{corollary}

\section{本章总结}

我们在本章学习了无穷限的反常积分和无界函数的反常积分这两类反常积分的基本概念%
(\cref{definition:定积分.无穷限的反常积分的定义1,%
definition:定积分.无穷限的反常积分的定义3%
},以及\cref{definition:定积分.无界函数的反常积分的定义1}).
它们是对常义积分的定义的扩展.

我们可以利用莱布尼茨公式
\labelcref{equation:定积分.利用牛顿莱布尼茨公式计算无穷限的反常积分1,%
equation:定积分.利用牛顿莱布尼茨公式计算无穷限的反常积分2,%
equation:定积分.利用牛顿莱布尼茨公式计算无穷限的反常积分3,%
equation:定积分.利用牛顿莱布尼茨公式计算无界函数的反常积分1,%
equation:定积分.利用牛顿莱布尼茨公式计算无界函数的反常积分2%
}
计算反常积分.

\begin{table}[hb]
	\centering
	\begin{tblr}{*2cl}
		\hline
		名称 & 表达式 & 敛散条件 \\
		\hline
		{\hyperref[example:定积分.p积分]{p 积分}}
			& \(\int_a^{+\infty} \frac{\dd{x}}{x^p}\ (a>0)\)
			& 当\(p > 1\)时收敛于\(\frac{1}{p-1} a^{1-p}\),当\(p \leq 1\)时发散 \\
		{\hyperref[example:定积分.q积分]{q 积分}}
			& \(\int_a^b \frac{\dd{x}}{(x-a)^q}\ (a<b)\)
			& 当\(0 < q < 1\)时收敛于\(\frac{1}{1-q} (b-a)^{1-q}\),当\(q \geq 1\)时发散 \\
		& \(\int_0^{+\infty} e^{-ax} \dd{x}\) %\cref{example:反常积分.重要反常积分公式1}
			& 当\(a>0\)时收敛于\(\frac1a\),当\(a\leq0\)时发散 \\
		%@see: https://www.bilibili.com/video/BV1pgsteTEBB
		%@see-comment: 这个视频有点问题!反常积分\(\int_0^{+\infty} \frac{\dd{x}}{1-x^n}\)发散,除非明说是柯西主值积分,否则没有意义!
		%TODO 反常积分\(\int_0^{+\infty} \frac{\dd{x}}{1+x^n}\)的收敛性还没有讨论,以后有空再补完吧.
		%@Mathematica: Integrate[1/(1 + x^n), {x, 0, +Infinity}, Assumptions -> {n > 1}]
		& \(\int_0^{+\infty} \frac{\dd{x}}{1+x^n}\)
			& \(n>1\)时收敛于\(\frac\pi{n} \csc\frac\pi{n}\) \\
		\hline
	\end{tblr}
	\caption{重要反常积分及其敛散条件}
\end{table}

常见反常积分:
\begin{gather*}
	%\cref{example:反常积分.指数函数与三角函数之积的反常积分}
	\int_0^{+\infty} e^{-ax} \cos bx \dd{x} = \frac{a}{a^2+b^2}. \\
	\int_0^{+\infty} e^{-ax} \sin bx \dd{x} = \frac{b}{a^2+b^2}.
\end{gather*}

在本章我们还学习了一个特殊函数:
\hyperref[equation:特殊函数.伽马函数的积分定义]{\(\Gamma\)函数}\begin{equation*}
%\cref{equation:特殊函数.伽马函数的积分定义}
	\Gamma(s)
	\defeq
	\int_0^{+\infty} t^{s-1} e^{-t} \dd{t}
	\quad(s>0).
\end{equation*}
它具有几个重要性质:\begin{itemize}
	%\cref{equation:伽马函数.递推公式}
	\item 对于任意\(s > 0\),总有\(\Gamma(s+1) = s~\Gamma(s)\).
	\item \(\Gamma(1) = 1\).
	%\cref{equation:定积分.伽马函数与阶乘的联系}
	\item \(\Gamma(n+1) = n!\ (n\in\mathbb{N})\).
	%\cref{equation:定积分.余元公式}
	\item \(\Gamma(s) \cdot \Gamma(1-s) = \frac{\pi}{\sin{\pi s}} \quad (0 < s < 1)\).
	\item \(\Gamma\left(1/2\right) = \sqrt{\pi}\).
	%\cref{equation:定积分.伽马函数与双阶乘的联系1}
	\item \((2n)!! = 2^n \cdot \Gamma(n+1)\).
	%\cref{equation:定积分.伽马函数与双阶乘的联系2}
	\item \((2n-1)!! = \frac{\Gamma(2n)}{2^{n-1} \cdot \Gamma(n)}\).
\end{itemize}


\chapter{微分方程概论}
函数是客观事物的内部联系在数量方面的反映.
利用函数关系又可以对客观事物的规律性进行研究.
因此如何寻求函数关系,在实践中具有重要的意义.
在许多问题中,往往不能直接找出所需要的函数关系,
但是根据问题所提供的情况,有时可以列出含有要找的函数及其导数的关系式.
这样的关系式就是所谓微分方程.
微分方程建立以后,对它进行研究,找出未知函数来,这就是解微分方程.
本章主要介绍微分方程的一些基本概念和几种常用的微分方程的解法.

\section{微分方程的基本概念}
\subsection{微分方程的基本概念}
一般地,含有自变量、未知函数、未知函数的导数(或偏导数)的方程,
叫做\DefineConcept{微分方程}(differential equation).

如果未知函数是一元函数,
那么我们把这个微分方程称为\DefineConcept{常微分方程}%
(ordinary differential equation).

如果未知函数是多元函数,
那么我们把这个微分方程称为\DefineConcept{偏微分方程}%
(partial differential equation).

未知函数的最高阶导数(或偏导数)的阶数,
称为“微分方程的\DefineConcept{阶}(order)”.

本章主要研究常微分方程.
它的一般形式为:
\begin{equation}\label[differential-equation]{equation:微分方程.微分方程的一般形式}
	F\left( x,y,y',y'',\dotsc,y^{(n)} \right)=0.
\end{equation}
根据定义,
\cref{equation:微分方程.微分方程的一般形式} 中所出现的未知函数
\(y = y(x)\)的最高阶导数\(y^{(n)}\)的阶数\(n\),
就是\cref{equation:微分方程.微分方程的一般形式} 的阶.

如果\cref{equation:微分方程.微分方程的一般形式} 左边是
关于未知函数\(y\)及其导数\(y',y'',\dotsc,y^{(n)}\)的一次有理整式,那么称其为
“\(n\)阶\DefineConcept{线性常微分方程}%
(n-th order linear ordinary differential equation)”;
否则称其为
“\(n\)阶\DefineConcept{非线性常微分方程}%
(n-th order nonlinear ordinary differential equation)”.

如果能从方程 \labelcref{equation:微分方程.微分方程的一般形式} 中解出最高阶导数,则可得微分方程
\begin{equation}\label[differential-equation]{equation:微分方程.分离出最高阶导数}
	y^{(n)} = f\left( x,y,y',y'',\dotsc,y^{(n-1)} \right).
\end{equation}
像\cref{equation:微分方程.分离出最高阶导数} 这样的微分方程称为
“\(n\)阶\DefineConcept{显式常微分方程}%
(n-th order explicit ordinary differential equation)”.
本章讨论的微分方程都是显式常微分方程.

\(n\)阶线性常微分方程的一般形式为
\begin{equation}\label[differential-equation]{equation:微分方程.线性常微分方程的一般形式}
	y^{(n)} + a_1 \cdot y^{(n-1)} + \dotsb + a_{n-1} \cdot y' + a_n \cdot y = f,
\end{equation}
其中\(a_i\ (i=1,2,\dotsc,n)\)和\(f\)是已知函数.
函数\(f\)称为“\cref{equation:微分方程.线性常微分方程的一般形式}
的\DefineConcept{非齐次项}(non-homogeneous term)”.
当\(f(x) \equiv 0\)时,我们称\cref{equation:微分方程.线性常微分方程的一般形式} 为
“\(n\)阶\DefineConcept{线性齐次常微分方程}%
(n-th order linear homogeneous ordinary differential equation)”;
否则,称其为“\(n\)阶\DefineConcept{线性非齐次常微分方程}%
(n-th order linear non-homogeneous ordinary differential equation)”.

当我们需要求解的未知函数是\(\AutoTuple{y}{k}\),且它们都只依赖于一个自变量\(x\)时,
我们要给出关于\(k\)个未知函数的常微分方程,联立为方程组,
于是得到如下的\(n = \max\{\AutoTuple{n}{k}\}\)阶\DefineConcept{常微分方程组}的一般形式:\begin{equation*}
	\def\y#1{y_{#1},y_{#1}',\dotsc,y_{#1}^{(n_#1)}}%
	F_i\left(x,\y{1},\y{2},\dotsc,\y{k}\right)=0
	\quad(i=1,2,\dotsc,k).
\end{equation*}

\subsection{微分方程的解}
\begin{definition}
设函数\(\phi\)在区间\(I\subseteq\mathbb{R}\)上有\(n\)阶连续导数.
如果对于\(\forall x \in I\)有\begin{equation*}
	F\left[x,\phi(x),\phi'(x),\dotsc,\phi^{(n)}(x)\right]\equiv0,
\end{equation*}
那么称函数\(\phi\)为
“\cref{equation:微分方程.微分方程的一般形式}
在区间\(I\)上的\DefineConcept{解}(solution)”.

如果\(n\)阶微分方程的解中含有任意常数\(\AutoTuple{C}{n}\),
且任意常数的个数与微分方程的阶数相同\footnote{%
这里所说的任意常数是相互\DefineConcept{独立的}(independent),
就是说,它们不能合并而使得任意常数的个数减少;
或者说,\(\phi,\phi',\dotsc,\phi^{(n-1)}\)
关于\(\AutoTuple{C}{n}\)的雅克比行列式\begin{equation*}
	\jacobi{\phi,\phi',\dotsc,\phi^{(n-1)}}{\AutoTuple{C}{n}}
\end{equation*}不等于零.},
这样的解\begin{equation*}
	y = \phi(x;\AutoTuple{C}{n})
\end{equation*}叫做微分方程的\DefineConcept{通解}(general solution);
否则称其为微分方程的\DefineConcept{特解}(special solution).
\end{definition}

由于通解中含有任意常数,
所以它还不能完全确定地反映某一客观事物的规律性.
要完全确定地反映客观事物的规律性,
必须确定这些常数的值.
为此,要根据问题的实际情况,提出确定这些常数的条件.

如果微分方程是一阶的,
通常用来确定任意常数的条件是\begin{equation*}
	y(x_0) = y_0,
\end{equation*}
其中\(x_0,y_0\)都是给定的值;
如果微分方程是二阶的,
通常用来确定任意常数的条件是\begin{equation*}
	y(x_0) = y_0,
	\qquad
	y'(x_0) = y'_0,
\end{equation*}
其中\(x_0,y_0,y'_0\)都是给定的值;
以此类推.上述这种条件叫做\DefineConcept{初始条件}.
确定了通解中的任意常数以后,就得到微分方程的\DefineConcept{特解}.

特别地,求一阶微分方程\(y'=f(x,y)\)满足初始条件\(y(x_0) = y_0\)的特解这样一个问题,
叫做一阶微分方程的\DefineConcept{初值问题},记作\begin{equation*}
	\left\{ \begin{array}{l}
		y' = f(x,y), \\
		y(x_0) = y_0.
	\end{array} \right.
\end{equation*}

微分方程\(y'=f(x,y)\)的任意特解在平面上表现为一条曲线,
叫做“微分方程\(y'=f(x,y)\)的\DefineConcept{积分曲线}(integral curve)”
或“微分方程\(y'=f(x,y)\)的\DefineConcept{解曲线}(solution curve)”.
同样地,微分方程\(y'=f(x,y)\)的通解在平面上表现为一族曲线,
叫做“微分方程\(y'=f(x,y)\)的\DefineConcept{积分曲线族}(family of integral curves)”
或“微分方程\(y'=f(x,y)\)的\DefineConcept{解曲线族}(family of solution curves)”.

一阶微分方程的初值问题\begin{equation*}
	\left\{ \begin{array}{l}
		y' = f(x,y), \\
		y(x_0) = y_0
	\end{array} \right.
\end{equation*}的几何意义,
就是求通过点\((x_0,y_0)\)的那条积分曲线.

二阶微分方程的初值问题\begin{equation*}
	\left\{ \begin{array}{l}
		y'' = f(x,y,y'), \\
		y(x_0) = y_0, \\
		y'(x_0) = y'_0
	\end{array} \right.
\end{equation*}的几何意义,
是求微分方程的通过点\((x_0,y_0)\)
且在点\((x_0,y_0)\)的切线斜率为\(y'_0\)的那条积分曲线.

\begin{example}
%@see: 《高等数学(第六版 下册)》 P297 例3
试验证:函数\begin{equation*}
	x = C_1 \cos kt + C_2 \sin kt
\end{equation*}是微分方程\begin{equation*}
	\dv[2]{x}{t} + k^2 x = 0
\end{equation*}的解.
\begin{solution}
求出函数\(x(t)\)的导数,得\begin{gather*}
	\dv{x}{t} = -k C_1 \sin kt + k C_2 \cos kt, \\
	\dv[2]{x}{t} = -k^2 C_1 \cos kt - k^2 C_2 \sin kt.
\end{gather*}
代入微分方程得\begin{equation*}
	-k^2 (C_1 \cos kt + C_2 \sin kt) + k^2 (C_1 \cos kt + C_2 \sin kt)
	\equiv 0.
\end{equation*}
函数\(t \mapsto x(t)\)代入微分方程后成为一个恒等式,因此这个函数就是微分方程的解.
\end{solution}
\end{example}

\section{微分方程的解的存在性和唯一性}
\subsection{微分方程的解的存在性}
\begin{definition}\label{definition:微分方程.函数系的一致有界性}
设函数系\(S\)是由定义在某个区间\(D = [a,b]\)上的一些函数组成的集合.
如果\begin{equation*}
	(\exists M>0)
	(\forall x \in D)
	(\forall F \in S)
	[\abs{F(x)} < M],
\end{equation*}
则称“函数系\(S\)在区间\(D\)上是\DefineConcept{一致有界的}(uniformly bounded)”.
%@see: \cref{definition:微分方程.函数系的一致有界性}
\end{definition}

\begin{definition}\label{definition:微分方程.函数系的等度连续性}
设函数系\(S\)是由定义在某个区间\(D = [a,b]\)上的一些函数组成的集合.
如果\begin{equation*}
	(\forall\epsilon>0)
	(\exists\delta>0)
	(\forall F \in S)
	(\forall x_1,x_2 \in D)
	[\abs{x_1-x_2} < \delta \implies \abs{F(x_1)-F(x_2)} < \epsilon],
\end{equation*}
%@see: https://mathworld.wolfram.com/Equicontinuous.html
则称“函数系\(S\)在区间\(D\)上是\DefineConcept{等度连续的}(equicontinuous)%
\footnote{尽管有很多相似之处,但是一定注意“等度连续”与“一致连续”的区别:
“等度连续”描述的对象是函数系,
“一致连续”描述的对象是函数、函数列或函数项级数
(见\cref{definition:极限.函数的一致连续性} 和
\cref{definition:无穷级数.函数项级数的一致收敛性}).}”;
称\begin{equation*}
	\sup_{F_1,F_2 \in S} \abs{F_1(x)-F_2(x)}
\end{equation*}为“函数系\(S\)在区间\([a,b]\)上的\DefineConcept{宽度}%
\footnote{注意与\hyperref[definition:极限.函数在集合上的振幅]{函数在集合上的振幅}相区别.}”.
\end{definition}

\begin{lemma}[阿斯科拉--阿尔泽拉引理]\label{theorem:微分方程概论.阿斯科拉--阿尔泽拉引理}
任何在区间\([a,b]\)上一致有界且等度连续的函数系\(S\)都包含在此区间上一致收敛的函数列.
\end{lemma}

\begin{theorem}
设函数\(f(x,y)\)在\(Oxy\)平面的某个有界闭区域\(D\)上连续,
那么对于\(D\)中任意一个内点\((x_0,y_0)\),
总存在着函数\(y = \phi(x)\),在点\(x_0\)的某个邻域内满足微分方程\begin{equation*}
	\dv{y}{x} = f(x,y),
\end{equation*}且同时有\(y_0 = \phi(x_0)\).
\end{theorem}

\subsection{微分方程的解的唯一性}
\begin{theorem}
设函数\(f(x,y)\)在区域\(D\)上连续,
且满足“利普希茨条件”,即\begin{equation*}
	(\exists k>0)
	[
		\abs{f(x,y_1) - f(x,y_2)}
		\leq
		k \abs{y_1 - y_2}
	],
\end{equation*}
那么对于\(D\)中任意一个内点\((x_0,y_0)\),
存在唯一的函数\(\phi\)满足微分方程\begin{equation*}
	\dv{y}{x} = f(x,y),
\end{equation*}
且同时有\(y_0 = \phi(x_0)\).
\end{theorem}

\section{可分离变量的微分方程}
我们首先讨论一阶微分方程\begin{equation}\label[first-order-ordinary-differential-equation]{equation:微分方程.一阶微分方程的一般形式}
%@see: 《高等数学(第六版 上册)》 P299 (1)
	y' = f(x,y)
\end{equation}的一些解法.

\cref{equation:微分方程.一阶微分方程的一般形式} 有时也写成如下的
对称形式\begin{equation}\label[differential-equation]{equation:微分方程.一阶微分方程的对称形式}
%@see: 《高等数学(第六版 上册)》 P299 (2)
	P(x,y) \dd{x} + Q(x,y) \dd{y} = 0.
\end{equation}
在\cref{equation:微分方程.一阶微分方程的对称形式} 中,
变量\(x\)与\(y\)对称,
它既可看作是以\(x\)为自变量、\(y\)为因变量的方程\begin{equation*}
	\dv{y}{x} = -\frac{P(x,y)}{Q(x,y)},
	\quad Q(x,y)\neq0,
\end{equation*}
也可看作是以\(y\)为自变量、\(x\)为因变量的方程\begin{equation*}
	\dv{x}{y} = -\frac{Q(x,y)}{P(x,y)},
	\quad P(x,y)\neq0.
\end{equation*}

对于有的一阶微分方程(例如\(\dv{y}{x} = 2x\))可以直接在两端积分得到它的通解.
但是并不是所有的一阶微分方程都能这样求解,
例如,对于一阶微分方程\begin{equation*}
%@see: 《高等数学(第六版 上册)》 P299 (3)
	\dv{y}{x} = 2xy^2
\end{equation*}就不能这样做,
这是因为这个方程的右端含有与\(x\)存在函数关系的变量\(y\),
积分\(\int 2xy^2 \dd{x}\)求不出来.
但如果将这个方程变形为\(\frac{\dd{y}}{y^2}=2x\dd{x}\),
使得变量\(x\)与\(y\)分离在等式的两端,
再两端积分即得\begin{equation*}
%@see: 《高等数学(第六版 上册)》 P299 (4)
	-\frac{1}{y} = x^2+C
	\quad\text{或}\quad
	y = -\frac{1}{x^2+C},
\end{equation*}
其中\(C\)是任意常数.
可以验证上述函数确实满足原来的一阶微分方程\(\dv{y}{x} = 2xy^2\),
且含有一个任意常数,所以它是原方程的通解.

\begin{definition}
如果\cref{equation:微分方程.一阶微分方程的一般形式}
能化成\begin{equation}\label[first-order-ordinary-differential-equation]{equation:微分方程.一阶微分方程的分离变量形式}
%@see: 《高等数学(第六版 上册)》 P300 (5)
	g(y) \dd{y} = f(x) \dd{x}
\end{equation}的形式,
那么称“\cref{equation:微分方程.一阶微分方程的一般形式}
是一个\DefineConcept{可分离变量的微分方程}”.
\end{definition}

假定\cref{equation:微分方程.一阶微分方程的分离变量形式} 中的函数\(g(y)\)和\(f(x)\)是连续的.
设\(y=\phi(x)\)是\cref{equation:微分方程.一阶微分方程的分离变量形式} 的解,
将它代入\cref{equation:微分方程.一阶微分方程的分离变量形式} 中得到恒等式\begin{equation*}
	g[\phi(x)] \phi'(x) \dd{x} = f(x) \dd{x}.
\end{equation*}
将上式两端积分,并由\(y=\phi(x)\)引进变量\(y\),得\begin{equation*}
	\int g(y) \dd{y} = \int f(x) \dd{x}.
\end{equation*}
设\(G(y)\)及\(F(x)\)依次为\(g(y)\)及\(f(x)\)的原函数,
于是有\begin{equation}\label{equation:微分方程.一阶微分方程的分离变量形式.隐式解}
%@see: 《高等数学(第六版 上册)》 P300 (6)
	G(y) = F(x) + C.
\end{equation}
因此\cref{equation:微分方程.一阶微分方程的分离变量形式} 的解
满足\cref{equation:微分方程.一阶微分方程的分离变量形式.隐式解} .
反之,如果\(y = \Phi(x)\)是
由\cref{equation:微分方程.一阶微分方程的分离变量形式.隐式解} 所确定的隐函数,
那么在\(g(y)\neq0\)的条件下,
\(y=\Phi(x)\)也是\cref{equation:微分方程.一阶微分方程的分离变量形式} 的解,
事实上,由\hyperref[theorem:多元函数微分法.隐函数存在定理1]{隐函数的求导法}可知,
当\(g(y)\neq0\)时,\begin{equation*}
	\Phi'(x) = \frac{F'(x)}{G'(y)} = \frac{f(x)}{g(y)},
\end{equation*}
这就表示函数\(y = \Phi(x)\)满足\cref{equation:微分方程.一阶微分方程的分离变量形式}.
所以,如果已分离变量的\cref{equation:微分方程.一阶微分方程的分离变量形式} 中,
\(g(y)\)和\(f(x)\)是连续的,且\(g(y)\neq0\),
那么\cref{equation:微分方程.一阶微分方程的分离变量形式}
两端积分后得到的\cref{equation:微分方程.一阶微分方程的分离变量形式.隐式解} ,
就用隐式给出了\cref{equation:微分方程.一阶微分方程的分离变量形式} 的解,
\cref{equation:微分方程.一阶微分方程的分离变量形式.隐式解} 就叫做
“\cref{equation:微分方程.一阶微分方程的分离变量形式} 的\DefineConcept{隐式解}”.
又由于\cref{equation:微分方程.一阶微分方程的分离变量形式.隐式解} 中含有任意常数,
因此\cref{equation:微分方程.一阶微分方程的分离变量形式.隐式解} 所确定的隐函数
是\cref{equation:微分方程.一阶微分方程的分离变量形式} 的通解,
所以\cref{equation:微分方程.一阶微分方程的分离变量形式.隐式解} 叫做
“\cref{equation:微分方程.一阶微分方程的分离变量形式} 的\DefineConcept{隐式通解}”.
当\(f(x)\neq0\)时,\cref{equation:微分方程.一阶微分方程的分离变量形式.隐式解}
所确定的隐函数\(x=\Psi(y)\)也可认为是\cref{equation:微分方程.一阶微分方程的分离变量形式} 的解.

\begin{example}
%@see: 《高等数学(第六版 上册)》 P301 例2
放射性元素铀由于不断地有原子放射出微粒子而变成其他元素,铀的含量就不断减少,这种现象叫衰变.
由原子物理学知道,铀的衰变速度与当时未衰变的铀原子的含量\(M\)成正比.
已知\(t=0\)时,铀的含量为\(M_0\),求在衰变过程中铀含量\(M(t)\)随时间\(t\)变化的规律.
\begin{solution}
铀的衰变速度就是\(M(t)\)对时间\(t\)的导数\(\dv{M}{t}\).
由于铀的衰变速度与其含量成正比,故得微分方程\begin{equation*}
%@see: 《高等数学(第六版 上册)》 P301 (8)
	\dv{M}{t} = -\lambda M,
\end{equation*}
其中\(\lambda > 0\)是常数,叫做衰变系数,
\(\lambda\)前置负号是由于当\(t\)增加时\(M\)单调减少,
即\(\dv{M}{t} < 0\)的缘故.对这个微分方程分离变量,得\begin{equation*}
	\frac{\dd{M}}{M} = -\lambda \dd{t}.
\end{equation*}两端积分\begin{equation*}
	\int \frac{\dd{M}}{M} = -\lambda \int \dd{t},
\end{equation*}
以\(\ln C\)表示任意常数,考虑到\(M>0\),得\begin{equation*}
	\ln M = -\lambda t + \ln C,
\end{equation*}
即\begin{equation*}
	M = C e^{-\lambda t}.
\end{equation*}
这就是原方程的通解.将初始条件\begin{equation*}
	\eval{M}_{t=0} = M_0
\end{equation*}
代入上式,得\begin{equation*}
	M_0 = C e^0 = C,
\end{equation*}
所以\(M = M_0 e^{-\lambda t}\).
这就是所求铀的衰变规律 --- 铀的含量随时间的增加而按指数规律衰减.
\end{solution}
\end{example}

\begin{example}
%@see: 《高等数学(第六版 上册)》 P315 习题7-4 6.
验证形如\begin{equation*}
	y f(xy) \dd{x} + x g(xy) \dd{y} = 0
\end{equation*}的微分方程,
可经变量代换\(v = xy\)化为可分离变量的方程,并求出其通解.
\begin{solution}
令\(v = xy\),则\(y = v/x\),
\(\dd{y} = \frac{x \dd{v} - v \dd{x}}{x^2}\),
上述微分方程化为\begin{equation*}
	\frac{v}{x} f(v) \dd{x} + x g(v) \frac{x \dd{v} - v \dd{x}}{x^2} = 0,
\end{equation*}
整理得\begin{equation*}
	\frac{\dd{x}}{x}
	= \frac{g(v) \dd{v}}{v (g(v) - f(v))}.
\end{equation*}
积分得\begin{equation*}
	\ln\abs{x}
	= \int \frac{g(v)}{v (g(v) - f(v))} \dd{v} + C_1,
\end{equation*}
即\begin{equation*}
	x = C \exp\int \frac{g(v)}{v (g(v) - f(v))} \dd{v}.
\end{equation*}
最后代回\(v = xy\)即可.
\end{solution}
\end{example}

\begin{example}
%@see: 《2024年全国硕士研究生入学统一考试(数学一)》二填空题/第14题
求微分方程\(y'=\frac1{(x+y)^2}\)满足条件\(y(1)=0\)的解.
\begin{solution}
令\(x+y=u\),
对\(x\)求导得\(u'=1+y'\),
代入原方程并整理得\(\dv{u}{x}=\frac{1+u^2}{u^2}\),
分离变量得\begin{equation*}
	\left( 1-\frac1{1+u^2} \right) \dd{u}
	= \dd{x},
\end{equation*}
积分得\(u-\arctan u=x+C\ (\text{$C$是常数})\),
用\(x+y\)代\(u\)得\(y-\arctan(x+y)=C\).
代入初始条件可得\(C=-\frac\pi4\),
于是原方程的解为\(x=\tan\left(y+\frac\pi4\right)-y\).
\end{solution}
\end{example}

\section{齐次方程}
有的一阶线性齐次微分方程可以化成\begin{equation}\label[differential-equation]{equation:微分方程.一阶齐次方程}
%@see: 《高等数学(第六版 上册)》 P305 (1)
	\dv{y}{x}=\phi\left(\frac{y}{x}\right)
\end{equation}的形式.

在解这类方程的时候,我们可以使用换元法,
在方程中引入新的未知函数\begin{equation}\label[differential-equation]{equation:微分方程.一阶齐次方程.中间步骤1}
%@see: 《高等数学(第六版 上册)》 P305 (2)
	u=\frac{y}{x},
\end{equation}
就可以将其化为可分离变量的方程.
这是因为由\cref{equation:微分方程.一阶齐次方程.中间步骤1} 有\begin{equation*}
	y = ux,
	\qquad
	\dv{y}{x} = u + x \dv{u}{x},
\end{equation*}
代入\cref{equation:微分方程.一阶齐次方程},便得方程\begin{equation*}
	u + x \dv{u}{x} = \phi(u),
\end{equation*}
即\begin{equation*}
	x \dv{u}{x} = \phi(u) - u.
\end{equation*}
分离变量,得\begin{equation*}
	\frac{\dd{u}}{\phi(u) - u} = \frac{\dd{x}}{x}.
\end{equation*}
对两端积分,得\begin{equation*}
	\int \frac{\dd{u}}{\phi(u) - u} = \int \frac{\dd{x}}{x}.
\end{equation*}
求出积分后,再以\(\frac{y}{x}\)代替\(u\),便得齐次方程的通解.

\begin{example}
解方程\begin{equation*}y^2+x^2 \dv{y}{x} = xy \dv{y}{x}.\end{equation*}
\begin{solution}
原方程可写成\begin{equation*}
	\dv{y}{x} = \frac{y^2}{xy-x^2} = \frac{(y/x)^2}{y/x-1},
\end{equation*}
因此原方程是齐次方程.
令\(u=y/x\),
则\begin{equation*}
	y=ux, \qquad \dv{y}{x} = u + x \dv{u}{x},
\end{equation*}
于是原方程变为\begin{equation*}
	u+x \dv{u}{x} = \frac{u^2}{u-1},
\end{equation*}
即\begin{equation*}
	x \dv{u}{x} = \frac{u}{u-1}.
\end{equation*}
分离变量,得\begin{equation*}
	\left(1-\frac{1}{u}\right) \dd{u} = \frac{\dd{x}}{x}.
\end{equation*}
两端积分,得\begin{equation*}
	u - \ln\abs{u} + C = \ln\abs{x},
\end{equation*}或\begin{equation*}
	\ln\abs{xu} = u + C.
\end{equation*}
以\(y/x\)代上式中的\(u\),
便得方程的通解为\begin{equation*}
	\ln\abs{y} = \frac{y}{x} + C.
\end{equation*}
\end{solution}
\end{example}

\begin{example}
设曲线\(y = y(x)\ (x>0)\)经过点\((1,2)\),
该曲线上任一点\(P(x,y)\)到\(y\)轴的距离等于该点处的切线在\(y\)轴上的截距.
求曲线方程\(y(x)\).
\begin{solution}
设\((x,y)\)处的切线方程为\(Y-y=y'(X-x)\),
令\(X=0\)得\(Y=y-y'x\),
这就是切线在\(y\)轴上的截距,
因此得到微分方程\(x = y-y'x\).
整理得\(\frac{y}{x}-1=y'\),
可见这是一个齐次方程,
解得\(y=x(C-\ln x)\ (x>0)\).
代入\(y(1)=2\),
得\(C=2\),
于是曲线方程为\(y=x(2-\ln x)\).
\end{solution}
\end{example}

\begin{example}
探照灯的聚光镜的镜面是一张旋转曲面,
它的形状由\(xOy\)坐标面上的一条曲线\(L\)绕\(x\)轴旋转而成.
按照工程师对聚光镜性能的要求,从点\(O\)处发出的一切光线,经它反射后都与旋转轴平行.
求曲线\(L\)的方程.
\begin{figure}[htb]
	\centering
	\begin{tikzpicture}
		\begin{axis}[
			xmin=-3.5,xmax=5,
			ymin=0,ymax=5,
			axis equal=true,
			axis lines=middle,
			xlabel=$x$,
			ylabel=$y$,
			xtick=\empty,
			ytick=\empty,
		]
			\pgfmathsetmacro{\pMx}{1}
			\pgfmathsetmacro{\pMy}{sqrt(4*(\pMx+1))}
			\pgfmathsetmacro{\pSx}{\pMx+3}
			\pgfmathsetmacro{\dvM}{2*\pMx/sqrt(4*(\pMx+1))}
			\pgfmathsetmacro{\pTx}{4}
			\pgfmathsetmacro{\pTy}{\pMy+\dvM*(\pTx-\pMx)}
			\pgfmathsetmacro{\pAx}{\pMx-\pMy/\dvM}
			\pgfmathsetmacro{\pNx}{\dvM*\pMy+\pMx}
			\draw[red](0,0)node[black,below left]{$O$}coordinate(O)
				--(\pMx,\pMy)node[black,above left]{$M$}coordinate(M)
				--(\pSx,\pMy)coordinate(S)node[black,below]{$S$};
			\draw(4,4)node{$L$};
			\begin{scope}[color=black!30]
				\draw(M)--(\pTx,\pTy)coordinate(T)node[black,above right]{$T$};
				\draw(M)--(\pAx,0)coordinate(A)node[black,below]{$A$};
				\draw(M)--(M|-O)coordinate(P)node[black,below]{$P$};
				\draw(M)--(\pNx,0)coordinate(N)node[black,below]{$N$};
			\end{scope}
			\begin{scope}[angle eccentricity=2,angle radius=.4cm]
				\draw pic["$\alpha$",draw=orange,-]{angle=P--A--M};
				\draw pic["$\alpha$",draw=orange,-]{angle=S--M--T};
				\draw pic["$\alpha$",draw=orange,-]{angle=A--M--O};
				\draw pic[draw=brown,angle radius=.2cm]{right angle=N--P--M};
				\draw pic[draw=brown,angle radius=.2cm]{right angle=N--M--T};
			\end{scope}
			\begin{scope}[color=blue,smooth]
				\addplot[domain=-1:0,samples=100]{sqrt(4*(x+1))};
				\addplot[domain=0:10,samples=50]{sqrt(4*(x+1))};
			\end{scope}
		\end{axis}
	\end{tikzpicture}
	\caption{}
	\label{figure:微分方程.齐次方程.例图1}
\end{figure}
\begin{solution}
如\cref{figure:微分方程.齐次方程.例图1},建立平面直角坐标系.
根据对称性,我们只考虑曲线方程中\(y\geq0\)的部分.
设点\(M(x,y)\)是\(L\)上任意一点,
点\(O\)发出的某条光线经点\(M\)反射后是一条与\(x\)轴平行的直线\(MS\).
又设过点\(M\)的切线\(AT\)与\(x\)轴的夹角为\(\alpha\).
根据题意有,\(\angle SMT = \alpha\).
另一方面,\(\angle OMA\)是入射角的余角,\(\angle SMT\)是反射角的余角,
于是由光学中的反射定律有\(\angle OMA = \angle SMT = \alpha\);
从而\(AO = OM\),但\(AO = AP - OP = PM \cot\alpha - OP = \frac{y}{y'} - x\),
而\(OM = \sqrt{x^2+y^2}\).
于是得微分方程\begin{equation*}
	\frac{y}{y'} - x = \sqrt{x^2+y^2}.
\end{equation*}
当\(y>0\)时,上式即为\begin{equation*}
	\dv{x}{y} = \frac{x}{y} + \sqrt{\left(\frac{x}{y}\right)^2 + 1},
\end{equation*}
这是齐次方程.
令\(v = \frac{x}{y}\),
则\(x = yv\),\(\dv{x}{y} = v + y \dv{v}{y}\),
代入上式,得\begin{equation*}
	v + y \dv{v}{y} = v + \sqrt{v^2 + 1},
\end{equation*}
即\begin{equation*}
	y \dv{v}{y} = \sqrt{v^2 + 1}.
\end{equation*}
分离变量,得\begin{equation*}
	\frac{\dd{v}}{\sqrt{v^2+1}}
	= \frac{\dd{y}}{y}.
\end{equation*}
积分,得\begin{equation*}
	\ln(v+\sqrt{v^2+1}) = \ln y - \ln C,
\end{equation*}
或\begin{equation*}
	v + \sqrt{v^2+1} = \frac{y}{C}.
\end{equation*}

由\begin{equation*}
\left(\frac{y}{C} - v\right)^2 = v^2 + 1,
\end{equation*}得\begin{equation*}
	\frac{y^2}{C^2} - \frac{2yv}{C} = 1,
\end{equation*}
以\(yv=x\)代入上式,得\begin{equation*}
	y^2 = 2C(x+\frac{C}{2}).
\end{equation*}
可以看出,这是以\(x\)轴为对称轴、焦点在原点的抛物线.
\end{solution}
\end{example}

\section{可化为齐次的微分方程}
方程\begin{equation}\label[differential-equation]{equation:微分方程.可化为齐次的方程}
%@see: 《高等数学(第六版 上册)》 P307 (3)
	\dv{y}{x} = \frac{ax + by + c}{a_1 x + b_1 y + c_1}
\end{equation}
当\(c=c_1=0\)时是齐次的,否则不是齐次的.
在非齐次的情形下,可用下列代换把它化为齐次方程:
先令\(x = X + h\),\(y = Y + k\),
其中\(h\)和\(k\)是待定的常数.
于是\(\dd{x}=\dd{X}\),\(\dd{y}=\dd{Y}\),
而原方程化为\begin{equation*}
	\dv{Y}{X} = \frac{aX+bY+(ah+bk+c)}{a_1 X+b_1 Y+(a_1 h+b_1 k+c_1)}.
\end{equation*}

根据\hyperref[theorem:线性方程组.克拉默法则]{克拉默法则},
如果关于\(h,k\)的代数方程组\begin{equation}\label{equation:微分方程.可化为齐次的方程.待定系数的代数方程}
	\left\{ \begin{array}{l}
		ah+bk+c = 0, \\
		a_1 h+b_1 k+c_1 = 0
	\end{array} \right.
\end{equation}的系数行列式\(\begin{vmatrix}
	a & b \\
	a_1 & b_1
\end{vmatrix} \neq 0\),
那么可以解出唯一的\(h\)及\(k\),
即\begin{equation*}
	h = -\frac{\begin{vmatrix}
		c & b \\
		c_1 & b_1
	\end{vmatrix}}{\begin{vmatrix}
		a & b \\
		a_1 & b_1
	\end{vmatrix}},
	\qquad
	k = -\frac{\begin{vmatrix}
		a & c \\
		a_1 & c_1
	\end{vmatrix}}{\begin{vmatrix}
		a & b \\
		a_1 & b_1
	\end{vmatrix}},
\end{equation*}
从而将\cref{equation:微分方程.可化为齐次的方程}
化为齐次方程\begin{equation}\label{equation:微分方程.可化为齐次的方程.换元得到的齐次方程1}
	\dv{Y}{X} = \frac{aX+bY}{a_1 X+b_1 Y}.
\end{equation}
求出齐次方程 \labelcref{equation:微分方程.可化为齐次的方程.换元得到的齐次方程1} 的通解后,
在通解中以\(x-h\)代\(X\),\(y-k\)代\(Y\),
便得\cref{equation:微分方程.可化为齐次的方程} 的通解.

如果代数方程组 \labelcref{equation:微分方程.可化为齐次的方程.待定系数的代数方程} 的
系数行列式\(\begin{vmatrix} a & b \\ a_1 & b_1 \end{vmatrix} = 0\),
那么上述方法不能应用,不过可以令\footnote{这里假设\(a,b \neq 0\).}\begin{equation*}
	\frac{a_1}{a} = \frac{b_1}{b} = \lambda,
\end{equation*}
从而将\cref{equation:微分方程.可化为齐次的方程}
化为\begin{equation}\label{equation:微分方程.可化为齐次的方程.换元得到的齐次方程2}
	\dv{y}{x} = \frac{(ax+by)+c}{\lambda(ax+by)+c_1}.
\end{equation}
引入新变量\(v=ax+by\),则\begin{equation*}
	\dv{v}{x} = a + b \dv{y}{x}
	\quad\text{或}\quad
	\dv{y}{x} = \frac{1}{b} \left(\dv{v}{x} - a\right).
\end{equation*}
于是\cref{equation:微分方程.可化为齐次的方程} 成为可分离变量的微分方程\begin{equation*}
	\frac{1}{b} \left(\dv{v}{x} - a\right)
	= \frac{v + c}{\lambda v + c_1}.
\end{equation*}

上述解法可以应用于更一般的微分方程\begin{equation}
	\dv{y}{x} = f\left(\frac{ax+by+c}{a_1 x+b_1 y+c_1}\right).
\end{equation}

对形如\begin{equation}
	\dv{y}{x} = x f\left(\frac{y}{x^2}\right)
\end{equation}的微分方程,
可令\(u = y/x^2\).

对形如\begin{equation}
	\dv{y}{x} = \frac{1}{x^2} f(xy)
	\quad\text{或}\quad
	\dv{y}{x} = \frac{y}{x} f(xy)
\end{equation}的微分方程,
可令\(u = xy\).

对形如\begin{equation}
	\dv{y}{x} = f(ax+by+c)
\end{equation}的微分方程,
可令\(u = ax+by+c\).

\section{一阶线性微分方程}\label{section:微分方程.一阶线性微分方程}
对于一阶线性非齐次微分方程\begin{equation}\label[first-order-linear-nonhomogeneous-ordinary-differential-equation]{equation:微分方程.一阶线性非齐次微分方程}
%@see: 《高等数学(第六版 上册)》 P310 (1)
	\dv{y}{x} + P(x) y = Q(x),
	\quad Q(x) \not\equiv 0,
\end{equation}
我们有如下的解法.

首先,我们把\(Q(x)\)换成零,
写出方程\begin{equation}\label[first-order-linear-homogeneous-ordinary-differential-equation]{equation:微分方程.一阶线性齐次微分方程}
%@see: 《高等数学(第六版 上册)》 P310 (2)
	\dv{y}{x} + P(x) y = 0.
\end{equation}
\cref{equation:微分方程.一阶线性齐次微分方程}
叫做“对应于\cref{equation:微分方程.一阶线性非齐次微分方程} 的线性齐次方程”.

\cref{equation:微分方程.一阶线性齐次微分方程} 是可分离变量的,分离变量后得\begin{equation*}
	\frac{\dd{y}}{y} = -P(x) \dd{x},
\end{equation*}
两端积分,得\begin{equation*}
	\ln{\abs{y}} = -\int P(x) \dd{x} + C_1,
\end{equation*}
或\begin{equation*}
	y = C e^{ -\int P(x) \dd{x} }
	\quad(C = \pm e^{C_1}),
\end{equation*}
这是\cref{equation:微分方程.一阶线性齐次微分方程} 的通解
\footnote{
	这里记号\(\int P(x) \dd{x}\)表示\(P(x)\)的某个确定的原函数.
}.

现在我们使用所谓“常数变易法”来求\cref{equation:微分方程.一阶线性非齐次微分方程} 的通解.
这方法是把\cref{equation:微分方程.一阶线性齐次微分方程} 的通解中的\(C\)换成\(x\)的未知函数\(u(x)\),
即作变换\begin{equation}\label{equation:微分方程.一阶线性非齐次微分方程.中间步骤1}
%@see: 《高等数学(第六版 上册)》 P310 (3)
	y = u e^{ -\int P(x)\dd{x} },
\end{equation}
于是\begin{equation}\label{equation:微分方程.一阶线性非齐次微分方程.中间步骤2}
%@see: 《高等数学(第六版 上册)》 P311 (4)
	\dv{y}{x} = \dv{u}{x} e^{ -\int P(x) \dd{x} } - u P(x) e^{ -\int P(x) \dd{x} }.
\end{equation}
将\cref{equation:微分方程.一阶线性非齐次微分方程.中间步骤1}
和\cref{equation:微分方程.一阶线性非齐次微分方程.中间步骤2}
代入\cref{equation:微分方程.一阶线性非齐次微分方程}
得\begin{equation*}
	\dv{u}{x} e^{ -\int P(x) \dd{x} }
	- u P(x) e^{ -\int P(x) \dd{x} }
	+ P(x) u e^{ -\int P(x) \dd{x} }
	= Q(x),
\end{equation*}
即\begin{equation*}
	\dv{u}{x} e^{ -\int P(x) \dd{x} } = Q(x),
	\qquad
	\dv{u}{x} = Q(x) e^{ \int P(x) \dd{x} }.
\end{equation*}
两端积分,得\begin{equation*}
	u = \int Q(x) e^{ \int P(x) \dd{x} } \dd{x} + C.
\end{equation*}
将上式代入\cref{equation:微分方程.一阶线性非齐次微分方程.中间步骤1},
便得\cref{equation:微分方程.一阶线性非齐次微分方程} 的通解\begin{equation}\label{equation:微分方程.一阶线性非齐次微分方程的通解}
	y = e^{ -\int P(x) \dd{x} } \left( \int Q(x) e^{ \int P(x) \dd{x} } \dd{x} + C \right).
\end{equation}
上式右端第一项是\cref{equation:微分方程.一阶线性齐次微分方程} 的通解,
第二项是\cref{equation:微分方程.一阶线性非齐次微分方程} 的一个特解.
由此可知,一阶非齐次线性方程的通解等于对应的齐次方程的通解与非齐次方程的一个特解之和.

\begin{example}
%@see: 《高等数学(第六版 上册)》 P311 例1
求方程\begin{equation*}
	\dv{y}{x} - \frac{2y}{x+1} = (x+1)^{5/2}
\end{equation*}的通解.
\begin{solution}
这是一个一阶非齐次线性方程.
先求对应的齐次方程\(\dv{y}{x} - \frac{2}{x+1} y = 0\)的通解.
分离变量得\begin{equation*}
	\frac{\dd{y}}{y} = \frac{2 \dd{x}}{x+1},
\end{equation*}
积分得\begin{equation*}
	\ln\abs{y} = 2\ln\abs{x+1}+ \ln\abs{C_1},
\end{equation*}
即\begin{equation*}
	y = C_1 (x+1)^2.
\end{equation*}

用常数变易法,把\(C_1\)换成\(u\),即令\(y = u(x+1)^2\),那么\begin{equation*}
	\dv{y}{x} = \dv{u}{x} (x+1)^2 + 2u(x+1),
\end{equation*}
代入所给非齐次方程,得\begin{equation*}
	u' = (x+1)^{1/2}.
\end{equation*}
两端积分,得\begin{equation*}
	u = \frac{2}{3} (x+1)^{3/2} + C.
\end{equation*}
故所求方程通解为\begin{equation*}
	y = (x+1)^2 \left[
	\frac{2}{3} (x+1)^{3/2} + C
	\right].
\end{equation*}
\end{solution}
\end{example}

在上一小节中,对于齐次方程\(y' = f(y/x)\),我们通过变量代换\(y = xu\),把它化为变量可分离的方程,然后分离变量,经积分求得通解.
在本小节中,对于一阶非齐次线性方程\begin{equation*}
y' + P(x) y = Q(x),
\end{equation*}我们通过解对应的齐次线性方程找到变量代换\begin{equation*}
y = u e^{- \int P(x) \dd{x}},
\end{equation*}利用这一代换,把非齐次线性方程化为变量可分离的方程,然后经积分求得通解.

利用变量代换(因变量的变量代换或自变量的变量代换),把一个微分方程化为变量可分离的方程,或化为已知其求解步骤的方程,这是解微分方程最常用的方法.

\begin{example}
解方程\begin{equation*}
	x y' + y = x^2 + 3x + 2.
\end{equation*}
\begin{solution}
将原方程变形为\begin{equation*}
	y' + \frac{1}{x} y = x + 3 + \frac{2}{x}.
\end{equation*}
对应的齐次方程\begin{equation*}
	y' + \frac{1}{x} y = 0
\end{equation*}
又可变形为\begin{equation*}
	\frac{\dd{y}}{y} = -\frac{\dd{x}}{x},
\end{equation*}
积分得\begin{equation*}
	\ln\abs{y} = -\ln\abs{x} + C_1
	\quad\text{或}\quad
	y = C_2 e^{-\ln\abs{x}}
	= C_2 \frac{1}{\abs{x}}.
\end{equation*}
应用常数变易法,令\(y = \frac{u}{\abs{x}}\),那么\begin{equation*}
	\dv{y}{x} = \frac{u' \abs{x} - u(\abs{x})'}{x^2}.
\end{equation*}
代回原方程得\begin{equation*}
	\frac{u' \abs{x} - u(\abs{x})'}{x^2}
	+ \frac{1}{x} \frac{u}{\abs{x}}
	= x + 3 + \frac{2}{x},
\end{equation*}
或\begin{equation*}
	\def\arraystretch{1.5}
	\left\{ \begin{array}{cc}
		\frac{u' x - u}{x^2}
		+ \frac{1}{x} \frac{u}{x}
		= x + 3 + \frac{2}{x},
		& x>0, \\
		\frac{u' (-x) + u}{x^2}
		- \frac{1}{x} \frac{u}{x}
		= x + 3 + \frac{2}{x},
		& x<0.
	\end{array} \right.
\end{equation*}
整理得\begin{equation*}
	\def\arraystretch{1.5}
	u' = \left\{ \begin{array}{rc}
		x^2 + 3x + 2,
		& x>0, \\
		-(x^2 + 3x + 2),
		& x<0.
	\end{array} \right.
\end{equation*}
积分得\begin{equation*}
	\def\arraystretch{1.5}
	u = \left\{ \begin{array}{rc}
		\frac{1}{3} x^3 + \frac{3}{2} x^2 + 2x + C_3,
		& x>0, \\
		-(\frac{1}{3} x^3 + \frac{3}{2} x^2 + 2x) + C_4,
		& x<0.
	\end{array} \right.
\end{equation*}
由此,原方程的通解为\begin{equation*}
	y = \frac{1}{3} x^2 + \frac{3}{2} x + 2 + \frac{C}{x}.
\end{equation*}
\end{solution}
\end{example}

\begin{example}
%@see: https://www.bilibili.com/video/BV1ECxGepEAZ/
求方程\begin{equation*}
	\dv{y}{x} = \frac{y}{x + y^4 \ln y}
\end{equation*}的通解.
\begin{solution}
取倒数得\begin{equation*}
	\dv{x}{y} - \frac{x}{y} = y^3 \ln y.
\end{equation*}
由对应的齐次方程\begin{equation*}
	\dv{x}{y} - \frac{x}{y} = 0
\end{equation*}解得\(x = C_1 y\ (\text{$C_1$是常数})\).
用常数变易法,令\(x = u y\),
则\(\dv{x}{y} = u + y \dv{u}{y}\),
代入原方程得\begin{equation*}
	u + y \dv{u}{y} = u + y^3 \ln y,
\end{equation*}
整理得\begin{equation*}
	\dd{u} = y^2 \ln y \dd{y},
\end{equation*}
积分得\begin{equation*}
	u = \frac13 y^3 \ln y - \frac19 y^3 + C.
\end{equation*}
那么原方程的通解为\begin{equation*}
	x = y \left( \frac13 y^3 \ln y - \frac19 y^3 + C \right).
\end{equation*}
%@Mathematica: DSolve[1/D[x[y], y] == y/(x[y] + y^4 Log[y]), x[y], y]
\end{solution}
\end{example}

\begin{example}
%@see: 《2018年全国硕士研究生入学统一考试(数学一)》三解答题/第18题
已知微分方程\(y' + y = f(x)\),
其中\(f\)是\(\mathbb{R}\)上的周期为\(T\)的连续函数.
证明:方程存在唯一的以\(T\)为周期的解.
\begin{proof}
记\(P = 1,
Q = f(x)\),
那么\(y' + y = f(x)\)的通解为\begin{equation*}
	%\cref{equation:微分方程.一阶线性非齐次微分方程的通解}
	Y = e^{-x} \left( \int f(x) e^x \dd{x} + C \right),
\end{equation*}
其中\(C\)是任意常数.
根据题意,我们需要证明通解\(Y\)中存在一个以\(T\)为周期的函数\(y\),即\(y(x+T) = y(x)\).
于是可以假设函数\begin{equation*}
	y(x) = e^{-x} \left( \int_0^x f(t) e^t \dd{t} + C \right)
\end{equation*}是所求的以\(T\)为周期的函数,
其中\(C\)是待定参数.
因为\begin{align*}
	y(x+T) &= e^{-x-T} \left( \int_0^{x+T} f(t) e^t \dd{t} + C \right) \\
	&= e^{-x-T} \left( \int_0^T f(t) e^t \dd{t} + \int_T^{x+T} f(t) e^t \dd{t} + C \right) \\
	&\xlongequal{u=t-T}
	e^{-x-T} \left( \int_0^T f(t) e^t \dd{t} + \int_0^x f(u+T) e^{u+T} \dd{u} + C \right) \\
	% 利用\(f(u+T) = f(u)\)化简,把因子\(e^{-T}\)乘进去
	&= e^{-x} \left( e^{-T} \int_0^T f(t) e^t \dd{t} + \int_0^x f(u) e^u \dd{u} + C e^{-T} \right),
\end{align*}
所以\begin{align*}
	0 &= (y(x+T) - y(x)) e^x \\
	&= \left( e^{-T} \int_0^T f(t) e^t \dd{t} + \int_0^x f(u) e^u \dd{u} + C e^{-T} \right)
	- \left( \int_0^x f(t) e^t \dd{t} + C \right) \\
	&= e^{-T} \int_0^T f(t) e^t \dd{t} + C e^{-T} - C,
\end{align*}
整理得\begin{equation*}
	C (1 - e^{-T}) = e^{-T} \int_0^T f(t) e^t \dd{t},
\end{equation*}
解得\begin{equation*}
	C = \frac{e^{-T}}{1 - e^{-T}} \int_0^T f(t) e^t \dd{t}.
\end{equation*}
注意到线性微分方程的通解包含它的所有解,
故只要\(C\)唯一,则周期解\(y\)唯一.
\end{proof}
\end{example}

\begin{example}
%@see: http://www.youtube.com/watch?v=IZqwi0wJovM
解方程组\begin{equation*}
	\begin{cases}
		u_1' = -u_1 + 2u_2, \\
		u_2' = u_1 - 2u_2,
	\end{cases}
\end{equation*}
其中\(
	u_1(0) = 1,
	u_2(0) = 0
\).
\begin{solution}
这是一个一阶常系数线性方程组.
记\begin{equation*}
	U \defeq \begin{bmatrix}
		u_1 \\ u_2
	\end{bmatrix},
	\qquad
	U' \defeq \begin{bmatrix}
		u_1' \\ u_2'
	\end{bmatrix},
	\qquad
	D \defeq \begin{bmatrix}
		-1 & 2 \\
		1 & -2
	\end{bmatrix},
\end{equation*}
便得\(D' = D U\).
%@Mathematica: A = {{-1, 2}, {1, -2}};
%@Mathematica: Eigenvalues[A]
%@Mathematica: Eigenvectors[A]
易知\(\lambda_1=-3\)和\(\lambda_2=0\)是\(D\)的特征值,
\(x_1=(-1,1)^T\)和\(x_2=(2,1)^T\)是\(D\)的特征向量,
于是所求方程组的通解为\begin{equation*}
	U = C_1 e^{\lambda_1 t} x_1 + C_2 e^{\lambda_2 t} x_2
	= C_1 e^{-3t} \begin{bmatrix}
		-1 \\ 1
	\end{bmatrix}
	+ C_2 \begin{bmatrix}
		2 \\ 1
	\end{bmatrix},
\end{equation*}
其中\(C_1,C_2\)是任意常数.
代入\(U(0) \defeq (1,0)^T\)得\begin{equation*}
	U(0)
	= C_1 \begin{bmatrix}
		-1 \\ 1
	\end{bmatrix}
	+ C_2 \begin{bmatrix}
		2 \\ 1
	\end{bmatrix}
	= \begin{bmatrix}
		-1 & 2 \\
		1 & 1
	\end{bmatrix}
	\begin{bmatrix}
		C_1 \\ C_2
	\end{bmatrix}
	= \begin{bmatrix}
		1 \\ 0
	\end{bmatrix},
\end{equation*}
解得\(
	C_1 = -\frac13,
	C_2 = \frac13
\),
即\(
	U(t)
	= -\frac13 e^{-3t} \begin{bmatrix}
		-1 \\ 1
	\end{bmatrix}
	+ \frac13 \begin{bmatrix}
		2 \\ 1
	\end{bmatrix}
\).
%@Mathematica: DSolve[{u1'[t] == -u1[t] + 2 u2[t], u2'[t] == u1[t] - 2 u2[t], u1[0] == 1, u2[0] == 0}, {u1[t], u2[t]}, t] // FullSimplify
\end{solution}
\end{example}

\section{伯努利方程}
方程\begin{equation}\label[differential-equation]{equation:微分方程.伯努利方程}
%@see: 《高等数学(第六版 上册)》 P314 (10)
	\dv{y}{x} + P(x) y = Q(x) y^n
\end{equation}
叫做\DefineConcept{伯努利方程}.
当\(n=0\)或\(n=1\)时,
\cref{equation:微分方程.伯努利方程} 就是一阶线性微分方程,
可以直接按\cref{section:微分方程.一阶线性微分方程}讲述的步骤解出该方程.
当\(n\neq0,1\)时,这方程不是线性的,
但是通过变量的代换,可以把它化为线性的.
事实上,以\(y^n\)除\cref{equation:微分方程.伯努利方程} 两端,
得\begin{equation}\label[differential-equation]{equation:微分方程.伯努利方程.中间步骤}
%@see: 《高等数学(第六版 上册)》 P314 (11)
	y^{-n} \dv{y}{x} + P(x) y^{1-n} = Q(x),
\end{equation}
容易看出,上式左端第一项与\(\dv{x}(y^{1-n})\)只差一个常数因子\(1-n\),
因此我们引入新的因变量\(z = y^{1-n}\),
那么\begin{equation*}
	\dv{z}{x} = (1-n) y^{-n} \dv{y}{x}.
\end{equation*}
用\((1-n)\)乘\cref{equation:微分方程.伯努利方程.中间步骤} 的两端,
再通过上述代换便得一阶线性微分方程\begin{equation}\label[differential-equation]{equation:微分方程.伯努利方程.换元后得到的方程}
	\dv{z}{x} + (1-n) P(x) z = (1-n) Q(x).
\end{equation}
求出\cref{equation:微分方程.伯努利方程.换元后得到的方程} 的通解后,
以\(y^{1-n}\)代\(z\)便得到伯努利方程的通解.

特别地,当\(n>0\)时,\(y\)可以取为\(0\),
而\(y=0\)恰好也满足\cref{equation:微分方程.伯努利方程} .

\begin{example}
%@see: 《高等数学(第六版 上册)》 P314 例4
求方程\begin{equation*}
	\dv{y}{x} + \frac{y}{x} = a y^2 \ln x
\end{equation*}的通解.
\begin{solution}
以\(y^2\)除方程的两端,得\begin{equation*}
	y^{-2} \dv{y}{x} + \frac{1}{2} y^{-1} = a \ln x,
\end{equation*}
即\begin{equation*}
	- \dv{x}(y^{-1}) + \frac{1}{x} y^{-1} = a \ln x.
\end{equation*}
令\(z = y^{-1}\),则上述方程成为\begin{equation*}
	\dv{z}{x} - \frac{1}{x} z = -a \ln x.
\end{equation*}
这是一个线性方程,它的通解为\begin{equation*}
	z = x \left[ C - \frac{a}{2} \ln^2 x \right].
\end{equation*}
以\(y^{-1}\)代\(z\),得所求方程的通解为\begin{equation*}
	yx \left[ C - \frac{a}{2} \ln^2 x \right] = 1.
\end{equation*}
\end{solution}
\end{example}

% \section{一阶微分方程的幂级数解法}
% 给定一阶微分方程\begin{equation*}
% \def\arraystretch{1.5}
% \left\{ \begin{array}{l}
% \dv{y}{x} = f(x,y) \\
% \eval{y}_{x=x_0} = y_0 \\
% \end{array} \right.
% \end{equation*}其中函数\(f(x,y)\)是\((x-x_0)\)、\((y-y_0)\)的多项式\begin{equation*}
% f(x,y) = a_{00} + a_{10} (x-x_0) + a_{01} (y-y_0) + \dotsb + a_{lm} (x-x_0)^l (y-y_0)^m.
% \end{equation*}那么可以设所求的特解可展开为\((x-x_0)\)的幂级数:\begin{equation*}
% y = y_0 + a_1 (x-x_0) + a_2 (x-x_0)^2 + \dotsb + a_n (x-x_0)^n + \dotsb,
% \end{equation*}其中\(\AutoTuple{a}{n},\dotsc\)是待定系数.把上式代回原微分方程(组)中,
% 便得一恒等式,比较所得恒等式两端\((x-x_0)\)的同次幂的系数,就可确定待定系数取值,即求得原微分方程(组)的特解.

\section{可降阶的高阶微分方程}
从这一节起我们将讨论二阶及二阶以上的微分方程,即所谓的“高阶微分方程”.

对于有些高阶微分方程,我们可以通过代换将它化成较低阶的方程来求解.
以二阶微分方程\begin{equation}\label[differential-equation]{equation:微分方程.二阶微分方程.1111型}
	y'' = f(x,y,y')
\end{equation}而论,
如果我们能设法作代换把它从二阶降为一阶,
那么就有可能应用前面几节中所讲的方法来求出它的解了.

下面介绍三种容易降阶的高阶微分方程的求解方法.

\subsection{\texorpdfstring{形如\(y^{(n)} = f(x)\)}{由自变量确定n阶导数}的微分方程}
微分方程\begin{equation}\label[differential-equation]{equation:微分方程.简单高阶微分方程}
	y^{(n)} = f(x)
\end{equation}的右端仅含有自变量\(x\).
容易看出,只要把\(y^{(n-1)}\)作为新的未知函数,
那么\cref{equation:微分方程.简单高阶微分方程} 就是新未知函数的一阶微分方程.
两边积分,就得到一个\(n-1\)阶的微分方程\begin{equation*}
	y^{(n-1)} = \int f(x) \dd{x} \dd{x} + C_1.
\end{equation*}
同理可得\begin{equation*}
	y^{(n-2)} = \int \left[ \int f(x) \dd{x} + C_1 \right] \dd{x} + C_2.
\end{equation*}
依此法继续进行,接连积分\(n\)次,
便得\cref{equation:微分方程.简单高阶微分方程} 的含有\(n\)个任意常数的通解.

\begin{example}
求微分方程\begin{equation*}
	y''' = e^{2x} - \cos x
\end{equation*}的通解.
\begin{solution}
对所给方程接连积分三次,得\begin{gather*}
	y'' = \frac{1}{2} e^{2x} - \sin x + C, \\
	y' = \frac{1}{4} e^{2x} + \cos x + C x + C_2, \\
	y = \frac{1}{8} e^{2x} + \sin x + C_1 x^2 + C_2 x + C_3.
\end{gather*}
这就是所求的通解.
\end{solution}
\end{example}

\subsection{\texorpdfstring{形如\(y'' = f(x,y')\)}{由自变量与一阶导数确定二阶导数}的微分方程}
微分方程\begin{equation}\label[differential-equation]{equation:微分方程.二阶微分方程.1011型}
	y'' = f(x, y')
\end{equation}的右端不显含未知函数\(y\).
如果设\(y' = p\),那么\begin{equation*}
	y'' = \dv[2]{y}{x} = \dv{p}{x} = p',
\end{equation*}
而\cref{equation:微分方程.二阶微分方程.1011型} 就成为\begin{equation*}
	p' = f(x, p),
\end{equation*}
这是一个关于变量\(x\)、\(p\)的一阶微分方程.
设其通解为\begin{equation*}
	p = \phi(x, C_1),
\end{equation*}
但是\(p = \dv{y}{x}\),因此又得到一个一阶微分方程\begin{equation*}
	\dv{y}{x} = \phi(x, C_1).
\end{equation*}
对它进行积分,便得\cref{equation:微分方程.二阶微分方程.1011型} 的
通解为\begin{equation}\label{equation:微分方程.二阶微分方程.1011型.通解}
	y = \int \phi(x, C_1) \dd{x} + C_2.
\end{equation}

\begin{example}
求微分方程\begin{equation*}
	(1+x^2) y'' = 2xy'
\end{equation*}满足初始条件\begin{equation*}
	\eval{y}_{x=0} = 1,
	\qquad
	\eval{y'}_{x=0} = 3
\end{equation*}的特解.
\begin{solution}
所给方程是\(y'' = f(x, y')\)型的.
设\(y' = p\),代入方程并分离变量后,有\begin{equation*}
	\frac{\dd{p}}{p} = \frac{2x}{1+x^2} \dd{x}.
\end{equation*}
两端积分,得\begin{equation*}
	\ln\abs{p} = \ln(1+x^2) + C,
\end{equation*}
即\(p = y' = C_1(1+x^2)\ (C_1 = \pm e^C)\).
由条件\(\eval{y'}_{x=0}=3\),得\(C_1 = 3\),所以\begin{equation*}
	y' = 3(1+x^2).
\end{equation*}
两端再积分,得\begin{equation*}
	y = x^3 + 3x + C_2.
\end{equation*}
又由条件\(\eval{y}_{x=0}=1\),得\(C_2=1\),于是所求的特解为\begin{equation*}
	y = x^3 + 3x + 1.
\end{equation*}
\end{solution}
\end{example}

\begin{example}
设有一均匀、柔软的绳索,两端固定,绳索仅受重力的作用而下垂.
试问该绳索在平衡状态时是怎样的曲线?
\begin{solution}
设绳索的最低点为\(A\).
取\(y\)轴通过点\(A\)铅直向上,并取\(x\)轴水平向右,且\(\abs{OA}\)等于某个定值.
设绳索曲线的方程为\(y = \phi(x)\).
考察绳索上点\(A\)到另一点\(M(x,y)\)间的一段弧\(\Arc{AM}\),设其长为\(s\).
假定绳索的线密度为\(\rho\),则弧\(\Arc{AM}\)所受重力为\(\rho gs\).
由于绳索是柔软的,因而在点\(A\)处的张力沿水平的切线方向,其大小设为\(H\);
在点\(M\)处的张力沿该点处的切线方向,设其水平倾角为\(\theta\),其大小为\(T\).
因作用于弧段\(\Arc{AM}\)的外力相互平衡,
把作用于弧\(\Arc{AM}\)上的力沿铅直、水平两方向分解,得\begin{equation*}
	T \sin\theta = \rho gs,
	\qquad
	T \cos\theta = H.
\end{equation*}
将此两式相除,得\begin{equation*}
	\tan\theta = \frac{1}{a} s
	\quad(a = \frac{H}{\rho g}).
\end{equation*}
由于\(\tan\theta = y'\),\(s = \int_0^x \sqrt{1+(y')^2} \dd{x}\),代入上式即得\begin{equation*}
	y' = \frac{1}{a} \int_0^x \sqrt{1+(y')^2} \dd{x}.
\end{equation*}
将上式两端对\(x\)求导,便得\(y = \phi(x)\)满足的微分方程\begin{equation*}
	y'' = \frac{1}{a} \sqrt{1+(y')^2}.
\end{equation*}

取原点\(O\)到点\(A\)的距离为定值\(a\),即\(\abs{OA}=a\),那么初始条件为\begin{equation*}
	\eval{y}_{x=0} = a, \qquad \eval{y'}_{x=0} = 0.
\end{equation*}

设\(y' = p\),则\(y'' = \dv{p}{x}\),代回微分方程,并分离变量,得\begin{equation*}
	\frac{\dd{p}}{\sqrt{1+p^2}} = \frac{\dd{x}}{a}.
\end{equation*}
两端积分,得\begin{equation*}
	\ln(p+\sqrt{1+p^2}) = \frac{x}{a} + C_1.
\end{equation*}
把条件\(\eval{y'}_{x=0} = \eval{p}_{x=0} = 0\)代入上式,得\(C_1 = 0\),于是上式成为\begin{equation*}
	\ln(p+\sqrt{1+p^2}) = \frac{x}{a},
\end{equation*}
解得\begin{equation*}
	y' = p = \frac{1}{2} \left( e^{x/a} - e^{-x/a} \right).
\end{equation*}
再积分,便得\begin{equation*}
	y = \frac{a}{2} \left( e^{x/a} + e^{-x/a} \right) + C_2.
\end{equation*}
将条件\(\eval{y}_{x=0} = a\)代入上式,得\(C_2 = 0\).
于是该绳索的形状可由曲线方程
\begin{equation}\label{equation:微分方程.悬链线}
	y = \frac{a}{2} \left( e^{x/a} + e^{-x/a} \right)
	= a \cosh(\frac{x}{a}).
\end{equation}来表示.
这曲线叫做\DefineConcept{悬链线}.
\end{solution}
\end{example}

\subsection{\texorpdfstring{形如\(y'' = f(y,y')\)}{由因变量与一阶导数确定二阶导数}的微分方程}
方程\begin{equation}\label[differential-equation]{equation:微分方程.二阶微分方程.0111型}
	y'' = f(y, y')
\end{equation}
中不显含自变量\(x\).
为了求出它的解,我们令\(y'=p\),并利用复合函数的求导法则把\(y''\)化为对\(y\)的导数,即\begin{equation*}
	y'' = \dv{p}{x} = \dv{p}{y} \cdot \dv{y}{x} = p~\dv{p}{y}.
\end{equation*}
这样,\cref{equation:微分方程.二阶微分方程.0111型} 就成为\begin{equation*}
	p~\dv{p}{y} = f(y, p).
\end{equation*}
这是一个关于变量\(y\)、\(p\)的一阶微分方程.
设它的通解为\begin{equation*}
	y' = p = \phi(y, C_1),
\end{equation*}
分离变量并积分,便得\cref{equation:微分方程.二阶微分方程.0111型} 的
通解为\begin{equation}\label{equation:微分方程.二阶微分方程.0111型.通解}
	\int \frac{\dd{y}}{\phi(y,C_1)} = x + C_2.
\end{equation}

\begin{example}
设有一质量为\(m\)的物体,在空中由静止开始下落,
如果空气阻力为\(R = cv\)(其中\(c\)是常数,\(v\)是物体运动的速度),
试求物体下落的距离\(s\)与时间\(t\)的函数关系.
\begin{solution}
物体的加速度\(\dv{v}{t}\)与速度\(v\)满足\begin{equation*}
	m \dv{v}{t} = mg - cv.
\end{equation*}
当\(mg>cv\)时,有\begin{equation*}
	\frac{m}{mg-cv} \dd{v} = \dd{t},
\end{equation*}
即\begin{equation*}
	-\frac{m}{c} \frac{1}{mg-cv} \dd(mg-cv) = \dd{t},
\end{equation*}
两端积分,得\begin{equation*}
	-\frac{m}{c} \ln(mg-cv) = t + C_1
	\quad\text{或}\quad
	v = \frac{mg}{c} - \frac{1}{c} C_2 e^{-\frac{c}{m} t}.
\end{equation*}
代入初始条件\(\eval{v}_{t=0} = 0\),则有\(C_2 = mg\),那么上式即为\begin{equation*}
	v = \frac{mg}{c} \left( 1 - e^{-\frac{c}{m} t} \right).
\end{equation*}
再积分,得\begin{equation*}
	s(t) = \int_0^t v \dd{t}
	= \int_0^t \frac{mg}{c} \left( 1 - e^{-\frac{c}{m} t} \right) \dd{t} \\
	= \frac{mg}{c} \left( t + \frac{m}{c} e^{-\frac{c}{m} t} \right) + C_3.
\end{equation*}
代入初始条件\(\eval{s}_{t=0} = 0\),
则有\(C_3 = -\frac{m^2}{c^2} g\),
那么上式即为\begin{equation*}
	s(t) = \frac{mg}{c} \left( t + \frac{m}{c} e^{-\frac{c}{m} t} - \frac{m}{c} \right).
\end{equation*}
\end{solution}
\end{example}

\begin{example}
设函数\(y(x)\)具有二阶导数,且曲线\(l: y=y(x)\)与直线\(y=x\)相切于原点.
记\(\alpha\)为曲线\(l\)在点\((x,y)\)处的倾角,
且\(\dv{\alpha}{x}=\dv{y}{x}\),
求\(y(x)\)的表达式.
\begin{solution}
由导数的几何意义可知,\(\tan\alpha=y'\),
那么\(\sec^2\alpha \dv{\alpha}{x} = y''\),
即\begin{equation*}
	\dv{\alpha}{x} = \frac{y''}{\sec^2\alpha}
	= \frac{y''}{1+\tan^2\alpha}
	= \frac{y''}{1+(y')^2}.
\end{equation*}
由题意有,\(\frac{y''}{1+(y')^2} = y'\),
令\(y' = p,
y'' = p\dv{p}{y}\),
得\begin{gather*}
	p\dv{p}{y} = p(1+p^2), \\
	\dv{p}{y} = 1+p^2, \\
	\dd{y} = \frac{\dd{p}}{1+p^2}, \\
	\arctan p = y + C_1, \\
	p = \tan(y+C_1) = y',
\end{gather*}
因为曲线\(l: y=y(x)\)与直线\(y=x\)相切于原点,
\(y(0) = 0\),\(y'(0) = 1\),
代入得\(\tan C_1 = 1\),
那么可取\(C_1 = \pi/4\),
于是\(y' = \tan(y+\pi/4)\).
因此\begin{gather*}
	\cot(y+\frac{\pi}{4}) \dd{y} = \dd{x}, \\
	\ln\abs{\sin(y+\frac{\pi}{4})} = x + C_2, \\
	\sin(y+\frac{\pi}{4}) = C_3 e^x,
\end{gather*}
又因为\(y(0) = 0\),
代入得\(\sqrt{2}/2 = C_3\),
于是\begin{equation*}
	y = \arcsin(\frac{\sqrt{2}}{2} e^x) - \frac{\pi}{4}.
\end{equation*}
\end{solution}
\end{example}

\begin{example}
%@see: 《2025年全国硕士研究生入学统一考试(数学一)》三解答题/第18题
已知函数\(f(u)\)在区间\((0,+\infty)\)内具有二阶导数,
函数\(g(x,y) = f(x/y)\)
满足\(x^2 \pdv[2]{g}{x} + xy \pdv[2]{g}{x}{y} + y^2 \pdv[2]{g}{y} = 1\),
且\(g(x,x) = 1,
\eval{\pdv{g}{x}}_{(x,x)} = \frac2x\).
求\(f(u)\).
\begin{solution}
对\(g\)求导得\begin{gather*}
	\pdv{g}{x}
	= f'\left( \frac{x}{y} \right) \cdot \frac1y, \\
	\pdv{g}{y}
	= -\frac{x}{y^2} f'\left( \frac{x}{y} \right), \\
	\pdv[2]{g}{x}
	= f''\left( \frac{x}{y} \right) \cdot \frac1{y^2}, \\
	\pdv[2]{g}{x}{y}
	= -\frac{x}{y^3} f''\left( \frac{x}{y} \right) - \frac1{y^2} \cdot f'\left( \frac{x}{y} \right), \\
	\pdv[2]{g}{y}
	= \frac{x^2}{y^4} f''\left( \frac{x}{y} \right) + \frac{2x}{y^3} \cdot f'\left( \frac{x}{y} \right),
\end{gather*}
代入\(x^2 \pdv[2]{g}{x} + xy \pdv[2]{g}{x}{y} + y^2 \pdv[2]{g}{y} = 1\),得\begin{equation*}
	\frac{x^2}{y^2} f''(\frac{x}{y}) + \frac{x}{y} f'\left( \frac{x}{y} \right) = 1.
\end{equation*}
令\(u=x/y\),得\(u^2 f''(u) + u f'(u) = 1\).
令\(p=f'(u)\),得\(p' + \frac1u p = \frac1{u^2}\),
这是一个一阶线性微分方程,
解得\(p = \frac1u (\ln u + C_1)\),
其中\(C_1\)是待定常数.

由\(\eval{\pdv{g}{x}}_{(x,x)} = \frac2x\)可得\(f'(1)=2\).
代入\(f'(u) = \frac1u (\ln u + C_1)\),得\(C_1=2\),
于是\begin{equation*}
	f'(u) = \frac1u (\ln u + 2).
\end{equation*}
积分得\(f(u) = \frac12 \ln^2 u + 2 \ln u + C_2\),
其中\(C_2\)是待定常数.

由\(g(x,x) = 1\)可得\(f(1)=1\).
代入\(f(u) = \frac12 \ln^2 u + 2 \ln u + C_2\),得\(C_2=1\),
于是\begin{equation*}
	f(u) = \frac12 \ln^2 u + 2 \ln u + 1.
\end{equation*}
\end{solution}
\end{example}

\section{高阶线性微分方程的解的结构}
本节讨论在实际问题中应用得较多的所谓高阶线性微分方程.

\begin{figure}[htb]
	\centering
	\begin{tikzpicture}
		\draw[ultra thick](-1,0)--(1,0);
		\foreach \i in {-5,...,4} {
			\draw({\i*.2},0)--++(5pt,5pt);
		}
		\pgfmathsetmacro{\b}{2}
		\draw(0,0)--(0,-\b)coordinate(A);
		\pgfmathsetmacro{\c}{.4}
		\foreach \i in {0,...,9} {
			\draw(-\c,-\b*.1*\i)--++(2*\c,-.1)--++(-2*\c,-.1);
		}
		\pgfmathsetmacro{\a}{1}
		\draw(A)--++(.5*\a,0)--++(0,-\a)--++(-\a,0)--++(0,\a)--(A);
		\coordinate(B)at($(A)+(0,-.5*\a)$);
		\fill(B)circle(2pt)node[left]{\(O\)};
		\draw[>=Stealth,->](B)--++(0,-1)node[right]{\(\vb{x}\)};
	\end{tikzpicture}
	\caption{}
	\label{figure:微分方程.重力场中的弹簧振子模型}
\end{figure}
\begin{example}
设有一个弹簧,它的上端固定,下端挂着一个质量为\(m\)的物体.
当物体处于静止状态时,作用在物体上的重力与弹力大小相等、方向相反.
这个位置就是物体的平衡位置.
如\cref{figure:微分方程.重力场中的弹簧振子模型},
取\(x\)轴铅直向下,并取物体的平衡位置为坐标原点.

如果使物体具有一个初始速度\(v_0\neq0\),
那么物体便离开平衡位置,并在平衡位置附近作上下振动.
在振动过程中,物体的位置\(x\)随时间\(t\)变化,
即\(x\)是\(t\)的函数:\(x=x(t)\).
要确定物体的振动规律,就要求出函数\(x=x(t)\).

由力学知道,弹簧使物体回到平衡位置的弹力\(f\)
(它不包括物体在平衡位置时和重力\(mg\)相平衡的那一部分弹力)
和物体离开平衡位置的位移\(x\)成正比:\begin{equation*}
	f = -cx,
\end{equation*}
其中\(c\)是弹簧的弹性系数,
负号表示弹力的方向和物体位移的方向相反.

另外,物体在运动过程中还收到阻尼介质(如空气)的阻力的作用,使得振动逐渐趋向停止.
由实验指导,阻力\(R\)的方向总与运动方向相反,
当运动速度不大时,其大小与物体运动的速度成正比:\begin{equation*}
	R = -\mu \dv{x}{t},
\end{equation*}
其中\(\mu\)是阻尼系数.

根据上述关于物体受力情况的分析,由牛顿第二定律得\begin{equation*}
	m \dv[2]{x}{t} = -cx -\mu \dv{x}{t}.
\end{equation*}
移项,并记\(2n=\frac{\mu}{m},
k^2=\frac{c}{m}\),
则上式化为\begin{equation*}
	\dv[2]{x}{t}+2n\dv{x}{t}+k^2x=0.
\end{equation*}
这就是在有阻尼的情况下,物体自由振动的微分方程.

如果物体在振动过程中,还收到铅直干扰力\begin{equation*}
	F = H \sin pt
\end{equation*}的作用,则有\begin{equation*}
	\dv[2]{x}{t} + 2n \dv{x}{t} + k^2x = h \sin pt,
\end{equation*}
其中\(h=\frac{H}{m}\).
这就是受迫振动的微分方程.
\end{example}

下面来讨论形如
\begin{equation}\label{equation:微分方程.二阶非齐次线性微分方程的一般形式}
	\dv[2]{y}{x} + P(x) \dv{y}{x} + Q(x) y = f(x)
\end{equation}
的二阶线性微分方程的解的一些性质,这些性质可以推广到\(n\)阶线性方程
\begin{equation}\label{equation:微分方程.n阶线性微分方程的一般形式}
	y^{(n)} + a_1(x) y^{(n-1)} + \dotsb + a_{n-1}(x) y' + a_n(x) y = f(x).
\end{equation}

要研究线性微分方程的解的结构,
首先讨论二阶齐次线性方程\begin{equation}\label{equation:微分方程.二阶齐次线性微分方程的一般形式}
	y'' + P(x) y' + Q(x) y = 0.
\end{equation}

\begin{theorem}
%@see: 《高等数学(第六版 上册)》 P325 定理1
如果函数\(y_1(x)\)与\(y_2(x)\)是
二阶齐次线性方程 \labelcref{equation:微分方程.二阶齐次线性微分方程的一般形式} 的两个解,
那么\begin{equation}\label{equation:微分方程.二阶齐次线性微分方程的通解的结构}
	y = C_1 y_1 + C_2 y_2
\end{equation}
也是二阶齐次线性方程 \labelcref{equation:微分方程.二阶齐次线性微分方程的一般形式} 的解,
其中\(C_1\)、\(C_2\)是任意常数.
\begin{proof}
将\cref{equation:微分方程.二阶齐次线性微分方程的通解的结构}
代入\cref{equation:微分方程.二阶齐次线性微分方程的一般形式} 左端,
得\begin{align*}
	&\hspace{-20pt}
	[C_1 y_1'' + C_2 y_2'']
	+ P(x) [C_1 y_1' + C_2 y_2']
	+ Q(x) [C_1 y_1 + C_2 y_2] \\
	&= C_1 [y_1'' + P(x) y_1' + Q(x) y_1]
	+ C_2 [y_2'' + P(x) y_2' + Q(x) y_2].
\end{align*}
由于\(y_1,y_2\)都是二阶齐次线性方程 \labelcref{equation:微分方程.二阶齐次线性微分方程的一般形式} 的解,
上式右端方括号中的表达式都恒等于零,
因而整个式子恒等于零,
所以\cref{equation:微分方程.二阶齐次线性微分方程的通解的结构}
是\cref{equation:微分方程.二阶齐次线性微分方程的一般形式} 的解.
\end{proof}
\end{theorem}
从\cref{equation:微分方程.二阶齐次线性微分方程的通解的结构} 可以看出,
二阶齐次线性方程 \labelcref{equation:微分方程.二阶齐次线性微分方程的一般形式} 的解
成为一个线性空间.

%@see: 《高等数学(第六版 上册)》 P326
解 \labelcref{equation:微分方程.二阶齐次线性微分方程的通解的结构} 从形式上来看
含有\(C_1\)与\(C_2\)两个任意常数,
但它不一定是二阶齐次线性方程 \labelcref{equation:微分方程.二阶齐次线性微分方程的一般形式} 的通解.
例如,设\(y_1(x)\)是 \labelcref{equation:微分方程.二阶齐次线性微分方程的一般形式} 的一个解,
则\(y_2(x)=2 y_1(x)\)也是 \labelcref{equation:微分方程.二阶齐次线性微分方程的一般形式} 的解.
这时解 \labelcref{equation:微分方程.二阶齐次线性微分方程的通解的结构} 成为\(y = C_1 y_1(x) + 2 C_2 y_1(x)\),
可以把它改写成\(y = C y_1(x)\),其中\(C = C_1 + 2 C_2\).
这显然不是 \labelcref{equation:微分方程.二阶齐次线性微分方程的一般形式} 的通解.
那么在什么情况下 \labelcref{equation:微分方程.二阶齐次线性微分方程的通解的结构} 才是%
二阶齐次线性方程 \labelcref{equation:微分方程.二阶齐次线性微分方程的一般形式} 的通解呢?
要解决这个问题,还得引入一个新的概念,即所谓“函数组的线性相关与线性无关”.

\begin{definition}
%@see: 《高等数学(第六版 上册)》 P326
设\(y_1(x),y_2(x),\dotsc,y_n(x)\)为定义在区间\(I\)上的\(n\)个函数.
如果存在\(n\)个不全为零的常数\(\AutoTuple{k}{n}\),
使得当\(x \in I\)时,
有恒等式\begin{equation*}
	k_1 y_1+k_2 y_2+ \dotsb +k_n y_n \equiv 0
\end{equation*}成立,
那么称这\(n\)个函数在区间\(I\)上\DefineConcept{线性相关};
否则称之为\DefineConcept{线性无关}.
\end{definition}

\begin{example}
%@see: 《高等数学(第六版 上册)》 P326
函数\(1,\cos^2 x, \sin^2 x\)在整个数轴上是线性相关的.
因为取\(k_1=1,k_2=k_3=-1\),就有恒等式\begin{equation*}
	1 - \cos^2 x - \sin^2 x \equiv 0.
\end{equation*}

函数\(1,x,x^2\)在任何区间\((a,b)\)内是线性无关的.
因为如果\(k_1,k_2,k_3\)不全为零,
那么在该区间内至多只有两个\(x\)值能使二次三项式\begin{equation*}
	k_1 + k_2 x + k_3 x^2
\end{equation*}为零;而要使它恒等于零,必须\(k_1,k_2,k_3\)全为零.
\end{example}

%@see: 《高等数学(第六版 上册)》 P326
应用上述概念可知,对于两个函数的情形,它们线性相关与否,只要看它们的比是否为常数:
如果比为常数,那么它们就线性相关;否则就线性无关.

有了一组函数线性相关或线性无关的概念后,
我们有如下关于二阶齐次线性微分方程的通解结构的定理.
\begin{theorem}\label{theorem:微分方程.二阶齐次线性微分方程的通解结构}
%@see: 《高等数学(第六版 上册)》 P326 定理2
如果\(y_1(x)\)与\(y_2(x)\)是
二阶齐次线性方程 \labelcref{equation:微分方程.二阶齐次线性微分方程的一般形式} 的
两个线性无关的特解,那么\begin{equation*}
	y = C_1 y_1(x) + C_2 y_2(x)
	\quad\text{\(C_1,C_2\)是任意常数}
\end{equation*}
就是二阶齐次线性方程 \labelcref{equation:微分方程.二阶齐次线性微分方程的一般形式} 的通解.
\end{theorem}

\begin{example}
%@see: 《高等数学(第六版 上册)》 P326
方程\(y'' + y = 0\)是二阶齐次线性方程.
这里\begin{equation*}
	P(x)\equiv0,
	\qquad
	Q(x)\equiv1.
\end{equation*}
容易验证\begin{equation*}
	y_1 = \cos x
	\quad\text{与}\quad
	y_2 = \sin x
\end{equation*}
是所给方程的两个解,
且\begin{equation*}
	\frac{y_2}{y_1}
	= \frac{\sin x}{\cos x}
	= \tan x
\end{equation*}不是常数函数,
即它们是线性无关的.
因此方程\(y'' + y = 0\)的通解为\begin{equation*}
	y = C_1 \cos x + C_2 \sin x.
\end{equation*}
\end{example}
\begin{example}
%@see: 《高等数学(第六版 上册)》 P327
方程\((x-1) y'' - x y' + y = 0\)是二阶齐次线性方程.
这里\begin{equation*}
	P(x)=-\frac{x}{x-1},
	\qquad
	Q(x)=\frac1{x-1}.
\end{equation*}
容易验证\begin{equation*}
	y_1 = x
	\quad\text{与}\quad
	y_2 = e^x
\end{equation*}是所给方程的两个解,
且\begin{equation*}
	\frac{y_2}{y_1}
	= \frac{e^x}{x}
\end{equation*}不是常数函数,
即它们是线性无关的.
因此方程\((x-1) y'' - x y' + y = 0\)的通解为\begin{equation*}
	y = C_1 x + C_2 e^x.
\end{equation*}
\end{example}

\cref{theorem:微分方程.二阶齐次线性微分方程的通解结构} 不难推广到\(n\)阶齐次线性方程.
\begin{corollary}\label{theorem:微分方程.n阶齐次线性微分方程的通解结构}
%@see: 《高等数学(第六版 上册)》 P327 推论
如果\(y_1(x),y_2(x),\dotsc,y_n(x)\)是\(n\)阶齐次线性方程\begin{equation*}
	y^{(n)} + a_1(x) y^{(n-1)} + \dotsb + a_{n-1}(x) y' + a_n(x) y = 0
\end{equation*}的\(n\)个线性无关的解,
那么此方程的通解为\begin{equation*}
	y = C_1 y_1 + C_2 y_2 + \dotsb + C_n y_n,
\end{equation*}
其中\(\AutoTuple{C}{n}\)为任意常数.
\end{corollary}

下面讨论二阶非齐次线性方程 \labelcref{equation:微分方程.二阶非齐次线性微分方程的一般形式}.
我们把二阶齐次线性方程 \labelcref{equation:微分方程.二阶齐次线性微分方程的一般形式} 叫做%
“与非齐次方程 \labelcref{equation:微分方程.二阶非齐次线性微分方程的一般形式} 对应的齐次方程”.


在\cref{section:微分方程.一阶线性微分方程}我们已经看到,一阶非齐次线性微分方程的通解由两部分构成:
一部分是对应的齐次方程的通解;另一部分是非齐次方程本身的一个特解.
实际上,不仅一阶非齐次线性微分方程的通解具有这样的结构,
而且二阶及更高阶的非齐次线性微分方程的特解也具有同样的结构.
\begin{theorem}\label{theorem:微分方程.二阶非齐次线性微分方程的通解结构}
%@see: 《高等数学(第六版 上册)》 P327 定理3
设\(\hat{y}(x)\)是二阶非齐次线性微分方程 \labelcref{equation:微分方程.二阶非齐次线性微分方程的一般形式} 的一个特解,
\(Y(x)\)是与 \labelcref{equation:微分方程.二阶非齐次线性微分方程的一般形式} 对应的
二阶齐次线性方程 \labelcref{equation:微分方程.二阶齐次线性微分方程的一般形式} 的通解,
那么\begin{equation}\label{equation:微分方程.二阶非齐次线性微分方程的通解结构}
	y=Y(x)+\hat{y}(x)
\end{equation}是二阶非齐次线性微分方程的通解.
\begin{proof}
把\cref{equation:微分方程.二阶非齐次线性微分方程的通解结构}
代入\cref{equation:微分方程.二阶非齐次线性微分方程的一般形式} 左端,
得\begin{align*}
	&\hspace{-20pt}
	(Y'' + \hat{y}'')
	+ P(x) (Y' + \hat{y}')
	+ Q(x) (Y + \hat{y}) \\
	&= [Y'' + P(x) Y' + Q(x) Y]
	+ [\hat{y}'' + P(x) \hat{y}' + Q(x) \hat{y}].
\end{align*}
由于\(Y\)是\cref{equation:微分方程.二阶齐次线性微分方程的一般形式} 的解,
\(\hat{y}\)是\cref{equation:微分方程.二阶非齐次线性微分方程的一般形式} 的解,
可知上式右端第一个方括号内表达式恒等于零,第二个恒等于\(f(x)\).
这样,\(y = Y + \hat{y}\)使
\cref{equation:微分方程.二阶非齐次线性微分方程的一般形式} 的两端恒等,
说明\cref{equation:微分方程.二阶非齐次线性微分方程的通解结构}
是\cref{equation:微分方程.二阶非齐次线性微分方程的一般形式} 的解.

由于对应的二阶齐次线性方程 \labelcref{equation:微分方程.二阶齐次线性微分方程的一般形式} 的通解
\(Y = C_1 y_1 + C_2 y_2\)中含有两个任意常数,
所以\(y = Y + \hat{y}\)中也含有两个任意常数,
因此它就是二阶非齐次线性微分方程 \labelcref{equation:微分方程.二阶非齐次线性微分方程的一般形式} 的通解.
\end{proof}
\end{theorem}

\begin{example}
%@see: 《高等数学(第六版 上册)》 P327
方程\(y'' + y = x^2\)是二阶非齐次线性微分方程.
已知\(Y = C_1 \cos x + C_2 \sin x\)是对应的齐次方程\(y'' + y = 0\)的通解.
又容易验证\(\hat{y} = x^2 - 2\)是所给方程的一个特解.
因此\begin{equation*}
	y = C_1 \cos x + C_2 \sin x + x^2 - 2
\end{equation*}是所给方程的通解.
\end{example}

非齐次线性微分方程 \labelcref{equation:微分方程.二阶非齐次线性微分方程的一般形式} 的特解
有时可用下述定理来帮助求出.
\begin{theorem}\label{theorem:微分方程.二阶非齐次线性微分方程的解的叠加原理}
设非齐次线性方程 \labelcref{equation:微分方程.二阶非齐次线性微分方程的一般形式}
的右端\(f(x)\)是两个函数之和,即\begin{equation*}
	y''+P(x)y'+Q(x)y
	=f(x)
	=f_1(x)+f_2(x);
\end{equation*}而\(\hat{y}_1(x)\)与\(\hat{y}_2(x)\)分别是方程\begin{equation*}
	y''+P(x)y'+Q(x)y=f_1(x)
\end{equation*}与\begin{equation*}
	y''+P(x)y'+Q(x)y=f_2(x)
\end{equation*}的特解,那么\(\hat{y}_1(x)+\hat{y}_2(x)\)就是原方程的特解.
\end{theorem}

\cref{theorem:微分方程.二阶非齐次线性微分方程的解的叠加原理}
通常称为线性微分方程的解的\DefineConcept{叠加原理}.

\cref{theorem:微分方程.二阶非齐次线性微分方程的通解结构}
和\cref{theorem:微分方程.二阶非齐次线性微分方程的解的叠加原理}
也可推广到\(n\)阶非齐次线性方程,这里不再赘述.

在\cref{section:微分方程.一阶线性微分方程}为了解一阶非齐次线性方程,我们采用了常数变易法.
该方法的特点是:
如果\(C y_1(x)\)是齐次线性方程的通解,那么,
可以利用变换\(y = u y_1(x)\)(这变换是把齐次方程的通解中的
任意常数\(C\)换成未知函数\(u(x)\)而得到的)去解非齐次线性方程.
这一方法也适用于解高阶线性方程.
下面就二阶线性方程来作讨论.

\begin{enumerate}
	\item%第一种情况
	如果已知二阶齐次线性方程 \labelcref{equation:微分方程.二阶齐次线性微分方程的一般形式} 的通解为\begin{equation*}
		Y(x) = C_1 y_1(x) + C_2 y_2(x),
	\end{equation*}
	那么,可以用如下的“常数变易法”去求非齐次方程 \labelcref{equation:微分方程.二阶非齐次线性微分方程的一般形式} 的通解:
	令\begin{equation}\label{equation:微分方程.二阶非齐次线性微分方程的通解设想}
		y = y_1 v_1 + y_2 v_2,
	\end{equation}
	要确定未知函数\(v_1\)及\(v_2\)
	使函数 \labelcref{equation:微分方程.二阶非齐次线性微分方程的通解设想} 满足
	非齐次方程 \labelcref{equation:微分方程.二阶非齐次线性微分方程的一般形式}.
	为此,对函数 \labelcref{equation:微分方程.二阶非齐次线性微分方程的通解设想} 求导,得\begin{equation*}
		y' = y_1 v_1' + y_2 v_2' + y_1' v_1 + y_2' v_2.
	\end{equation*}
	由于两个未知函数\(v_1,v_2\)
	只需使函数 \labelcref{equation:微分方程.二阶非齐次线性微分方程的通解设想} 满足
	一个关系式 \labelcref{equation:微分方程.二阶非齐次线性微分方程的一般形式},
	所以可规定它们再满足一个关系式.
	从\(y'\)的上述表示式可以看出,
	为了使\(y''\)的表示式中不含\(v_1''\)和\(v_2''\),
	可设\begin{equation}\label{equation:微分方程.二阶非齐次线性微分方程的额外条件1}
		y_1 v_1' + y_2 v_2' = 0,
	\end{equation}
	从而\begin{equation*}
		y' = y_1' v_1 + y_2' v_2,
	\end{equation*}
	再求导,得\begin{equation*}
		y'' = y_1' v_1' + y_2' v_2' + y_1'' v_1 + y_2'' v_2.
	\end{equation*}
	把\(y,y',y''\)代入非齐次方程 \labelcref{equation:微分方程.二阶非齐次线性微分方程的一般形式},
	得\begin{equation*}
		(y_1' v_1' + y_2' v_2' + y_1'' v_1 + y_2'' v_2)
		+P\cdot(y_1' v_1 + y_2' v_2) + Q\cdot(y_1 v_1 + y_2 v_2) = f,
	\end{equation*}
	整理得\begin{equation*}
		y_1' v_1' + y_2' v_2'
		+ (y_1'' + P y_1' + Q y_1) v_1
		+ (y_2'' + P y_2' + Q y_2) v_2
		= f.
	\end{equation*}
	注意到\(y_1,y_2\)是
	二阶齐次线性方程 \labelcref{equation:微分方程.二阶齐次线性微分方程的一般形式} 的解,
	故上式即为
	\begin{equation}\label{equation:微分方程.二阶非齐次线性微分方程的额外条件2}
		y_1' v_1' + y_2' v_2' = f.
	\end{equation}
	联立方程 \labelcref{equation:微分方程.二阶非齐次线性微分方程的额外条件1}
	与 \labelcref{equation:微分方程.二阶非齐次线性微分方程的额外条件2},
	得\begin{equation*}
		\left\{ \begin{array}{l}
			y_1 v_1' + y_2 v_2' = 0, \\
			y_1' v_1' + y_2' v_2' = f,
		\end{array} \right.
		\quad\text{即}\quad
		\begin{bmatrix}
			y_1 & y_2 \\
			y_1' & y_2'
		\end{bmatrix}
		\begin{bmatrix}
			v_1' \\ v_2'
		\end{bmatrix}
		= \begin{bmatrix}
			0 \\ f
		\end{bmatrix},
	\end{equation*}
	在系数行列式\begin{equation*}
		W = \begin{vmatrix} y_1 & y_2 \\ y_1' & y_2' \end{vmatrix}
		= y_1 y_2' - y_1 y_2 \neq0
	\end{equation*}时,
	利用\hyperref[theorem:线性方程组.克拉默法则]{克拉默法则}可解得\begin{equation*}
		v_1'
		= \frac1W \begin{vmatrix}
			0 & y_2 \\
			f & y_2'
		\end{vmatrix}
		= -\frac{y_2 f}{W},
		\qquad
		v_2'
		= \frac1W \begin{vmatrix}
			y_1 & 0 \\
			y_1' & f
		\end{vmatrix}
		= \frac{y_1 f}{W}.
	\end{equation*}
	对上两式积分(假定\(f\)连续),得\begin{equation*}
		v_1 = C_1 + \int \left(-\frac{y_2 f}{W}\right) \dd{x},
		\qquad
		v_2 = C_2 + \int \frac{y_1 f}{W} \dd{x}.
	\end{equation*}

	将\(v_1,v_2\)代回\cref{equation:微分方程.二阶非齐次线性微分方程的通解设想},
	于是得到非齐次方程 \labelcref{equation:微分方程.二阶非齐次线性微分方程的一般形式} 的通解
	\begin{equation}\label{equation:微分方程.二阶非齐次线性微分方程的通解.形式1}
		y = C_1 y_1 + C_2 y_2
		- y_1 \int \frac{y_2 f}{W} \dd{x}
		+ y_2 \int \frac{y_1 f}{W} \dd{x}.
	\end{equation}

	\item%第二种情况
	如果只知齐次方程 \labelcref{equation:微分方程.二阶齐次线性微分方程的一般形式}
	的一个不恒为零的解\(y_1(x)\),
	那么利用变换\(y = u y_1(x)\),
	可把非齐次方程 \labelcref{equation:微分方程.二阶非齐次线性微分方程的一般形式}
	化为一阶线性方程.

	事实上,把\begin{equation*}
		y = y_1 u, \qquad y' = y_1 u' + y_1' u, \qquad y'' = y_1 u'' + 2 y_1' u' + y_1'' u
	\end{equation*}
	代入方程 \labelcref{equation:微分方程.二阶非齐次线性微分方程的一般形式},
	得\begin{equation*}
		(y_1 u'' + 2y_1' u' + y_1'' u) + P\cdot(y_1 u' + y_1' u) + Q \cdot y_1 u = f,
	\end{equation*}
	即\begin{equation*}
		y_1 u'' + (2y_1' + P y_1) u' + (y_1'' + P y_1' + Q y_1) u = f,
	\end{equation*}
	由于\(y_1\)是
	二阶齐次线性方程 \labelcref{equation:微分方程.二阶齐次线性微分方程的一般形式} 的解,
	即有\(y_1'' + P y_1' + Q y_1 \equiv 0\),
	故上式化为\begin{equation*}
		y_1 u'' + (2 y_1' + P y_1) u' = f.
	\end{equation*}
	令\(u' = z\),上式即化为一阶线性方程
	\begin{equation}\label{equation:微分方程.降为一阶的二阶线性微分方程}
		y_1 z' + (2 y_1' + P y_1) z = f.
	\end{equation}

	把二阶非齐次方程 \labelcref{equation:微分方程.二阶非齐次线性微分方程的一般形式}
	化为一阶方程 \labelcref{equation:微分方程.降为一阶的二阶线性微分方程} 以后,
	按\hyperref[section:微分方程.一阶线性微分方程]{一阶线性方程的解法},
	设求得一阶方程 \labelcref{equation:微分方程.降为一阶的二阶线性微分方程} 的通解为\begin{equation*}
		u' = z = C_2 Z(x) + \hat{z}(x),
	\end{equation*}
	积分得\begin{equation*}
		u = C_1 + C_2 U(x) + \hat{u}(x),
	\end{equation*}
	其中\(U'(x) = Z(x), [\hat{u}(x)]' = \hat{z}(x)\).
	上式乘以\(y_1(x)\),便得方程 \labelcref{equation:微分方程.二阶非齐次线性微分方程的一般形式} 的通解
	\begin{equation}\label{equation:微分方程.二阶非齐次线性微分方程的通解.形式2}
		y = C_1 y_1(x) + C_2 U(x) y_1(x) + \hat{u}(x) y_1(x).
	\end{equation}
	上述方法显然也适用于求
	二阶齐次线性方程 \labelcref{equation:微分方程.二阶齐次线性微分方程的一般形式} 的通解.
\end{enumerate}

\begin{example}
%@see: 《高等数学(第六版 上册)》 P329 例3
已知齐次方程\((x-1)y'' - xy' + y = 0\)的通解为
\(Y(x) = C_1 x + C_2 e^x\),
求非齐次方程\((x-1)y'' - xy' + y = (x-1)^2\)的通解.
\begin{solution}
把所给方程写成标准形式\begin{equation*}
	y'' - \frac{x}{x-1} y' + \frac{1}{x-1} y = x-1.
\end{equation*}

应用常数变易法,令\(y = x v_1 + e^x v_2\).由\begin{equation*}
	\left\{ \begin{array}{l}
		x v_1' + e^x v_2' = 0, \\
		v_1' + e^x v_2' = x-1,
	\end{array} \right.
\end{equation*}
解得\(v_1' = -1, v_2' = x e^{-x}\).
再积分,得\(v_1 = C_1 - x, v_2 = C_2 - (x+1) e^{-x}\).
于是所求非齐次方程的通解为\begin{equation*}
	y = C_1 x + C_2 e^x - (x^2+x+1).
\end{equation*}
\end{solution}
\end{example}

\begin{example}
%@see: 《高等数学(第六版 上册)》 P330 例4
已知\(y_1(x) = e^x\)是齐次方程\(y'' - 2y' + y = 0\)的解,
求非齐次方程\(y'' - 2y' + y = x^{-1} e^x\)的通解.
\begin{solution}
令\(y = e^x u\),
则\(y' = e^x (u'+u), y'' = e^x (u''+2u'+u)\).
代入非齐次方程,得\begin{equation*}
	e^x (u''+2u'+u) - 2e^x (u'+u) + e^x u = \frac{1}{x} e^x,
\end{equation*}
即\(e^x u'' = x^{-1} e^x, u'' = x^{-1}\).

这里不需再作变换去化为一阶线性方程,只要直接积分,便得\begin{equation*}
	u' = C + \ln\abs{x},
	\qquad
	u = C_1 + C x + x\ln\abs{x} - x,
\end{equation*}
即\(u = C_1 + C_2 x + x \ln\abs{x}\ (C_2=C-1)\).
\end{solution}
\end{example}

\begin{example}
%@see: 《高等数学(第六版 上册)》 P331 习题7-6 5.
已知\(y_1(x) = e^x\)是齐次线性方程\begin{equation*}
	(2x-1)y'' - (2x+1)y' + 2y = 0
\end{equation*}的一个解,求此方程的通解.
\begin{solution}
将\(y_2(x) = u(x) e^x,
y_2'(x) = (u(x) + u'(x)) e^x,
y_2''(x) = (u(x) + 2 u'(x) + u''(x)) e^x\)代入原方程得\begin{equation*}
	(2x-1) (u(x) + 2 u'(x) + u''(x)) e^x
	- (2x+1) (u(x) + u'(x)) e^x
	+ 2 u(x) e^x
	= 0,
\end{equation*}
整理得\begin{equation*}
	(2x-1)u'' + (2x-3)u' = 0,
\end{equation*}
解得\(u = C_1 e^{-x} (2x+1) + C_2\),
于是原方程的通解为\begin{equation*}
	Y(x) = (C_1 e^{-x} (2x+1) + C_2) e^x
	= C_1 (2x+1) + C_2 e^x.
\end{equation*}
\end{solution}
\end{example}

\section{二阶常系数齐次线性微分方程}\label{section:微分方程.常系数齐次线性微分方程}
先讨论二阶常系数齐次线性微分方程的解法,再把二阶方程的解法推广到\(n\)阶.

\begin{definition}
在二阶齐次线性微分方程 \labelcref{equation:微分方程.二阶齐次线性微分方程的一般形式} 中,
如果\(y'\)、\(y\)的系数\(P(x)\)、\(Q(x)\)均为常数,
即方程 \labelcref{equation:微分方程.二阶齐次线性微分方程的一般形式} 成为
\begin{equation}\label{equation:微分方程.二阶常系数齐次线性微分方程的一般形式}
	y''+py'+qy=0.
\end{equation}
其中\(p\)、\(q\)是常数,
则称方程 \labelcref{equation:微分方程.二阶常系数齐次线性微分方程的一般形式}
为\DefineConcept{二阶常系数齐次线性微分方程}.
如果\(p\)、\(q\)不全为常数,
称方程 \labelcref{equation:微分方程.二阶齐次线性微分方程的一般形式}
为\DefineConcept{二阶变系数齐次线性微分方程}.
\end{definition}

由上节讨论可知,要找微分方程 \labelcref{equation:微分方程.二阶常系数齐次线性微分方程的一般形式} 的通解,
可以先求出它的两个解\(y_1, y_2\),如果\(\frac{y_2}{y_1} \not\equiv \text{常数}\),
即\(y_1\)与\(y_2\)线性无关,
那么\(y = C_1 y_1 + C_2 y_2\)就是方程 \labelcref{equation:微分方程.二阶常系数齐次线性微分方程的一般形式} 的通解.

当\(r\)为常数时,指数函数\(y = e^{r x}\)和它的各阶导数都只相差一个常数因子.
由于指数函数有这个特点,因此我们用\(y = e^{rx}\)来尝试,看能否选取适当的常数\(r\),
使\(y = e^{rx}\)满足方程 \labelcref{equation:微分方程.二阶常系数齐次线性微分方程的一般形式}.

对\(y = e^{rx}\)求导\footnote{当\(r\in\mathbb{C}\),\(x\)为实变数时,
导数公式\(\dv{x} e^{rx} = r e^{rx}\)仍成立.},得到\begin{equation*}
	y' = r e^{rx}, \qquad y'' = r^2 e^{rx}.
\end{equation*}
把\(y,y',y''\)代入方程 \labelcref{equation:微分方程.二阶常系数齐次线性微分方程的一般形式},得\begin{equation*}
	(r^2 + pr + q) e^{rx} = 0.
\end{equation*}
由于\(e^{rx}\neq0\),
所以\begin{equation}\label{equation:微分方程.二阶常系数齐次线性微分方程的特征方程}
	r^2 + pr + q = 0.
\end{equation}

由此可见,只要\(r\)满足代数方程 \labelcref{equation:微分方程.二阶常系数齐次线性微分方程的特征方程},
函数\(y = e^{rx}\)就是微分方程 \labelcref{equation:微分方程.二阶常系数齐次线性微分方程的一般形式} 的解,
我们把代数方程 \labelcref{equation:微分方程.二阶常系数齐次线性微分方程的特征方程} 叫做
“微分方程 \labelcref{equation:微分方程.二阶常系数齐次线性微分方程的一般形式} 的\DefineConcept{特征方程}”.
注意到特征方程 \labelcref{equation:微分方程.二阶常系数齐次线性微分方程的特征方程} 是一个二次代数方程,
其中\(r^2, r\)的系数及常数项恰好依次是
微分方程 \labelcref{equation:微分方程.二阶常系数齐次线性微分方程的一般形式} 中\(y'',y',y\)的系数.

特征方程 \labelcref{equation:微分方程.二阶常系数齐次线性微分方程的特征方程} 的两个根\(r_1,r_2\)可以用公式\begin{equation*}
r_{1,2} = \frac{-p \pm\sqrt{p^2-4q}}{2}
\end{equation*}求出.
它们有三种不同的情形:\begin{enumerate}
	\item 当\(p^2-4q>0\)时,\(r_1,r_2\)是两个不相等的实根:\begin{equation*}
		r_1 = \frac{-p +\sqrt{p^2-4q}}{2}, \qquad
		r_2 = \frac{-p -\sqrt{p^2-4q}}{2};
	\end{equation*}
	\item 当\(p^2-4q=0\)时,\(r_1,r_2\)是两个相等的实根:\begin{equation*}
		r_1 = r_2 = -\frac{p}{2};
	\end{equation*}
	\item 当\(p^2-4q<0\)时,\(r_1,r_2\)是一对共轭复根:\begin{equation*}
		r_1 = \alpha+\iu\beta, \qquad r_2 = \alpha-\iu\beta,
	\end{equation*}
	其中\(\alpha=-p/2\),\(\beta=\sqrt{4q-p^2}/2\).
\end{enumerate}

相应地,微分方程 \labelcref{equation:微分方程.二阶常系数齐次线性微分方程的一般形式} 的通解也有三种不同的情形.
分别讨论如下:\begin{enumerate}
	\item 特征方程有两个不相等的实根:\(r_1 \neq r_2\).

	由上面的讨论知道,\(y_1 = e^{r_1 x}\)和\(y_2 = e^{r_2 x}\)是
	微分方程 \labelcref{equation:微分方程.二阶常系数齐次线性微分方程的一般形式} 的两个解,
	并且\begin{equation*}
		\frac{y_2}{y_1} = \frac{e^{r_2 x}}{e^{r_1 x}} = e^{(r_2-r_1) x}
	\end{equation*}不是常数,
	因此微分方程 \labelcref{equation:微分方程.二阶常系数齐次线性微分方程的一般形式} 的
	通解为\begin{equation}
		y = C_1 e^{r_1 x} + C_2 e^{r_2 x}.
	\end{equation}

	\item 特征方程有两个相等的实根:\(r_1 = r_2\).

	这时,只得到微分方程 \labelcref{equation:微分方程.二阶常系数齐次线性微分方程的一般形式} 的一个解\begin{equation*}
		y_1 = e^{r_1 x}.
	\end{equation*}
	为了得出微分方程 \labelcref{equation:微分方程.二阶常系数齐次线性微分方程的一般形式} 的通解,
	还需求出另一个解\(y_2\),
	并且,根据\cref{theorem:微分方程.二阶齐次线性微分方程的通解结构},
	我们要求这两个解的比\(\frac{y_2}{y_1}\)不是常数.

	设\(\frac{y_2}{y_1} = u(x)\),即\(y_2 = e^{r_1 x} u(x)\).
	下面来求\(u(x)\).

	对\(y_2\)求导,得\begin{equation*}
		y_2' = e^{r_1 x} (u' + r_1 u),
		\qquad
		y_2'' = e^{r_1 x} (u'' + 2 r_1 u' + r_1^2 u),
	\end{equation*}
	将\(y_2,y_2',y_2''\)
	代入微分方程 \labelcref{equation:微分方程.二阶常系数齐次线性微分方程的一般形式},
	得\begin{equation*}
		e^{r_1 x} [(u'' + 2 r_1 u' + r_1^2 u) + p(u' + r_1 u) + qu] = 0,
	\end{equation*}
	约去\(e^{r_1 u}\),
	并以\(u'',u',u\)为准合并同类项,
	得\begin{equation*}
		u'' + (2 r_1 + p) u' + (r_1^2 + p r_1 + q) u = 0.
	\end{equation*}
	由于\(r_1\)是特征方程 \labelcref{equation:微分方程.二阶常系数齐次线性微分方程的特征方程} 的二重根,
	因此\(r_1^2 + p r_1 + q = 0\),且\(2 r_1 + p = 0\),
	于是得\begin{equation*}
		u'' = 0.
	\end{equation*}
	因为这里只要得到一个不为常数的解,所以不妨选取\(u = x\),
	由此得到微分方程 \labelcref{equation:微分方程.二阶常系数齐次线性微分方程的一般形式} 的另一个解\begin{equation*}
		y_2 = x e^{r_1 x}.
	\end{equation*}

	从而微分方程 \labelcref{equation:微分方程.二阶常系数齐次线性微分方程的一般形式} 的
	通解为\begin{equation}
		y = (C_1 + C_2 x) e^{r_1 x}.
	\end{equation}

	\item 特征方程有一对共轭复根:\(r_1 = \alpha+\iu\beta, r_2 = \alpha-\iu\beta \ (\beta\neq0)\).

	这时,\(y_1 = e^{(\alpha+\iu\beta)x}, y_2 = e^{(\alpha-\iu\beta)x}\)是微分方程 \labelcref{equation:微分方程.二阶常系数齐次线性微分方程的一般形式} 的两个解,但它们是复值函数形式.
	为了得出实值函数形式的解,
	先利用\hyperref[equation:复数.欧拉公式]{欧拉公式},
	即\(e^{\iu\theta} \equiv \cos\theta+\iu\sin\theta\),
	把\(y_1,y_2\)改写为\begin{equation*}
		y_1 = e^{(\alpha+\iu\beta)x} = e^{\alpha x} \cdot e^{\iu\beta x}
		= e^{\alpha x} (\cos\beta x + \iu \sin\beta x),
	\end{equation*}\begin{equation*}
		y_2 = e^{(\alpha-\iu\beta)x} = e^{\alpha x} \cdot e^{-\iu\beta x}
		= e^{\alpha x} (\cos\beta x - \iu \sin\beta x),
	\end{equation*}
	由于复值函数\(y_1\)与\(y_2\)之间成共轭关系,
	它们的实部相等,虚部恰好相反,\begin{equation*}
		\Re y_1 = \Re y_2, \qquad
		\Im y_1 = -\Im y_2,
	\end{equation*}
	于是\(y_1,y_2\)可以由它们的实部、虚部线性表出,
	那么根据\hyperref[theorem:微分方程.二阶非齐次线性微分方程的解的叠加原理]{叠加原理},
	它们的实部、虚部\begin{equation*}
		z_1 = \frac{1}{2} (y_1 + y_2) = e^{\alpha x} \cos\beta x,
		\qquad
		z_2 = \frac{1}{2\iu} (y_1 - y_2) = e^{\alpha x} \sin\beta x
	\end{equation*}也是方程 \labelcref{equation:微分方程.二阶常系数齐次线性微分方程的一般形式} 的解.
	又因为\begin{equation*}
		\frac{z_1}{z_2}
		= \frac{e^{\alpha x} \cos\beta x}{e^{\alpha x} \sin\beta x}
		= \cot\beta x
	\end{equation*}不是常数,
	所以微分方程 \labelcref{equation:微分方程.二阶常系数齐次线性微分方程的一般形式} 的
	通解为\begin{equation}
		y = e^{\alpha x} (C_1 \cos\beta x + C_2 \sin\beta x).
	\end{equation}
\end{enumerate}

综上所述,
求二阶常系数齐次线性微分方程 \labelcref{equation:微分方程.二阶常系数齐次线性微分方程的一般形式} 的通解的步骤如下:
\begin{algorithm}[求解二阶常系数齐次线性微分方程]
\hfill
\begin{enumerate}
	\item
	写出微分方程 \labelcref{equation:微分方程.二阶常系数齐次线性微分方程的一般形式} 的
	特征方程 \labelcref{equation:微分方程.二阶常系数齐次线性微分方程的特征方程}.

	\item
	求出特征方程 \labelcref{equation:微分方程.二阶常系数齐次线性微分方程的特征方程} 的
	两个根\(r_1,r_2\).

	\item
	根据特征方程 \labelcref{equation:微分方程.二阶常系数齐次线性微分方程的特征方程} 的两个根的不同情形,
	按照\cref{table:二阶常系数齐次线性微分方程.通解}
	写出微分方程 \labelcref{equation:微分方程.二阶常系数齐次线性微分方程的一般形式} 的通解.
	\begin{table}[htb]
		\centering
		\begin{tblr}{c|c}
			\hline
			特征方程\(r^2+pr+q=0\)的两个根\(r_1,r_2\)
				& 微分方程\(y''+py'+qy = 0\)的通解 \\ \hline
			两个不相等的实根\(r_1,r_2\)
				& \(y = C_1 e^{r_1 x} + C_2 e^{r_2 x}\) \\
			两个相等的实根\(r_1=r_2=r\)
				& \(y = (C_1 + C_2 x) e^{r x}\) \\
			一对共轭复根\(r_{1,2}=\alpha\pm\iu\beta\)
				& \(y = e^{ax} (C_1 \cos{\beta x} + C_2 \sin{\beta x})\) \\ \hline
		\end{tblr}
		\caption{}
		\label{table:二阶常系数齐次线性微分方程.通解}
	\end{table}
\end{enumerate}
\end{algorithm}
\begin{example}
求微分方程\(y'' - 2y' - 3y = 0\)的通解.
\begin{solution}
所给微分方程的特征方程为\(r^2-2r-3=0\),
其根\(r_1=-1, r_2=3\)是两个不同的实根,因此所求通解为\begin{equation*}
	y = C_1 e^{-x} + C_2 e^{3x}.
\end{equation*}
\end{solution}
\end{example}

\begin{example}
%@see: 《2023年全国硕士研究生入学统一考试(数学一)》一选择题/第2题
若微分方程\(y''+ay'+by=0\)的解在\((-\infty,+\infty)\)上有界,求常数\(a,b\)的取值范围.
\begin{solution}
微分方程\(y''+ay'+by=0\)的特征方程为\(\lambda^2 + a\lambda+b = 0\).

当\(\Delta=a^2-4b>0\)时,特征方程有两个不同的实根\(\lambda_1,\lambda_2\),
且\(\lambda_1,\lambda_2\)至少有一个不等于零.
只要常数\(C_1,C_2\)都不为零,
那么微分方程的通解\(y = C_1 e^{\lambda_1 x} + C_2 e^{\lambda_2 x}\)在\((-\infty,+\infty)\)内无界.

当\(\Delta=a^2-4b=0\)时,特征方程有两个相同的实根\(\lambda=-a/2\).
只要常数\(C_2\neq0\),
那么微分方程的通解\(y = C_1 e^{-ax/2} + C_2 x e^{-ax/2}\)在\((-\infty,+\infty)\)内无界.

当\(\Delta=a^2-4b<0\)时,
特征方程有两个共轭复根\(\lambda_{1,2}=-\frac{a}{2}\pm\frac{\sqrt{4b-a^2}}{2}\iu\),
微分方程的通解为\begin{equation*}
	y = e^{-\frac{a}{2}x} \left( C_1 \cos\frac{\sqrt{4b-a^2}}{2}x + C_2 \sin\frac{\sqrt{4b-a^2}}{2}x \right),
\end{equation*}
此时,要使该通解在\((-\infty,+\infty)\)内有界,则必有\(a = 0, b > 0\).
\end{solution}
\end{example}

\begin{example}
%@see: 《2016年全国硕士研究生入学统一考试(数学一)》三解答题/第16题
设函数\(y(x)\)满足方程\(y'' + 2 y' + k y = 0\),
其中\(0<k<1\).
\begin{itemize}
	\item 证明:反常积分\(\int_0^{+\infty} y(x) \dd{x}\)收敛;
	\item 若\(y(0) = 1,
	y'(0) = 1\),
	求\(\int_0^{+\infty} y(x) \dd{x}\)的值.
\end{itemize}
\begin{solution}
写出方程\(y'' + 2 y' + k y = 0\)的特征方程\begin{equation*}
	f(r) = r^2 + 2 r + k = 0,
\end{equation*}
其判别式为\(\Delta = 4 - 4 k = 4 (1-k) > 0\),
故特征方程有两个相异实根\(r_1,r_2\).
将通解\begin{equation*}
	y(x) = C_1 e^{r_1 x} + C_2 e^{r_2 x}\ (\text{$C_1,C_2$是任意常数})
	\eqno(1)
\end{equation*}
代入\(\int_0^{+\infty} y(x) \dd{x}\)得\begin{equation*}
	\int_0^{+\infty} \left( C_1 e^{r_1 x} + C_2 e^{r_2 x} \right) \dd{x}
	= \frac{C_1}{r_1} \eval{e^{r_1 x}}_0^{+\infty}
	+ \frac{C_2}{r_2} \eval{e^{r_2 x}}_0^{+\infty}.
	\eqno(2)
\end{equation*}
因为\(r_1 + r_2 = -2 < 0\)且\(f(0) = k > 0\),
所以\(r_1,r_2\)均小于零,
从而有\begin{equation*}
	\lim_{x\to+\infty} e^{r_1 x}
	= \lim_{x\to+\infty} e^{r_2 x}
	= 0,
	\eqno(3)
\end{equation*}
于是(2)式等于\(-\frac{C_1}{r_1}-\frac{C_2}{r_2}\).

由方程\(y'' + 2 y' + k y = 0\)
得\(y(x) = -\frac1k \left( y''(x) + 2 y'(x) \right)\),
积分得\begin{align*}
	\int_0^{+\infty} y(x) \dd{x}
	&= -\frac1k \int_0^{+\infty} \left( y''(x) + 2 y'(x) \right) \dd{x}
	= -\frac1k \eval{\left( y'(x) + 2 y(x) \right)}_0^{+\infty} \\
	&= -\frac1k \left[
		\lim_{x\to+\infty} \left( y'(x) + 2 y(x) \right)
		- \left( y'(0) + 2 y(0) \right)
	\right].
	\tag4
\end{align*}
由(3)式可知\begin{equation*}
	\lim_{x\to+\infty} y(x)
	= \lim_{x\to+\infty} \left( C_1 e^{r_1 x} + C_2 e^{r_2 x} \right)
	= 0,
	\qquad
	\lim_{x\to+\infty} y'(x)
	= \lim_{x\to+\infty} \left( r_1 C_1 e^{r_1 x} + r_2 C_2 e^{r_2 x} \right)
	= 0,
\end{equation*}
于是,当\(y(0)=y'(0)=1\)时,
(4)式等于\(
	\frac1k (y'(0) + 2 y(0))
	= \frac3k
\).
\end{solution}
\end{example}
\begin{example}
%@see: 《2020年全国硕士研究生入学统一考试(数学一)》二填空题/第11题
设函数\(f\)满足方程\(f''(x) + a f'(x) + f(x) = 0\ (a>0)\),
且\(f(0) = m,
f'(0) = n\).
计算反常积分\(\int_0^{+\infty} f(x) \dd{x}\).
\begin{solution}
由\(f''(x) + a f'(x) + f(x) = 0\)
有\(f(x) = - f''(x) - a f'(x)\),
积分得\begin{align*}
	\int_0^{+\infty} f(x) \dd{x}
	&= \int_0^{+\infty} (- f''(x) - a f'(x)) \dd{x}
	= \eval{(- f'(x) - a f(x))}_0^{+\infty} \\
	&= \lim_{x\to+\infty} (- f'(x) - a f(x))
	- (- f'(0) - a f(0)).
	\tag1
\end{align*}

写出方程\(f''(x) + a f'(x) + f(x) = 0\)的特征方程\begin{equation*}
	f(r) = r^2 + a r + 1 = 0,
\end{equation*}
其判别式为\(\Delta = a^2 - 4\).

当\(\Delta > 0\)即\(a>2\)时,
特征方程有两个实根\(r_1,r_2\),
由韦达定理可知\(r_1 + r_2 = -a < 0\),
又因为\(f(0) = 1 > 0\),所以\(r_1,r_2\)均小于零,
于是通解\(f(x) = C_1 e^{r_1 x} + C_2 e^{r_2 x}\)满足\begin{equation*}
	\lim_{x\to+\infty} f(x)
	= \lim_{x\to+\infty} f'(x)
	= 0.
	\eqno(2)
\end{equation*}

当\(\Delta = 0\)即\(a=2\)时,
特征方程只有一个实根\(r=-1\),
于是通解\(f(x) = (C_1 + C_2 x) e^{-x}\)也满足(2)式.

当\(\Delta < 0\)即\(0<a<2\)时,
特征方程有一对共轭复根\(r_1 = -\frac{a}2 - b \iu,
r_2 = -\frac{a}2 + b \iu\),
于是通解\(f(x) = e^{-\frac{a}2 x} (C_1 \cos b x + C_2 \sin b x)\)也满足(2)式.

综上所述,由(1)式得\begin{equation*}
	\int_0^{+\infty} f(x) \dd{x}
	= f'(0) + a f(0)
	= n + a m.
\end{equation*}
\end{solution}
\end{example}

上面讨论二阶常系数齐次线性微分方程所用的方法以及方程的通解的形式,
可推广到\(n\)阶常系数齐次线性微分方程上去,对此我们不再详细讨论,只简单地叙述于下:

\(n\)阶常系数齐次线性微分方程的一般形式是
\begin{equation}\label{equation:微分方程概论.n阶常系数齐次线性微分方程的一般形式}
	y^{(n)} + p_1 y^{(n-1)} + p_2 y^{(n-2)} + \dotsb + p_{n-1} y' + p_n y = 0,
\end{equation}
其中\(p_1,p_2,\dotsc,p_{n-1},p_n\)都是常数.

有时我们用记号\(D\)(叫做\DefineConcept{微分算子})表示对\(x\)求导的运算\(\dv{x}\),
把\(\dv{y}{x}\)记作\(Dy\),把\(\dv[n]{y}{x}\)记作\(D^n y\),并把上述方程记作
\begin{equation}\label{equation:微分方程概论.n阶常系数齐次线性微分方程的一般形式.用微分算子改写}
	(D^n + p_1 D^{n-1} + p_2 D^{n-2} + \dotsb + p_{n-1} D + p_n) y = 0.
\end{equation}

再记\begin{equation*}
	L(D) \defeq D^n + p_1 D^{n-1} + p_2 D^{n-2} + \dotsb + p_{n-1} D + p_n,
\end{equation*}
称\(L(D)\)为“微分算子\(D\)的\(n\)次多项式”.
于是方程 \labelcref{equation:微分方程概论.n阶常系数齐次线性微分方程的一般形式.用微分算子改写} 又可记作\begin{equation*}
	L(D) y = 0.
\end{equation*}

如同讨论二阶常系数齐次线性微分方程那样,令\(y = e^{rx}\),
由于\begin{equation*}
	D e^{rx} = r e^{rx},\dotsc,D^n e^{rx} = r^n e^{rx},
\end{equation*}
故\(L(D) e^{rx} = L(r) e^{rx}\).
因此把\(y = e^{rx}\)%
代入方程 \labelcref{equation:微分方程概论.n阶常系数齐次线性微分方程的一般形式.用微分算子改写},
得\begin{equation*}
	L(r) e^{rx} = 0.
\end{equation*}
由此可见,如果选取\(r\)是\(n\)次代数方程
\begin{equation}\label{equation:微分方程概论.n阶常系数齐次线性微分方程的一般形式.特征方程}
	L(r) = 0
	\quad\text{即}\quad
	r^n + p_1 r^{n-1} + p_2 r^{n-2} + \dotsb + p_{n-1} r + p_n = 0
\end{equation}的根,
那么作出的函数\(y = e^{rx}\)就是微分方程的一个解.

方程 \labelcref{equation:微分方程概论.n阶常系数齐次线性微分方程的一般形式.特征方程} 叫做
方程 \labelcref{equation:微分方程概论.n阶常系数齐次线性微分方程的一般形式.用微分算子改写} 的特征方程.

如\cref{table:n阶常系数齐次线性微分方程.通解} 所示,
根据特征方程的根,可以写出其对应的微分方程的解.

\begin{table}[htb]
	\centering
	\begin{tblr}{p{3cm}|p{9cm}}
		\hline
		特征方程的根
			& 微分方程通解中的对应项 \\ \hline
		单实根\(r\)
			& 给出一项:\(C e^{rx}\) \\ \hline
		一对单复根\newline\(r_{1,2}=\alpha\pm\iu\beta\)
			& 给出两项:\(e^{\alpha x} (C_1 \cos\beta x + C_2 \sin\beta x)\) \\ \hline
		\(k\)重实根\(r\)
			& 给出\(k\)项:\(e^{rx} (C_1 + C_2 x + \dotsb + C_k x^{k-1})\) \\ \hline
		一对\(k\)重复根\newline\(r_{1,2}=\alpha\pm\iu\beta\)
			& 给出\(2k\)项:
			\(e^{\alpha x} [
				(C_1+C_2 x+\dotsb+C_k x^{k-1}) \cos\beta x
				+ (D_1+D_2 x+\dotsb+D_k x^{k-1})\sin\beta x
			]\)
		\\ \hline
	\end{tblr}
	\caption{}
	\label{table:n阶常系数齐次线性微分方程.通解}
\end{table}

根据代数基本定理,\(n\)次代数方程有\(n\)个根.
而特征方程的每一个根都对应着通解中的一项,且每项各含一个任意常数.
这样就得到\(n\)阶常系数齐次线性微分方程的通解\begin{equation*}
	y = C_1 y_1 + C_2 y_2 + \dotsb + C_n y_n.
\end{equation*}

\section{二阶常系数非齐次线性微分方程}\label{section:微分方程.常系数非齐次线性微分方程}
本节着重讨论二阶常系数非齐次线性微分方程的解法,并对\(n\)阶方程的解法作必要的说明.

二阶常系数非齐次线性微分方程的一般形式是
\begin{equation}\label{equation:微分方程.二阶常系数非齐次线性微分方程的一般形式}
y'' + p y' + q y = f(x),
\end{equation}
其中\(p,q\)是常数.

由\cref{theorem:微分方程.二阶非齐次线性微分方程的通解结构} 可知,求二阶常系数非齐次线性微分方程的通解,
归结为求对应的齐次方程 \labelcref{equation:微分方程.二阶常系数齐次线性微分方程的一般形式} 的通解
和非齐次方程 \labelcref{equation:微分方程.二阶常系数非齐次线性微分方程的一般形式} 本身的一个特解.
由于\hyperref[section:微分方程.常系数齐次线性微分方程]{二阶常系数齐次线性微分方程的通解的求法}已经得到解决,
所以这里只需讨论求二阶常系数非齐次线性微分方程的一个特解\(\hat{y}\)的方法.

本节只介绍当非齐次方程 \labelcref{equation:微分方程.二阶常系数非齐次线性微分方程的一般形式} 中的
\(f(x)\)取两种常见形式时求\(\hat{y}\)的方法.
这种方法的特点是不用积分就可求出\(\hat{y}\)来,它叫做“待定系数法”.
\(f(x)\)的两种形式是\begin{enumerate}
	\item \(f(x) = e^{\lambda x} P_m(x)\),
	其中\(\lambda\)是常数,
	\(P_m(x)\)是一个\(m\)次多项式:\begin{equation*}
		P_m(x) = a_0 x^m + a_1 x^{m-1} + \dotsb + a_{m-1} x + a_m;
	\end{equation*}

	\item \(f(x) = e^{\lambda x} [ \phi_m(x) \cos\omega x + \psi_n(x) \sin \omega x ]\),
	其中\(\lambda,\omega\)是常数,
	\(\phi_m(x),\psi_n(x)\)分别是\(x\)的\(m\)次、\(n\)次多项式,
	且至多有一个为零.
\end{enumerate}

\subsection{\texorpdfstring{\(f(x) = e^{\lambda x} P_m(x)\)型}{第一类:指数函数与多项式函数的乘积}}
我们知道,非齐次方程 \labelcref{equation:微分方程.二阶常系数非齐次线性微分方程的一般形式} 的特解\(\hat{y}\)是
使 \labelcref{equation:微分方程.二阶常系数非齐次线性微分方程的一般形式} 成为恒等式的函数.
怎样的函数能使 \labelcref{equation:微分方程.二阶常系数非齐次线性微分方程的一般形式} 称为恒等式呢?
因为 \labelcref{equation:微分方程.二阶常系数非齐次线性微分方程的一般形式} 右端\(f(x)\)
是多项式\(P_m(x)\)与指数函数\(e^{\lambda x}\)的乘积,
而多项式与指数函数乘积的导数仍然是多项式与指数函数的乘积,
因此,我们推测\(\hat{y} = e^{\lambda x} Q(x)\)(其中\(Q(x)\)是某个多项式)
可能是非齐次方程 \labelcref{equation:微分方程.二阶常系数非齐次线性微分方程的一般形式} 的特解.
把\(\hat{y}, \hat{y}', \hat{y}''\)代入
非齐次方程 \labelcref{equation:微分方程.二阶常系数非齐次线性微分方程的一般形式},
然后考虑能否选取适当的多项式\(Q(x)\),使得\(\hat{y} = e^{\lambda x} Q(x)\)满足
非齐次方程 \labelcref{equation:微分方程.二阶常系数非齐次线性微分方程的一般形式}.
为此,将\begin{align*}
	\hat{y} &= e^{\lambda x} Q(x), \\
	\hat{y}' &= e^{\lambda x} [ \lambda Q(x) + Q'(x) ], \\
	\hat{y}'' &= e^{\lambda x} [ \lambda^2 Q(x) + 2\lambda Q'(x) + Q''(x) ]
\end{align*}
代入非齐次方程 \labelcref{equation:微分方程.二阶常系数非齐次线性微分方程的一般形式}
并消去\(e^{\lambda x}\),
得\begin{equation*}
	[ \lambda^2 Q(x) + 2\lambda Q'(x) + Q''(x) ]
	+ p [ \lambda Q(x) + Q'(x) ]
	+ q Q(x)
	= P_m(x),
\end{equation*}
整理得\begin{equation}\label{equation:微分方程.常系数非齐次线性微分方程.中间步骤1}
	Q''(x) + (2\lambda+p) Q'(x) + (\lambda^2+p\lambda+q) Q(x) = P_m(x).
\end{equation}
\begin{enumerate}
	\item 如果\(\lambda\)不是
	齐次方程 \labelcref{equation:微分方程.二阶常系数齐次线性微分方程的一般形式} 的
	特征方程\(r^2+pr+q=0\)的根,
	那么\(\lambda^2+p\lambda+q\neq0\);
	由于\(P_m(x)\)是一个\(m\)次多项式,
	要使 \labelcref{equation:微分方程.常系数非齐次线性微分方程.中间步骤1} 的两端恒等,
	那么可令\(Q(x)\)为另一个\(m\)次多项式\begin{equation}
		Q_m(x) = b_0 x^m + b_1 x^{m-1} + b_2 x^{m-2}
			+ \dotsb + b_{m-2} x^2 + b_{m-1} x + b_m.
	\end{equation}
	于是\begin{gather*}
		Q_m(x) = (x^m,x^{m-1},x^{m-2},\dotsc,x^2,x,1)
		\begin{bmatrix}
			b_0 \\ b_1 \\ b_2 \\ \vdots \\ b_{m-2} \\ b_{m-1} \\ b_m
		\end{bmatrix}, \\
		Q_m'(x) = (x^m,x^{m-1},x^{m-2},\dotsc,x^2,x,1)
		\begin{bmatrix}
			0 \\ m b_0 \\ (m-1) b_1 \\ \vdots \\
			3 b_{m-3} \\ 2 b_{m-2} \\ b_{m-1}
		\end{bmatrix}, \\
		Q_m''(x) = (x^m,x^{m-1},x^{m-2},\dotsc,x^2,x,1)
		\begin{bmatrix}
			0 \\ 0 \\ m(m-1) b_0 \\ \vdots \\
			4\cdot3 b_{m-4} \\ 3\cdot2 b_{m-3} \\ 2\cdot1 b_{m-2}
		\end{bmatrix}.
	\end{gather*}
	把\(Q_m,Q_m',Q_m''\)
	代入方程 \labelcref{equation:微分方程.常系数非齐次线性微分方程.中间步骤1},
	比较等式两端\(x\)同次幂的系数,
	就得到以\(\lambda,p,q,\AutoTuple{a}[0]{m}\)为已知数、
	\(\AutoTuple{b}[0]{m}\)为未知数的\(m+1\)个方程的联立方程组\begin{equation*}
		\begin{bmatrix}
			0 \\ 0 \\ m(m-1) b_0 \\ \vdots \\
			4\cdot3 b_{m-4} \\ 3\cdot2 b_{m-3} \\ 2\cdot1 b_{m-2}
		\end{bmatrix}
		+ (2\lambda+p)
		\begin{bmatrix}
			0 \\ m b_0 \\ (m-1) b_1 \\ \vdots \\
			3 b_{m-3} \\ 2 b_{m-2} \\ b_{m-1}
		\end{bmatrix}
		+ (\lambda^2+p\lambda+q)
		\begin{bmatrix}
			b_0 \\ b_1 \\ b_2 \\ \vdots \\
			b_{m-2} \\ b_{m-1} \\ b_m
		\end{bmatrix}
		= \begin{bmatrix}
			a_0 \\ a_1 \\ a_2 \\ \vdots \\
			a_{m-2} \\ a_{m-1} \\ a_m
		\end{bmatrix}.
	\end{equation*}
	只要定出这些\(b_i\ (i=0,1,\dotsc,m)\),
	就能得到所求的特解\begin{equation}
		\hat{y} = e^{\lambda x} Q_m(x).
	\end{equation}

	\item 如果\(\lambda\)是特征方程\(r^2+pr+q=0\)的单根,
	即\(\lambda^2+p\lambda+q=0\),
	但\(2\lambda+p\neq0\),
	要使 \labelcref{equation:微分方程.常系数非齐次线性微分方程.中间步骤1} 的两端恒等,
	那么\(Q'(x)\)必须是\(m\)次多项式.
	此时可令\begin{equation*}
		Q(x) = x Q_m(x),
	\end{equation*}
	于是\begin{gather*}
		Q'(x) = Q_m(x) + x Q_m'(x), \\
		Q''(x) = 2 Q_m'(x) + x Q_m''(x),
	\end{gather*}
	代入方程 \labelcref{equation:微分方程.常系数非齐次线性微分方程.中间步骤1},得\begin{equation*}
		(2 Q_m'(x) + x Q_m''(x))
		+ (2\lambda+p) (Q_m(x) + x Q_m'(x))
		= P_m(x),
	\end{equation*}
	即\begin{equation*}
		x Q_m''(x) + [(2\lambda+p)x+2] Q_m'(x)
		+ (2\lambda+p) Q_m(x)
		= P_m(x),
	\end{equation*}
	比较等式两边\(x\)同次幂的系数,就得到方程组\begin{equation*}
		\begin{bmatrix}
			0 \\ m(m-1) b_0 \\ (m-1)(m-2) b_1 \\ \vdots \\
			3\cdot2 b_{m-3} \\ 2\cdot1 b_{m-2} \\ 0
		\end{bmatrix}
		+ (2\lambda+p)
		\begin{bmatrix}
			m b_0 \\ (m-1) b_1 \\ (m-2) b_2 \\ \vdots \\
			2 b_{m-2} \\ b_{m-1} \\ 0
		\end{bmatrix}
		+ 2 \begin{bmatrix}
			0 \\ m b_0 \\ (m-1) b_1 \\ \vdots \\
			3 b_{m-3} \\ 2 b_{m-2} \\ b_{m-1}
		\end{bmatrix}
		= \begin{bmatrix}
			a_0 \\ a_1 \\ a_2 \\ \vdots \\
			a_{m-2} \\ a_{m-1} \\ a_m
		\end{bmatrix}.
	\end{equation*}
	用同样的方法来确定\(b_i\ (i=0,1,\dotsc,m)\).
	于是所求的特解为\begin{equation}
		\hat{y} = x e^{\lambda x} Q_m(x).
	\end{equation}

	\item 如果\(\lambda\)是特征方程\(r^2+pr+q=0\)的重根,
	即\(\lambda^2+p\lambda+q=2\lambda+p=0\),
	要使 \labelcref{equation:微分方程.常系数非齐次线性微分方程.中间步骤1} 的两端恒等,
	那么\(Q''(x)\)必须是\(m\)次多项式.
	此时可令\begin{equation*}
		Q(x) = x^2 Q_m(x),
	\end{equation*}
	并用同样的方法来确定\(Q_m(x)\)中的系数.
	于是所求的特解为\begin{equation}
		\hat{y} = x^2 e^{\lambda x} Q_m(x).
	\end{equation}
\end{enumerate}

综上所述,我们有如下结论:
\begin{proposition}
如果非齐次项\(f(x)\)可以表达为\(P_m(x) e^{\lambda x}\)的形式,
则二阶常系数非齐次线性微分方程 \labelcref{equation:微分方程.二阶常系数非齐次线性微分方程的一般形式}
具有形如\begin{equation}\label{equation:微分方程.二阶常系数齐次线性微分方程.特解1}
	\hat{y} = e^{\lambda x} Q(x)
	= e^{\lambda x} \cdot x^k Q_m(x)
\end{equation}的特解,
其中\(Q_m(x)\)是与\(P_m(x)\)同次(\(m\)次)的多项式,\begin{equation*}
	k = \left\{ \begin{array}{ll}
		0, & \text{\(\lambda\)不是特征方程的根}, \\
		1, & \text{\(\lambda\)是特征方程的单根}, \\
		2, & \text{\(\lambda\)是特征方程的重根}. \\
	\end{array} \right.
\end{equation*}
\end{proposition}

上述结论可推广到\(n\)阶常系数非齐次线性微分方程,
但要注意 \labelcref{equation:微分方程.二阶常系数齐次线性微分方程.特解1} 中的
\(k\)是特征方程含根\(\lambda\)的\emph{重复次数}
(即若\(\lambda\)不是特征方程的根,则\(k=0\);
若\(\lambda\)是特征方程的单根,则\(k=1\);
若\(\lambda\)是特征方程的\(s\)重根,则\(k=s\)).

\begin{example}
%@see: 《高等数学(第六版 上册)》 P342 例1
求微分方程\(y''-2y'-3y=3x+1\)的一个特解.
\begin{solution}
这是二阶常系数非齐次线性微分方程,
且函数\(f\)是\(P_m(x) e^{\lambda x}\)型
(其中\(P_m(x) = 3x+1, \lambda=0\)).

与所给方程对应的齐次方程为\begin{equation*}
	y''-2y'-3y=0,
\end{equation*}
它的特征方程为\begin{equation*}
	r^2-2r-3=0.
\end{equation*}

由于这里\(\lambda=0\)不是特征方程的根,
所以应设特解为\begin{equation*}
	\hat{y} = b_0 x + b_1.
\end{equation*}
将其代入所给方程,得\begin{equation*}
	-3 b_0 x - 2 b_0 - 3 b_1 = 3x+1,
\end{equation*}
比较两端\(x\)同次幂的系数,得\begin{equation*}
	-3 b_0 = 3, \qquad -2 b_0 - 3 b_1 = 1,
\end{equation*}
即\(b_0=-1, b_1=1/3\).
于是求得一个特解为\begin{equation*}
	\hat{y} = -x + \frac{1}{3}.
\end{equation*}
\end{solution}
\end{example}

\begin{example}
求微分方程\(y'' - \lambda^2 y = e^{\lambda x} + e^{-\lambda x}\ (\lambda>0)\)的特解.
\begin{solution}
解齐次方程\(y'' - \lambda^2 y = 0\)的特征方程\(r^2 - \lambda^2 = 0\)
得\(r = \pm\lambda\).

因为\(\lambda\)是特征方程的单根,
那么可以设非齐次方程\(y'' - \lambda^2 y = e^{\lambda x}\)的特解为\begin{equation*}
	\hat{y}_1 = x e^{\lambda x} a,
\end{equation*}
其中\(a\)为常数,且满足\(2\lambda a = 1\)即\(a = \frac{1}{2\lambda}\).

同理可设非齐次方程\(y'' - \lambda^2 y = e^{-\lambda x}\)的特解为\begin{equation*}
	\hat{y}_2 = x e^{-\lambda x} b,
\end{equation*}
其中\(b\)为常数,且满足\(-2\lambda b = 1\)即\(b = -\frac{1}{2\lambda}\).

综上,由\hyperref[theorem:微分方程.二阶非齐次线性微分方程的解的叠加原理]{叠加原理},
微分方程\(y'' - \lambda^2 y = e^{\lambda x} + e^{-\lambda x}\ (\lambda>0)\)的特解为\begin{equation*}
	\hat{y}_1 + \hat{y}_2 = x e^{\lambda x} a + x e^{-\lambda x} b
	= x(a e^{\lambda x} + b e^{-\lambda x}).
\end{equation*}
\end{solution}
\end{example}

\subsection{\texorpdfstring{\(f(x) = e^{\lambda x} [ \phi_m(x) \cos\omega x + \psi_n(x) \sin \omega x ]\)型}{第二类:指数函数与广义三角函数的乘积}}
应用欧拉公式\begin{equation*}
	\cos\theta = \frac{1}{2}(e^{\iu\theta}+e^{-\iu\theta}),
	\qquad
	\sin\theta = \frac{1}{2\iu}(e^{\iu\theta}-e^{-\iu\theta}),
\end{equation*}
把\(f(x)\)表成复变指数函数的形式,有\begin{align*}
	f(x) &= e^{\lambda x} ( \phi_m \cos \omega x + \psi_n \sin \omega x ) \\
	&= e^{\lambda x} \left(
	\phi_m \frac{e^{\iu\omega x}+e^{-\iu\omega x}}{2}
	+ \psi_n \frac{e^{\iu\omega x}-e^{-\iu\omega x}}{2\iu}
	\right) \\
	&= \left(\frac{\phi_m}{2}+\frac{\psi_n}{2\iu}\right) e^{(\lambda+\iu\omega)x}
	+ \left(\frac{\phi_m}{2}-\frac{\psi_n}{2\iu}\right) e^{(\lambda-\iu\omega)x} \\
	&= P e^{(\lambda+\iu\omega)x}
	+ \overline{P} e^{(\lambda-\iu\omega)x},
\end{align*}
其中\begin{equation*}
	P = P(x) = \frac{\phi_m}{2} - \frac{\psi_n}{2}\iu,
	\qquad
	\overline{P} = \overline{P}(x) = \frac{\phi_m}{2}+\frac{\psi_n}{2}\iu
\end{equation*}是互成共轭\footnote{%
如果说两个\(n\)次多项式\(\alpha_n(x) = a_0 x^n + a_1 x^{n-1} + \dotsb + a_n\)
与\(\beta_n(x) = b_0 x^n + b_1 x^{n-1} + \dotsb + b_n\) \DefineConcept{互成共轭},
那么它们的对应项系数是共轭复数,
即\(a_i = \overline{b_i}\ (i=0,1,\dotsc,n)\).}的\(N = \max\{m,n\}\)次多项式.

应用上一目的结果,对于\(f(x)\)中的第一项\(P e^{(\lambda+\iu\omega)x}\),
可求出一个\(N\)次多项式\(Q = Q_N(x)\),
使得\(\hat{y}_1 = x^k Q e^{(\lambda+\iu\omega)x}\)为方程\begin{equation*}
	y'' + p y' + q y = P e^{(\lambda+\iu\omega)x}
\end{equation*}的特解,
其中\(k\)按“\(\lambda+\iu\omega\)不是特征方程的根”
或“\(\lambda+\iu\omega\)是特征方程的单根”依次取0或1.
由于\(f(x)\)的第二项\(\overline{P} e^{(\lambda-\iu\omega)x}\)
与第一项\(P e^{(\lambda+\iu\omega)x}\)成共轭,
所以与\(\hat{y}_1\)成共轭的函数\(\hat{y}_2 = x^k \overline{Q} e^{(\lambda-\iu\omega)x}\)
必然是方程\begin{equation*}
	y'' + p y' + q y = \overline{P}(x) e^{(\lambda-\iu\omega)x}
\end{equation*}的特解,
这里\(\overline{Q}\)表示与\(Q\)成共轭的\(N\)次多项式.

于是,根据\cref{theorem:微分方程.二阶非齐次线性微分方程的解的叠加原理},
非齐次方程 \labelcref{equation:微分方程.二阶常系数非齐次线性微分方程的一般形式} 具有形如\begin{equation*}
\hat{y} = x^k Q e^{(\lambda+\iu\omega)x} + x^k \overline{Q} e^{(\lambda-\iu\omega)x}
\end{equation*}的特解.
因为\(Q e^{(\lambda+\iu\omega)x}\)与\(\overline{Q} e^{(\lambda-\iu\omega)x}\)共轭,
两者相加后虚部为零,
所以上式可写为实函数的形式:
\begin{equation}
\hat{y} = x^k e^{\lambda x} (\Phi \cos\omega x + \Psi \sin\omega x),
\end{equation}
其中\(\Phi,\Psi\)都是\(N\)次多项式.

综上所述,我们有如下的结论:

如果非齐次项\(f(x)\)可以表达为
\(e^{\lambda x} [\phi_m(x) \cos\omega x + \psi_n(x) \sin\omega x]\)
的形式,
则二阶常系数非齐次线性微分方程
\labelcref{equation:微分方程.二阶常系数非齐次线性微分方程的一般形式}
具有形如
\begin{equation}\label{equation:微分方程.二阶常系数齐次线性微分方程.特解2}
\hat{y} = x^k e^{\lambda x} [\Phi_N(x) \cos\omega x + \Psi_N(x) \sin\omega x]
\end{equation}
的特解,
其中\(\Phi_N(x)\)、\(\Psi_N(x)\)是\(N=\max\{m,n\}\)次多项式,
而\(k\)按“\(\lambda\pm\iu\omega\)不是特征方程的根”%
或“\(\lambda\pm\iu\omega\)是特征方程的单根”%
两种情形依次取为0或1.

上述结论可以推广到\(n\)阶常系数非齐次线性微分方程,
但要注意\cref{equation:微分方程.二阶常系数齐次线性微分方程.特解2} 中,
\(k\)是特征方程中含根\(\lambda\pm\iu\omega\)的重复次数.

\begin{example}
%@see: 《高等数学(第六版 上册)》 P345 例3
求微分方程\(y'' + y = x \cos2x\)的一个特解.
\begin{solution}
所给方程是二阶常系数非齐次线性方程,
且非齐次项\(f(x)\)属于\(e^{\lambda x} [\phi_m(x) \cos\omega x + \psi_n(x) \sin\omega x]\)型
(其中\(\lambda=0,
\omega=2,
\phi_m(x) = x,
\psi_n(x) = 0\)).
所给方程对应的齐次方程\(y''+y=0\)的特征方程为\(r^2+1=0\),解得\(r=\pm\iu\).
由于这里\(\lambda+\iu\omega=2\iu\)不是特征方程的根,
所以应设原方程的特解为\begin{equation*}
	\hat{y} = (ax+b) \cos2x + (cx+d) \sin2x.
\end{equation*}
将其代入所给方程,得\begin{equation*}
	(-3ax-3b+4c) \cos2x - (3cx+3d+4a) \sin2x = x \cos2x.
\end{equation*}
比较两端同类项的系数,得线性方程\begin{equation*}
	\left\{ \begin{array}{l}
		-3a=1, \\
		-3b+4c=0, \\
		-3c=0, \\
		-3d-4a=0,
	\end{array} \right.
\end{equation*}
由此解得\begin{equation*}
	a = -1/3, \qquad
	b = 0, \qquad
	c = 0, \qquad
	d = 4/9.
\end{equation*}
于是求得一个特解为\begin{equation*}
	\hat{y} = -\frac{1}{3} x\cos2x + \frac{4}{9} \sin2x.
\end{equation*}
\end{solution}
\end{example}

\begin{example}
%@see: 《高等数学(第六版 上册)》 P345 例4
求微分方程\(y'' - y = e^x \cos2x\)的一个特解.
\begin{solution}
所给方程是二阶常系数非齐次线性方程,
且非齐次项\(f(x)\)属于\(e^{\lambda x} [\phi_m(x) \cos\omega x + \psi_n(x) \sin\omega x]\)型
(其中\(\lambda=1,
\omega=2,
\phi_m(x) = 1,
\psi_n(x) = 0\)).
特征方程为\(r^2-1=0\),解得\(r=\pm1\).
由于\(\lambda+\iu\omega=1+2\iu\)不是特征方程的根,故应设特解为\begin{equation*}
	\hat{y} = e^x (a \cos2x + b \sin2x).
\end{equation*}
求导得\begin{equation*}
	\hat{y}' = e^x [(a+2b) \cos2x + (-2a+b) \sin2x],
\end{equation*}\begin{equation*}
	\hat{y}'' = e^x [(-3a+4b) \cos2x + (-4a-3b) \sin2x].
\end{equation*}
代入所给方程,得\begin{equation*}
	e^x [(-4a+4b) \cos2x + (-4a-4b) \sin2x]
	= e^x \cos2x,
\end{equation*}
比较两端同类项系数,有\begin{equation*}
	\left\{ \begin{array}{l}
		-4a+4b=1, \\
		-4a-4b=0
	\end{array} \right.
	\quad\text{或}\quad
	\left\{ \begin{array}{l}
		-a+b=1/4, \\
		a+b=0,
	\end{array} \right.
\end{equation*}
得\(a = -1/8,
b = 1/8\).
因此,所给方程的一个特解为\begin{equation*}
	\hat{y} = \frac{1}{8} e^x (\sin2x - \cos2x).
\end{equation*}
\end{solution}
\end{example}

\begin{example}
设\(f(x) = \sin x - \int_0^x (x-t) f(t) \dd{t}\),
其中\(f\)是连续函数,求\(f\).
\begin{solution}
对\(f\)求导,得\begin{align*}
	f'(x)
	&= \cos x - \dv{x}\left[ x \int_0^x f(t) \dd{t} \right]
		+ \dv{x}\left[\int_0^x t f(t) \dd{t}\right] \\
	&= \cos x - \int_0^x f(t) \dd{t} - x f(x) + x f(x) \\
	&= \cos x - \int_0^x f(t) \dd{t}.
\end{align*}
令\(y = \int_0^x f(t) \dd{t}\),
得\(y' = f(x)\),
\(y'' = f'(x)\).
于是有微分方程\begin{equation*}
	y'' + y = \cos x.
\end{equation*}
解特征方程\begin{equation*}
	r^2 + 1 = 0
\end{equation*}
得\(r = \pm\iu\).
设原方程的特解为\begin{equation*}
	\hat{y} = x(a \cos x + b \sin x),
\end{equation*}
得\begin{gather*}
	\hat{y}' = (a \cos x + b \sin x)
				+ x(b \cos x - a \sin x), \\
	\hat{y}'' = 2(b \cos x - a \sin x)
				- x(a \cos x + b \sin x).
\end{gather*}
代入原方程,得\begin{equation*}
	2(b \cos x - a \sin x)
	= \cos x,
\end{equation*}
于是\(2b = 1, -2a = 0\),即\(a=0, b=1/2\),
所以特解为\begin{equation*}
	\hat{y} = \frac{1}{2} x \sin x.
\end{equation*}
那么原方程的通解为\begin{equation*}
	y = C_1 \cos x + C_2 \sin x + \frac{1}{2} x \sin x.
\end{equation*}
那么\begin{gather*}
	f(x) = y'
	= \left( C_2 + \frac{1}{2} x \right) \cos x
	+ \left( \frac{1}{2} - C_1 \right) \sin x, \\
	\int_0^x (x-t) f(t) \dd{t}
	= -C_1 x + C_2 - \left(\frac{1}{2}x+C_2\right) \cos x
	+ \left(\frac{1}{2}+C_1\right) \sin x.
\end{gather*}
代回题设条件得\begin{equation*}
	\left( C_2 + \frac{1}{2} x \right) \cos x
	+ \left( \frac{1}{2} - C_1 \right) \sin x
	= C_1 x - C_2 + \left(\frac{1}{2} x + C_2\right) \cos x
	+ \left(\frac{1}{2}-C_1\right) \sin x.
\end{equation*}
于是\(C_1 = C_2 = 0\),因此\begin{equation*}
	f(x) = \frac{1}{2} (x \cos x + \sin x).
\end{equation*}
\end{solution}
\end{example}

\section{欧拉方程}
%@see: 《高等数学(第六版 上册)》 P348
变系数的线性微分方程一般来说都是不容易求解的.
但是有些特殊的变系数线性微分方程则可以通过变量代换化为常系数线性微分方程,因而容易求解,欧拉方程就是其中的一种.

形如\begin{equation}\label{equation:微分方程.欧拉方程的一般形式}
%@see: 《高等数学(第六版 上册)》 P348 (1)
	x^n y^{(n)} + p_1 x^{n-1} y^{(n-1)} + \dotsb + p_{n-1} x y' + p_n y = f(x)
\end{equation}的方程(其中\(p_1,p_2,\dotsc,p_n\)为常数),
叫做\DefineConcept{欧拉方程}.

作变换\begin{equation*}
	x = e^t \quad\text{或}\quad t = \ln x,
\end{equation*}
将自变量\(x\)换成\(t\)\footnote{这里仅在\(x>0\)范围内求解.
如果要在\(x<0\)内求解,则可作变换\(x=-e^t\)或\(t=\ln(-x)\),所得结果与\(x>0\)内的结果相类似.},
我们有\begin{align*}
	\dv{y}{x}
	&= \dv{y}{t} \cdot \dv{t}{x} = \frac{1}{x} \dv{y}{t}, \\
	\dv[2]{y}{x}
	&= \frac{1}{x^2} \left( \dv[2]{y}{t} - \dv{y}{t} \right), \\
	\dv[3]{y}{x}
	&= \frac{1}{x^3} \left( \dv[3]{y}{t} - 3 \dv[2]{y}{t} + 2 \dv{y}{t} \right).
\end{align*}

如果采用记号\(D\)表示对\(t\)求导的运算\(\dv{t}\),
那么上述计算结果可以写成\begin{align*}
	x y' &= Dy, \\
	x^2 y'' &= \dv[2]{y}{t} - \dv{y}{t}
		= \left(\dv[2]{t} - \dv{t}\right)y \\
		&= (D^2 - D)y = D(D-1)y, \\
	x^3 y''' &= \dv[3]{y}{t} - 3 \dv[2]{y}{t} + 2 \dv{y}{t} \\
		&= (D^3-3D^2+2D)y = D(D-1)(D-2)y.
\end{align*}
一般地,有\(x^k y^{(k)} = D(D-1)\dotsm(D-k+1)y\).

把它代入欧拉方程 \labelcref{equation:微分方程.欧拉方程的一般形式},
便得一个以\(t\)为自变量的常系数线性微分方程.
在求出这个方程的解后,把\(t\)换成\(\ln x\),即得原方程的解.

\begin{example}
%@see: 《高等数学(第六版 上册)》 P349 例
求欧拉方程\(x^3 y''' + x^2 y'' - 4 x y' = 3 x^2\)的通解.
\begin{solution}
作变换\(x = e^t\),原方程化为\begin{equation*}
	D(D-1)(D-2)y + D(D-1)y - 4 Dy = 3 e^{2t},
\end{equation*}
即\begin{equation*}
	D^3 y - 2 D^2 y - 3 Dy = 3 e^{2t}.
	\eqno(1)
\end{equation*}
方程(1)对应的齐次方程为\begin{equation*}
	\dv[3]{y}{t} - 2 \dv[2]{y}{t} - 3 \dv{y}{t} = 0,
	\eqno(2)
\end{equation*}
其特征方程为\begin{equation*}
	r^3 - 2 r^2 - 3 r = 0,
\end{equation*}
它有三个根,分别为\begin{equation*}
	r_1 = 0,
	\qquad
	r_2 = -1,
	\qquad
	r_3 = 3.
\end{equation*}
于是方程(2)的通解为\begin{equation*}
	Y = C_1 + C_2 e^{-t} + C_3 e^{3t}.
\end{equation*}
假设方程(1)的特解为\begin{equation*}
	y^* = b e^{2t}
	= b x^2,
\end{equation*}
代入原方程,求得\(b = -\frac12\),
即\begin{equation*}
	y^* = -\frac12 x^2.
\end{equation*}
于是,所给欧拉方程的通解为\begin{equation*}
	y = C_1 + C_2 e^{-t} + C_3 e^{3t} - \frac12 x^2.
\end{equation*}
\end{solution}
%@Mathematica: DSolve[x^3 y'''[x] + x^2 y''[x] - 4 x y'[x] == 3 x^2, y[x], x]
\end{example}

\begin{example}
%@see: 《2004年全国硕士研究生入学统一考试(数学一)》一填空题/第4题
求欧拉方程\(x^2 y'' + 4 x y' + 2 y = 0\ (x>0)\)的通解.
\begin{solution}
作变换\(x = e^t\),原方程化为\begin{equation*}
	D(D-1)y + 4 Dy + 2y = 0,
\end{equation*}
即\(D^2y + 3 Dy + 2y = 0\),
其特征方程为\begin{equation*}
	r^2 + 3r + 2 = 0,
\end{equation*}
它有两个根,分别是\(r_1 = -2,r_2 = -1\).
于是变换后的微分方程的通解为\(y(t) = C_1 e^{-2t} + C_2 e^{-t}\).
把\(t\)换回\(\ln x\),得\begin{equation*}
	y(x) = \frac{C_1}{x^2} + \frac{C_2}{x}.
\end{equation*}
\end{solution}
%@Mathematica: DSolve[x^2 y''[x] + 4 x y'[x] + 2 y[x] == 0, y[x], x]
\end{example}
\begin{example}
%@see: 《2021年全国硕士研究生入学统一考试(数学一)》二填空题/第13题
求欧拉方程\(x^2 y'' + x y' - 4 y = 0\)
满足条件\(y(1) = 1,y'(1) = 2\)的解.
\begin{solution}
作变换\(x = e^t\),原方程化为\begin{equation*}
	D(D-1)y + Dy - 4y = 0,
\end{equation*}
即\begin{equation*}
	D^2y - 4y = 0,
\end{equation*}
其特征方程为\begin{equation*}
	r^2 - 4 = 0,
\end{equation*}
它有两个根,分别为\(r_1 = -2,r_2 = 2\).
于是变换后的微分方程的通解为\(y(t) = C_1 e^{-2t} + C_2 e^{2t}\).
把\(t\)换回\(\ln x\),得\begin{equation*}
	y(x) = C_1 x^{-2} + C_2 x^2.
\end{equation*}
代入条件\(y(1) = 1,y'(1) = 2\)得\begin{equation*}
	C_1 + C_2 = 1,
	\qquad
	-2 C_1 + 2 C_2 = 2,
\end{equation*}
解得\(C_1 = 0,C_2 = 1\),
故所求欧拉方程的解为\(y(x) = x^2\).
\end{solution}
%@Mathematica: DSolve[x^2 y''[x] + x y'[x] - 4 y[x] == 0 && y[1] == 1 && y'[1] == 2, y[x], x]
\end{example}

% \input{微积分/微分方程/常系数线性微分方程组的解法}
% \begin{landscape}
\section{本章总结}

在解微分方程时,我们可以按以下步骤解题:
首先要找出最高阶导数,
确定它是一阶方程、二阶方程,还是三阶以上方程.
然后利用代数方法将最高阶导数的系数化为\(1\).
接下来分情况研究方程,
当它是一阶方程时,
观察它是不是可分离变量,是不是齐次方程,是不是可以化为齐次方程,
是不是线性方程,是不是伯努利方程;
当它是二阶方程时,
观察它是不是\(y'' = f(x,y')\)(缺\(y\))
或\(y'' = f(y,y')\)(缺\(x\))这两类方程,
是不是常系数线性方程;
当它是三阶以上方程时,
观察它是不是欧拉方程.

\begin{table}[!htp]
	\centering
	\begin{tblr}{l|l|l}
		\hline%
		大类 & 小类 & 解法(通解) \\ \hline%
		\begin{tblr}{l}
			一阶微分方程 \\
			\(y'=f(x,y)\) \\
		\end{tblr}
			& \begin{tblr}{l}
				可分离变量的微分方程\\
				\(g(y) \dd{y} = f(x) \dd{x}\) \\
			\end{tblr}
			& \begin{tblr}{l}
				\(\int g(y) \dd{y} = \int f(x) \dd{x}\) \\
			\end{tblr} \\ \cline{2-3}%
			& \begin{tblr}{l}
				齐次方程 \\
				\(\dv{y}{x} = \phi\left(\frac{y}{x}\right)\) \\
			\end{tblr}
			& \begin{tblr}{l}
				计算\(\int \frac{\dd{u}}{\phi(u) - u} = \int \frac{\dd{x}}{x}\), \\
				再换元\(u=\frac{y}{x}\) \\
			\end{tblr} \\ \cline{2-3}%
			& \begin{tblr}{l}
				可化齐次方程 \\
				\(\dv{y}{x} = \frac{ax+by+c}{a_1x+b_1y+c_1}\) \\
			\end{tblr}
			& \begin{tblr}{l}
				化成齐次方程 \\
				%\cref{equation:微分方程.可化为齐次的方程.换元得到的齐次方程1}
				\(\dv{Y}{X} = \frac{aX+bY}{a_1 X+b_1 Y}\); \\
				%\cref{equation:微分方程.可化为齐次的方程.换元得到的齐次方程2}
				或化为\(\dv{y}{x} = \frac{(ax+by)+c}{\lambda(ax+by)+c_1}\) \\
				(再用\(v = a x + b y\)换元) \\
			\end{tblr} \\ \cline{2-3}%
			& \begin{tblr}{l}
				一阶线性微分方程 \\
				\(\dv{y}{x} + P(x) y = Q(x)\) \\ %\cref{equation:微分方程.一阶线性非齐次微分方程}
			\end{tblr}
			& \begin{tblr}{l}
				利用常数变易法 \\
				%\cref{equation:微分方程.一阶线性非齐次微分方程的通解}
				\(y = e^{ -\int P(x) \dd{x} } \left( \int Q(x) e^{ \int P(x) \dd{x} } \dd{x} + C \right)\) \\
			\end{tblr} \\ \cline{2-3}%
			& \begin{tblr}{l}
				伯努利方程 \\
				\(\dv{y}{x} + P(x) y = Q(x) y^n\) \\
			\end{tblr}
			& \begin{tblr}{l}
				用\(z = y^{1-n}\)换元, \\
				解方程\(\dv{z}{x} + (1-n) P(x) z = (1-n) Q(x)\) \\
			\end{tblr}
		\\ \hline
		\SetCell[r=2]{c}
		\begin{tblr}{l}
			二阶微分方程 \\
			\(y'' = f(x,y,y')\) \\ %\cref{equation:微分方程.二阶微分方程.1111型}
		\end{tblr}
			& \begin{tblr}{l}
				\(y'' = f(x,y')\) \\ %\cref{equation:微分方程.二阶微分方程.1011型}
			\end{tblr}
			& \begin{tblr}{l}
				\(y = \int \phi(x, C_1) \dd{x} + C_2\) \\
			\end{tblr} \\ \cline{2-3} %\cref{equation:微分方程.二阶微分方程.1011型.通解}
			& \begin{tblr}{l}
				\(y'' = f(y, y')\) \\ %\cref{equation:微分方程.二阶微分方程.0111型}
			\end{tblr}
			& \begin{tblr}{l}
				\(\int \frac{\dd{y}}{\phi(y,C_1)} = x + C_2\) \\ %\cref{equation:微分方程.二阶微分方程.0111型.通解}
			\end{tblr}
		\\ \hline
		\SetCell[c=2]{l}
		\begin{tblr}{l}
			欧拉方程 \\ %\cref{equation:微分方程.欧拉方程的一般形式}
			\(x^n y^{(n)} + p_1 x^{n-1} y^{(n-1)}\) \\
			\(+ \dotsb + p_{n-1} x y' + p_n y = f(x)\) \\
		\end{tblr}
			&& \begin{tblr}{l}
				依次用\(x=e^t\)和\(x=-e^t\)换元 \\
			\end{tblr}
		\\ \hline
	\end{tblr}
\end{table}
% \end{landscape}


\chapter{常数项级数}
无穷级数是高等数学的一个重要组成部分;
它是表示函数,研究函数性质,以及进行数值计算的一种工具.
本章先讨论常数项级数,介绍无穷级数的一些基本内容,然后讨论函数项级数,
着重讨论如何将函数展开成幂级数和三角级数的问题.

\section{常数项级数的概念和性质}
人们认识事物在数量方面的特性,往往有一个由近似到精确的过程.
在这种认识过程中,会遇到由有限个数量相加到无穷多个数量相加的问题.

例如计算半径为\(R\)的圆面积\(A\),具体做法如下:
如\cref{figure:无穷级数.用内接正多边形覆盖圆},
作圆的内接正六边形,算出这六边形的面积\(a_1\),它是圆面积\(A\)的一个粗糙的近似值.
为了比较准确地计算出\(A\)的值,
我们以这个正六边形的每一边为底分别作一个顶点在圆周上的等腰三角形,
算出这六个等腰三角形的面积之和\(a_2\).
那么\(a_1+a_2\)(即内接正十二边形的面积)就是\(A\)的一个较好的近似值.
同样地,在这正十二边形的每一边上分别作一个顶点在圆周上的等腰三角形,
算出这十二个等腰三角形的面积之和\(a_3\).
那么\(a_1+a_2+a_3\)(即内接正二十四边形的面积)是\(A\)的一个更好的近似值.
如此继续下去,内接\(3\cdot2^n\)边形的面积就逐步逼近圆面积:\begin{equation*}
	A \approx a_1 + a_2 + \dotsb + a_n.
\end{equation*}

\begin{figure}[h]
%@see: 《高等数学(第六版 下册)》 P248 图12-1
	\centering
	\begin{tikzpicture}
		\pgfmathsetmacro{\radius}{4}
		\def\PlotPolygon#1{
			\foreach \i in {1,...,#1}{
				\draw({\radius*cos(\i*360/#1)},{\radius*sin(\i*360/#1)})
				--({\radius*cos((\i-1)*360/#1)},{\radius*sin((\i-1)*360/#1)});
			}
		}
		\begin{scope}
			\PlotPolygon{3}
			\PlotPolygon{6}
			\PlotPolygon{12}
		\end{scope}
		\draw(0,0)circle(\radius)node[left]{\(O\)}
			--(\radius,0)node[midway,above]{\(R\)};
		\fill(0,0)circle(1pt);
	\end{tikzpicture}
	\caption{用内接正多边形覆盖圆}
	\label{figure:无穷级数.用内接正多边形覆盖圆}
\end{figure}

如果内接正多边形的边数无限增多,即\(n\)无限增大,
则和\(a_1+a_2+\dotsb+a_n\)的极限就是所要求的圆面积\(A\).
这时和式中的项数无限增多,于是出现了无穷多个数量依次相加的数学式子.

\subsection{常数项级数的概念}
\begin{definition}\label{definition:无穷级数.常数项级数的定义}
设\(\{u_n\}\)是数列.
定义:\begin{equation*}
	\sum_{n=1}^\infty u_n
	\defeq
	\lim_{n\to\infty} \sum_{k=1}^n u_k,
\end{equation*}
称之为\DefineConcept{常数项无穷级数}(infinite series with constant terms),
简称\DefineConcept{常数项级数},
或者进一步简称为\DefineConcept{级数}.
把\(u_n\)称为
“级数\(\sum_{n=1}^\infty u_n\)的\DefineConcept{一般项}”.
把\(\sum_{k=1}^n u_k\)称为
“级数\(\sum_{n=1}^\infty u_n\)的\DefineConcept{部分和}(partial sum)”.

如果级数\(\sum_{n=1}^\infty u_n\)的
部分和\(\sum_{k=1}^n u_k\)
构成的数列收敛于\(s\),
则称“级数\(\sum_{n=1}^\infty u_n\)~\DefineConcept{收敛}(converge)”,
我们把极限\(s\)叫做“级数\(\sum_{n=1}^\infty u_n\)的\DefineConcept{和}(sum)”.
反之,如果这个级数的部分和数列发散,
则称“级数\(\sum_{n=1}^\infty u_n\)~\DefineConcept{发散}(diverge)”.

当级数收敛时,
我们把收敛级数的和\(s\)与级数的部分和\(\sum_{k=1}^n u_k\)的差值\begin{equation*}
	\sum_{n=1}^\infty u_n - \sum_{k=1}^n u_k
\end{equation*}
叫做“级数\(\sum_{n=1}^\infty u_n\)的\DefineConcept{余项}”.
\end{definition}

\begin{example}\label{example:无穷级数.等比级数的收敛性}
%@see: 《数学分析(第二版 下册)》(陈纪修) P2 例9.1.1
级数\(\sum_{n=0}^\infty q^n\)
叫做\DefineConcept{等比级数}或\DefineConcept{几何级数}(geometric series).
%@see: https://mathworld.wolfram.com/GeometricSeries.html
我们把常数\(q\)叫做“级数\(\sum_{n=0}^\infty q^n\)的\DefineConcept{公比}”.
试讨论上述等比级数的收敛性.
\begin{solution}
根据\(q\)的取值范围,分情况讨论.
\begin{enumerate}
	\item 当\(q = 1\)时,则部分和为\begin{equation*}
		\sum_{i=0}^n q^n
		= n,
	\end{equation*}
	那么\(\lim_{n\to\infty} \sum_{i=0}^n q^n
	= \lim_{n\to\infty} n
	= \infty\),
	即级数发散.

	\item 当\(q \neq 1\),
	根据\cref{theorem:等比数列.前n项和} 有\begin{equation*}
		\sum_{i=0}^n q^n
		= \frac{1-q^n}{1-q}
		= \frac{1}{1-q} - \frac{q^n}{1-q}.
	\end{equation*}
	\begin{enumerate}
		\item 当\(\abs{q} < 1\)时,
		由于\(\lim_{n\to\infty} q^n=0\),
		从而\(\lim_{n\to\infty} \sum_{i=0}^n q^n
		=\frac{a}{1-q}\),
		因此级数收敛,
		其和为\(\frac{a}{1-q}\).

		\item 当\(\abs{q} > 1\)时,
		由于\(\lim_{n\to\infty} q^n=\infty\),
		从而\(\lim_{n\to\infty} \sum_{i=0}^n q^n
		=\infty\),
		级数发散.

		\item 当\(q = -1\)时,部分和为\begin{equation*}
			\sum_{i=0}^n q^n
			= \begin{cases}[cl]
				1, & \text{\(n\)是偶数}, \\
				0, & \text{\(n\)是奇数}.
			\end{cases}
		\end{equation*}
		根据\cref{theorem:子列极限.数列收敛的充分必要条件},
		\(\sum_{i=0}^n q^n\)的极限不存在,级数发散.
	\end{enumerate}
\end{enumerate}

综上所述,{\color{red}当\(\abs{q} < 1\)时,几何级数收敛;
当\(\abs{q} \geq 1\)时,几何级数发散.}
\end{solution}
\end{example}

\begin{example}\label{example:无穷级数.等差级数的收敛性}
试证:\DefineConcept{等差级数}(arithmetic series)\begin{equation*}
	1+2+3+\dotsb+n+\dotsb
\end{equation*}是发散的.
\begin{proof}
级数的部分和为\begin{equation*}
	s_n = 1+2+3+\dotsb+n = \frac{n(n+1)}{2}.
\end{equation*}
显然,\(\lim_{n\to\infty} s_n=\infty\),级数是发散的.
\end{proof}
%@see: https://mathworld.wolfram.com/ArithmeticSeries.html
\end{example}
从上例也可看出:
\begin{proposition}
除非初项\(a_0\)与公差\(d\)都等于\(0\),
否则等差级数\(\sum_{n=0}^\infty(a_0+nd)\)总是发散的.
\end{proposition}

\begin{example}
判定无穷级数\begin{equation*}
	\frac{1}{1\cdot2}+\frac{1}{2\cdot3}+\dotsb+\frac{1}{n(n+1)}+\dotsb
\end{equation*}的收敛性.
\begin{solution}
记\begin{equation*}
	u_n = \frac{1}{n(n+1)} = \frac{1}{n}-\frac{1}{n+1},
\end{equation*}
因此\begin{align*}
	s_n &= \frac{1}{1\cdot2}+\frac{1}{2\cdot3}+\dotsb+\frac{1}{n(n+1)} \\
	&= \left(1-\frac{1}{2}\right)+\left(\frac{1}{2}-\frac{1}{3}\right)
	+\dotsb+\left(\frac{1}{n}-\frac{1}{n+1}\right) \\
	&= 1-\frac{1}{n+1}.
\end{align*}
从而\begin{equation*}
	\lim_{n\to\infty} s_n
	= \lim_{n\to\infty} \left(1-\frac{1}{n+1}\right)
	= 1,
\end{equation*}
即该级数收敛,它的和为\(1\).
\end{solution}
\end{example}

\begin{example}\label{example:无穷级数.与调和数列有关的级数1}
%@see: https://www.bilibili.com/video/BV1c2yBYwEvt/
证明级数\(\sum_{n=1}^\infty \frac1{(n+1)(n+2)} \left( 1+\frac12+\dotsb+\frac1n \right)\)收敛并求和.
\begin{solution}
记\(a_n = 1+\frac12+\dotsb+\frac1n\),
那么\(a_n - a_{n-1} = \frac1n\ (n=2,3,\dotsc)\),
而级数的部分和为\begin{align*}
	S_n &= \sum_{k=1}^n \frac{a_k}{(k+1)(k+2)}
	% 裂项
	= \sum_{k=1}^n \left( \frac{a_k}{k+1} - \frac{a_k}{k+2} \right) \\
	&= \left( \frac{a_1}{2} - \frac{a_1}{3} \right)
	+ \left( \frac{a_2}{3} - \frac{a_2}{4} \right)
	+ \left( \frac{a_3}{4} - \frac{a_3}{5} \right)
	+ \dotsb + \left( \frac{a_{n-1}}{n} - \frac{a_{n-1}}{n+1} \right)
	+ \left( \frac{a_n}{n+1} - \frac{a_n}{n+2} \right) \\
	&= \frac{a_1}{2} + \frac{a_2 - a_1}{3}
	+ \frac{a_3 - a_4}{4} + \frac{a_4 - a_3}{5}
	+ \dotsb + \frac{a_n - a_{n-1}}{n+1}
	- \frac{a_n}{n+2} \\
	&= \frac1{2\cdot1} + \frac1{3\cdot2} + \frac1{4\cdot3} + \frac1{5\cdot4}
	+ \dotsb + \frac1{(n+1)n} - \frac{a_n}{n+2} \\
	% 裂项
	&= \left( 1-\frac12 \right) + \left( \frac12 - \frac13 \right)
	+ \dotsb + \left( \frac1n - \frac1{n+1} \right) - \frac{a_n}{n+2} \\
	&= 1 - \frac1{n+1} - \frac{a_n}{n+2}.
\end{align*}
这里\(\frac1{n+1}\to0\ (n\to\infty)\),
因此,要证部分和\(S_n\)收敛,需证数列\(\frac{a_n}{n+2}\)收敛.
由\cref{example:微分中值定理.拉格朗日中值定理.重要不等式1} 可知,
对于任意正整数\(n\)有\(\frac1{n+1} < \ln(1+\frac1n) < \frac1n\),
所以\begin{align*}
	a_n &= 1+\frac12+\frac13+\dotsb+\frac1n \\
	&< 1+\ln(1+\frac11)+\ln(1+\frac12)+\dotsb+\ln(1+\frac1{n-1}) \\
	&= 1+\ln2+\ln\frac32+\dotsb+\ln\frac{n}{n-1}
	= 1+\ln n,
\end{align*}
从而有\begin{equation*}
	0 < \frac{a_n}{n+2}
	< \frac{1+\ln n}{n+2}
	\to 0
	\quad(n\to\infty),
\end{equation*}
于是\(S_n\to1\ (n\to\infty)\),
即\(\sum_{n=1}^\infty \frac1{(n+1)(n+2)} \left( 1+\frac12+\dotsb+\frac1n \right) = 1\).
\end{solution}
\end{example}

\subsection{级数收敛的必要条件}
\begin{proposition}[级数收敛的必要条件]\label{theorem:无穷级数.级数收敛的必要条件}
%@see: 《高等数学(第六版 上册)》 P253 性质5(级数收敛的必要条件)
%@see: 《数学分析(第二版 上册)》(陈纪修) P3 定理9.1.1(级数收敛的必要条件)
如果级数\(\sum_{n=1}^\infty u_n\)收敛,
则它的一般项\(u_n\)收敛于\(0\),
即\begin{equation*}
	\lim_{n\to\infty} u_n = 0.
\end{equation*}
\begin{proof}
设级数\(\sum_{n=1}^\infty u_n\)的部分和为\(s_n\),
且\(\lim_{n\to\infty} s_n = s\),
则\begin{equation*}
	\lim_{n\to\infty} u_n
	= \lim_{n\to\infty}(s_n - s_{n-1})
	= \lim_{n\to\infty} s_n - \lim_{n\to\infty} s_{n-1}
	= s - s
	= 0.
	\qedhere
\end{equation*}
\end{proof}
\end{proposition}
\begin{remark}
数列\(\{u_n\}\)收敛\(\iff\)级数\(\sum_{n=1}^\infty (u_n - u_{n+1})\)收敛.
\end{remark}

值得注意的是,级数的一般项趋于零并不是级数收敛的充分条件.
有些级数虽然一般项趋于零,但仍然是发散的.
\begin{example}\label{example:无穷级数.调和级数的敛散性}
试证:\DefineConcept{调和级数}(harmonic series)\begin{equation*}
%@see: 《高等数学(第六版 上册)》 P253 (5)
	\sum_{n=1}^\infty u_n
	= 1+\frac12+\frac13+\dotsb+\frac1n+\dotsb
\end{equation*}是发散的.
\begin{proof}
虽然调和级数的一般项\(\lim_{n\to\infty} u_n
= \lim_{n\to\infty} 1/n
= 0\),
但是它是发散的.
现在我们用反证法证明.

假设级数\(\sum_{n=1}^\infty u_n\)收敛.
设级数的部分和为\(s_n\),
且\(\lim_{n\to\infty} s_n
= s\).
显然,对级数\(\sum_{n=1}^\infty u_n\)的部分和\(s_{2n}\)
也有\(\lim_{n\to\infty} s_{2n}
= s\).
于是\begin{equation*}
	\lim_{n\to\infty} (s_{2n}-s_n)
	= \lim_{n\to\infty} s_{2n} - \lim_{n\to\infty} s_n
	= s - s
	= 0.
\end{equation*}
但另一方面\begin{equation*}
	s_{2n} - s_n
	= \frac{1}{n+1}+\frac{1}{n+2}+\dotsb+\frac{1}{2n}
	> \underbrace{\frac{1}{2n}+\frac{1}{2n}+\dotsb+\frac{1}{2n}}_{\text{$n$项}}
	= n \cdot \frac1{2n}
	= \frac12,
\end{equation*}
由\cref{theorem:极限.收敛数列的保序性2} 可知\begin{equation*}
	\lim_{n\to\infty} (s_{2n}-s_n)
	\geq \frac12
	> 0,
\end{equation*}与假设矛盾,说明级数\(\sum_{n=1}^\infty u_n\)必定发散.
\end{proof}
%@see: https://mathworld.wolfram.com/HarmonicSeries.html
%@see: https://math.stackexchange.com/questions/1160527/the-series-sum-n-1-infty-frac1n-diverges
\end{example}

\begin{corollary}\label{theorem:无穷级数.级数收敛的必要条件.推论}
%@see: 《高等数学(第六版 上册)》 P253
%@see: https://mathworld.wolfram.com/DivergenceTests.html
%@see: https://mathworld.wolfram.com/LimitTest.html
如果级数\(\sum_{n=1}^\infty u_n\)的一般项不收敛于\(0\),
即\(u_n \not\to 0\ (n\to\infty)\),
那么\(\sum_{n=1}^\infty u_n\)发散.
\begin{proof}
\cref{theorem:无穷级数.级数收敛的必要条件} 的逆否命题.
\end{proof}
\end{corollary}

\begin{example}
%@see: 《高等数学(第六版 上册)》 P253
级数\begin{equation*}
	\frac12-\frac23+\frac34-\dotsb+(-1)^{n-1}\frac{n}{n+1}+\dotsb,
\end{equation*}的一般项
\(u_n = (-1)^{n-1} \frac{n}{n+1}\)满足\begin{equation*}
	\varlimsup_{n\to\infty} u_n = 1,
	\qquad
	\varliminf_{n\to\infty} u_n = -1,
\end{equation*}
于是\(u_n \not\to 0\ (n\to\infty)\),
由\cref{theorem:无穷级数.级数收敛的必要条件.推论} 可知,该级数是发散的.
\end{example}

即便在“级数的一般项趋于零”之外,再加上条件“级数的部分和数列有界”,也依然无法保证“级数收敛”.
\begin{example}
%@see: https://www.bilibili.com/video/BV1EkiGeJET3/
举例说明:当级数\(\sum_{n=1}^\infty a_n\)的部分和数列\(\{S_n\}\)有界
且\(\lim_{n\to\infty} a_n = 0\)时,
级数\(\sum_{n=1}^\infty a_n\)不收敛.
\begin{solution}
取\begin{equation*}
	a_n = \sin\sqrt{n+1} - \sin\sqrt{n}
	\quad(n=1,2,\dotsc),
\end{equation*}
那么由\hyperref[theorem:微分中值定理.拉格朗日中值定理]{拉格朗日中值定理}可知,
存在实数\(\xi\in(\sqrt{n},\sqrt{n+1})\),
使得\begin{equation*}
	\sin\sqrt{n+1} - \sin\sqrt{n}
	= \cos\xi \cdot (\sqrt{n+1} - \sqrt{n}).
\end{equation*}
由于\(\lim_{n\to\infty} (\sqrt{n+1} - \sqrt{n})
= \lim_{n\to\infty} \frac1{\sqrt{n+1} + \sqrt{n}}
= 0\),
而\(\cos\xi\)是有界量,
所以由\cref{theorem:函数极限.无穷小.有界函数与无穷小的乘积是无穷小}
可知\(\lim_{n\to\infty} a_n = 0\).
另一方面\begin{equation*}
	S_n = a_n + a_{n-1} + \dotsb + a_2 + a_1
	= \sin\sqrt{n+1} - \sin1
	\quad(n=1,2,\dotsc).
\end{equation*}
显然\(\{S_n\}\)有界但不收敛.
\end{solution}
\end{example}

\subsection{收敛级数的基本性质}
%@see: https://mathworld.wolfram.com/ConvergentSeries.html
\begin{property}\label{theorem:无穷级数.收敛级数性质1}
%@see: 《高等数学(第六版 上册)》 P251 性质1
%@see: 《数学分析(第二版 上册)》(陈纪修) P4 定理9.1.2(线性性)
如果级数\(\sum_{n=1}^\infty u_n\)收敛于和\(s\),
则级数\(\sum_{n=1}^\infty k u_n\)收敛于和\(ks\).
\begin{proof}
设级数\(\sum_{n=1}^\infty u_n\)
与级数\(\sum_{n=1}^\infty k u_n\)的
部分和分别为\(s_n\)与\(\sigma_n\),
则\begin{equation*}
	\sigma_n
	= k u_1 + k u_2 + \dotsb + k u_n
	= k(u_1 + u_2 + \dotsb + u_n) = k s_n,
\end{equation*}\begin{equation*}
	\lim_{n\to\infty} \sigma_n
	= \lim_{n\to\infty} k s_n
	= k \lim_{n\to\infty} s_n = ks,
\end{equation*}
也就是说,级数\(\sum_{n=1}^\infty k u_n\)收敛于\(ks\).
\end{proof}
\end{property}
\begin{remark}
由关系式\(\sigma_n = k s_n\)知道,
如果\(\{s_n\}\)没有极限且\(k\neq0\),
那么\(\{\sigma_n\}\)也不可能有极限.
因此我们得到如下结论:
{\color{red}级数的每一项同乘一个非零常数后,它的敛散性不会改变.}
\end{remark}
\begin{remark}
对于任意发散级数\(\sum_{n=1}^\infty u_n\),
级数\(\sum_{n=1}^\infty 0 \cdot u_n\)的每一项都是零,
故级数\(\sum_{n=1}^\infty 0 \cdot u_n\)收敛于\(0\).
\end{remark}

\begin{example}
判断\begin{equation*}
\sin\frac{\pi}{6}+\sin\frac{2\pi}{6}+\dotsb+\sin\frac{n\pi}{6}+\dotsb
\end{equation*}的收敛性.
\begin{solution}
记\(u_n = \sin\frac{n\pi}{6},
v_n = 2\sin\frac{\pi}{12} \cdot u_n\).
因为\begin{equation*}
	v_n = \cos\frac{\pi-2n\pi}{12} - \cos\frac{\pi+2n\pi}{12},
\end{equation*}\begin{equation*}
	\sigma_n
	= v_1 + v_2 + \dotsb + v_n
	= \cos\frac{\pi}{12} - \cos\frac{(2n+1)\pi}{12},
\end{equation*}
所以\begin{equation*}
	s_n
	= \left(2\sin\frac{\pi}{12}\right)^{-1} \cdot \sigma_n
	= 1+\frac{\sqrt{3}}{2} - \frac{\sqrt{2}}{\sqrt{3}-1} \cos\frac{(2n+1)\pi}{12}.
\end{equation*}
可见\(\lim_{n\to\infty} s_n\)不存在,级数发散.
\end{solution}
\end{example}

\begin{property}\label{theorem:无穷级数.收敛级数性质2}
%@see: 《高等数学(第六版 上册)》 P251 性质2
%@see: 《数学分析(第二版 上册)》(陈纪修) P4 定理9.1.2(线性性)
如果级数\(\sum_{n=1}^\infty u_n\)、\(\sum_{n=1}^\infty v_n\)
分别收敛于\(s\)、\(\sigma\),
则级数\(\sum_{n=1}^\infty(u_n \pm v_n)\)收敛于\(s \pm \sigma\).
\begin{proof}
设级数\(\sum_{n=1}^\infty u_n\)与级数\(\sum_{n=1}^\infty v_n\)的部分和
分别为\(s_n\)与\(\sigma_n\),
则级数\(\sum_{n=1}^\infty(u_n \pm v_n)\)的部分和\begin{align*}
	\tau_n &= (u_1 \pm v_1) + (u_2 \pm v_2) + \dotsb + (u_n + v_n) \\
	&= (u_1 + u_2 + \dotsb + u_n) \pm (v_1 + v_2 + \dotsb + v_n) \\
	&= s_n \pm \sigma_n,
\end{align*}
于是\begin{equation*}
	\lim_{n\to\infty} \tau_n
	= \lim_{n\to\infty} (s_n \pm \sigma_n)
	= s + \sigma.
\end{equation*}

这就表明级数\(\sum_{n=1}^\infty(u_n \pm v_n)\)收敛于\(s \pm \sigma\).
\end{proof}
\end{property}
\begin{remark}
从\cref{theorem:无穷级数.收敛级数性质1,theorem:无穷级数.收敛级数性质2} 可以看出:
对收敛级数可以进行加法和数乘运算.
\end{remark}

\begin{example}
设级数\(\sum_{n=1}^\infty u_n\)收敛,\(\sum_{n=1}^\infty v_n\)发散.
试证:级数\(\sum_{n=1}^\infty (u_n + v_n)\)发散.
\begin{proof}
设级数\(\sum_{n=1}^\infty u_n\)与\(\sum_{n=1}^\infty v_n\)的部分和
分别为\(s_n\)和\(\sigma_n\),
又设\begin{equation*}
	\lim_{n\to\infty} s_n = s.
\end{equation*}
用反证法.
假设级数\(\sum_{n=1}^\infty (u_n + v_n)\)收敛于\(s + \sigma\),
即\begin{equation*}
	\lim_{n\to\infty} (s_n + \sigma_n) = s+\sigma,
\end{equation*}
那么根据\hyperref[theorem:极限.极限的四则运算法则]{极限的四则运算法则},
有\begin{equation*}
	\lim_{n\to\infty} \sigma_n
	= \lim_{n\to\infty} [(s_n + \sigma_n) - s_n]
	= \lim_{n\to\infty} (s_n + \sigma_n) - \lim_{n\to\infty} s_n
	= (s + \sigma) - s
	= \sigma,
\end{equation*}
也就是说级数\(\sum_{n=1}^\infty v_n\)的部分和数列\(\{\sigma_n\}\)收敛,
于是级数\(\sum_{n=1}^\infty v_n\)收敛,矛盾!
因此级数\(\sum_{n=1}^\infty (u_n + v_n)\)必定发散.
\end{proof}
\end{example}

从上例我们可以看出:
\begin{proposition}\label{theorem:无穷级数.收敛级数性质2.推论1}
一个收敛级数与一个发散级数相加(或相减)所得级数必定发散.
\end{proposition}

另一方面,任给一个发散级数\(\sum_{n=1}^\infty u_n\),
级数\(\sum_{n=1}^\infty u_n + \sum_{n=1}^\infty u_n
= \sum_{n=1}^\infty 2 u_n\)也是发散的,
而级数\(\sum_{n=1}^\infty u_n - \sum_{n=1}^\infty u_n
= 0\)却是收敛的,
于是我们还可以看出:
{\color{red}两个发散级数相加(或相减)所得级数可能收敛也可能发散.}

\begin{example}
已知级数\(\sum_{n=1}^\infty (-1)^{n-1} a_n = 2,
\sum_{n=1}^\infty a_{2n-1} = 5\),
求\(\sum_{n=1}^\infty a_n\).
\begin{solution}
由于\(\sum_{n=1}^\infty [a_n + (-1)^{n-1} a_n]
= 2 \sum_{n=1}^\infty a_{2n-1}\)收敛,
所以\(\sum_{n=1}^\infty a_n\)收敛,且有
\begin{equation*}
\sum_{n=1}^\infty a_n
= 2 \sum_{n=1}^\infty a_{2n-1}
- \sum_{n=1}^\infty (-1)^{n-1} a_n
= 10 - 2 = 8.
\end{equation*}
\end{solution}
\end{example}

\begin{property}\label{theorem:无穷级数.收敛级数性质3}
%@see: 《高等数学(第六版 上册)》 P252 性质3
在级数中去掉、加上或改变有限项,不会改变级数的收敛性.
\begin{proof}
我们只需证明“在级数的前面部分去掉或加上有限项,不会改变级数的收敛性”,
因为其他情形(即在级数中任意去掉、加上或改变有限项的情形)都可以看成在级数的前面部分先去掉有限项,
然后再加上有限项的结果.

将级数\begin{equation*}
u_1+u_2+\dotsb+u_k+u_{k+1}+\dotsb+u_{k+n}+\dotsb
\end{equation*}的前\(k\)项去掉,则得级数\begin{equation*}
u_{k+1}+u_{k+2}+\dotsb+u_{k+n}+\dotsb.
\end{equation*}于是新得到的级数的部分和为\begin{equation*}
\sigma_n = u_{k+1}+u_{k+2}+\dotsb+u_{k+n} = s_{k+n} - s_k,
\end{equation*}其中\(s_{k+n}\)是原级数的前\(k+n\)项的和.
因为\(s_k\)是常数,所以当\(n\to\infty\)时,
\(\sigma_n\)与\(s_{k+n}\)这两个量,要么同时具有极限,要么同时没有极限.

同理可证在级数的前面加上有限项,不会改变级数的收敛性.
由此可见,在级数中去掉、加上或改变有限项,不会改变级数的收敛性.

另外,我们可以观察到,在收敛级数中去掉、加上或改变有限项,
虽不会改变级数的收敛性,但可能会改变收敛级数的和.
\end{proof}
\end{property}

\begin{property}\label{theorem:无穷级数.收敛级数性质4}
%@see: 《高等数学(第六版 上册)》 P252 性质4
%@see: 《数学分析(第二版 上册)》(陈纪修) P4 定理9.1.3
如果级数\(\sum_{n=1}^\infty u_n\)收敛,
则对这级数的项任意加括号后所成的级数\begin{equation*}
%@see: 《高等数学(第六版 上册)》 P252 (4)
	(u_1+\dotsb+u_{n_1})
	+ (u_{n_1+1}+\dotsb+u_{n_2})
	+ \dotsb
	+ (u_{n_{k-1}+1}+\dotsb+u_{n_k}) + \dotsb
\end{equation*}仍收敛,且其和不变.
\begin{proof}
设\(\sum_{n=1}^\infty u_n\)添加括号后表示为\begin{align*}
	&(u_1 + u_2 + \dotsb + u_{n_1}) \\
	&+ (u_{n_1+1} + u_{n_1+2} + \dotsb + u_{n_2}) \\
	&+ \dotsb + (u_{n_{k-1}+1} + u_{n_{k-1}+2} + \dotsb + u_{n_k}) + \dotsb.
\end{align*}
令\begin{equation*}
	\begin{array}{l}
		v_1 = u_1 + u_2 + \dotsb + u_{n_1}, \\
		v_2 = u_{n_1+1} + u_{n_1+2} + \dotsb + u_{n_2}, \\
		\hdotsfor{1} \\
		v_k = u_{n_{k-1}+1} + u_{n_{k-1}+2} + \dotsb + u_{n_k}, \\
		\hdotsfor{1}
	\end{array}
\end{equation*}
则\(\sum_{n=1}^\infty u_n\)按上面上述方式添加括号后所得的级数为\(\sum_{n=1}^\infty v_n\).

设\(\sum_{n=1}^\infty u_n\)的部分和数列为\(\{S_n\}\),
\(\sum_{n=1}^\infty v_n\)的部分和数列为\(\{S'_n\}\),
则\begin{equation*}
	S'_1 = S_{n_1},
	S'_2 = S_{n_2},
	\dotsc,
	S'_k = S_{n_k},
	\dotsc
\end{equation*}
显然\(\{S'_n\}\)是\(\{S_n\}\)的一个子列,
于是由\(\{S_n\}\)的收敛性可得\(\{S'_n\}\)的收敛性,
且两者极限相同.
\end{proof}
\end{property}
\begin{remark}
\cref{theorem:无穷级数.收敛级数性质4} 说明:
收敛的级数满足加法结合律.
\end{remark}
\begin{remark}
在极限论中我们已经知道,一个数列的某个子列收敛并不能保证数列本身收敛.
因此,相应地,在一个级数的和式中,添加了括号后所得的级数收敛,并不能保证原来的级数收敛.
例如,级数\begin{equation*}
	1-1+1-1+\dotsb
\end{equation*}是发散的,
但是,加上括号以后,级数\begin{equation*}
	(1-1)+(1-1)+\dotsb
\end{equation*}收敛于\(0\),
而级数\begin{equation*}
	1+(-1+1)+(-1+1)+\dotsb
\end{equation*}收敛于\(1\).
这就说明,发散的级数不满足加法结合律.
\end{remark}
\begin{corollary}
如果加括号后所成的级数发散,则原来的级数也发散.
\begin{proof}
用反证法.
倘若原级数收敛,
则根据\cref{theorem:无穷级数.收敛级数性质4} 知道,
加括号后的级数就应该收敛了.
\end{proof}
\end{corollary}

\subsection{柯西审敛原理}
由于无穷级数\(\sum_{n=1}^\infty u_n\)收敛
即为极限\(\lim_{n\to\infty} \sum_{k=1}^n u_k\)存在,
对无穷级数收敛性的最本质的刻画,
就是极限论中的\hyperref[theorem:极限.函数的柯西极限存在准则]{柯西极限存在准则},
它可以表述为如下形式:
\begin{theorem}[柯西审敛原理]\label{theorem:无穷级数.级数的柯西审敛原理}
%@see: 《高等数学(第六版 下册)》 P254 柯西审敛原理
%@see: 《数学分析(第二版 下册)》(陈纪修) P29 定理9.4.1(级数的Cauchy收敛原理)
%@see: 《数学分析教程(第3版 下册)》(史济怀) P180 定理14.4.1(Cauchy收敛原理)
%@see: 《数学分析(第2册)》(周民强) P170 定理9.14
级数\(\sum_{n=1}^\infty u_n\)收敛的充分必要条件为:
对于任意给定的正数\(\epsilon\),
总存在正整数\(N\),
使得当\(n>N\)时,
对于任意的正整数\(m\),
总是成立\begin{equation*}
	\abs{ \sum_{k=1}^m u_{n+k} }
	< \epsilon.
\end{equation*}
\end{theorem}

我们可以利用极限记号,
把\hyperref[theorem:无穷级数.级数的柯西审敛原理]{柯西审敛原理}改写为如下命题.
\begin{proposition}
级数\(\sum_{n=1}^\infty u_n\)收敛的充分必要条件是:
对于任意正整数\(m\),总有\begin{equation*}
	\lim_{n\to\infty} \sum_{k=1}^m u_{n+k} = 0.
\end{equation*}
\end{proposition}

\begin{example}\label{example:无穷级数.zeta2的敛散性}
试求级数\(\sum_{n=1}^\infty \frac{1}{n^2}\)的收敛性.
\begin{solution}
因为对任意正整数\(p\),\begin{align*}
	&\abs{u_{n+1}+u_{n+2}+\dotsb+u_{n+p}} \\
	&=\frac{1}{(n+1)^2}+\frac{1}{(n+2)^2}+\dotsb+\frac{1}{(n+p)^2} \\
	&<\frac{1}{n(n+1)}+\frac{1}{(n+1)(n+2)}+\dotsb+\frac{1}{(n+p-1)(n+p)} \\
	&=\left(\frac{1}{n}-\frac{1}{n+1}\right)
		+\left(\frac{1}{n+1}-\frac{1}{n+2}\right)
		+\dotsb+\left(\frac{1}{n+p-1}-\frac{1}{n+p}\right) \\
	&=\frac{1}{n}-\frac{1}{n+p} < \frac{1}{n},
\end{align*}
所以对于\(\forall \epsilon > 0\),
取正整数\(N \geq \frac{1}{\epsilon}\),
则当\(n > N\)时,
对任何正整数\(p\),
都有\begin{equation*}
	\abs{u_{n+1}+u_{n+2}+\dotsb+u_{n+p}}
	< \frac{1}{n}
	< \frac{1}{N}
	\leq \epsilon
\end{equation*}成立.
按柯西审敛原理,级数\(\sum_{n=1}^\infty \frac{1}{n^2}\)收敛.
\end{solution}
\end{example}

\begin{example}
%@see: 《数学分析(第2册)》(周民强) P171 例1
设\(\{u_n\}\)是单调增加的无界的正数列.
证明:级数\(\sum_{n=1}^\infty \left(1-\frac{u_n}{u_{n+1}}\right)\)发散.
\begin{proof}
显然\(u_n\to+\infty\ (n\to\infty)\),
那么对于任意给定正整数\(N\),
当\(p\)充分大时,
有\begin{equation*}
	u_{N+p} > 2 u_N.
\end{equation*}
从而有\begin{align*}
	&\sum_{n=N}^{N+p} \left(1-\frac{u_n}{u_{n+1}}\right)
	= \sum_{n=N}^{N+p} u_n \left(\frac1{u_n}-\frac1{u_{n+1}}\right) \\
	&\geq u_N \left(\frac1{u_N}-\frac1{u_{N+p+1}}\right)
	= 1-\frac{u_N}{u_{N+p+1}}
	> \frac12.
\end{align*}
于是由{柯西审敛原理}可知级数\(\sum_{n=1}^\infty \left(1-\frac{u_n}{u_{n+1}}\right)\)发散.
\end{proof}
\end{example}

\begin{example}
%@see: 《数学分析(第2册)》(周民强) P171 例2
设\(a_n>0\ (n=1,2,\dotsc)\),\(p>0\),
级数\(\sum_{n=1}^\infty \frac{a_n}{n^p}\)收敛.
证明:\(\lim_{n\to\infty} \sum_{k=1}^n \frac{a_k}{n^p} = 0\).
\begin{proof}
因为级数\(\sum_{n=1}^\infty \frac{a_n}{n^p}\)收敛,
由{柯西审敛原理}可知,
对任意给定\(\epsilon>0\),
存在正整数\(N_1\),
对于任意正整数\(n \geq N_1\),
成立\begin{equation*}
	% N_1 \leq k \leq n
	% \implies
	% 0 < \frac1n \leq \frac1k
	% \implies
	\sum_{k=N_1+1}^n \frac{a_k}{n^p}
	< \sum_{k=N_1+1}^n \frac{a_k}{k^p}
	< \epsilon,
\end{equation*}
又因为\(\lim_{n\to\infty} \sum_{k=1}^{N_1} \frac{a_k}{n^p} = 0\),
故同时存在正整数\(N_2\),
当\(n > N_2\)时,
成立\begin{equation*}
	\sum_{k=1}^{N_1} \frac{a_k}{n^p}
	\leq \frac{N_1}{n^p} \cdot \max_{1 \leq k \leq N_1} a_k
	< \epsilon,
\end{equation*}
于是,当\(n>\max\{N_1,N_2\}\)时,成立\begin{equation*}
	\sum_{k=1}^n \frac{a_k}{n^p}
	= \sum_{k=N_1+1}^n \frac{a_k}{n^p}
	+ \sum_{k=1}^{N_1} \frac{a_k}{n^p}
	< 2\epsilon,
\end{equation*}
这就说明\(\lim_{n\to\infty} \sum_{k=1}^n \frac{a_k}{n^p} = 0\).
\end{proof}
\end{example}

\begin{example}
%@see: 《数学分析(第二版 下册)》(陈纪修) P43 习题 3.
%@see: 《数学分析(第2册)》(周民强) P150 定理9.6(Pringsheim)
设正项级数\(\sum_{n=1}^\infty u_n\)收敛,数列\(\{u_n\}\)单调减少.
利用柯西审敛原理证明:\(\lim_{n\to\infty} n u_n = 0\).
\begin{proof}
由于级数\(\sum_{n=1}^\infty u_n\)收敛,数列\(\{u_n\}\)单调减少,
所以对于任意给定\(\epsilon>0\),
存在正整数\(N\),使得当\(n>N\)时,
对于任意正整数\(m\),总是成立\begin{equation*}
	m \cdot u_{n+m}
	< \abs{\sum_{k=1}^m u_{n+k}}
	= u_{n+1} + u_{n+2} + \dotsb + u_{n+m}
	< \epsilon.
\end{equation*}

令\(m=n\),
则\(n u_{2n} < \epsilon\),
于是\(\lim_{n\to\infty} n u_{2n} = 0\).

又令\(m=n-1\),
则\((n-1) u_{2n-1} < \epsilon\),
于是\(\lim_{n\to\infty} (n-1) u_{2n-1} = 0\).

考虑到\(\lim_{n\to\infty} u_n = 0\),
于是又有\(\lim_{n\to\infty} (2n-1) u_{2n-1} = 0\).

综上所述,\(\lim_{n\to\infty} n u_n = 0\).
\end{proof}
\end{example}

%@see: [19世纪上半叶的无穷级数敛散性判别法(华东师范大学数学系,汪晓勤)](https://faculty.ecnu.edu.cn/picture/article/1284/f3/48/968f8e4243b885767e11fcf63faf/5e2f0d6b-5780-4ad8-92d3-7cf2516e7083.pdf.x)
\section{正项级数及其审敛法}
\subsection{正项级数的概念及其收敛条件}
一般的常数项级数,它的各项可以是正数、负数或零.
现在我们先讨论“正项级数”.
这种级数特别重要,以后将看到许多级数的收敛性问题可归结为正项级数的收敛性问题.

\begin{definition}
%@see: 《数学分析(第二版 下册)》(陈纪修) P16 定义9.3.1
如果级数\(\sum_{n=1}^\infty u_n\)的一般项是非负实数,
即\begin{equation*}
	u_n\geq0\ (n=1,2,\dotsc),
\end{equation*}
则称“级数\(\sum_{n=1}^\infty u_n\)是\DefineConcept{正项级数}(positive term series)”.
\end{definition}

\begin{property}\label{theorem:无穷级数.正项级数部分和数列单调增加}
正项级数的部分和数列是单调增加的.
\begin{proof}
设\(\sum_{n=1}^\infty u_n\)是一个正项级数,
\(\{S_n\}\)是它的部分和数列,
那么\begin{equation*}
	S_{n+1} - S_n = a_{n+1} \geq 0
	\quad(n=1,2,\dotsc),
\end{equation*}
即\(S_{n+1} \geq S_n\ (n=1,2,\dotsc)\),
部分和数列\(\{S_n\}\)是单调增加的.
\end{proof}
\end{property}

\begin{theorem}\label{theorem:无穷级数.正项级数收敛的充分必要条件}
%@see: 《高等数学(第六版 下册)》 P256 定理1
%@see: 《数学分析教程 (第3版 下册)》(史济怀) P163 定理14.2.1
%@see: 《数学分析(第二版 下册)》(陈纪修) P16 定理9.3.1(正项级数的收敛原理)
正项级数收敛的充分必要条件是:它的部分和数列有界.
\begin{proof}
设\(\sum_{n=1}^\infty u_n\)是一个正项级数,
它的部分和数列为\(\{s_n\}\).

假设数列\(\{s_n\}\)有界.
那么由\cref{theorem:无穷级数.正项级数部分和数列单调增加} 可知,
数列\(\{s_n\}\)是一个单调有界数列,
于是根据\hyperref[theorem:极限.数列的单调有界定理]{单调有界定理},
数列\(\{s_n\}\)收敛,
根据\hyperref[definition:无穷级数.常数项级数的定义]{级数的定义},
级数\(\sum_{n=1}^\infty u_n\)收敛.

反之,如果正项级数\(\sum_{n=1}^\infty u_n\)收敛于和\(s\),
由\hyperref[definition:无穷级数.常数项级数的定义]{级数的定义}可知
\(\lim_{n\to\infty} s_n = s\),
根据\hyperref[theorem:极限.收敛数列的有界性]{收敛数列的有界性}可知,
数列\(\{s_n\}\)有界.
\end{proof}
\end{theorem}

\begin{example}
%@see: 《数学分析(第2册)》(周民强) P148 例1
设\(\{u_n\}\)是单调增加的有界的正数列.
证明:级数\(\sum_{n=1}^\infty \left(1-\frac{u_n}{u_{n+1}}\right)\)收敛.
\begin{proof}
不妨设\(0 < u_n \leq M\ (n=1,2,\dotsc)\).
因为\begin{align*}
	0 < S_n &= \sum_{k=1}^n \left(1-\frac{u_n}{u_{n+1}}\right)
	= \sum_{k=1}^n u_k \left(\frac1{u_k} - \frac1{u_{k+1}}\right) \\
	&= M \sum_{k=1}^n \left(\frac1{u_k} - \frac1{u_{k+1}}\right)
	= M \left(\frac1{u_1} - \frac1{u_{n+1}}\right)
	\leq \frac{M}{u_1},
\end{align*}
所以部分和数列\(\{S_n\}\)是有界的,
级数\(\sum_{n=1}^\infty \left(1-\frac{u_n}{u_{n+1}}\right)\)收敛.
\end{proof}
\end{example}
\begin{example}
%@see: 《2019年全国硕士研究生入学统一考试(数学一)》一选择题/第3题/选项(C)
设\(\{u_n\}\)是单调增加的有界数列.
举例说明:级数\(\sum_{n=1}^\infty \left(1-\frac{u_n}{u_{n+1}}\right)\)发散.
\begin{solution}
取\begin{equation*}
	u_n = -\frac1n
	\quad(n=1,2,\dotsc),
\end{equation*}
易见\(u_1 \leq u_n < u_{n+1} < 0\ (n=1,2,\dotsc)\),
\(\{u_n\}\)是单调增加的有界数列,
而级数\begin{equation*}
	\sum_{n=1}^\infty \left(1-\frac{u_n}{u_{n+1}}\right)
	= -\sum_{n=1}^\infty \frac1n
\end{equation*}发散.
\end{solution}
\end{example}

\begin{example}
%@see: 《数学分析(第2册)》(周民强) P148 例2
设\(\{u_n\}\)满足\(u_1 = 1, u_2 = 2\),且有递推公式\(u_n = u_{n-1} + u_{n-2}\ (n=3,4,\dotsc)\).
证明:级数\(\sum_{n=1}^\infty \frac1{u_n}\)收敛.
\begin{proof}
利用数学归纳法可证:\(u_{n+1} > u_n > 0\ (n=1,2,\dotsc)\)
且\(u_{n-2} < u_{n-1} < 2 u_{n-2}\ (n=3,4,\dotsc)\).
于是\begin{equation*}
	u_n > u_{n-1} + \frac12 u_{n-1} = \frac32 u_{n-1}
	\quad(n=2,3,\dotsc).
\end{equation*}
由此可知,\(\{u_n\}\)的前\(n\)项和\(S_n\)满足\begin{equation*}
	S_n - 1 = \sum_{k=2}^n \frac1{u_k}
	< \frac23 \sum_{k=2}^n \frac1{u_{k-1}}
	= \frac23 (S_n - u_n^{-1})
	\quad(n=1,2,\dotsc),
\end{equation*}
即\begin{equation*}
	S_n < 3 - \frac2{u_n} < 3
	\quad(n=1,2,\dotsc),
\end{equation*}
这说明部分和数列\(\{S_n\}\)有上界,级数\(\sum_{n=1}^\infty \frac1{u_n}\)收敛.
\end{proof}
\end{example}

\begin{proposition}
%@see: 《数学分析(第二版 下册)》(陈纪修) P16 定理9.3.1(正项级数的收敛原理)
正项级数发散的充分必要条件是:它的部分和数列无界.
\begin{proof}
\cref{theorem:无穷级数.正项级数收敛的充分必要条件} 的逆否命题.
\end{proof}
\end{proposition}

\begin{proposition}
正项级数发散的充分必要条件是:它的部分和\(S_n\)满足\(\lim_{n\to\infty} S_n = +\infty\).
\end{proposition}

\begin{example}
%@see: 《数学分析教程 (第3版 下册)》(史济怀) P163 例1
%@see: 《数学分析(第二版 下册)》(陈纪修) P29 习题 13
设正项级数\(\sum_{n=1}^\infty a_n\)的部分和是\(S_n\).
证明:\begin{equation*}
	\sum_{n=1}^\infty \frac{a_n}{S_n^2} < +\infty.
\end{equation*}
\begin{proof}
显然\(\sum_{n=1}^\infty \frac{a_n}{S_n^2}\)是正项级数.
根据\cref{theorem:无穷级数.正项级数收敛的充分必要条件},
只需证明它的部分和有界.
由于对于任意正整数\(N\),有\begin{align*}
	\sum_{n=2}^N \frac{a_n}{S_n^2}
	&= \sum_{n=2}^N \frac{S_n-S_{n-1}}{S_n^2}
	\leq \sum_{n=2}^N \frac{S_n-S_{n-1}}{S_{n-1} S_n} \\
	&= \sum_{n=2}^N \left(
			\frac{1}{S_{n-1}} - \frac{1}{S_n}
		\right)
	= \frac{1}{S_1} - \frac{1}{S_N}
	< \frac{1}{a_1},
\end{align*}
也就是说\(\sum_{n=1}^\infty \frac{a_n}{S_n^2}\)的部分和有界,
所以根据\cref{theorem:无穷级数.正项级数收敛的充分必要条件},该级数收敛.
\end{proof}
\end{example}

\subsection{柯西凝聚判别法}
\begin{theorem}\label{theorem:无穷级数.柯西凝聚判别法}
%@see: 《数学分析(第2册)》(周民强) P149 定理9.5
设\(\{a_n\}\)是单调减少的正数列,
则级数\(\sum_{n=1}^\infty a_n\)收敛的充分必要条件是:
级数\begin{equation*}
	\sum_{n=0}^\infty 2^n a_{2^n}
\end{equation*}收敛.
\begin{proof}
记\(S_n \defeq \sum_{k=1}^n a_k,
\sigma_n \defeq \sum_{k=1}^n 2^k a_{2^k}\).

当\(n \leq 2^k\)时,有\begin{align*}
	S_n &\leq a_1 + (a_2 + a_3)
	+ \dotsb + (a_{2^k} + a_{2^k+1} + \dotsb + a_{2^{k+1}-1}) \\
	&= a_1 + 2 a_2 + 4 a_4 + \dotsb + 2^k a_{2^k}
	= \sigma_k.
\end{align*}
若\(\sum_{n=0}^\infty 2^n a_{2^n}\)收敛于\(\sigma\),
则\(S_n \leq \sigma\),
说明\(\{S_n\}\)有界,
级数\(\sum_{n=1}^\infty a_n\)收敛.

当\(n > 2^k\)时,有\begin{align*}
	S_n &\geq a_1 + a_2 + (a_3 + a_4)
	+ \dotsb + (a_{2^{k-1}+1} + \dotsb + a_{2^k}) \\
	&\geq \frac12 a_1 + a_2 + 2 a_4 + \dotsb + 2^{k-1} a_{2^k} \\
	&\geq \frac12 (a_1 + 2 a_2 + 4 a_4 + \dotsb + a^k a_{2^k})
	= \frac12 \sigma_n.
\end{align*}
若\(\sum_{n=1}^\infty a_n\)收敛于\(S\),
则\(\sigma_k \leq 2 S\),
说明\(\{\sigma_k\}\)有界,
级数\(\sum_{n=0}^\infty 2^n a_{2^n}\)收敛.
\end{proof}
\end{theorem}

\subsection{比较审敛法}
利用\cref{theorem:无穷级数.正项级数收敛的充分必要条件} 直接证明某些级数的部分和有界不太容易,
因此我们需要一个判别级数敛散性的更简单的方法.

\begin{theorem}[比较审敛法]\label{theorem:无穷级数.正项级数的比较审敛法}
%@see: 《高等数学(第六版 下册)》 P256 定理2
%@see: 《数学分析教程 (第3版 下册)》(史济怀) P164 定理14.2.2
%@see: 《数学分析(第2册)》(周民强) P151 定理9.7
设\(\sum_{n=1}^\infty u_n\)
和\(\sum_{n=1}^\infty v_n\)都是正项级数,且\begin{equation*}
	u_n \leq v_n
	\quad(n=1,2,\dotsc).
\end{equation*}
若级数\(\sum_{n=1}^\infty v_n\)收敛,
则级数\(\sum_{n=1}^\infty u_n\)收敛;
反之,若级数\(\sum_{n=1}^\infty u_n\)发散,
则级数\(\sum_{n=1}^\infty v_n\)发散.
\begin{proof}
设级数\(\sum_{n=1}^\infty v_n\)收敛于和\(\sigma\),
则级数\(\sum_{n=1}^\infty u_n\)的部分和\begin{equation*}
	s_n = u_1 + u_2 + \dotsb u_n
	\leq
	v_1 + v_2 + \dotsb + v_n \leq \sigma
	\quad(n=1,2,\dotsc),
\end{equation*}
即部分和数列\(\{s_n\}\)有界,
由\cref{theorem:无穷级数.正项级数收敛的充分必要条件} 知级数\(\sum_{n=1}^\infty u_n\)收敛.
\end{proof}
\end{theorem}
\begin{remark}
%@see: 《数学分析教程 (第3版 下册)》(史济怀) P166 练习题14.2 1.
这里不能去掉前提条件“\(\sum_{n=1}^\infty u_n\)
和\(\sum_{n=1}^\infty v_n\)都是正项级数”.
取\begin{equation*}
	u_n = -\frac1n,
	\qquad
	v_n = \frac1{n^2},
\end{equation*}
则级数\(\sum_{n=1}^\infty v_n\)收敛,
而级数\(\sum_{n=1}^\infty u_n\)发散.
\end{remark}

\begin{example}
%@see: 《高等数学(第六版 下册)》 P268 习题12-2 1. (5)
判断级数\(\sum_{n=1}^\infty \frac{1}{1+a^n}\ (a>0)\)的收敛性.
\begin{solution}
显然有\begin{equation*}
	0 < \frac{1}{1+a^n} < \frac{1}{a^n}.
\end{equation*}
根据\hyperref[theorem:无穷级数.正项级数的比较审敛法]{比较审敛法},
如果级数\(\sum_{n=1}^\infty \frac{1}{a^n}\)收敛,
那么级数\(\sum_{n=1}^\infty \frac{1}{1+a^n}\)收敛.
然而,等比级数\(\sum_{n=1}^\infty \frac{1}{a^n}\)收敛
当且仅当\(\abs{\frac{1}{a}} < 1\)或\(a > 1\).

因此,当\(a > 1\)时,级数\(\sum_{n=1}^\infty \frac{1}{1+a^n}\)收敛;
当\(0 < a \leq 1\)时,级数\(\sum_{n=1}^\infty \frac{1}{1+a^n}\)发散.
\end{solution}
\end{example}

注意到级数的每一项同乘不为零的常数\(k\)
以及去掉级数前面部分的有限项不会影响级数的收敛性,
我们可得如下推论:
\begin{corollary}\label{theorem:无穷级数.正项级数的比较审敛法的推论}
%@see: 《高等数学(第六版 下册)》 P257 推论
%@see: 《数学分析(第二版 下册)》(陈纪修) P17 定理9.3.2(比较判别法)
设\(\sum_{n=1}^\infty u_n\)和\(\sum_{n=1}^\infty v_n\)都是正项级数.
\begin{itemize}
	\item 如果级数\(\sum_{n=1}^\infty v_n\)收敛,
	且从某一项开始\(u_n\)小于或等于\(v_n\)的正倍数,
	即\begin{equation*}
		(\exists k>0)
		(\exists N\in\mathbb{N})
		(\forall n\in\mathbb{N})
		[
			n > N
			\implies
			u_n \leq k v_n
		],
	\end{equation*}
	则级数\(\sum_{n=1}^\infty u_n\)收敛.

	\item 如果级数\(\sum_{n=1}^\infty v_n\)发散,
	且从某一项开始\(u_n\)大于或等于\(v_n\)的正倍数,
	即\begin{equation*}
		(\exists k>0)
		(\exists N\in\mathbb{N})
		(\forall n\in\mathbb{N})
		[
			n > N
			\implies
			u_n \geq k v_n
		],
	\end{equation*}
	则级数\(\sum_{n=1}^\infty u_n\)发散.
\end{itemize}
\end{corollary}

\begin{example}\label{example:无穷级数.p级数的收敛性}
%@see: 《高等数学(第六版 下册)》 P257 例1
%@see: 《数学分析(第二版 下册)》(陈纪修) P3 例9.1.3
讨论\(p\)级数\begin{equation*}
	1+\frac{1}{2^p}+\frac{1}{3^p}+\dotsb+\frac{1}{n^p}+\dotsb
\end{equation*}的收敛性,
其中常数\(p>0\).
\begin{solution}
当\(p \leq 1\)时,\(p\)级数各项均不小于调和级数对应项,
即\(\frac{1}{n^p} \geq \frac{1}{n}\),
但调和级数发散,
故根据\cref{theorem:无穷级数.正项级数的比较审敛法} 可知,
当\(p \leq 1\)时\(p\)级数发散.

当\(p > 1\)时,
因为\(k-1
\leq x
\leq k \implies \frac{1}{k}
\leq \frac{1}{x} \implies \frac{1}{k^p}
\leq \frac{1}{x^p}\),
所以\begin{equation*}
	\frac{1}{k^p}
	= \int_{k-1}^k \frac{1}{k^p} \dd{x}
	\leq \int_{k-1}^k \frac{1}{x^p} \dd{x}
	\quad(k=2,3,\dotsc),
\end{equation*}
从而级数的部分和
\begin{align*}
	s_n &= 1 + \sum_{k=2}^n{\frac{1}{k^p}}
	\leq 1 + \sum_{k=2}^n{ \int_{k-1}^k{\frac{1}{x^p}\dd{x}} }
	= 1 + \int_1^n{\frac{1}{x^p}\dd{x}} \\
	&= 1 + \frac{1}{p-1}\left(1-\frac{1}{n^{p-1}}\right)
	< 1 + \frac{1}{p-1}
	\quad(n=2,3,\dotsc),
\end{align*}
这表明数列\(\{s_n\}\)有界,因此\(p\)级数收敛.

综上所述,{\color{red} \(p\)级数\(\sum_{n=1}^\infty \frac{1}{n^p}\)
当\(p > 1\)时收敛,
当\(p \leq 1\)时发散.}
\end{solution}
\end{example}

\begin{example}
%@see: 《数学分析(第2册)》(周民强) P153 例2
设\(\{n a_n\}\)是有界数列,证明:\(\sum_{n=1}^\infty a_n^2\)收敛.
\begin{proof}
设\(\abs{n a_n} \leq M\ (n=1,2,\dotsc)\),
则\(a_n^2 = \frac{(n a_n)^2}{n^2} \leq \frac{M^2}{n^2}\),
由\cref{example:无穷级数.p级数的收敛性} 可知
级数\(\sum_{n=1}^\infty \frac1{n^2}\)收敛,
于是由\hyperref[theorem:无穷级数.正项级数的比较审敛法]{比较审敛法}可知
级数\(\sum_{n=1}^\infty a_n^2\)收敛.
\end{proof}
\end{example}

\begin{example}
%@see: 《高等数学(第六版 下册)》 P257 例2
试证:级数\(\sum_{n=1}^\infty \frac{1}{\sqrt{n(n+1)}}\)是发散的.
\begin{proof}
因为\(n(n+1) < (n+1)^2\),所以\(\frac{1}{\sqrt{n(n+1)}} > \frac{1}{n+1}\),
而级数\(\sum_{n=1}^\infty \frac{1}{n+1}\)是发散的,
根据比较审敛法可知级数\(\sum_{n=1}^\infty \frac{1}{\sqrt{n(n+1)}}\)是发散的.
\end{proof}
\end{example}

\begin{example}
%@see: 《数学分析教程 (第3版 下册)》(史济怀) P163 例1
设正项级数\(\sum_{n=1}^\infty a_n\)的部分和是\(S_n\).
证明:对任意的\(\alpha>1\),有\begin{equation*}
	\sum_{n=1}^\infty \frac{a_n}{S_n^\alpha} < +\infty.
\end{equation*}
\begin{proof}
当\(\alpha>1\)时,
有\begin{align*}
	\sum_{i=2}^n \frac{a_i}{S_i^\alpha}
	&= \sum_{i=2}^n \frac{S_i-S_{i-1}}{S_i^\alpha} \\
	&= \sum_{i=2}^n \frac{1}{S_i^\alpha} \int_{S_{i-1}}^{S_i} \dd{x}
			\tag{\cref{theorem:定积分.定积分性质4}} \\
	&\leq \sum_{i=2}^n \int_{S_{i-1}}^{S_i} \frac{\dd{x}}{x^\alpha}
			\tag{\cref{theorem:定积分.定积分性质6}} \\
	&= \int_{S_1}^{S_n} \frac{\dd{x}}{x^\alpha}
			\tag{\cref{theorem:定积分.定积分性质3}} \\
	&\leq \int_{S_1}^{+\infty} \frac{\dd{x}}{x^\alpha}
	< +\infty,
			\tag{\cref{example:定积分.p积分}}
\end{align*}
可见该级数的部分和有界,因此该级数收敛.
\end{proof}
\end{example}

\begin{example}
设正项级数\(\sum_{n=1}^\infty a_n\)发散.
证明:级数\(\sum_{n=1}^\infty \frac{a_n}{n^3+a_n^2}\)收敛.
\begin{proof}
由\hyperref[theorem:不等式.基本不等式2]{基本不等式}可知\begin{equation*}
	n^3+a_n^2\geq2\sqrt{n^3 a_n^2}=2n^{3/2}a_n,
\end{equation*}
那么\begin{equation*}
	\frac{a_n}{n^3+a_n^2}\leq\frac{a_n}{2n^{3/2}a_n}=\frac{1}{2n^{3/2}}.
\end{equation*}
由\cref{example:无穷级数.p级数的收敛性}
我们知道\(p=\frac{3}{2}>1\)时,\(p\)级数收敛;
那么根据\hyperref[theorem:无穷级数.正项级数的比较审敛法]{比较审敛法}可知
级数\(\sum_{n=1}^\infty \frac{a_n}{n^3+a_n^2}\)收敛.
\end{proof}
\end{example}

\begin{example}
设数列\(\{u_n\},\{v_n\},\{w_n\}\)满足\begin{equation*}
	v_n \leq u_n \leq w_n.
\end{equation*}
证明:如果\(\sum_{n=1}^\infty v_n\)和\(\sum_{n=1}^\infty w_n\)都收敛,
则\(\sum_{n=1}^\infty u_n\)也收敛.
\begin{proof}
由\hyperref[theorem:无穷级数.收敛级数性质2]{收敛级数的性质}可知
级数\(\sum_{n=1}^\infty (w_n - v_n)\)收敛.
由\cref{theorem:不等式.加法的单调性} 可知\begin{equation*}
	0 \leq u_n - v_n \leq w_n - v_n.
\end{equation*}
因此由\hyperref[theorem:无穷级数.正项级数的比较审敛法]{比较审敛法}可知
级数\(\sum_{n=1}^\infty (u_n - v_n)\)收敛.
于是再由\hyperref[theorem:无穷级数.收敛级数性质2]{收敛级数的性质}可知
级数\begin{equation*}
	\sum_{n=1}^\infty u_n
	= \sum_{n=1}^\infty ((u_n - v_n) + v_n)
\end{equation*}收敛.
\end{proof}
\end{example}

\subsection{比较审敛法的比值形式}
\begin{theorem}\label{theorem:无穷级数.比较审敛法的比值形式}
%@see: 《数学分析简明教程(第二版 下册)》(邓东皋、尹小玲) P14 定理10.8(比较判别法的另一形式)
%@see: 《数学分析(第2册)》(周民强) P152 推论9.2
设\(\sum_{n=1}^\infty u_n\)和\(\sum_{n=1}^\infty v_n\)都是正项级数.
设存在正整数\(N\),使得当\(n>N\)时,有\begin{equation*}
	\frac{u_{n+1}}{u_n} \leq \frac{v_{n+1}}{v_n}.
\end{equation*}
\begin{itemize}
	\item 如果\(\sum_{n=1}^\infty v_n\)收敛,则\(\sum_{n=1}^\infty u_n\)收敛.
	\item 如果\(\sum_{n=1}^\infty u_n\)发散,则\(\sum_{n=1}^\infty v_n\)发散.
\end{itemize}
%TODO proof
\end{theorem}

\subsection{比较审敛法的极限形式}
\begin{theorem}[比较审敛法的极限形式]\label{theorem:无穷级数.正项级数的比较审敛法的极限形式}
%@see: 《高等数学(第六版 下册)》 P258 定理3
%@see: 《数学分析(第二版 下册)》(陈纪修) P18 定理9.3.2'(比较判别法的极限形式)
%@see: 《数学分析(第2册)》(周民强) P155 定理9.8
%@see: https://mathworld.wolfram.com/LimitComparisonTest.html
设\(\sum_{n=1}^\infty u_n\)和\(\sum_{n=1}^\infty v_n\)都是正项级数,
记\begin{equation*}
	\rho
	\defeq
	\lim_{n\to\infty} \frac{u_n}{v_n}.
\end{equation*}
\begin{itemize}
	\item 如果\(\rho\in[0,+\infty)\),
	且级数\(\sum_{n=1}^\infty v_n\)收敛,
	则级数\(\sum_{n=1}^\infty u_n\)收敛.

	\item 如果\(\rho\in(0,+\infty]\),
	且级数\(\sum_{n=1}^\infty v_n\)发散,
	则级数\(\sum_{n=1}^\infty u_n\)发散.
\end{itemize}
\begin{proof}
如果\(\rho\in[0,+\infty)\),
那么由极限定义可知,
对\(\epsilon=1\),
存在正整数\(N\),
当\(n>N\)时,
有\begin{equation*}
	\frac{u_n}{v_n} < l+1,
\end{equation*}
即\(u_n < (l+1) v_n\).
而级数\(\sum_{n=1}^\infty v_n\)收敛,
根据\cref{theorem:无穷级数.正项级数的比较审敛法的推论} 可知,
级数\(\sum_{n=1}^\infty u_n\)收敛.

如果\(\rho\in(0,+\infty]\),
那么极限\(\lim_{n\to\infty} \frac{v_n}{u_n}\)存在且有限.
如果级数\(\sum_{n=1}^\infty u_n\)收敛,
那么由上可知,级数\(\sum_{n=1}^\infty v_n\)收敛;
但已知级数\(\sum_{n=1}^\infty v_n\)发散,矛盾!
因此级数\(\sum_{n=1}^\infty u_n\)不可能收敛,
即级数\(\sum_{n=1}^\infty u_n\)发散.
\end{proof}
\end{theorem}

极限形式的比较审敛法,在两个正项级数的一般项均趋于零的情况下,
其实是比较它们的一般项作为无穷小量的阶.
定理表明,当\(n \to \infty\)时,
如果\(u_n\)是与\(v_n\)同阶或是比\(v_n\)高阶的无穷小,
而级数\(\sum_{n=1}^\infty v_n\)收敛,则级数\(\sum_{n=1}^\infty u_n\)收敛;
如果\(u_n\)是与\(v_n\)同阶或是比\(v_n\)低阶的无穷小,
而级数\(\sum_{n=1}^\infty v_n\)发散,则级数\(\sum_{n=1}^\infty u_n\)发散.

\begin{example}
%@see: 《高等数学(第六版 下册)》 P258 例3
判断级数\(\sum_{n=1}^\infty \sin\frac{1}{n}\)的收敛性.
\begin{solution}
因为\begin{equation*}
	\lim_{n\to\infty} \frac{\sin(1/n)}{1/n} = 1 > 0,
\end{equation*}
而级数\(\sum_{n=1}^\infty \frac{1}{n}\)发散,
可知级数\(\sum_{n=1}^\infty \sin\frac{1}{n}\)发散.
\end{solution}
\end{example}

\begin{example}
%@see: 《高等数学(第六版 下册)》 P268 习题12-2 1. (1)
判断级数\begin{equation*}
	1 + \frac{1}{3} + \frac{1}{5} + \dotsb + \frac{1}{2n-1} + \dotsb
\end{equation*}的收敛性.
\begin{solution}
记\(u_n = \frac{1}{2n-1}\),
取\(v_n = \frac{1}{n}\).
因为\begin{equation*}
	\lim_{n\to\infty} \frac{u_n}{v_n}
	= \lim_{n\to\infty} \frac{n}{2n-1}
	= \lim_{n\to\infty} \frac{1}{2-1/n}
	= \frac{1}{2}
	> 0,
\end{equation*}
而级数\(\sum_{n=1}^\infty \frac{1}{n}\)发散,
所以级数\(\sum_{n=1}^\infty \frac{1}{2n-1}\)发散.
\end{solution}
\end{example}

\begin{example}
%@see: 《高等数学(第六版 下册)》 P268 习题12-2 1. (2)
判断级数\begin{equation*}
	1 + \frac{1+2}{1+2^2} + \frac{1+3}{1+3^2} + \dotsb + \frac{1+n}{1+n^2} + \dotsb
\end{equation*}的收敛性.
\begin{solution}
记\(u_n = \frac{1+n}{1+n^2}\),
取\(v_n = \frac{1}{1+n}\).
因为\begin{equation*}
	\lim_{n\to\infty} \frac{u_n}{v_n}
	= \lim_{n\to\infty} \frac{(1+n)^2}{1+n^2}
	= \lim_{n\to\infty} \frac{n^2 + 2n + 1}{n^2 + 1}
	= 1 > 0,
\end{equation*}
而\(\sum_{n=1}^\infty v_n\)发散,
所以级数\(\sum_{n=1}^\infty \frac{1+n}{1+n^2}\)发散.
\end{solution}
\end{example}

\begin{example}
%@see: 《高等数学(第六版 下册)》 P268 习题12-2 1. (3)
判断级数\begin{equation*}
	\frac{1}{2\cdot5} + \frac{1}{3\cdot6} + \dotsb + \frac{1}{(n+1)(n+4)} + \dotsb
\end{equation*}的收敛性.
\begin{solution}
记\(u_n = \frac{1}{(n+1)(n+4)}\),
取\(v_n = \frac{1}{n^2}\).
因为\begin{equation*}
	\lim_{n\to\infty} \frac{u_n}{v_n}
	= \lim_{n\to\infty} \frac{n^2}{(n+1)(n+4)}
	= 1,
\end{equation*}
而级数\(\sum_{n=1}^\infty v_n\)收敛,
所以级数\(\sum_{n=1}^\infty \frac{1+n}{1+n^2}\)收敛.
\end{solution}
\end{example}

\begin{example}
%@see: 《高等数学(第六版 下册)》 P268 习题12-2 1. (4)
\newcommand\sinfrac[1][]{\sin\frac{\pi}{2^{#1}}}
判断级数\begin{equation*}
	\sinfrac + \sinfrac[2] + \sinfrac[3] + \dotsb + \sinfrac[n] + \dotsb
\end{equation*}的收敛性.
\begin{solution}
记\(u_n = \sin\frac{\pi}{2^n}\),
取\(v_n = \frac{\pi}{2^n}\).
因为\begin{equation*}
	\lim_{n\to\infty} \frac{u_n}{v_n}
	= \lim_{n\to\infty} \frac{\sin(\pi/2^n)}{\pi/2^n}
	= 1,
\end{equation*}
而级数\(\sum_{n=1}^\infty v_n\)收敛,
所以级数\(\sum_{n=1}^\infty \sin\frac{\pi}{2^n}\)收敛.
\end{solution}
\end{example}

\begin{example}
%@see: 《数学分析(第二版 下册)》(陈纪修) P19 例9.3.4
判断正项级数\(\sum_{n=1}^\infty \left(\exp\frac1{n^2}-\cos\frac\pi{n}\right)\)的敛散性.
\begin{solution}
因为\begin{align*}
	\exp\frac1{n^2}-\cos\frac\pi{n}
	&= \left[1+\frac1{n^2}+o\left(\frac1{n^2}\right)\right]
	- \left[1-\frac12\left(\frac\pi{n}\right)^2+o\left(\frac1{n^2}\right)\right] \\
	&= \left(1+\frac{\pi^2}2\right) \frac1{n^2} + o\left(\frac1{n^2}\right),
\end{align*}
所以\begin{equation*}
	\lim_{n\to\infty} n^2 \left(\exp\frac1{n^2}-\cos\frac\pi{n}\right)
	= 1+\frac{\pi^2}2.
\end{equation*}
由于\(\sum_{n=1}^\infty \frac1{n^2}\)收敛,
所以\(\sum_{n=1}^\infty \left(\exp\frac1{n^2}-\cos\frac\pi{n}\right)\)收敛.
\end{solution}
\end{example}
\begin{remark}
从这里例子可以看出,对于一般项\(a_n = f(n)\)的形式比较复杂的级数,
可以令\(x = 1/n\)进行倒代换,得到一个函数\(g(x) = f(1/x)\),
首先排查\(x=0\)是不是\(g\)的一个奇点(如果\(x=0\)是\(g\)的一个奇点,那么级数必定发散),
然后求出函数\(g\)的带有拉格朗日余项的麦克劳林公式\begin{equation*}
	g(x) = \sum_{k=0}^n \frac{g^{(k)}(x_0)}{k!} (x-x_0)^k
	+ \frac{g^{(n+1)}(\xi)}{(n+1)!} (x-x_0)^{n+1}.
\end{equation*}
只要在点\(x=0\)的某个邻域内成立\(g(0) + g'(0) x = 0\),
就能证明级数\(\sum_{n=1}^\infty a_n\)收敛.
\end{remark}

\subsection{比值审敛法}
用比较审敛法审敛时,
需要适当地选取一个已知其收敛性的级数\(\sum_{n=1}^\infty v_n\)作为比较的基准.
最常选用作为基准级数的是正项等比级数\(\sum_{n=1}^\infty q^n\ (q>0)\)
和\(p\)级数\(\sum_{n=1}^\infty \frac1{n^p}\).

我们知道,\hyperref[example:无穷级数.等比级数的收敛性]{等比级数}
\(\sum_{n=1}^\infty q^n\)的敛散性只依赖于其相邻两项之比\(q\)是否小于\(1\).
利用\hyperref[theorem:无穷级数.正项级数的比较审敛法]{比较审敛法}可以得出以下结论:
设级数\(\sum_{n=1}^\infty u_n\)的后项与前项之比\(\frac{u_{n+1}}{u_n}\)
或前\(n\)项的“平均公比”\begin{equation*}
	\sqrt[n]{u_n}
	= \sqrt[n]{\frac{u_1}1\cdot\frac{u_2}{u_1}\dotsm\frac{u_n}{u_{n-1}}}
\end{equation*}的极限是\(\rho\),
如果\(\rho<1\),那么级数\(\sum_{n=1}^\infty u_n\)收敛;
如果\(\rho>1\),那么级数\(\sum_{n=1}^\infty u_n\)发散.
正是基于这样的思路,产生了如下的\hyperref[theorem:无穷级数.正项级数的比值审敛法]{比值审敛法}%
和\hyperref[theorem:无穷级数.正项级数的根值审敛法]{根值审敛法}.

\begin{theorem}[比值审敛法,达朗贝尔判别法]\label{theorem:无穷级数.正项级数的比值审敛法}
设\(\sum_{n=1}^\infty u_n\)是正项级数,
记\(\rho \defeq \lim_{n\to\infty} \frac{u_{n+1}}{u_n}\).
\begin{itemize}
	\item 当\(\rho<1\)时,级数\(\sum_{n=1}^\infty u_n\)收敛.
	\item 当\(\rho>1\)时,或当\(\rho=\infty\)时,级数\(\sum_{n=1}^\infty u_n\)发散.
	\item 当\(\rho=1\)时,级数\(\sum_{n=1}^\infty u_n\)可能收敛也可能发散.
\end{itemize}
\begin{proof}
当\(\rho<1\).
取一个适当小的正数\(\epsilon\),
使得\(\rho+\epsilon=r<1\),
根据极限定义,
存在正整数\(m\),
当\(n \geq m\)时有不等式\begin{equation*}
	\frac{u_{n+1}}{u_n} < \rho + \epsilon = r.
\end{equation*}
因此\begin{equation*}
	u_{m+1} < r u_m,
	u_{m+2} < r u_{m+1} < r^2 u_m,
	\dotsc,
	u_{m+k} < r^k u_m,
	\dotsc.
\end{equation*}
而因为公比\(r<1\),
故等比级数\(\sum_{k=1}^\infty r^k u_m\)收敛,
根据\cref{theorem:无穷级数.正项级数的比较审敛法的推论} 可知,
级数\(\sum_{n=1}^\infty u_n\)收敛.

当\(\rho>1\).
取一个适当小的正数\(\epsilon\),
使得\(\rho-\epsilon>1\).
根据极限定义,
当\(n \geq m\)时有不等式\begin{equation*}
	\frac{u_{n+1}}{u_n} > \rho-\epsilon > 1,
\end{equation*}
也就是\(u_{n+1}>u_n\).
所以当\(n \geq m\)时,
级数的一般项\(u_n\)是逐渐增大的,
从而\begin{equation*}
	\lim_{n\to\infty} u_n \neq 0.
\end{equation*}
根据\cref{theorem:无穷级数.级数收敛的必要条件} (即级数收敛的必要条件)可知,
级数\(\sum_{n=1}^\infty u_n\)发散.

类似地,可以证明当\(\lim_{n\to\infty} \frac{u_{n+1}}{u_n} = \infty\)时,
级数\(\sum_{n=1}^\infty u_n\)发散.

当\(\rho = 1\)时,
级数可能收敛也可能发散.
例如\(p\)级数不论\(p\)为何值都有\begin{equation*}
	\lim_{n\to\infty} \frac{u_{n+1}}{u_n}
	= \lim_{n\to\infty} \frac{1/(n+1)^p}{1/n^p} = 1.
\end{equation*}
但我们知道,
当\(p>1\)时\(p\)级数收敛,
当\(p\leq1\)时\(p\)级数发散,
因此只根据\(\rho=1\)不能判定级数的收敛性.
\end{proof}
\end{theorem}

\begin{example}\label{example:无穷级数.常数e的级数表示}
%@see: 《高等数学(第六版 下册)》 P260 例5
证明级数\begin{equation*}
	1+\frac{1}{1}+\frac{1}{1\cdot2}+\frac{1}{1\cdot2\cdot3}+\dotsb+\frac{1}{(n-1)!}+\dotsb
\end{equation*}是收敛的,
并估计以级数的部分和\(s_n\)近似代替和\(s\)所产生的误差.
\begin{solution}
因为\begin{equation*}
	\lim_{n\to\infty} \frac{u_{n+1}}{u_n}
	 \lim_{n\to\infty} \frac{(n-1)!}{n!}
	= \lim_{n\to\infty} \frac{1}{n} = 0 < 1,
\end{equation*}
根据比值审敛法可知,该级数收敛.

以该级数的部分和近似代替和\(s\)所产生的误差为\begin{align*}
	\abs{r_n} &= \frac{1}{n!} + \frac{1}{(n+1)!} + \frac{1}{(n+2)!} + \dotsb \\
	&= \frac{1}{n!} \left[ 1 + \frac{1}{n+1} + \frac{1}{(n+1)(n+2)} + \dotsb \right] \\
	&< \frac{1}{n!} \left( 1 + \frac{1}{n} + \frac{1}{n^2} + \dotsb \right) \\
	&= \frac{1}{n!} \frac{1}{1-1/n}
	= \frac{1}{(n-1)\cdot(n-1)!}.
\end{align*}
\end{solution}
\end{example}

\begin{example}
%@see: 《高等数学(第六版 下册)》 P268 习题12-2 2. (1)
判断级数\begin{equation*}
	\frac{3}{1\cdot2}
	+\frac{3^2}{2\cdot2^2}
	+\frac{3^3}{3\cdot2^3}
	+\dotsb
	+\frac{3^n}{n\cdot2^n}
	+\dotsb
\end{equation*}的收敛性.
\begin{solution}
记\(u_n = \frac{3^n}{n\cdot2^n}\).
因为\begin{equation*}
	\lim_{n\to\infty} \frac{u_{n+1}}{u_n}
	= \lim_{n\to\infty} \frac{3}{2}\cdot\frac{n}{n+1}
	= \frac{3}{2} > 1,
\end{equation*}
所以级数发散.
\end{solution}
\end{example}

\begin{example}
%@see: 《高等数学(第六版 下册)》 P268 习题12-2 2. (2)
判断级数\(\sum_{n=1}^\infty \frac{n^2}{3^n}\)的收敛性.
\begin{solution}
记\(u_n = \frac{n^2}{3^n}\).
因为\begin{equation*}
	\lim_{n\to\infty} \frac{u_{n+1}}{u_n}
	= \lim_{n\to\infty} \frac{1}{3} \cdot \frac{(n+1)^2}{n^2}
	= \frac{1}{3} < 1,
\end{equation*}
所以级数收敛.
\end{solution}
\end{example}

\begin{example}
%@see: 《高等数学(第六版 下册)》 P268 习题12-2 2. (3)
判断级数\(\sum_{n=1}^\infty \frac{2^n \cdot n!}{n^n}\)的收敛性.
\begin{solution}
记\(u_n = \frac{2^n \cdot n!}{n^n}\).
因为\begin{align*}
	\lim_{n\to\infty} \frac{u_{n+1}}{u_n}
	&= \lim_{n\to\infty} 2(n+1) \cdot \frac{n^n}{(n+1)^{n+1}} \\
	&= 2 \cdot \lim_{n\to\infty} \frac{n^n}{(n+1)^n}
	= 2 \cdot \lim_{n\to\infty} \left(1+\frac{1}{n}\right)^{-n}
	= \frac2e < 1,
\end{align*}
所以级数收敛.
\end{solution}
\end{example}

\begin{example}
%@see: 《高等数学(第六版 下册)》 P268 习题12-2 2. (4)
判断级数\(\sum_{n=1}^\infty n \tan\frac{\pi}{2^{n+1}}\)的收敛性.
\begin{solution}
记\(u_n = n \tan\frac{\pi}{2^{n+1}}\).
我们有\begin{equation*}
	\frac{u_{n+1}}{u_n}
	= \frac{n+1}{n} \frac{\tan(\frac{1}{2}\frac{\pi}{2^{n+1}})}{\tan\frac{\pi}{2^{n+1}}}.
\end{equation*}
根据二倍角公式\begin{equation*}
	\tan2\theta = \frac{2\tan\theta}{1-\tan^2\theta},
	\qquad
	\frac{\tan\theta}{\tan2\theta} = \frac{1-\tan^2\theta}{2},
\end{equation*}
有\begin{equation*}
	\frac{\tan(\frac{1}{2}\frac{\pi}{2^{n+1}})}{\tan\frac{\pi}{2^{n+1}}}
	= \frac{1}{2} \left(
		1-\tan^2\frac{\pi}{2^{n+2}}
	\right).
\end{equation*}
于是\begin{align*}
	\lim_{n\to\infty} \frac{u_{n+1}}{u_n}
	&= \lim_{n\to\infty} \frac{n+1}{n} \frac{1}{2} \left(
		1-\tan^2\frac{\pi}{2^{n+2}}
	\right) \\
	&= \frac{1}{2} \cdot \lim_{n\to\infty} \frac{n+1}{n} \cdot \left(
		1 - \lim_{n\to\infty} \tan^2\frac{\pi}{2^{n+2}}
	\right) \\
	&= \frac{1}{2} \cdot 1 \cdot (1 - 0) = \frac{1}{2} < 1.
\end{align*}
所以级数收敛.
\end{solution}
\end{example}

\subsection{比值审敛法的上、下极限形式}
\begin{corollary}[比值审敛法的上、下极限形式]\label{theorem:无穷级数.正项级数的比值审敛法的上下极限形式}
%@see: 《数学分析(第二版 下册)》(陈纪修) P20 定理9.3.4(d'Alembert判别法)
%@see: 《数学分析(第2册)》(周民强) P157 定理9.9
%@see: https://mathworld.wolfram.com/RatioTest.html
\def\orho{\overline{\rho}}
\def\urho{\underline{\rho}}
设\(\sum_{n=1}^\infty u_n\)是正项级数,%正项级数只隐含了\(u_n\geq0\)的条件,不保证\(u_n>0\).
\(u_n > 0\ (n=1,2,\dotsc)\),%保证\(u_n\)作为分母时,分式有意义!
记\begin{equation*}
	\orho
	\defeq
	\varlimsup_{n\to\infty} \frac{u_{n+1}}{u_n},
	\qquad
	\urho
	\defeq
	\varliminf_{n\to\infty} \frac{u_{n+1}}{u_n}.
\end{equation*}
\begin{itemize}
	\item 如果\(\orho < 1\),
	则级数\(\sum_{n=1}^\infty u_n\)收敛.

	\item 如果\(\urho > 1\),
	则级数\(\sum_{n=1}^\infty u_n\)发散.

	\item 如果\(\orho \geq 1\)或\(\urho \leq 1\),
	则级数\(\sum_{n=1}^\infty u_n\)可能收敛也可能发散.
\end{itemize}
%TODO proof
\end{corollary}

\begin{proposition}
%@see: 《数学分析(第2册)》(周民强) P162 引理9.1(对数比值型判别法)
设\(\sum_{n=1}^\infty a_n\)是正项级数,记\begin{equation*}
	\rho \defeq \lim_{n\to\infty} n \ln\frac{a_n}{a_{n+1}}.
\end{equation*}
\begin{itemize}
	\item 若\(\rho>1\),则\(\sum_{n=1}^\infty a_n\)收敛.
	\item 若\(\rho<1\),则\(\sum_{n=1}^\infty a_n\)发散.
\end{itemize}
%TODO proof
\end{proposition}

\subsection{根值审敛法}
\begin{theorem}[柯西根值审敛法]\label{theorem:无穷级数.正项级数的根值审敛法}
%@see: 《高等数学(第六版 下册)》 P260 定理5
%@see: 《数学分析(第二版 下册)》(陈纪修) P19 定理9.3.3(Cauchy判别法)
%@see: 《数学分析(第2册)》(周民强) P159 定理9.10
%@see: https://mathworld.wolfram.com/RootTest.html
设\(\sum_{n=1}^\infty u_n\)是正项级数,
记\(\rho \defeq \varlimsup_{n\to\infty} \sqrt[n]{u_n}\).
\begin{itemize}
	\item 当\(\rho<1\)时,级数\(\sum_{n=1}^\infty u_n\)收敛.
	\item 当\(\rho>1\)时,或当\(\rho=+\infty\)时,级数\(\sum_{n=1}^\infty u_n\)发散.
	\item 当\(\rho=1\)时,级数\(\sum_{n=1}^\infty u_n\)可能收敛也可能发散.
\end{itemize}
%TODO proof
\end{theorem}

\begin{example}
判定级数\(\sum_{n=1}^\infty \frac{2+(-1)^n}{2^n}\)的敛散性.
\begin{solution}
记\begin{equation*}
	u_n = \frac{2+(-1)^n}{2^n},
\end{equation*}
显然有\begin{equation*}
	\lim_{n\to\infty} \sqrt[n]{u_n}
	= \lim_{n\to\infty} \frac{1}{2} \sqrt[n]{2+(-1)^n}
	= \lim_{n\to\infty} \frac{1}{2} \exp{\frac{1}{n} \ln[2+(-1)^n]},
\end{equation*}
因为\(\ln[2+(-1)^n] \in \{ 0, \ln3 \}\)有界,
故\(\lim_{n\to\infty} \frac{1}{n} \ln[2+(-1)^n] = 0\),
从而\begin{equation*}
	\lim_{n\to\infty} \sqrt[n]{u_n} = \frac{1}{2} < 1.
\end{equation*}
根据\hyperref[theorem:无穷级数.正项级数的根值审敛法]{根值审敛法}可知,
级数\(\sum_{n=1}^\infty u_n\)收敛.
\end{solution}
\end{example}

\begin{example}
%@see: 《数学分析(第二版 下册)》(陈纪修) P20 例9.3.5
判定级数\(\sum_{n=1}^\infty \frac{n^3[\sqrt2 + (-1)^n]^n}{3^n}\)的敛散性.
\begin{solution}
由于\begin{equation*}
	\varlimsup_{n\to\infty} \sqrt[n]{
		\frac{n^3[\sqrt2 + (-1)^n]^n}{3^n}
	}
	= \frac{\sqrt2+1}3
	< 1,
\end{equation*}
由\hyperref[theorem:无穷级数.正项级数的根值审敛法]{根值审敛法}可知,
级数\(\sum_{n=1}^\infty \frac{n^3[\sqrt2 + (-1)^n]^n}{3^n}\)收敛.
\end{solution}
\end{example}

\begin{proposition}
%@see: 《数学分析(第2册)》(周民强) P165 定理9.12
设数列\(\{a_n\}\)满足\(a_n > 0\ (n=1,2,\dotsc)\),
记\begin{equation*}
	\rho \defeq \lim_{n\to\infty} \sqrt[\ln n]{a_n}.
\end{equation*}
\begin{itemize}
	\item 若\(\rho < 1/e\),则\(\sum_{n=1}^\infty a_n\)收敛.
	\item 若\(\rho > 1/e\),则\(\sum_{n=1}^\infty a_n\)发散.
\end{itemize}
%TODO proof
\end{proposition}

\subsection{比值审敛法与根值审敛法之间的联系}
\begin{theorem}\label{theorem:无穷级数.比值审敛法与根值审敛法之间的联系}
%@see: 《数学分析(第二版 下册)》(陈纪修) P20 引理9.3.1
设\(\{u_n\}\)是正项数列,
则\begin{equation*}
	\varliminf_{n\to\infty} \frac{u_{n+1}}{u_n}
	\leq
	\varliminf_{n\to\infty} \sqrt[n]{u_n}
	\leq
	\varlimsup_{n\to\infty} \sqrt[n]{u_n}
	\leq
	\varlimsup_{n\to\infty} \frac{u_{n+1}}{u_n}.
\end{equation*}
\end{theorem}
\begin{remark}
\cref{theorem:无穷级数.比值审敛法与根值审敛法之间的联系} 说明:
如果一个正项级数的敛散情况
可以利用\hyperref[theorem:无穷级数.正项级数的比值审敛法]{比值审敛法}判定,
那么它一定也能用\hyperref[theorem:无穷级数.正项级数的根值审敛法]{根值审敛法}判定.
但是,能用\hyperref[theorem:无穷级数.正项级数的根值审敛法]{根值审敛法}判定,
却未必能用\hyperref[theorem:无穷级数.正项级数的比值审敛法]{比值审敛法}判定.
\end{remark}

\begin{example}
%@see: 《数学分析(第二版 下册)》(陈纪修) P21 例9.3.7
考虑级数\begin{equation*}
	\sum_{n=1}^\infty x_n
	= \frac12 + \frac13
	+ \frac1{2^2} + \frac1{3^2}
	+ \frac1{2^3} + \frac1{3^3}
	+ \dotsb,
\end{equation*}
则\begin{align*}
	\varlimsup_{n\to\infty} \sqrt[n]{x_n}
	&= \lim_{n\to\infty} \sqrt[2n-1]{\frac1{2^n}}
	= \frac1{\sqrt2}, \\
	\varlimsup_{n\to\infty} \frac{x_{n+1}}{x_n}
	&= \lim_{n\to\infty} \frac{3^n}{2^{n+1}}
	= +\infty, \\
	\varliminf_{n\to\infty} \frac{x_{n+1}}{x_n}
	&= \lim_{n\to\infty} \frac{2^n}{3^n}
	= 0.
\end{align*}
可以看出,
由\hyperref[theorem:无穷级数.正项级数的根值审敛法]{根值审敛法}可知
级数\(\sum_{n=1}^\infty x_n\)收敛,
但\hyperref[theorem:无穷级数.正项级数的比值审敛法]{比值审敛法}却失效了.
于是我们说\hyperref[theorem:无穷级数.正项级数的根值审敛法]{根值审敛法}的适用范围
比\hyperref[theorem:无穷级数.正项级数的比值审敛法]{比值审敛法}更广泛.
\end{example}

但是,对于某些具体例子而言,虽然根值审敛法和比值审敛法这两种审敛法都适用,
但是比值审敛法比根值审敛法更方便一些,因此我们需要根据级数具体情况来选择合适的审敛法.

比值审敛法和根值审敛法的本质是比较判别法,
用来比较的基准级数是几何级数\(\sum_{n=1}^\infty q^n\):
在判断级数收敛时,要求级数的通项受到\(q^n\ (0<q<1)\)的控制;
而在判断级数发散时,则是依据其一般项不趋于\(0\).
由于这两者相去甚远,因此这两种审敛法在许多情况下会失效,
即便是对\(\sum_{n=1}^\infty \frac1{n^p}\)这样简单的级数,
它们也都无能为力.

\subsection{对数审敛法}
\begin{theorem}\label{theorem:无穷级数.正项级数的对数审敛法}
设\(\sum_{n=1}^\infty u_n\)是正项级数,
记\begin{equation*}
	\rho \defeq \frac{\ln(1/u_n)}{\ln n}.
\end{equation*}
\begin{itemize}
	\item 如果\(\rho>1\),则级数\(\sum_{n=1}^\infty u_n\)收敛.
	\item 如果\(\rho\leq1\),则级数\(\sum_{n=1}^\infty u_n\)发散.
\end{itemize}
\begin{proof}
假设极限\(\rho = \frac{\ln(1/u_n)}{\ln n}\)存在且有限,
那么对于任意给定\(\epsilon>0\),存在正整数\(N\),当\(n>N\)时,有\begin{equation*}
	\abs{\frac{\ln(1/u_n)}{\ln n} - \rho} < \epsilon,
	\quad\text{或}\quad
	\rho - \epsilon < \frac{\ln(1/u_n)}{\ln n} < \rho + \epsilon.
\end{equation*}

当\(0<\rho<1\)时,只要\(\epsilon\)满足\(\rho+\epsilon<1\),
就成立\(u_n > n^{-(\rho+\epsilon)}\).
因为\(\sum_{n=1}^\infty \frac1{n^{\rho+\epsilon}}\)发散,
所以级数\(\sum_{n=1}^\infty u_n\)发散.

当\(\rho>1\)时,只要\(\epsilon\)满足\(\rho-\epsilon>1\),
就成立\(u_n < n^{-(\rho-\epsilon)}\).
因为\(\sum_{n=1}^\infty \frac1{n^{\rho-\epsilon}}\)收敛,
所以级数\(\sum_{n=1}^\infty u_n\)收敛.
\end{proof}
\end{theorem}

\begin{example}
证明:级数\(\sum_{n=2}^\infty \frac1{\ln^{\ln n} n}\)收敛.
\begin{proof}
\begin{proof}[证法一]
易知:存在正整数\(N\),当\(n \geq N\)时,有\(\ln n \geq 3\),
从而有\begin{equation*}
	\ln^{\ln n} n
	\geq 3^{\ln n}
	= n^{\ln3},%\cref{equation:函数.真底互换公式}
	\quad\text{即}\quad
	\frac1{\ln^{\ln n} n} \leq \frac1{n^{\ln3}},
\end{equation*}
而\(\ln3>1\),
故\(\sum_{n=N}^\infty \frac1{n^{\ln3}}\)收敛,
\(\sum_{n=N}^\infty \frac1{\ln^{\ln n} n}\)收敛,
\(\sum_{n=2}^\infty \frac1{\ln^{\ln n} n}\)收敛.
\end{proof}
\begin{proof}[证法二]
因为\begin{equation*}
	\lim_{n\to\infty} \frac{\ln(\ln^{\ln n} n)}{\ln n}
	= \lim_{n\to\infty} \ln n
	= +\infty,
\end{equation*}
所以由\cref{theorem:无穷级数.正项级数的对数审敛法}可知,
\(\sum_{n=2}^\infty \frac1{\ln^{\ln n} n}\)收敛.
\end{proof}\let\qed\relax
\end{proof}
\end{example}

\begin{remark}
设级数\(\sum_{n=1}^\infty u_n\)的一般项是\(u_n = \frac1{n\sqrt[n]{n}}\).
那么\(\lim_{n\to\infty} \frac{\ln(1/u_n)}{\ln n} = 1\),
因此,我们无法利用\cref{theorem:无穷级数.正项级数的对数审敛法} 判别这个级数的敛散性.
\end{remark}

\subsection{极限审敛法}
将\(p\)级数作为基准级数,将所给正项级数与之作比较,
可得在实用上较方便的极限审敛法和拉贝审敛法.
\begin{theorem}[极限审敛法]\label{theorem:无穷级数.正项级数的极限审敛法}
%@see: 《高等数学(第六版 下册)》 P261 定理6
设\(\sum_{n=1}^\infty u_n\)为正项级数,
记\(\rho \defeq \lim_{n\to\infty} n^p u_n\).
\begin{itemize}
	\item 当\(\rho\in[0,+\infty)\)且\(p>1\)时,
	级数\(\sum_{n=1}^\infty u_n\)收敛.

	\item 当\(\rho\in(0,+\infty]\)且\(p\leq1\)时,
	级数\(\sum_{n=1}^\infty u_n\)发散.
\end{itemize}
\end{theorem}

\begin{example}
%@see: 《高等数学(第六版 下册)》 P261 例7
判定级数\(\sum_{n=1}^\infty \ln(1+\frac{1}{n^2})\)的收敛性.
\begin{solution}
因\(\ln(1+\frac{1}{n^2}) \sim \frac{1}{n^2}\ (n\to\infty)\),
故\begin{equation*}
	\lim_{n\to\infty} n^2 u_n
	= \lim_{n\to\infty} n^2 \ln(1+\frac{1}{n^2})
	= \lim_{n\to\infty} n^2 \cdot \frac{1}{n^2}
	= 1,
\end{equation*}
根据\hyperref[theorem:无穷级数.正项级数的极限审敛法]{极限审敛法}可知,所给级数收敛.
\end{solution}
\end{example}

\begin{example}
%@see: 《高等数学(第六版 下册)》 P261 例8
判定级数\(\sum_{n=1}^\infty \sqrt{n+1} \left(1-\cos\frac{\pi}{n}\right)\)的收敛性.
\begin{solution}
因为\(1 - \cos x \sim \frac{1}{2} x^2\ (x\to0)\),
故\begin{equation*}
	\lim_{n\to\infty} n^{\frac32} \sqrt{n+1} \cdot \left(1-\cos\frac{\pi}{n}\right)
	= \lim_{n\to\infty} n^2 \sqrt{\frac{n+1}{n}} \cdot \frac{1}{2} \left(\frac{\pi}{n}\right)^2
	= \frac{1}{2} \pi^2,
\end{equation*}
根据\hyperref[theorem:无穷级数.正项级数的极限审敛法]{极限审敛法}可知,所给级数收敛.
\end{solution}
\end{example}

\begin{example}
%@see: https://www.bilibili.com/video/BV1WQegeUEBc
举例说明:当正项级数\(\sum_{n=1}^\infty a_n\)收敛时,不成立\(\lim_{n\to\infty} n a_n = 0\).
\begin{solution}
取\begin{equation*}
	\def\arraystretch{1.5}
	a_n = \left\{ \begin{array}{cl}
		\frac1{n^2}, & \sqrt{n}\notin\mathbb{N}^+, \\
		\frac1n, & \sqrt{n}\in\mathbb{N}^+.
	\end{array} \right.
\end{equation*}
那么\begin{align*}
	\sum_{n=1}^\infty a_n
	&= \frac11 + \frac1{2^2} + \frac1{3^2}
		+ \frac14 + \frac1{5^2} + \frac1{6^2} + \frac1{7^2} + \frac1{8^2}
		+ \dotsb \\
	&= \left(\frac11 + \frac1{2^2} + \frac1{3^2}\right)
		+ \left(\frac14 + \frac1{5^2} + \frac1{6^2} + \frac1{7^2} + \frac1{8^2}\right)
		+ \dotsb \\
	&= \sum_{k=1}^\infty \left[
		\frac1{k^2}
		+ \sum_{k^2+1 \leq p \leq (k+1)^2-1} \frac1{p^2}
	\right].
\end{align*}
因为级数\(\sum_{k=1}^\infty \frac1{k^2}\)收敛,
所以要证\(\sum_{n=1}^\infty a_n\)收敛,
只需证\begin{equation*}
	\sum_{k=1}^\infty \sum_{k^2+1 \leq p \leq (k+1)^2-1} \frac1{p^2}
	\eqno(1)
\end{equation*}收敛.
因为级数(1)的一般项满足\begin{equation*}
	\sum_{k^2+1 \leq p \leq (k+1)^2-1} \frac1{p^2}
	\leq \left[(k+1)^2-k^2-1\right] \cdot \frac1{(k^2+1)^2}
	= \frac{2k}{(k^2+1)^2},
\end{equation*}
又因为\begin{equation*}
	\lim_{k\to\infty} k^2 \cdot \frac{2k}{(k^2+1)^2}
	= 0,
\end{equation*}
由\hyperref[theorem:无穷级数.正项级数的极限审敛法]{极限审敛法}可知,
级数\(\sum_{k=1}^\infty \frac{2k}{(k^2+1)^2}\)收敛,
再利用\hyperref[theorem:无穷级数.正项级数的比较审敛法]{比较审敛法}可知,
级数(1)收敛.
因此,级数\(\sum_{k=1}^\infty \frac1{k^2}\)收敛.

但是,数列\(\{a_n\}\)有两个子列极限:\begin{equation*}
	\varliminf_{n\to\infty} n a_n
	= \lim_{n\to\infty} n \cdot \frac1{n^2}
	= \lim_{n\to\infty} \frac1n
	= 0,
	\qquad
	\varlimsup_{n\to\infty} n a_n
	= \lim_{n\to\infty} n \cdot \frac1n
	= \lim_{n\to\infty} 1
	= 1.
\end{equation*}
而\hyperref[theorem:子列极限.具有不同子列极限的数列必定发散]{具有不同子列极限的数列必定发散},
于是得到\(\lim_{n\to\infty} n a_n\)不存在.
\end{solution}
\end{example}

\subsection{拉贝审敛法}
对于某些正项级数\(\sum_{n=1}^\infty u_n\),
成立\(\lim_{n\to\infty} \frac{x_{n+1}}{x_n} = 1\),
这时\hyperref[theorem:无穷级数.正项级数的根值审敛法]{根值审敛法}%
和\hyperref[theorem:无穷级数.正项级数的比值审敛法]{比值审敛法}都失效了,
下面给出一种针对这类情况的判别法.

\begin{lemma}\label{theorem:常数项级数的审敛法.拉贝审敛法的引理}
%@see: 《数学分析简明教程(第二版 下册)》(邓东皋、尹小玲) P17 引理
对于任意\(r>p>1\),存在正整数\(N\),当\(n>N\)时,有\begin{equation*}
	1+\frac{r}{n} > \left(1+\frac1n\right)^p.
\end{equation*}
\begin{proof}
因为\begin{equation*}
	\lim_{x\to0} \frac{(1+x)^p-1}x = p,
\end{equation*}
所以\begin{equation*}
	\lim_{n\to\infty} \frac{\left(1+1/n\right)^p-1}{1/n} = p.
\end{equation*}
又因为\(p<r\),
由\hyperref[theorem:极限.收敛数列的保序性1]{收敛数列的保序性} 可知,
存在正整数\(N\),当\(n>N\)时,有\begin{equation*}
	\frac{\left(1+1/n\right)^p-1}{1/n} < r
	\quad\text{或}\quad
	\left(1+\frac1n\right)^p-1
	< \frac{r}{n}.
	\qedhere
\end{equation*}
\end{proof}
\end{lemma}

\begin{theorem}[拉贝审敛法]
%@see: 《数学分析(第二版 下册)》(陈纪修) P22 定理9.3.5(Raabe判别法)
%@see: 《数学分析简明教程(第二版 下册)》(邓东皋、尹小玲) P17 定理10.11(拉阿比(Raabe)判别法)
%@see: 《数学分析(第2册)》(周民强) P164 定理9.11
%@see: 《数学分析:原理与方法》(胡适耕) P293 15.1.10推论 (i)
%@see: https://mathworld.wolfram.com/RaabesTest.html
设\(u_n > 0\),记\begin{equation*}
	\rho \defeq \lim_{n\to\infty} n \left(\frac{u_n}{u_{n+1}} - 1\right).
\end{equation*}
\begin{itemize}
	\item 当\(\rho>1\)时,级数\(\sum_{n=1}^\infty u_n\)收敛.
	\item 当\(\rho<1\)时,级数\(\sum_{n=1}^\infty u_n\)发散.
	\item 当\(\rho=1\)时,级数\(\sum_{n=1}^\infty u_n\)可能收敛也可能发散.
\end{itemize}
\begin{proof}
假设\(\rho>1\).
记\(r \defeq \rho-\epsilon_0\).
那么存在\(\epsilon_0>0\),
使得\(\rho>r=\rho-\epsilon_0>1\).
显然存在正整数\(N_1\),
使得当\(n>N_1\)时,
有\begin{equation*}
	n \left(\frac{u_n}{u_{n+1}}-1\right) > r,
	\quad\text{即}\quad
	\frac{u_n}{u_{n+1}} \geq 1+\frac{r}{n}.
\end{equation*}
取\(p\in(1,r)\),由\cref{theorem:常数项级数的审敛法.拉贝审敛法的引理} 可知,
存在正整数\(N_2\),当\(n>N_2\)时,有\begin{equation*}
	1+\frac{r}{n} \geq \left(1+\frac1n\right)^p.
\end{equation*}
故当\(n>N=\max\{N_1,N_2\}\)时,有\begin{equation*}
	\frac{u_n}{u_{n+1}}
	\geq 1+\frac{r}{n}
	\geq \left(1+\frac1n\right)^p
	= \frac{1/n^p}{1/(n+1)^p}.
\end{equation*}
根据\cref{theorem:无穷级数.比较审敛法的比值形式},
由于级数\(\sum_{n=1}^\infty \frac1{n^p}\ (p>1)\)收敛,
所以级数\(\sum_{n=1}^\infty u_n\)也收敛.

假设\(\rho<1\).
显然存在\(\epsilon_1>0\),
使得\(\rho+\epsilon_1<1\).
这时,存在正整数\(N\),当\(n>N\)时,有\begin{equation*}
	n \left(\frac{u_n}{u_{n+1}}-1\right)
	< \rho+\epsilon_1
	< 1,
\end{equation*}
从而有\begin{equation*}
	\frac{u_n}{u_{n+1}}
	\leq 1+\frac1n
	= \frac{1/n}{1/(n+1)}.
\end{equation*}
根据\cref{theorem:无穷级数.比较审敛法的比值形式},
由于级数\(\sum_{n=1}^\infty \frac1n\)发散,
所以级数\(\sum_{n=1}^\infty u_n\)也发散.
\end{proof}
\end{theorem}

\subsection{积分审敛法}
\begin{theorem}[积分审敛法]\label{theorem:无穷级数.积分审敛法}
%@see: 《数学分析(第二版 下册)》(陈纪修) P24 定理9.3.6(积分判别法)
%@see: 《数学分析教程 (第3版 下册)》(史济怀) P166 定理14.2.4(Cauchy积分判别法)
%@see: 《数学分析(第2册)》(周民强) P166 定理9.13
%@see: 《数学分析简明教程(第二版 下册)》(邓东皋、尹小玲) P19 定理10.12(柯西积分判别法)
%@see: https://mathworld.wolfram.com/IntegralTest.html
设非负函数\(f\colon[a,+\infty)\to\mathbb{R}\)在任意有限闭区间上黎曼可积,
取一个单调增加趋于\(+\infty\)的数列\(\{a_n\}\),
使得\begin{equation*}
	a_1 = a, \qquad
	u_n = \int_{a_n}^{a_{n+1}} f(x) \dd{x}.
\end{equation*}
反常积分\(\int_a^{+\infty} f(x) \dd{x}\)
与正项级数\(\sum_{n=1}^\infty u_n\)同时收敛或同时发散于\(+\infty\),
且\begin{equation*}
	\int_a^{+\infty} f(x) \dd{x}
	= \sum_{n=1}^\infty u_n
	= \sum_{n=1}^\infty \int_{a_n}^{a_{n+1}} f(x) \dd{x}.
\end{equation*}

特别地,若\(f\)单调减少,取\(a_n = n\),
则反常积分\(\int_a^{+\infty} f(x) \dd{x}\)
与正项级数\(\sum_{n=N}^\infty f(n)\)(其中\(N=\floor{a}+1\))
同时收敛或同时发散.
\begin{proof}
设正项级数\(\sum_{n=1}^\infty u_n\)的部分和数列为\(\{S_n\}\),
则对任意实数\(A>a\),
存在正整数\(n\),
成立\(a_n \leq A < a_{n+1}\),
于是\begin{equation*}
	S_{n-1} \leq \int_a^A f(x) \dd{x} \leq S_n.
\end{equation*}

当\(\{S_n\}\)有界,即\(\sum_{n=1}^\infty u_n\)收敛时,
有\(\lim_{A\to+\infty} \int_a^A f(x) \dd{x}\)收敛,
且根据夹逼准则,它们收敛于相同的极限.
当\(\{S_n\}\)无界,即\(\sum_{n=1}^\infty u_n\)发散于\(+\infty\)时,
则同样有\(\lim_{A\to+\infty} \int_a^A f(x) \dd{x} = +\infty\).
由此可得\begin{equation*}
	\int_a^{+\infty} f(x) \dd{x}
	= \sum_{n=1}^\infty u_n
	= \sum_{n=1}^\infty \int_{a_n}^{a_{n+1}} f(x) \dd{x}.
\end{equation*}

特别地,若\(f\)单调减少,取\(a_n = n\),
则当\(n \geq N = \floor{a}+1\)时,有\begin{equation*}
	f(n+1) \leq u_n = \int_n^{n+1} f(x) \dd{x} \leq f(n),
\end{equation*}
由比较审敛法可知\(\sum_{n=N}^\infty f(n)\)与\(\sum_{n=N}^\infty u_n\)同时收敛或同时发散,
从而与\(\int_a^{+\infty} f(x) \dd{x}\)同时收敛或同时发散.
\end{proof}
\end{theorem}

\begin{example}\label{example:无穷级数.对数负次幂级数}
%@see: 《数学分析(第二版 下册)》(陈纪修) P25 例9.3.9
%@see: 《数学分析简明教程(第二版 下册)》(邓东皋、尹小玲) P21 例8
证明:正项级数\(\sum_{n=2}^\infty \frac1{n \ln^q n}\)在\(q>1\)时收敛,在\(q\leq1\)时发散.
\begin{proof}
取\(f(x) = \frac1{x \ln^q x}\),
则在\([2,+\infty)\)上,
函数\(f\)单调减少且恒为正,
另外\begin{equation*}
	\sum_{n=2}^\infty f(n)
	= \sum_{n=2}^\infty \frac1{n \ln^q n}.
\end{equation*}
由\begin{equation*}
	\int_2^A f(x) \dd{x}
	= \left\{ \begin{array}{ll}
		\frac1{1-q} \ln^{1-q} A - \frac1{1-q} \ln^{1-q}2, & q\neq1, \\
		\ln\ln A - \ln\ln 2, & q=1,
	\end{array} \right.
\end{equation*}
令\(A\to+\infty\),可知
积分\(\int_2^{+\infty} f(x) \dd{x}\)在\(q>1\)时收敛,在\(q\leq1\)时发散,
由此得到\(\sum_{n=2}^\infty \frac1{n \ln^q n}\)在\(q>1\)时收敛,在\(q\leq1\)时发散.
\end{proof}
\end{example}

\begin{remark}
在应用\cref{theorem:无穷级数.积分审敛法} 时,
必须注意条件\(f(x)\geq0\).
如果缺少这一个条件,
虽然从反常积分\(\int_a^{+\infty} f(x) \dd{x}\)的收敛性,
仍然可以得到级数\(\sum_{n=1}^\infty u_n\)的收敛性.
但是反过来,结论就不一定成立.
例如,取\(f(x) = \sin x\),
显然\(\int_0^{+\infty} f(x) \dd{x}\)是发散的,
但是若取\(a_n = 2n\pi\),
则\(u_n = \int_{a_n}^{a_{n+1}} f(x) \dd{x} = 0\),
也就是说级数\(\sum_{n=1}^\infty u_n\)居然是收敛的.
\end{remark}

\begin{example}
证明:级数\(\sum_{n=1}^\infty \frac1{n \ln(n+1)}\)发散.
\begin{proof}
%@credit: {6f21d9e6-edca-4b6f-9dff-364f3d62dcce},{4dc9721f-7c76-4700-adc0-75d458e212ba}
由\cref{example:无穷级数.对数负次幂级数} 可知,
级数\(\sum_{n=2}^\infty \frac1{n \ln n}\)发散.
又因为\begin{equation*}
	\frac1{\ln n} \sim \frac1{\ln(n+1)}
	\quad(n\to\infty),
\end{equation*}
所以\(\sum_{n=1}^\infty \frac1{n \ln(n+1)}\)发散.
\end{proof}
\end{example}

\begin{example}
%@see: 《数学分析:原理与方法》(胡适耕) P291
证明:级数\(\sum_{n=3}^\infty \frac1{n \ln^p n (\ln\ln n)^q}\)
在\(p>1\)或\(p=1<q\)时收敛.
%TODO proof
\end{example}

\begin{example}
%@see: 《2019年全国硕士研究生入学统一考试(数学一)》一选择题/第3题/选项(A)
设\(\{u_n\}\)是单调增加的有界数列.
举例说明:级数\(\sum_{n=1}^\infty \frac{u_n}{n}\)发散.
\begin{solution}
取\begin{equation*}
	u_n = -\frac1{\ln(n+1)}
	\quad(n=1,2,\dotsc).
\end{equation*}
显然\(u_n<0\)
且\begin{align*}
	u_{n+1} - u_n
	&= \frac1{\ln(n+1)} - \frac1{\ln(n+2)}
	= \frac{\ln(n+2)-\ln(n+1)}{\ln(n+1) \ln(n+2)} \\
	&= \frac1{\ln(n+1) \ln(n+2)} \ln\left(1+\frac1{n+1}\right)
	> 0,
\end{align*}
即\(u_{n+1} > u_n\).
这就说明\(\{u_n\}\)是单调增加的有界数列.
又因为\begin{equation*}
	\frac{u_n}{n}
	= -\frac1{n \ln(n+1)},
\end{equation*}
所以级数\(\sum_{n=1}^\infty \frac{u_n}{n}\)发散.
\end{solution}
\end{example}

\begin{example}
%@see: 《2019年全国硕士研究生入学统一考试(数学一)》一选择题/第3题/选项(B)
设\(\{u_n\}\)是单调增加的有界数列.
举例说明:级数\(\sum_{n=1}^\infty \frac{(-1)^n}{u_n}\)发散.
\begin{solution}
取\begin{equation*}
	u_n = -\frac1n
	\quad(n=1,2,\dotsc),
\end{equation*}
易见\(u_1 \leq u_n < u_{n+1} < 0\ (n=1,2,\dotsc)\),
\(\{u_n\}\)是单调增加的有界数列,
而级数\begin{equation*}
	\sum_{n=1}^\infty \frac{(-1)^n}{u_n}
	= \sum_{n=1}^\infty (-1)^{n+1} n
\end{equation*}发散.
\end{solution}
\end{example}

\subsection{其他审敛法}
\begin{theorem}
%@see: 《数学分析:原理与方法》(胡适耕) P292 15.1.9定理
设\(a_n>0\),且\begin{equation*}
	\frac{a_{n+1}}{a_n}
	= r - \frac{p}{n} - \frac{q}{n \ln n} + o\left(\frac1{n \ln n}\right).
\end{equation*}
\begin{itemize}
	\item 若\(r<1\),或\(r=1<p\),或\(r=p=1<q\),则级数\(\sum a_n\)收敛.
	\item 若\(r>1\),或\(r=1>p\),或\(r=p=1>q\),则级数\(\sum a_n\)发散.
\end{itemize}
%TODO proof
\end{theorem}

\begin{corollary}[贝特朗审敛法]
%@see: 《数学分析:原理与方法》(胡适耕) P293 15.1.10推论 (ii)
%@see: https://mathworld.wolfram.com/BertrandsTest.html
设\(a_n>0\),记\begin{equation*}
	\rho
	\defeq
	\lim_{n\to\infty} \ln n \left[
		n \left( \frac{a_n}{a_{n+1}} - 1 \right) - 1
	\right].
\end{equation*}
\begin{itemize}
	\item 当\(\rho>1\)时,级数\(\sum a_n\)收敛.
	\item 当\(\rho<1\)时,级数\(\sum a_n\)发散.
\end{itemize}
%TODO proof
\end{corollary}

% \begin{corollary}[高斯审敛法]
% %@see: 《数学分析:原理与方法》(胡适耕) P293 15.1.10推论 (iii)
% %@see: https://mathworld.wolfram.com/GausssTest.html
% %TODO proof
% \end{corollary}

%@see: https://mathworld.wolfram.com/KummersTest.html

%@see: https://mathworld.wolfram.com/ErmakoffsTest.html

\section{交错级数及其审敛法}
\subsection{莱布尼茨定理}
\begin{definition}
%@see: 《数学分析(第二版 下册)》(陈纪修) P29 定义9.4.1
如果级数\(\sum_{n=1}^\infty u_n\)满足\begin{equation*}
	u_n = (-1)^{n+1} \abs{u_n},
	\quad\text{或}\quad
	u_n = (-1)^n \abs{u_n},
\end{equation*}
则称“级数\(\sum_{n=1}^\infty u_n\)是\DefineConcept{交错级数}(alternating series)”.
\end{definition}

从定义可以看出:
交错级数的各项是正负交错的,从而可以写成下面的形式:\begin{equation*}
	u_1 - u_2 + u_3 - u_4 + \dotsb,
\end{equation*}或\begin{equation*}
	-u_1 + u_2 - u_3 + u_4 - \dotsb,
\end{equation*}
其中\(u_1,u_2,\dotsc\)都是正数.

\begin{definition}
%@see: 《数学分析(第二版 下册)》(陈纪修) P29 定义9.4.1
设交错级数\(\sum_{n=1}^\infty u_n\)满足
\(\{\abs{u_n}\}\)单调减少且收敛于0,
则称“级数\(\sum_{n=1}^\infty u_n\)是\DefineConcept{莱布尼茨级数}”.
\end{definition}

\begin{theorem}[莱布尼茨审敛法]\label{theorem:无穷级数.莱布尼茨定理}
%@see: 《数学分析(第二版 下册)》(陈纪修) P30 定理9.4.2(Leibniz判别法)
%@see: 《数学分析教程(第3版 下册)》(史济怀) P181 定理14.4.2(Leibniz判别法)
莱布尼茨级数\(\sum_{n=1}^\infty u_n\)收敛,
它的和\(s\)满足\(\abs{s} \leq \abs{u_1}\),
它的余项\(r_n\)满足\(\abs{r_n} \leq \abs{u_{n+1}}\).
%TODO proof
\end{theorem}

\begin{example}\label{example:无穷级数.交错级数1}
交错级数\begin{equation*}
	1 - \frac{1}{2} + \frac{1}{3} - \frac{1}{4} + \dotsb + (-1)^{n-1} \frac{1}{n} + \dotsb
\end{equation*}
满足条件\begin{itemize}
	\item \(u_n = \frac{1}{n} > \frac{1}{n+1} = u_{n+1}\ (n=1,2,\dotsc)\);
	\item \(\lim_{n\to\infty} u_n = \lim_{n\to\infty} \frac{1}{n} = 0\),
\end{itemize}
所以它是收敛的,且其和\(s < 1\).
如果取前\(n\)项的和\begin{equation*}
	s_n = 1 - \frac{1}{2} + \frac{1}{3} - \dotsb + (-1)^{n-1} \frac{1}{n}
\end{equation*}作为\(s\)的近似值,
所产生的误差\(\abs{r_n} \leq \frac{1}{n+1}\).
\end{example}
\begin{remark}
将\cref{example:无穷级数.交错级数1}
与\cref{example:无穷级数.调和级数的敛散性} 进行比较可以得出结论:
即便级数\(\sum_{n=1}^\infty u_n\)收敛,
级数\(\sum_{n=1}^\infty (-1)^{n+1} u_n\)也可能发散.
\end{remark}

\begin{example}\label{example:无穷级数.交错级数2}
交错级数\begin{equation*}
	\frac{1}{2} - \frac{1}{3}
	+ \frac{1}{2^2} - \frac{1}{3^2}
	+ \dotsm + \frac{1}{2^n} - \frac{1}{3^n}
\end{equation*}是收敛的,
这是因为它的前\(2n\)项和\begin{align*}
	s_{2n} &= \frac{1}{2} - \frac{1}{3}
	+ \frac{1}{2^2} - \frac{1}{3^2}
	+ \dotsm + \frac{1}{2^n} - \frac{1}{3^n} \\
	&= \left(\frac{1}{2} + \frac{1}{2^2} + \dotsm + \frac{1}{2^n}\right)
	- \left(\frac{1}{3} + \frac{1}{3^2} + \dotsm + \frac{1}{3^n}\right) \\
	&= \left(1 - \frac{1}{2^n}\right)
	- \left(\frac{1}{2} - \frac{1}{2\cdot3^n}\right),
\end{align*}
从而\begin{equation*}
	\lim_{n\to\infty} s_{2n} = \frac{1}{2}.
\end{equation*}
但是这级数在\(n>1\)时总有\begin{equation*}
	u_{2n} = \frac{1}{3^n} < \frac{1}{2^{n+1}} = u_{2n+1},
\end{equation*}不符合\cref{theorem:无穷级数.莱布尼茨定理} 的条件.
\end{example}
\begin{remark}
\cref{example:无穷级数.交错级数2} 说明:
\cref{theorem:无穷级数.莱布尼茨定理} 是判断交错级数收敛性的一个充分不必要定理.
\end{remark}

\begin{example}
设数列\(\{u_n\}\)满足\(u_n = (-1)^n \frac1{\ln(n+1)}\ (n=1,2,\dotsc)\).
因为\(\ln\)是\((0,+\infty)\)上的严格单调增加函数,
对于\(n=1,2,\dotsc\)总有\(\ln(n+1) < \ln(n+2)\),
继而有\(\frac1{\ln(n+1)} > \frac1{\ln(n+2)}\),
且有\(\lim_{n\to\infty} \frac1{\ln(n+1)} = 0\),
所以交错级数\(\sum_{n=1}^\infty u_n\)是莱布尼茨级数.
但是\(\sum_{n=1}^\infty (-1)^n \frac{u_n}{n}
= \sum_{n=1}^\infty \frac1{n \ln(n+1)}\)发散.
\end{example}

\begin{example}
设级数\(\sum_{n=1}^\infty u_n\)收敛.
举例说明:\(\sum_{n=1}^\infty u_n u_{n+1}\)发散.
\begin{solution}
取\begin{equation*}
	u_n = \frac{(-1)^n}{\sqrt{n}}
	\quad(n=1,2,\dotsc),
\end{equation*}
则\begin{equation*}
	u_n u_{n+1}
	= -\frac1{\sqrt{n(n+1)}},
\end{equation*}
显然\(\sum_{n=1}^\infty u_n u_{n+1}\)发散.
%@see: 《高等数学(第六版 下册)》 P257 例2
\end{solution}
\end{example}

\begin{example}
设级数\(\sum_{n=1}^\infty u_n\)收敛.
举例说明:级数\(\sum_{n=1}^\infty (u_{2n-1} - u_{2n})\)发散.
\begin{solution}
取\begin{equation*}
	u_n = (-1)^{n+1} \frac1n
	\quad(n=1,2,\dotsc),
\end{equation*}
则\begin{equation*}
	u_{2n-1} - u_{2n}
	= (-1)^{2n} \frac1{2n-1} - (-1)^{2n+1} \frac1{2n}
	= \frac1{2n-1} + \frac1{2n},
\end{equation*}
显然\(\sum_{n=1}^\infty (u_{2n-1} - u_{2n})\)是调和级数,自然是发散的.
\end{solution}
\end{example}

\begin{example}
设级数\(\sum_{n=1}^\infty u_n\)收敛.
举例说明:级数\(\sum_{k=1}^\infty u_{2k-1}\)和\(\sum_{k=1}^\infty u_{2k}\)发散.
\begin{solution}
取\begin{equation*}
	u_n = (-1)^{n+1} \frac1n
	\quad(n=1,2,\dotsc),
\end{equation*}
则级数\begin{equation*}
	\sum_{k=1}^\infty u_{2k}
	= \sum_{k=1}^\infty \frac1{2k}
\end{equation*}显然发散.
因为\begin{equation*}
	\frac1{2k-1} > \frac1{2k},
	\quad k=1,2,\dotsc,
\end{equation*}
所以级数\begin{equation*}
	\sum_{k=1}^\infty u_{2k-1}
	= -\sum_{k=1}^\infty \frac1{2k-1}
\end{equation*}也发散.
\end{solution}
\end{example}

\subsection{由两个级数的一般项之积构成的级数的敛散性}
\begin{example}\label{example:交错级数.逐项平方以后发散的特例}
设\(u_n = \frac{(-1)^n}{\sqrt{n}}\).
试讨论级数\(\sum_{n=1}^\infty u_n\)和\(\sum_{n=1}^\infty u_n^2\)的敛散性.
\begin{proof}
因为\(u_n u_{n+1} < 0\),
\(\abs{u_n} > \abs{u_{n+1}}\),
\(\lim_{n\to\infty} \abs{u_n} = 0\),
那么根据\hyperref[theorem:无穷级数.莱布尼茨定理]{莱布尼茨定理}可知
\(\sum_{n=1}^\infty u_n\)收敛.
但是\(\sum_{n=1}^\infty u_n^2
= \sum_{n=1}^\infty \frac{1}{n}\)是调和级数,发散.
\end{proof}
\end{example}

从\cref{example:无穷级数.调和级数的敛散性,example:无穷级数.zeta2的敛散性} 可以看出,
虽然调和级数\(\sum_{n=1}^\infty \frac1n\)是发散的,
但是由它的一般项的平方构成的级数\(\sum_{n=1}^\infty \frac1{n^2}\)是收敛的,
并且由\cref{example:无穷级数.p级数的收敛性} 可知
\(\sum_{n=1}^\infty \frac1{\sqrt{n}}\)是发散的,
而\(\sum_{n=1}^\infty \frac1{n^3}\)
和\(\sum_{n=1}^\infty \frac1{n^4}\)是收敛的.
这就说明,由两个发散级数的一般项之积构成的级数可能是收敛的也可能是发散的,
由一个收敛级数与一个发散级数的一般项之积构成的级数可能是收敛的,
由两个收敛级数的一般项之积构成的级数可能是收敛的.

从\cref{example:交错级数.逐项平方以后发散的特例} 可以看出,
虽然级数\(\sum_{n=1}^\infty \frac{(-1)^n}{\sqrt{n}}\)是收敛的,
但是由它的一般项的平方构成的级数是调和级数,是发散的.
这就说明,由两个收敛级数的一般项之积构成的级数可能是发散的.

由\cref{example:无穷级数.等比级数的收敛性} 可知,
级数\(\sum_{n=1}^\infty 2^n\)是发散的,
又因为\(\lim_{n\to\infty} \frac{2^n}{n^2} = +\infty\),
所以\(\sum_{n=1}^\infty \frac{2^n}{n^2}\)是发散的.
这就说明,由一个收敛级数与一个发散级数的一般项之积构成的级数可能是发散的.

于是我们可以得出以下结论:
\begin{proposition}
对于\(\sum_{n=1}^\infty u_n\)和\(\sum_{n=1}^\infty v_n\)这两个级数,
不论它们是都收敛,还是都发散,还是一个收敛一个发散,
将它们的一般项对应相乘所得的级数\(\sum_{n=1}^\infty u_n v_n\)既可能收敛,也可能发散.
\end{proposition}

\section{阿贝尔--狄利克雷审敛法}
\begin{lemma}[阿贝尔变换]
%@see: 《数学分析(第二版 下册)》(陈纪修) P31 引理9.4.1(Abel变换)
%@see: 《数学分析教程(第3版 下册)》(史济怀) P182 引理14.4.1(Abel分部求和公式)
设\(\{a_n\},\{b_n\}\)是两个数列,
记\(B_n \defeq \sum_{k=1}^n b_k\ (n=1,2,\dotsc)\),
则\begin{equation}\label{equation:无穷级数.阿贝尔的分部求和公式}
	\sum_{k=1}^p a_k b_k
	= a_p B_p - \sum_{k=1}^{p-1} (a_{k+1} - a_k) B_k.
\end{equation}
\begin{proof}
直接计算得\begin{align*}
	\sum_{k=1}^p a_k b_k
	&= a_1 B_1 + \sum_{k=2}^p a_k (B_k - B_{k-1}) \\
	&= a_1 B_1 + \sum_{k=2}^p a_k B_k - \sum_{k=2}^p a_k B_{k-1} \\
	&= \sum_{k=1}^{p-1} a_k B_k - \sum_{k=1}^{p-1} a_{k+1} B_k + a_p B_p \\
	&= a_p B_p - \sum_{k=1}^{p-1} (a_{k+1} - a_k) B_k.
	\qedhere
\end{align*}
\end{proof}
\end{lemma}
我们把\cref{equation:无穷级数.阿贝尔的分部求和公式} 称为\DefineConcept{分部求和公式}.
事实上,阿贝尔变换就是离散形式的分部积分公式.
我们只要把数列的通项类比于被积函数
(\(a_n\)对应于\(u(x)\),\(b_n\)对应于\(v(x)\)),
求和类比于求积分,
求差类比于求微分
(\(a_{k+1} - a_k\)对应于函数的微分\(\dd{[u(x)]}\)),
那么两条公式就是一致的.

\begin{example}
%@see: https://www.bilibili.com/video/BV1JW4y1U7NF/
利用阿贝尔变换证明\cref{example:无穷级数.与调和数列有关的级数1} 中的级数收敛并求和.
\begin{solution}
记\(a_n = \frac1{n+1},
b_n = \frac1n,
B_n = \sum_{k=1}^n \frac1k\),
那么由\cref{equation:无穷级数.阿贝尔的分部求和公式} 有\begin{align*}
	&\sum_{n=1}^m \frac1{(n+1)(n+2)} \sum_{k=1}^n \frac1k
	= \sum_{n=1}^m \left( a_n - a_{n+1} \right) B_n
	= \sum_{n=1}^{m+1} a_n b_n - a_{m+1} B_{m+1} \\
	&= \sum_{n=1}^{m+1} \frac1{n+1} \cdot \frac1n - \frac1{m+2} \sum_{k=1}^{m+1} \frac1k
	= 1 - \frac1{m+2} - \frac1{m+2} \sum_{k=1}^{m+1} \frac1k.
\end{align*}
令\(m\to\infty\),
由\(\frac1{m+2} \to 0,
\frac1{m+2} \sum_{k=1}^{m+1} \frac1k \to 0\)可知
\(\sum_{n=1}^m \frac1{(n+1)(n+2)} \sum_{k=1}^n \frac1k \to 1\).
\end{solution}
\end{example}

\begin{lemma}[阿贝尔引理]\label{theorem:无穷级数.阿贝尔引理}
%@see: 《数学分析(第二版 下册)》(陈纪修) P32 引理9.4.2(Abel引理)
%@see: 《数学分析教程(第3版 下册)》(史济怀) P182 引理14.4.2(Abel引理)
设\begin{itemize}
	\item \(\{a_n\}\)是单调数列,
	\item 数列\(\{b_n\}\)的部分和数列\(\{B_n\}\)
	(其中\(B_n = \sum_{k=1}^n b_k\ (n=1,2,\dotsc)\))有界,
	即存在\(M>0\),成立\(\abs{B_n} \leq M\ (n=1,2,\dotsc)\),
\end{itemize}
则\begin{equation*}
	\abs{\sum_{k=1}^p a_k b_k}
	\leq M(\abs{a_1} + 2 \abs{a_p}).
\end{equation*}
\begin{proof}
由\hyperref[equation:无穷级数.阿贝尔的分部求和公式]{阿贝尔变换}得\begin{align*}
	\abs{\sum_{k=1}^p a_k b_k}
	&\leq \abs{a_p} \abs{B_p} + \sum_{k=1}^{p-1} \abs{a_{k+1} - a_k} \abs{B_k} \\
	&\leq M \left(
		\abs{a_p}
		+ \sum_{k=1}^{p-1} \abs{a_{k+1} - a_k}
	\right).
\end{align*}
由于\(\{a_n\}\)单调,
所以\begin{equation*}
	\sum_{k=1}^{p-1} \abs{a_{k+1} - a_k}
	= \abs{\sum_{k=1}^{p-1} (a_{k+1} - a_k)}
	= \abs{a_p - a_1}
	\leq \abs{a_p} + \abs{a_1},
\end{equation*}
于是\begin{equation*}
	\abs{\sum_{k=1}^p a_k b_k}
	\leq M(\abs{a_1} + 2 \abs{a_p}).
	\qedhere
\end{equation*}
\end{proof}
\end{lemma}

\begin{theorem}\label{theorem:无穷级数.无穷级数的阿贝尔--狄利克雷审敛法}
%@see: 《数学分析(第二版 下册)》(陈纪修) P33 定理9.4.3(级数的A-D判别法)
%@see: 《数学分析教程(第3版 下册)》(史济怀) P183 定理14.4.3(Dirichlet判别法)
%@see: 《数学分析教程(第3版 下册)》(史济怀) P184 定理14.4.4(Abel判别法)
设\(\{a_n\},\{b_n\}\)是两个数列.

若下列两个条件之一满足,
则级数\(\sum_{n=1}^\infty a_n b_n\)收敛:\begin{itemize}
	\item {\rm\bf 阿贝尔条件}
	数列\(\{a_n\}\)单调有界,
	级数\(\sum_{n=1}^\infty b_n\)收敛;

	\item {\rm\bf 狄利克雷条件}
	数列\(\{a_n\}\)单调趋于\(0\),
	级数\(\sum_{n=1}^\infty b_n\)的部分和数列有界.
	%@see: https://mathworld.wolfram.com/DirichletsTest.html
\end{itemize}
\begin{proof}
假设满足阿贝尔条件,
设\(\abs{a_n} \leq M\),
由于\(\sum_{n=1}^\infty b_n\)收敛,
则对于任意给定\(\epsilon>0\),
存在正整数\(N\),
使得对于一切\(n>N\)和一切正整数\(p\),
成立\begin{equation*}
	\abs{\sum_{k=n+1}^{n+p} b_k} < \epsilon.
\end{equation*}
对\(\sum_{k=n+1}^{n+p} a_k b_k\)应用\hyperref[theorem:无穷级数.阿贝尔引理]{阿贝尔引理},即得\begin{equation*}
	\abs{\sum_{k=n+1}^{n+p} a_k b_k}
	< \epsilon (\abs{a_{n+1} + 2 \abs{a_{n+p}}})
	\leq 3 M \epsilon.
\end{equation*}
根据\hyperref[theorem:无穷级数.级数的柯西审敛原理]{柯西审敛原理}便知
级数\(\sum_{n=1}^\infty a_n b_n\)收敛.

假设满足狄利克雷条件.
由于\(\lim_{n\to\infty} a_n = 0\),
因此对于任意给定\(\epsilon>0\),
存在正整数\(N\),
使得对于一切\(n>N\),
成立\begin{equation*}
	\abs{a_n} < \epsilon.
\end{equation*}
又设\(\abs{\sum_{k=1}^n b_k} \leq M\),
令\(B_p = \sum_{k=n+1}^{n+p} b_k\ (p=1,2,\dotsc)\),
则\begin{equation*}
	\abs{B_p}
	= \abs{\sum_{k=1}^{n+p} b_k - \sum_{k=1}^n b_k}
	\leq \abs{\sum_{k=1}^{n+p} b_k} + \abs{\sum_{k=1}^n b_k}
	= 2 M.
\end{equation*}
应用\hyperref[theorem:无穷级数.阿贝尔引理]{阿贝尔引理},同样可得\begin{equation*}
	\abs{\sum_{k=n+1}^{n+p} a_k b_k}
	\leq 2 M (\abs{a_{n+1}} + 2 \abs{a_{n+p}})
	< 6 M \epsilon
\end{equation*}对一切\(n>N\)与一切正整数\(p\)成立.
根据\hyperref[theorem:无穷级数.级数的柯西审敛原理]{柯西审敛原理}便知
级数\(\sum_{n=1}^\infty a_n b_n\)收敛.
\end{proof}
%\cref{theorem:反常积分.无穷限的反常积分的阿贝尔--狄利克雷审敛法}
\end{theorem}
\begin{remark}
在狄利克雷条件中,取\(b_n = (-1)^{n-1}\),
则级数\(\sum_{n=1}^\infty b_n\)的部分和数列有界,
那么当数列\(\{a_n\}\)单调趋于\(0\)时,
级数\(\sum_{n=1}^\infty (-1)^{n-1} a_n\)收敛.
这就是前面已经证明过的\hyperref[theorem:无穷级数.莱布尼茨定理]{莱布尼茨审敛法}.
可见莱布尼茨审敛法只是狄利克雷审敛法的特殊情形.

在阿贝尔条件中,由于数列\(\{a_n\}\)单调有界,不妨设\(\lim_{n\to\infty} a_n = a\),
则数列\(\{a_n - a\}\)单调趋于\(0\).
又因为级数\(\sum_{n=1}^\infty b_n\)收敛,
所以它的部分和数列必定有界,根据狄利克雷审敛法,
级数\(\sum_{n=1}^\infty (a_n - a) b_n\)收敛,
从而级数\(\sum_{n=1}^\infty a_n b_n\)收敛.
这就说明:阿贝尔审敛法是狄利克雷审敛法的特殊情形.
\end{remark}

\begin{example}\label{example:无穷级数.单调收敛数列与正弦函数列的乘积的级数收敛}
%@see: 《数学分析(第二版 下册)》(陈纪修) P34 例9.4.3
设数列\(\{a_n\}\)单调趋于\(0\),
则对一切实数\(x\),级数\(\sum_{n=1}^\infty a_n \sin n x\)收敛.
\begin{proof}
当\(x=2k\pi\)时,
有\(\sum_{n=1}^\infty a_n \sin n x = 0\),
于是对于一切实数\(x\),
级数\(\sum_{n=1}^\infty a_n \sin n x\)收敛.

当\(x\neq2k\pi\)时,有\begin{equation*}
	2\sin\frac{x}{2} \cdot \sum_{k=1}^n \sin kx
	= \cos\frac{x}{2} - \cos\frac{2n+1}{2}x,
\end{equation*}
于是对一切正整数\(n\)有\begin{equation*}
	\abs{\sum_{k=1}^n \sin k x}
	\leq \frac1{\abs{\sin(x/2)}},
\end{equation*}
由\hyperref[theorem:无穷级数.无穷级数的阿贝尔--狄利克雷审敛法]{狄利克雷审敛法}可知
级数\(\sum_{n=1}^\infty a_n \sin n x\)收敛.
\end{proof}
\end{example}

\section{绝对收敛与条件收敛}
\subsection{绝对收敛与条件收敛的概念}
\begin{definition}
%@see: 《高等数学(第六版 下册)》 P263
%@see: 《数学分析(第二版 下册)》(陈纪修) P35 定义9.4.2
设级数\(\sum_{n=1}^\infty u_n\)的各项为任意实数.
\begin{itemize}
	\item 如果级数\(\sum_{n=1}^\infty \abs{u_n}\)收敛,
	则称“级数\(\sum_{n=1}^\infty u_n\)~\DefineConcept{绝对收敛}”.
	\item 如果级数\(\sum_{n=1}^\infty u_n\)收敛,
	而级数\(\sum_{n=1}^\infty \abs{u_n}\)发散,
	则称“级数\(\sum_{n=1}^\infty u_n\)~\DefineConcept{条件收敛}”.
\end{itemize}
\end{definition}

\begin{example}
%@see: 《数学分析(第二版 下册)》(陈纪修) P35
\(\sum_{n=1}^\infty \frac{(-1)^{n+1}}n\)是一个条件收敛级数.
\end{example}

\begin{theorem}\label{theorem:无穷级数.绝对收敛级数必定收敛}
%@see: 《高等数学(第六版 下册)》 P263 定理8
%@see: 《数学分析教程 (第3版 下册)》(史济怀) P188 定理14.5.1
%@see: 《数学分析(第二版 下册)》(陈纪修) P36 定理9.4.4
设\(\{u_n\}\)是实数列,
令\(v_n \defeq \frac12 (u_n + \abs{u_n}),
w_n \defeq \frac12 (\abs{u_n} - u_n)
\ (n=1,2,\dotsc)\).
\begin{itemize}
	\item 如果级数\(\sum_{n=1}^\infty u_n\)绝对收敛,
	则级数\(\sum_{n=1}^\infty u_n\)、\(\sum_{n=1}^\infty v_n\)和\(\sum_{n=1}^\infty w_n\)都收敛.

	\item 如果级数\(\sum_{n=1}^\infty u_n\)条件收敛,
	则级数\(\sum_{n=1}^\infty v_n\)和\(\sum_{n=1}^\infty w_n\)都发散.
\end{itemize}
\begin{proof}
可见级数\(\sum_{n=1}^\infty v_n\)是
级数\(\sum_{n=1}^\infty u_n\)中的全体正项所构成的级数,
而级数\(\sum_{n=1}^\infty w_n\)是
级数\(\sum_{n=1}^\infty u_n\)中全体负项的绝对值所构成的级数.
级数\(\sum_{n=1}^\infty v_n\)和\(\sum_{n=1}^\infty w_n\)都是正项级数.

先设级数\(\sum_{n=1}^\infty u_n\)绝对收敛,
由于\begin{equation*}
	0 \leq v_n \leq \abs{u_n},
	0 \leq w_n \leq \abs{u_n},
	\quad n=1,2,\dotsc,
\end{equation*}
所以由\hyperref[theorem:无穷级数.正项级数的比较审敛法]{比较审敛法}可知
级数\(\sum_{n=1}^\infty v_n\)和\(\sum_{n=1}^\infty w_n\)都收敛.
又因为\begin{equation*}
	u_n = v_n - w_n,
\end{equation*}
所以由\hyperref[theorem:无穷级数.收敛级数性质2]{收敛级数的基本性质}可知\begin{equation*}
	\sum_{n=1}^\infty u_n
	= \sum_{n=1}^\infty v_n
	- \sum_{n=1}^\infty w_n,
\end{equation*}
那么级数\(\sum_{n=1}^\infty u_n\)收敛.

再设级数\(\sum_{n=1}^\infty u_n\)条件收敛.
用反证法,假设级数\(\sum_{n=1}^\infty v_n\)也收敛,
则由\begin{equation*}
	w_n = \frac12 (\abs{u_n} - u_n)
	= \frac12 (u_n + \abs{u_n} - 2 u_n)
	= v_n - u_n,
\end{equation*}可知\begin{equation*}
	\sum_{n=1}^\infty w_n
	= \sum_{n=1}^\infty v_n
	- \sum_{n=1}^\infty u_n,
\end{equation*}
所以级数\(\sum_{n=1}^\infty w_n\)也收敛,
于是级数\begin{equation*}
	\sum_{n=1}^\infty \abs{u_n}
	= \sum_{n=1}^\infty v_n
	+ \sum_{n=1}^\infty w_n
\end{equation*}也收敛,与假设矛盾!
因此级数\(\sum_{n=1}^\infty v_n\)必定发散.
同理级数\(\sum_{n=1}^\infty w_n\)也必定发散.
\end{proof}
\end{theorem}

建立\cref{theorem:无穷级数.绝对收敛级数必定收敛} 以后,
我们就把一大类级数的收敛性判定问题,
转化成为正项级数的收敛性判定问题.

一般说来,
如果级数\(\sum_{n=1}^\infty \abs{u_n}\)发散,
我们不能断定级数\(\sum_{n=1}^\infty u_n\)也发散.
正如我们在\cref{example:无穷级数.交错级数1} 看到的那样,
虽然\(\sum_{n=1}^\infty \frac{1}{n}\)发散,
但是\(\sum_{n=1}^\infty \frac{(-1)^{n+1}}{n}\)收敛.
但是,如果我们用比值审敛法或根值审敛法,
根据\(\lim_{n\to\infty} \abs{\frac{u_{n+1}}{u_n}} > 1\)
或\(\lim_{n\to\infty} \sqrt[n]{\abs{u_n}} > 1\)
判定级数\(\sum_{n=1}^\infty \abs{u_n}\)发散,
那么我们可以断定级数\(\sum_{n=1}^\infty u_n\)必然发散.

\begin{theorem}\label{theorem:无穷级数.绝对发散的特殊情况}
%@see: 《高等数学(第六版 下册)》 P264
当级数\(\sum_{n=1}^\infty u_n\)满足\begin{equation*}
	\lim_{n\to\infty} \abs{\frac{u_{n+1}}{u_n}} = \rho > 1
	\quad\text{或}\quad
	\lim_{n\to\infty} \sqrt[n]{\abs{u_n}} = \rho > 1
\end{equation*}时,
这个级数必定发散.
\begin{proof}
这是因为从\(\rho > 1\)可推知\(\abs{u_n} \not\to 0\ (n\to\infty)\),
从而\(u_n \not\to 0\ (n\to\infty)\),
因此级数\(\sum_{n=1}^\infty u_n\)是发散的.
\end{proof}
\end{theorem}

\begin{example}
设级数\(\sum_{n=1}^\infty u_n\)条件收敛,
且极限\(r \defeq \lim_{n\to\infty} \frac{u_{n+1}}{u_n}\)存在且有限.
证明:\(r = -1\).
\begin{proof}
易知\(\lim_{n\to\infty} \abs{\frac{u_{n+1}}{u_n}} = \abs{r}\).

假设\(\abs{r} < 1\),
则由\cref{theorem:无穷级数.正项级数的比值审敛法} 可知,
级数\(\sum_{n=1}^\infty \abs{u_n}\)收敛,
级数\(\sum_{n=1}^\infty u_n\)绝对收敛,矛盾!

假设\(\abs{r} > 1\),
则由\cref{theorem:无穷级数.绝对发散的特殊情况} 可知,
级数\(\sum_{n=1}^\infty u_n\)发散,矛盾!

假设\(r = 1\),
那么存在正整数\(N\),当\(n>N\)时,有\(\frac{u_{n+1}}{u_n} > 0\),
即\(u_{n+1}\)和\(u_n\)同号,
级数\(\sum_{n=N+1}^\infty u_n\)是正项级数或负项级数,
但凡收敛必定是绝对收敛,而不可能条件收敛.

综上所述,\(r\)只可能等于\(-1\).
\end{proof}
\end{example}

\begin{example}
%@see: 《高等数学(第六版 下册)》 P265 例9
判断级数\(\sum_{n=1}^\infty \frac{\sin n\alpha}{n^2}\)的敛散性.
\begin{solution}
因为\begin{equation*}
	\abs{\frac{\sin n\alpha}{n^2}}
	\leq \frac1{n^2},
\end{equation*}
而级数\(\sum_{n=1}^\infty \frac1{n^2}\)收敛,
所以级数\(\sum_{n=1}^\infty \abs{\frac{\sin n\alpha}{n^2}}\)收敛,
由\cref{theorem:无穷级数.绝对收敛级数必定收敛} 可知,
级数\(\sum_{n=1}^\infty \frac{\sin n\alpha}{n^2}\)收敛.
\end{solution}
\end{example}

\begin{example}
%@see: 《数学分析(第二版 下册)》(陈纪修) P35 例9.4.4
判断级数\(\sum_{n=1}^\infty \frac{x^n}{n^p}\)的敛散性.
\begin{solution}
对\(\sum_{n=1}^\infty \abs{\frac{x^n}{n^p}}
= \sum_{n=1}^\infty \frac{\abs{x}^n}{n^p}\)
应用\hyperref[theorem:无穷级数.正项级数的根值审敛法]{柯西审敛法},有\begin{equation*}
	\lim_{n\to\infty} \sqrt[n]{\frac{\abs{x}^n}{n^p}}
	= \abs{x}.
\end{equation*}
由此可知:\begin{itemize}
	\item 当\(\abs{x}<1\)时,对于任意实数\(p\),
	总有级数\(\sum_{n=1}^\infty \frac{x^n}{n^p}\)绝对收敛;

	\item 当\(\abs{x}>1\)时,对于任意实数\(p\),
	总有级数\(\sum_{n=1}^\infty \frac{x^n}{n^p}\)发散;

	\item 当\(x=1\)时,
	若\(p>1\)则级数\(\sum_{n=1}^\infty \frac{x^n}{n^p}\)收敛,
	若\(p\leq1\)则级数\(\sum_{n=1}^\infty \frac{x^n}{n^p}\)发散;

	\item 当\(x=-1\)时,
	若\(p>1\)则级数\(\sum_{n=1}^\infty \frac{x^n}{n^p}\)绝对收敛,
	若\(0<p\leq1\)则级数\(\sum_{n=1}^\infty \frac{x^n}{n^p}\)条件收敛,
	若\(p\leq0\)则级数\(\sum_{n=1}^\infty \frac{x^n}{n^p}\)发散.
\end{itemize}
\end{solution}
\end{example}

\begin{example}
%@see: 《数学分析(第二版 下册)》(陈纪修) P36 例9.4.5
判断级数\(\sum_{n=1}^\infty \frac{\sin nx}{n^p}\ (0<x<\pi)\)的敛散性.
\begin{solution}
当\(p>1\)时,由\(\frac{\abs{\sin nx}}{n^p} \leq \frac1{n^p}\),
可知级数\(\sum_{n=1}^\infty \frac{\sin nx}{n^p}\)绝对收敛.

当\(0<p\leq1\)时,
由\cref{example:无穷级数.单调收敛数列与正弦函数列的乘积的级数收敛} 可知
级数\(\sum_{n=1}^\infty \frac{\sin nx}{n^p}\)收敛,
进而有\begin{equation*}
	\frac{\abs{\sin nx}}{n^p}
	\geq \frac{\sin^2nx}{n^p}
	= \frac1{2n^p} - \frac{\cos2nx}{2n^p},
\end{equation*}
因为级数\(\sum_{n=1}^\infty \frac1{2n^p}\)发散,
所以级数\(\sum_{n=1}^\infty \frac{\abs{\sin nx}}{n^p}\)发散,
级数\(\sum_{n=1}^\infty \frac{\sin nx}{n^p}\)条件收敛.

当\(p\leq0\)时,由于级数的一般项不趋于\(0\),
级数\(\sum_{n=1}^\infty \frac{\sin nx}{n^p}\)发散.
\end{solution}
\end{example}

\begin{example}
%@see: 《高等数学(第六版 下册)》 P265 例10
判断级数\(\sum_{n=1}^\infty (-1)^n \frac1{2^n} \left(1+\frac1n\right)^{n^2}\)的敛散性.
\begin{solution}
记\(u_n = \frac1{2^n} \left(1+\frac1n\right)^{n^2}\),
有\begin{equation*}
	\sqrt[n]{u_n}
	= \frac12 \left(1+\frac1n\right)^n
	\to \frac12 e \quad(n\to\infty),
\end{equation*}
而\(e/2>1\),所以\(u_n \not\to 0\ (n\to\infty)\),
那么级数\(\sum_{n=1}^\infty (-1)^n \frac1{2^n} \left(1+\frac1n\right)^{n^2}\)发散.
\end{solution}
\end{example}

\begin{example}
%@see: 《2023年全国硕士研究生入学统一考试(数学一)》一选择题/第4题
已知\(a_n < b_n\ (n=1,2,\dotsc)\),
若级数\(\sum_{n=1}^\infty a_n\)与\(\sum_{n=1}^\infty b_n\)均收敛,
证明:“\(\sum_{n=1}^\infty a_n\)绝对收敛”是
“\(\sum_{n=1}^\infty b_n\)绝对收敛”的充分必要条件.
\begin{proof}
由题可知级数\(\sum_{n=1}^\infty (b_n - a_n)\)是收敛的正项级数,因而绝对收敛.

当级数\(\sum_{n=1}^\infty a_n\)绝对收敛时,
由\hyperref[theorem:不等式.三角不等式1]{三角不等式}有\begin{equation*}
	\abs{b_n} = \abs{(b_n - a_n) + a_n}
	\leq \abs{b_n - a_n} + \abs{a_n},
\end{equation*}
那么由\hyperref[theorem:无穷级数.正项级数的比较审敛法]{比较审敛法}可知,
\(\sum_{n=1}^\infty \abs{b_n}\)收敛,
\(\sum_{n=1}^\infty b_n\)绝对收敛.

同理,当级数\(\sum_{n=1}^\infty b_n\)绝对收敛时,
亦有\(\sum_{n=1}^\infty a_n\)绝对收敛.
\end{proof}
\end{example}

\begin{example}
%@see: 《1994年全国硕士研究生入学统一考试(数学一)》二选择题/第3题
设常数\(\lambda>0\),级数\(\sum_{n=1}^\infty a_n^2\)收敛,
考察级数\(\sum_{n=1}^\infty (-1)^n \frac{\abs{a_n}}{\sqrt{n^2+\lambda}}\)的敛散性.
\begin{solution}
由于\begin{equation*}
	\abs{(-1)^n \frac{\abs{a_n}}{\sqrt{n^2+\lambda}}}
	= \frac{\abs{a_n}}{\sqrt{n^2+\lambda}}
	\leq \frac{1}{2} \left( a_n^2 + \frac{1}{n^2+\lambda} \right)
	\leq \frac{1}{2} \left( a_n^2 + \frac{1}{n^2} \right),
\end{equation*}
并且级数\(\sum\limits_{n=1}^\infty a_n^2\)
和级数\(\sum\limits_{n=1}^\infty \frac{1}{n^2}\)收敛,
那么由\hyperref[theorem:无穷级数.收敛级数性质2]{收敛级数的性质}%
以及\hyperref[theorem:无穷级数.正项级数的比较审敛法的推论]{正项级数的比较判别法}可知,
级数\(\sum\limits_{n=1}^\infty (-1)^n \frac{\abs{a_n}}{\sqrt{n^2+\lambda}}\)绝对收敛.
\end{solution}
\end{example}
\begin{example}
%@see: 《2025年全国硕士研究生入学统一考试(数学一)》一选择题/第2题
考察级数\(\sum_{n=1}^\infty \sin\frac{n^3\pi}{n^2+1}\)的敛散性.
\begin{solution}
由\cref{equation:函数.三角函数.和积互化公式1} 可得\begin{equation*}
	\sin\frac{n^3\pi}{n^2+1}
	= \sin\left( \frac{n^3\pi}{n^2+1} - n\pi + n\pi \right)
	% \sin(n\pi) = 0
	% \cos(n\pi) = (-1)^n
	= (-1)^{n+1} \sin\frac{n\pi}{n^2+1},
\end{equation*}
加之\(\sin\frac{n\pi}{n^2+1}\to0\ (n\to\infty)\),
故由\hyperref[theorem:无穷级数.莱布尼茨定理]{莱布尼茨定理}可知
级数\(\sum_{n=1}^\infty \sin\frac{n^3\pi}{n^2+1}\)收敛.

但是因为\(\sin\frac{n\pi}{n^2+1}
\sim \frac{n\pi}{n^2+1}
\sim \frac\pi{n}\),
所以由\cref{theorem:无穷级数.正项级数的比较审敛法的极限形式} 可知
级数\(\sum_{n=1}^\infty \sin\frac{n^3\pi}{n^2+1}\)不绝对收敛.

综上所述,级数\(\sum_{n=1}^\infty \sin\frac{n^3\pi}{n^2+1}\)条件收敛.
\end{solution}
\end{example}
\begin{example}
%@see: 《2025年全国硕士研究生入学统一考试(数学一)》一选择题/第2题
考察级数\(\sum_{n=1}^\infty (-1)^n \left( \frac1{\sqrt[3]{n^2}} - \tan\frac1{\sqrt[3]{n^2}} \right)\)的敛散性.
\begin{solution}
当\(n\to\infty\)时,\(\frac1{\sqrt[3]{n^2}}\to0\).
令\(x=\frac1{\sqrt[3]{n^2}}\),
则\(x - \tan x \sim -\frac13 x^2\ (x\to0)\).
这就说明\begin{equation*}
	\abs{(-1)^n \left( \frac1{\sqrt[3]{n^2}} - \tan\frac1{\sqrt[3]{n^2}} \right)} \sim \frac13 \frac1{n^2}.
\end{equation*}
由\cref{theorem:无穷级数.正项级数的比较审敛法的极限形式} 可知
级数\(\sum_{n=1}^\infty (-1)^n \left( \frac1{\sqrt[3]{n^2}} - \tan\frac1{\sqrt[3]{n^2}} \right)\)绝对收敛.
\end{solution}
\end{example}

\subsection{绝对收敛级数的性质}
绝对收敛级数有很多性质是条件收敛级数所没有的.

%@see: 《数学分析(第二版 上册)》(陈纪修) P37 加法交换律
\cref{theorem:无穷级数.收敛级数性质4} 表明:收敛级数保持结合律.
那么,收敛级数是否也保持交换律呢?
也就是说,将一个收敛级数\(\sum_{n=1}^\infty u_n\)的项任意重新排列,
得到一个新级数\(\sum_{n=1}^\infty v_n\)
(我们把级数\(\sum_{n=1}^\infty v_n\)
称为“级数\(\sum_{n=1}^\infty u_n\)的\DefineConcept{更序级数}”),
这个新级数是否仍然收敛?如果这个新级数收敛,其和是否保持不变,即是否有\begin{equation*}
	\sum_{n=1}^\infty v_n = \sum_{n=1}^\infty u_n
\end{equation*}成立?
遗憾的是,答案是否定的.

%@see: 《数学分析(第二版 上册)》(陈纪修) P37 例9.4.6
考虑莱布尼茨级数\begin{equation*}
	\sum_{n=1}^\infty \frac{(-1)^{n+1}}n.
\end{equation*}
这是一个条件收敛级数,可以证明它的和为\(\ln2\).
现在重新排列这个级数的项,依次在每一个正项后面接两个负项,即\begin{equation*}
	\sum_{n=1}^\infty u'_n
	= 1 - \frac12 - \frac14
	+ \frac13 - \frac16 - \frac18
	+ \dotsb
	+ \frac1{2n-1} - \frac1{4n-2} - \frac1{4n}
	+ \dotsb.
\end{equation*}
设\(\sum_{n=1}^\infty \frac{(-1)^{n+1}}n\)的部分和数列为\(\{S_n\}\),
\(\sum_{n=1}^\infty u'_n\)的部分和数列为\(\{S'_n\}\),
则\begin{align*}
	S'_{3n}
	&= \sum_{k=1}^n \left(
		\frac1{2k-1} - \frac1{4k-2} - \frac1{4k}
	\right) \\
	&= \sum_{k=1}^n \left(
		\frac1{4k-2} - \frac1{4k}
	\right) \\
	&= \frac12 \sum_{k=1}^n \left(
		\frac1{2k-1} - \frac1{2k}
	\right)
	= \frac12 S_{2n},
\end{align*}
于是\begin{equation*}
	\lim_{n\to\infty} S'_{3n}
	= \frac12 \lim_{n\to\infty} S_{2n}
	= \frac12 \ln 2.
\end{equation*}
%@credit: {ce603838-a24d-4616-9395-d7b223e8cb72}
% 下面之所以要研究\(S'_{3n-1}\)和\(S'_{3n+1}\),
% 是因为\(S'_{3n}\)只是\(n=3k\)(模3得0)情况下的部分和.
% 要想进一步证明\(S'_n\)收敛于0,还要考虑模3得1或2的形式,
% 也就是要论证\(S'_{3n-1}\)和\(S'_{3n+1}\)也同样收敛于0.
% 下面两条等式说明当\(n\to\infty\)时,\(S'_{3n-1}\)和\(S'_{3n+1}\)两者与\(S'_{3n}\)相比,只是多了一个无穷小.
% 具体思路可以参考\cref{example:数列极限.奇偶子列收敛则数列收敛}.
由于\begin{equation*}
	S'_{3n-1} = S'_{3n} + \frac1{4n},
	\qquad
	S'_{3n+1} = S'_{3n} + \frac1{2n+1},
\end{equation*}
最终得到\begin{equation*}
	\sum_{n=1}^\infty u'_n
	= \lim_{n\to\infty} S'_n
	= \frac12 \ln 2.
\end{equation*}
也就是说,尽管级数\(\sum_{n=1}^\infty \frac{(-1)^{n+1}}n\)收敛,但它不保持交换律.

这个例子告诉我们,要使一个级数保持加法交换律,仅有收敛性是不够的.
事实上,是否保持加法交换律,是绝对收敛级数与条件收敛级数的一个本质区别.

\begin{theorem}[绝对收敛级数的可交换性]\label{theorem:无穷级数.绝对收敛级数的可交换性}
%@see: 《数学分析(第二版 上册)》(陈纪修) P38 定理9.4.5
%@see: 《高等数学(第六版 下册)》 P265 定理9
%@see: 《数学分析教程 (第3版 下册)》(史济怀) P189 定理14.5.2
绝对收敛级数的更序级数也绝对收敛,两者的和相同.
\begin{proof}
设级数\(\sum_{n=1}^\infty u_n\)绝对收敛,
级数\(\sum_{n=1}^\infty u'_n\)是它的更序级数.
接下来我们分两步证明这个定理.

首先假设\(\sum_{n=1}^\infty u_n\)是正项级数,
则对一切正整数\(n\),有\begin{equation*}
	\sum_{k=1}^n u'_k
	\leq \sum_{n=1}^\infty u_n,
\end{equation*}
于是级数\(\sum_{n=1}^\infty u'_n\)收敛,
且\begin{equation*}
	\sum_{n=1}^\infty u'_n
	\leq \sum_{n=1}^\infty u_n.
\end{equation*}
反过来,也可以将\(\sum_{n=1}^\infty u_n\)
看成是\(\sum_{n=1}^\infty u'_n\)的更序级数,
从而又有\begin{equation*}
	\sum_{n=1}^\infty u_n
	\leq \sum_{n=1}^\infty u'_n.
\end{equation*}
因此\begin{equation*}
	\sum_{n=1}^\infty u_n
	= \sum_{n=1}^\infty u'_n.
\end{equation*}

然后假设\(\sum_{n=1}^\infty u_n\)是任意级数,
则由\cref{theorem:无穷级数.绝对收敛级数必定收敛} 可知,
正项级数\(\sum_{n=1}^\infty v_n\)和\(\sum_{n=1}^\infty w_n\)都收敛,
且\begin{equation*}
	\sum_{n=1}^\infty u_n
	= \sum_{n=1}^\infty v_n
	- \sum_{n=1}^\infty w_n,
	\qquad
	\sum_{n=1}^\infty \abs{u_n}
	= \sum_{n=1}^\infty v_n
	+ \sum_{n=1}^\infty w_n.
\end{equation*}
对于更序级数\(\sum_{n=1}^\infty u'_n\),
同样可以构造对应的正项级数\(\sum_{n=1}^\infty v'_n\)和\(\sum_{n=1}^\infty w'_n\).
由于\(\sum_{n=1}^\infty v'_n\)和\(\sum_{n=1}^\infty v_n\)、
\(\sum_{n=1}^\infty w'_n\)和\(\sum_{n=1}^\infty w_n\)互为更序级数,
那么由前面的结论可知\begin{equation*}
	\sum_{n=1}^\infty v'_n
	= \sum_{n=1}^\infty v_n,
	\qquad
	\sum_{n=1}^\infty w'_n
	= \sum_{n=1}^\infty w_n.
\end{equation*}
于是级数\begin{equation*}
	\sum_{n=1}^\infty \abs{u'_n}
	= \sum_{n=1}^\infty v'_n
	+ \sum_{n=1}^\infty w'_n
\end{equation*}收敛,
即级数\(\sum_{n=1}^\infty u'_n\)绝对收敛,
且\begin{equation*}
	\sum_{n=1}^\infty u'_n
	= \sum_{n=1}^\infty v'_n
	- \sum_{n=1}^\infty w'_n
	= \sum_{n=1}^\infty v_n
	- \sum_{n=1}^\infty w_n
	= \sum_{n=1}^\infty u_n.
	\qedhere
\end{equation*}
\end{proof}
\end{theorem}

从\cref{theorem:无穷级数.绝对收敛级数的可交换性} 的证明过程中,
我们还可以看出如下命题也成立.
\begin{proposition}
若正项级数\(\sum_{n=1}^\infty u_n\)的更序级数\(\sum_{n=1}^\infty u'_n\)收敛,
则\(\sum_{n=1}^\infty u_n\)也收敛.
%@credit: {5a781423-ba4e-4629-ac1a-eac743a4d445}
\end{proposition}

\begin{theorem}[条件收敛级数的黎曼定理]
%@see: 《数学分析(第二版 上册)》(陈纪修) P39 定理9.4.6(Riemann)
%@see: 《数学分析教程 (第3版 下册)》(史济怀) P190 定理14.5.3(Riemann定理)
%@see: https://mathworld.wolfram.com/RiemannSeriesTheorem.html
若级数\(\sum_{n=1}^\infty u_n\)条件收敛,
则只要适当交换各项的次序,
就可使其收敛到任意一个有限实数,
也可以使其发散到\(+\infty\)或\(-\infty\).
% By a suitable rearrangement of terms, a conditionally convergent series may be made to converge to any desired value, or to diverge.
\begin{proof}
假设\(a\)是一个有限实数.
由于\(\sum_{n=1}^\infty u_n\)条件收敛,
由\cref{theorem:无穷级数.绝对收敛级数必定收敛} 可知\begin{equation*}
	\sum_{n=1}^\infty v_n = +\infty,
	\qquad
	\sum_{n=1}^\infty w_n = +\infty.
\end{equation*}
依次计算\(\sum_{n=1}^\infty v_n\)的部分和,
必定存在最小的正整数\(n_1\)满足\begin{equation*}
	\sum_{k=1}^{n_1} v_k > a.
\end{equation*}
再依次计算\(\sum_{n=1}^\infty w_n\)的部分和,
也必定存在最小的正整数\(m_1\)满足\begin{equation*}
	\sum_{k=1}^{n_1} v_k
	- \sum_{k=1}^{m_1} w_k
	< a.
\end{equation*}
类似地可以找到最小的正整数\(n_2>n_1\)和最小的正整数\(m_2>m_1\)满足\begin{equation*}
	\sum_{k=1}^{n_1} v_k
	- \sum_{k=1}^{m_1} w_k
	+ \sum_{k=n_1+1}^{n_2} v_k
	> a
\end{equation*}和\begin{equation*}
	\sum_{k=1}^{n_1} v_k
	- \sum_{k=1}^{m_1} w_k
	+ \sum_{k=n_1+1}^{n_2} v_k
	- \sum_{k=m_1+1}^{m_2} w_k
	< a.
\end{equation*}
这样的步骤可以一直继续下去,
由此得到\(\sum_{k=1}^\infty u_n\)的一个更序级数\(\sum_{n=1}^\infty u'_n\),
它的部分和在\(a + v_{n_k}\)与\(a - w_{m_k}\)之间振荡.
由于\(\sum_{k=1}^\infty u_n\)收敛,
可知\begin{equation*}
	\lim_{n\to\infty} v_n
	= \lim_{n\to\infty} w_n
	= 0,
\end{equation*}
于是\begin{equation*}
	\sum_{n=1}^\infty u'_n = a.
\end{equation*}
这就说明,只要适当交换\(\sum_{k=1}^\infty u_n\)各项的次序,
就可使其收敛到任意一个有限实数\(a\).
\end{proof}
\end{theorem}

%@see: 《数学分析(第二版 上册)》(陈纪修) P40 级数的乘法
有限和式\(\sum_{k=1}^n a_k\)和\(\sum_{k=1}^m b_k\)的乘积是
所有诸如\begin{equation*}
	a_i b_j\ (i=1,2,\dotsc,n;j=1,2,\dotsc,m)
\end{equation*}项的和,
显然,其最终结果与它们相加的次序、方式无关.

类似地,对于两个收敛的级数\(\sum_{k=1}^\infty a_k\)和\(\sum_{k=1}^\infty b_k\),
可以同样写出所有诸如\begin{equation*}
	a_i b_j\ (i=1,2,\dotsc;j=1,2,\dotsc)
\end{equation*}项,
再讲它们排列成如\cref{figure:无穷级数.无穷级数乘积各项的矩阵} 所示的无穷矩阵的形式.
\begin{figure}[htb]
	\centering
	\def\term#1#2{\draw({\number#2-1},{1-\number#1})node{$a_#1 b_#2$};}
	\begin{tikzpicture}
		\term11
		\term12
		\term13
		\term14
		\draw(4,0)node{$\dotso$};
		\term21
		\term22
		\term23
		\term24
		\draw(4,-1)node{$\dotso$};
		\term31
		\term32
		\term33
		\term34
		\draw(4,-2)node{$\dotso$};
		\term41
		\term42
		\term43
		\term44
		\draw(4,-3)node{$\dotso$};
		\draw(0,-4)node{$\vdots$};
		\draw(1,-4)node{$\vdots$};
		\draw(2,-4)node{$\vdots$};
		\draw(3,-4)node{$\vdots$};
		\draw(4,-4)node{$\ddots$};
	\end{tikzpicture}
	\caption{}
	\label{figure:无穷级数.无穷级数乘积各项的矩阵}
\end{figure}

由于级数运算一般不满足交换律和结合律,这就有一个排列的次序与方式的问题.
尽管排列的次序与方式多种多样,
但是最常用的最具应用价值的方式是“对角线”排列与“正方形”排列.

\begin{definition}\label{definition:无穷级数.绝对收敛级数的柯西乘积}
%@see: 《高等数学(第六版 下册)》 P267 定理10
定义级数\(\sum_{n=1}^\infty u_n\)
和\(\sum_{n=1}^\infty v_n\)的\DefineConcept{柯西乘积}为\begin{equation*}
	\left( \sum_{n=1}^\infty u_n \right)
	\cdot
	\left( \sum_{n=1}^\infty v_n \right)
	\defeq
	\sum_{n=1}^\infty \sum_{k=1}^n u_k v_{n-k+1}.
\end{equation*}
\end{definition}

%@see: 《数学分析(第二版 上册)》(陈纪修) P42
对于正方形排列所得的乘积,
只要\(\sum_{n=1}^\infty u_n\)和\(\sum_{n=1}^\infty v_n\)都收敛,
那么它们的乘积\(\sum_{n=1}^\infty w_n\)总是收敛的.

但是,仅有\(\sum_{n=1}^\infty u_n\)和\(\sum_{n=1}^\infty v_n\)的收敛性,
不足以保证它们的柯西乘积的收敛性.
%@see: 《数学分析(第二版 上册)》(陈纪修) P42 例9.4.7
例如,取\begin{equation*}
	a_n = b_n = \frac{(-1)^{n+1}}{\sqrt{n}}
	\quad(n=1,2,\dotsc),
\end{equation*}
则级数\(\sum_{n=1}^\infty a_n\)和\(\sum_{n=1}^\infty b_n\)都是条件收敛的,
它们的柯西乘积的一般项为\begin{equation*}
	c_n = (-1)^{n+1} \sum_{i+j=n+1} \frac1{\sqrt{ij}},
\end{equation*}
注意上面\(c_n\)的表达式中共有\(n\)项,
在每一项中都有\(i+j = n+1\),
因而\begin{equation*}
	\sqrt{ij} \leq \frac{i+j}2 = \frac{n+1}2,
\end{equation*}
于是得到\begin{equation*}
	\abs{c_n} \geq n \cdot \frac2{n+1},
\end{equation*}
因此\(\abs{c_n}\)不是无穷小,
级数\(\sum_{n=1}^\infty a_n\)和\(\sum_{n=1}^\infty b_n\)的
柯西乘积\(\sum_{n=1}^\infty c_n\)发散.

\begin{theorem}\label{theorem:无穷级数.绝对收敛级数的柯西乘积必收敛}
%@see: 《数学分析(第二版 上册)》(陈纪修) P42 定理9.4.7
%@see: 《高等数学(第六版 下册)》 P267 定理10
设级数\(\sum_{n=1}^\infty u_n\)和\(\sum_{n=1}^\infty v_n\)都绝对收敛,
其和分别为\(s\)和\(\sigma\),
则将\begin{equation*}
	a_i b_j
	\quad(i=1,2,\dotsc;j=1,2,\dotsc)
\end{equation*}按任意方式排列求和而成的级数也绝对收敛,
且其和为\(s \cdot \sigma\).
%TODO proof
\end{theorem}

\begin{proposition}\label{theorem:绝对收敛.命题1}
若级数\(\sum_{n=1}^\infty u_n\)绝对收敛,
级数\(\sum_{n=1}^\infty v_n\)条件收敛,
则级数\(\sum_{n=1}^\infty (u_n \pm v_n)\)条件收敛.
\begin{proof}
\begin{proof}[证法一]
因为级数\(\sum_{n=1}^\infty u_n\)和\(\sum_{n=1}^\infty v_n\)都收敛,
所以,由\cref{theorem:无穷级数.收敛级数性质2} 可知,
它们相加或相减所得级数\(\sum_{n=1}^\infty (u_n \pm v_n)\)一定收敛.

因为级数\(\sum_{n=1}^\infty v_n\)条件收敛,
所以级数\(\sum_{n=1}^\infty \abs{v_n}\)发散;
由\hyperref[theorem:不等式.三角不等式1]{三角不等式}有\begin{align*}
	\abs{v_n}
	= \abs{(v_n + u_n) - u_n}
	\leq \abs{u_n + v_n} + \abs{u_n}, \\
	\abs{v_n}
	= \abs{(v_n - u_n) + u_n}
	\leq \abs{u_n - v_n} + \abs{u_n};
\end{align*}
于是由\hyperref[theorem:无穷级数.正项级数的比较审敛法]{比较审敛法}可知
级数\(\sum_{n=1}^\infty \left( \abs{u_n \pm v_n} + \abs{u_n} \right)\)发散.

又因为级数\(\sum_{n=1}^\infty u_n\)绝对收敛,
也就是说级数\(\sum_{n=1}^\infty \abs{u_n}\)收敛.
那么由\cref{theorem:无穷级数.收敛级数性质2.推论1} 可知,
级数\begin{equation*}
	\sum_{n=1}^\infty \left[
		\left( \abs{u_n \pm v_n} + \abs{u_n} \right) - \abs{u_n}
	\right]
	= \sum_{n=1}^\infty \abs{u_n \pm v_n}
\end{equation*}发散.

综上所述,级数\(\sum_{n=1}^\infty (u_n \pm v_n)\)条件收敛.
\end{proof}
\begin{proof}[证法二]
用反证法.
假设级数\(\sum_{n=1}^\infty (u_n \pm v_n)\)绝对收敛,
故级数\(\sum_{n=1}^\infty \abs{u_n \pm v_n}\)收敛.
因为级数\(\sum_{n=1}^\infty u_n\)绝对收敛,
级数\(\sum_{n=1}^\infty \abs{u_n}\)收敛,
所以由\cref{theorem:无穷级数.收敛级数性质2} 可知,
级数\(\sum_{n=1}^\infty (\abs{u_n + v_n} + \abs{u_n})\)收敛.
又因为\begin{equation*}
	\abs{v_n}
	= \abs{(v_n + u_n) - u_n}
	\leq \abs{u_n + v_n} + \abs{u_n},
\end{equation*}
所以由\hyperref[theorem:无穷级数.正项级数的比较审敛法]{比较审敛法}可知,
级数\(\sum_{n=1}^\infty \abs{v_n}\)收敛,
也就是说级数\(\sum_{n=1}^\infty v_n\)绝对收敛,矛盾!
因此级数\(\sum_{n=1}^\infty (u_n \pm v_n)\)条件收敛.
\end{proof}\let\qed\relax
\end{proof}
\end{proposition}

\begin{proposition}\label{theorem:绝对收敛.命题2}
设正项级数\(\sum_{n=1}^\infty u_n\)收敛,
数列\(\{v_n\}\)有界,
则级数\(\sum_{n=1}^\infty u_n v_n\)绝对收敛.
\begin{proof}
设\(\abs{v_n} < M\ (n=1,2,\dotsc)\),
那么\begin{equation*}
	\abs{u_n v_n}
	= \abs{u_n} \abs{v_n}
	< M \abs{u_n}
	= M u_n
	\quad(n=1,2,\dotsc),
\end{equation*}
于是根据\hyperref[theorem:无穷级数.正项级数的比较审敛法的推论]{正项级数的比较审敛法}可知,
级数\(\sum_{n=1}^\infty u_n v_n\)绝对收敛.
\end{proof}
\end{proposition}

\begin{proposition}\label{theorem:绝对收敛.命题3}
设级数\(\sum_{n=1}^\infty u_n\)绝对收敛,
\(\lim_{n\to\infty} v_n = 0\),
则级数\(\sum_{n=1}^\infty u_n v_n\)绝对收敛.
\begin{proof}
根据\cref{theorem:极限.收敛数列的有界性},
数列\(\{v_n\}\)有界.
又因为\(\sum_{n=1}^\infty \abs{u_n}\)收敛,
那么根据\cref{theorem:绝对收敛.命题2} 可知
级数\(\sum_{n=1}^\infty u_n v_n\)绝对收敛.
\end{proof}
\end{proposition}

\begin{proposition}\label{theorem:绝对收敛.命题4}
设\(\sum_{n=1}^\infty u_n\)和\(\sum_{n=1}^\infty v_n\)这两个正项级数都收敛,
则正项级数\(\sum_{n=1}^\infty u_n v_n\)收敛.
\begin{proof}
因为\(\sum_{n=1}^\infty v_n\)收敛,
根据\hyperref[theorem:无穷级数.级数收敛的必要条件]{级数收敛的必要条件},
必有\(\lim_{n\to\infty} v_n = 0\).
于是,根据\cref{theorem:绝对收敛.命题3} 便有
级数\(\sum_{n=1}^\infty u_n v_n\)绝对收敛.
\end{proof}
\end{proposition}

\begin{proposition}\label{theorem:绝对收敛.命题5}
设级数\(\sum_{n=1}^\infty u_n\)绝对收敛,
级数\(\sum_{n=1}^\infty v_n\)条件收敛,
则级数\(\sum_{n=1}^\infty u_n v_n\)绝对收敛.
\begin{proof}
因为级数\(\sum_{n=1}^\infty u_n\)绝对收敛,
所以级数\(\sum_{n=1}^\infty \abs{u_n}\)收敛.
又因为级数\(\sum_{n=1}^\infty v_n\)条件收敛,
所以根据\hyperref[theorem:无穷级数.级数收敛的必要条件]{级数收敛的必要条件}有
\(\lim_{n\to\infty} v_n = 0\).
因此,根据\cref{theorem:绝对收敛.命题3},
级数\(\sum_{n=1}^\infty \abs{u_n} v_n\)绝对收敛,
也就是说,级数\begin{equation*}
	\sum_{n=1}^\infty \abs{\abs{u_n} v_n}
	= \sum_{n=1}^\infty \abs{u_n v_n}
\end{equation*}收敛,
级数\(\sum_{n=1}^\infty u_n v_n\)绝对收敛.
\end{proof}
\end{proposition}

\begin{example}
设级数\(\sum_{n=1}^\infty u_n^2\)收敛.
证明:级数\(\sum_{n=1}^\infty u_n^3\)绝对收敛.
\begin{proof}
因为级数\(\sum_{n=1}^\infty u_n^2\)收敛,
所以根据\hyperref[theorem:无穷级数.级数收敛的必要条件]{级数收敛的必要条件}有
\(\lim_{n\to\infty} u_n^2 = 0\),
于是\(\lim_{n\to\infty} u_n = 0\).
因此,根据\cref{theorem:绝对收敛.命题3},
级数\(\sum_{n=1}^\infty u_n^2 \cdot u_n
= \sum_{n=1}^\infty u_n^3\)绝对收敛.
\end{proof}
\end{example}
从这个例子出发,我们可以做一些推广.
例如,当级数\(\sum_{n=1}^\infty u_n^3\)收敛时,
级数\(\sum_{n=1}^\infty u_n^4\)未必收敛.
但是,当级数\(\sum_{n=1}^\infty u_n^3\)绝对收敛时,
级数\(\sum_{n=1}^\infty u_n^4\)必定收敛.
还有,当正项级数\(\sum_{n=1}^\infty u_n\)收敛时,
级数\(\sum_{n=1}^\infty \sqrt{u_n}\)未必收敛.
% 例如\(\sum_{n=1}^\infty \frac1{n^2}\)收敛,但\(\sum_{n=1}^\infty \frac1n\)发散.

\begin{example}
设级数\(\sum_{n=1}^\infty u_n^2\)收敛,
证明:级数\(\sum_{n=1}^\infty \frac{u_n}{n}\)绝对收敛.
\begin{proof}
因为\begin{equation*}
	\abs{\frac{u_n}{n}}
	= \abs{u_n} \cdot \frac{1}{n}
	\leq \frac12 \left( u_n^2 + \frac{1}{n^2} \right),
\end{equation*}
加之级数\(\sum_{n=1}^\infty \frac{1}{n^2}\)收敛,
所以级数\(\sum_{n=1}^\infty \abs{\frac{u_n}{n}}\)收敛.
\end{proof}
\end{example}

\section{无穷乘积}
\subsection{无穷乘积的概念}
\begin{definition}
%@see: 《数学分析(第二版 下册)》(陈纪修) P45 定义9.5.1
设\(\{p_n\}\)是数列.
定义:\begin{equation*}
	\prod_{n=1}^\infty p_n
	\defeq
	\lim_{n\to\infty} \prod_{k=1}^n p_k,
\end{equation*}
称之为\DefineConcept{无穷乘积}.
把\(\prod_{k=1}^n p_k\)称为“无穷乘积\(\prod_{n=1}^\infty p_n\)的\DefineConcept{部分积}”.
把\(p_n\)称为“无穷乘积\(\prod_{n=1}^\infty p_n\)的\DefineConcept{一般项}”.

设\(\{P_n\}\)是无穷乘积\(\prod_{n=1}^\infty p_n\)的部分积数列,
这里\begin{equation*}
	P_n = \prod_{k=1}^n p_k
	\quad(n=1,2,\dotsc).
\end{equation*}
如果\(\{P_n\}\)收敛于一个非零有限数\(P\),
则称“无穷乘积\(\prod_{n=1}^\infty p_n\)~\DefineConcept{收敛}”,
并且称“\(P\)是无穷乘积\(\prod_{n=1}^\infty p_n\)的\DefineConcept{积}”,
“无穷乘积\(\prod_{n=1}^\infty p_n\)收敛于\(P\)”.
如果\(\{P_n\}\)发散,或者\(\{P_n\}\)收敛于\(0\),
则称“无穷乘积\(\prod_{n=1}^\infty p_n\)~\DefineConcept{发散}”.
\end{definition}

\begin{theorem}\label{theorem:无穷乘积.无穷乘积收敛的必要条件}
%@see: 《数学分析(第二版 下册)》(陈纪修) P46 定理9.5.1
如果无穷乘积\(\prod_{n=1}^\infty p_n\)收敛,则\begin{itemize}
	\item \(\lim_{n\to\infty} p_n = 1\);
	\item \(\lim_{m\to\infty} \prod_{n=m+1}^\infty p_n = 1\).
\end{itemize}
%TODO proof
\end{theorem}

\subsection{无穷乘积与无穷级数的联系}
\cref{theorem:无穷乘积.无穷乘积收敛的必要条件} 告诉我们,
无穷乘积\(\prod_{n=1}^\infty p_n\)收敛的必要条件是\(\lim_{n\to\infty} p_n = 1\),
因此必定存在正整数\(N\),当\(n>N\)时,成立\(p_n>0\).
由于无穷乘积的敛散性与它的前\(N\)项非零因子无关,
所以在讨论无穷乘积\(\prod_{n=1}^\infty p_n\)的敛散性问题时,我们都假定\(p_n>0\),
如此,我们便可以对\(p_n\)取对数.
\begin{theorem}\label{theorem:无穷乘积.无穷乘积收敛性与无穷级数收敛性的联系}
%@see: 《数学分析(第二版 下册)》(陈纪修) P48 定理9.5.2
无穷乘积\(\prod_{n=1}^\infty p_n\)收敛的充分必要条件是
常数项无穷级数\(\sum_{n=1}^\infty \ln p_n\)收敛.
%TODO proof
\end{theorem}

\begin{corollary}\label{theorem:无穷乘积.无穷乘积收敛性与无穷级数收敛性的联系.推论1}
%@see: 《数学分析(第二版 下册)》(陈纪修) P49 推论1
设\(a_n>0\)(或\(a_n<0\)),
则无穷乘积\(\prod_{n=1}^\infty (1+a_n)\)收敛的充分必要条件是
常数项无穷级数\(\sum_{n=1}^\infty a_n\)收敛.
%TODO proof
\end{corollary}
\begin{remark}
如果\(\{a_n\}\)不保持定号,
则常数项无穷级数\(\sum_{n=1}^\infty a_n\)的收敛性
并不能保证无穷乘积\(\prod_{n=1}^\infty (1+a_n)\)的收敛性.
\end{remark}

\begin{corollary}\label{theorem:无穷乘积.无穷乘积收敛性与无穷级数收敛性的联系.推论2}
%@see: 《数学分析(第二版 下册)》(陈纪修) P49 推论2
设常数项无穷级数\(\sum_{n=1}^\infty a_n\)收敛,
则无穷乘积\(\prod_{n=1}^\infty (1+a_n)\)收敛的充分必要条件是
常数项无穷级数\(\sum_{n=1}^\infty a_n^2\)收敛.
%TODO proof
\end{corollary}
\begin{remark}
%@see: 《数学分析(第二版 下册)》(陈纪修) P50
\cref{theorem:无穷乘积.无穷乘积收敛性与无穷级数收敛性的联系.推论2} 的叙述不能改为
“无穷乘积\(\prod_{n=1}^\infty (1+a_n)\)收敛的充分必要条件是
常数项无穷级数\(\sum_{n=1}^\infty a_n\)
与\(\sum_{n=1}^\infty a_n^2\)收敛”.
%@see: 《数学分析(第二版 下册)》(陈纪修) P54 习题 7.
实际上,只要取\begin{equation*}
	\def\arraystretch{1.5}
	a_n = \left\{ \begin{array}{cl}
		-\frac1{\sqrt{n}}, & \text{$n$是奇数}, \\
		\frac1{\sqrt{n}}+\frac1n\left(1+\frac1{\sqrt{n}}\right), & \text{$n$是偶数},
	\end{array} \right.
\end{equation*}
就能看出,虽然无穷乘积\(\prod_{n=1}^\infty\)收敛,
但是\(\sum_{n=1}^\infty a_n\)
与\(\sum_{n=1}^\infty a_n^2\)却都是发散的.
\end{remark}
\begin{proposition}
%@see: 《数学分析(第二版 下册)》(陈纪修) P50
设常数项无穷级数\(\sum_{n=1}^\infty a_n\)收敛,
而\(\sum_{n=1}^\infty a_n^2 = +\infty\),
则无穷乘积\(\prod_{n=1}^\infty (1+a_n)\)发散于\(0\).
%TODO proof 证明需利用\cref{theorem:无穷乘积.无穷乘积收敛性与无穷级数收敛性的联系.推论2}
\end{proposition}

\subsection{无穷乘积绝对收敛的概念}
\begin{definition}
%@see: 《数学分析(第二版 下册)》(陈纪修) P50 定义9.5.2
如果常数项无穷级数\(\sum_{n=1}^\infty p_n\)绝对收敛,
则称“无穷乘积\(\prod_{n=1}^\infty p_n\)~\DefineConcept{绝对收敛}”.
\end{definition}

%@see: 《数学分析(第二版 下册)》(陈纪修) P50
显然,绝对收敛的无穷乘积必定收敛.

由于绝对收敛级数具有可交换性,
可知绝对收敛的无穷乘积也具有可交换性.
然而,收敛但非绝对收敛的无穷乘积,不一定具有可交换性.

\begin{theorem}
%@see: 《数学分析(第二版 下册)》(陈纪修) P50 定理9.5.3
设\(a_n>-1\ (n=1,2,\dotsc)\),
则下列三个命题等价:\begin{itemize}
	\item 无穷乘积\(\prod_{n=1}^\infty (1+a_n)\)绝对收敛,
	\item 无穷乘积\(\prod_{n=1}^\infty (1+\abs{a_n})\)收敛,
	\item 常数项无穷级数\(\sum_{n=1}^\infty \abs{a_n}\)收敛.
\end{itemize}
%TODO proof
\end{theorem}

\begin{example}\label{example:无穷乘积.斯特林公式}
%@see: 《数学分析(第二版 下册)》(陈纪修) P51 例9.5.5
证明斯特林公式\begin{equation}
	n! \sim \sqrt{2\pi n} \left( \frac{n}{e} \right)^n
	\quad(n\to\infty).
\end{equation}
%TODO proof
\end{example}

\section{本章总结}
\subsection*{边缘分布,随机变量的独立性}
%\cref{theorem:多维随机变量及其分布.联合密度、边缘密度的关系}
设\(F(x,y)\)为二维随机变量\((X,Y)\)的二维分布函数,
则\(X\)与\(Y\)的边缘分布函数分别为\begin{gather*}
	F_X(x) = F(x,+\infty)
	\quad(-\infty < x < +\infty), \\
	F_Y(y) = F(+\infty,y)
	\quad(-\infty < x < +\infty).
\end{gather*}

%\cref{definition:多维随机变量及其分布.随机变量的独立性}
设\(\AutoTuple{X}{n}\)是\(n\)维随机变量.
若对任意\(n\)个实数\(\AutoTuple{x}{n}\),
\(n\)个事件\((X_1 \leq x_1),\allowbreak\dotsc,\allowbreak(X_n \leq x_n)\)相互独立,
即有\begin{equation*}
	P(X_1 \leq x_1,\dotsc,X_n \leq x_n)
	= \prod_{i=1}^n P(X_i \leq x_i)
	= P(X_1 \leq x_1) \dotsm P(X_n \leq x_n)
\end{equation*}
或\begin{equation*}
	F(x_1,\dotsc,x_n)
	= \prod_{i=1}^n F_i(x_i)
	= F_1(x_1) \dotsm F_n(x_n),
\end{equation*}
其中\(F\)是\(n\)维随机变量\(\AutoTuple{X}{n}\)的联合分布函数,
而\(F_1,\dotsc,F_n\)分别是\(X_1,\dotsc,X_n\)的边缘分布函数,
则称“\(n\)个随机变量\(\AutoTuple{X}{n}\)相互独立”;
否则称“\(n\)个随机变量\(\AutoTuple{X}{n}\)不相互独立”
或“\(n\)个随机变量\(\AutoTuple{X}{n}\)相依”.

\(n\)个事件两两独立是它们相互独立的必要不充分条件.

%\cref{theorem:多维随机变量及其分布.两个离散型随机变量相互独立的充分必要条件}
设\((X,Y)\)是二维离散型随机变量,有二维概率分布\begin{equation*}
	p_{ij} = P(X=x_i,Y=y_j), \quad i,j=1,2,\dotsc,
\end{equation*}
和边缘分布\begin{gather*}
	p_{i*} = \sum_j p_{ij},
	\quad i=1,2,\dotsc; \\
	p_{*j} = \sum_i p_{ij},
	\quad j=1,2,\dotsc,
\end{gather*}
则随机变量\(X\)与\(Y\)相互独立的充分必要条件是:\begin{equation*}
	p_{ij} = p_{i*} p_{*j}, \quad i,j=1,2,\dotsc.
\end{equation*}

%\cref{theorem:多维随机变量及其分布.两个连续型随机变量相互独立的充分必要条件}
设二维连续型随机变量\((X,Y)\)的二维密度为\(f(x,y)\),
\(X\)与\(Y\)的边缘密度分别为\(f_X(x)\)和\(f_Y(y)\),
则\begin{align*}
	f_X(x) = \int_{-\infty}^{+\infty} f(x,y) \dd{y}, \\
	f_Y(y) = \int_{-\infty}^{+\infty} f(x,y) \dd{x}.
\end{align*}
而\(X\)与\(Y\)相互独立的充分必要条件是:\begin{equation*}
	f(x,y) = f_X(x) f_Y(y).
\end{equation*}在三个密度函数的公共连续点上成立.

\subsection*{联合分布、边缘分布与条件分布的联系}
设二维连续型随机变量\((X,Y)\)有二维密度\(f(x,y)\),
从而\(X\)及\(Y\)有边缘密度\(f_X(x),f_Y(y)\),
则\begin{gather*}
	F_{X \vert Y}(x \vert y)
	= \int_{-\infty}^x \frac{f(u,y)}{f_Y(y)}\dd{u}
	\quad(-\infty < x < +\infty), \\
	F_{Y \vert X}(y \vert x)
	= \int_{-\infty}^y \frac{f(x,v)}{f_X(x)}\dd{v}
	(\quad y \in \mathbb{R}).
\end{gather*}
\(X\)关于\(Y\)的条件密度函数为\begin{equation*}
	%\cref{equation:多维随机变量及其分布.条件密度、联合密度、边缘密度的关系1}
	f_{X \vert Y}(x \vert y)
	= \frac{f(x,y)}{f_Y(y)}.
\end{equation*}
\(Y\)关于\(X\)的条件密度函数为\begin{equation*}
	%\cref{equation:多维随机变量及其分布.条件密度、联合密度、边缘密度的关系2}
	f_{Y \vert X}(y \vert x)
	= \frac{f(x,y)}{f_X(x)}.
\end{equation*}
反过来,可以利用边缘密度函数和条件密度函数计算联合密度函数:\begin{equation*}
	f(x,y) = f_Y(y) \cdot f_{X \vert Y}(x \vert y)
	= f_X(x) \cdot f_{Y \vert X}(y \vert x).
\end{equation*}

\subsection*{分布的可加性}
%\cref{theorem:多维随机变量及其分布.二项分布的可加性1}
设\(X \sim B(n,p)\),
\(Y \sim B(m,p)\),
且\(X\)与\(Y\)相互独立,
则\begin{equation*}
	X+Y \sim B(n+m,p).
\end{equation*}

%\cref{theorem:多维随机变量及其分布.泊松分布的可加性1}
设\(X \sim P(\lambda_1)\),
\(Y \sim P(\lambda_2)\),
且\(X\)与\(Y\)相互独立,
则\begin{equation*}
	X+Y \sim P(\lambda_1 + \lambda_2).
\end{equation*}

%\cref{theorem:正态分布与自然指数分布族.正态分布的可加性1}
设\(X \sim N(\mu_1,\sigma_1^2)\),
\(Y \sim N(\mu_2,\sigma_2^2)\),
且\(X\)与\(Y\)相互独立,
则\begin{equation*}
	X+Y \sim N(\mu_1+\mu_2,\sigma_1^2+\sigma_2^2).
\end{equation*}

%\cref{theorem:正态分布与自然指数分布族.正态分布的可加性2}
设随机变量\(\AutoTuple{X}{n}\)相互独立,
且\begin{equation*}
	X_i \sim N(\mu_i,\sigma_i^2),
	\quad i=1,2,\dotsc,n,
\end{equation*}
且\(C_1,C_2,\dotsc,C_n\)为常数,
则\begin{equation*}
	\sum_{i=1}^n {C_i X_i}
	\sim N\left(
	\sum_{i=1}^n {C_i \mu_i},
	\sum_{i=1}^n {C_i^2 \sigma_i^2}
	\right).
\end{equation*}


\chapter{函数项级数}
\section{函数项级数}
\subsection{函数项级数的概念}
\begin{definition}\label{definition:无穷级数.实函数项级数的概念}
%@see: 《数学分析(第二版 下册)》(陈纪修) P55 定义10.1.1
%@see: 《数学分析(第7版 第二卷)》(卓里奇) P299 定义1
%@see: 《数学分析(第7版 第二卷)》(卓里奇) P299 定义2
%@see: 《数学分析(第7版 第二卷)》(卓里奇) P299 定义4
给定一个定义在区间\(I \subseteq \mathbb{R}\)上的函数列\begin{equation*}
	u_1,u_2,\dotsc,u_n,\dotsc,
\end{equation*}
则由该函数列构成的表达式\begin{equation*}
	u_1(x)+u_2(x)+\dotsb+u_n(x)+\dotsb
\end{equation*}
称为“定义在区间\(I\)上的\DefineConcept{函数项无穷级数}
(infinite series with function terms)”,
简称\DefineConcept{函数项级数},
或者进一步简称为\DefineConcept{级数},
记作\begin{equation*}
	x \mapsto \sum_{n=1}^\infty u_n(x),
\end{equation*}
或\begin{equation*}
	\sum_{n=1}^\infty u_n.
\end{equation*}
\end{definition}
\begin{remark}
函数项无穷级数本质上是一个映射.
它的定义域是各项函数\(u_k\)的共同定义域\begin{equation*}
	I \subseteq \bigcap\Set{ \dom u_k \given k=1,2,\dotsc }.
\end{equation*}
于是,对于每一个确定的值\(x_0 \in I\),
只要在函数项无穷级数的表达式\(\sum_{n=1}^\infty u_n(x)\)中,用\(x_0\)代入\(x\),
就能得到一个常数项无穷级数\(\sum_{n=1}^\infty u_n(x_0)\).
只不过,并非所有\(x_0\)都能保证
常数项无穷级数\(\sum_{n=1}^\infty u_n(x_0)\)是收敛的.
\end{remark}

\begin{definition}
如果常数项无穷级数\(\sum_{n=1}^\infty u_n(x_0)\)收敛,
则称“函数项级数\(\sum_{n=1}^\infty u_n\)在点\(x_0\)收敛”
或“点\(x_0\)是函数项级数\(\sum_{n=1}^\infty u_n\)的\DefineConcept{收敛点}
(point of convergence)”.

如果常数项无穷级数\(\sum_{n=1}^\infty u_n(x_0)\)发散,
就称“函数项级数\(\sum_{n=1}^\infty u_n\)在点\(x_0\)发散”
或“点\(x_0\)是函数项级数\(\sum_{n=1}^\infty u_n\)的\DefineConcept{发散点}
(point of divergence)”.

函数项级数\(\sum_{n=1}^\infty u_n\)的收敛点的全体称为
“函数项级数\(\sum_{n=1}^\infty u_n\)的\DefineConcept{收敛域}(domain of convergence)”,
记作\(\dom \sum_{n=1}^\infty u_n\).

函数项级数\(\sum_{n=1}^\infty u_n\)的发散点的全体称为
“函数项级数\(\sum_{n=1}^\infty u_n\)的\DefineConcept{发散域}(domain of divergence)”.
\end{definition}

\begin{definition}
记\(D \defeq \dom\sum_{n=1}^\infty u_n\).
把函数\begin{equation*}
	S\colon D\to\mathbb{R},
	x \mapsto \sum_{n=1}^\infty u_n(x),
\end{equation*}称为“函数项级数\(\sum_{n=1}^\infty u_n\)的\DefineConcept{和函数}”.

由于这个函数是通过逐点定义的方式得到的,
因此我们称“函数项级数\(\sum_{n=1}^\infty u_n\)在\(D\)上
具有\DefineConcept{点态收敛性}(pointwise convergence)”,
“函数项级数\(\sum_{n=1}^\infty u_n\)在\(D\)上\DefineConcept{点态收敛}于\(S\)”.

把函数\begin{equation*}
	S_n\colon I\to\mathbb{R},
	x \mapsto \sum_{k=1}^n u_k(x)
\end{equation*}称为“函数项级数\(\sum_{n=1}^\infty u_n\)的\DefineConcept{部分和函数}(partial sum)”.
\end{definition}

\begin{definition}
当函数项级数\(\sum_{n=1}^\infty u_n\)收敛时,
把函数\begin{equation*}
	R_n\colon D\to\mathbb{R},
	x \mapsto S(x) - S_n(x),
\end{equation*}称为“函数项级数\(\sum_{n=1}^\infty u_n\)的\DefineConcept{余项函数}”.
\end{definition}

\begin{definition}
%@see: 《数学分析(第7版 第二卷)》(卓里奇) P299 定义3
设函数列\(\{S_n\}\)在集合\(D\)上点态收敛于函数\(S\),
即对任意\(x \in D\)成立\begin{equation*}
	\lim_{n\to\infty} S_n(x) = S(x),
\end{equation*}
则把\(S\)称为“函数列\(\{S_n\}\)的\DefineConcept{极限函数}”,
记作\begin{equation*}
	S_n \to S\ (n\to\infty,x \in D)
	\quad\text{或}\quad
	S_n \overset{D}\to S,
\end{equation*}
不强调区间\(D\)时,也可以记作\(S_n \to S\).
\end{definition}

\begin{proposition}
%@see: 《数学分析(第二版 下册)》(陈纪修) P56
设\(\{S_n\}\)是函数项级数\(\sum_{n=1}^\infty u_n\)的部分和函数列,
则\begin{equation*}
	\Set{ x \given \text{数列$\{S_n(x)\}$收敛} } = D.
\end{equation*}
\end{proposition}
\begin{proposition}
%@see: 《数学分析(第二版 下册)》(陈纪修) P56
设\(S\)是函数项级数\(\sum_{n=1}^\infty u_n\)的和函数,
\(S_n\)是函数项级数\(\sum_{n=1}^\infty u_n\)的部分和函数,
则在收敛域\(D\)上有\begin{equation*}
	\lim_{n\to\infty} S_n(x) = S(x).
\end{equation*}
\end{proposition}
\begin{proposition}
设\(R_n\)是函数项级数\(\sum_{n=1}^\infty u_n\)的余项函数,
则在收敛域\(D\)上有\begin{equation*}
	\lim_{n\to\infty} R_n(x) = 0.
\end{equation*}
\end{proposition}
\begin{remark}
可以看出,函数项级数\(\sum_{n=1}^\infty u_n\)与函数列\(\{S_n\}\)的收敛性在本质上完全是一回事.
\end{remark}

\begin{example}
%@see: 《数学分析(第二版 下册)》(陈纪修) P55 例10.1.1
利用常数项无穷级数审敛法和\cref{definition:无穷级数.实函数项级数的概念},可知下述结论:
\begin{itemize}
	\item \(\sum_{n=1}^\infty x^n\)的收敛域为\((-1,1)\),
	和函数为\(S(x) = \frac{x}{1-x}\).
	\item \(\sum_{n=1}^\infty \frac{x^n}{n}\)的收敛域为\([-1,1)\).
	\item \(\sum_{n=1}^\infty \frac{x^n}{n^2}\)的收敛域为\([-1,1]\).
	\item \(\sum_{n=1}^\infty \frac{x^n}{n!}\)的收敛域是\((-\infty,+\infty)\).
	\item \(\sum_{n=1}^\infty n! x^n\)的收敛域是\(\{0\}\).
	\item \(\sum_{n=1}^\infty e^{-nx}\)的收敛域为\((0,+\infty)\),
	和函数为\(S(x) = \frac1{e^x-1}\).
\end{itemize}
\end{example}

\begin{example}
%@see: 《数学分析教程(第3版 上册)》(史济怀) P213 问题15.1 1.
设函数项级数\(\sum_{n=1}^\infty u_n(x)\)在有界闭区间\([a,b]\)上收敛于\(S(x)\),
且\(u_n\ (n=1,2,\dotsc)\)都是\([a,b]\)上的非负连续函数.
证明:\(S\)必在\([a,b]\)上有最小值.
\begin{proof}
记\(S_n(x) = \sum_{k=1}^n u_k(x)\),
则有\begin{equation*}
	0 \leq S_n(x) \leq S_{n+1}(x) \leq S(x),
\end{equation*}
所以\(S\)在\([a,b]\)上有下确界\(\alpha\).
于是对于每个正整数\(n\),
存在\(x_n\in[a,b]\),
使得\begin{equation*}
	\alpha \leq S(x_n) < \alpha + \frac1n,
\end{equation*}
即\begin{equation*}
	\lim_{n\to\infty} S(x_n) = \alpha.
\end{equation*}
因为\(x_n\in[a,b]\),
所以存在子列\(\{x_{n_k}\}\)使得\begin{equation*}
	\lim_{k\to\infty} x_{n_k} = x_0 \in [a,b].
\end{equation*}
我们要证明\(S(x_0) = \alpha\).
对任意的正整数\(m\),
有\begin{equation*}
	S_m(x_{n_k})
	\leq S(x_{n_k})
	< \alpha + \frac1{n_k}.
\end{equation*}
令\(n\to\infty\),
由于\(S_m\)在点\(x_0\)连续,
所以\begin{equation*}
	S_m(x_0) \leq \alpha.
\end{equation*}
再令\(m\to\infty\),
便得\begin{equation*}
	S(x_0) \leq \alpha.
\end{equation*}
但是\(S(x_0) \geq \alpha\),
所以\(S(x_0) = \alpha\).
\end{proof}
\end{example}

\subsection{函数项级数和函数列的基本问题}
%@see: 《数学分析(第二版 下册)》(陈纪修) P56
通过前面的学习我们已经知道,
若有限个函数\(u_1,u_2,\dotsc,u_n\)在\(D\)上有定义且具有某种分析性质,
如连续性、可导性和黎曼可积性,
则它们的和函数\begin{equation*}
	u_1(x) + u_2(x) + \dotsb + u_n(x)
\end{equation*}在\(D\)上仍保持同样的分析性质,
且其和函数的极限、导数或黎曼积分
可以通过对每个函数分别求极限、导数或黎曼积分后再求和来得到,
即\begin{gather*}
	\lim_{x \to x_0} [u_1(x) + u_2(x) + \dotsb + u_n(x)]
	= \lim_{x \to x_0} u_1(x)
	+ \lim_{x \to x_0} u_2(x)
	+ \dotsb
	+ \lim_{x \to x_0} u_n(x), \\
	\dv{x} [u_1(x) + u_2(x) + \dotsb + u_n(x)]
	= \dv{x} u_1(x)
	+ \dv{x} u_2(x)
	+ \dotsb
	+ \dv{x} u_n(x), \\
	\int_a^b [u_1(x) + u_2(x) + \dotsb + u_n(x)] \dd{x}
	= \int_a^b u_1(x) \dd{x}
	+ \int_a^b u_2(x) \dd{x}
	+ \dotsb
	+ \int_a^b u_n(x) \dd{x}.
\end{gather*}

在研究函数项级数时,我们面对的是无限个函数之和.
它们的和函数\(S\)大多是未知的,也就是说,
我们只能借助函数\(u_n\)的分析性质来间接地获得\(S\)的分析性质.
那么我们自然希望上述运算法则可以在一定条件下推广到无限个函数求和的情况.

这个问题是函数项级数和函数列研究中的基本问题,
其实质是极限、求导、求积分运算与无限求和运算在什么条件下可以交换次序.
由于求导、求积分与无限求和均可看作特殊的极限运算,因此更一般地,
可以将其统一视为两种极限运算的交换次序.
下面我们将会看到,仅要求\(\sum_{n=1}^\infty u_n(x)\)在\(D\)上点态收敛是不够的.

\begin{definition}
%@see: 《数学分析(第二版 下册)》(陈纪修) P57
设\(S\)是函数项级数\(\sum_{n=1}^\infty u_n\)的和函数.
如果当\(u_n\)在\(D\)上连续时,函数\(S\)也在\(D\)上连续,
并且成立\begin{equation*}
	\lim_{x \to x_0} \sum_{n=1}^\infty u_n(x)
	= \sum_{n=1}^\infty \lim_{x \to x_0} u_n(x),
\end{equation*}
则称“极限运算与无限求和运算可以交换次序”
或“函数项级数\(\sum_{n=1}^\infty u_n\)可以\DefineConcept{逐项求极限}”.
\end{definition}
对于函数列\(\{S_n\}\)而言,相应的结论是
极限函数\(S(x) = \lim_{n\to\infty} S_n(x)\)也在\(D\)上连续,
并且成立\begin{equation*}
	\lim_{x \to x_0} \lim_{n\to\infty} S_n(x)
	= \lim_{n\to\infty} \lim_{x \to x_0} S_n(x),
\end{equation*}
即两种极限运算可以交换次序.

下面的例子说明,在点态收敛的情况下,函数项级数不一定可以逐项求极限.
\begin{example}\label{example:函数项级数.点态收敛情况下和函数可能不连续}
%@see: 《数学分析(第二版 下册)》(陈纪修) P58 例10.1.2
%@see: 《数学分析(第7版 第二卷)》(卓里奇) P299 例1
设\(S_n(x) = x^n\),
则函数列\(\{S_n\}\)在区间\((-1,1]\)上收敛,
极限函数为\begin{equation*}
	S(x) = \lim_{n\to\infty} S_n(x)
	= \left\{ \begin{array}{ll}
		0, & -1<x<1, \\
		1, & x=1.
	\end{array} \right.
\end{equation*}
虽然对于任意正整数\(n\),函数\(S_n\)在\((-1,1]\)上连续(也是可导的),
但极限函数\(S\)在\(x=1\)不连续(当然更谈不上在\(x=1\)可导).
\end{example}

\begin{definition}
%@see: 《数学分析(第二版 下册)》(陈纪修) P58
设\(S\)是函数项级数\(\sum_{n=1}^\infty u_n\)的和函数.
如果当\(u_n\)在\(D\)上可导时,函数\(S\)也在\(D\)上可导,
并且成立\begin{equation*}
	\dv{x} \sum_{n=1}^\infty u_n(x)
	= \sum_{n=1}^\infty \dv{x} u_n(x),
\end{equation*}
则称“求导运算与无限求和运算可以交换次序”
或“函数项级数\(\sum_{n=1}^\infty u_n\)可以\DefineConcept{逐项求导}”.
\end{definition}
对于函数列\(\{S_n\}\)而言,相应的结论是
极限函数\(S(x) = \lim_{n\to\infty} S_n(x)\)也在\(D\)上可导,
并且成立\begin{equation*}
	\dv{x} \lim_{n\to\infty} S_n(x)
	= \lim_{n\to\infty} \dv{x} S_n(x),
\end{equation*}
即求导运算与极限运算可以交换次序.

\cref{example:函数项级数.点态收敛情况下和函数可能不连续} 已说明:
在点态收敛情况下,和函数(或极限函数)可能不可导.
下面将看到,即使和函数(或极限函数)可导,上述两个等式也不一定成立.
\begin{example}
%@see: 《数学分析(第二版 下册)》(陈纪修) P58 例10.1.3
设\(S_n(x) = \frac{\sin nx}{\sqrt{n}}\),
则函数列\(\{S_n\}\)在区间\((-\infty,+\infty)\)上收敛,
极限函数为\(S(x) = 0\),从而导函数\(S'(x) = 0\).
由于\begin{equation*}
	S_n'(x) = \sqrt{n} \cos nx,
\end{equation*}
因此\(S_n\)的导函数所构成的序列\(\{S_n'\}\)并不收敛于\(S'\),
例如当\(x=0\)时\begin{equation*}
	S_n'(0) = \sqrt{n}\to+\infty\ (n\to\infty).
\end{equation*}
\end{example}

\begin{definition}
%@see: 《数学分析(第二版 下册)》(陈纪修) P58
设\(S\)是函数项级数\(\sum_{n=1}^\infty u_n\)的和函数.
如果当\(u_n\)在闭区间\([a,b] \subseteq D\)上黎曼可积时,函数\(S\)也在\([a,b]\)上黎曼可积,
并且成立\begin{equation*}
	\int_a^b \sum_{n=1}^\infty u_n(x) \dd{x}
	= \sum_{n=1}^\infty \int_a^b u_n(x) \dd{x},
\end{equation*}
则称“求积分运算与无限求和运算可以交换次序”
或“函数项级数\(\sum_{n=1}^\infty u_n\)可以\DefineConcept{逐项求积分}”.
\end{definition}
对于函数列\(\{S_n\}\)而言,相应的结论是
极限函数\(S(x) = \lim_{n\to\infty} S_n(x)\)也在闭区间\([a,b]\)上黎曼可积,
并且成立\begin{equation*}
	\int_a^b \lim_{n\to\infty} S_n(x) \dd{x}
	= \lim_{n\to\infty} \int_a^b S_n(x) \dd{x},
\end{equation*}
即求积分运算与极限运算可以交换次序.

下面将看到,在点态收敛情况下,
和函数(或极限函数)可能不可积;
即使黎曼可积,上述两个等式也不一定成立.
\begin{example}
%@see: 《数学分析(第二版 下册)》(陈纪修) P59 例10.1.4
设函数\(S_n\colon[0,1]\to\{0,1\}\)满足\begin{equation*}
	S_n(x) = \left\{ \begin{array}{cl}
		1, & \text{$n! \cdot x$是整数},  \\
		0, & \text{其他}.
	\end{array} \right.
\end{equation*}
显然,对于每一个正整数\(n\),函数\(S_n\)在\([0,1]\)上有界,
至多只有有限个不连续点,因而是可积的.

但是,当\(x\)是无理数时,对一切正整数\(n\),有\(S_n(x) = 0\),
因此\(S(x) = \lim_{n\to\infty} S_n(x) = 0\);
当\(x\)是有理数时,不妨设\(x\)等于既约分数\(q/p\),
即\(p\in\mathbb{N}^+,
q\in\mathbb{N},
q \leq p\),
那么对于\(n \geq p\),有\(S_n(x) = 1\),
因此\(S(x) = \lim_{n\to\infty} S_n(x) = 1\).
所以\(\{S_n\}\)的极限函数\(S\)就是我们熟知的狄利克雷函数,
它在\([0,1]\)上是不可积的.
\end{example}
\begin{example}
%@see: 《数学分析(第二版 下册)》(陈纪修) P59 例10.1.5
设\(S_n(x) = nx(1-x^2)^n\),
则\(\{S_n\}\)在区间\([0,1]\)上收敛于极限函数\(S(x) = 0\).
但是\begin{align*}
	\int_0^1 S_n(x) \dd{x}
	&= \int_0^1 n x (1-x^2)^n \dd{x}
	= -\frac{n}{2} \int_0^1 (1-x^2)^n \dd(1-x^2) \\
	&= \frac{n}{2(n+1)}
	\not\to \int_0^1 S(x) \dd{x}
	\quad(n\to\infty).
\end{align*}
\end{example}
这些例子说明,为了解决这类交换运算次序问题,
需要引进比“点态收敛”要求更强的新的收敛概念.

\subsection{函数项级数或函数列的一致收敛性}
%@see: 《数学分析(第二版 下册)》(陈纪修) P59
\begin{definition}\label{definition:无穷级数.函数项级数的一致收敛性}
%@see: 《数学分析(第二版 下册)》(陈纪修) P60 定义10.1.2
%@see: 《高等数学(第六版 上册)》 P294 定义
设定义在区间\(D\)上的函数列\(\{S_n\}\).
若对任意给定\(\epsilon>0\),
存在仅与\(\epsilon\)有关的正整数\(N\),
当\(n>N\)时,
对\(D\)上的一切\(x\),
成立\begin{equation*}
	\abs{S(x) - S_n(x)} < \epsilon,
\end{equation*}
则称“函数列\(\{S_n\}\)在\(D\)上
具有\DefineConcept{一致收敛性}(uniform convergence)”,
%@see: https://mathworld.wolfram.com/UniformConvergence.html
“函数列\(\{S_n\}\)在\(D\)上\DefineConcept{一致收敛}于\(S\)
(\(\{S_n\}\) \emph{converges uniformly} to \(S\) on \(D\))”,
记为\begin{equation*}
	S_n \UniformlyConverge S\ (n\to\infty,x \in D)
	\quad\text{或}\quad
	S_n \UniformlyConverge[D] S,
\end{equation*}
不强调区间\(D\)时,也可以记为\(S_n \UniformlyConverge S\).

若函数项级数\(\sum_{n=1}^\infty u_n(x)\ (x \in D)\)的
部分和函数列\(S_n = \sum_{k=1}^n u_k(x)\)在\(D\)上一致收敛于\(S\),
则称“函数项级数\(\sum_{n=1}^\infty u_n\)在\(D\)上一致收敛于\(S\)”.
\end{definition}
我们可以用“\(\epsilon-\delta\)”语言将上述定义表述为\begin{equation*}
	S_n \UniformlyConverge[D] S
	\defiff
	(\forall\epsilon>0)
	(\exists N\in\mathbb{N})
	(\forall n\in\mathbb{N})
	[
		n>N
		\implies
		(\forall x \in D)
		[\abs{S_n(x) - S(x)} < \epsilon]
	].
\end{equation*}

如\cref{figure:无穷级数.函数项级数一致收敛的几何解释},
以上函数项级数和函数列一致收敛的定义在几何上可解释为:
对于任意给定\(\epsilon>0\),存在正整数\(N\),
当\(n>N\)时,函数\(y = S_n(x)\ (x \in D)\)的图像
都落在带状区域\begin{equation*}
	\Set{ (x,y) \given x \in D, S(x) - \epsilon < y < S(x) + \epsilon }
\end{equation*}之中.

\begin{figure}[htb]
	\centering
	\begin{tikzpicture}
		\draw[->](0,0)node[below left]{\(O\)} -- (7,0)node[below]{\(x\)};
		\draw[->](0,0) -- (0,4)node[left]{\(y\)};
		\draw[dashed](1,0) -- (1,3) (6,0) -- (6,3);
		\draw(1,1)--+(-.4,0);
		\draw(1,2)--+(-.4,0);
		\draw(1,3)--+(-.4,0);
		\begin{scope}[<->,xshift=-.2cm]
			\draw(1,1) -- (1,2)node[midway,left]{\(\epsilon\)};
			\draw(1,2) -- (1,3)node[midway,left]{\(\epsilon\)};
		\end{scope}
		\begin{scope}[blue]
			\draw[dashed] (1,1) .. controls (2,2) and (3,0) .. (4,1) .. controls (5,2) and (6,1) .. (6,1);
			\draw[yshift=1cm] (1,1) .. controls (2,2) and (3,0) .. (4,1) .. controls (5,2) and (6,1) .. (6,1)node[right]{\(y = S(x)\)};
			\draw[yshift=2cm,dashed] (1,1) .. controls (2,2) and (3,0) .. (4,1) .. controls (5,2) and (6,1) .. (6,1);
			\draw(3.3,3.5)node{\(y = S(x) + \epsilon\)};
			\draw(3.3,0.4)node{\(y = S(x) - \epsilon\)};
		\end{scope}
		\draw(1,1.5) .. controls (1.7,5) and (2.5,0) .. (3,2) .. controls (3.3,3.5) and (4.2,1) .. (6,2.6)node[right]{\(y = S_n(x)\)};
	\end{tikzpicture}
	\caption{函数项级数一致收敛的几何解释}
	\label{figure:无穷级数.函数项级数一致收敛的几何解释}
\end{figure}

\begin{proposition}
%@see: 《数学分析(第二版 下册)》(陈纪修) P60 推论10.1.1
若函数项级数\(\sum_{n=1}^\infty u_n\)在\(D\)上一致收敛,
则函数列\(\{u_n\}\)在\(D\)上一致收敛于\(u(x) \equiv 0\).
\begin{proof}
由\cref{definition:无穷级数.函数项级数的一致收敛性} 立即可得.
\end{proof}
\end{proposition}

由于函数项级数的一致收敛性本质上就是部分和函数列的一致收敛性,
所以下面我们仅对函数列举例讨论.

\begin{example}
%@see: 《高等数学(第六版 上册)》 P295 例2
设\(S_n(x) = \frac1{x+n}\).
研究\(\{S_n\}\)在区间\([0,+\infty)\)上的一致收敛性.
\begin{solution}
显然\(S(x) = \lim_{n\to\infty} S_n(x) = 0\).
对于\(\forall\epsilon>0\),
要使\begin{equation*}
	\abs{S(x) - S_n(x)}
	= \frac1{x+n}
	\leq \frac1{n}
	< \epsilon,
\end{equation*}
可以取\(N \geq \ceil*{\frac1{\epsilon}}\),
当\(n>N\)时,
就有\begin{equation*}
	\abs{S(x) - S_n(x)}
	< \epsilon
\end{equation*}对一切\(x\in[0,+\infty)\)成立,
因此\(\{S_n\}\)在\([0,+\infty)\)上一致收敛于\(S(x)\equiv0\).
\end{solution}
\end{example}

\begin{example}\label{example:函数项级数.一致收敛的函数列2}
%@see: 《数学分析(第二版 下册)》(陈纪修) P60 例10.1.6
设\(S_n(x) = \frac{x}{1+n^2x^2}\),
则\(\{S_n\}\)在\((-\infty,+\infty)\)上收敛于极限函数\(S(x) = 0\).
这是因为\begin{equation*}
	\abs{S_n(x) - S(x)}
	= \frac{\abs{x}}{1+n^2x^2}
	\leq \frac1{2n},%TODO QUESTION: 怎么放缩成这个只与\(n\)有关的分式的?
\end{equation*}
所以对任意给定的\(\epsilon>0\),只要取\(N = \floor*{\frac1{2\epsilon}}\),
当\(n>N\)时,就有\begin{equation*}
	\abs{S_n(x) - S(x)} \leq \frac1{2n} < \epsilon
\end{equation*}对一切\(x\in(-\infty,+\infty)\)成立.

\begin{figure}[htb]
%@see: 《数学分析(第二版 下册)》(陈纪修) P61 图10.1.2
	\centering
	\begin{tikzpicture}
		\begin{axis}[
			name=Plt,
			xscale=1.5,
			xmin=-2.2,xmax=2.2,
			ymin=-.6,ymax=.6,
			axis lines=middle,
			xlabel=$x$,
			ylabel=$y$,
		]
			\addplot[color=red,samples=50,smooth,domain=-2:2]
				{x/(1+x^2)};\label{pgfplots:函数项级数.函数图像1}
			\addplot[color=orange,samples=50,smooth,domain=-2:2]
				{x/(1+4*x^2)};\label{pgfplots:函数项级数.函数图像2}
			\addplot[color=green,samples=50,smooth,domain=-2:2]
				{x/(1+9*x^2)};\label{pgfplots:函数项级数.函数图像3}
			\addplot[color=blue,samples=50,smooth,domain=-2:2]
				{x/(1+25*x^2)};\label{pgfplots:函数项级数.函数图像4}
		\end{axis}
		\node[draw,fill=white,inner sep=0pt,right=1em]
		at(Plt.east){
			\small\begin{tblr}{cl}
				\ref{pgfplots:函数项级数.函数图像1} & \(n=1\) \\
				\ref{pgfplots:函数项级数.函数图像2} & \(n=2\) \\
				\ref{pgfplots:函数项级数.函数图像3} & \(n=3\) \\
				\ref{pgfplots:函数项级数.函数图像4} & \(n=5\) \\
			\end{tblr}
		};
	\end{tikzpicture}
	\caption{}
	\label{figure:函数项级数.图1}
\end{figure}

从几何上看(\cref{figure:函数项级数.图1}),
对任意给定\(\epsilon>0\),只要取\(N = \floor*{\frac1{2\epsilon}}\),
当\(n>N\)时,函数\(S_n\)的图像都落在带状区域\begin{equation*}
	\Set{ (x,y) \given \abs{y} < \epsilon }
\end{equation*}中,这正是一致收敛的几何描述.
\end{example}

\begin{example}\label{example:函数项级数.不一致收敛的函数列1}
%@see: 《数学分析(第二版 下册)》(陈纪修) P62 例10.1.7
%@see: 《高等数学(第六版 上册)》 P295 例3
设\(S_n(x) = x^n\),
考察\(\{S_n\}\)在区间\([0,1)\)上的一致收敛性.
显然\(S(x) = \lim_{n\to\infty} S_n(x) = 0\).
对任意给定\(\epsilon\in(0,1)\),
要使\begin{equation*}
	\abs{S_n(x) - S(x)} = x^n < \epsilon,
\end{equation*}
必须\begin{equation*}
	n > \frac{\ln\epsilon}{\ln x},
\end{equation*}
因此\(N = N(x,\epsilon)\)至少应该取\(\floor*{\frac{\ln\epsilon}{\ln x}}\).
由于当\(x\to1^-\)时,\(\frac{\ln\epsilon}{\ln x}\to+\infty\),
因此不可能找到对一切\(x\in[0,1)\)都适用的\(N = N(\epsilon)\),
换言之,\(\{S_n\}\)在\([0,1)\)上不是一致收敛的.

\begin{figure}[htb]
%@see: 《数学分析(第二版 下册)》(陈纪修) P62 图10.1.3
	\centering
	\begin{tikzpicture}
		\begin{axis}[
			name=Plt,
			xmin=0,xmax=1.2,
			ymin=0,ymax=1.2,
			axis equal=true,
			axis lines=middle,
			xlabel=$x$,
			ylabel=$y$,
		]
			\addplot[color=red,samples=50,smooth,domain=0:1]
				{x};\label{pgfplots:函数项级数.函数图像5}
			\addplot[color=orange,samples=50,smooth,domain=0:1]
				{x^2};\label{pgfplots:函数项级数.函数图像6}
			\addplot[color=green,samples=50,smooth,domain=0:1]
				{x^3};\label{pgfplots:函数项级数.函数图像7}
			\addplot[color=blue,samples=50,smooth,domain=0:1]
				{x^4};\label{pgfplots:函数项级数.函数图像8}
			\draw[dashed,black!30](0,1)--(1,1)--(1,0);
		\end{axis}
		\node[draw,fill=white,inner sep=0pt,right=1em]
		at(Plt.east){
			\small\begin{tblr}{cl}
				\ref{pgfplots:函数项级数.函数图像5} & \(n=1\) \\
				\ref{pgfplots:函数项级数.函数图像6} & \(n=2\) \\
				\ref{pgfplots:函数项级数.函数图像7} & \(n=3\) \\
				\ref{pgfplots:函数项级数.函数图像8} & \(n=4\) \\
			\end{tblr}
		};
	\end{tikzpicture}
	\caption{}
	\label{figure:函数项级数.图2}
\end{figure}

从几何上看(\cref{figure:函数项级数.图2}),
对于每个正整数\(n\),函数\(x \mapsto x^n\)的取值范围(即它的值域)都是\([0,1]\),
因此它们的图像不可能落在带状区域\begin{equation*}
	\Set{ (x,y) \given x\in[0,1), 0<y<\epsilon }
\end{equation*}中.
\end{example}

\begin{definition}
%@see: 《数学分析(第二版 下册)》(陈纪修) P62 定义10.1.3
若对于任意给定闭区间\([a,b] \subseteq D\),
函数列\(\{S_n\}\)在\([a,b]\)上一致收敛于\(S\),
则称“\(\{S_n\}\)在\(D\)上\DefineConcept{内闭一致收敛}于\(S\)”.
\end{definition}
显然,在\(D\)上一致收敛的函数列必定在\(D\)上内闭一致收敛,
但其逆命题不成立.
例如,将\cref{example:函数项级数.不一致收敛的函数列1} 中
考察的区间\([0,1)\)缩小为\([0,\rho]\),
其中\(0<\rho<1\)是任意的,
则由\begin{equation*}
	\abs{S_n(x) - S(x)} = x^n < \rho^n,
\end{equation*}
只要取\(N = N(\epsilon) = \floor*{\frac{\ln\epsilon}{\ln\rho}}\),
当\(n>N\)时,就有\begin{equation*}
	\abs{S_n(x) - S(x)} < \rho^n < \epsilon
\end{equation*}对一切\(x\in[0,\rho]\)成立,
即\(\{S_n\}\)在\([0,\rho]\ (\rho<1)\)上是一致收敛的.
也就是说,尽管\(\{S_n\}\)在\([0,1)\)上不是一致收敛的,
但是它是内闭一致收敛的.

这个例子也说明:函数项级数与函数列的一致收敛性与我们考察的区间有关.

下面我们建立关于一致收敛的两个充分必要条件,它们将有助于对一致收敛性进行判断.
\begin{theorem}\label{theorem:函数项级数.一致收敛的充分必要条件1}
%@see: 《数学分析(第二版 下册)》(陈纪修) P63 定理10.1.1
设函数列\(\{S_n\}\)在集合\(D\)上点态收敛于函数\(S\),
则\(\{S_n\}\)在\(D\)上一致收敛于\(S\)的充分必要条件是:
把\begin{equation*}
	\lim_{n\to\infty} \sup_{x \in D} \abs{S_n(x) - S(x)} = 0.
\end{equation*}
\begin{proof}
假设\(\{S_n\}\)在集合\(D\)上点态收敛于函数\(S\),
那么对任意给定\(\epsilon>0\),存在正整数\(N = N(\epsilon)\),当\(n>N\)时,有\begin{equation*}
	\abs{S_n(x) - S(x)} < \frac\epsilon2
\end{equation*}对一切\(x \in D\)成立,
于是\begin{equation*}
	\sup_{x \in D} \abs{S_n(x) - S(x)} \leq \frac\epsilon2 < \epsilon,
\end{equation*}
这就说明\(\lim_{n\to\infty} \sup_{x \in D} \abs{S_n(x) - S(x)} = 0\).

反过来,假设\(\lim_{n\to\infty} \sup_{x \in D} \abs{S_n(x) - S(x)} = 0\),
则对任意给定\(\epsilon>0\),存在正整数\(N = N(\epsilon)\),当\(n>N\)时,有\begin{equation*}
	\sup_{x \in D} \abs{S_n(x) - S(x)} < \epsilon,
\end{equation*}
于是\begin{equation*}
	\abs{S_n(x) - S(x)} < \epsilon
\end{equation*}对一切\(x \in D\)成立,
即\(\{S_n\}\)在\(D\)上一致收敛于\(S\).
\end{proof}
\end{theorem}
回过头来,对于\cref{example:函数项级数.一致收敛的函数列2} 中的
\(S_n(x) = \frac{x}{1+n^2x^2}\ (-\infty < x < +\infty)\),
由于\begin{equation*}
	\abs{S_n(x) - S(x)}
	= \frac{\abs{x}}{1+n^2x^2}
	\leq \frac1{2n}
\end{equation*}等号成立当且仅当\(x=\pm\frac1n\),
可知\begin{equation*}
	\sup_{-\infty < x < +\infty} \abs{S_n(x) - S(x)}
	= \frac1{2n}
	\to 0
	\quad(n\to\infty),
\end{equation*}
因此\(\{S_n\}\)在\((-\infty,+\infty)\)上一致收敛于\(S(x)\equiv0\).

对于\cref{example:函数项级数.不一致收敛的函数列1} 中的
\(S_n(x) = x^n\ (0 \leq x < 1)\),
由于\begin{equation*}
	\sup_{0 \leq x < 1} \abs{S_n(x) - S(x)}
	= \sup_{0 \leq x < 1} x^n
	= 1
	\not\to 0
	\quad(n\to\infty),
\end{equation*}
所以\(\{S_n\}\)在\([0,1)\)上不是一致收敛的.

\begin{example}\label{example:函数项级数.不一致收敛的函数列2}
%@see: 《数学分析(第二版 下册)》(陈纪修) P64 例10.1.8
设\(S_n(x) = \frac{nx}{1+n^2x^2}\),
则\(\{S_n\}\)在\((0,+\infty)\)上收敛于\(S(x)=0\).
由于\begin{equation*}
	\abs{S_n(x) - S(x)}
	= \frac{nx}{1+n^2x^2}
	\leq \frac12
\end{equation*}等号成立当且仅当\(x=\frac1n\),
可知\begin{equation*}
	\sup_{0 < x < +\infty} \abs{S_n(x) - S(x)}
	= \frac12
	\not\to 0
	\quad(n\to\infty),
\end{equation*}
因此\(\{S_n\}\)在\((0,+\infty)\)上不是一致收敛的.

\begin{figure}[htb]
%@see: 《数学分析(第二版 下册)》(陈纪修) P64 图10.1.4
	\centering
	\begin{tikzpicture}
		\begin{axis}[
			name=Plt,
			xscale=1.5,
			xmin=0,xmax=4.4,
			ymin=0,ymax=.6,
			axis lines=middle,
			xlabel=$x$,
			ylabel=$y$,
		]
			\addplot[color=red,samples=50,smooth,domain=0:4]
				{x/(1+x^2)};\label{pgfplots:函数项级数.函数图像9}
			\addplot[color=orange,samples=50,smooth,domain=0:4]
				{2*x/(1+4*x^2)};\label{pgfplots:函数项级数.函数图像10}
			\addplot[color=green,samples=50,smooth,domain=0:4]
				{4*x/(1+16*x^2)};\label{pgfplots:函数项级数.函数图像11}
			\addplot[color=blue,samples=50,smooth,domain=0:4]
				{8*x/(1+64*x^2)};\label{pgfplots:函数项级数.函数图像12}
		\end{axis}
		\node[draw,fill=white,inner sep=0pt,right=1em]
		at(Plt.east){
			\small\begin{tblr}{cl}
				\ref{pgfplots:函数项级数.函数图像9} & \(n=1\) \\
				\ref{pgfplots:函数项级数.函数图像10} & \(n=2\) \\
				\ref{pgfplots:函数项级数.函数图像11} & \(n=4\) \\
				\ref{pgfplots:函数项级数.函数图像12} & \(n=8\) \\
			\end{tblr}
		};
	\end{tikzpicture}
	\caption{}
	\label{figure:函数项级数.图3}
\end{figure}

从几何上看(\cref{figure:函数项级数.图3}),
对于每个正整数\(n\),
函数\(x \mapsto \frac{nx}{1+n^2x^2}\)在\(x=\frac1n\)取到最大值\(\frac12\).
因此它们的图像不可能落在带状区域\begin{equation*}
	\Set{ (x,y) \given 0<x<+\infty, \abs{y}<\epsilon<1/2 }
\end{equation*}中.
事实上,\(\{S_n\}\)在任意包含\(x=0\)或以\(x=0\)为端点的区间上都不是一致收敛的.

若将\(\{S_n\}\)限制在任意有限闭区间\([\rho,A]\ (0<\rho<A<+\infty)\)上,
则由\begin{equation*}
	\abs{S_n(x) - S(x)} = \frac{nx}{1+n^2x^2}
\end{equation*}及\begin{equation*}
	\dv{x} \left(\frac{nx}{1+n^2x^2}\right)
	= \frac{n(1-n^2x^2)}{(1+n^2x^2)^2},
\end{equation*}
可以知道当\(n>\frac1\rho\)时,\(\abs{S_n(x) - S(x)}\)在\([\rho,A]\)单调减少,
从而\begin{equation*}
	\sup_{\rho \leq x \leq A} \abs{S_n(x) - S(x)}
	= \frac{n\rho}{1+n^2\rho^2}
	\to 0
	\quad(n\to\infty),
\end{equation*}
即\(\{S_n\}\)在\([\rho,A]\)上一致收敛于\(S(x)=0\),
或者说\(\{S_n\}\)在\((0,+\infty)\)上内闭一致收敛.
\end{example}

\begin{theorem}\label{theorem:函数项级数.一致收敛的充分必要条件2}
%@see: 《数学分析(第二版 下册)》(陈纪修) P66 定理10.1.2
设函数列\(\{S_n\}\)在集合\(D\)上点态收敛于函数\(S\),
则\(\{S_n\}\)在\(D\)上一致收敛于\(S\)的充分必要条件是:
对任意一个数列\(\{x_n\}\ (x_n \in D)\),
成立\begin{equation*}
	\lim_{n\to\infty} (S_n(x_n) - S(x_n)) = 0.
\end{equation*}
%TODO proof
\end{theorem}
\cref{theorem:函数项级数.一致收敛的充分必要条件2}
常用于判断函数列的不一致收敛.

\begin{example}
%@credit: {e9b17d8d-3be5-4f44-9c7a-a5e6122a69e2} 提出问题
函数\(f\)在区间\(D\)上一致连续的充分必要条件是:
对于任意一个数列\(\{a_n\}\),
只要\(\lim_{n\to\infty} a_n = 0\),
就有函数列\(\{f(x-a_n)\}\)在\(D\)上一致收敛于\(f\).
\begin{proof}
%@credit: {5a781423-ba4e-4629-ac1a-eac743a4d445} 提供了证明必要性的思路,不用反证法就证明了充分性
%@see: https://math.stackexchange.com/a/4967991/591741 利用反证法,证明充分性
必要性.
假设函数\(f\)在区间\(D\)上一致连续,
则对任意给定\(\epsilon>0\),
存在\(\delta>0\),
对于区间\(D\)上的任意两点\(x_1,x_2\),
当\(\abs{x_1-x_2}<\delta\)时,
有\begin{equation*}
	\abs{f(x_1)-f(x_2)}<\epsilon.
\end{equation*}
再假设数列\(\{a_n\}\)是任意一个无穷小,
那么对于任意给定\(\delta>0\),
存在正整数\(N\),
当\(n>N\)时,
有\begin{equation*}
	\abs{a_n} < \delta.
\end{equation*}
于是,对任意给定\(\epsilon>0\),
存在正整数\(N\),
当\(n>N\)时,
对区间\(D\)上的任意一点\(x\),
有\begin{equation*}
	\abs{f(x)-f(x-a_n)}<\epsilon.
\end{equation*}
这就说明函数列\(\{f(x-a_n)\}\)在\(D\)上一致收敛于\(f\).

充分性.
用反证法.
假设\(f\)在区间\(D\)上不一致连续,
那么存在\(\epsilon_0>0\),
对于任意\(\delta>0\),
存在\(x,y \in D\),
满足\begin{equation*}
	\abs{x-y}<\delta
	\quad\land\quad
	\abs{f(x)-f(y)}\geq\epsilon_0.
\end{equation*}
于是对于任意正整数\(n\),
存在实数\(x_n,y_n \in D\),
使得\begin{equation*}
	\abs{x_n-y_n}<\frac1n
	\quad\land\quad
	\abs{f(x_n)-f(y_n)}\geq\epsilon_0.
\end{equation*}
记\(a_n \defeq y_n - x_n\).
显然\(\lim_{n\to\infty} a_n = 0\).
然而,如若假设函数列\(\{f(x-a_n)\}\)在\(D\)上一致收敛于\(f\)成立,
则必定存在正整数\(N\),
当\(n>N\)时,
有\begin{equation*}
	\abs{f(y_n)-f(y_n-a_n)}
	=\abs{f(x_n)-f(y_n)}
	<\epsilon_0,
\end{equation*}
这与假设矛盾!
这就说明,如果对于任意一个收敛于\(0\)的数列\(\{a_n\}\),
总是成立函数列\(\{f(x-a_n)\}\)在\(D\)上一致收敛于\(f\),
那么函数\(f\)必定在区间\(D\)上一致连续.
\end{proof}
\end{example}

\section{函数项级数的一致收敛性,一致收敛级数的基本性质}

\subsection{一致收敛的判别}
\begin{theorem}
%@see: 《数学分析(第二版 下册)》(陈纪修) P69 定理10.2.1(函数项级数一致收敛的Cauchy收敛原理)
%@see: 《数学分析(第7版 第二卷)》(卓里奇) P304 定理(一致收敛性的柯西准则)
函数项级数\(\sum_{n=1}^\infty u_n\)在\(D\)上一致收敛的充分必要条件是:
对于任意给定的\(\epsilon>0\),存在正整数\(N = N(\epsilon)\),
使得当\(n>N\)时,对于任意正整数\(m\)与一切\(x \in D\)成立
\begin{equation*}
	\abs{\sum_{k=1}^m u_{n+k}(x)} < \epsilon.
\end{equation*}
%TODO proof
\end{theorem}

\subsection{魏尔斯特拉斯判别法}
以上两例都是直接根据定义来判定级数的一致收敛性的,现在介绍一个在实用上较方便的判别法.
\begin{theorem}[魏尔斯特拉斯判别法]\label{theorem:无穷级数.魏尔斯特拉斯判别法}
%@see: 《数学分析(第二版 下册)》(陈纪修) P70 定理10.2.2(Weierstrass判别法)
如果函数项级数\(\sum_{n=1}^\infty u_n\)在区间\(I\)上满足条件\begin{itemize}
	\item \(\abs{u_n(x)} \leq a_n \quad(n=1,2,\dotsc)\);
	\item 正项级数\(\sum_{n=1}^\infty a_n\)收敛,
\end{itemize}
则函数项级数\(\sum_{n=1}^\infty u_n\)在区间\(I\)上一致收敛且绝对收敛.
% \(\sum_{n=1}^\infty u_n\) converges uniformly and absolutely on \(I\).
\begin{proof}
假设正项级数\(\sum_{n=1}^\infty a_n\)收敛,
那么根据\hyperref[theorem:无穷级数.级数的柯西审敛原理]{柯西审敛原理},
\(\forall\epsilon>0\),\(\exists N \in \mathbb{N}^+\),
使得当\(n > N\)时,\(\forall p \in \mathbb{N}^+\),都有\begin{equation*}
	a_{n+1} + a_{n+2} + \dotsb + a_{n+p} < \frac{\epsilon}{2}.
\end{equation*}

再假设\(\abs{u_n(x)} \leq a_n \quad(n=1,2,\dotsc)\),
那么对于\(\forall x \in I\),都有\begin{align*}
	&\hspace{-20pt}\abs{u_{n+1}(x) + u_{n+2}(x) + \dotsb + u_{n+p}(x)} \\
	&\leq \abs{u_{n+1}(x)} + \abs{u_{n+2}(x)} + \dotsb + \abs{u_{n+p}(x)} \\
	&\leq a_{n+1} + a_{n+2} + \dotsb + a_{n+p} < \frac{\epsilon}{2},
\end{align*}
令\(p\to\infty\),则由上式得\begin{equation*}
	\abs{r_n(x)} \leq \frac{\epsilon}{2} < \epsilon.
\end{equation*}

因此函数项级数\(\sum_{n=1}^\infty u_n\)在区间\(I\)上一致收敛.
%TODO proof 尚未证明:函数项级数\(\sum_{n=1}^\infty u_n\)在区间\(I\)上绝对收敛.
\end{proof}
\end{theorem}

\begin{example}
证明级数\begin{equation*}
	\frac{\sin x}{1^2}
	+ \frac{\sin 2^2 x}{2^2}
	+ \dotsb
	+ \frac{\sin n^2 x}{n^2}
	+ \dotsb
\end{equation*}在区间\((-\infty,+\infty)\)内一致收敛.
\begin{proof}
因为在\((-\infty,+\infty)\)内\begin{equation*}
	\abs{\frac{\sin n^2 x}{n^2}} \leq \frac{1}{n^2}
	\quad(n=1,2,\dotsc),
\end{equation*}
而\(\sum_{n=1}^\infty \frac{1}{n^2}\)收敛,
故由\hyperref[theorem:无穷级数.魏尔斯特拉斯判别法]{魏尔斯特拉斯判别法},
所给级数在\((-\infty,+\infty)\)内一致收敛.
\end{proof}
\end{example}

\subsection{阿贝尔--狄利克雷审敛法}
\begin{definition}\label{definition:函数项级数.函数列的一致有界性}
设函数列\(\{u_n\}\)满足\begin{equation*}
	(\exists M>0)
	(\forall x \in D)
	(\forall n\in\mathbb{N})
	[\abs{u_n(x)} \leq M],
\end{equation*}
则称“函数列\(\{u_n\}\)在\(D\)上\DefineConcept{一致有界}”.
%@see: \cref{definition:微分方程.函数系的一致有界性}
\end{definition}

\begin{theorem}\label{theorem:函数项级数.函数项级数的阿贝尔--狄利克雷审敛法}
%@see: 《数学分析(第二版 下册)》(陈纪修) P72 定理10.2.3
设\(\{a_n\},\{b_n\}\)是两个函数列.

若下列两个条件之一满足,
则函数项级数\(\sum_{n=1}^\infty a_n(x) b_n(x)\)收敛:\begin{itemize}
	\item {\rm\bf 阿贝尔条件}
	对于每一个固定的\(x \in D\)总有数列\(\{a_n(x)\}\)是单调的,
	函数列\(\{a_n\}\)在\(D\)上一致有界,
	函数项级数\(\sum_{n=1}^\infty b_n\)收敛;

	\item {\rm\bf 狄利克雷条件}
	对于每一个固定的\(x \in D\)总有数列\(\{a_n(x)\}\)是单调的,
	函数列\(\{a_n\}\)在\(D\)上一致收敛于\(0\),
	函数项级数\(\sum_{n=1}^\infty b_n(x)\)的部分和函数列在\(D\)上一致有界.
\end{itemize}
%TODO proof
\end{theorem}

\begin{example}
%@see: 《数学分析(第二版 下册)》(陈纪修) P83 习题 8.
设函数项级数\(\sum_{n=1}^\infty u_n\)在点\(a\)与点\(b\)收敛,
且函数\(u_n\ (n=1,2,\dotsc)\)在闭区间\([a,b]\)上单调增加.
证明:函数项级数\(\sum_{n=1}^\infty u_n\)在\([a,b]\)上一致收敛.
%TODO proof
\end{example}
\begin{example}
%@see: 《数学分析(第二版 下册)》(陈纪修) P83 习题 9.
设函数\(u_n\ (n=1,2,\dotsc)\)在点\(a\)右连续,
且函数项级数\(\sum_{n=1}^\infty u_n\)在点\(a\)发散.
证明:对于\(\forall\delta>0\)都有
函数项级数\(\sum_{n=1}^\infty u_n\)在\((a,a+\delta)\)上必定不一致收敛.
%TODO proof
\end{example}

\subsection{一致收敛级数的性质}
现在我们可以来回答之前提出的
关于函数项级数或函数列的基本问题,
即在什么条件下,
和函数或极限函数仍然保持连续性、可导性、可积性等分析性质.

\begin{theorem}\label{theorem:函数项级数.连续函数列的一致收敛性保证极限函数的连续性}
%@see: 《数学分析(第二版 下册)》(陈纪修) P74 定理10.2.4(连续性定理)
设函数列\(\{S_n\}\)满足\begin{itemize}
	\item 各项\(S_n\)在区间\([a,b]\)上连续,
	\item 函数列\(\{S_n\}\)在区间\([a,b]\)上一致收敛于函数\(S\),
\end{itemize}
则函数\(S\)在区间\([a,b]\)上也连续.
此时,两种极限运算可以交换次序,即\begin{equation*}
	\lim_{x \to x_0} \lim_{n\to\infty} S_n(x)
	= \lim_{n\to\infty} \lim_{x \to x_0} S_n(x).
\end{equation*}
\begin{proof}
设\(x_0\)是\([a,b]\)上任意两点.

因为\(\{S_n\}\)在区间\([a,b]\)上一致收敛于\(S\),
所以对任意给定\(\epsilon>0\),
存在正整数\(N = N(\epsilon)\),
使得当\(n>N\)时,
有\begin{equation*}
	\abs{S_n(x) - S(x)} < \frac\epsilon3
\end{equation*}对一切\(x\in[a,b]\)成立.
特别地,有\begin{equation*}
	\abs{S_n(x_0) - S(x_0)} < \frac\epsilon3.
\end{equation*}
对于每一个固定的大于\(N\)的\(n\),函数\(S_n\)是有限项连续函数之和,
故\(S_n(x)\)在区间\([a,b]\)上连续,
所以存在\(\delta>0\),
当\(\abs{x - x_0} < \delta\)时,
有\begin{equation*}
	\abs{S_n(x) - S_n(x_0)} < \frac\epsilon3.
\end{equation*}
于是有\begin{equation*}
	\abs{S(x) - S(x_0)}
	\leq \abs{S_n(x) - S(x)}
		+ \abs{S_n(x_0) - S(x_0)}
		+ \abs{S_n(x) - S_n(x_0)}
	< \epsilon,
\end{equation*}
即\(S\)在点\(x_0\)连续.
由\(x_0\)在区间\([a,b]\)中的任意性,就得到\(S\)在区间\([a,b]\)上连续.
\end{proof}
\end{theorem}
\begin{theorem}\label{theorem:函数项级数.连续函数项级数的一致收敛性保证和函数的连续性}
%@see: 《数学分析(第二版 下册)》(陈纪修) P75 定理10.2.4'
%@see: 《高等数学(第六版 上册)》 P297 定理1
设函数项级数\(\sum_{n=1}^\infty u_n\)满足\begin{itemize}
	\item 各项\(u_n\)在区间\([a,b]\)上连续,
	\item 函数项级数\(\sum_{n=1}^\infty u_n\)在区间\([a,b]\)上一致收敛于函数\(S\),
\end{itemize}
则函数\(S\)在区间\([a,b]\)上连续.
此时,极限运算与无限求和运算可以交换次序,
即对任意\(x_0\in[a,b]\)成立\begin{equation*}
	\lim_{x \to x_0} \sum_{n=1}^\infty u_n(x)
	= \sum_{n=1}^\infty \lim_{x \to x_0} u_n(x).
\end{equation*}
\end{theorem}
%@see: 《数学分析(第二版 下册)》(陈纪修) P75 注
由于连续性是函数的一种局部性质,它是逐点定义的,
因此,我们可以把“在闭区间\([a,b]\)上一致收敛”这个条件
修改为“在开区间\((a,b)\)上内闭一致收敛”,
就足以保证函数\(S\)在开区间\((a,b)\)上连续.
于是我们有下述两个命题:
\begin{proposition}\label{theorem:函数项级数.连续函数列的内闭一致收敛性保证极限函数的连续性}
%@see: 《数学分析(第二版 下册)》(陈纪修) P75 注
设函数列\(\{S_n\}\)满足\begin{itemize}
	\item 各项\(S_n\)在区间\((a,b)\)上连续,
	\item 函数列\(\{S_n\}\)在区间\((a,b)\)上内闭一致收敛于函数\(S\),
\end{itemize}
则函数\(S\)在\((a,b)\)上也连续.
\end{proposition}
\begin{proposition}\label{theorem:函数项级数.连续函数项级数的内闭一致收敛性保证和函数的连续性}
%@see: 《数学分析(第二版 下册)》(陈纪修) P75 注
设函数项级数\(\sum_{n=1}^\infty u_n\)满足\begin{itemize}
	\item 各项\(u_n\)在区间\((a,b)\)上连续,
	\item 函数项级数\(\sum_{n=1}^\infty u_n\)在区间\((a,b)\)上内闭一致收敛于函数\(S\),
\end{itemize}
则函数\(S\)在区间\((a,b)\)上连续.
\end{proposition}

\begin{example}
%@credit: {fc98e9bc-61fa-4633-8839-1ace1db0f985}
设函数项级数\(\sum_{n=1}^\infty u_n\)满足\begin{itemize}
	\item 各项\(u_n\)在区间\([a,b]\)上非负连续,
	\item 函数项级数\(\sum_{n=1}^\infty u_n\)在区间\([a,b]\)上点态收敛于函数\(S\).
\end{itemize}
证明:函数\(S\)在区间\([a,b]\)上也连续.
%TODO proof 要用到{控制收敛定理}吗?
%@see: https://www.maths.tcd.ie/~richardt/MA2224/MA2224-ch4.pdf
\end{example}

\begin{example}
%@see: 《数学分析(第二版 下册)》(陈纪修) P83 习题 7.
设\(u_n,v_n\)在区间\((a,b)\)上连续,
且\begin{equation*}
	(\forall n\in\mathbb{N}^+)
	(\forall x\in(a,b))
	[\abs{u_n(x)} \leq v_n(x)].
\end{equation*}
证明:若\(\sum_{n=1}^\infty v_n(x)\)在\((a,b)\)上点态收敛于一个连续函数,
则\(\sum_{n=1}^\infty u_n(x)\)也必然收敛于一个连续函数.
%TODO proof
\end{example}

\begin{theorem}\label{theorem:函数项级数.连续函数列的一致收敛性保证极限函数的可积性}
%@see: 《数学分析(第二版 下册)》(陈纪修) P75 定理10.2.5
设函数列\(\{S_n\}\)满足\begin{itemize}
	\item 各项\(S_n\)在区间\([a,b]\)上连续,
	\item 函数列\(\{S_n\}\)在区间\([a,b]\)上一致收敛于函数\(S\),
\end{itemize}
则函数\(S\)在区间\([a,b]\)上可积,
且\begin{equation*}
	\int_a^b S(x) \dd{x}
	= \lim_{n\to\infty} \int_a^b S_n(x) \dd{x}.
\end{equation*}
此时,求积分运算与极限运算可以交换次序,
即\begin{equation*}
	\int_a^b \lim_{n\to\infty} S_n(x) \dd{x}
	= \lim_{n\to\infty} \int_a^b S_n(x) \dd{x}.
\end{equation*}
\begin{proof}
由\cref{theorem:函数项级数.连续函数列的一致收敛性保证极限函数的连续性} 可知
函数\(S\)在区间\([a,b]\)上连续,
再由\cref{theorem:定积分.黎曼可积条件.闭区间上的连续函数必定可积} 可知
函数\(S\)在区间\([a,b]\)上可积.
由于函数列\(\{S_n\}\)在区间\([a,b]\)上一致收敛于函数\(S\),
所以对任意给定\(\epsilon>0\),
存在正整数\(N\),
当\(n>N\)时,
有\begin{equation*}
	\abs{S_n(x) - S(x)} < \epsilon
\end{equation*}对一切\(x\in[a,b]\)成立,
于是有\begin{equation*}
	\abs{\int_a^b S(x) \dd{x} - \int_a^b S_n(x) \dd{x}}
	\leq \int_a^b \abs{S(x) - S_n(x)} \dd{x}
	< (b-a) \epsilon.
	\qedhere
\end{equation*}
\end{proof}
\end{theorem}
\begin{theorem}\label{theorem:函数项级数.连续函数项级数的一致收敛性保证和函数的可积性}
%@see: 《数学分析(第二版 下册)》(陈纪修) P76 定理10.2.5'(逐项积分定理)
%@see: 《高等数学(第六版 上册)》 P298 定理2
设函数项级数\(\sum_{n=1}^\infty u_n\)满足\begin{itemize}
	\item 各项\(u_n\)在区间\([a,b]\)上连续,
	\item 函数项级数\(\sum_{n=1}^\infty u_n\)在区间\([a,b]\)上一致收敛于函数\(S\),
\end{itemize}
则函数\(S\)在区间\([a,b]\)上可积.
此时,求积分运算与无限求和运算可以交换次序,
即\begin{equation*}
	\int_a^b S(x) \dd{x}
	= \int_a^b \sum_{n=1}^\infty u_n(x) \dd{x}
	= \sum_{n=1}^\infty \int_a^b u_n(x) \dd{x}.
\end{equation*}
\end{theorem}
\begin{proposition}
%@see: 《数学分析(第二版 下册)》(陈纪修) P76 注
设函数列\(\{S_n\}\)满足\begin{itemize}
	\item 各项\(S_n\)在区间\([a,b]\)上连续,
	\item 函数列\(\{S_n\}\)在区间\([a,b]\)上一致收敛于\(S\),
\end{itemize}
则对于任意固定\(x_0\in[a,b]\),
函数列\begin{equation*}
	\left\{x \mapsto \int_{x_0}^x S_n(t) \dd{t}\right\}
\end{equation*}在区间\([a,b]\)上一致收敛于函数\(x \mapsto \int_{x_0}^x S(t) \dd{t}\).
%TODO proof
\end{proposition}
\begin{proposition}
%@see: 《数学分析(第二版 下册)》(陈纪修) P76 注
设函数项级数\(\sum_{n=1}^\infty u_n\)满足\begin{itemize}
	\item 各项\(u_n\)在区间\([a,b]\)上连续,
	\item 函数项级数\(\sum_{n=1}^\infty u_n\)在区间\([a,b]\)上一致收敛于函数\(S\),
\end{itemize}
则对于任意固定\(x_0\in[a,b]\),
函数项级数\begin{equation*}
	x \mapsto \sum_{n=1}^\infty \int_{x_0}^x u_n(t) \dd{t}
\end{equation*}在区间\([a,b]\)上一致收敛于函数\(x \mapsto \int_{x_0}^x S(t) \dd{t}\).
%TODO proof
\end{proposition}

\begin{theorem}\label{theorem:函数项级数.连续可导函数列的点态收敛性及其导函数列的一致收敛性保证极限函数的可微性}
%@see: 《数学分析(第二版 下册)》(陈纪修) P77 定理10.2.6
设函数列\(\{S_n\}\)满足\begin{itemize}
	\item 各项\(S_n\)在区间\([a,b]\)上连续可导,
	\item 函数列\(\{S_n\}\)在区间\([a,b]\)上点态收敛于函数\(S\),
	\item 导函数列\(\{S_n'\}\)在区间\([a,b]\)上一致收敛于函数\(\sigma\),
\end{itemize}
则
%@see: 《数学分析(第二版 下册)》(陈纪修) P78 注(1)
函数列\(\{S_n\}\)在区间\([a,b]\)上一致收敛于\(S\),
函数\(S\)在区间\([a,b]\)上可导,
且\begin{equation*}
	\dv{x} S(x) = \sigma(x).
\end{equation*}
此时,求导运算与极限运算可以交换次序,
即\begin{equation*}
	\dv{x} \lim_{n\to\infty} S_n(x)
	= \lim_{n\to\infty} \dv{x} S_n(x).
\end{equation*}
\begin{proof}
由\cref{theorem:函数项级数.连续函数列的一致收敛性保证极限函数的连续性,theorem:函数项级数.连续函数列的一致收敛性保证极限函数的可积性} 可知
函数\(\sigma\)在区间\([a,b]\)上连续,
且\begin{equation*}
	\int_a^x \sigma(t) \dd{t}
	= \lim_{n\to\infty} \int_a^x S_n'(t) \dd{t}
	= \lim_{n\to\infty} [S_n(x) - S_n(a)]
	= S(x) - S(a).
\end{equation*}
根据\cref{theorem:定积分.变限积分定理},
函数\(x \mapsto \int_a^x \sigma(t) \dd{t}\)可导,
所以函数\(S\)也可导,
且\(S'(x) = \sigma(x)\).
\end{proof}
\end{theorem}
\begin{theorem}
%@see: 《数学分析(第二版 下册)》(陈纪修) P77 定理10.2.6'(逐项求导定理)
设函数项级数\(\sum_{n=1}^\infty u_n\)满足\begin{itemize}
	\item 各项\(u_n\)在区间\([a,b]\)上具有连续导函数\(u_n'\),
	\item 函数项级数\(\sum_{n=1}^\infty u_n\)在区间\([a,b]\)上点态收敛于\(S\),
	\item 函数项级数\(\sum_{n=1}^\infty u_n'(x)\)在区间\([a,b]\)上一致收敛于\(\sigma\),
\end{itemize}
则
%@see: 《数学分析(第二版 下册)》(陈纪修) P78 注(1)
函数项级数\(\sum_{n=1}^\infty u_n\)在区间\([a,b]\)上一致收敛于\(S\),
函数\(S\)在区间\([a,b]\)上可导,
且求导运算与无限求和运算可以交换次序,
即\begin{equation*}
	\dv{x} \sum_{n=1}^\infty u_n(x)
	= \sum_{n=1}^\infty \dv{x} u_n(x).
\end{equation*}
\end{theorem}
与连续性类似,可导性也是函数的一种局部性质,它也是逐点定义的,
因此我们可以把“在闭区间\([a,b]\)上一致收敛”这个条件
修改为“在开区间\((a,b)\)上内闭一致收敛”,
就足以保证函数\(S\)在开区间\((a,b)\)上可导.
于是我们有下述两个命题:
\begin{proposition}
%@see: 《数学分析(第二版 下册)》(陈纪修) P78 注(2)
设函数列\(\{S_n\}\)满足\begin{itemize}
	\item 各项\(u_n\)在区间\((a,b)\)上具有连续导函数\(u_n'\),
	\item 函数列\(\{S_n\}\)在区间\((a,b)\)上点态收敛于\(S\),
	\item 导函数列\(\{S_n'\}\)在区间\((a,b)\)上内闭一致收敛于\(\sigma\),
\end{itemize}
则函数\(S\)在区间\((a,b)\)上可导.
\end{proposition}
\begin{proposition}
%@see: 《数学分析(第二版 下册)》(陈纪修) P78 注(2)
设函数项级数\(\sum_{n=1}^\infty u_n\)满足\begin{itemize}
	\item 各项\(u_n\)在区间\((a,b)\)上具有连续导函数\(u_n'\),
	\item 函数项级数\(\sum_{n=1}^\infty u_n\)在区间\((a,b)\)上点态收敛于\(S\),
	\item 函数项级数\(\sum_{n=1}^\infty u_n'(x)\)在区间\((a,b)\)上内闭一致收敛于\(\sigma\),
\end{itemize}
则函数\(S\)在区间\((a,b)\)上可导.
\end{proposition}

\begin{example}
%@see: 《数学分析(第二版 下册)》(陈纪修) P78 例10.2.9
证明:对于一切\(x\in(-1,1)\),成立\begin{equation*}
	\sum_{n=1}^\infty n x^n
	= x + 2x^2 + 3x^3 + \dotsb
	= \frac{x}{(1-x)^2}.
\end{equation*}
\begin{proof}
我们已经知道函数项级数\(\sum_{n=0}^\infty x^n\)在\((-1,1)\)上
点态收敛于函数\(S(x) = \frac1{1-x}\),
而\(\sum_{n=0}^\infty x^n\)经过逐项求导,
得到\(\sum_{n=1}^\infty n x^{n-1}\).
对于任意\(\rho\in(0,1)\),
当\(x\in[-\rho,\rho]\)时,
有\begin{equation*}
	\abs{n x^{n-1}} \leq n \rho^{n-1}.
\end{equation*}
应用\hyperref[theorem:无穷级数.魏尔斯特拉斯判别法]{魏尔斯特拉斯判别法}可知
\(\sum_{n=1}^\infty n x^{n-1}\)在\([-\rho,\rho]\)上一致收敛,
换言之,\(\sum_{n=1}^\infty n x^{n-1}\)在\((-1,1)\)上内闭一致收敛.
%TODO ref 定理10.2.6'
对\(\sum_{n=0}^\infty x^n = \frac1{1-x}\)进行逐项求导,
得到\begin{equation*}
	\sum_{n=1}^\infty n x^{n-1}
	= \frac1{(1-x)^2},
\end{equation*}
两边同时乘上\(x\),就得到\begin{equation*}
	\sum_{n=1}^\infty n x^n
	= \frac{x}{(1-x)^2}.
\end{equation*}
\end{proof}
\end{example}

需要指出的是,
\cref{theorem:函数项级数.连续函数列的一致收敛性保证极限函数的连续性,%
theorem:函数项级数.连续函数列的一致收敛性保证极限函数的可积性,%
theorem:函数项级数.连续可导函数列的点态收敛性及其导函数列的一致收敛性保证极限函数的可微性}
中的条件都是充分而不必要的.
对于\cref{theorem:函数项级数.连续函数列的一致收敛性保证极限函数的连续性,%
theorem:函数项级数.连续函数列的一致收敛性保证极限函数的可积性},
我们可以考虑\cref{example:函数项级数.不一致收敛的函数列2} 中的
函数\begin{equation*}
	S_n(x) = \frac{nx}{1+n^2x^2},
\end{equation*}
函数列\(\{S_n\}\)在\([0,1]\)上收敛于\(S(x)=0\),
但是它不是一致收敛的,
然而\(S\)在\([0,1]\)上连续且可积,
并且\begin{align*}
	\int_0^1 S_n(x) \dd{x}
	&= \frac1{2n} \int_0^1 \frac{\dd(1+n^2x^2)}{1+n^2x^2} \\
	&= \frac1{2n} \eval{\ln(1+n^2x^2)}_0^1
	= \frac1{2n} \ln(1+n^2) \\
	&\to 0 = \int_0^1 S(x) \dd{x}
	\quad(n\to\infty).
\end{align*}
对于\cref{theorem:函数项级数.连续可导函数列的点态收敛性及其导函数列的一致收敛性保证极限函数的可微性},
可以考虑函数\begin{equation*}
	\sigma_n(x) = \frac1{2n} \ln(1+n^2x^2),
\end{equation*}
函数列\(\{\sigma_n\}\)在\([0,1]\)上收敛于\(\sigma(x)=0\).
由于\begin{equation*}
	\sigma_n'(x)
	= S_n(x)
	= \frac{nx}{1+n^2x^2},
\end{equation*}
导函数列\(\{\sigma_n'\}\)在\([0,1]\)上收敛于\(S(x)=0\),
但并非一致收敛.
虽然\(\{\sigma_n\}\)不满足\cref{theorem:函数项级数.连续可导函数列的点态收敛性及其导函数列的一致收敛性保证极限函数的可微性} 的条件,
但仍然有\(\sigma'(x) = S(x)\)的结论.

经过上面的讨论,我们知道\cref{theorem:函数项级数.连续函数列的一致收敛性保证极限函数的连续性}
的逆命题一般来说不成立,
即区间\([a,b]\)上连续的函数序列\(\{S_n\}\)收敛于连续函数\(S\)
并不意味着收敛在\([a,b]\)具有一致性.
但是在一定的条件下,我们还是可以得到这个结论,这就是下面的定理.
\begin{theorem}[狄尼定理]\label{theorem:函数项级数.狄尼定理}
%@see: 《数学分析(第二版 下册)》(陈纪修) P79 定理10.2.7(Dini定理)
设函数列\(\{S_n\}\)在闭区间\([a,b]\)上点态收敛于函数\(S\).
如果\begin{itemize}
	\item 各项\(S_n\)在\([a,b]\)上连续,
	\item 函数\(S\)在\([a,b]\)上连续,
	\item 对任意固定的\(x\in[a,b]\),数列\(\{S_n(x)\}\)是单调的,
\end{itemize}
则函数列\(\{S_n\}\)在\([a,b]\)上一致收敛于\(S\).
%TODO proof
\end{theorem}
\begin{remark}
%@see: 《数学分析(第二版 下册)》(陈纪修) P80
\hyperref[theorem:函数项级数.狄尼定理]{狄尼定理}中
\(x\)的取值范围 --- 闭区间\([a,b]\) --- 不能换成开区间\((a,b)\).
\end{remark}
\begin{theorem}
%@see: 《数学分析(第二版 下册)》(陈纪修) P80 定理10.2.7'
设函数项级数\(\sum_{n=1}^\infty u_n\)在闭区间\([a,b]\)上点态收敛于函数\(S\).
如果\begin{itemize}
	\item 各项\(u_n\)在\([a,b]\)上连续,
	\item 函数\(S\)在\([a,b]\)上连续,
	\item 对任意固定的\(x\in[a,b]\),级数\(\sum_{n=1}^\infty u_n(x)\)是正项级数或负项级数,
\end{itemize}
则函数项级数\(\sum_{n=1}^\infty u_n\)在\([a,b]\)上一致收敛于\(S\).
%TODO proof
\end{theorem}

\subsection{处处不可导的连续函数}
一般说来,分析学所研究的连续函数在其绝大部分连续点上总是可导得.
因此在分析学的发展历史上,数学家们一只猜测:
连续函数在其定义区间中,至多除去可列个点外,都是可导的.
换言之,连续函数的不可导点至多是可列集.

后来,随着级数理论的发展,魏尔斯特拉斯利用函数项级数
构造出了第一个处处连续而处处不可导的函数,为上述猜测做了一个否定的终结.
下面我们叙述一个相对简易的反例,它是由荷兰数学家 范·德·瓦尔登 于1930年给出的.

设\(\phi(x)\)表示\(x\)与最邻近的整数之间的距离,
例如当\(x=1.26\)时有\(\phi(x)=0.26\),当\(x=3.67\)时有\(\phi(x)=0.33\).
显然\begin{equation*}
	\phi(x) = \min\{\abs{x-\floor{x}},\abs{x-\ceil{x}}\}
\end{equation*}是周期为\(1\)的连续函数,且\(\abs{\phi(x)}\leq1/2\).
%@Mathematica: Plot[Min[Abs[x - Floor[x]], Abs[x - Ceiling[x]]], {x, -5, 5}]
又令\begin{equation*}
	f(x) = \sum_{n=0}^\infty \frac{\phi(10^n x)}{10^n}.
\end{equation*}
由于\begin{equation*}
	\abs{\frac{\phi(10^n x)}{10^n}} \leq \frac1{2\cdot10^n},
\end{equation*}
级数\(\sum_{n=0}^\infty \frac1{2\cdot10^n}\)收敛,
应用\hyperref[theorem:无穷级数.魏尔斯特拉斯判别法]{魏尔斯特拉斯判别法}可知
函数项级数\(\sum_{n=0}^\infty \frac{\phi(10^n x)}{10^n}\)
在\((-\infty,+\infty)\)上一致收敛.
再由\(\phi\)的连续性,
应用\cref{theorem:函数项级数.连续函数项级数的一致收敛性保证和函数的连续性},
可知\(f\)在\((-\infty,+\infty)\)上连续.

现在考虑\(f\)在任意一点\(x\)的可导性.

\section{幂级数}
我们在本节学习函数项级数中简单而常见的一类函数项级数级数 --- “幂级数”.

\subsection{幂级数的概念}
\begin{definition}\label{definition:无穷级数.幂级数}
各项均是幂函数的函数项级数,
称为\DefineConcept{幂级数}(power series).

称级数\begin{equation*}
	\sum_{n=0}^\infty a_n (x-x_0)^n
\end{equation*}为“幂级数的\DefineConcept{一般形式}”.

称级数\begin{equation*}
	\sum_{n=0}^\infty a_n x^n
\end{equation*}为“幂级数的\DefineConcept{标准形式}”.

这里,我们把常数\(\AutoTuple{a}[0]{n},\dotsc\)称为
“幂级数的\DefineConcept{系数}”.
\end{definition}

由于幂级数的一般形式只要作变量代换\(t = x - x_0\)就可化为它的标准形式,
因此,即便我们取标准形式来讨论,也并不影响一般性.

\subsection{幂级数的收敛半径}
现在我们来讨论:
对于一个给定的幂级数,
它的收敛域与发散域是怎样的?
即\(x\)取数轴上哪些点时幂级数收敛,
取哪些点时幂级数发散?
这就是幂级数的收敛性问题.

先看一个例子.
考察幂级数\begin{equation*}
	1+x^2+x^3+\dotsb+x^n+\dotsb
\end{equation*}的收敛性.
我们已经知道,
当\(\abs{x}<1\)时,
该级数收敛于\(\frac{1}{1-x}\);
当\(\abs{x}\geq1\)时,
该级数发散.
因此,这个幂级数的收敛域是开区间\((-1,1)\),
发散域是\((-\infty,-1]\cup[1,+\infty)\),
并有\begin{equation*}
	\frac{1}{1-x} = 1+x+x^2+\dotsb+x^n+\dotsb
	\quad(-1<x<1).
\end{equation*}

在这个例子中我们看到,这个幂级数的收敛域是一个区间.
事实上,这个结论对于一般的幂级数也是成立的.
我们有如下的定理.

\begin{theorem}[阿贝尔第一定理]\label{theorem:无穷级数.阿贝尔定理1}
%@see: 《高等数学(第六版 下册)》 P271 定理1
%@see: 《数学分析(第二版 下册)》(陈纪修) P86 阿贝尔第一定理
%@see: 《数学分析教程(第3版 上册)》(史济怀) P234 定理15.4.1
如果级数\(\sum_{n=0}^\infty a_n x^n\)当\(x=x_0\neq0\)时收敛,
则满足\(\abs{x}<\abs{x_0}\)的一切\(x\)使之绝对收敛.
反之,如果级数\(\sum_{n=0}^\infty a_n x^n\)当\(x=x_0\)时发散,
则满足\(\abs{x}>\abs{x_0}\)的一切\(x\)使之发散.
\begin{proof}
先设\(x_0\)是幂级数\(\sum_{n=0}^\infty a_n x^n\)的收敛点,
即常数项级数\begin{equation*}
	a_0 + a_1 x_0 + a_2 x_0^2 + \dotsb + a_n x_0^n + \dotsb
\end{equation*}收敛.
根据级数收敛的必要条件,
这时有\begin{equation*}
	\lim_{n\to\infty} a_n x_0^n = 0;
\end{equation*}
于是\(\exists M > 0\),使得\begin{equation*}
	\abs{a_n x_0^n} \leq M
	\quad(n=0,1,2,\dotsc).
\end{equation*}
这样幂级数\(\sum_{n=0}^\infty a_n x^n\)的一般项的绝对值\begin{equation*}
	\abs{a_n x^n} = \abs{a_n x_0^n \cdot \frac{x^n}{x_0^n}}
	= \abs{a_n x_0^n} \cdot \abs{\frac{x}{x_0}}^n
	\leq M \abs{\frac{x}{x_0}}^n.
\end{equation*}
因为当\(\abs{x}<\abs{x_0}\)时,
等比级数\(\sum_{n=0}^\infty M \abs{\frac{x}{x_0}}^n\)收敛,
所以级数\(\sum_{n=1}^\infty \abs{a_n x^n}\)收敛,
也就是级数\(\sum_{n=0}^\infty a_n x^n\)绝对收敛.

定理的第二部分可用反证法证明.
假设幂级数当\(x=x_0\)时发散而有一点\(x_1\)适合\(\abs{x_1}>\abs{x_0}\)使级数收敛,
则根据本定理的第一部分,当\(x=x_0\)时级数应收敛,这与假设矛盾.
\end{proof}
\end{theorem}

\cref{theorem:无穷级数.阿贝尔定理1} 表明,
如果幂级数在\(x=x_0\)处收敛,
则对\(\forall x\in(-\abs{x_0},\abs{x_0})\),
幂级数都收敛;
如果幂级数在\(x=x_0\)处发散,
则对\(\forall x\in(-\infty,-\abs{x_0})\cup(\abs{x_0},+\infty)\),
幂级数都发散.

设已给幂级数在数轴上既有收敛点(不仅是原点)也有发散点.
现在从原点出发沿数轴正方向走,最初只遇到收敛点,然后就只遇到发散点.
这两部分的界点可能是收敛点,也可能是发散点.
从原点出发沿数轴负方向走,情形相同.
利用\cref{theorem:无穷级数.阿贝尔定理1} 可以证明:原点两侧的两个界点到原点的距离是相等的.
像这样,我们就得到以下重要推论.
\begin{corollary}[柯西--阿达玛定理]\label{theorem:无穷级数.阿贝尔定理1推论}
%@see: 《数学分析教程(第3版 上册)》(史济怀) P235 定理15.4.2
%@see: 《高等数学(第六版 下册)》 P271 推论
%@see: 《数学分析(第二版 下册)》(陈纪修) P84 定理10.3.1(Cauchy-Hadamard定理)
给定幂级数\(\sum_{n=0}^\infty a_n x^n\),
记\begin{equation*}
	\rho \defeq \varlimsup_{n\to\infty} \sqrt[n]{\abs{a_n}},
	\qquad
	R \defeq \frac1\rho,
\end{equation*}
那么\begin{itemize}
	\item 当\(R = 0\)时,级数\(\sum_{n=0}^\infty a_n x^n\)只在点\(x=0\)收敛;
	\item 当\(R = +\infty\)时,级数\(\sum_{n=0}^\infty a_n x^n\)在\((-\infty,+\infty)\)上绝对收敛;
	\item 当\(0 < R < +\infty\)时,级数\(\sum_{n=0}^\infty a_n x^n\)在\((-R,R)\)内绝对收敛,在\((-\infty,-R)\cup(R,+\infty)\)上发散.
\end{itemize}
%TODO proof
% \begin{proof}
% 当\(R=0\)时,

% 那么\begin{equation*}
% 	\varlimsup_{n\to\infty} \sqrt[n]{\abs{a_n x^n}}
% 	= \varlimsup_{n\to\infty} \sqrt[n]{\abs{a_n}} \cdot \abs{x}
% 	= \rho \abs{x}.
% 	\eqno(1)
% \end{equation*}

% 根据\hyperref[theorem:无穷级数.正项级数的根值审敛法]{正项级数的根值审敛法},
% 当极限(1)取值小于\(1\),
% 即\begin{equation*}
% 	\abs{x} < \frac1\rho
% \end{equation*}时,
% 级数\(\sum_{n=0}^\infty a_n x^n\)绝对收敛.

% 根据\cref{theorem:无穷级数.绝对发散的特殊情况},
% 当极限(1)取值大于\(1\),
% 即\begin{equation*}
% 	\abs{x} > \frac1\rho
% \end{equation*}时,
% 级数\(\sum_{n=0}^\infty a_n x^n\)发散.

% \begin{equation*}
% 	R = \left\{ \def\arraystretch{1.5} \begin{array}{cl}
% 		\frac1\rho, & 0<\rho<+\infty, \\
% 		+\infty, & \rho = 0, \\
% 		0, & \rho = +\infty \\
% 	\end{array} \right.
% \end{equation*}
% \end{proof}
\end{corollary}

对于\(\sum_{n=0}^\infty a_n (x-x_0)^n\),我们有平行的结论:
如果幂级数\(\sum_{n=0}^\infty a_n (x-x_0)^n\)不是仅在\(x=x_0\)一点收敛,
也不是在整个数轴上都收敛,
则必定存在正数\(R\),
使得幂级数\(\sum_{n=0}^\infty a_n (x-x_0)^n\)
在以\(x_0\)为中心、\(R\)为半径的对称区间\((x_0-R,x_0+R)\)内绝对收敛,
而在\((-\infty,x_0-R)\cup(x_0+R,+\infty)\)上发散.

我们把\cref{theorem:无穷级数.阿贝尔定理1推论} 中提到的正数\(R\)
称为“幂级数\(\sum_{n=0}^\infty a_n x^n\)的\DefineConcept{收敛半径}”.
把开区间\((-R,R)\)称为“幂级数\(\sum_{n=0}^\infty a_n x^n\)的\DefineConcept{收敛区间}”.

在已知幂级数的收敛半径或收敛区间的情况下,
我们可以根据幂级数在点\(x = \pm R\)处的收敛性,
就可以决定其收敛域是\((-R,R)\)、\([-R,R)\)、\((-R,R]\)或\([-R,R]\)四个区间之一.

如果幂级数只在\(x=0\)处收敛,这时收敛域为\(\{0\}\),规定收敛半径\(R=0\).
如果幂级数对任意实数都收敛,则规定收敛半径\(R=+\infty\),收敛域为\((-\infty,+\infty)\).

从\cref{theorem:无穷级数.阿贝尔定理1推论} 可以看出:
给定两个幂级数\(\sum_{n=0}^\infty a_n x^n\)和\(\sum_{n=0}^\infty b_n x^n\),
假设它们的收敛半径分别是\(R_1\)和\(R_2\),
如果\(\abs{a_n} \leq \abs{b_n}\ (n=1,2,\dotsc)\),
那么必有\(\sqrt[n]{\abs{a_n}} \leq \sqrt[n]{\abs{b_n}}\),
从而有\(R_1 \geq R_2\).
这体现了极限的保序性.

\begin{example}
%@see: 《数学分析(第二版 下册)》(陈纪修) P85 例10.3.2
考察幂级数\(\sum_{n=1}^\infty \frac{[2+(-1)^n]^n}{n} \left(x-\frac12\right)^n\)的敛散性.
\begin{solution}
因为\begin{equation*}
	\varlimsup_{n\to\infty} \sqrt[n]{\frac{[2+(-1)^n]^n}n} = 3,
\end{equation*}
所以收敛半径为\(R=\frac13\).
当\(x=\frac12+R=\frac56\)时,
级数\(\sum_{n=1}^\infty \frac{[2+(-1)^n]^n}{3^n n}\)是发散的.
当\(x=\frac12-R=\frac16\)时,
级数\(\sum_{n=1}^\infty \frac{[2+(-1)^n]^n}{6^n n}\)是发散的.
因此幂级数\(\sum_{n=1}^\infty \frac{[2+(-1)^n]^n}{n} \left(x-\frac12\right)^n\)的收敛域是
\(\left(\frac16,\frac56\right)\).
\end{solution}
\end{example}

\begin{theorem}\label{theorem:无穷级数.幂级数的收敛半径的求法2}
%@see: 《高等数学(第六版 下册)》 P272 定理2
%@see: 《数学分析(第二版 下册)》(陈纪修) P85 定理10.3.2(d'Alembert判别法)
如果\begin{equation*}
	\lim_{n\to\infty} \abs{\frac{a_{n+1}}{a_n}} = \rho,
\end{equation*}
则幂级数\(\sum_{n=0}^\infty a_n x^n\)的收敛半径为\begin{equation*}
	R = \left\{ \def\arraystretch{1.5} \begin{array}{cl}
		\frac1\rho, & 0<\rho<+\infty, \\
		+\infty, & \rho = 0, \\
		0, & \rho = +\infty. \\
	\end{array} \right.
\end{equation*}
\begin{proof}
考察幂级数\(\sum_{n=0}^\infty a_n x^n\)的各项取绝对值所成的级数\begin{equation*}
	\sum_{n=1}^\infty \abs{a_n x^n}
	= \abs{a_0} + \abs{a_1 x} + \abs{a_2 x^2} + \dotsb + \abs{a_n x^n} + \dotsb.
\end{equation*}
这级数相邻两项之比为\begin{equation*}
	\frac{\abs{a_{n+1} x^{n+1}}}{\abs{a_n x^n}}
	= \abs{\frac{a_{n+1}}{a_n}} \abs{x}.
\end{equation*}

\begin{enumerate}
	\item 如果极限\(\lim_{n\to\infty} \abs{\frac{a_{n+1}}{a_n}} = \rho\neq0\)存在,
	根据\hyperref[theorem:无穷级数.正项级数的比值审敛法]{比值审敛法},
	则当\(\rho \abs{x} < 1\)即\(\abs{x} < \frac1\rho\)时,
	级数\(\sum_{n=1}^\infty \abs{a_n x^n}\)收敛,
	从而级数\(\sum_{n=0}^\infty a_n x^n\)绝对收敛;
	再根据\cref{theorem:无穷级数.绝对发散的特殊情况},
	当\(\rho \abs{x} > 1\)即\(\abs{x} > \frac1\rho\)时,
	级数\(\sum_{n=1}^\infty \abs{a_n x^n}\)发散,
	并且\begin{equation*}
		(\exists N\in\mathbb{N})
		(\forall n\in\mathbb{N})
		[
			n > N
			\implies
			\abs{a_{n+1} x^{n+1}} > \abs{a_n x^n}
		].
	\end{equation*}

	\item 如果\(\rho=0\),
	则对\(\forall x\neq0\),
	有\(\lim_{n\to\infty} \abs{\frac{a_{n+1} x^{n+1}}{a_n x^n}} = 0\),
	所以级数\(\sum_{n=1}^\infty \abs{a_n x^n}\)收敛,
	从而级数\(\sum_{n=0}^\infty a_n x^n\)绝对收敛.
	于是\(R=+\infty\).
		\item 如果\(\rho=+\infty\),
	则对\(\forall x\neq0\),
	级数\(\sum_{n=0}^\infty a_n x^n\)必发散,
	否则由\cref{theorem:无穷级数.阿贝尔定理1} 知道,
	\(\exists x\neq0\)使得\(\sum_{n=1}^\infty \abs{a_n x^n}\)收敛.
	于是\(R=0\).
	\qedhere
\end{enumerate}
\end{proof}
\end{theorem}
\begin{remark}
当级数缺项(即某些项取值为零)时,
不能直接运用\cref{theorem:无穷级数.幂级数的收敛半径的求法2} 求解幂级数的收敛半径,
可以把\(x\)看成常数,把\(\sum_{n=0}^\infty a_n x^n\)看成常数项级数,
使用合适的审敛法(例如\hyperref[theorem:无穷级数.正项级数的比值审敛法]{比值审敛法},
或\hyperref[theorem:无穷级数.正项级数的根值审敛法]{根值审敛法}),
或者对幂级数使用换元法.
\end{remark}

\begin{example}
%@see: 《数学分析(第二版 下册)》(陈纪修) P86 例10.3.3
考察幂级数\(\sum_{n=0}^\infty \frac{n^n}{n!} x^n\)的收敛情况.
\begin{solution}
因为\begin{equation*}
	\lim_{n\to\infty} \abs{\frac{a_{n+1}}{a_n}}
	= \lim_{n\to\infty} \frac{(n+1)^{n+1} / (n+1)!}{n^n / n!}
	= e,
\end{equation*}
所以收敛半径为\(R=1/e\).

当\(x=1/e\)时,
\(\sum_{n=0}^\infty \frac{n^n}{n!} x^n\)是正项级数,
由\hyperref[example:无穷乘积.斯特林公式]{斯特林公式}
\(n! \sim \sqrt{2\pi n} (n/e)^n\ (n\to\infty)\)有\begin{equation*}
	\frac{n^n}{n!} x^n
	\sim
	\frac{n^n}{\sqrt{2\pi} n^{n+1/2} e^{-n}} \cdot \frac1{e^n}
	= \frac1{\sqrt{2\pi n}}
	\quad(n\to\infty),
\end{equation*}
可知\(\sum_{n=0}^\infty \frac{n^n}{n!} x^n\)在点\(x=1/e\)发散.

当\(x=-1/e\)时,
\(\sum_{n=0}^\infty \frac{n^n}{n!} x^n\)是交错级数,
由于\begin{equation*}
	\abs{\frac{(n+1)^{n+1} x^{n+1} / (n+1)!}{n^n x^n / n!}}
	= \frac1e \left( 1 + \frac1n \right)^n
	< 1,
\end{equation*}
且\begin{equation*}
	\abs{\frac{n^n}{n!} x^n}
	\sim
	\frac1{\sqrt{2\pi n}}
	\to 0
	\quad(n\to\infty),
\end{equation*}
可知\(\sum_{n=0}^\infty \frac{n^n}{n!} x^n\)
在点\(x=-1/e\)成为莱布尼茨级数,必定收敛.

综上所述,
\(\sum_{n=0}^\infty \frac{n^n}{n!} x^n\)的收敛域是\([-1/e,1/e)\).
\end{solution}
\end{example}

\begin{example}
%@see: 《高等数学(第六版 下册)》 P273 例1
求幂级数\begin{equation*}
	x-\frac{x^2}{2}+\frac{x^3}{3}-\dotsb+(-1)^{n-1}\frac{x^n}{n}+\dotsb
\end{equation*}的收敛半径与收敛域.
\begin{solution}
因为\begin{equation*}
	\rho = \lim_{n\to\infty} \abs{\frac{a_{n+1}}{a_n}}
	= \lim_{n\to\infty} \frac{n}{n+1} = 1,
\end{equation*}
所以收敛半径\begin{equation*}
	R = \frac1\rho = 1.
\end{equation*}

对于端点\(x=1\),级数成为交错级数\begin{equation*}
	1-\frac{1}{2}+\frac{1}{3}-\dotsb+(-1)^{n-1}\frac{1}{n}+\dotsb,
\end{equation*}
由\cref{example:无穷级数.交错级数1} 可知,此级数收敛;
对于端点\(x=-1\),级数成为\begin{equation*}
	-1-\frac{1}{2}-\frac{1}{3}-\dotsb-\frac{1}{n}-\dotsb,
\end{equation*}
此级数发散.
综上所述,收敛域是\((-1,1]\).
\end{solution}
\end{example}
利用上例的结果可以计算出交错级数\begin{equation*}
	1-\frac{1}{2}+\frac{1}{3}-\dotsb+(-1)^{n-1}\frac{1}{n}+\dotsb.
\end{equation*}
显然有\begin{align*}
	\sum_{n=1}^\infty (-1)^{n-1} \frac{1}{n}
	&= \eval{\sum_{n=1}^\infty (-1)^{n-1} \frac{x^n}{n}}_{x=1} \\
	&= \eval{\sum_{n=1}^\infty (-1)^{n-1} \int_0^x \left(\frac{x^n}{n}\right)' \dd{x}}_{x=1} \\
	&= \eval{\sum_{n=1}^\infty (-1)^{n-1} \int_0^x x^{n-1} \dd{x}}_{x=1} \\
	&= \eval{\int_0^x \sum_{n=1}^\infty (-1)^{n-1} x^{n-1} \dd{x}}_{x=1} \\
	&= \eval{\int_0^x \frac{1}{1+x} \dd{x}}_{x=1} \\
	&= \eval{\ln(1+x)}_{x=1} = \ln2.
\end{align*}
\begin{example}
%@see: 《2009年全国硕士研究生入学统一考试(数学一)》三解答题/第16题
设\(a_n\)是曲线\(y=x^n\)与\(y=x^{n+1}\)所围成区域的面积,
计算级数\(\sum_{n=1}^\infty a_{2n-1}\).
\begin{solution}
因为\(y = x^n\)与\(y = x^{n+1}\)只交于\((0,0)\)和\((1,1)\)两点,
且当\(0 < x < 1\)时总有\(x^n > x^{n+1}\),
所以\begin{equation*}
	a_n = \int_0^1 (x^n - x^{n+1}) \dd{x}
	= \frac1{n+1} - \frac1{n+2},
\end{equation*}
于是\begin{equation*}
	\sum_{n=1}^\infty a_{2n-1}
	= \frac12 - \frac13 + \frac14 - \frac15 + \dotsb.
\end{equation*}
而\begin{equation*}
	1 - \left(\frac12 - \frac13 + \frac14 - \frac15 + \dotsb\right)
	= \ln2,
\end{equation*}
因此\(\sum_{n=1}^\infty a_{2n-1} = 1 - \ln2\).
\end{solution}
\end{example}

\begin{example}
%@see: 《高等数学(第六版 下册)》 P273 例2
求幂级数\begin{equation*}
	1+x+\frac{1}{2!}x^2+\dotsb+\frac{1}{n!}x^n+\dotsb
\end{equation*}的收敛域.
\begin{solution}
因为\begin{equation*}
	\rho = \lim_{n\to\infty} \abs{\frac{a_{n+1}}{a_n}}
	= \lim_{n\to\infty} \frac{n!}{(n+1)!}
	= \lim_{n\to\infty} \frac{1}{n+1}
	= 0,
\end{equation*}
所以收敛半径\(R = +\infty\),从而收敛域是\((-\infty,+\infty)\).
\end{solution}
\end{example}

\begin{example}
%@see: 《高等数学(第六版 下册)》 P273 例3
求幂级数\(\sum_{n=0}^\infty n! x^n\)的收敛半径.
\begin{solution}
因为\begin{equation*}
	\rho
	= \lim_{n\to\infty} \abs{\frac{a_{n+1}}{a_n}}
	= \lim_{n\to\infty} \frac{(n+1)!}{n!}
	= \lim_{n\to\infty} (n+1)
	= +\infty,
\end{equation*}
所以收敛半径\(R = 0\),
即级数仅在点\(x = 0\)处收敛.
\end{solution}
\end{example}

\begin{example}
%@see: 《高等数学(第六版 下册)》 P274 例4
求幂级数\(\sum_{n=0}^\infty \frac{(2n)!}{(n!)^2} x^{2n}\)的收敛半径.
\begin{solution}
级数缺少奇次幂的项,\cref{theorem:无穷级数.幂级数的收敛半径的求法2} 不能直接应用,
我们根据\cref{theorem:无穷级数.正项级数的比值审敛法的上下极限形式} 来求收敛半径:
\begin{equation*}
	\lim_{n\to\infty} \abs{
		{\frac{[2(n+1)]!}{[(n+1)!]^2} x^{2(n+1)}}
		\Bigg/
		{\frac{(2n)!}{(n!)^2} x^{2n}}
	}
	= \lim_{n\to\infty} \abs{\frac{(2n+2)(2n+1)}{(n+1)^2} x^2}
	= 4 x^2.
\end{equation*}

当\(4 x^2 < 1\)即\(\abs{x} < 1/2\)时级数收敛;
当\(4 x^2 > 1\)即\(\abs{x} > 1/2\)时级数发散.
所以收敛半径\(R = 1/2\).
\end{solution}
\end{example}

\begin{example}
%@see: 《高等数学(第六版 下册)》 P274 例4
求幂级数\(\sum_{n=1}^\infty \frac{(x-1)^n}{2^n \cdot n}\)的收敛域.
\begin{solution}
令\(t = x-1\),上述级数变为\begin{equation*}
	\sum_{n=1}^\infty \frac{t^n}{2^n \cdot n}.
\end{equation*}
因为\begin{equation*}
	\rho
	= \lim_{n\to\infty} \abs{\frac{a_{n+1}}{a_n}}
	= \lim_{n\to\infty} \frac{2^n \cdot n}{2^{n+1} \cdot (n+1)}
	= \frac{1}{2},
\end{equation*}
所以收敛半径\(R_t = 2\),而原级数的收敛区间为\(-1<x<3\).

当\(x=3\)时,级数成为\(\sum_{n=1}^\infty \frac{1}{n}\),这级数发散;
当\(x=-1\)时,级数成为\(\sum_{n=1}^\infty \frac{(-1)^n}{n}\),这级数收敛.
因此原级数的收敛域为\([-1,3)\).
\end{solution}
\end{example}

\begin{example}
%@see: 《数学分析(第二版 下册)》(陈纪修) P94 习题 8.
设正项级数\(\sum_{n=1}^\infty a_n\)发散,
\(A_n = \sum_{k=1}^n a_k\),
且\(\lim_{n\to\infty} \frac{a_n}{A_n} = 0\).
求幂级数\(\sum_{n=1}^\infty a_n x^n\)的收敛半径.
\begin{solution}
设\(\sum_{n=1}^\infty a_n x^n\)的收敛半径为\(R_1\),
\(\sum_{n=1}^\infty A_n x^n\)的收敛半径为\(R_2\).

由于\(a_n\geq0\ (n=1,2,\dotsc)\),
所以\(a_n \leq A_n\),
从而有\(R_1 \geq R_2\).

又因为\(\sum_{n=1}^\infty a_n\)发散,
由\cref{theorem:无穷级数.正项级数的比值审敛法} 可知
\(\lim_{n\to\infty} \frac{a_{n+1}}{a_n} \geq 1\),
所以\(R_1 \leq 1\).

由于\begin{equation*}
	\lim_{n\to\infty} \frac{A_n}{A_{n+1}}
	= \lim_{n\to\infty} \frac{A_{n+1} - a_{n+1}}{A_{n+1}}
	= 1,
\end{equation*}
所以\(R_2=1\).

综上所述,既然\(1 \leq R_1 \leq 1\),则必有\(R_1=1\).
\end{solution}
\end{example}

\subsection{幂级数的性质}
\begin{theorem}[阿贝尔第二定理]\label{theorem:无穷级数.阿贝尔定理2}
%@see: 《数学分析(第二版 下册)》(陈纪修) P87 定理10.3.3(Abel第二定理)
设幂级数\(\sum_{n=0}^\infty a_n x^n\)的收敛半径为\(R\),
则\begin{itemize}
	\item \(\sum_{n=0}^\infty a_n x^n\)在\((-R,R)\)上内闭一致收敛;
	\item \(\sum_{n=0}^\infty a_n x^n\)在包含于它的收敛域的任意闭区间上一致收敛,
	即\begin{itemize}
		\item 若\(\sum_{n=0}^\infty a_n x^n\)在点\(x=R\)收敛,
		则它在任意闭区间\([a,R]\subseteq(-R,R]\)上一致收敛;
		\item 若\(\sum_{n=0}^\infty a_n x^n\)在点\(x=-R\)收敛,
		则它在任意闭区间\([-R,b]\subseteq[-R,R)\)上一致收敛;
		\item 若\(\sum_{n=0}^\infty a_n x^n\)在点\(x=\pm R\)都收敛,
		则它在闭区间\([-R,R]\)上一致收敛.
	\end{itemize}
\end{itemize}
%TODO proof
\end{theorem}
\begin{remark}
我们可以把\hyperref{theorem:无穷级数.阿贝尔定理2}{阿贝尔第二定理}的结论推广为:
幂级数在它的收敛域上内闭一致收敛.
\end{remark}

根据\hyperref[theorem:无穷级数.阿贝尔定理2]{阿贝尔第二定理},
我们可以得到幂级数的如下几个性质.
%\cref{theorem:无穷级数.幂级数的和函数的性质1}
%\cref{theorem:无穷级数.幂级数的和函数的性质2}
%\cref{theorem:无穷级数.幂级数的和函数的性质3}

\begin{property}\label{theorem:无穷级数.幂级数的和函数的性质1}
%@see: 《数学分析(第二版 下册)》(陈纪修) P87 定理10.3.4
%@see: 《高等数学(第六版 下册)》 P276 性质1
幂级数\(\sum_{n=0}^\infty a_n x^n\)的和函数在其收敛域上连续.
\begin{proof}
幂级数的一般项是幂函数,显然是连续函数.
由\hyperref[theorem:无穷级数.阿贝尔定理2]{阿贝尔第二定理},
\(\sum_{n=0}^\infty a_n x^n\)在其收敛域上内闭一致收敛.
根据\hyperref[theorem:函数项级数.连续函数项级数的内闭一致收敛性保证和函数的连续性]{一致收敛函数项级数的和函数的连续性},
\(\sum_{n=0}^\infty a_n x^n\)在包含于收敛域中的任意闭区间上连续,
因而在它的整个收敛域上连续.
\end{proof}
\end{property}

\begin{property}\label{theorem:无穷级数.幂级数的和函数的性质2}
%@see: 《数学分析(第二版 下册)》(陈纪修) P87 定理10.3.5
%@see: 《高等数学(第六版 下册)》 P276 性质2
幂级数\(\sum_{n=0}^\infty a_n x^n\)
在包含于其收敛域中的任意闭区间上
可以逐项求积分,
即对于其收敛域内的任意两点\(a,b\),
有\begin{equation}
	\int_a^b \sum_{n=0}^\infty a_n x^n \dd{x}
	= \sum_{n=0}^\infty \int_a^b a_n x^n \dd{x};
\end{equation}
特别地,有\begin{equation}
	\int_0^x \sum_{n=0}^\infty a_n t^n \dd{t}
	= \sum_{n=0}^\infty \frac{a_n}{n+1} x^{n+1},
\end{equation}
且逐项积分所得到的幂级数\(\sum_{n=0}^\infty \frac{a_n}{n+1} x^{n+1}\)
和原幂级数\(\sum_{n=0}^\infty a_n x^n\)
具有相同的收敛半径.
\begin{proof}
由\hyperref[theorem:无穷级数.阿贝尔定理2]{阿贝尔第二定理},
\(\sum_{n=0}^\infty a_n x^n\)在其收敛域上内闭一致收敛.
应用\hyperref[theorem:函数项级数.连续函数项级数的一致收敛性保证和函数的可积性]{一致收敛函数项级数的逐项积分定理},
就得到幂级数的逐项可积性.

由于\begin{equation*}
	\varlimsup_{n\to\infty} \sqrt[n+1]{\frac{\abs{a_n}}{n+1}}
	= \varlimsup_{n\to\infty} \sqrt[n]{\abs{a_n}},
\end{equation*}
所以\(\sum_{n=0}^\infty \frac{a_n}{n+1} x^{n+1}\)
与\(\sum_{n=0}^\infty a_n x^n\)具有相同的收敛半径.
\end{proof}
\end{property}
\begin{remark}
虽然逐项积分所得的幂级数\(\sum_{n=0}^\infty \frac{a_n}{n+1} x^{n+1}\)
和原幂级数\(\sum_{n=0}^\infty a_n x^n\)收敛半径相同,
但是它的收敛域相比于原幂级数可能扩大.

%@see: 《数学分析(第二版 下册)》(陈纪修) P89 例10.3.4
例如,幂级数\begin{equation*}
	\sum_{n=1}^\infty (-1)^{n-1} x^{2n-2}
\end{equation*}的收敛域是\((-1,1)\),
但逐项积分所得的幂级数\begin{equation*}
	\sum_{n=1}^\infty \frac{(-1)^{n-1}}{2n-1} x^{2n-1}
\end{equation*}的收敛域是\([-1,1]\).

%@see: 《数学分析(第二版 下册)》(陈纪修) P89 例10.3.5
又例如,幂级数\begin{equation*}
	\sum_{n=1}^\infty (-1)^{n-1} x^{n-1}
\end{equation*}的收敛域是\((-1,1)\),
但逐项积分所得的幂级数\begin{equation*}
	\sum_{n=1}^\infty \frac{(-1)^{n-1}}{n} x^n
\end{equation*}的收敛域是\((-1,1]\).
\end{remark}

\begin{property}\label{theorem:无穷级数.幂级数的和函数的性质3}
%@see: 《数学分析(第二版 下册)》(陈纪修) P89 定理10.3.6
%@see: 《高等数学(第六版 下册)》 P276 性质3
幂级数\(\sum_{n=0}^\infty a_n x^n\)
在其收敛区间上可以逐项求导,
即\begin{equation}
	\dv{x} \sum_{n=0}^\infty a_n x^n
	= \sum_{n=0}^\infty \dv{x} a_n x^n
	= \sum_{n=1}^\infty n a_n x^{n-1}
	\quad(-R<x<R),
\end{equation}
其中\(R\)是\(\sum_{n=0}^\infty a_n x^n\)的收敛半径,
且逐项求导所得的幂级数\(\sum_{n=1}^\infty n a_n x^{n-1}\)的收敛半径也是\(R\).
\end{property}
\begin{remark}
虽然逐项积分所得的幂级数\(\sum_{n=1}^\infty n a_n x^{n-1}\)
和原幂级数\(\sum_{n=0}^\infty a_n x^n\)收敛半径相同,
但是它的收敛域相比于原幂级数可能缩小.

例如,幂级数\begin{equation*}
	\sum_{n=1}^\infty \frac{(-1)^{n-1}}{2n-1} x^{2n-1}
\end{equation*}的收敛域是\([-1,1]\),
但逐项求导所得的幂级数\begin{equation*}
	\sum_{n=1}^\infty (-1)^{n-1} x^{2n-2}
\end{equation*}的收敛域是\((-1,1)\).
\end{remark}

反复利用\cref{theorem:无穷级数.幂级数的和函数的性质3} 可得以下结论.
\begin{proposition}
幂级数\(\sum_{n=0}^\infty a_n x^n\)的和函数
在其收敛区间\((-R,R)\)内具有任意阶导数.
\end{proposition}

\begin{example}
%@see: 《高等数学(第六版 下册)》 P276 例6
求幂级数\(\sum_{n=0}^\infty \frac{x^n}{n+1}\)的和函数.
\begin{solution}
先求收敛域.
由\begin{equation*}
	\lim_{n\to\infty} \abs{\frac{a_{n+1}}{a_n}}
	= \lim_{n\to\infty} \frac{n+1}{n+2}
	= 1,
\end{equation*}
得收敛半径\(R=1\).

当\(x = -1\)时,
幂级数成为\(\sum_{n=0}^\infty \frac{(-1)^n}{n+1}\),
是莱布尼茨级数,收敛.
当\(x = 1\)时,
幂级数成为\(\sum_{n=0}^\infty \frac{1}{n+1}\),
是发散的.
因此收敛域是\(I = [-1,1)\).

设和函数为\(s(x)\),即\begin{equation*}
	s(x) = \sum_{n=0}^\infty \frac{x^n}{n+1},
	\quad x\in[-1,1).
\end{equation*}
于是\begin{equation*}
	x \cdot s(x) = \sum_{n=0}^\infty \frac{x^{n+1}}{n+1}.
\end{equation*}

利用\cref{theorem:无穷级数.幂级数的和函数的性质3},逐项求导,并由\begin{equation*}
	\frac{1}{1-x} = 1+x+x^2+\dotsb+x^n+\dotsb
	\quad(-1<x<1),
\end{equation*}
得\begin{equation*}
	[x \cdot s(x)]'
	= \sum_{n=0}^\infty \left( \frac{x^{n+1}}{n+1} \right)'
	= \sum_{n=0}^\infty x^n
	= \frac{1}{1-x}
	\quad(\abs{x}<1).
\end{equation*}
对上式积分,
得\begin{equation*}
	x \cdot s(x) = \int_0^x \frac{1}{1-x} \dd{x} = -\ln(1-x)
	\quad(-1 \leq x < 1).
\end{equation*}
于是,当\(x\neq0\)时,有\(s(x) = -\frac{1}{x} \ln(1-x)\).

而\(s(0)\)可由\(s(0) = a_0 = 1\)得出,
或者由和函数的连续性得到,即\begin{equation*}
	s(0)
	= \lim_{x\to0} s(x)
	= \lim_{x\to0} \left[ -\frac{1}{x} \ln(1-x) \right]
	= 1.
\end{equation*}
故\begin{equation*}
	s(x) = \left\{ \begin{array}{cl}
		-\frac{1}{x} \ln(1-x), & x\in[-1,0)\cup(0,1), \\
		1, & x=0.
	\end{array} \right.
\end{equation*}
\end{solution}
%@Mathematica: Sum[x^n/(n + 1), {n, 0, Infinity}]
%@Mathematica: SumConvergence[x^n/(n + 1), n]
\end{example}

\begin{example}
%@see: 《2012年全国硕士研究生入学统一考试(数学一)》三解答题/第17题
求幂级数\(\sum_{n=0}^\infty \frac{4n^2+4n+3}{2n+1} x^{2n}\)的和函数.
%TODO
%@Mathematica: Sum[(4 n^2 + 4 n + 3)/(2 n + 1) x^(2 n), {n, 0, Infinity}] // FullSimplify
%@Mathematica: SumConvergence[(4 n^2 + 4 n + 3)/(2 n + 1) x^(2 n), n]
\end{example}
\begin{example}
%@see: 《2017年全国硕士研究生入学统一考试(数学一)》二填空题/第12题
求幂级数\(\sum_{n=1}^\infty (-1)^{n-1} n x^{n-1}\)的和函数.
%TODO
% 和函数为\(S(x) = \frac1{(1+x)^2}\ (-1<x<1)\).
%@Mathematica: Sum[(-1)^(n - 1) n x^(n - 1), {n, 0, Infinity}]
%@Mathematica: SumConvergence[(-1)^(n - 1) n x^(n - 1), n]
\end{example}

\begin{example}
%@see: 《2020年全国硕士研究生入学统一考试(数学一)》三解答题/第17题
设数列\(\{a_n\}\)满足\(a_1 = 1,
(n+1) a_{n+1} = \left( n + \frac12 \right) a_n\).
求幂级数\(\sum_{n=1}^\infty a_n x^n\)的收敛半径与和函数.
\begin{solution}
幂级数\(\sum_{n=1}^\infty a_n x^n\)的收敛半径为
\(\lim_{n\to\infty} \abs{\frac{a_n}{a_{n+1}}} = 1\).

假设它的和函数为\(S(x)\),
那么\begin{align*}
	S'(x)
	&= \sum_{n=1}^\infty \dv{x} a_n x^n \\
	&= \sum_{n=1}^\infty n a_n x^{n-1}
	= \sum_{n=0}^\infty (n+1) a_{n+1} x^n \\
	&= 1 + \sum_{n=1}^\infty (n+1) a_{n+1} x^n \\
	% 代入\((n+1) a_{n+1} = \left( n + \frac12 \right) a_n\)
	&= 1 + \sum_{n=1}^\infty \left( n + \frac12 \right) a_n x^n \\
	&= 1 + \sum_{n=1}^\infty n a_n x^n + \frac12 \sum_{n=1}^\infty a_n x^n.
\end{align*}
于是得到一个一阶微分方程\begin{equation*}
	S'(x) = 1 + x S'(x) + \frac12 S(x),
\end{equation*}
整理得\begin{equation*}
	S'(x) - \frac1{2(1-x)} S(x) = \frac1{1-x},
\end{equation*}
其通解为\begin{equation*}
	S(x) = e^{\int \frac1{2(1-x)} \dd{x}}
	\left[ C + \int \frac1{1-x} e^{-\int \frac1{2(1-x)} \dd{x}} \dd{x} \right]
	% 由于收敛区间为\(-1<x<1\),所以\(\ln\abs{1-x} = \ln(1-x)\).
	= \frac{C}{\sqrt{1-x}} - 2,
\end{equation*}
其中\(C\)是任意常数.
因为\(S(0) = \frac{C}{\sqrt{1-0}} - 2 = 0\),
所以\(C = 2\),
即\begin{equation*}
	S(x) = \frac2{\sqrt{1-x}} - 2
	\quad(-1<x<1).
\end{equation*}
\end{solution}
\end{example}
\begin{example}
%@see: 《2021年全国硕士研究生入学统一考试(数学一)》三解答题/第18题
设\(u_n(x) = e^{-nx} + \frac{x^{n+1}}{n(n+1)}\ (n=1,2,\dotsc)\).
求级数\(\sum_{n=1}^\infty u_n(x)\)的收敛域及和函数.
\begin{solution}
记\(S(x) = \sum_{n=1}^\infty u_n(x),
S_1(x) = \sum_{n=1}^\infty e^{-nx},
S_2(x) = \sum_{n=1}^\infty \frac{x^{n+1}}{n(n+1)}\).

我们知道,等比级数\(\sum_{n=1}^\infty e^{-nx}\)收敛当且仅当其公比的绝对值小于\(1\),
即\(\abs{e^{-x}} < 1\),
这就说明级数\(\sum_{n=1}^\infty e^{-nx}\)的收敛域为\((0,+\infty)\).
再由\cref{theorem:无穷级数.幂级数的收敛半径的求法2} 可知,
级数\(\sum_{n=1}^\infty \frac{x^{n+1}}{n(n+1)}\)的收敛半径为\begin{equation*}
	R = \lim_{n\to\infty} \abs{\frac{a_n}{a_{n+1}}}
	= \lim_{n\to\infty} \abs{\frac{(n+1)(n+2)}{n(n+1)}}
	= 1;
\end{equation*}
当\(x=1\)时,级数\(\sum_{n=1}^\infty \frac{x^{n+1}}{n(n+1)}\)成为
\(p\)级数\(\sum_{n=1}^\infty \frac1{n(n+1)}\),
由\cref{example:无穷级数.p级数的收敛性} 可知该级数收敛.
于是级数\(\sum_{n=1}^\infty u_n(x)\)的收敛域为
\((0,+\infty) \cap (-1,1] = (0,1]\).

由\cref{example:无穷级数.等比级数的收敛性} 可知\begin{equation*}
	S_1(x) = \frac1{e^x-1}.
\end{equation*}
特别地,有\begin{equation*}
	S_1(1) = \frac1{e-1}.
\end{equation*}

当\(0<x<1\)时,
有\begin{align*}
	S_2(x) &= \sum_{n=1}^\infty \left( \frac{x^{n+1}}{n} - \frac{x^{n+1}}{n+1} \right) \\
	&= x \sum_{n=1}^\infty \frac{x^n}{n} - \sum_{n=2}^\infty \frac{x^n}{n} \\
	&= x + (x-1) \sum_{n=1}^\infty \frac{x^n}{n} \\
	&= x + (x-1) \int_0^x \sum_{n=1}^\infty \dv{t}(\frac{t^n}{n}) \dd{t} \\
	&= x + (1-x) \ln(1-x).
\end{align*}
注意到在上式中函数\(S_2\)在收敛域的端点\(x=1\)没有定义,需要另行计算.
由于\hyperref[theorem:无穷级数.幂级数的和函数的性质1]{幂级数的和函数在收敛域上连续},
所以\begin{equation*}
	S_2(1) = \lim_{x\to1^-} S_2(x)
	= \lim_{x\to1^-} [x + (1-x) \ln(1-x)]
	%\cref{example:微分中值定理.洛必达法则.零乘无穷大型2}
	= 1.
\end{equation*}

综上所述,级数\(\sum_{n=1}^\infty u_n(x)\)的和函数为\begin{equation*}
	S(x)
	= S_1(x) + S_2(x) \\
	= \left\{ \begin{array}{cl}
		\frac1{e^x-1} + x + (1-x) \ln(1-x),
			& 0<x<1, \\
		\frac{e}{e-1},
			& x=1.
	\end{array} \right.
\end{equation*}
\end{solution}
\end{example}

\subsection{幂级数的运算}
\begin{definition}
设幂级数\(\sum_{n=0}^\infty a_n x^n\)
和\(\sum_{n=0}^\infty b_n x^n\)
分别在区间\((-R,R)\)和\((-R',R')\)内收敛.
定义:
\begin{gather}
	\left(\sum_{n=0}^\infty a_n x^n\right)
	+ \left(\sum_{n=0}^\infty b_n x^n\right)
	\defeq
	\sum_{n=0}^\infty (a_n+b_n) x^n, \\
	\left(\sum_{n=0}^\infty a_n x^n\right)
	- \left(\sum_{n=0}^\infty b_n x^n\right)
	\defeq
	\sum_{n=0}^\infty (a_n-b_n) x^n, \\
	\left(\sum_{n=0}^\infty a_n x^n\right)
	\cdot \left(\sum_{n=0}^\infty b_n x^n\right)
	\defeq
	\sum_{n=0}^\infty \left(
		\sum_{i=0}^n a_i b_{n-i}
	\right) x^n.
\end{gather}
\end{definition}

\begin{definition}
设幂级数\(\sum_{n=0}^\infty a_n x^n\)
和\(\sum_{n=0}^\infty b_n x^n\)
分别在区间\((-R,R)\)和\((-R',R')\)内收敛.

记\begin{equation*}
	\frac{
		\sum_{n=0}^\infty a_n x^n
	}{
		\sum_{n=0}^\infty b_n x^n
	}
	= \sum_{n=0}^\infty c_n x^n,
\end{equation*}
假设\(b_0 \neq 0\).
为了决定系数\(c_0,c_1,\dotsc,c_n,\dotsc\),
可以将级数\(\sum_{n=0}^\infty b_n x^n\)
与\(\sum_{n=0}^\infty c_n x^n\)相乘,
并令乘积中各项的系数分别等于级数\(\sum_{n=0}^\infty a_n x^n\)中同次幂的系数,
即得\begin{equation*}
	\begin{cases}
		a_0 = b_0 c_0, \\
		a_1 = b_1 c_0 + b_0 c_1, \\
		a_2 = b_2 c_0 + b_1 c_1 + b_0 c_2, \\
		\hdotsfor{1} \\
	\end{cases}
\end{equation*}
由这些方程就可以顺序地求出\(c_0,c_1,\dotsc,c_n,\dotsc\).

相除后所得的幂级数\(\sum_{n=0}^\infty c_n x^n\)的收敛区间可能比原来的两级数的收敛区间小得多.
\end{definition}

例如,级数\begin{equation*}
	\sum_{n=0}^\infty a_n x^n
	= 1 + 0x + \dotsb + 0x^n + \dotsb
\end{equation*}与\begin{equation*}
	\sum_{n=0}^\infty b_n x^n
	= 1 - x + 0x^2 + \dotsb + 0x^n + \dotsb
\end{equation*}在整个数轴上收敛,
但是这两个级数的商\(\sum_{n=0}^\infty c_n x^n
= \sum_{n=0}^\infty x^n\)仅在区间\((-1,1)\)内收敛.

\section{函数展开成幂级数}
前面讨论了幂级数的收敛域及其和函数的性质.
但在许多应用中,我们遇到的却是相反的问题:
给定函数\(f(x)\),
要考虑它是否能在某个区间内“展开成幂级数”,
就是说,是否能找到这样一个幂级数,它在某区间内收敛,且其和恰好就是给定的函数\(f(x)\).
如果能找到这样的幂级数,我们就说“函数\(f(x)\)在该区间内能展开成幂级数”,
也说“这个幂级数在该区间内表达了函数\(f(x)\)”.

\subsection{泰勒展式}
\begin{definition}
设函数\(f\colon D\to\mathbb{R}\).
如果在区间\(I \subseteq D\)内,
存在收敛的幂级数\(\sum_{n=0}^\infty a_n x^n\)使得\begin{equation*}
	f(x) = \sum_{n=0}^\infty a_n x^n
	\quad(x \in I)
\end{equation*}成立,
则称“函数\(f(x)\)在区间\(I\)内能展开成幂级数”.
\end{definition}

\begin{definition}
级数\begin{equation*}
	\sum_{n=0}^\infty \frac{1}{n!} f^{(n)}(x_0) (x-x_0)^n
\end{equation*}
为“函数\(f(x)\)在点\(x_0\)处的\DefineConcept{泰勒级数}(Taylor series)”.
%@see: https://mathworld.wolfram.com/TaylorSeries.html

等式\begin{equation*}
	f(x) = \sum_{n=0}^\infty {\frac{1}{n!} f^{(n)}(x_0) (x-x_0)^n},
	\quad x \in U(x_0).
\end{equation*}
称为“函数\(f(x)\)在点\(x_0\)处的\DefineConcept{泰勒展开式}”.
\end{definition}

可以看出,函数\(f(x)\)在\(U(x_0)\)内能展开成幂级数的充分必要条件为:
泰勒级数\begin{equation*}
	\sum_{n=0}^\infty {\frac{1}{n!} f^{(n)}(x_0) (x-x_0)^n}
\end{equation*}在\(U(x_0)\)内收敛,
且收敛到\(f(x)\).
但是“泰勒级数收敛到函数\(f(x)\)”这一叙述非常模糊,所以我们有以下定理:
\begin{theorem}
设函数\(f(x)\)在点\(x_0\)的某一邻域\(U(x_0)\)内具有各阶导数,
则\(f(x)\)在该邻域内能展开成泰勒级数的充分必要条件是:
在该邻域内\(f(x)\)的泰勒公式中的余项\(R_n(x)\)满足\begin{equation*}
	\lim_{n\to\infty} R_n(x) = 0,
	\quad x \in U(x_0).
\end{equation*}
\end{theorem}

\begin{definition}
级数\begin{equation*}
	\sum_{n=0}^\infty \frac{1}{n!} f^{(n)}(0) x^n
\end{equation*}
称为函数\(f(x)\)的\DefineConcept{麦克劳林级数}.

等式\begin{equation*}
	f(x) = \sum_{n=0}^\infty {\frac{1}{n!} f^{(n)}(0) x^n}
	\quad(\abs{x} < r)
\end{equation*}
称为函数\(f(x)\)的\DefineConcept{麦克劳林展开式}.
\end{definition}

\begin{theorem}
若函数\(f(x)\)在区间\(I\)内能展开成幂级数,
则它的幂级数展开式是唯一的.
\end{theorem}

要把函数\(f(x)\)展开成麦克劳林级数,可以按照下列步骤进行:
\begin{algorithm}[麦克劳林展开]\label{algorithm:函数项级数.麦克劳林展开}
\hfill
\begin{enumerate}
	\item 从一阶导数\(f'(x)\)开始,
	由低阶导数到高阶导数,
	逐个计算函数\(f(x)\)在点\(x=0\)的各阶导数\begin{equation*}
		f'(0),f''(0),\dotsc,f^{(n)}(0),\dotsc.
	\end{equation*}
	在上述计算过程中,
	\begin{itemize}
		\item 如果在\(x=0\)处某阶导数不存在,就停止展开,因为这就说明函数\(f(x)\)不能展开为麦克劳林级数;

		\item 如果各阶倒数在点\(x=0\)都存在,就继续展开.
	\end{itemize}

	\item 写出幂级数\begin{equation*}
		f(0) + f'(0) x + \frac{f''(0)}{2!} x^2 + \dotsb + \frac{f^{(n)}(0)}{n!} x^n + \dotsb,
	\end{equation*}
	并求出收敛半径\(R\).

	\item 利用余项\(R_n(x)\)的表达式\begin{equation*}
		R_n(x) = \frac{1}{(n+1)!} f^{(n+1)}(\theta x) x^{n+1}
		\quad(0 < \theta < 1),
	\end{equation*}
	考察当\(x\)在区间\((-R,R)\)内时余项的极限\(\lim_{n\to\infty} R_n(x)\)是否为零.
	\begin{itemize}
		\item 如果\(\lim_{n\to\infty} R_n(x) \neq 0\),则函数\(f(x)\)不能展开为麦克劳林级数.

		\item 如果\(\lim_{n\to\infty} R_n(x) = 0\),
		则函数\(f(x)\)在区间\((-R,R)\)内的麦克劳林展开式为\begin{equation*}
			f(x) = \sum_{n=0}^\infty \frac{1}{n!} f^{(n)}(0) x^n
			\quad(-R < x < R).
		\end{equation*}
	\end{itemize}
\end{enumerate}
\end{algorithm}

需要注意的是,不要错误地认为“如果一个函数的泰勒级数在点\(x_0\)处收敛,那么该级数就一定收敛到这个函数”.
\begin{example}\label{example:无穷级数.函数的泰勒级数不一定收敛到函数}
最常见的例子是:\begin{equation*}
	f(x) = \left\{ \begin{array}{ll}
		e^{-1/x^2}, & x\neq0, \\
		0, & x=0.
	\end{array} \right.
\end{equation*}
根据导数的定义,以及\(\forall k\in\mathbb{R}: x^k e^{-1/x^2} \to 0\ (x\to0)\),
可知\(f^{(n)}(0) = 0\ (n=0,1,2,\dotsc)\).
于是,函数\(f\)在点\(x=0\)处的泰勒级数的每一项都是零,其和恒等于零;
只不过,当\(x\neq0\)时,\(f(x)\neq0\).
\end{example}

\begin{example}
将函数\(f(x) = e^x\)展开成\(x\)的幂级数.
\begin{solution}
所给函数的各阶导数为\begin{equation*}
	f^{(n)}(x) = e^x
	\quad(n=1,2,\dotsc),
\end{equation*}
因此\begin{equation*}
	f^{(n)}(x) = 1
	\quad(n=0,1,2,\dotsc).
\end{equation*}
于是得级数\begin{equation*}
	1+x+\frac{x^2}{2!}+\dotsb+\frac{x^n}{n!}+\dotsb,
\end{equation*}它的收敛半径\(R = +\infty\).

对于任何有限的数\(x\)和\(\xi\)(\(\xi\)在\(0\)与\(x\)之间),
余项的绝对值为\begin{equation*}
	\abs{R_n(x)} = \abs{\frac{e^{\xi}}{(n+1)!} x^{n+1}}
	< e^{\abs{x}} \cdot \frac{\abs{x}^{n+1}}{(n+1)!}.
\end{equation*}
因为\(e^{\abs{x}}\)有限,
而\(\frac{\abs{x}^{n+1}}{(n+1)!}\)是
收敛级数\(\sum_{n=0}^\infty \frac{\abs{x}^{n+1}}{(n+1)!}\)的一般项,
所以当\(n\to\infty\)时,\(e^{\abs{x}} \cdot \frac{\abs{x}^{n+1}}{(n+1)!}\to0\),
即\begin{equation*}
	\lim_{n\to\infty} \abs{R_n(x)} = 0.
\end{equation*}
于是得展开式\begin{equation*}
	e^x = 1+x+\frac{x^2}{2!}+\dotsb+\frac{x^n}{n!}+\dotsb
	\quad(-\infty<x<+\infty).
\end{equation*}
\end{solution}
\end{example}

\begin{example}
将函数\(f(x) = \sin x\)展开成\(x\)的幂级数.
\begin{solution}
所给函数的各阶导数为\begin{equation*}
	f^{(n)}(x) = \sin\left(x + n\cdot\frac{\pi}{2}\right)
	\quad(n=1,2,\dotsc),
\end{equation*}
易见\(f^{(n)}(0)\)依顺序循环地取\(0,1,0,-1,\dotsc\),于是得级数\begin{equation*}
	x-\frac{x^3}{3!}+\frac{x^5}{5!}-\dotsb+(-1)^k \frac{x^{2k+1}}{(2k+1)!}+\dotsb,
\end{equation*}
它的收敛半径\(R=+\infty\).

对于任何有限的数\(x\)和\(\xi\)(\(\xi\)在\(0\)与\(x\)之间),余项的绝对值为\begin{equation*}
	\abs{R_n(x)}
	= \abs{ \frac{x^{n+1}}{(n+1)!} \sin\left[\xi+\frac{n+1}{2}\pi\right] }
	\leq \frac{\abs{x}^{n+1}}{(n+1)!}.
\end{equation*}
而\begin{equation*}
	\lim_{n\to\infty} \frac{\abs{x}^{n+1}}{(n+1)!} = 0.
\end{equation*}
因而得展开式\begin{equation*}
	\sin x = x-\frac{x^3}{3!}+\frac{x^5}{5!}-\dotsb+(-1)^k \frac{x^{2k+1}}{(2k+1)!}+\dotsb
	\quad(-\infty<x<+\infty).
\end{equation*}
\end{solution}
\end{example}

以上将函数展开成幂级数的例子,是直接按公式\begin{equation*}
a_n = \frac{1}{n!} f^{(n)}(0)
\end{equation*}计算幂级数的系数,最后考察余项\(R_n(x)\)是否趋于零.
这种直接展开的方法计算量较大,而且研究余项即使在初等函数中也不是一件容易的事.
下面介绍间接展开的方法,这就是利用一些已知的函数展开式,通过幂级数的运算(如四则运算、逐项求导、逐项积分)以及变量代换等,将所给函数展开成幂级数.
这样做不但计算简单,而且可以避免研究余项.

前面我们已经求得的幂级数展开式有\begin{gather}
	e^x = \sum_{n=0}^\infty \frac{x^n}{n!}
		\quad(-\infty<x<+\infty), \label{equation:无穷级数.幂级数展开式1} \\
	\sin x = \sum_{k=0}^\infty \frac{(-1)^k}{(2k+1)!} x^{2k+1}
		\quad(-\infty<x<+\infty), \label{equation:无穷级数.幂级数展开式2} \\
	\frac{1}{1+x} = \sum_{n=0}^\infty (-1)^n x^n
		\quad(-1<x<1). \label{equation:无穷级数.幂级数展开式3}
\end{gather}
利用这三个展开式,可以求得许多函数的幂级数展开式.
例如对\cref{equation:无穷级数.幂级数展开式3} 两边从\(0\)到\(x\)积分,可得
\begin{equation}\label{equation:无穷级数.幂级数展开式4}
	\ln(1+x) = \sum_{n=0}^\infty \frac{(-1)^n}{n+1} x^{n+1}
	= \sum_{n=1}^\infty \frac{(-1)^{n-1}}{n} x^n
	\quad(-1<x\leq1);
\end{equation}

对\cref{equation:无穷级数.幂级数展开式2} 两边求导,即得
\begin{equation}\label{equation:无穷级数.幂级数展开式5}
	\cos x = \sum_{k=0}^\infty \frac{(-1)^k}{(2k)!} x^{2k}
	\quad(-\infty<x<+\infty);
\end{equation}

把\cref{equation:无穷级数.幂级数展开式1} 中的\(x\)换成\(x \ln a\),可得
\begin{equation}\label{equation:无穷级数.幂级数展开式6}
	a^x = e^{x \ln a} = \sum_{n=0}^\infty \frac{\ln^n a}{n!} x^n
	\quad(-\infty<x<+\infty);
\end{equation}

把\cref{equation:无穷级数.幂级数展开式3} 中的\(x\)换成\(x^2\),可得
\begin{equation}\label{equation:无穷级数.幂级数展开式7}
	\frac{1}{1+x^2} = \sum_{n=0}^\infty (-1)^n x^{2n}
	\quad(-1<x<1);
\end{equation}

对\cref{equation:无穷级数.幂级数展开式7} 从\(0\)到\(x\)积分,可得
\begin{equation}
	\arctan x = \sum_{n=0}^\infty \frac{(-1)^n}{2n+1} x^{2n+1}
	\quad(-1 \leq x \leq 1).
\end{equation}

下面再举几个用间接法把函数展开成幂级数的例子.

\begin{example}
把函数\(f(x) = (1-x) \ln(1+x)\)展开成\(x\)的幂级数.
\begin{solution}
由\cref{equation:无穷级数.幂级数展开式4}
得\begin{align*}
	f(x) &= (1-x) \sum_{n=1}^\infty \frac{(-1)^{n-1}}{n} x^n \\
	&= \sum_{n=1}^\infty \frac{(-1)^{n-1}}{n} x^n
		- \sum_{n=1}^\infty \frac{(-1)^{n-1}}{n} x^{n+1} \\
	&= \sum_{n=1}^\infty \frac{(-1)^{n-1}}{n} x^n
		- \sum_{n=2}^\infty \frac{(-1)^n}{n-1} x^n \\
	&= x + \sum_{n=2}^\infty \frac{(-1)^{n-1} (2n-1)}{n(n-1)} x^n
	\quad(-1 < x \leq 1).
\end{align*}
\end{solution}
\end{example}

\begin{example}
将函数\(\sin x\)展开成\(\left(x-\frac{\pi}{4}\right)\)的幂级数.
\begin{solution}
因为\begin{align*}
	\sin x &= \sin\left[\frac{\pi}{4}+\left(x-\frac{\pi}{4}\right)\right] \\
	&= \sin\frac{\pi}{4} \cos\left(x-\frac{\pi}{4}\right) + \cos\frac{\pi}{4} \sin\left(x-\frac{\pi}{4}\right) \\
	&= \frac{1}{\sqrt{2}} \left[\cos\left(x-\frac{\pi}{4}\right) + \sin\left(x-\frac{\pi}{4}\right)\right],
\end{align*}
又由\begin{gather*}
	\cos\left(x-\frac{\pi}{4}\right)
	= 1 - \frac{1}{2!} \left(x-\frac{\pi}{4}\right)^2 + \frac{1}{4!} \left(x-\frac{\pi}{4}\right)^4 - \dotsb
	\quad(-\infty < x < +\infty), \\
	\sin\left(x-\frac{\pi}{4}\right)
	= \left(x-\frac{\pi}{4}\right) - \frac{1}{3!} \left(x-\frac{\pi}{4}\right)^3 + \frac{1}{5!} \left(x-\frac{\pi}{4}\right)^5 - \dotsb
	\quad(-\infty < x < +\infty),
\end{gather*}
所以\begin{equation*}
	\sin x = \frac{1}{\sqrt{2}} \left[
	1 + \left(x-\frac{\pi}{4}\right)
	- \frac{1}{2!} \left(x-\frac{\pi}{4}\right)^2
	- \frac{1}{3!} \left(x-\frac{\pi}{4}\right)^3
	+ \dotsb
	\right]
	\quad(-\infty < x < +\infty).
\end{equation*}
\end{solution}
\end{example}

\begin{example}
将函数\(f(x) = \frac{1}{x^2+4x+3}\)展开成\((x-1)\)的幂级数.
\begin{solution}
因为\begin{align*}
	f(x) &= \frac{1}{x^2+4x+3}
	= \frac{1}{(x+1)(x+3)} \\
	&= \frac{1}{2(1+x)} - \frac{1}{2(3+x)} \\
	&= \frac{1}{4\left(1+\frac{x-1}{2}\right)}
	- \frac{1}{8\left(1+\frac{x-1}{4}\right)},
\end{align*}
而\begin{gather*}
	\frac{1}{4\left(1+\frac{x-1}{2}\right)}
	= \frac{1}{4} \sum_{n=0}^\infty \frac{(-1)^n}{2^n} (x-1)^n
	\quad(-1<x<3), \\
	\frac{1}{8\left(1+\frac{x-1}{4}\right)}
	= \frac{1}{8} \sum_{n=0}^\infty \frac{(-1)^n}{4^n} (x-1)^n
	\quad(-3<x<5),
\end{gather*}
所以\begin{equation*}
	f(x) = \frac{1}{x^2+4x+3}
	= \sum_{n=0}^\infty (-1)^n \left(\frac{1}{2^{n+2}}-\frac{1}{2^{2n+3}}\right) (x-1)^n
	\quad(-1<x<3).
\end{equation*}
\end{solution}
\end{example}

\begin{example}
将函数\(\sinh x = \frac{e^x - e^{-x}}{2}\)展开成\(x\)的幂级数,并求展开式成立的区间.
\begin{solution}
由\cref{equation:无穷级数.幂级数展开式1} 可知,\begin{equation*}
	\sinh x
	= \frac{e^x - e^{-x}}{2}
	= \frac{1}{2} \left[
	\sum_{n=0}^\infty \frac{1}{n!} x^n
	- \sum_{n=0}^\infty \frac{1}{n!} (-x)^n
	\right].
\end{equation*}
因为\begin{equation*}
	(-x)^n = \left\{ \begin{array}{cl}
		x^n, & n=2k, \\
		-x^n, & n=2k+1,
	\end{array} \right.
	\quad(k\in\mathbb{N}),
\end{equation*}
所以\begin{equation*}
	\sinh x
	= \sum_{k=0}^\infty \frac{x^{2k+1}}{(2k+1)!}
	\quad(-\infty<x<+\infty).
\end{equation*}
\end{solution}
\end{example}

最后,再举一个用直接法展开的例子.

\begin{example}
将函数\(f(x) = (1+x)^m\)展开成\(x\)的幂级数,其中\(m\)为任意实数.
\begin{solution}
\(f(x)\)的各阶导数为\begin{align*}
	&f'(x) = m (1+x)^{m-1},
	f''(x) = m(m-1) (1+x)^{m-2},
	\dotsc, \\
	&f^{(n)}(x) = m(m-1)(m-2)\dotsm(m-n+1) (1+x)^{m-n},
	\dotsc,
\end{align*}
所以\begin{align*}
	&f(0) = 1,
	f'(0) = m,
	f''(0) = m(m-1),
	\dotsc, \\
	&f^{(n)}(0) = m(m-1)\dotsm(m-n+1),\dotsc,
\end{align*}
于是得级数\begin{equation*}
	1+mx+\frac{m(m-1)}{2!}x^2+\dotsb+\frac{m(m-1)\dotsm(m-n+1)}{n!}x^n+\dotsb.
\end{equation*}
这级数相邻的系数之比的绝对值\begin{align*}
	\abs{\frac{a_{n+1}}{a_n}}
	&= \abs{ \frac{m(m-1)\dotsm(m-n)}{(n+1)!} \Bigg/ \frac{m(m-1)\dotsm(m-n+1)}{n!} } \\
	&= \abs{ \frac{m-n}{n+1} }
	\to 1 \quad(n\to\infty),
\end{align*}
因此,对于任何实数\(m\)这级数在开区间\((-1,1)\)内收敛.

为了避免直接研究余项,设这级数在开区间\((-1,1)\)内收敛到函数\(F(x)\):\begin{align*}
	F(x)
	&= 1+mx+\frac{m(m-1)}{2!}x^2+\dotsb \\
	&\hspace{20pt}+\frac{m(m-1)\dotsm(m-n+1)}{n!}x^n+\dotsb
	\quad(-1<x<1),
\end{align*}
下面证明\(F(x) = (1+x)^m\ (-1<x<1)\).

逐项求导,得\begin{equation*}
	F'(x) = m \left[
		1+\frac{m-1}{1}x+\dotsb+\frac{(m-1)\dotsm(m-n+1)}{(n-1)!}x^{n-1}+\dotsb
	\right],
\end{equation*}
两边各乘以\((1+x)\),并把含有\(x^n\ (n=1,2,\dotsc)\)的两项合并起来.
根据\cref{theorem:组合数性质2}
\begin{align*}
	&\frac{(m-1)\dotsm(m-n+1)}{(n-1)!}
		+ \frac{(m-1)\dotsm(m-n)}{n!} \\
	&= \frac{m(m-1)\dotsm(m-n+1)}{n!}
		\quad(n=1,2,\dotsc),
\end{align*}
可得\begin{align*}
	(1+x) F'(x)
	&= m \Biggl[
	1+mx+\frac{m(m-1)}{2!}x^2+\dotsb \\
	&\hspace{30pt} +\frac{m(m-1)\dotsm(m-n+1)}{n!}x^n+\dotsb
	\Biggr] \\
	&= m F(x)
	\quad(-1<x<1).
\end{align*}

现在令\(\phi(x) = \frac{F(x)}{(1+x)^m}\),
于是\(\phi(0) = F(0) = 1\),
且\begin{align*}
	\phi'(x)
	&= \frac{(1+x)^m F'(x) - m(1+x)^{m-1} F(x)}{(1+x)^{2m}} \\
	&= \frac{(1+x)^{m-1} [(1+x) F'(x) - m F(x)]}{(1+x)^{2m}}
	= 0,
\end{align*}
所以\(\phi(x)\)是常数函数.
又因为\(\phi(0) = 1\),
所以\(\phi(x) = 1\),
即\begin{equation*}
	F(x) = (1+x)^m.
\end{equation*}
因此原级数在区间\((-1,1)\)内有展开式
\begin{equation}\label{equation:无穷级数.二项展开式}
\begin{split}
	(1+x)^m
	&= 1+mx+\frac{m(m-1)}{2!}x^2+\dotsb \\
	&\hspace{20pt} +\frac{m(m-1)\dotsm(m-n+1)}{n!}x^n+\dotsb
	\quad(-1<x<1).
\end{split}
\end{equation}
在区间的端点,展开式是否成立要看\(m\)的数值而定.
\end{solution}
\end{example}

特殊地,当\(m\)为正整数时,级数为\(x\)的\(m\)次多项式,这就是代数学中的二项式定理.

对应于\(m=\pm1/2\)的二项展开式分别为\begin{gather*}
	\sqrt{1+x}
	= 1+\frac{1}{2}x-\frac{1}{2\cdot4}x^2+\frac{1\cdot3}{2\cdot4\cdot6}x^3-\frac{1\cdot3\cdot5}{2\cdot4\cdot6\cdot8}x^4+\dotsb
	\quad(-1 \leq x \leq 1), \\
	\frac{1}{\sqrt{1+x}}
	= 1-\frac{1}{2}x+\frac{1\cdot3}{2\cdot4}x^2-\frac{1\cdot3\cdot5}{2\cdot4\cdot6}x^3+\frac{1\cdot3\cdot5\cdot7}{2\cdot4\cdot6\cdot8}x^4-\dotsb
	\quad(-1 < x \leq 1).
\end{gather*}

现在我们可以利用幂级数展开式计算定积分.
\begin{example}
计算\begin{equation*}
	\int_0^1 \frac{\sin\ln x}{\ln x} \dd{x}.
\end{equation*}
\begin{solution}
由\cref{equation:无穷级数.幂级数展开式2} \begin{equation*}
	\sin x = \sum_{k=0}^\infty \frac{(-1)^k}{(2k+1)!} x^{2k+1}
	\quad(-\infty<x<+\infty),
\end{equation*}
有\begin{equation*}
	\frac{\sin\ln x}{\ln x} = \sum_{k=0}^\infty \frac{(-1)^k}{(2k+1)!} (\ln x)^{2k},
\end{equation*}
那么\begin{align*}
	\int_0^1 \frac{\sin\ln x}{\ln x} \dd{x}
	&= \int_0^1 \sum_{k=0}^\infty \frac{(-1)^k}{(2k+1)!} (\ln x)^{2k} \dd{x} \\
	&= \sum_{k=0}^\infty \frac{(-1)^k}{(2k+1)!} \int_0^1 (\ln x)^{2k} \dd{x} \\
	&\xlongequal{\ln x = -t}
		\sum_{k=0}^\infty \frac{(-1)^k}{(2k+1)!}
		\int_{+\infty}^0 (-t)^{2k} \cdot (-e^{-t}) \dd{t} \\
	&= \sum_{k=0}^\infty \frac{(-1)^k}{(2k+1)!}
		\int^{+\infty}_0 e^{-t} t^{2k} \dd{t} \\
	&= \sum_{k=0}^\infty \frac{(-1)^k}{(2k+1)!} \Gamma(2k+1) \\
	&= \sum_{k=0}^\infty \frac{(-1)^k}{(2k+1)!} (2k)! \\
	&= \sum_{k=0}^\infty (-1)^k \frac{1}{2k+1} \\
	&= \sum_{k=0}^\infty (-1)^k \int_0^1 u^{2k} \dd{u} \\
	&= \int_0^1 \sum_{k=0}^\infty (-u^2)^k \dd{u} \\
	&= \int_0^1 \frac{1}{1-(-u^2)} \dd{u} \\
	&= \left(\arctan u\right)_0^1
	= \frac{\pi}{4}.
\end{align*}
\end{solution}
\end{example}

\begin{example}
求幂级数\(\sum_{n=1}^\infty \frac{(-1)^{n-1}}{2n-1} x^{2n}\)的收敛域及和函数.
%TODO
\end{example}

\section{函数的幂级数展开式的应用}
\subsection{微分方程的幂级数解法}
这里,我们简单介绍一阶微分方程和二阶齐次线性微分方程的幂级数解法.

\subsubsection{一阶微分方程的幂级数解法}
为求一阶微分方程\begin{equation*}
\dv{y}{x} = f(x,y)
\end{equation*}满足初始条件\(\eval{y}_{x=x_0} = y_0\)的特解,如果其中函数\(f(x,y)\)是\((x-x_0),(y-y_0)\)的多项式\begin{equation*}
f(x,y) = a_{00} + a_{10} (x-x_0) + a_{01} (y-y_0) + \dotsb + a_{lm} (x-x_0)^l (y-y_0)^m.
\end{equation*}那么可以设所求特解可展开为\((x-x_0)\)的幂级数:\begin{equation*}
y = y_0 + a_1 (x-x_0) + a_2 (x-x_0)^2 + \dotsb + a_n (x-x_0)^n + \dotsb,
\end{equation*}其中\(a_1,a_2,\dotsc\)是待定系数.
把上式代入微分方程中,便得一恒等式,比较所得恒等式两端\((x-x_0)\)的同次幂的系数,就可定出常数\(a_1,a_2,\dotsc\),以这些常数为系数的级数在其收敛区间内就是所求一阶微分方程满足初始条件的通解.

\begin{example}
求方程\(\dv{y}{x} = x + y^2\)满足\(\eval{y}_{x=0}=0\)的特解.
\begin{solution}
这时\(x_0=y_0=0\),故设\begin{equation*}
y = a_1 x + a_2 x^2 + a_3 x^3 + a_4 x^4 + \dotsb,
\end{equation*}把\(y\)及\(y'\)的幂级数展开式代入原方程,得\begin{align*}
a_1 + 2a_2 x + 3a_3 x^2 + 4a_4 x^3 + \dotsb
&= x + (a_1 x + a_2 x^2 + a_3 x^3 + a_4 x^4 + \dotsb)^2 \\
&= x + a_1^2 x^2 + 2a_1a_2 x^3 + (a_2^2 + 2a_1a_3) x^4 + \dotsb,
\end{align*}
比较上式两边\(x\)的同次幂的系数,得\begin{equation*}
a_1 = 0, \qquad
a_2 = \frac{1}{2}, \qquad
a_3 = 0, \qquad
a_4 = 0, \qquad
a_5 = \frac{1}{20},
\end{equation*}于是\begin{equation*}
y = \frac{1}{2} x^2 + \frac{1}{20} x^5 + \dotsb.
\end{equation*}
\end{solution}
\end{example}

\subsubsection{二阶齐次线性微分方程的幂级数解法}
关于二阶齐次线性方程\begin{equation*}
	y'' + P(x) y' + Q(x) y = 0
\end{equation*}用幂级数求解的问题,我们先叙述一个定理:
\begin{theorem}
如果二阶齐次线性方程中的系数\(P(x)\)与\(Q(x)\)
可在\(-R<x<R\)内展开为\(x\)的幂级数,
那么在\(-R<x<R\)内方程必有形如\begin{equation*}
	y = \sum_{n=0}^\infty a_n x^n
\end{equation*}的解.
\end{theorem}

\begin{example}
求微分方程\(y''-xy=0\)满足初始条件\(\eval{y}_{x=0}=0\)和\(\eval{y'}_{x=0}=1\)的特解.
\begin{solution}
这里\(P(x)=0, Q(x)=-x\)在整个数轴上满足定理的条件.
因此不妨设所求的解\(y(x)\)展开成\(x\)的幂级数为\begin{equation*}
	y = \sum_{n=0}^\infty a_n x^n.
\end{equation*}
由条件\(\eval{y}_{x=0}=0\),
得\(a_0=0\).
对幂级数逐项求导,有\begin{equation*}
	y' = \sum_{n=1}^\infty n a_n x^{n-1},
\end{equation*}
由条件\(\eval{y'}_{x=0}=1\),
得\(a_1=1\).
于是所求特解\(y\)及\(y'\)的展开式成为\begin{equation*}
	y = x + \sum_{n=2}^\infty a_n x^n,
\end{equation*}\begin{equation*}
	y' = 1 + \sum_{n=2}^\infty n a_n x^{n-1}.
\end{equation*}
再次逐项求导,得\begin{equation*}
	y'' = \sum_{n=2}^\infty n(n-1) a_n x^{n-2}.
\end{equation*}
将\(y,y',y''\)代入原方程,
得\begin{equation*}
	\sum_{n=2}^\infty n(n-1) a_n x^{n-2}
	- x \left( x + \sum_{n=2}^\infty a_n x^n \right)
	= 0,
\end{equation*}
按\(x\)的升幂顺序,合并同类项,得\begin{equation*}
	2 a_2 + 3\cdot2 a_3 x + (4\cdot3 a_4 - 1) x^2
	+ (5\cdot4 a_5 - a_2) x^3 + (6\cdot5 a_6 - a_3) x^4
	+ \dotsb + [(n+2)(n+1) a_{n+2} - a_{n-1}] x^n + \dotsb
	= 0.
\end{equation*}
因为上式是恒等式,
所以上式左端各项的系数必全为零,
于是又\begin{equation*}
	a_2 = 0,
	a_3 = 0,
	a_4 = \frac{1}{4\cdot3},
	a_5 = 0,
	a_6 = 0,
	\dotsc,
\end{equation*}
一般地,\begin{equation*}
	a_{n+2} = \frac{a_{n-1}}{(n+2)(n+1)}
	\quad(n=3,4,\dotsc).
\end{equation*}
可以推得\begin{equation*}
	a_{3m+1} = \frac{1}{(3m+1)(3m) \dotsm 7\cdot6\cdot4\cdot3}
	\quad(m=1,2,\dotsc).
\end{equation*}
于是所求特解为\begin{equation*}
	y = x + \frac{x^4}{4\cdot3} + \frac{x^7}{7\cdot6\cdot4\cdot3}
	+ \frac{x^{10}}{10\cdot9\cdot7\cdot6\cdot4\cdot3}
	+ \dotsb
	+ \frac{x^{3m+1}}{(3m+1)(3m) \dotsm 7\cdot6\cdot4\cdot3}
	+ \dotsb.
\end{equation*}
\end{solution}
\end{example}

\subsection{重新定义三角函数}
尽管我们是从三角函数的几何定义出发,
计算得到三角函数的幂级数展开式,
但是我们也可以反其道行之,将三角函数(特别是正、余弦函数)定义为对应的幂级数,
重新建立三角学的基础.

我们将\cref{equation:无穷级数.幂级数展开式2,equation:无穷级数.幂级数展开式5}
分别作为正弦函数和余弦函数的定义式:
\begin{equation*}
	\sin x \defeq \sum_{k=0}^\infty \frac{(-1)^k}{(2k+1)!} x^{2k+1}
	\quad(-\infty<x<+\infty),
\end{equation*}\begin{equation*}
	\cos x \defeq \sum_{k=0}^\infty \frac{(-1)^k}{(2k)!} x^{2k}
	\quad(-\infty<x<+\infty).
\end{equation*}

容易验证:
\((\sin x)' = \cos x\),
\((\cos x)' = - \sin x\),
\(\sin 0 = 0\),
\(\cos 0 = 1\).

现在我们来证明恒等式\(\sin^2 x + \cos^2 x \equiv 1\).
由于\(\sin^2 0 + \cos^2 0 = 0^2 + 1^2 = 1\),
所以只需证函数\(f(x) = \sin^2 x + \cos^2 x\)在\((-\infty,+\infty)\)上是常数函数,
即证\(f'(x) = 0\ (-\infty<x<+\infty)\)恒成立;
由于\(f'(x) = 2 \sin x \cos x - 2 \cos x \sin x \equiv 0\),
自然就有恒等式\(\sin^2 x + \cos^2 x \equiv 1\)成立.

现在我们来证明和积互化公式.
这里我们只挑出两个最基本的和差化积公式作出证明.
对于其他和积互化公式,我们只需应用初等的代数方法就可以证得.
例如,要证\begin{equation*}
\sin(x+y) \equiv \sin x \cos y + \cos x \sin y,
\qquad
\cos(x+y) \equiv \cos x \cos y - \sin x \sin y
\end{equation*}成立,
只需构造辅助函数\begin{equation*}
\phi(x)
= \sin(x+y) - (\sin x \cos y + \cos x \sin y),
\end{equation*}\begin{equation*}
\psi(x)
= \cos(x+y) - (\cos x \cos y - \sin x \sin y),
\end{equation*}再证明\(\phi(x),\psi(x) \equiv 0\).
令\begin{equation*}
\rho(x)
= [\phi(x)]^2 + [\psi(x)]^2.
\end{equation*}
并注意到\(\phi'(x) = \psi(x)\),
\(\psi'(x) = -\phi'(x)\),
就有\(\rho'(x) \equiv 0\),
也就是说\(\rho(x)\)也是常值函数.
由于\begin{equation*}
\rho(0) = [\phi(0)]^2 + [\psi(0)]^2 = 0,
\end{equation*}
所以\(\rho(x) \equiv 0\);
又因为\(\phi(x),\psi(x) \in \mathbb{R}\),
所以\(\phi(x),\psi(x) \equiv 0\),
也就是说,和积互化公式成立.

\section{傅里叶级数}
从本节开始,我们讨论由三角函数组成的函数项级数,
即所谓\DefineConcept{三角级数}(trigonometric series)\[
	\sum_{n=0}^\infty (a_n \cos nx + b_n \sin nx),
	\quad a_n,b_n\in\mathbb{R}.
\]
考虑到\(\cos0x=1,\sin0x=0\),
于是我们总是把三角级数表示成\begin{equation}\label[trigonometric-series]{equation:三角级数.三角级数}
	a_0 + \sum_{n=1}^\infty (a_n \cos nx + b_n \sin nx).
\end{equation}
着重研究如何把函数展开成三角级数.

\subsection{三角级数,三角函数系的正交性}
我们希望将周期为\(T = \frac{2\pi}{\omega}\)的周期函数用一系列以\(T\)为周期的正弦函数\[
	t \mapsto A_n \sin(n \omega t + \phi_n)
\]组成的级数来表示,
记为\begin{equation}
	f(t) = A_0 + \sum_{n=1}^\infty A_n \sin(n \omega t + \phi_n),
\end{equation}
其中\(A_n\ (n=0,1,2,\dotsc)\)和\(\phi_n\ (n=1,2,\dotsc)\)都是常数.

将周期函数按上述方式展开,它的物理意义是很明确的.
这就是把一个比较复杂的周期运动看成是许多不同频率的简谐运动的叠加.
在电学上,这种展开被称为\emph{谐波分析}.
这里的常数项\(A_0\)称为“函数\(f\)的\emph{直流分量}”;
函数\(t \mapsto A_1 \sin(\omega t+\phi_1)\)
称为“函数\(f\)的\emph{一次谐波}”或“函数\(f\)的\emph{基波}”;
函数\(t \mapsto A_2 \sin(\omega t+\phi_2)\)
称为“函数\(f\)的\emph{二次谐波}”;
以此类推.

为了以后讨论方便起见,
我们将正弦函数\(A_n \sin(n \omega t + \phi_n)\)按三角公式变形,得\[
A_n \sin(n \omega t + \phi_n)
= A_n \sin\phi_n \cos(n \omega t) + A_n \cos\phi_n \sin(n \omega t),
\]并且令\(\frac{a_0}{2} = A_0\),
\(a_n = A_n \sin\phi_n\),
\(b_n = A_n \cos\phi_n\),
\(\omega = \frac{\pi}{l}\)(即\(T = 2l\)),
则(1)式右端的级数就可以改写为
\[
	\frac{a_0}{2}
	+ \sum_{n=1}^\infty
		\left( a_n \cos{\frac{n \pi t}{l}} + b_n \sin{\frac{n \pi t}{l}} \right)
	\eqno(2)
\]
形如(2)式的级数称为“以\(2l\)为周期的\DefineConcept{三角级数}”,
其中\(a_0,a_n,b_n\ (n=1,2,\dotsc)\)都是常数.

令\(x = \frac{\pi t}{l}\),则(2)式称为
\[
	\frac{a_0}{2} + \sum_{n=1}^\infty ( a_n \cos{n x} + b_n \sin{n x} ),
	\eqno(3)
\]
这就把“以\(2l\)为周期的三角级数”转换成“以\(2\pi\)为周期的三角级数”.

下面我们讨论以\(2\pi\)为周期的三角级数(3).

如同讨论幂级数时一样,
我们必须讨论三角级数(3)的收敛问题,
以及给定周期为\(2\pi\)的周期函数%
如何把它展开成三角级数(3).
为此,我们首先介绍三角函数系的正交性.

\begin{definition}
所谓\DefineConcept{三角函数系}\[
1, \cos x, \sin x, \cos 2x, \sin 2x, \dotsc, \cos nx, \sin nx, \dotsc
\]在区间\([-\pi,\pi]\)上\DefineConcept{正交},
是指三角函数系中任意两个不同的函数的乘积在区间\([-\pi,\pi]\)上的积分等于零,
即\begin{gather*}
	\int_{-\pi}^\pi \cos{nx} \dd{x} = 0 \quad(n=1,2,\dotsc), \\
	\int_{-\pi}^\pi \sin{nx} \dd{x} = 0 \quad(n=1,2,\dotsc), \\
	\int_{-\pi}^\pi \sin{mx}\cos{nx} \dd{x} = 0 \quad(m,n=1,2,\dotsc), \\
	\int_{-\pi}^\pi \cos{mx}\cos{nx} \dd{x} = 0 \quad(m,n=1,2,\dotsc; m \neq n), \\
	\int_{-\pi}^\pi \sin{mx}\sin{nx} \dd{x} = 0 \quad(m,n=1,2,\dotsc; m \neq n).
\end{gather*}

在三角函数系中,两个相同函数的乘积在区间\([-\pi,\pi]\)的积分不等于零,即
\begin{gather*}
\int_{-\pi}^\pi 1^2 \dd{x} = 2\pi, \\
\int_{-\pi}^\pi \sin^2 nx \dd{x} = \int_{-\pi}^\pi \cos^2 nx \dd{x} = \pi \quad(n=1,2,\dotsc).
\end{gather*}
\end{definition}

\subsection{三角级数的性质}
\begin{property}\label{theorem:三角级数.具有绝对收敛系数的三角级数一致收敛且绝对收敛}
如果常数项级数\(\sum_{n=1}^\infty a_n\)和\(\sum_{n=1}^\infty b_n\)绝对收敛,
那么\cref{equation:三角级数.三角级数} 在区间\((-\infty,+\infty)\)上一致收敛且绝对收敛.
\begin{proof}
由\hyperref[theorem:无穷级数.魏尔斯特拉斯判别法]{魏尔斯特拉斯判别法}立即可得.
%TODO proof 需要细说
\end{proof}
\end{property}

\begin{property}
如果常数项级数\(\sum_{n=1}^\infty a_n\)和\(\sum_{n=1}^\infty b_n\)绝对收敛,
那么\cref{equation:三角级数.三角级数} 的和函数在区间\((-\infty,+\infty)\)上连续.
\begin{proof}
三角级数的各项是正弦函数或余弦函数,
它们都在区间\((-\infty,+\infty)\)上连续,
于是三角级数的部分和函数也在区间\((-\infty,+\infty)\)上连续.
根据\cref{theorem:三角级数.具有绝对收敛系数的三角级数一致收敛且绝对收敛},
\cref{equation:三角级数.三角级数} 在区间\((-\infty,+\infty)\)上一致收敛且绝对收敛.
那么由\cref{theorem:函数项级数.连续函数项级数的内闭一致收敛性保证和函数的连续性} 可知,
\cref{equation:三角级数.三角级数} 的和函数在区间\((-\infty,+\infty)\)上连续.
\end{proof}
\end{property}

\subsection{函数展开成傅里叶级数}
\begin{definition}\label{definition:无穷级数.傅里叶级数}
%@see: https://mathworld.wolfram.com/FourierSeries.html
设\(f\)是周期为\(2 \pi\)的周期函数.
如果积分\[
	a_n = \frac{1}{\pi} \int_{-\pi}^\pi f(x) \cos nx \dd{x} \quad(n=0,1,2,\dotsc)
\]和\[
	b_n = \frac{1}{\pi} \int_{-\pi}^\pi f(x) \sin nx \dd{x} \quad(n=1,2,3,\dotsc)
\]都存在,
那么它们定出的系数\(a_0,a_1,\dotsc,b_1,b_2,\dotsc\)叫做%
“函数\(f\)的\DefineConcept{傅里叶系数}(Fourier coefficient)”,
根据这些系数确定的三角级数\[
	\frac{a_0}{2} + \sum_{k=1}^\infty (a_k \cos{kx} + b_k \sin kx)
\]称为“函数\(f\)的\DefineConcept{傅里叶级数}(Fourier series)”
或“函数\(f\)的\DefineConcept{傅里叶展开}(Fourier expansion)”.
\end{definition}

一个定义在\((-\infty,+\infty)\)上周期为\(2\pi\)的函数\(f\),
如果它在一个周期上可积,
则一定可以作出\(f\)的傅里叶级数.
然而函数\(f\)的傅里叶级数是否一定收敛?
如果它收敛,它是否一定收敛于函数\(f\)?
一般说来,这两个问题的答案都不是肯定的.
那么,\(f\)在怎样的条件下,它的傅里叶级数不仅收敛,而且收敛于\(f\)?
也就是说,\(f\)满足什么条件可以展开成傅里叶级数?
这是我们面临的一个基本问题.

下面我们叙述一个收敛定理,它给出了上述问题的一个重要结论.
\begin{theorem}[收敛定理,狄利克雷充分条件]\label{theorem:无穷级数.傅里叶级数收敛的狄利克雷充分条件}
设\(f\)是周期为\(2 \pi\)的周期函数,如果它满足:\begin{itemize}
	\item 在一个周期内连续或只有有限个第一类间断点,
	\item 在一个周期内至多只有有限个极值点,
\end{itemize}
则\(f\)的傅里叶级数收敛,并且\begin{itemize}
	\item 当\(x\)是\(f\)的连续点时,
	级数收敛于\(f\).

	\item 当\(x\)是\(f\)的间断点时,
	级数收敛于\(\frac{1}{2} [ f(x^-) + f(x^+) ]\)\footnote{%
	函数\(f\)的傅里叶级数在点\(x=\pm\pi\)处收敛于%
	区间端点\(-\pi\)的右极限和\(\pi\)的左极限的算术平均值%
	\(\frac{1}{2} [ f(\pi^-) + f(-\pi^+) ]\).}.
\end{itemize}
\end{theorem}

收敛定理告诉我们:
只要函数在\([-\pi,\pi]\)上至多有有限个第一类间断点,
并且不作无限次振动,
函数的傅里叶级数在连续点处就收敛于该点的函数值,
在间断点处收敛于该点左、右极限的算术平均值.
可见,函数展开成傅里叶级数的条件比展开成幂级数的条件低得多.
记
\begin{equation}
	C = \Set*{
		x \in \mathbb{R}
		\given
		f(x) = \frac{1}{2} [ f(x^-) + f(x^+) ]
	},
\end{equation}
那么在\(C\)上就成立\(f\)的傅里叶级数展开式\[
	f(x) = \frac{a_0}{2} + \sum_{k=1}^\infty (a_k \cos{kx} + b_k \sin kx),
	\quad x \in C.
\]

\begin{theorem}
设以\(2\pi\)为周期的函数\(f\)的傅里叶系数为\(a_n,b_n\),
那么函数\(f(x+h)\ (h\text{是常数})\)的傅里叶系数为\[
	\alpha_n
	= a_n \cos nh + b_n \sin nh,
	\qquad
	\beta_n
	= b_n \cos nh - a_n \sin nh.
\]
\end{theorem}

\begin{example}[矩形波的谐波分析]
设\(f\)是周期为\(2\pi\)的周期函数,它在\([-\pi,\pi)\)上的表达式为\[
	f(x) = \left\{ \begin{array}{cc}
		-1, & -\pi \leq x < 0, \\
		1, & 0 \leq x < \pi.
	\end{array} \right.
\]将\(f\)展开成傅里叶级数.
\begin{solution}
所给函数满足\hyperref[theorem:无穷级数.傅里叶级数收敛的狄利克雷充分条件]{收敛定理}的条件,
它在点\(x = k\pi\ (k=0,\pm1,\pm2,\dotsc)\)处不连续,在其他点处连续;
从而由收敛定理知道\(f\)的傅里叶级数收敛,
并且当\(x = k\pi\)时,级数收敛于\(0\);
当\(x \neq k\pi\)时,级数收敛于\(f\).

计算傅里叶系数如下:
\begin{align*}
	a_n &= \frac{1}{\pi} \int_{-\pi}^\pi f(x) \cos nx \dd{x} \\
		&= \frac{1}{\pi} \int_{-\pi}^0 (-1) \cos nx \dd{x}
			+ \frac{1}{\pi} \int_0^\pi 1 \cdot \cos nx \dd{x} \\
		&= 0 \quad(n=0,1,2,\dotsc); \\
	b_n &= \frac{1}{\pi} \int_{-\pi}^\pi f(x) \sin nx \dd{x} \\
		&= \frac{1}{\pi} \int_{-\pi}^0 (-1) \sin nx \dd{x}
			+ \frac{1}{\pi} \int_0^\pi 1 \cdot \sin nx \dd{x} \\
		&= \frac{2}{n\pi} [1-(-1)^n]
		= \left\{ \begin{array}{cl}
			\frac{4}{n\pi}, & n=1,3,5,\dotsc, \\
			0, & n=2,4,6,\dotsc.
		\end{array} \right.
\end{align*}
所以\(f\)的傅里叶展开式为\[
	f(x) = \frac{4}{\pi} \sum_{k=1}^\infty \frac{1}{2k-1} \sin(2k-1) x
	\quad(-\infty<x<+\infty;x\neq0,\pm\pi,\pm2\pi,\dotsc).
\]
\end{solution}
\end{example}
如果把上例中的函数理解为矩形波的波形函数%
(周期\(T=2\pi\),振幅\(E=1\),自变量\(x\)表示时间),
那么上面所得到的展开式表明:
矩形波是由一系列不同频率的正弦波叠加而成的,
这些正弦波的频率依次为基波频率的奇数倍.

\begin{example}
设\(f\)是周期为\(2\pi\)的周期函数,它在\([-\pi,\pi)\)上的表达式为\[
f(x) = \left\{ \begin{array}{cc}
x, & -\pi \leq x < 0, \\
0, & 0 \leq x < \pi.
\end{array} \right.
\]
将\(f\)展成傅里叶级数.
\begin{solution}
所给函数满足\hyperref[theorem:无穷级数.傅里叶级数收敛的狄利克雷充分条件]{收敛定理}的条件.
它在点\(x=(2k+1)\pi\ (k=0,\pm1,\pm2,\dotsc)\)处不连续.
因此,\(f\)的傅里叶级数在\(x=(2k+1)\pi\)处收敛于\[
\frac{f(\pi^-)+f(-\pi^+)}{2} = \frac{0-\pi}{2} = -\frac{\pi}{2},
\]在连续点\(x\ (x\neq(2k+1)\pi)\)处收敛于\(f\).

计算傅里叶系数如下:\begin{align*}
	a_n &= \frac{1}{\pi} \int_{-\pi}^\pi f(x) \cos nx \dd{x}
	= \frac{1}{\pi} \int_{-\pi}^0 x \cos nx \dd{x} \\
	&= \frac{1}{\pi} \left[ \frac{x \sin nx}{n} + \frac{\cos nx}{n^2} \right]_{-\pi}^0 \\
	&= \frac{1}{n^2 \pi} (1-\cos nx) \\
	&= \left\{ \begin{array}{cc}
	\frac{2}{n^2 \pi}, & n=1,3,5,\dotsc, \\
	0, & n=2,4,6,\dotsc;
	\end{array} \right. \\
	a_0 &= \frac{1}{\pi} \int_{-\pi}^\pi f(x) \dd{x}
	= \frac{1}{\pi} \int_{-\pi}^0 x \dd{x}
	= \frac{1}{\pi} \left[ \frac{x^2}{2} \right]_{-\pi}^0 = -\frac{\pi}{2}; \\
	b_n &= \frac{1}{\pi} \int_{-\pi}^\pi f(x) \sin nx \dd{x}
	= \frac{1}{\pi} \int_{-\pi}^0 x \sin nx \dd{x} \\
	&= \frac{1}{\pi} \left[ -\frac{x \cos nx}{n} + \frac{\sin nx}{n^2} \right]_{-\pi}^0 \\
	&= -\frac{\cos n\pi}{n} = \frac{(-1)^{n+1}}{n}.
\end{align*}
得到\(f\)的傅里叶级数展开式为\begin{align*}
	f(x) &= -\frac{\pi}{4} + \left(\frac{2}{\pi} \cos x + \sin x\right) \\
	&\hspace{20pt}-\frac{1}{2}\sin 2x + \left(\frac{2}{3^2\pi}\cos 3x + \frac{1}{3}\sin 3x\right) \\
	&\hspace{20pt}-\frac{1}{4}\sin 4x + \left(\frac{2}{5^2\pi}\cos 5x + \frac{1}{5}\sin 5x\right)
	-\dotsb \\
	&= -\frac{\pi}{4} + \frac{2}{\pi} \sum_{k=1}^\infty \frac{1}{(2k-1)^2} \cos(2k-1)x \\
	&\hspace{20pt}+\sum_{n=1}^\infty \frac{(-1)^{n-1}}{n} \sin nx
		\qquad(-\infty<x<+\infty; x\neq\pm\pi,\pm3\pi,\dotsc).
\end{align*}
\end{solution}
\end{example}

\subsection{周期延拓}
应注意,如果函数\(f\)只在\([-\pi,\pi]\)上有定义,
并且满足收敛定理的条件,那么\(f\)也可以展开成傅里叶级数.
事实上,我们可在\([-\pi,\pi)\)或\((-\pi,\pi]\)外补充函数\(f\)的定义,
使它拓广成周期为\(2\pi\)的周期函数\(F\).
按这种方式拓广函数的定义域的过程称为\DefineConcept{周期延拓}(periodic extension).
再将\(F\)展开成傅里叶级数.
最后限制\(x\)在\((-\pi,\pi)\)内,此时\(F(x) \equiv f(x)\),
这样便得到\(f\)的傅里叶级数展开式.
根据\hyperref[theorem:无穷级数.傅里叶级数收敛的狄利克雷充分条件]{收敛定理},
这级数在区间端点\(x=\pm\pi\)处收敛于\(\frac{1}{2} [f(\pi^-) + f(-\pi^+)]\).

\subsection{正弦级数和余弦级数}
一般说来,一个函数的傅里叶级数既含有正弦项,又含有余弦项.
但是,也有一些函数的傅里叶级数只含有正弦项或或者只含有常数项和余弦项.
这是什么原因呢?实际上这些情况是与所给函数\(f\)的奇偶性有密切关系的.
对于周期为\(2\pi\)的函数\(f\),
它的傅里叶系数计算公式为\[
	a_n = \frac{1}{\pi} \int_{-\pi}^\pi f(x) \cos nx \dd{x} \quad(n=0,1,2,\dotsc),
\]\[
	b_n = \frac{1}{\pi} \int_{-\pi}^\pi f(x) \sin nx \dd{x} \quad(n=1,2,3,\dotsc).
\]
由于奇函数在对称区间上的积分为零,偶函数在对称区间上的积分等于半区间上积分的两倍,因此:

当\(f\)为奇函数时,
\(x \mapsto f(x) \cos nx\)是奇函数,
\(x \mapsto f(x) \sin nx\)是偶函数,
故\[
	\begin{array}{ll}
		a_n = 0 & (n=0,1,2,\dotsc), \\
		b_n = \frac{2}{\pi} \int_0^\pi f(x) \sin nx \dd{x} & (n=1,2,3,\dotsc).
	\end{array}
\]
即知奇函数的傅里叶级数是
只含有正弦项的\DefineConcept{正弦级数}(sine series)\[
	\sum_{n=1}^\infty b_n \sin nx.
\]

当\(f\)为偶函数时,
\(x \mapsto f(x) \cos nx\)是偶函数,
\(x \mapsto f(x) \sin nx\)是奇函数,
故\[
	\begin{array}{ll}
		a_n = \frac{2}{\pi} \int_0^\pi f(x) \cos nx \dd{x} & (n=0,1,2,\dotsc), \\
		b_n = 0 & (n=1,2,3,\dotsc).
	\end{array}
\]
即知偶函数的傅里叶级数是
只含有常数项和余弦项的\DefineConcept{余弦级数}(cosine series)\[
	\frac{a_0}{2} + \sum_{n=1}^\infty a_n \cos{nx}.
\]

\begin{example}
%@see: 《2023年全国硕士研究生入学统一考试(数学一)》二填空题/第13题
设\(f\)是周期为2的周期函数,且\begin{equation*}
	f(x) = 1-x
	\quad(0 \leq x \leq 1).
\end{equation*}
若\(f(x) = \frac{a_0}2 + \sum_{n=1}^\infty a_n \cos n\pi x\),
计算\(\sum_{n=1}^\infty a_{2n}\).
\begin{solution}
由于函数\(f\)的傅里叶级数只含常数项和余弦项,
所以\(f\)一定是偶函数,
那么当\(n\geq1\)时,有\begin{equation*}
	a_n = 2 \int_0^1 f(x) \cos n\pi x \dd{x}
	= 2 \int_0^1 \cos n\pi x \dd{x}
	- 2 \int_0^1 x \cos n\pi x \dd{x}
	= \frac2{n^2\pi^2} (1 - \cos n\pi).
\end{equation*}
% 对于任意整数\(n\)总成立\(\cos 2n\pi = 1\).
于是\(a_{2n} = \frac2{n^2\pi^2} (1 - \cos 2n\pi) = 0\),
从而有\(\sum_{n=1}^\infty a_{2n} = 0\).
\end{solution}
%@Mathematica: a[n_] := 2 Integrate[(1 - x) Cos[n Pi x], {x, 0, 1}]
\end{example}

\subsection{奇延拓与偶延拓}
在实际应用(如研究某种波动问题,热的传导、扩散问题)中,
有时还需要把定义在区间\([0,\pi]\)上的函数\(f\)展开成正弦级数或余弦级数.
根据前面讨论的结果,这类展开问题可以按如下的方法解决:
\begin{enumerate}
	\item 设函数\(f\)定义在区间\([0,\pi]\)上
	并且满足\hyperref[theorem:无穷级数.傅里叶级数收敛的狄利克雷充分条件]{收敛定理}的条件,
	我们在区间\([-\pi,0)\)内补充函数\(f\)的定义,
	得到定义在\([-\pi,\pi]\)上的函数\(F\).
	我们可以采用以下两种延拓方式:
	\begin{itemize}
		\item 令\[
			F_1(x)
			\defeq
			\left\{ \begin{array}{cl}
				f(-x), & -\pi \leq x < 0, \\
				f(x), & 0 \leq x \leq \pi.
			\end{array} \right.
		\]
		我们把这种延拓函数定义域的方式称为\DefineConcept{偶延拓}(even extension).

		\item 令\[
			F_2(x)
			\defeq
			\left\{ \begin{array}{cl}
				-f(-x), & -\pi \leq x < 0, \\
				0, & x = 0, \\
				f(x), & 0 < x \leq \pi.
			\end{array} \right.
		\]
		我们把这种延拓函数定义域的方式称为\DefineConcept{奇延拓}(odd extension).
	\end{itemize}

	\item 然后将奇延拓(或偶延拓)后的函数展开成傅里叶级数,这个级数必定是正弦级数(或余弦级数).

	\item 再限制\(x\)在\((0,\pi]\)上,
	此时\(F(x)\equiv f(x)\),这样便得到\(f\)的正弦级数(或余弦级数)展开式.
\end{enumerate}
%@see: https://math24.net/even-odd-extensions.html

\begin{figure}[htb]
	\centering
	\begin{subfigure}[b]{.4\textwidth}
		\centering
		\begin{tikzpicture}[scale=.9]
			\begin{axis}[
				enlargelimits=.2,
				xmin=-pi,xmax=pi,
				ymin=-1,ymax=1,
				axis lines=middle,
				xlabel=$x$,
				ylabel=$y$,
				xtick={-pi,-.5*pi,.5*pi,pi},
				xticklabels={$-\pi\vphantom{\frac12}$,$-\frac12\pi$,$\frac12\pi$,$\pi\vphantom{\frac12}$},
			]
				\begin{scope}[color=blue,samples=50,smooth]
					\addplot[domain=0:{.5*pi}]{cos(deg(x))};
					\addplot[domain={.5*pi}:{pi}]{0};

					\addplot[domain=0:{-.5*pi}]{cos(deg(x))};
					\addplot[domain={-.5*pi}:{-pi}]{0};
				\end{scope}
			\end{axis}
		\end{tikzpicture}
		\caption{}
		\label{figure:傅里叶级数.偶延拓1}
	\end{subfigure}~\begin{subfigure}[b]{.4\textwidth}
		\centering
		\begin{tikzpicture}[scale=.9]
			\begin{axis}[
				enlargelimits=.2,
				xmin=-pi,xmax=pi,
				ymin=-1,ymax=1,
				axis lines=middle,
				xlabel=$x$,
				ylabel=$y$,
				xtick={-pi,-.5*pi,.5*pi,pi},
				xticklabels={$-\pi\vphantom{\frac12}$,$-\frac12\pi$,$\frac12\pi$,$\pi\vphantom{\frac12}$},
			]
				\begin{scope}[color=blue,samples=50,smooth]
					\addplot[domain=0:{.5*pi}]{cos(deg(x))};
					\addplot[domain={.5*pi}:{pi}]{0};

					\addplot[domain=0:{-.5*pi}]{-cos(deg(x))};
					\addplot[domain={-.5*pi}:{-pi}]{0};
				\end{scope}
			\end{axis}
		\end{tikzpicture}
		\caption{}
		\label{figure:傅里叶级数.奇延拓1}
	\end{subfigure}
	\caption{}
\end{figure}

\begin{example}
将函数\[
	f(x) = \left\{ \begin{array}{cc}
		\cos x, & 0 \leq x < \frac{\pi}{2}, \\
		0, & \frac{\pi}{2} \leq x \leq \pi
	\end{array} \right.
\]分别展开成正弦级数和余弦级数.
\begin{solution}
先展开成正弦级数.
为此对函数\(f\)作奇延拓,
如\cref{figure:傅里叶级数.奇延拓1},
那么傅里叶系数为\begin{align*}
	b_n &= \frac{2}{\pi} \int_0^\pi f(x) \sin nx \dd{x}
	= \frac{2}{\pi}
		\int_0^{\frac{\pi}{2}} \cos x \sin nx \dd{x} \\
	&= \frac{1}{\pi}
		\int_0^{\frac{\pi}{2}} [\sin(n-1)x + \sin(n+1)x] \dd{x} \\
	&= \frac{1}{\pi}
		\left[
			-\frac{1}{n-1} \cos(n-1)x - \frac{1}{n+1} \cos(n+1)x
		\right]_0^{\frac{\pi}{2}} \\
	&= \frac{1}{\pi}
		\left(
			\frac{1}{n-1}
			+ \frac{1}{n+1}
			- \frac{1}{n-1} \cos\frac{n-1}{2}\pi
			- \frac{1}{n+1} \cos\frac{n+1}{2}\pi
		\right) \\
	&= \frac{1}{\pi}
		\left(
			\frac{2n}{n^2-1}
			- \frac{1}{n-1}
			\sin\frac{n\pi}{2}
			+ \frac{1}{n+1}\sin\frac{n\pi}{2}
		\right) \\
	&= \frac{2}{\pi(n^2-1)} \left(n-\sin\frac{n\pi}{2}\right)
		\quad(n=2,3,\dotsc); \\
	b_1 &= \frac{2}{\pi} \int_0^\pi f(x) \sin x \dd{x}
	= \frac{2}{\pi} \int_0^\pi \cos x \sin x \dd{x} = \frac{1}{\pi}.
\end{align*}
从而\(f\)的正弦级数展开式为\[
f(x) = \frac{1}{\pi} \left[
		\sin x
		+ 2\sum_{n=2}^\infty
			\frac{1}{n^2-1} \left(n-\sin\frac{n\pi}{2}\right) \sin nx
	\right]
	\quad(0 < x \leq \pi).
\]
在端点\(x=0\)处级数收敛到令,它不等于\(f(0)\).

再展开成余弦级数.
为此对函数\(f\)作偶延拓,
如\cref{figure:傅里叶级数.偶延拓1},
那么傅里叶系数为\begin{align*}
	a_n &= \frac{2}{\pi} \int_0^\pi f(x) \cos nx \dd{x}
	= \frac{2}{\pi} \int_0^{\frac{\pi}{2}} \cos x \cos nx \dd{x} \\
	&= \frac{1}{\pi} \int_0^{\frac{\pi}{2}} [\cos(n-1)x + \cos(n+1)x] \dd{x} \\
	&= \frac{1}{\pi} \left[
			\frac{1}{n-1} \sin\frac{n-1}{2}\pi
			+ \frac{1}{n+1} \sin\frac{n+1}{2}\pi
		\right] \\
	&= \frac{2}{\pi(n^2-1)} \sin\frac{n-1}{2}\pi \\
	&= \left\{ \begin{array}{cl}
		0, & n=2k-1 \land n\neq1, \\
		\frac{2(-1)^{k-1}}{\pi(4k^2-1)}, & n=2k;
	\end{array} \right. \\
	a_1 &= \frac{2}{\pi} \int_0^{\frac{\pi}{2}} \cos^2 x \dd{x}
	= \frac{1}{\pi} \int_0^{\frac{\pi}{2}} (1+\cos 2x) \dd{x}
	= \frac{1}{2}.
\end{align*}
从而\(f\)的余弦级数展开式为\[
	f(x) = \frac{1}{\pi} + \frac{1}{2} \cos x + \frac{2}{\pi}
		\sum_{k=1}^\infty \frac{(-1)^k-1}{4k^2-1} \cos 2kx
	\quad(0 \leq x \leq \pi).
\]
\end{solution}
\end{example}

\begin{example}
设周期函数\(f\)的周期为\(2\pi\),
证明:\begin{enumerate}
\item 如果\(f(x-\pi) = -f(x)\),
则\(f\)的傅里叶系数满足\[
	a_0 = 0,
	a_{2k} = 0,
	b_{2k} = 0
	\quad(k=1,2,\dotsc);
\]
\item 如果\(f(x-\pi) = f(x)\),
则\(f\)的傅里叶系数满足\[
	a_{2k+1} = 0,
	b_{2k+1} = 0
	\quad(k=0,1,2,\dotsc).
\]
\end{enumerate}
\begin{proof}
写出函数\(f\)的傅里叶级数如下:\[
	f(x) = \frac{a_0}{2}
	+ \sum_{k=1}^\infty \left(
		a_k \cos kx
		+ b_k \sin kx
	\right).
\]

还要注意到,当\(k\in\mathbb{Z}\)时,有以下恒等式成立:\[
	\sin(x+2k\pi) \equiv \sin x, \qquad
	\cos(x+2k\pi) \equiv \cos x,
\]\[
	\sin(x+\overline{2k+1}\pi) \equiv -\sin x, \qquad
	\cos(x+\overline{2k+1}\pi) \equiv -\cos x.
\]

如果\(f(x-\pi) = -f(x)\),
那么\[
	\frac{a_0}{2} + \sum_{k=1}^\infty \left[
		a_k \cos k(x-\pi)
		+ b_k \sin k(x-\pi)
	\right]
	=
	-\frac{a_0}{2} - \sum_{k=1}^\infty \left(
		a_k \cos kx
		+ b_k \sin kx
	\right),
\]
即\[
	\left\{ \begin{array}{ll}
		\frac{a_0}{2} = -\frac{a_0}{2}, \\
		a_k \cos k(x-\pi) = - a_k \cos kx, &(k=1,2,\dotsc) \\
		b_k \sin k(x-\pi) = - b_k \sin kx, &(k=1,2,\dotsc)
	\end{array} \right.
\]
解得\[
	a_0 = 0, \qquad
	a_{2k} = 0, \qquad
	b_{2k} = 0
	\quad(k=1,2,\dotsc).
\]

如果\(f(x-\pi) = f(x)\),那么\[
	\frac{a_0}{2} + \sum_{k=1}^\infty \left[
		a_k \cos k(x-\pi)
		+ b_k \sin k(x-\pi)
	\right]
	=
	\frac{a_0}{2} + \sum_{k=1}^\infty \left(
		a_k \cos kx
		+ b_k \sin kx
	\right),
\]
即\[
	\left\{ \begin{array}{ll}
		\frac{a_0}{2} = \frac{a_0}{2}, \\
		a_k \cos k(x-\pi) = a_k \cos kx, &(k=1,2,\dotsc) \\
		b_k \sin k(x-\pi) = b_k \sin kx, &(k=1,2,\dotsc)
	\end{array} \right.
\]
解得\[
	a_{2k+1} = 0, \qquad
	b_{2k+1} = 0
	\quad(k=0,1,2,\dotsc).
	\qedhere
\]
\end{proof}
\end{example}

\section{一般周期函数的傅里叶级数}
\subsection{周期为2\texorpdfstring{\(l\)}{l}的周期函数的傅里叶级数}
\begin{theorem}
\def\f{\frac{n \pi x}{l}}
设周期为\(2l\)的周期函数\(f(x)\)满足\hyperref[theorem:无穷级数.傅里叶级数收敛的狄利克雷充分条件]{收敛定理}的条件,
则它的傅里叶级数展开式为\begin{equation*}
f(x) = \frac{a_0}{2} + \sum_{n=1}^\infty \left( a_n \cos\f + b_n \sin\f \right)
\quad(x \in C),
\end{equation*}其中\begin{gather*}
a_n = \frac{1}{l} \int_{-l}^{l} f(x) \cos\f \dd{x}
\quad(n=0,1,2,\dotsc), \\
b_n = \frac{1}{l} \int_{-1}^{l} f(x) \sin\f \dd{x}
\quad(n=1,2,3,\dotsc), \\
C = \Set*{ x \in \mathbb{R} \given f(x) = \frac{1}{2} [ f(x^-) + f(x^+) ] }.
\end{gather*}

特别地,当\(f(x)\)是奇函数时,\begin{equation*}
f(x) = \sum_{n=1}^\infty b_n \sin\frac{n \pi x}{l}
\quad(x \in C),
\end{equation*}其中\begin{equation*}
b_n = \frac{2}{l} \int_0^l f(x) \sin\f \dd{x}
\quad(n=1,2,3,\dotsc).
\end{equation*}

当\(f(x)\)是偶函数时,\begin{equation*}
f(x) = \frac{a_0}{2} + \sum_{n=1}^\infty a_n \cos\f
\quad(x \in C),
\end{equation*}其中\begin{equation*}
a_n = \frac{2}{l} \int_0^l f(x) \cos\f \dd{x}
\quad(n=0,1,2,\dotsc).
\end{equation*}
\begin{proof}
作变量代换\(z = \frac{\pi x}{l}\),于是区间\(-l \leq x \leq l\)就变换成\(-\pi \leq z \leq \pi\).
设函数\(f(x) = f\left(\frac{lz}{\pi}\right) = F(z)\),从而\(F(z)\)是周期为\(2\pi\)的周期函数,并且它满足\hyperref[theorem:无穷级数.傅里叶级数收敛的狄利克雷充分条件]{收敛定理}的条件,将\(F(z)\)展开成傅里叶级数\begin{equation*}
F(z) = \frac{a_0}{2} + \sum_{n=1}^\infty (a_n \cos nz + b_n \sin nz),
\end{equation*}其中\(a_n = \frac{1}{\pi} \int_{-\pi}^\pi F(z) \cos nz \dd{z},\ b_n = \frac{1}{\pi} \int_{-\pi}^\pi F(z) \sin nz \dd{z}\).

在以上式子中令\(z=\frac{\pi x}{l}\),并注意到\(F(z) = f(x)\),于是有\begin{equation*}
f(x) = \frac{a_0}{2} + \sum_{n=1}^\infty \left( a_n \cos\frac{n\pi x}{l} + b_n \sin\frac{n\pi x}{l} \right),
\end{equation*}而且\begin{equation*}
a_n = \frac{1}{l} \int_{-l}^l f(x) \cos\frac{n\pi x}{l} \dd{x},
\qquad
b_n = \frac{1}{l} \int_{-l}^l f(x) \sin\frac{n\pi x}{l} \dd{x}.
\end{equation*}

类似地,可以证明定理的其余部分.
\end{proof}
\end{theorem}

要将定义在非对称区间\([a,b]\)上的函数\(f(x)\)展开成傅里叶级数,
既可以先作变换\(x = z + \frac{b+a}{2}\),
使得\(z \in \left[-\frac{b-a}{2},\frac{b-a}{2}\right]\),再进行周期延拓;
也可以先做变换\(x = z + a\),使得\(z \in [0,b-a]\),再进行奇延拓(或偶延拓).

\begin{example}
设周期函数\(f(x)\)的周期为\(2\),且\(f(x) = 1-x\ (0 \leq x \leq 1)\).
若\(f(x)\)可展为傅里叶级数\(\frac{a_0}{2} + \sum_{n=1}^\infty a_n \cos n\pi x\),求\(\sum_{n=1}^\infty a_{2n}\).
\begin{solution}
观察\(f(x)\)的傅里叶级数的形式可知,它是余弦级数,函数\(f(x)\)是偶函数,那么\begin{align*}
\frac{a_n}{2} &= \int_0^1 (1-x) \cos n\pi x \dd{x} \\
&= \int_0^1 \cos n\pi x \dd{x}
- \frac{1}{n\pi} \int_0^1 x \dd(\sin n\pi x) \\
&= \frac{1}{\pi} \int_0^\pi \cos nt \dd{t}
- \frac{1}{n\pi} \left[
(x \sin n\pi x)_0^1
- \frac{1}{\pi} \int_0^\pi \sin nt \dd{t}
\right] \\
&= \frac{1}{n^2\pi^2} \int_0^\pi \sin nt \dd(nt) \\
&= \frac{1}{n^2\pi^2} \left( 1 - \cos n\pi \right),
\end{align*}
于是\(a_{2n} = \frac{2}{n^2\pi^2} \left( 1 - \cos 2n\pi \right) = 0\),从而\(\sum_{n=1}^\infty a_{2n} = 0\).
\end{solution}
\end{example}

\subsection{傅里叶级数的复数形式}
\begingroup
\def\f{\frac{n \pi x}{l}}
\def\eif#1{e^{#1\iu\f}}

傅里叶级数还可以用复数形式表示.
在电子技术中,经常应用这种形式.

设周期为\(2l\)的周期函数\(f(x)\)的傅里叶级数为\begin{equation*}
	\frac{a_0}{2} + \sum_{n=1}^\infty \left( a_n \cos\f + b_n \sin\f \right),
\end{equation*}
其中系数\(a_n,b_n\)为\begin{gather*}
	a_n = \frac{1}{l} \int_{-l}^{l} f(x) \cos\f \dd{x}
	\quad(n=0,1,2,\dotsc), \\
	b_n = \frac{1}{l} \int_{-1}^{l} f(x) \sin\f \dd{x}
	\quad(n=1,2,3,\dotsc).
\end{gather*}
利用欧拉公式\begin{equation*}
	\cos t = \frac{e^{\iu t}+e^{-\iu t}}{2},
	\qquad
	\sin t = \frac{e^{\iu t}-e^{-\iu t}}{2\iu},
\end{equation*}
可将上述傅里叶级数化为\begin{equation*}
	\begin{split}
		&\frac{a_0}{2} + \sum_{n=1}^\infty \left[
			\frac{a_n}{2} \left( \eif{} + \eif- \right)
			-\frac{\iu b_n}{2} \left( \eif{} - \eif- \right)
		\right] \\
		&= \frac{a_0}{2} + \sum_{n=1}^\infty \left[
			\frac{a_n - \iu b_n}{2} \eif{}
			+ \frac{a_n + \iu b_n}{2} \eif-
		\right].
	\end{split}
\end{equation*}
记\begin{equation*}
	\frac{a_0}{2} = c_0,
	\qquad
	\frac{a_n-\iu b_n}{2} = c_n,
	\qquad
	\frac{a_n+\iu b_n}{2} = c_{-n},
\end{equation*}
其中\(n=1,2,\dotsc\),
则可进一步将傅里叶级数化为\begin{equation*}
	\begin{split}
		&c_0 + \sum_{n=1}^\infty \left(
			c_n \eif{} + c_{-n} \eif-
		\right) \\
		&= (c_n \eif{})_{n=0}
		+ \sum_{n=1}^\infty \left(
			c_n \eif{} + c_{-n} \eif-
	\right).
\end{split}
\end{equation*}
即得\DefineConcept{傅里叶级数的复数形式}\footnote{这是一个罗朗级数.}为
\begin{equation}\label{equation:无穷级数.傅里叶级数的复数形式}
	\sum_{n=-\infty}^{+\infty} c_n \eif{}.
\end{equation}
可以求出系数满足
\begin{equation}\label{equation:无穷级数.傅里叶系数的复数形式}
	c_n = \frac{1}{2l} \int_{-l}^l f(x) \eif- \dd{x}
	\quad(n=0,\pm1,\pm2,\dotsc).
\end{equation}
这就是\DefineConcept{傅里叶系数的复数形式}.

傅里叶级数的两种形式,本质上是一样的,但复数形式比较简洁,且只用一个算式计算系数.
\endgroup

\section{本章总结}

我们在本章学习了无穷级数的基本概念.

依据级数的一般项是不是常数,我们将无穷级数分类为%
\hyperref[definition:无穷级数.常数项级数的定义]{常数项级数}%
和\hyperref[definition:无穷级数.实函数项级数的概念]{函数项级数}.

常数项级数具有许多重要的历史意义与应用价值.
一方面,
很多常数(例如圆周率\(\pi\))的严格定义就是依靠级数理论建立的.
另一方面,
由于一个函数项级数\(\sum_{n=1}^\infty u_n\)%
在给定自变量\(x=x_0\)时会成为一个常数项级数\(\sum_{n=1}^\infty a_n
= \sum_{n=1}^\infty u_n(x_0)\),
所以常数项级数又是我们研究函数项级数的基础.

\begin{table}[h]
	\centering
	\begin{tabular}{*3l}
		\hline
		名称 & 一般项 & 收敛条件 \\ \hline
		几何级数 & \(a_n = a q^n\ (a\neq0)\) & \(\abs{q}<1\) \\ %\cref{example:无穷级数.等比级数的收敛性}
		等差级数 & \(a_n = a + n d\) & \(a = d = 0\) \\ %\cref{example:无穷级数.等差级数的收敛性}
		调和级数 & \(a_n = 1/n\) & 不收敛 \\ %\cref{example:无穷级数.调和级数的敛散性}
		\(p\)级数 & \(a_n = 1/n^p\) & \(p > 1\) \\ %\cref{example:无穷级数.p级数的收敛性}
		& \(a_n = 1/(n \ln^q n)\) & \(q > 1\) \\ %\cref{example:无穷级数.对数负次幂级数}
		\hline
	\end{tabular}
	\caption{常见的常数项级数及其收敛条件}
\end{table}

\subsection*{收敛级数的性质}
收敛的常数项级数具有下述美妙性质:
\begin{enumerate}
	\item 由\cref{theorem:无穷级数.收敛级数性质1} 我们知道,
	任意一个级数的每一项同乘以一个非零常数后,所得级数的敛散性与原级数一致;
	任意一个级数的每一项同乘以零后,所得级数收敛于零.

	\item 由\cref{theorem:无穷级数.收敛级数性质2} 我们知道,
	任意两个收敛级数相加(或相减),所得级数也收敛;
	一个收敛级数与一个发散级数相加(或相减),所得级数必发散;
	两个发散级数相加(或相减),所得级数可能收敛也可能发散.

	\item 由\cref{theorem:无穷级数.收敛级数性质3} 我们知道,
	在级数中去掉、加上或改变有限项,不会改变级数的收敛性.

	\item 由\cref{theorem:无穷级数.收敛级数性质4} 我们知道,
	对收敛级数的项任意加括号后所成的级数仍收敛,且其和不变;
	如果加括号后所成的级数收敛,则原级数可能收敛也可能发散;
	如果加括号后所成的级数发散,则原级数必定发散.

	\item 由\cref{theorem:无穷级数.级数收敛的必要条件} 我们知道,
	收敛级数的一般项必定收敛于零,一般项不收敛于零的级数必定发散.
\end{enumerate}

\subsection*{常数项级数的审敛法}
在判断一个常数项级数是否收敛时,我们有许多工具可供利用:
\begin{enumerate}
	\item “级数\(\sum_{n=1}^\infty u_n\)收敛”的等价命题%
	就是“部分和数列\(\left\{ S_n = \sum_{i=1}^n u_i \right\}\)收敛”.

	\item
	对于一般的级数,我们可以利用\hyperref[theorem:无穷级数.级数的柯西审敛原理]{柯西审敛原理}.

	\item
	对于正项级数,我们知道,“正项级数收敛”的充分必要条件是“它的部分和数列有界”(\cref{theorem:无穷级数.正项级数收敛的充分必要条件}).
	除此以外,我们还有比较审敛法(\cref{theorem:无穷级数.正项级数的比较审敛法}、%
	\cref{theorem:无穷级数.正项级数的比较审敛法的推论}、%
	\cref{theorem:无穷级数.正项级数的比较审敛法的极限形式}),
	比值审敛法(\cref{theorem:无穷级数.正项级数的比值审敛法}、%
	\cref{theorem:无穷级数.正项级数的比值审敛法的上下极限形式}),
	根值审敛法(\cref{theorem:无穷级数.正项级数的根值审敛法}),
	极限审敛法(\cref{theorem:无穷级数.正项级数的极限审敛法})等方法,
	可以用于判断正项级数是否收敛.

	\item
	对于交错级数,我们可以利用\hyperref[theorem:无穷级数.莱布尼茨定理]{莱布尼茨定理}判断级数的敛散性.

	\item
	在某些情况下,为了判断一个常数项级数\(\sum_{n=0}^\infty u_n\)是否收敛,
	我们会首先研究这个级数各项的绝对值所构成的正项级数\(\sum_{n=0}^\infty \abs{u_n}\)是否收敛.
	这是因为%
	“级数\(\sum_{n=0}^\infty u_n\)绝对收敛”%
	(或者说“级数\(\sum_{n=0}^\infty \abs{u_n}\)收敛”)%
	必定有“级数\(\sum_{n=0}^\infty u_n\)收敛”(\cref{theorem:无穷级数.绝对收敛级数必定收敛});
	尽管反之不然(\cref{theorem:无穷级数.绝对发散的特殊情况}).
	此外,我们还可以利用两条绝对收敛级数具有的、条件收敛级数不具有的性质,即%
	“绝对收敛级数具有可交换性”(\cref{theorem:无穷级数.绝对收敛级数的可交换性})%
	和“两个绝对收敛级数的柯西乘积也是绝对收敛的”(\cref{theorem:无穷级数.绝对收敛级数的柯西乘积必收敛}),
	帮助我们构造辅助级数,用来证明给定的级数是否收敛.
\end{enumerate}

\subsection*{函数项级数的基本问题}
在研究完常数项级数以后,我们就可以着手研究函数项级数了.
常见的函数项级数按形式可以分为幂级数和三角级数.
对于函数项级数,我们最关心的是:
\begin{enumerate}
	\item 任给一个函数,能否将它展开成函数项级数?
	展得的级数的收敛域是什么?
	\item 任给一个函数项级数,能否求出它的和函数?
	和函数的定义域是什么?
\end{enumerate}

\subsection*{一致收敛级数的性质}
\begin{table}[htb]
	\centering
	% \scalebox{.8}{
	\begin{tblr}{*5c||c}
		\hline
		依据 & 考察区间 & 各项\(S_n\) & 函数列\(\{S_n\}\) & 导函数列\(\{S_n'\}\) & 极限函数\(S\) \\
		\hline
		\cref{theorem:函数项级数.连续函数列的一致收敛性保证极限函数的连续性}
		& \([a,b]\)
		& 连续
		& 一致收敛于\(S\)
		&& 连续 \\
		\cref{theorem:函数项级数.连续函数列的内闭一致收敛性保证极限函数的连续性}
		& \((a,b)\)
		& 连续
		& 内闭一致收敛于\(S\)
		&& 连续 \\
		\cref{theorem:函数项级数.连续函数列的一致收敛性保证极限函数的可积性}
		& \([a,b]\)
		& 连续
		& 一致收敛于\(S\)
		&& 可积 \\
		\cref{theorem:函数项级数.连续可导函数列的点态收敛性及其导函数列的一致收敛性保证极限函数的可微性}
		& \([a,b]\)
		& 连续可导
		& 点态收敛于\(S\)
		& 一致收敛于\(S'\)
		& 可导 \\
		\hline
	\end{tblr}
	% }
	\caption{}
\end{table}

\subsection*{计算函数的泰勒展开式}
要把函数\(f(x)\)展开成幂级数(麦克劳林级数),可以按照下列步骤进行:
\begin{enumerate}
	\item 求出函数\(f(x)\)的各阶导数\begin{equation*}
		f'(x),f''(x),\dotsc,f^{(n)}(x),\dotsc.
	\end{equation*}
	如果在\(x=0\)处某阶导数不存在,就停止展开,
	因为这就说明函数\(f(x)\)不能展开为麦克劳林级数.

	\item 求出函数及其各阶导数在\(x=0\)处的值:\begin{equation*}
		f(0),f'(0),f''(0),\dotsc,f^{(n)}(0),\dotsc.
	\end{equation*}

	\item 写出幂级数\begin{equation*}
		f(0) + f'(0) x + \frac{f''(0)}{2!} x^2
		+ \dotsb + \frac{f^{(n)}(0)}{n!} x^n + \dotsb,
	\end{equation*}
	并求出收敛半径\(R\).

	\item 利用余项\(R_n(x)\)的表达式\begin{equation*}
		R_n(x) = \frac{1}{(n+1)!} f^{(n+1)}(\theta x) x^{n+1}
		\quad(0 < \theta < 1),
	\end{equation*}
	考察当\(x\)在区间\((-R,R)\)内时余项的极限\(\lim_{n\to\infty} R_n(x)\)是否为零.
	\begin{enumerate}
		\item 如果\(\lim_{n\to\infty} R_n(x) = 0\),
		则函数\(f(x)\)在区间\((-R,R)\)内的麦克劳林展开式为
		\begin{equation*}
			f(x) = \sum_{n=0}^\infty \frac{1}{n!} f^{(n)}(0) x^n
			\quad(-R < x < R).
		\end{equation*}

		\item 如果\(\lim_{n\to\infty} R_n(x) \neq 0\),
		则函数\(f(x)\)不能展开为麦克劳林级数
		(以后在学习了复变函数的级数表示之后,我们可以将这类函数展开为罗朗级数).
	\end{enumerate}
\end{enumerate}

\subsection*{求解幂级数的和函数}
求解幂级数的和函数的具体思路如下:
\begin{enumerate}
	\item 首先判断幂级数的形式是不是标准形式.
	如果给定的幂级数是\begin{equation*}
		\sum_{n=0}^\infty a_n (t-t_0)^n,
	\end{equation*}
	则应先尝试运用“换元法”,
	令\(x=t-t_0\),
	将其化为标准形式\begin{equation*}
		\sum_{n=0}^\infty a_n x^n.
	\end{equation*}

	\item 利用\cref{theorem:无穷级数.阿贝尔定理1推论,theorem:无穷级数.幂级数的收敛半径的求法2}%
	(或\hyperref[theorem:无穷级数.正项级数的比值审敛法]{比值审敛法},
	或\hyperref[theorem:无穷级数.正项级数的根值审敛法]{根值审敛法})%
	求出幂级数的收敛半径\(R\),
	研究幂级数在点\(\pm R\)处的收敛性,
	定出幂级数的收敛域\(C\subseteq[-R,R]\).

	\item 最后我们要写出幂级数的和函数\(s(x)\)的解析式.
	这里我们要灵活地应用以下几种方法:\begin{enumerate}
		\item 比照已知的初等函数的幂级数(列举如下),写出幂级数的和函数.
		\begin{gather*}
			%@Mathematica: Sum[x^n/n!, {n, 0, Infinity}]
			e^x = \sum_{n=0}^\infty \frac{x^n}{n!}
				\quad(-\infty<x<+\infty), \\
			%@Mathematica: Sum[(-1)^k x^(2 k + 1)/(2 k + 1)!, {k, 0, Infinity}]
			\sin x = \sum_{k=0}^\infty \frac{(-1)^k}{(2k+1)!} x^{2k+1}
				\quad(-\infty<x<+\infty), \\
			%@Mathematica: Sum[(-1)^k x^(2 k)/(2 k)!, {k, 0, Infinity}]
			\cos x = \sum_{k=0}^\infty \frac{(-1)^k}{(2k)!} x^{2k}
				\quad(-\infty<x<+\infty), \\
			%@Mathematica: Sum[x^n, {n, 0, Infinity}]
			\frac{1}{1-x} = \sum_{n=0}^\infty x^n
				\quad(-1<x<1), \\
			%@Mathematica: Sum[(n + 1) x^n, {n, 0, Infinity}]
			\frac{1}{(1-x)^2} = \sum_{n=0}^\infty (n+1) x^n
				\quad(-1<x<1), \\
			%@Mathematica: Sum[(n + 2) (n + 1) x^n, {n, 0, Infinity}]
			\frac{2}{(1-x)^3} = \sum_{n=0}^\infty (n+2)(n+1) x^n
				\quad(-1<x<1), \\
			%@Mathematica: Sum[(-1)^n x^n, {n, 0, Infinity}]
			\frac{1}{1+x} = \sum_{n=0}^\infty (-1)^n x^n
				\quad(-1<x<1), \\
			%@Mathematica: Sum[(-1)^n/(n + 1) x^(n + 1), {n, 0, Infinity}]
			\ln(1+x) = \sum_{n=0}^\infty \frac{(-1)^n}{n+1} x^{n+1}
				\quad(-1<x\leq1), \\
			%@Mathematica: -Sum[1/n x^n, {n, 1, Infinity}]
			\ln(1-x) = -\sum_{n=1}^\infty \frac{x^n}{n}
				\quad(-1\leq x<1), \\
			%@Mathematica: Sum[x^(2 n + 1)/(2 n + 1), {n, 0, Infinity}]
			\artanh x = \sum_{n=0}^\infty \frac{x^{2n+1}}{2n+1}
				\quad(-1<x<1), \\
			%@Mathematica: Sum[(-1)^n x^(2 n), {n, 0, Infinity}]
			\frac{1}{1+x^2} = \sum_{n=0}^\infty (-1)^n x^{2n}
				\quad(-1<x<1), \\
			%@Mathematica: Sum[(-1)^n/(2 n + 1) x^(2 n + 1), {n, 0, Infinity}]
			\arctan x = \sum_{n=0}^\infty \frac{(-1)^n}{2n+1} x^{2n+1}
				\quad(-1 \leq x \leq 1), \\
			%@Mathematica: Sum[x^(2 k + 1)/(2 k + 1)!, {k, 0, Infinity}]
			\sinh x = \sum_{k=0}^\infty \frac{x^{2k+1}}{(2k+1)!}
				\quad(-\infty<x<+\infty), \\
			%@Mathematica: Sum[x^(2 k)/(2 k)!, {k, 0, Infinity}]
			\cosh x = \sum_{k=0}^\infty \frac{x^{2k}}{(2k)!}
				\quad(-\infty<x<+\infty), \\
			%@Mathematica: Sum[Gamma[m + 1]/(n! Gamma[m - n + 1]) x^n, {n, 0, Infinity}]
			%@Mathematica: Sum[Product[k, {k, m - n + 1, m}]/n! x^n, {n, 0, Infinity}]
			(1+x)^m = \sum_{n=0}^\infty \frac{m(m-1)\dotsm(m-n+1)}{n!} x^n
				\quad(-1<x<1,m\in\mathbb{R}).
		\end{gather*}

		\item 改变幂级数的求和指标,从而舍弃(或添加)有限项,
		即\begin{equation*}
			\sum_{k=0}^{m-1} b_k x^k + \sum_{n=m}^\infty a_n x^n
			= \sum_{n=0}^\infty a_n x^n + \sum_{k=0}^{m-1} (b_k-a_k) x^k.
		\end{equation*}

		\item 对幂级数进行适当的拆分,
		即\begin{equation*}
			\sum_{n=0}^\infty (a_n + b_n) x^n
			= \sum_{n=0}^\infty a_n x^n
			+ \sum_{n=0}^\infty b_n x^n.
		\end{equation*}
		特别是当幂级数的系数是关于\(n\)的有理分式函数时,
		更应该运用多项式的带余除法把系数拆成一个多项式与一个真分式之和.

		\item 对幂级数进行若干次逐项求导(或逐项积分),
		再按其他几种方法变换幂级数,
		最后进行若干次逐项积分(或逐项求导),
		即\begin{align*}
			s(x) &= \dv{x} \sum_{n=0}^\infty \int_0^x a_n t^n \dd{t}, \\
			s(x) &= s(0) + \int_0^x \left(\sum_{n=0}^\infty a_n \dv{t} t^n\right) \dd{t},
		\end{align*}
		其中\(s(0) = \eval{\sum_{n=0}^\infty a_n x^n}_{x=0}\).

		\item 对于形如\(\sum_{n=0}^\infty a_n x^{kn+b}\ (\text{$k,b$是常数})\),
		可以先求出级数\(\sum_{n=0}^\infty a_n y^n\)的和函数\(\sigma\),
		那么原级数的和函数就是\(x \mapsto x^b \cdot \sigma(x^k)\).

		\item 当系数\(a_n\)中存在形如\(\frac{1}{n+p}\)或\(n+q\)这样的因子时,
		构造辅助函数\begin{equation*}
			g(x) = x^m \cdot s(x) = \sum_{n=0}^\infty a_n x^{n+m}
			\quad(m=\pm1,\pm2,\dotsc),
		\end{equation*}
		使得我们在对新的幂级数\(\sum_{n=0}^\infty a_n x^{n+m}\)逐项求导(或逐项积分)后,
		系数\(a_n\)中的因子与\(x^{n+m}\)在求导(或积分)时带出的因子\(n+m\)(或\(\frac{1}{n+m+1}\))恰好消去;
		像这样,只要求出\(g(x)\),就有\begin{equation*}
			s(x) = x^{-m} \cdot g(x);
		\end{equation*}
		这里要注意辅助函数中\(m\)的取值,如果\(m>0\),那么上式隐含\(x\neq0\)这一限定条件.
	\end{enumerate}
	在运用上述方法时,需要注意函数解析式是否存在奇点(例如点\(x=0\));
	在发现奇点后,可以将奇点\(x=x_0\)代入幂级数\(\sum_{n=0}^\infty a_n x^n\),
	求出特殊值\(s(x_0) = \sum_{n=0}^\infty a_n x_0^n\);
	最后把和函数写成分段函数的形式.
\end{enumerate}

\subsection*{计算函数的傅里叶展开式}
求解任意函数的傅里叶级数展开式的具体思路如下:
\begin{enumerate}
	\item 对于任意一个周期为\(2\pi\)的函数\(f(x)\),
	我们只要利用积分表,
	例如\begin{gather*}
		\int_{-\pi}^\pi \cos{nx} \dd{x} = 0 \quad(n=1,2,\dotsc), \\
		\int_{-\pi}^\pi \sin{nx} \dd{x} = 0 \quad(n=1,2,\dotsc), \\
		\int_{-\pi}^\pi \sin{mx}\cos{nx} \dd{x} = 0 \quad(m,n=1,2,\dotsc), \\
		\int_{-\pi}^\pi \cos{mx}\cos{nx} \dd{x} = 0 \quad(m,n=1,2,\dotsc; m \neq n), \\
		\int_{-\pi}^\pi \sin{mx}\sin{nx} \dd{x} = 0 \quad(m,n=1,2,\dotsc; m \neq n),
	\end{gather*}
	和\begin{gather*}
		\int x \sin ax \dd{x}
		= \frac{1}{a^2} \sin ax - \frac{1}{a} x \cos ax + C, \\
		\int x^2 \sin ax \dd{x}
		= -\frac{1}{a} x^2\cos ax + \frac{2}{a^2} x\sin ax + \frac{2}{a^3} \cos ax + C, \\
		\int x \cos ax \dd{x}
		= \frac{1}{a^2} \cos ax + \frac{1}{a} x \sin ax + C, \\
		\int x^2 \cos ax \dd{x}
		= \frac{1}{a} x^2 \sin ax + \frac{2}{a^2} x \cos ax - \frac{2}{a^3} \sin ax + C, \\
		\int e^{ax} \sin bx \dd{x}
		= \frac{1}{a^2+b^2} e^{ax} (a \sin bx - b \cos bx) + C, \\
		\int e^{ax} \cos bx \dd{x}
		= \frac{1}{a^2+b^2} e^{ax} (b \sin bx + a \cos bx) + C,
	\end{gather*}
	计算出它的傅里叶系数\begin{equation*}
		a_n = \frac{1}{\pi} \int_{-\pi}^\pi f(x) \cos nx \dd{x}
		\quad(n=0,1,2,\dotsc),
	\end{equation*}\begin{equation*}
		b_n = \frac{1}{\pi} \int_{-\pi}^\pi f(x) \sin nx \dd{x}
		\quad(n=1,2,3,\dotsc),
	\end{equation*}
	就可以写出它的傅里叶级数\begin{equation*}
		\frac{a_0}{2} + \sum_{n=1}^\infty (a_n \cos nx + b_n \sin nx)
		\quad(x \in C).
	\end{equation*}
	这里,函数\(f(x)\)的傅里叶级数的收敛域\(C = \Set*{
		x \given
		f(x) = \frac{1}{2} [ f(x^-) + f(x^+) ]
	}\).

	\item 对于任意一个周期为\(2l\ (l\neq\pi,l>0)\)的函数\(\phi(t)\),
	令\(\frac{x}{\pi} = \frac{t}{l}\),
	得到函数\(f(x) = \phi(xl/\pi)\),
	再按上述步骤将\(f(x)\)展开成傅里叶级数,
	最后利用变量代换得到\(\phi(t)\)的傅里叶级数.

	\item 对于只在闭区间\([-\pi,\pi]\)上有定义的函数\(f(x)\),
	我们可以应用周期延拓方法,在原有的定义域以外补充函数定义,
	再将其展开成傅里叶级数,
	最后将定义域限制为开区间\((-\pi,\pi)\).

	\item 对于只在闭区间\([0,\pi]\)上有定义的函数\(f(x)\),
	我们可以应用奇延拓(或偶延拓)方法,在区间\((-\pi,0)\)内补充函数定义,
	得到定义在\((-\pi,\pi]\)上的函数\(F(x)\)(注意要令\(F(0) = 0\)),
	使得它在\((-\pi,\pi)\)上成为奇函数(或偶函数),
	然后将\(F(x)\)展开成傅里叶级数,
	最后将定义域限制为\((0,\pi]\).

	\item 对于定义在非对称区间\([a,b]\)上的函数\(f(x)\),
	可以采用以下两种方法之一展成傅里叶级数:
		\begin{enumerate}
			\item 先作变换\(x = z + \frac{b+a}{2}\),
			记\(c = (b-a)/2\),
			使得\(z\in[-c,c]\),再进行周期延拓,展成傅里叶级数.
			\item 先作变换\(x = z + a\),
			使得\(z\in[0,b-a]\),再进行奇延拓或偶延拓,展成正弦级数或余弦级数.
		\end{enumerate}
\end{enumerate}


\chapter{积分变换}
\section{傅里叶变换}
\subsection{对连续函数的傅里叶变换}
\begin{definition}
设函数\(f(x)\)是可积函数,则广义积分\begin{equation*}
	\int_{-\infty}^{+\infty}
	f(x) e^{-2 \pi i x \xi}
	\dd{x},
	\quad \xi \in \mathbb{R}
\end{equation*}
称为“对函数\(f(x)\)的\DefineConcept{傅里叶变换}”,
通常记为\(\hat{f}(\xi)\),
即\begin{equation*}
	\hat{f}(\xi)
	= \int_{-\infty}^{+\infty} f(x) e^{-2 \pi i x \xi} \dd{x},
	\quad \xi \in \mathbb{R}.
\end{equation*}
\end{definition}

\begin{theorem}
设函数\(\hat{f}(x)\)是对函数\(f(x)\)的傅里叶变换,
则有\begin{equation*}
	f(x)
	= \int_{-\infty}^{+\infty}
	\hat{f}(\xi) e^{2 \pi i \xi x} \dd{\xi},
	\quad x \in \mathbb{R}.
\end{equation*}
\end{theorem}

\section{拉普拉斯变换}
\subsection{对周期函数的拉普拉斯变换}
\begin{definition}
设\(f\colon[0,+\infty)\to\mathbb{C}\),
\(D\subseteq\mathbb{C}\).
若对\(\forall p \in D\),
都有含参反常积分\(F(p)=\int_0^{+\infty} f(t) e^{-pt} \dd{t}\)收敛,
则把\(F(p)\)称为“\(f(t)\)的\DefineConcept{拉普拉斯变换}”,
记作\(\lt f(p)\);
同时,相对地,把\(f(t)\)称为“\(F(p)\)的\DefineConcept{拉普拉斯逆变换}”,
记作\(\lt^{-1} F(t)\).
%@see: https://ins.sjtu.edu.cn/people/songtingli/resources/slides/MathPhysics/Lecture11.pdf
%@see: https://www.lamda.nju.edu.cn/yehj/dsp2021/09.pdf
\end{definition}

\begin{theorem}
设函数\(f(x)\)是周期为\(T\)的连续函数,
\(s>0\),
则\begin{equation*}
	\int_0^{+\infty} f(x) e^{-sx} \dd{x}
	= \frac1{1-e^{-sT}}
	\int_0^T f(x) e^{-sx} \dd{x}.
\end{equation*}
%@see: https://www.bilibili.com/video/BV1ja4y1Q7eF
\end{theorem}


\chapter{卷积分}

\section{卷积的概念与性质}
\begin{definition}
设函数\(f(x)\)、\(g(x)\)都是可积函数,则广义积分\begin{equation*}
\int_{-\infty}^{+\infty} f(\tau) g(x - \tau) \dd{\tau}
\end{equation*}称为\(f\)和\(g\)的卷积,记作\((f * g)(x)\),简记为\(f*g\).
\end{definition}

\begin{property}[交换律]
\(f*g = g*f\).
\end{property}

\begin{property}[结合律 I]
\(f*(g*h) = (f*g)*h\).
\end{property}

\begin{property}[结合律 II]
\(a (f*g) = (a f)*g = f*(a g), \quad a \in \mathbb{R}\).
\end{property}

\begin{property}[分配律]
\(f*(g+h)=(f*g)+(f*h)\).
\end{property}

\section{微分定理}
\begin{theorem}
设函数\(f\)和\(g\)都是可积函数,则它们的微分满足关系\begin{equation*}
\dd(f*g) = \dd{f}*g = g*\dd{f}.
\end{equation*}
\end{theorem}

\section{卷积定理}
\begin{theorem}
设函数\(f\)和\(g\)都是可积函数,则它们的傅里叶变换满足关系\begin{equation*}
\ft(f*g) = \ft(f) \cdot \ft(g).
\end{equation*}
\end{theorem}

