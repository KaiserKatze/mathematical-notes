%线性代数的核心内容是研究有限维线性空间的结构和线性空间的线性变换.
%由于数域\(K\)上的任意一个\(n\)维线性空间\(V\)都与\(n\)维向量空间\(K^n\)同构,
%\(V\)上的全体线性变换构成的集合\(L(V,V)\)%
%与数域\(K\)上的全体\(n \times n\)矩阵构成的集合\(K^{n \times n}\)同构,
%因此,在本书中线性代数主要研究:
%矩阵理论、\(n\)维向量的线性关系、
%线性方程组、行列式、二次型、矩阵的特征值与特征向量、内积等内容.

\chapter{线性方程组}
\section{线性方程组}
我们把含有\(n\)个未知量\(\AutoTuple{x}{n}\)的一次方程组
\begin{equation}\label[equation-system]{equation:线性方程组.线性方程组的代数形式}
	\left\{ \begin{array}{l}
		a_{11} x_1 + a_{12} x_2 + \dotsb + a_{1n} x_n = b_1, \\
		a_{21} x_1 + a_{22} x_2 + \dotsb + a_{2n} x_n = b_2, \\
		\hdotsfor{1} \\
		a_{s1} x_1 + a_{s2} x_2 + \dotsb + a_{sn} x_n = b_s
	\end{array} \right.
\end{equation}
称为“\(n\)元\DefineConcept{线性方程组}(\emph{linear system of equations} in \(n\) variables)”.
这里,\(s\)为方程的个数\footnote{%
在\cref{equation:线性方程组.线性方程组的代数形式} 中,
方程的数目\(s\)可以等于未知数的数目\(n\),
也可以不相等(即\(s<n\)或\(s>n\)).};
我们把数\[
	a_{ij}
	\quad(i=1,2,\dotsc,s; j=1,2,\dotsc,n)
\]称为“第\(i\)个方程中\(x_j\)的\DefineConcept{系数}(coefficient)”;
把数\[
	b_i
	\quad(i=1,2,\dotsc,s)
\]叫做“第\(i\)个方程的\DefineConcept{常数项}(constant term)”.

\begin{definition}
我们把常数项全为零的线性方程组
称为\DefineConcept{齐次线性方程组}(homogeneous linear systems of equations),
把常数项中有非零数的线性方程组
称为\DefineConcept{非齐次线性方程组}(nonhomogeneous linear systems of equations,
inhomogeneous linear systems of equations).
\end{definition}

\begin{definition}
如果存在\(n\)个数\(\AutoTuple{c}{n}\)满足\cref{equation:线性方程组.线性方程组的代数形式},即\[
	a_{i1} c_1 + a_{12} c_2 + \dotsb + a_{in} c_n \equiv b_i
	\quad(i=1,2,\dotsc,s),
\]
则称“\cref{equation:线性方程组.线性方程组的代数形式} 有解”,
或“\cref{equation:线性方程组.线性方程组的代数形式} 是相容的”;
否则,称“\cref{equation:线性方程组.线性方程组的代数形式} 无解”,
或“\cref{equation:线性方程组.线性方程组的代数形式} 是不相容的”.

称这\(n\)个数\(\AutoTuple{c}{n}\)构成的列向量\((\AutoTuple{c}{n})^T\)为%
\cref{equation:线性方程组.线性方程组的代数形式} 的一个\DefineConcept{解}(solution)%
或\DefineConcept{解向量}(solution vector).
\cref{equation:线性方程组.线性方程组的代数形式} 的解的全体构成的集合,
称为“\cref{equation:线性方程组.线性方程组的代数形式} 的\DefineConcept{解集}”.
\end{definition}

\begin{definition}
元素全为零的解向量称为\DefineConcept{零解}(zero solution).
元素不全为零的解向量称为\DefineConcept{非零解}(nonzero solution).
\end{definition}

\begin{theorem}
任意一个齐次线性方程组都有零解.
\end{theorem}

\begin{definition}
解集相等的两个线性方程组称为\DefineConcept{同解方程组}.
\end{definition}

需要注意的是,线性方程组的解集与数域有关.

\begin{theorem}
\(n\)元线性方程组的解的情况有且只有三种可能:
无解、有唯一解、有无穷多个解.
\end{theorem}

\section{克拉默法则}
首先研究有两个二元一次方程的方程组\begin{equation}\label[equation-system]{equation:克拉默法则.两个二元一次方程}
%@see: 《高等代数(第三版 上册)》(丘维声) P18 (1)
	\left\{ \begin{array}{l}
		a_{11} x_1 + a_{12} x_2 = b_1, \\
		a_{21} x_1 + a_{22} x_2 = b_2,
	\end{array} \right.
\end{equation}
其中\(a_{11},a_{21}\)不全为零.
不妨设\(a_{11}\neq0\).
将这个方程组的增广矩阵经过初等行变换化成阶梯形矩阵:\begin{equation*}
	\begin{bmatrix}
		a_{11} & a_{12} & b_1 \\
		a_{21} & a_{22} & b_2
	\end{bmatrix}
	\to \begin{bmatrix}
		a_{11} & a_{12} & b_1 \\
		0 & a_{22}-\frac{a_{21}}{a_{11}} a_{12} & b_2-\frac{a_{21}}{a_{11}} b_1
	\end{bmatrix}.
\end{equation*}
可以看出:\begin{itemize}
	\item 当\(a_{11} a_{22} - a_{12} a_{21} = 0\)时,
	\begin{itemize}
		\item 如果\(b_2-\frac{a_{21}}{a_{11}} b_1=0\),
		那么\cref{equation:克拉默法则.两个二元一次方程} 有无穷多解;
		\item 如果\(b_2-\frac{a_{21}}{a_{11}} b_1\neq0\),
		那么\cref{equation:克拉默法则.两个二元一次方程} 无解.
	\end{itemize}

	\item 当\(a_{11} a_{22} - a_{12} a_{21} \neq 0\)时,
	\cref{equation:克拉默法则.两个二元一次方程} 有唯一解:\begin{equation*}
	%@see: 《高等代数(第三版 上册)》(丘维声) P18 (2)
		\begin{bmatrix}
			\frac{b_1 a_{22} - b_2 a_{12}}{a_{11} a_{22} - a_{12} a_{21}},
			\frac{a_{11} b_2 - a_{21} b_1}{a_{11} a_{22} - a_{12} a_{21}}
		\end{bmatrix}^T.
	\end{equation*}
\end{itemize}

我们想要知道,上述结论能否推广到
数域\(K\)上由\(n\)个\(n\)元线性方程组成的方程组:
\begin{equation}\label[equation-system]{equation:克拉默法则.线性方程组}
	\left\{ \begin{array}{l}
		a_{11}x_1 + a_{12}x_2 + \dotsb + a_{1n}x_n = b_1, \\
		a_{21}x_1 + a_{22}x_2 + \dotsb + a_{2n}x_n = b_2, \\
		\hdotsfor{1} \\
		a_{n1}x_1 + a_{n2}x_2 + \dotsb + a_{nn}x_n = b_n.
	\end{array} \right.
\end{equation}
我们把\cref{equation:克拉默法则.线性方程组} 的系数矩阵、增广矩阵分别记为\(\vb{A}\)和\(\widetilde{\vb{A}}\).
对增广矩阵\(\widetilde{\vb{A}}\)施行初等行变换,化成阶梯形矩阵\(\widetilde{\vb{J}}\),
则系数矩阵\(\vb{A}\)经过这些初等行变换也被化成阶梯形矩阵\(\vb{J}\).

假设\(\abs{\vb{A}}\neq0\),则\(\abs{\vb{J}}\neq0\),
于是\(\vb{J}\)没有零行,或者说\(\vb{J}\)的每一行都是非零行,因此\(\vb{J}\)有\(n\)个主元.
由于\(\vb{J}\)只有\(n\)列,因此\(\vb{J}\)的\(n\)个主元分别位于第\(1,2,\dotsc,n\)行,
即\begin{equation*}
	\vb{J} = \begin{bmatrix}
		c_{11} & c_{12} & \dots & c_{1n} \\
		0 & c_{22} & \dots & c_{2n} \\
		\vdots & \vdots & & \vdots \\
		0 & 0 & \dots & c_{nn}
	\end{bmatrix},
\end{equation*}
其中\(c_{11} c_{22} \dotsm c_{nn} \neq 0\).
由于\(\widetilde{\vb{J}}\)比\(\vb{J}\)多一列,因此\begin{equation*}
	\widetilde{\vb{J}} = \begin{bmatrix}
		c_{11} & c_{12} & \dots & c_{1n} & d_1 \\
		0 & c_{22} & \dots & c_{2n} & d_2 \\
		\vdots & \vdots & & \vdots & \vdots \\
		0 & 0 & \dots & c_{nn} & d_n
	\end{bmatrix}.
\end{equation*}
由此看出,原\cref{equation:克拉默法则.线性方程组} 有解,
由于\(\widetilde{\vb{J}}\)的非零行数目等于未知量数目,
所以原\cref{equation:克拉默法则.线性方程组} 有唯一解.

假设\(\abs{\vb{A}}=0\),则\(\abs{\vb{J}}=0\),于是我们可以断言\(\vb{J}\)必有零行,
那么\(\vb{J}\)的非零行数目\(r\)必定小于行数\(n\),
% 即\begin{equation*}
% 	\vb{J} = \begin{bmatrix}
% 		c_{11} & \dots & \dots & \dots
% 	\end{bmatrix}
% \end{equation*}
%TODO

\begin{theorem}[克拉默法则]\label{theorem:线性方程组.克拉默法则}
%@see: 《高等代数(第三版 上册)》(丘维声) P46 定理1
%@see: 《高等代数(第三版 上册)》(丘维声) P48 定理3
%@see: https://mathworld.wolfram.com/CramersRule.html
给定一个未知量个数与方程个数相同的\cref{equation:克拉默法则.线性方程组}.
如果它的系数行列式满足\begin{equation*}
	d
	=\begin{vmatrix}
		a_{11} & a_{12} & \dots & a_{1n} \\
		a_{21} & a_{22} & \dots & a_{2n} \\
		\vdots & \vdots & & \vdots \\
		a_{n1} & a_{n2} & \dots & a_{nn}
	\end{vmatrix}
	\neq 0,
\end{equation*}
则线性方程组 \labelcref{equation:克拉默法则.线性方程组} 存在唯一解:
\begin{equation}
	\vb{x}_0
	= \left( \frac{d_1}{d},\frac{d_2}{d},\dotsc,\frac{d_n}{d} \right)^T,
\end{equation}
其中\begin{equation}
	d_k
	= \begin{vmatrix}
		a_{11} & \dots & a_{1\ k-1} & b_1 & a_{1\ k+1} & \dots & a_{1n} \\
		a_{21} & \dots & a_{2\ k-1} & b_2 & a_{2\ k+1} & \dots & a_{2n} \\
		\vdots & & \vdots & \vdots & \vdots & & \vdots \\
		a_{n1} & \dots & a_{n\ k-1} & b_n & a_{n\ k+1} & \dots & a_{nn}
	\end{vmatrix},
	\quad k=1,2,\dotsc,n.
\end{equation}
%TODO 在不借助矩阵记号的情况下可能需要数学归纳法证明?
% \begin{proof}
% 上述原线性方程组可以改写为矩阵形式\(\vb{A} \vb{x} = \vb\beta\).
% 因为\(d = \abs{\vb{A}} \neq 0\),故\(\vb{A}\)可逆,那么线性方程组有唯一解\begin{equation*}
% 	\vb{x}_0 = \vb{A}^{-1} \vb\beta = \frac{1}{d} \vb{A}^* \vb\beta,
% \end{equation*}其中\(\vb{A}^*\)是\(\vb{A}\)的伴随矩阵.
% 设\(A_{ij}\)表示\(\vb{A}\)中元素\(a_{ij}\ (i,j=1,2,\dotsc,n)\)的代数余子式,
% 则行列式\(d_k\)可按第\(k\)列展开,得\begin{equation*}
% 	d_k = b_1 A_{1k} + b_2 A_{2k} + \dotsb + b_n A_{nk},
% 	\quad k=1,2,\dotsc,n,
% \end{equation*}
% 则\begin{equation*}
% 	\vb{x}_0 = \frac{1}{d} \vb{A}^* \vb\beta
% 	= \frac{1}{d} \begin{bmatrix}
% 		A_{11} & A_{21} & \dots & A_{n1} \\
% 		A_{12} & A_{22} & \dots & A_{n2} \\
% 		\vdots & \vdots & \ddots & \vdots \\
% 		A_{1n} & A_{2n} & \dots & A_{nn}
% 	\end{bmatrix} \begin{bmatrix} b_1 \\ b_2 \\ \vdots \\ b_n \end{bmatrix}
% 	= \frac{1}{d} \begin{bmatrix} d_1 \\ d_2 \\ \vdots \\ d_n \end{bmatrix}.
% 	\qedhere
% \end{equation*}
% \end{proof}
\end{theorem}

\section{高斯--若尔当消元法}
\begin{definition}
在数域\(K\)上,
设矩阵\(\vb{A} \in M_{s \times n}(K)\),
向量\(\vb{x} \in K^n\),
\(\vb\beta \in K^s\).
由\(n\)元线性方程组\begin{equation*}
	\vb{A} \vb{x} = \vb\beta
\end{equation*}的系数和常数项按原位置构成的\(s \times (n+1)\)矩阵\((\vb{A},\vb\beta)\),
称为该\(n\)元线性方程组的\DefineConcept{增广矩阵}(augmented matrix),
记作\(\widetilde{\vb{A}}\).
\end{definition}

\begin{theorem}
对线性方程组\(\vb{A} \vb{x} = \vb\beta\)的增广矩阵\(\widetilde{\vb{A}} = (\vb{A},\vb\beta)\)作初等行变换,
变为\(\widetilde{\vb{C}} = (\vb{C},\vb\gamma)\),
则相应的线性方程组\(\vb{C} \vb{x} = \vb\gamma\)与原线性方程组同解.
\begin{proof}
显然存在可逆矩阵\(\vb{P}\),
使得\(\widetilde{\vb{A}} \to \widetilde{\vb{C}} = \vb{P} \widetilde{\vb{A}} = (\vb{P}\vb{A},\vb{P}\vb\beta)\),
\(\vb{C} = \vb{P}\vb{A}\),\(\vb\gamma = \vb{P}\vb\beta\).
如果\(\vb{x}_0\)是原线性方程组的解,即\(\vb{A} \vb{x}_0 = \vb\beta\),
用\(\vb{P}\)左乘等式两端得\(\vb{C} \vb{x}_0 = \vb\gamma\);
反之,若\(\vb{x}_0\)满足\(\vb{C} \vb{x}_0 = \vb\gamma\),
用\(\vb{P}^{-1}\)左乘等式两端得\(\vb{A} \vb{x}_0 = \vb\beta\),
故两方程组同解.
\end{proof}
\end{theorem}
这就是消元法解线性方程组的理论根据.
具体化简\(\widetilde{\vb{A}}\)时,
可用一系列初等行变换将其变成一个较为简单的“阶梯形矩阵”(或更简单的“若尔当阶梯形矩阵”).

\begin{definition}
称如下形式的\(s \times n\)矩阵\begin{equation*}
	\vb{A} = \begin{bmatrix}
		0 & \dots & a_{1 j_1} & \dots & a_{1 j_2} & \dots & a_{1 j_r} & \dots & a_{1n} \\
		0 & \dots & 0 & \dots & a_{2 j_2} & \dots & a_{2 j_r} & \dots & a_{2n} \\
		\vdots & & \vdots & & \vdots & & \vdots & & \vdots \\
		0 & \dots & 0 & \dots & 0 & \dots & a_{r j_r} & \dots & a_{rn} \\
		0 & \dots & 0 & \dots & 0 & \dots & 0 & \dots & 0 \\
		\vdots & & \vdots & & \vdots & & \vdots & & \vdots \\
		0 & \dots & 0 & \dots & 0 & \dots & 0 & \dots & 0 \\
	\end{bmatrix}
\end{equation*}为\DefineConcept{阶梯形矩阵}(echelon form),
%@see: https://mathworld.wolfram.com/EchelonForm.html
其中\(a_{1 j_1},a_{2 j_2},\dotsc,a_{r j_r}\)均不为零,
\(j_1 < j_2 < \dotsb < j_r\),
\(\vb{A}\)的后\(s-r\)行全为零.

元素全为\(0\)的行,称为\DefineConcept{零行}.
元素不全为\(0\)的行,称为\DefineConcept{非零行}.

在非零行中,从左数起,第一个不为\(0\)的元素\(a_{i j_i}\ (i=1,2,\dotsc,r)\),
称为\DefineConcept{主元}(pivot)或\DefineConcept{非零首元}.

以主元为系数的未知量\(x_{j_1},x_{j_2},\dotsc,x_{j_r}\),
称为\DefineConcept{主变量};
不以主元为系数的未知量,称为\DefineConcept{自由未知量}.
\end{definition}

\begin{definition}
若阶梯形矩阵\(\vb{A}\)的非零行的非零首元全为\(1\),
它们所在列的其余元素全为零,
则称\(\vb{A}\)为\DefineConcept{若尔当阶梯形矩阵}或\DefineConcept{行约化矩阵}.
\end{definition}
% 在Mathematica中可以用RowReduce对矩阵进行初等行变换化为若尔当阶梯形矩阵.

\begin{lemma}
任何一个非零矩阵都可经初等行变换化为阶梯形矩阵.
\begin{proof}
设\(\vb{A}_{s \times n} \neq \vb0\),则\(\vb{A}\)经0次或1次交换两行的变换化为\(\vb{B}\),即\begin{equation*}
	\vb{A} \to \vb{B} = \begin{bmatrix}
		0 & \dots & 0 & b_{1 j_1} & \dots & b_{1n} \\
		0 & \dots & 0 & b_{2 j_1} & \dots & b_{2n} \\
		\vdots & & \vdots & \vdots & & \vdots \\
		0 & \dots & 0 & b_{s j_1} & \dots & b_{sn}
	\end{bmatrix},
\end{equation*}
其中\(b_{1 j_1} \neq 0\).
分别将\(\vb{B}\)的第一行的\(-b_{i j_1}/b_{1 j_1}\)倍加到第\(i\ (i=2,3,\dotsc,s)\)行,则\begin{equation*}
	\vb{B} \to \vb{C} = \begin{bmatrix}
		\vb0 & b_{1 j_1} & \vb{B}_1 \\
		\vb0 & \vb0 & \vb{C}_1
	\end{bmatrix},
\end{equation*}
其中\(\vb{C}_1\)是\((s-1)\times(n-j_1)\)矩阵,
再对\(\vb{C}\)的后面\(s-1\)行作类似的初等行变换化简.
因为矩阵行数有限,这样下去,最后总可化为阶梯形矩阵.
\end{proof}
\end{lemma}

\begin{corollary}\label{theorem:线性方程组.非零矩阵可经初等行变换化为若尔当阶梯形矩阵}
任何一个非零矩阵都可经初等行变换化为若尔当阶梯形矩阵.
\end{corollary}

\begin{theorem}
设\(\vb{A}\)为\(n\)阶方阵,
则齐次线性方程组\(\vb{A}\vb{x} = \vb0\)有非零解的充分必要条件是:
\(\abs{\vb{A}} = 0\).
\begin{proof}
必要性.
给定\(\vb{x}_0 \neq \vb0\)满足\(\vb{A} \vb{x}_0 = \vb0\).
假设\(\abs{\vb{A}} \neq 0\),
由克拉默法则可知方程组有唯一解\(\vb{x}_0 = \vb{A}^{-1} \vb0 = \vb0\),
矛盾,故\(\abs{\vb{A}} = 0\).

充分性.
用数学归纳法证明.
给定\(\abs{\vb{A}} = 0\).
当\(n=1\)时,\(\vb{A} = \vb0_{1 \times 1} = 0\),
\(0 \cdot x_1 = 0\)有非零解;
假设当\(n=k-1\geq1\)时,结论成立;
那么当\(n=k\)时,
设\(\vb{A}\)经初等行变换\(\vb{P}\)化为阶梯形矩阵\begin{equation*}
	\vb{B} = \begin{bmatrix}
		b & \vb\gamma \\
		\vb0 & \vb{C} \\
	\end{bmatrix} = \vb{P} \vb{A},
\end{equation*}
其中\(\vb{C}\)是\(n-1\)阶方阵,\(\vb{P}\)是\(n\)阶可逆矩阵.
取行列式得
\begin{equation*}
	\abs{\vb{B}} = \abs{\vb{P}} \abs{\vb{A}} = 0 = b \abs{\vb{C}}.
\end{equation*}
解同解方程组\(\vb{B} \vb{x} = \vb0\).
若\(b = 0\)则\((1,0,\dotsc,0)^T\)是一个非零解;
若\(b \neq 0\),则\(\abs{\vb{C}} = 0\),由归纳假设,齐次线性方程组\begin{equation*}
	\vb{C} \begin{bmatrix} x_2 \\ x_3 \\ \vdots \\ x_n \end{bmatrix} = \vb0
\end{equation*}有非零解\((k_2,k_3,\dotsc,k_n)^T\),
代入\(\vb{B} \vb{x} = \vb0\)的第一个方程,
因为\(x_1\)的系数\(b \neq 0\),可解出\(x_1\).
于是\((\AutoTuple{k}{n})^T\)是\(\vb{A} \vb{x} = \vb0\)的一个非零解.
\end{proof}
\end{theorem}

\begin{corollary}\label{theorem:线性方程组.方程个数少于未知量个数的齐次线性方程组必有非零解}
方程个数少于未知量个数的齐次线性方程组必有非零解.
\begin{proof}
设数域\(K\)上的线性方程组\(\vb{A} \vb{x} = \vb0\),
它的系数矩阵\(\vb{A} \in M_{s \times n}(K)\ (s<n)\).
在原方程组的后面添加\(n-s\)个\(0=0\)的方程,解不变,新方程组的系数矩阵为:
\begin{equation*}
	\vb{B}_n = \begin{bmatrix} \vb{A}_{s \times n} \\ \vb0_{(n-s) \times n} \end{bmatrix},
\end{equation*}
由于\(s < n\),有\(\abs{\vb{B}} = 0\),故\(\vb{B} \vb{x} = \vb0\)有非零解,从而\(\vb{A} \vb{x} = \vb0\)有非零解.
\end{proof}
%\cref{theorem:线性方程组.齐次线性方程组只有零解的充分必要条件}
%\cref{theorem:线性方程组.齐次线性方程组有非零解的充分必要条件}
\end{corollary}
\begin{remark}
\cref{theorem:线性方程组.方程个数少于未知量个数的齐次线性方程组必有非零解} 的逆命题不成立,
即方程组有非零解,不能得出未知量个数与方程个数的关系.
\end{remark}


\chapter{向量、矩阵及其基本运算}
\chapter{向量、矩阵及其基本运算}
\section{矩阵与向量的概念}
\subsection{矩阵与向量的基本概念}
%@see: https://math.stackexchange.com/a/1162585/591741
\begin{definition}
设\(s,n\)都是正整数,\(K\)是数域.
我们把从\(\{1,\dotsc,s\}\times\{1,\dotsc,n\}\)到\(K\)的映射,
称为“数域\(K\)上的\(s \times n\)~\DefineConcept{矩阵}(matrix)”.
%@see: https://mathworld.wolfram.com/Matrix.html
\end{definition}

为了简化描述,我们定义:\[
	M_{s \times n}(K)
	\defeq
	\Set{
		\A \given
		\text{\(\A\)是数域\(K\)上的\(s \times n\)矩阵}
	}.
\]

设\(\A\in M_{s \times n}(K)\).
我们把\(s\)称为“\(\A\)的\DefineConcept{行数}”,
把\(n\)称为“\(\A\)的\DefineConcept{列数}”.
把\[
	\A(i,j)
	\quad(i=1,2,\dotsc,s;j=1,2,\dotsc,n)
\]称为“\(\A\)的\((i,j)\)元素”
或“\(\A\)的第\(i\)行第\(j\)列\DefineConcept{元素}(element)”.
%@see: https://mathworld.wolfram.com/MatrixElement.html

对于任意给定的两个矩阵\(\A\)和\(\B\),
如果它们行数、列数都相同,
则称“\(\A\)与\(\B\) \DefineConcept{同型}”.

如果矩阵\(\A\)的行数\(s\)恰好等于它的列数\(n\),
我们就把它称为“\(n\)阶矩阵”或“\(n\)阶\DefineConcept{方阵}(square matrix)”.
%@see: https://mathworld.wolfram.com/SquareMatrix.html
定义:\[
	M_n(K)
	\defeq
	M_{n \times n}(K).
\]

设\(\A\in M_{s \times n}(K)\).
我们把\[
	\A(k,k)
	\quad(k=1,2,\dotsc,\min\{s,n\})
\]称为“\(\A\)的\DefineConcept{主对角线}(main diagonal)上的元素”.
%@see: https://mathworld.wolfram.com/Diagonal.html
%@see: https://mathworld.wolfram.com/SkewDiagonal.html

我们把行数为\(1\)、列数为\(n\)的矩阵称为“\(n\)维\DefineConcept{行向量}(row vector)”;
%@see: https://mathworld.wolfram.com/RowVector.html
把列数为\(1\)、行数为\(n\)的矩阵称为“\(n\)维\DefineConcept{列向量}(column vector)”.
%@see: https://mathworld.wolfram.com/ColumnVector.html
\(n\)维行向量和\(n\)维列向量统称为“\(n\)维\DefineConcept{向量}(vector)”.
通常用一个黑体小写希腊字母(如\(\a,\b,\g\)等)表示向量.

设\(\a\in M_{s \times 1}(K)\).
我们把\(s\)称为“\(\a\)的\DefineConcept{维数}”.
定义:\[
	\a(i) \defeq \a(i,1).
\]
把\(\a(i)\ (i=1,2,\dotsc,s)\)称为“\(\a\)的第\(i\)个\DefineConcept{分量}”.

设\(\b\in M_{1 \times n}(K)\).
把\(n\)称为“\(\b\)的\DefineConcept{维数}”.
定义:\[
	\b(i) \defeq \b(1,i).
\]
把\(\b(i)\ (i=1,2,\dotsc,n)\)称为“\(\b\)的第\(i\)个\DefineConcept{分量}”.

有时候为了强调矩阵的行数和列数,
我们会在表示矩阵的字母的右下角标注矩阵的形状,例如,
如果\(\A\in M_{s \times n}(K)\),那么我们可以写出\(\A_{s \times n}\);
如果\(\B\in M_n(K)\),那么我们可以写出\(\B_n\).

“矩阵”之所以得名,是因为我们可以把矩阵\(\A\)写成\(s\)行\(n\)列的矩形表,再加上括号:\[
	\begin{bmatrix}
		a_{11} & a_{12} & \dots & a_{1n} \\
		a_{21} & a_{22} & \dots & a_{2n} \\
		\vdots & \vdots & & \vdots \\
		a_{s1} & a_{s2} & \dots & a_{sn}
	\end{bmatrix},
\]
其中\(a_{ij}=\A(i,j)\ (i=1,2,\dotsc,s;j=1,2,\dotsc,n)\).
有时候为了强调矩阵的元素,
我们会把\(\A_{s \times n}\)记作\(\A=(a_{ij})_{s \times n}\).

类似地,我们把行向量表示为\[
	\begin{bmatrix}
		a_1 & a_2 & \dots & a_n
	\end{bmatrix};
\]
把列向量表示为\[
	\begin{bmatrix}
		a_1 \\ a_2 \\ \vdots \\ a_n
	\end{bmatrix}.
\]

\subsection{子矩阵}
\begin{definition}
在矩阵\(\A=(a_{ij})_{s \times n}\)中,
任取\(k\)行\(l\)列,
位于这些行与列交叉处的\(kl\)个元素,按原顺序排成的\(k \times l\)矩阵\[
	\begin{vmatrix}
		a_{i_1,j_1} & a_{i_1,j_2} & \dots & a_{i_1,j_l} \\
		a_{i_2,j_1} & a_{i_2,j_2} & \dots & a_{i_2,j_l} \\
		\vdots & \vdots & & \vdots \\
		a_{i_k,j_1} & a_{i_k,j_2} & \dots & a_{i_k,j_l}
	\end{vmatrix},
	\quad
	\begin{array}{c}
		1 \leq i_1 < i_2 < \dotsb < i_k \leq s; \\
		1 \leq j_1 < j_2 < \dotsb < j_l \leq n
	\end{array}
\]称为“\(\A\)的一个\(k\)阶\DefineConcept{子矩阵}(submatrix)”.
%@see: https://mathworld.wolfram.com/Submatrix.html
\end{definition}

\subsection{矩阵的分块}
\begin{definition}
设\(\A\in M_{s \times n}(K)\);
给定介于\(1\)和\(s\)之间的两个正整数\(a,b\),
以及介于\(1\)和\(n\)之间的两个正整数\(c,d\).
令\[
	X=\Set{ x\in\mathbb{Z}^+ \given 1 \leq a \leq x \leq b \leq s },
	\qquad
	Y=\Set{ y\in\mathbb{Z}^+ \given 1 \leq c \leq y \leq d \leq n }.
\]
我们把映射\(\A\)在\(X \times Y\)上的限制\[
	\A\setrestrict(X \times Y)
\]称为“\(\A\)的\DefineConcept{子块}”.
以子块为元素的矩阵称为\DefineConcept{分块矩阵}.
\end{definition}

我们可以把矩阵\(\A\)分解成如下形式:
\[
	\begin{matrix}
		& \begin{matrix} n_1 & n_2 & \dots & n_r \end{matrix} \\
			\begin{matrix} s_1 \\ s_2 \\ \vdots \\ s_t \end{matrix} & \begin{bmatrix}
			\A_{11} & \A_{12} & \dots & \A_{1r} \\
			\A_{21} & \A_{22} & \dots & \A_{2r} \\
			\vdots & \vdots & \ddots & \vdots \\
			\A_{t1} & \A_{t2} & \dots & \A_{tr}
		\end{bmatrix}.
	\end{matrix}
\]

\subsection{矩阵的行向量组、列向量组}
设\(\A\in M_{s \times n}(K)\).
现在我们以\(\A\)为对象,研究两种特殊的分块方式.

如果我们只把第\(i\ (i=1,2,\dotsc,s)\)行上的元素取出来,
构成一个行向量\[
	\a=\begin{bmatrix}
		a_{i1} & a_{i2} & \dots & a_{in}
	\end{bmatrix},
\]
或者说,我们取\[
	\a=\A\setrestrict(\{i\}\times\{1,\dotsc,n\}),
\]
那么称“\(\a\)是\(\A\)的第\(i\) \DefineConcept{行向量}”,
记作\(\MatrixEntry\A{i,*}\).

如果我们只把第\(j\ (j=1,2,\dotsc,n)\)列上的元素取出来,
构成一个列向量\[
	\b=\begin{bmatrix}
		a_{1j} \\ a_{2j} \\ \vdots \\ a_{sj}
	\end{bmatrix},
\]
或者说,我们取\[
	\b=\A\setrestrict(\{1,\dotsc,s\}\times\{j\}),
\]
那么称“\(\b\)是\(\A\)的第\(j\) \DefineConcept{列向量}”,
记作\(\MatrixEntry\A{*,j}\).

把\(\A\)的全部行向量\[
	\Set{ \a \given \a=\A\setrestrict(\{i\}\times\{1,\dotsc,n\}), i=1,2,\dotsc,s }
\]或者说\[
	\Set{\MatrixEntry\A{1,*},\MatrixEntry\A{2,*},\dotsc,\MatrixEntry\A{s,*}}
\]
称为“\(\A\)的\DefineConcept{行向量组}”.

把\(\A\)的全部列向量\[
	\Set{ \b \given \b=\A\setrestrict(\{1,\dotsc,s\}\times\{j\}), j=1,2,\dotsc,n }
\]或者说\[
	\Set{\MatrixEntry\A{*,1},\MatrixEntry\A{*,2},\dotsc,\MatrixEntry\A{*,n}}
\]
称为“\(\A\)的\DefineConcept{列向量组}”.

\subsection{矩阵的转置}
\begin{definition}
设\(\A\in M_{s \times n}(K),
\B\in M_{n \times s}(K)\).
如果\[
	\B(i,j)=\A(j,i),
	\quad i=1,2,\dotsc,s;j=1,2,\dotsc,n,
\]
那么把\(\B\)称为“\(\A\)的\DefineConcept{转置矩阵}(transpose)”,
记作\(\A^T\).
\end{definition}
\begin{remark}
矩阵的转置运算可以看作一个从\(M_{s \times n}(K)\)到\(M_{n \times s}(K)\)上的映射.
\end{remark}

\begin{property}\label{theorem:矩阵的转置.性质1}
设\(\A \in M_n(K)\),
则\begin{equation}
	(\A^T)^T = \A.
\end{equation}
\end{property}

\begin{property}\label{theorem:矩阵的转置.性质2}
设\(\A,\B \in M_n(K)\),
则\begin{equation}
	(\A+\B)^T = \A^T + \B^T.
\end{equation}
\end{property}

\begin{property}\label{theorem:矩阵的转置.性质3}
设\(\A \in M_n(K)\),\(k \in K\),
则\begin{equation}
	(k \A)^T = k \A^T.
\end{equation}
\end{property}

\begin{definition}
设矩阵\(\A \in M_{s \times n}(\mathbb{C})\).
把对\(\A\)的各元素取共轭得到的矩阵
称为“矩阵\(\A\)的\DefineConcept{共轭矩阵}(conjugate)”,
%@see: https://mathworld.wolfram.com/ConjugateMatrix.html
记作\(\overline{\A}\).
\end{definition}

\begin{definition}
设矩阵\(\A \in M_{s \times n}(\mathbb{C})\).
将\(\A\)转置后,再对各元素取共轭,
把这样的矩阵
称为“矩阵\(\A\)的\DefineConcept{共轭转置矩阵}(conjugate transpose)”,
%@see: https://mathworld.wolfram.com/ConjugateTranspose.html
记作\(\A^H\),即\[
    \A^H \defeq \overline{\A^T} \equiv \overline{\A}^T.
\]
\end{definition}

\begin{property}
设\(\A \in M_{s \times n}(\mathbb{R})\),
则\(\overline{\A} = \A\).
\end{property}

\begin{property}
设\(\A \in M_{s \times n}(\mathbb{R})\),
则\(\A^H = \A^T\).
\end{property}

\section{向量的线性运算}
\begin{definition}
对于\(n\)维向量\(\a = (\AutoTuple{a}{n})\)和\(\b = (\AutoTuple{b}{n})\).
\begin{enumerate}
	\item {\rm\bf 加法}:
	把向量\((a_1+b_1,a_2+b_2,\dotsc,a_n+b_n)\)
	称为“\(\a\)与\(\b\)的\DefineConcept{和}”,记作\[
		\a+\b=(a_1+b_1,a_2+b_2,\dotsc,a_n+b_n)
	\]
	\item {\rm\bf 数量乘法}:
	设\(k\)为数,把向量\((k a_1, k a_2, \dotsc, k a_n)\)
	称为“\(k\)与\(\a\)的\DefineConcept{数乘}”,记作\[
		k\a = (k a_1, k a_2, \dotsc, k a_n)
	\]
	\item 分量全为零的向量\((0,0,\dotsc,0)\)称为\DefineConcept{零向量},记作\(\z\).
	\item 称\((-a_1,-a_2,\dotsc,-a_n)\)为\(\a\)的\DefineConcept{负向量},记作\(-\a\).
\end{enumerate}

向量的加法、数乘统称为向量的\DefineConcept{线性运算}.
\end{definition}

\begin{theorem}
由上述定义可知,向量的线性运算满足下面八条运算规律:
\begin{enumerate}
	\item \(\a + \b = \b + \a\)
	\item \((\a + \b) + \g = \a + (\b + \g)\)
	\item \(\a + \z = \a\)
	\item \(\a + (-\a) = \z\)
	\item \(1 \a = \a\)
	\item \(k(l \a) = (kl) \a\)
	\item \(k(\a + \b) = k\a + k\b\)
	\item \((k+l)\a = k\a + l\a\)
\end{enumerate}
\end{theorem}

\begin{property}
向量的运算还满足以下性质:
\begin{enumerate}
	\item \(0 \a = \z\)
	\item \((-1) \a = -\a\)
	\item \(k \z = \z\)
	\item \(k \a = \z \implies k = 0 \lor \a = \z\)
\end{enumerate}
\end{property}

\begin{definition}
对于\(n\)维向量\(\a = (\AutoTuple{a}{n})\)和\(\b = (\AutoTuple{b}{n})\).
称向量\(\a\)与向量\(\b\)的负向量\(-\b\)的和为向量\(\a\)与向量\(\b\)的\DefineConcept{差},即\[
	\a - \b = \a + (-\b).
\]
\end{definition}

\section{矩阵的线性运算}
\begin{definition}
设\(\A,\B\in M_{s\times n}(K)\),
\(\A=(a_{ij})_{s \times n}\),
\(\B=(b_{ij})_{s \times n}\).
\begin{enumerate}
	\item 称矩阵\[
		(a_{ij} + b_{ij})_{s \times n} = \begin{bmatrix}
			a_{11}+b_{11} & a_{12}+b_{12} & \dots & a_{1n}+b_{1n} \\
			a_{21}+b_{21} & a_{22}+b_{22} & \dots & a_{2n}+b_{2n} \\
			\vdots & \vdots & & \vdots \\
			a_{s1}+b_{s1} & a_{s2}+b_{s2} & \dots & a_{sn}+b_{sn}
		\end{bmatrix}
	\]为“\(\A\)与\(\B\)的\DefineConcept{和}(sum)”,
	记作\(\A+\B\).

	\item 任取\(k\in K\),称矩阵\[
		(ka_{ij})_{s \times n} = \begin{bmatrix}
			ka_{11} & ka_{12} & \dots & ka_{1n} \\
			ka_{21} & ka_{22} & \dots & ka_{2n} \\
			\vdots & \vdots & & \vdots \\
			ka_{s1} & ka_{s2} & \dots & ka_{sn}
		\end{bmatrix}
	\]为“\(k\)与矩阵\(\A\)的\DefineConcept{数乘}”,
	记作\(k\A\).

	\item 称元素全为零的矩阵为\DefineConcept{零矩阵}(zero matrix),记作\(\z\).

	\item 称矩阵\[
		(-a_{ij})_{s \times n}=\begin{bmatrix}
			-a_{11} & -a_{12} & \dots & -a_{1n} \\
			-a_{21} & -a_{22} & \dots & -a_{2n} \\
			\vdots & \vdots & & \vdots \\
			-a_{s1} & -a_{s2} & \dots & -a_{sn}
		\end{bmatrix}
	\]为\(\A\)的\DefineConcept{负矩阵},记作\(-\A\).
\end{enumerate}
\end{definition}

定义\DefineConcept{非零矩阵}:\[
	M_{s \times n}^*(K) \defeq M_{s \times n}(K)-\{\z\},
	\qquad
	M_n^*(K) \defeq M_n(K)-\{\z\}.
\]

\begin{theorem}
矩阵的线性运算满足以下运算规律:
\begin{gather}
	(\forall\A,\B\in M_{s\times n}(K))[\A + \B = \B + \A], \\
	(\forall\A,\B,\C\in M_{s\times n}(K))[(\A + \B) + \C = \A + (\B + \C)], \\
	(\forall\A\in M_{s\times n}(K))[\A + \z = \A], \\
	(\forall\A\in M_{s\times n}(K))[\A + (-\A) = \z], \\
	(\forall\A\in M_{s\times n}(K))[1 \A = \A], \\
	(\forall\A\in M_{s\times n}(K))(\forall k,l\in K)[k(l \A) = (kl) \A], \\
	(\forall\A,\B\in M_{s\times n}(K))(\forall k\in K)[k(\A + \B) = k\A + k\B], \\
	(\forall\A\in M_{s\times n}(K))(\forall k,l\in K)[(k+l)\A = k\A + l\A], \\
	(\forall\A\in M_{s\times n}(K))[0\A = \z], \\
	(\forall\A\in M_{s\times n}(K))[(-1)\A = -\A], \\
	(\forall k\in K)[k\z = \z], \\
	k\A = \z \implies k = 0 \lor \A = \z.
\end{gather}
\end{theorem}

\section{矩阵的乘法}
\subsection{矩阵乘法的概念}
\begin{definition}
设\(\A = (a_{ij})_{s \times n}\),
\(\B = (b_{ij})_{n \times m}\),
\(\C = (c_{ij})_{s \times m}\).
如果满足\[
	c_{ij} = \sum_{k=1}^n {a_{ik} b_{kj}},
	\quad
	i=1,2,\dotsc,s;j=1,2,\dotsc,m,
\]
则称矩阵\(\C\)是“\(\A\)与\(\B\)的\DefineConcept{乘积}”,
记作\(\C = \A \B\).
\end{definition}
\begin{remark}
如果我们分别对\(\A\)和\(\B\)做行分块和列分块,得\begin{equation*}
	\A=(\AutoTuple{\a}{s}[,][T])^T, \qquad
	\B=(\AutoTuple{\b}{m}),
\end{equation*}
那么\begin{itemize}
	%@see: 《Linear Algebra Done Right (Fourth Eidition)》(Sheldon Axler) P75 3.46
	\item \(\A\)与\(\B\)的乘积\(\A\B\)的\((i,j)\)元素,
	等于\(\A\)的第\(i\)行向量\(\a_i^T\)与\(\B\)的第\(j\)列向量\(\b_j\)的乘积,
	即\begin{equation*}
		c_{ij} = \a_i^T \b_j,
		\quad
		i=1,2,\dotsc,s;j=1,2,\dotsc,m;
	\end{equation*}

	%@see: 《Linear Algebra Done Right (Fourth Eidition)》(Sheldon Axler) P75 3.48
	\item \(\A\B\)的第\(k\)列向量,
	等于\(\A\)与\(\B\)的第\(k\)列向量\(\b_k\)的乘积,
	即\begin{equation*}
		\MatrixEntry{(\A\B)}{*,k}
		= \A \MatrixEntry{\B}{*,k},
		\quad k=1,2,\dotsc,m.
	\end{equation*}

	%@see: 《Linear Algebra Done Right (Fourth Eidition)》(Sheldon Axler) P76 3.50
\end{itemize}
\end{remark}
\begin{remark}
如果我们分别对\(\A\)和\(\B\)做列分块和行分块,
\begingroup%
\def\mx{\vb{\xi}}%
\def\mz{\vb{\zeta}}%
得\begin{equation*}
	\A=(\AutoTuple{\mx}{n}), \qquad
	\B=(\AutoTuple{\mz}{n}[,][T])^T,
\end{equation*}
那么有\begin{equation}
	\mx_i \mz_i^T
	= \begin{bmatrix}
		a_{1i} b_{i1} & a_{1i} b_{i2} & \dots & a_{1i} b_{im} \\
		a_{2i} b_{i1} & a_{2i} b_{i2} & \dots & a_{2i} b_{im} \\
		\vdots & \vdots & & \vdots \\
		a_{si} b_{i1} & a_{si} b_{i2} & \dots & a_{si} b_{im} \\
	\end{bmatrix},
	\quad
	i=1,2,\dotsc,n,
\end{equation}
于是\begin{equation*}
	\A\B=\sum_{i=1}^n \mx_i \mz_i^T.
\end{equation*}
\endgroup%
\end{remark}

\begin{proposition}
矩阵乘法不满足交换律.
\end{proposition}

\begin{definition}
设矩阵\(\A,\B \in M_n(K)\).
如果\[
	\A \B = \B \A,
\]
则称“\(\A\)与\(\B\)~\DefineConcept{可交换}”
或“\(\A\)与\(\B\)的乘积服从交换律”.
\end{definition}

\begin{example}
矩阵\[
	\begin{bmatrix}
		1 & 0 & 0 \\
		0 & -1 & 0 \\
		0 & 0 & -1
	\end{bmatrix}
	\quad\text{与}\quad
	\begin{bmatrix}
		1 & 0 & 0 \\
		0 & 0 & 1 \\
		0 & 1 & 0
	\end{bmatrix}
\]可交换,这是因为\[
	\begin{bmatrix}
		1 & 0 & 0 \\
		0 & -1 & 0 \\
		0 & 0 & -1
	\end{bmatrix}
	\begin{bmatrix}
		1 & 0 & 0 \\
		0 & 0 & 1 \\
		0 & 1 & 0
	\end{bmatrix}
	= \begin{bmatrix}
		1 & 0 & 0 \\
		0 & 0 & -1 \\
		0 & -1 & 0
	\end{bmatrix}
	= \begin{bmatrix}
		1 & 0 & 0 \\
		0 & 0 & 1 \\
		0 & 1 & 0
	\end{bmatrix}
	\begin{bmatrix}
		1 & 0 & 0 \\
		0 & -1 & 0 \\
		0 & 0 & -1
	\end{bmatrix}.
\]
\end{example}

\begin{example}
举例说明:非零矩阵的乘积可能是零矩阵.
\begin{solution}
矩阵\[
	\A = \begin{bmatrix}
		0 & 0 & 0 \\
		a_{21} & 0 & 0 \\
		a_{31} & a_{32} & 0 \\
	\end{bmatrix}
	\quad\text{和}\quad
	\B = \begin{bmatrix}
		b_{11} & b_{12} & b_{13} \\
		0 & b_{22} & b_{23} \\
		0 & 0 & b_{33} \\
	\end{bmatrix}
\]可以都不是零矩阵,
但他们的乘积\(\A\B\)一定是零矩阵.
\end{solution}
\end{example}

\begin{example}
举例说明:矩阵乘法不满足消去律,即\[
	\A \B = \A \C
	\notimplies
	\A = \z \lor \B = \C.
\]
\begin{solution}
取\[
	\A = \begin{bmatrix}
		1 & 0 \\
		1 & 0
	\end{bmatrix},
	\qquad
	\B = \begin{bmatrix}
		0 & 0 \\
		0 & 1
	\end{bmatrix},
	\qquad
	\C = \begin{bmatrix}
		0 & 0 \\
		0 & 2
	\end{bmatrix},
\]
显然\[
	\A\B
	= \A\C
	= \begin{bmatrix}
		0 & 0 \\
		0 & 0
	\end{bmatrix},
\]
但是\(\A\neq\vb0\)且\(\B\neq\C\).
\end{solution}
\end{example}

\subsection{矩阵乘法的运算规则}
\begin{theorem}
矩阵乘法满足结合律.
\begin{proof}
设\(\A = (a_{ij})_{s \times n},
\B = (b_{ij})_{n \times m},
\C = (c_{ij})_{m \times r}\).
显然\((\A\B)\C\)与\(\A(\B\C)\)同型,都是\(s \times r\)矩阵.
由于\begin{align*}
	\MatrixEntry{((\A\B)\C)}{i,j}
	&= \sum_{l=1}^m (\MatrixEntry{(\A\B)}{i,l}) \cdot c_{lj} \\
	&= \sum_{l=1}^m \left( \sum_{k=1}^n a_{ik} b_{kl} \right) c_{lj} \\
	&= \sum_{l=1}^m \left( \sum_{k=1}^n a_{ik} b_{kl} c_{lj} \right), \\
	\MatrixEntry{(\A(\B\C))}{i,j}
	&= \sum_{k=1}^n a_{ik} \cdot (\MatrixEntry{(\B\C)}{k,j}) \\
	&= \sum_{k=1}^n a_{ik} \left( \sum_{l=1}^m b_{kl} c_{lj} \right) \\
	&= \sum_{k=1}^n \left( \sum_{l=1}^m a_{ik} b_{kl} c_{lj} \right) \\
	&= \sum_{l=1}^m \left( \sum_{k=1}^n a_{ik} b_{kl} c_{lj} \right),
\end{align*}
于是\((\A\B)\C = \A(\B\C)\).
\end{proof}
\end{theorem}

\begin{definition}
设\(\A\in M_n(K)\).
若有\[
	\A(i,j) = \left\{ \begin{array}{cl}
		1, & i=j, \\
		0, & i\neq j,
	\end{array} \right.
\]
则称“\(\A\)是\DefineConcept{单位矩阵}(identity matrix)”,记作\(\E\).
%@see: https://mathworld.wolfram.com/IdentityMatrix.html
\end{definition}

\begin{property}
矩阵的乘法满足以下性质:
\begin{gather}
	(\forall\A,\B\in M_{s\times n}(K))(\forall k\in K)[k(\A\B) = (k\A)\B = \A(k\B)], \\
	(\forall\A,\B,\C\in M_{s\times n}(K))[\A(\B+\C) = \A\B + \A\C], \label{equation:矩阵的乘法.左分配律} \\
	(\forall\A,\B,\C\in M_{s\times n}(K))[(\A+\B)\C = \A\C + \B\C], \label{equation:矩阵的乘法.右分配律} \\
	(\forall\A\in M_{s\times n}(K))[\z_{q \times s} \A = \z_{q \times n}], \\
	(\forall\A\in M_{s\times n}(K))[\A \z_{n \times p} = \z_{s \times p}], \\
	(\forall\A\in M_{s\times n}(K))[\E_s \A = \A], \\
	(\forall\A\in M_{s\times n}(K))[\A \E_n = \A].
\end{gather}
\end{property}

\subsection{矩阵的幂}
\begin{definition}
设\(\A\in M_n(K)\).
定义:
\begin{align}
	\A^0 &\defeq \E, \\
	\A^k &\defeq \underbrace{\A\A\dotsm\A}_{\text{$k$个}}.
\end{align}
%@see: https://mathworld.wolfram.com/MatrixPower.html
\end{definition}

\begin{theorem}
指数律成立,
即\begin{gather}
	(\forall\A \in M_n(K))
	(k,l \in \mathbb{N})
	[\A^k\A^l = \A^{k+l}], \\
	(\forall\A \in M_n(K))
	(k,l \in \mathbb{N})
	[(\A^k)^l = \A^{kl}].
\end{gather}
\end{theorem}

\begin{proposition}
设\(\A,\B \in M_n(K)\),
则\begin{equation}
	(\A\B)^k = \A(\B\A)^{k-1}\B,
	\quad k=1,2,\dotsc.
\end{equation}
\end{proposition}
\begin{proposition}
设\(\A,\B \in M_n(K)\).
若\(\A\)与\(\B\)可交换,则\begin{equation*}
	(\A\B)^k = \A^k\B^k.
\end{equation*}
\end{proposition}
\begin{remark}
注意,当\(\A\)、\(\B\)不可交换时,通常有\[
	(\A\B)^k \neq \A^k\B^k.
\]
\end{remark}

\begin{example}
设\(\A=\diag(\AutoTuple{a}{n}),
\B=\diag(\AutoTuple{b}{n})\).
那么\[
	\A\B = \diag(a_1b_1,a_2b_2,\dotsc,a_nb_n).
\]
\end{example}
\begin{remark}
如果\(\A\)是一个对角阵,其主对角线上的元素各不相同,
则与\(\A\)可交换的矩阵必定也是一个对角阵.
\end{remark}

\begin{example}
设二阶矩阵\(\A=\begin{bmatrix} 1 & \lambda \\ 0 & 1 \end{bmatrix}\).
试证\(\A^n=\begin{bmatrix} 1 & n\lambda \\ 0 & 1 \end{bmatrix}\).
\begin{proof}
用数学归纳法.
显然\(n=1\)时,命题成立.
接下来我们再验证\(n=2\)时,命题是否成立.
因为\[
	\A^2
	= \begin{bmatrix}
		1 & \lambda \\
		0 & 1
	\end{bmatrix}^2
	= \begin{bmatrix}
		1\cdot0+\lambda\cdot0 & 1\cdot\lambda+\lambda\cdot1 \\
		0\cdot1+1\cdot0 & 0\cdot\lambda+1\cdot1
	\end{bmatrix}
	= \begin{bmatrix}
		1 & 2\lambda \\
		0 & 1
	\end{bmatrix},
\]
于是\(n=2\)时命题成立.

假设\(n=k\ (k\geq1)\)时命题成立,
那么,当\(n=k+1\)时,有\[
	A^{k+1}
	= A A^k
	= \begin{bmatrix}
		1 & \lambda \\
		0 & 1
	\end{bmatrix}
	\begin{bmatrix}
		1 & k\lambda \\
		0 & 1
	\end{bmatrix}
	= \begin{bmatrix}
		1\cdot1+\lambda\cdot0 & 1\cdot k\lambda+\lambda\cdot1 \\
		0\cdot1+1\cdot0 & 0\cdot k\lambda+1\cdot1
	\end{bmatrix}
	= \begin{bmatrix}
		1 & (k+1)\lambda \\
		0 & 1
	\end{bmatrix}.
\]
因此,命题\(\A^n=\begin{bmatrix} 1 & n\lambda \\ 0 & 1 \end{bmatrix}\)
当\(n=1,2,\dotsc\)时总成立.
\end{proof}
\end{example}

\begin{example}
设\(\A,\B,\C \in M_n(K)\),则有\[
	(\A + \B + \C)^2
	= \A^2 + \B^2 + \C^2 + \A\B + \B\A + \A\C + \C\A + \B\C + \C\B.
\]
\end{example}

\begin{theorem}
设矩阵\(\A \in M_n(K)\),
则\begin{itemize}
	\item 对于任意非负整数\(k\),\(\A\)与\(\A^k\)可交换;
	\item 对于数域\(K\)上的任意一个一元多项式\(f(x)\),\(\A\)与\(f(\A)\)可交换.
	\item 对于数域\(K\)上的任意两个一元多项式\(f(x)\)和\(g(x)\),\(f(\A)\)与\(g(\A)\)可交换.
\end{itemize}
\end{theorem}

\begin{theorem}
如果\(g(x)\)和\(h(x)\)是两个多项式,
设\(l(x) = g(x) + h(x)\),\(m(x) = g(x) h(x)\),
则\[
	l(\A) = g(\A) + h(\A),
	\quad
	m(\A) = g(\A) h(\A).
\]
\end{theorem}

\begin{example}
设\[
	\A = \begin{bmatrix}
	\cos t & \sin t \\
	-\sin t & \cos t
	\end{bmatrix}.
\]
令\[
	\B = \begin{bmatrix}
		\cos t & 0 \\
		0 & \cos t
	\end{bmatrix},
	\qquad
	\C = \begin{bmatrix}
		0 & \sin t \\
		-\sin t & 0
	\end{bmatrix},
\]
则\(\A=\B+\C\).
因为\[
	\B\C = \begin{bmatrix}
		\cos t & 0 \\
		0 & \cos t
	\end{bmatrix}
	\begin{bmatrix}
		0 & \sin t \\
		-\sin t & 0
	\end{bmatrix}
	= \begin{bmatrix}
		0 & \cos t \sin t \\
		-\cos t \sin t & 0
	\end{bmatrix},
\]\[
	\C\B = \begin{bmatrix}
		0 & \sin t \\
		-\sin t & 0
	\end{bmatrix}
	\begin{bmatrix}
		\cos t & 0 \\
		0 & \cos t
	\end{bmatrix}
	= \begin{bmatrix}
		0 & \cos t \sin t \\
		-\cos t \sin t & 0
	\end{bmatrix},
\]
所以\(\B\C=\C\B\),\(\B\)与\(\C\)可交换.
由牛顿二项式定理有,\[
	\A^n=(\B+\C)^n
	=\sum_{k=0}^n C_n^k \B^{n-k} \C^k.
\]
\end{example}

\begin{example}
设\(\A,\B,\x \in M_n(K)\).
证明:若\(\A\x=\x\B\),则对任意多项式\[
	f(x) = a_0 + a_1 x + a_2 x^2 + \dotsb + a_k x^k,
	\quad
	a_0,a_1,a_2,\dotsc,a_k \in K,
\]
总有\[
	f(\A) \x = \x f(\B).
\]
\begin{proof}
因为\[
	\A\x = \x\B,
\]
所以\[
	\A^2 \x = \A(\A\x) = \A(\x\B),
	\qquad
	\x \B^2 = (\x\B)\B = (\A\x)\B.
\]
以此类推,可证\[
	\A^n \x = \x \B^n,
	\quad n=1,2,\dotsc.
\]

因为\[
	f(\A) = a_0 \E + a_1 \A + a_2 \A^2 + \dotsb + a_k \A^k,
\]
根据左分配律,有\[
	f(\A) \x = a_0 \x + a_1 \A\x + a_2 \A^2 \x + \dotsb + a_k \A^k \x.
	\eqno(1)
\]
同理,根据右分配律,有\[
	\x f(\B) = a_0 \x + a_1 \x\B + a_2 \x \B^2 + \dotsb + a_k \x \B^k.
	\eqno(2)
\]
因为(1)与(2)中各项逐项相等,
故\(f(\A) \x = \x f(\B)\).
\end{proof}
\end{example}

\subsection{矩阵乘积的转置}
\begin{theorem}\label{theorem:矩阵.矩阵乘积的转置}
设\(\A\in M_{s\times n}(K),
\B\in M_{n \times t}(K)\),
则有\[
	(\A\B)^T = \B^T \A^T.
\]
\begin{proof}
假设\[
	\A=(a_{ij})_{s \times n}
	=(\AutoTuple{\a}{s})^T, \qquad
	\B=(b_{ij})_{n \times t}
	=(\AutoTuple{\b}{t}),
\]
其中\(\a_i\in K^n\ (i=1,2,\dotsc,s)\)是行向量,
\(\b_j\in K^n\ (j=1,2,\dotsc,t)\)是列向量.
又假设\[
	\A\B=(c_{ij})_{s \times t}, \qquad
	\B^T\A^T=(d_{ij})_{t \times s}.
\]
那么\begin{gather*}
	c_{ij}  % \(\A\)的第\(i\)行,\(\B\)的第\(j\)列
	= \VectorInnerProductDot{\vb\alpha_i}{\vb\beta_j}
	= \sum_{k=1}^n a_{ik}b_{kj}, \\
	d_{ij}  % \(\B^T\)的第\(i\)行,\(\A^T\)的第\(j\)列
	= \VectorInnerProductDot{\vb\beta_i}{\vb\alpha_j}  % 相当于\(\B\)的第\(i\)列,\(\A\)的第\(j\)行
	= \sum_{k=1}^n a_{jk}b_{ki},
\end{gather*}
可见\(c_{ij}=d_{ji}\ (i=1,2,\dotsc,s;j=1,2,\dotsc,t)\).
因此,\((\A\B)^T = \B^T \A^T\).
\end{proof}
\end{theorem}

\begin{corollary}
\((\A_1 \A_2 \dotsb \A_n)^T = \A_n^T \dots \A_2^T \A_1^T\).
\end{corollary}

\section{特殊矩阵}
\subsection{基本矩阵}
\begin{definition}
%@see: 《高等代数(第三版 上册)》(丘维声) P117 定义3
只有一个元素是\(1\),其余元素全为\(0\)的矩阵,称为\DefineConcept{基本矩阵}.
\((i,j)\)元素为\(1\)的基本矩阵,记作\(\vb{E}_{ij}\).
\end{definition}
\begin{example}
%@see: 《高等代数(第三版 上册)》(丘维声) P117
对于\(2\times3\)矩阵\(\vb{A}\),有\begin{align*}
	\vb{A}
	&= \begin{bmatrix}
		a_{11} & a_{12} & a_{13} \\
		a_{21} & a_{22} & a_{23}
	\end{bmatrix} \\
	&= a_{11}
	\begin{bmatrix}
		1 & 0 & 0 \\
		0 & 0 & 0
	\end{bmatrix}
	+ a_{12}
	\begin{bmatrix}
		0 & 1 & 0 \\
		0 & 0 & 0
	\end{bmatrix}
	+ a_{13}
	\begin{bmatrix}
		0 & 0 & 1 \\
		0 & 0 & 0
	\end{bmatrix} \\
	&\hspace{20pt}
	+ a_{21}
	\begin{bmatrix}
		0 & 0 & 0 \\
		1 & 0 & 0
	\end{bmatrix}
	+ a_{22}
	\begin{bmatrix}
		0 & 0 & 0 \\
		0 & 1 & 0
	\end{bmatrix}
	+ a_{23}
	\begin{bmatrix}
		0 & 0 & 0 \\
		0 & 0 & 1
	\end{bmatrix} \\
	&=
	a_{11} \vb{E}_{11} + a_{12} \vb{E}_{12} + a_{13} \vb{E}_{13}
	+ a_{21} \vb{E}_{21} + a_{22} \vb{E}_{22} + a_{23} \vb{E}_{23}.
\end{align*}
\end{example}

一般地,
对于任意一个\(s \times n\)矩阵\(\vb{A}\),
有\begin{equation*}
	\vb{A} = \sum_{i=1}^s \sum_{j=1}^n a_{ij} \vb{E}_{ij},
\end{equation*}
其中\(a_{ij}\)是\(\vb{A}\)的\((i,j)\)元素.

\subsection{三角矩阵}
\begin{definition}
设矩阵\(\vb{A}=(a_{ij})_n \in M_n(K)\).
\begin{enumerate}
	\item 如果\(\vb{A}\)左下角的元素全为零,即\begin{equation*}
		a_{ij} = 0
		\quad(i>j),
	\end{equation*}
	则称之为\DefineConcept{上三角形矩阵}(upper triangular matrix),
	%@see: https://mathworld.wolfram.com/UpperTriangularMatrix.html
	简称\DefineConcept{上三角阵}.

	\item 如果\(\vb{A}\)右上角的元素全为零,即\begin{equation*}
		a_{ij} = 0
		\quad(i<j),
	\end{equation*}
	则称之为\DefineConcept{下三角形矩阵}(lower triangular matrix),
	%@see: https://mathworld.wolfram.com/LowerTriangularMatrix.html
	简称\DefineConcept{下三角阵}.
\end{enumerate}
%@see: https://mathworld.wolfram.com/TriangularMatrix.html
\end{definition}

\begin{example}
设\(\vb{A},\vb{B}\)都是\(n\)阶上三角阵,证明:\(\vb{A}\vb{B}\)是上三角阵.
\begin{proof}
利用数学归纳法.
当\(n=1\)时,\(\vb{A}=a\),\(\vb{B}=b\),\(\vb{A}\vb{B} = ab\),结论成立.

假设\(n=k\)时上三角阵的乘积是上三角阵.
当\(n=k+1\)时,对矩阵\(\vb{A}\)和\(b\)分块如下:\begin{equation*}
	\vb{A} = \begin{bmatrix}
		a_{11} & \vb{A}_2 \\
		\vb0 & \vb{A}_4
	\end{bmatrix},
	\qquad
	\vb{B} = \begin{bmatrix}
		b_{11} & \vb{B}_2 \\
		\vb0 & \vb{B}_4
	\end{bmatrix},
\end{equation*}
其中\(\vb{A}_4\)和\(\vb{B}_4\)都是\(k\)阶上三角阵,
由归纳假设,\(\vb{A}_4 \vb{B}_4\)是\(k\)阶上三角阵,
则\begin{equation*}
	\vb{A}\vb{B} = \begin{bmatrix}
		a_{11} b_{11} & a_{11} \vb{B}_2 + \vb{A}_2 \vb{B}_4 \\
		\vb0 & \vb{A}_4 \vb{B}_4
	\end{bmatrix},
\end{equation*}
即\(\vb{A}\vb{B}\)是\(k+1\)阶上三角阵.
\end{proof}
\end{example}

\subsection{对角矩阵}
\begin{definition}
如果矩阵\(\vb{A}=(a_{ij})_n \in M_n(K)\)除对角线以外的元素全为零,即\begin{equation*}
	a_{ij} = 0
	\quad(i \neq j),
\end{equation*}
那么称“\(\vb{A}\)是一个\(n\)阶\DefineConcept{对角矩阵}(diagonal matrix)”,
%@see: https://mathworld.wolfram.com/DiagonalMatrix.html
记作\(\diag(a_{11},a_{22},\dotsc,a_{nn})\).
\end{definition}

\begin{definition}
如果矩阵\(\vb{A}=(a_{ij})_{m \times n} \in M_{m \times n}(K)\)
除对角线以外的元素全为零,即\begin{equation*}
	a_{ij} = 0
	\quad(i \neq j),
\end{equation*}
那么称“\(\vb{A}\)是一个\(m \times n\) \DefineConcept{对角矩阵}”,
记作\(\diag_{m \times n}(a_{11},a_{22},\dotsc,a_{pp})\),
其中\(p = \min\{m,n\}\).
\end{definition}

\subsection{对称矩阵,反对称矩阵}
\begin{definition}
%@see: 《Linear Algebra Done Right (Fourth Eidition)》(Sheldon Axler) P337 9.11
若矩阵\(\vb{A} \in M_n(K)\)满足\begin{equation*}
    \vb{A}^T = \vb{A},
\end{equation*}
那么把\(\vb{A}\)称为\DefineConcept{对称矩阵}(symmetric matrix).
%@see: https://mathworld.wolfram.com/SymmetricMatrix.html
反之,如果\begin{equation*}
	\vb{A}^T \neq \vb{A},
\end{equation*}
那么把\(\vb{A}\)称为\DefineConcept{非对称矩阵}(asymmetric matrix).
%@see: https://mathworld.wolfram.com/AsymmetricMatrix.html
\end{definition}

\begin{definition}
如果矩阵\(\vb{A} \in M_n(K)\)满足\begin{equation*}
	\vb{A}^T = -\vb{A},
\end{equation*}
那么把\(\vb{A}\)称为\DefineConcept{反对称矩阵}(antisymmetric matrix)
%@see: https://mathworld.wolfram.com/AntisymmetricMatrix.html
或\DefineConcept{斜对称矩阵}.
\end{definition}

\begin{definition}
如果矩阵\(\vb{A} \in M_n(K)\)满足\begin{equation*}
    \vb{A}^H = \vb{A},
\end{equation*}
那么把\(\vb{A}\)称为\DefineConcept{厄米矩阵}(Hermitian matrix).
%@see: https://mathworld.wolfram.com/HermitianMatrix.html
\end{definition}

\begin{definition}
如果矩阵\(\vb{A} \in M_n(K)\)满足\begin{equation*}
	\vb{A}^H = -\vb{A},
\end{equation*}
那么把\(\vb{A}\)称为\DefineConcept{反厄米矩阵}(antihermitian matrix).
%@see: https://mathworld.wolfram.com/AntihermitianMatrix.html
\end{definition}

%\cref{theorem:对称矩阵.对称矩阵的合同类}
%\cref{theorem:反对称矩阵.反对称矩阵的合同类}

\begin{example}
设矩阵\(\vb{A} \in M_n(K)\).
试证:\(\vb{A}\vb{A}^T\)为对称矩阵.
\begin{proof}
因为\((\vb{A} \vb{A}^T)^T = (\vb{A}^T)^T \vb{A}^T = \vb{A} \vb{A}^T\),所以\(\vb{A} \vb{A}^T\)是对称矩阵.
\end{proof}
\end{example}
\begin{remark}
容易看出,\(\vb{A}^T\vb{A}\)也是对称矩阵.
\end{remark}

\begin{example}
设\(\vb{A}\)和\(\vb{B}\)是同阶对称矩阵.
试证:\(\vb{A}\vb{B}\)是对称矩阵的充分必要条件是\(\vb{A}\vb{B} = \vb{B}\vb{A}\).
\begin{proof}
因为\(\vb{A}\)和\(\vb{B}\)都是对称矩阵,所以\begin{equation*}
	\vb{A}^T = \vb{A},
	\qquad
	\vb{B}^T = \vb{B}.
\end{equation*}
在此条件下,有\begin{equation*}
	\text{$\vb{A}\vb{B}$是对称矩阵}
	\iff
	(\vb{A}\vb{B})^T
	= \vb{A}\vb{B}
	\iff
	\vb{B}^T\vb{A}^T
	= \vb{A}\vb{B}
	\iff
	\vb{B}\vb{A}
	= \vb{A}\vb{B}.
	\qedhere
\end{equation*}
\end{proof}
\end{example}
\begin{remark}
可以证明:同阶反对称矩阵的乘积是对称矩阵的充分必要条件也是\(\vb{A}\vb{B} = \vb{B}\vb{A}\).
类似地,我们还可以证明:同阶对称矩阵或同阶反对称矩阵的乘积是反对称矩阵的充分必要条件是
\(\vb{A}\vb{B} + \vb{B}\vb{A} = \vb0\).
\end{remark}

\begin{property}
反对称矩阵主对角线上的元素全为零.
\end{property}

\begin{example}
零矩阵\(\vb0\)是唯一一个既是实对称矩阵又是实反对称矩阵的矩阵.
\begin{proof}
\(\vb{A}^T = \vb{A} = -\vb{A} \implies 2\vb{A} = \vb{A}+\vb{A} = \vb0 \implies \vb{A} = \vb0\).
\end{proof}
\end{example}

\begin{example}
设\(\vb{A}\)是一个方阵,证明:\(\vb{A}+\vb{A}^T\)为对称矩阵,\(\vb{A}-\vb{A}^T\)为反对称矩阵.
\begin{proof}
因为\((\vb{A}+\vb{A}^T)^T = \vb{A}^T+\vb{A}\),
而\((\vb{A}-\vb{A}^T)^T = \vb{A}^T - \vb{A} = -(\vb{A}-\vb{A}^T)\),
所以\(\vb{A}+\vb{A}^T\)为对称矩阵,
\(\vb{A}-\vb{A}^T\)为反对称矩阵.
显然有\(\vb{A} = \frac{\vb{A} + \vb{A}^T}{2} + \frac{\vb{A} - \vb{A}^T}{2}\).
\end{proof}
\end{example}

\begin{example}
设\(\vb{A}\)是3阶实对称矩阵,\(\vb{B}\)是3阶实反对称矩阵,\(\vb{A}^2 = \vb{B}^2\).
试证:\(\vb{A} = \vb{B} = \vb0\).
\begin{proof}
设\(\vb{A} = (a_{ij})_n\),\(\vb{B} = (b_{ij})_n\).
因为\(\vb{A} = \vb{A}^T\),\(\vb{A}^2 = \vb{A}^T \vb{A}\),
所以\(\vb{A}^2\)的\(\opair{i,j}\)元素为
\(a_{1i} a_{1j} + a_{2i} a_{2j} + \dotsb + a_{ni} a_{nj}\).
因为\(\vb{B} = -\vb{B}^T\),\(\vb{B}^2 = -\vb{B}^T \vb{B}\),
所以\(\vb{B}^2\)的\(\opair{i,j}\)元素为
\(-(b_{1i} b_{1j} + b_{2i} b_{2j} + \dotsb + b_{ni} b_{nj})\).
因为\(\vb{A}^2 = \vb{B}^2\),
所以\begin{equation*}
	a_{1i} a_{1j} + a_{2i} a_{2j} + \dotsb + a_{ni} a_{nj}
	= -(b_{1i} b_{1j} + b_{2i} b_{2j} + \dotsb + b_{ni} b_{nj}).
\end{equation*}

当\(i=j\)时,上式变为\(a_{1i}^2 + a_{2i}^2 + \dotsb + a_{ni}^2
= -(b_{1i}^2 + b_{2i}^2 + \dotsb + b_{ni}^2)\),
又由\(a_{ij},b_{ij} \in \mathbb{R}\)
可知\(a_{1i}^2 + a_{2i}^2 + \dotsb + a_{ni}^2 \geq 0\),
\(-(b_{1i}^2 + b_{2i}^2 + \dotsb + b_{ni}^2) \leq 0\),
所以\begin{equation*}
	a_{1i}^2 + a_{2i}^2 + \dotsb + a_{ni}^2
	= -(b_{1i}^2 + b_{2i}^2 + \dotsb + b_{ni}^2) = 0,
\end{equation*}
进而有\begin{equation*}
	a_{1i} = a_{2i} = \dotsb = a_{ni} = b_{1i} = b_{2i} = \dotsb = b_{ni} = 0.
	\qedhere
\end{equation*}
\end{proof}
\end{example}

\subsection{幂零矩阵}
%@Mathematica: MatrixPowerTable[A_, n_] := Table[MatrixPower[A, k] // MatrixForm, {k, 1, n}] // DeleteDuplicates

\begin{definition}
设矩阵\(\vb{A} \in M_n(K)\).
若\begin{equation*}
	(\exists m\in\mathbb{N}^+)
	[\vb{A}^m = \vb0],
\end{equation*}
则称“\(\vb{A}\)是\DefineConcept{幂零矩阵}(nilpotent matrix)”.
%@see: https://mathworld.wolfram.com/NilpotentMatrix.html
称使得\(\vb{A}^m = \vb0\)成立的最小正整数\begin{equation*}
    \min\Set{ m\in\mathbb{N}^+ \given \vb{A}^m = \vb0 }
\end{equation*}为“\(\vb{A}\)的\DefineConcept{幂零指数}”.
\end{definition}
%\cref{example:幂零矩阵.幂零矩阵的行列式}
%\cref{example:幂零矩阵.幂零矩阵的特征值的性质}
%\cref{example:幂零矩阵.幂零矩阵的相似类}
%\cref{example:幂零矩阵.非零的幂零矩阵不可以相似对角化}

\begin{example}
%@Mathematica: MatrixPowerTable[{{0, 1, 1}, {0, 0, 1}, {0, 0, 0}}, 5]
矩阵\begin{equation*}
	\begin{bmatrix}
		0 & 1 & 1 \\
		0 & 0 & 1 \\
		0 & 0 & 0
	\end{bmatrix}
\end{equation*}是幂零矩阵,
因为\begin{equation*}
	\begin{bmatrix}
		0 & 1 & 1 \\
		0 & 0 & 1 \\
		0 & 0 & 0
	\end{bmatrix}^2
	=
	\begin{bmatrix}
		0 & 0 & 1 \\
		0 & 0 & 0 \\
		0 & 0 & 0
	\end{bmatrix},
	\qquad
	\begin{bmatrix}
		0 & 1 & 1 \\
		0 & 0 & 1 \\
		0 & 0 & 0
	\end{bmatrix}
	\begin{bmatrix}
		0 & 0 & 1 \\
		0 & 0 & 0 \\
		0 & 0 & 0
	\end{bmatrix}
	=
	\begin{bmatrix}
		0 & 0 & 0 \\
		0 & 0 & 0 \\
		0 & 0 & 0
	\end{bmatrix},
\end{equation*}
%@Mathematica: MatrixPowerTable[{{0, 1, 1}, {0, 0, 1}, {0, 0, 0}}, 5] // Length
它的幂零指数是\(3\).
\end{example}
\begin{example}
%@Mathematica: MatrixPowerTable[{{0, 1, 0, 0}, {0, 0, 1, 0}, {0, 0, 0, 1}, {0, 0, 0, 0}}, 5]
矩阵\(
	\begin{bmatrix}
		0 & 1 & 0 & 0 \\
		0 & 0 & 1 & 0 \\
		0 & 0 & 0 & 1 \\
		0 & 0 & 0 & 0
	\end{bmatrix}
\)是幂零矩阵,
%@Mathematica: MatrixPowerTable[{{0, 1, 0, 0}, {0, 0, 1, 0}, {0, 0, 0, 1}, {0, 0, 0, 0}}, 5] // Length
它的幂零指数是\(4\).
\end{example}

\subsection{幂幺矩阵}
\begin{definition}
设矩阵\(\vb{A} \in M_n(K)\),\(\vb{E}\)是数域\(K\)上的\(n\)阶单位矩阵.
若\begin{equation*}
	(\exists m\in\mathbb{N}^+)
	[(\vb{A}-\vb{E})^m=\vb0],
\end{equation*}
则称“\(\vb{A}\)是\DefineConcept{幂幺矩阵}(unipotent matrix)”.
%@see: https://mathworld.wolfram.com/Unipotent.html
称使得\((\vb{A}-\vb{E})^m=\vb0\)成立的最小正整数\begin{equation*}
    \min\Set{ m\in\mathbb{N}^+ \given (\vb{A}-\vb{E})^m=\vb0 }
\end{equation*}为“\(\vb{A}\)的\DefineConcept{幂幺指数}”.
\end{definition}
%\cref{example:幂幺矩阵.幂幺矩阵的特征值的性质}

\subsection{幂等矩阵}
\begin{definition}\label{definition:幂等矩阵.幂等矩阵的定义}
设矩阵\(\vb{A} \in M_n(K)\).
若\(\vb{A}^2=\vb{A}\),
则称“\(\vb{A}\)是\DefineConcept{幂等矩阵}(idempotent matrix)”.
%@see: https://mathworld.wolfram.com/IdempotentMatrix.html
\end{definition}
%\cref{example:幂等矩阵.幂等矩阵的秩的性质1}
%\cref{example:幂等矩阵.幂等矩阵的特征值的性质}
%\cref{example:幂等矩阵.幂等矩阵的相似类}

\begin{example}
单位矩阵是幂等矩阵.
\end{example}

\begin{example}
%@Mathematica: MatrixPowerTable[{{0, 1, 1}, {0, 1, 1}, {0, 0, 0}}, 5]
矩阵\(
	\begin{bmatrix}
		0 & 1 & 1 \\
		0 & 1 & 1 \\
		0 & 0 & 0
	\end{bmatrix}
\)是一个幂等矩阵.
\end{example}

\subsection{对合矩阵}
\begin{definition}
设矩阵\(\vb{A} \in M_n(K)\),\(\vb{E}\)是数域\(K\)上的\(n\)阶单位矩阵.
若\(\vb{A}^2=\vb{E}\),
则称“\(\vb{A}\)是\DefineConcept{对合矩阵}(involutory matrix)”.
%@see: https://mathworld.wolfram.com/InvolutoryMatrix.html
\end{definition}
%\cref{example:对合矩阵.对合矩阵的秩的性质1}
%\cref{example:对合矩阵.对合矩阵的逆矩阵}
%\cref{example:对合矩阵.对合矩阵的相似类}

\begin{example}
%@Mathematica: MatrixPowerTable[{{0, 1}, {1, 0}}, 5]
矩阵\(
	\begin{bmatrix}
		0 & 1 \\
		1 & 0
	\end{bmatrix}
\)满足\begin{equation*}
	\begin{bmatrix}
		0 & 1 \\
		1 & 0
	\end{bmatrix}^2
	=
	\begin{bmatrix}
		0 & 1 \\
		1 & 0
	\end{bmatrix}
	\begin{bmatrix}
		0 & 1 \\
		1 & 0
	\end{bmatrix}
	= \begin{bmatrix}
		1 & 0 \\
		0 & 1
	\end{bmatrix},
\end{equation*}是一个对合矩阵.
\end{example}
\begin{example}
%@Mathematica: MatrixPowerTable[{{0, I}, {-I, 0}}, 5]
矩阵\(
	\begin{bmatrix}
		0 & \iu \\
		-\iu & 0
	\end{bmatrix},
	\qquad
	\begin{bmatrix}
		0 & -\iu \\
		\iu & 0
	\end{bmatrix}
\)都是对合矩阵.
\end{example}
\begin{example}
%@Mathematica: MatrixPowerTable[{{0, 0, 1}, {0, 1, 0}, {1, 0, 0}}, 5]
%@Mathematica: MatrixPowerTable[{{0, 1, 0}, {1, 0, 0}, {0, 0, 1}}, 5]
矩阵\begin{equation*}
	\begin{bmatrix}
		0 & 0 & 1 \\
		0 & 1 & 0 \\
		1 & 0 & 0
	\end{bmatrix},
	\qquad
	\begin{bmatrix}
		0 & 1 & 0 \\
		1 & 0 & 0 \\
		0 & 0 & 1
	\end{bmatrix}
\end{equation*}都是对合矩阵.
\end{example}

\subsection{周期矩阵}
\begin{definition}
设矩阵\(\vb{A} \in M_n(K)\),
\(\vb{E}\)是数域\(K\)上的\(n\)阶单位矩阵.
若\begin{equation*}
	(\exists m\in\mathbb{N}^+)
	[\vb{A}^m = \vb{E}],
\end{equation*}
则称“\(\vb{A}\)是\DefineConcept{周期矩阵}(periodic matrix)”.
使\(\vb{A}^m = \vb{E}\)成立的最小正整数\begin{equation*}
	\min\Set{ m\in\mathbb{N}^+ \given \vb{A}^m = \vb{E} }
\end{equation*}称为“\(\vb{A}\)的\DefineConcept{周期}”.
%@see: https://mathworld.wolfram.com/PeriodicMatrix.html
\end{definition}

% \subsection{汉克尔矩阵}
%TODO
%@see: https://mathworld.wolfram.com/HankelMatrix.html

\section{向量的内积}
\begin{definition}
设\(\a=(\AutoTuple{a}{n})^T\)和\(\b=(\AutoTuple{b}{n})^T\)都是\(n\)维复向量.
我们把复数\[
	a_1b_1 + a_2b_2 + \dotsb + a_nb_n
\]
称为“\(\a\)与\(\b\)的\DefineConcept{内积}(inner product)”,
记作\(\VectorInnerProductDot{\a}{\b}\).
\end{definition}

\begin{definition}
若向量\(\a\)与\(\b\)满足\(\VectorInnerProductDot{\a}{\b}=0\),
则称\(\a\)与\(\b\)正交(orthogonal),
记作\(\a\perp\b\).
\end{definition}

\begin{property}
向量内积具有以下性质:
\begin{enumerate}
	\item \(\VectorInnerProductDot{\a}{\b} = \VectorInnerProductDot{\b}{\a}\);
	\item \(\VectorInnerProductDot{(\a+\b)}{\g} = \VectorInnerProductDot{\a}{\g} + \VectorInnerProductDot{\b}{\g}\);
	\item \(\VectorInnerProductDot{(k\a)}{\b} = k (\VectorInnerProductDot{\a}{\b})\ (k\in\mathbb{R})\);
	\item \(\a\neq\z \iff \VectorInnerProductDot{\a}{\a} > 0\);\(\a=\z \iff \VectorInnerProductDot{\a}{\a} = 0\);
	\item \(\VectorInnerProductDot{\z}{\a} = 0\);
\end{enumerate}
\end{property}

\subsection{向量的长度(模、范数)与单位向量}
\begin{definition}
设\(n\)维向量\(\a = (\AutoTuple{a}{n})\).
定义向量的\DefineConcept{长度}为\[
	\sqrt{\VectorInnerProductDot{\a}{\a}} = \sqrt{a_1^2+a_2^2+\dotsb+a_n^2}.
\]同样地可以定义\(n\)维列向量的长度.
2维向量、3维向量的长度常被称作向量的\DefineConcept{模}(module),记作\(\abs{\a}\).
高维(\(n > 3\))向量的长度常被称作向量的\DefineConcept{范数}(norm),记作\(\norm{\a}\).
\end{definition}

\begin{property}
显然有向量的长度为非负实数,即\(\abs{\a}\geq0\).
\end{property}

\begin{definition}
长度为1的向量被称为\DefineConcept{单位向量}.
\end{definition}

\begin{definition}
\def\f{\frac{1}{\abs{\a}}}
设\(\a\)满足\(\abs{\a}>0\).
用\(\f\)数乘\(\a\),
称为“将\(\a\) \DefineConcept{单位化}”,
得单位向量\(\f\a\).
\end{definition}

尽管我们通常出于几何(特别是欧氏几何)的考量,
像上面一样将向量\(\a\)的模(或范数)定义为\(\sqrt{\VectorInnerProductDot{\a}{\a}}\),
不过我们还可以定义其他形式的模(或范数).
观察上面的模(或范数)的定义,我们可以发现,
向量的模(或范数)实际上是满足以下3条性质的映射
\begingroup%
\def\x{\vb{x}}%
\def\y{\vb{y}}%
\(f\colon K^n \to K, \x \mapsto m\):
\begin{enumerate}
	\item {\rm\bf 非负性},
	即\((\forall \x \in K^n)[f(\x) \geq 0]\);
	\item {\rm\bf 齐次性},
	即\((\forall \x \in K^n)(\forall c \in K)[f(c \x) = \abs{c} f(\x)]\);
	\item {\rm\bf 三角不等式},
	即\((\forall \x,\y \in K^n)[f(\x+\y) \leq f(\x) + f(\y)]\).
\end{enumerate}

\begin{definition}\label{definition:向量与矩阵.p范数}
形如\[
	f\colon\mathbb{R}^n \to \mathbb{R},
	\x = \opair{\AutoTuple{x}{n}}
	\mapsto
	\sqrt[p]{\abs{x_1}^p + \abs{x_2}^p + \dotsb + \abs{x_n}^p}
\]的这一类映射,
称为 \DefineConcept{\(p\)范数},
记作\(\norm{\x}_p\).
\end{definition}

易见
\begin{gather}
	\norm{\x}_1 = \abs{x_1} + \abs{x_2} + \dotsb + \abs{x_n}, \\
	\norm{\x}_2 = \sqrt{x_1^2 + x_2^2 + \dotsb + x_n^2}, \\
	\norm{\x}_\infty = \max\{\abs{x_1},\abs{x_2},\dotsc,\abs{x_n}\}.
\end{gather}
\endgroup%

\section{分块矩阵的运算}
分块阵的运算服从以下规律:
\begin{enumerate}
	\item {\rm\bf 分块阵的加法}

	设\(\vb{A},\vb{B} \in M_{s \times n}(K)\),
	若将\(\vb{A}\)和\(\vb{B}\)按同样的规则分块为\begin{equation*}
		\vb{A}=(\vb{A}_{ij})_{t \times r}, \qquad
		\vb{B}=(\vb{B}_{ij})_{t \times r},
	\end{equation*}
	其中\(\vb{A}_{ij},\vb{B}_{ij}\in M_{s_i \times n_j}(K)\ (i=1,2,\dotsc,t;j=1,2,\dotsc,r)\),
	则\begin{equation*}
		\vb{A}+\vb{B}=(\vb{A}_{ij}+\vb{B}_{ij})_{t \times r}.
	\end{equation*}

	\item {\rm\bf 分块阵的数乘}

	设\(\vb{A}\in M_{s \times n}(K)\),
	若将\(\vb{A}\)分块为\begin{equation*}
		\vb{A}=(\vb{A}_{ij})_{t \times r},
	\end{equation*}
	其中\(\vb{A}_{ij}\in M_{s_i \times n_j}(K)\ (i=1,2,\dotsc,t;j=1,2,\dotsc,r)\),
	则\begin{equation*}
		k\vb{A}=(k\vb{A}_{ij})_{t \times r}.
	\end{equation*}

	\item {\rm\bf 分块阵的转置}

	设\(\vb{A}\in M_{s \times n}(K)\),
	若将\(\vb{A}\)分块为\begin{equation*}
		\vb{A}=(\vb{A}_{ij})_{t \times r},
	\end{equation*}
	其中\(\vb{A}_{ij}\in M_{s_i \times n_j}(K)\ (i=1,2,\dotsc,t;j=1,2,\dotsc,r)\),
	则\begin{equation*}
		\vb{A}^T=(\vb{A}_{ji}^T)_{r \times t}.
	\end{equation*}
	这就是说,在转置分块阵时,要将每个子块转置.

	\item {\rm\bf 分块阵的乘法}

	设\(\vb{A}\in M_{s \times n}(K),
	\vb{B}\in M_{n \times m}(K)\),
	若将\(\vb{A}\)、\(\vb{B}\)分别分块为\begin{equation*}
		\vb{A}=(\vb{A}_{ij})_{t \times r}, \qquad
		\vb{B}=(\vb{B}_{jk})_{r \times p},
	\end{equation*}
	且\(\vb{A}\)的列的分块法与\(\vb{B}\)的行的分块法一致,即\begin{equation*}
		\vb{A} = \begin{matrix}
			& \begin{matrix} n_1 & n_2 & \dots & n_r \end{matrix} \\
			\begin{matrix} s_1 \\ s_2 \\ \vdots \\ s_t \end{matrix} & \begin{bmatrix}
			\vb{A}_{11} & \vb{A}_{12} & \dots & \vb{A}_{1r} \\
			\vb{A}_{21} & \vb{A}_{22} & \dots & \vb{A}_{2r} \\
			\vdots & \vdots & & \vdots \\
			\vb{A}_{t1} & \vb{A}_{t2} & \dots & \vb{A}_{tr}
			\end{bmatrix}
		\end{matrix},
		\qquad
		\vb{B} = \begin{matrix}
			& \begin{matrix} m_1 & m_2 & \dots & m_p \end{matrix} \\
			\begin{matrix} n_1 \\ n_2 \\ \vdots \\ n_r \end{matrix} & \begin{bmatrix}
			\vb{B}_{11} & \vb{B}_{12} & \dots & \vb{B}_{1p} \\
			\vb{B}_{21} & \vb{B}_{22} & \dots & \vb{B}_{2p} \\
			\vdots & \vdots & & \vdots \\
			\vb{B}_{r1} & \vb{B}_{r2} & \dots & \vb{B}_{rp}
			\end{bmatrix},
		\end{matrix}
	\end{equation*}
	则\begin{equation*}
		\vb{A}\vb{B} = \begin{matrix}
			& \begin{matrix} m_1 & m_2 & \dots & m_p \end{matrix} \\
			\begin{matrix} s_1 \\ s_2 \\ \vdots \\ s_t \end{matrix} & \begin{bmatrix}
			\vb{C}_{11} & \vb{C}_{12} & \dots & \vb{C}_{1p} \\
			\vb{C}_{21} & \vb{C}_{22} & \dots & \vb{C}_{2p} \\
			\vdots & \vdots & & \vdots \\
			\vb{C}_{t1} & \vb{C}_{t2} & \dots & \vb{C}_{tp}
			\end{bmatrix}
		\end{matrix}.
	\end{equation*}
	其中\(\vb{C}_{ij}=\sum_{k=1}^r \vb{A}_{ik} \vb{B}_{kj}\ (i=1,2,\dotsc,t;j=1,2,\dotsc,p)\).
\end{enumerate}
\begin{remark}
下面列举几个十分常用的矩阵乘法运算:\begin{gather*}
	\vb{A} (\AutoTuple{\vb\alpha}{m})
	= (\AutoTuple{\vb{A} \vb\alpha}{m})
	\quad(\vb{A} \in M_{s \times n}(K),\vb\alpha_i \in K^n,i=1,2,\dotsc,m), \\
	(\AutoTuple{\vb\alpha}{m})
	\begin{bmatrix}
		\vb{B}_1 \\
		\vdots \\
		\vb{B}_m
	\end{bmatrix}
	= \sum_{i=1}^m \vb\alpha_i \vb{B}_i
	\quad(\vb{B}_i \in K^t,\vb\alpha_i \in K^n,i=1,2,\dotsc,m), \\
	\begin{bmatrix}
		\vb{A}_1 \\ \vb{A}_2 \\ \vdots \\ \vb{A}_s
	\end{bmatrix}
	\vb{B}
	= \begin{bmatrix}
		\vb{A}_1 \vb{B} \\
		\vb{A}_2 \vb{B} \\
		\vdots \\
		\vb{A}_s \vb{B}
	\end{bmatrix}
	\quad(\text{$\AutoTuple{\vb{A}}{s}$的列数与$\vb{B}$的行数相同}), \\
	\begin{bmatrix}
		\vb{A}_1 & \vb0 \\
		\vb0 & \vb{A}_2
	\end{bmatrix}
	\begin{bmatrix}
		\vb{B}_1 & \vb0 \\
		\vb0 & \vb{B}_2
	\end{bmatrix}
	= \begin{bmatrix}
		\vb{A}_1 \vb{B}_1 & \vb0 \\
		\vb0 & \vb{A}_2 \vb{B}_2
	\end{bmatrix}.
\end{gather*}
\end{remark}

\begin{example}
设\(n\)阶矩阵\(
	\vb{A}_k
	\defeq \begin{bmatrix}
		\vb0 & \vb{E}_{n-k} \\
		0 & \vb0
	\end{bmatrix}
	\ (n>k\geq1)
\),
其中\(\vb{E}_k\)是\(k\)阶单位矩阵.
% 这是一个幂零矩阵
证明:\begin{equation*}
	\vb{A}_1^k
	= \begin{cases}[cl]
		\vb{A}_k,	& 1 \leq k \leq n-1, \\
		\vb0,			& k = n.
	\end{cases}
\end{equation*}
%TODO proof
\end{example}

\section{初等矩阵}
\subsection{初等变换}
对矩阵施行以下变换,称为矩阵的\DefineConcept{初等行变换}(elementary row operation):
\begin{enumerate}
	\item 互换两行的位置;
	\item 用一非零数\(c\)乘以某行;
	\item 将某行的\(k\)倍加到另一行.
\end{enumerate}

类似地,可以定义矩阵的\DefineConcept{初等列变换}(elementary column operation):
\begin{enumerate}
	\item 互换两列的位置;
	\item 用一非零数\(c\)乘以某列;
	\item 将某列的\(k\)倍加到另一列.
\end{enumerate}

矩阵的初等行变换、初等列变换统称为矩阵的\DefineConcept{初等变换}(elementary operation).

%我们约定:
%矩阵\(\vb{A}\)经过一次初等行变换\(\sigma_1\)化为矩阵\(\vb{B}\)的过程
%可以表示为在连接矩阵\(\vb{A}\)和\(\vb{B}\)的箭头上方标记\(\sigma_1\),即\begin{equation*}
%	\vb{A} \xlongrightarrow{\sigma_1} \vb{B};
%\end{equation*}而矩阵\(\vb{A}\)经过一次初等列变换\(\sigma_2\)化为矩阵\(\vb{B}\)的过程
%可以表示为在连接矩阵\(\vb{A}\)和\(\vb{B}\)的箭头下方标记\(\sigma_2\),即\begin{equation*}
%	\vb{A} \xlongrightarrow[\sigma_2]{} \vb{B}.
%\end{equation*}

\subsection{初等矩阵的概念}
若矩阵\(\vb{A}\)可以经过一系列初等变换化为矩阵\(\vb{B}\),
则称“\(\vb{A}\)与\(\vb{B}\)~\DefineConcept{等价}(equivalent)”,
或“\(\vb{A}\)与\(\vb{B}\)~\DefineConcept{相抵}”,
记作\(\vb{A}\cong\vb{B}\).

\begin{definition}
由\(n\)阶单位矩阵\(\vb{E}\)经过\emph{一次}初等变换所得矩阵
称为\(n\)阶\DefineConcept{初等矩阵}(elementary matrix).
\end{definition}

对应于矩阵的三类初等变换,有三种类型的初等矩阵:
\begin{enumerate}
	\item 互换\(\vb{E}\)的\(i\),\(j\)两行(列)所得的矩阵\begin{equation*}
		\vb{P}(i,j) = \begin{bmatrix}
			\vb{E}_{i-1} & & & \\
			& 0 & & 1 & \\
			& & \vb{E}_{j-i-1} & & \\
			& 1 & & 0 & \\
			& & & & \vb{E}_{n-j}
		\end{bmatrix}_n;
	\end{equation*}
	\item 用非零数\(c\)乘以\(\vb{E}\)的第\(i\)行(列)所得的矩阵\begin{equation*}
		\vb{P}(i(c)) = \begin{bmatrix}
			\vb{E}_{i-1} & & \\
			& c & \\
			& & \vb{E}_{n-i}
		\end{bmatrix}_n;
	\end{equation*}
	\item 把\(\vb{E}\)的第\(j\)行(第\(i\)列)的\(k\)倍加到第\(i\)行(第\(j\)列)所得的矩阵\begin{equation*}
		\vb{P}(i,j(k)) = \begin{bmatrix}
			\vb{E}_{i-1} & & & \\
			& 1 & & k & \\
			& & \vb{E}_{j-i-1} & & \\
			& 0 & & 1 & \\
			& & & & \vb{E}_{n-j}
		\end{bmatrix}_n.
	\end{equation*}
\end{enumerate}

\subsection{初等矩阵的性质}
\begin{property}\label{theorem:逆矩阵.初等矩阵的性质1}
初等矩阵具有以下性质:\begin{gather}
	\abs{\vb{P}(i,j)} = -1, \\
	\abs{\vb{P}(i(c))} = c, \\
	\abs{\vb{P}(i,j(k))} = 1, \\
	\vb{P}(i,j)^T = \vb{P}(i,j), \\
	\vb{P}(i(c))^T = \vb{P}(i(c)), \\
	\vb{P}(i,j(k))^T = \vb{P}(j,i(k)), \\
	\vb{P}(i,j)^{-1} = \vb{P}(i,j), \\
	\vb{P}(i(c))^{-1} = \vb{P}(i(c^{-1})), \\
	\vb{P}(i,j(k))^{-1} = \vb{P}(i,j(-k)).
\end{gather}
\end{property}

\begin{property}\label{theorem:逆矩阵.初等矩阵的性质2}
对\(n \times t\)矩阵\(\vb{A}\)施行一次初等行变换,相当于用一个相应的\(n\)阶初等矩阵左乘\(\vb{A}\);
对\(\vb{A}\)施行一次初等列变换,相当于用一个相应的\(t\)阶初等矩阵右乘\(\vb{A}\).
\begin{proof}
用\(n\)阶矩阵\(\vb{P}(i,j)\)左乘\(\vb{A}\),将矩阵\(\vb{A}\)作相应分块,有\begin{equation*}
	\vb{P}(i,j) \vb{A} = \begin{bmatrix}
		\vb{E}_{i-1} \\
		& 0 & & 1 \\
		& & \vb{E}_{j-i-1} \\
		& 1 & & 0 \\
		& & & & \vb{E}_{n-j}
	\end{bmatrix}
	\begin{bmatrix}
		\vb{A}_1 \\ \vb\alpha_i \\ \vb{A}_2 \\ \vb\alpha_j \\ \vb{A}_3
	\end{bmatrix}
	= \begin{bmatrix}
		\vb{A}_1 \\ \vb\alpha_j \\ \vb{A}_2 \\ \vb\alpha_i \\ \vb{A}_3
	\end{bmatrix},
\end{equation*}
即\(\vb{A}\)交换\(i\)、\(j\)两行.

用\(n\)阶矩阵\(\vb{P}(i(c))\)左乘\(\vb{A}\),将矩阵\(\vb{A}\)作相应分块,有\begin{equation*}
	\vb{P}(i(c)) \vb{A} = \begin{bmatrix}
		\vb{E}_{i-1} \\
		& c \\
		& & \vb{E}_{n-i}
	\end{bmatrix}
	\begin{bmatrix}
		\vb{A}_1 \\ \vb\alpha_i \\ \vb{A}_2
	\end{bmatrix}
	= \begin{bmatrix}
		\vb{A}_1 \\ c \vb\alpha_i \\ a_2
	\end{bmatrix},
\end{equation*}
即用一非零数\(c\)乘以第\(i\)行.

用\(n\)阶矩阵\(\vb{P}(i,j(k))\ (i < j)\)左乘\(\vb{A}\),将矩阵\(\vb{A}\)作相应分块,有\begin{equation*}
	\vb{P}(i,j(k)) \vb{A} = \begin{bmatrix}
		\vb{E}_{i-1} \\
		& 1 & & k \\
		& & \vb{E}_{j-i-1} \\
		& 0 & & 1 \\
		& & & & \vb{E}_{n-j}
	\end{bmatrix}
	\begin{bmatrix}
		\vb{A}_1 \\ \vb\alpha_i \\ \vb{A}_2 \\ \vb\alpha_j \\ \vb{A}_3
	\end{bmatrix}
	= \begin{bmatrix}
		\vb{A}_1 \\ \vb\alpha_i + k \vb\alpha_j \\ \vb{A}_2 \\ \vb\alpha_j \\ \vb{A}_3
	\end{bmatrix},
\end{equation*}
即把\(\vb{A}\)第\(j\)行的\(k\)倍加到第\(i\)行.
\end{proof}
\end{property}

\section{广义初等变换}
设\(\vb{A} \in M_{m \times s}(K),
\vb{B} \in M_{m \times t}(K),
\vb{C} \in M_{n \times s}(K),
\vb{D} \in M_{n \times t}(K)\).

\def\OriginalMatrix{
	\begin{bmatrix}
		\vb{A} & \vb{B} \\
		\vb{C} & \vb{D}
	\end{bmatrix}
}
分块矩阵有以下三种\DefineConcept{广义初等行变换}:
\begin{enumerate}
	\item 交换两行,\begin{equation*}
		\OriginalMatrix
		\mapsto \begin{bmatrix}
			\vb{C} & \vb{D} \\
			\vb{A} & \vb{B}
		\end{bmatrix}
		= {\color{red} \begin{bmatrix}
			\vb0 & \vb{E}_n \\
			\vb{E}_m & \vb0
		\end{bmatrix}}
		\OriginalMatrix.
	\end{equation*}

	\item 用一个可逆矩阵\(\vb{P}_m\)左乘某一行,\begin{equation*}
		\OriginalMatrix
		\mapsto \begin{bmatrix}
			\vb{P}\vb{A} & \vb{P}\vb{B} \\
			\vb{C} & \vb{D}
		\end{bmatrix}
		= {\color{red} \begin{bmatrix}
			\vb{P} & \vb0 \\
			\vb0 & \vb{E}_n
		\end{bmatrix}}
		\OriginalMatrix.
	\end{equation*}

	\item 用一个矩阵\(\vb{Q}_{n \times m}\)左乘某一行后加到另一行,\begin{equation*}
		\OriginalMatrix
		\mapsto \begin{bmatrix}
		\vb{A} & \vb{B} \\
		\vb{C}+\vb{Q}\vb{A} & \vb{D}+\vb{Q}\vb{B}
		\end{bmatrix}
		= {\color{red} \begin{bmatrix}
		\vb{E}_m & \vb0 \\
		\vb{Q} & \vb{E}_n
		\end{bmatrix}}
		\OriginalMatrix.
	\end{equation*}
\end{enumerate}

类似地,有\DefineConcept{广义初等列变换}:
\begin{enumerate}
	\item 交换两列,\begin{equation*}
		\OriginalMatrix
		\mapsto \begin{bmatrix}
			\vb{B} & \vb{A} \\
			\vb{D} & \vb{C}
		\end{bmatrix}
		= \OriginalMatrix {\color{red} \begin{bmatrix}
			\vb0 & \vb{E}_s \\
			\vb{E}_t & \vb0
		\end{bmatrix}}.
	\end{equation*}

	\item 用一个可逆矩阵\(\vb{P}_t\)右乘某一列,\begin{equation*}
		\OriginalMatrix
		\mapsto \begin{bmatrix}
			\vb{A} & \vb{B}\vb{P} \\
			\vb{C} & \vb{D}\vb{P}
		\end{bmatrix}
		= \OriginalMatrix {\color{red} \begin{bmatrix}
			\vb{E}_s & \vb0 \\
			\vb0 & \vb{P}
		\end{bmatrix}}.
	\end{equation*}

	\item 用一个矩阵\(\vb{Q}_{t \times s}\)右乘某一列后加到另一列,\begin{equation*}
		\OriginalMatrix
		\mapsto \begin{bmatrix}
			\vb{A} + \vb{B}\vb{Q} & \vb{B} \\
			\vb{C} + \vb{D}\vb{Q} & \vb{D}
		\end{bmatrix}
		= \OriginalMatrix {\color{red} \begin{bmatrix}
			\vb{E}_s & \vb0 \\
			\vb{Q} & \vb{E}_t
		\end{bmatrix}}.
	\end{equation*}
\end{enumerate}

广义初等行变换与广义初等列变换统称为\DefineConcept{广义初等变换}.

类比于初等矩阵,我们定义分块初等矩阵如下:
将\(n\)阶单位矩阵\(\vb{E}\)分为\(m\)块后,
进行\emph{一次}广义初等变换所得的矩阵称为\DefineConcept{分块初等矩阵}.

\begin{property}
分块初等矩阵都是可逆矩阵.
\end{property}

\section{张量积}
\subsection{张量积的定义}
\begin{definition}
%@see: 《矩阵论》(詹兴致) P13
设\(\vb{A} = (a_{ij}) \in M_{m \times n}(\mathbb{C}),
\vb{B} \in M_{s \times t}(\mathbb{C})\).
定义:\begin{equation}
	\MatrixKroneckerTensorProduct{\vb{A}}{\vb{B}}
	\defeq
	\begin{bmatrix}
		a_{11} \vb{B} & a_{12} \vb{B} & \dots & a_{1n} \vb{B} \\
		a_{21} \vb{B} & a_{22} \vb{B} & \dots & a_{2n} \vb{B} \\
		\vdots & \vdots & & \vdots \\
		a_{m1} \vb{B} & a_{m2} \vb{B} & \dots & a_{mn} \vb{B}
	\end{bmatrix}
	\in M_{ms \times nt},
\end{equation}
称之为“矩阵\(\vb{A}\)与矩阵\(\vb{B}\)的\DefineConcept{克罗内克张量积}(Kronecker tensor product)”.
\end{definition}

\begin{property}
%@see: 《矩阵论》(詹兴致) P13
克罗内克张量积具有以下性质:\begin{gather*}
	(\forall \vb{A} \in M_{m \times n}(\mathbb{C}))
	(\forall \vb{B} \in M_{s \times t}(\mathbb{C}))
	(\forall k \in \mathbb{C})
	[
		\MatrixKroneckerTensorProduct{(k\vb{A})}{\vb{B}}
		= \MatrixKroneckerTensorProduct{\vb{A}}{(k\vb{B})}
		= k(\MatrixKroneckerTensorProduct{\vb{A}}{\vb{B}})
	], \\
	(\forall \vb{A} \in M_{m \times n}(\mathbb{C}))
	(\forall \vb{B} \in M_{s \times t}(\mathbb{C}))
	[
		(\MatrixKroneckerTensorProduct{\vb{A}}{\vb{B}})^T
		= \MatrixKroneckerTensorProduct{\vb{A}^T}{\vb{B}^T}
	], \\
	(\forall \vb{A} \in M_{m \times n}(\mathbb{C}))
	(\forall \vb{B} \in M_{s \times t}(\mathbb{C}))
	[
		(\MatrixKroneckerTensorProduct{\vb{A}}{\vb{B}})^H
		= \MatrixKroneckerTensorProduct{\vb{A}^H}{\vb{B}^H}
	], \\
	(\forall \vb{A} \in M_{m \times n}(\mathbb{C}))
	(\forall \vb{B} \in M_{s \times t}(\mathbb{C}))
	(\forall \vb{C} \in M_{p \times q}(\mathbb{C}))
	[
		\MatrixKroneckerTensorProduct{(\MatrixKroneckerTensorProduct{\vb{A}}{\vb{B}})}{\vb{C}}
		= \MatrixKroneckerTensorProduct{\vb{A}}{(\MatrixKroneckerTensorProduct{\vb{B}}{\vb{C}})}
	], \\
	(\forall \vb{A} \in M_{m \times n}(\mathbb{C}))
	(\forall \vb{B},\vb{C} \in M_{s \times t}(\mathbb{C}))
	[
		\MatrixKroneckerTensorProduct{\vb{A}}{(\vb{B}+\vb{C})}
		= (\MatrixKroneckerTensorProduct{\vb{A}}{\vb{B}})
		+ (\MatrixKroneckerTensorProduct{\vb{A}}{\vb{C}})
	], \\
	(\forall \vb{A},\vb{B} \in M_{m \times n}(\mathbb{C}))
	(\forall \vb{C} \in M_{s \times t}(\mathbb{C}))
	[
		\MatrixKroneckerTensorProduct{(\vb{A}+\vb{B})}{\vb{C}}
		= (\MatrixKroneckerTensorProduct{\vb{A}}{\vb{C}})
		+ (\MatrixKroneckerTensorProduct{\vb{B}}{\vb{C}})
	], \\
	(\forall \vb{A} \in M_{m \times n}(\mathbb{C}))
	(\forall \vb{B} \in M_{s \times t}(\mathbb{C}))
	[
		\MatrixKroneckerTensorProduct{\vb{A}}{\vb{B}} = 0
		\iff
		\vb{A} = 0 \lor \vb{B} = 0
	], \\
	(\forall \vb{A} \in M_m(\mathbb{C}))
	(\forall \vb{B} \in M_n(\mathbb{C}))
	[
		\text{$\vb{A},\vb{B}$都是对称矩阵}
		\implies
		\text{$\MatrixKroneckerTensorProduct{\vb{A}}{\vb{B}}$是对称矩阵}
	], \\
	(\forall \vb{A} \in M_m(\mathbb{C}))
	(\forall \vb{B} \in M_n(\mathbb{C}))
	[
		\text{$\vb{A},\vb{B}$都是厄米矩阵}
		\implies
		\text{$\MatrixKroneckerTensorProduct{\vb{A}}{\vb{B}}$是厄米矩阵}
	].
\end{gather*}
\end{property}

\begin{lemma}
%@see: 《矩阵论》(詹兴致) P14 引理2.1
设矩阵\(\vb{A} \in M_{m \times n}(\mathbb{C}),
\vb{B} \in M_{s \times t}(\mathbb{C}),
\vb{C} \in M_{n \times k}(\mathbb{C}),
\vb{D} \in M_{t \times r}(\mathbb{C})\),
则\begin{equation}
	(\MatrixKroneckerTensorProduct{\vb{A}}{\vb{B}})
	(\MatrixKroneckerTensorProduct{\vb{C}}{\vb{D}})
	= \MatrixKroneckerTensorProduct{(\vb{A}\vb{C})}{(\vb{B}\vb{D})}.
\end{equation}
%TODO proof
\end{lemma}

\begin{property}
%@see: 《矩阵论》(詹兴致) P14 定理2.2(i)
设矩阵\(\vb{A} \in M_m(\mathbb{C}),
\vb{B} \in M_n(\mathbb{C})\).
若\(\vb{A},\vb{B}\)都可逆,
则\(\MatrixKroneckerTensorProduct{\vb{A}}{\vb{B}}\)也可逆,
且\begin{equation}
	(\MatrixKroneckerTensorProduct{\vb{A}}{\vb{B}})^{-1}
	= \MatrixKroneckerTensorProduct{\vb{A}^{-1}}{\vb{B}^{-1}}.
\end{equation}
\end{property}

\begin{property}
%@see: 《矩阵论》(詹兴致) P14 定理2.2(ii)
设矩阵\(\vb{A} \in M_m(\mathbb{C}),
\vb{B} \in M_n(\mathbb{C})\).
若\(\vb{A},\vb{B}\)都是正规矩阵,
则\(\MatrixKroneckerTensorProduct{\vb{A}}{\vb{B}}\)也是正规矩阵.
\end{property}

\begin{property}
%@see: 《矩阵论》(詹兴致) P14 定理2.2(iii)
设矩阵\(\vb{A} \in M_m(\mathbb{C}),
\vb{B} \in M_n(\mathbb{C})\).
若\(\vb{A},\vb{B}\)都是酉矩阵,
则\(\MatrixKroneckerTensorProduct{\vb{A}}{\vb{B}}\)也是酉矩阵.
\end{property}

\begin{property}
%@see: 《矩阵论》(詹兴致) P14 定理2.2(iv)
%@see: 《矩阵论》(詹兴致) P14 定理2.2(v)
设矩阵\(\vb{A} \in M_m(\mathbb{C}),
\vb{B} \in M_n(\mathbb{C})\).
若\(\lambda\)是\(\vb{A}\)的一个特征值,
\(\vb{x}\)是\(\vb{A}\)的属于\(\lambda\)的一个特征向量,
\(\mu\)是\(\vb{B}\)的一个特征值,
\(\vb{y}\)是\(\vb{B}\)的属于\(\mu\)的一个特征向量,
则\(\lambda\mu\)是\(\MatrixKroneckerTensorProduct{\vb{A}}{\vb{B}}\)的一个特征值,
\(\MatrixKroneckerTensorProduct{\vb{x}}{\vb{y}}\)是
\(\MatrixKroneckerTensorProduct{\vb{A}}{\vb{B}}\)的
属于\(\lambda\mu\)的特征向量.
\end{property}

\begin{property}
%@see: 《矩阵论》(詹兴致) P14 定理2.2(vi)
设矩阵\(\vb{A} \in M_m(\mathbb{C}),
\vb{B} \in M_n(\mathbb{C})\),
则\begin{equation}
	\det(
		\MatrixKroneckerTensorProduct{\vb{A}}{\vb{B}}
	)
	= (\det\vb{A})^n (\det\vb{B})^m.
\end{equation}
\end{property}

\begin{property}
%@see: 《矩阵论》(詹兴致) P14 定理2.2(vii)
设矩阵\(\vb{A} \in M_m(\mathbb{C}),
\vb{B} \in M_n(\mathbb{C})\).
若\(\lambda\)是\(\vb{A}\)的一个奇异值,
\(\mu\)是\(\vb{B}\)的一个奇异值,
则\(\lambda\mu\)是\(\MatrixKroneckerTensorProduct{\vb{A}}{\vb{B}}\)的一个奇异值.
\end{property}

\begin{property}
%@see: 《矩阵论》(詹兴致) P14 定理2.2(viii)
设矩阵\(\vb{A} \in M_m(\mathbb{C}),
\vb{B} \in M_n(\mathbb{C})\),
则\begin{equation}
	\rank(
		\MatrixKroneckerTensorProduct{\vb{A}}{\vb{B}}
	)
	= (\rank\vb{A}) (\rank\vb{B}).
\end{equation}
\end{property}


\chapter{行列式}
\section{排列与逆序数}
\begin{definition}
\(n\)个不同的自然数按一定顺序排列组成的一个有序数组\begin{equation*}
	(\AutoTuple{k}{n})
\end{equation*}称为一个\(n\)阶\DefineConcept{排列}.

当\(1 \leq i<j \leq n\)时,
如果\(k_i>k_j\),则称“\(k_i,k_j\)构成一个\DefineConcept{逆序}”;
如果\(k_i<k_j\),则称“\(k_i,k_j\)构成一个\DefineConcept{顺序}”.
此排列中的逆序的总数叫它的\DefineConcept{逆序数},记作\(\tau(\AutoTuple{k}{n})\).

逆序数为偶数的排列叫\DefineConcept{偶排列},逆序数为奇数的排列叫\DefineConcept{奇排列}.
\end{definition}

计算一个排列的逆序数时,排列中的逆序不能重复计算,也不能漏掉.
可按公式\begin{equation*}
	\tau(\AutoTuple{k}{n})=m_1+m_2+\dotsb+m_n
\end{equation*}计算,其中\(m_i\)为排列中排在\(k_i\)后面比它小的数的个数.
\begin{example}
在4阶排列2341中,2与3形成的数对\((2,3)\),小的数在前,大的数在后,这一对数构成一个顺序;
而2与1形成的数对\((2,1)\),大的数在前,小的数在后,这一对数构成一个逆序.

\begin{figure}[htb]
	\centering
	\begin{tikzpicture}
		\foreach \j in {0,...,3} {
			\fill[ballblue](\j cm+1pt,1pt)rectangle(\j cm+1cm-1pt,1cm-1pt);
		}
		\foreach \i in {0,...,3} {
			\fill[orangepeel](-1pt,-\i cm-1pt)rectangle(-1cm+1pt,-\i cm-1cm+1pt);
		}
		\tiny
		\draw[black](-.5,5pt)node{\(k_i\)}
				(-5pt,.5)node{\(k_j\)};
		\normalsize
		\draw(-.5,-.5)node{2}
			(-.5,-1.5)node{3}
			(-.5,-2.5)node{4}
			(-.5,-3.5)node{1};
		\draw(.5,.5)node{2}
			(1.5,.5)node{3}
			(2.5,.5)node{4}
			(3.5,.5)node{1};
		\draw[applegreen](1.5,-.5)node{(2,3)}
			(2.5,-.5)node{(2,4)}
			(2.5,-1.5)node{(3,4)};
		\draw[tangelo](3.5,-.5)node{(2,1)}
			(3.5,-1.5)node{(3,1)}
			(3.5,-2.5)node{(4,1)};
		\draw[black!30,dashed](0,0)--(4,-4);
	\end{tikzpicture}
	\caption{}
	\label{figure:行列式.4阶排列2341的所有数对}
\end{figure}

如\cref{figure:行列式.4阶排列2341的所有数对},
构成逆序的数对有\((2,1),(3,1),(4,1)\),
构成顺序的数对有\((2,3),(2,4),(3,4)\),
这个4阶排列的逆序数是3,即\(\tau(2341)=3\),
它是一个奇排列.
\end{example}

\begin{property}
\(n\)阶排列共有\(n!\)个.
\end{property}

\begin{definition}
排列\(1,2,\dotsc,n\)由小到大按自然顺序排列,叫做\(n\)阶\DefineConcept{自然排列}.
\end{definition}

\begin{property}
自然排列中没有逆序,即\begin{equation}
	\tau(1,2,\dotsc,n)=0.
\end{equation}
\end{property}

\begin{example}
证明:\begin{equation}
	\tau(n,n-1,\dotsc,1)=\frac{n(n-1)}{2}.
\end{equation}
\begin{proof}
由于对于排列中的每一个数来说,其后的所有数都比它小,所以\begin{equation*}
	\tau(n,n-1,\dotsc,1)
	= (n-1) + (n-2) + \dotsb + 1 + 0
	= \frac{(n-1)n}{2}.
	\qedhere
\end{equation*}
\end{proof}
\end{example}

\begin{definition}
把排列中的两个数的位置互换,其余数字不动,得到另一个排列;
像这样的变换称为\DefineConcept{对换}.
\end{definition}

\begin{theorem}
排列经一次对换后奇偶性改变.
\begin{proof}
我们首先讨论对换的两个数在\(n\)阶排列中相邻的情形:
排列\(k_1,\dotsc,k_i,k_{i+1},\dotsc,k_n\)
对换\(k_i\)与\(k_{i+1}\)这两个数会得到\(k_1,\dotsc,k_{i+1},k_i,\dotsc,k_n\);
除\(k_i,k_{i+1}\)以外的数构成的数对是顺序还是逆序,
在变换前与变换后是一样的;
\(k_i\)和\(k_{i+1}\)以外的数与\(k_i\)或\(k_{i+1}\)构成的数对是顺序还是逆序,
在变换前后也是一样的.
只有数对\((k_i,k_{i+1})\),
如果它在变换前是顺序,那么它在变换后是逆序,
这时变换后排列的逆序数比变换前排列的逆序数多1,即\begin{equation*}
	\tau(k_1,\dotsc,k_{i+1},k_i,\dotsc,k_n)
	= \tau(k_1,\dotsc,k_i,k_{i+1},\dotsc,k_n) + 1;
\end{equation*}
如果它在变换前是逆序,那么它在变换后是顺序,
这时变换后排列的逆序数比变换前排列的逆序数少1,即\begin{equation*}
	\tau(k_1,\dotsc,k_{i+1},k_i,\dotsc,k_n)
	= \tau(k_1,\dotsc,k_i,k_{i+1},\dotsc,k_n) - 1.
\end{equation*}
因此,在对换的两个数在\(n\)阶排列中相邻的情形下,变换前后排列的奇偶性相反.

再讨论一般情形:
排列\(k_1,\dotsc,k_{i-1},k_i,k_{i+1},\dotsc,k_{j-1},k_j,k_{j+1},\dotsc,k_n\)
对换\(k_i,k_j\)这两个数会得到
\(k_1,\dotsc,k_{i-1},k_j,k_{i+1},\dotsc,k_{j-1},k_i,k_{j+1},\dotsc,k_n\);
由于这次对换可以视作一系列相邻两数的对换,
即“对换\(k_i\)与\(k_{i+1}\)”“对换\(k_{i+1}\)与\(k_{i+2}\)”%
......%
“对换\(k_{j-2}\)与\(k_{j-1}\)”“对换\(k_{j-1}\)与\(k_j\)”,
而这就是作了\(s+1+s=2s+1\)次相邻两数的对换,
像这样奇数次相邻两数的对换回改变排列的奇偶性,也就是说,变换前后排列的奇偶性相反.
\end{proof}
\end{theorem}

\begin{corollary}
排列经奇数次对换后奇偶性改变,经偶数次对换后奇偶性不变.
\end{corollary}
我们可以把这个推论表述为如下形式:
设对换次数为\(s\),变换前后的排列分别为\begin{equation*}
	\AutoTuple{\mu}{n}
	\quad\text{和}\quad
	\AutoTuple{\nu}{n},
\end{equation*}
则\begin{equation*}
	(-1)^{\tau(\AutoTuple{\nu}{n})} = (-1)^s (-1)^{\tau(\AutoTuple{\mu}{n})}.
\end{equation*}

\begin{theorem}\label{theorem:行列式.任意排列可化为自然序}
任意一个\(n\)阶排列\(\AutoTuple{k}{n}\)都可经一系列对换变成自然顺序排列,
且对换的次数\(s\)与\(\tau(\AutoTuple{k}{n})\)同奇偶,即\begin{equation*}
	(-1)^s = (-1)^{\tau(\AutoTuple{k}{n})}.
\end{equation*}
\begin{proof}
设\(n\)阶排列\(\AutoTuple{k}{n}\)经过\(s\)次对换变成\(1,2,\dotsc,n\).
考虑到\(1,2,\dotsc,n\)是偶排列,
因此,如果\(\AutoTuple{k}{n}\)是奇排列,则\(s\)必为奇数,才能把奇排列变成偶排列;
如果\(\AutoTuple{k}{n}\)是偶排列,则\(s\)必为偶数,才能保持排列的奇偶性不变.
显然,如果\(n\)阶排列\(\AutoTuple{k}{n}\)经过\(s\)次对换变成自然排列\(1,2,\dotsc,n\),
那么\(1,2,\dotsc,n\)经过上述\(n\)次对换(次序相反)就变成排列\(\AutoTuple{k}{n}\).
\end{proof}
\end{theorem}

\begin{example}
证明:在全部\(n\)阶排列中,奇偶排列各占一半.
%TODO
\end{example}

\begin{example}
试证:\(\tau(\AutoTuple{i}{n})+\tau(i_n,i_{n-1},\dotsc,i_1)=\frac{n(n-1)}{2}\).
\begin{proof}
记\(I=\Set{i_1,i_2,\dotsc,i_{n-1},i_n}\).
对于\(\forall p,q \in I\),
根据排列的定义必有\(p \neq q\),即有\(p<q\)或\(p>q\)成立.
因此,对于数对\((i_k,i_{k+1})\)和\((i_{k+1},i_k)\),
有且仅有以下两种情况:\begin{itemize}
	\item \(i_k,i_{k+1}\)构成一个顺序,\(i_{k+1},i_k\)构成一个逆序;
	\item \(i_k,i_{k+1}\)构成一个逆序,\(i_{k+1},i_k\)构成一个顺序.
\end{itemize}
不论是哪种情况,都有\(\tau(i_k,i_{k+1})+\tau(i_{k+1},i_k)=1\).

由上可知,排列\((i_1,i_2,\dotsc,i_{n-1},i_n)\)
与\((i_n,i_{n-1},\dotsc,i_2,i_1)\)的逆序数之和
相当于是从集合\(I\)中任取两个数构成一个逆序的取法,
那么\begin{equation*}
	\tau(i_1,i_2,\dotsc,i_{n-1},i_n)+\tau(i_n,i_{n-1},\dotsc,i_2,i_1)
	= C_n^2
	= \frac{n(n-1)}{2}.
	\qedhere
\end{equation*}
\end{proof}
\end{example}

\section{行列式}
\subsection{行列式的概念}
\begin{definition}
设\begin{equation*}
	\vb{A} = \begin{bmatrix}
		a_{11} & a_{12} & \dots & a_{1n} \\
		a_{21} & a_{22} & \dots & a_{2n} \\
		\vdots & \vdots & & \vdots \\
		a_{n1} & a_{n2} & \dots & a_{nn}
	\end{bmatrix}
\end{equation*}是数域\(K\)上的一个\(n\)阶方阵.
从矩阵\(\vb{A}\)中取出不同行又不同列的\(n\)个元素作乘积
\begin{equation}\label{equation:行列式.行列式的项1}
	(-1)^{\tau(\AutoTuple{j}{n})}
	a_{1 j_1} a_{2 j_2} \dotsm a_{n j_n},
\end{equation}
构成一项;%
我们可以像这样构造\(n!\)项,
并且称这\(n!\)项之和\begin{equation*}
	\sum_{\AutoTuple{j}{n}}
	(-1)^{\tau(\AutoTuple{j}{n})}
	a_{1 j_1} a_{2 j_2} \dotsm a_{n j_n}
\end{equation*}为“矩阵\(\vb{A}\)的\DefineConcept{行列式}(determinant)”,
%@see: https://mathworld.wolfram.com/Determinant.html
记作\begin{equation*}
	\begin{vmatrix}
		a_{11} & a_{12} & \dots & a_{1n} \\
		a_{21} & a_{22} & \dots & a_{2n} \\
		\vdots & \vdots & & \vdots \\
		a_{n1} & a_{n2} & \dots & a_{nn}
	\end{vmatrix},
\end{equation*}或\(\det\vb{A}\),或\(\abs{\vb{A}}\);
即
\begin{equation}\label{equation:行列式.行列式的定义式}
	\begin{vmatrix}
		a_{11} & a_{12} & \dots & a_{1n} \\
		a_{21} & a_{22} & \dots & a_{2n} \\
		\vdots & \vdots & & \vdots \\
		a_{n1} & a_{n2} & \dots & a_{nn}
	\end{vmatrix}
	\defeq
	\sum_{\AutoTuple{j}{n}}
	(-1)^{\tau(\AutoTuple{j}{n})}
	a_{1 j_1} a_{2 j_2} \dotsm a_{n j_n}.
\end{equation}
这里,求和指标\(\AutoTuple{j}{n}\)表示遍取所有\(n\)阶排列.

我们称\cref{equation:行列式.行列式的定义式}
为“行列式\(\abs{A}\)的\DefineConcept{完全展开式}”.
\end{definition}

特别地,
一阶行列式为
\begin{equation}
	\begin{vmatrix} a \end{vmatrix} = a.
\end{equation}

二阶行列式为
\begin{equation}
	\begin{vmatrix}
		a_{11} & a_{12} \\
		a_{21} & a_{22}
	\end{vmatrix}
	= a_{11} a_{22} - a_{12} a_{21}.
\end{equation}

三阶行列式为
\begin{equation}
	\begin{vmatrix}
		a_{11} & a_{12} & a_{13} \\
		a_{21} & a_{22} & a_{23} \\
		a_{31} & a_{32} & a_{33}
	\end{vmatrix}
	= \begin{array}[t]{l}
		(a_{11} a_{22} a_{33} + a_{12} a_{23} a_{31} + a_{13} a_{21} a_{32} \\
		\hspace{20pt}
		- a_{13} a_{22} a_{31} - a_{12} a_{21} a_{33} - a_{11} a_{23} a_{32})
	\end{array}.
\end{equation}

我们还可以用数学归纳法证明以下两条公式:
\begin{gather}
	\begin{vmatrix}
		a_{11} & a_{12} & \dots & a_{1n} \\
		& a_{22} & \dots & a_{2n} \\
		& & \ddots & \vdots \\
		& & & a_{nn}
	\end{vmatrix}
	= a_{11} a_{22} \dotsm a_{nn}, \\%
	\begin{vmatrix}
		& & & & a_{1n} \\
		& & & a_{2,n-1} & a_{2n} \\
		& & & \vdots & \vdots \\
		& a_{n-1,2} & \dots & a_{n-1,n-1} & a_{n-1,n} \\
		a_{n1} & a_{n2} & \dots & a_{n,n-1} & a_{nn}
	\end{vmatrix}
	=(-1)^{\frac{1}{2}n(n-1)} a_{1n} a_{2,n-1} \dotsm a_{n-1,2} a_{n1}.
\end{gather}

\begin{lemma}
设\(\vb{A}=(a_{ij})_n\),而\(\AutoTuple{i}{n}\)与\(\AutoTuple{j}{n}\)是两个\(n\)阶排列,则
\begin{equation}\label{equation:行列式.行列式的项2}
	(-1)^{\tau(\AutoTuple{i}{n})+\tau(\AutoTuple{j}{n})}
	a_{i_1j_1} a_{i_2j_2} \dotsm a_{i_nj_n}
\end{equation}
是\(\abs{\vb{A}}\)的项.
\begin{proof}
由乘法交换律,\cref{equation:行列式.行列式的项2} 可以经过\(s\)次互换两个因子的次序写成\begin{equation*}
(-1)^{\tau(\AutoTuple{i}{n})+\tau(\AutoTuple{j}{n})}
	a_{1 l_1} a_{2 l_2} \dotsm a_{n l_n},
\end{equation*}其中,\(\AutoTuple{l}{n}\)是一个\(n\)阶排列.

同时,行标排列\(\AutoTuple{i}{n}\)与列标排列\(\AutoTuple{j}{n}\)
分别经过\(s\)次对换变到\(1,2,\dotsc,n\)与\(\AutoTuple{l}{n}\),
而它们的奇偶性都分别改变了\(s\)次,总共改变了\(2s\)次(偶数次),故\begin{equation*}
	(-1)^{\tau(\AutoTuple{i}{n})+\tau(\AutoTuple{j}{n})}
	= (-1)^{\tau(1,2,\dotsc,n)+\tau(\AutoTuple{l}{n})}
	= (-1)^{\tau(\AutoTuple{l}{n})},
\end{equation*}这说明\cref{equation:行列式.行列式的项2} 是行列式\(\abs{\vb{A}}\)的项.
\end{proof}
\end{lemma}

\begin{corollary}
给定行指标的一个排列\(\AutoTuple{i}{n}\),则\(n\)阶矩阵\(\vb{A}\)的行列式为
\begin{equation}\label{equation:行列式.给定行指标排列下的行列式的完全展开式}
\abs{\vb{A}}
= \sum_{\AutoTuple{j}{n}}
(-1)^{\tau(\AutoTuple{i}{n})+\tau(\AutoTuple{j}{n})}
a_{i_1 j_1} a_{i_2 j_2} \dotsm a_{i_n j_n};
\end{equation}
或者给定列指标的一个排列\(\AutoTuple{j}{n}\),则\(n\)阶矩阵\(\vb{A}\)的行列式为
\begin{equation}\label{equation:行列式.给定列指标排列下的行列式的完全展开式}
	\abs{\vb{A}}
	= \sum_{\AutoTuple{i}{n}}
	(-1)^{\tau(\AutoTuple{i}{n})+\tau(\AutoTuple{j}{n})}
	a_{i_1 j_1} a_{i_2 j_2} \dotsm a_{i_n j_n}.
\end{equation}

特别地,\(n\)阶行列式\(\abs{\vb{A}}\)的每一项可以按列指标成自然序排好位置,
这时用行指标所成排列的奇偶性来决定该项前面所带的符号,即
\begin{equation}\label{equation:行列式.给定列指标为自然序下行列式的完全展开式}
	\abs{\vb{A}} =
	\sum_{\AutoTuple{i}{n}}
	(-1)^{\tau(\AutoTuple{i}{n})}
	a_{i_1 1} a_{i_2 2} \dotsm a_{i_n n}.
\end{equation}
\end{corollary}

\begin{example}
若\(n\)阶行列式\(\det\vb{A}\)中为零的元多于\(n^2-n\)个,证明:\(\det\vb{A}=0\).
%TODO
\end{example}

\begin{example}
证明:如果\(n\ (n\geq2)\)阶矩阵\(\vb{A}\)的元素为\(1\)或\(-1\),则\(\abs{\vb{A}}\)必为偶数.
%TODO
\end{example}

\subsection{行列式的性质}
\begin{property}\label{theorem:行列式.性质1}
设\(\vb{A} \in M_n(K)\),则\(\det\vb{A} = \det\vb{A}^T\).
\begin{proof}
由\cref{equation:行列式.行列式的定义式,equation:行列式.给定列指标为自然序下行列式的完全展开式}
立即可得.
\end{proof}
\end{property}
这就说明,行列互换,行列式的值不变.

\begin{property}\label{theorem:行列式.性质2}
设\(\AutoTuple{\vb\alpha}{n} \in K^n\),\(k \in K\).
那么\begin{equation*}
	\det(\vb\alpha_1,\dotsc,k\vb\alpha_j,\dotsc,\vb\alpha_n)
	= k \cdot \det(\vb\alpha_1,\dotsc,\vb\alpha_j,\dotsc,\vb\alpha_n).
\end{equation*}
\end{property}
这就说明,行列式某一列(或某一行)各元素的公因子可以提到行列式外.

\begin{corollary}\label{theorem:行列式.性质2.推论1}
设\(\AutoTuple{\vb\alpha}{n} \in K^n\),
\(\vb0\)是\(K^n\)的零向量.
那么\begin{equation*}
	\det(\vb\alpha_1,\dotsc,\vb0,\dotsc,\vb\alpha_n) = 0.
\end{equation*}
\end{corollary}
也就是说,如果行列式中某一列(或某一行)元素全为零,则行列式等于零.

\begin{corollary}\label{theorem:行列式.性质2.推论2}
设\(k \in K\),\(\vb{A} \in M_n(K)\).
那么\(\det(k\vb{A}) = k^n \det \vb{A}\).
\end{corollary}

应该注意到,一般说来,\(\det(k\vb{A}) \neq k \det\vb{A}\).

\begin{property}\label{theorem:行列式.性质3}
%@see: 《高等代数(第三版 上册)》(丘维声) P28 性质3
设\(\AutoTuple{\vb\alpha}{n} \in K^n\),且\(\vb\beta,\vb\gamma \in K^n\).
那么\begin{equation*}
	\det(\vb\alpha_1,\dotsc,\vb\beta + \vb\gamma,\dotsc,\vb\alpha_n)
	= \det(\vb\alpha_1,\dotsc,\vb\beta,\dotsc,\vb\alpha_n)
	+ \det(\vb\alpha_1,\dotsc,\vb\gamma,\dotsc,\vb\alpha_n).
\end{equation*}
\begin{proof}
直接计算得
\begin{align*}
	\det(\vb\alpha_1,\dotsc,\vb\beta + \vb\gamma,\dotsc,\vb\alpha_n)
	&= \sum_{\AutoTuple{i}{n}}{
		(-1)^{\tau(\AutoTuple{i}{n})}
		a_{i_1 1} \dotsm (b_{i_j} + c_{i_j}) \dotsm a_{i_n n}
	} \\
	&= \sum_{\AutoTuple{i}{n}}{
		(-1)^{\tau(\AutoTuple{i}{n})}
		a_{i_1 1} \dotsm b_{i_j} \dotsm a_{i_n n}
	} \\
	&\hspace{20pt}+ \sum_{\AutoTuple{i}{n}}{
		(-1)^{\tau(\AutoTuple{i}{n})}
		a_{i_1 1} \dotsm c_{i_j} \dotsm a_{i_n n}
	} \\
	&= \det(\vb\alpha_1,\dotsc,\vb\beta,\dotsc,\vb\alpha_n)
		+ \det(\vb\alpha_1,\dotsc,\vb\gamma,\dotsc,\vb\alpha_n).
	\qedhere
\end{align*}
\end{proof}
\end{property}
注:一般地,\(\det(\vb\alpha_1+\vb\beta_1,\vb\alpha_2+\vb\beta_2,\dotsc,\vb\alpha_n+\vb\beta_n)\)可以拆成\(2^n\)个行列式之和.

\begin{property}\label{theorem:行列式.性质4}
设\(\AutoTuple{\vb\alpha}{n} \in K^n\).
那么\begin{equation*}
	\det(\vb\alpha_1,\dotsc,\vb\alpha_s,\dotsc,\vb\alpha_t,\dotsc,\vb\alpha_n)
	= -\det(\vb\alpha_1,\dotsc,\vb\alpha_t,\dotsc,\vb\alpha_s,\dotsc,\vb\alpha_n).
\end{equation*}
\end{property}
也就是说,交换两列(行),行列式变号.

\begin{property}\label{theorem:行列式.性质5}
设\(\AutoTuple{\vb\alpha}{n} \in K^n\),且\(\vb\alpha \in K^n\),\(k,l \in K\).
那么\begin{equation*}
	\det(\vb\alpha_1,\dotsc,k\vb\alpha,\dotsc,l\vb\alpha,\dotsc,\vb\alpha_n) = 0.
\end{equation*}
\end{property}
这就说明,行列式中若有两列(行)成比例,则行列式等于零.

\begin{property}\label{theorem:行列式.性质6}
设\(\AutoTuple{\vb\alpha}{n} \in K^n\),且\(k \in K\).
那么\begin{equation*}
	\det(\vb\alpha_1,\dotsc,\vb\alpha_s,\dotsc,\vb\alpha_t,\dotsc,\vb\alpha_n)
	= \det(\vb\alpha_1,\dotsc,\vb\alpha_s,\dotsc,\vb\alpha_t + k\vb\alpha_s,\dotsc,\vb\alpha_n).
\end{equation*}
\end{property}
这说明,将一列的\(k\)倍加到另一列,行列式的值不变.

\begin{example}
设\(\vb{A}\)为奇数阶反对称矩阵,即\(\vb{A}^T = -\vb{A}\),则\(\det\vb{A}=0\).
\begin{proof}
假设\(\vb{A} \in M_n(K)\),其中\(n\)是奇数.
因为\(\vb{A}^T = -\vb{A}\),根据行列式的性质,有\begin{equation*}
	\det\vb{A}
	= \det\vb{A}^T
	= \det(-\vb{A})
	= (-1)^n \det\vb{A}
	= -\det\vb{A},
\end{equation*}
于是\(\det\vb{A} = 0\).
\end{proof}
\end{example}

\begin{example}
计算\(n\)阶行列式\begin{equation*}
	D_n = \begin{vmatrix}
		k & \lambda & \lambda & \dots & \lambda \\
		\lambda & k & \lambda & \dots & \lambda \\
		\lambda & \lambda & k & \dots & \lambda \\
		\vdots & \vdots & \vdots & & \vdots \\
		\lambda & \lambda & \lambda & \dots & k
	\end{vmatrix},
	\quad k\neq\lambda.
\end{equation*}
\begin{solution}
当\(n>1\)时,有\begin{align*}
	D_n &= \begin{vmatrix}
		k+(n-1)\lambda & \lambda & \lambda & \dots & \lambda \\
		k+(n-1)\lambda & k & \lambda & \dots & \lambda \\
		k+(n-1)\lambda & \lambda & k & \dots & \lambda \\
		\vdots & \vdots & \vdots & & \vdots \\
		k+(n-1)\lambda & \lambda & \lambda & \dots & k
	\end{vmatrix} \\
	&= [k+(n-1)\lambda] \begin{vmatrix}
		1 & \lambda & \lambda & \dots & \lambda \\
		1 & k & \lambda & \dots & \lambda \\
		1 & \lambda & k & \dots & \lambda \\
		\vdots & \vdots & \vdots & & \vdots \\
		1 & \lambda & \lambda & \dots & k
	\end{vmatrix} \\
	&= [k+(n-1)\lambda] \begin{vmatrix}
		1 & \lambda & \lambda & \dots & \lambda \\
		0 & k-\lambda & 0 & \dots & 0 \\
		0 & 0 & k-\lambda & \dots & 0 \\
		\vdots & \vdots & \vdots & & \vdots \\
		0 & 0 & 0 & \dots & k-\lambda
	\end{vmatrix} \\
	&= [k+(n-1)\lambda] (k-\lambda)^{n-1}.
	\tag1
\end{align*}

当\(n=1\)时,\(D_1 = k\)符合(1)式.
\end{solution}
\end{example}

\begin{example}\label{example:行列式.两个向量的乘积矩阵的行列式}
设\(\vb\alpha=(\AutoTuple{a}{n})^T,
\vb\beta=(\AutoTuple{b}{n})^T\)是\(n\)维列向量.
求:\(\abs{\vb\alpha\vb\beta^T}\).
\begin{solution}
根据\cref{theorem:行列式.性质2},
有\begin{align*}
	\abs{\vb\alpha\vb\beta^T} = \begin{vmatrix}
		a_1 b_1 & a_1 b_2 & \dots & a_1 b_n \\
		a_2 b_1 & a_2 b_2 & \dots & a_2 b_n \\
		\vdots & \vdots & & \vdots \\
		a_n b_1 & a_n b_2 & \dots & a_n b_n
	\end{vmatrix}
	= a_1 a_2 \dotsm a_n \cdot \begin{vmatrix}
		b_1 & b_2 & \dots & b_n \\
		b_1 & b_2 & \dots & b_n \\
		\vdots & \vdots & & \vdots \\
		b_1 & b_2 & \dots & b_n
	\end{vmatrix}.
\end{align*}
而\begin{equation*}
\begin{vmatrix}
	b_1 & b_2 & \dots & b_n \\
	b_1 & b_2 & \dots & b_n \\
	\vdots & \vdots & & \vdots \\
	b_1 & b_2 & \dots & b_n
\end{vmatrix}
\end{equation*}各行成比例,
根据\cref{theorem:行列式.性质5},那么该行列式等于0,可知\(\abs{\vb\alpha\vb\beta^T} = 0\).
\end{solution}
%\cref{theorem:矩阵乘积的秩.多行少列矩阵与少行多列矩阵的乘积的行列式}
\end{example}

\section{行列式按行(或列)展开及其计算}
\subsection{子式}
\begin{definition}
在矩阵\(\A=(a_{ij})_{s \times n}\)中,
任取\(k\)行\(k\)列,
位于这些行与列交叉处的\(k^2\)个元素,按原顺序排成的\(k\)阶矩阵的行列式\[
	\begin{vmatrix}
		a_{i_1,j_1} & a_{i_1,j_2} & \dots & a_{i_1,j_k} \\
		a_{i_2,j_1} & a_{i_2,j_2} & \dots & a_{i_2,j_k} \\
		\vdots & \vdots & & \vdots \\
		a_{i_k,j_1} & a_{i_k,j_2} & \dots & a_{i_k,j_k}
	\end{vmatrix},
	\quad
	\begin{array}{c}
		1 \leq i_1 < i_2 < \dotsb < i_k \leq s; \\
		1 \leq j_1 < j_2 < \dotsb < j_k \leq n
	\end{array}
\]称为“矩阵\(\A\)的一个\(k\)阶\DefineConcept{子式}(minor)”,
%@see: https://mathworld.wolfram.com/Minor.html
记作\[
	\MatrixMinor\A{
		\AutoTuple{i}{k} \\
		\AutoTuple{j}{k}
	}.
\]
如果进一步有\[
	\MatrixMinor\A{
		\AutoTuple{i}{k} \\
		\AutoTuple{j}{k}
	}
	\neq 0,
\]
则称之为“矩阵\(\A\)的一个\(k\)阶\DefineConcept{非零子式}(nonzero minor)”.
\end{definition}

\begin{property}
设矩阵\(\A = (a_{ij})_{s \times n}\).
如果存在\(r < \min\{s,n\}\),
使得所有\(r\)阶子式都等于零,
则对任意\(k > r\)有\(\A\)的所有\(k\)阶子式全为零.
\end{property}

\subsection{主子式,顺序主子式}
\begin{definition}
设\(\A=(a_{ij})_n\),
\(k\)阶子式\[
	\MatrixMinor\A{
		i_1,i_2,\dotsc,i_k \\
		i_1,i_2,\dotsc,i_k
	}
\]
称为“\(\A\)的\(k\)阶\DefineConcept{主子式}(principal minor)”.

\(\A\)位于左上角的\(k\)阶主子式\[
	\MatrixMinor\A{
		1,2,\dotsc,k \\
		1,2,\dotsc,k
	}
\]称为“\(\A\)的\(k\)阶\DefineConcept{顺序主子式}(ordinal principal minor)”.
\end{definition}

\subsection{余子式、代数余子式}
\begin{definition}
在\(n\)阶矩阵\(\A=(a_{ij})_n\)中,
称子式\[
	\MatrixMinor\A{
		\AutoTuple{\mu}{n-k} \\
		\AutoTuple{\nu}{n-k}
	},
\]为“子式\(\MatrixMinor\A{
	\AutoTuple{i}{k} \\
	\AutoTuple{j}{k}
}\)的\DefineConcept{余子式}(cofactor)”,
其中\[
	\Set{ \AutoTuple{\mu}{n-k} } = \Set{ 1,2,\dotsc,n } - \Set{ \AutoTuple{i}{k} },
\]\[
	\Set{ \AutoTuple{\nu}{n-k} } = \Set{ 1,2,\dotsc,n } - \Set{ \AutoTuple{j}{k} },
\]
且\(\mu_1<\mu_2<\dotsb<\mu_{n-k},
\nu_1<\nu_2<\dotsb<\nu_{n-k}\).

把\[
	(-1)^{i_1+\dotsb+i_k+j_1+\dotsb+j_k}
	\MatrixMinor\A{
		\AutoTuple{\mu}{n-k} \\
		\AutoTuple{\nu}{n-k}
	}
\]称作“子式\(\MatrixMinor\A{
	\AutoTuple{i}{k} \\
	\AutoTuple{j}{k}
}\)的\DefineConcept{代数余子式}(algebraic cofactor)”.

特别地,称子式\[
	\MatrixMinor\A{
		1,\dotsc,i-1,i+1,\dotsc,n \\
		1,\dotsc,j-1,j+1,\dotsc,n
	}
\]为“元素\(a_{ij}\)的\DefineConcept{余子式}”,记作\(M_{ij}\).
又称\[
(-1)^{i+j} M_{ij}
\]为“\(a_{ij}\)的\DefineConcept{代数余子式}”,记作\(A_{ij}\).
\end{definition}

\subsection{伴随矩阵}
\begin{definition}\label{definition:伴随矩阵.伴随矩阵的定义}
设\(\A=(a_{ij})_n\),
\(A_{ij}\)为元素\(a_{ij}\ (i,j=1,2,\dotsc,n)\)的代数余子式.
以\(A_{ij}\)作为第\(j\)行第\(i\)列元素构成的\(n\)阶矩阵,
称为“\(\A\)的\DefineConcept{伴随矩阵}(adjoint, adjugate matrix)”,
记为\(\A^*\),即\[
	\A^*
	\defeq
	\begin{bmatrix}
		A_{11} & A_{21} & \dots & A_{n1} \\
		A_{12} & A_{22} & \dots & A_{n2} \\
		\vdots & \vdots & & \vdots \\
		A_{1n} & A_{2n} & \dots & A_{nn}
	\end{bmatrix}.
\]
\end{definition}

\begin{example}
设\(\A=(a_{ij})_n\)的伴随矩阵为\(\A^*\),求\((k\A)^*\).
\begin{solution}
设\(\A\)的元素\(a_{ij}\)的代数余子式是\(A_{ij}\),
那么矩阵\(k\A = (b_{ij})_n\)的元素\(b_{ij} = k a_{ij}\)的代数余子式是\[
	B_{ij}
	= (-1)^{i+j}
	\begin{vmatrix}
		k a_{11} & \dots & k a_{1,j-1} & k a_{1,j+1} & \dots & k a_{1n} \\
		\vdots & & \vdots & \vdots & & \vdots \\
		k a_{i-1,1} & \dots & k a_{i-1,j-1} & k a_{i-1,j+1} & \dots & k a_{i-1,n} \\
		k a_{i+1,1} & \dots & k a_{i+1,j-1} & k a_{i+1,j+1} & \dots & k a_{i+1,n} \\
		\vdots & & \vdots & \vdots & & \vdots \\
		k a_{n1} & \dots & k a_{n,j-1} & k a_{n,j+1} & \dots & k a_{nn} \\
	\end{vmatrix}
	= k^{n-1} A_{ij}.
\]
因此\(k\A\)的伴随矩阵是\((B_{ji})_n\),
即\begin{equation}\label{equation:行列式.伴随矩阵.数与矩阵乘积的伴随}
	(k \A)^* = k^{n-1} \A^*.
\end{equation}
\end{solution}
%\cref{theorem:逆矩阵.数与矩阵乘积的逆}
\end{example}

\begin{example}
证明:对角矩阵的伴随矩阵仍是对角矩阵.
%TODO proof
\end{example}

\subsection{行列式按一行(或一列)展开}
\begin{theorem}\label{theorem:行列式.行列式按行展开}
设\(\A=(a_{ij})_n\),
\(A_{ij}\)为\(a_{ij}\ (i,j=1,2,\dotsc,n)\)的代数余子式.
\begin{enumerate}
	\item 行列式等于它的任一行的各元与其代数余子式乘积之和,
	即\begin{equation}
		\abs{\A} = \sum_{j=1}^n a_{ij} A_{ij}
		\quad(i=1,2,\dotsc,n).
	\end{equation}

	\item 行列式的任一行的各元与另一行对应元素的代数余子式乘积之和为零,
	即\begin{equation}
		\sum_{j=1}^n a_{ij} A_{kj} = 0
		\quad(i \neq k;
		i,k=1,2,\dotsc,n).
	\end{equation}
\end{enumerate}
\begin{proof}
注意\[
	\tau(i,1,2,\dotsc,i-1,i+1,\dotsc,n) = i-1,
\]\[
	\tau(j,j_1,j_2,\dotsc,j_{i-1},j_{i+1},\dotsc,j_n)
	= j-1+\tau(j_1,j_2,\dotsc,j_{i-1},j_{i+1},\dotsc,j_n),
\]于是有\begin{align*}
	\abs{\A}
	&= \sum_{j,j_1,j_2,\dotsc,j_{i-1},j_{i+1},\dotsc,j_n}
		(-1)^{\tau(i,1,2,\dotsc,i-1,i+1,\dotsc,n) + \tau(j,j_1,j_2,\dotsc,j_{i-1},j_{i+1},\dotsc,j_n)}
		a_{ij} \prod_{\substack{k=1 \\ k \neq i}}^n a_{k j_k} \\
	&= \sum_{j=1}^n a_{ij} (-1)^{(i-1)+(j-1)}
		\sum_{j_1,j_2,\dotsc,j_{i-1},j_{i+1},\dotsc,j_n}
			(-1)^{\tau(j_1,j_2,\dotsc,j_{i-1},j_{i+1},\dotsc,j_n)}
				\prod_{\substack{k=1 \\ k \neq i}}^n a_{k j_k} \\
	&= \sum_{j=1}^n a_{ij} (-1)^{i+j} M_{ij}
	= \sum_{j=1}^n a_{ij} A_{ij}.
	\qedhere
\end{align*}
\end{proof}
\end{theorem}

由于行列式中行与列的地位平等,因此又可以行列式按某一列展开.
\begin{theorem}
设\(\A=(a_{ij})_n\),\(A_{ij}\)为\(a_{ij}\ (i,j=1,2,\dotsc,n)\)的代数余子式.
\begin{enumerate}
	\item 行列式等于它的任一列的各元与其代数余子式乘积之和,即\begin{equation}
		\abs{\A} = \sum_{i=1}^n a_{ij} A_{ij}
		\quad(j=1,2,\dotsc,n).
	\end{equation}

	\item 行列式的任一列的各元与另一列对应元素的代数余子式乘积之和为零,即\begin{equation}
		\sum_{i=1}^n a_{ij} A_{ik} = 0
		\quad(j \neq k;
		j,k=1,2,\dotsc,n).
	\end{equation}
\end{enumerate}
\end{theorem}

我们可以将上述两个定理中的公式分别改写成以下形式:
\begin{gather}
	\sum_{j=1}^n a_{ij} A_{kj}
	= \left\{ \begin{array}{cl}
		\abs{\A}, & k = i, \\
		0, & k \neq i,
	\end{array} \right.
	\quad i=1,2,\dotsc,n, \\
	\sum_{i=1}^n a_{ij} A_{ik}
	= \left\{ \begin{array}{cl}
		\abs{\A}, & k = j, \\
		0, & k \neq j,
	\end{array} \right.
	\quad j=1,2,\dotsc,n,
\end{gather}

\begin{theorem}
设方阵\(\A\)的伴随矩阵为\(\A^*\),
则\begin{gather}
	\A \A^* = \A^* \A = \abs{\A} \E, \label{equation:行列式.伴随矩阵.恒等式1} \\
	(\A^*)^T = (\A^T)^*, \label{equation:行列式.伴随矩阵.恒等式2} \\
	(\A \B)^* = \B^* \A^*. \label{equation:行列式.伴随矩阵.恒等式3}
\end{gather}
\begin{proof}
这里只证\cref{equation:行列式.伴随矩阵.恒等式1}.
设\(\A=(a_{ij})_n\),\(\A^*=(\A_{ji})_n\).
那么\begin{align*}
	\A\A^*
	&= \begin{bmatrix}
		a_{11} & a_{12} & \dots & a_{1n} \\
		a_{21} & a_{22} & \dots & a_{2n} \\
		\vdots & \vdots & & \vdots \\
		a_{n1} & a_{n2} & \dots & a_{nn} \\
	\end{bmatrix}
	\begin{bmatrix}
		A_{11} & A_{21} & \dots & A_{n1} \\
		A_{12} & A_{22} & \dots & A_{n2} \\
		\vdots & \vdots & & \vdots \\
		A_{1n} & A_{2n} & \dots & A_{nn}
	\end{bmatrix} \\
	&= \begin{bmatrix}
		\abs{\A} & 0 & \dots & 0 \\
		0 & \abs{\A} & \dots & 0 \\
		\vdots & \vdots & & \vdots \\
		0 & 0 & \dots & \abs{\A}
	\end{bmatrix}
	= \abs{\A} \E.
\end{align*}
利用对称性,立即可得\(\A^*\A\)也等于\(\abs{\A} \E\).
\end{proof}
\end{theorem}

\begin{example}
设\(\A \in M_n(K)\)是对称矩阵,
\(\A^T\)是\(\A\)的转置矩阵.
证明:\(\A\)的伴随矩阵\(\A^*\)也是对称矩阵.
\begin{proof}
因为\(\A^T = \A\),
所以\((\A^T)^* = \A^*\).
又因为 \hyperref[equation:行列式.伴随矩阵.恒等式2]{\((\A^*)^T = (\A^T)^*\)},
所以\(\A^* = (\A^*)^T\).
\end{proof}
\end{example}
\begin{example}
设\(\A \in M_n(K)\)是反对称矩阵,
\(\A^T\)是\(\A\)的转置矩阵.
讨论\(\A\)的伴随矩阵\(\A^*\)的对称性.
\begin{proof}
因为\(\A^T = -\A\),
所以\((\A^T)^* = (-\A)^*\).
由\cref{equation:行列式.伴随矩阵.数与矩阵乘积的伴随} 可知
\((-\A)^* = (-1)^{n-1} \A^*\).
又因为 \hyperref[equation:行列式.伴随矩阵.恒等式2]{\((\A^*)^T = (\A^T)^*\)},
所以\((\A^*)^T = (-1)^{n-1} \A^*\).
因此,当\(n\)是奇数时,\(\A^*\)是对称矩阵;
当\(n\)是偶数时,\(\A^*\)是反对称矩阵.
\end{proof}
\end{example}

\begin{example}
%@see: 《线性代数》(张慎语、周厚隆) P34 例3
试证范德蒙德行列式:
\begin{equation}\label{equation:行列式.范德蒙德行列式}
	V_n = \begin{vmatrix}
		1 & 1 & 1 & \dots & 1 \\
		x_1 & x_2 & x_3 & \dots & x_n \\
		x_1^2 & x_2^2 & x_3^2 & \dots & x_n^2 \\
		\vdots & \vdots & \vdots& & \vdots \\
		x_1^{n-1} & x_2^{n-1} & x_3^{n-1} & \dots & x_n^{n-1}
	\end{vmatrix}
	= \prod_{1 \leq j < i \leq n}(x_i-x_j).
\end{equation}
\begin{proof}
利用数学归纳法.
当\(n=2\)时,\(V_2 = \begin{vmatrix}
	1 & 1 \\ x_1 & x_2
\end{vmatrix} = x_2 - x_1\),结论成立.

假设\(n=k-1\)时,结论成立;
那么当\(n=k\)时,在\(V_k\)中,
依次将第\(k-1\)行的\(-x_1\)倍加到第\(k\)行,
将第\(k-2\)行的\(-x_1\)倍加到第\(k-1\)行,
以此类推,直到最后将第1行的\(-x_1\)倍加到第\(2\)行,得\[
	V_k = \begin{vmatrix}
		1 & 1 & 1 & \dots & 1 \\
		0 & x_2 - x_1 & x_3 - x_1 & \dots & x_k - x_1 \\
		0 & x_2(x_2 - x_1) & x_3(x_3 - x_1) & \dots & x_k(x_k - x_1) \\
		\vdots & \vdots & \vdots & & \vdots \\
		0 & x_2^{k-2}(x_2 - x_1) & x_3^{k-2}(x_3 - x_1) & \dots & x_k^{k-2}(x_k - x_1) \\
	\end{vmatrix};
\]
按第1列展开,得\[
	V_k = 1 \cdot (-1)^{1+1} \cdot \begin{vmatrix}
		x_2 - x_1 & x_3 - x_1 & \dots & x_k - x_1 \\
		x_2(x_2 - x_1) & x_3(x_3 - x_1) & \dots & x_k(x_k - x_1) \\
		\vdots & \vdots & & \vdots \\
		x_2^{k-2}(x_2 - x_1) & x_3^{k-2}(x_3 - x_1) & \dots & x_k^{k-2}(x_k - x_1) \\
	\end{vmatrix};
\]
提取各列公因子,得
\begin{align*}
	V_k &= (x_2 - x_1)(x_3 - x_1)\dotsm(x_k - x_1) \begin{vmatrix}
		1 & 1 & \dots & 1 \\
		x_2 & x_3 & \dots & x_k \\
		\vdots & \vdots & & \vdots \\
		x_2^{k-2} & x_3^{k-2} & \dots & x_k^{k-2} \\
	\end{vmatrix} \\
	&= (x_2 - x_1)(x_3 - x_1)\dotsm(x_k - x_1)
		\prod_{2 \leq j < i \leq k}(x_i - x_j) \\
	&= \prod_{1 \leq j < i \leq k}(x_i - x_j).
	\qedhere
\end{align*}
\end{proof}
\end{example}
我们可以从范德蒙德行列式的表达式 \labelcref{equation:行列式.范德蒙德行列式} 看出,
\(n\)阶范德蒙德行列式\(V_n\)等于零的充分必要条件是:
\(\AutoTuple{x}{n}\)这\(n\)个数中至少有两个相等,即\[
	(\exists b,c\in\Set{\AutoTuple{x}{n}})[b=c].
\]
因此,如果\(\AutoTuple{x}{n}\)两两不等,则范德蒙德行列式不等于零.

\begin{example}\label{example:行列式的展开.三对角行列式}
%@see: 《线性代数》(张慎语、周厚隆) P35 例4
计算\(n\)阶三对角行列式\[
	D_n = \begin{vmatrix}
		a+b & ab & \\
		1 & a+b & ab & \\
		& 1 & a + b & \ddots & \\
		& & \ddots & \ddots & ab \\
		& & & 1 & a+b \\
	\end{vmatrix}_n.
\]
\begin{proof}
将\(D_n\)按第一行展开,得\begin{align*}
	D_n
	= (a+b) D_{n-1} - ab \begin{vmatrix}
		1 & ab \\
		0 & a+b & ab \\
		& 1 & a+b & \ddots \\
		& & \ddots & \ddots & ab \\
		& & & 1 & a+b \\
	\end{vmatrix}_{n-1};
\end{align*}
对于上式等号右边第二项中的行列式,
再按第一列展开,得\begin{equation*}
	\begin{vmatrix}
		1 & ab \\
		0 & a+b & ab \\
		& 1 & a+b & \ddots \\
		& & \ddots & \ddots & ab \\
		& & & 1 & a+b \\
	\end{vmatrix}_{n-1}
	= \begin{vmatrix}
		a+b & ab \\
		1 & a+b & \ddots \\
		& \ddots & \ddots & ab \\
		& & 1 & a+b \\
	\end{vmatrix}_{n-2};
\end{equation*}
于是\begin{equation*}
	D_n
	= (a+b) D_{n-1} - ab D_{n-2}.
\end{equation*}
把上式改写为\(D_n - a D_{n-1} = b(D_{n-1} - a D_{n-2})\),
继续递推下去,得\begin{align*}
	D_n - a D_{n-1} &= b(D_{n-1} - a D_{n-2}) = b^2(D_{n-2} - a D_{n-3}) \\
	&= \dotsb = b^{n-2}(D_2 - a D_1) \\
	&= b^{n-2} [(a^2 + ab + b^2) - a(a+b)] = b^n,
\end{align*}
所以\[
	D_n - a D_{n-1} = b^n,
\]
又由\(a\)和\(b\)的对称性可得\[
	D_n - b D_{n-1} = a^n.
\]

当\(a \neq b\)时,可解得\[
	D_n = \frac{a^{n+1} - b^{n+1}}{a - b}
	= a^n + a^{n-1} b + a^{n-2} b^2 + \dotsb + a b^{n-1} + b^n.
\]
当\(a = b\)时,由\[
	D_n - a D_{n-1} = a^n
\]可继续递推得\begin{align*}
	D_n &= a D_{n-1} + a^n
	= a(a D_{n-2} + a^{n-1}) + a^n
	= a^2 D_{n-2} + 2 a^n \\
	&= a^3 D_{n-3} + 3 a^n
	= \dotsb
	= (n+1) a^n.
\end{align*}
综上所述,对任意\(a,b\in\mathbb{R}\)都有\[
	D_n = a^n + a^{n-1} b + a^{n-2} b^2 + \dotsb + a b^{n-1} + b^n.
	\qedhere
\]
\end{proof}
\end{example}

% \begin{example}
% 设矩阵\(\A = (a_{ij})_n\),\(A_{ij}\)为\(a_{ij}\)的代数余子式.
% 把\(\A\)的每个元素都加上同一个数\(t\),
% 得到的矩阵记作\(\A(t) = (a_{ij} + t)_n\).
% 证明:\[
% 	\abs{\A(t)}
% 	= \abs{\A} + t \sum_{i=1}^n \sum_{j=1}^n A_{ij}.
% \]
% \begin{proof}
% 对\(\A\)列分块得\[
% 	\A = (\AutoTuple{\a}{n}),
% \]
% 又令\(n\)维列向量\(\b=(t,t,\dotsc,t)^T\),
% 那么根据\cref{theorem:行列式.性质3,theorem:行列式.性质5} 有
% \begin{align*}
% 	\abs{\A(t)}
% 	&= \abs{(\a_1+\b,\a_2+\b,\dotsc,\a_n+\b)} \\
% 	&= \abs{(\a_1,\a_2+\b,\dotsc,\a_n+\b)}
% 		+ \abs{(\b,\a_2+\b,\dotsc,\a_n+\b)} \\
% 	&= \dotsb = \abs{(\a_1,\a_2,\dotsc,\a_n)}
% 		\begin{aligned}[t]
% 			&+ \abs{(\b,\a_2,\dotsc,\a_{n-1},\a_n)} \\
% 			&+ \abs{(\a_1,\b,\dotsc,\a_{n-1},\a_n)} \\
% 			&+ \dotsb
% 			+ \abs{(\a_1,\a_2,\dotsc,\a_{n-1},\b)}
% 		\end{aligned}
% 	\qedhere
% \end{align*}
% \end{proof}
% \end{example}

通常在计算行列式时我们常采用各种方法(例如按行或列展开)降低行列式的阶数,
但是有时候我们也可以反其道而行之,适当地增加行、列,反而可以简化问题.
\begin{example}
%@see: 《高等代数(第四版)》(谢启鸿 姚慕生) P27 例1.31
计算\(n\)阶行列式\[
	\abs{\A} = \begin{vmatrix}
		1 + x_1 & 1 + x_1^2 & \dots & 1 + x_1^n \\
		1 + x_2 & 1 + x_2^2 & \dots & 1 + x_2^n \\
		\vdots & \vdots & & \vdots \\
		1 + x_n & 1 + x_n^2 & \dots & 1 + x_n^n
	\end{vmatrix}.
\]
\begin{solution}
增加一列升阶得到\[
	\abs{\A} = \begin{vmatrix}
		1 & 0 & 0 & \dots & 0 \\
		1 & 1 + x_1 & 1 + x_1^2 & \dots & 1 + x_1^n \\
		1 & 1 + x_2 & 1 + x_2^2 & \dots & 1 + x_2^n \\
		\vdots & \vdots & \vdots & & \vdots \\
		1 & 1 + x_n & 1 + x_n^2 & \dots & 1 + x_n^n
	\end{vmatrix}
	= \begin{vmatrix}
		1 & -1 & -1 & \dots & -1 \\
		1 & x_1 & x_1^2 & \dots & x_1^n \\
		1 & x_2 & x_2^2 & \dots & x_2^n \\
		\vdots & \vdots & \vdots & & \vdots \\
		1 & x_n & x_n^2 & \dots & x_n^n
	\end{vmatrix}.
\]
将第一行拆开,可得\[
	\abs{\A} = \begin{vmatrix}
		2 & 0 & 0 & \dots & 0 \\
		1 & x_1 & x_1^2 & \dots & x_1^n \\
		1 & x_2 & x_2^2 & \dots & x_2^n \\
		\vdots & \vdots & \vdots & & \vdots \\
		1 & x_n & x_n^2 & \dots & x_n^n
	\end{vmatrix}
	+ \begin{vmatrix}
		-1 & -1 & -1 & \dots & -1 \\
		1 & x_1 & x_1^2 & \dots & x_1^n \\
		1 & x_2 & x_2^2 & \dots & x_2^n \\
		\vdots & \vdots & \vdots & & \vdots \\
		1 & x_n & x_n^2 & \dots & x_n^n
	\end{vmatrix}.
\]
上式等号右边第二个行列式只要提出公因子\((-1)\)
就变成一个\hyperref[equation:行列式.范德蒙德行列式]{范德蒙德行列式},
从而可得\[
	\abs{\A}
	= \left[ 2 x_1 x_2 \dotsm x_n - (x_1 - 1)(x_2 - 1)\dotsm(x_n - 1) \right]
	\prod_{1 \leq i < j \leq n} (x_j - x_i).
\]
\end{solution}
\end{example}

\begin{example}
%@see: 《2021年全国硕士研究生入学统一考试(数学一)》二填空题/第15题
设\(\A = (a_{ij})\)是3阶矩阵,
\(A_{ij}\)是元素\(a_{ij}\)的代数余子式.
若\(\A\)的每行元素之和均为\(2\),
且\(\abs{\A} = 3\),
计算\(A_{11} + A_{21} + A_{31}\).
\begin{solution}
由\cref{theorem:行列式.行列式按行展开} 可知
\(\abs{\A} = a_{11} A_{11} + a_{21} A_{21} + a_{31} A_{31}\),
那么\begin{equation*}
	A_{11} + A_{21} + A_{31}
	= \begin{vmatrix}
		1 & a_{12} & a_{13} \\
		1 & a_{22} & a_{23} \\
		1 & a_{32} & a_{33}
	\end{vmatrix}.
\end{equation*}
由\hyperref[theorem:行列式.性质6]{行列式的性质}有\begin{align*}
	3 &= \abs{\A} = \begin{vmatrix}
		a_{11} & a_{12} & a_{13} \\
		a_{21} & a_{22} & a_{23} \\
		a_{31} & a_{32} & a_{33}
	\end{vmatrix} \\
	&= \begin{vmatrix}
		a_{11} + a_{12} + a_{13} & a_{12} & a_{13} \\
		a_{21} + a_{22} + a_{23} & a_{22} & a_{23} \\
		a_{31} + a_{32} + a_{33} & a_{32} & a_{33}
	\end{vmatrix}
	= \begin{vmatrix}
		2 & a_{12} & a_{13} \\
		2 & a_{22} & a_{23} \\
		2 & a_{32} & a_{33}
	\end{vmatrix}
	= 2 \begin{vmatrix}
		1 & a_{12} & a_{13} \\
		1 & a_{22} & a_{23} \\
		1 & a_{32} & a_{33}
	\end{vmatrix},
\end{align*}
所以\(A_{11} + A_{21} + A_{31} = \frac32\).
\end{solution}
\end{example}

\subsection{行列式按\texorpdfstring{\(k\)}{k}行(或\texorpdfstring{\(k\)}{k}列)展开}
\begin{theorem}[拉普拉斯定理]\label{theorem:行列式.拉普拉斯定理}
%@see: 《高等代数(大学高等代数课程创新教材 第二版 下册)》(丘维声) P68 定理1(Laplace定理)
在\(n\)阶矩阵\(\A\)中,
取定第\(\AutoTuple{i}{k}\)行
(\(i_1<i_2<\dotsb<i_k\)
且\(1 \leq k < n\)),
则这\(k\)行元素形成的所有\(k\)阶子式与它们自己的代数余子式的乘积之和等于\(\abs{\A}\),
即\begin{equation}
	\abs{\A} =
	\sum_{1 \leq j_1 < j_2 < \dotsb < j_k \leq n}
	\MatrixMinor\A{
		\AutoTuple{i}{k} \\
		\AutoTuple{j}{k}
	}
	(-1)^{(i_1+\dotsb+i_k)+(j_1+\dotsb+j_k)}
	\MatrixMinor\A{
		\AutoTuple{\mu}{n-k} \\
		\AutoTuple{\nu}{n-k}
	},
\end{equation}
其中\[
	\Set{ \AutoTuple{\mu}{n-k} } = \Set{ 1,2,\dotsc,n } - \Set{ \AutoTuple{i}{k} },
\]\[
	\Set{ \AutoTuple{\nu}{n-k} } = \Set{ 1,2,\dotsc,n } - \Set{ \AutoTuple{j}{k} },
\]
且\(\mu_1<\mu_2<\dotsb<\mu_{n-k},
\nu_1<\nu_2<\dotsb<\nu_{n-k}\).
\begin{proof}
根据\cref{equation:行列式.给定行指标排列下的行列式的完全展开式},
给定\(\abs{\A}\)的行指标排列\(\AutoTuple{i}{k},\AutoTuple{\mu}{n-k}\),
\(\abs{\A}\)的表达式为\begin{align*}
	\abs{\A}
	&= \sum_{\AutoTuple{p}{k},\AutoTuple{q}{n-k}}
	(-1)^{\tau(\AutoTuple{i}{k},\AutoTuple{\mu}{n-k}) + \tau(\AutoTuple{p}{k},\AutoTuple{q}{n-k})}
	a_{i_1 p_1} \dotsm a_{i_k p_k}
	a_{\mu_1 q_1} \dotsm a_{\mu_{n-k} q_{n-k}} \\
	&= \sum_{\AutoTuple{p}{k},\AutoTuple{q}{n-k}}
	(-1)^{(i_1+\dotsb+i_k) - \frac{1}{2}k(k+1) + \tau(\AutoTuple{p}{k},\AutoTuple{q}{n-k})}
	a_{i_1 p_1} \dotsm a_{i_k p_k}
	a_{\mu_1 q_1} \dotsm a_{\mu_{n-k} q_{n-k}}.
\end{align*}

通过任意给定\(\AutoTuple{j}{k}\),
其中\(1 \leq j_1 < j_2 < \dotsb < j_k \leq n\),
可以把\(n!\)个\(n\)阶排列分成\(C_n^k\)组,
对应于\(\AutoTuple{j}{k}\)这一组中的\(n\)阶排列形如\(\AutoTuple{p}{k},\AutoTuple{q}{n-k}\),
其中\(\AutoTuple{p}{k}\)是\(\AutoTuple{j}{k}\)形成的\(k\)阶排列,
\(\AutoTuple{q}{n-k}\)是\(\AutoTuple{\mu}{n-k}\)形成的\(n-k\)阶排列.
于是对于第\(\AutoTuple{i}{k}\)行,
通过任意取定\(k\)列,例如第\(\AutoTuple{j}{k}\)列,
可以把\(\abs{\A}\)的表达式中的\(n!\)项分成\(C_n^k\)组;
再根据\cref{theorem:行列式.任意排列可化为自然序} 有
\begin{align*}
	&\hspace{-20pt}
	(-1)^{\tau(\AutoTuple{p}{k},\AutoTuple{q}{n-k})}
	= (-1)^{\tau(\AutoTuple{p}{k})}
	(-1)^{\tau(\AutoTuple{i}{k},\AutoTuple{q}{n-k})} \\
	&= (-1)^{\tau(\AutoTuple{p}{k})}
	(-1)^{[(i_1-1)+(i_2-2)+\dotsb+(i_k-k)] + \tau(\AutoTuple{q}{n-k})} \\
	&= (-1)^{(i_1+\dotsb+i_k) - \frac{1}{2}k(k+1)}
	(-1)^{\tau(\AutoTuple{p}{k}) + \tau(\AutoTuple{q}{n-k})},
\end{align*}
于是\begin{align*}
	\abs{\A}
	&= \sum_{i \leq j_1 < \dotsb < j_k \leq n}
			\sum_{\AutoTuple{p}{k}}
			\sum_{\AutoTuple{q}{n-k}}
			(-1)^{(i_1+\dotsb+i_k) - \frac{1}{2}k(k+1)}
			(-1)^{(j_1+\dotsb+j_k) - \frac{1}{2}k(k+1)} \\
		&\hspace{40pt}\cdot(-1)^{\tau(\AutoTuple{p}{k}) + \tau(\AutoTuple{q}{n-k})}
			a_{i_1 p_1} \dotsm a_{i_k p_k}
			a_{\mu_1 q_1} \dotsm a_{\mu_{n-k} q_{n-k}} \\
	&= \sum_{i \leq j_1 < \dotsb < j_k \leq n}
		(-1)^{(i_1+\dotsb+i_k)+(j_1+\dotsb+j_k)}
		\biggl\{
			\biggl[
				\sum_{\AutoTuple{p}{k}}
				(-1)^{\tau(\AutoTuple{p}{k})}
				a_{i_1 p_1} \dotsm a_{i_k p_k}
			\biggr] \\
			&\hspace{40pt}\cdot\biggl[
				\sum_{\AutoTuple{q}{n-k}}
				(-1)^{\tau(\AutoTuple{q}{n-k})}
				a_{\mu_1 q_1} \dotsm a_{\mu_{n-k} q_{n-k}}
			\biggr]
		\biggr\} \\
	&= \sum_{i \leq j_1 < \dotsb < j_k \leq n}
		(-1)^{(i_1+\dotsb+i_k)+(j_1+\dotsb+j_k)}
		\MatrixMinor\A{
			\AutoTuple{i}{k} \\
			\AutoTuple{j}{k}
		}
		\MatrixMinor\A{
			\AutoTuple{\mu}{n-k} \\
			\AutoTuple{\nu}{n-k}
		}.
	\qedhere
\end{align*}
\end{proof}
\end{theorem}
\cref{theorem:行列式.拉普拉斯定理} 称为“拉普拉斯定理”或“行列式按\(k\)行展开定理”.

由于行列式中行与列的地位平等,因此也有行列式按\(k\)列展开的定理:
\begin{theorem}\label{theorem:行列式.行列式按k列展开}
%@see: 《高等代数(第三版)》(丘维声) P52 定理2
\(n\)阶行列式\(\abs{\A}\)中,取定\(k\ (1 \leq k < n)\)列,
则这\(k\)列元素形成的所有\(k\)阶子式与它们自己的代数余子式的乘积之和等于\(\abs{\A}\).
\end{theorem}

\section{矩阵乘积的行列式}
\begin{lemma}
%@see: 《线性代数》(张慎语、周厚隆) P38 引理
%@see: 《高等代数(第三版 上册)》(丘维声) P52 例1
设\(\vb{A}\in M_n^*(K),
\vb{B}\in M_m^*(K),
\vb{C}\in M_{m\times n}(K),
\vb{D}\in M_{n\times m}(K)\),
则\begin{align}
	\begin{vmatrix}
		\vb{A} & \vb0 \\
		\vb{C} & \vb{B}
	\end{vmatrix}
	&= \begin{vmatrix}
		\vb{A} & \vb{D} \\
		\vb0 & \vb{B}
	\end{vmatrix}
	= \abs{\vb{A}} \abs{\vb{B}}, \label{equation:行列式.广义三角阵的行列式1} \\
	\begin{vmatrix}
		\vb0 & \vb{A} \\
		\vb{B} & \vb{C}
	\end{vmatrix}
	&= \begin{vmatrix}
		\vb{D} & \vb{A} \\
		\vb{B} & \vb0
	\end{vmatrix}
	= (-1)^{mn} \abs{\vb{A}} \abs{\vb{B}}. \label{equation:行列式.广义三角阵的行列式2}
\end{align}
%TODO proof
\end{lemma}

\begin{theorem}[矩阵乘积的行列式定理]\label{theorem:行列式.矩阵乘积的行列式}
%@see: 《线性代数》(张慎语、周厚隆) P39 定理5(矩阵乘积的行列式定理)
%@see: 《高等代数(第三版 上册)》(丘维声) P123 定理3
设\(\vb{A},\vb{B}\in M_n(K)\),
则\begin{equation}
	\abs{\vb{A} \vb{B}} = \abs{\vb{A}} \abs{\vb{B}}.
\end{equation}
%TODO proof
\end{theorem}

\begin{example}\label{example:幂零矩阵.幂零矩阵的行列式}
证明:幂零矩阵的行列式等于\(0\).
\begin{proof}
设\(\vb{A}\)是数域\(K\)上的\(n\)阶幂零矩阵,\(\vb{A}^m = \vb0\ (m\in\mathbb{N}^+)\).
于是由\cref{theorem:行列式.矩阵乘积的行列式} 可知\begin{equation*}
	\abs{\vb{A}}^m
	= \abs{\vb{A}^m}
	= \abs{\vb0}
	= 0,
\end{equation*}
解得\(\abs{\vb{A}} = 0\).
\end{proof}
\end{example}

\begin{example}\label{example:正交矩阵.行列式小于零的正交矩阵与单位矩阵之和的行列式等于零}
%@see: 《线性代数》(张慎语、周厚隆) P39 例1
%@see: 《1995年全国硕士研究生入学统一考试(数学一)》九解答题
设矩阵\(\vb{A} \in M_n(K)\),
\(\vb{E}\)是数域\(K\)上的\(n\)阶单位矩阵,
且\(\vb{A} \vb{A}^T = \vb{E}\),\(\abs{\vb{A}}<0\).
证明:\begin{equation*}
	\abs{\vb{E}+\vb{A}}=0.
\end{equation*}
\begin{proof}
等式\(\vb{A} \vb{A}^T=\vb{E}\)的两端取行列式,得\begin{equation*}
	\abs{\vb{A}} \abs{\vb{A}^T}
	= \abs{\vb{E}}.
\end{equation*}
由\(\abs{\vb{A}} = \abs{\vb{A}^T}\)可知\begin{equation*}
	\abs{\vb{A}} \abs{\vb{A}^T}
	= \abs{\vb{A}}^2,
\end{equation*}
从而有\begin{equation*}
	\abs{\vb{A}}^2 = \abs{\vb{E}}.
\end{equation*}
由于\(\abs{\vb{A}} < 0\),
所以\begin{equation*}
	\abs{\vb{A}} = -1.
\end{equation*}
于是\begin{align*}
	\abs{\vb{E}+\vb{A}}
	&= \abs{(\vb{A} \vb{A}^T)+(\vb{A} \vb{E})}
	= \abs{\vb{A}(\vb{A}^T+\vb{E})}
	= \abs{\vb{A}} \abs{\vb{A}^T+\vb{E}} \\
	&= -1 \cdot \abs{(\vb{A}^T+\vb{E})^T}
	= -\abs{\vb{A}+\vb{E}},
\end{align*}
因此\(\abs{\vb{E}+\vb{A}}=0\).
\end{proof}
\end{example}

\begin{example}
设矩阵\(\vb{A},\vb{B}\)满足\(\abs{\vb{A}}=3,\abs{\vb{B}}=2,\abs{\vb{A}^{-1}+\vb{B}}=2\),
试求\(\abs{\vb{A}+\vb{B}^{-1}}\).
\begin{solution}
由于\(\abs{\vb{E}+\vb{A}\vb{B}} = \abs{\vb{A}(\vb{A}^{-1}+\vb{B})} = \abs{\vb{A}} \abs{\vb{A}^{-1}+\vb{B}} = 6\),
所以\begin{equation*}
	\abs{\vb{A}+\vb{B}^{-1}}
	= \frac{\abs{(\vb{A}+\vb{B}^{-1})\vb{B}}}{\abs{\vb{B}}}
	= \frac{\abs{\vb{A}\vb{B}+\vb{E}}}{\abs{\vb{B}}}
	= 3.
\end{equation*}
\end{solution}
\end{example}

\begin{example}
用\(\abs{\vb{A}}^2 = \abs{\vb{A}} \abs{\vb{A}^T}\)的方法计算行列式\begin{equation*}
	\abs{\vb{A}} = \begin{vmatrix}
		a & b & c & d \\
		-b & a & d & -c \\
		-c & -d & a & b \\
		-d & c & -b & a
	\end{vmatrix}.
\end{equation*}
\begin{solution}
因为\begin{align*}
	\abs{\vb{A}}^2 &= \abs{\vb{A}} \abs{\vb{A}^T} = \abs{\vb{A} \vb{A}^T} \\
	&= \abs{\begin{bmatrix}
		a & b & c & d \\
		-b & a & d & -c \\
		-c & -d & a & b \\
		-d & c & -b & a
	\end{bmatrix}
	\begin{bmatrix}
		a & -b & -c & -d \\
		b & a & -d & c \\
		c & d & a & -b \\
		d & -c & b & a
	\end{bmatrix}} \\
	&= \abs{(a^2+b^2+c^2+d^2) \vb{E}}_4
	= (a^2+b^2+c^2+d^2)^4,
\end{align*}
所以\(\abs{\vb{A}} = \pm(a^2+b^2+c^2+d^2)^2\),
再由\(\abs{\vb{A}}\)中含有项\(a^4\),
得\begin{equation*}
	\abs{\vb{A}} = (a^2+b^2+c^2+d^2)^2.
\end{equation*}
\end{solution}
%@Mathematica: Det[({{a, b, c, d},{-b, a, d, -c},{-c, -d, a, b},{-d, c, -b, a}})] // Factor
\end{example}

\begin{example}
计算:\begin{equation*}
	D = \begin{vmatrix}
		a & a & a & a \\
		a & a & -a & -a \\
		a & -a & a & -a \\
		a & -a & -a & a
	\end{vmatrix}.
\end{equation*}
\begin{solution}
因为\begin{equation*}
	\begin{bmatrix}
		1 & 0 & 0 & 0 \\
		0 & 0 & 1 & 0 \\
		0 & 1 & 0 & 0 \\
		0 & 0 & 0 & 1
	\end{bmatrix} \begin{bmatrix}
		a & a & a & a \\
		a & a & -a & -a \\
		a & -a & a & -a \\
		a & -a & -a & a
	\end{bmatrix}
	= \begin{bmatrix}
		1 & 0 & 0 & 0 \\
		1 & 1 & 0 & 0 \\
		1 & 0 & 1 & 0 \\
		1 & 1 & 1 & 1
	\end{bmatrix} \begin{bmatrix}
		a & a & a & a \\
		0 & -2 a & 0 & -2 a \\
		0 & 0 & -2 a & -2 a \\
		0 & 0 & 0 & 4 a
	\end{bmatrix},
\end{equation*}而\begin{equation*}
	\begin{vmatrix}
		1 & 0 & 0 & 0 \\
		0 & 0 & 1 & 0 \\
		0 & 1 & 0 & 0 \\
		0 & 0 & 0 & 1
	\end{vmatrix} = -1,
	\qquad
	\begin{vmatrix}
		1 & 0 & 0 & 0 \\
		1 & 1 & 0 & 0 \\
		1 & 0 & 1 & 0 \\
		1 & 1 & 1 & 1
	\end{vmatrix} = 1,
\end{equation*}\begin{equation*}
	\begin{vmatrix}
		a & a & a & a \\
		0 & -2 a & 0 & -2 a \\
		0 & 0 & -2 a & -2 a \\
		0 & 0 & 0 & 4 a
	\end{vmatrix}
	= a\cdot(-2a)\cdot(-2a)\cdot(4a) = 16a^2,
\end{equation*}
所以\begin{equation*}
	\begin{vmatrix}
		a & a & a & a \\
		a & a & -a & -a \\
		a & -a & a & -a \\
		a & -a & -a & a
	\end{vmatrix}
	= -16a^2.
\end{equation*}
\end{solution}
\end{example}

\begin{example}
设\(s_k = a_1^k + a_2^k + a_3^k + a_4^k\ (k=1,2,3,4,5,6)\).
计算:\begin{equation*}
	D = \begin{vmatrix}
		4 & s_1 & s_2 & s_3 \\
		s_1 & s_2 & s_3 & s_4 \\
		s_2 & s_3 & s_4 & s_5 \\
		s_3 & s_4 & s_5 & s_6
	\end{vmatrix}.
\end{equation*}
\begin{solution}
令矩阵\begin{equation*}
	\vb{A} = \begin{bmatrix}
		1 & 1 & 1 & 1 \\
		a_1 & a_2 & a_3 & a_4 \\
		a_1^2 & a_2^2 & a_3^2 & a_4^2 \\
		a_1^3 & a_2^3 & a_3^3 & a_4^3
	\end{bmatrix},
\end{equation*}
显然\begin{equation*}
	D = \det(\vb{A} \vb{A}^T) = \abs{\vb{A}}^2.
\end{equation*}
而根据\cref{equation:行列式.范德蒙德行列式},
\(\abs{\vb{A}}
= \prod_{1 \leq j < i \leq n} (a_i - a_j)\),
故\(D = \prod_{1 \leq j < i \leq n} (a_i - a_j)^2\).
\end{solution}
\end{example}

\section{柯西--比内公式}
\begin{theorem}
%@see: 《高等代数(第三版 上册)》(丘维声) P141 定理1
已知数域\(K\).
设矩阵\(\vb{A} \in M_{m \times n}(K),
\vb{B} \in M_{n \times m}(K)\).
如果\(m < n\),
那么\begin{equation}\label{equation:线性方程组.柯西比内公式}
	\abs{\vb{A}\vb{B}}
	= \sum_{1 \leq i_1 < i_2 < \dotsb < i_m \leq n}
	\MatrixMinor{\vb{A}}{
		1,2,\dotsc,m \\
		i_1,i_2,\dotsc,i_m
	}
	\MatrixMinor{\vb{B}}{
		i_1,i_2,\dotsc,i_m \\
		1,2,\dotsc,m
	}.
\end{equation}
\begin{proof}
考虑\(m+n\)阶分块矩阵\[
	\begin{bmatrix}
		\vb{E}_n & \vb{B} \\
		\vb0 & \vb{A}\vb{B}
	\end{bmatrix},
\]
其中\(\vb{E}_n\)是数域\(K\)上的\(n\)阶单位矩阵.
由于\[
	\begin{vmatrix}
		\vb{E}_n & \vb{B} \\
		\vb0 & \vb{A}\vb{B}
	\end{vmatrix}
	= \abs{\vb{E}_n} \abs{\vb{A}\vb{B}}
	= \abs{\vb{A}\vb{B}},
\]
所以\[
	\begin{bmatrix}
		\vb{E}_n & \vb{B} \\
		\vb0 & \vb{A}\vb{B}
	\end{bmatrix}
	\to
	\begin{bmatrix}
		\vb{E}_n & \vb{B} \\
		-\vb{A} & \vb0
	\end{bmatrix}
	= \begin{bmatrix}
		\vb{E}_n & \vb0 \\
		-\vb{A} & \vb{E}_m
	\end{bmatrix} \begin{bmatrix}
		\vb{E}_n & \vb{B} \\
		\vb0 & \vb{A}\vb{B}
	\end{bmatrix},
\]\[
	\begin{vmatrix}
		\vb{E}_n & \vb{B} \\
		-\vb{A} & \vb0
	\end{vmatrix}
	= \begin{vmatrix}
		\vb{E}_n & \vb0 \\
		-\vb{A} & \vb{E}_m
	\end{vmatrix} \begin{vmatrix}
		\vb{E}_n & \vb{B} \\
		\vb0 & \vb{A}\vb{B}
	\end{vmatrix}
	= \begin{vmatrix}
		\vb{E}_n & \vb{B} \\
		\vb0 & \vb{A}\vb{B}
	\end{vmatrix},
\]
其中\(\vb{E}_m\)是数域\(K\)上的\(m\)阶单位矩阵.
利用\hyperref[theorem:行列式.拉普拉斯定理]{拉普拉斯定理}把上式最左端行列式按后\(m\)行展开得\[
	\begin{vmatrix}
		\vb{E}_n & \vb{B} \\
		-\vb{A} & \vb0
	\end{vmatrix}
	= \sum_{1 \leq i_1 < \dotsb < i_m \leq n}
	\MatrixMinor{(-\vb{A})}{
		1,2,\dotsc,m \\
		i_1,i_2,\dotsc,i_m
	}
	(-1)^{[(n+1)+\dotsb+(n+m)]+(i_1+\dotsb+i_m)}
	\abs{(\vb\epsilon_{\mu_1},\dotsc,\vb\epsilon_{\mu_{n-m}},\vb{B})},
\]
其中\(\Set{\mu_1,\dotsc,\mu_{n-m}}
= \Set{1,\dotsc,n}-\Set{i_1,\dotsc,i_s}\),
且\(\mu_1<\dotsb<\mu_{n-m}\).

把\(\abs{(\vb\epsilon_{\mu_1},\dotsc,\vb\epsilon_{\mu_{n-m}},\vb{B})}\)
按前\(n-m\)行展开得\[
	\abs{(\vb\epsilon_{\mu_1},\dotsc,\vb\epsilon_{\mu_{n-m}},\vb{B})}
	= \abs{\vb{E}_{n-m}}
	(-1)^{(\mu_1+\dotsb+\mu_{n-m})+[1+\dotsb+(n-m)]}
	\MatrixMinor{\vb{B}}{
		i_1,i_2,\dotsc,i_m \\
		1,2,\dotsc,m
	}.
\]
因此\begin{align*}
	\begin{vmatrix}
		\vb{E}_n & \vb{B} \\
		-\vb{A} & \vb0
	\end{vmatrix}
	&= \sum_{1 \leq i_1 < \dotsb < i_m \leq n}
	(-1)^{m+m^2+n+n^2}
	\MatrixMinor{\vb{A}}{
		1,2,\dotsc,m \\
		i_1,i_2,\dotsc,i_m
	}
	\MatrixMinor{\vb{B}}{
		i_1,i_2,\dotsc,i_m \\
		1,2,\dotsc,m
	} \\
	&= \sum_{1 \leq i_1 < \dotsb < i_m \leq n}
	\MatrixMinor{\vb{A}}{
		1,2,\dotsc,m \\
		i_1,i_2,\dotsc,i_m
	}
	\MatrixMinor{\vb{B}}{
		i_1,i_2,\dotsc,i_m \\
		1,2,\dotsc,m
	}.
\end{align*}
综上所述,\[
	\abs{\vb{A}\vb{B}}
	= \sum_{1 \leq i_1 \leq i_2 \leq \dotsb \leq i_m \leq n}
	\MatrixMinor{\vb{A}}{
		1,2,\dotsc,m \\
		i_1,i_2,\dotsc,i_m
	}
	\MatrixMinor{\vb{B}}{
		i_1,i_2,\dotsc,i_m \\
		1,2,\dotsc,m
	}.
	\qedhere
\]
\end{proof}
%\cref{theorem:矩阵乘积的秩.多行少列矩阵与少行多列矩阵的乘积的行列式}
\end{theorem}
\cref{equation:线性方程组.柯西比内公式} 称为\DefineConcept{柯西--比内公式}.


\begin{example}
设\(\vb{A} = (\vb{B},\vb{C}) \in M_{n \times m}(\mathbb{R})\),
其中\(\vb{B} \in M_{n \times s}(\mathbb{R})\),
\(\vb{C} \in M_{n \times (m-s)}(\mathbb{R})\).
证明:\begin{equation}
\abs{\vb{A}^T \vb{A}} \leq \abs{\vb{B}^T \vb{B}} \abs{\vb{C}^T \vb{C}}.
\end{equation}
%TODO
\end{example}

\begin{example}
设\(\vb{A} = (a_{ij})_n \in M_n(\mathbb{R})\).
证明:\begin{equation}\label{equation:线性方程组.Hadamard不等式}
	\abs{\vb{A}}^2 \leq \prod_{j=1}^n \sum_{i=1}^n a_{ij}^2.
\end{equation}
%TODO
\end{example}

\begin{example}
设\(\vb{A} = (a_{ij})_n \in M_n(\mathbb{R})\),
且\(\abs{a_{ij}} < M\ (i,j=1,2,\dotsc,n)\).
证明:\begin{equation}
	\abs{\det\vb{A}} \leq M^n n^{n/2}.
\end{equation}
%TODO
\end{example}

\section{本章总结}
\subsection*{行列式的计算方法}
我们可以采用以下方法计算行列式:
\begin{enumerate}
	\item 利用行列式的定义计算.
	\item 利用初等变换,将行列式化为上三角形,如此行列式就等于主对角线上元素之积.
	\item 拆成若干个行列式的和.
	\item 按行(或列)展开.
	\item 数学归纳法.
	\item 递推关系法.
	\item 升阶法.
	\item \hyperref[theorem:逆矩阵.行列式降阶定理]{降阶法}.
	\item 利用矩阵乘法,例如\begin{gather*}
		\abs{\vb{A}}^2 = \abs{\vb{A}} \abs{\vb{A}^T}, \\
		\abs{\vb{A}} = \frac{\abs{\vb{A} \vb{B}}}{\abs{\vb{B}}} \quad(\abs{\vb{B}}\neq0).
	\end{gather*}
	\item 利用\hyperref[equation:行列式.范德蒙德行列式]{范德蒙德行列式}等特殊行列式计算.
\end{enumerate}

\subsection*{重要公式}
\begin{gather*}
	\sum_{j=1}^n a_{ij} A_{kj}
	= \left\{ \begin{array}{cl}
		\abs{\vb{A}}, & k = i, \\
		0, & k \neq i,
	\end{array} \right.
	\quad i=1,2,\dotsc,n, \\
	\sum_{i=1}^n a_{ij} A_{ik}
	= \left\{ \begin{array}{cl}
		\abs{\vb{A}}, & k = j, \\
		0, & k \neq j,
	\end{array} \right.
	\quad j=1,2,\dotsc,n, \\
	\vb{A} \vb{A}^* = \vb{A}^* \vb{A} = \abs{\vb{A}} \vb{E}, \\ %\cref{equation:行列式.伴随矩阵.恒等式1}
	(\vb{A}^*)^T = (\vb{A}^T)^*, \\ %\cref{equation:行列式.伴随矩阵.恒等式2}
	(\vb{A} \vb{B})^* = \vb{B}^* \vb{A}^*, \\ %\cref{equation:行列式.伴随矩阵.恒等式3}
	(k \vb{A})^* = k^{n-1} \vb{A}^*. %\cref{equation:行列式.伴随矩阵.数与矩阵乘积的伴随}
\end{gather*}

\subsection*{重要行列式}
\begin{gather*}
	\begin{vmatrix}
		a_{11} & a_{12} & \dots & a_{1n} \\
		& a_{22} & \dots & a_{2n} \\
		& & \ddots & \vdots \\
		& & & a_{nn}
	\end{vmatrix}
	= a_{11} a_{22} \dotsm a_{nn}, \\%
	\begin{vmatrix}
		& & & & a_{1n} \\
		& & & a_{2,n-1} & a_{2n} \\
		& & & \vdots & \vdots \\
		& a_{n-1,2} & \dots & a_{n-1,n-1} & a_{n-1,n} \\
		a_{n1} & a_{n2} & \dots & a_{n,n-1} & a_{nn}
	\end{vmatrix}
	=(-1)^{\frac{1}{2}n(n-1)} a_{1n} a_{2,n-1} \dotsm a_{n-1,2} a_{n1}. \\
	\begin{vmatrix}
		k & \lambda & \lambda & \dots & \lambda \\
		\lambda & k & \lambda & \dots & \lambda \\
		\lambda & \lambda & k & \dots & \lambda \\
		\vdots & \vdots & \vdots & & \vdots \\
		\lambda & \lambda & \lambda & \dots & k
	\end{vmatrix}_n
	= [k+(n-1)\lambda] (k-\lambda)^{n-1},
	\quad k\neq\lambda,n=1,2,\dotsc. \\
	%\cref{equation:行列式.范德蒙德行列式}
	\begin{vmatrix}
		1 & 1 & 1 & \dots & 1 \\
		x_1 & x_2 & x_3 & \dots & x_n \\
		x_1^2 & x_2^2 & x_3^2 & \dots & x_n^2 \\
		\vdots & \vdots & \vdots& & \vdots \\
		x_1^{n-1} & x_2^{n-1} & x_3^{n-1} & \dots & x_n^{n-1}
	\end{vmatrix}
	= \prod_{1 \leq j < i \leq n}(x_i-x_j). \\
	%\cref{example:行列式的展开.三对角行列式}
	\begin{vmatrix}
		a+b & ab & \\
		1 & a+b & ab & \\
		& 1 & a + b & \ddots & \\
		& & \ddots & \ddots & ab \\
		& & & 1 & a+b \\
	\end{vmatrix}_n
	= a^n + a^{n-1} b + a^{n-2} b^2 + \dotsb + a b^{n-1} + b^n. \\
	\begin{vmatrix}
		a & b & c & d \\
		-b & a & d & -c \\
		-c & -d & a & b \\
		-d & c & -b & a
	\end{vmatrix}
	= (a^2+b^2+c^2+d^2)^2.
\end{gather*}


\chapter{矩阵的逆}
\section{可逆矩阵}
% \begin{lemma}
% 设\(\vb{A},\vb{B}\)是数域\(K\)上的\(n\)阶矩阵,
% \(\vb{E}\)是数域\(K\)上的\(n\)阶单位矩阵.
% 如果\(\vb{A}\vb{B}=\vb{E}\),则\(\vb{B}\vb{A}=\vb{E}\).
% \begin{proof}
% 只要在等式\(\vb{A}\vb{B}=\vb{E}\)等号两边同时左乘\(\vb{B}\)并右乘\(\vb{A}\),
% 就有\[
% 	(\vb{B}\vb{A})(\vb{B}\vb{A})
% 	= \vb{B}(\vb{A}\vb{B})\vb{A}
% 	= \vb{B}\vb{E}\vb{A}
% 	= \vb{B}\vb{A}.
% \]
% 记\(\vb{X}\defeq\vb{B}\vb{A}\),
% 则\(\vb{X}^2=\vb{X}\),
% 整理得\(\vb{X}(\vb{X}-\vb{E})=\vb0\),
% 解得\(\vb{X}=\vb0\)或\(\vb{X}=\vb{E}\).
% 假设\(\vb{B}\vb{A}=\vb0\),
% 于是\(\abs{\vb{B}\vb{A}}=\abs{\vb{B}}\abs{\vb{A}}=0\),
% 这与\(\abs{\vb{A}\vb{B}}=\abs{\vb{A}}\abs{\vb{B}}=1\)矛盾,
% 因此\(\vb{B}\vb{A}=\vb{E}\).
% \end{proof}
% \end{lemma}
%DELETE: 这个引理是错误的!由\(\vb{X}^2=\vb{X}\)只能得到\(\vb{X}\)是幂等矩阵,不能说明它是零矩阵或单位矩阵!

\begin{definition}\label{definition:可逆矩阵.可逆矩阵的定义}
%@see: 《线性代数》(张慎语、周厚隆) P43 定义1
%@see: 《高等代数(第三版 上册)》(丘维声) P128 定义1
设\(\vb{E}\)是数域\(K\)上的\(n\)阶单位矩阵.
对于数域\(K\)上的矩阵\(\vb{A}\),
如果存在数域\(K\)上的矩阵\(\vb{B}\),使得\[
	\vb{A}\vb{B}=\vb{B}\vb{A}=\vb{E},
\]
则称“\(\vb{A}\)是一个\DefineConcept{可逆矩阵}(\(\vb{A}\) is an invertible matrix)”,
%@see: https://mathworld.wolfram.com/InvertibleMatrix.html
或称“\(\vb{A}\) \DefineConcept{可逆}(\(\vb{A}\) is invertible)”;
并称“\(\vb{B}\)是\(\vb{A}\)的\DefineConcept{逆矩阵}(inverse matrix)”,
记作\(\vb{A}^{-1}\),
即\[
	(\forall \vb{A},\vb{B}\in M_n(K))
	[\vb{A}^{-1} = \vb{B} \defiff \vb{A}\vb{B}=\vb{B}\vb{A}=\vb{E}].
\]
\end{definition}

从定义可知,如果矩阵\(\vb{A},\vb{B}\)满足\(\vb{A}\vb{B}=\vb{B}\vb{A}=\vb{E}\),
那么这两个矩阵都是可逆矩阵,且两者互为逆矩阵.

\begin{definition}
设\(\vb{A} \in M_n(K)\).
\begin{itemize}
	\item 若\(\abs{\vb{A}}=0\),
	则称“\(\vb{A}\)是\DefineConcept{奇异矩阵}(singular matrix)”.
	%@see: https://mathworld.wolfram.com/SingularMatrix.html
	\item 若\(\abs{\vb{A}} \neq 0\),
	则称“\(\vb{A}\)是\DefineConcept{非奇异矩阵}(nonsingular matrix)”.
	%@see: https://mathworld.wolfram.com/NonsingularMatrix.html
\end{itemize}
\end{definition}

\begin{definition}
若\(\vb{A}\)是可逆矩阵,那么规定:
对于正整数\(k\),有
\begin{equation}
	\vb{A}^{-k} = (\vb{A}^{-1})^k
	= \underbrace{\vb{A}^{-1}\vb{A}^{-1}\dotsm\vb{A}^{-1}}_{\text{$k$个}}.
\end{equation}
\end{definition}

\begin{theorem}\label{theorem:逆矩阵.矩阵可逆的充分必要条件1}
%@see: 《线性代数》(张慎语、周厚隆) P43 定理1
设\(\vb{A}\)是\(n\)阶方阵,则“\(\vb{A}\)可逆”的充分必要条件是“\(\vb{A}\)是非奇异矩阵”.
\begin{proof}
必要性.
假设矩阵\(\vb{A}\)可逆,那么存在\(n\)阶方阵\(\vb{B}\),使得\(\vb{A}\vb{B}=\vb{E}\),于是\(\abs{\vb{A}\vb{B}}=\abs{\vb{E}}\);
而根据\cref{theorem:行列式.矩阵乘积的行列式},
\(\abs{\vb{A}\vb{B}}=\abs{\vb{A}}\abs{\vb{B}}=1\),\(\abs{\vb{A}}\neq0\).

充分性.
设\(\abs{\vb{A}}\neq0\),\(\vb{A}^*\)是\(\vb{A}\)的伴随矩阵.
根据\cref{equation:行列式.伴随矩阵.恒等式1},
若令\[
	\vb{B}=\frac{1}{\abs{\vb{A}}} \vb{A}^*,
\]
则有\(\vb{A}\vb{B} = \vb{B}\vb{A} = \vb{E}\),
故由可逆矩阵的定义可知,矩阵\(\vb{A}\)可逆,
且有\(\vb{A}^{-1} = \vb{B}\).
\end{proof}
\end{theorem}

\begin{property}\label{theorem:逆矩阵.逆矩阵的唯一性}
设\(\vb{A}\)是可逆矩阵,则它的逆矩阵存在且唯一,
且有\begin{equation}
	\vb{A}^{-1} = \abs{\vb{A}}^{-1} \vb{A}^*.
\end{equation}
\begin{proof}
存在性.
在\cref{theorem:逆矩阵.矩阵可逆的充分必要条件1} 的证明过程中,
我们看到矩阵\(\vb{B}=\abs{\vb{A}}^{-1} \vb{A}^*\)是可逆矩阵\(\vb{A}\)的一个逆矩阵,即\(\vb{A}\vb{B}=\vb{E}\).

唯一性.
设矩阵\(\vb{C}\)也是\(\vb{A}\)的逆矩阵,即\(\vb{C}\vb{A}=\vb{E}\),于是\[
	\vb{C}=\vb{C}\vb{E}=\vb{C}(\vb{A}\vb{B})=(\vb{C}\vb{A})\vb{B}=\vb{E}\vb{B}=\vb{B}.
	\qedhere
\]
\end{proof}
\end{property}

\begin{property}\label{theorem:逆矩阵.单位矩阵可逆}
单位矩阵\(\vb{E}\)可逆,且\(\vb{E}^{-1}=\vb{E}\).
\end{property}

\begin{property}\label{theorem:逆矩阵.逆矩阵的行列式}
%@see: 《线性代数》(张慎语、周厚隆) P44 性质3
设\(\vb{A}\)可逆,则\(\abs{\vb{A}^{-1}} = \abs{\vb{A}}^{-1}\).
\end{property}

\begin{property}\label{theorem:逆矩阵.逆矩阵的逆}
%@see: 《线性代数》(张慎语、周厚隆) P44 性质4
设\(\vb{A}\)可逆,则\(\vb{A}^{-1}\)可逆,且\((\vb{A}^{-1})^{-1} = \vb{A}\).
\end{property}

\begin{property}\label{theorem:逆矩阵.矩阵乘积的逆1}
%@see: 《线性代数》(张慎语、周厚隆) P44 性质5
设\(\vb{A}\)、\(\vb{B}\)都是\(n\)阶可逆矩阵,则\(\vb{A}\vb{B}\)可逆,且\begin{equation}
	(\vb{A} \vb{B})^{-1} = \vb{B}^{-1} \vb{A}^{-1}.
\end{equation}
\begin{proof}
因为\(\vb{A},\vb{B}\)都可逆,可设它们的逆矩阵分别为\(\vb{A}^{-1},\vb{B}^{-1}\),
于是\[
	(\vb{A}\vb{B})(\vb{B}^{-1}\vb{A}^{-1})
	= \vb{A}(\vb{B}\vb{B}^{-1})\vb{A}^{-1}
	= \vb{A}\vb{E}\vb{A}^{-1}
	= \vb{A}\vb{A}^{-1}
	= \vb{E}.
	\qedhere
\]
\end{proof}
\end{property}

\cref{theorem:逆矩阵.矩阵乘积的逆1} 可以推广到有限个\(n\)阶可逆矩阵乘积的情形.
\begin{property}\label{theorem:逆矩阵.矩阵乘积的逆2}
设\(\AutoTuple{\vb{A}}{n}\)都是\(n\)阶可逆矩阵,
则\(\vb{A}_1 \vb{A}_2 \dotsm \vb{A}_{n-1} \vb{A}_n\)可逆,且\begin{equation}
	(\vb{A}_1 \vb{A}_2 \dotsm \vb{A}_{n-1} \vb{A}_n)^{-1}
	= \vb{A}_n^{-1} \vb{A}_{n-1}^{-1} \dotsm \vb{A}_2^{-1} \vb{A}_1^{-1}.
\end{equation}
\end{property}

\begin{property}\label{theorem:逆矩阵.数与矩阵乘积的逆}
%@see: 《线性代数》(张慎语、周厚隆) P44 性质6
设数域\(K\)上的\(n\)阶矩阵\(\vb{A}\)可逆,
\(k \in K-\{0\}\),则\(k\vb{A}\)可逆,且
\begin{equation}
	(k \vb{A})^{-1} = k^{-1} \vb{A}^{-1}.
\end{equation}
\begin{proof}
由\cref{theorem:行列式.性质2.推论2},
\(\abs{k\vb{A}} = k^n\abs{\vb{A}}\).
因为\(\vb{A}\)可逆,所以\(\abs{\vb{A}}\neq0\);
又因为\(k\neq0\),所以\(\abs{k\vb{A}}\neq0\),即\(k\vb{A}\)可逆.
因此\[
	(k^{-1}\vb{A}^{-1})(k\vb{A})
	= (k^{-1} \cdot k)(\vb{A}^{-1}\vb{A})
	= 1 \vb{E} = \vb{E},
\]
也就是说\(k^{-1}\vb{A}^{-1}\)是\(k\vb{A}\)的逆矩阵.
\end{proof}
%\cref{equation:行列式.伴随矩阵.数与矩阵乘积的伴随}
\end{property}

\begin{property}\label{theorem:逆矩阵.转置矩阵的逆与逆矩阵的转置}
%@see: 《线性代数》(张慎语、周厚隆) P44 性质7
设\(\vb{A}\)可逆,则\(\vb{A}^T\)可逆,且\begin{equation}
	(\vb{A}^T)^{-1} = (\vb{A}^{-1})^T.
\end{equation}
\begin{proof}
由\cref{theorem:行列式.性质1},
\(\abs{\vb{A}^T}=\abs{\vb{A}}\neq0\),
于是\(\vb{A}^T\)可逆.
由\cref{theorem:矩阵.矩阵乘积的转置},
\((\vb{A} \vb{A}^{-1})^T = (\vb{A}^{-1})^T \vb{A}^T\).
既然\(\vb{A} \vb{A}^{-1} = \vb{E}, \vb{E}^T = \vb{E}\),
于是\((\vb{A}^{-1})^T \vb{A}^T = \vb{E}\),
那么由逆矩阵的定义可知,
\((\vb{A}^T)^{-1}=(\vb{A}^{-1})^T\).
\end{proof}
\end{property}

\begin{example}
设矩阵\(\vb{A}\)、\(\vb{B}\)可交换,\(\vb{A}\)可逆.
证明:\(\vb{A}^{-1}\)与\(\vb{B}\)可交换.
\begin{proof}
因为\(\vb{A}\vb{B} = \vb{B}\vb{A}\),在等式两边同时左乘\(\vb{A}^{-1}\),得\[
	\vb{B} = (\vb{A}^{-1}\vb{A})\vb{B} = \vb{A}^{-1}(\vb{A}\vb{B}) = \vb{A}^{-1}(\vb{B}\vb{A});
\]
再在等式两边右乘\(\vb{A}^{-1}\),得\[
	\vb{B}\vb{A}^{-1} = (\vb{A}^{-1}\vb{B}\vb{A})\vb{A}^{-1} = \vb{A}^{-1}\vb{B}(\vb{A}\vb{A}^{-1}) = \vb{A}^{-1}\vb{B}.
	\qedhere
\]
\end{proof}
\end{example}

\begin{example}
下面看一些常见矩阵的逆矩阵:\begin{gather*}
	\begin{bmatrix}
		a_{11} & a_{12} \\
		a_{21} & a_{22}
	\end{bmatrix}
	= \begin{vmatrix}
		a_{11} & a_{12} \\
		a_{21} & a_{22}
	\end{vmatrix}^{-1}
	\begin{bmatrix}
		a_{22} & -a_{12} \\
		-a_{21} & a_{11}
	\end{bmatrix}, \\
	\begin{bmatrix}
		\lambda_1 \\
		& \lambda_2 \\
		&& \ddots \\
		&&& \lambda_n
	\end{bmatrix}^{-1}
	= \begin{bmatrix}
		\lambda_1^{-1} \\
		& \lambda_2^{-1} \\
		&& \ddots \\
		&&& \lambda_n^{-1}
	\end{bmatrix}, \\
	\begin{bmatrix}
		& & & & \lambda_1 \\
		& & & \lambda_2 \\
		& & \iddots \\
		& \lambda_{n-1} \\
		\lambda_n
	\end{bmatrix}^{-1}
	= \begin{bmatrix}
		& & & & \lambda_n^{-1} \\
		& & & \lambda_{n-1}^{-1} \\
		& & \iddots \\
		& \lambda_2^{-1} \\
		\lambda_1^{-1}
	\end{bmatrix}.
\end{gather*}
\end{example}

\begin{example}\label{theorem:逆矩阵.伴随矩阵的逆与逆矩阵的伴随}
%@see: 《线性代数》(张慎语、周厚隆) P46 例4
设\(\vb{A}\)可逆.
证明:\(\vb{A}\)的伴随矩阵\(\vb{A}^*\)可逆,
且\begin{equation}
	(\vb{A}^*)^{-1}
	= \abs{\vb{A}}^{-1} \vb{A}
	= (\vb{A}^{-1})^*.
\end{equation}
\begin{proof}
因为\begin{align*}
	(\vb{A}^*)^{-1}
	&= \left( \abs{\vb{A}} \vb{A}^{-1} \right)^{-1}
		\tag{\cref{theorem:逆矩阵.逆矩阵的唯一性}} \\
	&= \abs{\vb{A}}^{-1} (\vb{A}^{-1})^{-1}
		\tag{\cref{theorem:逆矩阵.数与矩阵乘积的逆}} \\
	&= \abs{\vb{A}}^{-1} \vb{A}
		\tag{\cref{theorem:逆矩阵.逆矩阵的逆}} \\
	&= \abs{\vb{A}^{-1}} (\vb{A}^{-1})^{-1}
		\tag{\cref{theorem:逆矩阵.逆矩阵的行列式}} \\
	&= (\vb{A}^{-1})^*,
		\tag{\cref{theorem:逆矩阵.逆矩阵的唯一性}}
\end{align*}
所以\((\vb{A}^*)^{-1}
= \abs{\vb{A}}^{-1} \vb{A}
= (\vb{A}^{-1})^*\).
\end{proof}
\end{example}
\begin{example}
%@see: 《高等代数学习指导书(第三版)》(姚慕生、谢启鸿) P62 例2.23
设\(\vb{A} \in M_n(K)\)满足\(\vb{A}^m = \vb{E}\),
其中\(\vb{E}\)是数域\(K\)上的\(n\)阶单位矩阵.
证明:\((\vb{A}^*)^m = \vb{E}\).
\begin{proof}
由\(\vb{A}^m = \vb{E}\)得\(\abs{\vb{A}}^m = 1\),\(\vb{A}\)可逆,
那么\(\vb{A}^* = \abs{\vb{A}} \vb{A}^{-1}\),
于是\[
	(\vb{A}^*)^m
	= (\abs{\vb{A}} \vb{A}^{-1})^m
	= \abs{\vb{A}}^m (\vb{A}^{-1})^m
	= (\vb{A}^m)^{-1}
	= \vb{E}.
	\qedhere
\]
\end{proof}
\end{example}

\begin{example}\label{example:对合矩阵.对合矩阵的逆矩阵}
设\(\vb{A}\)是数域\(K\)上的\(n\)阶对合矩阵.
证明\(\vb{A}^{-1} = \vb{A}\).
\begin{proof}
假设\(\vb{A}^2=\vb{E}\),
其中\(\vb{E}\)是数域\(K\)上的\(n\)阶单位矩阵,
那么由\cref{theorem:行列式.矩阵乘积的行列式} 可知\[
	\abs{\vb{A}^2}
	= \abs{\vb{A}}^2
	= 1,
\]
从而\(\abs{\vb{A}}\neq0\),
\(\vb{A}\)可逆,
因此\[
	\vb{A}^{-1}
	= \vb{A}^{-1}\vb{E}
	= \vb{A}^{-1}(\vb{A}^2)
	= (\vb{A}^{-1}\vb{A})\vb{A}
	= \vb{E}\vb{A}
	= \vb{A}.
	\qedhere
\]
\end{proof}
\end{example}

\begin{example}\label{example:可逆矩阵.分块上三角矩阵的逆}
%@see: 《线性代数》(张慎语、周厚隆) P46 例5
设\(\vb{A} \in M_s(K),
\vb{B} \in M_n(K),
\vb{C} \in M_{s \times n}(K)\),
\(\vb{A}\)和\(\vb{B}\)都可逆.
证明:矩阵\[
	\vb{M} = \begin{bmatrix}
		\vb{A} & \vb{C} \\
		\vb0 & \vb{B}
	\end{bmatrix}
\]可逆,且\[
	\vb{M}^{-1} = \begin{bmatrix}
		\vb{A}^{-1} & -\vb{A}^{-1} \vb{C} \vb{B}^{-1} \\
		\vb0 & \vb{B}^{-1}
	\end{bmatrix}.
\]
\begin{proof}
因为\(\vb{A}\)、\(\vb{B}\)为可逆矩阵,\(\abs{\vb{A}} \neq 0\),\(\abs{\vb{B}} \neq 0\).
所以\(\abs{\vb{M}}=\abs{\vb{A}}\abs{\vb{B}} \neq 0\),即\(\vb{M}\)可逆.

令\(\vb{M}\vb{x}=\vb{E}\),即\[
	\begin{bmatrix}
		\vb{A} & \vb{C} \\
		\vb0 & \vb{B}
	\end{bmatrix}
	\begin{bmatrix}
		\vb{x}_1 & \vb{x}_2 \\
		\vb{x}_3 & \vb{x}_4
	\end{bmatrix}
	= \begin{bmatrix}
		\vb{E} & \vb0 \\
		\vb0 & \vb{E}
	\end{bmatrix}
\]则\[
	\begin{bmatrix}
		\vb{A}\vb{x}_1+\vb{C}\vb{x}_3 & \vb{A}\vb{x}_2+\vb{C}\vb{x}_4 \\
		\vb{B}\vb{x}_3 & \vb{B}\vb{x}_4
	\end{bmatrix}
	= \begin{bmatrix}
		\vb{E} & \vb0 \\
		\vb0 & \vb{E}
	\end{bmatrix}
\]
进而有\[
	\left\{ \begin{array}{l}
		\vb{A}\vb{x}_1+\vb{C}\vb{x}_3 = \vb{E} \\
		\vb{A}\vb{x}_2+\vb{C}\vb{x}_4 = \vb0 \\
		\vb{B}\vb{x}_3 = \vb0 \\
		\vb{B}\vb{x}_4 = \vb{E}
	\end{array} \right.
\]
由第4式可得\(\vb{x}_4 = \vb{B}^{-1}\).
代入第2式得\(\vb{A}\vb{x}_2=-\vb{C}\vb{B}^{-1}\),
\(\vb{x}_2=-\vb{A}^{-1}\vb{C}\vb{B}^{-1}\).
用\(\vb{B}^{-1}\)左乘第3式左右两端,\(\vb{B}^{-1}\vb{B}\vb{x}_3=\vb{x}_3=\vb0\).
则第1式化为\(\vb{A}\vb{x}_1=\vb{E}\),显然\(\vb{x}_1=\vb{A}^{-1}\),所以\[
	\vb{M}^{-1} = \vb{x} = \begin{bmatrix}
		\vb{A}^{-1} & -\vb{A}^{-1}\vb{C}\vb{B}^{-1} \\
		\vb0 & \vb{B}^{-1}
	\end{bmatrix}.
	\qedhere
\]
\end{proof}
\end{example}

\begin{remark}
从\cref{example:可逆矩阵.分块上三角矩阵的逆} 的结论\[
	\begin{bmatrix}
		\vb{A} & \vb{C} \\
		\vb0 & \vb{B}
	\end{bmatrix}^{-1}
	= \begin{bmatrix}
		\vb{A}^{-1} & -\vb{A}^{-1} \vb{C} \vb{B}^{-1} \\
		\vb0 & \vb{B}^{-1}
	\end{bmatrix}
\]出发,
任取\(\vb{D} \in M_{n \times s}(K)\),
我们还可以得到\[
	\begin{bmatrix}
		\vb{A} & \vb0 \\
		\vb{D} & \vb{B}
	\end{bmatrix}^{-1}
	= \begin{bmatrix}
		\vb{A}^{-1} & \vb0 \\
		-\vb{B}^{-1} \vb{D} \vb{A}^{-1} & \vb{B}^{-1}
	\end{bmatrix},
\]\[
	\begin{bmatrix}
		\vb{C} & \vb{A} \\
		\vb{B} & \vb0
	\end{bmatrix}^{-1}
	= \begin{bmatrix}
		\vb0 & \vb{B}^{-1} \\
		\vb{A}^{-1} & -\vb{A}^{-1}\vb{C}\vb{B}^{-1}
	\end{bmatrix},
\]\[
	\begin{bmatrix}
		\vb0 & \vb{A} \\
		\vb{B} & \vb{D}
	\end{bmatrix}^{-1}
	= \begin{bmatrix}
		-\vb{B}^{-1}\vb{D}\vb{A}^{-1} & \vb{B}^{-1} \\
		\vb{A}^{-1} & \vb0
	\end{bmatrix}.
\]

我们还可以进一步利用\cref{theorem:逆矩阵.逆矩阵的唯一性}
以及\cref{equation:行列式.广义三角阵的行列式1,equation:行列式.广义三角阵的行列式2},
得到\[
	\begin{bmatrix}
		\vb{A} & \vb{C} \\
		\vb0 & \vb{B}
	\end{bmatrix}^*
	= \begin{bmatrix}
		\abs{\vb{B}} \vb{A}^* & -\vb{A}^*\vb{C}\vb{B}^* \\
		\vb0 & \abs{\vb{A}} \vb{B}^*
	\end{bmatrix},
\]\[
	\begin{bmatrix}
		\vb{A} & \vb0 \\
		\vb{D} & \vb{B}
	\end{bmatrix}^*
	= \begin{bmatrix}
		\abs{\vb{B}} \vb{A}^* & \vb0 \\
		-\vb{B}^* \vb{D} \vb{A}^* & \abs{\vb{A}} \vb{B}^*
	\end{bmatrix},
\]\[
	\begin{bmatrix}
		\vb{C} & \vb{A} \\
		\vb{B} & \vb0
	\end{bmatrix}^*
	= (-1)^{sn} \begin{bmatrix}
		\vb0 & \abs{\vb{A}} \vb{B}^* \\
		\abs{\vb{B}} \vb{A}^* & -\vb{A}^*\vb{C}\vb{B}^*
	\end{bmatrix},
\]\[
	\begin{bmatrix}
		\vb0 & \vb{A} \\
		\vb{B} & \vb{D}
	\end{bmatrix}^*
	= (-1)^{sn} \begin{bmatrix}
		-\vb{B}^*\vb{D}\vb{A}^* & \abs{\vb{A}} \vb{B}^* \\
		\abs{\vb{B}} \vb{A}^* & \vb0
	\end{bmatrix}.
\]
\end{remark}

\section{应用初等变换求解逆矩阵}
\begin{property}\label{theorem:逆矩阵.初等矩阵的性质3}
初等矩阵可逆.
\end{property}

\begin{theorem}\label{theorem:逆矩阵.可逆矩阵与初等矩阵的关系}
设\(\vb{A}=(a_{ij})_n\),则\(\vb{A}\)可逆的充分必要条件是:
\(\vb{A}\)可经一系列初等行变换化为单位矩阵\(\vb{E}_n\),
即\(\vb{A} \cong \vb{E}_n\).
\begin{proof}
\def\Ps{\vb{P}_t \vb{P}_{t-1} \dotsm \vb{P}_2 \vb{P}_1}
存在与\(t\)次初等行变换对应的\(t\)个初等矩阵\(\vb{P}_t,\vb{P}_{t-1},\dotsc,\vb{P}_2,\vb{P}_1\),使\[
	\vb{A} \to \vb{E}_n = \Ps \vb{A},
\]
则\(\vb{A}\)可逆且\(\vb{A}^{-1} = \Ps\).

对矩阵\((\vb{A},\vb{E}_n)\)作以上初等行变换,则\begin{align*}
	(\vb{A},\vb{E}_n) \to &\Ps(\vb{A},\vb{E}_n) = \vb{A}^{-1}(\vb{A},\vb{E}_n) \\
	&= (\vb{A}^{-1}\vb{A},\vb{A}^{-1}\vb{E}_n) = (\vb{E}_n,\vb{A}^{-1}).
	\qedhere
\end{align*}
\end{proof}
\end{theorem}

\begin{corollary}\label{theorem:逆矩阵.计算逆矩阵的方法}
如果方阵\(\vb{A}\)经\(t\)次初等行变换为\(\vb{E}_n\),
那么同样的初等行变换会将\(\vb{E}_n\)变为\(\vb{A}^{-1}\).
\end{corollary}

\begin{corollary}
可逆矩阵\(\vb{A}\)可以表示成若干个初等矩阵的乘积.
\end{corollary}

\begin{corollary}
\(n\)阶方阵\(\vb{A}\)可逆的充分必要条件是:
\(\vb{A}\)可经过一系列初等列变换变为\(\vb{E}_n\),
且同样的初等列变换将\(\begin{bmatrix}\vb{A}\\\vb{E}_n\end{bmatrix}\)变为
\(\begin{bmatrix}\vb{E}_n\\\vb{A}^{-1}\end{bmatrix}\).
\end{corollary}

当\(\vb{A}\)可逆时,我们可以利用初等行变换解矩阵方程\(\vb{A} \vb{X} = \vb{B}\):\[
	\vb{A}^{-1} (\vb{A},\vb{B})
	= (\vb{A}^{-1} \vb{A},\vb{A}^{-1} \vb{B})
	= (\vb{E},\vb{A}^{-1} \vb{B}),
\]
其中\(\vb{X} = \vb{A}^{-1} \vb{B}\)就是原方程的解.
\begin{example}
%@see: https://www.bilibili.com/video/BV1qT421275n/
设\[
	\vb{A} = \begin{bmatrix}
		1 & 1 & 1 \\
		2 & 1 & 0 \\
		1 & -1 & 0
	\end{bmatrix},
	\qquad
	\vb{B} = \begin{bmatrix}
		0 & 1 \\
		1 & 2 \\
		-1 & 1
	\end{bmatrix},
\]
解矩阵方程\(\vb{A} \vb{X} = \vb{B}\).
\begin{solution}
对\((\vb{A},\vb{B})\)作初等行变换得\[
	\begin{bmatrix}
		1 & 1 & 1 & 0 & 1 \\
		2 & 1 & 0 & 1 & 2 \\
		1 & -1 & 0 & -1 & 1
	\end{bmatrix}
	\to \begin{bmatrix}
		1 & & & 0 & 1 \\
		& 1 & & 1 & 0 \\
		& & 1 & -1 & 0
	\end{bmatrix},
\]
于是\(\vb{X} = \begin{bmatrix}
	0 & 1 \\
	1 & 0 \\
	-1 & 0
\end{bmatrix}\).
%@Mathematica: RowReduce[{{1, 1, 1, 0, 1}, {2, 1, 0, 1, 2}, {1, -1, 0, -1, 1}}]
\end{solution}
\end{example}

\begin{theorem}
设\(\vb{A}\)与\(\vb{B}\)都是\(s \times n\)矩阵,
则\(\vb{A}\)与\(\vb{B}\)等价的充分必要条件是:
存在\(s\)阶可逆矩阵\(\vb{P}\)与\(n\)阶可逆矩阵\(\vb{Q}\),使得\(\vb{B}=\vb{P}\vb{A}\vb{Q}\).
\end{theorem}

\begin{example}
初等矩阵的逆:\begin{gather*}
	[\vb{P}(i,j)]^{-1}=\vb{P}(i,j), \\
	[\vb{P}(i(c))]^{-1}=\vb{P}(i(c^{-1})), \\
	[\vb{P}(i,j(k))]^{-1}=\vb{P}(i,j(-k)).
\end{gather*}
\end{example}

\section{舒尔定理}
\begin{theorem}\label{theorem:逆矩阵.舒尔定理}
设\(\vb{A}\)是\(m\)阶可逆矩阵,
\(\vb{B},\vb{C},\vb{D}\)分别是\(m \times p, n \times m, n \times p\)矩阵,
则有\begin{gather}
	\begin{bmatrix}
		\vb{E}_m & \vb0 \\
		-\vb{C}\vb{A}^{-1} & \vb{E}_n
	\end{bmatrix}
	\begin{bmatrix}
		\vb{A} & \vb{B} \\
		\vb{C} & \vb{D}
	\end{bmatrix}
	= \begin{bmatrix}
		\vb{A} & \vb{B} \\
		\vb0 & \vb{D} - \vb{C} \vb{A}^{-1} \vb{B}
	\end{bmatrix},
	\\
	\begin{bmatrix}
		\vb{A} & \vb{B} \\
		\vb{C} & \vb{D}
	\end{bmatrix}
	\begin{bmatrix}
		\vb{E}_m & -\vb{A}^{-1} \vb{B} \\
		\vb0 & \vb{E}_p
	\end{bmatrix}
	= \begin{bmatrix}
		\vb{A} & \vb0 \\
		\vb{C} & \vb{D} - \vb{C} \vb{A}^{-1} \vb{B}
	\end{bmatrix},
	\\
	\begin{bmatrix}
		\vb{E}_m & \vb0 \\
		-\vb{C} \vb{A}^{-1} & \vb{E}_n
	\end{bmatrix}
	\begin{bmatrix}
		\vb{A} & \vb{B} \\
		\vb{C} & \vb{D}
	\end{bmatrix}
	\begin{bmatrix}
		\vb{E}_m & -\vb{A}^{-1} \vb{B} \\
		\vb0 & \vb{E}_p
	\end{bmatrix}
	= \begin{bmatrix}
		\vb{A} & \vb0 \\
		\vb0 & \vb{D} - \vb{C} \vb{A}^{-1} \vb{B}
	\end{bmatrix},
\end{gather}
\rm
其中\(\vb{D} - \vb{C} \vb{A}^{-1} \vb{B}\)
称为“矩阵\(\begin{bmatrix}
	\vb{A} & \vb{B} \\
	\vb{C} & \vb{D}
\end{bmatrix}\)
关于\(\vb{A}\)的\DefineConcept{舒尔补}(Schur complement)”.
%TODO proof
\end{theorem}
我们把\cref{theorem:逆矩阵.舒尔定理} 称为“舒尔定理”.

可以看到只要\(\vb{A}\)可逆,
就能通过初等分块矩阵直接对分块矩阵进行分块相似对角化,
而在操作过程中就把一些原本不为0的分块矩阵变成了零矩阵,
这个过程可以形象地称为“矩阵打洞”,
即让矩阵出现尽可能多的0.

利用\hyperref[theorem:逆矩阵.舒尔定理]{舒尔定理}可以证明下述行列式降阶定理:
\begin{theorem}[行列式降阶定理]\label{theorem:逆矩阵.行列式降阶定理}
设\(\vb{M} = \begin{bmatrix}
	\vb{A} & \vb{B} \\
	\vb{C} & \vb{D}
\end{bmatrix}\)是方阵.
\begin{itemize}
	\item 若\(\vb{A}\)可逆,则\begin{equation}\label{equation:逆矩阵.行列式降阶公式1}
		\abs{\vb{M}} = \abs{\vb{A}} \abs{\vb{D} - \vb{C} \vb{A}^{-1} \vb{B}}.
	\end{equation}

	\item 若\(\vb{D}\)可逆,则\begin{equation}\label{equation:逆矩阵.行列式降阶公式2}
		\abs{\vb{M}} = \abs{\vb{D}} \abs{\vb{A} - \vb{B} \vb{D}^{-1} \vb{C}}.
	\end{equation}

	\item 若\(\vb{A},\vb{D}\)均可逆,则\begin{equation}
		\abs{\vb{A}} \abs{\vb{D} - \vb{C} \vb{A}^{-1} \vb{B}}
		= \abs{\vb{D}} \abs{\vb{A} - \vb{B} \vb{D}^{-1} \vb{C}}.
	\end{equation}
\end{itemize}
\begin{proof}
由\cref{theorem:逆矩阵.舒尔定理,equation:行列式.广义三角阵的行列式1} 立即可得.
\end{proof}
\end{theorem}
\begin{remark}
这里给出一个有利于记忆\hyperref[theorem:逆矩阵.行列式降阶定理]{行列式降阶定理}的口诀:
对于\cref{equation:逆矩阵.行列式降阶公式1} 中等号右边的第二个行列式,
我们从\(\vb{D}\)出发,先以顺时针方向依次写出分块矩阵\(\vb{D},\vb{C},\vb{A},\vb{B}\),
再在第一个和第二个分块阵之间插入一个减号,最后取第三个分块阵的逆矩阵.
\cref{equation:逆矩阵.行列式降阶公式2} 也可以相同的记忆方法.
\end{remark}

\begin{example}\label{example:逆矩阵.行列式降阶定理的重要应用1}
设\(\vb{A} \in M_{s \times n}(K),
\vb{B} \in M_{n \times s}(K)\).
证明:\begin{equation*}
	\begin{vmatrix}
		\vb{E}_n & \vb{B} \\
		\vb{A} & \vb{E}_s
	\end{vmatrix}
	= \abs{\vb{E}_s - \vb{A} \vb{B}}
	= \abs{\vb{E}_n - \vb{B} \vb{A}}.
\end{equation*}
\begin{proof}
由于\(\vb{E}_n,\vb{E}_s\)都是单位矩阵,必可逆,
那么由\hyperref[equation:逆矩阵.行列式降阶公式1]{行列式降阶定理}有\begin{equation*}
	\begin{vmatrix}
		\vb{E}_n & \vb{B} \\
		\vb{A} & \vb{E}_s
	\end{vmatrix}
	= \abs{\vb{E}_n} \abs{\vb{E}_s - \vb{A}(\vb{E}_n)^{-1}\vb{B}}
	= \abs{\vb{E}_s - \vb{A} \vb{B}}.
\end{equation*}
同理,由\hyperref[equation:逆矩阵.行列式降阶公式2]{行列式降阶定理}有\begin{equation*}
	\begin{vmatrix}
		\vb{E}_n & \vb{B} \\
		\vb{A} & \vb{E}_s
	\end{vmatrix}
	= \abs{\vb{E}_n - \vb{B} \vb{A}}.
\end{equation*}
因此\(\abs{\vb{E}_s - \vb{A} \vb{B}} = \abs{\vb{E}_n - \vb{B} \vb{A}}\).
\end{proof}
%\cref{example:单位矩阵与两矩阵乘积之差.单位矩阵与两矩阵乘积之差的秩}
%\cref{example:单位矩阵与两矩阵乘积之差.单位矩阵与两矩阵乘积之差的行列式}
\end{example}

\begin{example}
%@see: https://www.bilibili.com/video/BV1ki4y1a7dL/
设\(\vb{A},\vb{B},\vb{C},\vb{D} \in M_n(K)\),
且\(\vb{A} \vb{C} = \vb{C} \vb{A}\),
则\begin{equation*}
	\begin{vmatrix}
		\vb{A} & \vb{B} \\
		\vb{C} & \vb{D}
	\end{vmatrix}
	= \abs{\vb{A} \vb{D} - \vb{C} \vb{B}}.
\end{equation*}
\begin{proof}
当\(\vb{A}\)可逆时,由\hyperref[equation:逆矩阵.行列式降阶公式1]{行列式降阶定理}有\begin{align*}
	\begin{vmatrix}
		\vb{A} & \vb{B} \\
		\vb{C} & \vb{D}
	\end{vmatrix}
	&= \abs{\vb{A}} \abs{\vb{D} - \vb{C} \vb{A}^{-1} \vb{B}} \\
	&= \abs{\vb{A} (\vb{D} - \vb{C} \vb{A}^{-1} \vb{B})} \\
	&= \abs{\vb{A} \vb{D} - \vb{A} \vb{C} \vb{A}^{-1} \vb{B}} \\
	&= \abs{\vb{A} \vb{D} - \vb{C} \vb{A} \vb{A}^{-1} \vb{B}} \\
	&= \abs{\vb{A} \vb{D} - \vb{C} \vb{B}}.
	\qedhere
\end{align*}
%TODO 还没有证明“\(\vb{A}\)不可逆”的情形
%@credit: {1ef6baa6-d8ce-4b2c-97d3-f6156721c52f} 说可以用摄动法
\end{proof}
\end{example}

\begin{example}
\def\M{\vb{M}}
求解行列式\(\det \M\),其中\begin{equation*}
	\M = \begin{bmatrix}
		1+a_1 b_1 & a_1 b_2 & \dots & a_1 b_n \\
		a_2 b_1 & 1+a_2 b_2 & \dots & a_2 b_n \\
		\vdots & \vdots & & \vdots \\
		a_n b_1 & a_n b_2 & \dots & 1+a_n b_n
	\end{bmatrix}.
\end{equation*}
\begin{solution}
记\(\vb\alpha = (\AutoTuple{a}{n})^T,
\vb\beta = (\AutoTuple{b}{n})\),
则\(\M = \vb{E}_n + \vb\alpha\vb\beta\).

注意到\(\M = \vb{E}_n + \vb\alpha\vb\beta\)形似某个矩阵的舒尔补,因此考虑下面的矩阵:\begin{equation*}
	\vb{A} = \begin{bmatrix}
		1 & -\vb\beta \\
		\vb\alpha & \vb{E}_n
	\end{bmatrix}.
\end{equation*}由于\begin{equation*}
	\begin{bmatrix}
		1 & \vb0 \\
		-\vb\alpha & \vb{E}_n
	\end{bmatrix} \vb{A}
	= \begin{bmatrix}
		1 & \vb0 \\
		-\vb\alpha & \vb{E}_n
	\end{bmatrix}
	\begin{bmatrix}
		1 & -\vb\beta \\
		\vb\alpha & \vb{E}_n
	\end{bmatrix}
	= \begin{bmatrix}
		1 & -\vb\beta \\
		\vb0 & \M
	\end{bmatrix},
\end{equation*}
且\begin{equation*}
	\begin{bmatrix}
		1 & \vb\beta \\
		\vb0 & \vb{E}_n
	\end{bmatrix} \vb{A}
	= \begin{bmatrix}
		1 & \vb\beta \\
		\vb0 & \vb{E}_n
	\end{bmatrix}
	\begin{bmatrix}
		1 & -\vb\beta \\
		\vb\alpha & \vb{E}_n
	\end{bmatrix}
	= \begin{bmatrix}
		1+\vb\beta\vb\alpha & \vb0 \\
		\vb\alpha & \vb{E}_n
	\end{bmatrix},
\end{equation*}
所以\begin{equation*}
	\abs{\vb{A}}
	= \abs{\M}
	= \abs{1+\vb\beta\vb\alpha}
	= 1+\vb\beta\vb\alpha
	= 1 + \sum_{k=1}^n a_k b_k.
\end{equation*}
\end{solution}
\end{example}

\section{本章总结}
\subsection*{矩阵可逆的等价条件}
%@see: https://mathworld.wolfram.com/InvertibleMatrixTheorem.html
矩阵\(\vb{A} \in M_n(K)\)可逆的充分必要条件:\begin{itemize}
	\item 矩阵\(\vb{A}\)等价于数域\(K\)上的\(n\)阶单位矩阵.
	\item 存在数域\(K\)上的\(n\)阶矩阵\(\vb{B}\),使得\(\vb{B}\vb{A}\)或\(\vb{A}\vb{B}\)等于数域\(K\)上的\(n\)阶单位矩阵.
	\item 方程\(\vb{A}\vb{x}=\vb0\)只有零解\(\vb{x}=\vb0\).
	\item 矩阵\(\vb{A}\)的列向量线性无关.
	\item 线性变换\(\vb{x} \mapsto \vb{A}\vb{x}\)是双射.
	\item 线性变换\(\vb{x} \mapsto \vb{A}\vb{x}\)是满射.
	\item 对于任意向量\(\vb\beta \in K^n\),方程\(\vb{A}\vb{x}=\vb\beta\)有唯一解.
	\item 矩阵\(\vb{A}\)的行空间是\(K^n\).
	\item 矩阵\(\vb{A}\)的列空间是\(K^n\).
	\item 矩阵\(\vb{A}\)的列向量组是向量空间\(K^n\)的一组基.
	\item 矩阵\(\vb{A}\)的列向量组可以张成向量空间\(K^n\).
	\item 矩阵\(\vb{A}\)的转置矩阵\(\vb{A}^T\)可逆.
	\item 矩阵\(\vb{A}\)的列空间的维数等于\(n\).
	\item 矩阵\(\vb{A}\)的秩等于\(n\).
	\item 矩阵\(\vb{A}\)的零空间是\(\{\vb0\}\).
	\item 矩阵\(\vb{A}\)的零空间的维数等于\(0\).
	\item \(0\)不是矩阵\(\vb{A}\)的特征值.
	\item 矩阵\(\vb{A}\)的行列式不等于零.
	\item 矩阵\(\vb{A}\)是非奇异矩阵.
	% \item 矩阵\(\vb{A}\)的列空间的正交补是\(\{\vb0\}\).
	% \item 矩阵\(\vb{A}\)的零空间的正交补是向量空间\(K^n\).
\end{itemize}

\subsection*{重要公式}
假设\(\vb{A},\vb{B}\)可逆,则\begin{gather*}
	\vb{A}^{-1} = \abs{\vb{A}}^{-1} \vb{A}^*, \\ %\cref{theorem:逆矩阵.逆矩阵的唯一性}
	\abs{\vb{A}^{-1}} = \abs{\vb{A}}^{-1}, \\ %\cref{theorem:逆矩阵.逆矩阵的行列式}
	(\vb{A}^{-1})^{-1} = \vb{A}, \\ %\cref{theorem:逆矩阵.逆矩阵的逆}
	(\vb{A} \vb{B})^{-1} = \vb{B}^{-1} \vb{A}^{-1}, \\ %\cref{theorem:逆矩阵.矩阵乘积的逆1,theorem:逆矩阵.矩阵乘积的逆2}
	(k \vb{A})^{-1} = k^{-1} \vb{A}^{-1}, \\ %\cref{theorem:逆矩阵.数与矩阵乘积的逆}
	(\vb{A}^T)^{-1} = (\vb{A}^{-1})^T, \\ %\cref{theorem:逆矩阵.转置矩阵的逆与逆矩阵的转置}
	\diag(\AutoTuple{\lambda}{n}) = \diag(\AutoTuple{\lambda^{-1}}{n}), \\
	(\vb{A}^*)^{-1}
	= \abs{\vb{A}}^{-1} \vb{A}
	= (\vb{A}^{-1})^*, \\ %\cref{theorem:逆矩阵.伴随矩阵的逆与逆矩阵的伴随}
	%\cref{example:可逆矩阵.分块上三角矩阵的逆}
	\begin{bmatrix}
		\vb{A} & \vb{C} \\
		\vb0 & \vb{B}
	\end{bmatrix}^{-1}
	= \begin{bmatrix}
		\vb{A}^{-1} & -\vb{A}^{-1} \vb{C} \vb{B}^{-1} \\
		\vb0 & \vb{B}^{-1}
	\end{bmatrix}, \\
	\begin{bmatrix}
		\vb{A} & \vb0 \\
		\vb{D} & \vb{B}
	\end{bmatrix}^{-1}
	= \begin{bmatrix}
		\vb{A}^{-1} & \vb0 \\
		-\vb{B}^{-1} \vb{D} \vb{A}^{-1} & \vb{B}^{-1}
	\end{bmatrix}, \\
	\begin{bmatrix}
		\vb{C} & \vb{A} \\
		\vb{B} & \vb0
	\end{bmatrix}^{-1}
	= \begin{bmatrix}
		\vb0 & \vb{B}^{-1} \\
		\vb{A}^{-1} & -\vb{A}^{-1}\vb{C}\vb{B}^{-1}
	\end{bmatrix}, \\
	\begin{bmatrix}
		\vb0 & \vb{A} \\
		\vb{B} & \vb{D}
	\end{bmatrix}^{-1}
	= \begin{bmatrix}
		-\vb{B}^{-1}\vb{D}\vb{A}^{-1} & \vb{B}^{-1} \\
		\vb{A}^{-1} & \vb0
	\end{bmatrix}, \\
	\begin{bmatrix}
		\vb{A} & \vb{C} \\
		\vb0 & \vb{B}
	\end{bmatrix}^*
	= \begin{bmatrix}
		\abs{\vb{B}} \vb{A}^* & -\vb{A}^*\vb{C}\vb{B}^* \\
		\vb0 & \abs{\vb{A}} \vb{B}^*
	\end{bmatrix}, \\
	\begin{bmatrix}
		\vb{A} & \vb0 \\
		\vb{D} & \vb{B}
	\end{bmatrix}^*
	= \begin{bmatrix}
		\abs{\vb{B}} \vb{A}^* & \vb0 \\
		-\vb{B}^* \vb{D} \vb{A}^* & \abs{\vb{A}} \vb{B}^*
	\end{bmatrix}, \\
	\vb{A} \in M_s(K),
	\vb{B} \in M_n(K)
	\implies
	\begin{bmatrix}
		\vb{C} & \vb{A} \\
		\vb{B} & \vb0
	\end{bmatrix}^*
	= (-1)^{sn} \begin{bmatrix}
		\vb0 & \abs{\vb{A}} \vb{B}^* \\
		\abs{\vb{B}} \vb{A}^* & -\vb{A}^*\vb{C}\vb{B}^*
	\end{bmatrix}, \\
	\vb{A} \in M_s(K),
	\vb{B} \in M_n(K)
	\implies
	\begin{bmatrix}
		\vb0 & \vb{A} \\
		\vb{B} & \vb{D}
	\end{bmatrix}^*
	= (-1)^{sn} \begin{bmatrix}
		-\vb{B}^*\vb{D}\vb{A}^* & \abs{\vb{A}} \vb{B}^* \\
		\abs{\vb{B}} \vb{A}^* & \vb0
	\end{bmatrix}. \\
	%\cref{equation:逆矩阵.行列式降阶公式1}
	\text{$\vb{A}$可逆}
	\implies
	\begin{vmatrix}
		\vb{A} & \vb{B} \\
		\vb{C} & \vb{D}
	\end{vmatrix}
	= \abs{\vb{A}} \abs{\vb{D} - \vb{C} \vb{A}^{-1} \vb{B}}, \\
	%\cref{equation:逆矩阵.行列式降阶公式2}
	\text{$\vb{D}$可逆}
	\implies
	\begin{vmatrix}
		\vb{A} & \vb{B} \\
		\vb{C} & \vb{D}
	\end{vmatrix}
	= \abs{\vb{D}} \abs{\vb{A} - \vb{B} \vb{D}^{-1} \vb{C}}. \\
	%\cref{example:逆矩阵.行列式降阶定理的重要应用1}
	\vb{A} \in M_{s \times n}(K),
	\vb{B} \in M_{n \times s}(K)
	\implies
	\begin{vmatrix}
		\vb{E}_n & \vb{B} \\
		\vb{A} & \vb{E}_s
	\end{vmatrix}
	= \abs{\vb{E}_s - \vb{A}\vb{B}}.
\end{gather*}


\chapter{向量空间}
\section{向量空间}
为了直接用线性方程组的系数和常数项判断方程组有没有解,有多少解,
我们在前面给出了用系数行列式判断\(n\)个方程的\(n\)元线性方程组有唯一解的充分必要条件.
这一判定方法只适用于方程数目与未知量数目相等的线性方程组;
而且,当系数行列式等于零时,只能得出方程组无解或有无穷多解的结论,
没有办法区分什么时候无解,什么时候有无穷多解.
对于任意的线性方程组,有没有这样一种判定方法:
直接依据它的系数和常数项,给出它有没有解,有多少解呢?
为此我们需要探讨和建立线性方程组的进一步的理论.
这一理论还将使我们弄清楚线性方程组有无穷多个解时解的结构.

\subsection{向量空间}
设\(K\)是数域,\(n\)是任意给定的一个正整数.
令\[
	K^n \defeq \Set{ (\AutoTuple{a}{n}) \given a_i \in K\ (i=1,2,\dotsc,n) }.
\]

如果\[
	a_i=b_i
	\quad(i=1,2,\dotsc,n),
\]
则称“\(K^n\)中的两个元素\((\AutoTuple{a}{n})\)与\((\AutoTuple{b}{n})\)相等”.

在\(K^n\)中规定“加法”运算如下:
\begin{equation}\label{equation:向量空间.向量的加法.定义式}
	(\AutoTuple{a}{n}) + (\AutoTuple{b}{n})
	\defeq (a_1+b_1,a_2+b_2,\dotsc,a_n+b_n).
\end{equation}

在\(K\)的元素与\(K^n\)的元素之间规定“数量乘法”运算如下:
\begin{equation}\label{equation:向量空间.向量的数量乘法.定义式}
	k (\AutoTuple{a}{n})
	\defeq (k a_1,k a_2,\dotsc,k a_n).
\end{equation}

容易验证,上述加法和数量乘法满足下述8条运算法则:
\begin{enumerate}
	\item 加法交换律,即\((\forall \a,\b \in K^n)[\a+\b=\b+\a]\).

	\item 加法结合律,即\((\forall \a,\b,\g \in K^n)[(\a+\b)+\g=\a+(\b+\g)]\).

	\item 记\(\z=(0,0,\dotsc,0)\),\[
		(\forall\a \in K^n)[\a+\z = \z+\a = \a].
	\]
	称\(\z\)为“\(K^n\)的\DefineConcept{零元}(zero element)”.

	\item \(\forall\a=(\AutoTuple{a}{n}) \in K^n\),令\[
		-\a \defeq (\AutoTuple{-a}{n}),
	\]
	则\(-\a \in K^n\)且\[
		\a+-\a
		= -\a+\a
		= \z;
	\]
	称\(-\a\)为“\(\a\)的\DefineConcept{负元}(negative element)”.

	\item \((\forall \a \in K^n)[1 \a=\a]\).

	\item \((\forall \a \in K^n)(\forall k,l \in K)[k (l \a)=(kl) \a]\).

	\item \((\forall \a \in K^n)(\forall k,l \in K)[(k+l) \a=k \a+l \a]\).

	\item \((\forall \a,\b \in K^n)(\forall k \in K)[k (\a+\b)=k \a+k \b]\).
\end{enumerate}

\begin{definition}
数域\(K\)上全体\(n\)元组组成的集合\(K^n\),
连同定义在它上面的加法运算和数量乘法运算,
及其满足的8条运算法则一起,
称为“数域\(K\)上的一个\(n\)维\DefineConcept{向量空间}(vector space)”.
\(K^n\)的元素称为“\(n\)维\DefineConcept{向量}(vector)”.

对于\(K^n\)中的任意一个向量\(\a=(\AutoTuple{a}{n})\),
称数\[
	a_i\quad(i=1,2,\dotsc,n)
\]为“\(\a\)的第\(i\)个\DefineConcept{分量}”.
\end{definition}

在\(n\)维向量空间\(K^n\)中,我们可以额外定义减法运算如下:
\begin{equation}\label{equation:向量空间.向量的减法.定义式}
	\a-\b \defeq \a+(-\b).
\end{equation}

在\(n\)维向量空间\(K^n\)中,容易验证下述4条性质:
\begin{property}
\((\forall\a \in K^n)[0\cdot\a=\z]\).
\end{property}

\begin{property}
\((\forall\a \in K^n)[(-1)\cdot\a=-\a]\).
\end{property}

\begin{property}
\((\forall k \in K)[k\z=\z]\).
\end{property}

\begin{property}
\(k\a=\z \implies k=0 \lor \a=\z\).
\end{property}

把\(n\)元组写成一行,得\[
	(\AutoTuple{a}{n})
	\quad\text{或}\quad
	\begin{bmatrix}
		a_1 & a_2 & \dots & a_n
	\end{bmatrix},
\]
称之为“\(n\)维\DefineConcept{行向量}(row vector)”.

把\(n\)元组写成一列,得\[
	\begin{bmatrix} a_1 \\ a_2 \\ \vdots \\ a_n \end{bmatrix},
\]
称之为“\(n\)维\DefineConcept{列向量}(column vector)”;
不过,我们有时候会为了方便排版,把列向量写成\[
	(a_1,a_2,\dotsc,a_n)^T.
\]

\(K^n\)可以看成是
全体\(n\)维行向量
组成的向量空间,
也可以看成是
全体\(n\)维列向量
组成的向量空间.
两者并没有本质的区别,
只是它们的元素的写法不同而已.

由有限个\(n\)维行向量构成的集合,
称为“\(n\)维\DefineConcept{行向量组}”.
由有限个\(n\)维列向量构成的集合,
称为“\(n\)维\DefineConcept{列向量组}”.
\(n\)维行向量组和\(n\)维列向量组
统称\(n\)维\DefineConcept{向量组}.
从本质上看,向量组就是\(n\)维向量空间\(K^n\)的有限子集.

称满足
\[
	e_{ij} = \left\{ \begin{array}{ll}
		1, & i=j, \\
		0, & i \neq j
	\end{array} \right.
\]
的向量组
\[
	\e_i = \begin{bmatrix}
		e_{1i} \\ e_{2i} \\ \vdots \\ e_{ni}
	\end{bmatrix}
	\quad(i=1,2,\dotsc,n)
\]为“\(K^n\)的\DefineConcept{基本向量组}”.

\subsection{线性组合,线性表出}
%@see: 《线性代数》(张慎语、周厚隆) P67 定义5
\begin{definition}\label{definition:向量空间.线性组合}
在\(K^n\)中,给定向量组\(A=\{\AutoTuple{\a}{s}\}\).
任给\(K\)中一组数\(\AutoTuple{k}{s}\),
我们把\[
	k_1 \a_1 + k_2 \a_2 + \dotsb + k_s \a_s
\]
称为“向量组\(A\)的一个\DefineConcept{线性组合}(linear combination)”,
把\(\AutoTuple{k}{s}\)称为\DefineConcept{系数}.
\end{definition}

\begin{definition}\label{definition:向量空间.线性表出1}
在\(K^n\)中,给定向量组\(A=\{\AutoTuple{\a}{s}\}\).
对于向量\(\b \in K^n\),
如果存在\(K\)中一组数\(\AutoTuple{c}{s}\),
使得\[
	\b = c_1 \a_1 + c_2 \a_2 + \dotsb + c_s \a_s,
\]
则称“向量\(\b\)可由向量组\(A\)~\DefineConcept{线性表出}”;
否则称“向量\(\b\)不可由向量组\(A\)线性表出”.
\end{definition}

现在,利用向量的加法运算和数量乘法运算,
我们可以把数域\(K\)上\(n\)元线性方程组 \labelcref{equation:线性方程组.线性方程组的代数形式}
写成
\begin{equation}\label{equation:线性方程组.线性方程组的向量形式}
	x_1 \a_1 + x_2 \a_2 + \dotsb + x_n \a_n = \b,
\end{equation}
其中\[
	\a_j=(a_{1j},a_{2j},\dotsc,a_{sj})^T,
	\quad
	j=1,2,\dotsc,n.
\]
于是,\begin{align*}
	&\text{数域\(K\)上线性方程组\(x_1 \a_1 + x_2 \a_2 + \dotsb + x_n \a_n = \b\)有解} \\
	&\iff \text{\(K\)中存在一组数\(\AutoTuple{c}{n}\),使得\(c_1 \a_1 + c_2 \a_2 + \dotsb + c_n \a_n = \b\)成立} \\
	&\iff \text{\(\b\)可以由\(\AutoTuple{\a}{n}\)线性表出}.
\end{align*}
这样我们把线性方程组有没有解的问题归结为:
常数项列向量\(\b\)能不能由系数矩阵的列向量组线性表出.
这个结论有两方面的意义:
一方面,为了从理论上研究线性方程组有没有解,
就需要去研究\(\b\)能否由\(\AutoTuple{\a}{n}\)线性表出;
另一方面,对于\(K^n\)中给定的向量组\(\AutoTuple{\a}{n}\),
以及给定的\(\b\),
为了判断\(\b\)能否由\(\AutoTuple{\a}{n}\)线性表出,
就可以去判断线性方程组\(x_1 \a_1 + x_2 \a_2 + \dotsb + x_n \a_n = \b\)是否有解.

\subsection{线性子空间}
在\(K^n\)中,从理论上如何判断任一向量\(\b\)能否由向量组\(\AutoTuple{\a}{n}\)线性表出?
从线性表出的定义知道,这需要考察\(\b\)是否等于\(\AutoTuple{\a}{n}\)的某一个线性组合.
为此,我们把\(\AutoTuple{\a}{n}\)的所有线性组合组成一个集合\(W\),即\[
	W \defeq \Set{ k_1 \a_1 + k_2 \a_2 + \dotsb + k_s \a_s \given k_i \in K, i=1,2,\dotsc,s }.
\]
如果我们能够把\(W\)的结构研究清楚,那么就比较容易判断\(\b\)是否属于\(W\),
也就是判断\(\b\)能否由\(\AutoTuple{\a}{n}\)线性表出.

现在我们来研究\(W\)的结构.
任取\(\a,\g\in W\),设\[
	\a=a_1\a_1+a_2\a_2+\dotsb+a_s\a_s, \qquad
	\g=b_1\a_1+b_2\a_2+\dotsb+b_s\a_s,
\]
则\begin{align*}
	\a+\g
	&=(a_1\a_1+a_2\a_2+\dotsb+a_s\a_s)+(b_1\a_1+b_2\a_2+\dotsb+b_s\a_s) \\
	&=(a_1+b_1)\a_1+(a_2+b_2)\a_2+\dotsb+(a_s+b_s)\a_s,
\end{align*}
从而\(\a+\g\in W\).

再任取\(k\in W\),则\begin{align*}
	k\a
	&=k(a_1\a_1+a_2\a_2+\dotsb+a_s\a_s) \\
	&=(ka_1)\a_1+(ka_2)\a_2+\dotsb+(ka_s)\a_s,
\end{align*}
从而\(k\a\in W\).

受此启发,我们引出如下概念.
\begin{definition}
\(K^n\)的一个非空子集\(U\)如果满足:
\begin{enumerate}
	\item \(U\)对\(K^n\)的加法封闭,即\[
		(\forall \a,\b \in U)[\a+\b \in U];
	\]
	\item \(U\)对\(K^n\)的数量乘法封闭,即\[
		(\a \in U)(k \in K)[k\a \in U];
	\]
\end{enumerate}
那么称\(U\)是“\(K^n\)的一个\DefineConcept{线性子空间}(linear subspace)”,
简称为\DefineConcept{子空间}(subspace).
\end{definition}
零空间\(\{\vb0\}\)是\(K^n\)的一个子空间,
因此我们又称之为“\(K^n\)的\DefineConcept{零子空间}(zero subspace)”.

类\(\Set{ x \given x\neq\{\vb0\} \land \text{\(x\)是\(K^n\)的子空间} }\)%
中的每一个\(x\)都称为“\(K^n\)的\DefineConcept{非零子空间}”.

\(K^n\)也是其自身的一个子空间.

\begin{proposition}
任意一个线性子空间总含有零向量.
\begin{proof}
假设存在一个线性子空间\(U\),不含有零向量.
由于\(U\)非空,不妨设非零向量\(\a\)是\(U\)的元素,即\(\a \in U\).
那么根据线性子空间的定义,有\[
	(-1)\cdot\a = -\a \in U.
\]
从而\[
	\vb0 = \a+(-\a) \in U.
\]
矛盾!
因此,\((\forall U)[\text{\(U\)是线性子空间} \implies \vb0 \in U]\).
\end{proof}
\end{proposition}

从上面的讨论知道,在\(K^n\)中,
向量组\(A=\{\AutoTuple{\a}{s}\}\)的所有线性组合组成的集合\(W\)是\(K^n\)的一个子空间,
称它为“\(\AutoTuple{\a}{s}\)生成的子空间”,
或“\(\AutoTuple{\a}{s}\)的\DefineConcept{线性生成空间}(linear span)”,
记作\(\opair{\AutoTuple{\a}{s}}\)或\(\Span A\),即\[
	\Span A
	\defeq
	\Set*{
		\sum_{i=1}^s k_i \a_i
		\given
		\AutoTuple{k}{s} \in K
	}.
\]
%@see: https://mathworld.wolfram.com/VectorSpaceSpan.html
%@see: https://math.stackexchange.com/questions/185255/span-of-an-empty-set-is-the-zero-vector/

%@see: 《代数学(一)》(李方、邓少强、冯荣权、刘东文) P102
易见\(\Span\emptyset = \{\vb0\}\).

于是,我们得出结论,以下三个命题等价:
\begin{enumerate}
	\item 数域\(K\)上的\(n\)元线性方程组\(x_1 \a_1 + x_2 \a_2 + \dotsb + x_n \a_n = \b\)有解.
	\item 向量\(\b\)可以由向量组\(A=\{\AutoTuple{\a}{n}\}\)线性表出.
	\item 向量\(\b\in\Span A=\opair{\AutoTuple{\a}{n}}\).
\end{enumerate}

\begin{theorem}\label{theorem:向量空间.任一向量可由基本向量组唯一线性表出}
\(K^n\)中任一向量都可由基本向量组唯一地线性表出.
\begin{proof}
对于任意一个向量\(\a=(\AutoTuple{a}{n})^T\),
线性方程组\(x_1 \e_1 + x_2 \e_2 + \dotsb + x_n \e_n = \a\)的系数行列式为
\[
\begin{vmatrix}
	1 & 0 & \dots & 0 \\
	0 & 1 & \dots & 0 \\
	\vdots & \vdots & & \vdots \\
	0 & 0 & \dots & 1
\end{vmatrix}
= 1 \neq 0,
\]
那么,根据\hyperref[theorem:线性方程组.克拉默法则]{克拉默法则},
上述线性方程组有唯一解.
由此可知,\(K^n\)中任一向量\(\a\)都能由基本向量组线性表出,且表出方式唯一.
事实上,由于\[
	a_1 \begin{bmatrix}
		1 \\ 0 \\ 0 \\ \vdots \\ 0
	\end{bmatrix}
	+ a_2 \begin{bmatrix}
		0 \\ 1 \\ 0 \\ \vdots \\ 0
	\end{bmatrix}
	+ \dotsb + a_n \begin{bmatrix}
		0 \\ 0 \\ 0 \\ \vdots \\ 1
	\end{bmatrix}
	= \begin{bmatrix}
		a_1 \\ a_2 \\ a_3 \\ \vdots \\ a_n
	\end{bmatrix},
\]
因此,用基本向量组标出向量\(\a\)的方式为\[
	\a = a_1 \e_1 + a_2 \e_2 + \dotsb + a_n \e_n.
	\qedhere
\]
\end{proof}
\end{theorem}

\section{向量组的线性相关性}
在上一节,我们把线性方程组有没有解的问题归结为:
常数项列向量\(\vb\beta\)能否由系数矩阵的列向量组\(\AutoTuple{\vb\alpha}{s}\)线性表出.
那么,如何研究\(K^n\)中一个向量能不能由一个向量组线性表出呢?

\subsection{线性相关性的概念}
我们首先回顾\cref{theorem:解析几何.两向量共线的充分必要条件1,%
theorem:解析几何.三向量共面的充分必要条件1},
以及\cref{theorem:解析几何.两向量不共线的充分必要条件1,%
theorem:解析几何.三向量不共面的充分必要条件1}.

受此启发,我们提出以下两个概念.
\begin{definition}\label{definition:线性方程组.线性相关与线性无关的定义}
%@see: 《线性代数》(张慎语、周厚隆) P68 定义6
设\(A=\Set{\AutoTuple{\vb\alpha}{s}}\)是\(n\)维向量空间\(K^n\)中的一个向量组.

如果\(K\)中存在不全为零的数\(\AutoTuple{k}{s}\),使得\begin{equation*}
	k_1 \vb\alpha_1 + k_2 \vb\alpha_2 + \dotsb + k_s \vb\alpha_s = \vb0,
\end{equation*}
则称“向量组\(A\) \DefineConcept{线性相关}(linearly dependent)”;
否则,称“向量组\(A\) \DefineConcept{线性无关}(linearly independent)”.
%@see: https://mathworld.wolfram.com/LinearlyIndependent.html
\end{definition}

显然,从\cref{definition:线性方程组.线性相关与线性无关的定义} 立即可得
\begin{equation*}
	\text{向量组\(A\)线性无关}
	\iff
	[k_1 \vb\alpha_1 + k_2 \vb\alpha_2 + \dotsb + k_s \vb\alpha_s = \vb0
	\implies
	(\AutoTuple{k}{s}) = \vb0].
\end{equation*}

特别地,我们规定:
\begin{axiom}
空集\(\emptyset\)线性无关.
\end{axiom}

\subsection{线性相关性的判定条件}
根据线性相关、线性无关的定义和解析几何的结论,
在几何空间中,共线的两个向量是线性相关的,
共面的三个向量是线性相关的,
不共面的三个向量是线性无关的,
不共线的两个向量是线性无关的.

下面我们再来看几个例子.
\begin{example}\label{example:线性方程组.含有零向量的向量组线性相关}
%@see: 《线性代数》(张慎语、周厚隆) P68 例1
向量空间\(K^n\)中的零向量可以由任意向量组\(\AutoTuple{\vb\beta}{t}\)线性表出,
这是因为恒等式\begin{equation*}
	0\vb\beta_1+0\vb\beta_2+\dotsb+0\vb\beta_t=\vb0.
\end{equation*}
进一步,
含有零向量\(\vb0\)的向量组\begin{equation*}
	\Set{\vb0,\AutoTuple{\vb\alpha}{s}}
\end{equation*}总是线性相关的,
这是因为\begin{equation*}
	1 \vb0 + 0 \vb\alpha_1 + 0 \vb\alpha_2 + \dotsb + 0 \vb\alpha_s = \vb0.
\end{equation*}
\end{example}

\begin{example}\label{example:线性方程组.基本向量组线性无关}
%@see: 《线性代数》(张慎语、周厚隆) P68 例2
\(K^n\)的基本向量组\(\AutoTuple{\vb\epsilon}{n}\)线性无关.
\begin{proof}
令\(k_1 \vb\epsilon_1 + k_2 \vb\epsilon_2 + \dotsb + k_n \vb\epsilon_n = \vb0\),即\begin{equation*}
	k_1 (1,0,\dotsc,0)^T + k_2 (0,1,\dotsc,0)^T + \dotsb k_n (0,0,\dotsc,1)^T = \vb0.
\end{equation*}
进一步,有\begin{equation*}
	(\AutoTuple{k}{n})^T = (0,\dotsc,0)^T,
\end{equation*}
于是\(k_1 = k_2 = \dotsb = k_n = 0\),
因此\(\vb\epsilon_1,\vb\epsilon_2,\dotsc,\vb\epsilon_n\)线性无关.
\end{proof}
\end{example}

\begin{proposition}\label{theorem:线性方程组.单向量组线性相关的充分必要条件}
设\(\vb\alpha \in K^n\),
则\begin{align*}
	\text{向量组\(\{\vb\alpha\}\)线性相关}
	\iff
	\vb\alpha=\vb0, \\
	\text{向量组\(\{\vb\alpha\}\)线性无关}
	\iff
	\vb\alpha\neq\vb0.
\end{align*}
\begin{proof}
必要性.
设\(\{\vb\alpha\}\)线性相关,存在数\(k \neq 0\)使得\(k\vb\alpha = \vb0\),可得\(\vb\alpha = \vb0\).

充分性.
设\(\vb\alpha = \vb0\),则\(1\vb\alpha = \vb0\),而数\(1 \neq 0\),故\(\{\vb\alpha\}\)线性相关.
\end{proof}
\end{proposition}

\begin{theorem}\label{theorem:线性方程组.向量组线性相关的充分必要条件1}
设向量组\(A=\{\AutoTuple{\vb\alpha}{s}\}\ (s>1)\),
则\begin{align*}
	\text{\(A\)线性相关}
	&\iff
	\text{\(A\)中至少有一个向量可由其余\(s-1\)个向量线性表出} \\
	&\iff
	(\exists \vb\alpha\in A)[\vb\alpha \in \Span(A-\{\vb\alpha\})], \\
	\text{\(A\)线性无关}
	&\iff
	\text{\(A\)中每一个向量都不能由其余向量线性表出} \\
	&\iff
	(\forall \vb\alpha\in A)[\vb\alpha \notin \Span(A-\{\vb\alpha\})].
\end{align*}
\begin{proof}
必要性.
\(A\)线性相关,则存在不全为零的数\(\AutoTuple{k}{s}\),使得\begin{equation*}
	k_1 \vb\alpha_1 + k_2 \vb\alpha_2 + \dotsb + k_s \vb\alpha_s = \vb0.
\end{equation*}
设\(k_i\neq0\ (1 \leq i \leq s)\),于是\begin{equation*}
	\vb\alpha_i = -\frac{1}{k_i} (
		k_1 \vb\alpha_1 + k_2 \vb\alpha_2 + \dotsb
		+ k_{i-1} \vb\alpha_{i-1} + k_{i+1} \vb\alpha_{i+1}
		+ \dotsb + k_s \vb\alpha_s
	),
\end{equation*}
即\(\vb\alpha_i\)可由其余\(s-1\)个向量线性表出.

充分性.
若\(\vb\alpha_j \in A\)可由其余\(s-1\)个向量线性表出,即\begin{equation*}
	\vb\alpha_j = l_1 \vb\alpha_1 + \dotsb + l_{j-1} \vb\alpha_{j-1} + l_{j+1} \vb\alpha_{j+1} + \dotsb + l_s \vb\alpha_s,
\end{equation*}
移项得\begin{equation*}
	l_1 \vb\alpha_1 + \dotsb
	+ l_{j-1} \vb\alpha_{j-1} + (-1) \vb\alpha_j + l_{j+1} \vb\alpha_{j+1}
	+ \dotsb + l_s \vb\alpha_s = \vb0,
\end{equation*}
上式等号左边的系数中至少有一个数\(-1\neq0\),
因此\(A\)线性相关.
\end{proof}
\end{theorem}


\begin{theorem}\label{theorem:向量空间.增加一个向量对线性相关性的影响1}
%@see: 《线性代数》(张慎语、周厚隆) P68 例4
%@see: 《高等代数(第三版 上册)》(丘维声) P69 命题1
%@see: 《高等代数(第三版 上册)》(丘维声) P69 推论2
设向量组\(A\)线性无关,
则\begin{align*}
	\text{向量\(\vb\beta\)可以由\(A\)线性表出}
	\iff
	\text{向量组\(B=A\cup\{\vb\beta\}\)线性相关}, \\
	\text{向量\(\vb\beta\)不能由\(A\)线性表出}
	\iff
	\text{向量组\(B=A\cup\{\vb\beta\}\)线性无关}.
\end{align*}
\begin{proof}
先证必要性.
由\cref{theorem:线性方程组.向量组线性相关的充分必要条件1} 可知,
“\(\vb\beta\)可以由\(A\)线性表出”显然蕴含“\(B\)线性相关”.

再证充分性.
由于向量组\(B\)线性相关,
则存在不全为零的数\(\AutoTuple{k}{s},k\)使得\begin{equation*}
	k_1 \vb\alpha_1 + k_2 \vb\alpha_2 + \dotsb + k_s \vb\alpha_s + k \vb\beta = \vb0.
\end{equation*}
用反证法.
假设\(k = 0\),
则\(\AutoTuple{k}{s}\)不全为零,
且有\(k_1 \vb\alpha_1 + k_2 \vb\alpha_2 + \dotsb + k_s \vb\alpha_s = \vb0\),
即\(A\)线性相关,
与题设矛盾,说明\(k \neq 0\).
于是\begin{equation*}
	\vb\beta = -\frac{1}{k} (k_1 \vb\alpha_1 + k_2 \vb\alpha_2 + \dotsb + k_s \vb\alpha_s).
	\qedhere
\end{equation*}
\end{proof}
\end{theorem}

\begin{theorem}\label{theorem:线性方程组.部分组线性相关则全组线性相关}
若向量组\(A\)的一个部分组线性相关,则\(A\)线性相关.
\begin{proof}
设\(A=\{\AutoTuple{\vb\alpha}{s}\}\).
假设\(A\)的部分组\(B=\{\AutoTuple{\vb\alpha}{t}\}\ (t \leq s)\)线性相关,
即存在不全为零的数\(\AutoTuple{k}{t}\)使得\begin{equation*}
	k_1 \vb\alpha_1 + k_2 \vb\alpha_2 + \dotsb + k_t \vb\alpha_t = \vb0;
\end{equation*}
从而有\begin{equation*}
	k_1 \vb\alpha_1 + k_2 \vb\alpha_2 + \dotsb + k_t \vb\alpha_t + 0 \vb\alpha_{t+1} + \dotsb + 0 \vb\alpha_s = \vb0;
\end{equation*}
由于上式等号左边的系数\(\AutoTuple{k}{t},0,\dotsc,0\)不全为零,
因此向量组\(A\)线性相关.
\end{proof}
\end{theorem}

由\cref{theorem:线性方程组.部分组线性相关则全组线性相关} 立即得到:
\begin{corollary}\label{theorem:线性方程组.全组线性无关则任一部分组线性无关}
如果向量组\(A\)线性无关,
那么\(A\)的任意一个部分组也线性无关.
\end{corollary}

%@see: 《高等代数(第三版 上册)》(丘维声) P69
给定\(n\)维向量组\(\AutoTuple{\vb\alpha}{s}\),
为其中的每个向量都添上\(m\)个分量,
所添分量的位置对于每个向量都一样,
把得到的\(n+m\)维向量组\(\AutoTuple{\vb\beta}{s}\)称为
“\(\AutoTuple{\vb\alpha}{s}\)的\DefineConcept{延伸组}”;
反过来,把\(\AutoTuple{\vb\alpha}{s}\)称为
“\(\AutoTuple{\vb\beta}{s}\)的\DefineConcept{缩短组}”.

线性无关向量组的延伸组线性无关.
线性相关向量组的缩短组线性相关.

\begin{theorem}\label{theorem:线性方程组.n个n维向量组线性相关的充分必要条件}
%@see: 《线性代数》(张慎语、周厚隆) P71 性质5
\(n\)个\(n\)维列向量\(\AutoTuple{\vb\alpha}{n}\)线性相关的充分必要条件是:\begin{equation*}
	\det(\AutoTuple{\vb\alpha}{n})=0.
\end{equation*}
\begin{proof}
由\hyperref[theorem:线性方程组.克拉默法则]{克拉默法则}可知\begin{align*}
	\text{\(n\)个\(n\)维列向量\(\AutoTuple{\vb\alpha}{n}\)线性相关}
	&\iff \text{方程\(x_1\vb\alpha_1+\dotsb+x_n\vb\alpha_n=\vb0\)有非零解} \\
	&\iff \text{方程\(x_1\vb\alpha_1+\dotsb+x_n\vb\alpha_n=\vb0\)有无穷多解} \\
	&\iff \det(\AutoTuple{\vb\alpha}{n})=0.
	\qedhere
\end{align*}
\end{proof}
\end{theorem}
\begin{corollary}
\(n\)个\(n\)维列向量\(\AutoTuple{\vb\alpha}{n}\)线性无关的充分必要条件是:\begin{equation*}
	\det(\AutoTuple{\vb\alpha}{n})\neq0.
\end{equation*}
\begin{proof}
这是\cref{theorem:线性方程组.n个n维向量组线性相关的充分必要条件} 的逆否命题.
\end{proof}
\end{corollary}
\begin{remark}
考察\(n\)个\(n\)维行向量的线性相关性时,只需将各个向量转置,化为列向量组即可.
\end{remark}
\begin{remark}
需要注意的是,当\(s \neq n\)时,
\(s\)个\(n\)维向量\(\AutoTuple{\vb\alpha}{s}\)不能构成行列式,
只能用其他方法判断其线性相关性.
\end{remark}

\begin{theorem}[替换定理]
设向量组\(\AutoTuple{\vb\alpha}{s}\)线性无关,
\(\vb\beta=b_1\vb\alpha_1+\dotsb+b_s\vb\alpha_s\).
如果\(b_j\neq0\),
那么用\(\vb\beta\)替换\(\vb\alpha_j\)以后得到的向量组
\(\AutoTuple{\vb\alpha}{j-1},\vb\beta,\AutoTuple{\vb\alpha}[j+1]{s}\)
也线性无关.
%TODO proof
\end{theorem}

\begin{example}
%@see: 《线性代数》(张慎语、周厚隆) P71 性质5
设\(\vb{A}\)是3阶矩阵,\(\vb\alpha_1,\vb\alpha_2,\vb\alpha_3\)为3维列向量组,
若\(\vb{A}\vb\alpha_1,\vb{A}\vb\alpha_2,\vb{A}\vb\alpha_3\)线性无关,
证明:\(\vb\alpha_1,\vb\alpha_2,\vb\alpha_3\)线性无关,且\(\vb{A}\)为可逆矩阵.
\begin{proof}
因为\(\vb{A}\vb\alpha_1,\vb{A}\vb\alpha_2,\vb{A}\vb\alpha_3\)线性无关,所以\begin{equation*}
	\abs{\vb{A}} \cdot \det(\vb\alpha_1,\vb\alpha_2,\vb\alpha_3)
	= \det(\vb{A}\vb\alpha_1,\vb{A}\vb\alpha_2,\vb{A}\vb\alpha_3) \neq 0,
\end{equation*}
从而有\(\abs{\vb{A}} \neq 0\),
且\(\det(\vb\alpha_1,\vb\alpha_2,\vb\alpha_3) \neq 0\),
因此\(\vb{A}\)是可逆矩阵,
而齐次线性方程组\(x_1 \vb\alpha_1 + x_2 \vb\alpha_2 + x_3 \vb\alpha_3 = \vb0\)只有零解,
也即向量组\(\vb\alpha_1,\vb\alpha_2,\vb\alpha_3\)线性无关.
\end{proof}
\end{example}
\begin{remark}
从上例可以看出:如果矩阵\(\vb{A},\vb{B}\)的乘积\(\vb{A} \vb{B}\)的列向量组线性无关,
则\(\vb{A}\)可逆且\(\vb{B}\)的列向量组线性无关.
\end{remark}

\begin{example}
设向量组\(\{\AutoTuple{\vb\alpha}{s}\}\)线性相关,
去掉任一向量后线性无关.
证明:方程\begin{equation*}
	x_1 \vb\alpha_1 + \dotsb + x_s \vb\alpha_s = \vb0
\end{equation*}的解\(\AutoTuple{x}{s}\)要么全为零,要么全不为零.
\begin{proof}
显然方程\(x_1 \vb\alpha_1 + \dotsb + x_s \vb\alpha_s = \vb0\)有零解,
并且由于\(\{\AutoTuple{\vb\alpha}{s}\}\)线性相关,
所以这个方程一定有非零解,于是我们只需证明:
这个方程的非零解向量\((\AutoTuple{x}{s})^T\)的各个分量全不为零.

% 首先证明向量组\(\{\AutoTuple{\vb\alpha}{s}\}\)中每一个向量都不是零向量.
% 用反证法.
% 假设向量组\(\{\AutoTuple{\vb\alpha}{s}\}\)中存在一个向量是零向量,
% 那么由题设可知,去掉这个零向量后,剩下\(s-1\)个向量线性无关,
% 但是,如果去掉的向量不是这个零向量,
% 则剩下的\(s-1\)个向量中含有一个零向量,必定线性相关,与题设矛盾!

用反证法.
假设方程\(x_1 \vb\alpha_1 + \dotsb + x_s \vb\alpha_s = \vb0\)的
非零解\((\AutoTuple{x}{s})^T\)的第\(i\)个分量\(x_i\)等于\(0\),
即\begin{equation*}
	x_1 \vb\alpha_1 + \dotsb + x_{i-1} \vb\alpha_{i-1}
	+ 0 \vb\alpha_i + x_{i+1} \vb\alpha_{i+1} + \dotsb
	+ x_s \vb\alpha_s
	= \vb0.
\end{equation*}
这相当于去掉了\(\vb\alpha_i\),由题设可知,剩下的\(s-1\)个向量
\(\{\vb\alpha_1,\dotsc,\vb\alpha_{i-1},\vb\alpha_{i+1},\dotsc,\vb\alpha_s\}\)线性无关,
那么这个方程只有零解,即\(x_1 = \dotsb = x_s = 0\),矛盾!
因此\((\AutoTuple{x}{s})^T\)的各个分量全不为零.
\end{proof}
\end{example}

%\begin{example}
%证明:\(\mathbb{R}^n\)中的任意正交组线性无关.
%\begin{proof}
%设\(A=\{\AutoTuple{\vb\alpha}{m}\}\)是\(\mathbb{R}^n\)的一个正交组,令\begin{equation*}
%	k_1 \vb\alpha_1 + k_2 \vb\alpha_2 + \dotsb + k_m \vb\alpha_m = \vb0,
%\end{equation*}
%两端分别与\(\vb\alpha_1\)作内积,
%即\begin{equation*}
%	\VectorInnerProductDot{(k_1 \vb\alpha_1 + k_2 \vb\alpha_2 + \dotsb + k_m \vb\alpha_m)}{\vb\alpha_1}
%	= \VectorInnerProductDot{\vb0}{\vb\alpha_1};
%\end{equation*}
%由内积性质,\begin{equation*}
%	k_1 (\VectorInnerProductDot{\vb\alpha_1}{\vb\alpha_1})
%	+ k_2 (\VectorInnerProductDot{\vb\alpha_2}{\vb\alpha_1})
%	+ \dotsb
%	+ k_m (\VectorInnerProductDot{\vb\alpha_m}{\vb\alpha_1})
%	= \VectorInnerProductDot{\vb0}{\vb\alpha_1},
%\end{equation*}
%其中\(\VectorInnerProductDot{\vb0}{\vb\alpha_1} = 0\),
%\(\VectorInnerProductDot{\vb\alpha_j}{\vb\alpha_1} = 0\ (j=2,3,\dotsc,m)\),
%故\(k_1 \VectorInnerProductDot{\vb\alpha_1}{\vb\alpha_1} = 0\),
%而\(\VectorInnerProductDot{\vb\alpha_1}{\vb\alpha_1} > 0\),
%所以\(k_1=0\).
%同理可得\(k_2=k_3=\dotsb=k_m=0\),从而\(A\)线性无关.
%\end{proof}
%\end{example}


最后我们对本节内容作一个小结.
\(K^n\)中线性相关的向量组与线性无关的向量组的本质区别可以从以下几个方面刻画.
\begin{enumerate}
	\item 从线性组合的角度看.
	\begin{enumerate}
		\item \(\begin{aligned}[t]
			&\text{向量组\(\AutoTuple{\vb\alpha}{s}\ (s\geq1)\)线性相关} \\
			&\iff
			\text{它们有系数不全为零的线性组合等于零向量}.
		\end{aligned}\)
		\item \(\begin{aligned}[t]
			&\text{向量组\(\AutoTuple{\vb\alpha}{s}\ (s\geq1)\)线性无关} \\
			&\iff
			\text{它们只有系数全为零的线性组合才会等于零向量}.
		\end{aligned}\)
	\end{enumerate}
	\item 从线性表出的角度看.
	\begin{enumerate}
		\item \(\begin{aligned}[t]
			&\text{向量组\(\AutoTuple{\vb\alpha}{s}\ (s\geq2)\)线性相关} \\
			&\iff
			\text{其中至少有一个向量可以由其余向量线性表出}.
		\end{aligned}\)
		\item \(\begin{aligned}[t]
			&\text{向量组\(\AutoTuple{\vb\alpha}{s}\ (s\geq2)\)线性无关} \\
			&\iff
			\text{其中每一个向量都不能由其余向量线性表出}.
		\end{aligned}\)
	\end{enumerate}
	\item 从齐次线性方程的角度看.
	\begin{enumerate}
		\item \(\begin{aligned}[t]
			&\text{列向量组\(\AutoTuple{\vb\alpha}{s}\ (s\geq1)\)线性相关} \\
			&\iff
			\text{有\(K\)中不全为零的数\(\AutoTuple{k}{s}\)使得\(k_1\vb\alpha_1+\dotsb+k_s\vb\alpha_s=\vb0\)} \\
			&\iff
			\text{齐次线性方程组\(x_1\vb\alpha_1+\dotsb+x_s\vb\alpha_s=\vb0\)有非零解}.
		\end{aligned}\)
		\item \(\begin{aligned}[t]
			&\text{列向量组\(\AutoTuple{\vb\alpha}{s}\ (s\geq1)\)线性无关} \\
			&\iff
			\text{齐次线性方程组\(x_1\vb\alpha_1+\dotsb+x_s\vb\alpha_s=\vb0\)只有零解}.
		\end{aligned}\)
	\end{enumerate}
	\item 从行列式的角度看.
	\begin{enumerate}
		\item \(\begin{aligned}[t]
			&\text{\(n\)个\(n\)维列向量\(\AutoTuple{\vb\alpha}{n}\)线性相关} \\
			&\iff
			\text{以\(\AutoTuple{\vb\alpha}{n}\)为列向量组的矩阵的行列式等于零}. \\
			&\text{\(n\)个\(n\)维行向量\(\AutoTuple{\vb\alpha}{n}[,][T]\)线性相关} \\
			&\iff
			\text{以\(\AutoTuple{\vb\alpha}{n}\)为行向量组的矩阵的行列式等于零}.
		\end{aligned}\)
		\item \(\begin{aligned}[t]
			&\text{\(n\)个\(n\)维列向量\(\AutoTuple{\vb\alpha}{n}\)线性无关} \\
			&\iff
			\text{以\(\AutoTuple{\vb\alpha}{n}\)为列向量组的矩阵的行列式不等于零}. \\
			&\text{\(n\)个\(n\)维行向量\(\AutoTuple{\vb\alpha}{n}[,][T]\)线性无关} \\
			&\iff
			\text{以\(\AutoTuple{\vb\alpha}{n}\)为行向量组的矩阵的行列式不等于零}.
		\end{aligned}\)
	\end{enumerate}
\end{enumerate}

\section{向量组的秩}
\subsection{向量组的等价关系}
\begin{definition}\label{definition:向量空间.线性表出2}
%@see: 《线性代数》(张慎语、周厚隆) P72 定义7
%@see: 《高等代数(第三版 上册)》(丘维声) P72 定义2
在\(K^n\)中,如果向量组\(A\)的每个向量都可由向量组\(B\)线性表出,
即\begin{equation*}
	(\forall \vb\alpha \in A)[\vb\alpha \in \Span B],
\end{equation*}或\begin{equation*}
	A\subseteq\Span B,
\end{equation*}
则称“向量组\(A\)可由向量组\(B\)~\DefineConcept{线性表出}”
或“向量组\(B\)可以\DefineConcept{线性表出}向量组\(A\)”;
否则称“向量组\(A\)不可由向量组\(B\)~线性表出”
或“向量组\(B\)不可以线性表出向量组\(A\)”.
\end{definition}

\begin{proposition}\label{theorem:向量空间.线性表出2的等价条件}
在\(K^n\)中,向量组\(A=\{\AutoTuple{\vb\alpha}{s}\}\)
可由向量组\(B=\{\AutoTuple{\vb\beta}{t}\}\)线性表出,
当且仅当存在矩阵\(\vb{Q} \in M_{t \times s}(K)\),
使得\begin{equation*}
	(\AutoTuple{\vb\alpha}{s}) = (\AutoTuple{\vb\beta}{t})~\vb{Q}.
\end{equation*}
\begin{proof}
假设向量组\(\{\AutoTuple{\vb\alpha}{s}\}\)由向量组\(\{\AutoTuple{\vb\beta}{t}\}\)线性表出,
即存在\begin{equation*}
	x_{ij} \in K\ (i=1,2,\dotsc,s;j=1,2,\dotsc,t),
\end{equation*}
使得\begin{equation*}
	\left\{ \begin{array}{l}
		\vb\alpha_1
		= x_{11} \vb\beta_1 + x_{12} \vb\beta_2 + \dotsb + x_{1t} \vb\beta_t
		= (\AutoTuple{\vb\beta}{t})
		(x_{11},\dotsc,x_{1t})^T, \\
		\vb\alpha_2
		= x_{21} \vb\beta_1 + x_{22} \vb\beta_2 + \dotsb + x_{2t} \vb\beta_t
		= (\AutoTuple{\vb\beta}{t})
		(x_{21},\dotsc,x_{2t})^T, \\
		\hdotsfor{1} \\
		\vb\alpha_s
		= x_{s1} \vb\beta_1 + x_{s2} \vb\beta_2 + \dotsb + x_{st} \vb\beta_t
		= (\AutoTuple{\vb\beta}{t})
		(x_{s1},\dotsc,x_{st})^T,
	\end{array} \right.
\end{equation*}
那么\begin{align*}
	(\AutoTuple{\vb\alpha}{s})
	&= (
		(\AutoTuple{\vb\beta}{t})
		(x_{11},\dotsc,x_{1t})^T,
		\dotsc,
		(\AutoTuple{\vb\beta}{t})
		(x_{s1},\dotsc,x_{st})^T
	) \\
	&= (\AutoTuple{\vb\beta}{t})
	((x_{11},\dotsc,x_{1t})^T,\dotsc,(x_{s1},\dotsc,x_{st})^T),
\end{align*}
这里\(\vb{Q} = ((x_{11},\dotsc,x_{1t})^T,\dotsc,(x_{s1},\dotsc,x_{st})^T)\).
\end{proof}
\end{proposition}
\begin{remark}
%@see: 《Linear Algebra Done Right (Fourth Eidition)》(Sheldon Axler) P76 3.51
\cref{theorem:向量空间.线性表出2的等价条件} 说明:
矩阵\(\vb{A}\)与\(\vb{B}\)的乘积\(\vb{A} \vb{B}\)的列向量组可以由\(\vb{A}\)的列向量组线性表出.
同理可证:
\(\vb{A} \vb{B}\)的行向量组可以由\(\vb{B}\)的行向量组线性表出.
\end{remark}

\begin{proposition}\label{theorem:向量空间.线性表出2的自反性}
非空向量组\(A\)总可由它本身线性表出,
即\begin{equation*}
	A \subseteq \Span A.
\end{equation*}
\begin{proof}
设\(A=\{\AutoTuple{\vb\alpha}{s}\}\ (s\geq1)\),
显然\begin{equation*}
	\vb\alpha_i=0\vb\alpha_1+\dotsb+0\vb\alpha_{i-1}+1\vb\alpha_i+0\vb\alpha_{i+1}+\dotsb+0\vb\alpha_s,
	\quad i=1,2,\dotsc,s;
\end{equation*}
这就是说\(\vb\alpha_i\ (i=1,2,\dotsc,s)\)可由\(A\)线性表出,即\begin{equation*}
	\vb\alpha_i\in\opair{\AutoTuple{\vb\alpha}{s}},
	\quad i=1,2,\dotsc,s.
\end{equation*}
于是\(A\)可由\(A\)线性表出.
\end{proof}
\end{proposition}

\begin{proposition}\label{theorem:向量空间.线性表出2的必要条件}
设\(A,B\)都是向量组,
则\begin{equation*}
	\text{\(A\)可由\(B\)线性表出}
	\implies
	\Span A \subseteq \Span B.
\end{equation*}
\begin{proof}
假设\(A = \{\AutoTuple{\vb\alpha}{s}\},
B = \{\AutoTuple{\vb\beta}{t}\}\).

任意取定\(\vb\alpha\in\Span A\),
那么一定存在\(\AutoTuple{k}{s}\in K\)满足\begin{equation*}
	\vb\alpha = \sum_{i=1}^s k_i \vb\alpha_i;
\end{equation*}
又设\(l_{ij}\in K\ (i=1,2,\dotsc,s;j=1,2,\dotsc,t)\)满足\begin{equation*}
	\vb\alpha_i = \sum_{j=1}^t l_{ij} \vb\beta_j,
	\quad i=1,2,\dotsc,s;
\end{equation*}
那么\begin{equation*}
	\vb\alpha = \sum_{i=1}^s k_i \left(
		\sum_{j=1}^t l_{ij} \vb\beta_j
	\right)
	= \sum_{j=1}^t \left(
		\sum_{i=1}^s k_i l_{ij}
	\right) \vb\beta_j,
\end{equation*}
这就是说\(\vb\alpha\in\Span B\),
于是\(\Span A\subseteq\Span B\).
\end{proof}
\end{proposition}

\begin{proposition}\label{theorem:向量空间.线性表出2的充分条件}
如果\(\Span A\subseteq\Span B\),
那么向量组\(A\)可由\(B\)线性表出.
\begin{proof}
因为\(\Span A\subseteq\Span B\),\(A\subseteq\Span A\),
所以\(A\subseteq\Span B\).
\end{proof}
\end{proposition}

于是,依据\cref{theorem:向量空间.线性表出2的必要条件,%
theorem:向量空间.线性表出2的充分条件},
我们可以说:\begin{equation}
	\text{向量组\(A\)可由\(B\)线性表出}
	\iff
	\Span A\subseteq\Span B.
\end{equation}
从而“线性表出”和集合的“包含”关系一样,具有自反性和传递性:
\begin{enumerate}
	\item \(\Span A\subseteq\Span A\).

	这由\cref{theorem:向量空间.线性表出2的自反性} 立即可得.

	\item \(\Span A\subseteq\Span B\land\Span B\subseteq\Span C\implies\Span A\subseteq\Span C\).

	即便不利用集合的包含关系,我们也可以证明线性表出的传递性.
	具体来说,
	假设向量组\(A=\Set{\AutoTuple{\vb\alpha}{s}}\)可以由向量组\(B=\Set{\AutoTuple{\vb\beta}{r}}\)线性表出,
	且\(B\)可以由向量组\(C=\Set{\AutoTuple{\vb\gamma}{m}}\)线性表出.
	在向量组\(A\)中任取一个向量\(\vb\alpha_i\),则\begin{equation*}
		\vb\alpha_i = \sum_{j=1}^r k_j \vb\beta_j.
	\end{equation*}
	又由于\(\vb\beta_j\)可以由向量组\(C\)线性表出,因此\begin{equation*}
		\vb\beta_j = \sum_{t=1}^m l_{jt} \vb\gamma_t,
		\quad j=1,\dotsc,r.
	\end{equation*}
	从而\begin{equation*}
		\vb\alpha_i = \sum_{j=1}^r k_j \vb\beta_j
		= \sum_{j=1}^r k_j \left(
			\sum_{t=1}^m l_{jt} \vb\gamma_t
		\right)
		= \sum_{t=1}^m \left(
			\sum_{j=1}^r k_j l_{jt}
		\right) \vb\gamma_t.
	\end{equation*}
	于是\(\vb\alpha_i\)可以由\(C\)线性表出,
	从而\(A\)可以由\(C\)线性表出.
\end{enumerate}

\begin{definition}\label{definition:向量空间.向量组等价的定义}
%@see: 《线性代数》(张慎语、周厚隆) P72 定义7
%@see: 《高等代数(第三版 上册)》(丘维声) P72 定义2
如果向量组\(A\)与\(B\)可以相互线性表出,
即\begin{equation*}
	\Span A\subseteq\Span B
	\land
	\Span B\subseteq\Span A,
\end{equation*}
或\begin{equation*}
	\Span A = \Span B,
\end{equation*}
则称“\(A\)与\(B\) \DefineConcept{等价}”,
记作\(A \cong B\).
\end{definition}
\begin{proposition}\label{theorem:向量空间.向量组等价的等价条件}
在\(K^n\)中,向量组\(A=\{\AutoTuple{\vb\alpha}{s}\}\)
与向量组\(B=\{\AutoTuple{\vb\beta}{t}\}\)等价,
当且仅当存在
矩阵\(\vb{P} \in M_{s \times t}(K)\)、
矩阵\(\vb{Q} \in M_{t \times s}(K)\),
使得\begin{equation*}
	(\AutoTuple{\vb\beta}{t}) = (\AutoTuple{\vb\alpha}{s})~\vb{P},
	\quad\text{且}\quad
	(\AutoTuple{\vb\alpha}{s}) = (\AutoTuple{\vb\beta}{t})~\vb{Q}.
\end{equation*}
\begin{proof}
由\cref{theorem:向量空间.线性表出2的等价条件} 易得.
\end{proof}
%@credit: {5f4d2f8a-fc8b-4798-85d6-98670f6761e7} 说
% 不能把上述条件浓缩为“存在一个行满秩矩阵或列满秩矩阵\(\vb{Q}\)使得\((\AutoTuple{\vb\alpha}{s}) = (\AutoTuple{\vb\beta}{t})~\vb{Q}\)”.
% 特别是考虑到向量组\(A\)和向量组\(B\)可能是线性相关的,那样就完全不需要一个满秩过渡矩阵.
\end{proposition}

\begin{property}\label{theorem:向量空间.向量组的等价的性质}
对于任意向量组\(A,B,C\subseteq K^n\)来说,
\begin{itemize}
	\item “向量组的等价”具有自反性,即\(A \cong A\).
	\item “向量组的等价”具有对称性,即\(A \cong B \implies B \cong A\).
	\item “向量组的等价”具有传递性,即\(A \cong B \land B \cong C \implies A \cong C\).
\end{itemize}
\end{property}
“向量组的等价”是向量组之间的一种等价关系.

另外,应该注意到,即便有\begin{equation*}
	A \subseteq \Span A = \Span B \supseteq B,
\end{equation*}成立,不见得就有\(A=B\)一定成立.

\begin{theorem}\label{theorem:线性方程组.部分组可由全组线性表出}
%@see: 《线性代数》(张慎语、周厚隆) P72
部分组可由全组线性表出.
\begin{proof}
设数域\(K\)上一个向量组\(A=\{\AutoTuple{\vb\alpha}{s}\}\),
从中任取\(t\ (t \leq s)\)个向量组成向量组\begin{equation*}
	B=\{\AutoTuple{\vb\alpha}{t}\}.
\end{equation*}
欲证部分组可由全组线性表出,
即证\(\forall \vb\alpha_j \in B\),
\(\exists \AutoTuple{k}{j},\dotsc,k_s \in K\),
使得\begin{equation*}
	\vb\alpha_j = k_1 \vb\alpha_1 + k_2 \vb\alpha_2 + \dotsb + k_j \vb\alpha_j + \dotsb + k_s \vb\alpha_s.
\end{equation*}
显然只要取\begin{equation*}
	k_i = \left\{ \begin{array}{cl}
		1, & i=j, \\
		0, & i \neq j,
	\end{array} \right.
\end{equation*}
便可令上式成立.
\end{proof}
\end{theorem}
对于\cref{theorem:线性方程组.部分组可由全组线性表出},
从\(A \subseteq B\)出发,
我们还可以利用\cref{theorem:向量空间.线性表出2的自反性},
结合\(B \subseteq \Span B\),
根据集合包含关系的传递性,
就可以得到\(A \subseteq \Span B\).
因此\begin{equation*}
	A \subseteq B \implies A \subseteq \Span B.
\end{equation*}

\begin{theorem}
%@see: 《线性代数》(张慎语、周厚隆) P72
设\(A\)是向量组,且\(\card A > 1\).
\(A\)线性相关的充分必要条件是:
\(A\)可由某个部分组线性表出.
\begin{proof}
必要性.
因为\(\card A > 1\),
所以由\cref{theorem:线性方程组.向量组线性相关的充分必要条件1} 可知
\begin{equation*}
	\text{\(A\)线性相关}
	\iff
	(\exists \vb\alpha \in A)[\vb\alpha \in \Span(A-\{\vb\alpha\})].
\end{equation*}
又因为\(\card A > 1\),
所以\((\forall\vb\beta\in A)[\card(A-\{\vb\beta\})>0]\),
那么只要令\(B=A-\{\vb\alpha\}\),
就必然有\begin{equation*}
	\emptyset \neq B \subseteq A = \{\vb\alpha\} \cup B, \qquad
	\{\vb\alpha\}\subseteq\Span B, \qquad
	B \subseteq \Span B
\end{equation*}同时成立.
因此\(A\subseteq\Span B\),
这就是说\(A\)可由\(B\)线性表出.

充分性.
因为\(\card A > 1\),
所以\((\exists B)[\emptyset \neq B \subset A]\).
假设\(B\)是\(A\)的一个非空真子集,且\(A\)可由它线性表出,
即\(\emptyset \neq B \subset A\)且\(\Span A \subseteq \Span B\).
显然\(A-B\neq\emptyset\).
由于\hyperref[theorem:线性方程组.部分组可由全组线性表出]{部分组可由全组线性表出},
所以\begin{equation*}
	A-B \subseteq A
	\implies
	A-B \subseteq \Span A
	\implies
	A-B \subseteq \Span B.
\end{equation*}
我们可以笃定:存在向量\(\vb\gamma \in A-B\),使得\(\vb\gamma\)可由向量组\(B\)线性表出,
即\(\vb\gamma \in \Span B\).
因为\(\vb\gamma \in A-B \implies A-\{\vb\gamma\} \supseteq B\),
所以\(\Span B \subseteq \Span(A-\{\vb\gamma\})\),
那么\(\vb\gamma\)也可由向量组\((A-\{g\})\)线性表出,
即\(\vb\gamma \in \Span(A-\{g\})\).
于是由\cref{theorem:线性方程组.向量组线性相关的充分必要条件1} 可知
向量组\(A\)线性相关.
\end{proof}
\end{theorem}

\begin{theorem}\label{theorem:向量空间.可由比自己基数小的向量组线性表出的向量组线性相关}
%@see: 《线性代数》(张慎语、周厚隆) P72 定理2
%@see: 《高等代数(第三版 上册)》(丘维声) P74 引理1
设向量组\(A=\{\AutoTuple{\vb\alpha}{s}\}\)可由\(B=\{\AutoTuple{\vb\beta}{t}\}\)线性表出.
如果\(s>t\),则\(A\)线性相关.
\begin{proof}
欲证\(A\)线性相关,须找到不全为零的\(s\)个数\(\AutoTuple{k}{s}\)使得\begin{equation*}
	k_1 \vb\alpha_1 + k_2 \vb\alpha_2 + \dotsb + k_s \vb\alpha_s = \vb0.
\end{equation*}
因为向量组\(A\)可由\(B\)线性表出,即有\begin{equation*}
	\left\{ \begin{array}{l}
		\vb\alpha_1 = c_{11} \vb\beta_1 + c_{21} \vb\beta_2 + \dotsb + c_{t1} \vb\beta_t, \\
		\vb\alpha_2 = c_{12} \vb\beta_1 + c_{22} \vb\beta_2 + \dotsb + c_{t2} \vb\beta_t, \\
		\hdotsfor{1} \\
		\vb\alpha_s = c_{1s} \vb\beta_1 + c_{2s} \vb\beta_2 + \dotsb + c_{ts} \vb\beta_t.
	\end{array} \right.
\end{equation*}代入可得\begin{equation*}
	\sum_{j=1}^s k_j \vb\alpha_j
	=\sum_{j=1}^s k_j \sum_{i=1}^t c_{ij} \vb\beta_i
	=\sum_{j=1}^s \sum_{i=1}^t k_j c_{ij} \vb\beta_i
	=\sum_{i=1}^t \vb\beta_i \sum_{j=1}^s k_j c_{ij}
	=\vb0.
\end{equation*}
如此只需证存在不全为零的\(s\)个数\(\AutoTuple{k}{s}\)
使得对于任意\(i=1,2,\dotsc,t\)都有\begin{equation*}
	\sum_{j=1}^s k_j c_{ij} = 0.
\end{equation*}
而关于\(k_i\ (i=1,2,\dotsc,s)\)的齐次线性方程组
\begin{equation*}
	\left\{ \begin{array}{l}
		c_{11} k_1 + c_{12} k_2 + \dotsb + c_{1s} k_s = 0, \\
		c_{21} k_1 + c_{22} k_2 + \dotsb + c_{2s} k_s = 0, \\
		\hdotsfor{1} \\
		c_{t1} k_1 + c_{t2} k_2 + \dotsb + c_{ts} k_s = 0.
	\end{array} \right.
\end{equation*}中方程数\(t\)小于未知量个数\(s\),必有非零解.
\end{proof}
\end{theorem}

\begin{corollary}
%@see: 《线性代数》(张慎语、周厚隆) P73 推论1
任意\(n+1\)个\(n\)维向量线性相关.
\begin{proof}
由\cref{theorem:向量空间.任一向量可由基本向量组唯一线性表出},
\(K^n\)中任意\(n+1\)个\(n\)维向量\(A=\{\AutoTuple{\vb\alpha}{n+1}\}\)
可由基本向量组线性表出;
这两个向量组中的向量个数满足\(n+1>n\),
由\cref{theorem:向量空间.可由比自己基数小的向量组线性表出的向量组线性相关},
向量组\(A\)线性相关.
\end{proof}
\end{corollary}

\begin{corollary}\label{theorem:向量空间.线性无关向量组的基数不大于可以线性表出它的任意向量组的基数}
%@see: 《线性代数》(张慎语、周厚隆) P73 推论2
%@see: 《高等代数(第三版 上册)》(丘维声) P74 推论3
若线性无关向量组\(A=\{\AutoTuple{\vb\alpha}{s}\}\)可由\(B=\{\AutoTuple{\vb\beta}{t}\}\)线性表出,
则\(s \leq t\).
\begin{proof}
用反证法.
假设\(s > t\),
由\cref{theorem:向量空间.可由比自己基数小的向量组线性表出的向量组线性相关},
因为向量组\(A\)可由\(B\)线性表出,
所以向量组\(A\)线性相关,矛盾!
故\(s \leq t\).
\end{proof}
\end{corollary}

\begin{corollary}\label{theorem:向量空间.两个等价的线性无关向量组含有相同的向量个数}
%@see: 《线性代数》(张慎语、周厚隆) P73 推论3
%@see: 《高等代数(第三版 上册)》(丘维声) P74 推论4
两个等价的线性无关向量组含有相同的向量个数,即\begin{equation*}
	A \cong B \implies \card A = \card B.
\end{equation*}
\begin{proof}
设\(A=\{\AutoTuple{\vb\alpha}{s}\}\)%
与\(B=\{\AutoTuple{\vb\beta}{t}\}\)%
都线性无关,且\(A \cong B\).
因为\(A\)可由\(B\)线性表出,
由\cref{theorem:向量空间.线性无关向量组的基数不大于可以线性表出它的任意向量组的基数},
\(s \leq t\);
同理可得\(t \leq s\);
于是\(s = t\).
\end{proof}
\end{corollary}
\begin{remark}
含有相同个数向量的两个向量组不一定等价.
例如,取向量组\(A=\{(0,1)\},
B=\{(1,0)\}\).
易知向量组\(B\)不可以线性表出向量组\(A\),
向量组\(A\)也不可以线性表出向量组\(B\).
\end{remark}

%\begin{example}
%在数域\(K\)上,满足\begin{equation*}
%\abs{a_{ii}} > \sum_{\substack{1 \leq j \leq n \\ j \neq i}} \abs{a_{ij}}
%\quad (i=1,2,\dotsc,n)
%\end{equation*}的\(n\)阶矩阵\(\vb{A} = (a_{ij})_n\)称为\DefineConcept{主对角占优矩阵}.
%证明:\(\vb{A}\)的列向量组\(\AutoTuple{\vb\alpha}{n}\)的秩等于\(n\).
%\begin{proof}
%假设\(\AutoTuple{\vb\alpha}{n}\)线性相关,
%则在\(K\)中有一组不全为0的数\(\AutoTuple{k}{n}\),
%使得\begin{equation*}
%	k_1 \vb\alpha_1 + k_2 \vb\alpha_2 + \dotsb + k_n \vb\alpha_n = \vb0.
%\end{equation*}
%不妨设\(\abs{k_l} = \max\{\abs{k_1},\abs{k_2},\dotsc,\abs{k_n}\}\neq0\).
%由\begin{equation*}
%	k_1 a_{l1} + k_2 a_{l2} + \dotsb + k_l a_{ll} + \dotsb + k_n a_{ln} = 0,
%\end{equation*}
%可得\begin{equation*}
%	a_{ll} = -\frac{1}{k_l} (k_1 a_{l1} + \dotsb + k_{l-1} a_{l,l-1} + k_{l+1} a_{l,l+1} + \dotsb + k_n a_{ln})
%	= - \sum_{\substack{1 \leq j \leq n \\ j \neq l}} \frac{k_j}{k_l} a_{lj},
%\end{equation*}\begin{equation*}
%	\abs{a_{ll}} \leq \sum_{\substack{1 \leq j \leq n \\ j \neq l}} \frac{\abs{k_j}}{\abs{k_l}} \abs{a_{lj}}
%	\leq \sum_{\substack{1 \leq j \leq n \\ j \neq l}} \abs{a_{lj}}.
%\end{equation*}
%这与已知条件矛盾!
%因此\(\AutoTuple{\vb\alpha}{n}\)线性无关,
%\(\rank\{\AutoTuple{\vb\alpha}{n}\} = n\).
%\end{proof}
%\end{example}
%TODO 后移到【向量的秩】

\subsection{极大线性无关组的概念}
\begin{definition}\label{definition:线性方程组.极大线性无关组的定义}
%@see: 《线性代数》(张慎语、周厚隆) P73 定义8
%@see: 《高等代数(第三版 上册)》(丘维声) P72 定义1
在\(K^n\)中,设\(B\)是\(A\)的一个部分组.
如果\begin{itemize}
	\item \(B\)线性无关,
	\item \(A\)可由\(B\)线性表出,
\end{itemize}
则称“\(B\)是\(A\)的一个\DefineConcept{极大线性无关组}(maximally linearly independent subset)”.
%@see: https://mathworld.wolfram.com/MaximallyLinearlyIndependent.html
\end{definition}

\begin{example}\label{example:向量空间.单向量组的极大线性无关组}
求向量组\(\{\vb\alpha\}\)的极大线性无关组.
\begin{solution}
显然有\(\Powerset\{\vb\alpha\} = \{ \emptyset, \{\vb\alpha\} \}\),
也就是说\(\{\vb\alpha\}\)的部分组只有\(\emptyset\)和\(\{\vb\alpha\}\),
于是它的极大线性无关组也只能是这两者中的一个.
因为\(\card\{\vb\alpha\} = 1 > 0 = \card\emptyset\),
所以只需要讨论\(\{\vb\alpha\}\)是不是极大线性无关组.

当\(\vb\alpha=\vb0\)时,\(\{\vb\alpha\}\)线性相关,
不能满足\hyperref[definition:线性方程组.极大线性无关组的定义]{极大线性无关组的定义},
故\(\{\vb\alpha\}\)的极大线性无关组是\(\emptyset\).

当\(\vb\alpha\neq\vb0\)时,\(\{\vb\alpha\}\)线性无关,
所以\(\{\vb\alpha\}\)的极大线性无关组是它本身.
\end{solution}
\end{example}

\begin{theorem}\label{theorem:线性方程组.向量组与其极大线性无关组等价}
%@see: 《高等代数(第三版 上册)》(丘维声) P73 命题1
向量组与其极大线性无关组等价.
\begin{proof}
由\cref{theorem:线性方程组.部分组可由全组线性表出} 可知,
作为部分组,极大线性无关组可由全组线性表出;
再根据\hyperref[definition:线性方程组.极大线性无关组的定义]{极大线性无关组的定义},
全组可由极大线性无关组线性表出;
因此,根据\hyperref[definition:向量空间.向量组等价的定义]{向量组等价的定义},
全组与极大线性无关组等价.
\end{proof}
\end{theorem}
\begin{remark}
\cref{theorem:线性方程组.向量组与其极大线性无关组等价} 说明,
向量组\(\AutoTuple{\vb\alpha}{s}\)可以由它的一个极大线性无关组线性表出.
再根据线性表出的传递性得,
\(W=\opair{\AutoTuple{\vb\alpha}{s}}\)中的每个向量%
可以由\(\AutoTuple{\vb\alpha}{s}\)的一个极大线性无关组线性表出,
此时表出方式就唯一了.
\end{remark}

\begin{corollary}\label{theorem:线性空间.向量组的任意两个极大线性无关组等价且等势}
%@see: 《线性代数》(张慎语、周厚隆) P73 定理3
%@see: 《高等代数(第三版 上册)》(丘维声) P73 推论2
%@see: 《高等代数(第三版 上册)》(丘维声) P75 推论5
向量组的任何两个极大线性无关组等价,
且包含相同个数的向量.
\begin{proof}
设\(A\)与\(B\)是向量组\(C\)的两个极大线性无关组.
根据\cref{theorem:线性方程组.向量组与其极大线性无关组等价},
\(A \cong C\),\(B \cong C\).
再由\cref{theorem:向量空间.向量组的等价的性质},向量组等价具有对称性和传递性,
于是\(A \cong B\).
\end{proof}
\end{corollary}

\begin{theorem}
在\(K^n\)中,任意向量组的极大线性无关组的向量个数不大于\(n\)个.
\begin{proof}
根据定义,任意向量组的极大线性无关组是线性无关的,
而向量个数大于维数的向量组总是线性相关,
故任意向量组的极大线性无关组的向量个数总是不大于其维数\(n\)的.
\end{proof}
\end{theorem}

\subsection{向量组的秩}
\begin{definition}
向量组\(A = \{\AutoTuple{\vb\alpha}{s}\}\)的极大线性无关组所含向量的个数,
称为向量组的\DefineConcept{秩}(rank),
记为\(\rank A\)或\(\rank\{\AutoTuple{\vb\alpha}{s}\}\),
即\begin{equation*}
	\text{\(A'\)是\(A\)的极大线性无关组}
	\implies
	[\rank A = \card A'].
\end{equation*}
\end{definition}

\begin{property}
空集\(\emptyset\)的秩为零,即\(\rank\emptyset = 0\).
\begin{proof}
由于\(A\subseteq\emptyset\iff A=\emptyset\),
所以\(\emptyset\)的极大线性无关组就是它本身,
\(\rank\emptyset=\card\emptyset=0\).
\end{proof}
\end{property}

\begin{property}
零向量组的秩为零,即\(\rank\{\vb0\}=0\).
\begin{proof}
由\cref{example:向量空间.单向量组的极大线性无关组},
\(\{\vb0\}\)的极大线性无关组是\(\emptyset\),
故\(\rank\{\vb0\} = \card\emptyset = 0\).
\end{proof}
\end{property}

\begin{proposition}\label{theorem:向量组的秩.并集的秩}
设\(A\)是\(K^n\)中的向量组.
\begin{itemize}
	\item 如果向量\(\vb\alpha\)可由\(A\)线性表出,
	则\begin{equation*}
		\rank(A \cup \{\vb\alpha\}) = \rank A.
	\end{equation*}

	\item 如果向量\(\vb\alpha\)不可由\(A\)线性表出,
	则\begin{equation*}
		\rank(A \cup \{\vb\alpha\}) = \rank A + 1.
	\end{equation*}
\end{itemize}
\begin{proof}
假设\(A'\)是\(A\)的一个极大线性无关组.

如果向量\(\vb\alpha\)可由\(A\)线性表出,
那么\(\vb\alpha\)可由\(A'\)线性表出,
所以\(A \cup \{\vb\alpha\}\)中的每一个向量都可由\(A'\)线性表出,
于是\begin{equation*}
	\rank(A \cup \{\vb\alpha\})
	= \card A'
	= \rank A.
\end{equation*}

如果向量\(\vb\alpha\)不可由\(A\)线性表出,
那么\(A' \cup \{\vb\alpha\}\)线性无关,
且\(A \cup \{\vb\alpha\}\)可由\(A' \cup \{\vb\alpha\}\)线性表出,
这就说明\(A' \cup \{\vb\alpha\}\)一定是\(A \cup \{\vb\alpha\}\)的一个极大线性无关组.
于是\begin{equation*}
	\rank(A \cup \{\vb\alpha\})
	= \card(A' \cup \{\vb\alpha\})
	= \card A' + 1
	= \rank A + 1.
	\qedhere
\end{equation*}
\end{proof}
\end{proposition}

\begin{corollary}\label{theorem:向量空间.秩与线性相关性的关系}
%@see: 《线性代数》(张慎语、周厚隆) P73 推论4
%@see: 《高等代数(第三版 上册)》(丘维声) P75 命题6
设向量组\(A\).
\begin{itemize}
	\item 如果\(\rank A=\card A\),则向量组\(A\)线性无关.
	\item 如果\(\rank A<\card A\),则向量组\(A\)线性相关.
\end{itemize}
\begin{proof}
\(\text{\(A\)线性无关}
	\iff \text{\(A\)的极大线性无关组是它本身}
	\iff \rank A = \card A\).
\end{proof}
\end{corollary}

\begin{theorem}\label{theorem:向量空间.向量组的秩的比较1}
%@see: 《线性代数》(张慎语、周厚隆) P73 推论5
%@see: 《高等代数(第三版 上册)》(丘维声) P75 命题7
设向量组\(A\)可由\(B\)线性表出,
则\(\rank A \leq \rank B\).
\begin{proof}
设\(A=\{\AutoTuple{\vb\alpha}{s}\}\),
\(B=\{\AutoTuple{\vb\beta}{t}\}\),
\(\rank A = r\),\(\rank B = u\).
因为\(A\)可由\(B\)线性表出,即\begin{equation*}
	\vb\alpha_k = \sum_{i=1}^t l_{ki} \vb\beta_i,
	\quad k=1,2,\dotsc,s.
\end{equation*}
设\(A'=\{\AutoTuple{\vb\alpha}{r}\}\)%
和\(B'=\{\AutoTuple{\vb\beta}{u}\}\)%
分别是\(A\)和\(B\)的极大线性无关组,
则\(B\)可由\(B'\)线性表出,即\begin{equation*}
	\vb\beta_i = \sum_{j=1}^u b_{ij} \vb\beta_j,
	\quad i=1,2,\dotsc,t;
\end{equation*}
所以有\begin{equation*}
	\vb\alpha_k = \sum_{i=1}^t l_{ki} \sum_{j=1}^u b_{ij} \vb\beta_j
	= \sum_{i=1}^t \sum_{j=1}^u l_{ki} b_{ij} \vb\beta_j
	= \sum_{j=1}^u \vb\beta_j \sum_{i=1}^t l_{ki} b_{ij},
	\quad k=1,2,\dotsc,s.
\end{equation*}

特别地,\(A'\)可由\(B'\)线性表出,
由\cref{theorem:向量空间.线性无关向量组的基数不大于可以线性表出它的任意向量组的基数},
则有\(r \leq u\),即\(\rank A \leq \rank B\).
\end{proof}
\end{theorem}
\begin{remark}
我们可以把\cref{theorem:向量空间.向量组的秩的比较1} 的证明思路绘制如下:\begin{equation*}
	\color{gray}
	\left. \begin{array}{r}
		\rank A = \rank A' = \card A' \\
		\left. \begin{array}{r}
			A' \subseteq A \\
			{\color{black} A \subseteq \Span B} \\
			\Span B \subseteq \Span B'
		\end{array} \right\}
		\implies
		A' \subseteq \Span B'
		\implies
		\card A' \leq \card B' \\
		\rank B = \rank B' = \card B'
	\end{array} \right\}
	\implies
	{\color{black} \rank A \leq \rank B}.
\end{equation*}
\end{remark}
\begin{example}
举例说明:纵然向量组\(A,B\)满足\(\rank A \leq \rank B\),也不能断定\(A\)可以由\(\vb{B}\)线性表出.
\begin{solution}
取\(A = \{(1,0,0),(0,1,0)\},
B = \{(0,0,1)\}\),
可见\(\rank A = 2,
\rank B = 1\),
但是\(B\)不能由\(A\)线性表出.
\end{solution}
\end{example}

\begin{corollary}\label{theorem:向量空间.向量组的秩的比较2}
部分组的秩总是小于或等于全组的秩.
\begin{proof}
因为\hyperref[theorem:线性方程组.部分组可由全组线性表出]{部分组总可由全组线性表出},
所以\cref{theorem:向量空间.向量组的秩的比较1} 可知,
部分组的秩总是小于或等于全组的秩.
\end{proof}
\end{corollary}

\begin{theorem}\label{theorem:向量组的秩.等价向量组的秩相等}
%@see: 《线性代数》(张慎语、周厚隆) P74 推论6
%@see: 《高等代数(第三版 上册)》(丘维声) P76 推论8
等价向量组的秩相等.
秩相等的向量组却不一定等价.
\begin{proof}
先证“等价向量组的秩相等”.
设向量组\(A\)与\(B\)等价,
则\(A\)可由\(B\)线性表出,
那么由\cref{theorem:向量空间.向量组的秩的比较1} 可得%
\(\rank A \leq \rank B\);
同理可得\(\rank A \geq \rank B\);
因此,\(\rank A = \rank B\).

再证“秩相等的向量组却不一定等价”.
设\(A=\{(0,1)\},
B=\{(1,0)\}\).
虽然\begin{equation*}
	\rank A = \rank B = 1,
\end{equation*}
但\(A\)与\(B\)显然不等价.
\end{proof}
\end{theorem}

\begin{proposition}\label{theorem:向量组的秩.向量组等价的充分必要条件}
设\(A,B\)都是\(K^n\)中的向量组,
则\(A\)与\(B\)等价的充分必要条件是\begin{equation*}
	\rank A = \rank B = \rank(A \cup B).
\end{equation*}
\begin{proof}
必要性.
假设\(A \cong B\)成立.
因为\hyperref[theorem:向量组的秩.等价向量组的秩相等]{等价向量组的秩相等},
所以\(\rank A = \rank B\).
因为\(A\)和\(B\)均可由\(B\)线性表出,
从而\(A \cup B\)也可由\(B\)线性表出;
同时\(B\)作为\(A \cup B\)的部分组自然可由\(A \cup B\)线性表出,
所以\(A \cup B\)与\(B\)等价,
于是\(\rank(A \cup B) = \rank B\).

充分性.
用反证法.
假设\(\rank A = \rank(A \cup B)\)成立,
但是\(B\)不可由\(A\)线性表出.
因为\(B\)不可由\(A\)线性表出,
所以\begin{equation*}
	(\exists\vb\beta \in B)  		% 向量组 B 中存在向量 \vb\beta
	[\vb\beta \notin \Span A],	% 向量 \vb\beta 不在向量组 A 的线性生成空间中
\end{equation*}
于是根据\cref{theorem:向量组的秩.并集的秩}
可知\(\rank(A \cup B) > \rank A\),
与假设矛盾,
因此\begin{equation*}
	\rank A = \rank(A \cup B)
	\implies
	\text{\(B\)可由\(A\)线性表出}.
\end{equation*}
同理\begin{equation*}
	\rank B = \rank(A \cup B)
	\implies
	\text{\(A\)可由\(B\)线性表出}.
\end{equation*}
综上所述\begin{equation*}
	\rank A = \rank(A \cup B) = \rank B
	\implies
	A \cong B.
	\qedhere
\end{equation*}
\end{proof}
\end{proposition}

\begin{example}\label{example:向量空间.若部分组向量个数多于全组的秩则部分组必线性相关}
证明:在秩为\(r\)的向量组中,任意\(r+1\)个向量必线性相关.
\begin{proof}
设向量组\(\AutoTuple{\vb\alpha}{s}\)的秩为\(r\).
假设部分组\(\AutoTuple{\vb\alpha}{r+1}\)线性无关,
那么由\cref{theorem:向量空间.秩与线性相关性的关系} 得\begin{equation*}
	\rank\{\AutoTuple{\vb\alpha}{r+1}\} = r+1.
\end{equation*}
因为\hyperref[theorem:向量空间.向量组的秩的比较2]{部分组的秩总是小于或等于全组的秩},
而这与\begin{equation*}
	r+1 = \rank\{\AutoTuple{\vb\alpha}{r+1}\} \leq \rank\{\AutoTuple{\vb\alpha}{s}\} = r,
\end{equation*}矛盾,
所以部分组\(\AutoTuple{\vb\alpha}{r+1}\)一定线性相关.
\end{proof}
\end{example}

\begin{example}
设向量组\(\AutoTuple{\vb\alpha}{s}\)的秩为\(r\).
如果\(\AutoTuple{\vb\alpha}{r}\)线性无关,证明:
\(\AutoTuple{\vb\alpha}{r}\)
是\(\AutoTuple{\vb\alpha}{s}\)的一个极大线性无关组.
\begin{proof}
设\begin{equation*}
	A=\{\AutoTuple{\vb\alpha}{s}\},
	\qquad
	B=\{\AutoTuple{\vb\alpha}{r}\}.
\end{equation*}
要证\(B\)是\(A\)的一个极大线性无关组,
须证\(A\)的任意向量可由\(B\)线性表出.

\begin{enumerate}
	\item 显然地,\(\vb\alpha_i\ (i=1,2,\dotsc,r)\)可由\(B\)线性表出.

	\item 根据上例,在秩为\(r\)的向量组中,
	任意\(r+1\)个向量必线性相关,
	那么向量组\begin{equation*}
		A_i = \{\AutoTuple{\vb\alpha}{r},\vb\alpha_i\}\quad(i=r+1,\dotsc,s)
	\end{equation*}必线性相关.

	又因为\(B\)线性无关,
	所以\(\vb\alpha_i\ (i=r+1,\dotsc,s)\)可由\(B\)线性表出.
\end{enumerate}

综上所述,\(A\)的任意向量可由\(B\)线性表出,且\(B\)线性无关,
根据极大线性无关组的定义,\(B\)是\(A\)的一个极大线性无关组.
\end{proof}
\end{example}

\begin{example}
向量组\(\AutoTuple{\vb\alpha}{r+1}\)与部分组\(\AutoTuple{\vb\alpha}{r}\)的秩相等.
证明:\(\vb\alpha_{r+1}\)可由\(\AutoTuple{\vb\alpha}{r}\)线性表出.
\begin{proof}
记\(A=\{\AutoTuple{\vb\alpha}{r+1}\}\),
\(B=\{\AutoTuple{\vb\alpha}{r}\}\).
设\(B\)的极大线性无关组为\begin{equation*}
	B'=\{\AutoTuple{\vb\alpha}{t}\},
	\quad 0 \leq t \leq r.
\end{equation*}
由题意有\(\rank A = \rank B = \rank B' = \card B' = t\).

由上例可知,因为\(\rank A = t\),\(B'\)线性无关,
所以\(B'\)是\(A\)的一个极大线性无关组.
那么向量组\(A\)中的向量\(\vb\alpha_{r+1}\)可以由极大线性无关组\(B'\)线性表出.
又由于\(B'\)是\(B\)的部分组,故\(B'\)可由\(B\)线性表出.
总而言之,\(A\)可由\(B\)线性表出.
\end{proof}
\end{example}

\begin{example}
%@see: https://www.bilibili.com/video/BV19N4y1b7aS/
设向量组\(A = \{\AutoTuple{\vb\alpha}{r}\}\)线性无关,
且可以由向量组\(B = \{\AutoTuple{\vb\beta}{s}\}\ (s>r)\)线性表出.
证明:存在\(\vb\beta \in B\),
使得\begin{equation*}
	\rank(\vb\beta,\AutoTuple{\vb\alpha}[2]{r}) = r.
\end{equation*}
%TODO proof
\end{example}

\subsection{极大线性无关组的求解}
\begin{theorem}\label{theorem:向量空间.利用初等行变换求取列极大线性无关组的依据}
%@see: 《线性代数》(张慎语、周厚隆) P75 定理4
设矩阵\begin{equation*}
	\vb{A}=(\AutoTuple{\vb\alpha}{m})
\end{equation*}
经一系列初等行变换化为矩阵\begin{equation*}
	\vb{B}=(\AutoTuple{\vb\beta}{m}),
\end{equation*}
则\(\vb\alpha_{j1},\vb\alpha_{j2},\dotsc,\vb\alpha_{jk}\)为\(\vb{A}\)的列极大线性无关组的充分必要条件是:
\(\vb\beta_{j1},\vb\beta_{j2},\dotsc,\vb\beta_{jk}\)为\(\vb{B}\)的列极大线性无关组.
\begin{proof}
假设矩阵\(\widetilde{\vb{A}}=(\vb\alpha_{j1},\vb\alpha_{j2},\dotsc,\vb\alpha_{jk},\vb\alpha_l)\)经相同的初等行变换化为\begin{equation*}
	\widetilde{\vb{B}}=(\vb\beta_{j1},\vb\beta_{j2},\dotsc,\vb\beta_{jk},\vb\beta_l) \quad(l=1,2,\dotsc,m).
\end{equation*}
考虑以下四个向量形式的线性方程组
\begin{gather}
	x_1 \vb\alpha_{j1} + x_2 \vb\alpha_{j2} + \dotsb + x_k \vb\alpha_{jk} = \vb0, \tag1 \\
	x_1 \vb\beta_{j1} + x_2 \vb\beta_{j2} + \dotsb + x_k \vb\beta_{jk} = \vb0, \tag2 \\
	y_1 \vb\alpha_{j1} + y_2 \vb\alpha_{j2} + \dotsb + y_k \vb\alpha_{jk} = \vb\alpha_l, \tag3 \\
	y_1 \vb\beta_{j1} + y_2 \vb\beta_{j2} + \dotsb + y_k \vb\beta_{jk} = \vb\beta_l, \tag4
\end{gather}
其中(1)与(2)同解,(3)与(4)同解.

必要性.
当\(\vb\alpha_{j1},\vb\alpha_{j2},\dotsc,\vb\alpha_{jk}\)是\(\vb{A}\)的列极大线性无关组时,
(1)仅有零解,(3)有解.于是(2)仅有零解,(4)有解,
从而\(\vb\beta_{j1},\vb\beta_{j2},\dotsc,\vb\beta_{jk}\)线性无关,
\(\vb\beta_l\ (l=1,2,\dotsc,m)\)可由其线性表出;
由极大线性无关组定义,
\(\vb\beta_{j1},\vb\beta_{j2},\dotsc,\vb\beta_{jk}\)是\(\vb{B}\)的列极大线性无关组.

同理可证充分性.
\end{proof}
\end{theorem}

\cref{theorem:向量空间.利用初等行变换求取列极大线性无关组的依据} 告诉我们,
要想求出一组向量\(\AutoTuple{\vb\alpha}{s}\)的极大线性无关组,
可以构造矩阵\(\vb{A}=(\AutoTuple{\vb\alpha}{s})\),
再利用高斯消元法得到阶梯形矩阵\(\vb{B}\),
找出\(\vb{B}\)的非零首元所在的列\(\vb\beta_{j1},\vb\beta_{j2},\dotsc,\vb\beta_{jk}\),
回过头找出\(\vb{A}\)中对应的列\(\vb\alpha_{j1},\vb\alpha_{j2},\dotsc,\vb\alpha_{jk}\),
那么\(\vb\alpha_{j1},\vb\alpha_{j2},\dotsc,\vb\alpha_{jk}\)就是\(\AutoTuple{\vb\alpha}{s}\)的极大线性无关组.
%@see: https://math.stackexchange.com/a/164021/591741

\begin{example}
求列向量组\begin{equation*}
	\vb\alpha_1 = \begin{bmatrix} -1 \\ 1 \\ 0 \\ 0 \end{bmatrix},
	\vb\alpha_2 = \begin{bmatrix} -1 \\ 2 \\ -1 \\ 1 \end{bmatrix},
	\vb\alpha_3 = \begin{bmatrix} 0 \\ -1 \\ 1 \\ -1 \end{bmatrix},
	\vb\alpha_4 = \begin{bmatrix} 1 \\ -1 \\ 2 \\ 3 \end{bmatrix},
	\vb\alpha_5 = \begin{bmatrix} 2 \\ -6 \\ 4 \\ 1 \end{bmatrix}
\end{equation*}的秩与一个极大线性无关组.
\begin{solution}
对矩阵\(\vb{A} = (\AutoTuple{\vb\alpha}{5})\)作初等行变换化为阶梯形矩阵:
\begin{align*}
	\vb{A} &= \begin{bmatrix}
		-1 & -1 & 0 & 1 & 2 \\
		1 & 2 & -1 & -3 & -6 \\
		0 & -1 & 1 & 2 & 4 \\
		0 & 1 & -1 & 3 & 1 \\
	\end{bmatrix}
	\xlongrightarrow{\begin{array}{c}
		(2\text{行}) \addeq 1 \times (1\text{行}) \\
		(4\text{行}) \addeq (3\text{行})
	\end{array}}
	\begin{bmatrix}
		-1 & -1 & 0 & 1 & 2 \\
		0 & 1 & -1 & -2 & -4 \\
		0 & -1 & 1 & 2 & 4 \\
		0 & 0 & 0 & 5 & 5 \\
	\end{bmatrix} \\
	&\xlongrightarrow{\begin{array}{c}
		(3\text{行}) \addeq (2\text{行}) \\
		(4\text{行}) \diveq 5
	\end{array}}
	\begin{bmatrix}
		-1 & -1 & 0 & 1 & 2 \\
		0 & 1 & -1 & -2 & -4 \\
		0 & 0 & 0 & 0 & 0 \\
		0 & 0 & 0 & 1 & 1 \\
	\end{bmatrix} \\
	&\xlongrightarrow{\begin{array}{c} \text{交换(3行)与(4行)} \end{array}}
	\begin{bmatrix}
		-1 & -1 & 0 & 1 & 2 \\
		0 & 1 & -1 & -2 & -4 \\
		0 & 0 & 0 & 1 & 1 \\
		0 & 0 & 0 & 0 & 0 \\
	\end{bmatrix}
	= \vb{B}.
\end{align*}
若按列分块有\(\vb{B} = (\AutoTuple{\vb\beta}{5})\).
阶梯形矩阵\(\vb{B}\)有3行不为零,故\begin{equation*}
	\rank\{\AutoTuple{\vb\alpha}{5}\}=3.
\end{equation*}
又因为\(\vb{B}\)的非零首元分别位于1、2、4列,
则\(\vb\beta_1,\vb\beta_2,\vb\beta_4\)是\(\vb{B}\)的一个列极大线性无关组,
相应地,\(\vb\alpha_1,\vb\alpha_2,\vb\alpha_4\)是\(\vb{A}\)的一个列极大线性无关组,
即\(\{\AutoTuple{\vb\alpha}{5}\}\)的极大线性无关组.
\end{solution}
\end{example}

\section{向量空间及其子空间的基与维数}
\begin{definition}\label{definition:向量空间.子空间的基的定义}
%@see: 《高等代数(第三版 上册)》(丘维声) P77 定义1
设\(U\)是\(K^n\)的一个子空间.
如果\(U\)的有限子集\(A\)
满足\begin{itemize}
	\item \(A\)线性无关,
	\item \(U\)中的每一个向量都可以由\(A\)线性表出,
\end{itemize}
那么称\(A\)是\(U\)的一个\DefineConcept{基}.
%@see: https://mathworld.wolfram.com/VectorBasis.html
\end{definition}

在\(K^n\)中,基本向量组\(\AutoTuple{\vb\epsilon}{n}\)线性无关,
并且根据\cref{theorem:向量空间.任一向量可由基本向量组唯一线性表出},
每一个向量\(\vb\alpha=(\AutoTuple{a}{n})^T\)可由基本向量组线性表出,
于是基本向量组是\(K^n\)的一个基,
称之为\(K^n\)的\DefineConcept{标准基}.

\begin{theorem}\label{theorem:线性方程组.向量空间1}
%@see: 《高等代数(第三版 上册)》(丘维声) P77 定理1
\(K^n\)的任一非零子空间\(U\)都有一个基.
\begin{proof}
因为\(U\neq\{\vb0\}\),
所以\(U\)中至少有一个非零向量\(\vb\alpha_1\).
由\cref{theorem:线性方程组.单向量组线性相关的充分必要条件} 可知,
向量组\(\{\vb\alpha_1\}\)是线性无关的.
若\(\opair{\vb\alpha_1} \neq U\),
则\((\exists \vb\alpha_2 \in U)[\vb\alpha_2 \notin \opair{\vb\alpha_1}]\).
于是\(\vb\alpha_2\)不能由\(\vb\alpha_1\)线性表出,
由\cref{theorem:向量空间.增加一个向量对线性相关性的影响1},
\(\{\vb\alpha_1,\vb\alpha_2\}\)线性无关.
若\(\opair{\vb\alpha_1,\vb\alpha_2} \neq U\),
则\((\exists \vb\alpha_3 \in U)[\vb\alpha_3 \notin \opair{\vb\alpha_1,\vb\alpha_2}]\).
同理\(\{\vb\alpha_1,\vb\alpha_2,\vb\alpha_3\}\)线性无关.
以此类推,
根据\cref{theorem:向量空间.线性无关向量组的基数不大于可以线性表出它的任意向量组的基数},
由于\(K^n\)的任一线性无关向量组所含向量个数不超过\(n\),
因此上述过程不能无限进行下去,到某一步必定终止.
即将我们得到了\(U\)中一个线性无关向量组\(\{\AutoTuple{\vb\alpha}{s}\}\)以后,
有\(\opair{\AutoTuple{\vb\alpha}{s}} = U\),
则\(\{\AutoTuple{\vb\alpha}{s}\}\)就是\(U\)的一个基.
\end{proof}
\end{theorem}
\cref{theorem:线性方程组.向量空间1} 的证明过程也表明,
从子空间\(U\)的一个非零向量出发,可以扩充成\(U\)的一个基.

\begin{theorem}\label{theorem:线性方程组.向量空间2}
\(K^n\)的非零子空间\(U\)的任意两个基所含向量的个数相等.
\begin{proof}
等价的线性无关的向量组含有相同个数的向量.
\end{proof}
\end{theorem}

\begin{definition}\label{theorem:向量空间.子空间的维数}
%@see: 《高等代数(第三版 上册)》(丘维声) P77 定义2
设\(U\)是\(K^n\)的一个非零子空间.
\(U\)的一个基所含向量的个数
称为“子空间\(U\)的\DefineConcept{维数}(the \emph{dimension} of \(U\))”,
记作\(\dim_K U\),
简记为\(\dim U\).
\end{definition}

由于基本向量组\(\AutoTuple{\vb\epsilon}{n}\)是\(K^n\)的一个基,
所以\(\dim K^n = n\).
这就是为什么我们把\(K^n\)称为“\(n\)维向量空间”.

由于零空间的基是空集,
所以零子空间的维数等于零,
即\begin{equation}
	\dim\{\vb0\}
	= \card\emptyset
	= 0.
\end{equation}

在几何空间中,
任意三个不共面的向量是它的一个基,
因此几何空间是三维的空间.
对于过原点的一个平面,它上面不共线的两个向量是它的一个基,
因此这个平面是二维的子空间.
对于过原点的一条直线,
它的一个方向向量是它的一个基,
因此这条直线是一维的子空间.

为了判断线性方程组有没有解,为了研究解集的结构,
我们就必须研究维数更高的向量空间.
我们会发现,对于子空间的结构,基和维数都起到了决定性作用.

\begin{proposition}
%@see: 《高等代数(第三版 上册)》(丘维声) P78 命题3
设\(U\)是\(K^n\)的一个非零子空间,
\(A\)是\(U\)的一个基,
那么\(U\)中每一个向量\(\vb\alpha\)可以由\(A\)线性表出,
并且表出方式是唯一的.
%TODO
\end{proposition}

\begin{definition}
%@see: 《高等代数(第三版 上册)》(丘维声) P78
设\(\AutoTuple{\vb\alpha}{r}\)是\(K^n\)的子空间\(U\)的一个基,
则\(U\)的每一个向量\(\vb\alpha\)都可以由\(\AutoTuple{\vb\alpha}{r}\)唯一地线性表出:\begin{equation*}
	\vb\alpha = x_1 \vb\alpha_1 + x_2 \vb\alpha_2 + \dotsb + x_r \vb\alpha_r.
\end{equation*}
把数域\(K\)上的\(n\)维向量\((x_1,x_2,\dotsc,x_r)^T\)
称为“向量\(\vb\alpha\)在基\(\AutoTuple{\vb\alpha}{r}\)下的\DefineConcept{坐标}”.
\end{definition}

\begin{proposition}\label{theorem:向量空间.子空间维数.命题4}
%@see: 《高等代数(第三版 上册)》(丘维声) P78 命题4
设\(U\)是\(K^n\)的\(r\)维子空间,
那么\(U\)中任意\(r+1\)个向量都线性相关.
\begin{proof}
在\(U\)中任取\(r+1\)个向量\(B=\{\AutoTuple{\vb\beta}{r+1}\}\).
设\(A=\{\AutoTuple{\vb\alpha}{r}\}\)是\(U\)的一个基,
则\(B\)可以由\(A\)线性表出.
由于\(r+1>r\),
因此根据\cref{theorem:向量空间.可由比自己基数小的向量组线性表出的向量组线性相关},
\(B\)线性相关.
\end{proof}
\end{proposition}

\begin{proposition}\label{theorem:向量空间.子空间维数.r维向量空间中任意r个线性无关向量就是该空间的一个基}
%@see: 《高等代数(第三版 上册)》(丘维声) P78 命题5
设\(U\)是\(K^n\)的\(r\)维子空间,
则\(U\)中任意\(r\)个线性无关的向量都是\(U\)的一个基.
\begin{proof}
设\(A=\{\AutoTuple{\vb\alpha}{r}\}\)是\(U\)中线性无关的向量组.
任意取定\(\vb\beta\in U\).
根据\cref{theorem:向量空间.子空间维数.命题4},
向量组\(\{\AutoTuple{\vb\alpha}{r},\vb\beta\}\)必线性相关.
那么由\cref{theorem:向量空间.增加一个向量对线性相关性的影响1},
\(\vb\beta\)可以由\(A\)线性表出.
因此\(A\)是\(U\)的一个基.
\end{proof}
\end{proposition}

\begin{proposition}\label{theorem:向量空间.两个非零子空间的关系1}
%@see: 《高等代数(第三版 上册)》(丘维声) P78 命题6
设\(U\)和\(W\)都是\(K^n\)的非零子空间.
如果\(U \subseteq W\),那么\(\dim U \leq \dim W\).
\begin{proof}
在\(U\)中取一个基\(A=\{\AutoTuple{\vb\alpha}{r}\}\),
在\(W\)中取一个基\(B=\{\AutoTuple{\vb\beta}{t}\}\).
因为\(U \subseteq W\),
所以\(\{\AutoTuple{\vb\alpha}{r}\}\)可由\(\{\AutoTuple{\vb\beta}{t}\}\)线性表出.
那么由\cref{theorem:向量空间.线性无关向量组的基数不大于可以线性表出它的任意向量组的基数}
可知\(r \leq t\),
即\(\dim U \leq \dim W\).
\end{proof}
\end{proposition}

\begin{proposition}\label{theorem:向量空间.两个非零子空间的关系2}
%@see: 《高等代数(第三版 上册)》(丘维声) P78 命题7
设\(U\)和\(W\)都是\(K^n\)的非零子空间,且\(U \subseteq W\).
若\(\dim U = \dim W\),则\(U = W\).
\begin{proof}
\(U\)中取一个基\(A=\{\AutoTuple{\vb\alpha}{r}\}\).
由于\(U \subseteq W\),因此\(A\)是\(W\)中\(r\)个线性无关的向量.
又由于\(\dim W = \dim U = r\),
因此由\cref{theorem:向量空间.子空间维数.r维向量空间中任意r个线性无关向量就是该空间的一个基}
可知\(A\)又是\(W\)的一个基.
从而\(W\)中任一向量\(\vb\beta\)可由\(A\)线性表出.
于是\(\vb\beta\in U\).
因此\(W \subseteq U\),从而\(W = U\).
\end{proof}
\end{proposition}

\begin{theorem}
%@see: 《高等代数(第三版 上册)》(丘维声) P79 定理8
\(K^n\)中,向量组\(A\)的一个极大线性无关组是这个向量组生成的子空间\(\Span A\)的一个基,
从而
\begin{equation}\label{equation:线性方程组.子空间的维数与向量组的秩的联系}
	\dim(\Span A) = \rank A.
\end{equation}
\begin{proof}
设\(U=\opair{\AutoTuple{\vb\alpha}{s}}\),
\(B=\{\vb\alpha_{i_1},\vb\alpha_{i_2},\dotsc,\vb\alpha_{i_r}\}\)是\(A=\{\AutoTuple{\vb\alpha}{s}\}\)的一个极大线性无关组.
由线性表出的传递性得,\(U\)中任一向量\(\vb\beta\)可由\(B\)线性表出.
因此\(B\)是\(U\)的一个基.
\end{proof}
\end{theorem}
这里要注意区分“子空间的维数\(\dim(\Span A)\)”
和“向量组的秩\(\rank A\)”这两个概念:
维数是对子空间而言,秩是对向量组而言;
在子空间\(\Span A\)这个集合中有无穷多个向量,
而向量组\(A\)这个集合中只有有限的\(s\)个向量.

\begin{example}
%@see: 《高等代数(第三版 上册)》(丘维声) P79 例1
设\(r < n\).
证明:\begin{equation*}
	W = \Set{ (\AutoTuple{a}{r},0,\dotsc,0) \given a_i \in K, i=1,2,\dotsc,r }
\end{equation*}是\(K^n\)的子空间.
\begin{proof}
显然\(\vb0\in W\).
任取\(k\in K\).
再任取两个向量\begin{equation*}
	\vb\alpha = (\AutoTuple{a}{r},0,\dotsc,0), \qquad
	\vb\beta = (\AutoTuple{b}{r},0,\dotsc,0).
\end{equation*}
显然\(\vb\alpha,\vb\beta\in W\),且有\begin{equation*}
	\vb\alpha+\vb\beta = (a_1+b_1,a_2+b_2,\dotsc,a_r+b_r,0,\dotsc,0) \in W,
\end{equation*}\begin{equation*}
	k \vb\alpha = (k a_1,k a_2,\dotsc,k a_r,0,\dotsc,0) \in W;
\end{equation*}
这就是说\(W\)对于加法、数量乘法都封闭.
因此\(W\)是\(K^n\)的一个子空间.
\end{proof}
\end{example}

\begin{example}
%@see: 《高等代数(第三版 上册)》(丘维声) P80 习题3.4 4.
设\(\dim U = r\),且\(\AutoTuple{\vb\alpha}{r} \in U\).
证明:如果\(U\)中每一个向量都可以由\(\AutoTuple{\vb\alpha}{r}\)线性表出,
那么\(\AutoTuple{\vb\alpha}{r}\)是\(U\)的一个基.
\begin{proof}
由\cref{definition:向量空间.子空间的基的定义},
倘若\(U\)中每一个向量都可以由\(\AutoTuple{\vb\alpha}{r}\)线性表出,
那么要证\(\AutoTuple{\vb\alpha}{r}\)是\(U\)的一个基,
就只需证\(\AutoTuple{\vb\alpha}{r}\)线性无关.

用反证法.
假设\(\AutoTuple{\vb\alpha}{r}\)线性相关,
那么由\cref{equation:线性方程组.子空间的维数与向量组的秩的联系,theorem:向量空间.秩与线性相关性的关系},
有\begin{equation*}
	\dim\opair{\AutoTuple{\vb\alpha}{r}}=\rank\{\AutoTuple{\vb\alpha}{r}\}<r=\dim U,
\end{equation*}
于是\(\opair{\AutoTuple{\vb\alpha}{r}} \neq U\),
即存在\(\vb\alpha \in U\),\(\vb\alpha\)无法由\(\AutoTuple{\vb\alpha}{r}\)线性表出,矛盾!
因此\(\AutoTuple{\vb\alpha}{r}\)线性无关,也是\(U\)的一个基.
\end{proof}
\end{example}

\section{矩阵的秩}
\subsection{矩阵的行秩与列秩}
为了求解向量组的秩,我们可以把向量组看成矩阵的行向量组或列向量组,
利用矩阵的性质,得出这个矩阵的行向量组、列向量组的秩,
最后得到所求向量组的秩.

\begin{definition}\label{definition:线性方程组.行秩与列秩的定义}
%@see: 《线性代数》(张慎语、周厚隆) P76 定义11(1)
设\(\vb{A}\)是数域\(K\)上的\(s \times n\)矩阵.
\begin{itemize}
	\item \(\vb{A}\)的列向量组生成的子空间\begin{equation*}
		\Set{
			\vb{A}\vb{x} \given \vb{x}=(x_1,\dotsc,x_n)^T \in K^n
		}
	\end{equation*}
	称为“\(\vb{A}\)的\DefineConcept{列空间}(column space)”,
	记为\(\SpanC\vb{A}\).

	有时候我们把矩阵的列空间称为它的\DefineConcept{像空间},
	记作\(\Img\vb{A}\).

	\item \(\vb{A}\)的行向量组生成的子空间\begin{equation*}
		\Set{
			\vb{x}^T\vb{A} \given \vb{x}=(x_1,\dotsc,x_s)^T \in K^s
		}
	\end{equation*}
	称为“\(\vb{A}\)的\DefineConcept{行空间}(row space)”,
	记为\(\SpanR\vb{A}\).

	\item \(\vb{A}\)的行向量组的秩
	称为“\(\vb{A}\)的\DefineConcept{行秩}(row rank)”,
	记为\(\RankR\vb{A}\).

	\item \(\vb{A}\)的列向量组的秩
	称为“\(\vb{A}\)的\DefineConcept{列秩}(column rank)”,
	记为\(\RankC\vb{A}\).
\end{itemize}
%@see: https://mathworld.wolfram.com/RowSpace.html
%@see: https://mathworld.wolfram.com/ColumnSpace.html
\end{definition}

矩阵\(\vb{A}\)的列空间就是它的转置的行空间,
它的行空间就是它的转置的列空间,
即\begin{equation*}
	\SpanC\vb{A} = \SpanR\vb{A}^T,
	\qquad
	\SpanR\vb{A} = \SpanC\vb{A}^T.
\end{equation*}

矩阵\(\vb{A}\)的列空间不一定与它的行空间相等,
例如,取\(
	\vb{A}
	\defeq
	\begin{bmatrix}
		0 & 1 \\
		0 & 0
	\end{bmatrix}
\),
那么\(
	\SpanC\vb{A}
	= \Span\{(1,0)^T\},
	\SpanR\vb{A}
	= \Span\{(0,1)^T\}
\),
显然\(\SpanC\vb{A} \neq \SpanR\vb{A}\).

矩阵的列秩等于它的列空间的维数,它的行秩等于它的行空间的维数,即\begin{equation*}
	\RankC\vb{A} = \dim(\SpanC\vb{A}),
	\qquad
	\RankR\vb{A} = \dim(\SpanR\vb{A}).
\end{equation*}

\begin{proposition}\label{theorem:向量空间.矩阵的行秩与列秩分别等于它的转置矩阵的列秩与行秩}
设\(\vb{A}\)是矩阵,那么\begin{gather}
	\RankR\vb{A} = \RankC\vb{A}^T, \\
	\RankC\vb{A} = \RankR\vb{A}^T.
\end{gather}
\begin{proof}
显然\begin{gather*}
	\RankR\vb{A}
	= \dim(\SpanR\vb{A})
	= \dim(\SpanC\vb{A}^T)
	= \RankC\vb{A}^T, \\
	\RankC\vb{A}
	= \dim(\SpanC\vb{A})
	= \dim(\SpanR\vb{A}^T)
	= \RankR\vb{A}^T.
	\qedhere
\end{gather*}
\end{proof}
\end{proposition}

当\(K = \mathbb{R}\)时,
于\(\vb{A} \in M_{s \times n}(K)\)而言,
最重要的四个子空间就是:
它的列空间\(\SpanC\vb{A}\)(即\(\Im\vb{A}\)),
它的行空间\(\SpanR\vb{A}\)(即\(\Im\vb{A}^T\)),
它的核空间\(\Ker\vb{A}\),
和它的左核空间\(\Ker\vb{A}^T\).
显然\(
	\Im\vb{A}^T \leq K^n,
	\Ker\vb{A} \leq K^n,
	\Im\vb{A} \leq K^s,
	\Ker\vb{A}^T \leq K^s
\).
相对地,当\(K = \mathbb{C}\)时,
于\(\vb{A} \in M_{s \times n}(K)\)而言,
最重要的四个子空间分别是:
\(\Im\vb{A}\)、
\(\Im\vb{A}^H\)、
\(\Ker\vb{A}\)、
\(\Ker\vb{A}^H\).
显然\(
	\Im\vb{A}^H \leq K^n,
	\Ker\vb{A} \leq K^n,
	\Im\vb{A} \leq K^s,
	\Ker\vb{A}^H \leq K^s
\).

\begin{figure}[hbt]
%@see: https://www.cs.utexas.edu/~flame/laff/alaff-beta/chapter04-four-fundamental-spaces.html
	\centering
	\begin{tikzpicture}
		\begin{scope}[rotate=40]
			\fill[fill=blue!30] (0,0)rectangle(2,3);
			\fill[fill=yellow!30] (0,0)rectangle(-3,-2);
			\coordinate (rsnw) at (0,3);
			\coordinate (rsne) at (2,3);
			\coordinate (nsnw) at (-3,0);
			\coordinate (nssw) at (-3,-2);
			\fill (1,1.5)coordinate(xr)circle(2pt)node[above]{$\vb{x}_1$};
			\fill (-1.8,-.8)coordinate(xn)circle(2pt)node[below]{$\vb{x}_0$};
			\fill (xr|-xn)coordinate(x)circle(2pt)node[below right]{$\vb{x} = \vb{x}_1 + \vb{x}_0$};
			\draw (0,0)node[left]{$\vb0$};
		\end{scope}
		\begin{scope}[rotate=-30,xshift=5cm,yshift=2.5cm]
			\fill[fill=red!30] (0,0)rectangle(-2,3);
			\fill[fill=green!30] (0,0)rectangle(3,-2);
			\coordinate (csnw) at (-2,3);
			\coordinate (csne) at (0,3);
			\coordinate (lsne) at (3,0);
			\coordinate (lsse) at (3,-2);
			\fill (-.6,1.4)coordinate(b)circle(2pt)node[right]{$\vb{b}$};
			\draw (0,0)coordinate(o)node[right]{$\vb0$};
		\end{scope}
		\begin{scope}[-{Stealth[length=4.5mm,width=0pt 8]}]
			\draw (xr)--(b)node[sloped,midway,above]{$\vb{A} \vb{x}_1 = \vb{b}$};
			\draw (x)--(b)node[sloped,pos=.35,above]{$\vb{A} \vb{x} = \vb{b}$};
			\draw (xn)--(o)node[sloped,pos=.55,below]{$\vb{A} \vb{x}_0 = \vb0$};
		\end{scope}
		\draw[white] (rsnw)--(rsne)node[sloped,midway,below,black]{$\Im\vb{A}^T$};
		\draw[white] (nsnw)--(nssw)node[sloped,midway,above,black]{$\Ker\vb{A}$};
		\draw[white] (csnw)--(csne)node[sloped,midway,below,black]{$\Im\vb{A}$};
		\draw[white] (lsne)--(lsse)node[sloped,midway,above,black]{$\Ker\vb{A}^T$};
		\draw[dashed] (xr)--(x)--(xn);
		\draw (-2,0)node{\Huge$\mathbb{R}^n$};
		\draw (7.5,0)node{\Huge$\mathbb{R}^s$};
	\end{tikzpicture}
	% 在图中,我们把\(\Ker\vb{A}\)和\(\Im\vb{A}^T\)都画在左侧,靠近符号\(K^n\),就是表示它们都是向量空间\(K^n\)的子空间;
	% 我们把\(\Ker\vb{A}^T\)和\(\Im\vb{A}\)都画在右侧,靠近符号\(K^s\),就是表示它们都是向量空间\(K^s\)的子空间.
	% 根据\cref{example:齐次线性方程组的解集的结构.矩阵的像空间与它的左核空间互为正交补},
	% 我们把代表\(\Ker\vb{A}\)和\(\Im\vb{A}^T\)的两个矩形的长边画成互相垂直,就是表示这两个子空间中的向量互相正交;
	% 同样地,我们把代表\(\Ker\vb{A}^T\)和\(\Im\vb{A}\)的两个矩形的长边画成互相垂直,就是表示这两个子空间中的向量互相正交.
	\caption{}
\end{figure}

\begin{definition}
设矩阵\(\vb{A} \in M_{s \times n}(K)\).
\begin{itemize}
	\item 若\(\RankR\vb{A} = s\),
	则称“\(\vb{A}\)为\DefineConcept{行满秩矩阵}(row full rank matrix)”
	或“\(\vb{A}\)具有满行秩(\(\vb{A}\) has full row rank)”.
	\item 若\(\RankC\vb{A} = n\),
	则称“\(\vb{A}\)为\DefineConcept{列满秩矩阵}(column full rank matrix)”
	或“\(\vb{A}\)具有满列秩(\(\vb{A}\) has full column rank)”.
\end{itemize}
\end{definition}

现在我们来研究一个问题:矩阵的列秩与行秩之间有什么联系?

\subsection{矩阵的秩}
\begin{definition}\label{definition:线性方程组.矩阵的秩的定义}
%@see: 《线性代数》(张慎语、周厚隆) P76 定义11(2)
数域\(K\)上的\(s \times n\)矩阵\(\vb{A}\)的最高阶非零子式的阶数,
% the highest order of nonzero minor of \(\vb{A}\)
称为“矩阵\(\vb{A}\)的\DefineConcept{秩}(rank)”,
记为\(\rank\vb{A}\).
%@see: https://mathworld.wolfram.com/MatrixRank.html
\end{definition}

\begin{proposition}\label{theorem:向量空间.零矩阵的秩为零}
零矩阵的秩为零,即\(\rank\vb0 = 0\).
\end{proposition}
\begin{proposition}\label{theorem:向量空间.秩为零的矩阵必为零矩阵}
秩为零的矩阵必为零矩阵.
\end{proposition}

\begin{property}\label{theorem:线性方程组.矩阵的秩的性质2}
设\(\vb{A}\)是数域\(K\)上的\(s \times n\)矩阵,则\(0 \leq \rank\vb{A} \leq \min\{s,n\}\).
\end{property}

\begin{property}\label{theorem:线性方程组.矩阵的秩的性质3}
设\(\vb{A}\)是数域\(K\)上的矩阵,\(k \in K-\{0\}\),
则\(\rank(k\vb{A}) = \rank\vb{A}\).
\end{property}

\begin{property}
设矩阵\(\vb{A} \in M_{s \times n}(K)\),
则\begin{equation*}
	\rank(\vb{A},\vb0) = \rank\vb{A}.
\end{equation*}
\end{property}

\begin{theorem}
设\(\vb{A}\in M_{s \times n}(K)\).
如果\(\vb{A}\)有一个\(k\)阶非零子式,那么\(\rank\vb{A}\geq k\).
\begin{proof}
用反证法.
假设\(\rank\vb{A}<k\),
也就是说\(\vb{A}\)的非零子式的最高阶数\(r\)小于\(k\),
但是根据前提条件,\(\vb{A}\)有一个\(k\)阶非零子式,\(k>r\),
这就和\hyperref[definition:线性方程组.矩阵的秩的定义]{矩阵的秩的定义}矛盾!
因此\(\rank\vb{A}\geq k\).
\end{proof}
\end{theorem}
\begin{corollary}
设矩阵\(\vb{A} \in M_{s \times n}(K)\),
\(\vb{E}\)是数域\(K\)上的\(s\)阶单位矩阵,
则\begin{equation*}
	\rank(\vb{A},\vb{E}) = s.
\end{equation*}
\end{corollary}
\begin{corollary}
设矩阵\(\vb{A} \in M_{s \times n}(K)\),
\(\vb{E}\)是数域\(K\)上的\(n\)阶单位矩阵,
则\begin{equation*}
	\rank\begin{bmatrix}
		\vb{A} \\ \vb{E}
	\end{bmatrix}
	= n.
\end{equation*}
\end{corollary}

\begin{example}
求矩阵\(\vb{A} = \begin{bmatrix} 1 & 7 \\ 2 & 6 \\ -3 & 1 \end{bmatrix}\)的秩、行秩与列秩.
\begin{solution}
因为至少存在一个\(\vb{A}\)的二阶子式不为零,即\begin{equation*}
	\begin{vmatrix} 1 & 7 \\ 2 & 6 \end{vmatrix} = -8 \neq 0,
\end{equation*}
且\(\vb{A}\)没有三阶子式,所以\(\rank\vb{A} = 2\).

写出\(\vb{A}\)的行向量组\begin{equation*}
	(1,7), \qquad
	(2,6), \qquad
	(-3,1),
\end{equation*}
分别转置后,列成一个矩阵(显然就是\(\vb{A}^T\)),
作初等行变换化成阶梯形矩阵\begin{equation*}
	\vb{A}^T = \begin{bmatrix}
		1 & 2 & -3 \\
		7 & 6 & 1
	\end{bmatrix} \to \begin{bmatrix}
		1 & 0 & 17/2 \\
		0 & 4 & -11
	\end{bmatrix} = \vb{B}_1.
\end{equation*}阶梯形矩阵\(\vb{B}_1\)有2行不为零,故矩阵\(\vb{A}\)的行秩为2.

写出\(\vb{A}\)的列向量组\begin{equation*}
	(1,2,-3)^T,
	(7,6,1)^T,
\end{equation*}列成一个矩阵(显然就是\(\vb{A}\)本身),
作初等行变换化成阶梯形矩阵\begin{equation*}
	\vb{A} = \begin{bmatrix} 1 & 7 \\ 2 & 6 \\ -3 & 1 \end{bmatrix}
	\to \begin{bmatrix} 1 & 0 \\ 0 & 1 \\ 0 & 0 \end{bmatrix} = \vb{B}_2.
\end{equation*}
阶梯形矩阵\(\vb{B}_2\)有2行不为零,故矩阵\(\vb{A}\)的列秩为2.
\end{solution}
\end{example}
\begin{remark}
从这个例子可以看出,矩阵\(\vb{A} = \begin{bmatrix} 1 & 7 \\ 2 & 6 \\ -3 & 1 \end{bmatrix}\)的秩、行秩与列秩全都相等.
这让我们不禁好奇:是否任意矩阵的秩、行秩与列秩也全都相等?
\end{remark}

\subsection{矩阵的秩、行秩、列秩的关系}
\begin{lemma}\label{theorem:向量空间.矩阵的秩与行秩和列秩的关系.引理}
%@see: 《线性代数》(张慎语、周厚隆) P77 引理
设矩阵\(\vb{A} = (a_{ij})_{s \times n}\)的列秩等于\(\vb{A}\)的列数\(n\),
则\(\vb{A}\)行秩、秩都等于\(n\).
\begin{proof}
对\(\vb{A}\)分别进行列分块与行分块:\begin{equation*}
	\vb{A} = (\AutoTuple{\vb\alpha}{n})
	= (\AutoTuple{\vb{A}}{s}[,][T])^T.
\end{equation*}
根据\cref{theorem:向量空间.秩与线性相关性的关系},
由于\(\RankC\vb{A}=n\),
所以\(\vb{A}\)的列向量组线性无关,
也就是说齐次线性方程组\begin{equation*}
	x_1 \vb\alpha_1 + x_2 \vb\alpha_2 + \dotsb + x_n \vb\alpha_n = \vb0
\end{equation*}只有零解,或者说\(\vb{A}\vb{x}=\vb0\)只有零解.
进而根据\cref{theorem:线性方程组.方程个数少于未知量个数的齐次线性方程组必有非零解},
在\(\vb{A}\vb{x}=\vb0\)中,方程个数\(s\)不少于未知量个数\(n\),即\(s \geq n\).

设\(\RankR\vb{A}=t\).
根据\cref{theorem:向量空间.子空间维数.命题4},
因为\(\vb{A}\)的行向量的维数是\(n\),
所以\(t \leq n\).

设\(\vb{A}\)的行极大线性无关组是
\(\vb{A}_{i_1},\vb{A}_{i_2},\dotsc,\vb{A}_{i_t}\),
其中\(1 \leq i_1 < i_2 < \dotsb < i_t \leq s\).

假设\(t < n\),
则\(\vb{A}\)可通过初等行变换化为\begin{equation*}
	\begin{bmatrix}
		\vb{A}_{i_1} \\ \vb{A}_{i_2} \\ \vdots \\ \vb{A}_{i_t} \\ \vb0_{(s-t) \times n}
	\end{bmatrix},
\end{equation*}
于是,\(\vb{A}\vb{x}=\vb0\)表示的齐次线性方程组中,
非零方程的个数\(t\)小于未知量个数\(n\),
那么根据\cref{theorem:线性方程组.方程个数少于未知量个数的齐次线性方程组必有非零解},
\(\vb{A}\vb{x}=\vb0\)有非零解,
这与我们前面得到的结论“\(\vb{A}\vb{x}=\vb0\)只有零解”矛盾,
所以假设\(t < n\)不成立,
因此\(t = n\).
这就是说\begin{equation*}
	\RankR\vb{A}=\RankC\vb{A}=n.
\end{equation*}

既然\(t=n\),
那么\(\vb{A}\)的行极大线性无关组恰好可以构成一个方阵,
而这个方阵的行列式为\begin{equation*}
	d=\begin{vmatrix} \vb{A}_{i_1} \\ \vb{A}_{i_2} \\ \vdots \\ \vb{A}_{i_n} \end{vmatrix}.
\end{equation*}
根据\cref{theorem:线性方程组.克拉默法则},
因为\(\vb{A}\vb{x}=\vb0\)只有零解作为它的唯一解,
所以\(d\neq0\).
于是\(d\)是\(\vb{A}\)的一个\(n\)阶非零子式.
\(\vb{A}\)是一个\(s \times n\)矩阵,
不可能有阶数大于\(n\)的子式,
因此\(\rank\vb{A} = n\).
\end{proof}
\end{lemma}

\begin{theorem}\label{theorem:向量空间.矩阵的秩与行秩和列秩的关系.定理}
%@see: 《线性代数》(张慎语、周厚隆) P77 定理5
矩阵的行秩、列秩、秩都相等.
\begin{proof}
如果矩阵\(\vb{A} = \vb0\),
则\(\rank\vb{A}=\RankR\vb{A}=\RankC\vb{A}=0\),
结论成立.

当\(\vb{A}\neq\vb0\)时,
设\(\rank\vb{A}=r\),
则根据\hyperref[definition:线性方程组.矩阵的秩的定义]{矩阵的秩的定义},
\(\vb{A}\)的所有\(t\ (t > r)\)阶子式全为零;
且\(\vb{A}\)有一个\(r\)阶子式不为零,
从而该\(r\)阶子式的列向量组线性无关,
它们的延伸组也线性无关,
\(\vb{A}\)有\(r\)列线性无关,
于是\(\RankC\vb{A}=p \geq r\);
由\cref{theorem:向量空间.矩阵的秩与行秩和列秩的关系.引理},
\(\vb{A}\)的列极大线性无关组构成的矩阵有一非零\(p\)阶子式,
也是\(\vb{A}\)的子式,
故\(p \leq r\);
综上得\(p = r\).
因此\(\rank\vb{A}=\RankC\vb{A}\).
同样地,将这一结论用于\(\vb{A}^T\),
得\(\rank\vb{A}^T=\RankC\vb{A}^T=\RankR\vb{A}\).
\end{proof}
\end{theorem}

\begin{theorem}\label{theorem:向量空间.转置不变秩}
%@see: 《高等代数(第三版 上册)》(丘维声) P83 推论6
设\(\vb{A}\)是矩阵,那么\(\rank\vb{A} = \rank\vb{A}^T\).
\begin{proof}
根据\cref{theorem:向量空间.矩阵的秩与行秩和列秩的关系.定理,theorem:向量空间.矩阵的行秩与列秩分别等于它的转置矩阵的列秩与行秩},
有\(\rank\vb{A}=\RankC\vb{A}=\RankR\vb{A}^T=\rank\vb{A}^T\).
\end{proof}
\end{theorem}

\begin{example}
设\(\vb{A},\vb{B} \in M_n(K)\),
则\begin{equation*}
	\rank\begin{bmatrix}
		\vb{A} \\ \vb{B}
	\end{bmatrix}
	= \rank\begin{bmatrix}
		\vb{A} \\ \vb{B}
	\end{bmatrix}^T
	= \rank(\vb{A}^T,\vb{B}^T).
\end{equation*}
\end{example}
\begin{remark}
%@credit: {ba6d5dc2-c9a6-4590-bfd1-1cd33fae3bd3}
需要注意的是,矩阵\((\vb{A},\vb{B})\)与\((\vb{A}^T,\vb{B}^T)\)的秩不一定相等.
例如,取\begin{equation*}
	\vb{A} = \begin{bmatrix}
		1 & 0 \\
		0 & 0
	\end{bmatrix},
	\qquad
	\vb{B} = \begin{bmatrix}
		0 & 1 \\
		0 & 0
	\end{bmatrix},
\end{equation*}
然而\begin{equation*}
	\rank(\vb{A},\vb{B})
	= \rank\begin{bmatrix}
		1 & 0 & 0 & 1 \\
		0 & 0 & 0 & 0
	\end{bmatrix}
	= 1
	\neq
	\rank(\vb{A}^T,\vb{B}^T)
	= \rank\begin{bmatrix}
		1 & 0 & 0 & 0 \\
		0 & 0 & 1 & 0
	\end{bmatrix}
	= 2.
\end{equation*}
\end{remark}

\subsection{初等变换对矩阵的秩的影响}
\begin{theorem}\label{theorem:线性方程组.初等变换不变秩}
%@see: 《高等代数(第三版 上册)》(丘维声) P81 定理2
%@see: 《高等代数(第三版 上册)》(丘维声) P82 定理3
%@see: 《高等代数(第三版 上册)》(丘维声) P83 推论7
初等变换不改变矩阵的秩.
\begin{proof}
由\cref{theorem:向量空间.利用初等行变换求取列极大线性无关组的依据},
初等行变换不改变矩阵的列秩;
同理,初等列变换不改变矩阵的行秩;
再由\cref{theorem:向量空间.矩阵的秩与行秩和列秩的关系.定理},
初等变换不改变矩阵的秩.
\end{proof}
\end{theorem}

\begin{example}
%@see: 《高等代数(第三版 上册)》(丘维声) P86 习题3.5 12.
%@see: 《高等代数学习指导书(第二版 上册)》(丘维声) P115 例8
设\(\vb{A} \in M_{s \times n}(K),
\vb{B} \in M_{l \times m}(K)\).
试证:\begin{equation}
	\rank\begin{bmatrix}
		\vb{A} & \vb0 \\
		\vb0 & \vb{B}
	\end{bmatrix}
	= \rank\vb{A} + \rank\vb{B}.
	\label{equation:矩阵的秩.分块矩阵的秩的等式1}
\end{equation}
\begin{proof}
若\(\vb{A}\)或\(\vb{B}\)是零矩阵,
则必有\(\rank\vb{A}=0\)或\(\rank\vb{B}=0\),
这时候\cref{equation:矩阵的秩.分块矩阵的秩的等式1} 显然成立.
下面我们假设\(\vb{A}\)和\(\vb{B}\)的秩都大于零,
对矩阵\(\begin{bmatrix}
	\vb{A} & \vb0 \\
	\vb0 & \vb{B}
\end{bmatrix}\)的前\(s\)行、后\(l\)行分别作初等行变换,化成\begin{equation*}
	\begin{bmatrix}
		\vb{J}_r & \vb0 \\
		\vb0 & \vb0 \\
		\vb0 & \vb{J}_t \\
		\vb0 & \vb0
	\end{bmatrix},
	\eqno(1)
\end{equation*}
其中\(\vb{J}_r\)是\(r \times n\)阶的阶梯形矩阵,且\(r\)行都是非零行,\(\rank\vb{A}=r\);
\(\vb{J}_t\)是\(t \times m\)阶的阶梯形矩阵,且\(t\)行都是非零行,\(\rank\vb{B}=t\).
对矩阵(1)作一系列的两行互换,化成\begin{equation*}
	\begin{bmatrix}
		\vb{J}_r & \vb0 \\
		\vb0 & \vb{J}_t \\
		\vb0 & \vb0
	\end{bmatrix},
	\eqno(2)
\end{equation*}
矩阵(2)是阶梯形矩阵,它有\(r+t\)个非零行.
因此\begin{equation*}
	\rank\begin{bmatrix}
		\vb{A} & \vb0 \\
		\vb0 & \vb{B}
	\end{bmatrix}
	= r+t
	= \rank\vb{A}+\rank\vb{B}.
	\qedhere
\end{equation*}
\end{proof}
\end{example}

现在我们来对\cref{equation:矩阵的秩.分块矩阵的秩的等式1} 做一番推广,
我们用一个非零矩阵\(\vb{C}\)取代分块对角矩阵\(\begin{bmatrix}
	\vb{A} & \vb0 \\
	\vb0 & \vb{B}
\end{bmatrix}\)的一个副对角,
得到分块上三角矩阵\(\begin{bmatrix}
	\vb{A} & \vb{C} \\
	\vb0 & \vb{B}
\end{bmatrix}\)或分块下三角矩阵\(\begin{bmatrix}
	\vb{A} & \vb0 \\
	\vb{C} & \vb{B}
\end{bmatrix}\),
接下来我们研究这类矩阵的秩的取值范围.
\begin{example}
%@see: 《高等代数(第三版 上册)》(丘维声) P87 习题3.5 13.
%@see: 《高等代数学习指导书(第二版 上册)》(丘维声) P116 例9
%@see: 《高等代数(第四版)》(谢启鸿 姚慕生) P157 例3.62
%@see: 《高等代数(第四版)》(谢启鸿 姚慕生) P170 例3.90
设\(\vb{A} \in M_{s \times n}(K),
\vb{B} \in M_{l \times m}(K),
\vb{C} \in M_{s \times m}(K)\).
试证:\begin{equation}
	\rank\begin{bmatrix}
		\vb{A} & \vb{C} \\
		\vb0 & \vb{B}
	\end{bmatrix} \geq \rank\vb{A} + \rank\vb{B}.
	\label{equation:矩阵的秩.分块矩阵的秩的不等式}
\end{equation}
当且仅当,
关于\(\vb{X} \in M_{n \times m}(K)\)和\(\vb{Y} \in M_{s \times l}(K)\)的矩阵方程
\(\vb{C}=\vb{A}\vb{X}+\vb{Y}\vb{B}\)有解时,
上式取“\(=\)”号.
\begin{proof}
首先证明不等式成立.
设\(\rank\vb{A}=r_1,
\rank\vb{B}=r_2\),
则存在可逆矩阵\(\vb{P}_1,\vb{P}_2,\vb{Q}_1,\vb{Q}_2\),
使得\begin{equation*}
	\vb{P}_1\vb{A}\vb{Q}_1 = \begin{bmatrix}
		\vb{E}_{r_1} & \vb0 \\
		\vb0 & \vb0
	\end{bmatrix},
	\qquad
	\vb{P}_2\vb{B}\vb{Q}_2 = \begin{bmatrix}
		\vb{E}_{r_2} & \vb0 \\
		\vb0 & \vb0
	\end{bmatrix}.
\end{equation*}
于是\begin{equation*}
	\begin{bmatrix}
		\vb{P}_1 & \vb0 \\
		\vb0 & \vb{P}_2
	\end{bmatrix}
	\begin{bmatrix}
		\vb{A} & \vb{C} \\
		\vb0 & \vb{B}
	\end{bmatrix}
	\begin{bmatrix}
		\vb{Q}_1 & \vb0 \\
		\vb0 & \vb{Q}_2
	\end{bmatrix}
	= \begin{bmatrix}
		\vb{P}_1\vb{A}\vb{Q}_1 & \vb{P}_1\vb{C}\vb{Q}_2 \\
		\vb0 & \vb{P}_2\vb{B}\vb{Q}_2
	\end{bmatrix}
	= \begin{bmatrix}
		\vb{E}_{r_1} & \vb0 & \vb{C}_{11} & \vb{C}_{12} \\
		\vb0 & \vb0 & \vb{C}_{21} & \vb{C}_{22} \\
		\vb0 & \vb0 & \vb{E}_{r_2} & \vb0 \\
		\vb0 & \vb0 & \vb0 & \vb0
	\end{bmatrix}.
\end{equation*}
做初等变换,
用\(\vb{E}_{r_1}\)消去同行的矩阵,
用\(\vb{E}_{r_2}\)消去同列的矩阵,
再将\(\vb{C}_{22}\)对换到第二行第二列,
得到:\begin{equation*}
	\begin{bmatrix}
		\vb{E}_{r_1} & \vb0 & \vb{C}_{11} & \vb{C}_{12} \\
		\vb0 & \vb0 & \vb{C}_{21} & \vb{C}_{22} \\
		\vb0 & \vb0 & \vb{E}_{r_2} & \vb0 \\
		\vb0 & \vb0 & \vb0 & \vb0
	\end{bmatrix}
	\to \begin{bmatrix}
		\vb{E}_{r_1} & \vb0 & \vb0 & \vb0 \\
		\vb0 & \vb{C}_{22} & \vb0 & \vb0 \\
		\vb0 & \vb0 & \vb{E}_{r_2} & \vb0 \\
		\vb0 & \vb0 & \vb0 & \vb0
	\end{bmatrix}.
\end{equation*}
于是,由\cref{equation:矩阵的秩.分块矩阵的秩的等式1} 有\begin{equation*}
	\rank\begin{bmatrix}
		\vb{A} & \vb{C} \\
		\vb0 & \vb{B}
	\end{bmatrix}
	= \rank\vb{E}_{r_1} + \rank\vb{C}_{22} + \rank\vb{E}_{r2}
	\geq r_1 + r_2
	= \rank\vb{A} + \rank\vb{B}.
\end{equation*}

然后证明取等条件.
\begin{itemize}
	\item 充分性.
	假设\(\vb{X} = \vb{X}_0, \vb{Y} = \vb{Y}_0\)是矩阵方程\(\vb{A}\vb{X}+\vb{Y}\vb{B}=\vb{C}\)的解,
	则\begin{align*}
		&\begin{bmatrix}
			\vb{E} & -\vb{Y}_0 \\
			\vb0 & \vb{E}
		\end{bmatrix}
		\begin{bmatrix}
			\vb{A} & \vb{C} \\
			\vb0 & \vb{B}
		\end{bmatrix}
		\begin{bmatrix}
			\vb{E} & -\vb{X}_0 \\
			\vb0 & \vb{E}
		\end{bmatrix}
		= \begin{bmatrix}
			\vb{A} & \vb{C}-\vb{Y}_0\vb{B} \\
			\vb0 & \vb{B}
		\end{bmatrix}
		\begin{bmatrix}
			\vb{E} & -\vb{X}_0 \\
			\vb0 & \vb{E}
		\end{bmatrix} \\
		&\hspace{20pt}= \begin{bmatrix}
			\vb{A} & \vb{C}-\vb{Y}_0\vb{B}-\vb{A}\vb{X}_0 \\
			\vb0 & \vb{B}
		\end{bmatrix}
		= \begin{bmatrix}
			\vb{A} & \vb0 \\
			\vb0 & \vb{B}
		\end{bmatrix},
	\end{align*}
	于是\begin{equation*}
		\rank\begin{bmatrix}
			\vb{A} & \vb{C} \\
			\vb0 & \vb{B}
		\end{bmatrix}
		= \rank\begin{bmatrix}
			\vb{A} & \vb0 \\
			\vb0 & \vb{B}
		\end{bmatrix}
		= \rank \vb{A} + \rank \vb{B}.
	\end{equation*}

	\item 必要性.
	假设\(\rank \vb{A}=r_1,
	\rank \vb{B}=r_2\),
	且\begin{equation*}
		\vb{P}_1 \vb{A} \vb{Q}_1
		= \begin{bmatrix}
			\vb{E}_{r_1} & \vb0 \\
			\vb0 & \vb0
		\end{bmatrix},
		\qquad
		\vb{P}_2 \vb{B} \vb{Q}_2
		= \begin{bmatrix}
			\vb{E}_{r_2} & \vb0 \\
			\vb0 & \vb0
		\end{bmatrix},
	\end{equation*}
	其中\(\vb{P}_1,\vb{Q}_1,\vb{P}_2,\vb{Q}_2\)是可逆矩阵.
	注意到问题的条件和结论在等价变换\begin{gather*}
		\vb{A} \mapsto \vb{P}_1 \vb{A} \vb{Q}_1, \qquad
		\vb{B} \mapsto \vb{P}_2 \vb{B} \vb{Q}_2, \qquad
		\vb{C} \mapsto \vb{P}_1 \vb{C} \vb{Q}_2, \\
		\vb{X} \mapsto \vb{Q}_1^{-1} \vb{X} \vb{Q}_2, \qquad
		\vb{Y} \mapsto \vb{P}_1 \vb{Y} \vb{P}_2^{-1},
	\end{gather*}
	下保持不变,
	故不妨从一开始就假设\(\vb{A},\vb{B}\)是等价标准型,
	即\begin{equation*}
		\vb{A} = \begin{bmatrix}
			\vb{E}_{r_1} & \vb0 \\
			\vb0 & \vb0
		\end{bmatrix}, \qquad
		\vb{B} = \begin{bmatrix}
			\vb{E}_{r_2} & \vb0 \\
			\vb0 & \vb0
		\end{bmatrix}.
	\end{equation*}
	再对\(\vb{C},\vb{X},\vb{Y}\)相应地分块,
	得到\begin{equation*}
		\vb{C} = \begin{bmatrix}
			\vb{C}_{11} & \vb{C}_{12} \\
			\vb{C}_{21} & \vb{C}_{22}
		\end{bmatrix},
		\qquad
		\vb{X} = \begin{bmatrix}
			\vb{X}_{11} & \vb{X}_{12} \\
			\vb{X}_{21} & \vb{X}_{22}
		\end{bmatrix},
		\qquad
		\vb{Y} = \begin{bmatrix}
			\vb{Y}_{11} & \vb{Y}_{12} \\
			\vb{Y}_{21} & \vb{Y}_{22}
		\end{bmatrix}.
	\end{equation*}
	作初等变换:\begin{equation*}
		\begin{bmatrix}
			\vb{A} & \vb{C} \\
			\vb0 & \vb{B}
		\end{bmatrix}
		= \begin{bmatrix}
			\vb{E}_{r_1} & \vb0 & \vb{C}_{11} & \vb{C}_{12} \\
			\vb0 & \vb0 & \vb{C}_{21} & \vb{C}_{22} \\
			\vb0 & \vb0 & \vb{E}_{r_2} & \vb0 \\
			\vb0 & \vb0 & \vb0 & \vb0
		\end{bmatrix}
		\to \begin{bmatrix}
			\vb{E}_{r_1} & \vb0 & \vb0 & \vb0 \\
			\vb0 & \vb0 & \vb0 & \vb{C}_{22} \\
			\vb0 & \vb0 & \vb{E}_{r_2} & \vb0 \\
			\vb0 & \vb0 & \vb0 & \vb0
		\end{bmatrix}.
	\end{equation*}
	因为\(\rank\begin{bmatrix}
		\vb{A} & \vb{C} \\
		\vb0 & \vb{B}
	\end{bmatrix}
	= \rank \vb{A} + \rank \vb{B}
	= r_1 + r_2\),
	所以\(\vb{C}_{22}\)必为零矩阵.
	于是矩阵方程\(\vb{A}\vb{X}+\vb{Y}\vb{B}=\vb{C}\)或者说\begin{equation*}
		\vb{C} = \begin{bmatrix}
			\vb{C}_{11} & \vb{C}_{12} \\
			\vb{C}_{21} & \vb0
		\end{bmatrix}
		= \begin{bmatrix}
			\vb{X}_{11} + \vb{Y}_{11} & \vb{X}_{12} \\
			\vb{Y}_{21} & \vb0
		\end{bmatrix}
		= \begin{bmatrix}
			\vb{X}_{11} & \vb{X}_{12} \\
			\vb0 & \vb0
		\end{bmatrix}
		+ \begin{bmatrix}
			\vb{Y}_{11} & \vb0 \\
			\vb{Y}_{21} & \vb0
		\end{bmatrix}
	\end{equation*}有解.
	例如\(\vb{X}_{11} = \vb{C}_{11}, \vb{X}_{12} = \vb{C}_{12}, \vb{Y}_{11} = \vb0, \vb{Y}_{21} = \vb{C}_{21}\).
	\qedhere
\end{itemize}
\end{proof}
%@see: https://math.stackexchange.com/questions/4941561/whats-the-equality-condition-of-inequality-def-rank-operatornamerank-rank
\end{example}

\subsection{满秩矩阵}
\begin{definition}
%@see: 《线性代数》(张慎语、周厚隆) P76
设矩阵\(\vb{A} \in M_n(K)\).
若\(\rank\vb{A} = n\),
则称“\(\vb{A}\)是\DefineConcept{满秩矩阵}(full rank matrix)”,
或“\(\vb{A}\)是\DefineConcept{非退化矩阵}(non-degenerate matrix)”.
\end{definition}

%应该注意到,“可逆矩阵”“非奇异矩阵”“满秩矩阵”“非退化矩阵”是从不同侧重点对同一类矩阵的四种称谓.

\begin{theorem}\label{theorem:向量空间.满秩方阵的行列式非零}
%@see: 《高等代数(第三版 上册)》(丘维声) P84 推论9
设\(\vb{A}\in M_n(K)\).
\(\vb{A}\)是满秩矩阵的充分必要条件是\(\abs{\vb{A}}\neq0\).
\begin{proof}
\(\rank\vb{A}=n
\iff \text{\(A\)的非零子式的最高阶数为\(n\)}
\iff \abs{\vb{A}}\neq0\).
\end{proof}
\end{theorem}

\begin{corollary}
%@see: 《高等代数(第三版 上册)》(丘维声) P84 推论10
设\(\vb{A}\in M_{s \times n}(K)\)且\(\rank\vb{A}=r\).
那么,\(\vb{A}\)的\(r\)阶非零子式
所在的列构成\(\vb{A}\)的列向量组的一个极大线性无关组,
所在的行构成\(\vb{A}\)的行向量组的一个极大线性无关组.
\begin{proof}
将\(\vb{A}\)按列分块为\((\vb\alpha_1,\vb\alpha_2,\dotsc,\vb\alpha_n)\),
并假设\(\rank\vb{A} = r\),
且\begin{equation*}
	\MatrixMinor{\vb{A}}{
		i_1,i_2,\dotsc,i_r \\
		j_1,j_2,\dotsc,j_r
	} \neq 0.
\end{equation*}
于是这个\(r\)阶子式的列向量组线性无关.
那么它的延伸组\(\vb\alpha_{j_1},\vb\alpha_{j_2},\dotsc,\vb\alpha_{j_r}\)线性无关.
由于\(\vb{A}\)的列秩为\(r\),
因此\(\vb\alpha_{j_1},\vb\alpha_{j_2},\dotsc,\vb\alpha_{j_r}\)构成\(\vb{A}\)的列向量组的一个极大线性无关组.
\end{proof}
\end{corollary}

\section{矩阵乘积的秩}
令\[
	\vb{A}=\begin{bmatrix}
		1 & 0 \\
		0 & 0
	\end{bmatrix}, \qquad
	\vb{B}=\begin{bmatrix}
		0 & 0 \\
		1 & 0
	\end{bmatrix}, \qquad
	\vb{C}=\begin{bmatrix}
		1 & 1 \\
		0 & 1
	\end{bmatrix},
\]
则\[
	\vb{A} \vb{B}=\begin{bmatrix}
		1 & 0 \\
		0 & 0
	\end{bmatrix}
	\begin{bmatrix}
		0 & 0 \\
		1 & 0
	\end{bmatrix}
	= \begin{bmatrix}
		0 & 0 \\
		0 & 0
	\end{bmatrix};
\]
于是\(\rank(\vb{A} \vb{B})=0\),而\(\rank\vb{A}=1\),\(\rank\vb{B}=1\).

又\[
	\vb{A}\vb{C}=\begin{bmatrix}
		1 & 0 \\
		0 & 0
	\end{bmatrix}
	\begin{bmatrix}
		1 & 1 \\
		0 & 1
	\end{bmatrix}
	= \begin{bmatrix}
		1 & 1 \\
		0 & 0
	\end{bmatrix};
\]
于是\(\rank(\vb{A}\vb{C})=1\),而\(\rank\vb{A}=1\),\(\rank\vb{C}=2\).

从上述例子,我们猜测:
对于任意矩阵\(\vb{A},\vb{B}\),总有\[
	\rank(\vb{A} \vb{B}) \leq \rank\vb{A}, \qquad
	\rank(\vb{A} \vb{B}) \leq \rank\vb{B}.
\]

\subsection{矩阵乘积的秩}
\begin{theorem}\label{theorem:线性方程组.矩阵乘积的秩}
%@see: 《线性代数》(张慎语、周厚隆) P78 定理7
%@see: 《高等代数(第三版 上册)》(丘维声) P121 定理1
设\(\vb{A} \in M_{s \times t}(K),
\vb{B} \in M_{t \times n}(K)\),
那么\[
	\rank(\vb{A}\vb{B}) \leq \min\{\rank\vb{A},\rank\vb{B}\}.
\]
\begin{proof}
记\(\vb{C} \defeq \vb{A}\vb{B}\).
显然\(\vb{C} \in M_{s \times n}(K)\).
将\(\vb{C}\)、\(\vb{B}\)分别进行列分块得\[
	\vb{C} = (\AutoTuple{\vb\gamma}{n}),
	\qquad
	\vb{B} = (\AutoTuple{\vb\beta}{n}),
\]
其中\(\vb\beta_i \in K^t\ (i=1,2,\dotsc,n)\),
\(\vb\gamma_i \in K^s\ (i=1,2,\dotsc,n)\),
则\[
	(\AutoTuple{\vb\gamma}{n})
	= \vb{A} (\AutoTuple{\vb\beta}{n})
	= (\AutoTuple{\vb{A}\vb\beta}{n}),
\]
于是\(\vb\gamma_i = \vb{A} \vb\beta_i\ (i=1,2,\dotsc,n)\).

假设\(\rank\vb{B} = r\).
由\cref{example:向量空间.若部分组向量个数多于全组的秩则部分组必线性相关},
\(\vb{B}\)的任意\(r+1\)个列向量
\(\vb\beta_{k_1},\vb\beta_{k_2},\dotsc,\vb\beta_{k_{r+1}}\)线性相关,
也就是说,存在不全为零的数\(l_1,l_2,\dotsc,l_{r+1}\in K\),使得\[
	l_1 \vb\beta_{k_1} + l_2 \vb\beta_{k_2} + \dotsb + l_{r+1} \vb\beta_{k_{r+1}} = \vb0,
\]
从而有\[
	l_1 \vb\gamma_{k_1} + l_2 \vb\gamma_{k_2} + \dotsb + l_{r+1} \vb\gamma_{k_{r+1}}
	= \vb{A}(l_1 \vb\beta_{k_1} + l_2 \vb\beta_{k_2} + \dotsb + l_{r+1} \vb\beta_{k_{r+1}})
	= \vb0,
\]
这就是说\(\vb{C}\)的任意\(r+1\)个列向量也线性相关,
那么\(\RankC\vb{C} \ngeq r+1\),
因此\[
	\rank(\vb{A}\vb{B})
	\leq r = \rank\vb{B}.
\]
利用这个结论,我们还可以得到\[
	\rank(\vb{A}\vb{B})
	= \rank(\vb{A}\vb{B})^T
	= \rank(\vb{B}^T \vb{A}^T)
	\leq \rank \vb{A}^T
	= \rank \vb{A}.
\]
综上所述\(\rank(\vb{A}\vb{B}) \leq \min\{\rank\vb{A},\rank\vb{B}\}\).
\end{proof}
\end{theorem}
\begin{remark}
我们可以看出\cref{theorem:向量空间.向量组的秩的比较1}
与\cref{theorem:线性方程组.矩阵乘积的秩} 之间存在内在联系:
由\cref{theorem:向量空间.线性表出2的等价条件} 可知\[
	\rank(\AutoTuple{\vb\alpha}{s})
	= \rank((\AutoTuple{\vb\beta}{t})\vb{Q}),
\]
而由可知\[
	\rank((\AutoTuple{\vb\beta}{t})\vb{Q})
	\leq \rank(\AutoTuple{\vb\beta}{t}),
\]
于是\[
	\rank(\AutoTuple{\vb\alpha}{s})
	\leq \rank(\AutoTuple{\vb\beta}{t}).
\]
\end{remark}

\begin{corollary}\label{theorem:矩阵乘积的秩.与可逆矩阵相乘不变秩}
%@see: 《线性代数》(张慎语、周厚隆) P78 推论1
设矩阵\(\vb{A} \in M_{s \times n}(K)\).
\begin{itemize}
	\item 如果\(\vb{P}\)是数域\(K\)上的\(s\)阶可逆矩阵,则\(\rank\vb{A} = \rank(\vb{P}\vb{A})\).
	\item 如果\(\vb{Q}\)是数域\(K\)上的\(n\)阶可逆矩阵,则\(\rank\vb{A} = \rank(\vb{A}\vb{Q})\).
\end{itemize}
\begin{proof}
因为\(\vb{A} = (\vb{P}^{-1} \vb{P}) \vb{A} = \vb{P}^{-1} (\vb{P} \vb{A})\),
由\cref{theorem:线性方程组.矩阵乘积的秩},
有\[
	\rank\vb{A} = \rank(\vb{P}^{-1}(\vb{P}\vb{A})) \leq \rank(\vb{P}\vb{A}) \leq \rank\vb{A},
\]
所以\(\rank\vb{A} = \rank(\vb{P}\vb{A})\).
同理可得\(\rank\vb{A} = \rank(\vb{A}\vb{Q})\).
\end{proof}
\end{corollary}

\begin{theorem}\label{theorem:矩阵乘积的秩.多行少列矩阵与少行多列矩阵的乘积的行列式}
设矩阵\(\vb{A} \in M_{m \times n}(K),
\vb{B} \in M_{n \times m}(K)\),
\(m > n\),
那么\[
	\abs{\vb{A} \vb{B}} = 0.
\]
\begin{proof}
当\(m>n\)时,
根据\cref{theorem:线性方程组.矩阵的秩的性质2},有\[
	\rank\vb{A},\rank\vb{B} \leq \min\{m,n\} = n;
\]
再根据\cref{theorem:线性方程组.矩阵乘积的秩},有\[
	\rank(\vb{A} \vb{B}) \leq \min\{\rank\vb{A},\rank\vb{B}\} = n < m,
\]
也就是说,矩阵\(\vb{A} \vb{B}\)不满秩;
那么根据\cref{theorem:向量空间.满秩方阵的行列式非零} 可知\(\abs{\vb{A} \vb{B}} = 0\).
\end{proof}
%\cref{equation:线性方程组.柯西比内公式}
\end{theorem}
\begin{remark}
在\cref{example:行列式.两个向量的乘积矩阵的行列式} 我们看到
维数相同的一个列向量与一个行向量的乘积的行列式等于零.
\end{remark}

\subsection{等价标准型}
\begin{definition}
设矩阵\(\vb{A} \in M_{s \times n}(K)\),
\(\rank\vb{A} = r\),
\(\vb{E}_r\)是\(r\)阶单位矩阵.
我们把分块矩阵\[
	\begin{bmatrix}
		\vb{E}_r & \vb0 \\
		\vb0 & \vb0
	\end{bmatrix}
\]
称为“\(\vb{A}\)的\DefineConcept{等价标准型}”.
\end{definition}

下面我们证明任意一个矩阵的等价标准型总是存在.
\begin{theorem}\label{theorem:矩阵乘积的秩.等价标准型的存在性}
矩阵\(\vb{A}\)满足\(\rank\vb{A}=r\)的充分必要条件是:
存在可逆矩阵\(\vb{P},\vb{Q}\),使得\[
	\vb{P} \vb{A} \vb{Q}
	= \begin{bmatrix}
		\vb{E}_r & \vb0 \\
		\vb0 & \vb0
	\end{bmatrix} = \vb{B}.
\]
\begin{proof}
充分性.
如果可逆矩阵\(\vb{P},\vb{Q}\)使得\[
	\vb{P}\vb{A}\vb{Q}
	= \begin{bmatrix}
		\vb{E}_r & \vb0 \\
		\vb0 & \vb0
	\end{bmatrix},
\]
把上式等号左边的可逆矩阵\(\vb{P}\)、\(\vb{Q}\)分别视作对矩阵\(\vb{A}\)的初等行变换和初等列变换,
那么,根据\cref{theorem:线性方程组.初等变换不变秩},
所得矩阵\(\vb{B}\)的秩与原矩阵\(\vb{A}\)相同,
即\[
	\rank\vb{A} = \rank\vb{B} = r.
	\qedhere
\]
%\cref{theorem:线性方程组.非零矩阵可经初等行变换化为若尔当阶梯形矩阵}
%TODO proof 未证明必要性
\end{proof}
\end{theorem}

\begin{theorem}\label{theorem:矩阵乘积的秩.矩阵等价的充分必要条件}
设\(\vb{A}\)与\(\vb{B}\)都是\(s \times n\)矩阵,
则\(\vb{A} \cong \vb{B}\)的充分必要条件是:
\(\rank\vb{A} = \rank\vb{B}\).
\begin{proof}
必要性.
因为\(\vb{A}\)可经一系列初等变换化为\(\vb{B}\),
根据\cref{theorem:线性方程组.初等变换不变秩},初等变换不改变矩阵的秩,
所以\(\rank\vb{A} = \rank\vb{B}\).

充分性.
已知\(\rank\vb{A} = \rank\vb{B} = r\).
对\(\vb{A}\)作初等行变换可将其化简为仅前\(r\)行不为零的阶梯形矩阵\(\vb{C}\),
同样对\(\vb{C}\)作初等列变换可化简为\(\vb{A}\)的等价标准型.
对\(\vb{B}\)也可作初等变换化为等价标准型.
那么存在\(s\)阶可逆矩阵\(\vb{P}_1\)和\(\vb{P}_2\),
存在\(n\)阶可逆矩阵\(\vb{Q}_1\)和\(\vb{Q}_2\),
使得\[
	\vb{P}_1 \vb{A} \vb{Q}_1 = \vb{P}_2 \vb{A} \vb{Q}_2
	= \begin{bmatrix} \vb{E}_r & \vb0 \\ \vb0 & \vb0 \end{bmatrix},
\]
令\(\vb{P} = \vb{P}_2^{-1} \vb{P}_1\),\(\vb{Q} = \vb{Q}_1 \vb{Q}_2^{-1}\),
则\(\vb{P}\)和\(\vb{Q}\)可逆,
\(\vb{P} \vb{A} \vb{Q} = \vb{B}\),
从而\(\vb{A} \cong \vb{B}\).
\end{proof}
\end{theorem}

\begin{example}\label{example:矩阵乘积的秩.可交换矩阵之和的秩}
设\(\vb{A},\vb{B}\in M_n(K)\),
且\(\vb{A} \vb{B}=\vb{B}\vb{A}\).
证明:\[
	\rank(\vb{A}+\vb{B})\leq\rank\vb{A}+\rank\vb{B}-\rank(\vb{A} \vb{B}).
\]
\begin{proof}
考虑\[
	\begin{bmatrix}
		\vb{E} & \vb{E} \\
		\vb0 & \vb{E}
	\end{bmatrix}
	\begin{bmatrix}
		\vb{A} & \vb0 \\
		\vb0 & \vb{B}
	\end{bmatrix}
	\begin{bmatrix}
		\vb{E} & -\vb{B} \\
		\vb{E} & \vb{A}
	\end{bmatrix}
	= \begin{bmatrix}
		\vb{A}+\vb{B} & -\vb{A} \vb{B}+\vb{B}\vb{A} \\
		\vb{B} & \vb{B}\vb{A}
	\end{bmatrix}
	= \begin{bmatrix}
		\vb{A}+\vb{B} & \vb0 \\
		\vb{B} & \vb{A} \vb{B}
	\end{bmatrix}.
\]
由\cref{equation:矩阵的秩.分块矩阵的秩的等式1} 可知\[
	\rank\vb{A}+\rank\vb{B}
	= \rank\begin{bmatrix}
		\vb{A} & \vb0 \\
		\vb0 & \vb{B}
	\end{bmatrix}
	\geq \rank\begin{bmatrix}
		\vb{A}+\vb{B} & \vb0 \\
		\vb{B} & \vb{B}\vb{A}
	\end{bmatrix}
	\geq \rank(\vb{A}+\vb{B}) + \rank(\vb{A} \vb{B}).
	\qedhere
\]
\end{proof}
\end{example}

\begin{example}\label{example:矩阵乘积的秩.两个向量的乘积的秩}
设\(\vb\alpha,\vb\beta\)是\(n\)维非零列向量.
证明:\(\rank(\vb\alpha\vb\beta^T)=1\).
\begin{proof}
因为\(\vb\alpha,\vb\beta\neq\vb0\),
所以\(\rank\vb\alpha=\rank\vb\beta=1\),
且\(\vb\alpha\vb\beta^T\neq\vb0\),
从而\(\rank(\vb\alpha\vb\beta^T)>0\),
再根据\cref{theorem:线性方程组.矩阵乘积的秩} 可知
\(\rank(\vb\alpha\vb\beta^T)
\leq \min\{\rank\vb\alpha,\rank\vb\beta^T\}
= 1\),
因此\(\rank(\vb\alpha\vb\beta^T)=1\).
\end{proof}
\end{example}

\begin{example}\label{example:矩阵乘积的秩.分块矩阵的秩的等式2}
设\(\vb{A} \in M_{s \times n}(K),
\vb{B} \in M_{s \times m}(K)\).
证明:\begin{equation}
	\max\{\rank\vb{A},\rank\vb{B}\} \leq \rank(\vb{A},\vb{B}) \leq \rank\vb{A} + \rank\vb{B}.
\end{equation}
%@credit: {e9b17d8d-3be5-4f44-9c7a-a5e6122a69e2} 提出取等条件如下:
当且仅当\(\vb{A}\)的列空间包含于\(\vb{B}\)的列空间,或\(\vb{B}\)的列空间包含于\(\vb{A}\)的列空间时,成立\begin{equation*}
	\max\{\rank\vb{A},\rank\vb{B}\} = \rank(\vb{A},\vb{B}).
\end{equation*}
当且仅当\((\vb{A},\vb{B})\)的列空间等于\(\vb{A}\)的列空间与\(\vb{B}\)的列空间的直和时,成立\begin{equation*}
	\rank(\vb{A},\vb{B}) = \rank\vb{A} + \rank\vb{B}.
\end{equation*}
%TODO 取等条件留待以后证明
\begin{proof}
\def\as{\AutoTuple{\vb\alpha}{n}}
\def\bs{\AutoTuple{\vb\beta}{m}}
\def\asi{\vb\alpha_{i_1},\dotsc,\vb\alpha_{i_r}}
\def\bsj{\vb\beta_{j_1},\dotsc,\vb\beta_{j_t}}
设\(\rank\vb{A} = r,
\rank\vb{B} = t\).
对\(\vb{A}\)、\(\vb{B}\)分别按列分块得\[
	\vb{A} = (\as),
	\qquad
	\vb{B} = (\bs).
\]
易见\[
	(\vb{A},\vb{B}) = (\as,\bs),
\]且\[
	\rank\{\as\} = r,
	\qquad
	\rank\{\bs\} = t.
\]

假设\(\as\)可由其极大线性无关组\(\asi\)线性表出,
\(\bs\)可由其极大线性无关组\(\bsj\)线性表出,
那么向量组\[
	V_1=\{\as\}\cup\{\bs\}
\]可由向量组\[
	V_2=\{\asi\}\cup\{\bsj\}
\]线性表出,
即\(V_1 \subseteq \Span V_2\),
则由\cref{theorem:向量空间.向量组的秩的比较1} 有\[
	\rank V_1
	\leq
	\rank V_2
	\leq
	r+t;
\]
于是\(\rank(\vb{A},\vb{B}) = \rank V_1 \leq r+t\).

又因为\hyperref[theorem:向量空间.向量组的秩的比较2]{部分组的秩总是小于或等于全组的秩},
而\[
	\{\asi\},\{\bsj\} \subseteq V_1,
\]
所以\[
	\rank\{\asi\},\rank\{\bsj\} \leq \rank V_1,
\]
于是\(\max\{\rank\vb{A},\rank\vb{B}\} \leq \rank(\vb{A},\vb{B})\).
\end{proof}
\end{example}

\begin{example}\label{example:矩阵乘积的秩.任意同型矩阵之和的秩}
%\cref{example:矩阵乘积的秩.可交换矩阵之和的秩}
设\(\vb{A}\)、\(\vb{B}\)都是\(s \times n\)矩阵.
证明:\begin{equation*}
	\rank(\vb{A}+\vb{B}) \leq \rank\vb{A} + \rank\vb{B}.
\end{equation*}
\begin{proof}
\def\asi{\vb\alpha_{i_1},\vb\alpha_{i_2},\dotsc,\vb\alpha_{i_r}}
\def\bsj{\vb\beta_{j_1},\vb\beta_{j_2},\dotsc,\vb\beta_{j_t}}
设\(\rank\vb{A} = r\),\(\rank\vb{B} = t\).
对\(\vb{A}\)、\(\vb{B}\)分别按列分块得\[
	\vb{A} = (\AutoTuple{\vb\alpha}{n}), \qquad
	\vb{B} = (\AutoTuple{\vb\beta}{m}),
\]
则\[
	\vb{A} + \vb{B} = (\vb\alpha_1 + \vb\beta_1,\vb\alpha_2 + \vb\beta_2,\dotsc,\vb\alpha_n + \vb\beta_n).
\]
由于\(\AutoTuple{\vb\alpha}{n}\)可由其极大线性无关组\(\asi\)线性表出,
\(\AutoTuple{\vb\beta}{m}\)可由其极大线性无关组\(\bsj\)线性表出,
故\[
	\vb\alpha_1 + \vb\beta_1,\vb\alpha_2 + \vb\beta_2,\dotsc,\vb\alpha_n + \vb\beta_n
\]可由向量组\[
	\asi,\bsj
\]线性表出,
结论显然成立.
\end{proof}
\end{example}
\begin{remark}
\cref{example:矩阵乘积的秩.任意同型矩阵之和的秩} 的意义在于:
既然由\cref{theorem:线性方程组.矩阵的秩的性质3}
可知一个矩阵乘以非零常数不变秩,
那么对于一个由矩阵构成的一次多项式,
我们可以把其中的某些项消掉,
像这样:\begin{equation*}
	\rank(x_1\vb{A}+x_2\vb{B}) + \rank(y_1\vb{A}+y_2\vb{B})
	\geq
	\max\left\{
		\rank\left( k \vb{A} \right),
		\rank\left( k \vb{B} \right)
	\right\},
\end{equation*}
其中\(k = \begin{vmatrix}
	x_1 & x_2 \\
	y_1 & y_2
\end{vmatrix}\);
如果进一步有\(k \neq 0\),
则有\begin{equation*}
	\rank(x_1\vb{A}+x_2\vb{B}) + \rank(y_1\vb{A}+y_2\vb{B})
	\geq
	\max\left\{
		\rank\vb{A},
		\rank\vb{B}
	\right\}.
\end{equation*}
\end{remark}

\begin{example}\label{example:矩阵乘积的秩.矩阵的一次多项式的秩之和}
%\cref{example:矩阵乘积的秩.矩阵的一次多项式的秩之和.取等条件1}
设\(\vb{A}\)是\(n\)阶矩阵.
证明:\(n \leq \rank(\vb{A} + \vb{E}) + \rank(\vb{A} - \vb{E})\).
\begin{proof}
由于\(\rank(\vb{A} - \vb{E}) = \rank(\vb{E} - \vb{A})\),
所以\[
	\rank(\vb{A} + \vb{E}) + \rank(\vb{A} - \vb{E})
	= \rank(\vb{A} + \vb{E}) + \rank(\vb{E} - \vb{A}).
\]
由\cref{example:矩阵乘积的秩.可交换矩阵之和的秩} 可知\[
	\rank(\vb{A} + \vb{E}) + \rank(\vb{E} - \vb{A})
	\geq \rank(\vb{A} + \vb{E} + \vb{E} - \vb{A})
	= \rank(2\vb{E})
	= \rank\vb{E}
	= n.
\]
因此\(\rank(\vb{A} + \vb{E}) + \rank(\vb{A} - \vb{E}) \geq n\).
\end{proof}
\end{example}
\begin{example}\label{example:矩阵乘积的秩.矩阵的多项式的各个互素因式的秩之和}
设\(n\)阶矩阵\(\vb{A}\)满足\(\vb{A}^2 - 3 \vb{A} - 10 \vb{E} = \vb0\).
证明:\[
	\rank(\vb{A} - 5\vb{E}) + \rank(\vb{A} + 2\vb{E}) = n.
\]
\begin{proof}
因为\(\vb{A}\)满足\(\vb{A}^2 - 3 \vb{A} - 10 \vb{E} = (\vb{A} - 5\vb{E})(\vb{A} + 2\vb{E}) = \vb0\),
所以由\cref{example:矩阵乘积的秩.乘积为零的两个矩阵的秩之和} 可知\[
	\rank(\vb{A} - 5\vb{E}) + \rank(\vb{A} + 2\vb{E}) \leq n.
\]
又因为\begin{align*}
	&\rank(\vb{A} - 5\vb{E}) + \rank(\vb{A} + 2\vb{E}) \\
	&\geq \rank[(5\vb{E} - \vb{A}) + (\vb{A} + 2\vb{E})] \\
	&= \rank(7\vb{E})
	= n,
\end{align*}
所以\(\rank(\vb{A} - 5\vb{E}) + \rank(\vb{A} + 2\vb{E}) = n\).
\end{proof}
\end{example}
\begin{example}
%@credit: {5a781423-ba4e-4629-ac1a-eac743a4d445},{8b6edada-f2fd-4ae5-9020-eb533149a54c}
设数域\(K\)上的一个一元多项式\(f(x)\)可以分解为\(m\)个互素的多项式的乘积,
即\[
	f(x) = p_1(x) \cdot p_2(x) \dotsm p_m(x).
\]
证明:如果矩阵\(\vb{A} \in M_n(K)\)满足\(f(\vb{A}) = \vb0\),
则\[
	\sum_{k=1}^m \rank(p_k(A))
	= (m-1)n.
\]
%TODO proof 具体证明过程参考2024年10月22日凌晨在[小飞机群]的聊天记录
\end{example}

\begin{example}
设\(\vb{A} \in M_{s \times n}(K)\ (s \neq n)\).
证明:\(\det(\vb{A} \vb{A}^T) \det(\vb{A}^T \vb{A}) = 0\).
\begin{proof}
由\cref{theorem:矩阵乘积的秩.多行少列矩阵与少行多列矩阵的乘积的行列式} 可知,
要么成立\(\det(\vb{A} \vb{A}^T) = 0\),
要么成立\(\det(\vb{A}^T \vb{A}) = 0\),
但总归有\[
	\det(\vb{A} \vb{A}^T) \det(\vb{A}^T \vb{A}) = 0.
	\qedhere
\]
\end{proof}
\end{example}

\begin{example}
设\(\vb{A}\)是\(m \times n\)矩阵,
\(\vb{B}\)是\(n \times m\)矩阵,
\(\vb{E}\)是\(m\)阶单位矩阵,
且\(\vb{A} \vb{B} = \vb{E}\).
求\(\vb{A}\)与\(\vb{B}\)的秩.
\begin{solution}
假设\(m > n\),
由\cref{theorem:矩阵乘积的秩.多行少列矩阵与少行多列矩阵的乘积的行列式}
可知\(\abs{\vb{A} \vb{B}} = 0\),
但是由题设条件\(\vb{A} \vb{B} = \vb{E}\)可知\[
	\abs{\vb{A} \vb{B}} = \abs{\vb{E}} = 1 \neq 0,
\]
矛盾,故必有\(m \leq n\).
那么\[
	\rank\vb{A},\rank\vb{B} \leq \min\{m,n\} = m,
\]
从而有\[
	\min\{\rank\vb{A},\rank\vb{B}\} \leq m.
\]
由\cref{theorem:线性方程组.矩阵乘积的秩} 可知\[
	\rank(\vb{A} \vb{B}) \leq \min\{\rank\vb{A},\rank\vb{B}\},
\]
即\(\rank(\vb{A} \vb{B}) \leq m\).
由题设条件\(\vb{A} \vb{B} = \vb{E}\)可知\[
	\rank(\vb{A} \vb{B}) = \rank\vb{E} = m.
\]
那么\[
	m \leq \rank\vb{A} \leq m,
	\qquad
	m \leq \rank\vb{B} \leq m,
\]
于是\(\rank\vb{A} = \rank\vb{B} = m\).
\end{solution}
\end{example}

\begin{example}
计算行列式\(\det\vb{D}\),
其中\[
	\vb{D} = \begin{bmatrix}
		1 & \cos(\alpha_1-\alpha_2) & \cos(\alpha_1-\alpha_3) & \dots & \cos(\alpha_1-\alpha_n) \\
		\cos(\alpha_1-\alpha_2) & 1 & \cos(\alpha_2-\alpha_3) & \dots & \cos(\alpha_2-\alpha_n) \\
		\cos(\alpha_1-\alpha_3) & \cos(\alpha_2-\alpha_3) & 1 & \dots & \cos(\alpha_3-\alpha_n) \\
		\vdots & \vdots & \vdots & & \vdots \\
		\cos(\alpha_1-\alpha_n) & \cos(\alpha_2-\alpha_n) & \cos(\alpha_3-\alpha_n) & \dots & 1
	\end{bmatrix}.
\]
\begin{solution}
记\(\vb{D} = (d_{ij})_n\).
由\cref{equation:函数.三角函数.和积互化公式2} 可知\[
	d_{ij} = \cos(\alpha_i-\alpha_j)
	= \cos\alpha_i\cos\alpha_j+\sin\alpha_i\sin\alpha_j,
	\quad i,j=1,2,\dotsc,n.
\]
记\[
	\vb{A} = \begin{bmatrix}
		\cos\alpha_1 & \cos\alpha_2 & \dots & \cos\alpha_n \\
		\sin\alpha_1 & \sin\alpha_2 & \dots & \sin\alpha_n
	\end{bmatrix},
\]那么\(\vb{D} = \vb{A}^T \vb{A}\).

由\cref{theorem:线性方程组.矩阵的秩的性质2} 可知,
\(\rank\vb{A} = \rank\vb{A}^T \leq \min\{n,2\}\).
由\cref{theorem:线性方程组.矩阵乘积的秩} 可知,\[
	\rank(\vb{A}^T\vb{A}) \leq \min\left\{\rank\vb{A}^T,\rank\vb{A}\right\} = \rank\vb{A}.
\]

当\(n=1\)时,\(\abs{\vb{D}}=1\).

当\(n=2\)时,\[
	\abs{\vb{D}}
	= \begin{vmatrix}
		1 & \cos(\alpha_1-\alpha_2) \\
		\cos(\alpha_1-\alpha_2) & 1
	\end{vmatrix}
	= 1 - \cos^2(\alpha_1-\alpha_2).
\]

当\(n>2\)时,
\(\rank(\vb{A}^T\vb{A}) \leq \rank\vb{A} \leq2\),
\(\vb{D} = \vb{A}^T \vb{A}\)不满秩,
故由\cref{theorem:向量空间.满秩方阵的行列式非零} 有\(\abs{\vb{D}}=0\).
\end{solution}
\end{example}

\begin{example}
设\(\vb\alpha_1=(1,2,-1,0)^T,\vb\alpha_2=(1,1,0,2)^T,\vb\alpha_3=(2,1,1,a)^T\).
若\(\dim\opair{\AutoTuple{\vb\alpha}{3}}=2\),求\(a\).
\begin{solution}
除了利用\cref{equation:线性方程组.子空间的维数与向量组的秩的联系} 在将矩阵\[
	\vb{A} = (\vb\alpha_1,\vb\alpha_2,\vb\alpha_3)
	= \begin{bmatrix}
		1 & 1 & 2 \\
		2 & 1 & 1 \\
		-1 & 0 & 1 \\
		0 & 2 & a
	\end{bmatrix}
\]化为阶梯形矩阵以后,
根据\(\rank\{\AutoTuple{\vb\alpha}{3}\}=2\)求出\(a\)的值这种方法以外,
我们还可以利用本节\hyperref[definition:线性方程组.矩阵的秩的定义]{矩阵的秩的定义},
得出\(\rank\vb{A}=\dim\opair{\AutoTuple{\vb\alpha}{3}}=2\)这一结论,
从而根据\cref{definition:线性方程组.矩阵的秩的定义} 可知,
\(\vb{A}\)中任意3阶子式全都为零.
对于\(\vb{A}\)这么一个\(4\times3\)矩阵,
任意去掉不含\(a\)的一行(不妨去掉第一行)得到一个行列式必为零:\[
	\begin{vmatrix}
	2 & 1 & 1 \\
	-1 & 0 & 1 \\
	0 & 2 & a
	\end{vmatrix}
	= a - 6 = 0,
\]
解得\(a = 6\).
\end{solution}
\end{example}

\section{西尔维斯特不等式}
\begin{theorem}
设\(\A \in M_{s \times n}(K),
\B \in M_{n \times t}(K)\),
则\begin{equation}\label{equation:线性方程组.西尔维斯特不等式}
	\rank\A + \rank\B - n \leq \rank(\A\B).
\end{equation}
当且仅当\[
	\rank\begin{bmatrix}
		\A & \vb0 \\
		\E_n & \B
	\end{bmatrix}
	= \rank\begin{bmatrix}
		\A & \vb0 \\
		\vb0 & \B
	\end{bmatrix}
\]时,
\cref{equation:线性方程组.西尔维斯特不等式} 取“=”号.
\begin{proof}
由\cref{equation:矩阵的秩.分块矩阵的秩的等式1},\[
	\rank\begin{bmatrix}
		\E_n & \z \\
		\z & \A\B
	\end{bmatrix}
	= n + \rank(\A\B).
	\eqno(1)
\]
又因为\[
	\begin{bmatrix}
		\B & \E_n \\
		\z & \A
	\end{bmatrix}
	= \begin{bmatrix}
		\E_n & \z \\
		\A & \E_s
	\end{bmatrix}
	\begin{bmatrix}
		\E_n & \z \\
		\z & \A\B
	\end{bmatrix}
	\begin{bmatrix}
		\E_n & -\B \\
		\z & \E_t
	\end{bmatrix}
	\begin{bmatrix}
		\z & \E_s \\
		-\E_t & \z
	\end{bmatrix},
\]
而\[
	\begin{bmatrix}
		\E_n & \z \\
		\A & \E_s
	\end{bmatrix}, \qquad
	\begin{bmatrix}
		\E_n & -\B \\
		\z & \E_t
	\end{bmatrix},
	\quad\text{和}\quad
	\begin{bmatrix}
		\z & \E_s \\
		-\E_t & \z
	\end{bmatrix}
\]这三个矩阵都是满秩矩阵,
所以\[
	\rank\begin{bmatrix}
		\E_n & \z \\
		\z & \A\B
	\end{bmatrix}
	= \rank\begin{bmatrix}
		\B & \E_n \\
		\z & \A
	\end{bmatrix}.
	\eqno(2)
\]
再由\cref{equation:矩阵的秩.分块矩阵的秩的不等式} 有\[
	\rank\begin{bmatrix}
		\B & \E_n \\
		\z & \A
	\end{bmatrix}
	\geq \rank\A+\rank\B.
	\eqno(3)
\]
因此,\(\rank\A + \rank\B \leq n + \rank(\A\B)\).
%@see: https://math.stackexchange.com/a/2414197/591741
%@see: http://www.m-hikari.com/imf-password2009/33-36-2009/luIMF33-36-2009.pdf
\end{proof}
%@see: https://math.stackexchange.com/questions/872587/equality-case-in-the-frobenius-rank-inequality
\end{theorem}

我们把\cref{equation:线性方程组.西尔维斯特不等式}
称为\DefineConcept{西尔维斯特不等式}(Sylvester rank inequality).

\begin{example}\label{example:西尔维斯特不等式.可逆矩阵的正整数次幂可逆}
设\(\A \in M_n(K)\)是可逆矩阵.
证明:\(\A^m\ (m=2,3,\dotsc)\)都是可逆矩阵.
\begin{proof}
首先证明\(\A^2\)是可逆矩阵.
由\cref{theorem:线性方程组.矩阵乘积的秩} 可知
\(\rank\A^2 \leq \rank\A = n\).
再由\hyperref[equation:线性方程组.西尔维斯特不等式]{西尔维斯特不等式}可知
\(\rank\A^2 \geq 2\rank\A - n = n\).
因此\(\rank\A^2 = n\).
接下来运用数学归纳法容易证得\(\rank\A^m = n\ (m=2,3,\dotsc)\).
\end{proof}
\end{example}

\begin{example}\label{example:矩阵乘积的秩.乘积为零的两个矩阵的秩之和}
%@see: 《高等代数(第三版 上册)》(丘维声) P143 习题4.5 1.
设\(\A \in M_{s \times n}(K),
\B \in M_{n \times m}(K)\).
如果\(\A\B=\vb0\),
那么\[
	\rank\A + \rank\B \leq n.
\]
\begin{proof}
由\hyperref[equation:线性方程组.西尔维斯特不等式]{西尔维斯特不等式}立即可得.
\end{proof}
\end{example}

我们可以利用\hyperref[equation:线性方程组.西尔维斯特不等式]{西尔维斯特不等式}证明一个重要结论:
\begin{proposition}\label{theorem:向量空间.用列满秩矩阵左乘任一矩阵不变秩}
设\(\A \in M_{m \times s}(K),
\B \in M_{s \times n}(K)\).
\begin{itemize}
	\item 如果\(\A\)是列满秩矩阵,则\(\rank(\A\B) = \rank\B\).
	\item 如果\(\B\)是行满秩矩阵,则\(\rank(\A\B) = \rank\A\).
\end{itemize}
\begin{proof}
假设\(\A\)是列满秩矩阵,
即\(\rank\A = s\).
由\hyperref[equation:线性方程组.西尔维斯特不等式]{西尔维斯特不等式}有\[
	\rank(\A\B) \geq \rank\B + \rank\A - s
	= \rank\B. % 代入\(\rank\A = s\)
	\eqno(1)
\]
又由\cref{theorem:线性方程组.矩阵乘积的秩} 可知\[
	\rank(\A\B) \leq \rank\B.
	\eqno(2)
\]
由(1)(2)两式便有\(\rank(\A\B) = \rank\B\).

同理可证:如果\(\B\)是行满秩矩阵,则\(\rank(\A\B) = \rank\A\).
\end{proof}
\end{proposition}
\begin{remark}
\cref{theorem:向量空间.用列满秩矩阵左乘任一矩阵不变秩} 说明:
对于任意一个矩阵,我们用一个列满秩矩阵左乘它,不变秩;
用一个行满秩矩阵右乘它,也不变秩.
%@credit: {439f21f7-fd12-4996-b112-dcbb8b467950} 给出逆命题不成立的反例
但是要注意\cref{theorem:向量空间.用列满秩矩阵左乘任一矩阵不变秩} 的逆命题并不成立,
只要取\(\A = \B = \vb0\),就有\(\rank(\A \B) = \rank\A = \rank\B = 0\),
但是零矩阵\(\A,\B\)显然既不是列满秩矩阵也不是行满秩矩阵.
%@credit: {523653db-1ec1-4b3a-972f-e44311ded599} 给出了条件增强后逆命题仍不成立的反例(\(\A = \B\)是一个幂等矩阵)
%@credit: {de3029b8-10a6-4ae5-8f64-108dae1c10a9} 给出了下面用到的具体的幂等矩阵
即便增加一个条件 --- “\(\rank(\A \B)\neq0\)”,
也不能断定\cref{theorem:向量空间.用列满秩矩阵左乘任一矩阵不变秩} 的逆命题一定成立,
这是因为只要取\(\A = \B
= \begin{bmatrix}
	1 & 0 \\
	0 & 0
\end{bmatrix}\)
(实际上对于任意一个不满秩的\hyperref[definition:幂等矩阵.幂等矩阵的定义]{幂等矩阵}~\(\A\),
逆命题始终不成立),
就有\(\A \B = \A\),
从而有\(\rank(\A \B) = \rank\A = \rank\B\),
但是\(\A,\B\)显然既不是列满秩矩阵也不是行满秩矩阵.
% 我们只要把\cref{theorem:向量空间.用列满秩矩阵左乘任一矩阵不变秩} 中的各个矩阵转置,
% 得到\(\A_1 = \A^T \in M_{n \times s}(K),
% \B_1 = \B^T \in M_{s \times m}(K),
% \A_1\B_1 = \A^T\B^T = (\B\A)^T\),
% 易见\(\B_1\)是行满秩矩阵,且\[
% 	\rank(\A_1\B_1)
% 	= \rank(\B\A)^T
% 	= \rank(\B\A)
% 	= \rank\A
% 	= \rank\A^T.
% \]
\end{remark}
\begin{corollary}\label{theorem:西尔维斯特不等式.分块矩阵的秩的等式3}
%@see: 《2018年全国硕士研究生入学统一考试(数学一)》一选择题/第6题
设\(\A \in M_{m \times n}(K),
\B \in M_{n \times t}(K),
\C \in M_{s \times m}(K)\),
则\begin{gather*}
	\rank\begin{bmatrix}
		\A \\
		\C \A
	\end{bmatrix}
	= \rank\A, \\
	\rank(\A,\A \B)
	= \rank\A.
\end{gather*}
\begin{proof}
由\cref{theorem:向量空间.用列满秩矩阵左乘任一矩阵不变秩} 立即可得.
\end{proof}
\end{corollary}
\begin{example}
设\(\A \in M_{m \times n}(K),
\B \in M_{n \times t}(K)\).
举例说明:\(\rank(\A,\B \A) \neq \rank\A\).
\begin{solution}
取\[
%@Mathematica: A = {{1, 0}, {0, 0}}
	\A = \begin{bmatrix}
		1 & 0 \\
		0 & 0
	\end{bmatrix},
	\qquad
%@Mathematica: B = {{1, 0}, {1, 1}}
	\B = \begin{bmatrix}
		1 & 0 \\
		1 & 1
	\end{bmatrix},
\]
那么\[
%@Mathematica: B.A
	\B \A = \begin{bmatrix}
		1 & 0 \\
		1 & 0
	\end{bmatrix},
	\qquad
%@Mathematica: Join[A, B.A, 2]
	(\A,\B \A) = \begin{bmatrix}
		1 & 0 & 1 & 0 \\
		0 & 0 & 1 & 0
	\end{bmatrix},
\]
%@Mathematica: MatrixRank[Join[A, B.A, 2]]
%@Mathematica: MatrixRank[A]
于是\(\rank(\A,\B \A) = 2 \neq 1 = \rank\A\).
\end{solution}
\end{example}
\begin{example}\label{example:向量空间.等秩矩阵的行向量组的等价性}
设\(\A \in M_{s \times n}(K),
\B \in M_{m \times n}(K)\),
且\(\rank\A
= \rank\begin{bmatrix}
	\A \\ \B
\end{bmatrix}\),
则\(\A\)的行向量组的任意一个极大线性无关组
都是\(\begin{bmatrix}
	\A \\ \B
\end{bmatrix}\)的行向量组的极大线性无关组.
\begin{proof}
\begin{proof}[证法一]
%@credit: {5a781423-ba4e-4629-ac1a-eac743a4d445},{5f4d2f8a-fc8b-4798-85d6-98670f6761e7},{6f21d9e6-edca-4b6f-9dff-364f3d62dcce}
设\(A' = \AutoTuple{\vb\alpha}{r}\)是\(\A\)的行向量组的一个极大线性无关组.
用反证法.
假设\(\B\)的某一个行向量\(\vb\beta\)不可以由\(A'\)线性表出,
即向量组\(A' \cup \{\vb\beta\}\)线性无关,
那么\begin{equation*}
	\rank\begin{bmatrix}
		\A \\ \B
	\end{bmatrix}
	\geq \card(A' \cup \{\vb\beta\})
	= r + 1,
\end{equation*}
与题设矛盾!
因此\(\B\)的行向量组可以由\(A'\)线性表出,
\(A'\)是\(\begin{bmatrix}
	\A \\ \B
\end{bmatrix}\)的极大线性无关组.
\end{proof}
\begin{proof}[证法二]
%@credit: {6c964576-9569-472e-969e-54699e35974b}
显然\(\A\)的行向量组张成的空间\(\SpanR\A\)是\(\begin{bmatrix}
	\A \\ \B
\end{bmatrix}\)的行向量组张成的空间\(\SpanR\begin{bmatrix}
	\A \\ \B
\end{bmatrix}\)的子集.
因为\(\rank\A
= \rank\begin{bmatrix}
	\A \\ \B
\end{bmatrix}\),
所以由\cref{theorem:线性方程组.齐次线性方程组的解向量个数} 可知
\(\dim\SpanR\A
= \dim\SpanR\begin{bmatrix}
	\A \\ \B
\end{bmatrix}\).
再由\cref{theorem:向量空间.两个非零子空间的关系2} 可知
\(\SpanR\A = \SpanR\begin{bmatrix}
	\A \\ \B
\end{bmatrix}\).
\end{proof}\let\qed\relax
\end{proof}
\end{example}

\begin{example}
%@see: 《高等代数(第三版 上册)》(丘维声) P143 习题4.5 2.
设\(\A\)是数域\(K\)上的\(n\)阶非零矩阵.
证明:“存在一个\(n \times m\)非零矩阵\(\B\),使得\(\A\B=\vb0\)”的充分必要条件为
\(\abs{\A}=0\).
\begin{proof}
%@credit: {8b6edada-f2fd-4ae5-9020-eb533149a54c},{c5f76621-34f0-4114-845f-e36475300576},{5f4d2f8a-fc8b-4798-85d6-98670f6761e7},{ce603838-a24d-4616-9395-d7b223e8cb72}
必要性.
假设存在一个\(n \times m\)非零矩阵\(\B\),使得\(\A\B=\vb0\).
那么只要任取\(\B\)的一个非零列向量\(\vb\beta\),
就有\(\A\vb\beta = \vb0\),
即\(\vb\beta\)是齐次方程\(\A\vb{x}=\vb0\)的非零解,
故\(\rank\A<n\),
从而有\(\abs{\A}=0\).

充分性.
假设\(\abs{\A}=0\),
则\(\rank\A = r < n\),
在\(\A\)的核空间\(\Ker\A\)中
任取一个非零向量\(\vb\beta\),
% 从而成立\(\A\vb\beta=\vb0\),
构成一个\(n \times m\)矩阵\(\B\),
使得\(\B\)的每一列都是\(\vb\beta\),
那么\(\A\B=\vb0\).
\end{proof}
\end{example}
\begin{example}
%@see: 《高等代数(第三版 上册)》(丘维声) P143 习题4.5 3.
设\(\A\)是数域\(K\)上的\(n\)阶方阵,
\(\B\)是数域\(K\)上的\(n \times m\)行满秩矩阵,
\(\E\)是数域\(K\)上的\(n\)阶单位矩阵.
证明:\begin{itemize}
	\item 如果\(\A\B=\vb0\),则\(\A=\vb0\).
	\item 如果\(\A\B=\B\),则\(\A=\E\).
\end{itemize}
\begin{proof}
因为\(\B\)是行满秩矩阵,
由\cref{theorem:向量空间.用列满秩矩阵左乘任一矩阵不变秩} 可知,
\(\rank(\A\B) = \rank\A\).
假设\(\A\B=\vb0\),
则\[
	\rank\A=\rank(\A\B)=\rank\vb0=0,
\]
那么由\cref{theorem:向量空间.秩为零的矩阵必为零矩阵} 可知,
\(\A=\vb0\).

假设\(\A\B=\B\),
则\((\A-\E)\B=\vb0\),
那么由上述结论可知\(\A-\E=\vb0\),
因此\(\A=\E\).
\end{proof}
\end{example}

\begin{example}
%@see: 《高等代数(第三版 上册)》(丘维声) P143 习题4.5 9.
设矩阵\(\A \in M_n(K)\ (n\geq2)\),\(\A^*\)是\(\A\)的伴随矩阵.
证明:\begin{equation}\label{equation:伴随矩阵.伴随矩阵的秩}
	\rank\A^* = \left\{ \begin{array}{cl}
		n, & \rank\A=n, \\
		1, & \rank\A=n-1, \\
		0, & \rank\A<n-1.
	\end{array} \right.
\end{equation}
\begin{proof}
根据恒等式 \labelcref{equation:行列式.伴随矩阵.恒等式1} 有\[
	\A \A^* = \abs{\A} \E,
	\eqno(1)
\]
于是\[
	\abs{\A \A^*} = \abs{\abs{\A} \E}.
	\eqno(2)
\]
根据\cref{theorem:行列式.矩阵乘积的行列式},
(2)式左边可以分解为\[
	\abs{\A \A^*} = \abs{\A} \abs{\A^*}.
\]
根据\cref{theorem:行列式.性质2.推论2},
(2)式右边可以分解为\[
	\abs{\abs{\A} \E} = \abs{\A}^n \abs{\E} = \abs{\A}^n.
\]
于是我们得到\[
	\abs{\A} \abs{\A^*} = \abs{\A}^n.
	\eqno(3)
\]

下面我们根据矩阵\(\A\)的秩的取值分类讨论.
\begin{enumerate}
	\item 当\(\rank\A = n\)时,
	由\cref{theorem:向量空间.满秩方阵的行列式非零} 有\(\abs{\A} \neq 0\);
	于是(3)式可化简得\[
		\abs{\A^*}
		= \abs{\A}^{n-1} \neq 0,
	\]
	再次利用\cref{theorem:向量空间.满秩方阵的行列式非零} 便知\(\rank\A^* = n\).

	\item 当\(\rank\A = n-1\)时,
	因为\(\rank\A<n\),
	所以\(\abs{\A} = 0\),
	那么由(1)式可知\(\A \A^* = \z\),
	那么利用\hyperref[equation:线性方程组.西尔维斯特不等式]{西尔维斯特不等式}便得\[
		\rank\A + \rank\A^* \leq n,
	\]
	移项得\[
		\rank\A^*
		\leq n - \rank\A
		= n - (n-1)
		= 1.
	\]
	又因为\(n > 1\),
	\(\rank\A = n-1 > 0\),
	根据矩阵的秩的定义,\(\A\)有一个\(n-1\)阶子式不等于零,
	再根据伴随矩阵的定义,这个子式是\(\A^*\)的一个元素,从而\(\A^*\neq\z\),
	\(\rank\A^*>0\).
	因此\(\rank\A^* = 1\).

	\item 当\(\rank\A < n-1\)时,
	\(\A\)的所有\(n-1\)阶子式全为零,
	也就是说\(\A\)的任意一个元素的代数余子式为零,
	根据伴随矩阵的定义,\(\A^* = \z\),
	因此\(\rank\A^* = 0\).
	\qedhere
\end{enumerate}
\end{proof}
\end{example}
\begin{remark}
从\cref{equation:伴随矩阵.伴随矩阵的秩} 还推导出以下结论:
%@see: 《高等代数(第三版 上册)》(丘维声) P143 习题4.5 8.
%@see: 《高等代数学习指导书(第三版)》(姚慕生、谢启鸿) P63 例2.26
\begin{equation}\label{equation:伴随矩阵.伴随矩阵的行列式}
	\abs{\A^*}
	= \abs{\A}^{n-1}.
\end{equation}
\end{remark}

\begin{example}
%@see: 《高等代数(第三版 上册)》(丘维声) P143 习题4.5 10.
%@see: 《高等代数学习指导书(第三版)》(姚慕生、谢启鸿) P64 例2.27
设矩阵\(\A \in M_n(K)\ (n\geq2)\),\(\A^*\)是\(\A\)的伴随矩阵.
证明:\begin{equation}\label{equation:伴随矩阵.伴随矩阵的伴随}
	(\A^*)^* = \left\{ \begin{array}{cl}
		\abs{\A}^{n-2} \A, & n\geq3, \\
		\A, & n=2.
	\end{array} \right.
\end{equation}
\begin{proof}
当\(n=2\)时,
设\(\A = \begin{bmatrix}
	a_{11} & a_{12} \\
	a_{21} & a_{22}
\end{bmatrix}\),
那么\[
	\A^* = \begin{bmatrix}
		A_{11} & A_{21} \\
		A_{12} & A_{22}
	\end{bmatrix}
	= \begin{bmatrix}
		a_{22} & - a_{12} \\
		- a_{21} & a_{11}
	\end{bmatrix}.
\]
从而\[
	(\A^*)^* = \begin{bmatrix}
		a_{11} & a_{12} \\
		a_{21} & a_{22}
	\end{bmatrix}
	= \A.
\]

下面证明当\(n\geq3\)时,成立\((\A^*)^* = \abs{\A}^{n-2} \A\).

假设\(\A\)可逆.
那么由\cref{theorem:逆矩阵.逆矩阵的唯一性}
可知\(\A^* = \abs{\A} \A^{-1}\).
由\cref{equation:伴随矩阵.伴随矩阵的行列式}
可知\(\abs{\A^*} = \abs{\A}^{n-1}\).
由\cref{theorem:逆矩阵.伴随矩阵的逆与逆矩阵的伴随}
可知\((\A^*)^{-1} = \abs{\A}^{-1} \A\).
于是\begin{align*}
	(\A^*)^*
	&= \abs{\A^*} (\A^*)^{-1} \\
	&= \abs{\A}^{n-1} \cdot \abs{\A}^{-1} \A \\
	&= \abs{\A}^{n-2} \A
	\quad(n\geq2).
\end{align*}

假设\(\A\)不可逆,即\(\rank\A<n\),
那么由\cref{equation:伴随矩阵.伴随矩阵的秩}
可知\(\rank\A^* \leq 1 < 2 \leq n-1\),
从而\(\rank(\A^*)^* = 0\),
即\((\A^*)^* = \vb0\).
这时\(\abs{\A}=0\),
因此\((\A^*)^* = \abs{\A}^{n-2} \A\)仍然成立.
\end{proof}
\end{example}

我们还可以将\hyperref[equation:线性方程组.西尔维斯特不等式]{西尔维斯特不等式}进行如下的推广.
\begin{theorem}
设\(\A \in M_{s \times n}(K),
\B \in M_{n \times m}(K),
\C \in M_{m \times t}(K)\),
则\begin{equation}\label{equation:线性方程组.弗罗贝尼乌斯不等式}
	\rank(\A\B\C) \geq \rank(\A\B) + \rank(\B\C) - \rank\B.
\end{equation}
\begin{proof}
利用初等变换,有\[
	\begin{bmatrix}
		\B & \z \\
		\z & \A\B\C
	\end{bmatrix}
	\to \begin{bmatrix}
		\B & \z \\
		\A\B & \A\B\C
	\end{bmatrix}
	\to \begin{bmatrix}
		\B & -\B\C \\
		\A\B & \z
	\end{bmatrix}
	\to \begin{bmatrix}
		\B\C & \B \\
		\z & \A\B
	\end{bmatrix},
\]
于是\[
	\rank\B + \rank(\A\B\C)
	= \rank\begin{bmatrix}
		\B & \z \\
		\z & \A\B\C
	\end{bmatrix}
	= \rank\begin{bmatrix}
		\B\C & \B \\
		\z & \A\B
	\end{bmatrix}
	\geq \rank(\A\B) + \rank(\B\C).
	\qedhere
\]
\end{proof}
\end{theorem}

我们把\cref{equation:线性方程组.弗罗贝尼乌斯不等式}
称为\DefineConcept{弗罗贝尼乌斯不等式}(Frobenius rank inequality).

\begin{example}
%@see: 《2025年全国硕士研究生入学统一考试(数学一)》一选择题/第7题
设\(n\)阶矩阵\(\A,\B,\C\)满足
\(\rank\A + \rank\B + \rank\C = \rank(\A\B\C) + 2n\).
证明:\begin{gather*}
	\rank(\A\B\C) + n = \rank(\A\B) + \rank\C, \\
	\rank(\A\B) + n = \rank\A + \rank\B.
\end{gather*}
\begin{proof}
由\hyperref[equation:线性方程组.西尔维斯特不等式]{西尔维斯特不等式}
\(\rank\A + \rank\B - n \leq \rank(\A\B)\)可知\begin{gather*}
	\rank(\A\B\C) + n
	\geq \rank(\A\B) + \rank\C, \\
	\rank(\A\B\C) + n
	= (\rank\A + \rank\B - n) + \rank\C
	\leq \rank(\A\B) + \rank\C,
\end{gather*}
从而有\(\rank(\A\B\C) + n = \rank(\A\B) + \rank\C\),
进而有\begin{equation*}
	\rank(\A\B) + n
	= \rank(\A\B\C) + n - \rank\C + n
	= \rank\A + \rank\B.
	\qedhere
\end{equation*}
\end{proof}
\end{example}

\begin{example}\label{example:对合矩阵.对合矩阵的秩的性质1}
%@see: 《高等代数(第三版 上册)》(丘维声) P143 习题4.5 4.
设\(\A\)是数域\(K\)上的\(n\)阶对合矩阵,
即有\(\A^2=\E\),
则\[
	\rank(\E+\A)+\rank(\E-\A)=n.
\]
\begin{proof}
因为利用初等变换可以得到\begin{align*}
	&\hspace{-20pt}
	\begin{bmatrix}
		\E+\A & \z \\
		\z & \E-\A
	\end{bmatrix}
	\to \begin{bmatrix}
		\E+\A & \z \\
		\A(\E+\A) & \E-\A
	\end{bmatrix}
	= \begin{bmatrix}
		\E+\A & \z \\
		\A+\E & \E-\A
	\end{bmatrix} \\
	&\to \begin{bmatrix}
		\E+\A & \E-\A \\
		\z & \z
	\end{bmatrix}
	\to \begin{bmatrix}
		\E+\A & 2\E \\
		\z & \z
	\end{bmatrix}
	\to \begin{bmatrix}
		\E+\A & \E \\
		\z & \z
	\end{bmatrix}
	\to \begin{bmatrix}
		\A & \E \\
		\z & \z
	\end{bmatrix},
\end{align*}
所以\[
	\rank(\E+\A)+\rank(\E-\A)
	=\rank\begin{bmatrix}
		\E+\A & \z \\
		\z & \E-\A
	\end{bmatrix}
	= \rank\begin{bmatrix}
		\A & \E \\
		\z & \z
	\end{bmatrix}
	= n.
	\qedhere
\]
\end{proof}
\end{example}

\begin{example}\label{example:幂等矩阵.幂等矩阵的秩的性质1}
%@see: 《高等代数(第三版 上册)》(丘维声) P143 习题4.5 5.
设\(\A\)是数域\(K\)上的\(n\)阶幂等矩阵,
即有\(\A^2=\A\).
证明:\[
	\rank\A+\rank(\E-\A)=n.
\]
\begin{proof}
由于\[
	\A^2=\A
	\iff
	\A^2-\A=\z
	\iff
	\rank(\A^2-\A)=0.
\]
又因为利用初等变换可以得到\[
	\begin{bmatrix}
		\A & \z \\
		\z & \E_n-\A
	\end{bmatrix}
	\to \begin{bmatrix}
		\A & \z \\
		\A & \E_n-\A
	\end{bmatrix}
	\to \begin{bmatrix}
		\A & \A \\
		\A & \E_n
	\end{bmatrix}
	\to \begin{bmatrix}
		\A-\A^2 & \z \\
		\A & \E_n
	\end{bmatrix}
	\to \begin{bmatrix}
		\A-\A^2 & \z \\
		\z & \E_n
	\end{bmatrix},
\]
所以\(\rank\A+\rank(\E_n-\A)
=\rank(\A-\A^2)+n
=n\).
\end{proof}
\end{example}

\begin{example}\label{example:单位矩阵与两矩阵乘积之差.单位矩阵与两矩阵乘积之差的秩}
设\(\A \in M_{m \times n}(K),
\B \in M_{n \times m}(K)\).
证明:\[
	\rank(\E_m-\A\B)-\rank(\E_n-\B\A)=m-n.
\]
\begin{proof}
因为利用初等变换可以得到\begin{align*}
	\begin{bmatrix}
		\E_m-\A\B & \z \\
		\z & \E_n
	\end{bmatrix}
	&\to \begin{bmatrix}
		\E_m-\A\B & \A \\
		\z & \E_n
	\end{bmatrix}
	\to \begin{bmatrix}
		\E_m & \A \\
		\B & \E_n
	\end{bmatrix} \\
	&\to \begin{bmatrix}
		\E_m & \z \\
		\B & \E_n-\B\A
	\end{bmatrix}
	\to \begin{bmatrix}
		\E_m & \z \\
		\z & \E_n-\B\A
	\end{bmatrix},
\end{align*}
所以\[
	\rank(\E_m-\A\B)+n=m+\rank(\E_n-\B\A),
\]
移项便得\(\rank(\E_m-\A\B)-\rank(\E_n-\B\A)=m-n\).
\end{proof}
%\cref{example:逆矩阵.行列式降阶定理的重要应用1}
%\cref{example:单位矩阵与两矩阵乘积之差.单位矩阵与两矩阵乘积之差的行列式}
\end{example}

\begin{example}
设\(\A,\B,\C \in M_n(K)\),
\(\rank\C=n\),
\(\A(\B\A+\C)=\z\).
证明:\[
	\rank(\B\A+\C)=n-\rank\A.
\]
\begin{proof}
因为\(\A(\B\A+\C)=\z\),
所以根据\cref{equation:线性方程组.西尔维斯特不等式} 有\[
	\rank(\B\A+\C)+\rank\A \leq n.
\]
又利用初等变换可以得到\[
	\begin{bmatrix}
		\B\A+\C & \z \\
		\z & \A
	\end{bmatrix}
	\to \begin{bmatrix}
		\B\A+\C & \B\A \\
		\z & \A
	\end{bmatrix}
	\to \begin{bmatrix}
		\C & \B\A \\
		-\A & \A
	\end{bmatrix},
\]
于是\[
	\rank(\B\C+\C)
	+\rank\A
	= \rank\begin{bmatrix}
		\B\A+\C & \z \\
		\z & \A
	\end{bmatrix}
	= \rank\begin{bmatrix}
		\C & \B\A \\
		-\A & \A
	\end{bmatrix}
	\geq \rank\C = n.
	\qedhere
\]
\end{proof}
\end{example}

\begin{example}
%@see: 《2021年全国硕士研究生入学统一考试(数学一)》一选择题/第7题
设\(\A,\B \in M_n(\mathbb{R})\),
证明:\begin{equation*}
	\rank\begin{bmatrix}
		\A & \vb0 \\
		\vb0 & \A^T \A
	\end{bmatrix}
	= \rank\begin{bmatrix}
		\A & \A \B \\
		\vb0 & \A^T
	\end{bmatrix}
	= \rank\begin{bmatrix}
		\A & \vb0 \\
		\B \A & \A^T
	\end{bmatrix}
	= 2 \rank\A,
\end{equation*}
并举例说明\(\rank\begin{bmatrix}
	\A & \B \A \\
	\vb0 & \A \A^T
\end{bmatrix}
\neq 2 \rank\A\).
\begin{proof}
因为\begin{gather*}
	\rank\begin{bmatrix}
		\A & \vb0 \\
		\vb0 & \A^T \A
	\end{bmatrix}
	= \rank\A + \rank(\A^T \A), \\
	\begin{bmatrix}
		\A & \A \B \\
		\vb0 & \A^T
	\end{bmatrix}
	= \begin{bmatrix}
		\A & \vb0 \\
		\vb0 & \A^T
	\end{bmatrix}
	\begin{bmatrix}
		\E & \B \\
		\vb0 & \E
	\end{bmatrix}
	\implies
	\rank\begin{bmatrix}
		\A & \A \B \\
		\vb0 & \A^T
	\end{bmatrix}
	= \rank\begin{bmatrix}
		\A & \vb0 \\
		\vb0 & \A^T
	\end{bmatrix}
	= \rank\A + \rank\A^T, \\
	\begin{bmatrix}
		\A & \vb0 \\
		\B \A & \A^T
	\end{bmatrix}
	= \begin{bmatrix}
		\E & \vb0 \\
		\B & \E
	\end{bmatrix}
	\begin{bmatrix}
		\A & \vb0 \\
		\vb0 & \A^T
	\end{bmatrix}
	\implies
	\rank\begin{bmatrix}
		\A & \vb0 \\
		\B \A & \A^T
	\end{bmatrix}
	= \rank\begin{bmatrix}
		\A & \vb0 \\
		\vb0 & \A^T
	\end{bmatrix}
	= \rank\A + \rank\A^T,
\end{gather*}
而由\cref{theorem:向量空间.转置不变秩} 可知\(\rank\A = \rank\A^T\),
且由\cref{equation:矩阵乘积的秩.实矩阵及其转置矩阵的乘积的秩} 可知\(\rank(\A^T \A) = \rank\A\),
所以\begin{equation*}
	\rank\begin{bmatrix}
		\A & \vb0 \\
		\vb0 & \A^T \A
	\end{bmatrix}
	= \rank\begin{bmatrix}
		\A & \A \B \\
		\vb0 & \A^T
	\end{bmatrix}
	= \rank\begin{bmatrix}
		\A & \vb0 \\
		\B \A & \A^T
	\end{bmatrix}
	= 2 \rank\A.
\end{equation*}

取\(\A = \begin{bmatrix}
	0 & 1 \\
	0 & 0
\end{bmatrix},
\B = \begin{bmatrix}
	0 & 1 \\
	1 & 0
\end{bmatrix}\),
则\(\rank\A = 1\),
而\begin{equation*}
	\rank\begin{bmatrix}
		\A & \B \A \\
		\vb0 & \A \A^T
	\end{bmatrix}
	= \rank\begin{bmatrix}
		0 & 1 & 0 & 0 \\
		0 & 0 & 0 & 1 \\
		0 & 0 & 1 & 0 \\
		0 & 0 & 0 & 0
	\end{bmatrix}
	= 3 \neq 2,
\end{equation*}
所以\(\rank\begin{bmatrix}
	\A & \B \A \\
	\vb0 & \A \A^T
\end{bmatrix}
\neq 2 \rank\A\).
\end{proof}
\end{example}
\begin{example}
%@see: 《2023年全国硕士研究生入学统一考试(数学一)》一选择题/第5题
\def\M{\vb{M}}
设矩阵\(\A,\B,\C \in M_n(K)\)满足\(\A \B \C = \vb0\),
\(\E\)是数域\(K\)上的\(n\)阶单位矩阵.
试比较以下三个矩阵的秩的序关系:\begin{equation*}
	\M_1 \defeq \begin{bmatrix}
		\vb0 & \A \\
		\B \C & \E
	\end{bmatrix},
	\qquad
	\M_2 \defeq \begin{bmatrix}
		\A \B & \C \\
		\vb0 & \E
	\end{bmatrix},
	\qquad
	\M_3 \defeq \begin{bmatrix}
		\E & \A \B \\
		\A \B & \vb0
	\end{bmatrix}.
\end{equation*}
\begin{solution}
因为\begin{align*}
	\begin{bmatrix}
		\E & -\A \\
		\vb0 & \E
	\end{bmatrix}
	\begin{bmatrix}
		\vb0 & \A \\
		\B \C & \E
	\end{bmatrix}
	&= \begin{bmatrix}
		% 第一行第一列的元素等于\(\vb0 - \A \B \C = \vb0\)
		% 第一行第二列的元素等于\(\A - \A = \vb0\)
		\vb0 & \vb0 \\
		\B \C & \E
	\end{bmatrix}, \\
	\begin{bmatrix}
		\E & -\C \\
		\vb0 & \E
	\end{bmatrix}
	\begin{bmatrix}
		\A \B & \C \\
		\vb0 & \E
	\end{bmatrix}
	&= \begin{bmatrix}
		\A \B & \vb0 \\
		\vb0 & \E
	\end{bmatrix}, \\
	\begin{bmatrix}
		\E & \vb0 \\
		-\A \B & \E
	\end{bmatrix}
	\begin{bmatrix}
		\E & \A \B \\
		\A \B & \vb0
	\end{bmatrix}
	\begin{bmatrix}
		\E & -\A \B \\
		\vb0 & \E
	\end{bmatrix}
	&= \begin{bmatrix}
		\E & \vb0 \\
		\vb0 & -\A \B \A \B
	\end{bmatrix},
\end{align*}
而\begin{align*}
	\rank\M_1
	&= \rank\begin{bmatrix}
		\vb0 & \vb0 \\
		\B \C & \E
	\end{bmatrix}
	= n, \\
	\rank\M_2
	&= \rank\begin{bmatrix}
		\A \B & \vb0 \\
		\vb0 & \E
	\end{bmatrix}
	= \rank(\A\B) + n
	%\cref{theorem:线性方程组.矩阵的秩的性质2}
	\geq n, \\
	\rank\M_3
	&= \rank\begin{bmatrix}
		\E & \vb0 \\
		\vb0 & -\A \B \A \B
	\end{bmatrix}
	= \rank(\A \B \A \B) + n
	%\cref{theorem:线性方程组.矩阵的秩的性质2}
	\geq n,
\end{align*}
%\cref{theorem:线性方程组.矩阵乘积的秩}
这里\(\rank(\A \B) \geq \rank(\A \B \A \B)\),
所以\(\rank\M_2 \geq \rank\M_3 \geq \rank\M_1\).
\end{solution}
\end{example}

\section{线性方程组与向量空间的联系}
我们可以利用向量、矩阵表记线性方程组.

\begin{definition}
将\cref{equation:线性方程组.线性方程组的代数形式} 的系数按原位置构成的\(s \times n\)矩阵\[
	\vb{A} = \begin{bmatrix}
		a_{11} & a_{12} & \dots & a_{1n} \\
		a_{21} & a_{22} & \dots & a_{2n} \\
		\vdots & \vdots & & \vdots \\
		a_{n1} & a_{n2} & \dots & a_{nn}
	\end{bmatrix}
\]叫做\DefineConcept{系数矩阵}(coefficient matrix).

特别地,如果\(s = n\)(即系数矩阵\(\vb{A}\)是一个方阵),
则系数矩阵的行列式\(\abs{\vb{A}}\)叫做\DefineConcept{系数行列式}.
\end{definition}

为使表述简明,常用向量、矩阵表示线性方程组.
若记\[
	\vb{x}=\begin{bmatrix}
		x_1 \\ x_2 \\ \vdots \\ x_n
	\end{bmatrix},
	\quad
	\vb\beta=\begin{bmatrix}
		b_1 \\ b_2 \\ \vdots \\ b_n
	\end{bmatrix},
	\quad
	\vb\alpha_j=\begin{bmatrix}
		a_{1j} \\ a_{2j} \\ \vdots \\ a_{sj}
	\end{bmatrix},
	\quad
	j=1,2,\dotsc,n.
\]
\(\vb{A}\)的列分块阵为\(\vb{A} = (\vb\alpha_1,\vb\alpha_2,\dotsc,\vb\alpha_n)\),
则\cref{equation:线性方程组.线性方程组的代数形式} 有以下两种等价表示:
\begin{enumerate}
	\item {\rm\bf 矩阵形式}:
	\begin{equation}
		\vb{A} \vb{x} = \vb\beta.
	\end{equation}
	\item {\rm\bf 向量形式}:
	\begin{equation}
		x_1 \vb\alpha_1 + x_2 \vb\alpha_2 + \dotsb + x_n \vb\alpha_n = \vb\beta.
	\end{equation}
\end{enumerate}

\section{线性方程组有解的充分必要条件}
现在我们可以来回答直接根据线性方程组的系数和常数项判断方程组有没有解,有多少解的问题.

\begin{theorem}\label{theorem:向量空间.线性方程组有解判别定理}
%@see: 《高等代数(第三版 上册)》(丘维声) P87 定理1
线性方程组\begin{equation*}
	x_1 \vb\alpha_1 + x_2 \vb\alpha_2 + \dotsb + x_n \vb\alpha_n = \vb\beta
\end{equation*}有解的充分必要条件是:
它的系数矩阵与增广矩阵有相同的秩,
即\begin{equation*}
	\rank(\AutoTuple{\vb\alpha}{n})
	= \rank(\AutoTuple{\vb\alpha}{n},\vb\beta).
\end{equation*}
\begin{proof}
记系数矩阵\(\vb{A}=(\AutoTuple{\vb\alpha}{n})\),
增广矩阵\(\widetilde{\vb{A}}=(\vb{A},\vb\beta)\),
那么
\begin{align*}
	&\hspace{-20pt}
	\text{线性方程组\(x_1 \vb\alpha_1 + x_2 \vb\alpha_2 + \dotsb + x_n \vb\alpha_n = \vb\beta\)有解} \\
	&\iff \vb\beta\in\opair{\AutoTuple{\vb\alpha}{n}} \\
	&\iff \opair{\AutoTuple{\vb\alpha}{n},\vb\beta}\subseteq\opair{\AutoTuple{\vb\alpha}{n}} \\
	&\iff \opair{\AutoTuple{\vb\alpha}{n},\vb\beta}=\opair{\AutoTuple{\vb\alpha}{n}} \\
	&\iff \dim\opair{\AutoTuple{\vb\alpha}{n},\vb\beta}=\dim\opair{\AutoTuple{\vb\alpha}{n}} \\
	&\iff \rank\vb{A}=\rank\widetilde{\vb{A}}.
	\qedhere
\end{align*}
\end{proof}
\end{theorem}

从\cref{theorem:向量空间.线性方程组有解判别定理} 看出,
判断线性方程组有没有解,只要去比较它的系数矩阵与增广矩阵的秩是否相等.
这种判别方法有几种优越之处:
首先,求矩阵的秩有多种方法,不一定要把系数矩阵和增广矩阵化成阶梯形矩阵.
其次,有时不用求出系数矩阵的秩和增广矩阵的秩,也能比较它们的秩是否相等.
由于系数矩阵\(\vb{A}\)是增广矩阵\(\widetilde{\vb{A}}\)的子矩阵,总有\(\rank\vb{A}\leq\rank\widetilde{\vb{A}}\),
那么只要能够证明\(\rank\widetilde{\vb{A}}\leq\rank\vb{A}\),就能得出\(\rank\vb{A}=\rank\widetilde{\vb{A}}\).

现在我们想知道,当线性方程组\(x_1 \vb\alpha_1 + x_2 \vb\alpha_2 + \dotsb + x_n \vb\alpha_n = \vb\beta\)有解时,
能不能用系数矩阵的秩去判别它有唯一解,还是有无穷多个解?

\begin{theorem}\label{theorem:向量空间.有解的非齐次线性方程组的解的个数定理}
%@see: 《高等代数(第三版 上册)》(丘维声) P88 定理2
设线性方程组\(x_1 \vb\alpha_1 + x_2 \vb\alpha_2 + \dotsb + x_n \vb\alpha_n = \vb\beta\)有解,
\(\vb{A}=(\AutoTuple{\vb\alpha}{n})\)是它的系数矩阵.
\begin{itemize}
	\item 如果\(\rank\vb{A}=n\),则这个方程组有唯一解.
	\item 如果\(\rank\vb{A}<n\),则这个方程组有无穷多个解.
\end{itemize}
%TODO proof
\end{theorem}

把\cref{theorem:向量空间.有解的非齐次线性方程组的解的个数定理}
应用到齐次线性方程组上,便得出以下两个推论.

\begin{corollary}\label{theorem:线性方程组.齐次线性方程组只有零解的充分必要条件}
齐次线性方程组只有零解的充分必要条件是:它的系数矩阵的秩等于未知量的数目.
%TODO proof
%\cref{theorem:线性方程组.方程个数少于未知量个数的齐次线性方程组必有非零解}
\end{corollary}

\begin{corollary}\label{theorem:线性方程组.齐次线性方程组有非零解的充分必要条件}
%@see: 《高等代数(第三版 上册)》(丘维声) P88 推论3
%@see: 《线性代数》(张慎语、周厚隆) P80 定理8
齐次线性方程组有非零解的充分必要条件是:它的系数矩阵的秩小于未知量的数目.
%TODO proof
%\cref{theorem:线性方程组.方程个数少于未知量个数的齐次线性方程组必有非零解}
\end{corollary}

\begin{table}[hbt]
	\centering
	\begin{tblr}{*5c}
		% 系数矩阵\(\vb{A}\)的秩\(r\)与它的行数\(m\)和列数\(n\)的关系
		\hline
		& \(r = m = n\) & \(r = n < m\) & \(r = m < n\) & \(r < m, r < n\) \\ \hline
		\(\vb{A}\)的行最简形
			& \(\vb{E}\)
			& \(\begin{bmatrix} \vb{E} \\ \vb0 \end{bmatrix}\)
			& \(\begin{bmatrix} \vb{E} & \vb{F} \end{bmatrix} \vb{P}\)
			& \(\begin{bmatrix} \vb{E} & \vb{F} \\ \vb0 & \vb0 \end{bmatrix} \vb{P}\) \\
		解的情况 & 有且仅有1个解 & 有0个或1个解 & 有无穷多解 & 有0个或无穷多解 \\
		\hline
	\end{tblr}
	\caption{非齐次线性方程组$\vb{A} \vb{x} = \vb\beta$的解的情况($\vb{A} \in M_{m \times n}(K)$)}
\end{table}

\begin{example}
%@see: 《1998年全国硕士研究生入学统一考试(数学一)》二选择题/第4题
%@see: 《2020年全国硕士研究生入学统一考试(数学一)》一选择题/第6题
设欧氏空间\(\mathbb{R}^3\)中有两条直线\begin{gather*}
	l_1: \frac{x - x_1}{a_1}
		= \frac{y - y_1}{b_1}
		= \frac{z - z_1}{c_1}, \\
	l_2: \frac{x - x_2}{a_2}
		= \frac{y - y_2}{b_2}
		= \frac{z - z_2}{c_2}.
\end{gather*}
记\begin{equation*}
	\vb\alpha_1 = \begin{bmatrix}
		a_1 \\ b_1 \\ c_1
	\end{bmatrix},
	\qquad
	\vb\alpha_2 = \begin{bmatrix}
		a_2 \\ b_2 \\ c_2
	\end{bmatrix},
	\qquad
	\vb\beta = \begin{bmatrix}
		x_2 - x_1 \\
		y_2 - y_1 \\
		z_2 - z_1
	\end{bmatrix},
\end{equation*}
则这两条直线的位置关系与线性方程\((\vb\alpha_1,\vb\alpha_2) \vb{x} = \vb\beta\)的解的情况
有如下对应关系:\begin{center}
	\begin{tblr}{*3{c|}c}
		\hline
		\(\rank(\vb\alpha_1,\vb\alpha_2)\)
		& \(\rank(\vb\alpha_1,\vb\alpha_2,\vb\beta)\)
		& 解的情况 & 位置关系 \\
		\hline
		1 & 1 & 有无穷多解 & 重合 \\
		1 & 2 & 无解 & 平行 \\
		2 & 2 & 有唯一解 & 相交 \\
		2 & 3 & 无解 & 异面 \\
		\hline
	\end{tblr}
\end{center}
\end{example}

\begin{example}
%@see: 《2024年全国硕士研究生入学统一考试(数学一)》一选择题/第5题
设欧氏空间\(\mathbb{R}^3\)中有三个平面\begin{gather*}
	\Pi_1: A_1 x + B_1 y + C_1 z + D_1 = 0, \\
	\Pi_2: A_2 x + B_2 y + C_2 z + D_2 = 0, \\
	\Pi_3: A_3 x + B_3 y + C_3 z + D_3 = 0.
\end{gather*}
记\begin{equation*}
	\vb{A} = \begin{bmatrix}
		A_1 & B_1 & C_1 \\
		A_2 & B_2 & C_2 \\
		A_3 & B_3 & C_3
	\end{bmatrix},
	\qquad
	\vb\beta = \begin{bmatrix}
		-D_1 \\
		-D_2 \\
		-D_3
	\end{bmatrix},
	\qquad
	\widetilde{\vb{A}} = (\vb{A},\vb\beta),
\end{equation*}
则这三个平面的位置关系与线性方程\(\vb{A}\vb{x}=\vb\beta\)的解的情况
有如下对应关系:\begin{center}
	\begin{tblr}{*3{c|}l}
		\hline
		\(\rank\vb{A}\) & \(\rank\widetilde{\vb{A}}\) & 解的情况 & \SetCell{c} 位置关系 \\
		\hline
		1 & 1 & 有无穷多解 & 三平面重合 \\
		1 & 2 & 无解 & 至少有一个平面不与其他平面重合 \\
		2 & 2 & 有无穷多解 & 三平面相交于一条直线 \\
		2 & 3 & 无解 & 两平面平行,第三个平面与两者相交; \\
				   &&& 三平面两两相交,三条交线相互平行 \\
		3 & 3 & 有唯一解 & 三平面相交于一点 \\
		\hline
	\end{tblr}
\end{center}
\end{example}

\section{齐次线性方程组的解集的结构}
\subsection{解空间的概念}
\begin{proposition}\label{theorem:线性方程组.齐次线性方程组的解的线性组合也是解}
%@see: 《高等代数(第三版 上册)》(丘维声) P90 性质1
%@see: 《高等代数(第三版 上册)》(丘维声) P90 性质2
%@see: 《线性代数》(张慎语、周厚隆) P80 性质1
%@see: 《线性代数》(张慎语、周厚隆) P81 推论
齐次线性方程组\(\vb{A}\vb{x}=\vb0\)的解的任意线性组合也是解.
\begin{proof}
设\(\vb{x}_1\)与\(\vb{x}_2\)是齐次线性方程组\(\vb{A}\vb{x}=\vb0\)的任意两个解,
即\begin{equation*}
	\vb{A}\vb{x}_1=\vb0, \qquad
	\vb{A}\vb{x}_2=\vb0.
\end{equation*}
又设\(k\)是任意常数,那么有\begin{equation*}
	\vb{A} (\vb{x}_1 + \vb{x}_2) = \vb{A} \vb{x}_1 + \vb{A} \vb{x}_2 = \vb0 + \vb0 = \vb0,
\end{equation*}\begin{equation*}
	\vb{A} (k \vb{x}_1) = k (\vb{A} \vb{x}_1) = k \vb0 = \vb0,
\end{equation*}
所以\(\vb{x}_1 + \vb{x}_2\)与\(k \vb{x}_1\)都是\(\vb{A} \vb{x} = \vb0\)的解.
\end{proof}
\end{proposition}

\cref{theorem:线性方程组.齐次线性方程组的解的线性组合也是解} 表明,
\(n\)元齐次线性方程组\(x_1\vb\alpha_1+x_2\vb\alpha_2+\dotsb+x_n\vb\alpha_n=\vb0\)的解集\begin{equation*}
	W = \Set{
		\vb{x} \in K^n
		\given
		\vb{A}\vb{x}=\vb0
	}
\end{equation*}是\(K^n\)的一个子空间.
我们把它称为“方程组\(\vb{A}\vb{x}=\vb0\)的\DefineConcept{解空间}(space of solution)”
或者“矩阵\(\vb{A}\)的\DefineConcept{核空间}(kernel)\footnote{
	有的地方会把核空间称为\DefineConcept{零空间}(null space),
	%@see: https://mathworld.wolfram.com/NullSpace.html
	还特别把核空间的维数称为\DefineConcept{零度}(nullity),
	但是由于这个称谓容易与另一个同样叫做“零空间”
	但涵义是只含零空间的线性空间\(\{\vb0\}\)混淆,
	因此我们不采用这种命名方式.
}”,
并记为\(\Ker\vb{A}\).
如果这个方程组只有零解,那么\(W\)是零子空间.
如果这个方程组有非零解,那么\(W\)是非零子空间,从而\(W\)有基.
%@see: 《高等代数(大学高等代数课程创新教材 第二版 上册)》(丘维声) P120 定义1
%@see: 《线性代数》(张慎语、周厚隆) P83
我们把解空间\(W\)的一个基称为这个方程组的一个\DefineConcept{基础解系}(basic set of solutions).

如果我们找到了齐次线性方程组的一个基础解系\(\{\AutoTuple{\vb{x}}{t}\}\),
\def\tongjie{k_1\vb{x}_1+k_2\vb{x}_2+\dotsb+k_t\vb{x}_t}%
那么这个方程组的解集为\begin{align*}
	W &= \Ker\vb{A}
	= \Span\{\AutoTuple{\vb{x}}{t}\} \\
	&= \Set{ \tongjie \given \AutoTuple{k}{t} \in K }.
\end{align*}
我们把表达式\((\tongjie)\)称为这个方程组的\DefineConcept{通解}(general solution).

\subsection{解空间的维数}
如何找出齐次线性方程组的一个基础解系?
解空间\(W\)的维数是多少?

\begin{theorem}\label{theorem:线性方程组.齐次线性方程组的解向量个数}
%@see: 《高等代数(第三版 上册)》(丘维声) P91 定理1
%@see: 《高等代数(大学高等代数课程创新教材 第二版 上册)》(丘维声) P120 定理1
%@see: 《线性代数》(张慎语、周厚隆) P82 定理9
数域\(K\)上\(n\)元齐次线性方程组\(\vb{A}\vb{x}=\vb0\)的解空间的维数与系数矩阵的秩
满足\begin{equation}
	\rank\vb{A} + \dim\Ker\vb{A} = n.
\end{equation}
当方程组有非零解时,它的每一个基础解系所含的解向量的数目都等于\(\dim W\).
\begin{proof}
首先假设\(\vb{A}\)是可逆矩阵,
那么\(\rank\vb{A} = n\),
方程\(\vb{A} \vb{x} = \vb0\)只有零解,
矩阵\(\vb{A}\)的核空间是零子空间,
从而\(\dim\Ker\vb{A} = 0\),
因此\begin{equation*}
	\rank\vb{A} + \dim\Ker\vb{A}
	= n + 0
	= n.
\end{equation*}

再假设\(\vb{A}\)不是可逆矩阵,
让\(\vb{A}\)经一系列初等行变换化为行约化矩阵\(\vb{B}\),
那么\(\vb{A}\)与\(\vb{B}\)等价,
关于\(\vb{x}\)的齐次线性方程组\(\vb{A} \vb{x} = \vb0\)与\(\vb{B} \vb{x} = \vb0\)同解,
且\(\rank\vb{A} = \rank\vb{B} = r < n\).
显然存在置换矩阵\(\vb{P}\),
使得\begin{equation*}
	\vb{B} \vb{P}
	= \begin{bmatrix}
		\vb{E}_r & \vb{F} \\
		\vb0 & \vb0
	\end{bmatrix},
\end{equation*}
其中\(\vb{E}_r\)是\(r\)阶单位矩阵,
\(\vb{F} \in M_{r\times(n-r)}(K)\).
令\(\vb{x} = \vb{P} \vb{y}\),
则\(\vb{y}_0\)是\((\vb{B} \vb{P}) \vb{y} = \vb0\)的一个解,
当且仅当\(\vb{x}_0 \defeq \vb{P} \vb{y}_0\)是\(\vb{B} \vb{x} = \vb0\)的一个解.
注意到\begin{equation*}
	\vb{B} \vb{P}
	\begin{bmatrix}
		-\vb{F} \\
		\vb{E}_{n-r}
	\end{bmatrix}
	= \begin{bmatrix}
		\vb{E}_r & \vb{F} \\
		\vb0 & \vb0
	\end{bmatrix}
	\begin{bmatrix}
		-\vb{F} \\
		\vb{E}_{n-r}
	\end{bmatrix}
	= \begin{bmatrix}
		\vb{E}_r (-\vb{F}) + \vb{F} \vb{E}_{n-r} \\
		\vb0
	\end{bmatrix}
	= \vb0,
\end{equation*}
其中\(\vb{E}_{n-r}\)是\(n-r\)阶单位矩阵,
因此矩阵\(
	\begin{bmatrix}
		-\vb{F} \\
		\vb{E}_{n-r}
	\end{bmatrix}
\)的每一个列向量都是关于\(\vb{y}\)的齐次线性方程组\((\vb{B} \vb{P}) \vb{y} = \vb0\)的一个解,
换句话说矩阵\(
	\vb{X}
	\defeq
	\vb{P}
	\begin{bmatrix}
		-\vb{F} \\
		\vb{E}_{n-r}
	\end{bmatrix}
\)的每一个列向量都是关于\(\vb{x}\)的齐次线性方程组\(\vb{B} \vb{x} = \vb0\)或\(\vb{A} \vb{x} = \vb0\)的一个解.
由于置换矩阵\(\vb{P}\)是可逆矩阵,
所以\begin{equation*}
	\rank\vb{X}
	= \rank\begin{bmatrix}
		-\vb{F} \\
		\vb{E}_{n-r}
	\end{bmatrix}
	= \rank\vb{E}_{n-r}
	= n-r,
\end{equation*}
从而\(\vb{X}\)的列秩等于\(n-r\),
% 根据\(\RankC\vb{A}=\dim(\SpanC\vb{A})\)
于是\(\vb{X}\)列空间的维数是\(\dim(\SpanC\vb{X}) = n-r\).
假设\(\vb{y}_0 = (\AutoTuple{y}{n})^T\)是\((\vb{B} \vb{P}) \vb{y} = \vb0\)的任意一个解,
那么\begin{equation*}
	\vb{y}_0
	- \begin{bmatrix}
		-\vb{F} \\
		\vb{E}_{n-r}
	\end{bmatrix}
	(y_{r+1},\dotsc,y_n)^T
	= (z_1,\dotsc,z_r,0,\dotsc,0)^T
\end{equation*}
也是\((\vb{B} \vb{P}) \vb{y} = \vb0\)的一个解,
即\begin{equation*}
	\vb{B} \vb{P} (z_1,\dotsc,z_r,0,\dotsc,0)^T
	= z_1 \vb\epsilon_1 + \dotsb + z_r \vb\epsilon_r
	= \vb0,
\end{equation*}
其中\(\AutoTuple{\vb\epsilon}{r}\)是\(\vb{E}_r\)的列向量组,
于是\(z_1 = \dotsb = z_r = 0\),
即\begin{equation*}
	\vb{y}_0
	= \begin{bmatrix}
		-\vb{F} \\
		\vb{E}_{n-r}
	\end{bmatrix}
	(y_{r+1},\dotsc,y_n)^T,
\end{equation*}
这就说明\((\vb{B} \vb{P}) \vb{y} = \vb0\)的任意一个解均可由\(
	\begin{bmatrix}
		-\vb{F} \\
		\vb{E}_{n-r}
	\end{bmatrix}
\)的列向量组线性表出,
继而说明\(\vb{A} \vb{x} = \vb0\)的任意一个解均可由\(\vb{X}\)的列向量组线性表出,
\(\vb{A}\)的核空间就是\(\vb{X}\)的的列空间,
因此\(\dim\Ker\vb{A} =\allowbreak n-r\).
综上所述\begin{equation*}
	\rank\vb{A} + \dim\Ker\vb{A}
	= r + (n-r)
	= n.
	\qedhere
\end{equation*}
\end{proof}
\end{theorem}

\cref{theorem:线性方程组.齐次线性方程组的解向量个数} 的证明过程给出了
求解齐次线性方程组\(\vb{A}\vb{x}=\vb0\)的基础解系的方法:
\begin{algorithm}[求解齐次线性方程组]
\hfill
\begin{enumerate}
	\item 把齐次线性方程组的系数矩阵\(\vb{A}\)经过初等行变换化简成行约化矩阵\(\vb{J}\);

	\item 从\(\vb{J}\)中,找出非零首元对应的列\(\vb\beta_{i_1},\dotsc,\vb\beta_{i_r}\),
	将这些列称为\DefineConcept{主列}(pivot column);

	\item 从\(\vb{J}\)中,找出没有非零首元的列\(\vb\beta_{j_1},\dotsc,\vb\beta_{j_{n-r}}\),
	将这些列称为\DefineConcept{自由列}(free column);

	\item 各个自由列的负向量\(-\vb\beta_{j_1},\dotsc,-\vb\beta_{j_{n-r}}\)
	就是齐次线性方程组\(\vb{A}\vb{x}=\vb0\)的一个基础解系.
\end{enumerate}
\end{algorithm}

\begin{corollary}
%@see: 《线性代数》(张慎语、周厚隆) P83 推论
%@see: 《高等代数(第三版 上册)》(丘维声) P95 习题3.7 3.
设齐次线性方程组\(\vb{A}\vb{x}=\vb0\)的系数矩阵\(\vb{A}\)是\(s \times n\)矩阵.
若\(\rank\vb{A} = r < n\),
则\begin{itemize}
	\item \(\vb{A}\vb{x}=\vb0\)的每个基础解系都含有\(n-r\)个解向量;
	\item \(\vb{A}\vb{x}=\vb0\)的任意\(n-r+1\)个解向量线性相关;
	\item \(\vb{A}\vb{x}=\vb0\)的任意\(n-r\)个线性无关的解都是它的一个基础解系.
\end{itemize}
\end{corollary}

\begin{example}
%@see: 《线性代数》(张慎语、周厚隆) P84 例2
求齐次线性方程组\begin{equation*}
	\left\{ \begin{array}{*{11}{r}}
		x_1 &-& 2 x_2 &-& x_3 &+& 2 x_4 &+& 4 x_5 &=& 0 \\
		2 x_1 &-& 2 x_2 &-& 3 x_3 && &+& 2 x_5 &=& 0 \\
		4 x_1 &-& 2 x_2 &-& 7 x_3 &-& 4 x_4 &-& 2 x_5 &=& 0
	\end{array} \right.
\end{equation*}的通解.
\begin{solution}
写出系数矩阵\(\vb{A}\),并作初等行变换化简
\begin{align*}
	\vb{A} &= \begin{bmatrix}
		1 & -2 & -1 & 2 & 4 \\
		2 & -2 & -3 & 0 & 2 \\
		4 & -2 & -7 & -4 & -2
	\end{bmatrix} \\
	&\xlongrightarrow{\begin{array}{c}
		-2\times\text{(1行)}+\text{(2行)} \\
		-4\times\text{(1行)}+\text{(3行)}
	\end{array}}
	\begin{bmatrix}
		1 & -2 & -1 & 2 & 4 \\
		0 & 2 & -1 & -4 & -6 \\
		0 & 6 & -3 & -12 & -18
	\end{bmatrix} \\
	&\xlongrightarrow{\begin{array}{c}
		-3\times\text{(2行)}+\text{(3行)} \\
		1\times\text{(2行)}+\text{(1行)}
	\end{array}}
	\begin{bmatrix}
		1 & 0 & -2 & -2 & -2 \\
		0 & 2 & -1 & -4 & -6 \\
		0 & 0 & 0 & 0 & 0
	\end{bmatrix}
	= \vb{B},
\end{align*}
因为\(\rank\vb{A}=\rank\vb{B}=2\),所以基础解系含\(5-2=3\)个向量.
分别将\(x_3,x_4,x_5\)的3组值\((2,0,0),(0,1,0),(0,0,1)\)代入\(\vb{B}\vb{x}=\vb0\),
得基础解系:\begin{equation*}
	\vb{x}_1 = (4,1,2,0,0)^T, \quad
	\vb{x}_2 = (2,2,0,1,0)^T, \quad
	\vb{x}_3 = (2,3,0,0,1)^T.
\end{equation*}
原方程组的通解为\(k_1 \vb{x}_1 + k_2 \vb{x}_2 + k_3 \vb{x}_3\),其中\(k_1,k_2,k_3\)为任意常数.
\end{solution}
%@Mathematica: A = {{1, -2, -1, 2, 4}, {2, -2, -3, 0, 2}, {4, -2, -7, -4, -2}}
%@Mathematica: RowReduce[A]
%@Mathematica: MatrixRank[A]
%@Mathematica: NullSpace[A]
\end{example}

\subsection{线性方程有公共解的条件}
\begin{proposition}
%@see: 《线性代数》(张慎语、周厚隆) P85 习题4.5 6(2)
设\(\vb{A},\vb{B} \in M_{s \times n}(K)\),
则\begin{equation*}
	\text{$\vb{A}\vb{x}=\vb0$的解都是$\vb{B}\vb{x}=\vb0$的解}
	\implies
	\rank\vb{A} \geq \rank\vb{B}.
\end{equation*}
\begin{proof}
假设\(\vb{A}\vb{x}=\vb0\)的解都是\(\vb{B}\vb{x}=\vb0\)的解,
即\(\Ker\vb{A} \subseteq \Ker\vb{B} \subseteq K^n\).
由\cref{theorem:向量空间.两个非零子空间的关系1,theorem:线性方程组.齐次线性方程组的解向量个数} 可知\begin{equation*}
	%\rank\vb{A} + \dim\Ker\vb{A} = n.
	n - \rank\vb{A} = \dim\Ker\vb{A} \leq \dim\Ker\vb{B} = n - \rank\vb{B}.
\end{equation*}
即\(\rank\vb{B} \leq \rank\vb{A}\).
\end{proof}
\end{proposition}

%@see: [两个线性方程组的公共解与同解](https://zhuanlan.zhihu.com/p/665121966)
\begin{definition}
设\(\vb{A},\vb{B} \in M_{s \times n}(K)\),
\(\vb\beta_1,\vb\beta_2 \in K^s\).
\begin{itemize}
	\item 如果存在\(\vb{x}_0 \in K^n\),
	使得\begin{equation*}
		\vb{A} \vb{x}_0 = \vb\beta_1
		\quad\text{和}\quad
		\vb{B} \vb{x}_0 = \vb\beta_2
	\end{equation*}同时成立,
	则称“\(\vb{A} \vb{x} = \vb\beta_1\)与\(\vb{B} \vb{x} = \vb\beta_2\)有\DefineConcept{公共解}”
	或者“\(\vb{x}_0\)是\(\vb{A} \vb{x} = \vb\beta_1\)与\(\vb{B} \vb{x} = \vb\beta_2\)的一个\DefineConcept{公共解}”.

	\item 如果存在\(\vb{x}_0 \in K^n-\{\vb0\}\),
	使得\begin{equation*}
		\vb{A} \vb{x}_0 = \vb0
		\quad\text{和}\quad
		\vb{B} \vb{x}_0 = \vb0
	\end{equation*}同时成立,
	则称“\(\vb{A} \vb{x} = \vb0\)与\(\vb{B} \vb{x} = \vb0\)有\DefineConcept{非零公共解}”
	或者“\(\vb{x}_0\)是\(\vb{A} \vb{x} = \vb0\)与\(\vb{B} \vb{x} = \vb0\)的一个\DefineConcept{非零公共解}”.
\end{itemize}
\end{definition}

\begin{proposition}
设\(\vb{A},\vb{B} \in M_{s \times n}(K)\),
则\begin{align*}
	\text{\(\vb{A}\vb{x}=\vb0\)与\(\vb{B}\vb{x}=\vb0\)有非零公共解}
	&\iff
	\text{$\begin{bmatrix}
		\vb{A} \\ \vb{B}
	\end{bmatrix}
	\vb{x}
	= \vb0$有非零解} \\
	&\iff
	\rank\begin{bmatrix}
		\vb{A} \\ \vb{B}
	\end{bmatrix}
	< n.
\end{align*}
\begin{proof}
%@credit: {de3029b8-10a6-4ae5-8f64-108dae1c10a9}
首先,假设\(\vb{A}\vb{x}=\vb0\)与\(\vb{B}\vb{x}=\vb0\)有非零公共解\(\vb{x}_0\),
即成立\(\vb{A}\vb{x}_0=\vb0,
\vb{B}\vb{x}_0=\vb0\),
那么\begin{equation*}
	\begin{bmatrix}
		\vb{A} \\ \vb{B}
	\end{bmatrix}
	\vb{x}_0
	= \begin{bmatrix}
		\vb{A} \vb{x}_0 \\
		\vb{B} \vb{x}_0
	\end{bmatrix}
	= \begin{bmatrix}
		\vb0_{s\times1} \\
		\vb0_{s\times1}
	\end{bmatrix}
	= \vb0_{(2s)\times1}.
\end{equation*}

反过来,假设\(\begin{bmatrix}
	\vb{A} \\ \vb{B}
\end{bmatrix}
\vb{x}
= \vb0_{(2s)\times1}\)有非零解\(\vb{x}_0\),
即\(\begin{bmatrix}
	\vb{A} \\ \vb{B}
\end{bmatrix}
\vb{x}_0
= \vb0_{(2s)\times1}\),
于是\begin{equation*}
	\begin{bmatrix}
		\vb{A} \vb{x}_0 \\
		\vb{B} \vb{x}_0
	\end{bmatrix}
	= \begin{bmatrix}
		\vb0_{s\times1} \\
		\vb0_{s\times1}
	\end{bmatrix}.
	\qedhere
\end{equation*}
\end{proof}
\end{proposition}

\begin{corollary}
设\(\vb{A},\vb{B} \in M_{s \times n}(K)\),
\(\vb\beta_1,\vb\beta_2 \in K^s\),
则\begin{align*}
	\text{$\vb{A}\vb{x}=\vb\beta_1$与$\vb{B}\vb{x}=\vb\beta_2$有公共解}
	&\iff \text{$\begin{bmatrix}
		\vb{A} \\ \vb{B}
	\end{bmatrix}
	\vb{x}
	= \begin{bmatrix}
		\vb\beta_1 \\
		\vb\beta_2
	\end{bmatrix}$有解} \\
	&\iff
	\rank\begin{bmatrix}
		\vb{A} \\ \vb{B}
	\end{bmatrix}
	= \rank\begin{bmatrix}
		\vb{A} & \vb\beta_1 \\
		\vb{B} & \vb\beta_2
	\end{bmatrix}.
\end{align*}
\begin{proof}
%@credit: {de3029b8-10a6-4ae5-8f64-108dae1c10a9}
假设\(\vb{A}\vb{x}=\vb\beta_1\)与\(\vb{B}\vb{x}=\vb\beta_2\)有公共解\(\vb{x}_0\),
即成立\(\vb{A} \vb{x}_0 = \vb\beta_1,
\vb{B} \vb{x}_0 = \vb\beta_2\),
那么\begin{equation*}
	\begin{bmatrix}
		\vb{A} \\ \vb{B}
	\end{bmatrix}
	\vb{x}_0
	= \begin{bmatrix}
		\vb{A} \vb{x}_0 \\
		\vb{B} \vb{x}_0
	\end{bmatrix}
	= \begin{bmatrix}
		\vb\beta_1 \\
		\vb\beta_2
	\end{bmatrix}.
\end{equation*}

假设\(\begin{bmatrix}
	\vb{A} \\ \vb{B}
\end{bmatrix}
\vb{x}
= \begin{bmatrix}
	\vb\beta_1 \\
	\vb\beta_2
\end{bmatrix}\)有解\(\vb{x}_0\),
即\(\begin{bmatrix}
	\vb{A} \\ \vb{B}
\end{bmatrix}
\vb{x}_0
= \begin{bmatrix}
	\vb\beta_1 \\
	\vb\beta_2
\end{bmatrix}\),
于是\(\vb{A} \vb{x}_0 = \vb\beta_1,
\vb{A} \vb{x}_0 = \vb\beta_2\).

由\cref{theorem:向量空间.线性方程组有解判别定理} 可知,
\(\text{$\begin{bmatrix}
	\vb{A} \\ \vb{B}
\end{bmatrix}
\vb{x}
= \begin{bmatrix}
	\vb\beta_1 \\
	\vb\beta_2
\end{bmatrix}$有解}
\iff
\rank\begin{bmatrix}
	\vb{A} \\ \vb{B}
\end{bmatrix}
= \rank\begin{bmatrix}
	\vb{A} & \vb\beta_1 \\
	\vb{B} & \vb\beta_2
\end{bmatrix}\).
\end{proof}
\end{corollary}

\begin{example}
%@see: 《2023年全国硕士研究生入学统一考试(数学一)》一选择题/第7题
已知向量\(\vb\gamma\)既可由\begin{equation*}
	\vb\alpha_1 = (1,2,3)^T,
	\qquad
	\vb\alpha_2 = (2,1,1)^T
\end{equation*}线性表出,
也可由\begin{equation*}
	\vb\beta_1 = (2,5,9)^T,
	\qquad
	\vb\beta_2 = (1,0,1)^T
\end{equation*}线性表出,
求\(\vb\gamma\).
\begin{solution}\let\qed\relax
\begin{proof}[解法一]
由题意有\begin{equation*}
	\vb\gamma = k_1 \vb\alpha_1 + k_2 \vb\alpha_2
	= k_3 \vb\beta_1 + k_4 \vb\beta_2,
\end{equation*}
其中\(\AutoTuple{k}{4}\)是常数.
建立方程\begin{equation*}
	k_1 \vb\alpha_1 + k_2 \vb\alpha_2 + k_3 (-\vb\beta_1) + k_4 (-\vb\beta_2) = 0,
\end{equation*}
写出系数矩阵得\begin{equation*}
	\vb{A} = (\vb\alpha_1,\vb\alpha_2,-\vb\beta_1,-\vb\beta_2)
	= \begin{bmatrix}
		1 & 2 & -2 & -1 \\
		2 & 1 & -5 & 0 \\
		3 & 1 & -9 & -1
	\end{bmatrix}
	\to \begin{bmatrix}
		1 & 0 & 0 & 3 \\
		0 & 1 & 0 & 1 \\
		0 & 0 & 1 & 1
	\end{bmatrix},
\end{equation*}
解得\((k_1,k_2,k_3,k_4)^T = k (3,-1,1,-3)^T\ (\text{$k$是常数})\),
那么\begin{equation*}
	\vb\gamma
	= k_1 \vb\alpha_1 + k_2 \vb\alpha_2
	= k \left( 3 \vb\alpha_1 - \vb\alpha_2 \right)
	= k (1,5,8)^T
	\quad(\text{$k$是常数}).
\end{equation*}
\end{proof}
\begin{proof}[解法二]
在空间解析几何视角下,
向量\(\vb\gamma\)在向量\(\vb\alpha_1,\vb\alpha_2\)张成的平面上,
也在向量\(\vb\beta_1,\vb\beta_2\)张成的平面上,
也就是说向量\(\vb\gamma\)是这两个平面的交线的一个方向向量,
或者说\(\vb\gamma\)同时垂直于这两个平面的法向量,
于是\(\vb\gamma = k(
	\VectorOuterProduct
	{(\VectorOuterProduct{\vb\alpha_1}{\vb\alpha_2})}
	{(\VectorOuterProduct{\vb\beta_1}{\vb\beta_2})}
)
\ (\text{$k$是任意常数})\).
这里\begin{gather*}
	\VectorOuterProduct{\vb\alpha_1}{\vb\alpha_2}
	= \begin{vmatrix}
		\vb{i} & \vb{j} & \vb{k} \\
		1 & 2 & 3 \\
		2 & 1 & 1
	\end{vmatrix}
	= \begin{bmatrix}
		-1 \\
		5 \\
		-3
	\end{bmatrix}, \\
	\VectorOuterProduct{\vb\beta_1}{\vb\beta_2}
	= \begin{vmatrix}
		\vb{i} & \vb{j} & \vb{k} \\
		2 & 5 & 9 \\
		1 & 0 & 1
	\end{vmatrix}
	= \begin{bmatrix}
		5 \\
		7 \\
		-5
	\end{bmatrix},
\end{gather*}
最后得到\begin{equation*}
	\vb\gamma
	= k_1 \begin{vmatrix}
		\vb{i} & \vb{j} & \vb{k} \\
		-1 & 5 & -3 \\
		5 & 7 & -5
	\end{vmatrix}
	= k_1 \begin{bmatrix}
		-4 \\
		-20 \\
		-32
	\end{bmatrix}
	= k \begin{bmatrix}
		1 \\ 5 \\ 8
	\end{bmatrix}.
\end{equation*}
\end{proof}
\end{solution}
\end{example}

\subsection{线性方程同解的条件}
\begin{proposition}\label{theorem:齐次线性方程组的解集的结构.两个方程同解的充分必要条件1}
%@credit: {b8a6b30d-44bc-4d7a-a6b5-574e615c5be0}
设\(\vb{A} \in M_{s \times n}(K), \vb{B} \in M_{t \times n}(K)\),
则\(\vb{A} \vb{x} = \vb0\)的解都是\(\vb{B} \vb{x} = \vb0\)的解,
当且仅当\(\vb{A} \vb{x} = \vb0\)与\(
	\begin{bmatrix}
		\vb{A} \\ \vb{B}
	\end{bmatrix}
	\vb{x}
	= \vb0
\)同解.
\begin{proof}
假设\(\vb{A} \vb{x} = \vb0\)的解都是\(\vb{B} \vb{x} = \vb0\)的解,
即\(
	\Set{
		\vb{x}
		\given
		\vb{A} \vb{x} = \vb0
	}
	\subseteq
	\Set{
		\vb{x}
		\given
		\vb{B} \vb{x} = \vb0
	}
\),
那么由\cref{equation:集合论.集合代数公式7-3} 可知\begin{equation*}
	\Set{
		\vb{x}
		\given
		\vb{A} \vb{x} = \vb0
	}
	=
	\Set{
		\vb{x}
		\given
		\vb{A} \vb{x} = \vb0
	}
	\cap
	\Set{
		\vb{x}
		\given
		\vb{B} \vb{x} = \vb0
	}
	% 交集的定义
	= \Set{
		\vb{x}
		\given
		\vb{A} \vb{x} = \vb0,
		\vb{B} \vb{x} = \vb0
	},
\end{equation*}
这就说明\(\vb{A} \vb{x} = \vb0\)与\(
	\begin{bmatrix}
		\vb{A} \\ \vb{B}
	\end{bmatrix}
	\vb{x}
	= \vb0
\)同解.
\end{proof}
\end{proposition}

\begin{proposition}
设\(\vb{A},\vb{B} \in M_{s \times n}(K)\),
则\begin{align*}
	\text{$\vb{A}\vb{x}=\vb0$与$\vb{B}\vb{x}=\vb0$同解}
	&\iff
	\text{$\vb{A}\vb{x}=\vb0$、
	$\vb{B}\vb{x}=\vb0$
	与$\begin{bmatrix}
		\vb{A} \\ \vb{B}
	\end{bmatrix}
	\vb{x}
	= \vb0$同解} \\
	&\iff
	\rank\vb{A}
	= \rank\vb{B}
	= \rank\begin{bmatrix}
		\vb{A} \\ \vb{B}
	\end{bmatrix} \\
	&\iff
	\text{$\vb{A}$的行向量组与$\vb{B}$的行向量组等价}.
\end{align*}
%TODO proof
\end{proposition}

\begin{proposition}
设\(\vb{A},\vb{B} \in M_{s \times n}(K)\),
\(\vb\beta_1,\vb\beta_2 \in K^s\),
方程\(\vb{A}\vb{x}=\vb\beta_1\)和\(\vb{B}\vb{x}=\vb\beta_2\)都有解,
则\begin{align*}
	&\text{$\vb{A}\vb{x}=\vb\beta_1$与$\vb{B}\vb{x}=\vb\beta_2$同解} \\
	&\iff
	\text{$\vb{A}\vb{x}=\vb0$与$\vb{B}\vb{x}=\vb0$同解,
	且$\vb{A}\vb{x}=\vb\beta_1$与$\vb{B}\vb{x}=\vb\beta_2$有公共解} \\
	&\iff
	\rank\vb{A} = \rank\vb{B}
	= \rank\begin{bmatrix}
		\vb{A} \\ \vb{B}
	\end{bmatrix}
	= \rank\begin{bmatrix}
		\vb{A} & \vb\beta_1 \\
		\vb{B} & \vb\beta_2
	\end{bmatrix} \\
	&\iff
	\text{$(\vb{A},\vb\beta_1)$的行向量组与$(\vb{B},\vb\beta_2)$的行向量组等价}.
\end{align*}
%TODO proof
%\cref{example:向量空间.等秩矩阵的行向量组的等价性}
\end{proposition}

\begin{proposition}\label{theorem:线性方程组.同解方程组的系数矩阵的秩相同}
设\(\vb{A},\vb{B} \in M_{s \times n}(K)\),
则“\(\vb{A}\vb{x}=\vb0\)与\(\vb{B}\vb{x}=\vb0\)同解”
是“\(\rank\vb{A}=\rank\vb{B}\)”的充分不必要条件.
\begin{proof}
假设\(\vb{A}\vb{x}=\vb0\)与\(\vb{B}\vb{x}=\vb0\)同解,
那么方程\(\vb{A}\vb{x}=\vb0\)的解空间与\(\vb{B}\vb{x}=\vb0\)的解空间相同,
那么由\cref{theorem:线性方程组.齐次线性方程组的解向量个数}
有\(n-\rank\vb{A}=n-\rank\vb{B}\),\(\rank\vb{A}=\rank\vb{B}\).

反过来,
取\(\vb{A} = (1,0),
\vb{B} = (0,1)\).
显然\(\rank\vb{A} = \rank\vb{B} = 1\).
但是线性方程组\(\vb{A}\vb{x}=\vb0\)的解是\(k_1(0,1)^T\ (\text{$k_1$是常数})\),
而\(\vb{B}\vb{x}=\vb0\)的解是\(k_2(1,0)^T\ (\text{$k_2$是常数})\).
\end{proof}
\end{proposition}

\begin{example}
%@see: 《2022年全国硕士研究生入学统一考试(数学一)》一选择题/第6题/选项(D)
设矩阵\(\vb{A},\vb{B} \in M_n(K)\).
举例说明:即便方程\(\vb{A} \vb{x} = \vb0\)与\(\vb{B} \vb{x} = \vb0\)同解,
但是方程\(\vb{A} \vb{B} \vb{x} = \vb0\)与\(\vb{B} \vb{A} \vb{x} = \vb0\)不同解.
\begin{solution}
取\(\vb{A} = \begin{bmatrix}
	0 & 1 \\
	0 & 0
\end{bmatrix},
\vb{B} = \begin{bmatrix}
	0 & 1 \\
	0 & 1
\end{bmatrix}\),
则\begin{equation*}
	\vb{A} \vb{B} = \begin{bmatrix}
		0 & 1 \\
		0 & 0
	\end{bmatrix},
	\qquad
	\vb{B} \vb{A} = \begin{bmatrix}
		0 & 0 \\
		0 & 0
	\end{bmatrix},
\end{equation*}
既然\(\rank(\vb{A} \vb{B}) = 1 \neq \rank(\vb{B} \vb{A}) = 0\),
所以方程\(\vb{A} \vb{B} \vb{x} = \vb0\)与\(\vb{B} \vb{A} \vb{x} = \vb0\)不同解.
\end{solution}
\end{example}

\begin{proposition}\label{theorem:线性方程组.同解方程组.特例1}
设\(\vb{A} \in M_{s \times n}(K),
\vb{B} \in M_{n \times m}(K)\),
则\begin{equation*}
	\text{$(\vb{A}\vb{B})\vb{x}=\vb0$与$\vb{B}\vb{x}=\vb0$同解}
	\iff
	\rank(\vb{A}\vb{B})=\rank\vb{B}.
\end{equation*}
\begin{proof}
充分性.
由\cref{theorem:线性方程组.同解方程组的系数矩阵的秩相同} 可知\begin{equation*}
	\text{$(\vb{A}\vb{B})\vb{x}=\vb0$与$\vb{B}\vb{x}=\vb0$同解}
	\implies
	\rank(\vb{A}\vb{B})=\rank\vb{B}.
\end{equation*}

必要性.
设\(\vb\xi\)是\(\vb{B}\vb{x}=\vb0\)的一个解,
即\(\vb{B}\vb\xi=\vb0\),
那么左乘\(\vb{A}\)便得\(\vb{A}\vb{B}\vb\xi=\vb0\),
这就说明\(\vb\xi\)也是\((\vb{A}\vb{B})\vb{x}=\vb0\)的一个解.
由于\(\vb\xi\)的任意性,
所以\(\vb{B}\vb{x}=\vb0\)的解都是\((\vb{A}\vb{B})\vb{x}=\vb0\)的解,
也就是说\begin{equation*}
	\Ker\vb{B}
	\subseteq
	\Ker(\vb{A}\vb{B}).
\end{equation*}
因为\(\rank(\vb{A}\vb{B})=\rank\vb{B}\),
所以由\cref{theorem:线性方程组.齐次线性方程组的解向量个数} 可知\begin{equation*}
	\dim(\Ker(\vb{A}\vb{B}))
	=\dim(\Ker\vb{B}).
\end{equation*}
因此由\cref{theorem:向量空间.两个非零子空间的关系2} 可知\begin{equation*}
	\Ker\vb{B}
	=\Ker(\vb{A}\vb{B}),
\end{equation*}
也就是说\(\vb{A}\vb{B}\vb{x}=\vb0\)与\(\vb{B}\vb{x}=\vb0\)同解.
\end{proof}
\end{proposition}
\begin{remark}
由\cref{theorem:线性方程组.同解方程组.特例1,theorem:向量空间.用列满秩矩阵左乘任一矩阵不变秩} 可知
如果\(\vb{A}\)是列满秩矩阵,
那么\((\vb{A}\vb{B})\vb{x}=\vb0\)与\(\vb{B}\vb{x}=\vb0\)同解.
\end{remark}
%@see: https://www.bilibili.com/video/BV1eYqHYnE7B/

\begin{example}\label{example:线性方程组.左乘系数矩阵的转置矩阵同解}
%@see: 《高等代数(第三版 上册)》(丘维声) P122 命题2
设\(\vb{A}\in M_{s \times n}(\mathbb{R})\).
证明:齐次线性方程组\(\vb{A}\vb{x}=\vb0\)与\((\vb{A}^T\vb{A})\vb{x}=\vb0\)同解.
\begin{proof}
设\(\vb\xi\)是\(\vb{A}\vb{x}=\vb0\)的任意一个解,
则\(\vb{A}\vb\xi=\vb0\),于是\begin{equation*}
	(\vb{A}^T\vb{A})\vb\xi=\vb{A}^T(\vb{A}\vb\xi)=\vb{A}^T\vb0=\vb0,
\end{equation*}
这就是说\(\vb\xi\)是\((\vb{A}^T\vb{A})\vb{x}=\vb0\)的一个解.

又设\(\vb\eta\)是\((\vb{A}^T\vb{A})\vb{x}=\vb0\)的任意一个解,
则\begin{equation*}
	(\vb{A}^T\vb{A})\vb\eta=\vb0.
	\eqno(1)
\end{equation*}
在(1)式等号两边同时左乘\(\vb\eta^T\)得\begin{equation*}
	\vb\eta^T(\vb{A}^T\vb{A})\vb\eta=(\vb{A}\vb\eta)^T(\vb{A}\vb\eta)=0.
	\eqno(2)
\end{equation*}

假设\(\vb{A}\vb\eta=(\AutoTuple{c}{s})^T\in\mathbb{R}^s\).
由(2)式有\begin{equation*}
	(\AutoTuple{c}{s}) (\AutoTuple{c}{s})^T
	= \AutoTuple{c}{n}[+][2]
	= 0.
\end{equation*}
由于\(\AutoTuple{c}{n}\in\mathbb{R}\),
所以\(\AutoTuple{c}{n}[=]=0\),
\(\vb{A}\vb\eta=\vb0\),
这就是说\(\vb\eta\)是\(\vb{A}\vb{x}=\vb0\)的一个解.

综上所述,\((\vb{A}^T\vb{A})\vb{x}=\vb0\)与\(\vb{A}\vb{x}=\vb0\)同解.
\end{proof}
\end{example}
\begin{example}
%@credit: {de3029b8-10a6-4ae5-8f64-108dae1c10a9} 指出\((\vb{A}\vb{A}^T)\vb{x}=\vb0\)与\(\vb{A}\vb{x}=\vb0\)不一定同解
设\(\vb{A} \in M_n(K)\),\(\vb{A}^T\)是\(\vb{A}\)的转置矩阵.
举例说明:\((\vb{A}\vb{A}^T)\vb{x}=\vb0\)与\(\vb{A}\vb{x}=\vb0\)不同解.
\begin{solution}
%@Mathematica: A = {{0, 1}, {0, 0}}
%@Mathematica: Transpose[A]
%@Mathematica: A.Transpose[A]
取\(\vb{A} = \begin{bmatrix}
	0 & 1 \\
	0 & 0
\end{bmatrix}\),
则\(\vb{A}^T = \begin{bmatrix}
	0 & 0 \\
	1 & 0
\end{bmatrix},
\vb{A} \vb{A}^T = \begin{bmatrix}
	1 & 0 \\
	0 & 0
\end{bmatrix}\),
可以解得\begin{gather*}
	\Ker\vb{A} = \Set{ k (1,0)^T \given k \in K }, \\
	\Ker(\vb{A} \vb{A}^T) = \Ker\vb{A}^T = \Set{ k (0,1)^T \given k \in K }.
\end{gather*}
\end{solution}
\end{example}
\begin{example}
%@see: 《高等代数(第三版 上册)》(丘维声) P122 命题2
设\(\vb{A} \in M_{s \times n}(\mathbb{R})\).
求证:\begin{equation}\label{equation:矩阵乘积的秩.实矩阵及其转置矩阵的乘积的秩}
	\rank\vb{A} = \rank(\vb{A} \vb{A}^T) = \rank(\vb{A}^T \vb{A}).
\end{equation}
\begin{proof}
由\cref{example:线性方程组.左乘系数矩阵的转置矩阵同解} 可知
\(\vb{A} \vb{x} = \vb0\)与\((\vb{A}^T \vb{A}) \vb{x} = \vb0\)同解,
所以由\cref{theorem:线性方程组.同解方程组的系数矩阵的秩相同} 可知
\(\rank\vb{A} = \rank(\vb{A}^T \vb{A})\).
又由\cref{theorem:向量空间.转置不变秩} 可知
\(\rank(\vb{A} \vb{A}^T)
= \rank[(\vb{A}^T)^T (\vb{A}^T)]
= \rank\vb{A}^T
= \rank\vb{A}\).
\end{proof}
\end{example}
\begin{remark}
应该注意到\cref{equation:矩阵乘积的秩.实矩阵及其转置矩阵的乘积的秩} 成立的前提条件是:
矩阵\(\vb{A}\)是实矩阵.
如果矩阵\(\vb{A}\)不是实矩阵,\cref{equation:矩阵乘积的秩.实矩阵及其转置矩阵的乘积的秩} 就不一定成立.
例如,取矩阵\(\vb{A} = \begin{bmatrix}
	1 & \iu \\
	0 & 0
\end{bmatrix}
\in M_2(\mathbb{C})\),
易见\(\vb{A} \vb{A}^T = \begin{bmatrix}
	0 & 0 \\
	0 & 0
\end{bmatrix}\),
于是\(\rank\vb{A}=1\)而\(\rank(\vb{A}\vb{A}^T)=0\),
也就是说\(\rank\vb{A}\neq\rank(\vb{A}\vb{A}^T)\).
\end{remark}
\begin{example}
%@see: 《高等代数与解析几何(上册)》(盛为民、李方) P193 本章拓展题 2.
设\(\vb{A} \in M_{s \times n}(\mathbb{C})\).
求证:\begin{equation}
	\rank\vb{A} = \rank(\vb{A} \vb{A}^H) = \rank(\vb{A}^H \vb{A}).
\end{equation}
%TODO
\end{example}

\begin{example}
%@see: 《2025年全国硕士研究生入学统一考试(数学一)》二填空题/第15题
设矩阵\(\vb{A} = \begin{bmatrix}
	4 & 2 & -3 \\
	a & 3 & -4 \\
	b & 5 & -7
\end{bmatrix}\),
且\(\vb{A}^2 \vb{X} = \vb0\)
与\(\vb{A} \vb{X} = \vb0\)不同解.
求\(a-b\).
\begin{solution}
%@Mathematica: A = {{4, 2, -3}, {a, 3, -4}, {b, 5, -7}}
假设\(\rank\vb{A} = 3\),
则由\cref{example:西尔维斯特不等式.可逆矩阵的正整数次幂可逆} 可知
\(\rank\vb{A}^2\)也是满秩矩阵,
从而\(\vb{A}^2 \vb{X} = \vb0\)
与\(\vb{A} \vb{X} = \vb0\)均只有零解,
不满足题目要求.
因此\(\rank\vb{A} < 3\),
那么由\cref{theorem:逆矩阵.矩阵可逆的充分必要条件1} 有\begin{equation*}
	\abs{\vb{A}}
	= \begin{vmatrix}
		4 & 2 & -3 \\
		a & 3 & -4 \\
		b & 5 & -7
	\end{vmatrix}
	= \begin{vmatrix}
		4 & 2 & -3 \\
		a & 3 & -4 \\
		-4-a+b & 0 & 0
	\end{vmatrix}
	= -4 - a + b
	= 0,
\end{equation*}
%@Mathematica: Solve[Det[A] == 0 /. {a -> x + b}, x]
于是\(a-b=-4\).
\end{solution}
\end{example}

\section{非齐次线性方程组的解集的结构}
数域\(K\)上\(n\)元非齐次线性方程组
\begin{equation}\label[equation-system]{equation:向量空间.非齐次线性方程组}
	x_1\vb\alpha_1+x_2\vb\alpha_2+\dotsb+x_n\vb\alpha_n=\vb\beta
\end{equation}
的每一个解都是\(K^n\)中的一个向量,
我们称这个向量为该方程组的一个解向量.
因此这个\(n\)元非齐次线性方程组的解集\(U\)是\(K^n\)的一个子集.
当该方程组无解时,\(U\)是空集.
现在我们想要知道,当该方程组有无穷多个解时,解集\(U\)的结构如何?

我们把齐次线性方程组\(\vb{A} \vb{x} = \vb0\)
称为“非齐次线性方程组\(\vb{A} \vb{x} = \vb\beta\)的\DefineConcept{导出组}”.

\begin{proposition}\label{theorem:非齐次线性方程组的解集的结构.解集的结构1}
%@see: 《高等代数(第三版 上册)》(丘维声) P97 性质1
%@see: 《高等代数(大学高等代数课程创新教材 第二版 上册)》(丘维声) P127 性质1
非齐次线性方程组\(\vb{A} \vb{x} = \vb\beta\)的两个解的差是它的导出组的一个解.
\begin{proof}
设\(\vb{x}_1,\vb{x}_2\)是非齐次线性方程组\(\vb{A} \vb{x} = \vb\beta\)的任意两个解,
即\begin{equation*}
	\vb{A} \vb{x}_1 = \vb\beta,
	\qquad
	\vb{A} \vb{x}_2 = \vb\beta,
\end{equation*}
则\begin{equation*}
	\vb{A}(\vb{x}_1 - \vb{x}_2)
	= \vb{A} \vb{x}_1 - \vb{A} \vb{x}_2
	= \vb\beta - \vb\beta
	= \vb0;
\end{equation*}
所以\(\vb{x}_1 - \vb{x}_2\)是\(\vb{A} \vb{x} = \vb0\)的解.
\end{proof}
\end{proposition}

\begin{proposition}\label{theorem:非齐次线性方程组的解集的结构.解集的结构2}
%@see: 《高等代数(第三版 上册)》(丘维声) P97 性质2
%@see: 《高等代数(大学高等代数课程创新教材 第二版 上册)》(丘维声) P127 性质2
非齐次线性方程组\(\vb{A} \vb{x} = \vb\beta\)的一个解与它的导出组的一个解之和仍是它的一个解.
\begin{proof}
设\(\vb\xi\)是\(\vb{A} \vb{x} = \vb\beta\)的一个解,
\(\vb\zeta\)是\(\vb{A} \vb{x} = \vb0\)的一个解,
即\begin{equation*}
	\vb{A}\vb\xi = \vb\beta,
	\qquad
	\vb{A}\vb\zeta = \vb0,
\end{equation*}
则\begin{equation*}
	\vb{A}(\vb\xi + \vb\zeta)
	= \vb{A}\vb\xi + \vb{A}\vb\zeta
	= \vb\beta + \vb0
	= \vb\beta;
\end{equation*}
所以\(\vb\xi + \vb\zeta\)是\(\vb{A} \vb{x} = \vb\beta\)的解.
\end{proof}
\end{proposition}

\begin{example}
设\(\vb{A} \in M_{s \times n}(K)\).
证明:非齐次线性方程组\(\vb{A} \vb{x} = \vb\beta\)有解的充分必要条件是:
如果矩阵\(\vb{A}\)的行向量组的某个线性组合是零向量,
那么向量\(\vb\beta\)的各个分量的相同线性组合也是零.
\begin{proof}
必要性.
假设\(\vb{x}_0\)是\(\vb{A} \vb{x} = \vb\beta\)的一个解,
即\(\vb{A} \vb{x}_0 = \vb\beta\),
那么\begin{equation*}
	(\AutoTuple{k}{s}) \vb{A} \vb{x}_0
	= (\AutoTuple{k}{s}) \vb\beta,
\end{equation*}
显然当\((\AutoTuple{k}{s}) \vb{A} = \vb0\)时,
\(
	(\AutoTuple{k}{s}) \vb\beta
	= \vb0 \vb{x}_0
	= 0
\).

充分性.
假设\begin{equation*}
	\vb{y}_0^T \vb{A} = \vb0
	\implies
	\vb{y}_0^T \vb\beta = 0,
\end{equation*}
其中\(\vb{y}_0 = (\AutoTuple{y}{s})^T\),
那么由\cref{theorem:齐次线性方程组的解集的结构.两个方程同解的充分必要条件1} 可知,
\(\vb{A}^T \vb{y}_0 = \vb0\)
与\(
	\begin{bmatrix}
		\vb{A}^T \\
		\vb\beta^T
	\end{bmatrix}
	\vb{y}_0 = 0
\)同解,
再由\cref{theorem:齐次线性方程组的解集的结构.两个方程同解的充分必要条件2} 可知\begin{equation*}
	\rank\vb{A}^T
	= \rank\begin{bmatrix}
		\vb{A}^T \\
		\vb\beta^T
	\end{bmatrix}.
\end{equation*}
因为\(
	\rank\vb{A} = \rank\vb{A}^T,
	\rank\begin{bmatrix}
		\vb{A} & \vb\beta
	\end{bmatrix}
	= \rank\begin{bmatrix}
		\vb{A}^T \\
		\vb\beta^T
	\end{bmatrix}
\),
所以\(
	\rank\vb{A}
	= \rank\begin{bmatrix}
		\vb{A} & \vb\beta
	\end{bmatrix}
\),
因此由\cref{theorem:向量空间.线性方程组有解判别定理} 可知,
\(\vb{A} \vb{x} = \vb\beta\)有解.
\end{proof}
\end{example}
\begin{remark}
%@credit: {b8a6b30d-44bc-4d7a-a6b5-574e615c5be0}
上例说明:\(\vb{A} \vb{x} = \vb\beta\)有解,
当且仅当\(\vb{A}^T\)的核空间\(\Ker\vb{A}^T\)中每一个向量都与\(\vb\beta\)正交.
%@credit: {5f4d2f8a-fc8b-4798-85d6-98670f6761e7}
结合\cref{example:齐次线性方程组的解集的结构.矩阵的像空间与它的转置的核空间互为正交补} 给出的结论,
又可以推出:\(\vb{A} \vb{x} = \vb\beta\)有解,
当且仅当\(\vb\beta\)是\(\vb{A}\)的像空间的一个元素.
\end{remark}

\begin{theorem}\label{theorem:向量空间.非齐次线性方程组的解集的结构}
%@see: 《高等代数(第三版 上册)》(丘维声) P97 定理1
%@see: 《高等代数(大学高等代数课程创新教材 第二版 上册)》(丘维声) P127 定理1
数域\(K\)上\(n\)元非齐次线性方程组 \labelcref{equation:向量空间.非齐次线性方程组} 有解,
则它的解集为\begin{equation}
	\vb{x}_0 + W \defeq \Set{ \vb{x}_0+\vb{x} \given \vb{x} \in W },
\end{equation}
其中\(\vb{x}_0\)是\cref{equation:向量空间.非齐次线性方程组} 的一个解,
\(W\)是\cref{equation:向量空间.非齐次线性方程组} 的导出组的解空间.
\begin{proof}
记\(U \defeq \Set{ \vb{x} \given \vb{A} \vb{x} = \vb\beta }\).
由题意有\(\vb{A} \vb{x}_0 = \vb\beta\),即\(\vb{x}_0 \in U\),
并且\(W = \Set{ \vb{x} \given \vb{A} \vb{x} = \vb0 }\).

假设\(\vb\eta \in W\),
由\cref{theorem:非齐次线性方程组的解集的结构.解集的结构2} 得\(\vb{x}_0 + \vb\eta \in U\),
因此\(\vb{x}_0 + W \subseteq U\).

反之,假设\(\vb\gamma \in U\),
由\cref{theorem:非齐次线性方程组的解集的结构.解集的结构1} 得\(\vb\gamma - \vb{x}_0 \in W\).
记\(\vb\eta \defeq \vb\gamma - \vb{x}_0\),
则\(\vb\gamma = \vb{x}_0 + \vb\eta \in \vb{x}_0 + W\),
因此\(\vb{x}_0 + W \supseteq U\).

综上所述,\(U = \vb{x}_0 + W\).
\end{proof}
\end{theorem}

我们把\cref{theorem:向量空间.非齐次线性方程组的解集的结构} 中的解向量\(\vb{x}_0\)
称为“\cref{equation:向量空间.非齐次线性方程组} 的\DefineConcept{特解}(particular solution)”.

\begin{corollary}\label{theorem:向量空间.非齐次线性方程组的解集的结构.推论1}
%@see: 《高等代数(第三版 上册)》(丘维声) P98 推论2
%@see: 《高等代数(大学高等代数课程创新教材 第二版 上册)》(丘维声) P128 推论1
如果\(n\)元非齐次线性方程组 \labelcref{equation:向量空间.非齐次线性方程组} 有解,
则它的解唯一的充分必要条件是它的导出组只有零解.
\begin{proof}
当\(n\)元非齐次线性方程组 \labelcref{equation:向量空间.非齐次线性方程组} 有解时,
有\begin{align*}
	\text{\cref{equation:向量空间.非齐次线性方程组} 有唯一解}
	&\iff
	\text{\cref{equation:向量空间.非齐次线性方程组} 的解集为$U = \vb{x}_0 + W = \{\vb{x}_0\}$} \\
	&\iff
	W = \{\vb0\}.
	\qedhere
\end{align*}
\end{proof}
\end{corollary}

从\cref{theorem:向量空间.非齐次线性方程组的解集的结构.推论1} 立即得出,
当非齐次线性方程组有无穷多个解时,
它的导出组必有非零解.
此时取导出组的一个基础解系\(\AutoTuple{k}{n-r}\),
其中\(r\)是导出组的系数矩阵的秩,
则非齐次线性方程组的解集为\begin{equation*}
	U = \Set{ \vb{x}_0+k_1\vb{x}_1+k_2\vb{x}_2+ \dotsb +k_{n-r}\vb{x}_{n-r} \given \AutoTuple{k}{n-r} \in K },
\end{equation*}
其中\(\vb{x}_0\)是非齐次线性方程组的一个特解.

解集\(U\)的代表元素\begin{equation*}
	\vb{x}_0+k_1\vb{x}_1+k_2\vb{x}_2+ \dotsb +k_{n-r}\vb{x}_{n-r}
	\quad(\AutoTuple{k}{n-r} \in K)
\end{equation*}
称为“\cref{equation:向量空间.非齐次线性方程组} 的\DefineConcept{通解}”.

同\cref{theorem:线性方程组.齐次线性方程组的解向量个数} 的证明过程中给出的
对系数矩阵\(\vb{A}\)进行初等行变换、初等列变换一样,
我们可以对非齐次线性方程组 \labelcref{equation:向量空间.非齐次线性方程组} 的
增广矩阵\((\vb{A},\vb\beta)\)进行类似的初等变换,
将其化为\((\vb{J},\vb\gamma)\),
其中\(
	\vb{J} = \begin{bmatrix}
		\vb{E}_r & \vb{F} \\
		\vb0 & \vb0
	\end{bmatrix}
\),
于是原方程化为\(\vb{J} \vb{x} = \vb\gamma\),
其中\(\vb\gamma = (\AutoTuple{c}{s})\),
从而有\begin{equation*}
	\vb{E}_r
	\begin{bmatrix}
		x_1 \\ \vdots \\ x_r
	\end{bmatrix}
	+ \vb{F}
	\begin{bmatrix}
		x_{r+1} \\ \vdots \\ x_n
	\end{bmatrix}
	= \begin{bmatrix}
		c_1 \\ \vdots \\ c_r
	\end{bmatrix}.
\end{equation*}
令\(x_{r+1} = \dotsb = x_n = 0\),
那么有\(
	x_1 = c_1,
	x_2 = c_2,
	\dotsc,
	x_r = c_r
\).
因此\(\vb{x}_0 \defeq (\AutoTuple{c}{r},0,\dotsc,0)^T\)
是\cref{equation:向量空间.非齐次线性方程组} 的一个特解.
接下来只要加上\cref{equation:向量空间.非齐次线性方程组} 的导出组的通解,
便可得到\cref{equation:向量空间.非齐次线性方程组} 的通解.

根据上述讨论,我们可以总结出
求解非齐次线性方程组\(\vb{A} \vb{x} = \vb\beta\)的基础解系的方法:
\begin{algorithm}[求解非齐次线性方程组]
\hfill
\begin{enumerate}
	\item 把非齐次线性方程组 \labelcref{equation:向量空间.非齐次线性方程组} 的
	系数矩阵\(\vb{A} \in M_{s \times n}(K)\)和常数项\(\vb\beta\)
	组成的增广矩阵\(\widetilde{\vb{A}} \defeq (\vb{A},\vb\beta)\)
	经过初等行变换化简成行约化矩阵\((\vb{J},\vb\gamma)\);

	\item 检查\(\rank\vb{J}\)与\(\rank(\vb{J},\vb\gamma)\)是否相等,
	判断\(\vb{A} \vb{x} = \vb\beta\)是否有解;
	如果无解,停止计算;

	\item 将\(\vb\gamma\)的前\(r = \rank\vb{J}\)个分量\(\AutoTuple{c}{r}\)与\(n-r\)个\(0\)一起,
	组成向量\(\vb{x}_0 \defeq (\AutoTuple{c}{r},0,\dotsc,0)^T\),
	作为\cref{equation:向量空间.非齐次线性方程组} 的一个特解;

	\item 从\(\vb{J}\)中,找出非零首元对应的列\(\vb\beta_{i_1},\dotsc,\vb\beta_{i_r}\),
	将这些列称为\DefineConcept{主列}(pivot column);

	\item 从\(\vb{J}\)中,找出没有非零首元的列\(\vb\beta_{j_1},\dotsc,\vb\beta_{j_{n-r}}\),
	将这些列称为\DefineConcept{自由列}(free column);

	\item 各个自由列的负向量\(-\vb\beta_{j_1},\dotsc,-\vb\beta_{j_{n-r}}\)
	就是导出组\(\vb{A} \vb{x} = \vb0\)的一个基础解系;

	\item 写出\cref{equation:向量空间.非齐次线性方程组} 的通解:\begin{equation*}
		\vb{x}_0 - k_1 \vb\beta_{j_1} - \dotsb - k_{n-r} \vb\beta_{j_{n-r}}.
	\end{equation*}
\end{enumerate}
\end{algorithm}

\begin{example}
%@see: 《线性代数》(张慎语、周厚隆) P87 例1
求线性方程组\begin{equation*}
	\left\{ \begin{array}{*{9}{r}}
		x_1 &-& 2 x_2 &-& x_3 &+& 2 x_4 &=& 4 \\
		2 x_1 &-& 2 x_2 &-& 3 x_3 && &=& 2 \\
		4 x_1 &-& 2 x_2 &-& 7 x_3 &-& 4 x_4 &=& -2
	\end{array} \right.
\end{equation*}的通解.
\begin{solution}
写出增广矩阵\(\widetilde{\vb{A}}\),并作初等行变换化简
\begin{align*}
	\widetilde{\vb{A}}
	&= \begin{bmatrix}
		1 & -2 & -1 & 2 & 4 \\
		2 & -2 & -3 & 0 & 2 \\
		4 & -2 & -7 & -4 & -2
	\end{bmatrix}
	\to \begin{bmatrix}
		1 & -2 & -1 & 2 & 4 \\
		0 & 2 & -1 & -4 & -6 \\
		0 & 6 & -3 & -12 & -18
	\end{bmatrix} \\
	&\to \begin{bmatrix}
		1 & 0 & -2 & -2 & -2 \\
		0 & 1 & -\tfrac12 & -2 & -3 \\
		0 & 0 & 0 & 0 & 0
	\end{bmatrix}
	= \widetilde{\vb{B}}
	= (\vb{B},\vb\gamma),
\end{align*}
故\(\rank\vb{A} = \rank\widetilde{\vb{A}} = \rank\widetilde{\vb{B}} = 2\),于是原方程组有解.

解同解方程组\begin{equation*}
	\left\{ \begin{array}{*{9}{c}}
		x_1 && &-& 2 x_3 &-& 2 x_4 &=& -2 \\
		&& x_2 &-& \frac12 x_3 &-& 2 x_4 &=& -3
	\end{array} \right..
\end{equation*}
令\(x_3 = 0\),\(x_4 = 0\),
解得\(x_1 = -2\),\(x_2 = -3\),
得特解\begin{equation*}
	\vb{x}_0 = (-2,-3,0,0)^T.
\end{equation*}

又因为导出组的基础解系含\(4 - \rank\vb{A} = 2\)个向量.
将\(x_3,x_4\)的两组值\((2,0),(0,1)\)分别代入\begin{equation*}
	\left\{ \begin{array}{*{9}{c}}
		x_1 && &-& 2 x_3 &-& 2 x_4 &=& 0 \\
		&& x_2 &-& \frac12 x_3 &-& 2 x_4 &=& 0
	\end{array} \right.
\end{equation*}
得基础解系
\(\vb{x}_1 = (4,1,2,0)^T,
\vb{x}_2 = (2,2,0,1)^T\).

于是原方程的通解为\begin{equation*}
	\vb{x} = \vb{x}_0 + k_1 \vb{x}_1 + k_2 \vb{x}_2
	= \begin{bmatrix} -2 \\ -3 \\ 0 \\ 0 \end{bmatrix}
	+ k_1 \begin{bmatrix} 4 \\ 1 \\ 2 \\ 0 \end{bmatrix}
	+ k_2 \begin{bmatrix} 2 \\ 2 \\ 0 \\ 1 \end{bmatrix},
\end{equation*}
其中\(k_1,k_2\)是任意常数.
\end{solution}
%@Mathematica: A = RowReduce[({{1, -2, -1, 2, 4}, {2, -2, -3, 0, 2}, {4, -2, -7, -4, -2}})]
%@Mathematica: LinearSolve[A[[All, 1 ;; 4]], A[[All, 5]]]
\end{example}

\begin{example}
%@see: 《线性代数》(张慎语、周厚隆) P88 例2
写出线性方程组\begin{equation*}
	\left\{ \begin{array}{l}
		x_1 - x_2 = a_1, \\
		x_2 - x_3 = a_2, \\
		x_3 - x_4 = a_3, \\
		x_4 - x_1 = a_4
	\end{array} \right.
\end{equation*}有解的充分必要条件,并求解.
\begin{solution}
对增广矩阵\(\widetilde{\vb{A}}\)作初等行变换化简
\begin{align*}
	\widetilde{\vb{A}}
	&= \begin{bmatrix}
		1 & -1 & 0 & 0 & a_1 \\
		0 & 1 & -1 & 0 & a_2 \\
		0 & 0 & 1 & -1 & a_3 \\
		-1 & 0 & 0 & 0 & a_4
	\end{bmatrix} \to \begin{bmatrix}
		1 & -1 & 0 & 0 & a_1 \\
		0 & 1 & -1 & 0 & a_2 \\
		0 & 0 & 1 & -1 & a_3 \\
		0 & 0 & 0 & 0 & \sum_{i=1}^4 a_i
	\end{bmatrix} \\
	&\to \begin{bmatrix}
		1 & 0 & 0 & -1 & a_1 + a_2 + a_3 \\
		0 & 1 & 0 & -1 & a_2 + a_3 \\
		0 & 0 & 1 & -1 & a_3 \\
		0 & 0 & 0 & 0 & \sum_{i=1}^4 a_i
	\end{bmatrix}.
\end{align*}
可见\(\rank\vb{A} = 3\).
方程组有解的充分必要条件是\(\rank\widetilde{\vb{A}} = \rank\vb{A} = 3\),
那么充分必要条件就是\(\sum_{i=1}^4 a_i = 0\).

当方程组有解时,通解为\begin{equation*}
	\begin{bmatrix}
		a_1 + a_2 + a_3 \\ a_2 + a_3 \\ a_3 \\ 0
	\end{bmatrix}
	+ k \begin{bmatrix}
		1 \\ 1 \\ 1 \\ 1
	\end{bmatrix},
\end{equation*}
其中\(k\)为任意常数.
\end{solution}
\end{example}

至此,我们讨论了线性方程组的解的存在性、解的性质、解的结构及求解方法,建立起了线性方程组的完整理论.
解线性方程组是线性代数的基本问题之一,现代科学技术方面用到的数学问题也有很多要归结到解线性方程组.

\begin{example}
%@see: 《线性代数》(张慎语、周厚隆) P89 习题4.6 4.
设\(\vb{x}_0\)是非齐次线性方程组\(\vb{A} \vb{x} = \vb\beta\)的一个解,
\(\AutoTuple{\vb{x}}{n-r}\)是其导出组\(\vb{A} \vb{x} = \vb0\)的一个基础解系.
证明:\(\vb{x}_0,\AutoTuple{\vb{x}}{n-r}\)线性无关.
\begin{proof}
因为\(\AutoTuple{\vb{x}}{n-r}\)是其导出组\(\vb{A} \vb{x} = \vb0\)的一个基础解系,
根据基础解系的定义,显然有\(\AutoTuple{\vb{x}}{n-r}\)线性无关.
假设\(\vb{x}_0,\AutoTuple{\vb{x}}{n-r}\)线性相关,
那么\(\vb{x}_0\)可由\(\AutoTuple{\vb{x}}{n-r}\)线性表出,
即存在数\(k_1,k_2,\dotsc,k_{n-r}\)使得\begin{equation*}
	\vb{x}_0 = k_1 \vb{x}_1 + k_2 \vb{x}_2 + \dotsb + k_{n-r} \vb{x}_{n-r},
\end{equation*}
进而有\begin{align*}
	&\vb{A} \vb{x}_0 = \vb{A}(k_1 \vb{x}_1 + k_2 \vb{x}_2 + \dotsb + k_{n-r} \vb{x}_{n-r}) \\
	&= k_1 \vb{A} \vb{x}_1 + k_2 \vb{A} \vb{x}_2 + \dotsb + k_{n-r} \vb{A} \vb{x}_{n-r}
	= \vb0 + \vb0 + \dotsb + \vb0 = \vb0,
\end{align*}
即\(\vb{x}_0\)是\(\vb{A} \vb{x} = \vb0\)的一个解,
这与\(\vb{x}_0\)是\(\vb{A} \vb{x} = \vb\beta\neq\vb0\)的一个解矛盾,
所以\(\vb{x}_0,\AutoTuple{\vb{x}}{n-r}\)线性无关.
\end{proof}
\end{example}

\begin{example}
%@see: 《线性代数》(张慎语、周厚隆) P89 习题4.6 6.
设线性方程组\(\vb{A} \vb{x} = \vb\beta\)的增广矩阵
\(\widetilde{\vb{A}} = (\vb{A},\vb\beta)\)是一个\(n\)阶可逆矩阵.
证明:方程组无解.
\begin{proof}
因为\(\widetilde{\vb{A}}\)是\(n\)阶方阵,
所以\(\vb{A}\)是\(n \times (n-1)\)矩阵,
从而\(\rank\vb{A} \leq \min\{n-1,n\} = n-1\).
又因为\(\widetilde{\vb{A}}\)可逆,所以\(\rank\widetilde{\vb{A}} = n\).
因为\(\rank\widetilde{\vb{A}} = n > n-1 \geq \rank\vb{A}\),
所以方程组\(\vb{A} \vb{x} = \vb\beta\)无解.
\end{proof}
\end{example}

\begin{example}
%@see: 《2025年全国硕士研究生入学统一考试(数学一)》一选择题/第6题
设\(\AutoTuple{\vb\alpha}{4}\)是\(n\)维向量,
\(\vb\alpha_1,\vb\alpha_2\)线性无关,
\(\vb\alpha_1,\vb\alpha_2,\vb\alpha_3\)线性相关,
且\begin{equation*}
	\vb\alpha_1 + \vb\alpha_2 + \vb\alpha_4 = \vb0.
\end{equation*}
试判别关于\(x,y,z\)的方程组
\(x \vb\alpha_1 + y \vb\alpha_2 + z \vb\alpha_3 = \vb\alpha_4\)的几何图形.
\begin{solution}
由题意有\(\vb\alpha_3,\vb\alpha_4\)可以由\(\vb\alpha_1,\vb\alpha_2\)线性表出.
那么\begin{equation*}
	\rank(\vb\alpha_1,\vb\alpha_2,\vb\alpha_3)
	= \rank(\vb\alpha_1,\vb\alpha_2,\vb\alpha_3,\vb\alpha_4)
	= 2 < 3,
\end{equation*}
%\cref{theorem:线性方程组.齐次线性方程组的解向量个数}
因此\(x \vb\alpha_1 + y \vb\alpha_2 + z \vb\alpha_3 = \vb0\)的基础解系含有\(3-2=1\)个向量,
这就说明该方程组的几何图形是一条直线.

假设方程组\(x \vb\alpha_1 + y \vb\alpha_2 + z \vb\alpha_3 = \vb\alpha_4\)的几何图形是一条经过原点的直线,
则\((0,0,0)\)是该方程组的一个解,
即\begin{equation*}
	\vb\alpha_4
	= 0 \vb\alpha_1 + 0 \vb\alpha_2 + 0 \vb\alpha_3
	= \vb0.
\end{equation*}
再由\(\vb\alpha_1 + \vb\alpha_2 + \vb\alpha_4 = \vb0\)
可知\(\vb\alpha_1 + \vb\alpha_2 = \vb0\),
这说明\(\vb\alpha_1,\vb\alpha_2\)线性相关,矛盾!
因此方程组\(x \vb\alpha_1 + y \vb\alpha_2 + z \vb\alpha_3 = \vb\alpha_4\)的几何图形是一条不经过原点的直线.
\end{solution}
\end{example}
\begin{example}
%@see: 《2016年全国硕士研究生入学统一考试(数学一)》三解答题/第20题
设矩阵\(\vb{A} = \begin{bmatrix}
	1 & -1 & -1 \\
	2 & a & 1 \\
	-1 & 1 & a
\end{bmatrix},
\vb{B} = \begin{bmatrix}
	2 & 2 \\
	1 & a \\
	-a-1 & -2
\end{bmatrix}\).
当\(a\)为何值时,方程\(\vb{A}\vb{X}=\vb{B}\)无解、有唯一解、有无穷多解?
在有解时,求解此方程.
\begin{solution}
%@Mathematica: A = {{1, -1, -1, 2, 2}, {2, a, 1, 1, a}, {-1, 1, a, -a - 1, -2}}
对增广矩阵\((\vb{A},\vb{B})\)作初等行变换,
得\begin{equation*}
	\begin{bmatrix}
		1 & -1 & -1 & 2 & 2 \\
		2 & a & 1 & 1 & a \\
		-1 & 1 & a & -a-1 & -2
	\end{bmatrix}
%@Mathematica: A1 = ({ {1, 0, 0}, {-2, 1, 0}, {1, 0, 1} }).A
	\to \begin{bmatrix}
		1 & -1 & -1 & 2 & 2 \\
		0 & a+2 & 3 & -3 & a-4 \\
		0 & 0 & a-1 & 1-a & 0
	\end{bmatrix}.
\end{equation*}

当\(a+2=0\)即\(a=-2\)时,继续对增广矩阵作初等行变换,
%@Mathematica: A /. a -> -2 // RowReduce // MatrixForm
得\begin{equation*}
	\begin{bmatrix}
		1 & -1 & -1 & 2 & 2 \\
		0 & 0 & 3 & -3 & -6 \\
		0 & 0 & -3 & 3 & 0
	\end{bmatrix}
	\to \begin{bmatrix}
		1 & -1 & 0 & 1 & 0 \\
		0 & 0 & 1 & -1 & 0 \\
		0 & 0 & 0 & 0 & 1
	\end{bmatrix},
\end{equation*}
由于\(\rank\vb{A} = 2 \neq \rank(\vb{A},\vb{B}) = 3\),
所以方程\(\vb{A}\vb{X}=\vb{B}\)无解.

当\(a\neq-2\)时,继续对增广矩阵作初等行变换,
得\begin{equation*}
	\begin{bmatrix}
		1 & -1 & -1 & 2 & 2 \\
		0 & a+2 & 3 & -3 & a-4 \\
		0 & 0 & a-1 & 1-a & 0
	\end{bmatrix}
	\to
%@Mathematica: A2 = ({ {1, 1, 0}, {0, 1, 0}, {0, 0, 1} }).({ {1, 0, 0}, {0, 1/(a + 2), 0}, {0, 0, 1} }).A1 // Factor // Simplify
	\def\arraystretch{1.5}
	\begin{bmatrix}
		1 & 0 & \frac{1-a}{a+2} & \frac{2a+1}{a+2} & \frac{3a}{a+2} \\
		0 & 1 & \frac3{a+2} & -\frac3{a+2} & \frac{a-4}{a+2} \\
		0 & 0 & a-1 & 1-a & 0
	\end{bmatrix}.
\end{equation*}

当\(a-1=0\)即\(a=1\)时,
增广矩阵化为\begin{equation*}
	\begin{bmatrix}
		1 & 0 & 0 & 1 & 1 \\
		0 & 1 & 1 & -1 & -1 \\
		0 & 0 & 0 & 0 & 0
	\end{bmatrix}.
\end{equation*}
由于\(\rank\vb{A} = \rank(\vb{A},\vb{B}) = 2 < 3\),
所以方程\(\vb{A}\vb{X}=\vb{B}\)有无穷多解.
解方程\begin{equation*}
	\begin{bmatrix}
		1 & 0 & 0 \\
		0 & 1 & 1 \\
		0 & 0 & 0 \\
	\end{bmatrix}
	\vb{x}
	= \begin{bmatrix}
		1 \\ -1 \\ 0
	\end{bmatrix}
\end{equation*}
得\begin{equation*}
	\vb{x}
	= \begin{bmatrix}
		1 \\ -1 \\ 0
	\end{bmatrix}
	+ k \begin{bmatrix}
		0 \\ -1 \\ 1
	\end{bmatrix}
	\quad(\text{$k$是任意常数}).
\end{equation*}
于是方程\(\vb{A}\vb{X}=\vb{B}\)的解为\begin{equation*}
	\vb{X} = \begin{bmatrix}
		1 & 1\\
		-1-k_1 & -1-k_2 \\
		k_1 & k_2
	\end{bmatrix}
	\quad(\text{$k_1,k_2$是任意常数}).
\end{equation*}

当\(a\neq1\)且\(a\neq-2\)时,继续对增广矩阵作初等行变换,
得\begin{equation*}
%@Mathematica: ({ {1, 0, (a - 1)/(2 + a)}, {0, 1, -3/(2 + a)}, {0, 0, 1} }).({ {1, 0, 0}, {0, 1, 0}, {0, 0, 1/(a - 1)} }).A2 // Factor // Simplify // MatrixForm
	\def\arraystretch{1.5}
	\begin{bmatrix}
		1 & 0 & 0 & 1 & \frac{3a}{a+2} \\
		0 & 1 & 0 & 0 & \frac{a-4}{a+2} \\
		0 & 0 & 1 & -1 & 0
	\end{bmatrix}.
\end{equation*}
由于\(\rank\vb{A} = \rank(\vb{A},\vb{B}) = 3\),
所以方程\(\vb{A}\vb{X}=\vb{B}\)有唯一解\begin{equation*}
	\def\arraystretch{1.5}
	\vb{X} = \begin{bmatrix}
		1 & \frac{3a}{a+2} \\
		0 & \frac{a-4}{a+2} \\
		-1 & 0
	\end{bmatrix}.
\end{equation*}
\end{solution}
\end{example}

\section{广义逆矩阵}
对于线性方程组\(\vb{A}\vb{x}=\vb\beta\),如果\(\vb{A}\)可逆,那么它有\(\vb{x}=\vb{A}^{-1}\vb\beta\).
如果\(\vb{A}\)不可逆,但\(\vb{A}\vb{x}=\vb\beta\)有解,那么它的解是否也可表达为类似的简洁公式呢?
我们接下来带着这个问题,开始对\(\vb{A}^{-1}\)的性质的分析.

如果\(\vb{A}\)可逆,那么\(\vb{A}\vb{A}^{-1}=\vb{E}\).
显然,只要在等式两端同时右乘\(\vb{A}\),便得\(\vb{A}\vb{A}^{-1}\vb{A}=\vb{A}\).
这就表明:当\(\vb{A}\)可逆时,\(\vb{A}^{-1}\)是矩阵方程\(\vb{A} \vb{X} \vb{A} = \vb{A}\)的一个解.
受此启发,当\(\vb{A}\)不可逆时,为了找到\(\vb{A}^{-1}\)的替代物,应当去找矩阵方程\(\vb{A} \vb{X} \vb{A} = \vb{A}\)的解.

\begin{theorem}[广义逆存在定理]\label{theorem:线性方程组.广义逆1}
设\(\vb{A}\)是数域\(K\)上的\(s \times n\)非零矩阵,
则矩阵方程\begin{equation}\label{equation:线性方程组.广义逆1矩阵方程}
	\vb{A} \vb{X} \vb{A} = \vb{A}
\end{equation}一定有解.
如果\(\rank\vb{A}=r\),并且\[
	\vb{A}
	= \vb{P}
	\begin{bmatrix}
		\vb{E}_r & \vb0 \\
		\vb0 & \vb0
	\end{bmatrix}
	\vb{Q},
\]
其中\(\vb{P},\vb{Q}\)分别是\(K\)上\(s\)阶、\(n\)阶可逆矩阵,
那么矩阵方程 \labelcref{equation:线性方程组.广义逆1矩阵方程} 的通解为
\begin{equation}\label{equation:线性方程组.广义逆1矩阵方程的通解}
	\vb{X} = \vb{Q}^{-1} \begin{bmatrix} \vb{E}_r & \vb{B} \\ \vb{C} & \vb{D} \end{bmatrix} \vb{P}^{-1},
\end{equation}
其中\(\vb{B}\in M_{r\times(s-r)}(K),
\vb{C}\in M_{(n-r)\times r}(K),
\vb{D}\in M_{(n-r)\times(s-r)}(K)\).
\end{theorem}

\begin{definition}
设\(\vb{A}\)是数域\(K\)上的\(s \times n\)矩阵,
矩阵方程\(\vb{A} \vb{X} \vb{A} = \vb{A}\)的每一个解都称为
“\(\vb{A}\)的一个\DefineConcept{广义逆矩阵}(generalized inverse)”,
简称“\(\vb{A}\)的广义逆”,记作\(\vb{A}^-\).
\end{definition}

\begin{property}\label{theorem:线性方程组.广义逆的性质1}
广义逆满足以下性质:\begin{gather}
	\vb{A}\vb{A}^-\vb{A}=\vb{A}, \\
	\vb{A}^-\vb{A}\vb{A}^-=\vb{A}^-, \\
	(\vb{A}\vb{A}^-)^H=\vb{A}\vb{A}^-, \\
	(\vb{A}^-\vb{A})^H=\vb{A}^-\vb{A}.
\end{gather}
\end{property}

\begin{property}\label{theorem:线性方程组.广义逆的性质2}
任意一个\(n \times s\)矩阵都是\(\vb0_{s \times n}\)的广义逆.
\end{property}

\begin{theorem}[非齐次线性方程组的相容性定理]\label{theorem:线性方程组.非齐次线性方程组的相容性定理}
非齐次线性方程组\(\vb{A} \vb{X} = \vb\beta\)有解的充分必要条件是
\(\vb\beta = \vb{A} \vb{A}^- \vb\beta\).
\begin{proof}
必要性.
设\(\vb\alpha\)是\(\vb{A} \vb{X} = \vb\beta\)的一个解,
则\[
	\vb\beta
	= \vb{A} \vb\alpha
	= (\vb{A} \vb{A}^- \vb{A}) \vb\alpha
	= \vb{A} \vb{A}^- \vb\beta.
\]

充分性.
设\(\vb\beta = \vb{A} \vb{A}^- \vb\beta\),
则\(\vb{A}^- \vb\beta\)是\(\vb{A} \vb{X} = \vb\beta\)的解.
\end{proof}
\end{theorem}

\begin{theorem}[非齐次线性方程组的解的结构定理]\label{theorem:线性方程组.非齐次线性方程组的解的结构定理}
非齐次线性方程组\(\vb{A} \vb{X} = \vb\beta\)有解时,
它的通解为\begin{equation}\label{equation:线性方程组.非齐次线性方程组的通解1}
	\vb{X} = \vb{A}^- \vb\beta.
\end{equation}
\end{theorem}
从\cref{theorem:线性方程组.非齐次线性方程组的解的结构定理} 看出,
任意非齐次线性方程组\(\vb{A} \vb{X} = \vb\beta\)有解时,
它的通解有简洁漂亮的形式 \labelcref{equation:线性方程组.非齐次线性方程组的通解1}.

\begin{theorem}[齐次线性方程组的解的结构定理]\label{theorem:线性方程组.齐次线性方程组的解的结构定理}
数域\(K\)上\(n\)元齐次线性方程组\(\vb{A}\vb{X}=\vb0\)的通解为\begin{equation}\label{equation:线性方程组.齐次线性方程组的通解}
	\vb{X}=(\vb{E}_n - \vb{A}^- \vb{A}) \vb{Z},
\end{equation}
其中\(\vb{A}^-\)是\(\vb{A}\)的任意给定的一个广义逆,
\(\vb{Z}\)取遍\(K^n\)中任意列向量.
\end{theorem}

\begin{corollary}\label{theorem:线性方程组.齐次线性方程组的解的结构定理.推论1}
设数域\(K\)上\(n\)元非齐次线性方程组\(\vb{A} \vb{X} = \vb\beta\)有解,则它的通解为\begin{equation}\label{equation:线性方程组.非齐次线性方程组的通解2}
	\vb{X} = \vb{A}^- \vb\beta + (\vb{E}_n - \vb{A}^- \vb{A}) \vb{Z},
\end{equation}
其中\(\vb{A}^-\)是\(\vb{A}\)的任意给定的一个广义逆,\(\vb{Z}\)取遍\(K^n\)中任意列向量.
\end{corollary}

一般情况下,矩阵方程\(\vb{A} \vb{X} \vb{A} = \vb{A}\)的解不唯一,从而\(\vb{A}\)的广义逆不唯一.
但是我们有时候希望\(\vb{A}\)的满足特殊条件的广义逆是唯一的,这就引出以下概念:
\begin{definition}
设\(\vb{A} \in M_{s \times n}(\mathbb{C})\).
关于矩阵\(\vb{X}\)的矩阵方程组\begin{equation}\label{equation:线性方程组.彭罗斯方程组}
	\begin{cases}
		\vb{A} \vb{X} \vb{A} = \vb{A}, \\
		\vb{X}\vb{A}\vb{X}=\vb{X}, \\
		(\vb{A}\vb{X})^H = \vb{A}\vb{X}, \\
		(\vb{X}\vb{A})^H = \vb{X}\vb{A}
	\end{cases}
\end{equation}
称为\(\vb{A}\)的\DefineConcept{彭罗斯方程组},
它的解称为\(\vb{A}\)的\DefineConcept{穆尔--彭罗斯广义逆},记作\(\vb{A}^+\).
%@see: https://mathworld.wolfram.com/Moore-PenroseMatrixInverse.html
\end{definition}

\begin{theorem}[穆尔--彭罗斯广义逆的唯一性]\label{theorem:线性方程组.穆尔--彭罗斯广义逆的唯一性}
如果\(\vb{A} \in M_{s \times n}(\mathbb{C})\),则\(\vb{A}\)的彭罗斯方程组 \labelcref{equation:线性方程组.彭罗斯方程组} 总是有解,并且它的解唯一.

设\(\vb{A}=\vb{B}\vb{C}\),其中\(\vb{B}\)、\(\vb{C}\)分别是列满秩矩阵、行满秩矩阵,则方程组 \labelcref{equation:线性方程组.彭罗斯方程组} 的唯一解是
\begin{equation}\label{equation:线性方程组.彭罗斯方程组的唯一解}
	\vb{X} = \vb{C}^H (\vb{C} \vb{C}^H)^{-1} (\vb{B}^H \vb{B})^{-1} \vb{B}^H.
\end{equation}
\begin{proof}
首先考虑\(\vb{A}\neq\vb0\).
把\cref{equation:线性方程组.彭罗斯方程组的唯一解}
代入彭罗斯方程组 \labelcref{equation:线性方程组.彭罗斯方程组} 的每一个方程,
不难验证每一个方程都将变成恒等式\footnote{由于\(\rank(\vb{C}\vb{C}^H)=\rank\vb{C}=r\),
所以\(\vb{C}\vb{C}^H\)是\(r\)阶满秩矩阵,可逆;
同理\(\vb{B}^H\vb{B}\)也可逆.},
由此可知\cref{equation:线性方程组.彭罗斯方程组的唯一解} 的确是
彭罗斯方程组 \labelcref{equation:线性方程组.彭罗斯方程组} 的解.

要证明这种广义逆的唯一性,
先设\(\vb{X}_1\)和\(\vb{X}_2\)都是
彭罗斯方程组 \labelcref{equation:线性方程组.彭罗斯方程组} 的解,
则\begin{align*}
	\vb{X}_1
	&= \vb{X}_1\vb{A}\vb{X}_1
	= \vb{X}_1(\vb{A}\vb{X}_2\vb{A})\vb{X}_1
	= \vb{X}_1(\vb{A}\vb{X}_2)(\vb{A}\vb{X}_1)
	= \vb{X}_1(\vb{A}\vb{X}_2)^H(\vb{A}\vb{X}_1)^H \\
	&= \vb{X}_1(\vb{A}\vb{X}_1\vb{A}\vb{X}_2)^H
	= \vb{X}_1(\vb{A}\vb{X}_2)^H
	= \vb{X}_1\vb{A}\vb{X}_2
	= \vb{X}_1(\vb{A}\vb{X}_2\vb{A})\vb{X}_2 \\
	&= (\vb{X}_1\vb{A})(\vb{X}_2\vb{A})\vb{X}_2
	= (\vb{X}_1\vb{A})^H(\vb{X}_2\vb{A})^H\vb{X}_2
	= (\vb{X}_2\vb{A}\vb{X}_1\vb{A})^H \vb{X}_2 \\
	&= (\vb{X}_2\vb{A})^H \vb{X}_2
	= \vb{X}_2\vb{A}\vb{X}_2
	= \vb{X}_2.
\end{align*}
这就证明了彭罗斯方程组 \labelcref{equation:线性方程组.彭罗斯方程组} 的解的唯一性.

现在考虑\(\vb{A}=\vb0\).
设\(\vb{X}_0\)是零矩阵\(\vb0\)的穆尔--彭罗斯广义逆,
则\[
	\vb{X}_0 = \vb{X}_0 \vb0 \vb{X}_0 = \vb0.
\]
显然\(\vb0\)是零矩阵的彭罗斯方程组的解,因此零矩阵的穆尔--彭罗斯广义逆是零矩阵本身.

综上,对任意复矩阵\(\vb{A}\),它的穆尔--彭罗斯广义逆存在且唯一.
\end{proof}
\end{theorem}

\section{本章总结}
\begin{gather*}
	%\cref{theorem:线性方程组.矩阵乘积的秩}
	\rank(\vb{A}\vb{B}) \leq \min\{\rank\vb{A},\rank\vb{B}\}. \\
	%\cref{theorem:矩阵乘积的秩.与可逆矩阵相乘不变秩}
	\text{$\vb{P}$可逆} \implies \rank\vb{A} = \rank(\vb{P}\vb{A}). \\
	\text{$\vb{Q}$可逆} \implies \rank\vb{A} = \rank(\vb{A}\vb{Q}). \\
	%\cref{theorem:矩阵乘积的秩.多行少列矩阵与少行多列矩阵的乘积的行列式}
	\vb{A} \in M_{m \times n}(K),
	\vb{B} \in M_{n \times m}(K),
	m>n
	\implies
	\abs{\vb{A}\vb{B}} = 0. \\
	%\cref{example:矩阵乘积的秩.可交换矩阵之和的秩}
	\vb{A}\vb{B}=\vb{B}\vb{A}
	\implies
	\rank(\vb{A}+\vb{B})\leq\rank\vb{A}+\rank\vb{B}-\rank(\vb{A}\vb{B}). \\
	%\cref{example:矩阵乘积的秩.任意同型矩阵之和的秩}
	\rank(\vb{A}+\vb{B}) \leq \rank\vb{A} + \rank\vb{B}. \\
	%\cref{example:矩阵乘积的秩.分块矩阵的秩的等式2}
	\max\{\rank\vb{A},\rank\vb{B}\} \leq \rank(\vb{A},\vb{B}) \leq \rank\vb{A} + \rank\vb{B}. \\
	%\cref{theorem:西尔维斯特不等式.分块矩阵的秩的等式3}
	\rank\begin{bmatrix}
		\vb{A} \\
		\vb{C} \vb{A}
	\end{bmatrix}
	= \rank\vb{A}. \\
	\rank(\vb{A},\vb{A} \vb{B})
	= \rank\vb{A}. \\
	%\cref{example:矩阵乘积的秩.矩阵的一次多项式的秩之和}
	\rank(\vb{A} + \vb{E}) + \rank(\vb{A} - \vb{E}) \geq n. \\
	%\cref{equation:线性方程组.西尔维斯特不等式}
	\rank\vb{A} + \rank\vb{B} - n \leq \rank(\vb{A}\vb{B}), \\
	%\cref{equation:线性方程组.弗罗贝尼乌斯不等式}
	\rank(\vb{A}\vb{B}\vb{C}) \geq \rank(\vb{A}\vb{B}) + \rank(\vb{B}\vb{C}) - \rank\vb{B}, \\
	%\cref{equation:伴随矩阵.伴随矩阵的秩}
	\rank\vb{A}^* = \left\{ \begin{array}{cl}
		n, & \rank\vb{A}=n, \\
		1, & \rank\vb{A}=n-1, \\
		0, & \rank\vb{A}<n-1,
	\end{array} \right. \\
	%\cref{equation:伴随矩阵.伴随矩阵的行列式}
	\abs{\vb{A}^*} = \abs{\vb{A}}^{n-1}, \\
	%\cref{equation:伴随矩阵.伴随矩阵的伴随}
	(\vb{A}^*)^* = \left\{ \begin{array}{cl}
		\abs{\vb{A}}^{n-2} \vb{A}, & n\geq3, \\
		\vb{A}, & n=2.
	\end{array} \right. \\
	%\cref{equation:矩阵乘积的秩.实矩阵及其转置矩阵的乘积的秩}
	\vb{A} \in M_{s \times n}(\mathbb{R})
	\implies
	\rank\vb{A} = \rank(\vb{A}\vb{A}^T) = \rank(\vb{A}^T\vb{A}).
\end{gather*}

% \begin{landscape}
\begin{table}[htb]
	\centering
	\begin{tblr}{cp{11cm}}
		\hline
		几何语言 & \SetCell{c} 代数语言 \\
		\hline
		%\cref{theorem:解析几何.两向量共线的充分必要条件1}
		向量\(\vb{a}\)与\(\vb{b}\)共线 & \(\vb{a},\vb{b}\)线性相关 \\
		%\cref{theorem:解析几何.两向量不共线的充分必要条件1}
		向量\(\vb{a}\)与\(\vb{b}\)不共线 & \(\vb{a},\vb{b}\)线性无关 \\
		%\cref{theorem:解析几何.三向量共面的充分必要条件1}
		向量\(\vb{a},\vb{b},\vb{c}\)共面 & \(\vb{a},\vb{b},\vb{c}\)线性相关 \\
		%\cref{theorem:解析几何.三向量不共面的充分必要条件1}
		向量\(\vb{a},\vb{b},\vb{c}\)不共面 & \(\vb{a},\vb{b},\vb{c}\)线性无关 \\
		%\cref{theorem:解析几何.点在线段上的充分必要条件1}
		点\(M\)在线段\(AB\)上 &
		存在非负实数\(\lambda,\mu\),
		使得对于任意一点\(P\),\newline
		总有\(\lambda+\mu=1\),
		且\(\vec{PM} = \lambda \vec{PA} + \mu \vec{PB}\) \\
		%\cref{theorem:解析几何.点在直线上的充分必要条件1}
		点\(M\)在直线\(AB\)上 &
		存在实数\(\lambda,\mu\),
		使得对于任意一点\(P\),\newline
		总有\(\lambda+\mu=1\),
		且\(\vec{PM} = \lambda \vec{PA} + \mu \vec{PB}\) \\
		%\cref{theorem:解析几何.三点共线的充分必要条件1}
		三点\(A,B,C\)共线 &
		存在不全为零的实数\(\lambda,\mu,\nu\),
		使得对于任意一点\(P\),\newline
		总有\(\lambda+\mu+\nu=0\),
		且\(\lambda \vec{PA} + \mu \vec{PB} + \nu \vec{PC} = \vb{0}\) \\
		%\cref{theorem:解析几何.四点共面的充分必要条件1}
		四点\(A,B,C,D\)共面 &
		存在不全为零的实数\(\lambda,\mu,\nu,\omega\),
		使得对于任意一点\(P\),\newline
		总有\(\lambda+\mu+\nu+\omega=0\),
		且\(\lambda \vec{PA} + \mu \vec{PB} + \nu \vec{PC} + \omega \vec{PD} = \vb{0}\) \\
		%\cref{theorem:解析几何.点在平面上的充分必要条件1}
		点\(M\)在平面\(ABC\)上 &
		存在实数\(\lambda,\mu,\nu\),
		使得对于任意一点\(P\),\newline
		总有\(\lambda+\mu+\nu=1\),
		且\(\vec{PM} = \lambda \vec{PA} + \mu \vec{PB} + \nu \vec{PC}\) \\
		%\cref{theorem:解析几何.点在三角形上的充分必要条件2}
		点\(M\)在\(\triangle ABC\)上 &
		存在非负实数\(\lambda,\mu,\nu\),
		使得对于任意一点\(P\),\newline
		总有\(\lambda+\mu+\nu=1\),
		且\(\vec{PM} = \lambda \vec{PA} + \mu \vec{PB} + \nu \vec{PC}\) \\
		\hline
	\end{tblr}
	\caption{线性代数在空间解析几何中的应用}
\end{table}
% \end{landscape}


\chapter{矩阵的特征值与特征向量}
\section{矩阵的迹}
\begin{definition}
矩阵\(\vb{A}=(a_{ij})_{s \times n}\)
主对角线上元素之和称为\(\vb{A}\)的\DefineConcept{迹}(trace),
%@see: https://mathworld.wolfram.com/MatrixTrace.html
记作\(\tr\vb{A}\),
即\begin{equation*}
	\tr\vb{A}
	\defeq
	\sum_{i=1}^m a_{ii},
\end{equation*}
其中\(m = \min\{s,n\}\).
\end{definition}

\begin{property}\label{theorem:矩阵的迹.性质1}
已知矩阵\(\vb{A},\vb{B} \in M_{s \times n}(K)\),
则\begin{gather}
	%@see: 《高等代数(第三版 上册)》(丘维声) P170 (2)
	\tr(\vb{A}+\vb{B}) = \tr\vb{A} + \tr\vb{B}, \\
	%@see: 《高等代数(第三版 上册)》(丘维声) P170 (3)
	(\forall k \in K)[\tr(k \vb{A}) = k \tr\vb{A}].
\end{gather}
\begin{proof}
设\(\vb{A}=(a_{ij})_{s \times n},
\vb{B}=(b_{ij})_{s \times n}\),
取\(m = \min\{s,n\}\),
那么\begin{equation*}
	\tr(\vb{A}+\vb{B}) = \sum_{i=1}^m (a_{ii}+b_{ii})
	= \sum_{i=1}^m a_{ii}
	+ \sum_{i=1}^m b_{ii}
	= \tr\vb{A} + \tr\vb{B},
\end{equation*}\begin{equation*}
	\tr(k \vb{A}) = \sum_{i=1}^m (k a_{ii})
	= k \sum_{i=1}^m a_{ii}
	= k \tr\vb{A}.
	\qedhere
\end{equation*}
\end{proof}
\end{property}
\begin{remark}
\cref{theorem:矩阵的迹.性质1} 说明:
矩阵的迹具有“线性性”.
\end{remark}

\begin{property}\label{theorem:矩阵的迹.矩阵转置不变迹}
已知矩阵\(\vb{A} \in M_{s \times n}(K)\),
则\begin{equation}
	\tr\vb{A} = \tr(\vb{A}^T).
\end{equation}
%TODO proof
\end{property}

\begin{property}\label{theorem:矩阵的迹.矩阵乘积交换次序不变迹}
%@see: 《高等代数(第三版 上册)》(丘维声) P170 (4)
已知矩阵\(\vb{A},\vb{B} \in M_n(K)\),
则\begin{equation}
	\tr(\vb{A} \vb{B}) = \tr(\vb{B} \vb{A}).
\end{equation}
\begin{proof}
设\(\vb{A} = (a_{ij})_n,
\vb{B} = (b_{ij})_n\),
则\begin{gather*}
	\tr(\vb{A} \vb{B})
	= \sum_{i=1}^n (\vb{A} \vb{B})(i,i)
	= \sum_{i=1}^n \sum_{k=1}^n a_{ik} b_{ki}, \\
	\tr(\vb{B} \vb{A})
	= \sum_{k=1}^n (\vb{B} \vb{A})(k,k)
	= \sum_{k=1}^n \sum_{i=1}^n b_{ki} a_{ik},
\end{gather*}
利用加法结合律可得\begin{equation*}
	\sum_{i=1}^n \sum_{k=1}^n a_{ik} b_{ki}
	= \sum_{k=1}^n \sum_{i=1}^n b_{ki} a_{ik},
\end{equation*}
于是\(\tr(\vb{A} \vb{B}) = \tr(\vb{B} \vb{A})\).
\end{proof}
\end{property}

\begin{example}
举例说明:实矩阵\(\vb{A}\)满足\(\tr(\vb{A}^2)<0\).
\begin{solution}
取\(\vb{A} = \begin{bmatrix}
	0 & 1 \\
	-1 & 0
\end{bmatrix}\),
则\(\vb{A}^2 = \begin{bmatrix}
	-1 & 0 \\
	0 & -1
\end{bmatrix}\),
从而有\(\tr(\vb{A}^2) = -2 < 0\).
\end{solution}
%@Mathematica: Tr[MatrixPower[{{0, 1}, {-1, 0}}, 2]]
\end{example}

\begin{example}
%@see: 《高等代数(第三版 上册)》(丘维声) P171 习题5.4 9.
证明:如果数域\(K\)上的\(n\)阶矩阵\(\vb{A},\vb{B}\)满足\begin{equation*}
	\vb{A} \vb{B} - \vb{B} \vb{A}=\vb{A},
\end{equation*}
则\(\vb{A}\)不可逆.
\begin{proof}
用反证法.
假设\(\vb{A}\)可逆,\(\vb{E}\)是数域\(K\)上的\(n\)阶单位矩阵,
那么\begin{gather*}
	\vb{E} = \vb{A}\vb{A}^{-1}
	= (\vb{A} \vb{B} - \vb{B} \vb{A})\vb{A}^{-1} % 把\(\vb{A} \vb{B} - \vb{B} \vb{A}\)代入\(\vb{A}\)
	= \vb{A} \vb{B} \vb{A}^{-1}-\vb{B},
\end{gather*}
从而有\(\tr(\vb{A} \vb{B} \vb{A}^{-1}-\vb{B}) = \tr\vb{E}\),
但是\begin{align*}
	\tr(\vb{A} \vb{B} \vb{A}^{-1}-\vb{B})
	&= \tr(\vb{A} \vb{B} \vb{A}^{-1})-\tr\vb{B}
		\tag{\cref{theorem:矩阵的迹.性质1}} \\
	&= \tr(\vb{A}^{-1}(\vb{A} \vb{B}))-\tr\vb{B}
		\tag{\cref{theorem:矩阵的迹.矩阵乘积交换次序不变迹}} \\
	&= \tr\vb{B}-\tr\vb{B}
	= 0
	< n = \tr\vb{E},
\end{align*}
所以\(\vb{A}\)不可逆.
\end{proof}
\end{example}

\begin{property}
设\(\vb{A}\)是可逆矩阵,
则\begin{equation}
	\tr(\vb{A}^*) = \abs{\vb{A}}~\tr(\vb{A}^{-1}).
\end{equation}
\begin{proof}
由\cref{theorem:逆矩阵.逆矩阵的唯一性} 可知,
\(\vb{A}^* = \abs{\vb{A}}~\vb{A}^{-1}\).
于是由\cref{theorem:矩阵的迹.性质1} 可知\begin{equation*}
	\tr(\vb{A}^*) = \tr(\abs{\vb{A}}~\vb{A}^{-1}) = \abs{\vb{A}}~\tr(\vb{A}^{-1}).
	\qedhere
\end{equation*}
\end{proof}
\end{property}

\begin{property}
已知矩阵\(\vb{A} \in M_{s \times n}(K)\),
则\begin{equation}
	\tr(\vb{A}\vb{A}^T) = \tr(\vb{A}^T\vb{A}).
\end{equation}
\begin{proof}
在\cref{theorem:矩阵的迹.矩阵乘积交换次序不变迹} 中,用\(\vb{A}^T\)代\(\vb{B}\)便得.
\end{proof}
\end{property}

\begin{property}
已知矩阵\(\vb{A},\vb{B} \in M_n(K)\),
且\(\vb{A},\vb{B}\)均为实对称矩阵,
则\begin{equation}
	\tr(\vb{A} \vb{B})^2 \leq \tr(\vb{A}^2 \vb{B}^2).
\end{equation}
%TODO proof
% \begin{proof}
% %@see: https://www.bilibili.com/video/BV1s12RY9EMx/
% 因为\(\vb{A},\vb{B}\)均是实对称矩阵,
% 所以\begin{equation*}
% 	\vb{A} = \vb{A}^T,
% 	\qquad
% 	\vb{B} = \vb{B}^T.
% \end{equation*}
% 因为\hyperref[theorem:矩阵的迹.矩阵乘积交换次序不变迹]{矩阵乘积交换次序不变迹},
% 所以\begin{equation*}
% 	\tr(\vb{A}^2 \vb{B}^2)
% 	= \tr[(\vb{A} \vb{A} \vb{B}) \vb{B}]
% 	= \tr(\vb{B} \vb{A} \vb{A} \vb{B}).
% 	\qquad
% 	\tr(\vb{B} \vb{A})
% 	= \tr(\vb{A} \vb{B}).
% \end{equation*}
% % 因为\hyperref[theorem:矩阵的迹.矩阵转置不变迹]{矩阵转置不变迹},
% 所以\begin{equation*}
% \end{equation*}
% 由\hyperref[theorem:矩阵的迹.性质1]{迹的线性性}可知,
% 要证\(\tr(\vb{A} \vb{B})^2 \leq \tr(\vb{A}^2 \vb{B}^2)\),
% 即证\(\tr(\vb{A} \vb{A} \vb{B} \vb{B} - \vb{A} \vb{B} \vb{A} \vb{B}) \geq 0\)
% \end{proof}
\end{property}

\begin{example}
设\(\vb{A}\)是数域\(K\)上的\(n\)阶对称矩阵,
\(\vb{B}\)是数域\(K\)上的\(n\)阶反对称矩阵.
证明:\begin{equation}
	\tr(\vb{A} \vb{B}) = 0.
\end{equation}
\begin{proof}
设\(\vb{A}\vb{B}\)的\((i,j)\)元素为\(c_{ij}\),
\(\vb{B}\vb{A}\)的\((i,j)\)元素为\(d_{ij}\),
即\begin{equation*}
	c_{ij} = \sum_{k=1}^n a_{ik} b_{kj},
	\qquad
	d_{ij} = \sum_{k=1}^n b_{ik} a_{kj}.
\end{equation*}
那么由迹的定义有\begin{align*}
	\tr(\vb{A}\vb{B})
	&= \sum_{i=1}^n c_{ii}
	= \sum_{i=1}^n \sum_{j=1}^n a_{ij} b_{ji}, \\
	\tr(\vb{B}\vb{A})
	&= \sum_{i=1}^n d_{ii}
	= \sum_{i=1}^n \sum_{j=1}^n b_{ij} a_{ji}.
\end{align*}
相加得\begin{equation*}
	\tr(\vb{A}\vb{B}) + \tr(\vb{B}\vb{A})
	= \sum_{i=1}^n \sum_{j=1}^n (a_{ij} b_{ji} + b_{ij} a_{ji}).
	\eqno(1)
\end{equation*}
因为\(\vb{A}\)是对称矩阵,所以\(a_{ij} = a_{ji}\).
因为\(\vb{B}\)是反对称矩阵,所以\(b_{ij} = -b_{ji}\).
那么(1)式化为\begin{equation*}
	\tr(\vb{A}\vb{B}) + \tr(\vb{B}\vb{A})
	= \sum_{i=1}^n \sum_{j=1}^n (a_{ij} b_{ji} - b_{ji} a_{ij})
	= 0.
\end{equation*}
因为\(\tr(\vb{A}\vb{B}) = \tr(\vb{B}\vb{A})\),
所以\(\tr(\vb{A}\vb{B}) = 0\).
\end{proof}
\end{example}

\section{矩阵的特征值与特征向量}
首先看一个实际例子.
\begin{example}
%@see: 《线性代数》(张慎语、周厚隆) P91 例1
设点\(M\)在新、旧坐标系的坐标分别为\((x',y',z')^T\)与\((x,y,z)^T\),
则在旋转变换下有\[
	\left\{ \begin{array}{l}
		x' = l_1 x + l_2 y + l_3 z, \\
		y' = m_1 x + m_2 y + m_3 z, \\
		z' = n_1 x + n_2 y + n_3 z
	\end{array} \right.
	\quad\text{或}\quad
	\begin{bmatrix}
		x' \\ y' \\ z'
	\end{bmatrix}
	= \begin{bmatrix}
		l_1 & l_2 & l_3 \\
		m_1 & m_2 & m_3 \\
		n_1 & n_2 & n_3
	\end{bmatrix}
	\begin{bmatrix}
		x \\ y \\ z
	\end{bmatrix},
\]
记\[
	\vb{Y} \defeq \begin{bmatrix}
		x' \\ y' \\ z'
	\end{bmatrix},
	\qquad
	\vb{A} \defeq \begin{bmatrix}
		l_1 & l_2 & l_3 \\
		m_1 & m_2 & m_3 \\
		n_1 & n_2 & n_3
	\end{bmatrix},
	\qquad
	\vb{X} \defeq \begin{bmatrix}
		x \\ y \\ z
	\end{bmatrix},
\]
于是又有\[
	\vb{Y}=\vb{A}\vb{X},
\]
这里\(\vb{A}\)的第一、二、三列分别是坐标轴\(Ox,Oy,Oz\)在\(Ox'y'z'\)的方向余弦,
以后我们会证明\(\vb{A}\)是一个正交矩阵.
可以注意到,如果点\(M\)在\(z\)轴上,当它绕\(z\)轴旋转时,
点\(M\)保持不动,也就是说\(\vb{A}\vb{X}=1\vb{X}\).
\end{example}
从这个例子可以看出,有的时候,一个向量左乘一个矩阵得到的结果,就跟它乘以一个标量一样.

\subsection{特征值,特征向量}
\begin{definition}
%@see: 《高等代数(第三版 上册)》(丘维声) P172 定义1
%@see: 《线性代数》(张慎语、周厚隆) P92 定义1
设矩阵\(\A \in M_n(K)\).
如果\[
	(\exists \lambda \in K)
	(\exists \x \in K^n - \{\vb0\})
	[\A\x = \lambda\x],
\]
则称“数\(\lambda\)是矩阵\(\A\)的一个\DefineConcept{特征值}(eigenvalue)”
%@see: https://mathworld.wolfram.com/Eigenvalue.html
“向量\(\x\)是矩阵\(\A\)的(属于特征值\(\lambda\)的)一个\DefineConcept{特征向量}(eigenvector)”.
%@see: https://mathworld.wolfram.com/Eigenvector.html
\end{definition}
\begin{remark}
零向量不是任何一个矩阵的特征向量.
\end{remark}
\begin{example}
设\[
	\vb{A} = \begin{bmatrix}
		1 & 1 \\ 1 & 1
	\end{bmatrix},
	\qquad
	\vb\alpha = \begin{bmatrix}
		1 \\ 1
	\end{bmatrix}.
\]
由于\[
	\vb{A} \vb\alpha
	= \begin{bmatrix}
		1 & 1 \\ 1 & 1
	\end{bmatrix}
	\begin{bmatrix}
		1 \\ 1
	\end{bmatrix}
	= \begin{bmatrix}
		2 \\ 2
	\end{bmatrix}
	= 2 \begin{bmatrix}
		1 \\ 1
	\end{bmatrix}
	= 2 \vb\alpha,
\]
所以\(2\)是\(\vb{A}\)的一个特征值,
\(\vb\alpha\)是\(\vb{A}\)的属于特征值\(2\)的一个特征向量.
\end{example}
\begin{example}
设\(\vb{E}\)是数域\(K\)上的\(n\)阶单位矩阵,
那么对于\(K^n\)中的任意一个非零向量\(\vb{x}\),
都成立\[
	\vb{E} \vb{x} = 1 \vb{x},
\]
也就是说,\(1\)是单位矩阵的唯一一个特征值,
任意一个非零向量都是单位矩阵的特征向量.
\end{example}

\begin{example}
%@see: 《高等代数(第三版 上册)》(丘维声) P172
根据\cref{equation:解析几何.平面坐标系的点的右手直角坐标变换公式I到II.矩阵形式1},
在平面上绕原点\(O\)的转角为\(\pi/3\)的旋转变换可以表示为实数域上的2阶矩阵\[
	\vb{A}
	= \begin{bmatrix}
		\cos(\pi/3) & -\sin(\pi/3) \\
		\sin(\pi/3) & \cos(\pi/3)
	\end{bmatrix}
	= \begin{bmatrix}
		1/2 & -\sqrt3/2 \\
		\sqrt3/2 & 1/2
	\end{bmatrix}.
\]
显然平面上任一非零向量在旋转一次后都不会变成它的倍数,
因此在\(\mathbb{R}^2\)中不存在非零列向量\(\vb\alpha\)
满足\(\vb{A}\vb\alpha=\lambda\vb\alpha\),
也就是说\(\vb{A}\)没有特征值、特征向量.

这个例子说明:不是所有数域上的矩阵都有特征值、特征向量.
但是我们只要把这个矩阵看成是复数域上的2阶矩阵,
就会发现\(\frac12(1\pm\iu\sqrt3)\)是它的两个特征值.
%@Mathematica: Eigenvalues[{{Cos[Pi/3], -Sin[Pi/3]}, {Sin[Pi/3], Cos[Pi/3]}}]
%@Mathematica: Eigenvectors[{{Cos[Pi/3], -Sin[Pi/3]}, {Sin[Pi/3], Cos[Pi/3]}}]
我们将在下面证明:复数域上的矩阵必定存在特征值、特征向量.
\end{example}

\begin{example}\label{example:特征值与特征向量.各行元素之和相同的矩阵的特征值与特征向量}
设矩阵\(\A \in M_n(K)\),
\(\A\)的各行元素之和均为\(\lambda_0\).
证明:\(\lambda_0\)是矩阵\(\A\)的特征值,
\(n\)维列向量\(\vb{x}_0=(1,\dotsc,1)^T\)是矩阵\(\A\)属于\(\lambda_0\)的特征向量.
\begin{proof}
记\[
	\A
	= \begin{bmatrix}
		a_{11} & a_{12} & \dots & a_{1n} \\
		a_{21} & a_{22} & \dots & a_{2n} \\
		\vdots & \vdots & & \vdots \\
		a_{n1} & a_{n2} & \dots & a_{nn}
	\end{bmatrix},
\]
那么\[
	\A \vb{x}_0
	= \begin{bmatrix}
		a_{11} & a_{12} & \dots & a_{1n} \\
		a_{21} & a_{22} & \dots & a_{2n} \\
		\vdots & \vdots & & \vdots \\
		a_{n1} & a_{n2} & \dots & a_{nn}
	\end{bmatrix}
	\begin{bmatrix}
		1 \\ 1 \\ \vdots \\ 1
	\end{bmatrix}
	= \begin{bmatrix}
		a_{11} + a_{12} + \dotsb + a_{1n} \\
		a_{21} + a_{22} + \dotsb + a_{2n} \\
		\vdots \\
		a_{n1} + a_{n2} + \dotsb + a_{nn}
	\end{bmatrix}
	= \begin{bmatrix}
		\lambda_0 \\ \lambda_0 \\ \vdots \\ \lambda_0
	\end{bmatrix}
	= \lambda_0
	\begin{bmatrix}
		1 \\ 1 \\ \vdots \\ 1
	\end{bmatrix},
\]
由此可见,\(\lambda_0\)是\(\A\)的特征值,
\(n\)维列向量\(\vb{x}_0=(1,\dotsc,1)^T\)是\(\A\)属于\(\lambda_0\)的特征向量.
\end{proof}
\end{example}

\begin{example}\label{example:矩阵的特征值与特征向量.零是奇异矩阵的特征值}
%@see: 《高等代数(第三版 上册)》(丘维声) P179 习题5.5 9.
设\(\A \in M_n(K)\).
证明:\(0\)是\(\A\)的一个特征值当且仅当\(\abs{\A}=0\).
\begin{proof}
由\cref{theorem:矩阵的特征值与特征向量.与特征多项式和特征子空间的联系} 有\begin{align*}
	&\text{\(0\)是\(\A\)的一个特征值} \\
	&\iff
	\abs{0\E-\A}
	= \abs{-\A}
	= (-1)^n \abs{\A}
	= 0 \\
	&\iff
	\abs{\A} = 0.
	\qedhere
\end{align*}
\end{proof}
\end{example}
\begin{remark}
如果\(0\)是矩阵\(\A\)的一个特征值,
那么矩阵\(\A\)的属于特征值\(0\)的每个特征向量都是齐次方程\(\A\x=\vb0\)的解,
这就说明\(\A\x=\vb0\)有非零解,
由\cref{theorem:线性方程组.齐次线性方程组有非零解的充分必要条件} 可知\(\rank\A < n\).
但是,即便一个矩阵的特征值全是零,也不能说明它的秩为零.
例如,幂零矩阵\[
	\begin{bmatrix}
		0 & 1 & 0 & 0 \\
		0 & 0 & 1 & 0 \\
		0 & 0 & 0 & 1 \\
		0 & 0 & 0 & 0
	\end{bmatrix}
\]的特征值全为零(见\cref{example:幂零矩阵.幂零矩阵的特征值的性质}),
但是它有一个三阶非零子式\[
	\begin{vmatrix}
		1 & 0 & 0 \\
		0 & 1 & 0 \\
		0 & 0 & 1
	\end{vmatrix},
\]
所以它的秩为\(3\).
要想保证一个矩阵的秩等于它的非零特征值个数,
需要验证它是否\hyperref[definition:相似对角化.相似对角化]{可以相似对角化},
这是因为\hyperref[theorem:特征值与特征向量.相似矩阵的迹的不变性]{秩是相似不变量}.
如果一个矩阵不可以相似对角化,则我们至多像上面提到的那样,
根据这个矩阵是否存在零特征值,判断它是否不满秩.
\end{remark}
\begin{example}\label{example:矩阵的特征值与特征向量.零不是非奇异矩阵的特征值}
%@see: 《高等代数(第三版 上册)》(丘维声) P179 习题5.5 8.(1)
设\(\A\)是数域\(K\)上一个\(n\)阶可逆矩阵.
证明:如果\(\A\)有特征值,则\(\A\)的特征值不等于\(0\).
\begin{proof}
由\cref{example:矩阵的特征值与特征向量.零是奇异矩阵的特征值} 可知,
\(0\)是奇异矩阵的特征值,
而可逆矩阵必定是非奇异矩阵,
所以\(\A\)的特征值只要存在就不可能等于\(0\).
\end{proof}
\end{example}

\begin{proposition}\label{theorem:矩阵的特征值与特征向量.特征子空间1}
%@see: 《线性代数》(张慎语、周厚隆) P92 性质1
设\(\vb{A} \in M_n(K)\).
如果\(\vb{x}_1,\vb{x}_2\)是\(\vb{A}\)的属于同一个特征值\(\lambda_0\)的特征向量,
且\(\vb{x}_1 + \vb{x}_2 \neq \vb0\),
则\(\vb{x}_1 + \vb{x}_2\)也是\(\vb{A}\)的属于\(\lambda_0\)的特征向量.
\begin{proof}
\(\vb{A} (\vb{x}_1 + \vb{x}_2)
= \vb{A} \vb{x}_1 + \vb{A} \vb{x}_2
= \lambda_0 \vb{x}_1 + \lambda_0 \vb{x}_2
= \lambda_0 (\vb{x}_1 + \vb{x}_2)\).
\end{proof}
\end{proposition}
\begin{proposition}\label{theorem:矩阵的特征值与特征向量.特征子空间2}
%@see: 《线性代数》(张慎语、周厚隆) P92 性质2
设\(\vb{A} \in M_n(K)\).
如果\(\vb{x}_0\)是\(\vb{A}\)的属于特征值\(\lambda_0\)的特征向量,
且\(k \in K-\{0\}\),
则\(k \vb{x}_0\)也是\(\vb{A}\)的属于\(\lambda_0\)的特征向量.
\begin{proof}
\(\vb{A} (k \vb{x}_0)
= k (\vb{A} \vb{x}_0)
= k (\lambda_0 \vb{x}_0)
= \lambda_0 (k \vb{x}_0)\).
\end{proof}
\end{proposition}
\begin{example}\label{example:矩阵的特征值与特征向量.矩阵的多项式的特征值1}
%@see: 《线性代数》(张慎语、周厚隆) P96 例6
%@see: 《高等代数(第三版 上册)》(丘维声) P180 习题5.5 11.(1)
设\(\vb{A} \in M_n(K)\).
证明:如果\(\lambda_0\)是\(\vb{A}\)的特征值,
则对于任意\(k \in K\),
都有\(k\lambda_0\)是\(k\vb{A}\)的特征值.
\begin{proof}
假设\(\vb{A} \vb{x}_0 = \lambda_0 \vb{x}_0\),
则\((k\vb{A}) \vb{x}_0
= k(\vb{A} \vb{x}_0)
= k(\lambda_0 \vb{x}_0)
= (k\lambda_0) \vb{x}_0\).
\end{proof}
\end{example}
\begin{example}\label{example:矩阵的特征值与特征向量.矩阵的多项式的特征值2}
%@see: 《线性代数》(张慎语、周厚隆) P92 性质3
%@see: 《高等代数(第三版 上册)》(丘维声) P180 习题5.5 11.(2)
设\(\vb{A} \in M_n(K)\).
证明:如果\(\lambda_0\)是\(\vb{A}\)的特征值,
则对于任意\(m\in\mathbb{N}\),
都有\(\lambda_0^m\)是\(\vb{A}^m\)的特征值.
\begin{proof}
由定义,存在\(\vb{x}_0\neq\z\),
使得\(\vb{A}\vb{x}_0 = \lambda_0\vb{x}_0\),
则\begin{gather*}
	\vb{A}^0 \vb{x}_0
	= \vb{E} \vb{x}_0
	= 1 \vb{x}_0
	= \lambda_0^0 \vb{x}_0, \\
	\vb{A}^2\vb{x}_0 = \vb{A}(\vb{A}\vb{x}_0)
	= \vb{A}(\lambda_0\vb{x}_0)
	= \lambda_0(\vb{A}\vb{x}_0)
	= \lambda_0(\lambda_0\vb{x}_0)
	= \lambda_0^2\vb{x}_0.
\end{gather*}
用数学归纳法.
假设\(\vb{A}^{m-1}\vb{x}_0 = \lambda_0^{m-1}\vb{x}_0\ (m\geq3)\)成立,
则\[
	\vb{A}^m\vb{x}_0 = \vb{A}(\vb{A}^{m-1}\vb{x}_0)
	= \vb{A}(\lambda_0^{m-1}\vb{x}_0)
	= \lambda_0^{m-1}(\vb{A}\vb{x}_0)
	= \lambda_0^{m-1}(\lambda_0\vb{x}_0)
	= \lambda_0^m\vb{x}_0.
	\qedhere
\]
\end{proof}
\end{example}
\begin{example}\label{example:矩阵的特征值与特征向量.矩阵的多项式的特征值3}
%@see: 《线性代数》(张慎语、周厚隆) P97 习题5.1 4.
%@see: 《高等代数(第三版 上册)》(丘维声) P180 习题5.5 8.(2)
设\(\vb{A}\)是数域\(K\)上的一个可逆矩阵.
证明:如果\(\lambda_0\)是\(\vb{A}\)的一个特征值,
则\(\lambda_0^{-1}\)是\(\vb{A}^{-1}\)的一个特征值.
\begin{proof}
假设非零向量\(\vb{x}_0\)满足
\(\lambda_0 \vb{x}_0 = \vb{A} \vb{x}_0\),
左乘\(\vb{A}^{-1}\)得\begin{equation*}
	\vb{A}^{-1} (\lambda_0 \vb{x}_0)
	= \vb{A}^{-1} (\vb{A} \vb{x}_0),
\end{equation*}
这里\begin{equation*}
	\vb{A}^{-1} (\lambda_0 \vb{x}_0)
	= \lambda_0 (\vb{A}^{-1} \vb{x}_0),
	\qquad
	\vb{A}^{-1} (\vb{A} \vb{x}_0)
	= (\vb{A}^{-1} \vb{A}) \vb{x}_0
	= \E \vb{x}_0
	= \vb{x}_0,
\end{equation*}
于是\(\lambda_0 (\vb{A}^{-1} \vb{x}_0) = \vb{x}_0\),
由\cref{example:矩阵的特征值与特征向量.零不是非奇异矩阵的特征值} 可知
\(\lambda_0 \neq 0\),
那么有\(\vb{A}^{-1} \vb{x}_0 = \lambda_0^{-1} \vb{x}_0\).
\end{proof}
\end{example}
\begin{example}\label{example:矩阵的特征值与特征向量.伴随矩阵的特征值}
设\(\vb{A}\)是数域\(K\)上的一个可逆矩阵.
证明:如果\(\lambda_0\)是\(\vb{A}\)的一个特征值,
\(\vb{x}_0\)是\(\vb{A}\)的属于\(\lambda_0\)的一个特征向量,
则\(\lambda_0^{-1} \abs{\vb{A}}\)是\(\vb{A}^*\)的一个特征值,
\(\vb{x}_0\)是\(\vb{A}^*\)的属于\(\lambda_0^{-1} \abs{\vb{A}}\)的一个特征向量.
\begin{proof}
% 由\cref{example:矩阵的特征值与特征向量.矩阵的多项式的特征值3} 可知
% \(\lambda_0^{-1}\)是\(\vb{A}^{-1}\)的一个特征值.
% 由\cref{example:矩阵的特征值与特征向量.矩阵的多项式的特征值1} 可知,
% 对于任意\(k \in K\),
% \(k \lambda_0^{-1}\)是\(k \vb{A}^{-1}\)的一个特征值.
% 于是由 \hyperref[theorem:逆矩阵.逆矩阵的唯一性]{\(\vb{A}^* = \abs{\vb{A}} \vb{A}^{-1}\)}
% 可知\(\lambda_0^{-1} \abs{\vb{A}}\)是\(\vb{A}^*\)的一个特征值.
在 \hyperref[equation:行列式.伴随矩阵.恒等式1]{\(\vb{A}^* \vb{A} = \abs{\vb{A}} \E\)}
等号两边右乘向量\(\vb{x}_0\),
得\begin{equation*}
	\vb{A}^* \vb{A} \vb{x}_0
	= \abs{\vb{A}} \vb{x}_0.
\end{equation*}
因为\(\vb{A} \vb{x}_0 = \lambda_0 \vb{x}_0\),
所以上式可以化为\begin{equation*}
	\vb{A}^* (\lambda_0 \vb{x}_0)
	= \abs{\vb{A}} \vb{x}_0.
\end{equation*}
由\cref{example:矩阵的特征值与特征向量.零不是非奇异矩阵的特征值} 可知
\(\lambda_0 \neq 0\),
所以上式又可以化为\begin{equation*}
	\vb{A}^* \vb{x}_0
	= \frac{\abs{\vb{A}}}{\lambda_0} \vb{x}_0,
\end{equation*}
这就说明\(\lambda_0^{-1} \abs{\vb{A}}\)是\(\vb{A}^*\)的一个特征值,
\(\vb{x}_0\)是\(\vb{A}^*\)的属于\(\lambda_0^{-1} \abs{\vb{A}}\)的一个特征向量.
\end{proof}
\end{example}
\begin{example}\label{example:矩阵的特征值与特征向量.矩阵的多项式的特征值4}
%@see: 《高等代数(第三版 上册)》(丘维声) P180 习题5.5 11.(3)
%@see: 《矩阵论》(詹兴致) P4 定理1.3(谱映射定理)
设\(\vb{A} \in M_n(K)\),
\(\lambda_0\)是\(\vb{A}\)的一个特征值,
\(f(x)\)是数域\(K\)上的一个一元多项式.
证明:\(f(\lambda_0)\)是\(f(\vb{A})\)的一个特征值.
\begin{proof}
假设\(\vb{A} \vb{x}_0 = \lambda_0 \vb{x}_0\),
那么由\cref{example:矩阵的特征值与特征向量.矩阵的多项式的特征值1,example:矩阵的特征值与特征向量.矩阵的多项式的特征值2}
可知对于任意\(a_n \in K\)和任意\(n\in\mathbb{N}\)总成立\begin{equation*}
	(a_n \vb{A}^n) \vb{x}_0
	= (a_n \lambda_0^n) \vb{x}_0.
\end{equation*}
于是\begin{align*}
	f(\vb{A}) \vb{x}_0
	&= (a_0 \E + a_1 \vb{A} + a_2 \vb{A}^2 + \dotsb + a_m \vb{A}^m) \vb{x}_0 \\
	&= a_0 \E \vb{x}_0 + a_1 \vb{A} \vb{x}_0 + a_2 \vb{A}^2 \vb{x}_0 + \dotsb + a_m \vb{A}^m \vb{x}_0 \\
	&= a_0 \vb{x}_0 + a_1 \lambda_0 \vb{x}_0 + a_2 \lambda_0^2 \vb{x}_0 + \dotsb + a_m \lambda_0^m \vb{x}_0 \\
	&= (a_0 + a_1 \lambda_0 + a_2 \lambda_0^2 + \dotsb + a_m \lambda_0^m) \vb{x}_0 \\
	&= f(\lambda_0) \vb{x}_0.
	\qedhere
\end{align*}
\end{proof}
\end{example}
\begin{example}
%@see: 《高等代数(第三版 上册)》(丘维声) P179 习题5.5 3.
设矩阵\(\A \in M_n(\mathbb{C})\)的各个元素均为实数,
\(\lambda_0\)是\(\A\)的一个特征值,
\(\vb{x}_0\)是\(\A\)的属于特征值\(\lambda_0\)的一个特征向量.
证明:\(\overline{\lambda_0}\)也是\(\A\)的一个特征值,
\(\overline{\vb{x}_0}\)是\(\A\)的属于特征值\(\overline{\lambda_0}\)的一个特征向量.
\begin{proof}
根据定义有\(\overline{\A} = \A\)和\(\A \vb{x}_0 = \lambda_0 \vb{x}_0\),
那么\[
	\A~\overline{\vb{x}_0}
	= \overline{\A~\vb{x}_0}
	= \overline{\lambda_0~\vb{x}_0}
	= \overline{\lambda_0}~\overline{\vb{x}_0}.
	\qedhere
\]
\end{proof}
\end{example}

\begin{example}\label{example:特征值与特征向量.各行元素之和相同的矩阵的伴随矩阵的特征值与特征向量}
设矩阵\(\A \in M_n(K)\),
\(\A\)的各行元素之和均为\(\lambda_0\).
证明:\(\A\)的伴随矩阵\(\A^*\)的各行元素之和等于\(\lambda_0^{-1} \abs{\A}\).
\begin{proof}
由\cref{example:特征值与特征向量.各行元素之和相同的矩阵的特征值与特征向量} 可知,
\(\lambda_0\)是矩阵\(\A\)的特征值,
\(n\)维列向量\(\vb{x}_0=(1,\dotsc,1)^T\)是矩阵\(\A\)属于\(\lambda_0\)的特征向量.
那么由\cref{example:矩阵的特征值与特征向量.伴随矩阵的特征值} 可知,
\(\lambda_0^{-1} \abs{\vb{A}}\)是\(\vb{A}^*\)的一个特征值,
\(\vb{x}_0\)是\(\vb{A}^*\)的属于\(\lambda_0^{-1} \abs{\vb{A}}\)的一个特征向量,
即\begin{equation*}
	\A^*
	\begin{bmatrix}
		1 \\ 1 \\ \vdots \\ 1
	\end{bmatrix}
	= \frac{\abs{\vb{A}}}{\lambda_0}
	\begin{bmatrix}
		1 \\ 1 \\ \vdots \\ 1
	\end{bmatrix}.
\end{equation*}
又因为\begin{equation*}
	\A^*
	\begin{bmatrix}
		1 \\ 1 \\ \vdots \\ 1
	\end{bmatrix}
	= \begin{bmatrix}
		A_{11} & A_{21} & \dots & A_{n1} \\
		A_{12} & A_{22} & \dots & A_{n2} \\
		\vdots & \vdots & & \vdots \\
		A_{1n} & A_{2n} & \dots & A_{nn}
	\end{bmatrix}
	\begin{bmatrix}
		1 \\ 1 \\ \vdots \\ 1
	\end{bmatrix}
	= \begin{bmatrix}
		A_{11} + A_{21} + \dotsb + A_{n1} \\
		A_{12} + A_{22} + \dotsb + A_{n2} \\
		\vdots \\
		A_{1n} + A_{2n} + \dotsb + A_{nn}
	\end{bmatrix},
\end{equation*}
所以\begin{equation*}
	\begin{bmatrix}
		A_{11} + A_{21} + \dotsb + A_{n1} \\
		A_{12} + A_{22} + \dotsb + A_{n2} \\
		\vdots \\
		A_{1n} + A_{2n} + \dotsb + A_{nn}
	\end{bmatrix}
	= \frac{\abs{\vb{A}}}{\lambda_0}
	\begin{bmatrix}
		1 \\ 1 \\ \vdots \\ 1
	\end{bmatrix}
	= \begin{bmatrix}
		\lambda_0^{-1} \abs{\vb{A}} \\ \lambda_0^{-1} \abs{\vb{A}} \\ \vdots \\ \lambda_0^{-1} \abs{\vb{A}}
	\end{bmatrix}.
	\qedhere
\end{equation*}
\end{proof}
\end{example}

\subsection{特征矩阵,特征多项式,特征方程}
\begin{proposition}
设\(\A \in M_n(K)\),
\(\AutoTuple{\lambda}{n}\)都是\(\A\)的特征值,
\(\AutoTuple{\vb{x}}{n}\)分别是\(\A\)的属于特征值\(\AutoTuple{\lambda}{n}\)的特征向量.
记\(\vb{X} = (\AutoTuple{\vb{x}}{n})\),
则\[
	\A \vb{X} = \vb{X} \vb{\Lambda},
\]
其中\(\vb{\Lambda} = \diag(\AutoTuple{\lambda}{n})\).
\end{proposition}

\begin{definition}
%@see: 《线性代数》(张慎语、周厚隆) P93 定义2
%@see: 《高等代数(第三版 上册)》(丘维声) P173
设矩阵\(\A \in M_n(K)\),\(\lambda \in K\),
\(\E\)是数域\(K\)上的\(n\)阶单位矩阵.
把\[
	\lambda\E-\A
\]称为“\(\A\)的属于特征值\(\lambda\)的\DefineConcept{特征矩阵}”.
把\(\A\)的属于特征值\(\lambda\)的特征矩阵的行列式\[
	\abs{\lambda\E-\A}
\]称为“\(\A\)的属于特征值\(\lambda\)的\DefineConcept{特征多项式}(characteristic polynomial)”.
%@see: https://mathworld.wolfram.com/CharacteristicPolynomial.html
把关于\(\lambda\)的方程\[
	\abs{\lambda\E-\A}=0
\]称为“\(\A\)的属于特征值\(\lambda\)的\DefineConcept{特征方程}(characteristic equation)”.
%@see: https://mathworld.wolfram.com/CharacteristicEquation.html
\end{definition}

\begin{theorem}\label{theorem:矩阵的特征值与特征向量.与特征多项式和特征子空间的联系}
%@see: 《高等代数(第三版 上册)》(丘维声) P174 定理1
%@see: 《线性代数》(张慎语、周厚隆) P93 性质4
设矩阵\(\A \in M_n(K)\),\(\lambda_0 \in K\),
\(\E\)是数域\(K\)上的\(n\)阶单位矩阵,
则\begin{align*}
	&\text{\(\lambda_0\)是矩阵\(\A\)的一个特征值} \\
	&\iff \text{\(\lambda_0\)是矩阵\(\A\)的特征多项式\(\abs{\lambda\E-\A}\)在\(K\)中的一个根} \\
	&\iff \abs{\lambda_0 \E - \A} = 0, \\
	&\text{\(\vb{x}_0\)是矩阵\(\A\)的属于特征值\(\lambda_0\)的一个特征向量} \\
	&\iff \text{\(\vb{x}_0\)是齐次线性方程组\((\lambda_0\E-\A)\x=\vb0\)的一个非零解} \\
	&\iff \vb{x}_0 \in \Ker(\lambda_0 \E - \A) - \{\vb0\}.
\end{align*}
%TODO proof
\end{theorem}

\begin{example}\label{example:特征值与特征向量.交换矩阵乘积后非零特征值不变}
%@see: 《高等代数(第三版 上册)》(丘维声) P180 习题5.5 13.
设\(\A \in M_{s \times n}(K),
\B \in M_{n \times s}(K)\).
证明:\(\A \B\)与\(\B \A\)有相同的非零特征值.
\begin{proof}
\begin{proof}[证法一]
%@see: 《高等代数(第三版 上册)》(丘维声) P260 习题5.5 13.
假设\(\lambda\)是\(\A \B\)的一个非零特征值,
即存在非零向量\(\vb\alpha\),
使得\begin{equation*}
	(\A \B) \vb\alpha = \lambda \vb\alpha.
\end{equation*}
左乘\(B\),得\begin{equation*}
	(\B \A) (\B \vb\alpha) = \lambda (B \vb\alpha).
\end{equation*}
因此\(\lambda\)也是\(\B \A\)的非零特征值.
\end{proof}
\begin{proof}[证法二]
%@see: https://www.bilibili.com/video/BV17UDtYrExu/
% 假设\(\lambda=0\),
% 那么\begin{equation*}
% 	\abs{\lambda \E_s - \A \B}
% 	= \abs{\A \B},
% 	\qquad
% 	\abs{\lambda \E_n - \B \A}
% 	= \abs{\B \A}.
% \end{equation*}
假设\(\lambda\neq0\),
那么由\hyperref[theorem:行列式.性质2.推论2]{行列式的性质}有\begin{equation*}
	\abs{\lambda \E_s - \A \B}
	= \abs{\lambda (\E_s - (\lambda^{-1} \A) \B)}
	= \lambda^s \abs{\E_s - (\lambda^{-1} \A) \B},
\end{equation*}
同理可得\begin{equation*}
	\abs{\lambda \E_n - \B \A}
	= \lambda^n \abs{\E_n - \B (\lambda^{-1} \A)}.
\end{equation*}
由\cref{example:逆矩阵.行列式降阶定理的重要应用1} 有\begin{equation*}
	\abs{\E_s - (\lambda^{-1} \A) \B}
	= \abs{\E_n - \B (\lambda^{-1} \A)},
\end{equation*}
那么\begin{equation*}
	\lambda^n \abs{\lambda \E_s - \A \B}
	= \lambda^s \abs{\lambda \E_n - \B \A}.
\end{equation*}
于是\(\abs{\lambda \E_s - \A \B} = 0
\iff
\abs{\lambda \E_n - \B \A} = 0\),
由\cref{theorem:矩阵的特征值与特征向量.与特征多项式和特征子空间的联系} 可知\begin{equation*}
	\text{\(\lambda\)是\(\A \B\)的非零特征值}
	\iff
	\text{\(\lambda\)是\(\B \A\)的非零特征值},
\end{equation*}
即\(\A \B\)与\(\B \A\)有相同的非零特征值,且它们的代数重数相同.
\end{proof}\let\qed\relax
\end{proof}
%\cref{example:单位矩阵与两矩阵乘积之差.单位矩阵与两矩阵乘积之差的行列式}
\end{example}
\begin{example}\label{example:单位矩阵与两矩阵乘积之差.单位矩阵与两矩阵乘积之差的行列式}
设\(\A \in M_{m \times n}(K),
\B \in M_{n \times m}(K)\),
且\(m \geq n\).
求证:\begin{equation}
	\abs{\lambda\E_m-\A\B} = \lambda^{m-n} \abs{\lambda\E_n-\B\A}.
\end{equation}
\begin{proof}
当\(\lambda\neq0\)时,考虑下列分块矩阵:\[
	\begin{bmatrix}
		\lambda\E_m & \A \\
		\B & \E_n
	\end{bmatrix}.
\]
因为\(\lambda\E_m,\E_n\)都是可逆矩阵,
故由\hyperref[theorem:逆矩阵.行列式降阶定理]{降阶公式}可得\[
	\abs{\E_n} \cdot \abs{\lambda\E_m - \A (\E_n)^{-1} \B}
	= \abs{\lambda\E_m} \cdot \abs{\E_n - \B (\lambda\E_m)^{-1} \A},
\]
即有\(\abs{\lambda\E_m-\A\B} = \lambda^{m-n} \abs{\lambda\E_n-\B\A}\)成立.

当\(\lambda=0\)时,若\(m>n\),则\[
	\rank(\A\B) \leq \min\{\rank\A,\rank\B\} \leq \min\{m,n\} < m,
\]
故\(\abs{-\A\B}=0\),结论也成立;
若\(m = n\),则由\cref{theorem:行列式.矩阵乘积的行列式} 可知结论也成立.

事实上,\(\lambda=0\)的情形也可通过摄动法由\(\lambda\neq0\)的情形来得到.
\end{proof}
%\cref{example:逆矩阵.行列式降阶定理的重要应用1}
%\cref{example:单位矩阵与两矩阵乘积之差.单位矩阵与两矩阵乘积之差的秩}
% 本例还有其他证法:
% 你可以将\(\A\)化为等价标准型来证明,
% 或先证\(\A\)非异的情形,再用摄动法进行讨论.
\end{example}

\begin{example}
%@see: https://www.bilibili.com/video/BV1vGmuYkEt3/
设\(\A\)是数域\(K\)上的2阶矩阵,
且\(\A^2 \vb\alpha - (p + q) \A \vb\alpha + p q \vb\alpha = \vb0\).
求\(\A\)的特征值与对应的特征向量.
\begin{solution}
由题意有\[
	(\A - p \E)(\A - q \E) \vb\alpha = \vb0.
	\eqno(1)
\]
% 矩阵\((\A - p \E)\)与\((\A - q \E)\)可交换
根据定义有\(0\)是\(\A - p \E\)的特征值,也是\(\A - q \E\)的特征值,
那么\[
	\abs{\A - p \E}
	= \abs{\A - q \E}
	= 0;
\]
由\cref{theorem:矩阵的特征值与特征向量.与特征多项式和特征子空间的联系}
可知\(p\)和\(q\)都是\(\A\)的特征值.
又因为\begin{gather*}
	(\A - p \E)(\A - q \E) \vb\alpha
	= \A (\A - q \E) \vb\alpha - p (\A - q \E) \vb\alpha
	= \vb0, \\
	\A (\A - q \E) \vb\alpha = p (\A - q \E) \vb\alpha,
\end{gather*}
所以\((\A - q \E) \vb\alpha\)是\(\A\)的属于\(p\)的特征向量.
同理可知\((\A - p \E) \vb\alpha\)是\(\A\)的属于\(q\)的特征向量.
\end{solution}
\end{example}
\begin{remark}
在上例中,可以根据等式\(\A^2 \vb\alpha - (p + q) \A \vb\alpha + p q \vb\alpha = \vb0\)
写出\[
	\A (\vb\alpha,\A \vb\alpha)
	= (\A \vb\alpha,\A^2 \vb\alpha)
	= (\A \vb\alpha,(p+q) \A \vb\alpha - p q \vb\alpha)
	= (\vb\alpha,\A \vb\alpha)
	\begin{bmatrix}
		0 & -p q \\
		1 & p+q
	\end{bmatrix}.
\]
\end{remark}

\begin{example}
%@see: 《高等代数(第三版 上册)》(丘维声) P179 习题5.5 7.
%@see: 《线性代数》(张慎语、周厚隆) P96 例7
设\(\A \in M_n(K)\),
证明:\[
	\text{\(\lambda\)是\(\A\)的特征值}
	\iff
	\text{\(\lambda\)是\(\A^T\)的特征值}.
\]
\begin{proof}
由\cref{theorem:行列式.性质1,theorem:矩阵的转置.性质1,theorem:矩阵的转置.性质2,theorem:矩阵的转置.性质3}
有\[
	\abs{\lambda\E-\A^T}
	= \abs{(\lambda\E-\A^T)^T}
	= \abs{(\lambda\E)^T-(\A^T)^T}
	= \abs{\lambda\E-\A},
\]
因此\(\A\)与\(\A^T\)具有相同的特征值.
\end{proof}
\end{example}
% \begin{example}
% 举例说明:矩阵\(\A \in M_n(K)\)与它的转置矩阵\(\A^T\)的特征向量不同.
% %TODO
% \end{example}
\begin{example}
试讨论:在什么条件下,一个矩阵的任一特征向量总是它的转置矩阵的特征向量.
\begin{solution}
设\(\vb{A} \in M_n(K)\).
假设\(\A\)的特征向量都是\(\A^T\)的特征向量,
即\[
	(\forall\lambda_1 \in K)
	(\forall\vb{x} \in K^n)
	[
		\vb{A} \vb{x} = \lambda_1 \vb{x}
		\implies
		\vb{A}^T \vb{x} = \lambda_2 \vb{x}
	],
\]
那么有\[
	(\A^T\x)^T=(\lambda_2\x)^T
	\implies
	\x^T\A=\lambda_2\x^T
	\implies
	(\x^T\A)\x=(\lambda_2\x^T)\x,
\]\[
	\x^T(\A\x)=\x^T(\lambda_1\x)=\lambda_1\x^T\x=\lambda_2\x^T\x
	\implies
	(\lambda_1-\lambda_2)\x^T\x=0,
\]
因为\(\x\neq\z\),\(\x^T\x\neq0\),
所以\(\lambda_1-\lambda_2=0\),\(\lambda_1=\lambda_2\).
那么\[
	\A\x=\lambda_1\x=\lambda_2\x=\A^T\x,
\]
从而\((\A-\A^T)\x=\z\),\(\A=\A^T\).
也就是说,当且仅当\(\A\)是对称矩阵时,\(\A\)的特征向量都是\(\A^T\)的特征向量.
\end{solution}
\end{example}

\begin{example}
%@see: 《高等代数(第三版 上册)》(丘维声) P175 例2
设\[
	\vb{A} = \begin{bmatrix}
		2 & -2 & 2 \\
		-2 & -1 & 4 \\
		2 & 4 & -1
	\end{bmatrix}
\]是数域\(K\)上的矩阵,
求\(\vb{A}\)的全部特征值和特征向量.
\begin{solution}
因为\[
	\abs{\lambda\vb{E}-\vb{A}}
	= \begin{vmatrix}
		\lambda-2 & 2 & -2 \\
		2 & \lambda+1 & -4 \\
		-2 & -4 & \lambda+1
	\end{vmatrix}
	= (\lambda-3)^2 (\lambda+6),
\]
所以\(\vb{A}\)的全部特征值是
\(3\ (\text{二重}),-6\).

对于特征值\(3\),解齐次线性方程组\((3\vb{E}-\vb{A}) \vb{x} = \vb0\):\[
	\begin{bmatrix}
		1 & 2 & -2 \\
		2 & 4 & -4 \\
		-2 & -4 & 4
	\end{bmatrix}
	\to \begin{bmatrix}
		1 & 2 -2 \\
		0 & 0 & 0 \\
		0 & 0 & 0
	\end{bmatrix},
\]
它的基础解系为\(\vb{x}_1 = (-2,1,0)^T,
\vb{x}_2 = (2,0,1)^T\).
因此\(\vb{A}\)的属于\(3\)的全部特征向量是\[
	\Set{
		k_1 \vb{x}_1 + k_2 \vb{x}_2
		\given
		\text{$k_1,k_2 \in K$且$k_1,k_2$不全为$0$}
	}.
\]

对于特征值\(-6\),解齐次线性方程组\((-6\vb{E}-\vb{A}) \vb{x} = \vb0\)
得它的基础解系为\(\vb{x}_3 = (1,2,-2)^T\).
因此\(\vb{A}\)的属于\(-6\)的全部特征向量是\[
	\Set{
		k_3 \vb{x}_3
		\given
		k_3 \in K-\{0\}
	}.
\]
\end{solution}
\end{example}

\begin{example}
%@see: 《2025年全国硕士研究生入学统一考试(数学一)》三解答题/第21题
设矩阵\(\A = \begin{bmatrix}
	0 & -1 & 2 \\
	-1 & 0 & 2 \\
	-1 & -1 & a
\end{bmatrix}\),
且\(1\)是\(\A\)的特征多项式的重根.
\begin{itemize}
	\item 求\(a\)的值.
	\item 求所有满足\(\A \vb\alpha = \vb\alpha + \vb\beta,
	\A^2 \vb\alpha = \vb\alpha + 2\vb\beta\)的非零列向量\(\vb\alpha,\vb\beta\).
\end{itemize}
\begin{solution}
由题意有\begin{equation*}
	\abs{\lambda\E-\A}
	= \begin{vmatrix}
		\lambda & 1 & -2 \\
		1 & \lambda & -2 \\
		1 & 1 & \lambda-a
	\end{vmatrix}
	%@Mathematica: -CharacteristicPolynomial[{{0, -1, 2}, {-1, 0, 2}, {-1, -1, a}}, \[Lambda]] // Factor
	= (\lambda-1)[\lambda^2+(1-a)\lambda+(4-a)].
\end{equation*}
因为\(1\)是\(\abs{\lambda\E-\A}\)的重根,
所以\(1\)是\(\lambda^2+(1-a)\lambda+(4-a)\)的根,
即\(1^2+(1-a)\cdot1+(4-a)
= 2a-6
= 0\),
解得\(a=3\).

由\(\A \vb\alpha = \vb\alpha + \vb\beta,
\A^2 \vb\alpha = \vb\alpha + 2\vb\beta\)
可得\(\A^2 \vb\alpha - 2 \A \vb\alpha = -\vb\alpha\),
即\((\A-\E)^2 \vb\alpha = \vb0\).
换言之,\(\vb\alpha\)是方程组\((\A-\E)^2 \vb{x} = \vb0\)的非零解.
因为\((\A-\E)^2 = \begin{bmatrix}
	-1 & -1 & 2 \\
	-1 & -1 & 2 \\
	-1 & -1 & 2
\end{bmatrix}^2
= \vb0\),
所以\(\vb\alpha\)可以是任意非零列向量,
即\(\vb\alpha = (k_1,k_2,k_3)^T\),
其中\(k_1,k_2,k_3\)是待定常数.

由\(\A \vb\alpha = \vb\alpha + \vb\beta\)
可得\begin{equation*}
	\vb\beta = (\A-\E) \vb\alpha
	= \begin{bmatrix}
		-1 & -1 & 2 \\
		-1 & -1 & 2 \\
		-1 & -1 & 2
	\end{bmatrix}
	\begin{bmatrix}
		k_1 \\ k_2 \\ k_3
	\end{bmatrix}
	= (2k_3 - k_2 - k_1)
	(1,1,1)^T,
\end{equation*}
其中\(k_1 + k_2 \neq 2 k_3\),且\(k_1,k_2,k_3\)不全为零.
\end{solution}
\end{example}

\subsection{特征子空间}
%\cref{theorem:矩阵的特征值与特征向量.特征子空间1,theorem:矩阵的特征值与特征向量.特征子空间2}
既然\(\A\)的属于\(\lambda_0\)的特征向量\(\vb{x}_0\)
就是齐次线性方程组\((\lambda_0\E-\A)\x=\vb0\)的非零解,
那么由\cref{theorem:线性方程组.齐次线性方程组的解的线性组合也是解} 可知,
\(\A\)的属于同一个特征值\(\lambda_0\)的
特征向量\(\AutoTuple{\x}{t}\)的非零线性组合
\(\sum_{i=1}^t k_i \x_i\)
也是\(\A\)的属于\(\lambda_0\)的特征向量.
于是可知,齐次线性方程组\((\lambda_0\E-\A)\x=\vb0\)的解集是\(K^n\)的子空间.

\begin{definition}
%@see: 《高等代数(第三版 上册)》(丘维声) P174
设\(\lambda \in K\)是矩阵\(\A \in M_n(K)\)的一个特征值,
我们把齐次线性方程\[
	(\lambda\E-\A)\x=\vb0
\]的解空间\(\Ker(\lambda\E-\A)\)
称为“矩阵\(\A\)的属于特征值\(\lambda\)的\DefineConcept{特征子空间}(eigenspace)”.
%@see: https://mathworld.wolfram.com/Eigenspace.html
\end{definition}

\begin{proposition}
%@see: 《高等代数(第三版 上册)》(丘维声) P174
矩阵\(\A\)的属于特征值\(\lambda\)的特征子空间的全体非零向量
恰是矩阵\(\A\)的属于特征值\(\lambda\)的全体特征向量.
\end{proposition}

\begin{proposition}
%@see: 《高等代数(第三版 上册)》(丘维声) P177 命题2
%@see: 《线性代数》(张慎语、周厚隆) P97 习题5.1 5.
设\(\A \in M_n(K)\),
则\(\A\)的特征多项式\(\abs{\lambda\E-\A}\)是\(\lambda\)的\(n\)次多项式,
\(\lambda^n\)的系数是\(1\),
\(\lambda^{n-1}\)的系数等于\(-\tr\A\),
常数项为\((-1)^n \abs{\A}\),
\(\lambda^{n-k}\ (1\leq k<n)\)的系数是\[
	(-1)^k \sum_{1 \leq i_1 < \dotsb < i_k \leq n} \MatrixMinor\A{
		i_1,i_2,\dotsc,i_k \\
		i_1,i_2,\dotsc,i_k
	}.
\]
\begin{proof}
\(\vb{A}\)的特征多项式为\[
	\abs{\lambda\vb{E}-\vb{A}}
	= \begin{vmatrix}
		\lambda-a_{11} & 0-a_{12} & \dots & 0-a_{1n} \\
		0-a_{21} & \lambda-a_{22} & \dots & 0-a_{2n} \\
		\vdots & \vdots && \vdots \\
		0-a_{n1} & 0-a_{n2} & \dots & \lambda-a_{nn}
	\end{vmatrix}.
\]
由\cref{theorem:行列式.性质3} 可知,
\(\vb{A}\)的特征多项式等于\(2^n\)个行列式之和,
其中两个是\[
	\begin{vmatrix}
		\lambda & 0 & \dots & 0 \\
		0 & \lambda & \dots & 0 \\
		\vdots & \vdots && \vdots \\
		0 & 0 & \dots & \lambda
	\end{vmatrix}
	= \lambda^n,
	\qquad
	\begin{vmatrix}
		-a_{11} & -a_{12} & \dots & -a_{1n} \\
		-a_{21} & -a_{22} & \dots & -a_{2n} \\
		\vdots & \vdots && \vdots \\
		-a_{n1} & -a_{n2} & \dots & -a_{nn}
	\end{vmatrix}
	= (-1)^n \abs{\vb{A}},
\]
其余行列式是这样的行列式 --- 它的第\(j_1,j_2,\dotsc,j_{n-k}\)列是含有\(\lambda\)的列,
其余列是不含\(\lambda\)的列(它们是\(-\vb{A}\)的列):\begin{equation*}
	\begin{vmatrix}
		-a_{11} & \dots & 0 & \dots & 0 & \dots & 0 & \dots & -a_{1n} \\
		\vdots & & \vdots & & \vdots & & \vdots & & \vdots \\
		-a_{j_1 1} & \dots & \lambda & \dots & 0 & \dots & 0 & \dots & -a_{j_1 n} \\
		\vdots & & \vdots & & \vdots & & \vdots & & \vdots \\
		-a_{j_2 1} & \dots & 0 & \dots & \lambda & \dots & 0 & \dots & -a_{j_2 n} \\
		\vdots & & \vdots & & \vdots & & \vdots & & \vdots \\
		-a_{j_{n-k} 1} & \dots & 0 & \dots & 0 & \dots & \lambda & \dots & -a_{j_{n-k} n} \\
		\vdots & & \vdots & & \vdots & & \vdots & & \vdots \\
		-a_{n1} & \dots & 0 & \dots & 0 & \dots & 0 & \dots & -a_{nn} \\
	\end{vmatrix}.
	\eqno(1)
\end{equation*}
利用\hyperref[theorem:行列式.拉普拉斯定理]{拉普拉斯定理},
按第\(j_1,j_2,\dotsc,j_{n-k}\)列展开(1)式,
这\(n-k\)列元素组成的\(n-k\)阶子式只有一个不为零,
其余\(n-k\)阶子式全为零(因为它们都含有元素全为零的行).
令\begin{equation*}
	\Set{ j_1',j_2',\dotsc,j_k' }
	= \Set{ 1,2,\dotsc,n } - \Set{ j_1,j_2,\dotsc,j_{n-k} },
\end{equation*}
且\(j_1'<j_2'<\dotsb<j_k'\),
则(1)式等于\begin{align*}
	&\hspace{-20pt}
	\begin{vmatrix}
		\lambda & 0 & \dots & 0 \\
		0 & \lambda & \dots & 0 \\
		\vdots & \vdots & & \vdots \\
		0 & 0 & \dots & \lambda
	\end{vmatrix}
	(-1)^{(j_1+j_2+\dotsb+j_{n-k})+(j_1+j_2+\dotsb+j_{n-k})}
	\MatrixMinor{(-\vb{A})}{
		j_1',j_2',\dotsc,j_k' \\
		j_1',j_2',\dotsc,j_k'
	} \\
	&= \lambda^{n-k} (-1)^k
	\MatrixMinor{\vb{A}}{
		j_1',j_2',\dotsc,j_k' \\
		j_1',j_2',\dotsc,j_k'
	}.
\end{align*}
由于\(1 \leq j_1' < j_2' < \dotsb < j_k' \leq n\),
因此\(\abs{\lambda\E-\A}\)中\(\lambda^{n-k}\)的系数就是
\(\A\)的所有\(k\)阶主子式之和与\((-1)^k\)的乘积\begin{equation*}
	(-1)^k
	\sum_{1 \leq j_1' < j_2' < \dotsb < j_k' \leq n}
	\MatrixMinor{\vb{A}}{
		j_1',j_2',\dotsc,j_k' \\
		j_1',j_2',\dotsc,j_k'
	}.
	\qedhere
\end{equation*}
\end{proof}
\end{proposition}

\begin{example}
%@see: https://www.bilibili.com/video/BV1tz4y177ys/
设矩阵\(\vb{A} \in M_3(K)\),
\(\vb{E}\)是数域\(K\)上的3阶单位矩阵,
且\[
	\abs{\vb{A}-2\vb{E}}
	= \abs{\vb{A}-3\vb{E}}
	= \abs{\vb{A}-4\vb{E}}
	= 3.
\]
求\(\abs{\vb{A}-\vb{E}}\).
\begin{solution}
记\(f(\lambda) = \abs{\vb{A}-\lambda\vb{E}}\),
那么由题意有\[
	f(2) = f(3) = f(4) = 3,
\]
由于\(f(\lambda)\)的变量\(\lambda\)的最高次项\(\lambda^3\)的系数是\(-1\),
所以\[
	f(\lambda) = 3 - (\lambda-2)(\lambda-3)(\lambda-4).
\]
用\(1\)代\(\lambda\)得\[
	\abs{\vb{A}-\vb{E}}
	= f(1)
	= 3 - (1-2)(1-3)(1-4)
	= 9.
\]
\end{solution}
\end{example}
\begin{example}
%@see: https://www.bilibili.com/video/BV1qQymYUEBj/
设矩阵\(\vb{A} \in M_3(K)\),
\(\vb{E}\)是数域\(K\)上的3阶单位矩阵,
且\[
	\abs{\vb{E}-\vb{A}} = 1,
	\qquad
	\abs{-\vb{E}-\vb{A}} = -1,
	\qquad
	\abs{2\vb{E}-\vb{A}} = 2.
\]
求\(\abs{3\vb{E}-\vb{A}}\).
\begin{solution}
记\(f(\lambda) = \abs{\lambda\vb{E}-\vb{A}}\),
那么由题意有\(\lambda=-1,1,2\)是方程\[
	f(\lambda) = \lambda
\]的根,
由于\(f(\lambda)\)的变量\(\lambda\)的最高次项\(\lambda^3\)的系数是\(+1\),
所以\[
	f(\lambda) - \lambda
	= (\lambda+1)(\lambda-1)(\lambda-2).
\]
用\(3\)代\(\lambda\)得\[
	\abs{3\vb{E}-\vb{A}} - 3
	= (3+1)(3-1)(3-2),
\]
整理得\(\abs{3\vb{E}-\vb{A}} = 11\).
\end{solution}
\end{example}

\subsection{求解特征值和特征向量的一般程序}
% 大数学家高斯在1799年证明了以下\DefineConcept{代数基本定理}:
\begin{lemma}[代数基本定理]
任何\(n\ (n>0)\)次多项式有且仅有\(n\)个复根,其中规定\(m\)重根算\(m\)个根.
\end{lemma}
我们将会在\cref{example:留数定理.利用儒歇定理证明代数基本定理} 给出代数基本定理的具体证明过程.

%@see: 《线性代数》(张慎语、周厚隆) P93
求解矩阵\(\A \in M_n(K)\)的特征值和特征向量的一般程序:\begin{enumerate}
	\item 计算特征多项式\(\abs{\lambda\E-\A}\).

	\item 判别多项式\(\abs{\lambda\E-\A}\)(即方程\(\abs{\lambda\E-\A}=0\))
	在数域\(K\)中有没有根.
	\begin{itemize}
		\item 如果它没有根,则没有特征值、特征向量.
		\item 如果它有根,则它在\(K\)中的全部根\(\AutoTuple{\lambda}{n}\)就是\(\A\)的全部特征值.
		\item 如果\(K = \mathbb{C}\),
		那么根据代数基本定理,任意\(n\)阶矩阵的特征多项式有且仅有\(n\)个复根.
	\end{itemize}

	\item 对于每个不同的特征值\(\lambda_j\ (j=1,2,\dotsc,n)\),
	求出齐次线性方程组\((\lambda_j \E - \A)\x = \vb0\)的一个基础解系\(\AutoTuple{\x}{t}\),
	则\(\A\)的属于\(\lambda_j\)的全部特征向量为\(k_1 \x_1 + k_2 \x_2 + \dotsb + k_t \x_t\)
	(其中\(\AutoTuple{k}{t}\)是不全为零的任意常数).
\end{enumerate}

\begin{example}
%@see: 《线性代数》(张慎语、周厚隆) P94 例3
求实数域上的3阶矩阵\[
	\A = \begin{bmatrix}
		2 & -1 & 2 \\
		5 & -3 & 3 \\
		-1 & 0 & -2
	\end{bmatrix}
\]的特征值与对应的特征向量.
\begin{solution}
\(\A\)的特征多项式\begin{align*}
	\abs{\lambda\E-\A}
	&= \begin{vmatrix}
		\lambda-2 & 1 & -2 \\
		-5 & \lambda+2 & -3 \\
		1 & 0 & \lambda+2
	\end{vmatrix} \\
	&= \lambda^3 + 3\lambda^2 + 3\lambda + 1
	= (\lambda+1)^3,
\end{align*}
%@Mathematica: CharacteristicPolynomial[{{2, -1, 2}, {5, -3, 3}, {-1, 0, -2}}, x]
特征值为\(\lambda=-1\ (\text{三重})\).
解方程组\((-\E-\A)\x = \z\),对系数矩阵施行初等行变换:\[
	-\E-\A = \begin{bmatrix}
		-3 & 1 & -2 \\
		-5 & 2 & -3 \\
		1 & 0 & 1
	\end{bmatrix} \to \begin{bmatrix}
		1 & 0 & 1 \\
		5 & 2 & -3 \\
		-3 & 1 & -2
	\end{bmatrix} \to \begin{bmatrix}
		1 & 0 & 1 \\
		0 & 1 & 1 \\
		0 & 0 & 0
	\end{bmatrix}.
\]
可知\(\rank(-\E-\A) = 2\),
\(\dim\Ker(-\E-\A) = 1\).
令\(x_3 = -1\),得基础解系\[
	\x_1 = (1,1,-1)^T,
\]
那么属于\(-1\)的全部特征向量为\(k (1,1,-1)^T\),
\(k\)是非零的任意常数.
\end{solution}
\end{example}

\begin{example}
求实数域上的3阶矩阵\[
	\A = \begin{bmatrix}
		-1 & 0 & 0 \\
		8 & 2 & 4 \\
		8 & 3 & 3
	\end{bmatrix}
\]的特征值与对应的特征向量.
\begin{solution}
\(\A\)的特征多项式\[
	\abs{\lambda\E-\A}
	= \begin{vmatrix}
		\lambda+1 & 0 & 0 \\
		-8 & \lambda-2 & -4 \\
		-8 & -3 & \lambda-3
	\end{vmatrix}
	= (\lambda+1)^2 (\lambda-6),
\]
故\(\A\)的特征值为\(\lambda_1=-1\ (\text{二重}),
\lambda_2=6\).

当\(\lambda=-1\)时,解方程组\((-\E-\A)\x = \z\),\[
	-\E-\A = \begin{bmatrix}
		0 & 0 & 0 \\
		-8 & -3 & -4 \\
		-8 & -3 & -4
	\end{bmatrix} \to \begin{bmatrix}
		-8 & -3 & -4 \\
		0 & 0 & 0 \\
		0 & 0 & 0
	\end{bmatrix},
\]
可知\(\rank(-\E-\A) = 1\),
方程组\((-\E-\A)\x = \z\)的解空间有2个基向量.
分别令\(x_2 = 8, x_3 = 0\)和\(x_2 = 0, x_3 = 2\),得基础解系\[
	\x_1 = \begin{bmatrix} -3 \\ 8 \\ 0 \end{bmatrix},
	\qquad
	\x_2 = \begin{bmatrix} -1 \\ 0 \\ 2 \end{bmatrix},
\]
故属于\(-1\)的全部特征向量为\(k_1 \x_1 + k_2 \x_2\),
\(k_1,k_2\)是不全为零的任意常数.

当\(\lambda=6\)时,解方程组\((6\E-\A)\x = \z\),\[
	6\E-\A = \begin{bmatrix}
		7 & 0 & 0 \\
		-8 & 4 & -4 \\
		-8 & -3 & 3
	\end{bmatrix} \to \begin{bmatrix}
		1 & 0 & 0 \\
		0 & 1 & -1 \\
		0 & 0 & 0
	\end{bmatrix},
\]
可知\(\rank(6\E-\A) = 2\),方程组\((6\E-\A)\x = \z\)的解空间有1个基向量.
令\(x_3 = 1\),得\(x_1 = 0, x_2 = 1\),基础解系\[
	\x_3 = (0,1,1)^T,
\]
故属于\(6\)的全部特征向量为\(k_3 \x_3\),\(k_3\)是任意非零常数.
\end{solution}
\end{example}

从特征值、特征向量的性质可以看出,
矩阵\(\A\)的一个特征值对应若干个线性无关的特征向量;
但反之,一个特征向量只能属于一个特征值.
事实上,设\(\vb{x}_0\)为某个矩阵\(\A\)的特征向量,若有\(\lambda_1,\lambda_2\)满足\[
	\A\vb{x}_0=\lambda_1\vb{x}_0,
	\quad
	\A\vb{x}_0=\lambda_2\vb{x}_0,
\]
则必有\(\lambda_1\vb{x}_0=\lambda_2\vb{x}_0\)或\((\lambda_1-\lambda_2)\vb{x}_0=\z\),
因为\(\vb{x}_0\neq\z\),所以\(\lambda_1-\lambda_2=0\),\(\lambda_1=\lambda_2\).

前面已经知道,
矩阵\(\A\)的同一个特征值\(\lambda_0\)对应的特征向量的非零线性组合
仍为\(\A\)的属于\(\lambda_0\)的特征向量.
那么,\(\A\)的不同特征值对应的特征向量的非零线性组合又如何呢?
\begin{example}
%@see: 《线性代数》(张慎语、周厚隆) P95 例5
设\(\lambda_1,\lambda_2\)是矩阵\(\A\)的两个不同的特征值,
\(\x_1,\x_2\)分别是\(\lambda_1,\lambda_2\)对应的特征向量.
证明:\(\x_1+\x_2\)不是\(\A\)的特征向量.
\begin{proof}
\(\A\x_1 = \lambda_1\x_1\),\(\A\x_2 = \lambda_2\x_2\),
假设\(\A(\x_1+\x_2) = \lambda_0(\x_1+\x_2)\),则\[
	\A\x_1+\A\x_2 =\lambda_1\x_1+\lambda_2\x_2 = \lambda_0\x_1+\lambda_0\x_2,
\]\[
	(\lambda_0-\lambda_1)\x_1+(\lambda_0-\lambda_2)\x_2 = \z,
\]
在上式左右两端同乘\(\A\)和\(\lambda_1\)可得\[
	\left\{ \begin{array}{l}
		(\lambda_0-\lambda_1)\A\x_1+(\lambda_0-\lambda_2)\A\x_2 = (\lambda_0-\lambda_1)\lambda_1\x_1 + (\lambda_0-\lambda_2)\lambda_2\x_2 = \z, \\
		(\lambda_0-\lambda_1)\lambda_1\x_1+(\lambda_0-\lambda_2)\lambda_1\x_2 = \z,
	\end{array} \right.
\]\[
	(\lambda_0-\lambda_2)(\lambda_2-\lambda_1)\x_2 = \z,
\]
因为\(\x_2\neq\z\),所以\((\lambda_0-\lambda_2)(\lambda_2-\lambda_1)=0\);
又因为\(\lambda_2\neq\lambda_1\),所以\(\lambda_0=\lambda_2\).
同理有\[
	(\lambda_0-\lambda_1)\lambda_2\x_1+(\lambda_0-\lambda_2)\lambda_2\x_2 = \z
	\implies
	(\lambda_0-\lambda_1)(\lambda_1-\lambda_2)\x_1 = \z,
\]
因为\(\x_1\neq\z\),所以\(\lambda_0=\lambda_1\).

于是导出\(\lambda_1=\lambda_2\),与题设矛盾,说明\(\x_1+\x_2\)不是\(\A\)的特征向量.
\end{proof}
\end{example}

\begin{example}\label{example:幂零矩阵.幂零矩阵的特征值的性质}
%@see: 《高等代数(第三版 上册)》(丘维声) P179 习题5.5 4.
证明:数域\(K\)上的\(n\)阶幂零矩阵的特征值都是\(0\).
\begin{proof}
设\(\A\)是数域\(K\)上的以正整数\(m\)为幂零指数的\(n\)阶幂零矩阵.
由\cref{example:幂零矩阵.幂零矩阵的行列式} 可知\[
	\abs{0\E-\A}
	= \abs{-\A}
	= (-1)^n \abs{\A}
	= 0,
\]
所以\(0\)是\(\A\)的一个特征值.
假设\(\lambda_0\)是\(\A\)的任意一个特征值,
即存在\(\vb{x}_0 \in K^n-\{\vb0\}\)
使得\[
	\A\vb{x}_0 = \lambda_0\vb{x}_0.
	\eqno(1)
\]
在(1)式等号两边同时左乘\(m-1\)次\(\A\),
便得\[
	\A^m\vb{x}_0 = \lambda_0^m\vb{x}_0.
	\eqno(2)
\]
因为\(\A^m=\vb0\)且\(\vb{x}_0\neq\vb0\),
所以由(2)式解得\(\lambda_0=0\),
这就说明:\(\A\)的特征值都是\(0\).
\end{proof}
\end{example}
\begin{remark}
\cref{example:幂零矩阵.幂零矩阵的特征值的性质} 说明:
特征值全为零的矩阵不一定是零矩阵,还可能是幂零矩阵.
\end{remark}

\begin{example}\label{example:幂幺矩阵.幂幺矩阵的特征值的性质}
证明:数域\(K\)上的\(n\)阶幂幺矩阵的特征值都是\(1\).
\begin{proof}
设\(\E\)是数域\(K\)上的\(n\)阶单位矩阵,
\(\A\)是数域\(K\)上的以正整数\(m\)为幂幺指数的\(n\)阶幂幺矩阵,
等价成立:\((\A-\E)\)是数域\(K\)上的以正整数\(m\)为幂零指数的\(n\)阶幂零矩阵.
由\cref{example:幂零矩阵.幂零矩阵的特征值的性质} 可知\((\A-\E)\)的特征值全为零,
即\(\abs{0\E-(\A-\E)}
= \abs{\E-\A}
= 0\),
可见\(\A\)的特征值都是\(1\).
\end{proof}
\end{example}
\begin{remark}
\cref{example:幂幺矩阵.幂幺矩阵的特征值的性质} 说明:
特征值全为一的矩阵不一定是单位矩阵,还可能是幂幺矩阵.
\end{remark}
%credit: {61d1026b-642e-438a-9506-08e3e7865f96} 说:“任意一个特征值全为0的矩阵,要么是零矩阵,要么是幂零矩阵”和“任意一个特征值全为1的矩阵,要么是单位矩阵,要么是幂幺矩阵”成立
%TODO 这两个命题的证明要用到若尔当标准型的构造或哈密顿--凯莱定理.

\begin{example}\label{example:幂等矩阵.幂等矩阵的特征值的性质}
%@see: 《高等代数(第三版 上册)》(丘维声) P179 习题5.5 5.
%@see: 《线性代数》(张慎语、周厚隆) P96 例8
证明:数域\(K\)上的\(n\)阶幂等矩阵一定有特征值,且其特征值必为0或1.
\begin{proof}
设\(\A\)是幂等矩阵,即有\(\A^2=\A\).
设\(\A\vb{x}_0=\lambda_0\vb{x}_0\ (\vb{x}_0\neq\z)\),
则\[
	\A^2\vb{x}_0
	=\A(\A\vb{x}_0)
	=\A(\lambda_0\vb{x}_0)
	=\lambda_0(\A\vb{x}_0)
	=\lambda_0(\lambda_0\vb{x}_0)
	=\lambda_0^2\vb{x}_0.
\]
因为\[
	\A^2=\A
	\iff
	\A^2-\A=\z
	\implies
	\A^2\vb{x}_0-\A\vb{x}_0
	=(\A^2-\A)\vb{x}_0
	=\z\vb{x}_0
	=\z,
\]
所以\[
	\lambda_0^2\vb{x}_0-\lambda_0\vb{x}_0
	=(\lambda_0^2-\lambda_0)\vb{x}_0
	=\lambda_0(\lambda_0-1)\vb{x}_0
	=\z,
\]
进一步有\(\lambda_0(\lambda_0-1)=0\),
所以\(\lambda_0=0\)或\(\lambda_0=1\).
\end{proof}
\end{example}

\begin{example}
%@see: 《高等代数(第三版 上册)》(丘维声) P179 习题5.5 6.
设数域\(\mathbb{C}\)上的\(n\)阶方阵\(\A\)
与数域\(\mathbb{C}\)上的\(n\)阶单位矩阵\(\E\)
满足\(\A^m=\E\),
其中\(m\)是正整数.
证明:\(\A\)的全体特征值是\(\Set{ z\in\mathbb{R} \given z^m = 1 }\).
%TODO proof
\end{example}

\begin{example}\label{example:矩阵乘积的秩.两个向量的乘积的特征值和特征向量}
设\(\a,\b\)是\(n\)维非零列向量.
证明:求矩阵\(\a\b^T\)的特征值和特征向量.
\begin{proof}
显然有\((\a\b^T)\a = \a(\b^T\a)\),
那么根据定义可知\(\b^T\a\)就是矩阵\(\a\b^T\)的特征值,
而\(\a\)是\(\a\b^T\)属于\(\b^T\a\)的一个特征向量.

又由\cref{example:行列式.两个向量的乘积矩阵的行列式} 可知\(\abs{\a\b^T} = 0\),
所以\(\abs{0\cdot\E-\a\b^T}=0\),
也就是说\(0\)也是矩阵\(\a\b^T\)的特征值.
再由\cref{example:矩阵乘积的秩.两个向量的乘积的秩} 可知\(\rank(\a\b^T) = 1\),
于是根据\cref{theorem:线性方程组.齐次线性方程组的解向量个数} 可知
\((\a\b^T)\x=\vb0\)的解空间的维数为\(n-1\),
这就是说\(0\)是矩阵\(\a\b^T\)的\(n-1\)重特征值.
最后,我们来求\(\a\b^T\)属于\(0\)的特征向量,
解方程组\((\a\b^T)\x=\vb0\),
左乘\(\a^T\)得\(\a^T\a\b^T\x=0\),
由于\(\a\neq\vb0\),\(\a^T\a>0\),
消去便得\(\b^T\x=0\),
因此由\cref{theorem:线性方程组.同解方程组.特例1} 可知
\(\b^T\x=0\)与\((\a\b^T)\x=\vb0\)同解,
\(\b^T\x=0\)的解空间就是\(\a\b^T\)的属于\(0\)的特征子空间.
\end{proof}
\end{example}

\begin{example}
设\(\A = \begin{bmatrix} 2 & -1 & 2 \\ 5 & -3 & 3 \\ -1 & 0 & -2 \end{bmatrix}\),
求\(\A\)的特征值与对应的特征向量.
\begin{solution}
\(\A\)的特征多项式\[
	\abs{\lambda\E-\A}
	= \begin{vmatrix}
		\lambda-2 & 1 & -2 \\
		-5 & \lambda+3 & -3 \\
		1 & 0 & \lambda+2
	\end{vmatrix}
	= (\lambda+1)^3,
\]
解\(\abs{\lambda\E-\A}=0\)得\(\lambda=-1\ (\text{三重})\).

当\(\lambda=-1\)时,解方程组\((-\E-\A)\x=\z\),\[
	-\E-\A = \begin{bmatrix} -3 & 1 & -2 \\ -5 & 2 & -3 \\ 1 & 0 & 1 \end{bmatrix}
	\to \begin{bmatrix} 1 & 0 & 1 \\ 5 & 2 & -3 \\ -3 & 1 & -2 \end{bmatrix}
	\to \begin{bmatrix} 1 & 0 & 1 \\ 0 & 1 & 1 \\ 0 & 0 & 0 \end{bmatrix},
\]
\(\rank(-\E-\A)=2\),
令\(x_3=1\),
得基础解系\[
	\x_1=\begin{bmatrix} -1 \\ -1 \\ 1 \end{bmatrix},
\]
属于\(-1\)的全部特征向量为\(k\x_1\)(\(k\)为非零的任意常数).
\end{solution}
\end{example}

\begin{example}
设\(\A = \begin{bmatrix} -1 & 0 & 0 \\ 8 & 2 & 4 \\ 8 & 3 & 3 \end{bmatrix}\),
求\(\A\)的特征值与对应的特征向量.
\begin{solution}
\(\A\)的特征多项式\[
	\a = \begin{vmatrix}
		\lambda+1 & 0 & 0 \\
		-8 & \lambda-2 & -4 \\
		-8 & -3 & \lambda-3
	\end{vmatrix}
	= (\lambda+1)(\lambda^2-5\lambda-6)
	= (\lambda+1)^2(\lambda-6).
\]
令\(\a = 0\)可得\(\A\)的特征值为\(\lambda_1=-1\ (\text{二重})\),\(\lambda_2=6\).

当\(\lambda=-1\)时,解方程组\((-\E-\A)\x=\z\),\[
	-\E-\A
	= \begin{bmatrix} 0 & 0 & 0 \\ -8 & -3 & -4 \\ -8 & -3 & -4 \end{bmatrix}
	\to \begin{bmatrix} -8 & -3 & -4 \\ 0 & 0 & 0 \\ 0 & 0 & 0 \end{bmatrix}.
\]
分别令\(\left\{ \begin{array}{l} x_2=8 \\ x_3=0 \end{array} \right.\)
和\(\left\{ \begin{array}{l} x_2=0 \\ x_3=2 \end{array} \right.\),
得基础解系\[
	\x_1 = \begin{bmatrix} -3 \\ 8 \\ 0 \end{bmatrix},
	\quad
	\x_2 = \begin{bmatrix} -1 \\ 0 \\ 2 \end{bmatrix},
\]
属于\(-1\)的全部特征向量为\(k_1\x_1+k_2\x_2\)(\(k_1,k_2\)为不全为零的任意常数);

当\(\lambda=6\)时,解方程组\((6\E-\A)\x=\z\),\[
	6\E-\A
	= \begin{bmatrix} 7 & 0 & 0 \\ -8 & 4 & -4 \\ -8 & -3 & 3 \end{bmatrix}
	\to \begin{bmatrix} 1 & 0 & 0 \\ 0 & 1 & -1 \\ 0 & 0 & 0 \end{bmatrix},
\]
令\(x_3=1\)得\(x_1=0\),\(x_2=1\),
基础解系\[
	\x_3 = \begin{bmatrix} 0 \\ 1 \\ 1 \end{bmatrix},
\]
属于\(6\)的全部特征向量为\(k_3\x_3\)(\(k_3\)为任意常数).
\end{solution}
\end{example}

\begin{example}
设矩阵\(\A = (a_{ij})_n \in \mathbb{C}^n\),
但\(a_{ij} \in \mathbb{R}\ (i,j=1,2,\dotsc,n)\).
证明:如果\(\lambda_0\in\mathbb{C}\)是\(\A\)的一个特征值,
\(\vb{x}_0\)是\(\A\)属于\(\lambda_0\)的一个特征向量,
那么\(\ComplexConjugate{\lambda_0}\)也是\(\A\)的一个特征值,
且\(\ComplexConjugate{\vb{x}_0}\)是\(\A\)属于\(\ComplexConjugate{\lambda_0}\)的一个特征向量.
\begin{proof}
在\(\A\vb{x}_0=\lambda_0\vb{x}_0\)两边取共轭得
\(\ComplexConjugate{\A}\ComplexConjugate{\vb{x}_0}
=\ComplexConjugate{\lambda_0}\ComplexConjugate{\vb{x}_0}\).
又因为\(\A=\ComplexConjugate{\A}\),
因此\(\A\ComplexConjugate{\vb{x}_0}=\ComplexConjugate{\lambda_0}\ComplexConjugate{\vb{x}_0}\).
这就表明\(\ComplexConjugate{\lambda_0}\)也是\(\A\)的一个特征值,
\(\ComplexConjugate{\vb{x}_0}\)是\(\A\)的属于\(\ComplexConjugate{\lambda_0}\)的一个特征向量.
\end{proof}
\end{example}

\begin{example}
求复数域上矩阵\[
	\A = \begin{bmatrix}
		4 & 7 & -3 \\
		-2 & -4 & 2 \\
		-4 & -10 & 4
	\end{bmatrix}
\]的全部特征值和特征向量.
\begin{solution}
\(\A\)的特征多项式为\[
	\abs{\lambda\E-\A}
	= \begin{vmatrix}
		\lambda-4 & -7 & 3 \\
		2 & \lambda+4 & -2 \\
		4 & 10 & \lambda-4
	\end{vmatrix}
	= \lambda^3 - 4\lambda^2 + 6\lambda - 4
	= (\lambda-2)(\lambda^2-2\lambda+2).
\]
令\(\abs{\lambda\E-\A}=0\)解得\(\lambda=2,1\pm\iu\).

当\(\lambda=2\)时,解方程组\((2\E-\A)\x=\z\),\[
	2\E-\A = \begin{bmatrix}
		-2 & -7 & 3 \\
		2 & 6 & -2 \\
		4 & 10 & -2
	\end{bmatrix} \to \begin{bmatrix}
		2 & 4 & 0 \\
		0 & 1 & -1 \\
		0 & 0 & 0
	\end{bmatrix}.
\]
令\(x_2=x_3=1\)得\(x_1=-2\),基础解系为\[
	\x_1 = (-2,1,1)^T,
\]
属于\(2\)的全部特征向量为\(k_1\x_1\ (k_1\in\mathbb{C}-\{0\})\).

当\(\lambda=1+\iu\)时,解方程组\([(1+\iu)\E-\A]\x=\z\),\[
	(1+\iu)\E-\A = \begin{bmatrix}
		-3+\iu & -7 & 3 \\
		2 & 5+\iu & -2 \\
		4 & 10 & -3+\iu
	\end{bmatrix}
	\to \def\arraystretch{1.5}\begin{bmatrix}
		1 & 0 & \frac{1}{2}-\iu \\
		0 & 1 & -\frac{1}{2}+\frac{1}{2}\iu \\
		0 & 0 & 0
	\end{bmatrix}.
\]
令\(x_3=-2\)得\(x_1=1-2\iu,x_2=-1+\iu\),
基础解系为\[
	\x_2 = (1-2\iu,-1+\iu,-2)^T,
\]
属于\(1+\iu\)的全部特征向量为\(k_2\x_2\ (k_2\in\mathbb{C}-\{0\})\).

当\(\lambda=1-\iu\)时,
\(\x_2\)也是它的一个特征向量,
那么属于\(1-\iu\)的全部特征向量为\(k_3\x_2\ (k_3\in\mathbb{C}-\{0\})\).
\end{solution}
\end{example}

\section{矩阵的相似}
有时候,我们会遇到这样的问题:
已知\(\vb{A}\)是数域\(K\)上的一个\(n\)阶方阵,求\(\vb{A}^m\).
这时候,如果存在数域\(K\)上的一个\(n\)阶可逆矩阵\(\vb{P}\),
使得\(\vb{P}^{-1}\vb{A}\vb{P} = \vb{B}\),并且\(\vb{B}^m\)容易计算,
那么我们就可以利用矩阵的乘法结合律得到以下结果:\begin{equation*}
	\vb{A}^m
	= (\vb{P}^{-1}\vb{B}\vb{P})^m
	= \underbrace{
			(\vb{P}^{-1}\vb{B}\vb{P})
			(\vb{P}^{-1}\vb{B}\vb{P})
			\dotsm
			(\vb{P}^{-1}\vb{B}\vb{P})
		}_{\text{$m$个}}
	= \vb{P}^{-1}\vb{B}^m\vb{P}.
\end{equation*}

\subsection{矩阵相似的概念}
\begin{definition}
%@see: 《线性代数》(张慎语、周厚隆) P97 定义3
%@see: 《高等代数(第三版 上册)》(丘维声) P169 定义1
设\(\vb{A}\)、\(\vb{B}\)是两个\(n\)阶矩阵.
若存在可逆矩阵\(\vb{P}\),
使得\begin{equation}\label{equation:特征值与特征向量.相似矩阵的定义}
	\vb{P}^{-1} \vb{A} \vb{P} = \vb{B}
\end{equation}
则称“\(\vb{A}\)与\(\vb{B}\)~\DefineConcept{相似}%
(\(\vb{A}\) is \emph{similar} to \(\vb{B}\))”,
%@see: https://mathworld.wolfram.com/SimilarMatrices.html
记作\(\vb{A}\sim\vb{B}\).
\end{definition}
\begin{example}
%@see: 《高等代数(第三版 上册)》(丘维声) P171 习题5.4 6.
证明:单位矩阵只与它本身相似.
\begin{proof}
设\(\vb{A} \in M_n(K)\),
\(\vb{E}\)是数域\(K\)上的\(n\)阶单位矩阵,
且\(\vb{E} \sim \vb{A}\).
根据矩阵相似的定义,
存在可逆\(\vb{P} \in M_n(K)\),
使得\(\vb{P}^{-1} \vb{E} \vb{P} = \vb{A}\).
又因为单位矩阵可以与任意同阶矩阵交换,
所以\(\vb{A}
= \vb{P}^{-1} \vb{E} \vb{P}
= \vb{P}^{-1} \vb{P} \vb{E}
= \vb{E} \vb{E}
= \vb{E}\).
\end{proof}
\end{example}

\begin{example}
%@see: https://www.bilibili.com/video/BV1dA29YeEz8/
设\begin{equation*}
	\vb{A} = \begin{bmatrix}
		1 & 2 & 0 \\
		0 & 1 & 3 \\
		0 & 0 & 1
	\end{bmatrix},
	\qquad
	\vb{B} = \begin{bmatrix}
		1 & -1 & -2 \\
		0 & 1 & 1 \\
		0 & 0 & 1
	\end{bmatrix}.
\end{equation*}
证明:\(\vb{A} \sim \vb{B}\).
\begin{solution}
要想证明\(\vb{A} \sim \vb{B}\),
我们需要找出满足等式\(\vb{P}^{-1} \vb{A} \vb{P} = \vb{B}\)的可逆矩阵\(\vb{P}\),
也就是要解矩阵方程\(\vb{A} \vb{P} = \vb{P} \vb{B}\).
将\(\vb{P}\)按列分块为\((\vb\alpha_1,\vb\alpha_2,\vb\alpha_3)\),
则有\((\vb{A} \vb\alpha_1,\vb{A} \vb\alpha_2,\vb{A} \vb\alpha_3) = (\vb\alpha_1,\vb\alpha_2,\vb\alpha_3) \vb{B}\),
即\begin{gather*}
	\vb{A} \vb\alpha_1 = \vb\alpha_1, \tag1 \\
	\vb{A} \vb\alpha_2 = -\vb\alpha_1 + \vb\alpha_2, \tag2 \\
	\vb{A} \vb\alpha_3 = -2\vb\alpha_1 + \vb\alpha_2 + a_3. \tag3
\end{gather*}

由(1)式有\begin{equation*}
	(\vb{A} - \vb{E}) \vb\alpha_1 = \vb0,
\end{equation*}
它的系数矩阵为\begin{equation*}
	\vb{A} - \vb{E}
	= \begin{bmatrix}
		0 & 2 & 0 \\
		0 & 0 & 3 \\
		0 & 0 & 0
	\end{bmatrix}
	\to \begin{bmatrix}
		0 & 1 & 0 \\
		0 & 0 & 1 \\
		0 & 0 & 0
	\end{bmatrix},
\end{equation*}
所以\(\vb\alpha_1 = \begin{bmatrix}
	k_1 \\
	0 \\
	0
\end{bmatrix}
\ (\text{$k_1$是任意常数})\).

由(2)式有\begin{equation*}
	(\vb{A} - \vb{E}) \vb\alpha_2 = -\vb\alpha_1,
\end{equation*}
它的增广矩阵为\begin{equation*}
	(\vb{A} - \vb{E},-\vb\alpha_1)
	= \begin{bmatrix}
		0 & 2 & 0 & -k_1 \\
		0 & 0 & 3 & 0 \\
		0 & 0 & 0 & 0
	\end{bmatrix},
\end{equation*}
所以\(\vb\alpha_2 = \begin{bmatrix}
	k_2 \\
	-\frac12 k_1 \\
	0
\end{bmatrix}
\ (\text{$k_2$是任意常数})\).

由(3)式有\begin{equation*}
	(\vb{A} - \vb{E}) \vb\alpha_3 = -2\vb\alpha_1 + \vb\alpha_2,
\end{equation*}
它的增广矩阵为\begin{equation*}
	(\vb{A} - \vb{E},-2\vb\alpha_1 + \vb\alpha_2)
	= \begin{bmatrix}
		0 & 2 & 0 & k_2 - 2 k_1 \\
		0 & 0 & 3 & -\frac12 k_1 \\
		0 & 0 & 0 & 0
	\end{bmatrix},
\end{equation*}
所以\(\vb\alpha_3 = \begin{bmatrix}
	k_3 \\
	\frac12 k_2 - k_1 \\
	-\frac16 k_1
\end{bmatrix}
\ (\text{$k_3$是任意常数})\).

于是\begin{equation*}
	\vb{P} = (\vb\alpha_1,\vb\alpha_2,\vb\alpha_3)
	= \begin{bmatrix}
		k_1 & k_2 & k_3 \\
		0 & -\frac12 k_1 & \frac12 k_2 - k_1 \\
		0 & 0 & -\frac16 k_1
	\end{bmatrix}
	\quad(\text{$k_1,k_2,k_3$是任意常数}).
\end{equation*}
由于\(\vb{P}\)是可逆矩阵,
所以\(\abs{\vb{P}} = \frac12 k_1^3 \neq 0\),
说明\(k_1\)必须满足约束性条件\(k_1 \neq 0\),
因此所求可逆矩阵\(\vb{P}\)为\begin{equation*}
	\vb{P} = \begin{bmatrix}
		k_1 & k_2 & k_3 \\
		0 & -\frac12 k_1 & \frac12 k_2 - k_1 \\
		0 & 0 & -\frac16 k_1
	\end{bmatrix}
	\quad(\text{$k_1,k_2,k_3$是任意常数,且$k_1\neq0$}).
\end{equation*}

既然可逆矩阵\(\vb{P}\)存在,
那么\(\vb{A} \sim \vb{B}\).
\end{solution}
\end{example}

\subsection{矩阵相似的性质}
\begin{proposition}
%@see: 《高等代数(第三版 上册)》(丘维声) P169 命题1
%@see: 《线性代数》(张慎语、周厚隆) P97 性质2
设矩阵\(\vb{A}_1,\vb{A}_2,\vb{B}_1,\vb{B}_2,\vb{P} \in M_n(K)\),\(\vb{P}\)可逆,
且\begin{equation*}
	\vb{B}_1=\vb{P}^{-1}\vb{A}_1\vb{P}, \qquad
	\vb{B}_2=\vb{P}^{-1}\vb{A}_2\vb{P},
\end{equation*}
则\begin{gather}
	\vb{B}_1 + \vb{B}_2 = \vb{P}^{-1} (\vb{A}_1 + \vb{A}_2) \vb{P}, \\
	\vb{B}_1 \vb{B}_2 = \vb{P}^{-1} (\vb{A}_1 \vb{A}_2) \vb{P}, \\
	m\in\mathbb{N} \implies \vb{B}_1^m = \vb{P}^{-1}\vb{A}_1^m\vb{P}.
		\label{equation:相似矩阵.利用相似性简化计算}
\end{gather}
\end{proposition}
\begin{corollary}\label{theorem:相似矩阵.相似矩阵的多项式相似}
设\(f(x)\)是数域\(K\)上的一个一元多项式,
矩阵\(\vb{A},\vb{B} \in M_n(K)\),且\(\vb{A} \sim \vb{B}\),
则\begin{equation*}
	f(\vb{A}) \sim f(\vb{B}).
\end{equation*}
\end{corollary}
\begin{example}
举例说明:数域\(K\)上的一个一元多项式\(f(x)\)和矩阵\(\vb{A},\vb{B} \in M_n(K)\)满足\begin{equation*}
	f(\vb{A}) \sim f(\vb{B}),
\end{equation*}
但不满足\(\vb{A} \sim \vb{B}\).
\begin{solution}
%@credit: {0275c083-b4f8-46fa-96e2-cce85388d500}
取\(f(x) = x^2 - x,
\vb{A} = \vb{E},
\vb{B} = \vb0\),
其中\(\vb{E}\)是数域\(K\)上的\(n\)阶单位矩阵.
显然\begin{equation*}
	f(\vb{A}) = \vb{E}^2 - \vb{E} = \vb0,
	\qquad
	f(\vb{B}) = \vb0^2 - \vb0 = \vb0,
\end{equation*}
但是\(\vb{E}\)与\(\vb0\)不相似.
\end{solution}
\end{example}

\begin{property}\label{theorem:特征值与特征向量.矩阵相似的必要条件1}
%@see: 《线性代数》(张慎语、周厚隆) P97 性质1
%@see: 《高等代数(第三版 上册)》(丘维声) P169 1°
相似矩阵的行列式相等.
\begin{proof}
设\(\vb{A},\vb{B} \in M_n(K)\).
假设\(\vb{A}\sim\vb{B}\),
那么存在数域\(K\)上\(n\)阶可逆矩阵\(\vb{P}\),
使得\begin{equation*}
	\vb{P}^{-1}\vb{A}\vb{P}=\vb{B},
\end{equation*}
两端取行列式,得\begin{equation*}
	\abs{\vb{B}} = \abs{\vb{P}^{-1}\vb{A}\vb{P}}
	= \abs{\vb{P}^{-1}}\abs{\vb{A}}\abs{\vb{P}}
	= \abs{\vb{P}}^{-1}\abs{\vb{A}}\abs{\vb{P}}
	= \abs{\vb{A}}.
	\qedhere
\end{equation*}
\end{proof}
\end{property}
\begin{proposition}
%@see: 《高等代数(第三版 上册)》(丘维声) P169 2°
设\(\vb{A},\vb{B} \in M_n(K)\).
若\(\vb{A}\sim\vb{B}\),则\(\vb{A}\)和\(\vb{B}\)同为可逆或不可逆.
\begin{proof}
由\cref{theorem:特征值与特征向量.矩阵相似的必要条件1} 立即可得.
\end{proof}
\end{proposition}
\begin{remark}
如果\(\vb{A},\vb{B}\)可逆,
那么对\begin{equation*}
	\vb{P}^{-1} \vb{A} \vb{P} = \vb{B}
\end{equation*}取逆,
由\cref{theorem:逆矩阵.矩阵乘积的逆2} 得\begin{equation*}
	\vb{P}^{-1} \vb{A}^{-1} \vb{P}
	= (\vb{P}^{-1} \vb{A} \vb{P})^{-1}
	= \vb{B}^{-1},
\end{equation*}
即\(\vb{A}^{-1} \sim \vb{B}^{-1}\),
于是\cref{equation:相似矩阵.利用相似性简化计算}
可以推广为\(m\in\mathbb{Z} \implies \vb{B}^m = \vb{P}^{-1}\vb{A}^m\vb{P}\).
这也说明:
相似矩阵的同次幂也相似.
同理,我们也可以把\cref{theorem:相似矩阵.相似矩阵的多项式相似} 的前提条件
“\(f(x)\)是数域\(K\)上的一个一元多项式”
推广为“\(f(x)\)是数域\(K\)上的一个一元罗朗多项式”.
%@see: https://mathworld.wolfram.com/LaurentPolynomial.html
\end{remark}

\begin{property}\label{theorem:特征值与特征向量.矩阵相似的必要条件3}
%@see: 《线性代数》(张慎语、周厚隆) P97 性质3
相似矩阵有相同的特征多项式,从而有相同的特征值.
\begin{proof}
设\(\vb{A},\vb{B} \in M_n(K)\).
假设\(\vb{A}\sim\vb{B}\),
那么存在数域\(K\)上\(n\)阶可逆矩阵\(\vb{P}\),
使得\begin{equation*}
	\vb{P}^{-1}\vb{A}\vb{P}=\vb{B},
\end{equation*}
于是\begin{equation*}
	\abs{\lambda\vb{E}-\vb{B}}
	=\abs{\vb{P}^{-1}(\lambda\vb{E}-\vb{A})\vb{P}}
	=\abs{\vb{P}^{-1}}\abs{\lambda\vb{E}-\vb{A}}\abs{\vb{P}}
	=\abs{\lambda\vb{E}-\vb{A}}.
	\qedhere
\end{equation*}
\end{proof}
\end{property}
\begin{remark}
\cref{theorem:特征值与特征向量.矩阵相似的必要条件3} 只是矩阵相似的必要不充分条件.
下面我们举出一条反例,不相似的两个矩阵有相同的特征多项式和特征值.
取\begin{equation*}
	\vb{A} = \begin{bmatrix} 2 & 1 \\ 0 & 2 \end{bmatrix},
	\quad\text{和}\quad
	\vb{B} = \begin{bmatrix} 2 & 0 \\ 0 & 2 \end{bmatrix},
\end{equation*}
显然两者的特征多项式相同,都是\((\lambda-2)^2\),
故而两者的特征值也相同,都是\(\lambda=2\ (\text{二重})\).
但\(\vb{A}\)与\(\vb{B}\)不相似,
这是因为\(\vb{B}=2\vb{E}\)是数乘矩阵,可以和所有二阶矩阵交换,
那么对任意二阶可逆矩阵\(\vb{P}\)都有\(\vb{P}^{-1}\vb{B}\vb{P}=\vb{B}\vb{P}^{-1}\vb{P}=\vb{B}\),
即\(\vb{B}\)只能与自身相似,
\(\vb{A}\)与\(\vb{B}\)不相似.
\end{remark}

\begin{property}\label{theorem:特征值与特征向量.相似矩阵的迹的不变性}
%@see: 《高等代数(第三版 上册)》(丘维声) P170 4°
相似矩阵有相同的迹.
\begin{proof}
假设\(\vb{A}\sim\vb{B}\),
那么存在数域\(K\)上\(n\)阶可逆矩阵\(\vb{P}\),
使得\begin{equation*}
	\vb{P}^{-1}\vb{A}\vb{P}=\vb{B},
\end{equation*}
于是由\cref{theorem:矩阵的迹.矩阵乘积交换次序不变迹} 有\begin{equation*}
	\tr\vb{B}
	= \tr(\vb{P}^{-1}\vb{A}\vb{P})
	= \tr(\vb{P}(\vb{P}^{-1}\vb{A}))
	= \tr\vb{A}.
	\qedhere
\end{equation*}
\end{proof}
\end{property}

\begin{property}\label{theorem:特征值与特征向量.相似矩阵的秩的不变性}
%@see: 《高等代数(第三版 上册)》(丘维声) P170 3°
相似矩阵有相同的秩.
\begin{proof}
由\cref{theorem:矩阵乘积的秩.与可逆矩阵相乘不变秩} 可得.
\end{proof}
\end{property}
\begin{example}
%@see: 《2018年全国硕士研究生入学统一考试(数学一)》一选择题/第5题/选项(B)
证明:矩阵\begin{equation*}
	\vb{A} = \begin{bmatrix}
		1 & 1 & 0 \\
		0 & 1 & 1 \\
		0 & 0 & 1
	\end{bmatrix}
	\quad\text{与}\quad
	\vb{B} = \begin{bmatrix}
		1 & 0 & -1 \\
		0 & 1 & 1 \\
		0 & 0 & 1
	\end{bmatrix}
\end{equation*}不相似.
\begin{proof}
%@see: https://www.bilibili.com/video/BV1E9s2eSEYR/
因为\begin{gather*}
	\vb{A}-\vb{E} = \begin{bmatrix}
		0 & 1 & 0 \\
		0 & 0 & 1 \\
		0 & 0 & 0
	\end{bmatrix}
	\qquad
	\vb{B}-\vb{E} = \begin{bmatrix}
		0 & 0 & -1 \\
		0 & 0 & 1 \\
		0 & 0 & 0
	\end{bmatrix}, \\
	\rank(\vb{A}-\vb{E}) = 2
	\neq
	\rank(\vb{B}-\vb{E}) = 1,
\end{gather*}
所以\(\vb{A}-\vb{E} \not\sim \vb{B}-\vb{E}\),
于是\(\vb{A} \not\sim \vb{B}\).
\end{proof}
\end{example}

\begin{property}
相似矩阵与原矩阵等价,即\(\vb{A}\sim\vb{B} \implies \vb{A}\cong\vb{B}\).
\begin{proof}
由于\hyperref[theorem:特征值与特征向量.相似矩阵的秩的不变性]{相似矩阵的秩的不变性},
而\hyperref[theorem:矩阵乘积的秩.矩阵等价的充分必要条件]{秩相等的矩阵等价},
所以相似矩阵必定等价.
\end{proof}
\end{property}

\begin{remark}
由\cref{theorem:特征值与特征向量.矩阵相似的必要条件1,theorem:特征值与特征向量.相似矩阵的迹的不变性,theorem:特征值与特征向量.相似矩阵的秩的不变性} 可知,
数域\(K\)上的\(n\)阶方阵的行列式、秩、迹
都是相似关系下的不变量,
我们把这三个量统称为\DefineConcept{相似不变量}.
\end{remark}

\subsection{相似类}
\begin{property}\label{theorem:特征值与特征向量.相似关系是等价关系}
%@see: 《线性代数》(张慎语、周厚隆) P97
%@see: 《高等代数(第三版 上册)》(丘维声) P169
数域\(K\)上的\(n\)阶矩阵之间的相似关系,
是数域\(K\)上的全体\(n\)阶矩阵\(M_n(K)\)上的等价关系,
因为它满足:\begin{itemize}
	\item {\rm\bf 反身性}:
	\((\forall \vb{A} \in M_n(K))
	[\vb{A}\sim\vb{A}]\).

	\item {\rm\bf 对称性}:
	\((\forall \vb{A},\vb{B} \in M_n(K))
	[\vb{A} \sim \vb{B} \implies \vb{B} \sim \vb{A}]\).

	\item {\rm\bf 传递性}:
	\((\forall \vb{A},\vb{B},\vb{C} \in M_n(K))
	[\vb{A} \sim \vb{B}, \vb{B} \sim \vb{C} \implies \vb{A} \sim \vb{C}]\).
\end{itemize}
\begin{proof}
在\cref{equation:特征值与特征向量.相似矩阵的定义} 中,
令\(\vb{A}=\vb{B}\)、\(\vb{P}=\vb{E}\),得\(\vb{E}\vb{A}\vb{E}=\vb{A}\),
即有相似矩阵的反身性成立.

再在\cref{equation:特征值与特征向量.相似矩阵的定义} 中取\(\vb{Q}=\vb{P}^{-1}\),
得\(\vb{A} = \vb{Q}^{-1}(\vb{P}^{-1}\vb{A}\vb{P})\vb{Q} = \vb{Q}^{-1}\vb{B}\vb{Q}\),即有相似矩阵的对称性成立.

设\(\vb{P}_1^{-1}\vb{A}\vb{P}_1=\vb{B},
\vb{P}_2^{-1}\vb{B}\vb{P}_2=\vb{C}\),
于是\(\vb{P}_2^{-1}(\vb{P}_1^{-1}\vb{A}\vb{P}_1)\vb{P}_2=\vb{C}\).
取\(\vb{Q}=\vb{P}_1\vb{P}_2\),\(\vb{Q}\)是可逆矩阵,且\(\vb{Q}^{-1}\vb{A}\vb{Q}=\vb{C}\),所以\(\vb{A}\sim\vb{C}\),
即有相似矩阵的传递性成立.
\end{proof}
\end{property}

\begin{definition}
%@see: 《高等代数(第三版 上册)》(丘维声) P169
把矩阵\(\vb{A} \in M_n(K)\)在相似关系下的等价类\begin{equation*}
	\Set{ \vb{B} \in M_n(K) \given \vb{A}\sim\vb{B} }
\end{equation*}称为“矩阵\(\vb{A}\)的\DefineConcept{相似类}”.
\end{definition}

\begin{example}
%@see: 《高等代数(第三版 上册)》(丘维声) P171 习题5.4 1.
设矩阵\(\vb{A},\vb{B} \in M_n(K)\).
证明:如果\(\vb{A} \sim \vb{B}\),则\begin{gather}
	(\forall k \in K)
	[k\vb{A} \sim k\vb{B}], \\
	\vb{A}^T \sim \vb{B}^T.
\end{gather}
\begin{proof}
假设可逆矩阵\(\vb{P}\)满足\begin{equation*}
	\vb{P}^{-1}\vb{A}\vb{P} = \vb{B},
\end{equation*}
那么由矩阵运算规律可知,
对于\(\forall k \in K\),
成立\begin{equation*}
	\vb{P}^{-1}(k\vb{A})\vb{P}
	= k(\vb{P}^{-1}\vb{A}\vb{P})
	= k\vb{B},
	\quad\text{和}\quad
	\vb{P}^T\vb{A}^T(\vb{P}^T)^{-1}
	= (\vb{P}^{-1}\vb{A}\vb{P})^T
	= \vb{B}^T,
\end{equation*}
因此\(k\vb{A} \sim k\vb{B}\)且\(\vb{A}^T \sim \vb{B}^T\).
\end{proof}
\end{example}
\begin{example}
举例说明:即使矩阵\(\vb{A},\vb{B} \in M_n(K)\)相似,还是有\(\vb{A} + \vb{A}^T\)与\(\vb{B} + \vb{B}^T\)不相似.
\begin{solution}
取\begin{equation*}
	\vb{A} = \begin{bmatrix}
		1 & 0 \\
		0 & -1
	\end{bmatrix},
	\qquad
	\vb{B} = \begin{bmatrix}
		-3 & -2 \\
		4 & 3
	\end{bmatrix}.
\end{equation*}
\end{solution}
%@Mathematica: A = {{1, 0}, {0, -1}}
%@Mathematica: B = {{-3, -2}, {4, 3}}
%@Mathematica: Eigenvalues[A + Transpose[A]]
%@Mathematica: Eigenvalues[B + Transpose[B]]
\end{example}
\begin{example}
设矩阵\(\vb{A}\)可逆,\(\vb{A}^T\)是\(\vb{A}\)的转置.
证明:\(\vb{A}\vb{A}^T \sim \vb{A}^T\vb{A}\).
\begin{proof}
取\(\vb{P}=\vb{A}^{-1}\),
则\(\vb{P}^{-1}=\vb{A}\),\(\vb{P}\vb{A}=\vb{A}\vb{P}=\vb{E}\),
于是\begin{equation*}
	\vb{P}(\vb{A}\vb{A}^T)\vb{P}^{-1}
	= (\vb{P}\vb{A})(\vb{A}^T\vb{P}^{-1})
	= \vb{A}^T\vb{P}^{-1}
	= \vb{A}^T\vb{A},
\end{equation*}
故\(\vb{A}\vb{A}^T \sim \vb{A}^T\vb{A}\).
\end{proof}
\end{example}
\begin{example}
%@see: 《高等代数(第三版 上册)》(丘维声) P171 习题5.4 2.
%@see: 《线性代数》(张慎语、周厚隆) P105 习题5.2 5.
设矩阵\(\vb{A},\vb{B} \in M_n(K)\).
证明:如果\(\vb{A}\)可逆,则\(\vb{A}\vb{B} \sim \vb{B}\vb{A}\).
\begin{proof}
因为\(\vb{A}^{-1}(\vb{A}\vb{B})\vb{A}
= \vb{B}\vb{A}\),
所以\(\vb{A}\vb{B} \sim \vb{B}\vb{A}\).
\end{proof}
\end{example}
\begin{example}\label{example:相似矩阵.分块对角矩阵的相似性}
%@see: 《高等代数(第三版 上册)》(丘维声) P171 习题5.4 3.
设矩阵\(\vb{A}_1,\vb{B}_1 \in M_s(K),
\vb{A}_2,\vb{B}_2 \in M_n(K)\).
证明:如果\(\vb{A}_1 \sim \vb{B}_1,\vb{A}_2 \sim \vb{B}_2\),
则\begin{equation*}
	\begin{bmatrix}
		\vb{A}_1 & \vb0 \\
		\vb0 & \vb{A}_2
	\end{bmatrix}
	\sim \begin{bmatrix}
		\vb{B}_1 & \vb0 \\
		\vb0 & \vb{B}_2
	\end{bmatrix}.
\end{equation*}
\begin{proof}
假设可逆矩阵\(\vb{P}_1 \in M_s(K)\)
和可逆矩阵\(\vb{P}_2 \in M_n(K)\)满足\begin{equation*}
	\vb{P}_1^{-1} \vb{A}_1 \vb{P}_1 = \vb{B}_1,
	\qquad
	\vb{P}_2^{-1} \vb{A}_2 \vb{P}_2 = \vb{B}_2,
\end{equation*}
那么\begin{equation*}
	\begin{bmatrix}
		\vb{P}_1^{-1} & \vb0 \\
		\vb0 & \vb{P}_2^{-1}
	\end{bmatrix}
	\begin{bmatrix}
		\vb{A}_1 & \vb0 \\
		\vb0 & \vb{A}_2
	\end{bmatrix}
	\begin{bmatrix}
		\vb{P}_1 & \vb0 \\
		\vb0 & \vb{P}_2
	\end{bmatrix}
	= \begin{bmatrix}
		\vb{P}_1^{-1} \vb{A}_1 \vb{P}_1 & \vb0 \\
		\vb0 & \vb{P}_2^{-1} \vb{A}_2 \vb{P}_2
	\end{bmatrix}
	= \begin{bmatrix}
		\vb{B}_1 & \vb0 \\
		\vb0 & \vb{B}_2
	\end{bmatrix}.
	\qedhere
\end{equation*}
\end{proof}
\end{example}
\begin{example}
%@see: 《高等代数(第三版 上册)》(丘维声) P171 习题5.4 7.
证明:数量矩阵只与它本身相似.
\begin{proof}
设\(\vb{A} \in M_n(K)\),
\(\vb{E}\)是数域\(K\)上的\(n\)阶单位矩阵,
\(k \in K\),
且\(k\vb{E} \sim \vb{A}\).
根据矩阵相似的定义,
存在可逆\(\vb{P} \in M_n(K)\),
使得\(\vb{P}^{-1} (k\vb{E}) \vb{P} = \vb{A}\),
于是\(\vb{A} = k\vb{E}\).
\end{proof}
\end{example}
\begin{example}\label{example:幂等矩阵.幂等矩阵的相似类}
%@see: 《高等代数(第三版 上册)》(丘维声) P171 习题5.4 10.
证明:与幂等矩阵相似的矩阵仍是幂等矩阵.
\begin{proof}
假设\(\vb{A}\)是数域\(K\)上的一个\(n\)阶幂等矩阵,
即\(\vb{A}^2=\vb{A}\).
假设\(\vb{A}\)与数域\(K\)上的某个\(n\)阶矩阵\(\vb{B}\)相似,
那么存在可逆矩阵\(\vb{P}\),使得\begin{equation*}
	\vb{P}^{-1}\vb{A}\vb{P} = \vb{B},
\end{equation*}
从而有\begin{equation*}
	\vb{P}\vb{B}^2\vb{P}^{-1}
	= (\vb{P}\vb{B}\vb{P}^{-1})(\vb{P}\vb{B}\vb{P}^{-1})
	= \vb{A}^2
	= \vb{A}
	= \vb{P}\vb{B}\vb{P}^{-1},
\end{equation*}
于是\(\vb{B}^2=\vb{B}\),
即\(\vb{B}\)也是幂等矩阵.
\end{proof}
\end{example}
\begin{example}\label{example:对合矩阵.对合矩阵的相似类}
%@see: 《高等代数(第三版 上册)》(丘维声) P171 习题5.4 11.
证明:与对合矩阵相似的矩阵仍是对合矩阵.
\begin{proof}
假设\(\vb{E}\)是数域\(K\)上的\(n\)阶单位矩阵,
\(\vb{A}\)是数域\(K\)上的一个\(n\)阶对合矩阵,
即\(\vb{A}^2=\vb{E}\).
假设\(\vb{A}\)与数域\(K\)上的某个\(n\)阶矩阵\(\vb{B}\)相似,
那么存在可逆矩阵\(\vb{P}\),使得\begin{equation*}
	\vb{P}^{-1}\vb{A}\vb{P} = \vb{B},
\end{equation*}
从而有\begin{equation*}
	\vb{P}\vb{B}^2\vb{P}^{-1}
	= (\vb{P}\vb{B}\vb{P}^{-1})(\vb{P}\vb{B}\vb{P}^{-1})
	= \vb{A}^2
	= \vb{E},
\end{equation*}
于是\(\vb{B}^2=\vb{E}\),
即\(\vb{B}\)也是对合矩阵.
\end{proof}
\end{example}
\begin{example}\label{example:幂零矩阵.幂零矩阵的相似类}
%@see: 《高等代数(第三版 上册)》(丘维声) P171 习题5.4 12.
证明:与幂零矩阵相似的矩阵仍是幂零矩阵.
\begin{proof}
假设\(\vb{A}\)是数域\(K\)上的一个以\(m\)为幂零指数的\(n\)阶幂零矩阵,
即\(\vb{A}^m=\vb0\).
假设\(\vb{A}\)与数域\(K\)上的某个\(n\)阶矩阵\(\vb{B}\)相似,
那么存在可逆矩阵\(\vb{P}\),使得\begin{equation*}
	\vb{P}^{-1}\vb{A}\vb{P} = \vb{B},
\end{equation*}
从而有\begin{equation*}
	\vb{P}\vb{B}^m\vb{P}^{-1}
	= (\vb{P}\vb{B}\vb{P}^{-1})^m
	= \vb{A}^m
	= \vb0,
\end{equation*}
于是\(\vb{B}^m=\vb0\),
即\(\vb{B}\)也是以\(m\)为幂零指数的幂零矩阵.
\end{proof}
\end{example}

\section{相似对角化}
\begin{definition}\label{definition:相似对角化.相似对角化}
%@see: 《高等代数(第三版 上册)》(丘维声) P171
设矩阵\(\vb{A} \in M_n(K)\).
\begin{itemize}
	\item 如果存在\(n\)阶对角矩阵\(\vb\Lambda\)相似于\(\vb{A}\),
	则称“矩阵\(\vb{A}\)可以\DefineConcept{相似对角化}”
	%@see: https://mathworld.wolfram.com/DiagonalizableMatrix.html
	“矩阵\(\vb\Lambda\)是\(\vb{A}\)的\DefineConcept{相似标准型}”.
	\item 如果不存在\(n\)阶对角矩阵相似于\(\vb{A}\),
	则称“矩阵\(\vb{A}\)不可以{相似对角化}”.
\end{itemize}
%@see: https://mathworld.wolfram.com/MatrixDiagonalization.html
\end{definition}

\begin{theorem}\label{theorem:矩阵相似对角化.矩阵可以相似对角化的充分必要条件}
%@see: 《高等代数(第三版 上册)》(丘维声) P171 定理2
%@see: 《高等代数(第三版 上册)》(丘维声) P180 定理1
矩阵\(\vb{A} \in M_n(K)\)可以相似对角化的充分必要条件是:
\(\vb{A}\)有\(n\)个线性无关的特征向量.
\begin{proof}
必要性.
假设\(\vb{A}\)可以相似对角化,
即存在数域\(K\)上的\(n\)阶可逆矩阵\(\vb{P}=(\AutoTuple{\vb{x}}{n})\)
和数域\(K\)上的\(n\)阶对角阵\(\vb\Lambda=\diag(\AutoTuple{\lambda}{n})\),
使\[
	\vb{P}^{-1}\vb{A}\vb{P}=\vb{\Lambda}.
\]
用\(\vb{P}\)左乘上式两端,得\[
	\vb{A}\vb{P}=\vb{P}\vb{\Lambda}.
\]
由于\(\vb{P}\)可逆,所以\(\AutoTuple{\vb{x}}{n}\)线性无关,有\[
	\vb{A}(\AutoTuple{\vb{x}}{n})
	=(\AutoTuple{\vb{x}}{n})\vb{\Lambda},
\]
由分块矩阵乘法法则,得\[
	(\vb{A}\vb{x}_1,\vb{A}\vb{x}_2,\dotsc,\vb{A}\vb{x}_n)
	=(\lambda_1\vb{x}_1,\lambda_2\vb{x}_2,\dotsc,\lambda_n\vb{x}_n),
\]
于是\[
	\vb{A}\vb{x}_i=\lambda_i\vb{x}_i,
	\quad i=1,2,\dotsc,n,
\]
即\(\AutoTuple{\vb{x}}{n}\)是矩阵\(\vb{A}\)的
分别属于\(\AutoTuple{\lambda}{n}\)的\(n\)个线性无关的特征向量.

同理可证充分性.
\end{proof}
\end{theorem}
\begin{remark}
%@see: 《线性代数》(张慎语、周厚隆) P99
从\cref{theorem:矩阵相似对角化.矩阵可以相似对角化的充分必要条件} 的证明过程可知:
{\color{red}当\(\vb{P}^{-1}\vb{A}\vb{P}=\vb{\Lambda}\)时,
\(\vb{\Lambda}\)的\(n\)个主对角元是\(\vb{A}\)的\(n\)个特征值,
可逆矩阵\(\vb{P}\)的\(n\)个列向量\(\AutoTuple{\vb{x}}{n}\)是
\(\vb{A}\)分别属于\(\lambda_1,\lambda_2,\dotsc,\lambda_n\)的线性无关的特征向量.}
\end{remark}

\begin{example}
设矩阵\(\vb{A} \in M_n(K)\)可以相似对角化.
证明:\(\vb{A}\)的伴随矩阵\(\vb{A}^*\)也可以相似对角化,
且存在可逆矩阵\(\vb{P}\),
使得\begin{equation*}
	\vb{P}^{-1} \vb{A} \vb{P} = \vb\Lambda,
	\qquad
	\vb{P}^{-1} \vb{A}^* \vb{P} = \vb\Lambda^*.
\end{equation*}
\begin{proof}
假设\(\vb{A}\)可以相似对角化,
即存在可逆矩阵\(\vb{P}\)使得\begin{equation*}
	\vb{P}^{-1} \vb{A} \vb{P} = \vb\Lambda,
	\eqno(1)
\end{equation*}
那么\begin{equation*}
	(\vb{P}^{-1} \vb{A} \vb{P})^*
	= \vb{P}^* \vb{A}^* (\vb{P}^{-1})^*.
	\eqno(2)
\end{equation*}
由\cref{theorem:逆矩阵.逆矩阵的唯一性}
可知\begin{equation*}
	\vb{P}^* = \abs{\vb{P}} \vb{P}^{-1},
	\qquad
	(\vb{P}^{-1})^* = \abs{\vb{P}^{-1}} (\vb{P}^{-1})^{-1}
	= \abs{\vb{P}}^{-1} \vb{P},
\end{equation*}
代入(2)式得\begin{equation*}
	(\vb{P}^{-1} \vb{A} \vb{P})^*
	= \vb{P}^{-1} \vb{A}^* \vb{P},
\end{equation*}
再代入(1)式得\begin{equation*}
	\vb\Lambda^*
	= \vb{P}^{-1} \vb{A}^* \vb{P}.
	\qedhere
\end{equation*}
\end{proof}
\end{example}

\begin{example}
%@see: 《2024年全国硕士研究生入学统一考试(数学一)》一选择题/第7题
设\(\vb{A}\)是秩为2的3阶矩阵,
\(\vb\alpha\)是满足\(\vb{A} \vb\alpha = \vb0\)的非零向量,
若对满足\(\vb\beta^T \vb\alpha = 0\)的3维列向量\(\vb\beta\),
均有\(\vb{A} \vb\beta = \vb\beta\),
求\(\tr\vb{A}^3\).
\begin{solution}
由于\(\vb{A} \vb\alpha = \vb0\)且\(\vb\alpha \neq \vb0\),
所以\(\lambda_1 = 0\)是\(\vb{A}\)的一个特征值.
但是\(\vb{A}\)的属于特征值\(\lambda_1\)的特征矩阵的秩为\[
	\rank(0\vb{E}-\vb{A})
	= \rank\vb{A}
	= 2,
\]
其中\(\vb{E}\)是3阶单位矩阵,
故\[
	\dim\Ker(0\vb{E}-\vb{A})
	= 3 - \rank(0\vb{E}-\vb{A})
	= 1,
\]
即\(\vb{A}\)只有1个属于\(\lambda_1\)的特征向量.

%TODO 这里用到正交补的概念,需要进一步阐述
易知子空间\(\Set{\vb\beta \given \vb\beta^T \vb\alpha = 0}\)的维数为\[
	\dim\Set{\vb\beta \given \vb\beta^T \vb\alpha = 0} = 2,
\]
%TODO 还可以这样说:当\(\vb\beta^T \vb\alpha = 0\)时,
% 成立\((\vb\beta^T \vb\alpha)^T = \vb\alpha^T \vb\beta = 0\),
% 又因为\(\rank(\vb\alpha^T) = 1\),
% 所以关于\(\vb\beta\)的方程\(\vb\alpha^T \vb\beta = 0\)有\(3-1=2\)个线性无关的解.
那么在其中必定存在两个线性无关的非零向量\(\vb\beta_1,\vb\beta_2\),
使得\begin{equation*}
	\vb\beta_1^T \vb\alpha
	= \vb\beta_2^T \vb\alpha
	= \vb0,
\end{equation*}
再根据题设条件有\[
	\vb{A}\vb\beta_1 = \vb\beta_1,
	\qquad
	\vb{A}\vb\beta_2 = \vb\beta_2,
\]
于是\(\lambda_2 = 1\ (\text{二重})\)是\(\vb{A}\)的一个特征值.

因此,由\hyperref[theorem:矩阵相似对角化.矩阵可以相似对角化的充分必要条件]{矩阵可以相似对角化的充分必要条件}可知
\(\vb{A} \sim \diag(1,1,0)\),
即存在可逆矩阵\(\vb{P}\),
使得\[
	\vb{P}^{-1} \vb{A} \vb{P}
	= \diag(1,1,0),
\]
于是\[
	\vb{P}^{-1} \vb{A}^n \vb{P}
	= (\vb{P}^{-1} \vb{A} \vb{P})^n
	= \diag(1^n,1^n,0^n)
	= \diag(1,1,0),
\]
即\(\vb{A}^n \sim \diag(1,1,0)\ (n\in\mathbb{N}^+)\),
那么由\hyperref[theorem:特征值与特征向量.相似矩阵的迹的不变性]{相似矩阵的迹的不变性}可知\[
	\tr\vb{A}^3
	= \tr\vb{A}^2
	= \tr\vb{A}
	= 1+1+0
	= 2.
\]
\end{solution}
\end{example}

我们已经知道,方程\((\lambda\vb{E}-\vb{A})\vb{x}=\vb0\)的基础解系是
矩阵\(\vb{A}\)的属于特征值\(\lambda\)的线性无关的特征向量.
于是自然提出问题:不同特征值的线性无关特征向量是否构成线性无关组?
关于这个问题,我们给出如下定理.
\begin{theorem}\label{theorem:矩阵相似对角化.不同特征值的特征向量线性无关}
%@see: 《线性代数》(张慎语、周厚隆) P99 定理2
%@see: 《线性代数》(张慎语、周厚隆) P100 定理3
%@see: 《高等代数(第三版 上册)》(丘维声) P181 定理2
%@see: 《高等代数(第三版 上册)》(丘维声) P181 定理3
%@see: 《高等代数(第三版 上册)》(丘维声) P181 推论4
矩阵的属于不同特征值的特征向量线性无关.
\begin{proof}
设矩阵\(\vb{A} \in M_n(K)\)的\(m\)个不同的特征值\(\lambda_1,\lambda_2,\dotsc,\lambda_m\)
对应的特征向量分别为\(\AutoTuple{\vb{x}}{m}\).

由上述所有特征向量构成的向量组,
记作\(X_m=\{\AutoTuple{\vb{x}}{m}\}\).

当\(m=1\)时,
由于\(\vb{x}_1 \neq 0\),
故向量组\(X_1=\{\vb{x}_1\}\)线性无关.

当\(m>1\)时,用数学归纳法,
假设\(m-1\)个不同特征值对应的特征向量\[
	X_{m-1}=\{\AutoTuple{\vb{x}}{m-1}\}
\]线性无关.
对于\(m\)个不同特征值对应的特征向量组\(X_m\),
考虑方程\[
	k_1\vb{x}_1+k_2\vb{x}_2+\dotsb+k_{m-1}\vb{x}_{m-1}+k_m\vb{x}_m=\vb0,
	\eqno(1)
\]
由于\(\vb{A}\vb{x}_j=\lambda_j\vb{x}_j\),
用\(\vb{A}\)左乘(1)式两端,
得\[
	k_1\lambda_1\vb{x}_1+k_2\lambda_2\vb{x}_2
	+\dotsb+k_{m-1}\lambda_{m-1}\vb{x}_{m-1}+k_m\lambda_m\vb{x}_m=\vb0.
	\eqno(2)
\]
再用\(\lambda_m\)数乘(1)式两端,得\[
	\lambda_mk_1\vb{x}_1+\lambda_mk_2\vb{x}_2
	+\dotsb+\lambda_mk_{m-1}\vb{x}_{m-1}+\lambda_mk_m\vb{x}_m=\vb0,
	\eqno(3)
\]
(2)(3)两式相减,得\[
	(\lambda_1-\lambda_m)k_1\vb{x}_1+(\lambda_2-\lambda_m)k_2\vb{x}_2
	+\dotsb+(\lambda_{m-1}-\lambda_m)k_{m-1}\vb{x}_{m-1}=\vb0.
	\eqno(4)
\]
根据归纳假设,向量组\(X_{m-1}\)线性无关,
则要使(4)式成立,必有\[
	(\lambda_i-\lambda_m)k_i=0\ (i=1,2,\dotsc,m-1).
	\eqno(5)
\]
由于\(\lambda_i\neq\lambda_m\ (i=1,2,\dotsc,m-1)\),
所以要使(5)式成立,必有\[
	k_1=k_2=\dotsb=k_{m-1}=0.
	\eqno(6)
\]
将(6)式代回(1)式得\(k_m\vb{x}_m=\vb0\),
考虑到特征向量\(\vb{x}_m\neq\vb0\),
于是解得\(k_m=0\).
可见方程(1)只有零解,
也就是说特征向量组\(X_m\)线性无关.
\end{proof}
\end{theorem}
\begin{remark}
由\cref{theorem:矩阵相似对角化.不同特征值的特征向量线性无关} 可知\[
	\bigcup\Set{
		\text{$\Ker(\lambda\vb{E}-\vb{A})$的基}
		\given
		\text{$\lambda$是$\vb{A}$的特征值}
	}
\]线性无关.
\end{remark}

\begin{example}
%@see: 《线性代数》(张慎语、周厚隆) P102 例2
设\[
	\vb{A} = \begin{bmatrix}
		1 & 0 & 0 \\
		-2 & 5 & -2 \\
		-2 & 4 & -1
	\end{bmatrix}.
\]
试问:\(\vb{A}\)能否相似对角化?
若能,则求出可逆矩阵\(\vb{P}\),使\(\vb{P}^{-1}\vb{A}\vb{P}\)为对角形矩阵.
\begin{solution}
\(\vb{A}\)的特征多项式为\[
	\abs{\lambda\vb{E}-\vb{A}} = \begin{bmatrix}
		\lambda-1 & 0 & 0 \\
		2 & \lambda-5 & 2 \\
		2 & -4 & \lambda+1
	\end{bmatrix}
	= (\lambda-1)^2 (\lambda-3),
\]
则\(\vb{A}\)的特征值为\(\lambda_1=1\ (\text{二重})\),\(\lambda_2=3\).

当\(\lambda_1=1\)时,解齐次线性方程组\((\vb{E}-\vb{A})\vb{x}=\vb0\),\[
	\vb{E}-\vb{A}=\begin{bmatrix}
		0 & 0 & 0 \\
		2 & -4 & 2 \\
		2 & -4 & 2
	\end{bmatrix}
	\to \begin{bmatrix}
		1 & -2 & 1 \\
		0 & 0 & 0 \\
		0 & 0 & 0
	\end{bmatrix},
\]
基础解系为\(\vb{x}_1 = \begin{bmatrix} 2 \\ 1 \\ 0 \end{bmatrix},
\vb{x}_2 = \begin{bmatrix} -1 \\ 0 \\ 1 \end{bmatrix}\).

对于\(\lambda_2=3\),解方程组\((3\vb{E}-\vb{A})\vb{x}=\vb0\),\[
	3\vb{E}-\vb{A}=\begin{bmatrix}
		2 & 0 & 0 \\
		2 & -2 & 2 \\
		2 & -4 & 4
	\end{bmatrix} \to \begin{bmatrix}
		2 & 0 & 0 \\
		0 & -2 & 2 \\
		0 & 0 & 0
	\end{bmatrix},
\]
基础解系为\(\vb{x}_3 = \begin{bmatrix} 0 \\ 1 \\ 1 \end{bmatrix}\).

特征向量\(\vb{x}_1,\vb{x}_2,\vb{x}_3\)线性无关,所以\(\vb{A}\)可以相似对角化.
令\[
	\vb{P} = \begin{bmatrix} \vb{x}_1 & \vb{x}_2 & \vb{x}_3 \end{bmatrix} = \begin{bmatrix}
		2 & -1 & 0 \\
		1 & 0 & 1 \\
		0 & 1 & 1
	\end{bmatrix},
	\quad\text{则有}\quad
	\vb{P}^{-1} \vb{A} \vb{P} = \begin{bmatrix} 1 \\ & 1 \\ && 3 \end{bmatrix}.
\]
\end{solution}
\end{example}

\begin{example}
%@see: 《线性代数》(张慎语、周厚隆) P102 例3
设\(\vb{A} = \begin{bmatrix}
	2 & 0 & 0 \\
	0 & 2 & 0 \\
	0 & 1 & 2
\end{bmatrix}\).
证明:\(\vb{A}\)不可以相似对角化.
\begin{proof}
\(\vb{A}\)的特征多项式为\[
	\abs{\lambda\vb{E}-\vb{A}} = \begin{vmatrix}
		\lambda-2 & 0 & 0 \\
		0 & \lambda-2 & 0 \\
		0 & -1 & \lambda-2
	\end{vmatrix} = (\lambda-2)^2,
\]
令\(\abs{\lambda\vb{E}-\vb{A}} = 0\)解得特征值\(\lambda_1=2\ (\text{三重})\).
由于\(\rank(\lambda_1\vb{E}-\vb{A})=1\),
那么对应于唯一的特征值\(\lambda_1=2\),
\(\vb{A}\)只有两个线性无关的特征向量,
因而不存在可逆矩阵\(\vb{P}\)使得\(\vb{P}^{-1}\vb{A}\vb{P}\)为对角形矩阵.
\end{proof}
\end{example}

从上述例子可以看出:
当矩阵\(\vb{A}\)的某个特征值\(\lambda_0\)是\(k\)重根时,
矩阵\(\vb{A}\)的属于特征值\(\lambda_0\)的线性无关的特征向量的个数
可能等于\(k\),也可能小于\(k\).
这个规律对于一般的矩阵是成立的.

\begin{definition}
%@see: 《高等代数(第三版 上册)》(丘维声) P183 习题5.6 11.
设\(\lambda \in K\)是矩阵\(\vb{A} \in M_n(K)\)的一个特征值.
\begin{itemize}
	\item 把\(\vb{A}\)的属于\(\lambda\)的特征子空间的维数\(\dim\Ker(\lambda\vb{E}-\vb{A})\)
	称为“\(\lambda\)的\DefineConcept{几何重数}”.
	\item 把\(\lambda\)作为\(\vb{A}\)的特征多项式的根的重数
	称为“\(\lambda\)的\DefineConcept{代数重数}”.
\end{itemize}
\end{definition}
\begin{theorem}\label{theorem:矩阵相似对角化.特征值的几何重数与代数重数的关系}
%@see: 《线性代数》(张慎语、周厚隆) P103 定理4
%@see: 《高等代数(第三版 上册)》(丘维声) P183 习题5.6 11.
矩阵的任意一个特征值的几何重数不大于它的代数重数.
%TODO proof
\end{theorem}
\begin{remark}
\cref{theorem:矩阵相似对角化.特征值的几何重数与代数重数的关系} 说明:
矩阵\(\vb{A}\)的\(k\)重特征值\(\lambda_0\)的线性无关的特征向量最多只有\(k\)个.
\end{remark}

\begin{theorem}\label{theorem:矩阵可以相似对角化的充分必要条件.定理2}
%@see: 《线性代数》(张慎语、周厚隆) P103 定理5
\(n\)阶矩阵\(\vb{A}\)可以相似对角化的充分必要条件是:
对于\(\vb{A}\)的每个\(k_i\)重特征值\(\lambda_i\),
\(\vb{A}\)有\(k_i\)个线性无关的特征向量,
即\[
	\dim\Ker(\lambda_i\vb{E}-\vb{A}) = k_i.
\]
%TODO proof
\end{theorem}
\begin{remark}
\cref{theorem:矩阵可以相似对角化的充分必要条件.定理2} 说明:
矩阵\(\vb{A}\)可以相似对角化当且仅当它的每个特征值的代数重数与几何重数相等.
\end{remark}

\begin{corollary}\label{theorem:矩阵可以相似对角化的充分必要条件.定理3}
%@see: 《线性代数》(张慎语、周厚隆) P103 推论
\(n\)阶矩阵\(\vb{A}\)可以相似对角化的充分必要条件是:
对于\(\vb{A}\)的每个\(k_i\)重特征值\(\lambda_i\),
都有\[
	\rank(\lambda_i\vb{E}-\vb{A}) = n-k_i.
\]
%TODO proof
\end{corollary}

\begin{theorem}\label{theorem:矩阵可以相似对角化的充分必要条件.定理4}
%@see: 《高等代数(第三版 上册)》(丘维声) P182 定理5
矩阵\(\vb{A} \in M_n(K)\)可以相似对角化的充分必要条件是:
\(\vb{A}\)的属于不同特征值的特征子空间的维数之和等于\(n\).
%TODO proof
\end{theorem}

由\cref{theorem:矩阵可以相似对角化的充分必要条件.定理4} 立即得到矩阵可以相似对角化的一个充分不必要条件:
\begin{corollary}\label{theorem:矩阵可以相似对角化的充分条件.定理1}
%@see: 《线性代数》(张慎语、周厚隆) P100 推论
%@see: 《高等代数(第三版 上册)》(丘维声) P182 推论6
设\(\vb{A} \in M_n(K)\).
“矩阵\(\vb{A}\)有\(n\)个不同的特征值”是
“\(\vb{A}\)可以相似对角化”的充分不必要条件.
\begin{proof}
首先证明充分性.
假设矩阵\(\vb{A}\)有\(n\)个不同的特征值\(\AutoTuple{\lambda}{n}\),
而\(\vb{x}_j\ (j=1,2,\dotsc,n)\)是矩阵\(\vb{A}\)的属于特征值\(\lambda_j\)的特征向量.
由\cref{theorem:矩阵相似对角化.不同特征值的特征向量线性无关},
\(\AutoTuple{\vb{x}}{n}\)线性无关.
再由\cref{theorem:矩阵相似对角化.矩阵可以相似对角化的充分必要条件},
\(\vb{A}\)可以相似对角化.

然后证伪必要性.
取\(\vb{A}\)为\(n\)阶单位矩阵,显然\(\vb{A}\)本来就是对角矩阵,当然可以相似对角化,
但是\(\vb{A}\)只有唯一一个特征值\(1\).
\end{proof}
\end{corollary}

\begin{example}
%@see: 《1988年全国硕士研究生入学统一考试(数学一)》八解答题
已知矩阵\(\vb{A} = \begin{bmatrix}
	2 & 0 & 0 \\
	0 & 0 & 1 \\
	0 & 1 & x
\end{bmatrix}\)与\(\vb{B} = \begin{bmatrix}
	2 & 0 & 0 \\
	0 & y & 0 \\
	0 & 0 & -1
\end{bmatrix}\)相似.
求\(x\)与\(y\).
\begin{solution}
因为\(\vb{A}\sim\vb{B}\),所以,由\cref{theorem:特征值与特征向量.矩阵相似的必要条件1},
\[
	\begin{vmatrix}
		2 & 0 & 0 \\
		0 & 0 & 1 \\
		0 & 1 & x
	\end{vmatrix}
	= -2 = -2y =
	\begin{vmatrix}
		2 & 0 & 0 \\
		0 & y & 0 \\
		0 & 0 & -1
	\end{vmatrix}
	\implies y = 1;
\]
又由\hyperref[theorem:特征值与特征向量.相似矩阵的迹的不变性]{相似矩阵的迹的不变性}可知\[
	\tr\vb{A} = 2+x
	= 1+y = \tr\vb{B}
	\implies
	x = 0.
\]
\end{solution}
\end{example}
\begin{example}
%@see: 《1992年全国硕士研究生入学统一考试(数学三)》三解答题/第17题
设矩阵\(\vb{A}\)与\(\vb{B}\)相似,
其中\begin{equation*}
%@Mathematica: A = ({ {-2, 0, 0}, {2, x, 2}, {3, 1, 1} })
%@Mathematica: B = ({ {-1, 0, 0}, {0, 2, 0}, {0, 0, y} })
	\vb{A} = \begin{bmatrix}
		-2 & 0 & 0 \\
		2 & x & 2 \\
		3 & 1 & 1
	\end{bmatrix},
	\qquad
	\vb{B} = \begin{bmatrix}
		-1 & 0 & 0 \\
		0 & 2 & 0 \\
		0 & 0 & y
	\end{bmatrix}.
\end{equation*}
求可逆矩阵\(\vb{P}\),使得\(\vb{P}^{-1} \vb{A} \vb{P} = \vb{B}\).
\begin{solution}
由于\(\vb{A} \sim \vb{B}\),
所以\begin{gather*}
	%\cref{theorem:特征值与特征向量.相似矩阵的迹的不变性}
	\tr\vb{A} = x - 1 = y + 1 = \tr\vb{B}, \\
	% 这里没有列出行列式的相等关系,是因为迹与行列式的相等关系成比例
	%\cref{theorem:特征值与特征向量.矩阵相似的必要条件3}
	\abs{\lambda\vb{E}-\vb{A}}
	= \begin{vmatrix}
		\lambda+2 & 0 & 0 \\
		-2 & \lambda-x & -2 \\
		-3 & -1 & \lambda-1
	\end{vmatrix}
	= \begin{vmatrix}
		\lambda+1 & 0 & 0 \\
		0 & \lambda-2 & 0 \\
		0 & 0 & \lambda-y
	\end{vmatrix}
	= \abs{\lambda\vb{E}-\vb{B}},
\end{gather*}
%@Mathematica: Solve[Tr[A] == Tr[B] && Det[k IdentityMatrix[3] - A] == Det[k IdentityMatrix[3] - B], {x, y}]
故\(x = 0,
y = -2\),
从而有\begin{equation*}
%@Mathematica: A = A /. x -> 0
%@Mathematica: B = B /. y -> -2
	\vb{A} = \begin{bmatrix}
		-2 & 0 & 0 \\
		2 & x & 2 \\
		3 & 1 & 1
	\end{bmatrix},
	\qquad
	\vb{B} = \begin{bmatrix}
		-1 & 0 & 0 \\
		0 & 2 & 0 \\
		0 & 0 & y
	\end{bmatrix}.
\end{equation*}
% 矩阵\(\vb{B}\)的特征值为\(-2,-1,2\).
考虑方程\(\vb{A} \vb{X} = \vb{X} \vb{B}\),
将\(\vb{X}\)按列分块为\(\vb\alpha_1,\vb\alpha_2,\vb\alpha_3\),
则有\begin{gather*}
	\vb{A} \vb\alpha_1 = -\vb\alpha_1, \tag1 \\
	\vb{A} \vb\alpha_2 = 2\vb\alpha_2, \tag2 \\
	\vb{A} \vb\alpha_3 = -2\vb\alpha_3. \tag3
\end{gather*}

由(1)式有\((\vb{A}+\vb{E}) \vb\alpha_1 = \vb0\),
它的系数矩阵为\begin{equation*}
%@Mathematica: A + IdentityMatrix[3] // MatrixForm
%@Mathematica: RowReduce[A + IdentityMatrix[3]] // MatrixForm
	\begin{bmatrix}
		-1 & 0 & 0 \\
		2 & 1 & 2 \\
		3 & 1 & 2
	\end{bmatrix}
	\to \begin{bmatrix}
		1 & 0 & 0 \\
		0 & 1 & 2 \\
		0 & 0 & 0
	\end{bmatrix},
\end{equation*}
所以\(\vb\alpha_1 = \begin{bmatrix}
	0 \\
	2k_1 \\
	-k_1
\end{bmatrix}
\ (\text{$k_1$是任意常数})\).

由(2)式有\((\vb{A}-2\vb{E}) \vb\alpha_2 = \vb0\),
它的系数矩阵为\begin{equation*}
%@Mathematica: A - 2 IdentityMatrix[3] // MatrixForm
%@Mathematica: RowReduce[A - 2 IdentityMatrix[3]] // MatrixForm
	\begin{bmatrix}
		-4 & 0 & 0 \\
		2 & -2 & 2 \\
		3 & 1 & -1
	\end{bmatrix}
	\to \begin{bmatrix}
		1 & 0 & 0 \\
		0 & 1 & -1 \\
		0 & 0 & 0
	\end{bmatrix},
\end{equation*}
所以\(\vb\alpha_2 = \begin{bmatrix}
	0 \\
	k_2 \\
	k_2
\end{bmatrix}
\ (\text{$k_2$是任意常数})\).

由(3)式有\((\vb{A}+2\vb{E}) \vb\alpha_3 = \vb0\),
它的系数矩阵为\begin{equation*}
%@Mathematica: A + 2 IdentityMatrix[3] // MatrixForm
%@Mathematica: RowReduce[A + 2 IdentityMatrix[3]] // MatrixForm
	\begin{bmatrix}
		0 & 0 & 0 \\
		2 & 2 & 2 \\
		3 & 1 & 3
	\end{bmatrix}
	\to \begin{bmatrix}
		1 & 0 & 1 \\
		0 & 1 & 0 \\
		0 & 0 & 0
	\end{bmatrix},
\end{equation*}
所以\(\vb\alpha_3 = \begin{bmatrix}
	k_3 \\
	0 \\
	-k_3
\end{bmatrix}
\ (\text{$k_3$是任意常数})\).

于是\begin{equation*}
	\vb{X} = (\vb\alpha_1,\vb\alpha_2,\vb\alpha_3)
	= \begin{bmatrix}
		0 & 0 & k_3 \\
		2k_1 & k_2 & 0 \\
		-k_1 & k_2 & -k_3
	\end{bmatrix}
	\quad(\text{$k_1,k_2,k_3$是任意常数}).
\end{equation*}
由于要求的是可逆矩阵,
所以\(\abs{\vb{X}} = 2 k_1 k_2 k_3 \neq 0\),
说明\(k_1,k_2,k_3\)全不为零,
那么所求可逆矩阵\(\vb{P}\)为\begin{equation*}
	\vb{P} = \begin{bmatrix}
		0 & 0 & k_3 \\
		2k_1 & k_2 & 0 \\
		-k_1 & k_2 & -k_3
	\end{bmatrix}
	\quad(\text{$k_1,k_2,k_3$是任意常数,$k_1,k_2,k_3$全不为零}).
\end{equation*}
\end{solution}
\end{example}
\begin{example}
%@see: 《2019年全国硕士研究生入学统一考试(数学一)》三解答题/第21题
已知矩阵\(\vb{A} = \begin{bmatrix}
	-2 & -2 & 1 \\
	2 & x & -2 \\
	0 & 0 & -2
\end{bmatrix}\)
与\(\vb{B} = \begin{bmatrix}
	2 & 1 & 0 \\
	0 & -1 & 0 \\
	0 & 0 & y
\end{bmatrix}\)相似.
求可逆矩阵\(\vb{P}\),使得\(\vb{P}^{-1} \vb{A} \vb{P} = \vb{B}\).
\begin{solution}
由于\(\vb{A} \sim \vb{B}\),
所以\begin{gather*}
	%\cref{theorem:特征值与特征向量.相似矩阵的迹的不变性}
	\tr\vb{A} = x - 4 = 1 + y = \tr\vb{B}, \\
	%\cref{theorem:特征值与特征向量.矩阵相似的必要条件1}
	\abs{\vb{A}} = 4(x-2) = -2y = \abs{\vb{B}},
\end{gather*}
解得\(x=3,
y=-2\),
那么\begin{equation*}
	\vb{A} = \begin{bmatrix}
		-2 & -2 & 1 \\
		2 & 3 & -2 \\
		0 & 0 & -2
	\end{bmatrix},
	\qquad
	\vb{B} = \begin{bmatrix}
		2 & 1 & 0 \\
		0 & -1 & 0 \\
		0 & 0 & -2
	\end{bmatrix}.
\end{equation*}
由\(\abs{\lambda\vb{E}-\vb{B}}=0\)解得\(\vb{B}\)的特征值为\(-2,-1,2\),
那么由\cref{theorem:矩阵可以相似对角化的充分条件.定理1} 可知
\(\vb{B}\)可以相似对角化,
%\cref{theorem:特征值与特征向量.矩阵相似的必要条件3}
故存在可逆矩阵\(\vb{P}_1,\vb{P}_2\),
它们的列向量分别是\(\vb{A},\vb{B}\)的特征向量,
使得\begin{equation*}
	\vb{P}_1^{-1} \vb{A} \vb{P}_1 = \vb\Lambda,
	\qquad
	\vb{P}_2^{-1} \vb{B} \vb{P}_2 = \vb\Lambda,
\end{equation*}
其中\(\vb\Lambda\)是某个对角矩阵,
那么\begin{equation*}
	\vb{P}_2 \vb{P}_1^{-1} \vb{A} \vb{P}_1 \vb{P}_2^{-1} = \vb{B},
\end{equation*}
因此所求可逆矩阵\(\vb{P}\)就是\(\vb{P}_1 \vb{P}_2^{-1}\).

计算\(\vb{A}\)的属于特征值\(2\)的特征向量,
解方程\((2\vb{E}-\vb{A}) \vb{x} = \vb0\)得
\(\vb{x}_{11} = (-1,2,0)^T\).

计算\(\vb{A}\)的属于特征值\(-1\)的特征向量,
解方程\((-\vb{E}-\vb{A}) \vb{x} = \vb0\)得
\(\vb{x}_{12} = (-2,1,0)^T\).

计算\(\vb{A}\)的属于特征值\(-2\)的特征向量,
解方程\((-2\vb{E}-\vb{A}) \vb{x} = \vb0\)得
\(\vb{x}_{13} = (-1,2,4)^T\).

因此\(\vb{P}_1\)可以取为\begin{math}
%@Mathematica: P1 = ({ {-1, -2, -1}, {2, 1, 2}, {0, 0, 4} })
	(\vb{x}_{11},\vb{x}_{12},\vb{x}_{13})
	= \begin{bmatrix}
		-1 & -2 & -1 \\
		2 & 1 & 2 \\
		0 & 0 & 4
	\end{bmatrix}.
\end{math}

计算\(\vb{B}\)的属于特征值\(2\)的特征向量,
解方程\((2\vb{E}-\vb{B}) \vb{x} = \vb0\)得
\(\vb{x}_{21} = (1,0,0)^T\).

计算\(\vb{B}\)的属于特征值\(-1\)的特征向量,
解方程\((-\vb{E}-\vb{B}) \vb{x} = \vb0\)得
\(\vb{x}_{22} = (-1,3,0)^T\).

计算\(\vb{B}\)的属于特征值\(-2\)的特征向量,
解方程\((-2\vb{E}-\vb{B}) \vb{x} = \vb0\)得
\(\vb{x}_{23} = (0,0,1)^T\).

因此\(\vb{P}_2\)可以取为\begin{math}
%@Mathematica: P2 = ({ {1, -1, 0}, {0, 3, 0}, {0, 0, 1} })
	(\vb{x}_{21},\vb{x}_{22},\vb{x}_{23})
	= \begin{bmatrix}
		1 & -1 & 0 \\
		0 & 3 & 0 \\
		0 & 0 & 1
	\end{bmatrix}.
\end{math}

接下来计算\(\vb{P} = \vb{P}_1 \vb{P}_2^{-1}\).
因为\begin{equation*}
%@Mathematica: Join[P1, P2, 1].({ {1, 1/3, 0}, {0, 1/3, 0}, {0, 0, 1} }) // MatrixForm
	\begin{bmatrix}
		\vb{P}_1 \\
		\vb{P}_2
	\end{bmatrix}
	= \begin{bmatrix}
		-1 & -2 & -1 \\
		2 & 1 & 2 \\
		0 & 0 & 4 \\
		1 & -1 & 0 \\
		0 & 3 & 0 \\
		0 & 0 & 1
	\end{bmatrix}
	\to \begin{bmatrix}
		-1 & -1 & -1 \\
		2 & 1 & 2 \\
		0 & 0 & 4 \\
		1 & 0 & 0 \\
		0 & 1 & 0 \\
		0 & 0 & 1
	\end{bmatrix}
	= \begin{bmatrix}
		\vb{P}_1 \vb{P}_2^{-1} \\
		\vb{E}
	\end{bmatrix},
\end{equation*}
所以\begin{equation*}
%@Mathematica: P1.Inverse[P2] // MatrixForm
	\vb{P} = \begin{bmatrix}
		-1 & -1 & -1 \\
		2 & 1 & 2 \\
		0 & 0 & 4
	\end{bmatrix}.
\end{equation*}
\end{solution}
\end{example}

\begin{example}
%@see: 《线性代数》(张慎语、周厚隆) P105 习题5.2 6.
设\(\vb{A}\)为可逆矩阵且可以相似对角化,证明:\(\vb{A}^{-1}\)也可以相似对角化.
\begin{proof}
设存在可逆矩阵\(\vb{P}\)使得\[
	\vb{P}^{-1}\vb{A}\vb{P} = \vb{\Lambda},
	\eqno(1)
\]
其中\(\vb{\Lambda}=\diag(\lambda_1,\lambda_2,\dotsc,\lambda_n)\),
\(n\)是矩阵\(\vb{A}\)的阶数,
\(\lambda_1,\lambda_2,\dotsc,\lambda_n\)是矩阵\(\vb{A}\)的特征值.
显然有\[
	\abs{\vb{\Lambda}}
	= \abs{\vb{P}^{-1}\vb{A}\vb{P}}
	= \abs{\vb{P}^{-1}}\abs{\vb{A}}\abs{\vb{P}}
	= (\abs{\vb{P}^{-1}}\abs{\vb{P}})\abs{\vb{A}}
	= 1 \cdot \abs{\vb{A}}
	= \abs{\vb{A}} \neq 0
\]
即\(\vb{\Lambda}\)可逆.
在(1)式两端左乘\(\vb{P}\)得\(\vb{P}(\vb{P}^{-1}\vb{A}\vb{P}) = \vb{P}\vb{\Lambda}\)
即\[
	\vb{A}\vb{P} = \vb{P}\vb{\Lambda}.
	\eqno(2)
\]
在(2)式两端左乘\(\vb{P}^{-1}\vb{A}^{-1}\),
右乘\(\vb{\Lambda}^{-1}\)得\[
	(\vb{P}^{-1}\vb{A}^{-1})(\vb{A}\vb{P})\vb{\Lambda}^{-1} = (\vb{P}^{-1}\vb{A}^{-1})(\vb{P}\vb{\Lambda})\vb{\Lambda}^{-1},
\]
即\(\vb{\Lambda}^{-1} = \vb{P}^{-1}\vb{A}^{-1}\vb{P}\).
\end{proof}
\end{example}

\begin{example}
%@see: 《线性代数》(张慎语、周厚隆) P105 习题5.2 7.
设\(m\)阶矩阵\(\vb{A}\)与\(n\)阶矩阵\(\vb{B}\)都可以相似对角化,证明:\(m+n\)阶矩阵\[
	\begin{bmatrix} \vb{A} & \vb0 \\ \vb0 & \vb{B} \end{bmatrix}
\]可以相似对角化.
\begin{proof}
设\(\vb{A} \sim \vb\Lambda_1,
\vb{B} \sim \vb\Lambda_2\),
那么由\cref{example:相似矩阵.分块对角矩阵的相似性} 可知\begin{equation*}
	\begin{bmatrix}
		\vb{A} & \vb0 \\
		\vb0 & \vb{B}
	\end{bmatrix}
	\sim \begin{bmatrix}
		\vb\Lambda_1 & \vb0 \\
		\vb0 & \vb\Lambda_2
	\end{bmatrix}.
	\qedhere
\end{equation*}
\end{proof}
\end{example}

\begin{example}\label{example:幂零矩阵.非零的幂零矩阵不可以相似对角化}
%@see: 《线性代数》(张慎语、周厚隆) P105 习题5.2 8.
%@see: 《高等代数(第三版 上册)》(丘维声) P183 习题5.6 6.
证明:非零的幂零矩阵不可以相似对角化.
\begin{proof}
用反证法.
假设\(\vb{A}\)可以相似对角化,
即存在可逆矩阵\(\vb{P}\)使得\[
	\vb{P}^{-1}\vb{A}\vb{P} = \diag(\lambda_1,\lambda_2,\dotsc,\lambda_n) = \vb0.
	\eqno(1)
\]
由\cref{example:幂零矩阵.幂零矩阵的特征值的性质}
可知\(\vb{A}\)的特征值全为零,
即\[
	\lambda_1 = \lambda_2 = \dotsb = \lambda_n = 0.
\]
在(1)式两边同时左乘\(\vb{P}\),并右乘\(\vb{P}^{-1}\),得\[
	\vb{A} = \vb{P}(\vb{P}^{-1}\vb{A}\vb{P})\vb{P}^{-1} = \vb{P}\vb0\vb{P}^{-1} = \vb0.
\]
矛盾,故\(\vb{A}\)不可以相似对角化.
\end{proof}
\end{example}
\begin{example}\label{example:幂零矩阵.幂等矩阵一定可以相似对角化}
%@see: 《高等代数(第三版 上册)》(丘维声) P183 习题5.6 7.
证明:幂等矩阵一定可以相似对角化.
\begin{proof}
设\(\vb{A} \in M_n(K)\)满足\(\vb{A}^2=\vb{A}\),且\(\rank\vb{A} = r\).

当\(r=0\)时,\(\vb{A}\)是零矩阵,可以相似对角化.

当\(r=n\)时,\(\vb{A}\)是单位矩阵,也可以相似对角化.

下面假设\(0<r<n\).
由\cref{example:幂等矩阵.幂等矩阵的特征值的性质} 可知\(\vb{A}\)特征值必为0或1.
由\cref{example:幂等矩阵.幂等矩阵的秩的性质1} 可知\[
	\rank\vb{A}+\rank(\vb{E}_n-\vb{A})=n.
\]
其中\(\vb{E}_n\)是数域\(K\)上的\(n\)阶单位矩阵.
由\cref{theorem:线性方程组.齐次线性方程组的解向量个数} 可知\begin{gather*}
	\rank\vb{A} + \dim\Ker\vb{A} = n, \\
	\rank(\vb{E}_n-\vb{A}) + \dim\Ker(\vb{E}_n-\vb{A}) = n,
\end{gather*}
从而有\[
	\dim\Ker\vb{A} + \dim\Ker(\vb{E}_n-\vb{A}) = n.
\]
那么由\cref{theorem:矩阵可以相似对角化的充分必要条件.定理4} 可知\(\vb{A}\)可以相似对角化.
又因为\[
	\dim\Ker(\vb{E}_n-\vb{A}) = \rank\vb{A},
\]
所以特征值\(1\)的代数重数和几何重数都是\(r\),
于是\(\vb{A}\)的相似标准型是分块对角阵\(\diag(\vb{E}_r,\vb0)\),
其中\(\vb{E}_r\)是数域\(K\)上的\(r\)阶单位矩阵.
\end{proof}
\end{example}
\begin{example}
%@see: 《高等代数(第三版 上册)》(丘维声) P183 习题5.6 8.
证明:幂等矩阵的秩等于它的迹.
\begin{proof}
由\cref{example:幂零矩阵.幂等矩阵一定可以相似对角化} 可知
秩为\(r\)的幂等矩阵\(\vb{A}\)的相似标准型是分块对角阵\(\diag(\vb{E}_r,\vb0)\),
其中\(\vb{E}_r\)是数域\(K\)上的\(r\)阶单位矩阵.
因为迹是相似不变量,所以\[
	\tr\vb{A} = \tr\diag(\vb{E}_r,\vb0) = r.
	\qedhere
\]
\end{proof}
\end{example}
\begin{example}
%@see: 《高等代数(第三版 上册)》(丘维声) P183 习题5.6 9.
证明:对合矩阵一定可以相似对角化.
%TODO proof
% \begin{proof}
% 设\(\vb{A} \in M_n(K)\)满足\(\vb{A}^2=\vb{E}\),且\(\rank\vb{A} = r\).

% 当\(\vb{A}\)与数域\(K\)上的\(n\)阶单位矩阵\(\vb{E}_n\)满足\(A = \pm\vb{E}_n\)时,
% 显然\(\vb{A}\)可以相似对角化,且它的相似标准型就是\(\vb{A}\).

% 下面假设\(\vb{A} \neq \pm\vb{E}_n\).
% \end{proof}
\end{example}

\begin{example}
%@see: 《高等代数(第三版 上册)》(丘维声) P171 习题5.4 8.
设矩阵\(\vb{A} \in M_n(K)\).
证明:如果\(\vb{A}\)可以相似对角化,则\(\vb{A} \sim \vb{A}^T\).
%TODO proof
\end{example}

\begin{example}
设\(\vb{A} \in M_n(K)\),
\(\rank\vb{A}=1\).
证明:\[
	\tr\vb{A}\neq0
	\iff
	\text{\(\vb{A}\)可相似对角化}.
\]
\begin{proof}
因为\(\rank\vb{A}=1
\iff
(\exists\vb\alpha,\vb\beta \in K^n-\{\vb0\})[\vb{A}=\vb\alpha\vb\beta^T]\),
所以根据\cref{example:矩阵乘积的秩.两个向量的乘积的特征值和特征向量},
\(\vb{A}\)的特征值为\(\tr\vb{A}\)和\(0\ (\text{$n-1$重})\).
又因为\(\rank(0\vb{E}-\vb{A})=\rank\vb{A}=1\),
所以根据\cref{theorem:矩阵可以相似对角化的充分必要条件.定理3} 可知,
\(\vb{A}\)可以相似对角化.
\end{proof}
\end{example}

\begin{example}
%@see: 《线性代数》(张慎语、周厚隆) P105 习题5.2 9.
形式为\[
	\vb{J}_n = \begin{bmatrix}
		\lambda_0 & 0 & 0 & \dots & 0 & 0 \\
		1 & \lambda_0 & 0 & \dots & 0 & 0 \\
		0 & 1 & \lambda_0 & \dots & 0 & 0 \\
		\vdots & \vdots & \vdots & \ddots & \vdots & \vdots \\
		0 & 0 & 0 & \dots & \lambda_0 & 0 \\
		0 & 0 & 0 & \dots & 1 & \lambda_0
	\end{bmatrix}_n
\]的复数三角形阵称为\DefineConcept{若尔当块}(Jordan block).
%@see: https://mathworld.wolfram.com/JordanBlock.html
%@see: https://mathworld.wolfram.com/JordanCanonicalForm.html
证明:\(n>1\)阶若尔当块不可以相似对角化.
\begin{proof}
令\(\abs{\lambda\vb{E}-\vb{J}_n}=(\lambda-\lambda_0)^n=0\),
解得\(\lambda=\lambda_0\ (\text{$n$重})\),
那么\[
	\lambda_0\vb{E}-\vb{J}_n = \begin{bmatrix}
		0 \\
		-1 & 0 \\
		& -1 & 0 \\
		& & \ddots & \ddots \\
		& & & -1 & 0
	\end{bmatrix}_n,
\]
\(\rank(\lambda_0\vb{E}-\vb{J}_n)=n-1 > 0\),
故当\(n>1\)时\(\vb{J}_n\)不可以相似对角化.
\end{proof}
\end{example}

\begin{definition}
由若干个若尔当块构成的准对角矩阵称为\DefineConcept{若尔当形矩阵}.
\end{definition}

\begin{theorem}
每个\(n\)阶复数矩阵不一定与对角阵相似,但必与一个若尔当形矩阵相似.
\end{theorem}

\begin{example}
%@see: 《2024年全国硕士研究生入学统一考试(数学二)》二填空题/第15题
设\(\vb{A} \in M_3(K)\),\(\vb{A}^*\)是\(\vb{A}\)的伴随矩阵,\(\vb{E}\)是数域\(K\)上的3阶单位矩阵,
且\[
	\rank(2\vb{E}-\vb{A}) = 1,
	\qquad
	\rank(\vb{E}+\vb{A}) = 2.
\]
计算行列式\(\abs{\vb{A}^*}\).
\begin{solution}
%@see: https://www.bilibili.com/video/BV1oR1kYKEkw/
由\(\rank(2\vb{E}-\vb{A}) = 1 < 3\)可知
\(2\)是\(\vb{A}\)的一个特征值,
它的几何重数为\[
	\dim\Ker(2\vb{E}-\vb{A})
	= 3 - \rank(2\vb{E}-\vb{A})
	= 2,
\]
它的代数重数不小于\(2\).

由\(\rank(\vb{E}+\vb{A}) = 2 < 3\)可知
\(-1\)是\(\vb{A}\)的一个特征值,
它的几何重数为\[
	\dim\Ker(\vb{E}+\vb{A})
	= 3 - \rank(\vb{E}+\vb{A})
	= 1,
\]
它的代数重数不小于\(1\).

于是特征值\(-1\)的代数重数是\(1\),
从而特征值\(-1\)的几何重数是\(1\),
特征值\(2\)的几何重数是\(3-1=2\),
故\(\vb{A}\)的相似标准型是\(\diag(2,2,-1)\),
从而有\(\abs{\vb{A}} = -4\),
由\cref{equation:伴随矩阵.伴随矩阵的行列式}
可知\(\abs{\vb{A}^*} = (-4)^{3-1} = 16\).
\end{solution}
\end{example}

\section{正交矩阵}
在平面上取一个直角坐标系\(Oxy\),
设向量\(\vb\alpha,\vb\beta\)的坐标分别是\((a_1,a_2),(b_1,b_2)\).
如果\(\vb\alpha,\vb\beta\)都是单位向量,并且互相垂直,
那么它们的坐标满足:\begin{equation*}
	\begin{split}
		a_1^2+a_2^2=1, \qquad
		a_1b_1+a_2b_2=0, \\
		b_1a_1+b_2a_2=0, \qquad
		b_1^2+b_2^2=1,
	\end{split}
\end{equation*}
这组等式可以写成一个矩阵等式:\begin{equation*}
	\begin{bmatrix}
		a_1 & a_2 \\
		b_1 & b_2
	\end{bmatrix}
	\begin{bmatrix}
		a_1 & b_1 \\
		a_2 & b_2
	\end{bmatrix}
	= \begin{bmatrix}
		1 & 0 \\
		0 & 1
	\end{bmatrix}.
\end{equation*}
如果记\(\vb{A}=(\vb\alpha^T,\vb\beta^T)\),
那么上式又可写为\begin{equation*}
	\vb{A}^T\vb{A}=\vb{E}.
\end{equation*}
根据\(\vb\alpha,\vb\beta\)的几何意义,
我们很自然地把矩阵\(\vb{A}\)称为“正交矩阵”.

这一节我们来研究正交矩阵的性质,尤其是它的行(列)向量组的特性.

\subsection{正交向量组}
\begin{definition}
在欧几里得空间中,如果
\begin{itemize}
	\item 向量组\(A=\{\AutoTuple{\vb\alpha}{m}\}\)不含零向量,即\(\vb0 \notin A\);
	\item \(A\)中向量两两正交,即\(\VectorInnerProductDot{\vb\alpha_i}{\vb\alpha_j} = 0\ (i \neq j)\),
\end{itemize}
则称\(A\)为一个\DefineConcept{正交向量组},简称\DefineConcept{正交组}.
由单位向量构成的正交组叫做\DefineConcept{规范正交组}或\DefineConcept{标准正交组}.
称含有\(n\)个向量的规范正交组
\begin{equation*}
	\AutoTuple{\vb\epsilon}{n}
\end{equation*}
为\(\mathbb{R}^n\)的一个\DefineConcept{规范正交基}%
或\DefineConcept{标准正交基}(orthonormal basis).
%@see: https://mathworld.wolfram.com/OrthonormalBasis.html
\end{definition}

\begin{example}
%@see: 《2023年全国硕士研究生入学统一考试(数学一)》二填空题/第15题
已知向量\(\vb\alpha_1 = \begin{bmatrix}
	1 \\ 0 \\ 1 \\ 1
\end{bmatrix},
\vb\alpha_2 = \begin{bmatrix}
	-1 \\ -1 \\ 0 \\ 1
\end{bmatrix},
\vb\alpha_3 = \begin{bmatrix}
	0 \\ 1 \\ -1 \\ 1
\end{bmatrix},
\vb\beta = \begin{bmatrix}
	1 \\ 1 \\ 1 \\ -1
\end{bmatrix},
\vb\gamma = k_1 \vb\alpha_1 + k_2 \vb\alpha_2 + k_3 \vb\alpha_3\),
若\(\vb\gamma^T \vb\alpha_i = \vb\beta^T \alpha_i\ (i=1,2,3)\),
计算\(k_1^2 + k_2^2 + k_3^2\).
\begin{solution}
注意到\(\vb\alpha_1^T \vb\alpha_2
= \vb\alpha_2^T \vb\alpha_3
= \vb\alpha_3^T \vb\alpha_1
= 0\),
\(\AutoTuple{\vb\alpha}{3}\)是一个正交向量组,
于是\begin{equation*}
	\vb\gamma^T \vb\alpha_i
	= k_i \abs{\vb\alpha_i}^2
	% \(\abs{\vb\alpha_i}^2 = 3\)
	= 3 k_i
	\quad(i=1,2,3).
\end{equation*}
又因为\begin{equation*}
	\vb\beta^T \vb\alpha_1 = 1,
	\qquad
	\vb\beta^T \vb\alpha_2 = -3,
	\qquad
	\vb\beta^T \vb\alpha_3 = -1,
\end{equation*}
所以\(k_1 = \frac13,
k_2 = -1,
k_3 = -\frac13\),
从而有\(k_1^2 + k_2^2 + k_3^2
= \frac{11}9\).
\end{solution}
\end{example}

\subsection{正交矩阵}
\begin{definition}\label{definition:正交矩阵.正交矩阵的定义}
%@see: 《高等代数(第三版 上册)》(丘维声) P145 定义1
%@see: 《线性代数》(张慎语、周厚隆) P107 定义6
设\(\vb{Q} \in M_n(\mathbb{R})\),
\(\vb{E}\)是实数域上的\(n\)阶单位矩阵.
若\(\vb{Q}\)满足\begin{equation}\label{equation:正交矩阵.正交矩阵的定义式}
	\vb{Q}^T \vb{Q} = \vb{E},
\end{equation}
则称“\(\vb{Q}\)是\(n\)阶\DefineConcept{正交矩阵}(orthogonal matrix)”.
%@see: https://mathworld.wolfram.com/OrthogonalMatrix.html
\end{definition}

\begin{example}%\label{example:正交矩阵.二阶旋转矩阵是正交矩阵}
%@see: 《高等代数(第三版 上册)》(丘维声) P145 例1
设\(\theta\)是实数,
判断矩阵\(
	\vb{A} \defeq \begin{bmatrix}
		\cos\theta & -\sin\theta \\
		\sin\theta & \cos\theta
	\end{bmatrix}
\)是否正交矩阵.
\begin{solution}
由于\begin{align*}
	\vb{A} \vb{A}^T
	&= \begin{bmatrix}
		\cos\theta & -\sin\theta \\
		\sin\theta & \cos\theta
	\end{bmatrix}
	\begin{bmatrix}
		\cos\theta & \sin\theta \\
		-\sin\theta & \cos\theta
	\end{bmatrix} \\
	&= \begin{bmatrix}
		\cos^2\theta + \sin^2\theta & \cos\theta \sin\theta - \sin\theta \cos\theta \\
		\sin\theta \cos\theta - \cos\theta \sin\theta & \sin^2\theta + \cos^2\theta
	\end{bmatrix} \\
	&= \begin{bmatrix}
		1 & 0 \\
		0 & 1
	\end{bmatrix},
\end{align*}
所以\(\vb{A}\)是正交矩阵.
\end{solution}
\end{example}

\begin{property}
若\(\vb{A},\vb{B}\)都是\(n\)阶正交矩阵,
则\begin{itemize}
	\item \(\vb{A}\)的行列式\(\det\vb{A}\)的绝对值等于\(1\),
	即\begin{equation}\label{equation:正交矩阵.正交矩阵的行列式}
		\abs{\det\vb{A}}=1.
	\end{equation}

	\item \(\vb{A}\)可逆.

	\item \(\vb{A}\)的转置\(\vb{A}^T\)以及它的逆\(\vb{A}^{-1}\)满足
	\begin{equation}\label{equation:正交矩阵.正交矩阵的转置等于正交矩阵的逆}
		\vb{A}^{-1}=\vb{A}^T.
	\end{equation}

	\item \(\vb{A}\vb{B}\)也是正交矩阵.
\end{itemize}
\begin{proof}
在正交矩阵的定义式 \labelcref{equation:正交矩阵.正交矩阵的定义式} 等号两端分别取行列式,
利用\cref{theorem:行列式.矩阵乘积的行列式,theorem:行列式.性质1} 得\begin{equation*}
	\abs{\det\vb{A}}^2
	=\abs{\det\vb{A}^T} \abs{\det\vb{A}\vphantom{^T}}
	=\abs{\det(\vb{A}^T \vb{A})}
	=\abs{\det\vb{E}}
	=1,
\end{equation*}
开方,得\(\abs{\det\vb{A}}=1\).

因为\(\det\vb{A}\neq0\),
\(\vb{A}\)是非奇异矩阵,
所以由\cref{theorem:逆矩阵.矩阵可逆的充分必要条件1} 可知,正交矩阵\(\vb{A}\)可逆.

由\hyperref[definition:可逆矩阵.可逆矩阵的定义]{可逆矩阵的定义}有\(\vb{A}^{-1}\vb{A}=\vb{E}\),
与\hyperref[equation:正交矩阵.正交矩阵的定义式]{正交矩阵的定义式}
\(\vb{A}^T\vb{A}=\vb{E}\)比较可知\(\vb{A}^{-1}=\vb{A}^T\).

利用矩阵乘法的结合律,
便得\begin{equation*}
	(\vb{A}\vb{B})(\vb{A}\vb{B})^T
	= (\vb{A}\vb{B})(\vb{B}^T\vb{A}^T)
	= \vb{A}(\vb{B}\vb{B}^T)\vb{A}^T
	= \vb{A}\vb{E}\vb{A}^T
	= \vb{A}\vb{A}^T
	= \vb{E}.
	\qedhere
\end{equation*}
\end{proof}
\end{property}

\begin{example}
由于单位矩阵\(\vb{E}\)满足\begin{equation*}
	\vb{E}^T=\vb{E}, \qquad
	\vb{E}^T \vb{E} = \vb{E} \vb{E}^T = \vb{E},
\end{equation*}
因此\(\vb{E}\)也是正交矩阵.
\end{example}

\begin{proposition}
正交矩阵\(\vb{Q}\)的伴随矩阵\(\vb{Q}^*\)满足\begin{equation*}
	\vb{Q}^*
	= \left\{ \begin{array}{rc}
		\vb{Q}^T, & \abs{\vb{Q}}>0, \\
		-\vb{Q}^T, & \abs{\vb{Q}}<0.
	\end{array} \right.
\end{equation*}
\begin{proof}
由\cref{theorem:逆矩阵.逆矩阵的唯一性}
可知\(\vb{Q}^* = \abs{\vb{Q} \vb{Q}^{-1}}\).
再由\cref{equation:正交矩阵.正交矩阵的转置等于正交矩阵的逆}
可知\(\vb{Q}^* = \abs{\vb{Q} \vb{Q}^T}\).
最后由\cref{equation:正交矩阵.正交矩阵的行列式}
就有\(\vb{Q}^* = \pm\vb{Q}^T\).
\end{proof}
\end{proposition}
\begin{example}
%@see: https://www.bilibili.com/video/BV1eG411L7xU/
%@see: 《2013年全国硕士研究生入学统一考试(数学一)》二填空题/第13题
设\(\vb{A}\)是实数域上的\(n\ (n>2)\)阶非零矩阵,
\(\vb{A}^T\)是\(\vb{A}\)的转置矩阵,\(\vb{A}^*\)是\(\vb{A}\)的伴随矩阵.
证明:\begin{itemize}
	\item 如果\(\vb{A}^T=\vb{A}^*\),则\(\vb{A}\)是正交矩阵,且\(\abs{\vb{A}}=1\).
	\item 如果\(\vb{A}^T+\vb{A}^*=0\),则\(\vb{A}\)是正交矩阵,且\(\abs{\vb{A}}=-1\).
\end{itemize}
%TODO proof
% \begin{proof}
% 假设\(\vb{A}^T=\vb{A}^*\),
% 那么\begin{equation*}
% 	\vb{A} \vb{A}^T
% 	= \vb{A} \vb{A}^*
% 	%\cref{equation:行列式.伴随矩阵.恒等式1}
% 	= \abs{\vb{A}} \vb{E}.
% 	\eqno(1)
% \end{equation*}
% 又因为\begin{equation*}
% 	\abs{\vb{A}} \abs{\vb{A}^T}
% 	%\cref{theorem:行列式.性质1}
% 	= \abs{\vb{A}}^2,
% 	\qquad
% 	\abs{\abs{\vb{A}} \vb{E}}
% 	%\cref{theorem:行列式.性质2.推论2}
% 	= \abs{\vb{A}}^n,
% \end{equation*}
% 所以\begin{equation*}
% 	\abs{\vb{A}}^2 (1 - \abs{\vb{A}}^{n-2}) = 0,
% \end{equation*}
% 解得\(\abs{\vb{A}}=0\)或\(\abs{\vb{A}}=1\).
% \end{proof}
\end{example}

\begin{proposition}\label{theorem:正交矩阵.正交矩阵的多项式的行列式1}
行列式小于零的正交矩阵\(\vb{A}\)与单位矩阵\(\vb{E}\)之和的行列式\(\abs{\vb{A}+\vb{E}}\)等于零.
\begin{proof}
见\cref{example:正交矩阵.行列式小于零的正交矩阵与单位矩阵之和的行列式等于零}.
\end{proof}
\end{proposition}
\begin{proposition}\label{theorem:正交矩阵.正交矩阵的多项式的行列式2}
%@see: https://www.bilibili.com/video/BV1eG411L7xU/
设\(\vb{A}\)是\(n\)阶正交矩阵.
证明:\begin{itemize}
	\item 如果\(n\)是偶数,且\(\abs{\vb{A}}<0\),则\(\abs{\vb{A}-\vb{E}} = 0\).
	\item 如果\(n\)是奇数,且\(\abs{\vb{A}}>0\),则\(\abs{\vb{A}-\vb{E}} = 0\).
\end{itemize}
\begin{proof}
由\cref{equation:正交矩阵.正交矩阵的行列式} 可知,
当\(\abs{\vb{A}}<0\)时\(\abs{\vb{A}}=\abs{\vb{A}^T}=-1\),
当\(\abs{\vb{A}}>0\)时\(\abs{\vb{A}}=\abs{\vb{A}^T}=1\).

假设\(\vb{A}\)是偶数阶正交矩阵,且\(\abs{\vb{A}}<0\),
则\begin{align*}
	\abs{\vb{A}^T} \abs{\vb{A}-\vb{E}}
	%\cref{theorem:行列式.矩阵乘积的行列式}
	= \abs{\vb{A}^T (\vb{A}-\vb{E})}
	%\cref{equation:矩阵的乘法.左分配律}
	= \abs{\vb{E}-\vb{A}^T}
	%\cref{theorem:行列式.性质1}
	%\cref{theorem:矩阵的转置.性质2}
	= \abs{\vb{E}-\vb{A}}
	%\cref{theorem:行列式.性质2.推论2}
	= (-1)^n \abs{\vb{A}-\vb{E}}
	= \abs{\vb{A}-\vb{E}};
\end{align*}
又因为\(\abs{\vb{A}^T} \abs{\vb{A}-\vb{E}} = -\abs{\vb{A}-\vb{E}}\),
所以\begin{equation*}
	\abs{\vb{A}-\vb{E}}
	= -\abs{\vb{A}-\vb{E}},
\end{equation*}
于是\(\abs{\vb{A}-\vb{E}} = 0\).

假设\(\vb{A}\)是奇数阶正交矩阵,且\(\abs{\vb{A}}>0\),
则同理可知\(\abs{\vb{A}-\vb{E}} = 0\)成立.
\end{proof}
\end{proposition}
\begin{remark}
由\cref{theorem:正交矩阵.正交矩阵的多项式的行列式1,theorem:正交矩阵.正交矩阵的多项式的行列式2} 可知,
当\(\vb{A}\)是奇数阶正交矩阵时,总有\begin{equation*}
	\abs{\vb{A}+\vb{E}} \abs{\vb{A}-\vb{E}} = 0.
\end{equation*}
另外还可以看出,
行列式小于零的正交矩阵必有一个特征值为\(-1\),
行列式大于零的奇数阶正交矩阵必有一个特征值为\(1\),
行列式小于零的偶数阶正交矩阵必有一个特征值为\(1\).
\end{remark}
\begin{proposition}
%@see: https://www.bilibili.com/video/BV1eG411L7xU/
设\(\vb{A}\)是正交矩阵,\(1\)和\(-1\)都是\(\vb{A}\)的特征值,
则\(\vb{A}\)的属于\(1\)的特征向量都与\(\vb{A}\)的属于\(-1\)的特征向量正交.
\begin{proof}
设\(\vb\alpha\)是\(\vb{A}\)的属于\(1\)的特征向量,
\(\vb\beta\)是\(\vb{A}\)的属于\(-1\)的特征向量,
即\begin{equation*}
	\vb{A} \vb\alpha = \vb\alpha,
	\qquad
	\vb{A} \vb\beta = -\vb\beta.
\end{equation*}
那么\begin{equation*}
	\vb\alpha^T = (\vb{A} \vb\alpha)^T = \vb\alpha^T \vb{A}^T,
	\qquad
	\vb\beta = -A \vb\beta,
\end{equation*}
从而有\begin{equation*}
	\vb\alpha^T \vb\beta
	= (\vb\alpha^T \vb{A}^T) (-A \vb\beta)
	= - \vb\alpha^T (\vb{A}^T \vb{A}) \vb\beta
	= - \vb\alpha^T \vb\beta,
\end{equation*}
因此\(\vb\alpha^T \vb\beta = 0\).
\end{proof}
\end{proposition}

\begin{example}
设\(\vb{Q}=(\AutoTuple{\vb\alpha}{n})\)是\(n\)阶实矩阵,
则\(\vb{Q}\)是正交矩阵的充分必要条件是\(\AutoTuple{\vb\alpha}{n}\)是\(\mathbb{R}^{n \times 1}\)的规范正交基.
\begin{proof}
在\(\vb{Q}\)是\(n\)阶实矩阵的前提下,\begin{align*}
	&\text{\(\vb{Q}\)是正交矩阵}
	\iff \vb{Q}^T\vb{Q} = \vb{Q}\vb{Q}^T = \vb{E} \\
	&\iff \vb{E} = \begin{bmatrix}
		\vb\alpha_1^T \\ \vb\alpha_2^T \\ \vdots \\ \vb\alpha_n^T
	\end{bmatrix} (\AutoTuple{\vb\alpha}{n})
	= \begin{bmatrix}
		\vb\alpha_1^T \vb\alpha_1 & \vb\alpha_1^T \vb\alpha_2 & \dots & \vb\alpha_1^T \vb\alpha_n \\
		\vb\alpha_2^T \vb\alpha_1 & \vb\alpha_2^T \vb\alpha_2 & \dots & \vb\alpha_2^T \vb\alpha_n \\
		\vdots & \vdots & & \vdots \\
		\vb\alpha_n^T \vb\alpha_1 & \vb\alpha_n^T \vb\alpha_2 & \dots & \vb\alpha_n^T \vb\alpha_n
	\end{bmatrix} \\
	&\iff \vb\alpha_i^T \vb\alpha_j = (\vb\alpha_i,\vb\alpha_j)
	= \left\{ \begin{array}{ll}
		1, & i=j, \\
		0, & i \neq j,
	\end{array} \right. i,j=1,2,\dotsc,n \\
	&\iff \text{\(\AutoTuple{\vb\alpha}{n}\)是规范正交基}.
	\qedhere
\end{align*}
\end{proof}
\end{example}

可以看出,正交矩阵是由一系列初等矩阵\(\vb{P}(i,j)\)的乘积.

\begin{example}
%@see: 《高等代数(第三版 上册)》(丘维声) P152 习题4.6 4.
设\(\vb{A}\)是实数域上的\(n\)阶矩阵.
证明:如果\(\vb{A}\)具有\begin{enumerate}
	\item \(\vb{A}\)是正交矩阵,
	\item \(\vb{A}\)是对称矩阵,
	\item \(\vb{A}\)是对合矩阵,
\end{enumerate}
这三个性质中的任意两个性质,
则必有第三个性质.
%TODO
\end{example}

\begin{example}
%@see: 《高等代数(第三版 上册)》(丘维声) P152 习题4.6 5.
证明:如果正交矩阵\(\vb{A}\)是上三角矩阵,
则\(\vb{A}\)一定是对角矩阵,
并且其主对角元是\(\pm1\).
%TODO
\end{example}

\section{幺正矩阵}
\begin{definition}\label{definition:幺正矩阵.幺正矩阵的定义}
设\(\Q \in M_n(\mathbb{C})\),
\(\E\)是复数域上的\(n\)阶单位矩阵.
若\(\Q\)满足\begin{equation}\label{equation:幺正矩阵.幺正矩阵的定义式}
	\Q^H \Q = \E,
\end{equation}
则称“\(\Q\)是\(n\)阶\DefineConcept{幺正矩阵}”
或“\(Q\)是\(n\)阶\DefineConcept{酉矩阵}(unitary matrix)”.
%@see: https://mathworld.wolfram.com/UnitaryMatrix.html
\end{definition}

\begin{property}
设\(\Q \in M_n(\mathbb{C})\),
则\begin{itemize}
	\item \(\A\)的行列式\(\det\A\)的模等于\(1\),
	即\begin{equation}\label{equation:幺正矩阵.幺正矩阵的行列式}
		\abs{\det\A}=1.
	\end{equation}

	\item \(\A\)可逆.

	\item \(\Q\)的共轭转置矩阵\(\Q^H\)和它的逆矩阵\(\Q^{-1}\)
	满足\begin{equation}\label{equation:幺正矩阵.幺正矩阵的转置等于幺正矩阵的逆}
		\Q^H = -\Q^{-1}.
	\end{equation}
\end{itemize}
%TODO proof
\end{property}

\begin{example}
%@see: 《线性代数》(张慎语、周厚隆) P114 第五章综合判断题 (14)
举例说明:元素全是实数的幺正矩阵的特征值不是\(\pm1\).
\begin{solution}
取\(\A
= \begin{bmatrix}
	0 & 1 \\
	-1 & 0
\end{bmatrix} \in M_n(\mathbb{C})\),
显然\(\A^T \A
= \begin{bmatrix}
	0 & -1 \\
	1 & 0
\end{bmatrix}
\begin{bmatrix}
	0 & 1 \\
	-1 & 0
\end{bmatrix}
= \begin{bmatrix}
	1 & 0 \\
	0 & 1
\end{bmatrix}\),
但是由\[
	\abs{\lambda\E-\A}
	= \begin{vmatrix}
		\lambda & -1 \\
		1 & \lambda
	\end{vmatrix}
	= \lambda^2 + 1
	= 0
\]解得\(\lambda=\pm\iu\).
\end{solution}
\end{example}
\begin{remark}
从上面这个例子可以看出:在实数域上,正交矩阵可能没有特征值.
\end{remark}
\begin{example}
%@see: 《线性代数》(张慎语、周厚隆) P114 第五章综合判断题 (15)
%@see: 《高等代数(第三版 上册)》(丘维声) P180 习题5.5 10.(1)
设\(\A\)是幺正矩阵.
证明:\(\A\)的特征值是模为\(1\)的复数.
\begin{proof}
设\(\lambda\)是幺正矩阵\(\vb{A}\)的一个特征值,
\(\vb\alpha\)是\(\vb{A}\)的属于特征值\(\lambda\)的一个特征向量,
即成立\[
	\vb{A} \vb\alpha
	= \lambda \vb\alpha.
	\eqno(1)
\]
对(1)式取转置得\[
	\vb\alpha^T \vb{A}^T
	= \lambda \vb\alpha^T.
	\eqno(2)
\]
将(1)(2)两式相乘,得\[
	(\vb\alpha^T \vb{A}^T) (\vb{A} \vb\alpha)
	= (\lambda \vb\alpha^T) (\lambda \vb\alpha),
\]
再利用矩阵乘法的结合律和幺正矩阵的定义,得\[
	\vb\alpha^T \vb\alpha
	= \lambda^2 \vb\alpha^T \vb\alpha,
\]
移项得\[
	(1-\lambda^2) \vb\alpha^T \vb\alpha = 0.
	\eqno(3)
\]
因为\(\vb\alpha\neq\vb0\),
所以\(\vb\alpha^T \vb\alpha \neq 0\),
那么(3)式可以化为\[
	1-\lambda^2 = 0,
	\quad\text{或}\quad
	\lambda^2 = 1,
\]
因此\(\A\)的特征值只要存在,就一定是模为\(1\)的复数.
\end{proof}
\end{example}
\begin{remark}
从上面这个例子可以看出:在实数域上,正交矩阵的特征值只要存在就只能是\(\pm1\).
\end{remark}

\section{正交规范化方法}
在许多实际问题中,我们需要构造正交矩阵,于是我们要设法求标准正交基.

\begin{figure}[htb]
%@see: 《高等代数(第三版 上册)》(丘维声) P149 图4-2
	\centering
	\begin{tikzpicture}[scale=3,>=Stealth,->]
		\draw(0,0)--(2,1)node[right]{\(\vb\alpha_2\)};
		\draw(0,0)--(0,1)node[left]{\(\vb\beta_2\)};
		\draw(2,1)--(0,1)node[midway,above]{\(k\vb\alpha_1\)};
		\draw(0,0)--(1,0)node[below]{\(\vb\alpha_1=\vb\beta_1\)};
	\end{tikzpicture}
	\caption{}
	\label{figure:正交规范化方法.二维几何空间中两不共线向量的正交规范化}
\end{figure}

平面上给定两个不共线的向量\(\vb\alpha_1,\vb\alpha_2\),
如\cref{figure:正交规范化方法.二维几何空间中两不共线向量的正交规范化},
我们很容易找到一个正交向量组:\begin{align*}
	\vb\beta_1 &= \vb\alpha_1, \\
	\vb\beta_2 &= \vb\alpha_2 + k \vb\beta_1,
\end{align*}
其中\(k\)是待定系数.

为了求出待定系数\(k\),
在上式两边用\(\vb\beta_1\)作内积,得\begin{equation*}
	\VectorInnerProductDot{\vb\beta_2}{\vb\beta_1}
	= \VectorInnerProductDot{(\vb\alpha_2 + k \vb\beta_1)}{\vb\beta_1}.
\end{equation*}
从而\begin{equation*}
	0 = \VectorInnerProductDot{\vb\alpha_2}{\vb\beta_1} + k \VectorInnerProductDot{\vb\beta_1}{\vb\beta_1}.
\end{equation*}
因此\begin{equation*}
%@see: 《高等代数(第三版 上册)》(丘维声) P149 (14)
	k = -\frac{\VectorInnerProductDot{\vb\alpha_2}{\vb\beta_1}}{\VectorInnerProductDot{\vb\beta_1}{\vb\beta_1}}.
\end{equation*}
于是\begin{equation*}
	\vb\beta_2
	= \vb\alpha_2
	- \frac{\VectorInnerProductDot{\vb\alpha_2}{\vb\beta_1}}{\VectorInnerProductDot{\vb\beta_1}{\vb\beta_1}}~\vb\beta_1.
\end{equation*}

从几何上的这个例子受到启发,
对于欧几里得空间\(\mathbb{R}^n\),
我们可以从一个线性无关的向量组出发,
构造一个正交向量组.

\begin{theorem}
%@see: 《高等代数(第三版 上册)》(丘维声) P149 定理4
设\(\vb\alpha_1,\vb\alpha_2,\dotsc,\vb\alpha_m\)是\(\mathbb{R}^n\)中的一个线性无关组,
令\begin{align*}
	\vb\beta_1 &= \vb\alpha_1, \\
	\vb\beta_k &= \vb\alpha_k - \sum_{i=1}^{k-1}
		\frac{\VectorInnerProductDot{\vb\alpha_k}{\vb\beta_i}}{\VectorInnerProductDot{\vb\beta_i}{\vb\beta_i}} \vb\beta_i,
	\quad k=2,3,\dotsc,m,
\end{align*}
则\(\AutoTuple{\vb\beta}{m}\)是一个正交向量组,
且\begin{equation*}
	\{\AutoTuple{\vb\alpha}{j}\}
	\cong
	\{\AutoTuple{\vb\beta}{j}\},
	\quad j=1,2,\dotsc,m.
\end{equation*}
再将之单位化(或规范化)得\begin{equation*}
	\vb\gamma_k = \frac1{\abs{\vb\beta_k}} \vb\beta_k,
	\quad k=1,2,\dotsc,m,
\end{equation*}
则\(\AutoTuple{\vb\gamma}{m}\)是一个规范正交组,且满足\begin{equation*}
	\{\AutoTuple{\vb\alpha}{j}\}
	\cong
	\{\AutoTuple{\vb\gamma}{j}\},
	\quad j=1,2,\dotsc,m.
\end{equation*}
\end{theorem}

\section{正规矩阵}
\begin{definition}
若矩阵\(\vb{A}\)满足\(\vb{A}\vb{A}^H = \vb{A}^H\vb{A}\),则称该矩阵为\DefineConcept{正规矩阵}.
\end{definition}

\begin{property}
实正交矩阵、实对称矩阵、实反对称矩阵都是实正规矩阵.
%TODO proof
\end{property}

\begin{property}
酉矩阵都是正规矩阵.
%TODO proof
\end{property}

\section{实对称矩阵的相似对角化}
由上节讨论我们知道,\(n\)阶矩阵分成可以相似对角化与不可以相似对角化两类.
实际上,矩阵的相似概念与数域有关,矩阵能否相似对角化也与数域有关.
因为一般矩阵的特征值是复数,即使\(\vb{A}\)的元素都是实数,也可能没有实数特征值.
例如,\(\vb{A} = \begin{bmatrix} 0 & -1 \\ 1 & 0 \end{bmatrix}\)是一个二阶实矩阵,
由于\begin{equation*}
	\abs{\lambda\vb{E}-\vb{A}}
	= \begin{vmatrix}
		\lambda & 1 \\
		-1 & \lambda
	\end{vmatrix}
	= (\lambda+\iu)(\lambda-\iu),
\end{equation*}
可以看出\(\vb{A}\)有两个特征值\(\pm\iu\),
其对应的特征向量\((\iu,1)^T\)和\((\iu,-1)^T\)是复向量,
因此\(\vb{A}\)在复数域上可以相似对角化,
但不存在可逆实阵\(\vb{P}\)使得\(\vb{P}^{-1}\vb{A}\vb{P}\)为对角阵.

\begin{theorem}\label{theorem:特征值与特征向量.实对称矩阵1}
%@see: 《线性代数》(张慎语、周厚隆) P109 定理6
实对称矩阵的特征值都是实数.
\begin{proof}
设\(\vb{A} \in M_n(\mathbb{R})\)满足\(\vb{A}^T=\vb{A}\).
显然\(\vb{A}\)的特征多项式\(\abs{\lambda\vb{E}-\vb{A}}\)在复数范围内有\(n\)个根.
假设\(\lambda_0\in\mathbb{C}\)是\(\vb{A}\)的任意一个特征值,
则存在\(n\)维复向量\(\vb{x}_0=(\AutoTuple{c}{n})^T \neq \vb0\),使得
\begin{gather}
	\vb{A}\vb{x}_0 = \lambda_0 \vb{x}_0, \tag1
\end{gather}
用\(\vb{x}_0\)的共轭转置向量\(\overline{\vb{x}_0}^T
=(\overline{c_1},\overline{c_2},\dotsc,\overline{c_n})\)
左乘(1)式两端,
得\begin{gather}
	\overline{\vb{x}_0}^T \vb{A} \vb{x}_0 = \lambda \overline{\vb{x}_0}^T \vb{x}_0, \tag2
\end{gather}
其中\(\vb{A}^T=\vb{A}=\overline{\vb{A}}\),
\(\overline{\vb{x}_0}^T \vb{x}_0
= \overline{c_1}c_1 + \overline{c_2}c_2 + \dotsb + \overline{c_n}{c_n} \in \mathbb{R}^+\).

又因为\(\overline{\vb{x}_0}^T \vb{A} \vb{x}_0 \in \mathbb{C}\)取转置不变,
且实对称矩阵满足\(\vb{A}^T = \vb{A} = \overline{\vb{A}}\),
所以\begin{equation*}
	\overline{\vb{x}_0}^T \vb{A} \vb{x}_0
	= (\overline{\vb{x}_0}^T \vb{A} \vb{x}_0)^T
	= \vb{x}_0^T \vb{A}^T \overline{\vb{x}_0}
	= \overline{\overline{\vb{x}_0}^T} \overline{\vb{A}} \overline{\vb{x}_0}
	= \overline{\overline{\vb{x}_0}^T \vb{A} \vb{x}_0},
\end{equation*}
说明\(\overline{\vb{x}_0}^T \vb{A} \vb{x}_0 \in \mathbb{R}\),进而有\(\lambda_0 \in \mathbb{R}\).
\end{proof}
\end{theorem}

\begin{theorem}\label{theorem:特征值与特征向量.实对称矩阵2}
%@see: 《线性代数》(张慎语、周厚隆) P109 定理7
实对称矩阵的不同特征值所对应的特征向量正交.
\begin{proof}
设\(\vb{A}\)是实对称矩阵,
\(\lambda_1\neq\lambda_2\)是\(\vb{A}\)的两个不同的特征值,
\(\vb{x}_1\neq\vb0\),
\(\vb{x}_2\neq\vb0\)分别是\(\vb{A}\)对应于\(\lambda_1\)、\(\lambda_2\)的特征向量,
则\(\vb{x}_1\)、\(\vb{x}_2\)都是实向量,
\begin{align*}
	\vb{A}\vb{x}_1 &= \lambda_1\vb{x}_1, \tag1 \\
	\vb{A}\vb{x}_2 &= \lambda_2\vb{x}_2. \tag2
\end{align*}
对(1)式左乘\(\vb{x}_2^T\),得\begin{gather}
	\vb{x}_2^T \vb{A} \vb{x}_1 = \lambda_1 \vb{x}_2^T \vb{x}_1, \tag3
\end{gather}
对(2)式左乘\(\vb{x}_1^T\),得\begin{gather}
	\vb{x}_1^T \vb{A} \vb{x}_2 = \lambda_2 \vb{x}_1^T \vb{x}_2, \tag4
\end{gather}
(3)式取转置,
得\(\lambda_1(\vb{x}_2^T \vb{x}_1)^T = (\vb{x}_2^T \vb{A} \vb{x}_1)^T\),
又由\(\vb{A}=\vb{A}^T\),
得\begin{gather}
	\lambda_1 \vb{x}_1^T \vb{x}_2 = \vb{x}_1^T \vb{A}^T \vb{x}_2 = \vb{x}_1^T \vb{A} \vb{x}_2, \tag5
\end{gather}
(5)式减(4)式,得
\begin{gather}
	(\lambda_2-\lambda_1)\vb{x}_1^T\vb{x}_2=0. \tag6
\end{gather}
因为\(\lambda_2 \neq \lambda_1\),
所以\(\vb{x}_1^T \vb{x}_2 = 0\),
即\((\vb{x}_1,\vb{x}_2) = 0\),
也就是\(\vb{x}_1\)与\(\vb{x}_2\)正交.
\end{proof}
\end{theorem}
\begin{remark}
一般地,对于\(n\)阶实对称矩阵\(\vb{A}\),
属于\(\vb{A}\)的同一特征值的一组线性无关的特征向量不一定相互正交,
可用施密特正交化方法将其正交化,得到\(\vb{A}\)的属于该特征值的正交特征向量组.
由以上定理,\(\vb{A}\)的几个属于不同特征值的正交特征向量组仍构成正交组.
特别地,\(\vb{A}\)有\(n\)个正交的特征向量,\(\vb{A}\)相似于对角形矩阵.
\end{remark}

\begin{example}
%@see: 《2022年全国硕士研究生入学统一考试(数学一)》一选择题/第5题/选项(D)
设矩阵\(\vb{A} \in M_n(K)\).
举例说明:“\(\vb{A}\)的属于不同特征值的特征向量相互正交”是
“\(\vb{A}\)可以相似对角化”的既不充分也不必要条件.
\begin{solution}
先证伪必要性.
取\(\vb\xi_1 = (1,0,0)^T,
\vb\xi_2 = (1,1,0)^T,
\vb\xi_3 = (1,0,1)^T\),
则\(\AutoTuple{\vb\xi}{3}\)是两两不正交的线性无关的向量组.
记矩阵\(\vb{P} = (\AutoTuple{\vb\xi}{3})\),
那么\(\vb{P}\)是可逆矩阵.
令\(\vb\Lambda = \diag(1,-1,0)\),
那么矩阵\(\vb{A} = \vb{P} \vb\Lambda \vb{P}^{-1}\)可以相似对角化,
但是\(\vb{A}\)的特征向量\(\AutoTuple{\vb\xi}{3}\)不相互正交.

再证伪充分性.
取\(\vb{A} = \begin{bmatrix}
	1 & 0 & 0 \\
	0 & 0 & 1 \\
	0 & 0 & 0
\end{bmatrix}\),
则\(\vb{A}\)有两个特征值\(1\)和\(0\ (\text{$2$重})\).
由于\(\Ker(0\vb{E}-\vb{A}) = \Ker\vb{A} = 3 - \rank\vb{A} = 1\),
所以由\cref{theorem:矩阵可以相似对角化的充分必要条件.定理3} 可知\(\vb{A}\)不可以相似对角化.
解\((0\vb{E}-\vb{A}) \vb{x} = \vb0\)得
\(\vb\xi_1 = (0,1,0)^T\)是\(\vb{A}\)的属于特征值\(0\)的一个特征向量.
解\((\vb{E}-\vb{A}) \vb{x} = \vb0\)得
\(\vb\xi_2 = (1,0,0)^T\)是\(\vb{A}\)的属于特征值\(1\)的一个特征向量.
显然\(\vb{A}\)的属于不同特征值的特征向量\(\vb\xi_1\)与\(\vb\xi_2\)是相互正交的.
\end{solution}
\end{example}

\begin{example}
%@see: 《1995年全国硕士研究生入学统一考试(数学一)》八解答题
设3阶实对称矩阵\(\vb{A}\)的特征值为
\(\lambda_1=-1,\lambda_2=\lambda_3=1\),
对应于\(\lambda_1\)的特征向量为\(\vb{\xi}_1=(0,1,1)^T\),
求\(\vb{A}\).
\begin{solution}
设属于\(\lambda=1\)的特征向量为\(\vb{\xi}=(x_1,x_2,x_3)^T\),
由于\hyperref[theorem:特征值与特征向量.实对称矩阵2]{实对称矩阵的不同特征值所对应的特征向量相互正交},
故\begin{equation*}
	\vb{\xi}^T \vb{\xi}_1 = x_2+x_3 = 0,
\end{equation*}
于是\(\vb{\xi}_2=(1,0,0)^T,
\vb{\xi}_3=(0,1,-1)^T\)
是属于\(\lambda=1\)的线性无关的特征向量,
也就是说\begin{equation*}
	\vb{A}(\vb{\xi}_1,\vb{\xi}_2,\vb{\xi}_3)
	=(-\vb{\xi}_1,\vb{\xi}_2,\vb{\xi}_3).
\end{equation*}
那么\begin{equation*}
	\vb{A}
	=(-\vb{\xi}_1,\vb{\xi}_2,\vb{\xi}_3)
	(\vb{\xi}_1,\vb{\xi}_2,\vb{\xi}_3)^{-1}
	=\begin{bmatrix}
		0 & 1 & 0 \\
		-1 & 0 & 1 \\
		-1 & 0 & -1
	\end{bmatrix}
	\begin{bmatrix}
		0 & \frac{1}{2} & \frac{1}{2} \\
		1 & 0 & 0 \\
		0 & \frac{1}{2} & -\frac{1}{2}
	\end{bmatrix}
	=\begin{bmatrix}
		1 & 0 & 0 \\
		0 & 0 & -1 \\
		0 & -1 & 0
	\end{bmatrix}.
\end{equation*}
\end{solution}
\end{example}

\begin{theorem}\label{theorem:特征值与特征向量.实对称矩阵3}
%@see: 《线性代数》(张慎语、周厚隆) P110 定理8
若\(\vb{A}\)为\(n\)阶实对称矩阵,则一定存在正交矩阵\(\vb{Q}\),使得\(\vb{\Lambda} = \vb{Q}^{-1}\vb{A}\vb{Q}\)为对角形矩阵.
\begin{proof}
用数学归纳法.
当\(n=1\)时,矩阵\(\vb{A}\)是一个实数\(a_{11}\),定理成立.
假设当\(n=k-1\)时定理成立,下面证明当\(n=k\)时定理也成立.

由\cref{theorem:特征值与特征向量.实对称矩阵1},
\(\vb{A}\)的特征值全为实数.
假设\(\lambda_1\)是\(\vb{A}\)的特征值,
并且相应地存在非零实向量\(\vb{x}_1=(\AutoTuple{c}{n})^T\),
使得\begin{equation*}
	\vb{A}\vb{x}_1=\lambda_1\vb{x}_1.
\end{equation*}
不妨设\(c_1\neq0\),则\(n\)元向量组\begin{equation*}
	\vb{x}_1,\vb{x}_2=(0,1,\dotsc,0)^T,\dotsc,\vb{x}_n=(0,0,\dotsc,1)^T
\end{equation*}线性无关.
对向量组\(X=\{\AutoTuple{\vb{x}}{n}\}\)
用施密特正交规范化方法可得规范正交组\(Y=\{\AutoTuple{\vb{y}}{n}\}\),
则\(\vb{P}=(\AutoTuple{\vb{y}}{n})\)是正交矩阵,
其中\(\vb{y}_1=\abs{\vb{x}_1}^{-1}\vb{x}_1\)是\(\vb{A}\)的特征向量.

因为\(\mathbb{R}^n\)中任意\(n+1\)个向量线性相关,
故任意向量都可由\(Y\)线性表出,
即\begin{align*}
	\vb{A}\vb{y}_1 &= \lambda_1\vb{y}_1 = \lambda_1\vb{y}_1 + 0\vb{y}_2 + \dotsb + 0\vb{y}_n, \\
	\vb{A}\vb{y}_k &= b_{1k}\vb{y}_1 + b_{2k}\vb{y}_2 + \dotsb + b_{nk}\vb{y}_n \quad(k=2,3,\dotsc,n),
\end{align*}
由分块矩阵乘法,\begin{equation*}
	\vb{A}(\AutoTuple{\vb{y}}{n})
	= (\AutoTuple{\vb{y}}{n})
	\begin{bmatrix}
		\lambda_1 & b_{11} & \dots & b_{1n} \\
		0 & b_{22} & \dots & b_{2n} \\
		\vdots & \vdots & & \vdots \\
		0 & b_{n2} & \dots & b_{nn}
	\end{bmatrix},
\end{equation*}
令\(\vb{P}=(\AutoTuple{\vb{y}}{n})\),
显然\(\vb{P}\)是正交阵,
而\(\vb{P}^{-1}\vb{A}\vb{P}=\vb{P}^T\vb{A}\vb{P}=\begin{bmatrix}
	\lambda_1 & \vb\alpha \\
	\vb0 & \vb{B}
\end{bmatrix}\).

由\((\vb{P}^T\vb{A}\vb{P})^T=\vb{P}^T\vb{A}^T\vb{P}=\vb{P}^T\vb{A}\vb{P}\)可知\(\begin{bmatrix}
	\lambda_1 & \vb\alpha \\
	\vb0 & \vb{B}
\end{bmatrix}
= \begin{bmatrix}
	\lambda_1 & \vb0 \\
	\vb\alpha^T & \vb{B}^T
\end{bmatrix}\),
于是\(\vb\alpha=\vb0\),\(\vb{B}=\vb{B}^T\),也就是说\(\vb{B}\)是\(n-1\)阶实对称矩阵.
又由归纳假设,存在\(n-1\)阶正交阵\(\vb{M}\),使得\(\vb{M}^{-1}\vb{B}\vb{M}\)成为对角阵.

令\(\vb{Q}=\vb{P}\begin{bmatrix} 1 & \vb0 \\ \vb0 & \vb{M} \end{bmatrix}\),
因为\(\vb{Q}\vb{Q}^T=\vb{Q}^T\vb{Q}=\vb{E}\),所以\(\vb{Q}\)是正交矩阵,
而\begin{align*}
	\vb{Q}^{-1}\vb{A}\vb{Q}
	&=\begin{bmatrix}
		1 & \vb0 \\
		\vb0 & \vb{M}^{-1}
	\end{bmatrix}\vb{P}^{-1}\vb{A}\vb{P}\begin{bmatrix}
		1 & \vb0 \\
		\vb0 & \vb{M}
	\end{bmatrix}=\begin{bmatrix}
		1 & \vb0 \\
		\vb0 & \vb{M}^{-1}
	\end{bmatrix}\begin{bmatrix}
		\lambda_1 & \vb0 \\
		\vb0 & \vb{B}
	\end{bmatrix}\begin{bmatrix}
		1 & \vb0 \\
		\vb0 & \vb{M}
	\end{bmatrix} \\
	&=\begin{bmatrix}
		\lambda_1 & \vb0 \\
		\vb0 & \vb{M}^{-1}\vb{B}\vb{M}
	\end{bmatrix}
	=\diag(\AutoTuple{\lambda}{n}).
\end{align*}
由上可知当\(n=k\)时定理也成立.
\end{proof}
\end{theorem}

\cref{theorem:特征值与特征向量.实对称矩阵3} 表明,
实对称矩阵总可以相似对角化,
或者说,实对称矩阵总是\DefineConcept{正交相似}({orthogonally similar})于某个对角形矩阵.

从\cref{theorem:特征值与特征向量.矩阵相似的必要条件3} 我们已经知道
\(\vb{A}\sim\vb{B} \implies \abs{\lambda\vb{E}-\vb{A}}=\abs{\lambda\vb{E}-\vb{B}}\).
但对于一般的同阶矩阵\(\vb{A},\vb{B}\),
我们不能肯定\cref{theorem:特征值与特征向量.矩阵相似的必要条件3} 的逆命题一定成立.
但是对于实对称矩阵来说,
只要加上一个额外条件“\(\vb{A}\)与\(\vb{B}\)的特征值相同”,
就可以依靠\cref{theorem:特征值与特征向量.实对称矩阵3} 证明\(\vb{A}\)与\(\vb{B}\)相似.
\begin{corollary}
%@see: 《线性代数》(张慎语、周厚隆) P113 习题5.3 1(2)
%@see: 《线性代数》(张慎语、周厚隆) P113 习题5.3 9(1)
设\(\vb{A},\vb{B}\)是同阶实对称矩阵,
则\begin{equation*}
	\vb{A}\sim\vb{B}
	\iff
	\abs{\lambda\vb{E}-\vb{A}}=\abs{\lambda\vb{E}-\vb{B}}.
\end{equation*}
\end{corollary}

\begin{corollary}
%@see: 《线性代数》(张慎语、周厚隆) P110 推论
\(n\)阶实对称矩阵\(\vb{A}\)存在\(n\)个正交的单位特征向量.
\end{corollary}

\begin{remark}
\color{red}
对于实对称矩阵\(\vb{A}\),求正交矩阵\(\vb{Q}\),使得\(\vb{Q}^{-1}\vb{A}\vb{Q}\)为对角形矩阵的方法:
\begin{enumerate}
	\item 求出\(\vb{A}\)的全部不同的特征值\(\AutoTuple{\lambda}{m}\);
	\item 求出\((\lambda_i\vb{E}-\vb{A})\vb{x}=\vb0\)的基础解系,将其正交化,
	得到\(\vb{A}\)属于\(\lambda_i\ (i=1,2,\dotsc,m)\)的正交特征向量,
	共求出\(\vb{A}\)的\(n\)个正交特征向量;
	\item 将以上\(n\)个正交特征向量单位化,由所得向量作为列构成正交矩阵\(\vb{Q}\),则\begin{equation*}
		\vb{Q}^{-1}\vb{A}\vb{Q} = \vb{Q}^T \vb{A} \vb{Q} = \diag(\AutoTuple{\lambda}{n}).
	\end{equation*}
\end{enumerate}
\end{remark}

\begin{example}
%@see: 《线性代数》(张慎语、周厚隆) P112 例5
设\(\vb{A}\)为\(n\)阶实对称矩阵,\(\vb{A}\)是对合矩阵,证明:存在正交矩阵\(\vb{Q}\),使得\begin{equation*}
	\vb{Q}^{-1}\vb{A}\vb{Q}=\begin{bmatrix} \vb{E}_r \\ & -\vb{E}_{n-r} \end{bmatrix}.
\end{equation*}
\begin{proof}
因为\(\vb{A}\)为\(n\)阶实对称矩阵,
则\(\vb{A}\)有\(n\)个实特征值,
\(\vb{A}\)有\(n\)个正交的单位特征向量,
适当调整它们的顺序,可以构成正交矩阵\(\vb{Q}\),
满足\begin{gather}
	\vb{Q}^{-1}\vb{A}\vb{Q}=\diag(\AutoTuple{\lambda}{n}), \tag1
\end{gather}
其中,\(\lambda_i>0\ (i=1,2,\dotsc,r),
\lambda_i\leq0\ (i=r+1,r+2,\dotsc,n)\).
对(1)式两端分别平方,
又由\(\vb{A}\)是对合矩阵,满足\(\vb{A}^2=\vb{E}\),
得\begin{equation*}
	\vb{Q}^{-1}\vb{A}^2\vb{Q}
	= \vb{Q}^{-1}\vb{E}\vb{Q}
	= \vb{E}
	= \diag(\lambda_1^2,\lambda_2^2,\dotsc,\lambda_n^2),
\end{equation*}
于是\(\lambda_i^2=1\ (i=1,2,\dotsc,n)\),
进而有\begin{equation*}
	\lambda_i= \begin{cases}
		1, & i=1,2,\dotsc,r, \\
		-1, & i=r+1,r+2,\dotsc,n.
	\end{cases}
	\qedhere
\end{equation*}
\end{proof}
\end{example}

\begin{example}
设\(\vb{A}\)是4阶实对称矩阵,且\(\vb{A}^2+\vb{A}=\vb0\).
若\(\rank\vb{A}=3\),求\(\vb{A}\)的特征值以及与\(\vb{A}\)相似的对角阵.
\begin{solution}
\(\vb{A}\)是实对称矩阵,根据\cref{theorem:特征值与特征向量.实对称矩阵3},
\(\vb{A}\)一定可以相似对角化,不妨设\(\vb{A}\vb{x}=\lambda\vb{x}\ (\vb{x}\neq0)\),
那么\(\vb{A}^2\vb{x}=\lambda^2\vb{x}\),\((\vb{A}^2+\vb{A})\vb{x}=(\lambda^2+\lambda)\vb{x}\).
因为\(\vb{A}^2+\vb{A}=\vb0\),所以\((\lambda^2+\lambda)\vb{x}=\vb0\),\(\lambda^2+\lambda=0\),
解得\(\vb{A}\)的特征值为\(\lambda=0,-1\).
又因为\(\rank\vb{A}=3\),所以\(\vb{A}\)具有3个非零特征值,
因此与\(\vb{A}\)相似的对角阵为\(\diag(-1,-1,-1,0)\).
\end{solution}
\end{example}

\begin{example}
设\(\vb{A}\)是特征值仅为1与0的\(n\)阶实对称矩阵,证明:\(\vb{A}\)是幂等矩阵.
\begin{proof}
\def\M{\begin{bmatrix} \vb{E}_r \\ & \vb0_{n-r} \end{bmatrix}}%
因为\(\vb{A}\)是实对称矩阵,所以存在正交矩阵\(\vb{Q}\)使得\begin{equation*}
	\vb{Q}^{-1}\vb{A}\vb{Q} = \M,
\end{equation*}
从而有\begin{equation*}
	\vb{A} = \vb{Q}\M\vb{Q}^{-1},
\end{equation*}
进而有\begin{equation*}
	\vb{A}^2 = \vb{Q}\M\vb{Q}^{-1}\vb{Q}\M\vb{Q}^{-1} = \vb{Q}\M\vb{Q}^{-1} = \vb{A}.
	\qedhere
\end{equation*}
\end{proof}
\end{example}

\begin{example}
%@see: 《线性代数》(张慎语、周厚隆) P113 习题5.3 7.
设\(\vb{A}\)为\(n\)阶实对称矩阵,满足\(\vb{A}^2=\vb0\),证明:\(\vb{A}=\vb0\).
\begin{proof}
因为\(\vb{A}\)是实对称矩阵,所以存在正交矩阵\(\vb{Q}\)使得\begin{equation*}
	\vb{Q}^{-1}\vb{A}\vb{Q} = \diag(\AutoTuple{\lambda}{n}) = \vb{\Lambda},
\end{equation*}
从而有\(\vb{A} = \vb{Q}\vb{\Lambda}\vb{Q}^{-1}\),\(\vb{A}^2 = (\vb{Q}\vb{\Lambda}\vb{Q}^{-1})^2 = \vb{Q}\vb{\Lambda}^2\vb{Q}^{-1} = \vb0\),那么\begin{equation*}
	\vb{\Lambda}^2 = \diag(\lambda_1^2,\lambda_2^2,\dotsc,\lambda_n^2) = \vb{Q}^{-1}\vb0\vb{Q} = \vb0,
\end{equation*}\begin{equation*}
	\lambda_1=\lambda_2=\dotsb=\lambda_n = 0,
\end{equation*}
所以\(\vb{A}=\vb0\).
\end{proof}
\end{example}

\begin{example}
%@see: 《2017年全国硕士研究生入学统一考试(数学一)》一选择题/第5题
设\(\vb\alpha\)是实数域上的\(n\)维单位列向量,
\(\vb{E}\)是实数域上的\(n\)阶单位矩阵.
判断矩阵\begin{equation*}
	\vb{E} - k \vb\alpha \vb\alpha^T
	\quad(k\in\mathbb{R})
\end{equation*}是否可逆.
\begin{solution}
由\cref{theorem:矩阵的迹.矩阵乘积交换次序不变迹} 可知
\(\tr(\vb\alpha \vb\alpha^T)
= \tr(\vb\alpha^T \vb\alpha)\).
因为\(\vb\alpha\)是单位向量,
所以\begin{equation*}
	\tr(\vb\alpha \vb\alpha^T)
	= \vb\alpha^T \vb\alpha
	= 1.
\end{equation*}
由\cref{example:矩阵乘积的秩.两个向量的乘积的秩} 可知\begin{equation*}
	\rank(\vb\alpha \vb\alpha^T) = 1.
\end{equation*}
由于\((\vb\alpha \vb\alpha^T)^T = \vb\alpha \vb\alpha^T\),
\(\vb\alpha \vb\alpha^T\)是实对称矩阵,
所以由\cref{theorem:特征值与特征向量.实对称矩阵3} 可知
\(\vb\alpha \vb\alpha^T\)可以相似对角化.
不妨设\begin{equation*}
	\vb\alpha \vb\alpha^T
	\sim
	\diag(\AutoTuple{\lambda}{n}).
\end{equation*}
由于\(\rank(\vb\alpha \vb\alpha^T) = 1\),
所以\(\AutoTuple{\lambda}{n}\)中必有\(n-1\)个是零,
因此不妨设\(\lambda_1 \neq 0\)
而\(\lambda_2 = \dotsb = \lambda_n = 0\).
又因为\begin{equation*}
	\tr(\vb\alpha \vb\alpha^T)
	= \lambda_1 + \lambda_2 + \dotsb + \lambda_n
	= \lambda_1
	= 1,
\end{equation*}
所以存在正交矩阵\(\vb{P}\)使得\begin{equation*}
	\diag(1,0,\dotsc,0)
	= \vb{P}^{-1} (\vb\alpha \vb\alpha^T) \vb{P}.
\end{equation*}
因为\begin{equation*}
	\vb{P}^{-1} (\vb{E} - k \vb\alpha \vb\alpha^T) \vb{P}
	= \vb{E} - k \diag(1,0,\dotsc,0)
	= \diag(1-k,1,\dotsc,1),
\end{equation*}
所以\begin{equation*}
	\abs{\vb{E} - k \vb\alpha \vb\alpha^T}
	= \abs{\vb{P}^{-1} (\vb{E} - k \vb\alpha \vb\alpha^T) \vb{P}}
	= \abs{\diag(1-k,1,\dotsc,1)}
	= 1-k,
\end{equation*}
因此\begin{equation*}
	\text{矩阵$\vb{E} - k \vb\alpha \vb\alpha^T$可逆}
	\iff
	k\neq1.
\end{equation*}
\end{solution}
\end{example}

\section{实反对称矩阵的相似对角化}
\begin{theorem}
%@see: 《线性代数》(张慎语、周厚隆) P113 习题5.3 6.
实反对称矩阵的特征值为零或纯虚数.
\begin{proof}
设\(\vb{A} \in M_n(\mathbb{R})\)满足\(\vb{A}^T=-\vb{A}\).
又设\(\mathbb{C} \ni \lambda_0 = a_0 + \iu b_0\ (a_0,b_0 \in \mathbb{R})\)是\(\vb{A}\)的任意一个特征值,
\(\mathbb{C}^{n \times 1} \ni \vb{x}_0=(\AutoTuple{c}{n})^T \neq \vb0\)是
\(\vb{A}\)属于特征值\(\lambda_0\)的特征向量,
即\begin{gather}
\vb{A}\vb{x}_0 = \lambda_0\vb{x}_0, \tag1
\end{gather}
在(1)式两端左乘\(\ComplexConjugate{\vb{x}_0}^T\),
得\begin{gather}
	\ComplexConjugate{\vb{x}_0}^T \vb{A} \vb{x}_0
	= \lambda_0\ \ComplexConjugate{\vb{x}_0}^T \vb{x}_0, \tag2
\end{gather}
取共轭转置,得\begin{gather}
\ComplexConjugate{\vb{x}_0}^T \vb{A}^T \vb{x}_0
= \ComplexConjugate{\lambda_0}\ \ComplexConjugate{\vb{x}_0}^T \vb{x}_0, \tag3
\end{gather}
由于\(\vb{A}\)是实反对称矩阵,即\(\vb{A}^T = -\vb{A}\),所以\begin{gather}
	\ComplexConjugate{\vb{x}_0}^T \vb{A} \vb{x}_0
	= -\ComplexConjugate{\lambda_0}\ \ComplexConjugate{\vb{x}_0}^T \vb{x}_0, \tag4
\end{gather}
其中,\(\ComplexConjugate{\vb{x}_0}^T \vb{x}_0
= \ComplexConjugate{c_1}c_1 + \ComplexConjugate{c_2}c_2 + \dotsb + \ComplexConjugate{c_n}{c_n} > 0\).
由(2)式与(4)式,得\begin{equation*}
	\lambda_0\ \ComplexConjugate{\vb{x}_0}^T \vb{x}_0
	= -\ComplexConjugate{\lambda_0}\ \ComplexConjugate{\vb{x}_0}^T \vb{x}_0,
\end{equation*}\begin{equation*}
	\lambda_0 = -\ComplexConjugate{\lambda_0},
\end{equation*}\begin{equation*}
	a_0 + \iu b_0 = -(a_0 - \iu b_0) = -a_0 + \iu b_0,
\end{equation*}\begin{equation*}
	\Re \lambda_0 = a_0 = 0.
\end{equation*}
也就是说\(\lambda_0\)要么为零要么为纯虚数.
\end{proof}
\end{theorem}


\chapter{二次型}
为了研究几何问题(特别是平面二次曲线、空间二次曲面的方程的化简)和物理问题,
我们抽象出“二次型”的概念,
利用代数方法对其进行研究.
\section{二次型的基本概念}
我们首先研究平面解析几何中以坐标原点为中心的二次曲线的方程:
\begin{center}
	\def\arraystretch{1.5}
	\begin{tblr}{cl}
		圆 & \(x^2+y^2=r^2\) \\
		椭圆 & \(\frac{x^2}{a^2}+\frac{y^2}{b^2}=1\) \\
		双曲线 & \(\frac{x^2}{a^2}-\frac{y^2}{b^2}=1\) \\
	\end{tblr}
\end{center}

可以看出,它们都具有\[
	a x^2 + 2b xy + c y^2 = d
\]的形式.
在研究二次曲线时,如果得到的方程不是标准方程,
我们通常希望通过旋转、平移等几何变换将其化为标准方程,
进而判别曲线的形状和几何性质.

\subsection{二次型的基本概念}
\begin{definition}
%@see: 《高等代数(第三版 上册)》(丘维声) P192 定义1
系数在数域\(K\)中的\(n\)个变量的
二次齐次多项式\begin{equation}\label{equation:二次型.二次型}
	f(\AutoTuple{x}{n})
	= \sum_{i=1}^n \sum_{j=1}^n a_{ij} x_i x_j
	\quad(a_{ji}=a_{ij},i,j=1,2,\dotsc,n),
\end{equation}
称为“数域\(K\)上的一个\(n\)元\DefineConcept{二次型}(quadratic form)”.
%@see: https://mathworld.wolfram.com/QuadraticForm.html
%@see: https://mathworld.wolfram.com/DiagonalQuadraticForm.html
%@see: https://mathworld.wolfram.com/SymmetricBilinearForm.html
\end{definition}

数域对于一个二次齐次多项式是否成为二次型是决定性的.
多项式\[
	f(x_1,x_2,x_3) = x_1^2 + 4 x_1 x_2 + 3 x_2^2 + 5 x_2 x_3 - x_3^2
\]和\[
	g(x_1,x_2,x_3) = x_1^2 + 2\sqrt{2} x_1 x_2 + 2 x_1 x_3 + 2 x_2^2 + 4\sqrt{3} x_2 x_3
\]都是实数域上的二次型;
但在有理数域上,只有\(f\)是二次型,\(g\)不是二次型.

本章不作特别声明时,所称“二次型”均指实二次型.

前面提到我们希望将一般方程化为标准方程,现在我们就要定义何种形式的方程应该被称为标准方程.
再次观察平面二次曲线的标准方程可以发现,标准方程的等号左边应该是二次齐次多项式(即若干个变量的平方和),等号右边则应该是任意(非零)常数.

\begin{definition}
若数域\(K\)上的\(n\)阶对称矩阵\(\vb{A}\)
满足\begin{equation}\label{equation:二次型.二次型的矩阵表示}
	f(\AutoTuple{x}{n}) = \vb{x}^T\vb{A}\vb{x},
\end{equation}
其中\(\vb{x} = (\AutoTuple{x}{n})^T\),
则把\cref{equation:二次型.二次型的矩阵表示}
称为“二次型\(f(\AutoTuple{x}{n})\)的\DefineConcept{矩阵表示}”,
把\(\vb{A}\)称为\(f\)的\DefineConcept{矩阵},
把\(\vb{A}\)的秩\(\rank\vb{A}\)称为\(f\)的\DefineConcept{秩}.
\end{definition}

显然,对于任一\(n\)阶矩阵\(\vb{B}\),\(\vb{x}^T\vb{B}\vb{x}\)必定是一个二次型.
需要注意的是,矩阵\(\vb{B}\)不必是对称矩阵,但“二次型\(\vb{x}^T\vb{B}\vb{x}\)的矩阵”必定是一个对称矩阵.

\begin{property}
二次型和它的矩阵是相互唯一确定的.
\begin{proof}
对于二次型\(f(\AutoTuple{x}{n})\),
设非零\(n\)阶对称矩阵\(\vb{A}\)和\(\vb{B}\)都是\(f\)的矩阵,
即\[
	\vb{x}^T\vb{A}\vb{x}
	=\vb{x}^T\vb{B}\vb{x}
	=f(\AutoTuple{x}{n}),
\]
则二次型\(\vb{x}^T\vb{A}\vb{x}\)与\(\vb{x}^T\vb{B}\vb{x}\)中\(x_i x_j\)的
系数\(2 a_{ij}\)与\(2 b_{ij}\ (1 \leq i < j \leq n)\)必相等,
\(x_i^2\)的系数\(a_{ii}\)与\(b_{ii}\ (i=1,2,\dotsc,n)\)必相等,
故\(\vb{A}=\vb{B}\).
\end{proof}
\end{property}

\begin{example}
将\(f(x_1,x_2,x_3) = x_1^2 + 4 x_1 x_2 + 3 x_2^2 + 5 x_2 x_3 - x_3^2\)写成矩阵形式.
\begin{solution}
\(f(x_1,x_2,x_3)
= \begin{bmatrix}
	x_1 & x_2 & x_3
\end{bmatrix}
\begin{bmatrix}
	1 & 2 & 0 \\
	2 & 3 & \frac{5}{2} \\
	0 & \frac{5}{2} & -1
\end{bmatrix}
\begin{bmatrix}
	x_1 \\ x_2 \\ x_3
\end{bmatrix}\).
\end{solution}
\end{example}

\begin{example}
写出二次型\(\begin{bmatrix}
	x_1 & x_2 & x_3
\end{bmatrix}
\begin{bmatrix}
	2 & -3 & 1 \\
	1 & 0 & 1 \\
	2 & 11 & 3
\end{bmatrix}
\begin{bmatrix}
	x_1 \\ x_2 \\ x_3
\end{bmatrix}\)的矩阵.
\begin{solution}
注意到矩阵\(\vb{A} = \begin{bmatrix}
	2 & -3 & 1 \\
	1 & 0 & 1 \\
	2 & 11 & 3
\end{bmatrix}\)不是对称矩阵,
二次型\(\vb{x}^T\vb{A}\vb{x}\)的矩阵应为\[
	\vb{B}
	= \frac{\vb{A}+\vb{A}^T}{2}
	= \begin{bmatrix}
		2 & -1 & \frac{3}{2} \\
		-1 & 0 & 6 \\
		\frac{3}{2} & 6 & 3
	\end{bmatrix}.
\]
\end{solution}
\end{example}

\begin{example}\label{example:二次型.反对称矩阵对应的二次型恒为零}
%@see: 《高等代数(第三版 上册)》(丘维声) P202 习题6.1 5.
设\(\vb{A}\)是数域\(K\)上的\(n\)阶矩阵.
证明:\(\vb{A}\)是反对称矩阵的充分必要条件是
“对于\(K^n\)中任一列向量\(\vb\alpha\),
有\(\vb\alpha^T \vb{A} \vb\alpha = 0\)”.
\begin{proof}
假设\(\vb{A}\)是反对称矩阵,
即\(\vb{A}^T = -\vb{A}\),
那么对于\(K^n\)中任一列向量\(\vb\alpha\),
有\[
	\vb\alpha^T \vb{A} \vb\alpha
	= - \vb\alpha^T \vb{A}^T \vb\alpha
	= - (\vb\alpha^T \vb{A} \vb\alpha)^T
	= - \vb\alpha^T \vb{A} \vb\alpha, % \(\vb\alpha^T \vb{A} \vb\alpha\)是一个数,数的转置还是这个数
\]
于是\(\vb\alpha^T \vb{A} \vb\alpha = 0\).

假设对于\(K^n\)中任一列向量\(\vb\alpha\),
有\(\vb\alpha^T \vb{A} \vb\alpha = 0\),
那么\[
	0 = \vb\alpha^T \vb{A} \vb\alpha
	= (\vb\alpha^T \vb{A} \vb\alpha)^T
	= \vb\alpha^T \vb{A}^T \vb\alpha
	= - (\vb\alpha^T \vb{A} \vb\alpha),
\]
从而有\[
	\vb\alpha^T \vb{A}^T \vb\alpha
	+ \vb\alpha^T \vb{A} \vb\alpha
	= \vb\alpha^T (\vb{A}^T + \vb{A}) \vb\alpha
	= 0,
\]
于是\(\vb{A} + \vb{A}^T = 0\),
\(\vb{A}\)是反对称矩阵.
\end{proof}
\end{example}
\begin{example}
%@see: 《高等代数(第三版 上册)》(丘维声) P202 习题6.1 6.
设\(\vb{A}\)是数域\(K\)上的\(n\)阶对称矩阵.
证明:如果对于\(K^n\)中任一列向量\(\vb\alpha\),
都有\(\vb\alpha^T \vb{A} \vb\alpha = 0\),
则\(\vb{A} = \vb0\).
\begin{proof}
由于对于\(K^n\)中任一列向量\(\vb\alpha\),
都有\[
	\vb\alpha^T \vb{A} \vb\alpha = \vb\alpha^T (0\vb{E}) \vb\alpha,
\]
其中\(\vb{E}\)是数域\(K\)上的\(n\)阶单位矩阵,
所以\[
	\vb\alpha^T \vb{A} \vb\alpha - \vb\alpha^T (0\vb{E}) \vb\alpha
	= \vb\alpha^T (\vb{A} - \vb0) \vb\alpha
	= 0,
\]
于是\(\vb{A} = \vb0\).
\end{proof}
\end{example}
\begin{example}
%@see: 《高等代数(第三版 上册)》(丘维声) P202 习题6.1 7.
证明:秩为\(r\)的对称矩阵可以表示成\(r\)个秩为\(1\)的对称矩阵之和.
%TODO proof
\end{example}
\begin{example}
%@see: 《高等代数(第三版 上册)》(丘维声) P202 习题6.1 9.
证明:数域\(K\)上的反对称矩阵一定合同于下述形式的分块对角矩阵:\[
	\diag\left(
		\begin{bmatrix}
			0 & 1 \\
			-1 & 0
		\end{bmatrix},
		\dotsc,
		\begin{bmatrix}
			0 & 1 \\
			-1 & 0
		\end{bmatrix},
		0,\dotsc,0
	\right).
\]
%TODO proof
\end{example}
\begin{example}
%@see: 《高等代数(第三版 上册)》(丘维声) P202 习题6.1 10.
证明:反对称矩阵的秩一定是偶数.
%TODO proof
\end{example}
\begin{example}\label{example:二次型.瑞利商的取值范围}
%@see: 《高等代数(第三版 上册)》(丘维声) P202 习题6.1 11.
设\(n\)阶实对称矩阵\(\vb{A}\)的全部特征值按大小顺序排成
\(\lambda_1 \geq \lambda_2 \geq \dotsb \geq \lambda_n\).
证明:对于\(\mathbb{R}^n\)中任一非零列向量\(\vb\alpha\),
都有\begin{equation*}
	\lambda_n \leq \frac{\vb\alpha^T \vb{A} \vb\alpha}{\vb\alpha^T \vb\alpha} \leq \lambda_1.
\end{equation*}
\begin{proof}
假设正交矩阵\(\vb{Q}\)满足\begin{equation*}
	\vb{Q}^T \vb{A} \vb{Q} = \vb\Lambda,
\end{equation*}
其中\(\vb\Lambda = \diag(\AutoTuple{\lambda}{n})\),
那么\begin{equation*}
	\vb\alpha^T \vb{A} \vb\alpha
	= \vb\alpha^T (\vb{Q} \vb\Lambda \vb{Q}^T) \vb\alpha
	= (\vb{Q}^T \vb\alpha)^T \vb\Lambda (\vb{Q}^T \vb\alpha)
	= \lambda_1 b_1^2 + \dotsb + \lambda_n b_n^2,
\end{equation*}
其中\(\vb{Q}^T \vb\alpha = (\AutoTuple{b}{n})\).
因为\(\lambda_1 \geq \lambda_2 \geq \dotsb \geq \lambda_n\),
所以\begin{gather*}
	\lambda_1 b_1^2 + \dotsb + \lambda_n b_n^2
	\geq \lambda_n b_1^2 + \dotsb + \lambda_n b_n^2
	= \lambda_n (b_1^2 + \dotsb + b_n^2), \\
	\lambda_1 b_1^2 + \dotsb + \lambda_n b_n^2
	\leq \lambda_1 b_1^2 + \dotsb + \lambda_1 b_n^2
	= \lambda_1 (b_1^2 + \dotsb + b_n^2),
\end{gather*}
又由于\begin{equation*}
	b_1^2 + \dotsb + b_n^2
	= (\vb{Q}^T \vb\alpha)^T (\vb{Q}^T \vb\alpha)
	= \vb\alpha^T (\vb{Q} \vb{Q}^T) \vb\alpha
	= \vb\alpha^T \vb\alpha
	\neq 0,
\end{equation*}
所以\(\lambda_n \vb\alpha^T \vb\alpha
\leq \vb\alpha^T \vb{A} \vb\alpha
\leq \lambda_1 \vb\alpha^T \vb\alpha\),
即\(\lambda_n \leq \frac{\vb\alpha^T \vb{A} \vb\alpha}{\vb\alpha^T \vb\alpha} \leq \lambda_1\).
\end{proof}
\end{example}
\begin{example}
%@see: 《高等代数(第三版 上册)》(丘维声) P202 习题6.1 12.
设\(\vb{A}\)是一个\(n\)阶实对称矩阵.
证明:存在一个正实数\(c\),使得对于\(\mathbb{R}^n\)中任一列向量\(\vb\alpha\),
都有\(\abs{\vb\alpha^T \vb{A} \vb\alpha} \leq c \vb\alpha^T \vb\alpha\).
\begin{proof}
由\cref{example:二次型.瑞利商的取值范围} 可知,
如果\(\vb{A}\)的最大特征值是\(\lambda_1\),最小特征值是\(\lambda_n\),
那么\begin{equation*}
	\lambda_n \vb\alpha^T \vb\alpha
	\leq \vb\alpha^T \vb{A} \vb\alpha
	\leq \lambda_1 \vb\alpha^T \vb\alpha,
\end{equation*}
从而由\cref{example:不等式.数的上下界} 可知,
成立\begin{equation*}
	\abs{\vb\alpha^T \vb{A} \vb\alpha}
	\leq \abs{\lambda_1 \vb\alpha^T \vb\alpha}
	= \abs{\lambda_1} \vb\alpha^T \vb\alpha
	\quad\text{或}\quad
	\abs{\vb\alpha^T \vb{A} \vb\alpha}
	\leq \abs{\lambda_n \vb\alpha^T \vb\alpha}
	= \abs{\lambda_n} \vb\alpha^T \vb\alpha,
\end{equation*}
因此只要取\(c = \max\{\abs{\lambda_1},\abs{\lambda_n}\}\),
那么对于\(\mathbb{R}^n\)中任一列向量\(\vb\alpha\),
便都有\(\abs{\vb\alpha^T \vb{A} \vb\alpha} \leq c \vb\alpha^T \vb\alpha\).
\end{proof}
\end{example}
\begin{example}
%@see: 《高等代数(第三版 上册)》(丘维声) P202 习题6.1 13.
设\(\vb{A},\vb{B}\)都是\(n\)阶实对称矩阵,并且\(\vb{A}\vb{B}=\vb{B}\vb{A}\).
证明:存在一个\(n\)阶正交矩阵\(\vb{Q}\),
使得\(\vb{Q}^T \vb{A} \vb{Q}\)与\(\vb{Q}^T \vb{B} \vb{Q}\)都是对角矩阵.
%TODO proof
\end{example}
\begin{example}
%@see: 《高等代数(第三版 上册)》(丘维声) P202 习题6.1 14.
设\(n\)元实二次型\(\vb{x}^T \vb{A} \vb{x}\)的矩阵\(\vb{A}\)的一个特征值是\(\lambda_i\).
证明:存在\(\mathbb{R}^n\)中非零列向量\(\vb\alpha\)使得\[
	\vb\alpha^T \vb{A} \vb\alpha = \lambda_i \vb\alpha^T \vb\alpha.
\]
%TODO proof
\end{example}

\subsection{线性替换}
\begin{definition}
因为平面二次曲线方程通过旋转变换化为标准方程,
实际上是用新变量的一次式代替原来的变量.
同样地,使用这种基本的方法来化简一般的\(n\)元二次型,
作如下的变量替换:\[
	\left\{ \begin{array}{l}
		x_1 = c_{11}y_1 + c_{12}y_2 + \dotsb + c_{1n}y_n \\
		x_2 = c_{21}y_1 + c_{22}y_2 + \dotsb + c_{2n}y_n \\
		\hdotsfor{1} \\
		x_n = c_{n1}y_1 + c_{n2}y_2 + \dotsb + c_{nn}y_n
	\end{array} \right.
\]
写成矩阵形式\[
	\begin{bmatrix}
		x_1 \\ x_2 \\ \vdots \\ x_n
	\end{bmatrix}
	= \begin{bmatrix}
		c_{11} & c_{12} & \dots & c_{1n} \\
		c_{21} & c_{22} & \dots & c_{2n} \\
		\vdots & \vdots & & \vdots \\
		c_{n1} & c_{n2} & \dots & c_{nn}
	\end{bmatrix}
	\begin{bmatrix}
		y_1 \\ y_2 \\ \vdots \\ y_n
	\end{bmatrix}
	\quad\text{或}\quad
	\vb{x}=\vb{C}\vb{y}.
\]

上述变量之间的替换称为\DefineConcept{线性替换}.

当矩阵\(\vb{C}\)可逆时,
称之为\DefineConcept{可逆线性替换}、\DefineConcept{满秩线性替换}或\DefineConcept{非退化线性替换}.

当矩阵\(\vb{C}\)是正交矩阵时,
称之为\DefineConcept{正交线性替换}.
\end{definition}

\begin{theorem}
对二次型\(f(\AutoTuple{x}{n})=\vb{x}^T\vb{A}\vb{x}\ (\vb{A}=\vb{A}^T)\)作可逆线性替换\(\vb{x}=\vb{C}\vb{y}\),则\(f\)化为新变量的二次型\(g(\AutoTuple{y}{n})=\vb{y}^T\vb{B}\vb{y}\),其中\(\vb{B}=\vb{C}^T\vb{A}\vb{C}\)为\(g\)的矩阵.
\begin{proof}
\(f(\AutoTuple{x}{n}) = \vb{x}^T\vb{A}\vb{x}%
\xlongequal{\vb{x}=\vb{C}\vb{y}} (\vb{C}\vb{y})^T\vb{A}(\vb{C}\vb{y})%
= \vb{y}^T (\vb{C}^T\vb{A}\vb{C}) \vb{y}\),
令\(\vb{B} = \vb{C}^T\vb{A}\vb{C}\),由于\(\vb{B}^T = (\vb{C}^T\vb{A}\vb{C})^T = \vb{C}^T\vb{A}\vb{C} = \vb{B}\),以及\(\vb{C}\)可逆,所以\(\vb{B}\)是对称矩阵.
\(f\)是二次型,它的矩阵\(\vb{A}\neq\vb0\),\(\vb{B}\cong\vb{A}\),故\(\vb{B}\neq\vb0\).
于是,\(g(\AutoTuple{y}{n})=\vb{y}^T\vb{B}\vb{y}\)是二次型,对称矩阵\(\vb{B}\)是\(g\)的矩阵.
\end{proof}
\end{theorem}

\section{矩阵的合同}
\subsection{矩阵合同的概念}
\begin{definition}
%@see: 《线性代数》(张慎语、周厚隆) P118 定义2
%@see: 《高等代数(第三版 上册)》(丘维声) P193 定义3
设\(\vb{A}\)和\(\vb{B}\)是数域\(K\)上的两个\(n\)阶矩阵.
若存在数域\(K\)上的一个可逆矩阵\(\vb{C}\),
使得\[
	\vb{B}=\vb{C}^T\vb{A}\vb{C},
\]
则称“\(\vb{A}\)与\(\vb{B}\)~\DefineConcept{合同}(congruent)”
“\(\vb{B}\)是\(\vb{A}\)的\DefineConcept{合同矩阵}”,
记为\(\vb{A}\simeq\vb{B}\).
\end{definition}

\subsection{矩阵合同的性质}
\begin{property}
合同矩阵与原矩阵等价,即\(\vb{A}\simeq\vb{B} \implies \vb{A}\cong\vb{B}\).
\begin{proof}
由\hyperref[definition:逆矩阵.矩阵等价]{矩阵等价的定义}显然有.
\end{proof}
\end{property}

\begin{property}
%@see: 《线性代数》(张慎语、周厚隆) P119 习题6.1 2(2)选项(A)
合同矩阵有相同的秩.
\begin{proof}
由\cref{theorem:矩阵乘积的秩.与可逆矩阵相乘不变秩} 可得.
\end{proof}
\end{property}

\begin{proposition}\label{theorem:矩阵合同.合同矩阵的行列式的关系}
设数域\(K\)上的\(n\)阶矩阵\(\vb{A},\vb{B}\)
和数域\(K\)上的\(n\)阶可逆矩阵\(\vb{C}\)
满足\(\vb{B} = \vb{C}^T \vb{A} \vb{C}\),
则\[
	\abs{\vb{A}} \abs{\vb{C}}^2
	= \abs{\vb{B}}.
\]
\begin{proof}
根据\cref{theorem:行列式.性质1,theorem:行列式.矩阵乘积的行列式},
必有\[
	\abs{\vb{B}}
	= \abs{\vb{C}^T \vb{A} \vb{C}}
	= \abs{\vb{C}^T} \abs{\vb{A}} \abs{\vb{C}}
	= \abs{\vb{A}} \abs{\vb{C}}^2.
	\qedhere
\]
\end{proof}
\end{proposition}

\begin{proposition}
%@see: 《线性代数》(张慎语、周厚隆) P119 习题6.1 2(2)选项(B)
合同矩阵的行列式同号.
\begin{proof}
设\(\vb{A},\vb{B} \in M_n(K)\).
假设\(\vb{A}\simeq\vb{B}\),
那么存在数域\(K\)上\(n\)阶可逆矩阵\(\vb{C}\),
使得\[
	\vb{C}^T \vb{A} \vb{C} = \vb{B}.
\]
由\cref{theorem:矩阵合同.合同矩阵的行列式的关系} 有\[
	\abs{\vb{A}} \abs{\vb{B}}
	= \abs{\vb{A}} (\abs{\vb{A}} \abs{\vb{C}}^2)
	= \abs{\vb{A}}^2 \abs{\vb{C}}^2
	\geq 0.
	\qedhere
\]
\end{proof}
\end{proposition}

\subsection{相似与合同的联系}
\begin{proposition}
相似的两个矩阵不一定合同.
\begin{proof}
例如,矩阵\(\vb{A}=\begin{bmatrix}
	1 & 1 \\
	0 & 2
\end{bmatrix}\)
和\(\vb{B}=\begin{bmatrix}
	0 & -1 \\
	2 & 3
\end{bmatrix}\)相似,
这是因为可逆矩阵\(\vb{P}=\begin{bmatrix}
	1 & 0 \\
	-1 & -1
\end{bmatrix}\)满足\(\vb{P}^{-1}\vb{A}\vb{P}=\vb{B}\).
%@Mathematica: Inverse[{{1, 0}, {-1, -1}}].{{1, 1}, {0, 2}}.{{1, 0}, {-1, -1}}
现在假设存在矩阵\(\vb{Q}=\begin{bmatrix}
	a & b \\
	c & d
\end{bmatrix}\)使得\(\vb{Q}^T\vb{A}\vb{Q}=\vb{B}\).
那么有\[
	\vb{Q}^T\vb{A}\vb{Q}-\vb{B}
	=\begin{bmatrix}
		a^2+ac+2c^2 & 1+ab+ad+2cd \\
		-2+ab+bc+2cd & -3+b^2+bd+2d^2
	\end{bmatrix}
	=\vb0,
\]
%@Mathematica: Expand[{{a, c}, {b, d}}.{{1, 1}, {0, 2}}.{{a, b}, {c, d}} - {{0, -1}, {2, 3}}]
得到关于\(a,b,c,d\)的方程组
\begin{align*}
	\left\{ \begin{array}{l}
		a^2+ac+2c^2 = 0, \\
		1+ab+ad+2cd = 0, \\
		-2+ab+bc+2cd = 0, \\
		-3+b^2+bd+2d^2 = 0,
	\end{array} \right.
	\tag*{$\begin{matrix}(1)\\(2)\\(3)\\(4)\end{matrix}$}
\end{align*}
(2)式减去(3)式得\[
	3+ad-bc=0,
\]
这就是说\(\abs{\vb{Q}}=-3\).
但是根据\cref{theorem:矩阵合同.合同矩阵的行列式的关系},
因为\(\abs{\vb{A}}=\abs{\vb{B}}=2\),所以\(\abs{\vb{Q}}^2=1\).
矛盾!因此\(\vb{A}\)与\(\vb{B}\)不合同!
\end{proof}
\end{proposition}

\begin{proposition}\label{example:矩阵合同.合同矩阵不一定相似}
合同的两个矩阵不一定相似.
\begin{proof}
例如,矩阵\(\vb{A}=\begin{bmatrix}
	1 & 0 \\
	0 & 1
\end{bmatrix}\)与\(\vb{B}=\begin{bmatrix}
	4 & 0 \\
	0 & 1
\end{bmatrix}\)合同,
这是因为可逆矩阵\(\vb{Q}=\begin{bmatrix}
	2 & 0 \\
	0 & 1
\end{bmatrix}\)满足\(\vb{Q}^T\vb{A}\vb{Q}=\vb{B}\).
%@Mathematica: {{2, 0}, {0, 1}}.{{1, 0}, {0, 1}}.{{2, 0}, {0, 1}}
但是\(\tr\vb{A}=2 \neq \tr\vb{B} = 5\),
根据\hyperref[theorem:特征值与特征向量.相似矩阵的迹的不变性]{相似矩阵的迹的不变性},
\(\vb{A}\)与\(\vb{B}\)必不相似!
\end{proof}
\end{proposition}

\begin{proposition}\label{theorem:二次型.实对称矩阵相似必合同}
设\(\vb{A},\vb{B}\)都是实对称矩阵,
则“\(\vb{A}\)与\(\vb{B}\)相似”是“\(\vb{A}\)与\(\vb{B}\)合同”的充分不必要条件.
\begin{proof}
因为\(\vb{A}\)、\(\vb{B}\)都是实对称矩阵,
\(\vb{A}^T=\vb{A}\),
\(\vb{B}^T=\vb{B}\),
且存在正交矩阵\(\vb{Q}_1,\vb{Q}_2\)使得\[
	\vb{Q}_1^{-1} \vb{A} \vb{Q}_1 = \vb{Q}_1^T \vb{A} \vb{Q}_1 = \vb{\Lambda}_1,
	\qquad
	\vb{Q}_2^{-1} \vb{B} \vb{Q}_2 = \vb{Q}_2^T \vb{B} \vb{Q}_2 = \vb{\Lambda}_2,
\]
其中\(\vb{\Lambda}_1,\vb{\Lambda}_2\)是对角阵.
又因为\(\vb{A}\sim\vb{B}\),
所以\(\vb{A}\)与\(\vb{B}\)有相同的特征多项式、特征值,
即\(\vb{\Lambda}_1=\vb{\Lambda}_2\),
或\[
	\vb{Q}_1^{-1} \vb{A} \vb{Q}_1 = \vb{Q}_2^{-1} \vb{B} \vb{Q}_2,
	\qquad
	(\vb{Q}_2 \vb{Q}_1^{-1}) \vb{A} (\vb{Q}_1 \vb{Q}_2^{-1}) = \vb{B}.
\]
令\(\vb{P} = \vb{Q}_1 \vb{Q}_2^{-1}\),
\(\vb{P}^T = (\vb{Q}_1 \vb{Q}_2^{-1})^T
= (\vb{Q}_2^{-1})^T \vb{Q}_1^T
= \vb{Q}_2 \vb{Q}_1^{-1}
= (\vb{Q}_1 \vb{Q}_2^{-1})^{-1}
= \vb{P}^{-1}\),
那么\[
	\vb{P}^T \vb{A} \vb{P} = \vb{B},
\]
也就是说\(\vb{A}\)与\(\vb{B}\)合同.

再根据\cref{example:矩阵合同.合同矩阵不一定相似},合同的两个实对称矩阵未必相似.
因此,对于实对称矩阵\(\vb{A},\vb{B}\)而言,
“\(\vb{A}\)与\(\vb{B}\)相似”是“\(\vb{A}\)与\(\vb{B}\)合同”的充分不必要条件.
\end{proof}
\end{proposition}

\begin{example}
设\(\AutoTuple{i}{n}\)是\(1,2,\dotsc,n\)的一个排列,
而\[
	\vb{A}=\diag(\AutoTuple{d}{n}),
	\qquad
	\vb{B}=\diag(d_{i_1},d_{i_2},\dotsc,d_{i_n}).
\]
证明:矩阵\(\vb{A},\vb{B}\)合同且相似.
\begin{proof}
显然\(\diag(d_{i_1},d_{i_2},\dotsc,d_{i_n})\)
可由\(\diag(\AutoTuple{d}{n})\)同时左乘和右乘若干个初等矩阵\[
	\vb{P}(i,j) \quad(1 \leq i < j \leq n)
\]得到,
又因为\(\vb{P}(i,j)^T = \vb{P}(i,j)^{-1} = \vb{P}(i,j)\),
所以只要令这些初等矩阵的乘积为\(\vb{P}\),
就有\[
	\vb{P}^{-1} \vb{A} \vb{P} = \vb{B},
	\qquad
	\vb{P}^T \vb{A} \vb{P} = \vb{B}.
\]
也就是说\(\vb{A}\simeq\vb{B}\land\vb{A}\sim\vb{B}\).
\end{proof}
\end{example}

\subsection{合同类}
\begin{property}\label{theorem:矩阵合同.合同关系是等价关系}
%@see: 《线性代数》(张慎语、周厚隆) P118
%@see: 《高等代数(第三版 上册)》(丘维声) P193
矩阵的合同关系是等价关系,
即具备下列三条性质:\begin{itemize}
	\item {\rm\bf 反身性}:
	\(\vb{A}\simeq\vb{A}\);
	\item {\rm\bf 对称性}:
	\(\vb{A}\simeq\vb{B} \implies \vb{B}\simeq\vb{A}\);
	\item {\rm\bf 传递性}:
	\(\vb{A}\simeq\vb{B} \land \vb{B}\simeq\vb{C} \implies \vb{A}\simeq\vb{C}\).
\end{itemize}
\begin{proof}
合同是集合\(M_n(K)\)上的一个二元关系.

由于\[
	\vb{A} = \vb{E}^T \vb{A} \vb{E},
\]
所以\(\vb{A} \simeq \vb{A}\).

假设\(\vb{A} \simeq \vb{B}\),
那么存在可逆矩阵\(\vb{C}\)使得\[
	\vb{B}=\vb{C}^T\vb{A}\vb{C},
\]
于是\[
	\vb{A}=(\vb{C}^T)^{-1}\vb{B}\vb{C}^{-1}
	=(\vb{C}^{-1})^T\vb{B}\vb{C}^{-1},
\]
即\(\vb{B} \simeq \vb{A}\).

假设\(\vb{A} \simeq \vb{B}\)且\(\vb{B} \simeq \vb{C}\),
那么存在可逆矩阵\(\vb{D}_1,\vb{D}_2\)
使得\[
	\vb{B} = \vb{D}_1^T \vb{A} \vb{D}_1, \qquad
	\vb{C} = \vb{D}_2^T \vb{B} \vb{D}_2,
\]
从而有\[
	\vb{C} = \vb{D}_2^T (\vb{D}_1^T \vb{A} \vb{D}_1) \vb{D}_2
	= (\vb{D}_1 \vb{D}_2)^T \vb{A} (\vb{D}_1 \vb{D}_2),
\]
因此\(\vb{A} \simeq \vb{C}\).
\end{proof}
\end{property}

\begin{definition}
%@see: 《高等代数(第三版 上册)》(丘维声) P193
把矩阵\(\vb{A} \in M_n(K)\)在合同关系下的等价类\[
	\Set{ \vb{B} \in M_n(K) \given \vb{A} \simeq \vb{B} }
\]称为“矩阵\(\vb{A}\)的\DefineConcept{合同类}”.
\end{definition}

\begin{proposition}\label{theorem:对称矩阵.对称矩阵的合同类}
对称矩阵的合同矩阵也是对称的,
即\[
	\vb{A}^T = \vb{A}
	\implies
	[\vb{A}\simeq\vb{B} \implies \vb{B}^T = \vb{B}].
\]
\begin{proof}
设可逆矩阵\(\vb{C}\)满足\(\vb{C}^T\vb{A}\vb{C}=\vb{B}\),
那么\[
	\vb{B}^T = (\vb{C}^T\vb{A}\vb{C})^T = \vb{C}^T\vb{A}^T\vb{C} = \vb{C}^T\vb{A}\vb{C} = \vb{B}.
	\qedhere
\]
\end{proof}
\end{proposition}

\begin{proposition}\label{theorem:反对称矩阵.反对称矩阵的合同类}
反对称矩阵的合同矩阵也是反对称的,
即\[
	\vb{A}^T = -\vb{A}
	\implies
	[\vb{A}\simeq\vb{B} \implies \vb{B}^T = -\vb{B}].
\]
\begin{proof}
设可逆矩阵\(\vb{C}\)满足\(\vb{C}^T\vb{A}\vb{C}=\vb{B}\),
那么\[
	\vb{B}^T = (\vb{C}^T\vb{A}\vb{C})^T = \vb{C}^T\vb{A}^T\vb{C} = -\vb{C}^T\vb{A}\vb{C} = -\vb{B}.
	\qedhere
\]
\end{proof}
\end{proposition}

\begin{remark}
从\cref{theorem:对称矩阵.对称矩阵的合同类,theorem:反对称矩阵.反对称矩阵的合同类}
可以看出:对称矩阵不可能与反对称矩阵合同,它们都不可能与非对称矩阵合同.
\end{remark}

\begin{proposition}
设\(\vb{A},\vb{B} \in M_n(K)\).
记\begin{gather*}
	\vb{A}_1 = \frac12 (\vb{A}+\vb{A}^T), \qquad
	\vb{A}_2 = \frac12 (\vb{A}-\vb{A}^T), \\
	\vb{B}_1 = \frac12 (\vb{B}+\vb{B}^T), \qquad
	\vb{B}_2 = \frac12 (\vb{B}-\vb{B}^T),
\end{gather*}
其中\(\vb{A}^T,\vb{B}^T\)分别为\(\vb{A},\vb{B}\)的转置.
那么\[
	\vb{A} \simeq \vb{B} \implies \vb{A}_1 \simeq \vb{B}_1 \land \vb{A}_2 \simeq \vb{B}_2.
\]
\begin{proof}
当\(\vb{A} \simeq \vb{B}\)时,
必存在可逆矩阵\(\vb{C}\),使得\[
	\vb{B}
	=\vb{C}^T\vb{A}\vb{C};
	\eqno(1)
\]
于是\[
	\vb{B}^T
	=(\vb{C}^T\vb{A}\vb{C})^T
	=\vb{C}^T\vb{A}^T\vb{C},
	\eqno(2)
\]
也就是说\(\vb{B}^T \simeq \vb{A}^T\).
由(1)(2)两式不难得到\[
	\vb{B}\pm\vb{B}^T
	=\vb{C}^T(\vb{A}\pm\vb{A}^T)\vb{C},
\]
于是\(\vb{A}_1\simeq\vb{B}_1,
\vb{A}_2\simeq\vb{B}_2\).
\end{proof}
\end{proposition}

\subsection{合同对角化}
\begin{definition}
%@see: 《高等代数(第三版 上册)》(丘维声) P194
设矩阵\(\vb{A} \in M_n(K)\).
如果存在\(n\)阶对角矩阵\(\vb\Lambda\)与\(\vb{A}\)合同,
则称“矩阵\(\vb{A}\)可以\DefineConcept{合同对角化}”
“矩阵\(\vb\Lambda\)是\(\vb{A}\)的\DefineConcept{合同标准型}”.
\end{definition}

\section{二次型化为标准型的三种方法}
二次型的基本问题是研究如何通过非退化的线性替换将一个二次型化简为平方和的形式,从而讨论其性质.

\begin{definition}
%@see: 《线性代数》(张慎语、周厚隆) P120 定义3
如果二次型\(f(\AutoTuple{x}{n})=\vb{x}^T\vb{A}\vb{x}\)
可以经过非退化的线性替换\(\vb{x} = \vb{C} \vb{y}\)化简为\begin{equation*}
	d_1 y_1^2 + d_2 y_2^2 + \dotsb + d_n y_n^2
\end{equation*}的形式,
则称上式为
“二次型\(f\)的\DefineConcept{标准型}”
或“二次型\(f\)的\DefineConcept{规范型}”.
\end{definition}
将二次型化为标准型的问题可以归结为对称矩阵合同于对角阵的问题.
下面介绍三种化二次型为标准型的方法.

\subsection{正交变换法}
根据\cref{theorem:特征值与特征向量.实对称矩阵3},
对于任何一个实对称矩阵\(\vb{A}\),
存在正交矩阵\(\vb{Q}\),
使得\begin{equation*}
	\vb{Q}^{-1}\vb{A}\vb{Q} = \vb{Q}^T\vb{A}\vb{Q} = \diag(\AutoTuple{\lambda}{n}),
\end{equation*}
即实对称矩阵\(\vb{A}\)与对角阵\(\diag(\AutoTuple{\lambda}{n})\)合同且相似.

\begin{theorem}
%@see: 《线性代数》(张慎语、周厚隆) P120 定理3
对于任一\(n\)元实二次型\(f(\AutoTuple{x}{n})=\vb{x}^T\vb{A}\vb{x}\ (\vb{A}=\vb{A}^T)\),
都存在正交矩阵\(\vb{Q}\),
由\(\vb{Q}\)构成的线性替换\(\vb{x}=\vb{Q}\vb{y}\)
(称为\DefineConcept{正交变换},{\rm orthogonal operator})
将\(f\)化为标准型\begin{equation*}
	f(\AutoTuple{x}{n})
	\xlongequal{\vb{x}=\vb{Q}\vb{y}}
	\lambda_1y_1^2+\lambda_2y_2^2+ \dotsb +\lambda_ny_n^2
\end{equation*}
其中\(\lambda_1,\lambda_2,\dotsc,\lambda_n\)是\(\vb{A}\)的全部特征值.
\end{theorem}

\begin{corollary}
%@see: 《线性代数》(张慎语、周厚隆) P120
正交变换的特点之一是保持向量的内积不变,也就是保持向量的长度不变,保持图形的形状不变.
\begin{proof}
设\(\vb{Q}\)为正交矩阵.
令\(\vb{x}_1=\vb{Q}\vb{y}_1\),
\(\vb{x}_2=\vb{Q}\vb{y}_2\),
必有\begin{align*}
	\VectorInnerProductDot{\vb{x}_1}{\vb{x}_2}
	&=\VectorInnerProductDot{(\vb{Q}\vb{y}_1)}{(\vb{Q}\vb{y}_2)}
	=(\vb{Q}\vb{y}_1)^T (\vb{Q}\vb{y}_2)
	=\vb{y}_1^T \vb{Q}^T \vb{Q} \vb{y}_2 \\
	&=\vb{y}_1^T \vb{E} \vb{y}_2
	=\vb{y}_1^T \vb{y}_2
	=\VectorInnerProductDot{\vb{y}_1}{\vb{y}_2},
\end{align*}
又令\(\vb{x}_1=\vb{x}_2=\vb{x}\),
\(\vb{y}_1=\vb{y}_2=\vb{y}\),
则\(\abs{\vb{x}}^2=\abs{\vb{y}}^2\),
\(\abs{\vb{x}}=\abs{\vb{y}}\).
\end{proof}
\end{corollary}

\begingroup
\color{red}
用正交变换法将二次型化为标准型的步骤如下:
\begin{enumerate}
	\item 首先根据\(f\)的表达式写出\(f\)的矩阵\(\vb{A}\);
	\item 写出\(f\)矩阵的特征多项式\(\abs{\lambda\vb{E}-\vb{A}}=0\),
	求解\(\vb{A}\)的特征值\(\AutoTuple{\lambda}{n}\)
	以及对应的特征向量\(\AutoTuple{\vb{x}}{n}\);
	\item 运用施密特规范化方法,
	将\(\vb{A}\)的特征向量正交单位化为\(\AutoTuple{\vb\gamma}{n}\),
	然后写成正交矩阵\(\vb{Q}=(\AutoTuple{\vb\gamma}{n})\);
	\item 计算得出标准型的矩阵
	\(\vb{B}=\vb{Q}^T\vb{A}\vb{Q}=\vb{Q}^{-1}\vb{A}\vb{Q}=\diag(\AutoTuple{\lambda}{n})\).
\end{enumerate}
\endgroup

\begin{example}
%@see: 《线性代数》(张慎语、周厚隆) P121 例1
设\(f(x_1,x_2,x_3) = -x_1^2-x_2^2-7x_3^2-4x_1x_2+8x_1x_3+8x_2x_3\).
利用正交变换将\(f\)化为标准型,并写出所用的正交变换.
\begin{solution}
\(f\)的矩阵为\begin{equation*}
	\vb{A} = \begin{bmatrix}
		-1 & -2 & 4 \\
		-2 & -1 & 4 \\
		4 & 4 & -7
	\end{bmatrix}.
\end{equation*}
令\begin{equation*}
	\abs{\lambda\vb{E}-\vb{A}}
	= \begin{bmatrix}
		\lambda+1 & 2 & -4 \\
		2 & \lambda+1 & -4 \\
		-4 & -4 & \lambda+7
	\end{bmatrix}
	= (\lambda-1)^2 (\lambda+11)
	= 0,
\end{equation*}
解得特征值\(\lambda_1=1\ (\text{二重})\),\(\lambda_2=-11\).

当\(\lambda=1\)时,解方程\((\vb{E}-\vb{A})\vb{x}=\vb0\),\begin{equation*}
	\vb{E}-\vb{A} = \begin{bmatrix}
		2 & 2 & -4 \\
		2 & 2 & -4 \\
		-4 & -4 & 8
	\end{bmatrix}
	\to \begin{bmatrix}
		2 & 2 & -4 \\
		0 & 0 & 0 \\
		0 & 0 & 0
	\end{bmatrix},
\end{equation*}基础解系为\begin{equation*}
	\vb{x}_1 = \begin{bmatrix} -1 \\ 1 \\ 0 \end{bmatrix},
	\qquad
	\vb{x}_2 = \begin{bmatrix} 2 \\ 0 \\ 1 \end{bmatrix}.
\end{equation*}
利用施密特方法将其正交化,得\begin{equation*}
	\vb{y}_1=\vb{x}_1,
	\qquad
	\vb{y}_2=\vb{x}_2-\frac{\vb{x}_2\cdot\vb{y}_1}{\vb{y}_1\cdot\vb{y}_1}\vb{y}_1
	=\begin{bmatrix} 1 \\ 1 \\ 1 \end{bmatrix};
\end{equation*}再将其单位化,得\begin{equation*}
	\vb{Z}_1
	= \frac{1}{\sqrt{2}} \begin{bmatrix} -1 \\ 1 \\ 0 \end{bmatrix},
	\qquad
	\vb{Z}_2
	= \frac{1}{\sqrt{3}} \begin{bmatrix} 1 \\ 1 \\ 1 \end{bmatrix}.
\end{equation*}

当\(\lambda=-11\)时,解方程\((-11\vb{E}-\vb{A})\vb{x}=\vb0\),\begin{equation*}
	-11\vb{E}-\vb{A} = \begin{bmatrix}
		-10 & 2 & -4 \\
		2 & -10 & -4 \\
		-4 & -4 & -4
	\end{bmatrix}
	\to \begin{bmatrix}
		1 & 1 & 1 \\
		0 & 2 & 1 \\
		0 & 0 & 0
	\end{bmatrix},
\end{equation*}
基础解系为\begin{equation*}
	\vb{x}_3 = \begin{bmatrix} 1 \\ 1 \\ -2 \end{bmatrix},
\end{equation*}
单位化得\begin{equation*}
	\vb{Z}_3 = \frac{1}{\sqrt{6}} \begin{bmatrix} 1 \\ 1 \\ -2 \end{bmatrix}.
\end{equation*}

令\begin{equation*}
	\vb{Q} = (\vb{Z}_1,\vb{Z}_2,\vb{Z}_3)
	= \begin{bmatrix}
		-\frac{1}{\sqrt{2}} & \frac{1}{\sqrt{3}} & \frac{1}{\sqrt{6}} \\
		\frac{1}{\sqrt{2}} & \frac{1}{\sqrt{3}} & \frac{1}{\sqrt{6}} \\
		0 & \frac{1}{\sqrt{3}} & -\frac{2}{\sqrt{6}} \\
	\end{bmatrix},
\end{equation*}\(\vb{Q}\)是正交矩阵,满足\(\vb{Q}^T\vb{A}\vb{Q}=\vb{Q}^{-1}\vb{A}\vb{Q}=\diag(1,1,-11)\).
作正交变换\(\vb{x}=\vb{Q}\vb{y}\),于是\(f\)化为标准型\(y_1^2+y_2^2-11y_3^2\).
\end{solution}
\end{example}

\begin{example}
%@see: 《线性代数》(张慎语、周厚隆) P122 例2
设实二次型\begin{equation*}
	f(\AutoTuple{x}{n})=\vb{x}^T\vb{A}\vb{x}
\end{equation*}的矩阵\(\vb{A}\)的特征值为\(\AutoTuple{\lambda}{n}\),
\(c=\max\{\AutoTuple{\lambda}{n}\}\).
证明:对于任意\(n\)维实向量\(\vb{x}\),都有\begin{equation*}
	f(\AutoTuple{x}{n}) \leq c \vb{x}^T\vb{x}.
\end{equation*}
\begin{proof}
因为\(\vb{A}\)是实对称矩阵,
即存在正交矩阵\(\vb{Q}\)
使得\(\vb{Q}^T\vb{A}\vb{Q} = \diag(\AutoTuple{\lambda}{n})\),
作正交变换\(\vb{x}=\vb{Q}\vb{y}\),
则\(f\)化为标准型\begin{align*}
	f(\AutoTuple{x}{n})
	&= \lambda_1 y_1^2 + \lambda_2 y_2^2 + \dotsb + \lambda_n y_n^2 \\
	&\leq c y_1^2 + c y_2^2 + \dotsb + c y_n^2 \\
	&= c \vb{y}^T \vb{y}
	= c \vb{x}^T \vb{x}.
	\qedhere
\end{align*}
\end{proof}
\end{example}

\subsection{拉格朗日配方法}
用正交变换能够化实二次型为标准型,这种方法是根据实对称矩阵的性质,
求出二次型矩阵的特征值和规范正交的特征向量,条件要求较强.
当研究一般数域\(K\)上的二次型(包括实二次型)的标准型时,
可以用\DefineConcept{拉格朗日配方法}.
这种方法不用解矩阵特征值问题,
只需反复利用以下两个初等代数公式\begin{equation*}
	a^2+2ab+b^2=(a+b)^2,
	\quad
	a^2-b^2=(a+b)(a-b)
\end{equation*}就能将二次型化为标准型.

但是在对二次型配方时,应该做到一点:
每配得一个完全平方式,余下的多项式就减少一个变量.
在完成配方以后,还应该写出线性替换矩阵\(\vb{C}\),检验它是不是可逆线性替换.
如果\(\vb{C}\)不是可逆的,那么配方结果就是错误的.
例如二次型\begin{equation*}
	f(x_1,x_2,x_3) = (x_1-x_2)^2 + (x_2-x_3)^2 + (x_3-x_1)^2
\end{equation*}看上去好像已经完成配方了,
但是它对应的线性替换矩阵\begin{equation*}
	\vb{C} = \begin{bmatrix}
		1 & -1 & 0 \\
		0 & 1 & -1 \\
		-1 & 0 & 1 \\
	\end{bmatrix}
	% \to \begin{bmatrix}
	% 	1 & 0 & -1 \\
	% 	0 & 1 & -1 \\
	% 	0 & 0 & 0
	% \end{bmatrix}
\end{equation*}是奇异矩阵,
于是我们只能把原式展开得到\begin{equation*}
	f(x_1,x_2,x_3)
	= 2 x_1^2 + 2 x_2^2 + 2 x_3^2
	- 2 x_1 x_2 - 2 x_1 x_3 - 2 x_2 x_3,
\end{equation*}
写出它的矩阵\begin{equation*}
	\vb{A} = \begin{bmatrix}
		2 & -1 & -1 \\
		-1 & 2 & -1 \\
		-1 & -1 & 2
	\end{bmatrix},
\end{equation*}
求出对应的相似标准型\(\diag(3,3,0)\),
最后写出标准型\begin{equation*}
	g(y_1,y_2,y_3) = 3 y_1^2 + 3 y_2^2.
\end{equation*}

\begin{example}
%@see: 《线性代数》(张慎语、周厚隆) P122 例3
用配方法化二次型\(f(x_1,x_2,x_3)
= x_1^2 + 2 x_1 x_2 + 2 x_2^2 - 3 x_2 x_3\)为标准型,并求出所用的可逆线性替换.
\begin{solution}
首先有\begin{align*}
	f(x_1,x_2,x_3)
	&= x_1^2 + 2 x_1 x_2 + 2 x_2^2 - 3 x_2 x_3 \\
	&= x_1^2 + 2 x_1 x_2 + x_2^2 + x_2^2 - 3 x_2 x_3 \\
	&= (x_1 + x_2)^2 + x_2^2 - 3 x_2 x_3 + \frac{9}{4} x_3^2 - \frac{9}{4} x_3^2 \\
	&= (x_1 + x_2)^2 + \left( x_2 - \frac{3}{2} x_3 \right)^2 - \frac{9}{4} x_3^2,
\end{align*}
令\begin{equation*}
	\left\{ \def\arraystretch{1.5} \begin{array}{*7r}
		y_1 &= &x_1 &+&x_2 \\
		y_2 &= & & & x_2 & -& \frac{3}{2} x_3 \\
		y_3 &= & & & & & x_3
	\end{array} \right.
	\eqno(1)
\end{equation*}
则\begin{equation*}
	\left\{ \def\arraystretch{1.5} \begin{array}{*7r}
	x_1 &= &y_1 &-&y_2 &-&\frac{3}{2} y_3 \\
	x_2 &= & & & y_2 & +& \frac{3}{2} y_3 \\
	x_3 &= & & & & & y_3
	\end{array} \right.
	\eqno(2)
\end{equation*}
(2)是可逆线性替换,使\(f(x_1,x_2,x_3) = y_1^2 + y_2^2 - \frac{9}{4} y_3^2\).
\end{solution}
\end{example}

\begin{theorem}
%@see: 《线性代数》(张慎语、周厚隆) P125 定理4
对于任意一个\(n\)元二次型\(f(\AutoTuple{x}{n})=\vb{x}^T\vb{A}\vb{x}\ (\vb{A}=\vb{A}^T)\),
都存在非退化的线性替换\(\vb{x}=\vb{C}\vb{y}\),
使之成为\begin{equation*}
	f(\AutoTuple{x}{n})=d_1 y_1^2 + d_2 y_2^2 + \dotsb + d_n y_n^2.
\end{equation*}
\begin{proof}
应用数学归纳法.
当\(n=1\)时,\(f(x_1) = a_{11} x_1^2 = d_1 y_1^2\),结论成立.

假设当\(n=k-1\ (k\geq2)\)时结论成立;那么当\(n=k\)时,我们分以下两种情况进行讨论:
\begin{enumerate}
\item 若\(f\)的平方项系数不全为零,不妨设\(a_{11}\neq0\),
从而有\begin{align*}
	f(\AutoTuple{x}{n})
	&= a_{11} x_1^2 + 2 x_1 \sum_{j=2}^n a_{1j} x_j
		+ \sum_{i=2}^n \sum_{j=2}^n a_{ij} x_i x_j \\
	&= a_{11} \left[
		x_1 + \frac{1}{a_{11}} \sum_{j=2}^n a_{1j} x_j
	\right]^2
	- \frac{1}{a_{11}} \left[
		\sum_{j=2}^n a_{1j} x_j
	\right]^2
	+ \sum_{i=2}^n \sum_{j=2}^n a_{ij} x_i x_j,
\end{align*}
令\begin{equation*}
	y_1 = x_1 + \frac{1}{a_{11}} \sum_{j=2}^n a_{1j} x_j, \qquad
	y_2 = x_2, \qquad
	\dotsc, \qquad
	y_n = x_n,
\end{equation*}
则\begin{equation*}
	f(\AutoTuple{x}{n}) = a_{11} y_1^2 + g(\AutoTuple{y}[2]{n}),
\end{equation*}
其中\(g(\AutoTuple{y}[2]{n})
= - \frac{1}{a_{11}} \left[
	\sum_{j=2}^n a_{1j} x_j
\right]^2
+ \sum_{i=2}^n \sum_{j=2}^n a_{ij} x_i x_j\)%
是一个\(n-1\)元二次型或零多项式.
由归纳假设,存在\(n-1\)阶可逆矩阵\(\vb{Q}\),
得到非退化线性变换\((\AutoTuple{y}[2]{n})^T = \vb{Q} (\AutoTuple{z}[2]{n})^T\),
使得\begin{equation*}
	g(\AutoTuple{y}[2]{n})
	= d_2 z_2^2 + d_3 z_3^2 + \dotsb + d_n z_n^2.
\end{equation*}
记\begin{equation*}
	\vb{P} = \begin{bmatrix}
		1 & -a_{11}^{-1} a_{12} & -a_{11}^{-1} a_{13} & \dots & -a_{11}^{-1} a_{1n} \\
		& 1 & 0 & \dots & 0 \\
		& & 1 & \dots & 0 \\
		& & & \ddots & \vdots \\
		& & & & 1
	\end{bmatrix},
\end{equation*}则非退化线性替换\begin{equation*}
	\begin{bmatrix}
		x_1 \\ x_2 \\ \vdots \\ x_n
	\end{bmatrix}
	= \vb{P} \begin{bmatrix}
		y_1 \\ y_2 \\ \vdots \\ y_n
	\end{bmatrix}
	= \vb{P} \begin{bmatrix} 1 & \vb0 \\ \vb0 & \vb{Q} \end{bmatrix} \begin{bmatrix}
		z_1 \\ z_2 \\ \vdots \\ z_n
	\end{bmatrix}
\end{equation*}
使得\(f(\AutoTuple{x}{n}) = a_{11} z_1^2 + d_2 z_2^2 + d_3 z_3^2 + \dotsb + d_n z_n^2\).

\item 若\(f\)不含平方项,则必存在\(1 \leq i < j \leq n\)使得\(a_{ij}\neq0\);
于是\begin{equation*}
	\left\{ \begin{array}{l}
		x_i = y_i + y_j \\
		x_j = y_i - y_j \\
		x_k = y_k\ (1 \leq k \leq n \land k \neq i \land k \neq j)
	\end{array} \right.
\end{equation*}是非退化的线性替换,
使得\begin{equation*}
	f(\AutoTuple{x}{n})
	= g(\AutoTuple{y}{n})
	= a_{ij} y_i^2 - a_{ij} y_j^2 + h(\AutoTuple{y}{n}),
\end{equation*}
其中\(h(\AutoTuple{y}{n})\)是不含平方项的二次型或零多项式,
故\(g(\AutoTuple{y}{n})\)含有平方项,这归结为第一种情形,从而可以化为标准型.
\qedhere
\end{enumerate}
\end{proof}
\end{theorem}

\begin{corollary}
%@see: 《线性代数》(张慎语、周厚隆) P126 推论1
任意\(n\)阶对称矩阵\(\vb{A}\)都与对角形矩阵合同.
\end{corollary}

\begin{example}
%@see: 《2018年全国硕士研究生入学统一考试(数学一)》三解答题/第20题
设实二次型\(f(x_1,x_2,x_3) = (x_1-x_2+x_3)^2 + (x_2+x_3)^2 + (x_1+ax_3)^2\),
其中\(a\)是未知参数.
求\(f(x_1,x_2,x_3)=0\)的解和规范型.
\begin{solution}
记\begin{equation*}
	\vb{C} \defeq \begin{bmatrix}
		1 & -1 & 1 \\
		0 & 1 & 1 \\
		1 & 0 & a
	\end{bmatrix}.
\end{equation*}
作初等行变换得\begin{equation*}
	\vb{C} \to \vb{C}_1 = \begin{bmatrix}
		1 & 0 & 2 \\
		0 & 1 & 1 \\
		0 & 0 & a-2
	\end{bmatrix}.
\end{equation*}

当\(a\neq2\)时,\(\rank\vb{C}=3\),方程\(\vb{C}\vb{x}=\vb0\)只有零解.

当\(a=2\)时,\(\rank\vb{C}=2\),方程\(\vb{C}\vb{x}=\vb0\)
解得\(k (-2,-1,1)^T\ (\text{$k$是任意常数})\).

注意到\(f(x_1,x_2,x_3) = (x_1-x_2+x_3)^2 + (x_2+x_3)^2 + (x_1+ax_3)^2 \geq 0\).

当\(a\neq2\)时,由于方程\(f(x_1,x_2,x_3)=0\)只有零解,\(f\)是正定的.
因此可令\begin{equation*}
	\left\{ \begin{array}{l}
		y_1 = x_1-x_2+x_3, \\
		y_2 = x_2+x_3, \\
		y_3 = x_1+ax_3,
	\end{array} \right.
\end{equation*}
于是\(f\)的规范型为\(g(y_1,y_2,y_3) = y_1^2 + y_2^2 + y_3^2\).

当\(a=2\)时,\(\vb{C}\)不满秩,将\(f\)展开得\begin{align*}
	f(x_1,x_2,x_3) &= (x_1-x_2+x_3)^2 + (x_2+x_3)^2 + (x_1+2x_3)^2 \\
	&= 2 x_1^2 - 2 x_1 x_2 + 2 x_2^2 + 6 x_1 x_3 + 6 x_3^2,
\end{align*}
写出对应的矩阵\begin{equation*}
	\vb{A} = \begin{bmatrix}
		2 & -1 & 3 \\
		-1 & 2 & 0 \\
		3 & 0 & 6
	\end{bmatrix},
\end{equation*}
解\(\vb{A}\)的特征方程\(\abs{\lambda\vb{E}-\vb{A}}=\lambda(\lambda^2-10\lambda+18)=0\)
得\(\lambda\in\Set{0,5\pm\sqrt7}\),
\(f\)的正惯性指数为\(2\),
于是\(f\)的规范型为\(h(z_1,z_2,z_3) = z_1^2+z_2^2\).
\end{solution}
\end{example}

\subsection{初等变换法}
\begin{definition}
%@see: 《线性代数》(张慎语、周厚隆) P126 定义4
设\(\vb{P}\)为初等矩阵.
对于任意矩阵\(\vb{B}\),称变换\(\vb{B} \to \vb{P}^T\vb{B}\vb{P}\)为
对\(\vb{B}\)作一次\DefineConcept{合同变换}.
\end{definition}

\begin{theorem}
因为对于任一对称阵\(\vb{A}\),存在可逆矩阵\(\vb{C}\),使得\begin{equation*}
	\vb{C}^T\vb{A}\vb{C}=\diag(\AutoTuple{d}{n}).
\end{equation*}
又因为存在初等矩阵\(\vb{P}_1,\vb{P}_2,\dotsc,\vb{P}_m\),使得\begin{equation*}
	\vb{C}=\vb{P}_1\vb{P}_2\dotsm\vb{P}_m,
\end{equation*}
那么\begin{equation*}
	\vb{C}^T\vb{A}\vb{C}
	= (\vb{P}_m^T\vb{P}_{m-1}^T\dotsm\vb{P}_1^T)\vb{A}(\vb{P}_1\vb{P}_2\dotsm\vb{P}_m)
	= \diag(\AutoTuple{d}{n}).
\end{equation*}

因为初等矩阵有三类:\begin{itemize}
	\item \(\vb{P}(i,j)^T\vb{B}\vb{P}(i,j)=\vb{P}(i,j)\vb{B}\vb{P}(i,j)\),
	相当于交换\(\vb{B}\)的\(i,j\)两行,再交换\(\vb{P}(i,j)\vb{B}\)的\(i,j\)两列.
	\item \(\vb{P}(i(c))^T\vb{B}\vb{P}(i(c))=\vb{P}(i(c))\vb{B}\vb{P}(i(c))\ (c \neq 0)\),
	相当于用\(c\)乘\(\vb{B}\)的\(i\)行,再用\(c\)乘\(\vb{P}(i(c))\vb{B}\)的\(i\)列.
	\item \(\vb{P}(i,j(k))^T\vb{B}\vb{P}(i,j(k))=\vb{P}(j,i(k))\vb{B}\vb{P}(i,j(k))\),
	相当于将\(\vb{B}\)的\(i\)行的\(k\)倍加到\(j\)行,
	再将\(\vb{P}(j,i(k))\vb{B}\)的\(i\)列的\(k\)倍加到\(j\)列.
\end{itemize}
可见,对称矩阵\(\vb{A}\)可以经过一系列合同变换化为对角形矩阵.
则\begin{equation*}
	\begin{bmatrix} \vb{A} \\ \vb{E} \end{bmatrix}
	\to
	\begin{bmatrix} \vb{C}^T & \vb0 \\ \vb0 & \vb{E} \end{bmatrix}
	\begin{bmatrix} \vb{A} \\ \vb{E} \end{bmatrix}
	\vb{C}
	= \begin{bmatrix} \vb{C}^T\vb{A}\vb{C} \\ \vb{C} \end{bmatrix},
\end{equation*}\begin{equation*}
	(\vb{A},\vb{E})
	\to
	\vb{C}^T (\vb{A},\vb{E}) \begin{bmatrix}
		\vb{C} & \vb0 \\
		\vb0 & \vb{E}
	\end{bmatrix}
	= (\vb{C}^T\vb{A}\vb{C},\vb{C}^T).
\end{equation*}
其中\(\vb{C}=\vb{P}_1\vb{P}_2\dotsm\vb{P}_m\),
即对\(\vb{A}\)作一系列合同变换化为对角阵\(\vb{C}^T\vb{A}\vb{C}\),
只对\(\vb{E}\)进行列变换,将\(\vb{E}\)变成\(\vb{C}\);
或者只对\(\vb{E}\)作其中的行变换,则将\(\vb{E}\)变为\(\vb{C}^T\).
\end{theorem}

\begin{example}
%@see: 《2023年全国硕士研究生入学统一考试(数学一)》三解答题/第21题
设二次型\begin{align*}
	f(x_1,x_2,x_3) &= x_1^2 + 2 x_2^2 + 2 x_3^2 + 2 x_1 x_2 - 2 x_1 x_3, \\
	g(y_1,y_2,y_3) &= y_1^2 + y_2^2 + y_3^2 + 2 y_2 y_3.
\end{align*}
求可逆变换\(\vb{x} = \vb{P} \vb{y}\)将\(f(x_1,x_2,x_3)\)化为\(g(y_1,y_2,y_3)\).
\begin{solution}\let\qed\relax
\begin{proof}[解法一]
配方,得\begin{align*}
	f(x_1,x_2,x_3) &= x_1^2 + 2 x_2^2 + 2 x_3^2 + 2 x_1 x_2 - 2 x_1 x_3 \\
	&= (x_1 + x_2 - x_3)^2 + x_2 + x_3^2 + 2 x_2 x_3 \\
	&= (x_1 + x_2 - x_3)^2 + (x_2 + x_3)^2, \\
	g(y_1,y_2,y_3) &= y_1^2 + y_2^2 + y_3^2 + 2 y_2 y_3 \\
	&= y_1^2 + (y_2 + y_3)^2,
\end{align*}

令\(\left\{ \begin{array}{l}
	z_1 = x_1 + x_2 - x_3, \\
	z_2 = x_2 + x_3, \\
	z_3 = x_3,
\end{array} \right.\)
即\(\begin{bmatrix}
	z_1 \\ z_2 \\ z_3
\end{bmatrix}
= \begin{bmatrix}
	1 & 1 & -1 \\
	0 & 1 & 1 \\
	0 & 0 & 1
\end{bmatrix}
\begin{bmatrix}
	x_1 \\ x_2 \\ x_3
\end{bmatrix}\),
可逆变换\(\vb{x} = \vb{P}_1 \vb{z}\)将\(f(x_1,x_2,x_3)\)化为\(h(z_1,z_2,z_3)\),
其中\begin{equation*}
	\vb{P}_1 = \begin{bmatrix}
		1 & 1 & -1 \\
		0 & 1 & 1 \\
		0 & 0 & 1
	\end{bmatrix}^{-1}.
\end{equation*}

令\(\left\{ \begin{array}{l}
	z_1 = y_1, \\
	z_2 = y_2 + y_3, \\
	z_3 = y_3,
\end{array} \right.\)
即\(\begin{bmatrix}
	z_1 \\ z_2 \\ z_3
\end{bmatrix}
= \begin{bmatrix}
	1 & 0 & 0 \\
	0 & 1 & 1 \\
	0 & 0 & 1
\end{bmatrix}
\begin{bmatrix}
	y_1 \\ y_2 \\ y_3
\end{bmatrix}\),
可逆变换\(\vb{z} = \vb{P}_2 \vb{y}\)将\(h(z_1,z_2,z_3)\)化为\(g(y_1,y_2,y_3)\),
其中\begin{equation*}
	\vb{P}_2 = \begin{bmatrix}
		1 & 0 & 0 \\
		0 & 1 & 1 \\
		0 & 0 & 1
	\end{bmatrix}.
\end{equation*}

因此,若要可逆变换\(\vb{x} = \vb{P} \vb{y}\)将\(f(x_1,x_2,x_3)\)化为\(g(y_1,y_2,y_3)\),
只需令\begin{equation*}
%@Mathematica: Inverse[{{1, 1, -1}, {0, 1, 1}, {0, 0, 1}}] . {{1, 0, 0}, {0, 1, 1}, {0, 0, 1}}
	\vb{P} = \vb{P}_1 \vb{P}_2
	= \begin{bmatrix}
		1 & 1 & -1 \\
		0 & 1 & 1 \\
		0 & 0 & 1
	\end{bmatrix}^{-1}
	\begin{bmatrix}
		1 & 0 & 0 \\
		0 & 1 & 1 \\
		0 & 0 & 1
	\end{bmatrix}
	= \begin{bmatrix}
		1 & -1 & 1 \\
		0 & 1 & 0 \\
		0 & 0 & 1
	\end{bmatrix}.
\end{equation*}
\end{proof}
\begin{proof}[解法二]
二次型\(f(x_1,x_2,x_3)\)的矩阵为\begin{equation*}
	\vb{A} = \begin{bmatrix}
		1 & 1 & -1 \\
		1 & 2 & 0 \\
		-1 & 0 & 2
	\end{bmatrix}.
\end{equation*}
二次型\(g(y_1,y_2,y_3)\)的矩阵为\begin{equation*}
	\vb{B} = \begin{bmatrix}
		1 & 0 & 0 \\
		0 & 1 & 1 \\
		0 & 1 & 1
	\end{bmatrix}.
\end{equation*}
将\(\vb{A}\)的第2行减去第1行,第2列减去第1列,
得到\begin{equation*}
	\vb{A} \to \begin{bmatrix}
		1 & 1 & -1 \\
		0 & 1 & 1 \\
		-1 & 0 & 2
	\end{bmatrix}
	\to \begin{bmatrix}
		1 & 0 & -1 \\
		0 & 1 & 1 \\
		-1 & 1 & 2
	\end{bmatrix};
\end{equation*}
再让第3行加上第1行,第3列加上第1列,
得到\begin{equation*}
	\vb{A} \to \begin{bmatrix}
		1 & 0 & -1 \\
		0 & 1 & 1 \\
		0 & 1 & 1
	\end{bmatrix}
	\to \begin{bmatrix}
		1 & 0 & 0 \\
		0 & 1 & 1 \\
		0 & 1 & 1
	\end{bmatrix}
	= \vb{B}.
\end{equation*}
于是所求可逆矩阵\(\vb{P}\)就是\(\vb{P}_1 = \begin{bmatrix}
	1 & -1 & 0 \\
	0 & 1 & 0 \\
	0 & 0 & 1
\end{bmatrix}\)和\(\vb{P}_2 = \begin{bmatrix}
	1 & 0 & 1 \\
	0 & 1 & 0 \\
	0 & 0 & 1
\end{bmatrix}\)这两个初等矩阵的乘积,
即\begin{equation*}
	\vb{P} = \begin{bmatrix}
		1 & -1 & 1 \\
		0 & 1 & 0 \\
		0 & 0 & 1
	\end{bmatrix}.
\end{equation*}
\end{proof}
\end{solution}
\end{example}

\section{实二次型的分类}
\subsection{实二次型的分类标准}
\begin{definition}\label{definition:实二次型的分类.实二次型的分类}
%@see: 《线性代数》(张慎语、周厚隆) P129 定义5
%@see: 《线性代数》(张慎语、周厚隆) P131 定义7
给定\(n\)元实二次型\(f(\vb{x}) = \vb{x}^T\vb{A}\vb{x}\).
\begin{enumerate}
	\item 如果\begin{equation*}
		(\forall\vb{x}\in\mathbb{R}^n-\{\vb0\})
		[f(\vb{x}) > 0],
	\end{equation*}
	则称“\(f\)是\DefineConcept{正定的}(positive definite)”,
	“\(\vb{A}\)是一个\DefineConcept{正定矩阵}(positive definite matrix)”,
	记为\(\vb{A}\succ\vb0\).

	\item 如果\begin{equation*}
		(\forall\vb{x}\in\mathbb{R}^n-\{\vb0\})
		[f(\vb{x}) \geq 0],
	\end{equation*}
	则称“\(f\)是\DefineConcept{半正定的}(positive semi-definite)”,
	“\(\vb{A}\)是一个\DefineConcept{半正定矩阵}(positive semi-definite matrix)”,
	记为\(\vb{A}\succeq\vb0\).

	\item 如果\begin{equation*}
		(\forall\vb{x}\in\mathbb{R}^n-\{\vb0\})
		[f(\vb{x}) < 0],
	\end{equation*}
	则称“\(f\)是\DefineConcept{负定的}(negative definite)”,
	“\(\vb{A}\)是一个\DefineConcept{负定矩阵}(negative definite matrix)”,
	记为\(\vb{A}\prec\vb0\).

	\item 如果\begin{equation*}
		(\forall\vb{x}\in\mathbb{R}^n-\{\vb0\})
		[f(\vb{x}) \leq 0],
	\end{equation*}
	则称“\(f\)是\DefineConcept{半负定的}(negative semi-definite)”,
	“\(\vb{A}\)是一个\DefineConcept{半负定矩阵}(negative semi-definite matrix)”,
	记为\(\vb{A}\preceq\vb0\).

	\item 否则,称\(f\)是\DefineConcept{不定的}(indefinite).
\end{enumerate}
\end{definition}

\begin{example}
%@see: 《线性代数》(张慎语、周厚隆) P129 例1
\(f(x_1,x_2,x_3) \defeq 3 x_1^2 + x_2^2 + 5 x_3^2\)是正定的.
\end{example}

\begin{example}
%@see: 《线性代数》(张慎语、周厚隆) P129 例1
\(f(x_1,x_2,x_3) \defeq -x_1^2 - x_3^2\)是半负定的.
\end{example}

\begin{example}
%@see: 《线性代数》(张慎语、周厚隆) P129 例1
\(f(x_1,x_2,x_3) \defeq x_1 x_2 + x_2^2 + 5 x_3^2\)是不定的.
\end{example}

\subsection{惯性定理}
\begin{theorem}\label{theorem:二次型.惯性定理}
%@see: 《线性代数》(张慎语、周厚隆) P129 惯性定理(inertial theorem)
\(n\)元实二次型\(f(\vb{x}) = \vb{x}^T\vb{A}\vb{x}\)经过任意满秩线性变换化为标准型,
所得的标准型的正平方项的项数\(p\)及负平方项的项数\(q\)都是唯一确定的.
\begin{proof}
设实二次型的秩为\(r\).
假设\(f\)经过两个不同的可逆线性替换\(\vb{x}=\vb{C}\vb{y},\vb{x}=\vb{D}\vb{z}\)分别化为标准型\begin{equation*}
	f \xlongequal{\vb{x}=\vb{C}\vb{y}}
	c_1 y_1^2 + c_2 y_2^2 + \dotsb + c_p y_p^2 - c_{p+1} y_{p+1}^2 - \dotsb - c_r y_r^2,
	\eqno(1)
\end{equation*}\begin{equation*}
	f \xlongequal{\vb{x}=\vb{D}\vb{z}}
	d_1 z_1^2 + d_2 z_2^2 + \dotsb + d_q z_q^2 - d_{q+1} z_{q+1}^2 - \dotsb - d_r z_r^2,
	\eqno(2)
\end{equation*}
其中\(c_i,d_i>0\ (i=1,2,\dotsc,r)\).

用反证法.
设\(p > q\),由\(\vb{x} = \vb{C}\vb{y} = \vb{D}\vb{z}\),\(\vb{D}\)可逆,得\(\vb{z} = \vb{D}^{-1} \vb{C} \vb{y}\).
\def\zexpr#1{h_{#1 1} y_1 + h_{#1 2} y_2 + \dotsb + h_{#1 n} y_n}%
记\(\vb{H} = (h_{ij})_n = \vb{B}^{-1} \vb{C}\),
则\(\vb{z} = \vb{H}\vb{y}\),
即\begin{equation*}
	z_i = \zexpr{i}
	\quad(i=1,2,\dotsc,n).
\end{equation*}
于是\begin{equation*}
	\begin{aligned}
		&\hspace{-40pt}
		c_1 y_1^2 + c_2 y_2^2 + \dotsb
			+ c_p y_p^2 - c_{p+1} y_{p+1}^2 - \dotsb - c_r y_r^2 \\
		&\hspace{-20pt}= d_1 (\zexpr{1})^2 + d_2 (\zexpr{2})^2 \\
		&+ \dotsb + d_q (\zexpr{q})^2 \\
		&- d_{q+1} (\zexpr{q+1})^2 - \dotsb \\
		&- d_r (\zexpr{r})^2.
	\end{aligned}
	\eqno(3)
\end{equation*}
由此可以构造齐次线性方程组\begin{equation*}
	\begin{cases}
		\zexpr{1} = 0, \\
		\hdotsfor{1} \\
		\zexpr{q} = 0, \\
		y_{p+1} = 0, \\
		\hdotsfor{1} \\
		y_n = 0.
	\end{cases}
	\eqno(4)
\end{equation*}
这个方程组中有\(n\)个未知量,\(q+n-p < n\)个方程,
于是它有非零解\((\AutoTuple{y}{p},0,\dotsc,0)^T\),
代入(3)式两端,得到左边大于零,右边小于等于零,矛盾,因此\(p \leq q\).
同理又有\(q \leq p\),于是\(p = q\).
\end{proof}
%@see: https://mathworld.wolfram.com/SylvestersInertiaLaw.html
\end{theorem}
我们把\cref{theorem:二次型.惯性定理} 称为“惯性定理(Inertial Theorem)”.

\begin{corollary}
%@see: 《线性代数》(张慎语、周厚隆) P130 推论
任意\(n\)元实二次型\(f(\vb{x}) = \vb{x}^T\vb{A}\vb{x}\),
总可经过满秩线性变换,化为以下形式:\begin{equation*}
	y_1^2+y_2^2+ \dotsb +y_p^2
	-y_{p+1}^2-\dotsb-y_r^2.
\end{equation*}
我们将其称为“\(f(\vb{x})\)的\DefineConcept{规范型}({\rm normal form})”,
且规范型是唯一的.
\begin{proof}
根据惯性定理,\(f\)经过可逆线性替换化为标准型:\begin{equation*}
	f \xlongequal{\vb{x}=\vb{D}\vb{z}}
	d_1 z_1^2 + d_2 z_2^2 + \dotsb + d_q z_q^2 - d_{q+1} z_{q+1}^2 - \dotsb - d_r z_r^2,
\end{equation*}
其中\(d_i>0\ (i=1,2,\dotsc,r)\).
令\begin{equation*}
	\vb{C} = \vb{D} \diag(d_1^{-1/2},\dotsc,d_r^{-1/2},1,\dotsc,1),
\end{equation*}
则\(\vb{x} = \vb{D}\vb{z} = \vb{C}\vb{y}\)是可逆线性替换,
使得\begin{equation*}
	f \xlongequal{\vb{x}=\vb{D}\vb{z}} y_1^2 + y_2^2 + \dotsb + y_q^2 - y_{q+1}^2 - \dotsb - y_r^2.
\end{equation*}
二次型的规范型的唯一性可以由惯性定理得到.
\end{proof}
\end{corollary}

\begin{definition}\label{definition:二次型.惯性系数的定义}
%@see: 《线性代数》(张慎语、周厚隆) P131 定义6
在秩为\(r\)的实二次型\(f(\vb{x})\)所化成的标准型(或规范型)中,
\begin{itemize}
	\item 正平方项的项数\(p\)
	称为“\(f\)的\DefineConcept{正惯性指数}(positive index of inertia)”;
	\item 负平方项的项数\(q=r-p\)
	称为“\(f\)的\DefineConcept{负惯性指数}(minus index of inertia)”;
	\item 正、负惯性指数之差\(d=p-q\)
	称为“\(f\)的\DefineConcept{符号差}(signature)”.
\end{itemize}
% 下面是一个数值算法,它的原理是:先计算矩阵的特征值,再分别统计正特征值和负特征值的个数.
%@Mathematica: InertiaIndices[A_?SymmetricMatrixQ] :=
%				Module[{eigvals},
%				eigvals = Eigenvalues[A];        (* 计算所有特征值 *)
%				{Count[eigvals, _?Positive],     (* 正惯性指数 *)
%				Count[eigvals, _?Negative]       (* 负惯性指数 *)}];
\end{definition}
由\hyperref[theorem:二次型.惯性定理]{惯性定理}%
和\cref{definition:二次型.惯性系数的定义} 可知:
可逆线性替换不改变二次型的正、负惯性指数.
因此我们可以根据二次型的正、负惯性指数确定二次型的类型.

\begin{theorem}
设\(\vb{A}\)和\(\vb{B}\)是同阶实对称矩阵.
这两个矩阵合同的充分必要条件是两者的秩、正负惯性指数均相等,
即\begin{equation*}
	\vb{A}\simeq\vb{B}
	\iff
	\rank\vb{A}=\rank\vb{B},
	p_{\vb{A}}=p_{\vb{B}},
	q_{\vb{A}}=q_{\vb{B}}.
\end{equation*}
\end{theorem}

\subsection{正定矩阵的等价条件}
\begin{theorem}
%@see: 《线性代数》(张慎语、周厚隆) P131 定理5
设\(\vb{A}\)为\(n\)阶实对称矩阵,
\(f(\AutoTuple{x}{n}) = \vb{x}^T\vb{A}\vb{x}\),
则下列命题相互等价:\begin{itemize}
	\item \(\vb{A}\)为正定矩阵;
	\item \(\vb{A}\)的特征值全是正实数;
	\item \(f(\AutoTuple{x}{n})\)的正惯性指数\(p=n\);
	\item \(\vb{A} \cong \vb{E}\);
	\item 存在可逆实阵\(\vb{P}\),使得\(\vb{A}=\vb{P}^T\vb{P}\).
\end{itemize}
%TODO proof
\end{theorem}

\begin{corollary}
%@see: 《线性代数》(张慎语、周厚隆) P132 推论1
正定矩阵的行列式大于零.
\begin{proof}
因为\(\vb{A}\)正定,
所以存在可逆实阵\(\vb{P}\),使得\(\vb{A}=\vb{P}^T\vb{P}\),
则\begin{equation*}
	\abs{\vb{A}}
	=\abs{\vb{P}^T\vb{P}}
	=\abs{\vb{P}^T} \abs{\vb{P}\vphantom{\vb{P}^T}}
	=\abs{\vb{P}}^2>0.
	\qedhere
\end{equation*}
\end{proof}
\end{corollary}

\begin{theorem}
%@see: 《线性代数》(张慎语、周厚隆) P132 定理6
\(n\)元实二次型\(f(\AutoTuple{x}{n}) = \vb{x}^T\vb{A}\vb{x}\ (\vb{A}=\vb{A}^T)\)正定的充分必要条件是:
矩阵\(\vb{A}\)的各阶顺序主子式\(\det\vb{A}_k\)均大于零,
即\begin{equation*}
	\det\vb{A}_k > 0
	\quad(k=1,2,\dotsc,n).
\end{equation*}
\begin{proof}
必要性.
对于任意不全为零的\(n\)个实数\(c_1,c_2,\dotsc,c_k,0,\dotsc,0\),
总有\begin{equation*}
	f(c_1,c_2,\dotsc,c_k,0,\dotsc,0)
	= \sum_{i=1}^k \sum_{j=1}^k c_i c_j a_{ij} > 0,
\end{equation*}
从而\(k\)元实二次型\(f_k(x_1,x_2,\dotsc,x_k)
=\sum_{i=1}^k
\sum_{j=1}^k
x_i x_j a_{ij}\)正定,
而\(f_k\)的矩阵为\(\vb{A}_k = (a_{ij})_k\),
那么\(\abs{\vb{A}_k} > 0\ (k=1,2,\dotsc,n)\).

充分性.当\(n=1\)时,\(a_{11} > 0\),\(f_1(x_1) = a_{11} x_1^2\)正定.
设\(n=k-1\)时结论成立,
当\(n=k\)时,
将\(\vb{A}\)分块得\(\vb{A} = \begin{bmatrix}
	\vb{A}_{k-1} & \vb\alpha \\
	\vb\alpha^T & a_{nn}
\end{bmatrix}\),
其中\(\vb{A}_{k-1}\)为各阶顺序主子式都大于零的\(k-1\)阶实对称矩阵.
由归纳假设,\(\vb{A}_{k-1}\)正定,故存在\(k-1\)阶可逆矩阵\(\vb{Q}\),
使得\(\vb{A}_{k-1} = \vb{Q}^T \vb{Q}\),\(\vb{A}_{k-1}\)可逆,
\(\vb{A}_{k-1}^{-1} = \vb{Q}^{-1}(\vb{Q}^{-1})^T\)是对称矩阵.
令\(\vb{P} = \begin{bmatrix}
	\vb{Q}^{-1} & -\vb{A}_{k-1}^{-1} \vb\alpha \\
	\vb0 & 1
\end{bmatrix}\),
则\(\vb{P}\)可逆,
于是\begin{align*}
	\vb{P}^T \vb{A} \vb{P} &= \begin{bmatrix}
		(\vb{Q}^{-1})^T & \vb0 \\
		-\vb\alpha^T \vb{A}_{k-1}^{-1} & 1
	\end{bmatrix}
	\begin{bmatrix}
		\vb{Q}^T \vb{Q} & \vb\alpha \\
		\vb\alpha^T & a_{nn}
	\end{bmatrix}
	\begin{bmatrix}
		\vb{Q}^{-1} & -\vb{A}_{k-1}^{-1} \vb\alpha \\
		\vb0 & 1
	\end{bmatrix} \\
	&= \begin{bmatrix}
		\vb{Q} & (\vb{Q}^{-1})^T \vb\alpha \\
		\vb0 & b
	\end{bmatrix}
	\begin{bmatrix}
		\vb{Q}^{-1} & -\vb{A}_{k-1}^{-1} \vb\alpha \\
		\vb0 & 1
	\end{bmatrix}
	= \begin{bmatrix}
		\vb{E}_{k-1} & \vb0 \\
		\vb0 & b
	\end{bmatrix} = \vb{B},
\end{align*}
其中\(b=a_{nn}-\vb\alpha^T \vb{A}_{k-1}^{-1} \vb\alpha\).
由于\(\vb{A}\)与\(\vb{B}\)合同,\(\abs{\vb{A}} > 0\),
得\(\abs{\vb{B}} = b > 0\),
作可逆线性替换\(\vb{x} = \vb{P}\vb{y}\),则\begin{equation*}
	f \xlongequal{\vb{x}=\vb{Q}\vb{y}} y_1^2 + y_2^2 + \dotsb + y_{n-1}^2 + b y_n^2,
\end{equation*}
故\(f\)的正惯性指数为\(n\),\(f\)正定.
\end{proof}
\end{theorem}

\begin{corollary}
%@see: 《线性代数》(张慎语、周厚隆) P133 推论2
\(n\)元实二次型\(f(\AutoTuple{x}{n}) = \vb{x}^T\vb{A}\vb{x}\ (\vb{A}=\vb{A}^T)\)负定的充分必要条件是:
矩阵\(\vb{A}\)的奇数阶顺序主子式为负,偶数阶顺序主子式为正,
即\begin{equation*}
	(-1)^k \det\vb{A}_k > 0
	\quad(k=1,2,\dotsc,n).
\end{equation*}
\end{corollary}

\begin{proposition}
设\(\vb{A} \in M_{s \times n}(\mathbb{R})\),
则\begin{equation*}
	\text{\(\vb{A}^T\vb{A}\)正定}
	\iff
	\rank\vb{A} = n.
\end{equation*}
\begin{proof}
显然矩阵\(\vb{A}^T\vb{A}\)是\(n\)阶实对称矩阵,
于是\begin{align*}
	&\text{矩阵\(\vb{A}^T\vb{A}\)正定} \\
	&\iff (\forall\vb{x}\in\mathbb{R}^n-\{\vb0\})[\vb{x}^T (\vb{A}^T \vb{A}) \vb{x} > 0]
		\tag{\hyperref[definition:实二次型的分类.实二次型的分类]{正定矩阵的定义}} \\
	&\iff (\forall\vb{x}\in\mathbb{R}^n-\{\vb0\})[\norm{\vb{A}\vb{x}} > 0] \\
	&\iff (\forall\vb{x}\in\mathbb{R}^n-\{\vb0\})[\vb{A}\vb{x}\neq\vb0] \\
	&\iff \text{\(\vb{A}\vb{x}=\vb0\)只有零解} \\
	&\iff \rank\vb{A} = n,
		\tag{\cref{theorem:向量空间.有解的非齐次线性方程组的解的个数定理}}
\end{align*}
也就是说,“矩阵\(\vb{A}^T\vb{A}\)正定”的充分必要条件是“\(\vb{A}\)是列满秩矩阵”.
\end{proof}
\end{proposition}

\begin{example}
设\(\vb{A} \in M_s(\mathbb{R}),
\vb{B} \in M_n(\mathbb{R})\).
证明:
若矩阵\(\begin{bmatrix}
	\vb{A} & \vb0 \\
	\vb0 & \vb{B}
\end{bmatrix}\)是正定矩阵,
则\(\vb{A}\)和\(\vb{B}\)都是正定矩阵.
\begin{proof}
由题意有,
对于\(\forall\vb{x}\in\mathbb{R}^s,
\forall\vb{y}\in\mathbb{R}^n\),
只要\((\vb{x},\vb{y})\neq\vb0\),
就有\begin{equation*}
	\begin{bmatrix}
		\vb{x}^T & \vb{y}^T
	\end{bmatrix}
	\begin{bmatrix}
		\vb{A} & \vb0 \\
		\vb0 & \vb{B}
	\end{bmatrix}
	\begin{bmatrix}
		\vb{x} \\ \vb{y}
	\end{bmatrix}
	= \vb{x}^T \vb{A} \vb{x} + \vb{y}^T \vb{B} \vb{y}
	> 0.
	\eqno(1)
\end{equation*}
当\(\vb{x}\neq\vb0,\vb{y}=\vb0\)时,
由(1)式有\(\vb{x}^T\vb{A}\vb{x}>0\);
这就是说,矩阵\(\vb{A}\)正定.
同理,当\(\vb{x}=\vb0,\vb{y}\neq\vb0\)时,
由(1)式有\(\vb{y}^T\vb{B}\vb{y}>0\);
这就是说,矩阵\(\vb{B}\)正定.
\end{proof}
\end{example}

\begin{example}
%@see: 《线性代数》(张慎语、周厚隆) P134 例4
设\(\vb{A}\)为实对称矩阵.
证明:当实数\(t\)充分大时,\(t\vb{E}+\vb{A}\)是正定矩阵.
\begin{proof}
因为\(\vb{A}\)为实对称矩阵,所以存在正交矩阵\(\vb{Q}\),使得\begin{equation*}
	\vb{Q}^T\vb{A}\vb{Q} = \diag(\AutoTuple{\lambda}{n}).
\end{equation*}
又因为\begin{equation*}
	\vb{Q}^T(t\vb{E}+\vb{A})\vb{Q}
	= \diag(t+\lambda_1,t+\lambda_2,\dotsc,t+\lambda_n),
\end{equation*}
所以当\(t+\lambda_1,t+\lambda_2,\dotsc,t+\lambda_n\)都大于零时,\(t\vb{E}+\vb{A}\)正定.
\end{proof}
\end{example}

\begin{example}
%@see: 《线性代数》(张慎语、周厚隆) P134 习题6.3 3.(1)
%@see: 《线性代数》(张慎语、周厚隆) P134 习题6.3 3.(2)
设\(\vb{A}\)、\(\vb{B}\)是同阶正定矩阵.
证明:\(\vb{A}+\vb{B}\)、\(\vb{A}^{-1}\)、\(\vb{A}^*\)是正定矩阵.
\begin{proof}
根据正定矩阵的定义,因为\(\vb{A}\)是正定矩阵,任取非零列向量\(\vb{x}\),都有\begin{equation*}
	\vb{x}^T\vb{A}\vb{x} > 0;
\end{equation*}
同样地,有\(\vb{x}^T\vb{B}\vb{x} > 0\).

又根据矩阵的乘法分配律,成立\begin{equation*}
	\vb{x}^T(\vb{A}+\vb{B})\vb{x} = \vb{x}^T\vb{A}\vb{x} + \vb{x}^T\vb{B}\vb{x} > 0,
\end{equation*}
即\(\vb{A}+\vb{B}\)是正定矩阵.

因为\(\vb{A}\)是正定矩阵,存在可逆实阵\(\vb{P}\)使得\(\vb{P}^T\vb{A}\vb{P}=\vb{E}\),
所以\begin{equation*}
	\vb{A} = (\vb{P}^T)^{-1}\vb{E}\vb{P}^{-1} = (\vb{P}^T)^{-1}\vb{P}^{-1}
	\implies
	\vb{A}^{-1} = \vb{P}\vb{P}^T,
\end{equation*}
说明\(\vb{A}^{-1}\)是正定矩阵.

由逆矩阵的定义,\(\vb{A}^{-1}=\frac{1}{\abs{\vb{A}}}\vb{A}^*\),
那么\(\vb{A}^*=\abs{\vb{A}}\vb{A}^{-1}\),\(\abs{\vb{A}}>0\),
显然\(\vb{A}^*\)也是正定矩阵.
\end{proof}
\end{example}

\begin{example}
%@see: 《线性代数》(张慎语、周厚隆) P134 习题6.3 3.(3)
设\(\vb{A}\)是正定矩阵.
试证:存在正定矩阵\(\vb{B}\),使得\(\vb{A}=\vb{B}^2\).
\begin{proof}
设\(\vb{A}\)是\(n\)阶正定矩阵,那么存在正交矩阵\(\vb{P}\)满足\begin{equation*}
	\vb{P}^T\vb{A}\vb{P} = \diag(\AutoTuple{\lambda}{n}) = \vb{\Lambda},
\end{equation*}
其中\(\AutoTuple{\lambda}{n} \in \mathbb{R}^+\).
又设矩阵\(\vb{B}\)满足\(\vb{B}^2=\vb{A}\),那么\begin{equation*}
	\vb{P}^T\vb{A}\vb{P} = \vb{P}^T\vb{B}^2\vb{P} = \vb{\Lambda}
	\iff
	\vb{B}^2 = \vb{P}\vb{\Lambda}\vb{P}^T,
\end{equation*}
只需要令\(\vb{B} = \vb{P} \diag(\sqrt{\lambda_1},\sqrt{\lambda_2},\dotsc,\sqrt{\lambda_n}) \vb{P}^T\)即可.
\end{proof}
\end{example}

\begin{theorem}
设\(\vb{A}\)为\(n\)阶实对称矩阵,
\(f(\AutoTuple{x}{n}) = \vb{x}^T\vb{A}\vb{x}\),
则下列命题相互等价:\begin{itemize}
	\item \(\vb{A}\)是半正定矩阵.
	\item \(\vb{A}\)的特征值全是非负实数.
	\item \(f(\AutoTuple{x}{n})\)的正惯性指数\(p\)等于\(\vb{A}\)的秩\(\rank\vb{A}\).
	\item \(\vb{A} \cong \diag(\vb{E}_r,\vb0)\),其中\(\vb{E}_r\)是\(r\)阶单位矩阵.
\end{itemize}
%TODO proof
\end{theorem}

\section{二次型的应用}
下面利用矩阵的运算及二次型理论讨论平面二次曲线、空间二次曲面的分类问题.

\subsection{平面二次曲线的分类问题}
首先考虑一般的二次曲线的方程:\begin{equation}\label{equation:二次型的应用.平面二次曲线的一般方程}
%@see: 《线性代数》(张慎语、周厚隆) P139 (1)
	a_{11} x_1^2
	+ 2 a_{12} x_1 x_2
	+ a_{22} x_2^2
	+ 2 b_1 x_1
	+ 2 b_2 x_2
	+ c
	= 0.
\end{equation}
设\begin{equation*}
	\vb{X}
	\defeq \begin{bmatrix}
		x_1 \\ x_2
	\end{bmatrix},
	\qquad
	\vb{A}
	\defeq \begin{bmatrix}
		a_{11} & a_{12} \\
		a_{12} & a_{22}
	\end{bmatrix},
	\qquad
	\vb{b}
	\defeq \begin{bmatrix}
		b_1 \\ b_2
	\end{bmatrix},
\end{equation*}
于是\cref{equation:二次型的应用.平面二次曲线的一般方程} 可以写成\begin{equation*}
	\vb{X}^T \vb{A} \vb{X}
	+ \vb{b}^T \vb{X}
	+ c
	= 0,
\end{equation*}
或\begin{equation}\label{equation:二次型的应用.平面二次曲线的一般方程.矩阵形式}
%@see: 《线性代数》(张慎语、周厚隆) P139 (2)
	\begin{bmatrix}
		\vb{X}^T & 1
	\end{bmatrix}
	\begin{bmatrix}
		\vb{A} & \vb{b} \\
		\vb{b}^T & c
	\end{bmatrix}
	\begin{bmatrix}
		\vb{X} \\ 1
	\end{bmatrix}
	= 0.
\end{equation}
记\begin{equation*}
	f(\vb{X})
	\defeq
	\begin{bmatrix}
		\vb{X}^T & 1
	\end{bmatrix}
	\begin{bmatrix}
		\vb{A} & \vb{b} \\
		\vb{b}^T & c
	\end{bmatrix}
	\begin{bmatrix}
		\vb{X} \\ 1
	\end{bmatrix}.
\end{equation*}
于是\(\vb{X}^T \vb{A} \vb{X}\)
可以经过正交变换\(\vb{X} = \vb{Q} \vb{Y}\)化为标准型:\begin{equation*}
	\lambda_1 y_1^2 + \lambda_2 y_2^2,
\end{equation*}
其中\(\vb{Q}\)是二阶正交矩阵,
\(\lambda_1,\lambda_2\)是\(\vb{A}\)的两个特征值.
将\(\vb{X} = \vb{Q} \vb{Y}\)
代入\cref{equation:二次型的应用.平面二次曲线的一般方程.矩阵形式} 便得\begin{align*}
	f(\vb{X})
	&\xlongequal{\vb{X} = \vb{Q} \vb{Y}}
	\begin{bmatrix}
		\vb{Y}^T \vb{Q}^T & 1
	\end{bmatrix}
	\begin{bmatrix}
		\vb{A} & \vb{b} \\
		\vb{b}^T & c
	\end{bmatrix}
	\begin{bmatrix}
		\vb{Q} \vb{Y} \\ 1
	\end{bmatrix} \\
	&= \lambda_1 y_1^2 + \lambda_2 y_2^2 + d_1 y_1 + d_2 y_2 + c,
\end{align*}
这里\(\lambda_1,\lambda_2\)不全为零.
对上式非零的平方项与相应的一次项进行配方,
再作平移变换\(\vb{Y} = \vb{Z} + \vb{X}_0\),
其中\(\vb{Z} = \begin{bmatrix}
	z_1 \\ z_2
\end{bmatrix},
\vb{X}_0 = \begin{bmatrix}
	x'_1 \\ x'_2
\end{bmatrix}\),
便可化简得\begin{equation*}
	f(\vb{X})
	= \lambda_1 z_1^2 + \lambda_2 z_2^2 + d,
	\quad\text{或}\quad
	f(\vb{X})
	= \lambda_1 z_1^2 + q z_2.
\end{equation*}
同时\begin{equation*}
	\begin{bmatrix}
		\vb{X} \\ 1
	\end{bmatrix}
	= \begin{bmatrix}
		\vb{Q} (\vb{Z} + \vb{X}_0) \\ 1
	\end{bmatrix}
	= \begin{bmatrix}
		\vb{Q} & \vb\delta \\
		\vb0 & 1
	\end{bmatrix}
	\begin{bmatrix}
		\vb{Z} \\ 1
	\end{bmatrix},
\end{equation*}
其中\(\vb\delta = \vb{Q} \vb{X}_0 \in M_{2\times1}(\mathbb{R})\).
\cref{equation:二次型的应用.平面二次曲线的一般方程.矩阵形式}
化为\begin{equation}\label{equation:二次型的应用.平面二次曲线的一般方程.矩阵标准形式}
%@see: 《线性代数》(张慎语、周厚隆) P140 (3)
	\begin{bmatrix}
		\vb{Z}^T & 1
	\end{bmatrix}
	\begin{bmatrix}
		\vb{A}_1 & \vb{b}_1 \\
		\vb{b}_1^T & c_1
	\end{bmatrix}
	\begin{bmatrix}
		\vb{Z} \\ 1
	\end{bmatrix}
	= 0,
\end{equation}
其中\begin{equation*}
	\begin{bmatrix}
		\vb{A}_1 & \vb{b}_1 \\
		\vb{b}_1^T & c_1
	\end{bmatrix}
	= \begin{bmatrix}
		\vb{Q} & \vb0^T \\
		\vb\delta^T & 1
	\end{bmatrix}
	\begin{bmatrix}
		\vb{A} & \vb{b} \\
		\vb{b}^T & c
	\end{bmatrix}
	\begin{bmatrix}
		\vb{Q} & \vb\delta \\
		\vb0 & 1
	\end{bmatrix},
	\qquad
	\vb{A}_1
	= \vb{Q}^T \vb{A} \vb{Q}
	= \diag(\lambda_1,\lambda_2).
\end{equation*}

\begin{lemma}
%@see: 《线性代数》(张慎语、周厚隆) P140 引理2
经过正交变换\(\vb{X} = \vb{Q} \vb{Y}\)与平移\(\vb{Y} = \vb{Z} + \vb{X}_0\),
平面上任意两点的距离保持不变.
\begin{proof}
设\(\vb{X}_1,\vb{X}_2\)是平面上任意两点,
则它们的距离为\[
	d(\vb{X}_1,\vb{X}_2)
	= \sqrt{(\vb{X}_1-\vb{X}_2)^T(\vb{X}_1-\vb{X}_2)}.
\]
经过正交变换和平移,得到新的两点为\[
	\vb{Z}_1 = \vb{Q}^T \vb{X}_1 - \vb{X}_0,
	\qquad
	\vb{Z}_2 = \vb{Q}^T \vb{X}_2 - \vb{X}_0.
\]
于是\begin{align*}
	d(\vb{Z}_1,\vb{Z}_2)
	&= \sqrt{(\vb{Z}_1-\vb{Z}_2)^T(\vb{Z}_1-\vb{Z}_2)} \\
	&= \sqrt{
		((\vb{Q}^T \vb{X}_1 - \vb{X}_0) - (\vb{Q}^T \vb{X}_2 - \vb{X}_0))^T
		((\vb{Q}^T \vb{X}_1 - \vb{X}_0) - (\vb{Q}^T \vb{X}_2 - \vb{X}_0))
	} \\
	&= \sqrt{
		(\vb{Q}^T \vb{X}_1 - \vb{Q}^T \vb{X}_2)^T
		(\vb{Q}^T \vb{X}_1 - \vb{Q}^T \vb{X}_2)
	} \\
	&= \sqrt{
		(\vb{Q}^T (\vb{X}_1 - \vb{X}_2))^T
		(\vb{Q}^T (\vb{X}_1 - \vb{X}_2))
	} \\
	&= \sqrt{
		(\vb{X}_1 - \vb{X}_2)^T
		\vb{Q} \vb{Q}^T
		(\vb{X}_1 - \vb{X}_2)
	} \\
	&= \sqrt{
		(\vb{X}_1 - \vb{X}_2)^T
		(\vb{X}_1 - \vb{X}_2)
	}
	= d(\vb{X}_1,\vb{X}_2).
	\qedhere
\end{align*}
\end{proof}
\end{lemma}

\begin{theorem}[平面二次曲线的分类定理]
%@see: 《线性代数》(张慎语、周厚隆) P140 定理8
设\(\vb{X} = \begin{bmatrix}
	x_1 \\ x_2
\end{bmatrix},
\vb{A} = \begin{bmatrix}
	a_{11} & a_{12} \\
	a_{12} & a_{22}
\end{bmatrix},
\vb{b} = \begin{bmatrix}
	b_1 \\ b_2
\end{bmatrix}\),
则二次曲线\[
	\begin{bmatrix}
		\vb{X}^T & 1
	\end{bmatrix}
	\begin{bmatrix}
		\vb{A} & \vb{b} \\
		\vb{b}^T & c
	\end{bmatrix}
	\begin{bmatrix}
		\vb{X} \\ 1
	\end{bmatrix}
	= 0
\]可以经过正交变换、平移变换化为\[
	\lambda_1 z_1^2 + \lambda_2 z_2^2 + d = 0,
	\quad\text{或}\quad
	\lambda_1 z_1^2 + q z_2 = 0.
\]
%TODO proof 教材省略了证明过程,只提示应该讨论\(\vb{A}\)、\((\vb{A},\vb{b})\)和\(\begin{bmatrix} \vb{A} & \vb{b} \\ \vb{b}^T & c \end{bmatrix}\)这三个矩阵的秩.
\end{theorem}

记\begin{equation*}
	\Delta = \begin{vmatrix}
		\vb{A} & \vb{b} \\
		\vb{b}^T & c
	\end{vmatrix},
	\qquad
	J \defeq \abs{\vb{A}},
	\qquad
	I \defeq \tr\vb{A},
	\qquad
	K \defeq \begin{vmatrix}
		a_{11} & b_1 \\
		b_1 & c
	\end{vmatrix}
	+ \begin{vmatrix}
		a_{22} & b_2 \\
		b_2 & c
	\end{vmatrix}.
	% \rho_2 \defeq \rank\vb{A},
	% \qquad
	% \rho_3 \defeq \rank\begin{bmatrix}
	% 	\vb{A} & \vb{b} \\
	% 	\vb{b}^T & c
	% \end{bmatrix}.
\end{equation*}
那么我们可以据此对平面二次曲线进行分类,
大致如\cref{table:二次型的应用.平面二次曲线的分类} 所示.


\begin{table}[hbt]
%@see: https://mathworld.wolfram.com/QuadraticCurve.html
	\centering
	\begin{tblr}{c*4{|c}}
		\hline
		曲线 & \(\Delta\) & \(J\) & \(\Delta/I\) & \(K\) \\
		\hline
		% 两条重合直线
		\begin{tblr}{c}
			两条重合直线 \\
			conincident lines \\
		\end{tblr}
		& 0 & 0 & & 0 \\
		% 虚椭圆
		\begin{tblr}{c}
			虚椭圆 \\
			ellipse (imaginary) \\
		\end{tblr}
		& \(\neq0\) & \(>0\) & \(>0\) & \\
		% 椭圆
		\begin{tblr}{c}
			椭圆 \\
			ellipse (real) \\
		\end{tblr}
		& \(\neq0\) & \(>0\) & \(<0\) & \\
		% 双曲线
		\begin{tblr}{c}
			双曲线 \\
			hyperbola \\
		\end{tblr}
		& \(\neq0\) & \(<0\) & & \\
		% 相交虚直线
		\begin{tblr}{c}
			相交虚直线 \\
			intersecting lines (imaginary) \\
		\end{tblr}
		& 0 & \(>0\) & & \\
		% 相交直线
		\begin{tblr}{c}
			相交直线 \\
			intersecting lines (real) \\
		\end{tblr}
		& 0 & \(<0\) & & \\
		% 抛物线
		\begin{tblr}{c}
			抛物线 \\
			parabola \\
		\end{tblr}
		& \(\neq0\) & 0 & & \\
		% 平行虚直线
		\begin{tblr}{c}
			平行虚直线 \\
			parallel lines (imaginary) \\
		\end{tblr}
		& 0 & 0 & & \(>0\) \\
		% 平行直线
		\begin{tblr}{c}
			平行直线 \\
			parallel lines (real) \\
		\end{tblr}
		& 0 & 0 & & \(<0\) \\
		\hline
	\end{tblr}
	\caption{}
	\label{table:二次型的应用.平面二次曲线的分类}
\end{table}

\subsection{空间二次曲面的分类问题}
首先考虑一般的二次曲面的方程:\begin{equation}\label{equation:二次型的应用.空间二次曲面的一般方程}
	a_{11} x_1^2
	+ a_{22} x_2^2
	+ a_{33} x_3^2
	+ 2 a_{12} x_1 x_2
	+ 2 a_{13} x_1 x_3
	+ 2 a_{23} x_2 x_3
	+ 2 b_1 x_1
	+ 2 b_2 x_2
	+ 2 b_3 x_3
	+ c
	= 0.
\end{equation}
设\begin{equation*}
	\vb{X} = \begin{bmatrix}
		x_1 \\ x_2 \\ x_3
	\end{bmatrix},
	\qquad
	\vb{A} = \begin{bmatrix}
		a_{11} & a_{12} & a_{13} \\
		a_{12} & a_{22} & a_{23} \\
		a_{13} & a_{23} & a_{33}
	\end{bmatrix},
	\qquad
	\vb{b} = \begin{bmatrix}
		b_1 \\ b_2 \\ b_3
	\end{bmatrix},
\end{equation*}
于是\cref{equation:二次型的应用.空间二次曲面的一般方程} 可以写成\begin{equation*}
	\vb{X}^T \vb{A} \vb{X}
	+ \vb{b}^T \vb{X}
	+ c
	= 0,
\end{equation*}
或\begin{equation}\label{equation:二次型的应用.空间二次曲面的一般方程.矩阵形式}
	\begin{bmatrix}
		\vb{X}^T & 1
	\end{bmatrix}
	\begin{bmatrix}
		\vb{A} & \vb{b} \\
		\vb{b}^T & c
	\end{bmatrix}
	\begin{bmatrix}
		\vb{X} \\ 1
	\end{bmatrix}
	= 0.
\end{equation}

\begin{theorem}[空间二次曲面的分类定理]
%@see: 《线性代数》(张慎语、周厚隆) P141 定理9
设\(\vb{X} = \begin{bmatrix}
	x_1 \\ x_2 \\ x_3
\end{bmatrix},
\vb{A} = \begin{bmatrix}
	a_{11} & a_{12} & a_{13} \\
	a_{12} & a_{22} & a_{23} \\
	a_{13} & a_{23} & a_{33}
\end{bmatrix},
\vb{b} = \begin{bmatrix}
	b_1 \\ b_2 \\ b_3
\end{bmatrix}\),
则二次曲面\[
	\begin{bmatrix}
		\vb{X}^T & 1
	\end{bmatrix}
	\begin{bmatrix}
		\vb{A} & \vb{b} \\
		\vb{b}^T & c
	\end{bmatrix}
	\begin{bmatrix}
		\vb{X} \\ 1
	\end{bmatrix}
	= 0
\]可以经过正交变换、平移变换化为\[
	\lambda_1 z_1^2 + \lambda_2 z_2^2 + \lambda_3 z_3^2 + d = 0,
	\qquad
	\lambda_1 z_1^2 + \lambda_2 z_2^2 + p z_3 = 0,
	\quad\text{或}\quad
	\lambda_1 z_1^2 + q z_2 + r z_3 = 0.
\]
%TODO proof
%@Mathematica: X = {{x1}, {x2}, {x3}}
%@Mathematica: A = {{a11, a12, a13}, {a12, a22, a23}, {a13, a23, a33}}
%@Mathematica: B = {{b1}, {b2}, {b3}}
%@Mathematica: Xstar = Join[X, {{1}}, 1]
%@Mathematica: Astar = Join[Join[A, B, 2], Join[Transpose[B], {{c}}, 2]]
%@Mathematica: Transpose[Xstar].Astar.Xstar // Expand
\end{theorem}

记\begin{equation*}
	\rho_3 \defeq \rank\vb{A},
	\qquad
	\rho_4 \defeq \rank\begin{bmatrix}
		\vb{A} & \vb{b} \\
		\vb{b}^T & c
	\end{bmatrix}.
\end{equation*}
那么我们可以依据
秩\(\rho_3,\rho_4\)以及正惯性指数\(p\)和负惯性指数\(q\),
对空间二次曲面进行分类,
大致如\cref{table:二次型的应用.空间二次曲面的分类} 所示.

\begin{table}[htb]
%@see: https://mathworld.wolfram.com/QuadraticSurface.html
	\centering
	\begin{tblr}{*5{c|}c}
		\hline
		曲面 & 标准方程 & \(\rho_3\) & \(\rho_4\) & \(p\) & \(q\) \\
		\hline
		% 椭球面
		\begin{tblr}{c}
			椭球面 \\
			ellipsoid \\
		\end{tblr}
		& \(\frac{x^2}{a^2}+\frac{y^2}{b^2}+\frac{z^2}{c^2}=1\)
		& 3 & 4
		& 3 & 0
		\\
		% 双曲面
		\begin{tblr}{c}
			单叶双曲面 \\
			hyperboloid of one sheet \\
		\end{tblr}
		& \(\frac{x^2}{a^2}+\frac{y^2}{b^2}-\frac{z^2}{c^2}=1\)
		& 3 & 4
		& 2 & 1
		\\
		\begin{tblr}{c}
			双叶双曲面 \\
			hyperboloid of two sheets \\
		\end{tblr}
		& \(\frac{x^2}{a^2}+\frac{y^2}{b^2}-\frac{z^2}{c^2}=-1\)
		& 3 & 4
		& 1 & 2
		\\
		% 抛物面
		\begin{tblr}{c}
			椭圆抛物面 \\
			elliptic paraboloid \\
		\end{tblr}
		& \(\frac{x^2}{p}+\frac{y^2}{q}=2z\)
		& 2 & 4
		& 2 & 0
		\\
		\begin{tblr}{c}
			双曲抛物面 \\
			hyperbolic paraboloid \\
		\end{tblr}
		& \(\frac{x^2}{p}-\frac{y^2}{q}=2z\)
		& 2 & 4
		& 1 & 1
		\\
		% 二次锥面
		\begin{tblr}{c}
			椭圆锥面 \\
			elliptic cone \\
		\end{tblr}
		& \(\frac{x^2}{a^2}+\frac{y^2}{b^2}-\frac{z^2}{c^2}=0\)
		& 3 & 3
		& 2 & 1
		\\
		% 二次柱面
		\begin{tblr}{c}
			椭圆柱面 \\
			elliptic cylinder \\
		\end{tblr}
		& \(\frac{x^2}{a^2}+\frac{y^2}{b^2}=1\)
		& 2 & 4
		& 2 & 0
		\\
		\begin{tblr}{c}
			双曲柱面 \\
			hyperbolic cylinder \\
		\end{tblr}
		& \(\frac{x^2}{a^2}-\frac{y^2}{b^2}=1\)
		& 2 & 3
		& 1 & 1
		\\
		\begin{tblr}{c}
			抛物柱面 \\
			parabolic cylinder \\
		\end{tblr}
		& \(x^2=2py\)
		& 1 & 3
		& 1 & 0
		\\
		\hline
	\end{tblr}
	\caption{}
	\label{table:二次型的应用.空间二次曲面的分类}
\end{table}

\begin{example}
%@see: 《2016年全国硕士研究生入学统一考试(数学一)》一选择题/第6题
设二次型\(f(x_1,x_2,x_3) = x_1^2 + x_2^2 + x_3^2 + 4 x_1 x_2 + 4 x_1 x_3 + 4 x_2 x_3\).
试判定\(f(x_1,x_2,x_3) = 2\)在空间直角坐标下表示的二次曲面的类型.
\begin{solution}
二次型的矩阵为\[
	\A = \begin{bmatrix}
		1 & 2 & 2 \\
		2 & 1 & 2 \\
		2 & 2 & 1
	\end{bmatrix}.
\]
它的特征多项式为\begin{align*}
	\abs{\lambda\E-\A}
	&= \begin{vmatrix}
		\lambda-1 & -2 & -2 \\
		-2 & \lambda-1 & -2 \\
		-2 & -2 & \lambda-1
	\end{vmatrix}
	= \begin{vmatrix}
		\lambda-5 & -2 & -2 \\
		\lambda-5 & \lambda-1 & -2 \\
		\lambda-5 & -2 & \lambda-1
	\end{vmatrix}
	= (\lambda-5)
	\begin{bmatrix}
		1 & -2 & -2 \\
		1 & \lambda-1 & -2 \\
		1 & -2 & \lambda-1
	\end{bmatrix} \\
	&= (\lambda-5)
	\begin{vmatrix}
		1 & 0 & 0 \\
		1 & \lambda+1 & 0 \\
		1 & 0 & \lambda+1
	\end{vmatrix}
	= (\lambda-5)(\lambda+1)^2.
\end{align*}
它的特征值就是\(\lambda=5\)和\(\lambda=-1\ (\text{二重})\).
二次型的正惯性指数和负惯性指数分别是\(p=1,q=2\),
查表可知%\cref{table:二次型的应用.空间二次曲面的分类}
\(f(x_1,x_2,x_3) = 2\)在空间直角坐标下表示的二次曲面
是一个双叶双曲面.
\end{solution}
\end{example}

\begin{example}
将曲面方程\(3x^2+3y^2+5z^2+2xy-2yz-2xz+2x+10y+6z+7=0\)化简为标准型,
说明该方程描绘的是哪一类图形.
\begin{solution}
令\begin{equation*}
	\vb{A} \defeq \begin{bmatrix}
		3 & 1 & -1 \\
		1 & 3 & -1 \\
		-1 & -1 & 5
	\end{bmatrix},
	\qquad
	\vb{b} \defeq \begin{bmatrix}
		1 \\ 5 \\ 3
	\end{bmatrix},
	\qquad
	c \defeq 7,
\end{equation*}
则\begin{equation*}
	\rho_3
	= \rank\vb{A}
	= 3,
	\qquad
	\rho_4
	= \rank\begin{bmatrix}
		\vb{A} & \vb{b} \\
		\vb{b}^T & c
	\end{bmatrix}
	= 4,
\end{equation*}
而\(\vb{A}\)的正负惯性指数分别为\(3\)和\(0\),
查\cref{table:二次型的应用.空间二次曲面的分类} 可知,
该方程描绘的是椭球面.
\end{solution}
%@Mathematica: F[x_, y_, z_] := 3 x^2 + 3 y^2 + 5 z^2 + 2 x y - 2 y z - 2 x z + 2 x + 10 y + 6 z + 7
%@Mathematica: GetArgumentCount[func_Symbol] :=  									(* 获取函数的自变量个数 *)
%				Module[{downValues, args}, downValues = DownValues[func];
%				If[downValues === {}, Return[0]];
%				args = Cases[downValues[[1, 1]],
%					Verbatim[Pattern][name_, _] :> name, Infinity];
%				Length[args]];
%@Mathematica: GenerateSubscriptedSequence[start_, stop_] :=						(* 生成一串带有下标的代数符号 *)
%				Module[{n, symbols}, n = stop - start + 1;
%				symbols = Table[Subscript[x, i], {i, start, stop}];
%				Return[symbols]]
%@Mathematica: QuadraticFormMatrix[f_] :=											(* 从函数中获取二次项系数,返回对称矩阵 *)
%				Module[{vars = GenerateSubscriptedSequence[1, GetArgumentCount[f]],
%				coeffs}, coeffs = Normal@CoefficientArrays[f @@ vars, vars][[3]];
%				(coeffs + Transpose[coeffs])/2]
\end{example}

\section{本章总结}

\begin{table}[htb]
	\centering
	\begin{tblr}{*4{|c}|}
		\hline
		& 等价\(\vb{A}\cong\vb{B}\) & 相似\(\vb{A}\sim\vb{B}\) & 合同\(\vb{A}\simeq\vb{B}\) \\ \hline
		秩 & \(\rank\vb{A}=\rank\vb{B}\) & 同左 & 同左 \\ \hline
		行列式 & & \(\abs{\vb{A}}=\abs{\vb{B}}\) & \(\sgn\abs{\vb{A}}=\sgn\abs{\vb{B}}\) \\ \hline
		特征多项式 & & \(\abs{\lambda\vb{E}-\vb{A}}=\abs{\lambda\vb{E}-\vb{B}}\) \\ \hline
		迹 & & \(\tr\vb{A}=\tr\vb{B}\) \\ \hline
		\SetCell[r=4]{c} 三者的关系 & \SetCell[c=3]{c} 相似一定等价,但不一定合同. \\
				& \SetCell[c=3]{c} 合同一定等价,但不一定相似. \\
				& \SetCell[c=3]{c} 等价既不一定相似,也不一定合同. \\
				& \SetCell[c=3]{c} 相似实对称阵必定合同. \\
		\hline
	\end{tblr}
	\caption{}
\end{table}

任给一个二次型\(f(\AutoTuple{x}{n}) = \vb{x}^T\vb{A}\vb{x}\),
我们想要求出正交变换\(\vb{x}=\vb{Q}\vb{y}\),把这个二次型化为它的标准型\(g\),
可以按以下步骤操作:
\begin{enumerate}
	\item 首先写出二次型\(f\)的矩阵\(\vb{A}\).

	\item 写出特征多项式\(\abs{\lambda\vb{E}-\vb{A}}\),
	令\(\abs{\lambda\vb{E}-\vb{A}}=0\),
	求出特征值\(\AutoTuple{\lambda}{n}\).

	\item 建立线性方程\((\lambda\vb{E}-\vb{A})\vb{x}=\vb0\),求出特征向量\(\vb{x}\).

	\item 利用施密特方法,将\(\vb{A}\)的特征向量正交化.

	注意到\(\vb{A}\)作为实对称矩阵,
	它的任意一对不同特征值\(\lambda_1,\lambda_2\)分别对应的特征向量\(\vb{x}_1,\vb{x}_2\)总是正交的,
	所以我们只需要针对它的每一组对应相同特征值的特征向量进行正交化,
	就可以取得\(\vb{A}\)的正交化特征向量组.

	不过,我们也可以将第3步、第4步合并成一个步骤,
	具体来说,就是在解方程\((\lambda\vb{E}-\vb{A})\vb{x}=\vb0\)时,
	只要解得一个解向量\(\vb{x}_0\),就从这个解向量出发,
	考察它的各个坐标,
	构造出一组既可以满足方程\((\lambda\vb{E}-\vb{A})\vb{x}=\vb0\),
	也可以与\(\vb{x}_0\)正交的向量.

	\item 规范化\(\vb{A}\)的全部特征向量,
	得到\(\vb{A}\)的正交规范化特征向量组\(\{\AutoTuple{\vb{\xi}}{n}\}\).

	\item 写出正交矩阵\(\vb{Q}=(\AutoTuple{\vb{\xi}}{n})\),
	还可以写出二次型\(f\)对应的标准型\begin{equation*}
		g(\AutoTuple{y}{n}) = \lambda_1 y_1^2 + \dotsb + \lambda_n y_n^2.
	\end{equation*}
\end{enumerate}


\chapter{矩阵的分解}
%@see: https://mathworld.wolfram.com/MatrixDecomposition.html
%@see: https://reference.wolfram.com/language/guide/MatrixDecompositions.html.zh
%@see: https://blog.csdn.net/Insomnia_X/article/details/126787580
\section{CR分解}
\DefineConcept{CR分解}(column-row factorization)的目的,
是将一个矩阵分解成一个列满秩矩阵和一个行满秩矩阵的乘积.
\begin{theorem}
%@see: 《Linear Algebra Done Right (Fourth Eidition)》(Sheldon Axler) P78 3.56
设\(\vb{A} \in M_{m \times n}(K)\)且\(\rank\vb{A} = c \geq 1\),
那么存在一个矩阵\(\vb{C} \in M_{m \times c}(K)\),
存在一个矩阵\(\vb{R} \in M_{c \times n}(K)\),
使得\(\vb{A} = \vb{C} \vb{R}\).
\begin{proof}
将矩阵\(\vb{A}\)按列分块,
得\(\vb{A} = (\AutoTuple{\alpha}{n})\).
由\cref{theorem:线性空间.生成向量组的约化} 可知
\(\vb{A}\)的列向量组可以约化为它的列空间的一个基,
由列秩与列空间的维数的定义以及两者之间的关系可知,
这个基的基数必定是\(c\).
这个基(不妨记为\(\AutoTuple{\alpha}{c}\))的全部\(c\)个基向量
可以组成一个\(m \times c\)矩阵\(\vb{C} = (\AutoTuple{\alpha}{c})\).

显然\(\vb{A}\)的每一个列向量都可以由\(\vb{C}\)的列向量组线性表出,
即\begin{equation*}
	\alpha_k = x_{1k} \alpha_1 + \dotsb + x_{ck} \alpha_c,
	\quad k=1,2,\dotsc,n.
\end{equation*}
那么只要记\begin{equation*}
	\vb{R} = \begin{bmatrix}
		x_{11} & \dots & x_{1n} \\
		\vdots & & \vdots \\
		x_{c1} & \dots & x_{cn}
	\end{bmatrix},
\end{equation*}
便有\(\vb{A} = \vb{C} \vb{R}\).
\end{proof}
\end{theorem}

%@Mathematica: (* 定义CR分解函数 *)
% CRDecomposition[A_?MatrixQ] := Module[
%     {rank, pivots, Cmatrix, Rmatrix},
%     rank = MatrixRank[A];
%     pivots = {};
%     Cmatrix = {};
%     For[i = 1, i <= Last[Dimensions[A]], i++,
%         col = A[[All, i]];
%         If[MatrixRank[Transpose[Join[Cmatrix, {col}]]] > Length[Cmatrix],
%             AppendTo[pivots, i];
%             AppendTo[Cmatrix, col];
%         ]
%     ];
%     Cmatrix = Transpose[Cmatrix];
%     Rmatrix = PseudoInverse[Cmatrix] . A; (* 最小二乘法求解R *)
%     {Cmatrix, Rmatrix}
% ];
% (* 测试 *)
% A = {{2, 1, 3}, {3, 1, 4}, {5, 7, 12}};
% {C, R} = CRDecomposition[A];
% Print["C = ", C // MatrixForm];
% Print["R = ", R // MatrixForm];
% Print["验证: A - C.R = ", Chop[A - C . R] // MatrixForm];

\section{QR分解}
QR分解的目的,是将一个实满秩矩阵分解成一个正定矩阵和一个主对角元都是正数的上三角矩阵的乘积.

\section{LU分解}
LU分解的目的,是将一个矩阵分解成一个下三角矩阵和一个上三角矩阵的乘积.

\begin{theorem}
设\(\vb{A} = (a_{ij})_n \in M_n(\mathbb{R})\),
存在下三角阵\(\vb{L} = (l_{ij})_n\)和上三角阵\(\vb{U} = (u_{ij})_n\),
使得\(\vb{A} = \vb{L} \vb{U}\),
其中\(l_{ii} = 1\ (i=1,2,\dotsc,n),
l_{ij} = 0\ (i<j),
u_{ij} = 0\ (i>j)\).
\end{theorem}

举例来说,令\begin{equation*}
	\vb{A} = \begin{bmatrix}
		a_{11} & a_{12} \\
		a_{21} & a_{22}
	\end{bmatrix}
	= \begin{bmatrix}
		1 & 0 \\
		l_{21} & 1
	\end{bmatrix}
	\begin{bmatrix}
		u_{11} & u_{12} \\
		0 & u_{22}
	\end{bmatrix}
	= \vb{L} \vb{U},
\end{equation*}
得\begin{equation*}
	\left.\begin{array}{r}
		1 \cdot u_{11} + 0 \cdot 0 = a_{11} \\
		1 \cdot u_{12} + 0 \cdot u_{22} = a_{12} \\
		l_{21} u_{11} + 1 \cdot 0 = a_{21} \\
		l_{21} u_{12} + 1 \cdot u_{22} = a_{22}
	\end{array}\right\}
	\implies
	\left\{\begin{array}{l}
		u_{11} = a_{11}, \\
		u_{12} = a_{12}, \\
		l_{21} = a_{21} / u_{11}, \\
		u_{22} = a_{22} - l_{21} u_{12}.
	\end{array}\right.
\end{equation*}

又令\begin{equation*}
	\vb{A} = \begin{bmatrix}
		a_{11} & a_{12} & a_{13} \\
		a_{21} & a_{22} & a_{23} \\
		a_{31} & a_{32} & a_{33}
	\end{bmatrix}
	= \begin{bmatrix}
		1 & 0 & 0 \\
		l_{21} & 1 & 0 \\
		l_{31} & l_{32} & 1
	\end{bmatrix}
	\begin{bmatrix}
		u_{11} & u_{12} & u_{13} \\
		0 & u_{22} & u_{23} \\
		0 & 0 & u_{33}
	\end{bmatrix} = \vb{L} \vb{U},
\end{equation*}
得\begin{equation*}
	\left\{\begin{array}{l}
		u_{11} = a_{11}, \\
		u_{12} = a_{12}, \\
		u_{13} = a_{13}, \\
		l_{21} = a_{21} / u_{11}, \\
		l_{31} = a_{31} / u_{11}, \\
		u_{22} = a_{22} - l_{21} \cdot u_{12}, \\
		u_{23} = a_{23} - l_{21} \cdot u_{13}, \\
		l_{32} = (a_{32} - l_{31} \cdot u_{12}) / u_{22}, \\
		u_{33} = a_{33} - (l_{31} \cdot u_{13} + l_{32} \cdot u_{23}).
	\end{array}\right.
\end{equation*}


\section{谱分解}
\begin{lemma}\label{theorem:矩阵分解.复方阵酉相似于上三角阵}
%@see: 《矩阵论》(詹兴致) P6 定理1.5(Schur酉三角化)
设矩阵\(\vb{A} \in M_n(\mathbb{C})\),
则\(\vb{A}\)酉相似于某个上三角矩阵.
\begin{proof}
用数学归纳法.
当\(n=1\)时,结论显然成立.
假设当\(n=k-1\ (k\geq2)\)时,结论成立.
当\(n=k\)时,
任取\(\vb{A}\)的某一个特征值\(\lambda_1\),
再任取属于\(\lambda_1\)的某一个单位特征向量\(\vb{x}_1\),
将\(\vb{x}_1\)扩充成\(\mathbb{C}^n\)的一个标准正交基\(\AutoTuple{\vb{x}}{n}\),
然后令\(\vb{U}_1 = (\AutoTuple{\vb{x}}{n})\),
则\(\vb{U}_1\)是酉矩阵,
且\begin{equation*}
	\vb{U}_1^H \vb{A} \vb{U}_1
	= \begin{bmatrix}
		\lambda & \vb{y}^H \\
		\vb0 & \vb{A}_1
	\end{bmatrix},
\end{equation*}
其中\(\vb{A}_1 \in M_{n-1}(\mathbb{C})\).
由归纳假设,
存在酉矩阵\(\vb{U}_2 \in M_{n-1}(\mathbb{C})\)使得\(\vb{U}_2^H \vb{A}_1 \vb{U}_2\)是上三角矩阵.
令\(\vb{U} = \vb{U}_1 \diag(1,\vb{U}_2) \in M_n(\mathbb{C})\),
则\(\vb{U}\)是酉矩阵,
且\begin{equation*}
	\vb{U}^H \vb{A} \vb{U}
	= \begin{bmatrix}
		\lambda & \vb{y}^H \vb{U}_2 \\
		\vb0 & \vb{U}_2^H \vb{A}_1 \vb{U}_2
	\end{bmatrix}
\end{equation*}是上三角矩阵.
\end{proof}
\end{lemma}

\begin{theorem}\label{theorem:矩阵分解.谱分解}
%@see: 《矩阵论》(詹兴致) P2 定理1.1
设\(\vb{A} \in M_n(\mathbb{C})\)是正规矩阵,
则存在酉矩阵\(\vb{U} \in M_n(\mathbb{C})\)满足\begin{equation*}
	\vb{A} = \vb{U} \diag(\AutoTuple{\lambda}{n}) \vb{U}^H,
\end{equation*}
其中\(\AutoTuple{\lambda}{n}\)是\(\vb{A}\)的特征值.
\begin{proof}
根据\cref{theorem:矩阵分解.复方阵酉相似于上三角阵},
存在酉矩阵\(\vb{U}\)和上三角阵\(\vb{T}\)满足\(\vb{A} = \vb{U} \vb{T} \vb{U}^H\).
因为\(\vb{A}\)是正规矩阵,\(\vb{A} \vb{A}^H = \vb{A}^H \vb{A}\),
所以\(\vb{T} \vb{T}^H = \vb{T}^H \vb{T}\).
逐个比较\(\vb{T} \vb{T}^H\)和\(\vb{T}^H \vb{T}\)的主对角线元素,
可以看出\(\vb{T}\)的每一个非主对角线元素都是零,
\(\vb{T}\)是对角阵.
\end{proof}
\end{theorem}
\begin{remark}
\cref{theorem:矩阵分解.谱分解} 说明
每一个正规矩阵都酉相似于一个对角矩阵.
\end{remark}

\section{奇异值分解}
\begin{theorem}
设矩阵\(\vb{A} \in M_{m \times n}(\mathbb{R})\),
则存在\(m\)阶正交矩阵\(\vb{U}\)、\(n\)阶正交矩阵\(\vb{V}\)和\(m \times n\)对角阵\(\vb\Sigma\),
使得\begin{equation*}
	\vb{A} = \vb{U} \vb\Sigma \vb{V}^T,
\end{equation*}
其中\(\vb\Sigma = (\sigma_{ij})_{m \times n}\)的元素\(\sigma_{ij}\)满足\begin{equation*}
	\sigma_{ij} = \left\{ \begin{array}{cc}
	0, & i \neq j, \\
	s_i \geq 0, & i = j.
	\end{array} \right.
\end{equation*}
\rm
这里,矩阵\(\vb\Sigma\)的对角元\(s_i\)
称为\(\vb{A}\)的\DefineConcept{奇异值}(通常按\(s_i \geq s_{i+1}\)排列),
\(\vb{U}\)的列分块向量
称为\(\vb{A}\)的\DefineConcept{左奇异向量},
\(\vb{V}\)的列分块向量
称为\DefineConcept{右奇异向量}.
\begin{proof}
由于\(\vb{A}^T \vb{A} \in M_n(\mathbb{R})\),故可作谱分解,即存在正交矩阵\(\vb{V}\),使得\begin{equation*}
	\vb{V}^{-1}\vb{A}\vb{V} = \vb{V}^T\vb{A}\vb{V} = \diag(\lambda_1,\lambda_2,\dotsc,\lambda_n),
\end{equation*}
其中\(\vb{V}=(\AutoTuple{\vb{v}}{n})\)中的列分块向量\(\vb{v}_i\)是\(\vb{A}^T \vb{A}\)对应于特征值\(\lambda_i\)的特征向量,
而\(\{\AutoTuple{\vb{v}}{n}\}\)构成\(\mathbb{R}^n\)的一组标准正交基.

注意到\(\vb{A}^T \vb{A}\)是半正定矩阵\footnote{当\(\vb{A}^T \vb{A}\)是可逆矩阵时,\(\vb{A}^T \vb{A}\)是正定矩阵.},
故其特征值\(\lambda_i\geq0\).

考虑映射\(\vb{A}_{m \times n}\colon \mathbb{R}^n \to \mathbb{R}^m, \vb{x} \mapsto \vb{A}\vb{x}\),
设\(\rank\vb{A} = r\),
将\(\vb{A}\)作用到\(\mathbb{R}^n\)的标准正交基\(\{\AutoTuple{\vb{v}}{n}\}\)上,
利用维数公式,得\begin{equation*}
\dim(\ker \vb{A}) + \dim(\Img \vb{A}) = n,
\end{equation*}
可知\(\vb{A}\vb{v}_1,\vb{A}\vb{v}_2,\dotsc,\vb{A}\vb{v}_n\)这\(n\)个向量中有\(r\)个向量构成\(\mathbb{R}^m\)的一组部分基,
而\(\vb{A}\vb{v}_{r+1} = \vb{A}\vb{v}_{r+2} = \dotsb = \vb{A}\vb{v}_n = 0\).

有\(\vb{A}^T \vb{A} \vb{v}_j = \lambda_j \vb{v}_j\),又有\begin{equation*}
	\VectorInnerProductDot{\vb{v}_i}{\vb{v}_j}
	= \vb{v}_i^T \vb{v}_j
	= \left\{ \begin{array}{lc}
		1, & i=j, \\
		0, & i \neq j.
	\end{array} \right.
\end{equation*}
所以,当\(i \neq j\)时,\begin{equation*}
	\VectorInnerProductDot{\vb{A} \vb{v}_i}{\vb{A} \vb{v}_j}
	= \vb{v}_i^T \vb{A}^T \vb{A} \vb{v}_j
	= \lambda_j \vb{v}_i^T \vb{v}_j
	= 0;
\end{equation*}
而当\(i = j\)时,\begin{equation*}
	\abs{\vb{A}\vb{v}_i}^2
	= \VectorInnerProductDot{\vb{A} \vb{v}_i}{\vb{A} \vb{v}_j}
	= \lambda_i \vb{v}_i^T \vb{v}_i
	= \lambda_i.
\end{equation*}
也就是说,向量组\(\{\vb{A}\vb{v}_1,\vb{A}\vb{v}_2,\dotsc,\vb{A}\vb{v}_r\}\)是两两正交的.
单位化该向量组,又记\begin{equation*}
	\vb{u}_i = \frac{\vb{A}\vb{v}_i}{\abs{\vb{A}\vb{v}_i}}
	= \frac{\vb{A}\vb{v}_i}{\sqrt{\lambda_i}}
	\vb{v}uad(i=1,2,\dotsc,r),
\end{equation*}
于是\(\vb{A}\vb{v}_i = s_i \vb{u}_i\),其中\(s_i = \sqrt{\lambda_i}\).

将\(\vb{u}_1,\vb{u}_2,\dotsc,\vb{u}_r\)扩充成\(\mathbb{R}^m\)的标准正交基
\(\{\vb{u}_1,\vb{u}_2,\dotsc,\vb{u}_r,\vb{u}_{r+1},\dotsc,\vb{u}_m\}\),
在这组基下,有\begin{equation*}
	\vb{A}\vb{V} = \vb{A}(\AutoTuple{\vb{v}}{n}) = \begin{bmatrix}
		s_1 \vb{u}_1 \\
		& \ddots \\
		& & s_r \vb{u}_r \\
		& & & 0 \\
		& & & & \ddots \\
		& & & & & 0
	\end{bmatrix}
	= \vb{U} \vb\Sigma,
\end{equation*}
其中\(\vb{U} = (\vb{u}_1,\vb{u}_2,\dotsc,\vb{u}_m)\),
\(\vb\Sigma = \diag(s_1,\dotsc,s_r,0,\dotsc,0)\),
将上式两边右乘\(\vb{V}^{-1}\),即得\(\vb{A} = \vb{U}\vb\Sigma\vb{V}^T\).
\end{proof}
\end{theorem}

\begin{example}
对矩阵\(\vb{A} = \begin{bmatrix} 0 & 1 \\ 1 & 1 \\ 1 & 0 \end{bmatrix}\)进行奇异值分解.
\begin{solution}
经计算\begin{equation*}
	\vb{A}^T \vb{A} = \begin{bmatrix} 2 & 1 \\ 1 & 2 \end{bmatrix},
\end{equation*}
其特征值是\(\lambda_1 = 3\)和\(\lambda_2 = 1\).
\(\vb{A}^T \vb{A}\)属于特征值\(\lambda_1\)的特征向量为
\(\vb{v}_1 = \begin{bmatrix} 1/\sqrt{2} \\ 1/\sqrt{2} \end{bmatrix}\);
\(\vb{A}^T \vb{A}\)属于特征值\(\lambda_2\)的特征向量为
\(\vb{v}_2 = \begin{bmatrix} -1/\sqrt{2} \\ 1/\sqrt{2} \end{bmatrix}\).

同时有\begin{equation*}
	\vb{A} \vb{A}^T = \begin{bmatrix} 1 & 1 & 0 \\ 1 & 2 & 1 \\ 0 & 1 & 1 \end{bmatrix},
\end{equation*}
其特征值是\(\mu_1 = 3\)、\(\mu_2 = 1\)、\(\mu_3 = 0\).
\(\vb{A} \vb{A}^T\)属于特征值\(\mu_1\)的特征向量为
\(\vb{u}_1 = \begin{bmatrix} 1/\sqrt{6} \\ 2/\sqrt{6} \\ 1/\sqrt{6} \end{bmatrix}\);
\(\vb{A} \vb{A}^T\)属于特征值\(\mu_2\)的特征向量为
\(\vb{u}_2 = \begin{bmatrix} 1/\sqrt{2} \\ 0 \\ -1/\sqrt{2} \end{bmatrix}\);
\(\vb{A} \vb{A}^T\)属于特征值\(\mu_3\)的特征向量为
\(\vb{u}_3 = \begin{bmatrix} 1/\sqrt{3} \\ -1/\sqrt{3} \\ 1/\sqrt{3} \end{bmatrix}\).

再根据\(s_i = \sqrt{\lambda_i}\)求得奇异值\(s_1 = \sqrt{3}\)和\(s_2 = 1\).

于是\begin{gather*}
	\vb{U} = (\vb{u}_1,\vb{u}_2,\vb{u}_3)
	= \begin{bmatrix}
		1/\sqrt{6} & 1/\sqrt{2} & 1/\sqrt{3} \\
		2/\sqrt{6} & 0 & -1/\sqrt{3} \\
		1/\sqrt{6} & -1/\sqrt{2} & 1/\sqrt{3}
	\end{bmatrix}, \\
	\vb{V} = (\vb{v}_1,\vb{v}_2,\vb{v}_3)
	= \begin{bmatrix}
		1/\sqrt{2} & -1/\sqrt{2} \\
		1/\sqrt{2} & 1/\sqrt{2}
	\end{bmatrix}, \\
	\vb\Sigma = \begin{bmatrix}
		\sqrt{3} & 0 \\
		0 & 1 \\
		0 & 0
	\end{bmatrix}.
\end{gather*}
\end{solution}
\end{example}

\begin{theorem}
%@see: 《矩阵论》(詹兴致) P7 定理1.6(奇异值分解)
设矩阵\(\vb{A} \in M_{m \times n}(\mathbb{C})\),
则存在酉矩阵\(\vb{U} \in M_m(\mathbb{C})\)、酉矩阵\(\vb{V} \in M_n(\mathbb{C})\)使得\begin{equation*}
	\vb{U} \vb{A} \vb{V} = \vb\Sigma,
\end{equation*}
其中\(\vb\Sigma = \diag_{m \times n}(\AutoTuple{s}{p}),
s_1 \geq \dotsb \geq s_p \geq 0,
p = \min\{m,n\}\).
%TODO proof
\end{theorem}

\section{极分解}
\begin{theorem}
任意实方阵\(\vb{A}\)可表为\begin{equation*}
	\vb{A} = \vb{S}\vb\Omega = \vb\Omega_1 \vb{S}_1,
\end{equation*}
其中\(\vb{S}\)和\(\vb{S}_1\)为半正定实对称方阵,
\(\vb\Omega\)与\(\vb\Omega_1\)为实正交方阵,
而且\(\vb{S}\)和\(\vb{S}_1\)都是唯一的.
\begin{proof}
当\(\vb{A}\)可逆时,
\(\vb{A}^T \vb{A}\)是正定阵,
存在正定阵\(\vb{S}_1\),
使得\(\vb{A}^T \vb{A} = \vb{S}_1^2\),
于是\(\vb{A} = \vb{A} \vb{S}_1^{-1} \vb{S}_1\),
注意到\((\vb{A} \vb{S}_1^{-1})^T (\vb{A} \vb{S}_1^{-1})
= (\vb{S}_1^{-1})^T \vb{A}^T \vb{A} \vb{S}_1^{-1} = \vb{E}\),
即\(\vb{A} \vb{S}_1^{-1}\)正交,
那么只需要令\(\vb\Omega_1 = \vb{A} \vb{S}_1^{-1}\)
即有\(\vb{A} = \vb\Omega_1 \vb{S}_1\).

当\(\vb{A}\)不可逆时,可以运用正交相似标准型;
也可以运用扰动法,
即令\(\vb{S}_1(t) = \vb{S}_1 + t\vb{E}\),
则当\(t\)充分大时,
\(\vb{S}_1(t)\)可逆.
\end{proof}
\end{theorem}

%@see: https://mathworld.wolfram.com/QRDecomposition.html
%@see: https://o-o-sudo.github.io/numerical-methods/qr-.html
%@see: https://reference.wolfram.com/language/guide/MatrixDecompositions.html.zh
%@see: https://blog.csdn.net/Insomnia_X/article/details/126787580
%@see: https://mathworld.wolfram.com/SchurDecomposition.html
%@see: https://mathworld.wolfram.com/JordanMatrixDecomposition.html

