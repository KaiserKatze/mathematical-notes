\chapter*{写在前面的话}
我在记笔记的时候参考了
% 集合论与数理逻辑
恩德尔顿 编写的《Elements of Set Theory》(ISBN 0-12-238440-7)、
《A Mathematical Introduction to Logic (2nd Edition)》(ISBN 0-12-238452-0),
Gaisi Takeuti、Wilson M. Zaring 编写的
《Introduction to Axiomatic Set Theory (Second Edition)》(ISBN 978-1-4613-8170-9),
% 数论
潘承洞 编写的《初等数论》(ISBN 978-7-301-21612-5),
% 离散数学
Ronald L. Graham、Donald E. Knuth、Oren Patashnik 编写的,
张明尧、张凡 翻译的
《具体数学\ 计算机科学基础(第2版)》(ISBN 978-7-115-30810-8),
% 组合数学
曾光、魏福山、杨本朝、王洪、马智 编写的《组合数学及其应用》(ISBN 978-7-03-075081-5),
% 欧氏几何
希尔伯特 编写的《几何基础》,
% 代数学
张慎语、周厚隆 编写的《线性代数》(ISBN 978-7-04-010545-2),
丘维声 编写的
《解析几何(第三版)》(ISBN 978-7-301-25921-4)、
《高等代数(第三版)》(ISBN 978-7-04-041880-4 和 ISBN 978-7-04-042235-1)、
《高等代数学习指导书(第二版)》(ISBN 978-7-302-48367-0 和 ISBN 978-7-302-44604-0)、
《近世代数》(ISBN 978-7-301-25580-3),
盛为民、李方等人 编写的
《高等代数与解析几何(上册)》(ISBN 978-7-03-079037-8)、
《高等代数与解析几何(下册)》(ISBN 978-7-03-079239-6),
谢启鸿、姚慕生 编写的
《高等代数(第四版)》(ISBN 978-7-309-16352-0)、
《高等代数学习指导书(第三版)》(ISBN 978-7-309-11776-9),
Peter D. Lax 编写的《Linear Algebra and Its Applications (Second Edition)》(ISBN 978-0-471-75156-4),
% 数学分析
同济大学数学系 编写的《高等数学(第六版)》(ISBN 978-7-04-020549-7 和 ISBN 978-7-04-021277-8),
李逸 编写的《基本分析讲义》,
辛钦 编写的《数学分析八讲》(ISBN 978-7-115-39747-8),
卓里奇 编写的《数学分析(第7版)》(ISBN 978-7-04-028755-4 和 ISBN 978-7-04-028756-1),
常庚哲、史济怀 编写的《数学分析教程(第3版)》(ISBN 978-7-312-03009-3 和 ISBN 978-7-312-03131-1),
谢惠民、恽自求、易法槐、钱定边 编写的
《数学分析习题课讲义(第2版 上册)》(ISBN 978-7-04-049851-6)、
《数学分析习题课讲义(第2版 下册)》(ISBN 978-7-04-051152-9),
陈纪修 编写的《数学分析(第二版)》(ISBN 978-7-040-13852-8 和 ISBN 978-7-040-15549-5),
邓东皋、尹小玲 编写的《数学分析简明教程(第二版)》(ISBN 7-04-018662-4 和 ISBN 7-04-019954-8),
胡适耕、张显文 编写的《数学分析:原理与方法》(ISBN 978-7-03-021797-4),
裴礼文 编写的《数学分析中的典型问题与方法(第3版)》(ISBN 978-7-04-051151-2),
潘承洞、于秀源 编写的《阶的估计基础》(ISBN 978-7-04-041350-2),
% 点集拓扑学
熊金城 编写的《点集拓扑讲义(第四版)》(ISBN 978-7-04-032237-8),
% 复分析
黄大奎、舒慕曾 编写的《数学物理方法》(ISBN 978-7-04-010326-7),
龚昇 编写的《简明复分析(第2版)》(ISBN 978-7-312-02169-5),
% 测度论
严加安 编写的《测度论讲义(第三版)》(ISBN 978-7-03-067803-4),
% 实分析
周民强 编写的
《数学分析(第1册)》(ISBN 978-7-03-042481-5)、
《数学分析(第2册)》(ISBN 978-7-03-042502-7)、
《数学分析(第3册)》(ISBN 978-7-03-042500-3)、
《实变函数论(第三版)》(ISBN 978-7-301-27647-1),
富兰德 编写的《Real Analysis Modern Techniques and Their Applications Second Edition》(ISBN 0-471-31716-0),
% 概率论与数理统计
陈鸿建、赵永红、翁洋 编写的《概率论与数理统计》(ISBN 978-7-04-024894-4),
施利亚耶夫 编写的《概率论(第3版)》(ISBN 978-7-04-022059-9 和 ISBN 978-7-04-022555-6),
茆诗松 编写的《概率论与数理统计(第二版)》、《概率论与数理统计(第四版)》(ISBN 978-7-5037-9345-5),
薛留根 编写的《概率论解题方法与技巧》(ISBN 7-118-01414-1),
% 数值分析
李庆扬、王能超、易大义 编写的《数值分析(第5版)》(ISBN 978-7-302-18565-9).

在此对先生们表达由衷的感激!

\cleardoublepage
\chapter*{序}
%@see: https://www.gushiwen.cn/shiwenv_c743b1310a1c.aspx
君子曰:学不可以已.

青,取之于蓝,而青于蓝;冰,水为之,而寒于水.
木直中绳,輮以为轮,其曲中规.
虽有槁暴,不复挺者,輮使之然也.
故木受绳则直,金就砺则利,君子博学而日参省乎己,则知明而行无过矣.

吾尝终日而思矣,不如须臾之所学也;吾尝跂而望矣,不如登高之博见也.
登高而招,臂非加长也,而见者远;顺风而呼,声非加疾也,而闻者彰.
假舆马者,非利足也,而致千里;假舟楫者,非能水也,而绝江河.
君子生非异也,善假于物也.

积土成山,风雨兴焉;积水成渊,蛟龙生焉;积善成德,而神明自得,圣心备焉.
故不积跬步,无以至千里;不积小流,无以成江海.
骐骥一跃,不能十步;驽马十驾,功在不舍.
锲而舍之,朽木不折;锲而不舍,金石可镂.
蚓无爪牙之利筋骨之强,上食埃土,下饮黄泉,用心一也.
蟹六跪而二螯,非蛇鳝之穴无可寄托者,用心躁也.
