\chapter*{序}
%@see: https://www.gushiwen.cn/shiwenv_c743b1310a1c.aspx
君子曰:学不可以已.

青,取之于蓝,而青于蓝;冰,水为之,而寒于水.
木直中绳,輮以为轮,其曲中规.
虽有槁暴,不复挺者,輮使之然也.
故木受绳则直,金就砺则利,君子博学而日参省乎己,则知明而行无过矣.

吾尝终日而思矣,不如须臾之所学也;吾尝跂而望矣,不如登高之博见也.
登高而招,臂非加长也,而见者远;顺风而呼,声非加疾也,而闻者彰.
假舆马者,非利足也,而致千里;假舟楫者,非能水也,而绝江河.
君子生非异也,善假于物也.

积土成山,风雨兴焉;积水成渊,蛟龙生焉;积善成德,而神明自得,圣心备焉.
故不积跬步,无以至千里;不积小流,无以成江海.
骐骥一跃,不能十步;驽马十驾,功在不舍.
锲而舍之,朽木不折;锲而不舍,金石可镂.
蚓无爪牙之利筋骨之强,上食埃土,下饮黄泉,用心一也.
蟹六跪而二螯,非蛇鳝之穴无可寄托者,用心躁也.

\hfill
--- 《劝学》(荀子)

路曼曼其修远兮,吾将上下而求索.

\hfill
--- 《离骚》(屈原)

故天将降大任于斯人也,
必先苦其心志,劳其筋骨,饿其体肤,空乏其身,行拂乱其所为,
所以动心忍性,曾益其所不能.

\hfill
--- 《生于忧患死于安乐》(孟子)

余幼时即嗜学.
家贫,无从致书以观.
每假借于藏书之家,手自笔录,计日以还.
天大寒,砚冰坚,手指不可屈伸,弗之怠.
录毕,走送之,不敢稍逾约.
以是人多以书假余,余因得遍观群书.

\hfill
--- 《送东阳马生序》(宋濂)

安得广厦千万间,大庇天下寒士俱欢颜,风雨不动安如山.

\hfill
--- 《茅屋为秋风所破歌》(杜甫)

\hspace{1pt}
