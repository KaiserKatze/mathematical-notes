\part{数值分析}
\chapter{数值分析引论}
\section{数值分析的对象、作用与特点}
数值分析研究用计算机求解各种数学问题的数值计算方法及其理论与软件实现,
用计算机求解科学技术问题通常经历以下步骤:
\begin{enumerate}
	\item 根据实际问题建立数学模型.
	\item 由数学模型给出数值计算方法.
	\item 根据计算方法编制算法程序,在计算机上算出结果.
\end{enumerate}

能用计算机计算的数值问题,
是指输入数据(即问题中的自变量与原始数据)
与输出数据(结果)之间函数关系的一个确定而无歧义的描述,
输入输出数据可用有限维向量表示.

有的问题,例如求解线性方程组,属于数值问题.
而有的问题,例如给定常微分方程,求未知函数的解析表达式,就不属于数值问题.

数值计算的基本单位称为算法元,
它由算子、输入元和输出元组成.
算子可以是简单操作,
例如算术运算、逻辑运算;
也可以是宏操作,
例如向量运算、数组传输、基本初等函数求值等.
输入元和输出元通常视作向量.
有限个算法元的序列称为一个进程.
一个数值问题的算法是指按规定顺序执行一个或多个完整的进程,
通过它们将输入元变换成输出元.
按同时运行的进程个数,
可以将算法分为串行算法和并行算法两类;
其中,同一时间只有一个进程在运行的算法,称为串行算法;
反之,同一时间有若干个进程在运行的算法,称为并行算法.

一个给定的数值问题,
可以有许多不同的算法.
虽然它们都能给出近似答案,
但是所需的计算量和得到的精确程度可能相差很大.
一个面向计算机、有可靠理论分析、计算复杂性好的算法,就是一个好算法.
理论分析主要是连续系统的离散化及离散型方程的数值问题求解,
它包括误差分析、稳定性、收敛性等基本概念,
它刻画了算法的可靠性、准确性.
计算复杂性包含计算时间复杂性与存储空间复杂性两个方面.
在同一规模、同一精度条件下,计算时间少的算法计算时间复杂性好,
而占用内存空间少的算法存储空间复杂性好.
计算复杂性实际上就是算法中计算量与存储量的分析.
同一问题的不同算法的计算复杂性可能差别很大.
例如在解\(n\)阶线性方程组时,
若依照克拉默法则用行列式解法则需要算\(n+1\)个\(n\)阶行列式的值,
具体到\(n=20\)的情况下就需要\num{9.7e21}次乘除法运算,
但是若用高斯列主元消去法,则只需要做3060次乘除运算,
且\(n\)越大两种算法所需运算次数差距就越大.
这表明算法研究的重要性,
也说明只提高计算机速度而不改进和选用好的算法是不行的.

综上所述,数值分析是研究数值问题的算法,概括起来有四点:\begin{enumerate}
	\item 面向计算机,要根据计算机的特点,提供切实可行的有效算法(即计算机能直接处理的逻辑运算和算术运算);
	\item 建立在相应的数学理论基础上,有可靠的理论分析,能任意逼近并达到精度要求,对近似算法要保证收敛性和数值稳定性,还要对误差进行分析;
	\item 要有好的计算复杂性,节省计算时间和存储空间;
	\item 要有数值实验,通过实验证明算法是行之有效的.
\end{enumerate}

\section{数值计算的误差}
\subsection{误差来源与分类}
用计算机解决科学计算问题首先要建立数学模型,
它是对被描述的实际问题进行抽象、简化得到的,因而是近似的.
我们把数学模型与实际问题之间出现的这种误差,
称为\DefineConcept{模型误差}.

在数学模型中往往还有一些根据观测得到的物理量,
例如温度、长度、电压等,
这些参量显然也包含误差.
这种由观测产生的误差称为\DefineConcept{观测误差}.

当数学模型不能得到精确解时,
通常需要用数值方法求它的近似解,
其近似解与精确解之间的误差称为\DefineConcept{截断误差}或\DefineConcept{方法误差}.

有了求解数学问题的计算公式以后,
就可以用计算机做数值计算了.
但是由于计算机的字长有限,
在计算机上表示原始数据、中间计算数据和输出数据,
以及将二进制数据转化为十进制数据,或将十进制数据转化为二进制数据时,
都会产生\DefineConcept{舍入误差}.

研究计算结果的误差是否满足精度要求就是误差估计问题.
数值分析主要讨论算法的截断误差、舍入误差.

\subsection{误差与有效数字}
%@see: 《数值分析(第5版)》(李庆扬、王能超、易大义) P4 定义1
假设\(x\)是准确值,\(x^*\)是\(x\)的一个近似值,
则称\begin{equation}
	e^* \defeq x^* - x
\end{equation}
是近似值的\DefineConcept{绝对误差}.

通常我们不能算出准确值\(x\),
也不能算出误差\(e^*\)的准确值,
只能根据测量工具或计算情况估计出误差的绝对值不超过某个整数\(\epsilon^*\),
也就是误差绝对值的一个上界.
我们把\(\epsilon^*\)称为近似值的\DefineConcept{绝对误差限},
它总是正数.
例如,用毫米刻度的米尺测量一根棍子的长度,假设棍子长度的准确值是\(x\),
测量时读出的与棍子长度接近的刻度是\(x^*\),
\(x^*\)是\(x\)的近似值,
它的绝对误差限是\(\qty{0.5}{\milli\meter}\),
可知\(\abs{x^* - x} \leq \qty{0.5}{\milli\meter}\).
如果读出的长度是\(\qty{765}{\milli\meter}\),
则有\(\abs{765-x} \leq 0.5\),
从这个不等式我们仍不知道准确的\(x\)是多少,
但是可以得出结论\(764.5 \leq x \leq 765.5\).
对于一般情形,我们常常把不等式\begin{equation}
	\abs{x^* - x} \leq \epsilon^*
\end{equation}
写成\begin{equation}
	x = x^* \pm \epsilon^*.
\end{equation}

绝对误差限的大小还不能完全表示近似值的好坏.
例如有两个量\(x = 10\pm1\)和\(y = 1000\pm5\),
虽然后者的绝对误差限是前者的5倍,
但是\(5/1000 = 0.5\%\)比\(1/10 = 10\%\)要小得多,
这说明\(y^*\)近似\(y\)的程度
比\(x^*\)近似\(x\)的程度要好得多.
也就是说,除了考虑误差的大小以外,
还应该考虑准确值\(x\)本身的大小.

我们把准确值的误差\(e^*\)与准确值\(x\)的比值\begin{equation}
	e^*_r
	\defeq \frac{e^*}{x}
	= \frac{x^* - x}{x}
\end{equation}
称为近似值\(x^*\)的\DefineConcept{相对误差}.

在实际计算中,由于真值\(x\)总是不知道的,
当\(e^* / x^*\)较小时,
通常取\begin{equation}
	e^*_r
	\defeq \frac{e^*}{x^*}
	= \frac{x^* - x}{x^*}
\end{equation}
作为\(x^*\)的相对误差,
此时\begin{equation*}
	\frac{e^*}{x}
	- \frac{e^*}{x^*}
	= \frac{e^* (x^* - x)}{x^* x}
	= \frac{(e^*)^2}{x^* (x^* - e^*)}
	= \frac{(e^* / x^*)^2}{1 - (e^* / x^*)}
\end{equation*}
是\(e^*_r\)的平方项级,故可忽略不计.

相对误差\((e^* / x^*)\)跟绝对误差\(e^*\)一样,也可正可负.
我们把相对误差的绝对值上界称为\DefineConcept{相对误差限},
记作\(\epsilon^*_r\),
即\begin{equation}
	\epsilon^*_r
	\defeq \frac{\epsilon^*}{\abs{x^*}}.
\end{equation}

根据定义,\(10\pm1\)和\(1000\pm5\)的相对误差限分别是\(10\%\)和\(0.5\%\),
由此可见后者的近似程度比前者的近视程度好.

需要注意的是,
绝对误差、绝对误差限是有量纲的,
相对误差、相对误差限是无量纲的.

在\(p\)进制下,
当准确值\(x\)有多位数时,
常常按四舍五入的原则得到\(x\)的前几位近似值\(x^*\).
如果近似值\(x^*\)的绝对误差限是某一位的半个单位,
且该位到\(x^*\)的第一位非零数字共有\(n\)位,
则称\(x^*\)有\(n\)位\DefineConcept{有效数字}.
我们常常把\(x^*\)表示为\begin{equation*}
	x^*
	= \pm p^m \times (
		a_1
		+ a_2 \times p^{-1}
		+ \dotsb
		+ a_n \times p^{-(n-1)}
	),
\end{equation*}
其中\(a_i\ (i=1,2,\dotsc,n)\)
是\(0\)到\(p-1\)中的一个数字,
\(a_1\neq0\),
\(m\)是整数,
且\begin{equation*}
	\abs{x - x^*}
	\leq \frac12 \times p^{m-n+1}.
\end{equation*}

\begin{example}
%@see: 《数值分析(第5版)》(李庆扬、王能超、易大义) P6 例1
按四舍五入原则写出下列十进制数字的具有5位有效数字的近似数:\begin{equation*}
	187.932~5, \qquad
	0.037~855~51, \qquad
	8.000~033, \qquad
	2.718~281~8.
\end{equation*}
\begin{solution}
根据定义,上述十进制数字的具有5位有效数字的近似数分别是:\begin{equation*}
	187.93, \qquad
	0.037~856, \qquad
	8.000~0, \qquad
	2.718~3.
\end{equation*}
\end{solution}
\end{example}

关于有效数字与相对误差限的关系,有下面的定理.
\begin{theorem}\label{theorem:数值分析.数值计算误差.有效数字与相对误差限的关系}
%@see: 《数值分析(第5版)》(李庆扬、王能超、易大义) P6 定理1
给定近似数\(x^*\),用\(p\)进制下的科学计数法可以将\(x^*\)表示为\begin{equation*}
	x^*
	= \pm p^m \times (
		a_1
		+ a_2 \times p^{-1}
		+ \dotsb
		+ a_n \times p^{-(n-1)}
	),
\end{equation*}
其中\(a_i\ (i=1,2,\dotsc,n)\)
是\(0\)到\(p-1\)中的一个数字,
\(a_1\neq0\),
\(m\)是整数.
\begin{itemize}
	\item 如果\(x^*\)具有\(n\)位有效数字,
	则\(x^*\)的相对误差限为\begin{equation}
		\epsilon^*_r
		\leq \frac1{2 a_1} \times p^{-(n-1)}.
	\end{equation}

	\item 如果\(x^*\)的相对误差限为\begin{equation*}
		\epsilon^*_r
		\leq \frac1{2(a_1+1)} \times p^{-(n-1)},
	\end{equation*}
	则\(x^*\)至少具有\(n\)位有效数字.
\end{itemize}
\end{theorem}
\cref{theorem:数值分析.数值计算误差.有效数字与相对误差限的关系} 说明,
一个近似数的有效位数越多,它的相对误差限就越小.

\begin{example}
%@see: 《数值分析(第5版)》(李庆扬、王能超、易大义) P7 例3
在十进制下,要使\(\sqrt{20}\)的近似值的相对误差限小于\(0.1\%\),应该取几位有效数字?
\begin{solution}
假设应取\(n\)位有效数字,
那么由\cref{theorem:数值分析.数值计算误差.有效数字与相对误差限的关系} 可知\begin{equation*}
	\epsilon^*_r \leq \frac1{2a_1} \times 10^{-(n-1)}.
\end{equation*}
由于\(4 < \sqrt{20} < 5\),
所以\(a_1 = 4\),
故只要取\(n = 4\),就有\begin{equation*}
	\epsilon^*_r
	\leq 0.125 \times 10^{-3}
	< 10^{-3}
	= 0.1\%,
\end{equation*}
因此,只要对\(\sqrt{20}\)的近似值取4位有效数字,
即\(\sqrt{20} \approx 4.472\),
其相对误差限就小于\(0.1\%\).
\end{solution}
\end{example}

\subsection{数值运算的误差估计}
设两个近似数\(x^*_1\)与\(x^*_2\)的误差限分别是\(\epsilon(x^*_1)\)与\(\epsilon(x^*_2)\),
则它们进行加减乘除运算,得到的误差限分别满足不等式\begin{gather*}
	\epsilon(x^*_1 \pm x^*_2)
	\leq \epsilon(x^*_1) + \epsilon(x^*_2), \\
	\epsilon(x^*_1 x^*_2)
	\leq \abs{x^*_1} \epsilon(x^*_2) + \abs{x^*_2} \epsilon(x^*_1), \\
	\epsilon(x^*_1 / x^*_2)
	\leq \frac{\abs{x^*_1} \epsilon(x^*_2) + \abs{x^*_2} \epsilon(x^*_1)}{\abs{x^*_2}^2}
	\quad(x^*_2 \neq 0).
\end{gather*}

更一般的情况是,当自变量有误差时,计算函数值也产生误差,
其误差值可以利用函数的泰勒展开式进行估计.

设\(f\)是一元可微函数,
\(x^*\)是\(x\)的近似值,
由\begin{equation*}
	% \cref{equation:微分中值定理.泰勒公式.多项式1}
	% \cref{equation:微分中值定理.泰勒公式.余项1}
	f(x) = f(x^*) + f'(x^*) (x-x^*)
	+ \dotsb
	+ \frac{f^{(n)}(x^*)}{n!} (x-x^*)^n
	+ \frac{f^{(n+1)}(\xi)}{(n+1)!} (x-x^*)^{n+1}
	\footnote{
		这里不写成\(f(x^*)\)的泰勒公式,是因为那样做会导致泰勒多项式中出现\(f'(x),f''(x)\)等含有未知真值的表达式.
		我们现在采取的写法,则\(f'(x^*),f''(x^*)\)等都是已知的,剩下的\(x-x^*\)可以通过放缩用绝对误差限代替.
	}
\end{equation*}
可知\begin{equation*}
	% \(f(x^*) - f(x)\)才是用\(f(x^*)\)近似\(f(x)\)产生的绝对误差,这里是绝对误差的相反数
	f(x) - f(x^*)
	= f'(x^*) (x-x^*)
	+ \dotsb
	+ \frac{f^{(n)}(x^*)}{n!} (x-x^*)^n
	+ \frac{f^{(n+1)}(\xi)}{(n+1)!} (x-x^*)^{n+1},
\end{equation*}
其中\(\xi\)在\(x\)与\(x^*\)之间.
对上式取绝对值,
考虑到\(\abs{x^* - x} \leq \epsilon^* \defeq \epsilon(x^*)\),
得\begin{align*}
	\abs{f(x^*) - f(x)}
	&= \abs{f'(x^*)} \abs{x-x^*}
	+ \dotsb
	+ \abs{\frac{f^{(n)}(x^*)}{n!}} \abs{x-x^*}^n
	+ \abs{\frac{f^{(n+1)}(\xi)}{(n+1)!}} \abs{x-x^*}^{n+1} \\
	&\leq \abs{f'(x^*)} \epsilon(x^*)
	+ \dotsb
	+ \abs{\frac{f^{(n)}(x^*)}{n!}} \epsilon^n(x^*)
	+ \abs{\frac{f^{(n+1)}(\xi)}{(n+1)!}} \epsilon^{n+1}(x^*).
\end{align*}
我们通常只需要按\((x-x^*)\)的幂展开的1阶泰勒公式,即\begin{equation*}
	\abs{f(x^*) - f(x)}
	\leq \abs{f'(x^*)} \epsilon(x^*)
	+ \frac{\abs{f''(\xi)}}{2!} \epsilon^2(x^*).
\end{equation*}
当\(f'(x^*)\)与\(f''(x^*)\)的比值不太大时,
我们还可以忽略\(\epsilon(x^*)\)的高次项,
于是得到计算函数的绝对误差限为\begin{equation}
	\epsilon(f(x^*))
	\approx \abs{f'(x^*)} \epsilon(x^*).
\end{equation}

设\(f\)是多元可微函数,
\(x^*_1,\dotsc,x^*_m\)分别是\(x_1,\dotsc,x_m\)的近似值,
那么由\hyperref[theorem:多元函数微分法.多元函数的泰勒公式]{多元函数的泰勒公式}可知,
用\(A^* \defeq f(x^*_1,\dotsc,x^*_m)\)近似\(A \defeq f(x_1,\dotsc,x_m)\)产生的绝对误差为\begin{align*}
	A^* - A
	&= f(x^*_1,\dotsc,x^*_m) - f(x_1,\dotsc,x_m) \\
	&\approx \sum_{k=1}^m \left( \pdv{f(x^*_1,\dotsc,x^*_m)}{x_k} \right) (x^*_k - x_k)
	= \sum_{k=1}^m \left( \pdv{f}{x_k} \right)^* e^*_k,
\end{align*}
于是绝对误差限为\begin{equation}\label{equation:数值计算的误差.数值运算的绝对误差限}
%@see: 《数值分析(第5版)》(李庆扬、王能超、易大义) P8 (2.3)
	\epsilon(A^*)
	\approx \sum_{k=1}^m \abs{\left( \pdv{f}{x_k} \right)^*} \epsilon(x^*_k),
\end{equation}
而\(A^*\)的相对误差限为\begin{equation}\label{equation:数值计算的误差.数值运算的相对误差限}
%@see: 《数值分析(第5版)》(李庆扬、王能超、易大义) P8 (2.4)
	\epsilon^*_r
	= \epsilon_r(A^*)
	= \frac{\epsilon(A^*)}{\abs{A^*}}
	\approx \sum_{k=1}^m \abs{\left( \pdv{f}{x_k} \right)^*} \frac{\epsilon(x^*_k)}{\abs{A^*}}.
\end{equation}

\begin{example}
%@see: 《数值分析(第5版)》(李庆扬、王能超、易大义) P8 例4
已经测得某场地长为\(l^* = \qty{110}{\meter}\),宽为\(d^* = \qty{80}{\meter}\),
且\(
	\abs{l-l^*} \leq \qty{0.2}{\meter},
	\abs{d-d^*} \leq \qty{0.1}{\meter}
\).
试求面积\(s = l d\)的绝对误差限和相对误差限.
\begin{solution}
由\(s = l d\)可知\(\pdv{s}{l} = d, \pdv{s}{d} = l\).
由\cref{equation:数值计算的误差.数值运算的绝对误差限} 可知\begin{equation*}
	\epsilon(s^*)
	\approx \abs{\left( \pdv{s}{l} \right)^*} \epsilon(l^*)
	+ \abs{\left( \pdv{s}{d} \right)^*} \epsilon(d^*),
\end{equation*}
其中\begin{equation*}
	\left( \pdv{s}{l} \right)^*
	= d^*
	= \qty{80}{\meter},
	\qquad
	\left( \pdv{s}{d} \right)^*
	= l^*
	= \qty{110}{\meter},
	\qquad
	\epsilon(l^*) = \qty{0.2}{\meter},
	\qquad
	\epsilon(d^*) = \qty{0.1}{\meter},
\end{equation*}
于是绝对误差限为\begin{equation*}
	\epsilon(s^*)
	\approx 80 \times (0.2) + 110 \times (0.1)
	= \qty{27}{\meter\squared},
\end{equation*}
相对误差限为\begin{equation*}
	\epsilon_r(s^*)
	= \frac{\epsilon(s^*)}{\abs{s^*}}
	= \frac{\epsilon(s^*)}{l^* d^*}
	\approx \frac{27}{8800}
	= 0.31\%.
\end{equation*}
\end{solution}
\end{example}

\section{误差定性分析与避免误差危害}
数值运算中的误差分析是个很重要又很复杂的问题.
上一节我们虽然讨论不精确数据运算结果的误差限,
但是它只适用于简单情形,
然而一个工程或科学计算问题往往需要运算成千上万次,
由于每一步运算都有误差,
所以每一步都做误差分析是不可能的,也是不科学的,
这是因为误差积累有正有负,
绝对值有大有小,
都按最坏情况估计误差限,
得到的结果就会比实际误差大很多,
这种保守的误差估计不能反映实际误差积累.
考虑到误差分布的随机性,
有人用概率统计方法,
将数据和运算中的舍入误差视为服从某种分布的随机变量,
然后确定计算结果的误差分布,
这样得到的误差估计更接近实际.
我们把这种方法称为\DefineConcept{概率分析法}.

\subsection{算法的数值稳定性}
%@see: 《数值分析(第5版)》(李庆扬、王能超、易大义) P10 定义3
用一个算法进行计算,
如果初始数据或输入数据的误差在计算中传播,使得计算结果的舍入误差增长很快,
那么就说这个算法是\DefineConcept{数值不稳定的};
反之,如果计算结果的舍入误差不增长,
则称这个算法是\DefineConcept{数值稳定的}.

\subsection{病态问题与条件数}
给定一个数值问题,
输入数据的微小扰动(即误差)
引起输出数据(即问题解)相对误差很大,
那么这个数值问题就叫做\DefineConcept{病态问题}.
例如,计算函数值\(f(x)\)时,
假设\(x\)有扰动\(\increment x = x^* - x\),
\(x\)的相对误差为\(\frac{\increment x}{x}\),
函数值\(f(x)\)的相对误差为\(\frac{f(x^*) - f(x)}{f(x)}\),
又假设差商\(\frac{f(x^*) - f(x)}{\increment x}\)约等于微商\(f'(x)\),
那么\(f(x)\)与\(x\)的相对误差比值为\begin{equation*}
%@see: 《数值分析(第5版)》(李庆扬、王能超、易大义) P10 (3.3)
	\abs{\frac{f(x^*) - f(x)}{f(x)}}
	\bigg/
	\abs{\frac{\increment x}{x}}
	= \abs{\frac{x}{f(x)} \cdot \frac{f(x^*) - f(x)}{\increment x}}
	\approx \abs{\frac{x f'(x)}{f(x)}}.
\end{equation*}
我们把\begin{equation*}
	C_p
	\defeq
	\abs{\frac{x f'(x)}{f(x)}}
\end{equation*}
称为计算函数值问题的\DefineConcept{条件数}.
自变量相对误差\(\frac{\increment x}{x}\)一般不会太大.
如果条件数\(C_p\)很大,
将引起函数值相对误差\(\frac{f(x^*) - f(x)}{f(x)}\)很大,
出现这种情况的问题就是病态问题.

例如,取\(f(x) \defeq x^n\),
则\(C_p = \abs{\frac{x \cdot n x^{n-1}}{x^n}} = n\),
这表示计算结果的相对误差相交于输入数据可能放大\(n\)倍.
具体地,取\(n \defeq 10\),
令准确值\(x \defeq 1\),令近似值\(x^* \defeq 1.02\),
则自变量相对误差为\(2\%\),
函数值相对误差为\(24\%\),
这时\(f(x)\)的计算问题可以认为是病态的.

一般地,当条件数\(C_p \geq 10\)时,
我们就认为问题是病态的,
并且\(C_p\)越大病态越严重.

\begin{example}

%@see: 《数值分析(第5版)》(李庆扬、王能超、易大义) P11 例6
求解线性方程组\begin{equation*}
	\begin{cases}
		x + \alpha y = 1, \\
		\alpha x + y = 0.
	\end{cases}
\end{equation*}
\begin{solution}
当\(\alpha = 1\)时,系数行列式为零,方程无解.
但是当\(\alpha \neq 1\)时,解得\begin{equation*}
	x = \frac{1}{1-\alpha^2},
	\qquad
	y = -\frac{\alpha}{1-\alpha^2}.
\end{equation*}
当\(\alpha \approx 1\)时,若输入数据\(\alpha\)有微小扰动(误差),则解的误差就会很大》
实际上,把\(x = \frac{1}{1-\alpha^2}\)看成\(\alpha\)的函数,
那么条件数为\begin{equation*}
	C_p = \abs{\frac{\alpha x'(\alpha)}{x(\alpha)}}
	= \abs{\frac{2\alpha^2}{1-\alpha^2}}.
\end{equation*}
假如输入数据的准确值为\(\alpha = 0.99\),
那么条件数为\(C_p \approx 100\),
因此本问题是病态的.
\end{solution}
\end{example}

应该注意到的是,病态问题不是计算方法引起的,
而是数值问题自身固有的,
因此,对数值问题首先要分清问题是否病态,
对于病态问题就必须采取相应的特殊方法以减少误差危害.

\subsection{避免误差危害}
数值计算中,通常不采用数值不稳定算法.
在设计算法时,还应该尽量避免误差危害,防止有效数字损失.
通常要避免两个相近的数相减、用绝对值很小的数做除数,
还要注意运算次序和减少运算次数.

\section{数值计算中算法设计的技术}
%@see: 《数值分析(第5版)》(李庆扬、王能超、易大义) P13
在数值计算中,算法设计的好坏,不但影响计算结果的精度,还会影响计算时间的长度.
下面给出几个具有代表性的算法,让我们一起学习它们的基本原则.

\subsection{多项式求值的秦九韶算法}
让我们考虑下面这个问题:
给定\(n\)次多项式\begin{equation}\label{equation:数值计算算法设计.秦九韶算法待计算的多项式}
	p(x)
	\defeq
	a_0 x^n + a_1 x^{n-1} + \dotsb + a_{n-1} x + a_n
	\quad(a_0\neq0),
\end{equation}
求\(p(x^*)\).

若直接计算每一项\(a_i x^{n-i}\)再相加,
总共需要\begin{equation*}
	\sum_{i=0}^n (n-i)
	= \frac{n(n+1)}{2}
	= O(n^2)
\end{equation*}
次乘法和\(n\)次加法.

但是,如果利用递推公式\begin{equation}\label{equation:数值计算算法设计.秦九韶算法递推公式1}
%@see: 《数值分析(第5版)》(李庆扬、王能超、易大义) P13 (4.1)
	\begin{cases}
		b_0 = a_0, \\
		b_i = b_{i-1} x^* + a_i
		\quad(i=1,2,\dotsc,n),
	\end{cases}
\end{equation}
计算出\(b_n\),那么就会发现\(p(x^*)\)恰与\(b_n\)相等.
这个算法便是秦九韶算法.
用它计算\(n\)次多项式\(p(x^*)\)只需要\(n\)次乘法和\(n\)次加法,
乘法次数由\(O(n^2)\)降为\(O(n)\),
并且只需要\(n+2\)个存储单元.
秦九韶算法是计算多项式函数值的最佳算法,
它是南宋数学家秦九韶于1247年提出的.
% 国外称此算法为 Horner 算法(1819年提出).

秦九韶算法还可以用于计算\(p'(x^*)\).
由递推公式 \labelcref{equation:数值计算算法设计.秦九韶算法递推公式1} 可知\begin{equation*}
	\begin{cases}
		a_0 = b_0, \\
		a_i = b_i - b_{i-1} x^*
		\quad(i=1,2,\dotsc,n),
	\end{cases}
\end{equation*}
代入\cref{equation:数值计算算法设计.秦九韶算法待计算的多项式} 得\begin{align*}
	p(x)
	&= b_0 x^n + (b_1 - b_0 x^*) x^{n-1} + \dotsb + (b_{n-1} - b_{n-2} x^*) x + (b_n - b_{n-1} x^*) \\
	&= (b_0 x^{n-1} + b_1 x^{n-2} + \dotsb + b_{n-1}) x
		- (b_0 x^{n-1} + \dotsb + b_{n-2} x + b_{n-1}) x^*
		+ b_n \\
	&= (b_0 x^{n-1} + b_1 x^{n-2} + \dotsb + b_{n-2} x + b_{n-1}) (x - x^*) + b_n,
\end{align*}
记\(q(x) \defeq b_0 x^{n-1} + b_1 x^{n-2} + \dotsb + b_{n-2} x + b_{n-1}\),
则\begin{equation*}
	p(x) = (x - x^*) q(x) + p(x^*),
\end{equation*}
求导得\begin{equation*}
	p'(x) = q(x) + (x - x^*) q'(x),
\end{equation*}
从而\begin{equation*}
	p'(x^*) = q(x^*),
\end{equation*}
正如我们刚刚对\(p(x)\)运用秦九韶算法得到
递推公式 \labelcref{equation:数值计算算法设计.秦九韶算法递推公式1} 一样,
现在对\(q(x)\)运用秦九韶算法可得递推公式\begin{equation*}
	\begin{cases}
		c_0 = b_0, \\
		c_i = c_{i-1} x^* + b_i
		\quad(i=1,2,\dotsc,n-1),
	\end{cases}
\end{equation*}
于是\(p'(x^*) = q(x^*) = c_{n-1}\).


\chapter{插值法}
历史上,天文学家、物理学家通过观测或实验获得了大量数据.
对于这些数据,我们总是希望从中找出表示某种内在规律的数量关系.
例如,天文学家开普勒在分析天文观测数据时发现行星轨道周期的平方
与它的轨道半长轴的立方成正比(此即开普勒第三定律).
%@see: https://mathshistory.st-andrews.ac.uk/HistTopics/Keplers_laws/
在计算机尚未发明的年代,
有的函数虽有解析表达式,
但是由于计算复杂,使用不方便,
所以数学家会编制函数表,
例如三角函数表、对数表、平方根表、立方根表等,
接着借助函数表,
构造一个既能反映数量关系又便于计算的简单函数\(P\),
用\(P\)近似\(f\).
通常选用的简单函数包括多项式函数、分段多项式函数、有理分式、三角多项式等,
我们把这类函数称为\DefineConcept{插值函数},
把自变量\(x_0,\dotsc,x_n\)称为\DefineConcept{插值节点},
把包含插值节点的区间\([a,b]\)称为\DefineConcept{插值区间},
把求解插值函数的方法称为\DefineConcept{插值法}.

\section{拉格朗日插值}
\subsection{多项式插值的存在性}
设\begin{equation*}
	a \leq x_0 < x_1 < \dotsb < x_n \leq b,
	\alpha \leq y_0 < y_1 < \dotsb < y_n \leq \beta,
\end{equation*}
现在要求次数不超过\(n\)的多项式\begin{equation*}
	P(x) = a_0 + a_1 x + a_2 x^2 + \dotsb + a_n x^n,
\end{equation*}
使得\begin{equation}\label{equation:插值法.用于插值的多项式需要满足的方程组}
%@see: 《数值分析(第5版)》(李庆扬、王能超、易大义) P23 (1.3)
	P(x_i) = y_i
	\quad(i=0,1,2,\dotsc,n)
\end{equation}成立,
那么可以建立关于\(a_0,\dotsc,a_n\)的\(n+1\)元线性方程组\begin{equation}\label{equation:插值法.用于插值的多项式的系数需要满足的线性方程组}
%@see: 《数值分析(第5版)》(李庆扬、王能超、易大义) P23 (1.4)
	\begin{cases}
		a_0 + a_1 x_0 + a_2 x_0^2 + \dotsb + a_n x_0^n = y_0, \\
		a_0 + a_1 x_1 + a_2 x_1^2 + \dotsb + a_n x_1^n = y_0, \\
		\hdotsfor{1}, \\
		a_0 + a_1 x_n + a_2 x_n^2 + \dotsb + a_n x_n^n = y_0, \\
	\end{cases}
\end{equation}
此方程组的系数矩阵就是范德蒙德矩阵\begin{equation*}
%@see: 《数值分析(第5版)》(李庆扬、王能超、易大义) P23 (1.5)
	A = \begin{bmatrix}
		1 & x_0 & \dots & x_0^n \\
		1 & x_1 & \dots & x_1^n \\
		\vdots & \vdots & & \vdots \\
		1 & x_n & \dots & x_n^n \\
	\end{bmatrix}.
\end{equation*}
由于\(x_i\)互不相等,所以范德蒙德行列式\begin{equation*}
	\det A
	= \prod_{0 \leq j < i \leq n+1} (x_i - x_j)
	\neq 0,
\end{equation*}
于是方程 \labelcref{equation:插值法.用于插值的多项式的系数需要满足的线性方程组} 的解存在且唯一,
也就是说,满足条件 \labelcref{equation:插值法.用于插值的多项式需要满足的方程组} 的多项式存在且唯一.

虽然只要求解方程 \labelcref{equation:插值法.用于插值的多项式的系数需要满足的线性方程组} 就能得到插值多项式函数\(P\),
但是这种解法十分繁琐,通常是不与采用的,下面我们介绍构造插值多项式的简便方法.

\subsection{拉格朗日插值多项式}
如前所述,给定\(n+1\)个节点\(x_0 < x_1 < \dotsb < x_n\),
要求\(n\)次插值多项式\(L_n\),
假定它满足\begin{equation*}
%@see: 《数值分析(第5版)》(李庆扬、王能超、易大义) P25 (2.6)
	L_n(x_j) = y_j
	\quad(j=0,1,2,\dotsc,n).
\end{equation*}

为了构造\(L_n\),我们首先需要定义\(n\)次插值基函数.
\begin{definition}
%@see: 《数值分析(第5版)》(李庆扬、王能超、易大义) P25 定义1
设\(n+1\)个\(n\)次多项式\(l_j\ (j=0,1,2,\dotsc,n)\)
在\(n+1\)个节点\(x_0 < x_1 < \dotsb < x_n\)上满足\begin{equation*}
%@see: 《数值分析(第5版)》(李庆扬、王能超、易大义) P25 (2.7)
	l_j(x_k) = \begin{cases}
		1, & k=j, \\
		0, & k \neq j,
	\end{cases}
	\quad(j,k=0,1,2,\dotsc,n),
\end{equation*}
则称多项式\(l_j\ (j=0,1,2,\dotsc,n)\)是
节点\(x_0,x_1,\dotsc,x_n\)上的
\DefineConcept{\(n\)次插值基函数}.
\end{definition}
\begin{remark}
拉格朗日插值法用到的基函数是一组正交基.
\end{remark}

显然\begin{equation}
%@see: 《数值分析(第5版)》(李庆扬、王能超、易大义) P26 (2.8)
	l_k(x)
	= \prod_{0 \leq i \leq n, i \neq k} \frac{(x - x_i)}{(x_k - x_i)}
	\quad(k=0,1,2,\dotsc,n).
\end{equation}
于是所求插值多项式为\begin{equation}
%@see: 《数值分析(第5版)》(李庆扬、王能超、易大义) P26 (2.9)
	L_n(x)
	= \sum_{k=0}^n y_k l_k(x).
\end{equation}
我们把\(L_n\)称为\DefineConcept{拉格朗日插值多项式}.

若记\begin{equation}
%@see: 《数值分析(第5版)》(李庆扬、王能超、易大义) P26 (2.10)
	\omega_{n+1}(x)
	\defeq
	\prod_{i=0}^n (x - x_i),
\end{equation}
则有\begin{equation*}
	\omega_{n+1}'(x_k)
	= \prod_{0 \leq i \leq n, i \neq k} (x_k - x_i),
\end{equation*}
于是\(L_n\)又可写为\begin{equation}
%@see: 《数值分析(第5版)》(李庆扬、王能超、易大义) P26 (2.11)
	L_n(x) = \sum_{k=0}^n y_k \frac{\omega_{n+1}(x)}{(x - x_k) \omega_{n+1}'(x_k)}.
\end{equation}

\begin{example}
给定数据\begin{center}
	\begin{tblr}{c|*4c}
		\hline
		\(i\) & 0 & 1 & 2 & 3 \\
		\hline
		\(x_i\) & 0 & 1 & 2 & 3 \\
		\(y_i\) & 0 & 1 & 5 & 14 \\
		\hline
	\end{tblr}
\end{center}
求三次拉格朗日插值多项式\(L_3\).
\begin{solution}
基函数为\begin{align*}
	l_1(x)
	&= \frac{(x-0)(x-2)(x-3)}{(1-0)(1-2)(1-3)}
	= \frac{x(x-2)(x-3)}{2}, \\
	l_2(x)
	&= \frac{(x-0)(x-1)(x-3)}{(2-0)(2-1)(2-3)}
	= \frac{x(x-1)(x-3)}{-2}, \\
	l_3(x)
	&= \frac{(x-0)(x-1)(x-2)}{(3-0)(3-1)(3-2)}
	= \frac{x(x-1)(x-2)}{6}.
\end{align*}
于是所求插值多项式为\begin{align*}
	L_3(x)
	&= \sum_{k=0}^3 y_k l_k(x)
	= 0 \cdot l_0(x)
	+ 1 \cdot l_1(x)
	+ 5 \cdot l_2(x)
	+ 14 \cdot l_3(x) \\
	&= \frac{x(x-2)(x-3)}{2}
	+ 5 \frac{x(x-1)(x-3)}{-2}
	+ 14 \frac{x(x-1)(x-2)}{6} \\
	&= \frac13 x^3 + \frac12 x^2 + \frac16 x
	= \frac16 x(x+1)(2x+1).
\end{align*}
\end{solution}
%@Mathematica: Expand[InterpolatingPolynomial[{{0,0},{1,1},{2,5},{3,14}},x]]
%@Mathematica: Expand[x(x-2)(x-3)/2+5x(x-1)(x-3)/(-2)+14x(x-1)(x-2)/6]
\end{example}

\subsection{插值余项与误差估计}
%@see: 《数值分析(第5版)》(李庆扬、王能超、易大义) P26
在闭区间\([a,b]\)上用\(L_n\)近似\(f\),
则其截断误差为\begin{equation*}
	R_n(x) \defeq f(x) - L_n(x).
\end{equation*}
我们把\(R_n\)称为“插值多项式\(L_n\)的\DefineConcept{余项}”.

%@see: 《数值分析(第5版)》(李庆扬、王能超、易大义) P26 定理2
给定结点\(a \leq x_0 < x_1 < \dotsb + x_n \leq b\),
设函数\(f \in C^n[a,b] \cap D^{n+1}(a,b)\),
\(L_n\)是用来近似\(f\)的插值多项式,
那么由\hyperref[equation:微分中值定理.泰勒公式.余项1]{带有拉格朗日余项的泰勒公式}可知,
对于任意\(x \in [a,b]\),
插值余项满足\begin{equation*}
%@see: 《数值分析(第5版)》(李庆扬、王能超、易大义) P26 (2.12)
	R_n(x) = \frac{f^{(n+1)}(\xi)}{(n+1)!} \omega_{n+1}(x),
\end{equation*}
其中\(\xi \in (a,b)\)且依赖于\(x\)的取值.
记\(M \defeq \max_{a \leq x \leq b} \abs{f^{(n+1)}(x)}\),
那么用插值多项式\(L_n\)近似\(f\)的误差限为\begin{equation*}
	% \sup_{a \leq x \leq b} \abs{R_n(x)}
	\frac{M}{(n+1)!} \abs{\omega_{n+1}(x)}.
\end{equation*}

\begin{example}
%@see: 《数值分析(第5版)》(李庆扬、王能超、易大义) P28 例2
已知\begin{equation*}
	\sin0.32 \approx \num{0.314567},
	\qquad
	\sin0.34 \approx \num{0.333487},
	\qquad
	\sin0.36 \approx \num{0.352274}.
\end{equation*}
用线性插值法和抛物插值法分别计算\(\sin\num{0.3367}\)的近似值,并估计截断误差.
\begin{solution}
首先采用线性插值法,
拉格朗日插值多项式为\begin{equation*}
	L_1(x) = \num{0.314567} \frac{x-0.34}{0.32-0.34}
	+ \num{0.333487} \frac{x-0.32}{0.34-0.32},
\end{equation*}
于是\begin{equation*}
	L_1(\num{0.3367})
	\approx \num{0.330365}.
\end{equation*}
其截断误差\begin{equation*}
	\abs{R_1(x)}
	\leq \frac{M_2}{2} \abs{(x-0.32)(x-0.34)},
\end{equation*}
其中\(M_2 = \max_{0.32 \leq x \leq 0.34} \abs{f''(x)}\).
因为对\(f(x) = \sin x\)求导可得\(f''(x) = -\sin x\),
所以\begin{equation*}
	M_2
	= \max_{0.32 \leq x \leq 0.34} \abs{\sin x}
	= \sin0.34
	\leq \num{0.3335},
\end{equation*}
于是截断误差为\begin{equation*}
	\abs{R_1(\num{0.3367})}
	\leq \frac{\num{0.3335}}{2} \abs{(\num{0.3367}-0.32)(\num{0.3367}-0.34)}
	\leq \num{9.2e-6},
\end{equation*}

接着采用抛物插值法,
拉格朗日插值多项式为\begin{align*}
	L_2(x) &= \num{0.314567} \frac{(x-0.34)(x-0.36)}{(0.32-0.34)(0.32-0.36)} \\
	&+ \num{0.333487} \frac{(x-0.32)(x-0.36)}{(0.34-0.32)(0.34-0.36)} \\
	&+ \num{0.352274} \frac{(x-0.32)(x-0.34)}{(0.36-0.32)(0.36-0.34)},
\end{align*}
%@Mathematica: 0.314567(x-0.34)(x-0.36)/((0.32-0.34)(0.32-0.36))+0.333487(x-0.32)(x-0.36)/((0.34-0.32)(0.34-0.36))+0.352274(x-0.32)(x-0.34)/((0.36-0.32)(0.36-0.34)) /. x->0.3367
于是\begin{equation*}
	L_2(\num{0.3367})
	\approx \num{0.330374}.
\end{equation*}
其截断误差为\begin{equation*}
	\abs{R_2(\num{0.3367})}
	\leq \num{2.0e-6}.
\end{equation*}
\end{solution}
\end{example}

\section{牛顿插值}
当数据增加或减少时,
拉格朗日插值法就非常不方便了,
这是因为之前的计算结果全都作废了,计算要全部从头开始.
因此,为了计算简便,就可以使用牛顿插值法.

\subsection{差商}
\begin{definition}
%@see: 《数值分析(第5版)》(李庆扬、王能超、易大义) P30 定义2
设函数\(f\colon \mathbb{R} \to \mathbb{R}\).
\begin{itemize}
	\item 把\(f(x_i)\)
	称为“函数\(f\)关于点\(x_i\)的\DefineConcept{零阶差商}”,
	记作\(f[x_i]\).

	\item 把\begin{equation*}
		\frac{f(x_j) - f(x_i)}{x_j - x_i}
	\end{equation*}
	称为“函数\(f\)关于点\(x_i,x_j\)的\DefineConcept{一阶差商}”,
	记作\(f[x_i,x_j]\).

	\item 把\begin{equation*}
		\frac{f[x_i,x_k] - f[x_i,x_j]}{x_k - x_j}
	\end{equation*}
	称为“函数\(f\)关于点\(x_i,x_j,x_k\)的\DefineConcept{二阶差商}”,
	记作\(f[x_i,x_j,x_k]\).

	\item 把\begin{equation*}
	%@see: 《数值分析(第5版)》(李庆扬、王能超、易大义) P30 (3.3)
		\frac{f[x_0,\dotsc,x_{k-2},x_k] - f[x_0,\dotsc,x_{k-2},x_{k-1}]}{x_k - x_{k-1}}
	\end{equation*}
	称为“函数\(f\)关于点\(x_0,\dotsc,x_k\)的 \DefineConcept{\(k\)阶差商}”,
	记作\(f[x_0,\dotsc,x_k]\).
\end{itemize}
\end{definition}
\begin{remark}
在\(k\)阶差商的定义式中,
分子中的被减数\(f[x_0,\dotsc,x_{k-2},x_k]\)中没有符号\(x_{k-1}\),
与此同时,
分子中的减数\(f[x_0,\dotsc,x_{k-2},x_{k-1}]\)中没有符号\(x_k\),
这正好与分母\(x_k - x_{k-1}\)的顺序相反.
\end{remark}

\begin{property}
%@see: 《数值分析(第5版)》(李庆扬、王能超、易大义) P30
\(k\)阶差商\(f[x_0,\dotsc,x_k]\)
可以表示为函数值\(f(x_0),\dotsc,f(x_k)\)的线性组合,
即\begin{equation}
%@see: 《数值分析(第5版)》(李庆扬、王能超、易大义) P30 (3.4)
	f[x_0,\dotsc,x_k]
	= \sum_{j=0}^k
	\frac{f(x_j)}{
		\prod_{0 \leq i \leq k, i \neq j}
		(x_j - x_i)
	}.
\end{equation}
\end{property}

\begin{property}[差商的对称性]
%@see: 《数值分析(第5版)》(李庆扬、王能超、易大义) P30
\(k\)阶差商\(f[x_0,\dotsc,x_k]\)的取值
与数据的编号次序无关,
即\begin{equation}
	f[x_0,\dotsc,x_i,\dotsc,x_j,\dotsc,x_k]
	= f[x_0,\dotsc,x_j,\dotsc,x_i,\dotsc,x_k].
\end{equation}
\end{property}

\begin{property}
\(k\)阶差商\(f[x_0,\dotsc,x_k]\)还可以表示为\begin{equation}\label{equation:牛顿插值.差商的计算式}
%@see: 《数值分析(第5版)》(李庆扬、王能超、易大义) P30 (3.3)'
	f[x_0,\dotsc,x_k]
	= \frac{f[x_1,\dotsc,x_{k-1},x_k] - f[x_0,x_1,\dotsc,x_{k-1}]}{x_k - x_0}.
\end{equation}
\end{property}

\begin{property}
若\(f\)在\([a,b]\)上存在\(n\)阶导数,
且\(x_0,\dotsc,x_n \in (a,b)\),
则\begin{equation*}%\label{equation:牛顿插值.差商与导数的关系}
%@see: 《数值分析(第5版)》(李庆扬、王能超、易大义) P30 (3.5)
	f[x_0,\dotsc,x_n]
	= \frac{f^{(n)}(\xi)}{n!}
	\quad(a \leq \xi \leq b).
\end{equation*}
\end{property}

当我们利用\cref{equation:牛顿插值.差商的计算式} 逐阶计算差商,
可得\begin{align*}
	f[x_0,x_1,x_2]
	&= \frac{f[x_1,x_2] - f[x_0,x_1]}{x_2 - x_0}, \\
	f[x_1,x_2,x_3]
	&= \frac{f[x_2,x_3] - f[x_1,x_2]}{x_3 - x_1}, \\
	f[x_0,x_1,x_2,x_3]
	&= \frac{f[x_1,x_2,x_3] - f[x_0,x_1,x_2]}{x_3 - x_0}.
\end{align*}
根据上述计算结果,我们可以列出一张表:\begin{center}
	\begin{tblr}{*3c|*3c}
		\hline
		\(k\) & \(x_k\) & \(f(x_k)\) & 一阶差商 & 二阶差商 & 三阶差商 \\
		\hline
		0 & \(x_0\) & \(f(x_0)\) \\
		1 & \(x_1\) & \(f(x_1)\) & \(f[x_0,x_1]\) \\
		2 & \(x_2\) & \(f(x_2)\) & \(f[x_1,x_2]\) & \(f[x_0,x_1,x_2]\) \\
		3 & \(x_3\) & \(f(x_3)\) & \(f[x_2,x_3]\) & \(f[x_1,x_2,x_3]\) & \(f[x_0,x_1,x_2,x_3]\) \\
		\hline
	\end{tblr}
\end{center}
我们把上述表格称为\DefineConcept{差商表}.

\subsection{牛顿差商插值多项式}
我们把\begin{align*}
%@see: 《数值分析(第5版)》(李庆扬、王能超、易大义) P31 (3.6)
	N_n(x)
	&\defeq
	\sum_{j=0}^n f[x_0,\dotsc,x_j] \prod_{i=0}^{j-1} (x-x_i) \\
	&= f(x_0)
	+ f[x_0,x_1] (x-x_0)
	+ f[x_0,x_1,x_2] (x-x_0)(x-x_1) \\
	&\hspace{20pt}+ \dotsb
	+ f[x_0,\dotsc,x_n] (x-x_0)\dotsm(x-x_{n-1})
\end{align*}
称为\DefineConcept{牛顿差商插值多项式},
把截断误差\begin{align*}
%@see: 《数值分析(第5版)》(李庆扬、王能超、易大义) P31 (3.7)
	R_n(x)
	&\defeq
	f(x) - N_n(x) \\
	&= f[x,x_0,\dotsc,x_n] \omega_{n+1}(x)
\end{align*}
称为“牛顿差商插值多项式\(N_n\)的\DefineConcept{余项}”.

\section{厄米插值}
%@see: 《数值分析(第5版)》(李庆扬、王能超、易大义) P35
有时候我们不仅希望插值函数\(P\)与目标函数\(f\)的函数值在插值节点上相等,
还希望它们的高阶导数值也在插值节点上相等.
满足上述要求的插值多项式,
称为\DefineConcept{厄米插值多项式}.

\subsection{重节点差商,泰勒插值}
\begin{theorem}
%@see: 《数值分析(第5版)》(李庆扬、王能超、易大义) P35 定理3
设\(f \in C^n[a,b]\),
给定相异的\(n+1\)个节点\(
	a \leq x_0 < x_1 < \dotsb < x_n \leq b
\),
则\((x_0,\dotsc,x_n) \mapsto f[x_0,\dotsc,x_n]\)是连续函数.
%TODO proof
\end{theorem}

根据差商的定义,如果\(f \in C^1[a,b]\)且\(x \neq x_0\),则有\begin{equation*}
	\lim_{x \to x_0} f[x_0,x]
	= \lim_{x \to x_0} \frac{f(x) - f(x_0)}{x - x_0}
	= f'(x_0).
\end{equation*}
由此定义\begin{equation}
	f[x_0,x_0]
	\defeq
	\lim_{x \to x_0} f[x_0,x],
\end{equation}
称之为\DefineConcept{重节点\(x_0\)的一阶差商}.
类似地,如果\(f \in C^2[a,b]\)且\(x \neq x_0\),则有\begin{equation*}
	f[x_0,x_0,x]
	= \frac{f[x_0,x] - f[x_0,x_0]}{x - x_0},
	\qquad
	\lim_{x \to x_0} f[x_0,x_0,x]
	= \frac12 f''(x_0).
\end{equation*}
由此定义\begin{equation}
	f[x_0,x_0,x_0]
	\defeq
	\lim_{x \to x_0} f[x_0,x_0,x],
\end{equation}
称之为\DefineConcept{重节点\(x_0\)的二阶差商}.
一般地,如果\(f \in C^n[a,b]\),定义\begin{equation}
	f[\underbrace{x_0,\dotsc,x_0}_{\text{$n+1$个}}]
	\defeq
	\lim_{x \to x_0} f[\underbrace{x_0,\dotsc,x_0}_{\text{$n$个}},x],
\end{equation}
称之为\DefineConcept{重节点\(x_0\)的\(n\)阶差商}.
显然有\begin{equation}\label{equation:厄米插值.重节点的高阶差商与高阶导数的关系}
%@see: 《数值分析(第5版)》(李庆扬、王能超、易大义) P35 (4.1)
	f[\underbrace{x_0,\dotsc,x_0}_{\text{$n+1$个}}]
	= \frac1{n!} f^{(n)}(x_0).
\end{equation}

在\hyperref[equation:牛顿插值.牛顿差商插值多项式]{牛顿差商插值多项式}中,
若令\(x_i \to x_0\ (i=1,2,\dotsc,n)\),
则由\cref{equation:厄米插值.重节点的高阶差商与高阶导数的关系}
可得泰勒多项式\begin{equation}\label{equation:厄米插值.泰勒插值多项式}
%@see: 《数值分析(第5版)》(李庆扬、王能超、易大义) P35 (4.2)
	P_n(x)
	= f(x_0)
	+ f'(x_0) (x-x_0)
	+ \dotsb
	+ \frac{f^{(n)}(x_0)}{n!} (x-x_0)^n.
\end{equation}
它实际上是在点\(x_0\)附近逼近\(f(x)\)的一个带导数的插值多项式,
它满足条件\begin{equation}\label{equation:厄米插值.泰勒插值多项式的高阶导数}
%@see: 《数值分析(第5版)》(李庆扬、王能超、易大义) P35 (4.3)
	P_n^{(k)}(x_0)
	= f^{(k)}(x_0)
	\quad(k=0,1,2,\dotsc,n).
\end{equation}
我们把\cref{equation:厄米插值.泰勒插值多项式} 称为\DefineConcept{泰勒插值多项式},
它就是一个厄米插值多项式,
它的余项为\begin{equation*}
%@see: 《数值分析(第5版)》(李庆扬、王能超、易大义) P35 (4.4)
	R_n(x)
	= \frac{f^{(n+1)}(\xi)}{(n+1)!} (x-x_0)^{n+1}
	\quad(a < \xi < b).
\end{equation*}
它与\hyperref[equation:拉格朗日插值.拉格朗日插值余项]{拉格朗日插值余项}中
令\(x_i \to x_0\ (i=1,2,\dotsc,n)\)的结果一致.
实际上泰勒插值是牛顿插值的极限形式,
是只在一点\(x_0\)给出\(n+1\)个插值条件 \labelcref{equation:厄米插值.泰勒插值多项式的高阶导数}
得到的\(n\)次厄米插值多项式.

一般地,只要给出\(m+1\)个插值条件(包括函数值和导数值),
就可以造出次数不超过\(m\)次的厄米插值多项式.
由于导数条件各不相同,这里就不给出一般的厄米插值公式,只讨论两个典型的例子.

\subsection{厄米插值的应用 --- 三点三次厄米插值}
先考虑满足条件\begin{equation*}
	P(x_i) = f(x_i)
	\ (i=0,1,2)
	\quad\text{和}\quad
	P'(x_1) = f'(x_1)
\end{equation*}
的插值多项式及其余项表达式.

由给定条件,可以确定次数不超过3的插值多项式.
由于该多项式通过点\(
	(x_0,f(x_0)),
	(x_1,f(x_1)),
	(x_2,f(x_2))
\),
所以它的形式为\begin{equation*}
	P(x)
	= f(x_0)
	+ f[x_0,x_1] (x-x_0)
	+ f[x_0,x_1,x_2] (x-x_0)(x-x_1)
	+ A (x-x_0)(x-x_1)(x-x_2),
\end{equation*}
其中\(A\)是待定常数,
可以根据条件\(P'(x_1) = f'(x_1)\)予以确定,
即\begin{equation*}
	A = \frac{
		f'(x_1) - f[x_0,x_1] - (x_1-x_0) f[x_0,x_1,x_2]
	}{
		(x_1-x_0)(x_1-x_2)
	}.
\end{equation*}
为了求出余项\(R(x) = f(x) - P(x)\)的表达式,
可以设\begin{equation*}
	R(x)
	\defeq f(x) - P(x)
	= k(x) (x-x_0) (x-x_1)^2 (x-x_2),
\end{equation*}
其中\(k(x)\)是待定函数.
构造\begin{equation*}
	\phi(t)
	\defeq
	f(t) - P(t) - k(x) (t-x_0) (t-x_1)^2 (t-x_2),
\end{equation*}
显然\(\phi(x_j) = 0\ (j=0,1,2)\)且\(\phi'(x_1) = \phi(x) = 0\).
故\(\phi(t)\)在\((a,b)\)内有5个零点(其中二重根算两个).
假设\(f\)具有较好的可微性,
那么只要反复应用罗尔定理,
可得\(\phi^{(4)}(t)\)在\((a,b)\)内至少有一个零点\(\xi\),
故\begin{equation*}
	\phi^{(4)}(\xi) = f^{(4)}(\xi) - 4! k(x) = 0,
\end{equation*}
于是\begin{equation*}
	k(x) = \frac1{4!} f^{(4)}(\xi),
\end{equation*}
余项表达式为\begin{equation*}
	R(x) = \frac1{4!} f^{(4)}(\xi) (x-x_0) (x-x_1)^2 (x-x_2),
\end{equation*}
其中\(\xi\)位于\(x_0,x_1,x_2\)和\(x\)所界定的范围内.

\section{分段低次插值}
\subsection{高次插值的病态性质}
%@see: 《数值分析(第5版)》(李庆扬、王能超、易大义) P39
人们一度认为:
根据区间\([a,b]\)上给出的节点构造插值多项式\(L_n\)近似\(f\),
\(L_n\)的次数\(n\)越高,逼近\(f\)的精度就越好.
但是事实并非如此,
这是因为对于任意插值节点,
当\(n\to\infty\)时,
\(L_n(x)\)不一定收敛于\(f(x)\).
%@see: https://personal.math.ubc.ca/~peirce/M406_Lecture_3_Runge_Phenomenon_Piecewise_Polynomial_Interpolation.pdf
龙格发现函数\(f(x) = \frac1{1+x^2}\)在闭区间\([-5,5]\)上存在各阶导数,
但是在\([-5,5]\)上取\(n+1\)个等距节点\(x_k = -5 + 10 \frac{k}{n}\ (k=0,1,2,\dotsc,n)\)
所构造的拉格朗日插值多项式为\begin{equation*}
	L_n(x)
	= \sum_{j=0}^n \frac{1}{1+x_j^2} \cdot \frac{\omega_{n+1}(x)}{(x - x_j) \omega_{n+1}'(x_j)}.
\end{equation*}
令\(x_{n-1/2} \defeq \frac12 (x_{n-1} + x_n)\),
则\(x_{n-1/2} = 5 - \frac5n\),
可以通过计算验证:随着\(n\)的增加,\(R(x_{n-1/2})\)的绝对值几乎在成倍增加.
这说明,当\(n\to\infty\)时,\(L_n\)在\([-5,5]\)上不收敛.
龙格进一步证明了:存在一个常数\(c \approx 3.63\),
使得当\(\abs{x} \leq c\)时,
有\(\lim_{n\to\infty} L_n(x) = f(x)\);
但是当\(\abs{x} > c\)时,
函数列\(\{L_n(x)\}_{n\geq1}\)发散.

\subsection{分段线性插值}
所谓“分段线性插值”,就是用折线段连接插值点,再把折线段作为插值函数逼近\(f\).

假设已知节点\(a = x_0 < x_1 < \dotsb < x_n = b\)上的函数值\(f_0,f_1,\dotsc,f_n\),
记\(
	h_k \defeq x_{k+1} - x_k,
	h \defeq \max_k h_k
\),
求折线函数\(I_h(x)\),
使之满足\begin{enumerate}
	\item \(I_h \in C[a,b]\);
	\item \(I_h(x_k) = f_k\ (k=0,1,2,\dotsc,n)\);
	\item \(I_h(x)\)在每个小区间\([x_k,x_{k+1}]\)上都是线性函数,
\end{enumerate}
则称“\(I_h\)是一个\DefineConcept{分段线性插值函数}”.

根据上述定义,\(I_h\)可以表示为\begin{equation}
%@see: 《数值分析(第5版)》(李庆扬、王能超、易大义) P40 (5.1)
	I_h(x)
	= f_k \frac{x - x_{k+1}}{x_k - x_{k+1}}
	+ f_{k+1} \frac{x - x_k}{x_{k+1} - x_k},
\end{equation}
其中\(
	x_k \leq x \leq x_{k+1},
	k=0,1,2,\dotsc,n-1
\).

分段线性插值的误差估计可以利用\hyperref[equation:拉格朗日插值.线性插值余项]{拉格朗日线性插值余项公式}得到\begin{equation*}
	\max_{x_k \leq x \leq x_{k+1}}
	\abs{f(x) - I_h(x)}
	\leq \frac{M_2}{2}
	\max_{x_k \leq x \leq x_{k+1}}
	\abs{(x - x_k) (x - x_{k+1})}
\end{equation*}
或\begin{equation*}
%@see: 《数值分析(第5版)》(李庆扬、王能超、易大义) P40 (5.2)
	\max_{a \leq x \leq b}
	\abs{f(x) - I_h(x)}
	\leq \frac{M_2}{8} h^2,
\end{equation*}
其中\(M_2 \defeq \max_{a \leq x \leq b} \abs{f''(x)}\).
由此还可以得到\begin{equation*}
	\lim_{h\to0} I_h(x) = f(x)
\end{equation*}
在\([a,b]\)上一致成立,
即\(I_h\)在\([a,b]\)上一致收敛于\(f\).

\subsection{分段三次厄米插值}
%@see: 《数值分析(第5版)》(李庆扬、王能超、易大义) P40
分段线性插值函数\(I_h\)的导数是间断的.
若在节点\(x_k\ (k=0,1,2,\dotsc,n)\)上,除了已知函数值\(f_k\)以外,
还给出导数值\(f'_k = m_k\ (k=0,1,2,\dotsc,n)\),
那么就可以构造一个导函数连续的分段插值函数\(I_h\),
它满足\begin{enumerate}
	\item \(I_h \in C^1[a,b]\);
	\item \(I_h(x_k) = f_k\ (k=0,1,2,\dotsc,n)\);
	\item \(I_h'(x_k) = f'_k\ (k=0,1,2,\dotsc,n)\);
	\item \(I_h(x)\)在每个小区间\([x_k,x_{k+1}]\)上都是三次多项式函数.
\end{enumerate}

根据\hyperref[equation:厄米插值.两点三次厄米插值多项式]{两点三次厄米插值多项式}可知,
\(I_h\)在区间\([x_k,x_{k+1}]\)上的表达式为\begin{equation*}
%@see: 《数值分析(第5版)》(李庆扬、王能超、易大义) P41 (5.3)
	\begin{aligned}
		I_h(x)
		&= \left( 1 + 2 \frac{x - x_k}{x_{k+1} - x_k} \right)
		\left( \frac{x - x_{k+1}}{x_k - x_{k+1}} \right)^2
		f_k \\
		&+ \left( 1 + 2 \frac{x - x_{k+1}}{x_k - x_{k+1}} \right)
		\left( \frac{x - x_k}{x_{k+1} - x_k} \right)^2
		f_{k+1} \\
		&+ (x - x_k)
		\left( \frac{x - x_{k+1}}{x_k - x_{k+1}} \right)^2
		f'_k
		+ (x - x_{k+1})
		\left( \frac{x - x_k}{x_{k+1} - x_k} \right)^2
		f'_{k+1},
	\end{aligned}
\end{equation*}
其中\(k=0,1,2,\dotsc,n-1\).

利用\hyperref[equation:厄米插值.两点三次厄米插值余项]{两点三次厄米插值余项}可得\begin{equation*}
	\abs{f(x) - I_h(x)}
	\leq \frac1{384} h_k^4
	\max_{x_k \leq x \leq x_{k+1}} \abs{f^{(4)}(x)}
	\quad(x_k \leq x \leq x_{k+1}),
\end{equation*}
其中\(h_k \defeq x_{k+1} - x_k\),
于是可得下述定理.

\begin{theorem}
%@see: 《数值分析(第5版)》(李庆扬、王能超、易大义) P41 定理4
设\(f \in C^4[a,b]\),
而\(I_h\)是\(f\)在节点\(a = x_0 < x_1 < \dotsb < x_n = b\)上的分段三次厄米插值多项式,
那么\begin{equation*}
	\max_{a \leq x \leq b} \abs{f(x) - I_h(x)}
	\leq \frac{h^4}{384} \max_{a \leq x \leq b} \abs{f^{(4)}(x)},
\end{equation*}
其中\(h \defeq \max_{0 \leq k \leq n-1}(x_{k+1} - x_k)\).
%TODO
\end{theorem}

上述定理表明分段三次厄米插值比分段线性插值效果明显更好.
但这种插值要求给出节点上的导数值,所需信息太多,其光滑度也不高(只有一阶导数连续,二阶以上导数不保证连续),
改进这种插值以克服其缺点就导致三次样条插值的提出.

\section{三次样条插值}

\chapter{函数逼近与快速傅里叶变换}
\section{函数逼近的基本概念}
函数逼近问题:
给定函数\(f\)(称之为目标函数),
我们想要在某个函数空间\(\Phi\)中找到一个函数\(\phi\)
使得误差\(f\)与\(\phi\)之间的距离在某种度量意义下最小.
\begin{definition}
% 这是我自己构思的定义
设\(D_f \subseteq D \subseteq \mathbb{R}\),
取定\(\Phi \subseteq \mathbb{R}^{D}\),
给定函数\(f\colon D_f \to \mathbb{R}\),
给定\(\Phi\)的一个度量\(\rho\),
把\begin{equation*}
	\argmin_{\phi \in \Phi} \rho(f,\phi \SetRestrict D_f)
\end{equation*}
称为“函数\(f\)在\(D\)上的\DefineConcept{最佳逼近函数}”.
\end{definition}

我们通常将函数空间\(\Phi\)限定为区间\([a,b]\)上的连续函数族\(C[a,b]\),
或者进一步限定为区间\([a,b]\)上的次数不超过\(n\)的多项式函数族\(H_n[a,b]\).
在这种情况下,把\begin{equation*}
	\argmin_{\phi \in H_n} \rho(f,\phi \SetRestrict D_f)
\end{equation*}
称为“函数\(f\)在\(D\)上的\DefineConcept{最佳逼近多项式}”.

如果度量\(\rho\)取为 \hyperref[equation:范数.连续函数的无穷范数]{\(L_\infty\)范数},
那么把\begin{equation*}
	\argmin_{\phi \in H_n} \max_{x \in D_f} \abs{f(x) - \phi(x)}
\end{equation*}
称为“函数\(f\)在\(D\)上的\DefineConcept{最佳一致逼近多项式}”.

如果度量\(\rho\)取为 \hyperref[equation:范数.连续函数的L2范数]{\(L_2\)范数},
且\(f\)是连续的,
那么把\begin{equation*}
	\argmin_{\phi \in H_n} \int_{D_f} (f(x) - \phi(x))^2 \dd{x}
\end{equation*}
称为“函数\(f\)在\(D\)上的\DefineConcept{最佳平方逼近多项式}”.

如果度量\(\rho\)取为 \hyperref[equation:范数.连续函数的L2范数]{\(L_2\)范数},
且\(f\)是离散的,
那么把\begin{equation*}
	\argmin_{\phi \in H_n} \sum_{x \in D_f} (f(x) - \phi(x))^2
\end{equation*}
称为“函数\(f\)在\(D\)上的\DefineConcept{最小二乘拟合}”.

本章将着重讨论实际应用多且便于计算的最佳平方逼近与最小二乘拟合.

\section{正交多项式}
\section{最佳平方逼近}
\section{曲线拟合的最小二乘法}
\section{有理逼近}
\section{三角多项式与快速傅里叶变换}

\chapter{数值积分与数值微分}
\section{数值积分}
\section{牛顿--柯特斯公式}
\section{复合求积公式}
\section{龙贝格求积公式}
%@see: https://mathworld.wolfram.com/RombergIntegration.html
\section{自适应积分方法}
\section{高斯求积公式}
\section{多重积分}
\section{数值微分}
