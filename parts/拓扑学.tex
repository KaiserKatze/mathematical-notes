\part{拓扑学}
%@see: https://math.berkeley.edu/~ltomczak/notes/Mich2022/AlgTop_Notes.pdf
\chapter{拓扑空间与连续映射}
%@see: 《基础拓扑学讲义》(尤承业) P1
% 拓扑学是一种几何学,
% 它是研究几何图形的.
% 但是拓扑学所研究的并不是图形的几何性质,而是所谓“拓扑性质”.

% 例如,让我们考虑“平面上,由曲线段构成的一个图形,能否一笔画成,保证线段不重复?”这个“一笔画”问题.
% 汉字“中”“日”都是可以一笔写出来的,
% 而“目”“田”则不能一笔写成.

%@see: 《点集拓扑讲义(第四版)》(熊金城) P45
% 拓扑学是分析学中极限这一概念的推广,通常分为点集拓扑学和代数拓扑学.
% 本章主要介绍点集拓扑学,它为几何学提供了基本语言.
在这一章中,我们首先将连续函数的定义域和值域的主要特征抽象出来用以定义度量空间,
将连续函数的主要特征抽象出来用以定义度量空间之间的连续映射.
然后将两者再度抽象,给出拓扑空间和拓扑空间之间的连续映射.
随后再逐步提出拓扑空间中的一些基本问题,
如邻域、闭包、内部、边界、基、子基、序列等.

\def\sfA{\mathscr{A}}
\def\sfB{\mathscr{B}}

\section{度量空间}
根据\cref{definition:极限.函数在一点的连续性} 我们知道,
“函数\(f\colon\mathbb{R}\to\mathbb{R}\)在点\(x_0\in\mathbb{R}\)连续”
当且仅当\begin{equation*}
	(\forall\epsilon>0)
	(\exists\delta>0)
	(\forall x\in\mathbb{R})
	[
		\abs{x - x_0}<\delta
		\implies
		\abs{f(x) - f(x_0)} < \epsilon
	].
\end{equation*}
在这个定义中只涉及两个实数之间的距离(即两个实数之差的绝对值)这个概念.
为了验证一个函数在某点处的连续性往往只要用到关于上述距离的最基本的性质,而与实数的其他性质无关.
关于多元函数的连续性情形也完全类似.
在此之前,我们一直是依靠几何直觉理解“距离”的概念,从现在开始,我们要抽象出度量和度量空间的概念.

\subsection{度量与度量空间的概念}
\begin{definition}
%@see: 《点集拓扑讲义(第四版)》(熊金城) P45 定义2.1.1
设\(X\)是一个集合,映射\(\rho\colon X \times X \to \mathbb{R}\).
如果对于\(\forall x,y \in X\),
有\(\rho(x,y) \geq 0\)和\(\rho(x,y) = 0 \iff x = y\),
则称“映射\(\rho\)具有\DefineConcept{正定性}”.
\end{definition}
\begin{definition}
%@see: 《点集拓扑讲义(第四版)》(熊金城) P45 定义2.1.1
设\(X\)是一个集合,映射\(\rho\colon X \times X \to \mathbb{R}\).
如果对于\(\forall x,y \in X\),
有\(\rho(x,y) = \rho(y,x)\),
则称“映射\(\rho\)具有\DefineConcept{对称性}”.
\end{definition}
\begin{definition}
%@see: 《点集拓扑讲义(第四版)》(熊金城) P45 定义2.1.1
设\(X\)是一个集合,映射\(\rho\colon X \times X \to \mathbb{R}\).
如果对于\(\forall x,y,z \in X\),
有\(\rho(x,z) \leq \rho(x,y) + \rho(y,z)\),
则称“映射\(\rho\)满足\DefineConcept{三角不等式}”.
\end{definition}
\begin{definition}
%@see: 《数学分析(第7版 第二卷)》(卓里奇) P1 定义1
%@see: 《点集拓扑讲义(第四版)》(熊金城) P45 定义2.1.1
设\(X\)是一个集合.
如果映射\(\rho\colon X \times X\to\mathbb{R}\)具有正定性、对称性,满足三角不等式,
那么称“映射\(\rho\)是集合\(X\)的一个\DefineConcept{度量}(metric)”.
\end{definition}
\begin{definition}
%@see: 《点集拓扑讲义(第四版)》(熊金城) P45 定义2.1.1
设\(\rho\)是集合\(X\)的一个度量,
则称“\((X,\rho)\)是一个\DefineConcept{度量空间}(metric space)”
或“集合\(X\)是一个{对于度量\(\rho\)而言的度量空间}”.
如果前文已经说明了度量\(\rho\),省略它不至于引起混淆,那么可以简称“\(X\)是一个度量空间”.
%@see: https://mathworld.wolfram.com/MetricSpace.html
\end{definition}
\begin{definition}
%@see: 《点集拓扑讲义(第四版)》(熊金城) P45 定义2.1.1
设\((X,\rho)\)是一个度量空间,
那么把实数\(\rho(x,y)\)
称为“在度量空间\((X,\rho)\)中,从点\(x\)到点\(y\)的\DefineConcept{距离}(distance)”
或“对于度量\(\rho\)而言,从点\(x\)到点\(y\)的距离”,
在不致混淆的情况下简称为“从点\(x\)到点\(y\)的距离”.
\end{definition}
%TODO 注意将“度量”与“测度(measure)”作区别
%@see: https://math.stackexchange.com/questions/1402847/
%@see: https://mathworld.wolfram.com/Measure.html
%@see: https://math.hws.edu/eck/metric-spaces/

\begin{remark}
假设一个映射\(f\)既具有对称性,又满足三角不等式,
并且对于\(\forall x,y \in X\)有\(\rho(x,y) = 0 \allowbreak\iff x = y\),
那就不必明说对于\(\forall x,y \in X\)有\(\rho(x,y) \geq 0\),
这是因为只要在三角不等式中,取\(x=z\),
便可得到\(
	0
	= \rho(x,x)
	\leq \rho(x,y) + \rho(y,x)
	= 2 \rho(x,y)
\),
即\(\rho(x,y)\geq0\).
\end{remark}

\begin{remark}
任意给定一个集合,我们总可以给出无穷多个度量.
例如,给定集合\(X\),对于\(x,y \in X\),
只要任意取定\(c\in\mathbb{R}^+\),
然后令\begin{equation*}
	d(x,y) \defeq \left\{ \begin{array}{cl}
		c, & x \neq y, \\
		0, & x=y,
	\end{array} \right.
\end{equation*}
那么映射\(d\)就是集合\(X\)的一个度量.
\end{remark}

\begin{example}
%@see: 《点集拓扑讲义(第四版)》(熊金城) P54 习题 1.
证明:映射\(
	\rho\colon \mathbb{R}\times\mathbb{R} \to \mathbb{R},
	(x,y) \mapsto (x-y)^2
\)不是\(\mathbb{R}\)的度量.
\begin{proof}
显然\(\rho(x,y) = (x-y)^2\)具有对称性和半正定性,
这是因为\begin{gather*}
	\rho(y,x)
	= (y-x)^2
	= (x-y)^2
	= \rho(x,y)
	\geq 0, \\
	\rho(x,y) = (x-y)^2 = 0 \iff x-y = 0 \iff x=y.
\end{gather*}
不过,只要取\(x=-1,y=0,z=1\),
就有\(
	\rho(x,z)
	% = (x-z)^2 = (-1-1)^2 = (-2)^2
	= 4,
	\rho(x,y)
	% = (x-y)^2 = (-1-0)^2 = (-1)^2
	= 1,
	\rho(y,z)
	% = (y-z)^2 = (0-1)^2 = (-1)^2
	= 1
\),
从而\(\rho(x,z) >\allowbreak \rho(x,y) +\allowbreak \rho(y,z)\),
所以\(\rho\)不满足三角不等式,
因此\(\rho\)不是\(\mathbb{R}\)的度量.
\end{proof}
\end{example}

\begin{example}
%@see: 《点集拓扑讲义(第四版)》(熊金城) P54 习题 1.
证明:映射\(
	\rho\colon \mathbb{R}\times\mathbb{R} \to \mathbb{R},
	(x,y) \mapsto \abs{x^2-y^2}
\)不是\(\mathbb{R}\)的度量.
\begin{proof}
显然\(\rho(x,y) = \abs{x^2-y^2}\)满足对称性和三角不等式,
这是因为\begin{gather*}
	\rho(y,x)
	= \abs{y^2-x^2}
	= \abs{x^2-y^2}
	= \rho(x,y), \\
	\rho(x,z)
	= \abs{x^2-z^2}
	= \abs{(x^2-y^2)-(z^2-y^2)}
	\leq \abs{x^2-y^2} + \abs{y^2-z^2}
	= \rho(x,y) + \rho(y,z).
\end{gather*}
不过,只要取\(x=1,y=-1\),
就有\(x \neq y\)
和\(
	\rho(x,y)
	% = \abs{x^2-y^2} = \abs{1^2-(-1)^2} = \abs{1-1}
	= 0
\),
所以\(\rho\)不满足半正定性,
因此\(\rho\)不是\(\mathbb{R}\)的度量.
\end{proof}
\end{example}

\subsection{常见的度量空间}
\begin{example}[实数空间\(\mathbb{R}\)]
%@see: 《点集拓扑讲义(第四版)》(熊金城) P46 例2.1.1
%@see: 《数学分析(第7版 第二卷)》(卓里奇) P2 例1
对于实数集\(\mathbb{R}\),
定义映射\(\rho\colon\mathbb{R}\times\mathbb{R}\to\mathbb{R}\)如下:\begin{equation*}
	\rho(x,y)
	\defeq
	\abs{x-y},
	\quad x,y\in\mathbb{R}.
\end{equation*}
显然\(\rho\)是\(\mathbb{R}\)的一个度量,
因此\((\mathbb{R},\rho)\)是一个度量空间.
特别地,这个度量空间被称为\DefineConcept{实数空间}或\DefineConcept{直线},
称度量\(\rho\)为“\(\mathbb{R}\)的\DefineConcept{通常度量}(usual metric)”.
\end{example}

%@see: 《数学分析(第7版 第二卷)》(卓里奇) P2 例2
我们可以假设一个定义在\([0,+\infty)\)上的非负函数\(f(x)\),
当且仅当\(x=0\)时\(f(x)=0\).
如果函数\(f(x)\)严格上凸,
则对于\(\forall x,y\in\mathbb{R}\),只要取\begin{equation*}
%@see: 《数学分析(第7版 第二卷)》(卓里奇) P2 (2)
	d(x,y)
	\defeq
	f(\abs{x-y}),
\end{equation*}
就得到\(\mathbb{R}\)的一个度量.

\begin{example}
%@see: 《数学分析(第7版 第二卷)》(卓里奇) P2
映射\(d(x,y) \defeq \sqrt{\abs{x-y}}\)是\(\mathbb{R}\)的一个度量.
\end{example}

\begin{example}
%@see: 《数学分析(第7版 第二卷)》(卓里奇) P2
映射\(d(x,y) \defeq \frac{\abs{x-y}}{1+\abs{x-y}}\)是\(\mathbb{R}\)的一个度量.
\end{example}

\begin{example}[欧氏空间\(\mathbb{R}^n\)]\label{example:度量空间.欧氏空间}
%@see: 《点集拓扑讲义(第四版)》(熊金城) P46 例2.1.2
对于实数集\(\mathbb{R}\)的\(n\)重笛卡尔积\(\mathbb{R}^n\),
定义映射\(\rho\colon\mathbb{R}^n\times\mathbb{R}^n\to\mathbb{R}\)如下:\begin{equation*}
	\rho(x,y)
	\defeq \sqrt{\sum_{i=1}^n (x_i-y_i)^2},
	\quad x=(\AutoTuple{x}{n}),y=(\AutoTuple{y}{n})\in\mathbb{R}^n.
\end{equation*}
显然\(\rho\)是\(\mathbb{R}^n\)的一个度量,
因此\((\mathbb{R}^n,\rho)\)是一个度量空间.
特别地,这个度量空间被称为(\(n\)维)\DefineConcept{欧氏空间},
称度量\(\rho\)为\(\mathbb{R}^n\)的\DefineConcept{通常度量}.
2维欧氏空间通常称为欧氏平面.
%@see: https://mathworld.wolfram.com/EuclideanSpace.html
\end{example}

\begin{example}
%@see: 《数学分析(第7版 第二卷)》(卓里奇) P2 例3
%@see: 《数学分析(第7版 第二卷)》(卓里奇) P3 例5
对于\(\mathbb{R}^n\),除了通常度量以外,我们还可以定义\begin{equation*}
%@see: 《数学分析(第7版 第二卷)》(卓里奇) P2 (4)
	d_p(x,y)
	\defeq \left(\sum_{i=1}^n \abs{x_i-y_i}^p\right)^{\frac{1}{p}},
	\quad x=(\AutoTuple{x}{n}),y=(\AutoTuple{y}{n})\in\mathbb{R}^n,
\end{equation*}
其中\(p\geq1\).
利用\hyperref[example:不等式.闵可夫斯基不等式]{闵可夫斯基不等式}%
可以证明\(d_p\)是\(\mathbb{R}^n\)的一个度量,
因此\((\mathbb{R}^n,d_p)\)也是一个度量空间,
把\(d_p\)称为\(\mathbb{R}^n\)的\DefineConcept{闵氏度量}.

%@see: 《数学分析(第7版 第二卷)》(卓里奇) P3 例6
特别地,如果取\(p\to+\infty\),则\begin{equation*}
%@see: 《数学分析(第7版 第二卷)》(卓里奇) P3 (5)
	d_\infty(x,y)
	\defeq \lim_{p\to+\infty} \left(\sum_{i=1}^n \abs{x_i-y_i}^p\right)^{\frac{1}{p}}
	= \max_{1 \leq i \leq n} \abs{x_i-y_i}.
\end{equation*}
\end{example}

\begin{example}
%@see: 《数学分析(第7版 第二卷)》(卓里奇) P3 例7
对于闭区间上的连续函数族\(C[a,b]\),任给其中两个函数\(f,g\),
定义:\begin{equation*}
%@see: 《数学分析(第7版 第二卷)》(卓里奇) P3 (6)
	d(f,g) \defeq \max_{a \leq x \leq b} \abs{f(x)-g(x)}.
\end{equation*}
我们把\(d\)称为\(C[a,b]\)的\DefineConcept{一致度量}%
或\DefineConcept{一致收敛性度量}%
或\DefineConcept{切比雪夫度量}.
在利用多项式代替任意给定函数以所需精度进行近似计算时,
可以用度量\(d\)刻画近似计算的精度.

%@see: 《数学分析(第7版 第二卷)》(卓里奇) P3 例8
对于闭区间上的连续函数族\(C[a,b]\),任给其中两个函数\(f,g\),
我们还可以定义:\begin{equation*}
	d_p(f,g) \defeq \left(
		\int_a^b \abs{f-g}^p(x) \dd{x}
	\right)^{\frac{1}{p}},
	\quad p\geq1.
\end{equation*}
当\(p=1\)时,我们把\(d_p\)称为\DefineConcept{积分度量}.
当\(p=2\)时,我们把\(d_p\)称为\DefineConcept{均方差度量}.
可以证明,当\(p\to+\infty\)时,\(d_\infty\)就是\(C[a,b]\)的一致度量,即\(d = d_\infty\).

我们常把度量空间\((C[a,b],d_p)\)简记为\(C_p[a,b]\),
把度量空间\((C[a,b],d)\)简记为\(C_\infty[a,b]\).
\end{example}

\begin{example}[希尔伯特空间\(\mathbb{H}\)]
%@see: 《点集拓扑讲义(第四版)》(熊金城) P47 例2.1.3
构造由所有的平方收敛的实数序列构成的集合,并记为\begin{equation*}
	\mathbb{H}
	\defeq \Set*{
		x=(\AutoTuple{x}{0})
		\given
		x_i\in\mathbb{R},
		i\in\mathbb{N}^+;
		\sum_{i=1}^\infty x_i^2<\infty
	}.
\end{equation*}
定义映射\(\rho\colon\mathbb{H}\times\mathbb{H}\to\mathbb{R}\)如下:\begin{equation*}
	\rho(x,y) \defeq \sqrt{\sum_{i=1}^\infty (x_i-y_i)^2},
	\quad x=(\AutoTuple{x}{0}),y=(\AutoTuple{y}{0})\in\mathbb{H}.
\end{equation*}
可以证明\(\rho\)是\(\mathbb{H}\)的一个度量,
因此\((\mathbb{H},\rho)\)是一个度量空间.
特别地,这个度量空间被称为\DefineConcept{希尔伯特空间},
称度量\(\rho\)为\(\mathbb{H}\)的\DefineConcept{通常度量}.
%@see: https://mathworld.wolfram.com/HilbertSpace.html
\end{example}

\begin{definition}
%@see: 《点集拓扑讲义(第四版)》(熊金城) P47 例2.1.4
设\((X,\rho)\)是一个度量空间.
如果总有\begin{equation*}
	(\forall x \in X)
	(\exists \delta_x > 0)
	(\forall y \in X - \{x\})
	[\rho(x,y) > \delta_x]
\end{equation*}成立,
则称“度量空间\((X,\rho)\)是\DefineConcept{离散的}”.
\end{definition}
\begin{example}
%@see: 《点集拓扑讲义(第四版)》(熊金城) P47 例2.1.4
设\(X\)是任意一个集合,映射\(\rho\colon X \times X\to\mathbb{R}\)满足\begin{equation*}
	\rho(x,y) \defeq \left\{ \begin{array}{ll}
		0, & x=y, \\
		1, & x\neq y.
	\end{array} \right.
\end{equation*}
容易验证:\(\rho\)是\(X\)的一个离散的度量,度量空间\((X,\rho)\)是离散的.
\end{example}

\begin{example}
%@see: 《点集拓扑讲义(第四版)》(熊金城) P54 习题 2.
证明:只含有限个点的度量空间一定是离散的.
\begin{proof}
设集合\(X\)含有\(n\)个点,
那么\(X\)中一定存在距离最短的两个点,
记\begin{equation*}
	\delta \defeq \min\Set{
		\rho(x_i,x_j)
		\given
		x_i,x_j \in X,
		x_i \neq x_j
	},
\end{equation*}
显然\begin{equation*}
	(\forall x \in X)
	(\exists \epsilon > 0)
	(\forall y \in X - \{x\})
	[
		\rho(x,y) > \delta - \epsilon > 0
	].
\end{equation*}
由此可见,只含有限个点的度量空间一定是离散的.
\end{proof}
\end{example}

\subsection{球形邻域的概念与性质}
\begin{definition}\label{definition:度量空间.球形邻域的概念}
%@see: 《点集拓扑讲义(第四版)》(熊金城) P47 定义2.1.2
%@see: 《数学分析(第7版 第二卷)》(卓里奇) P4 定义2
设\((X,\rho)\)是一个度量空间,\(x \in X\),\(\epsilon > 0\),
那么把集合\begin{equation*}
	\Set{ y \in X \given \rho(x,y) < \epsilon }
\end{equation*}
称为“在度量空间\((X,\rho)\)中,一个以\(x\)为\DefineConcept{中心}、
以\(\epsilon\)为\DefineConcept{半径}的\DefineConcept{球形邻域}%
(a \emph{ball} of \emph{radius} \(\epsilon\) about \(x\))”
或“点\(x\)对于度量\(\rho\)而言的一个\(\epsilon\)-邻域”,
记作\(B(x,\epsilon)\)
或\(B_{\epsilon}(x)\);
不特别强调度量\(\rho\)时,
可以简称为“一个以\(x\)为中心、以\(\epsilon\)为半径的球形邻域”
或“点\(x\)的一个\(\epsilon\)-邻域”;
进一步,不特别强调球形邻域的半径\(\epsilon\)时,
可以简称为“点\(x\)的一个球形邻域”,并简记为\(B(x)\).
\end{definition}

\begin{theorem}\label{theorem:度量空间.球形邻域的性质}
%@see: 《点集拓扑讲义(第四版)》(熊金城) P48 定理2.1.1
设\((X,\rho)\)是一个度量空间,
则\(x\)的球形邻域具有以下基本性质:
\begin{enumerate}
	\item 每一点\(x \in X\)至少有一个球形邻域,
	并且点\(x\)属于它的每一个球形邻域;

	\item 对于点\(x \in X\)的任意两个球形邻域,
	存在\(x\)的球形邻域同时包含于两者;

	\item 如果\(y \in X\)属于\(x \in X\)的某一个球形邻域,
	则\(y\)有一个球形邻域包含于\(x\)的这个球形邻域.
\end{enumerate}
\begin{proof}
\begin{enumerate}
	\item 设\(x \in X\).
	对于每一个实数\(\epsilon>0\),
	\(B(x,\epsilon)\)是\(x\)的一个球形邻域,
	所以\(x\)至少有一个球形邻域.
	由于\(\rho(x,x)=0\),
	所以\(x\)属于它的每一个球形邻域.

	\item 如果\(B(x,\epsilon_1)\)和\(B(x,\epsilon_2)\)是\(x \in X\)的两个球形邻域,
	任意选取实数\(\epsilon>0\),使得\(\epsilon<\min\{\epsilon_1,\epsilon_2\}\),
	则\begin{equation*}
		B(x,\epsilon)
		\subseteq
		B(x,\epsilon_1) \cap B(x,\epsilon_2).
	\end{equation*}

	\item 设\(y \in B(x,\epsilon)\).
	令\(\epsilon_1 = \epsilon - \rho(x,y)\).
	显然\(\epsilon_1>0\).
	如果\(z \in B(y,\epsilon_1)\),
	则\begin{equation*}
		\rho(z,x)
		\leq \rho(z,y) + \rho(y,x)
		< \epsilon_1 + \rho(y,x)
		= \epsilon,
	\end{equation*}
	所以\(z \in B(x,\epsilon)\).
	这证明\(B(y,\epsilon_1) \subseteq B(x,\epsilon)\).
	\qedhere
\end{enumerate}
\end{proof}
\end{theorem}

\begin{theorem}
%@see: 《基础拓扑学讲义》(尤承业) P14 引理
度量空间\((X,\rho)\)的任意两个球形邻域的交集是若干个球形邻域的并集.
\begin{proof}
只要任意取定\(X\)中两点\(x_1\)和\(x_2\),
并任意取定两个正数\(\epsilon_1\)和\(\epsilon_2\),
就可得到两个球形邻域\(B(x_1,\epsilon_1)\)和\(B(x_2,\epsilon_2)\),
再令\(U \defeq B(x_1,\epsilon_1) \cap B(x_2,\epsilon_2)\),
那么对于任意\(x \in U\),
根据\hyperref[definition:度量空间.球形邻域的概念]{球形邻域的定义}有\(
	\epsilon_1 > \rho(x,x_1),
	\epsilon_2 > \rho(x,x_2)
\),
若记\(
	\epsilon_x \defeq \min\{
		\epsilon_1 - \rho(x,x_1),
		\epsilon_2 - \rho(x,x_2)
	\}
\),
则有\(B(x,\epsilon_x) \subseteq U\).
于是\begin{equation*}
	U = \bigcup_{x \in U} B(x,\epsilon_x).
	\qedhere
\end{equation*}
\end{proof}
\end{theorem}

\subsection{度量空间与极限的联系}
\begin{definition}
%@see: 《Real Analysis Modern Techniques and Their Applications Second Edition》(Folland) P14
设\(\{x_n\}\)是度量空间\((X,\rho)\)中的一个序列.
若\begin{equation*}
	\lim_{n\to\infty} \rho(x_n,x) = 0,
\end{equation*}
则称“在度量空间\((X,\rho)\)中,序列\(\{x_n\}\) \DefineConcept{收敛于} 点\(x\)%
(\(\{x_n\}\) \emph{converges} to \(x\))”,
记作\(\lim_{n\to\infty} x_n = x\);
不特别强调度量\(\rho\)时,可以简称为“\(\{x_n\}\)收敛于\(x\)”.
\end{definition}

\subsection{开集的概念与性质}
\begin{definition}\label{definition:度量空间.开集的概念}
%@see: 《点集拓扑讲义(第四版)》(熊金城) P48 定义2.1.3
%@see: 《数学分析(第7版 第二卷)》(卓里奇) P5 定义3
设\(A\)是度量空间\(X\)的一个子集.
如果\(A\)中的每一个点都有一个球形邻域包含于\(A\),
即\begin{equation*}
	(\forall a \in A)
	(\exists\epsilon>0)
	[B(a,\epsilon) \subseteq A],
\end{equation*}
则称“\(A\)是度量空间\(X\)中的一个\DefineConcept{开集}(open set)”.
\end{definition}

\begin{definition}
%@see: 《数学分析(第7版 第二卷)》(卓里奇) P5 定义4
设\(A\)是度量空间\(X\)的一个子集.
如果\(A\)的补集\(X-A\)是度量空间\(X\)中的一个开集,
则称“\(A\)是度量空间\(X\)中的一个\DefineConcept{闭集}”.
\end{definition}

\begin{example}
%@see: 《点集拓扑讲义(第四版)》(熊金城) P48 例2.1.5
实数空间\(\mathbb{R}\)中,所有的开区间,不论是有限的还是无限的,都是开集;
闭区间或半开半闭区间都不是\(\mathbb{R}\)中的开集.
\end{example}

\begin{theorem}\label{theorem:度量空间.开集的性质}
%@see: 《点集拓扑讲义(第四版)》(熊金城) P49 定理2.1.2
%@see: 《数学分析(第7版 第二卷)》(卓里奇) P5 命题1
度量空间\(X\)中的开集具有以下性质:
\begin{enumerate}
	\item 集合\(X\)本身和空集\(\emptyset\)都是开集;
	\item 任意两个开集的交也是一个开集;
	\item 任意一个开集族的并是一个开集;
	\item 任意一个球形邻域都是开集.
\end{enumerate}
\begin{proof}
\begin{enumerate}
	\item 根据\cref{theorem:度量空间.球形邻域的性质},
	\(X\)中的每一个元素\(x\)都有一个球形邻域,
	这个球形邻域当然包含在\(X\)中,
	所以\(X\)满足开集的条件.
	空集\(\emptyset\)中不含任何点,
	也自然地可以认为它满足开集的条件.

	\item 设\(U,V\)都是\(X\)中的开集.
	如果\(x \in U \cap V\),
	则存在\(x\)的一个球形邻域\(B(x,\epsilon_1)\)包含于\(U\),
	也存在\(x\)的一个球形邻域\(B(x,\epsilon_2)\)包含于\(V\).
	根据\cref{theorem:度量空间.球形邻域的性质},
	\(x\)有一个球形邻域\(B(x,\epsilon)\)
	同时包含于\(B(x,\epsilon_1)\)和\(B(x,\epsilon_2)\),
	因此\begin{equation*}
		B(x,\epsilon)
		\subseteq
		B(x,\epsilon_1) \cap B(x,\epsilon_2)
		\subseteq
		U \cap V.
	\end{equation*}
	由于\(U \cap V\)中的每一点都是一个球形邻域包含于\(U \cap V\),
	所以\(U \cap V\)是一个开集.

	\item 设\(\sfA\)是一个由\(X\)中的开集构成的子集族.
	如果\(x \in \bigcup \sfA\),
	则存在\(A \in \sfA\)使得\(x \in A\).
	由于\(A\)是一个开集,
	所以\(x\)有一个球形邻域包含于\(A\),
	显然这个球形邻域也包含于\(\bigcup \sfA\).
	这证明\(\bigcup \sfA\)是\(X\)中的一个开集.
	\qedhere
\end{enumerate}
\end{proof}
\end{theorem}

根据\cref{theorem:度量空间.球形邻域的性质} 可以得知,
每一个球形邻域都是开集.

有时候为了方便讨论问题,我们将球形邻域的概念稍稍作一点推广.
\begin{definition}\label{definition:度量空间.邻域的概念}
%@see: 《点集拓扑讲义(第四版)》(熊金城) P50 定义2.1.4
设\(x\)是度量空间\(X\)中的一个点,集合\(U \subseteq X\).
如果存在一个开集\(V\)满足条件\(x \in V \subseteq U\),
就称“\(U\)是点\(x\)(在度量空间\(X\)中)的一个\DefineConcept{邻域}”.
%@see: 《数学分析(第7版 第二卷)》(卓里奇) P5 定义5
%@see: 《数学分析(第7版 第二卷)》(卓里奇) P5 定义6
\end{definition}

下面这个定理为邻域的定义提供了一个等价的说法,并且表明从球形邻域推广到邻域是自然的事情.
\begin{theorem}\label{theorem:度量空间.邻域的判定}
%@see: 《点集拓扑讲义(第四版)》(熊金城) P50 定理2.1.3
设\(x\)是度量空间\(X\)中的一个点,
则“\(X\)的子集\(U\)是\(x\)的一个邻域”的充分必要条件是:
\(x\)有某一个球形邻域包含于\(U\).
\begin{proof}
如果\(U\)是点\(x\)的一个邻域,
根据邻域的定义,存在开集\(V\),使得\(x \in V \subseteq U\).
又根据开集的定义,\(x\)有一个球形邻域包含于\(V\),
从而这个球形邻域也就包含于\(U\).
这证明\(U\)满足定理的条件.

反之,如果\(U\)满足定理中的条件,
由于球形邻域都是开集,
因此\(U\)是\(x\)的邻域.
\end{proof}
\end{theorem}

\subsection{集合的直径,集合的有界性}
\begin{definition}
%@see: 《数学分析(第7版 第一卷)》(卓里奇) P343 定义9
%@see: 《数学分析(第7版 第一卷)》(卓里奇) P343 定义10
%@see: 《点集拓扑讲义(第四版)》(熊金城) P219 定义7.5.1
设\((X,\rho)\)是一个度量空间,\(A \subseteq X\).
定义:\begin{equation*}
	\diam A \defeq \sup_{x,y \in A} \rho(x,y),
\end{equation*}
称其为“集合\(A\)(在度量空间\((X,\rho)\)中)的\DefineConcept{直径}(diameter)”.
\end{definition}
\begin{definition}
设\((X,\rho)\)是一个度量空间,\(A \subseteq X\).
若\begin{equation*}
	(\exists M>0)
	(\forall x,y \in A)
	[\rho(x,y) < M],
\end{equation*}
则称“集合\(A\)(在度量空间\((X,\rho)\)中)是\DefineConcept{有界的}(bounded)”,
记作\(\diam A < \infty\).
\end{definition}
\begin{definition}
设\((X,\rho)\)是一个度量空间,\(A \subseteq X\).
若\begin{equation*}
	(\forall M>0)
	(\exists x,y \in A)
	[\rho(x,y) > M],
\end{equation*}
则称“集合\(A\)(在度量空间\((X,\rho)\)中)是\DefineConcept{无界的}(unbounded)”,
记作\(\diam A = \infty\).
\end{definition}

\subsection{度量空间之间的连续映射}
现在我们把分析学中的连续函数的概念推广为度量空间之间的连续映射.

\begin{definition}
%@see: 《点集拓扑讲义(第四版)》(熊金城) P50 定义2.1.5
设\(X,Y\)都是度量空间,
映射\(f\colon X \to Y\),
点\(x_0 \in X\).
如果对于\(f(x_0)\)的任何一个球形邻域\(B(f(x_0),\epsilon)\),
存在\(x_0\)的某一个球形邻域\(B(x_0,\delta)\),
使得\begin{equation*}
	f(B(x_0,\delta))
	\subseteq
	B(f(x_0),\epsilon),
\end{equation*}
则称“映射\(f\)在点\(x_0\) \DefineConcept{连续}”.
\end{definition}

\begin{definition}\label{definition:度量空间.连续映射的概念}
%@see: 《点集拓扑讲义(第四版)》(熊金城) P50 定义2.1.5
设\(X,Y\)都是度量空间,
映射\(f\colon X \to Y\).
如果\begin{equation*}
	(\forall x \in X)
	[\text{$f$在点$x$连续}],
\end{equation*}
则称“\(f\)是一个\DefineConcept{连续映射}”.
\end{definition}

\begin{remark}
\cref{definition:度量空间.连续映射的概念} 是分析学中函数连续性定义的纯粹形式推广.
之所以这样说,是因为如果\(\rho\)和\(\sigma\)分别是\(X\)和\(Y\)的度量,
则“映射\(f\)在点\(x_0\)连续”可以说成是\begin{equation*}
	(\forall\epsilon>0)
	(\exists\delta>0)
	(\forall x \in X)
	[
		\rho(x,x_0)<\delta
		\implies
		\sigma(f(x),f(x_0))<\epsilon
	].
\end{equation*}
\end{remark}

下面这个定理是把度量空间和度量空间之间的连续映射的概念
推广为拓扑空间和拓扑空间之间的连续映射的出发点.
\begin{theorem}\label{theorem:度量空间.度量空间下的连续映射与邻域的联系}
%@see: 《点集拓扑讲义(第四版)》(熊金城) P50 定理2.1.4
设\(X\)、\(Y\)是两个度量空间.
映射\(f\colon X \to Y\).
取\(x_0 \in X\).
那么\begin{equation*}
	\text{\(f\)在点\(x_0\)连续}
	\iff
	\text{\(f(x_0)\)的每一个邻域的原像是\(x_0\)的一个邻域},
\end{equation*}\begin{equation*}
	\text{\(f\)是连续映射}
	\iff
	\text{\(Y\)中的每一个开集的原像是\(X\)中的一个开集}.
\end{equation*}
\begin{proof}
先证“\(\text{\(f\)在点\(x_0\)连续}
\implies
\text{\(f(x_0)\)的每一个邻域的原像是\(x_0\)的一个邻域}\)”.
假设\(f\)在点\(x_0\)连续.
令\(U\)为\(f(x_0)\)的一个邻域.
根据\cref{theorem:度量空间.邻域的判定},
\(f(x_0)\)有一个球形邻域\(B(f(x_0),\epsilon)\)包含于\(U\).
由于\(f\)在点\(x_0\)连续,
所以\(x_0\)有一个球形邻域\(B(x_0,\delta)\)
使得\(f(B(x_0,\delta)) \subseteq B(f(x_0),\epsilon)\).
然而,\(f^{-1}(B(f(x_0),\epsilon)) \subseteq f^{-1}(U)\),
所以\(B(x_0,\delta) \subseteq f^{-1}(U)\).
这证明\(f^{-1}(U)\)是\(x_0\)的一个邻域.

再证“\(\text{\(f(x_0)\)的每一个邻域的原像是\(x_0\)的一个邻域}
\implies
\text{\(f\)在点\(x_0\)连续}\)”.
假设\(f(x_0)\)的每一个邻域的原像是\(x_0\)的一个邻域.
任意给定\(f(x_0)\)的一个邻域\(B(f(x_0),\epsilon)\),
则\(f^{-1}(B(f(x_0),\epsilon))\)是\(x_0\)的一个邻域.
根据\cref{theorem:度量空间.邻域的判定},
\(x_0\)有一个球形邻域\(B(x_0,\delta)\)
包含于\(f^{-1}(B(f(x_0),\epsilon))\).
因此\(f(B(x_0,\delta)) \subseteq B(f(x_0),\epsilon)\).
这就证明\(f\)在点\(x_0\)连续.

接下来证“\(\text{\(f\)是连续映射}
\implies
\text{\(Y\)中的每一个开集的原像是\(X\)中的一个开集}\)”.
假设\(f\)是连续映射.
令\(V\)是\(Y\)中的一个开集,
又令\(U = f^{-1}(V)\).
对于每一个\(x \in U\),我们有\(f(x) \in V\).
由于\(V\)是一个开集,
所以\(V\)是\(f(x)\)的一个邻域.
由于\(f\)在每一点处都连续,
故根据本定理第一个结论,
\(U\)是\(x\)的一个邻域.
于是有包含\(x\)的某一个开集\(U_x\)使得\(U_x \subseteq U\).
易见\(U = \bigcup_{x \in U} U_x\).
由于每一个\(U_x\)都是开集,
根据\cref{theorem:度量空间.开集的性质},
\(U\)是一个开集.

最后证“\(\text{\(Y\)中的每一个开集的原像是\(X\)中的一个开集}
\implies
\text{\(f\)是连续映射}\)”.
假设\(Y\)中的每一个开集的原像是\(X\)中的一个开集.
对于任意\(x \in X\),
设\(U\)是\(f(x)\)的一个邻域,
即存在包含\(f(x)\)的一个开集\(V \subseteq U\).
从而\(x \in f^{-1}(V) \subseteq f^{-1}(U)\).
根据假设,\(f^{-1}(V)\)是一个开集,
所以\(f^{-1}(U)\)是\(x\)的一个邻域,
因此对于\(x\)而言,
\(f(x)\)的每一个邻域的原像是\(x\)的一个邻域,
因此根据本定理第一个结论可知,
\(f\)在点\(x\)连续.
由于点\(x\)是任意选取的,所以\(f\)是一个连续映射.
\end{proof}
\end{theorem}

%现在我们就明白了,对于积分的每一种定义都基于一种度量:
%黎曼积分基于若尔当度量,
%勒贝格积分基于勒贝格度量.

\begin{example}
%@see: 《点集拓扑讲义(第四版)》(熊金城) P54 习题 3.
设\((X,\rho)\)是一个离散的度量空间.
证明:\begin{itemize}
	\item \(X\)的每一个子集都是开集;
	\item 如果\(Y\)也是一个度量空间,则任意一个从\(X\)到\(Y\)的映射都是连续的.
\end{itemize}
\begin{proof}
因为\((X,\rho)\)是一个离散的度量空间,
那么由定义有\begin{equation*}
	(\forall x \in X)
	(\exists \delta_x > 0)
	(\forall y \in X - \{x\})
	[\rho(x,y) > \delta_x].
\end{equation*}

任取\(X\)的一个子集\(A\),
便有\begin{equation*}
	(\forall x \in A)
	[B(x,\epsilon_x) = \{x\} \subseteq A],
\end{equation*}
这就说明\(X\)的每一个子集都是开集.

设映射\(f\colon X \to Y\),
\(\sigma\)是\(Y\)的度量.
在\(X\)中任取一点\(x_0\),
对于任意正数\(\epsilon\),
只要取\(\delta = \delta_{x_0}\),
就会使得\begin{align*}
	f(B(x_0,\delta))
	&= f(\{x_0\})
	= \{f(x_0)\}
	= \Set{ y \in Y \given \sigma(y,f(x_0)) = 0 } \\
	&\subseteq
	B(f(x_0),\epsilon)
	= \Set{ y \in Y \given \sigma(y,f(x_0)) < \epsilon },
\end{align*}
这就说明映射\(f\)在点\(x_0\)连续.
\end{proof}
\end{example}

\begin{definition}
%@see: 《点集拓扑讲义(第四版)》(熊金城) P54 习题 4.
设\(\rho_1,\rho_2\)都是集合\(X\)的度量,\(A \subseteq X\).
若\begin{equation*}
	\text{\(A\)是度量空间\((X,\rho_1)\)中的开集}
	\iff
	\text{\(A\)是度量空间\((X,\rho_2)\)中的开集},
\end{equation*}
则称“度量\(\rho_1\)与\(\rho_2\)是\DefineConcept{等价的}”.
\end{definition}

\begin{example}
%@see: 《点集拓扑讲义(第四版)》(熊金城) P54 习题 4.
设\(X\)是一个非空集合,\(Y\)是一个度量空间,
\(f\)是从\(X\)到\(Y\)的一个映射,
映射\(\rho_1,\rho_2\)都是\(X\)的度量,
且\(\rho_1\)与\(\rho_2\)是等价的.
证明:\(f\)对于度量\(\rho_1\)而言是连续的,
当且仅当\(f\)对于度量\(\rho_2\)而言是连续的.
%TODO
\end{example}

\begin{example}
%@see: 《点集拓扑讲义(第四版)》(熊金城) P54 习题 5.
定义映射\(\rho_1\colon \mathbb{R}^2 \times \mathbb{R}^2 \to \mathbb{R}\),
使之满足\begin{equation*}
	\rho_1(x,y) \defeq \max\{ \abs{x_1-y_1}, \abs{x_2-y_2} \},
	\quad x=(x_1,x_2),y=(y_1,y_2).
\end{equation*}
定义映射\(\rho_2\colon \mathbb{R}^2 \times \mathbb{R}^2 \to \mathbb{R}\),
使之满足\begin{equation*}
	\rho_2(x,y) \defeq \abs{x_1-y_1} + \abs{x_2-y_2},
	\quad x=(x_1,x_2),y=(y_1,y_2).
\end{equation*}
证明:\(\rho_1\)、\(\rho_2\)和\(\mathbb{R}^2\)的通常度量\(\rho\)是等价的.
%TODO proof
\end{example}

\begin{example}
%@see: 《点集拓扑讲义(第四版)》(熊金城) P54 习题 6.
定义映射\(f\colon \mathbb{R}^2 \to \mathbb{R}\),使之满足\begin{equation*}
	f(x) \defeq \max\{ x_1, x_2 \},
	\quad x=(x_1,x_2).
\end{equation*}
证明:\(f\)是连续映射.
\begin{proof}
要证\(f\)是连续映射,
只需证对于\(\mathbb{R}^2\)中任意一点\(x\)和任意\(\epsilon > 0\),存在\(\delta > 0\),
当\(\mathbb{R}^2\)中任意一点\(y = (y_1,y_2)\)
满足\(\sqrt{(x_1-y_1)^2+(x_2-y_2)^2} < \delta\)时,
就有\(\abs{f(x) - f(y)} < \epsilon\)成立.
%@credit: {ce603838-a24d-4616-9395-d7b223e8cb72} 提起\(\max\)与\(\abs\)的关系
由\cref{equation:绝对值函数.两个数的最大值} \begin{equation*}
	\max\{a,b\} = \frac{a+b}{2} + \frac{\abs{a-b}}{2}
\end{equation*}
\def\ExpandMax#1#2{\frac{#1+#2}{2} + \frac{\abs{#1-#2}}{2}}
可知\begin{align*}
	\abs{f(x) - f(y)}
	&= \abs{
		\left( \ExpandMax{x_1}{x_2} \right)
		-
		\left( \ExpandMax{y_1}{y_2} \right)
	} \\
	&= \abs{
		\frac{
			(x_1-y_1) + (x_2-y_2)
			+ \abs{x_1-x_2} - \abs{y_1-y_2}
		}{2}
	} \\
	%\cref{theorem:不等式.三角不等式1}
	&\leq \abs{\frac{x_1-y_1}{2}} + \abs{\frac{x_2-y_2}{2}} + \frac{\abs{\abs{x_1-x_2} - \abs{y_1-y_2}}}{2} \\
	%\cref{theorem:不等式.三角不等式2}
	&\leq \abs{\frac{x_1-y_1}{2}} + \abs{\frac{x_2-y_2}{2}} + \frac{\abs{(x_1-x_2)-(y_1-y_2)}}{2} \\
	&= \frac{\abs{x_1-y_1}}{2} + \frac{\abs{x_2-y_2}}{2} + \frac{\abs{(x_1-y_1)-(x_2-y_2)}}{2} \\
	%\cref{theorem:不等式.三角不等式1}
	&\leq \frac{\abs{x_1-y_1}}{2} + \frac{\abs{x_2-y_2}}{2} + \frac{\abs{x_1-y_1}}{2} + \frac{\abs{x_2-y_2}}{2} \\
	&= \abs{x_1-y_1} + \abs{x_2-y_2}
	%\cref{theorem:不等式.基本不等式n算术平均数与平方平均数}
	\leq 2\sqrt{\frac{(x_1-y_1)^2+(x_2-y_2)^2}{2}}
	< \frac{2\delta}{\sqrt2},
\end{align*}
所以只要让\(\delta\)满足\(\epsilon = \sqrt2 \delta\),
或者说令\(\delta = \frac{\epsilon}{\sqrt2}\),
那么当\(\sqrt{(x_1-y_1)^2+(x_2-y_2)^2} < \delta\)时,
就有\(\abs{f(x) - f(y)} < \epsilon\)成立.
\end{proof}
\end{example}

\begin{example}
%@see: 《点集拓扑讲义(第四版)》(熊金城) P54 习题 6.
定义映射\(f\colon \mathbb{R}^2 \to \mathbb{R}\),使之满足\begin{equation*}
	f(x) \defeq x_1 + x_2,
	\quad x=(x_1,x_2).
\end{equation*}
证明:\(f\)是连续映射.
\begin{proof}
由于\begin{align*}
	\abs{f(x) - f(y)}
	&= \abs{
		(x_1 + x_2) - (y_1 + y_2)
	}
	= \abs{(x_1 - y_1) + (x_2 - y_2)} \\
	%\cref{theorem:不等式.三角不等式1}
	&\leq \abs{x_1 - y_1} + \abs{x_2 - y_2}
	%\cref{theorem:不等式.基本不等式n算术平均数与平方平均数}
	\leq \sqrt2 \sqrt{(x_1-y_1)^2 + (x_2-y_2)^2},
\end{align*}
所以对于\(\mathbb{R}^2\)中任意一点\(x = (x_1,x_2)\)和任意正数\(\epsilon\),
取\(\delta = \frac{\epsilon}{\sqrt2}\),
当\(\mathbb{R}^2\)中任意一点\(y = (y_1,y_2)\)
满足\(\sqrt{(x_1-y_1)^2 + (x_2-y_2)^2} < \delta\)时,
就有\(\abs{f(x) - f(y)} < \epsilon\)成立,
这就说明\(f\)是连续映射.
\end{proof}
\end{example}

\begin{example}
%@see: 《点集拓扑讲义(第四版)》(熊金城) P54 习题 7.
设\((X,\rho)\)是一个度量空间.
定义映射\(\rho_1\colon X \times X \to \mathbb{R}\),使之满足\begin{equation*}
	\rho_1(x,y) \defeq \frac{\rho(x,y)}{1+\rho(x,y)}.
\end{equation*}
定义映射\(\rho_2\colon X \times X \to \mathbb{R}\),使之满足\begin{equation*}
	\rho_2(x,y) \defeq \left\{ \begin{array}{cl}
		\rho(x,y), & \rho(x,y) \leq 1, \\
		1, & \rho(x,y) > 1.
	\end{array} \right.
\end{equation*}
证明:\(\rho_1,\rho_2,\rho\)是等价的.
%TODO proof
\end{example}


\begingroup
\def\T{{\mathfrak T}}%拓扑,\(X\)的全体开集
\def\oT{\overline{\T}}%\(X\)的全体闭集

\section{拓扑空间}
从\cref{theorem:度量空间.度量空间下的连续映射与邻域的联系} 可以看出:
度量空间之间的一个映射是否是连续的,或者在某一点处是否是连续的,
本质上只与度量空间中的开集有关(这是因为邻域是通过开集定义的).
这就导致我们可以抛弃度量这个概念,
参照\hyperref[theorem:度量空间.开集的性质]{度量空间中开集的基本性质},
抽象出拓扑空间和拓扑空间之间的连续映射的概念.
于是,\cref{theorem:度量空间.度量空间下的连续映射与邻域的联系}
成为了我们把度量空间和度量空间之间的连续映射的概念推广为拓扑空间和拓扑空间之间的连续映射的出发点.

\subsection{拓扑与拓扑空间的概念}
\begin{definition}\label{definition:拓扑学.开集公理定义的拓扑空间}
%@see: 《点集拓扑讲义(第四版)》(熊金城) P55 定义2.2.1
已知非空集合\(X\).
\(\T\)\footnote{\(\T\)是德文尖角体(Fraktur)的拉丁字母T.}
是\(X\)的一个子集族,
即\(\T \subseteq \Powerset X\).
若有\begin{itemize}
	\item \(\emptyset,X \in \T\);
	\item \(\T\)的有限交仍然属于\(\T\)%
	\footnote{有时候也将这个条件说成是“\(\T\)对有限交\DefineConcept{封闭}”.},
	即\(A,B \in \T \implies A \cap B \in \T\);
	\item \(\T\)的任意并仍然属于\(\T\),
	即\(\T_1 \subseteq \T \implies \bigcup_{A \in \T_1} A \in \T\);
\end{itemize}
则称“\(\T\)是\(X\)的一个\DefineConcept{拓扑}”
“集合\(X\)是一个(相对于拓扑\(\T\)而言的)\DefineConcept{拓扑空间}(topological space)”,
记作\((X,\T)\).
\(\T\)的任一元素都称为“拓扑空间\((X,\T)\)中的一个\DefineConcept{开集}(open set)”.
因此我们又把\(\T\)称为“集合\(X\)的一个\DefineConcept{开集族}(family of open sets)”.
%@see: https://mathworld.wolfram.com/TopologicalSpace.html
%@see: https://mathworld.wolfram.com/OpenSet.html
\end{definition}
如果我们将\cref{definition:拓扑学.开集公理定义的拓扑空间} 中的三个条件
与\cref{theorem:度量空间.开集的性质} 的三个结论对照一下,
并将“\(U\)属于\(\T\)”读作“\(U\)是一个开集”,
便会发现两者实际上是一样的.

只要经过简单的归纳立即可见,
\cref{definition:拓扑学.开集公理定义的拓扑空间} 中的第二个条件蕴含着以下结论:\begin{equation*}
	\AutoTuple{A}{n}\in\T\ (n\geq1)
	\implies
	A_1 \cap A_2 \cap \dotsb \cap A_n \in \T.
\end{equation*}

此外,如果在\cref{definition:拓扑学.开集公理定义的拓扑空间} 中的第三个条件中令\(\T_1=\emptyset\),
就会得到\(\emptyset = \bigcup_{A\in\T_1} A \in \T\),
而这一点在第一个条件中已经做了规定.
因此我们在验证任意集合\(X\)的一个子集族是否可以是\(X\)的一个拓扑,
在验证第三个条件是否满足时,总可以假定\(\T_1\neq\emptyset\).

\subsection{度量空间是特殊的拓扑空间}
现在首先将度量空间纳入拓扑空间的范畴.

%@see: 《点集拓扑讲义(第四版)》(熊金城) P55 定义2.2.2
设\((X,\rho)\)是一个度量空间.
%@see: 《基础拓扑学讲义》(尤承业) P14 引理
我们可以证明:\((X,\rho)\)的任意两个球形邻域的交集是若干个球形邻域的并集.
任意取定\(x_1,x_2 \in X\)和\(\epsilon_1,\epsilon_2>0\),
就可得到两个球形邻域\(B(x_1,\epsilon_1)\)和\(B(x_2,\epsilon_2)\).
令\(U=B(x_1,\epsilon_1) \cap B(x_2,\epsilon_2)\),
根据球形邻域的定义有\begin{equation*}
	(\forall x \in U)
	[
		\epsilon_1 - \rho(x,x_1) > 0
		\land
		\epsilon_2 - \rho(x,x_2) > 0
	].
\end{equation*}
若记\(\epsilon_x = \min\{
	\epsilon_1 - \rho(x,x_1),
	\epsilon_2 - \rho(x,x_2)
\}\),
则有\begin{equation*}
	(\forall x \in U)
	[B(x,\epsilon_x) \subseteq U].
\end{equation*}
于是\begin{equation*}
	U = \bigcup_{x \in U} B(x,\epsilon_x).
\end{equation*}

%@see: 《基础拓扑学讲义》(尤承业) P14 命题1.1
现在记\begin{equation*}
	\T_\rho = \Set{
		U
		\given
		\text{\(U\)是若干个球形邻域的并集}
	}
	\cup
	\{\emptyset,X\}.
\end{equation*}
我们来证明:\(\T_\rho\)是\(X\)的一个拓扑.
设\(U,V \in \T_\rho\),
记\begin{equation*}
	U = \bigcup_\alpha B(x_\alpha,\epsilon_\alpha), \qquad
	V = \bigcup_\beta B(x_\beta,\epsilon_\beta),
\end{equation*}
则\begin{align*}
	U \cap V
	&= \left(
		\bigcup_\alpha B(x_\alpha,\epsilon_\alpha)
	\right)
	\cap
	\left(
		\bigcup_\beta B(x_\beta,\epsilon_\beta)
	\right) \\
	&= \bigcup_{\alpha,\beta} \left[
		B(x_\alpha,\epsilon_\alpha)
		\cap
		B(x_\beta,\epsilon_\beta)
	\right].
\end{align*}
由于\begin{equation*}
	(\forall \alpha,\beta)
	[
		B(x_\alpha,\epsilon_\alpha)
		\cap
		B(x_\beta,\epsilon_\beta)
		\in
		\T_\rho
	],
\end{equation*}
那么有\(U \cap V \in \T_\rho\).

因此,假设\(\T_\rho\)是由\(X\)中的所有开集构成的集族,
根据\cref{theorem:度量空间.开集的性质} 可知\((X,\T_\rho)\)是\(X\)的一个拓扑.
我们称\(\T_\rho\)为“\(X\)的由度量\(\rho\)诱导出来的拓扑”.
此外,我们约定:
如果没有另外说明,
当我们提到“度量空间\((X,\rho)\)的拓扑”时,
指的就是拓扑\(\T_\rho\);
在称“度量空间\((X,\rho)\)是拓扑空间”时,
指的就是拓扑空间\((X,\T_\rho)\).

因此,实数空间\(\mathbb{R}\)、
\(n\)维欧氏空间\(\mathbb{R}^n\)(特别是欧氏平面\(\mathbb{R}^2\))
和希尔伯特空间\(\mathbb{H}\)都可以叫做拓扑空间,
它们各自的拓扑分别是由各自的通常度量所诱导出来的拓扑.

\subsection{闭集,闭集族}
\begin{definition}\label{definition:拓扑空间.闭集的定义}
%@see: 《基础拓扑学讲义》(尤承业) P15 定义1.2
%@see: 《点集拓扑讲义(第四版)》(熊金城) P69 定理2.4.2
拓扑空间\((X,\T)\)中任一开集\(A\)的补集\(X-A\)称为
“拓扑空间\((X,\T)\)中的一个\DefineConcept{闭集}(closed set)”.
\((X,\T)\)的全体闭集,
称为“\(X\)的一个\DefineConcept{闭集族}(family of closed sets)”,
记为\(\oT\).
%@see: https://mathworld.wolfram.com/ClosedSet.html
\end{definition}

\begin{property}
%@see: 《基础拓扑学讲义》(尤承业) P15 命题1.2
拓扑空间的闭集满足:\begin{itemize}
	\item \(X\)与\(\emptyset\)都是闭集.
	\item 任意多个闭集的交集是闭集.
	\item 有限个闭集的并集是闭集.
\end{itemize}
\begin{proof}
利用\cref{definition:拓扑学.开集公理定义的拓扑空间}
以及\cref{equation:集合论.集合代数公式4-3,equation:集合论.集合代数公式4-4}
易证.
\end{proof}
\end{property}

\begin{theorem}
设\((X,\T)\)是一个拓扑空间,
则\begin{gather*}
	\oT = \Set{ F \given X-F \in \T }, \\
	\T = \Set{ G \given X-G \in \oT }.
\end{gather*}
\end{theorem}
以后,当我们提到“拓扑空间\(X\)”,就默认配有对应的开集族\(\T\)和闭集族\(\oT\).

\subsection{常见的拓扑空间}
度量空间是拓扑空间中最为重要的一类.
于此,我们再举出一些拓扑空间的例子.

\begin{example}[平庸空间]
%@see: 《点集拓扑讲义(第四版)》(熊金城) P56 例2.2.1
设\(X\)是一个集合.
令\(\T=\{X,\emptyset\}\).
容易验证,\(\T\)是\(X\)的一个拓扑,称其为\(X\)的\DefineConcept{平庸拓扑};
称拓扑空间\((X,\T)\)为一个\DefineConcept{平庸空间}.
在平庸空间\((X,\T)\)中,有且仅有两个开集,即\(X\)本身和空集\(\emptyset\).
\end{example}

\begin{example}[离散空间]
%@see: 《点集拓扑讲义(第四版)》(熊金城) P57 例2.2.2
设\(X\)是一个集合.
令\(\T=\Powerset X\).
容易验证\(\T\)是\(X\)的一个拓扑,称其为\(X\)的\DefineConcept{离散拓扑};
称拓扑空间\((X,\T)\)为一个\DefineConcept{离散空间}.
在离散空间\((X,\T)\)中,\(X\)的每一个子集都是开集.
%@see: https://mathworld.wolfram.com/DiscreteTopology.html
\end{example}

\begin{example}\label{example:拓扑学.常见的拓扑空间3}
%@see: 《点集拓扑讲义(第四版)》(熊金城) P57 例2.2.3
设\(X = \{a,b,c\}\).
令\begin{equation*}
	\T = \{
		\emptyset,
		\{a\},
		\{a,b\},
		\{a,b,c\}
	\}.
\end{equation*}
容易验证\(\T\)是\(X\)的一个拓扑,
因此\((X,\T)\)是一个拓扑空间,
但这个拓扑空间既不是平庸空间也不是离散空间.
\end{example}

\begin{example}[有限补空间]
%@see: 《点集拓扑讲义(第四版)》(熊金城) P57 例2.2.4
%@see: 《基础拓扑学讲义》(尤承业) P13 例1
设\(X\)是一个集合.
令\begin{equation*}
	\T = \Set{
		X-U
		\given
		\text{\(U\)是\(X\)的一个有限子集}
	}
	\cup
	\{\emptyset\}.
\end{equation*}

因为\(X \subseteq X\),\(X - X = \emptyset\)是\(X\)的一个有限子集,
所以\(X \in \T\).
再根据这里对\(\T\)的定义,还有\(\emptyset \in \T\).

设\(A,B\in\T\).
如果\(A\)和\(B\)之中有一个是空集,
则\(A \cap B = \emptyset \in \T\);
如果\(A\)和\(B\)都不是空集,
\(X - (A \cap B) = (X - A) \cup (X - B)\)是\(X\)的一个有限子集,
所以\(A \cap B \in \T\).

设\(\T_1 \subseteq \T\),令\(\T_2 = \T_1 - \{ \emptyset \}\).
显然有\begin{equation*}
	\bigcup_{A \in \T_1} A
	= \bigcup_{A \in \T_2} A.
\end{equation*}
如果\(\T_2 = \emptyset\),
则\begin{equation*}
	\bigcup_{A \in \T_1} A
	= \bigcup_{A \in \T_2} A
	= \emptyset \in \T;
\end{equation*}
如果\(\T_2 \neq \emptyset\),
则对\(\forall A_0 \in \T_2\),
\begin{equation*}
	X - \bigcup_{A \in \T_1} A
	= X - \bigcup_{A \in \T_2} A
	= \bigcap_{A \in \T_2} (X - A)
	\subseteq X - A_0
\end{equation*}是\(X\)的一个有限子集;
所以\(\bigcup_{A \in \T_1} A \in \T\).

综上所述,\(\T\)是\(X\)的一个拓扑,
称其为\(X\)的\DefineConcept{有限补拓扑};
称拓扑空间\((X,\T)\)为一个\DefineConcept{有限补空间}.
\end{example}

\begin{example}[可数补空间]
%@see: 《点集拓扑讲义(第四版)》(熊金城) P58 例2.2.5
%@see: 《基础拓扑学讲义》(尤承业) P13 例2
设\(X\)是一个集合.
令\begin{equation*}
	\T = \Set{
		X-U
		\given
		\text{\(U\)是\(X\)的一个可数子集}
	}
	\cup
	\{\emptyset\}.
\end{equation*}
可以验证\(\T\)是\(X\)的一个拓扑,称其为\(X\)的\DefineConcept{可数补拓扑};
称拓扑空间\((X,\T)\)为一个\DefineConcept{可数补空间}.
\end{example}

\subsection{可度量化空间}
一个令人关心的问题是,
拓扑空间是否真的要比度量空间的范围更广一些?
是否每一个拓扑空间的拓扑都可以由某一个度量诱导出来?

\begin{definition}
%@see: 《点集拓扑讲义(第四版)》(熊金城) P58 定义2.2.3
设\((X,\T)\)是一个拓扑空间.
如果存在\(X\)的一个度量\(\rho\)
使得拓扑\(\T\)就是由度量\(\rho\)诱导出来的拓扑\(\T_\rho\),
则称“拓扑空间\((X,\T)\)是一个\DefineConcept{可度量化空间}”,
或称“拓扑空间\((X,\T)\)是\DefineConcept{可度量化的}”.
\end{definition}

根据这个定义,前述问题即是:
是否每一个拓扑空间都是可度量化空间?
我们知道,每一个只含有限个点的度量空间作为拓扑空间都是离散空间.
然而一个平庸空间如果含有多余一个点的话,它肯定不是离散空间,
因此含有多余一个点的有限的平庸空间不是可度量化的.
\cref{example:拓扑学.常见的拓扑空间3} 给出的那个空间只含有三个点,
但它既非离散空间也非可度量化空间.
由此可见,拓扑空间比度量空间的范围要更加广泛.
进一步的问题是,满足什么条件的拓扑空间是可度量化的?
这是点集拓扑学中的重要问题之一,以后我们将专门讨论.

\begin{example}
%@see: 《点集拓扑讲义(第四版)》(熊金城) P62 习题 5.
证明:每一个离散空间都是可度量化的.
%TODO
\end{example}

\subsection{拓扑空间之间的连续映射}
下面我们参考\cref{definition:度量空间.连续映射的概念}
和\cref{theorem:度量空间.度量空间下的连续映射与邻域的联系},
将度量空间之间的连续映射的概念推广为拓扑空间之间的连续映射.

\begin{definition}\label{definition:拓扑学.拓扑空间之间的连续映射}
%@see: 《点集拓扑讲义(第四版)》(熊金城) P58 定义2.2.4
设\(X\)和\(Y\)是两个拓扑空间,
映射\(f\colon X \to Y\).
如果\(Y\)中每一个开集\(U\)的原像\(f^{-1}(U)\)是\(X\)中的一个开集,
则称“\(f\)是从\(X\)到\(Y\)的一个\DefineConcept{连续映射}”,
简称“映射\(f\) \DefineConcept{连续}”.
\end{definition}
结合\cref{definition:拓扑学.拓扑空间之间的连续映射}
和\cref{theorem:度量空间.度量空间下的连续映射与邻域的联系} 可知:
当\(X\)、\(Y\)是两个度量空间时,
如果映射\(f\colon X \to Y\)是从度量空间\(X\)到度量空间\(Y\)的一个连续映射,
那么它也是从拓扑空间\(X\)到拓扑空间\(Y\)的一个连续映射;反之亦然.
注意到这里提到的拓扑都是指诱导拓扑.

下面我们给出连续映射的最重要的性质.

\begin{theorem}\label{theorem:拓扑学.拓扑空间之间的连续映射的性质}
%@see: 《点集拓扑讲义(第四版)》(熊金城) P59 定理2.2.1
设\(X\)、\(Y\)、\(Z\)都是拓扑空间,
那么\begin{itemize}
	\item 恒同映射\(i_X\colon X \to X\)是一个连续映射;
	\item 如果映射\(f\colon X \to Y\)和\(g\colon Y \to Z\)都是连续映射,
	则复合映射\(g \circ f\colon X \to Z\)也是连续映射.
\end{itemize}
\begin{proof}
\begin{itemize}
	\item 如果\(U\)是\(X\)的一个开集,
	则\(i_X^{-1}(U) = U\)当然也是\(X\)的开集,
	所以\(i_X\)连续.

	\item 设\(f\colon X \to Y\)和\(g\colon Y \to Z\)都是连续映射,
	\(W\)是\(Z\)的一个开集.
	由于\(g\)连续,
	\(g^{-1}(W)\)是\(Y\)的开集;
	又由于\(f\)连续,
	故\(f^{-1}(g^{-1}(W))\)是\(X\)的开集;
	因此,\begin{equation*}
		(g \circ f)^{-1}(W) = f^{-1}(g^{-1}(W))
	\end{equation*}是\(X\)的开集,
	\(g \circ f\)连续.
	\qedhere
\end{itemize}
\end{proof}
\end{theorem}

现在我们来将度量空间之间的连续映射在一点处的连续性的概念推广到拓扑空间之间的映射中去.

\begin{definition}
%@see: 《点集拓扑讲义(第四版)》(熊金城) P66 定义2.3.2
设\(X\)和\(Y\)是两个拓扑空间,
映射\(f\colon X \to Y\),
\(x \in X\).
如果\(f(x) \in Y\)的每一个邻域\(U\)的原像\(f^{-1}(U)\)是\(x \in X\)的一个邻域,
则称“\(f\)是一个在点\(x\)连续的映射”,
或称“映射\(f\)在点\(x\)连续”.
\end{definition}
与连续映射的情形一样,按这种方式定义拓扑空间之间的映射在某一点处的连续性
明显也是收到了\cref{theorem:度量空间.度量空间下的连续映射与邻域的联系} 的启发,
并且那个定理也保证了:
当\(X\)和\(Y\)是两个度量空间时,
如果\(f\)是从度量空间\(X\)到度量空间\(Y\)的一个映射,
它在某一点\(x \in X\)连续,
那么它也是从拓扑空间\(X\)到拓扑空间\(Y\)的一个在点\(x\)连续的映射;
反之亦然.

这里我们也有与\cref{theorem:拓扑学.拓扑空间之间的连续映射的性质} 类似地的定理.
\begin{theorem}
%@see: 《点集拓扑讲义(第四版)》(熊金城) P66 定理2.3.4
设\(X,Y,Z\)都是拓扑空间,
则\begin{itemize}
	\item 恒同映射\(i_X\colon X \to X\)在\(\forall x \in X\)连续;
	\item 如果\(f\colon X \to Y\)在点\(x \in X\)连续,
	\(g\colon Y \to Z\)在点\(f(x)\)连续,
	则\(g \circ f\colon X \to Z\)在点\(x\)连续.
\end{itemize}
%TODO proof
\end{theorem}

下面的\cref{theorem:拓扑学.连续性在局部与整体的连续}
建立了“局部的”连续性概念和“整体的”连续性概念之间的联系.

\begin{theorem}\label{theorem:拓扑学.连续性在局部与整体的连续}
%@see: 《点集拓扑讲义(第四版)》(熊金城) P66 定理2.3.5
设\(X\)和\(Y\)是拓扑空间,映射\(f\colon X \to Y\),
则映射\(f\)连续的充分必要条件是:
对于\(\forall x \in X\),映射\(f\)在点\(x\)连续.
%TODO proof
\end{theorem}

\subsection{同胚映射}
在数学的许多分支学科中都要涉及两种基本对象.
例如在线性代数中我们考虑线性空间和线性变换,
在群论中我们考虑群和同态,
在集合论中我们考虑集合和映射,
在不同的几何学中考虑各自的图形和各自的变换等.
并且对于后者都要提出一类来予以重点研究,
例如线性代数中的(线性)同构、
群论中的同构、
集合论中的双射
以及欧氏几何中的刚体运动(即平移、旋转)等.

既然我们已经提出了两种基本对象(即拓扑空间和连续映射),
那么我们也要从连续映射中挑出重要的一类来进行特别研究.

\begin{definition}\label{definition:拓扑学.同胚映射的概念}
%@see: 《点集拓扑讲义(第四版)》(熊金城) P60 定义2.2.5
设\(X,Y\)是两个拓扑空间.
如果映射\(f\colon X \to Y\)是一个双射,
并且\(f\)和逆映射\(f^{-1}\)都是连续的,
那么称“\(f\)是\(X\)与\(Y\)之间的一个\DefineConcept{同胚映射}(homeomorphism)”,
简称\DefineConcept{同胚};
%@see: 《点集拓扑讲义(第四版)》(熊金城) P60 定义2.2.6
又称“\(X\)与\(Y\)是\DefineConcept{同胚的}(homeomorphic)”,
或“\(X\)与\(Y\)同胚”,或“\(X\)同胚于\(Y\)”,
记作\(X \cong Y\).
%@see: https://mathworld.wolfram.com/Homeomorphism.html
\end{definition}

粗略地说,同胚的两个空间实际上就是两个具有相同拓扑结构的空间.

\begin{theorem}\label{theorem:拓扑学.同胚映射的性质}
%@see: 《点集拓扑讲义(第四版)》(熊金城) P60 定理2.2.2
设\(X\)、\(Y\)、\(Z\)都是拓扑空间,
那么\begin{enumerate}
	\item 恒同映射\(i_X\colon X \to X\)是一个同胚;
	\item 如果映射\(f\colon X \to Y\)是一个同胚,
	则逆映射\(f^{-1}\colon Y \to X\)也是同胚;
	\item 如果映射\(f\colon X \to Y\)、\(g\colon Y \to Z\)都是同胚,
	则复合映射\(g \circ f\colon X \to Z\)也是同胚.
\end{enumerate}
\begin{proof}
\begin{enumerate}
	\item 因为\(i_X\)是双射,
	并且\(i_X = i_X^{-1}\),
	再由\cref{theorem:拓扑学.拓扑空间之间的连续映射的性质}
	可知\(i_X\)是连续的,
	所以说\(i_X\)是同胚.

	\item 设\(f\colon X \to Y\)是一个同胚,
	则\(f\)是一个双射,
	并且\(f\)和\(f^{-1}\)都连续.
	于是\(f^{-1}\)也是一个双射,
	且\(f^{-1}\)和\((f^{-1})^{-1}\)也都连续,
	所以\(f^{-1}\)也是一个同胚.

	\item 设\(f\colon X \to Y\)和\(g\colon Y \to Z\)都是同胚.
	因此\(f\)和\(g\)都是双射,
	并且\(f,f^{-1},g,g^{-1}\)都是连续的.
	因此\(g \circ f\)也是双射,
	并且\(g \circ f\)和\((g \circ f)^{-1} = f^{-1} \circ g^{-1}\)都是连续的,
	所以说\(g \circ f\)是一个同胚.
	\qedhere
\end{enumerate}
\end{proof}
\end{theorem}

由\cref{theorem:拓扑学.同胚映射的性质} 立即可得如下定理.
\begin{theorem}\label{theorem:拓扑学.同胚关系是等价关系}
%@see: 《点集拓扑讲义(第四版)》(熊金城) P61 定理2.2.3
设\(X\)、\(Y\)、\(Z\)都是拓扑空间,
那么\begin{enumerate}
	\item \(X\)与\(X\)同胚;
	\item 如果\(X\)与\(Y\)同胚,则\(Y\)与\(X\)同胚;
	\item 如果\(X\)与\(Y\)同胚、\(Y\)与\(Z\)同胚,则\(X\)与\(Z\)同胚.
\end{enumerate}
\end{theorem}
根据\cref{theorem:拓扑学.同胚关系是等价关系},我们可以说:
在任意给定的一个由拓扑空间组成的族中,两个拓扑空间是否同胚这一关系是一个等价关系
\footnote{之所以不说“在由全体拓扑空间组成的族中,
两个拓扑空间是否同胚这一关系是一个等价关系”,
是因为“由全体拓扑空间组成的族”这样一个概念会引起逻辑矛盾:
若记这个族为\(T\),令\(\widetilde{T} = \Set{ X \given (X,\T) \in T }\),
赋予\(\widetilde{T}\)以平庸拓扑\(\T_0\),于是\((\widetilde{T},\T_0) \in T\),
从而\(\widetilde{T} \in \widetilde{T}\).
这就产生了“一个集合是它自己的元素”的悖论.}.
因此同胚关系将这个拓扑空间族分为互不相交的等价类,
使得属于同一类的拓扑空间彼此同胚,
属于不同类的拓扑空间彼此不同配.

拓扑空间的某种性质\(P\),
如果为某一个拓扑空间所具有,
则必为与其同胚的任何一个拓扑空间所具有,
那么称此性质\(P\)是一个\DefineConcept{拓扑不变性质}.
换言之,拓扑不变性质就是彼此同胚的拓扑空间所共有的性质.

\begin{remark}
{\color{red} 拓扑学的中心任务就是研究拓扑不变性质.}
\end{remark}

\begin{example}
%@see: 《点集拓扑讲义(第四版)》(熊金城) P63 习题 12.
设\(X\)和\(Y\)是两个同胚的拓扑空间.
证明:如果\(X\)是可度量化的,则\(Y\)也是可度量化的.
%TODO
% \begin{proof}
% 假设\(\T_1\)是\(X\)的由\(\rho_1\)诱导出来的拓扑,
% 映射\(f\colon X \to Y\)是\(X\)与\(Y\)之间的同胚映射.
% 要证\((Y,\T_2)\)是可度量化的,
% 须证存在\(Y\)的一个度量\(\rho_2\)使得拓扑\(\T_2\)是由度量\(\rho_2\)诱导出来的拓扑,
% 即证\(\T_2\)是\(Y\)中所有开集构成的集族.
% %cybcat:那把度量沿着同胚诱导过去不就完了
% %KK:?
% \end{proof}
\end{example}

\section{点的分类}
\subsection{基本概念}
\begin{definition}\label{definition:拓扑学.点的分类}
%@see: 《点集拓扑讲义(第四版)》(熊金城) P63 定义2.3.1
%@see: 《点集拓扑讲义(第四版)》(熊金城) P67 定义2.4.1
%@see: 《点集拓扑讲义(第四版)》(熊金城) P71 定义2.4.3
%@see: 《点集拓扑讲义(第四版)》(熊金城) P77 定义2.5.1
%@see: 《点集拓扑讲义(第四版)》(熊金城) P80 定义2.5.2
%@see: 《基础拓扑学讲义》(尤承业) P15 定义1.3
%@see: 《基础拓扑学讲义》(尤承业) P17 定义1.4
设\((X,\T)\)是一个拓扑空间,
\(A \subseteq X\).

对于任意取定的一点\(x \in X\),
如果\begin{equation*}
	(\exists U\in\T)
	[x \in U \subseteq A],
\end{equation*}
则称“\(x\)是\(A\)的一个\DefineConcept{内点}(interior point)”
“\(A\)是点\(x\)的一个\DefineConcept{邻域}(neighborhood)”.
%@see: https://mathworld.wolfram.com/Neighborhood.html

如果点\(x\)是\(A\)的补集\(X-A\)的一个内点,
则称“\(x\)是\(A\)的一个\DefineConcept{外点}(exterior point)”.

如果点\(x \in X\)的邻域\(U\)是\((X,\T)\)中的一个开集,
那么称“\(U\)是点\(x\)的一个\DefineConcept{开邻域}(open neighborhood)”.
%@see: https://mathworld.wolfram.com/OpenNeighborhood.html

\(X\)中任意一点\(x\)的全体邻域\begin{equation*}
	\Set{
		A \subseteq X \given (\exists U\in\T)[x \in U \subseteq A]
	},
\end{equation*}
称为“\(x\)的\DefineConcept{邻域系}(collection of neighborhoods)”.

集合\(A\)的全体内点\begin{equation*}
	\Set{
		x \in X \given (\exists U\in\T)[x \in U \subseteq A]
	},
\end{equation*}
称为“\(A\)的\DefineConcept{内部}(interior)”,
记作\(\TopoInterior{A}\).

对于任意取定的一点\(x \in X\),
如果\(x\)的每一个邻域\(U\)中总有异于\(x\)而属于\(A\)的点,
即\begin{equation*}
	U \cap (A - \{x\}) \neq \emptyset,
\end{equation*}
则称“\(x\)是\(A\)的一个\DefineConcept{聚点}(accumulation point, cluster point)”
或“\(x\)是\(A\)的一个\DefineConcept{极限点}(limit point)”;
否则,称“\(x\)是\(A\)的一个\DefineConcept{孤立点}(isolated point)”.
%@see: https://mathworld.wolfram.com/AccumulationPoint.html
%@see: https://mathworld.wolfram.com/LimitPoint.html
%@see: https://mathworld.wolfram.com/DiscreteSet.html

集合\(A\)的全体聚点,
称为“\(A\)的\DefineConcept{导集}(derived set)”,
记作\(\TopoDerived{A}\).
%@see: https://mathworld.wolfram.com/DerivedSet.html

我们把集合\(A\)及其导集\(\TopoDerived{A}\)的并\(A \cup \TopoDerived{A}\)
称为“集合\(A\)的\DefineConcept{闭包}(closure)”,
记作\(\TopoClosureL{A}\)或\(\TopoClosureM{A}\).

如果在\(x\)的任一邻域\(U\)中既有\(A\)中的点,又有\(X - A\)中的点,
即\begin{equation*}
	U \cap A \neq \emptyset
	\land
	U \cap (X-A) \neq \emptyset,
\end{equation*}
则称“\(x\)是集合\(A\)的一个\DefineConcept{边界点}(boundary point)”.
%@see: https://mathworld.wolfram.com/BoundaryPoint.html

集合\(A\)的全体边界点构成的集合,
称为“\(A\)的\DefineConcept{边界}(boundary)”,
记作\(\TopoBoundary{A}\).
%@see: https://mathworld.wolfram.com/Boundary.html
\end{definition}

容易看出,
\(X\)中任意一点\(x\)的邻域系是\(X\)的一个子集族.

\subsection{邻域系的性质}
\begin{theorem}\label{theorem:拓扑学.成为开集的充分必要条件1}
%@see: 《点集拓扑讲义(第四版)》(熊金城) P64 定理2.3.1
设\((X,\T)\)是一个拓扑空间,\(U \subseteq X\).
\(U\)是开集的充分必要条件是:
\(U\)是它的每一点的邻域,即对于\(\forall x \in U\),\(U\)都是\(x\)的一个邻域.
\begin{proof}
充分性.
\begin{itemize}
	\item 如果\(U\)是空集,当然\(U\)是一个开集.

	\item 如果\(U\neq\emptyset\),
	由于对于\(\forall x \in U\),\(\exists V_x \in \T\),
	使得\(x \in V_x \subseteq U\),
	所以\begin{equation*}
	U \equiv \bigcup_{x \in U} \{ x \}
	\subseteq \bigcup_{x \in U} V_x
	\subseteq U.
	\end{equation*}
	故\(U = \bigcup_{x \in U} V_x \subseteq \T\).
	根据\hyperref[definition:拓扑学.开集公理定义的拓扑空间]{拓扑的定义},\(U\)是一个开集.
	\qedhere
\end{itemize}
\end{proof}
\end{theorem}

\begin{theorem}\label{theorem:拓扑学.邻域系的基本性质}
%@see: 《点集拓扑讲义(第四版)》(熊金城) P64 定理2.3.2
设\(X\)是一个拓扑空间.
设\(A_x\)是任意一点\(x \in X\)的邻域系,则
\begin{itemize}
	\item \(A_x \neq \emptyset\);
	\item 如果\(U \in A_x\),则\(x \in U\);
	\item 如果\(U,V \in A_x\),则\(U \cap V \in A_x\);
	\item 如果\(U \in A_x\)且\(U \subseteq V\),则\(V \in A_x\);
	\item 如果\(U \in A_x\),则\(\exists V \in A_x\)满足:\begin{equation*}
		V \subseteq U
		\quad\land\quad
		(\forall y \in V)
		[V \in A_y].
	\end{equation*}
\end{itemize}
%TODO proof
\end{theorem}

\begin{theorem}\label{theorem:拓扑学.从邻域系出发定义拓扑}
%@see: 《点集拓扑讲义(第四版)》(熊金城) P65 定理2.3.3
设\(X\)是一个集合.
又设对于\(\forall x \in X\),指定\(X\)的一个子集族\(A_x\),
并且它们满足\cref{theorem:拓扑学.邻域系的基本性质} 中的全部条件,
则\(X\)有唯一的一个拓扑\(\T\)使得对于\(\forall x \in X\),
子集族\(A_x\)恰是点\(x\)在拓扑空间\((X,\T)\)中的邻域系.
%TODO proof
\end{theorem}

\cref{theorem:拓扑学.从邻域系出发定义拓扑}
表明,我们完全可以从邻域系的概念出发来建立拓扑空间理论.
这种做法在点集拓扑发展的早期常被采用,并且在一定程度上显得更加自然一些,
但不如现在流行的、从开集概念出发定义拓扑的做法来得简洁.

\subsection{聚点和导集的性质}
%@see: 《点集拓扑讲义(第四版)》(熊金城) P67
在\cref{definition:拓扑学.点的分类} 中,
聚点、导集以及孤立点的定义无一例外地依赖于它所在的拓扑空间的那个给定的拓扑\(\T\).
因此,当我们在讨论问题时,
如果涉及了多个拓扑而又提及聚点或孤立点时,
我们必须明确说明所称的聚点或孤立点是相对于哪个拓扑而言,不容许产生任何混响.
由于我们将要定义的许多概念绝大多数都是依赖于给定拓扑的,
因此类似于这里谈到的问题,今后几乎时时刻刻都会发生,
即便以后不作特别说明,也请留意这一问题.

应该注意到,尽管在欧氏空间中我们已经定义过聚点、孤立点的概念,
但绝不要以为某些在欧氏空间中有效的聚点或孤立点的性质对一般的拓扑空间都有效.

\begin{example}[离散空间中的聚点]\label{example:拓扑学.离散空间中的聚点}
%@see: 《点集拓扑讲义(第四版)》(熊金城) P67 例2.4.1
设\(X\)是一个离散空间,\(A\)是\(X\)的一个任意子集.
由于\(X\)中的每一个单点集都是开集,因此如果\(x \in X\),
则\(x\)有一个邻域\(\{x\}\)使得\(\{x\}\cap(A-\{x\})=\emptyset\),
于是\(x\)不是\(A\)的聚点,\(A\)没有聚点,从而\(A\)的导集是空集.
\end{example}

\begin{example}[平庸空间中的聚点]\label{example:拓扑学.平庸空间中的聚点}
%@see: 《点集拓扑讲义(第四版)》(熊金城) P68 例2.4.2
设\(X\)是一个平庸空间,\(A\)是\(X\)中的一个任意子集.
我们可以分三种情况讨论.
\begin{enumerate}
	\item 设\(\abs{A} = 0\).
	那么\(A = \emptyset\).
	这时\(A\)显然没有聚点,\(A\)的导集是空集.

	\item 设\(\abs{A} = 1\).
	不妨设\(A = \{x_0\}\).
	如果\(x \in X\),\(x \neq x_0\),点\(x\)只有唯一的一个邻域\(X\).
	这时\(x_0 \in X \cap (A - \{x\})\),
	所以\(X \cap (A - \{x\}) \neq \emptyset\).
	因此\(x\)是\(A\)的一个聚点.
	然而对于\(x_0\)的唯一邻域\(X\),
	有\(X \cap (A - \{x_0\}) = \emptyset\),
	所以\(x_0\)不是\(A\)的聚点.
	于是\(A\)的导集是\(X - A\).

	\item 设\(\abs{A} > 1\).
	这时\(X\)中的每一个点都是\(A\)的聚点.
\end{enumerate}
\end{example}

\begin{remark}
从\cref{example:拓扑学.离散空间中的聚点,example:拓扑学.平庸空间中的聚点} 可以看出,
离散空间中的任何一个子集都是闭集,而平庸空间中的任何一个非空真子集都不是闭集.
\end{remark}

\begin{theorem}
%@see: 《点集拓扑讲义(第四版)》(熊金城) P68 定理2.4.1
设\(X\)是一个拓扑空间,\(A,B \subseteq X\),则
\begin{itemize}
	\item \(\TopoDerived{\emptyset} = \emptyset\).
	\item \(A \subseteq B \implies \TopoDerived{A} \subseteq \TopoDerived{B}\).
	\item \(\TopoDerived{(A \cup B)} = \TopoDerived{A} \cup \TopoDerived{B}\).
	\item \(\TopoDerived{(\TopoDerived{A})} \subseteq \TopoClosureL{A}\).
\end{itemize}
%TODO proof
\end{theorem}

\begin{theorem}\label{theorem:点集拓扑.闭集的等价定义}
%@see: 《点集拓扑讲义(第四版)》(熊金城) P69 定义2.4.2
设\(X\)是一个拓扑空间,\(A \subseteq X\).
\(A\)是\(X\)中的一个闭集,
当且仅当\(A\)的每一个聚点都属于\(A\).
%TODO proof
\end{theorem}
\cref{theorem:点集拓扑.闭集的等价定义} 是\hyperref[definition:拓扑空间.闭集的定义]{闭集}的等价定义.

\begin{example}[实数空间\(\mathbb{R}\)中的闭集]
%@see: 《点集拓扑讲义(第四版)》(熊金城) P70 例2.4.3
设\(a,b\in\mathbb{R}\),\(a<b\).
闭区间\([a,b]\)是实数空间\(\mathbb{R}\)中的一个闭集,
因为\([a,b]\)的补集\(\mathbb{R}-[a,b]
=(-\infty,a)\cup(b,+\infty)\)是一个开集.
同理,\((-\infty,a]\)、\([b,+\infty)\)和\((-\infty,+\infty)\)也都是闭集.
但是,开区间\((a,b)\)却不是闭集,这是因为\(a\)是\(a,b\)的一个聚点,但\(a\notin(a,b)\).
同理,\((a,b]\)、\([a,b)\)、\((-\infty,a)\)和\((b,+\infty)\)都不是闭集.
\end{example}

\begin{theorem}\label{theorem:拓扑学.闭集族的性质}
%@see: 《点集拓扑讲义(第四版)》(熊金城) P70 定理2.4.3
设\(X\)是一个拓扑空间,\(F\)为所有闭集构成的族,则
\begin{itemize}
	\item \(\emptyset,X \in F\);
	\item \(A,B \in F \implies A \cup B \in F\);
	\item \(\emptyset \neq F_1 \subseteq F\)
	\footnote{%
		这里特别要求\(F_1 \neq \emptyset\)的原因在于
		当\(F_1 = \emptyset\)时所涉及的交运算没有定义.
	},则\(\bigcap_{A \in F_1} A \in F\).
\end{itemize}
%TODO proof
\end{theorem}

\subsection{闭包、内部与边界的关系}
\begin{theorem}\label{theorem:拓扑学.内部与闭包的联系}
%@see: 《基础拓扑学讲义》(尤承业) P17 命题1.4
设\(X\)是一个拓扑空间.
若\(X\)的子集\(A\)与\(B\)互为补集,
则\(A\)的闭包\(\TopoClosureM{A}\)与\(B\)的内部\(\TopoInterior{B}\)也互为补集,
即\begin{equation*}
	(\forall A,B \subseteq X)[A \cup B = X \implies (\TopoClosureM{A}) \cup (\TopoInterior{B}) = X].
\end{equation*}
\end{theorem}

\begin{theorem}
%@see: 《点集拓扑讲义(第四版)》(熊金城) P78 定理2.5.1
%@see: 《点集拓扑讲义(第四版)》(熊金城) P80 定理2.5.6
设\(X\)是一个拓扑空间.
对于\(X\)的任一子集\(A\),
它的闭包\(\TopoClosureM{A}\)、导集\(\TopoDerived{A}\)和内部\(\TopoInterior{A}\)满足以下性质:\begin{gather*}
	\TopoClosureM{A}
	= \TopoDerived{(\TopoInterior{(\TopoDerived{A})})}
	= (\TopoInterior{A}) \cup (\TopoBoundary{A}), \\
	\TopoInterior{A}
	= \TopoDerived{(\TopoClosureM{(\TopoDerived{A})})}
	= (\TopoClosureM{A}) - (\TopoBoundary{A}), \\
	\TopoBoundary{A}
	= (\TopoClosureM{A}) \cap (\TopoClosureM{(\TopoDerived{A})})
	= \TopoDerived{((\TopoInterior{A}) \cup (\TopoInterior{(\TopoDerived{A})}))}
	= \TopoBoundary(\TopoDerived{A}).
\end{gather*}
%TODO proof
\end{theorem}

\subsection{闭包的性质}
\begin{proposition}\label{theorem:拓扑学.一点属于闭包的充分必要条件}
%@see: 《点集拓扑讲义(第四版)》(熊金城) P71
设\(X\)是一个拓扑空间,\(A \subseteq X\),\(x \in X\).
\(x \in \TopoClosureL{A}\)的充分必要条件是:
对\(x\)的任一邻域\(U\)有\(U \cap A \neq \emptyset\).
\end{proposition}

\begin{theorem}\label{theorem:拓扑学.闭包的性质}
%@see: 《点集拓扑讲义(第四版)》(熊金城) P71 定理2.4.4
%@see: 《点集拓扑讲义(第四版)》(熊金城) P71 定理2.4.5
%@see: 《点集拓扑讲义(第四版)》(熊金城) P72 定理2.4.7
%@see: 《基础拓扑学讲义》(尤承业) P17 命题1.5
%@see: 《Real Analysis Modern Techniques and Their Applications Second Edition》(Gerald B. Folland) P13
设\(X\)是一个拓扑空间.
\begin{itemize}
	\item \(\TopoClosureL{\emptyset} = \emptyset\).
	\item \((\forall A\subseteq X)[A \subseteq \TopoClosureL{A}]\).
	\item \((\forall A\subseteq X)[\TopoClosureL{\TopoClosureL{A}} = \TopoClosureL{A}]\).
	\item \((\forall A,B\subseteq X)[A \subseteq B \implies \TopoClosureL{A} \subseteq \TopoClosureL{B}]\).
	\item 对于\(\forall A \subseteq X\),
	\(A\)的闭包\(\TopoClosureL{A}\)是\(X\)的包含\(A\)的全体闭集的交,
	或者说\(\TopoClosureL{A}\)是包含\(A\)的最小闭集,
	即\begin{equation}\label{equation:拓扑学.集合的闭包是含有该集的最小闭集}
		% the intersection of all closed sets V \supseteq E is the smallest closed set containing E
		\TopoClosureL{A}
		= \bigcap\Set{ V \supseteq A \given \text{$V$是$X$中的闭集} }.
	\end{equation}
	\item \((\forall A\subseteq X)[\TopoClosureL{A}=A \iff \text{\(A\)是闭集}]\).
	\item \((\forall A,B\subseteq X)[\TopoClosureL{A \cup B} = \TopoClosureL{A} \cup \TopoClosureL{B}]\).
	\item \((\forall A,B\subseteq X)[\TopoClosureL{A \cap B} \subseteq \TopoClosureL{A} \cap \TopoClosureL{B}]\).
\end{itemize}
%TODO proof
\end{theorem}

\begin{corollary}\label{theorem:拓扑学.拓扑空间子集闭包都是闭集}
%@see: 《点集拓扑讲义(第四版)》(熊金城) P72 定理2.4.6
拓扑空间\(X\)的任一子集\(A\)的闭包\(\TopoClosureL{A}\)都是闭集.
\begin{proof}
由\cref{theorem:拓扑学.闭包的性质} 立即可得.
\end{proof}
\end{corollary}

\subsection{内部的性质}
关于内部的基本性质,我们有与闭包的性质完全对偶的一组定理.
这些定理的证明过程都是将闭包的相应性质通过\cref{theorem:拓扑学.内部与闭包的联系}
转化为内部的性质.

\begin{theorem}\label{theorem:拓扑学.内部的性质}
%@see: 《点集拓扑讲义(第四版)》(熊金城) P78 定理2.5.2
%@see: 《点集拓扑讲义(第四版)》(熊金城) P78 定理2.5.3
%@see: 《点集拓扑讲义(第四版)》(熊金城) P79 定理2.5.5
%@see: 《基础拓扑学讲义》(尤承业) P16 命题1.3
%@see: 《Real Analysis Modern Techniques and Their Applications Second Edition》(Gerald B. Folland) P13
设\(X\)是一个拓扑空间,则\begin{itemize}
	\item \(\TopoInterior{X} = X\);
	\item \((\forall A \subseteq X)[A \supseteq \TopoInterior{A}]\);
	\item \((\forall A \subseteq X)[\TopoInterior{(\TopoInterior{A})} = \TopoInterior{A}]\);
	\item \((\forall A,B \subseteq X)[A \subseteq B \implies \TopoInterior{A} \subseteq \TopoInterior{B}]\);
	\item 对于\(\forall A \subseteq X\),
	\(A\)的内部\(\TopoInterior{A}\)是\(X\)的包含于\(A\)的全体开集的并,
	或者说\(\TopoInterior{A}\)是包含于\(A\)的最大开集,
	即\begin{equation}
		% the union of all open sets U \subseteq E is the largest open set contained in E
		\TopoInterior{A}
		= \bigcup\Set{ U \subseteq A \given \text{$U$是$X$中的开集} }.
	\end{equation}
	\item \((\forall A \subseteq X)[A=\TopoInterior{A} \iff \text{\(A\)是开集}]\);
	\item \((\forall A,B \subseteq X)[\TopoInterior{(A \cap B)} = \TopoInterior{A} \cap \TopoInterior{B}]\);
	\item \((\forall A,B \subseteq X)[\TopoInterior{(A \cup B)} \supseteq \TopoInterior{A} \cup \TopoInterior{B}]\).
\end{itemize}
%TODO proof
\end{theorem}

\begin{theorem}\label{theorem:拓扑学.拓扑空间子集内部都是开集}
%@see: 《点集拓扑讲义(第四版)》(熊金城) P79 定理2.5.4
拓扑空间\(X\)的任一子集\(A\)的内部\(\TopoInterior{A}\)都是开集.
%TODO proof
\end{theorem}

\subsection{闭包运算}
利用\cref{equation:拓扑学.集合的闭包是含有该集的最小闭集},
由一个集合求取它的闭包的步骤,
可以理解为空间\(X\)的幂集\(\Powerset X\)到自身的一个映射,
集合\(A \subseteq X\)在这个映射下的像便是\(A\)的闭包\(\TopoClosureL{A}\).

\begin{definition}\label{definition:拓扑学.闭包运算的概念}
%@see: 《点集拓扑讲义(第四版)》(熊金城) P73 定义2.4.4
设\(X\)是一个集合.
如果映射\(c^*\colon \Powerset X \to \Powerset X\)满足条件:
对于\(\forall A,B \in \Powerset X\),有\begin{itemize}
	\item \(c^*(\emptyset) = \emptyset\);
	\item \(A \subseteq c^*(A)\);
	\item \(c^*(A \cup B) = c^*(A) \cup c^*(B)\);
	\item \(c^*(c^*(A)) = c^*(A)\),
\end{itemize}
则称其为\(X\)的一个\DefineConcept{闭包运算}.
\end{definition}
\cref{definition:拓扑学.闭包运算的概念} 中给出的四个条件,
通常被称为“库拉托夫斯基闭包公理”.

根据\cref{theorem:拓扑学.闭包的性质},
将拓扑空间\(X\)的子集\(A\)映射为它的闭包\(\TopoClosureL{A}\)的那个
从\(X\)的幂集\(\Powerset X\)到自身的映射,便是一个闭包运算,
即这个映射满足库拉托夫斯基闭包公理.
不仅如此,下面的\cref{theorem:拓扑学.闭包公理与拓扑是等价的}
说明库拉托夫斯基闭包公理和我们定义拓扑的三个条件等价.
在一些点集拓扑发展的早期出现的文献就是从闭包运算出发来建立拓扑空间这一概念的.

\begin{theorem}\label{theorem:拓扑学.闭包公理与拓扑是等价的}
%@see: 《点集拓扑讲义(第四版)》(熊金城) P73 定理2.4.8
设\(X\)是一个集合,映射\(c^*\colon \Powerset X \to \Powerset X\)是集合\(X\)的一个闭包运算,
那么存在\(X\)的唯一一个拓扑\(\T\),使得在拓扑空间\((X,\T)\)中,
对于\(\forall A \subseteq X\),总有\(c^*(A) = \TopoClosureL{A}\).
%TODO proof
\end{theorem}

与闭包运算一样,
求取一个集合的内部也可以理解为从拓扑空间\(X\)的幂集\(\Powerset X\)到其自身的一个映射,
它将每一个\(A \in \Powerset X\)映射为\(\TopoInterior{A}\).
也同样可以像定义闭包运算一样定义\DefineConcept{内部运算},
并由内部运算导出拓扑和拓扑空间的概念.

同样地,映射的连续性也可通过内部这个概念作出等价的描述.

\subsection{度量空间中的点}

在度量空间中,集合的聚点、导集和闭包等概念都可以通过度量来刻画.

\begin{definition}\label{definition:拓扑学.点到点集的距离}
%@see: 《点集拓扑讲义(第四版)》(熊金城) P75 定义2.4.5
设\((X,\rho)\)是一个度量空间,\(A\)是\(X\)的非空子集,\(x \in X\).
定义:\begin{equation*}
	\rho(x,A) \defeq \inf\Set{ \rho(x,y) \given y \in A },
\end{equation*}
称之为“点\(x\)到\(A\)的\DefineConcept{距离}”.
\end{definition}

\begin{theorem}
%@see: 《点集拓扑讲义(第四版)》(熊金城) P75
设\((X,\rho)\)是一个度量空间,\(A\)是\(X\)的非空子集,\(x \in X\).
\(\rho(x,A) = 0\)的充分必要条件是:
\((\forall\epsilon>0)(\exists y \in A)[\rho(x,y)<\epsilon]\).
\end{theorem}

\begin{corollary}
%@see: 《点集拓扑讲义(第四版)》(熊金城) P75
设\((X,\rho)\)是一个度量空间,\(A\)是\(X\)的非空子集,\(x \in X\).
\(\rho(x,A) = 0\)的充分必要条件是:
对于\(x\)的任一邻域\(U\),总有\(U \cap A \neq \emptyset\).
\end{corollary}

\begin{theorem}
%@see: 《点集拓扑讲义(第四版)》(熊金城) P75 定理2.4.9
设\(A\)是度量空间\((X,\rho)\)中的一个非空子集,
则\begin{gather*}
	x \in \TopoDerived{A}
	\iff
	\rho(x,A-\{x\})=0, \\
	x \in \TopoClosureL{A}
	\iff
	\rho(x,A)=0.
\end{gather*}
\end{theorem}

以下定理既为连续映射提供了等价定义,
也为验证映射的连续性提供了另外的手段.

\begin{theorem}
%@see: 《点集拓扑讲义(第四版)》(熊金城) P75 定理2.4.10
设\(X\)和\(Y\)是两个拓扑空间,映射\(f\colon X \to Y\),则以下命题等价:
\begin{itemize}
	\item \(f\)是一个连续映射;
	\item \(Y\)中的任何一个闭集的原像\(f^{-1}(B)\)是一个闭集;
	\item 对于\(X\)中的任何一个子集\(A\),\(A\)的闭包的像包含于\(A\)的像的闭包,
	即\(
		f(\TopoClosureL{A})
		\subseteq
		\TopoClosureL{f(A)}
	\);
	\item 对于\(Y\)中的任何一个子集\(B\),\(B\)的闭包的原像包含\(B\)的原像的闭包,
	即\(
		f^{-1}(\TopoClosureL{B})
		\supseteq
		\TopoClosureL{f^{-1}(B)}
	\).
\end{itemize}
\end{theorem}


\def\B{\mathscr{B}}%拓扑\(\T\)的基
\def\S{\mathscr{S}}%拓扑\(\T\)的子基

\section{基,子基}
\subsection{基}
在讨论度量空间的拓扑的时候,球形邻域起着基础性的重要作用.
一方面,每一个球形邻域都是开集,从而任意多个球形邻域的并也是开集;
另一方面,假设\(U\)是度量空间\(X\)中的一个开集,
则对于每一个\(x\in U\)有一个球形邻域\(B(x,\epsilon) \subseteq U\),
因此\(U = \bigcup_{x \in U} B(x,\epsilon)\).
这就是说,一个集合时某度量空间中的一个开集,
当且仅当它是这个度量空间中的若干个球形邻域的并.
因此我们可以说,度量空间的拓扑是由它的所有球形邻域通过集族求并这一运算产生出来的.
留意了这个事实,我们对于下面再拓扑空间中提出“基”这个概念就不会感到突然了.

\begin{definition}
%@see: 《点集拓扑讲义(第四版)》(熊金城) P82 定义2.6.1
设\((X,\T)\)是一个拓扑空间,\(\B \subseteq \T\).
如果\(\T\)中的每一个元素都是\(\B\)中某些元素的并,
即\begin{equation*}
	(\forall U \in \T)
	(\exists \B_1 \subseteq \B)
	\left[U = \bigcup \B_1\right],
\end{equation*}
则称“\(\B\)是拓扑\(\T\)的一个\DefineConcept{基}”,
或称“\(\B\)是拓扑空间\(X\)的一个\DefineConcept{基}”.
\end{definition}

按照本节开头所作的论证立即可得.
\begin{theorem}
%@see: 《点集拓扑讲义(第四版)》(熊金城) P82 定理2.6.1
一个度量空间中的全体球形邻域,是这个度量空间作为拓扑空间时的一个基.
\end{theorem}

\begin{example}
%@see: 《点集拓扑讲义(第四版)》(熊金城) P82
由于实数空间\(\mathbb{R}\)中的开区间就是它的球形邻域,
因此\(\mathbb{R}\)的全体开区间是它的一个基.
\end{example}

\begin{example}
%@see: 《点集拓扑讲义(第四版)》(熊金城) P82
离散空间的基是它的全体单点子集.
\end{example}

\subsection{基的判别}
下面的定理,为判断某一个开集族是不是给定的拓扑的一个基,提供了一个易于验证的条件.
\begin{theorem}
%@see: 《点集拓扑讲义(第四版)》(熊金城) P83 定理2.6.2
设\(\B\)是拓扑空间\((X,\T)\)的一个开集族,即\(\B \subseteq \T\),
则“\(\B\)是拓扑空间\(X\)的一个基”的充分必要条件是:
对于每一个\(x \in X\)和\(x\)的每一个邻域\(U_x\),
存在\(V_x \in \B\),使得\(x \in V_x \subseteq U_x\).
%TODO proof
\end{theorem}

在度量空间中,通过球形邻域确定了度量空间的拓扑,
这个拓扑以全体球形邻域构成的集族作为基.
是不是一个集合的每一个子集族都可以确定一个拓扑以它为基?
答案是否定的.
以下定理告诉我们一个集合的子集族需要满足什么条件,才可以成为它的某一个拓扑的基.
\begin{theorem}\label{theorem:拓扑基.子集族成为拓扑基的条件}
%@see: 《点集拓扑讲义(第四版)》(熊金城) P83 定理2.6.3
设\(X\)是一个集合,\(\B\)是集合\(X\)的一个子集族,即\(\B \subseteq \Powerset X\).
如果\begin{itemize}
	\item \(\bigcup \B = X\);
	\item \(B_1,B_2 \in \B
	\implies
	(\forall x \in B_1 \cap B_2)
	(\exists B \in \B)
	[x \in B \subseteq B_1 \cap B_2]\)%
	\footnote{%
		如果\(\B\)满足\((\forall B_1,B_2 \in \B)[B_1 \cap B_2 \in \B]\),
		则\(\B\)必然满足第二个条件.%
	},
\end{itemize}
则\(X\)的子集族\begin{equation*}
	\T = \Set*{
		U \subseteq X
		\given
		(\exists \B_U \subseteq \B)\left[ U = \bigcup \B_U \right]
	}
\end{equation*}是集合\(X\)的唯一一个以\(\B\)为基的拓扑.
反之,如果\(X\)的一个子集族\(\B\)是\(X\)的某一个拓扑的基,
则\(\B\)一定满足上述两个条件.
%TODO proof
\end{theorem}

\begin{example}[实数下限拓扑空间]
%@see: 《点集拓扑讲义(第四版)》(熊金城) P85 例2.6.1
考虑实数集\(\mathbb{R}\).
令\begin{equation*}
	\B \defeq \Set{ [a,b) \given a,b \in \mathbb{R} \land a < b }.
\end{equation*}
容易验证\(\mathbb{R}\)的子集族\(\B\)满足\cref{theorem:拓扑基.子集族成为拓扑基的条件} 的所有条件,
因此\(\B\)是实数集\(\mathbb{R}\)的某个拓扑\(\S\)的基.
我们把\(\S\)称为“\(\mathbb{R}\)的\DefineConcept{下限拓扑}”,
拓扑空间\((\mathbb{R},\S)\)称为\DefineConcept{实数下限拓扑空间},记作\(\mathbb{R}_l\).
容易看出它与通常的实数空间\((\mathbb{R},\T)\)有很大区别.
对于每一个开区间\((a,b)\subseteq\mathbb{R}\),
其中\(a,b\in\mathbb{R}\)且\(a<b\),
如果对任意\(i \in \omega\),
任意选取\(b_i \in \mathbb{R}\),
使得\(a < \dotsb < b_2 < b_1 < b_0 < b\)
以及\(b_i - a < 1/i\),
那么\((a,b)=\bigcup_{i \in \omega} [b_i,b)\).
因此我们有\((a,b)\in\S\),
于是\(\T \subseteq \T_l\).
由于\(\T_l \subseteq \T\)显然不成立,
因此\(\T \subset \T_l\).
\end{example}

\subsection{子基}
在定义基的过程中,我们只是用到了集族的并运算.
如果再考虑集合的有限交运算\footnote{拓扑只是对有限交封闭的,所以只考虑有限交.},
便得到“子基”这个概念.

\begin{definition}
%@see: 《点集拓扑讲义(第四版)》(熊金城) P86 定义2.6.2
设\((X,\T)\)是一个拓扑空间,\(\S \subseteq \T\).
如果\(\S\)的全体非空有限子族之交\begin{equation*}
	\Set*{
		\bigcap S
		\given
		\text{$S$是$\S$的非空有限子集}
	}
\end{equation*}是拓扑\(\T\)的一个基,
则称“\(\S\)是拓扑\(\T\)的一个\DefineConcept{子基}”,
或称“\(\S\)是拓扑空间\(X\)的一个\DefineConcept{子基}”.
\end{definition}

\begin{example}
%@see: 《点集拓扑讲义(第四版)》(熊金城) P86 例2.6.2
\(\mathbb{R}\)的一个子集族\begin{equation*}
	\S \defeq \Set{ (a,+\infty) \given a\in\mathbb{R} } \cup \Set{ (-\infty,b) \given b\in\mathbb{R} }
\end{equation*}是\(\mathbb{R}\)的一个子基.
这是因为\(\S\)是实数空间的一个开集族,
并且\(\S\)的全体非空有限子族之交
恰好就是全体有限开区间\(\Set{ (a,b) \given a,b\in\mathbb{R} }\)、\(\S\)和\(\{\emptyset\}\)这三者的并.
显然它是实数空间\(\mathbb{R}\)的基.
\end{example}

\begin{theorem}
%@see: 《点集拓扑讲义(第四版)》(熊金城) P86 定理2.6.4
设\(X\)是一个集合,\(\S \subseteq \Powerset X\).
如果\(X = \bigcup \S\),
则\(X\)有唯一一个拓扑\(\T\)以\(\S\)为子基,
并且\begin{equation*}
	\T = \Set*{
		\bigcup B
		\given
		B \subseteq \B
	},
\end{equation*}
其中\begin{equation*}
	\B = \Set*{
		\bigcap S
		\given
		\text{$S$是$\S$的非空有限子集}
	}.
\end{equation*}
\end{theorem}

\subsection{邻域基,邻域子基}
\begin{definition}
%@see: 《点集拓扑讲义(第四版)》(熊金城) P87 定义2.6.3
\def\Ux{\mathscr{U}_x}
\def\Vx{\mathscr{V}_x}
\def\Wx{\mathscr{W}_x}
设\(X\)是一个拓扑空间,\(x \in X\).
记\(\Ux\)为\(x\)的邻域系,\(\Vx,\Wx \subseteq \Ux\).

如果\begin{equation*}
	(\forall U\in\Ux)
	(\exists V\in\Vx)
	[V \subseteq U],
\end{equation*}
则称“\(\Vx\)是点\(x\)的邻域系的一个基”
或称“\(\Vx\)是点\(x\)的一个\DefineConcept{邻域基}”.

如果\begin{equation*}
	\Set*{
		\bigcap W
		\given
		\text{$W$是$\Wx$的非空有限子集}
	}
\end{equation*}是\(\Ux\)的一个邻域基,
则称“\(\Wx\)是点\(x\)的邻域系的一个子基”,
或称“\(\Wx\)是点\(x\)的一个\DefineConcept{邻域子基}”.
\end{definition}

\begin{example}
%@see: 《点集拓扑讲义(第四版)》(熊金城) P88
在度量空间中以某一个点为中心的全体球形邻域是这个点的一个邻域基;
以某一个点为中心的全体以有理数为半径的球形邻域也是这个点的一个邻域基.
\end{example}

\subsection{基于邻域基、子基与邻域子基的关联}
\begin{theorem}
%@see: 《点集拓扑讲义(第四版)》(熊金城) P89 定理2.6.7
设\(X\)是一个拓扑空间,\(x \in X\).
\begin{itemize}
	\item 如果\(\B\)是\(X\)的一个基,
	则\begin{equation*}
		\B_x \defeq \Set{ B \in \B \given x \in B }
	\end{equation*}是点\(x\)的一个邻域基.

	\item 如果\(\S\)是\(x\)的一个子基,
	则\begin{equation*}
		\S_x \defeq \Set{ S \in \S \given x \in S }
	\end{equation*}是点\(x\)的一个邻域子基.
\end{itemize}
\end{theorem}

\subsection{应用:连续映射的判别}
映射的连续性可以用基或子基来验证.
一般来说,基或子基的基数不大于拓扑的基数.
因此,通过基或子基来验证映射的连续性,
有时可能会带来很大的方便.

\begin{theorem}
%@see: 《点集拓扑讲义(第四版)》(熊金城) P87 定理2.6.5
设\(X,Y\)都是拓扑空间,映射\(f\colon X\to Y\),
则以下命题等价:\begin{itemize}
	\item \(f\)连续;
	\item \(Y\)有一个基\(\B\),
	使得对于任何一个\(B \in \B\),
	原像\(f^{-1}(B)\)是\(X\)中的一个开集;
	\item \(Y\)有一个子基\(\S\),
	使得对于任何一个\(S \in \S\),
	原像\(f^{-1}(S)\)是\(X\)中的一个开集.
\end{itemize}
\end{theorem}

\begin{theorem}
%@see: 《点集拓扑讲义(第四版)》(熊金城) P88 定理2.6.6
\def\Vf{\mathscr{V}_{f(x)}}
\def\Wf{\mathscr{W}_{f(x)}}
设\(X,Y\)都是拓扑空间,映射\(f\colon X\to Y\),\(x \in X\),
则以下命题等价:\begin{itemize}
	\item \(f\)在点\(x\)连续;
	\item 点\(f(x)\)有一个邻域基\(\Vf\),
	使得对于任何一个\(V \in \Vf\),
	原像\(f^{-1}(V)\)是点\(x\)的一个邻域;
	\item 点\(f(x)\)有一个邻域子基\(\Wf\),
	使得对于任何一个\(W \in \Wf\),
	原像\(f^{-1}(W)\)是点\(x\)的一个邻域.
\end{itemize}
\end{theorem}

\section{拓扑空间中的序列}
\subsection{序列}
\begin{definition}
%@see: 《点集拓扑讲义(第四版)》(熊金城) P91 定义2.7.1
把从自然数集\(\omega\)的非空子集\(D\)到集合\(X\)的映射\begin{equation*}
	S\colon D \to X, n \mapsto x_n
\end{equation*}称为“\(X\)中的一个\DefineConcept{序列}”,
记作\(\{x_n\}_{n \in D}\).
\end{definition}

特别地,当\(D = \omega\)时,数列\(\{x_n\}_{n \in D}\)可以记作\(\{x_n\}_{n\geq0}\);
当\(D = \Set{ n \in \omega \given n \geq 1}\)时,
数列\(\{x_n\}_{n \in D}\)可以记作\(\{x_n\}_{n\geq1}\);
以此类推.

当序列\(\{x_n\}_{n \in D}\)的值域是一个单元素集时,
称“\(\{x_n\}_{n \in D}\)是一个\DefineConcept{常值序列}”.

\subsection{序列的单调性}
\begin{definition}
%@see: 《测度论讲义(第三版)》(严加安) P2 1.1.4
%@see: 《实变函数论(第三版)》(周民强) P9 定义1.8
设\(\{x_n\}_{n \in D}\)是集合\(X\)中的一个序列.

若\begin{equation*}
	(\forall n\in\omega)
	[x_n \subseteq x_{n+1}],
\end{equation*}
则称“\(\{x_n\}_{n \in D}\)是\(X\)中的一个\DefineConcept{单调增序列}”.

若\begin{equation*}
	(\forall n\in\omega)
	[x_n \supseteq x_{n+1}],
\end{equation*}
则称“\(\{x_n\}_{n \in D}\)是\(X\)中的一个\DefineConcept{单调减序列}”.

我们将“单调增序列”与“单调减序列”统称为\DefineConcept{单调序列}.
\end{definition}

\begin{definition}
设\(\{x_n\}_{n \in D}\)是集合\(X\)中的一个序列.
定义:\begin{align*}
	\bigcup_{k=n}^m x_k
	&\defeq
	\bigcup_{n \leq k \leq m, k\in\mathbb{N}} x_k, \\
	\bigcup_{k=n}^\infty x_k
	&\defeq
	\bigcup_{k \geq n, k\in\mathbb{N}} x_k, \\
	\bigcap_{k=n}^m x_k
	&\defeq
	\bigcap_{n \leq k \leq m, k\in\mathbb{N}} x_k, \\
	\bigcap_{k=n}^\infty x_k
	&\defeq
	\bigcap_{k \geq n, k\in\mathbb{N}} x_k.
\end{align*}
\end{definition}

\begin{definition}
%@see: 《实变函数论(第三版)》(周民强) P9 定义1.8
设\(\{x_n\}_{n \in D}\)是集合\(X\)中的一个序列.
定义:\begin{equation*}
	\lim_{n\to\infty} x_n
	\defeq
	\left\{ \def\arraystretch{2} \begin{array}{cl}
		\bigcup_{n=1}^\infty x_n, & \text{$\{x_n\}$是单调增序列}, \\
		\bigcap_{n=1}^\infty x_n, & \text{$\{x_n\}$是单调减序列},
	\end{array} \right.
\end{equation*}
称其为“\(\{x_n\}_{n \in D}\)的\DefineConcept{极限}”.
\end{definition}

\begin{definition}
%@see: 《测度论讲义(第三版)》(严加安) P2 1.1.4
设\(\{x_n\}_{x \in D}\)是集合\(X\)中的一个序列.

定义:\begin{equation*}
	\limsup_{n\to\infty} x_n
	\defeq
	\bigcap_{n=1}^\infty
	\bigcup_{k=n}^\infty
	x_k,
\end{equation*}
称其为“\(\{x_n\}_{x \in D}\)的\DefineConcept{上极限}”.

定义:\begin{equation*}
	\liminf_{n\to\infty} x_n
	\defeq
	\bigcup_{n=1}^\infty
	\bigcap_{k=n}^\infty
	x_k,
\end{equation*}
称其为“\(\{x_n\}_{x \in D}\)的\DefineConcept{下极限}”.
\end{definition}
容易看出,\(\limsup_{n\to\infty} x_n\)的任一元素属于无限多个\(x_n\),
而\(\liminf_{n\to\infty} x_n\)的任一元素至多不属于有限多个\(x_n\).

\begin{example}
%@see: 《实变函数论(第三版)》(周民强) P10 例7
设\(E,F\)都是集合,
集合\(X\)中的序列\(\{x_k\}_{k\geq1}\)满足\begin{equation*}
	x_k = \left\{ \begin{array}{cl}
		E, & \text{$k$是奇数}, \\
		F, & \text{$k$是偶数}
	\end{array} \right.
	\quad(k=1,2,\dotsc),
\end{equation*}
从而我们有\begin{equation*}
	\limsup_{k\to\infty} x_k
	= E \cup F, \qquad
	\liminf_{k\to\infty} x_k
	= E \cap F.
\end{equation*}
\end{example}

\begin{proposition}
%@see: 《实变函数论(第三版)》(周民强) P10
设\(E\)是集合,\(\{x_n\}_{n\geq1}\)是集合\(X\)中的一个序列,
则\begin{gather}
	E - \limsup_{n\to\infty} x_n = \liminf_{n\to\infty} (E - x_n); \\
	E - \liminf_{n\to\infty} x_n = \limsup_{n\to\infty} (E - x_n).
\end{gather}
\end{proposition}

\begin{proposition}
%@see: 《测度论讲义(第三版)》(严加安) P2 1.1.4
设\(\{x_n\}_{n\geq1}\)是集合\(X\)中的一个序列,
则\begin{equation}
	\liminf_{n\to\infty} x_n
	\subseteq
	\limsup_{n\to\infty} x_n.
\end{equation}
%@see: https://math.stackexchange.com/questions/107931/lim-sup-and-lim-inf-of-sequence-of-sets
\end{proposition}

\begin{theorem}
%@see: 《测度论讲义(第三版)》(严加安) P2 1.1.4
设\(\{x_n\}_{n\geq1}\)是集合\(X\)中的一个序列.
如果\begin{equation*}
	\liminf_{n\to\infty} x_n
	= \limsup_{n\to\infty} x_n
	= A,
\end{equation*}
则\(\lim_{n\to\infty} x_n = A\).
\end{theorem}

\subsection{收敛序列}
\begin{definition}\label{definition:序列.序列的聚点}
%@see: 《点集拓扑讲义(第四版)》(熊金城) P91 定义2.7.2
设\(\{x_n\}_{n \in D}\)是拓扑空间\(X\)中的一个序列,\(x \in X\).
如果对于\(x\)的每一个邻域\(U\),
存在\(N \in \omega\),
使得当\(n > N\)时
有\(x_n \in U\),
则称“点\(x\)是序列\(\{x_n\}_{n \in D}\)的一个\DefineConcept{聚点}”
“序列\(\{x_n\}_{n \in D}\)收敛于\(x\)”,
记作\(\lim_{n\to\infty} x_n = x\);
并称“序列\(\{x_n\}_{n \in D}\)是一个\DefineConcept{收敛序列}”.
\end{definition}

拓扑空间中序列的收敛性质与我们在数学分析中熟悉的有很大的差别.
例如,平庸空间中任何一个序列都收敛,并且它们收敛于这个空间中的每一个点.
这时极限的唯一性就无法保证了.

\subsection{子序列}
\begin{definition}\label{definition:序列.子序列}
%@see: 《点集拓扑讲义(第四版)》(熊金城) P91 定义2.7.3
设\(S\)和\(S_1\)是拓扑空间\(X\)中的两个序列.
如果存在一个严格单调增加的映射\(\sigma\colon \omega \to \omega\),
使得\(S_1 = S \circ \sigma\),
则称“序列\(S_1\)是序列\(S\)的一个\DefineConcept{子序列}”.
\end{definition}

\subsection{序列与子序列的性质}
我们已经看到,我们以前熟悉的序列的许多性质,对于拓扑空间中的序列来说,是不满足的.
但总有一些性质还保留着,其中最主要的,就是以下三个定理.

\begin{theorem}
%@see: 《点集拓扑讲义(第四版)》(熊金城) P92 定理2.7.1
设\(\{x_n\}_{n\geq0}\)是拓扑空间\(X\)中的一个序列,
则\begin{itemize}
	\item 如果\(\{x_n\}_{n\geq0}\)是一个常值序列,
	即\((\forall n\geq0)[x_n=x]\),
	则\(\lim_{n\to\infty} x_n = x\).
	\item 如果序列\(\{x_n\}_{n\geq0}\)收敛于\(x \in X\),
	则序列\(\{x_n\}_{n\geq0}\)的每一个子序列也收敛于\(x\).
\end{itemize}
%TODO proof
\end{theorem}

\begin{theorem}\label{theorem:序列.去心邻域内收敛于中心的序列的聚点1}
%@see: 《点集拓扑讲义(第四版)》(熊金城) P92 定理2.7.2
设\(X\)是一个拓扑空间,\(A \subseteq X\),\(x \in X\).
如果存在\(A-\{x\}\)中的一个序列\(\{x_n\}_{n\geq0}\),
\(\{x_n\}_{n\geq0}\)收敛于\(x\),
则\(x\)是\(A\)的一个聚点.
%TODO proof
\end{theorem}

\begin{proposition}\label{theorem:序列.去心邻域内收敛于中心的序列的聚点2}
%@see: 《点集拓扑讲义(第四版)》(熊金城) P92 例2.7.1
\cref{theorem:序列.去心邻域内收敛于中心的序列的聚点1} 的逆命题不成立.
%TODO proof
% \begin{proof}
% 设\(X\)是一个不可数集,
% 考虑它的拓扑为可数补拓扑.
% 这时\begin{equation*}
% 	\text{$X$的一个子集$X_1$是闭集}
% 	\iff
% 	X_1=X
% 	\lor
% 	\text{$X_1$是可数集}.
% \end{equation*}
% \end{proof}
\end{proposition}

\cref{theorem:序列.去心邻域内收敛于中心的序列的聚点2} 表明,
在一般的拓扑空间中,不能像在数学分析中那样,通过序列收敛的性质来刻画聚点.

\begin{theorem}\label{theorem:序列.连续映射的定义域中的序列与值域中的序列之间的关系1}
%@see: 《点集拓扑讲义(第四版)》(熊金城) P93 定理2.7.3
设\(X\)和\(Y\)是两个拓扑空间,映射\(f\colon X \to Y\),
则\begin{itemize}
	\item 如果\(f\)在点\(x_0 \in X\)连续,
	则\begin{equation*}
		\text{$X$中的一个序列$\{x_n\}_{n\geq0}$收敛于$x_0$}
		\implies
		\text{$Y$中的序列$\{f(x_n)\}_{n\geq0}$收敛于$f(x_0)$}.
	\end{equation*}

	\item 如果\(f\)连续,
	则\begin{equation*}
		\text{$X$中的一个序列$\{x_n\}_{n\geq0}$收敛于$x \in X$}
		\implies
		\text{$Y$中的序列$\{f(x_n)\}_{n\geq0}$收敛于$f(x)$}.
	\end{equation*}
\end{itemize}
%TODO proof
\end{theorem}

\begin{proposition}\label{theorem:序列.连续映射的定义域中的序列与值域中的序列之间的关系2}
%@see: 《点集拓扑讲义(第四版)》(熊金城) P93 例2.7.2
\cref{theorem:序列.连续映射的定义域中的序列与值域中的序列之间的关系1} 的逆命题不成立.
%TODO proof
\end{proposition}

\cref{theorem:序列.连续映射的定义域中的序列与值域中的序列之间的关系2} 表明,
在一般的拓扑空间中,不能像在数学分析中那样,通过序列收敛的性质来刻画映射的连续性.

至于在什么样的条件下,
\cref{theorem:序列.去心邻域内收敛于中心的序列的聚点1,theorem:序列.连续映射的定义域中的序列与值域中的序列之间的关系1} 的逆命题成立,
我们今后还要进行进一步的研究.

在度量空间中,序列的收敛可以通过度量来加以描述.
\begin{theorem}\label{theorem:序列.度量空间中的收敛序列}
%@see: 《点集拓扑讲义(第四版)》(熊金城) P94 定理2.7.4
设\((X,\rho)\)是一个度量空间,
\(\{x_n\}_{n\geq0}\)是\(X\)中的一个序列,\(x \in X\),
则以下命题等价:\begin{itemize}
	\item 序列\(\{x_n\}_{n\geq0}\)收敛于\(x\);
	\item 对于任意给定的实数\(\epsilon>0\),存在\(N\in\omega\),使得当\(n>N\)时有\(\rho(x_n,x)<\epsilon\);
	\item \(\lim_{n\to\infty} \rho(x_n,x) = 0\).
\end{itemize}
%TODO proof
\end{theorem}

% \begin{theorem}
% %@see: 《Real Analysis Modern Techniques and Their Applications Second Edition》(Folland) P14
% 设\(X\)是一个度量空间,\(E \subseteq X\),\(x \in X\),
% 则以下三个命题等价:\begin{itemize}
% 	\item \(x \in \overline{E}\).
% 	\item \((\forall r>0)[B(x,r) \cap E \neq \emptyset]\).
% 	\item 在\(E\)中存在一个序列\(\{x_n\}\)收敛于\(x\).
% \end{itemize}
% \begin{proof}
% 假设\(B(x,r) \cap E = \emptyset\),
% 则\(X-B(x,r)\)就是一个闭集,它包含\(E\)但不包括\(x\),
% 于是\(x \notin \overline{E}\).
% 再假设\(x \notin \overline{E}\),
% 因为\(X-\overline{E}\)是开集,
% 存在\(r>0\)使得\(B(x,r) \subseteq X-\overline{E} \subseteq X-E\).
% 因此\(x \in \overline{E} \iff (\forall r>0)[B(x,r) \cap E \neq \emptyset]\).

% 假设\((\forall r>0)[B(x,r) \cap E \neq \emptyset]\)成立,
% 对于\(\forall n\in\mathbb{N}\),
% 存在\(x_n \in B(x,n^{-1}) \cap E\),
% 使得\(x_n \to x\).
% 另一方面,假设\(B(x,r) \cap E = \emptyset\),
% 则\((\forall y \in E)[\rho(y,x) \geq r]\),
% 于是\(E\)中没有一个序列可以收敛于\(x\).
% 因此\((\forall r>0)[B(x,r) \cap E \neq \emptyset]
% \iff
% \text{在\(E\)中存在一个序列\(\{x_n\}\)收敛于\(x\)}\).
% \end{proof}
% \end{theorem}

\begin{example}
%@see: 《点集拓扑讲义(第四版)》(熊金城) P95 习题 3.
设\(X\)是一个度量空间.
证明:\begin{itemize}
	\item \(X\)中的任何一个收敛序列都只有唯一的极限;
	\item \cref{theorem:序列.去心邻域内收敛于中心的序列的聚点1} 的逆命题成立;
	\item 对于任何一个拓扑空间\(Y\)以及任何一个映射\(f\colon X \to Y\),\cref{theorem:序列.连续映射的定义域中的序列与值域中的序列之间的关系1} 的逆命题成立.
\end{itemize}
%TODO proof
\end{example}


\chapter{子空间,积空间,商空间}
在这一章中,我们介绍通过已知的拓扑空间构造新的拓扑空间的三种惯用办法.
为了便于理解,在讨论积空间的时候,我们将先讨论有限个空间的积空间,然后讨论一般情形.

\def\wT{\widetilde{\T}} % 拓扑子空间的相对拓扑
\def\woT{\widetilde{\oT}} % 拓扑子空间的闭集族
\section{子空间}
讨论拓扑空间的子空间的目的,在于对于拓扑空间中的一个给定的子集,
按某种“自然的方式”赋予它一个拓扑,使之成为一个拓扑空间,
以便将它作为一个独立的对象进行考察.
所谓“自然的方式”应当是什么样的方式?
为了回答这个问题,我们还是先从度量空间做起,以便得到必要的启发.

考虑一个度量空间和它的一个子集.
欲将这个子集看作一个度量空间,必须要为它的每一对点规定距离.
由于这个子集中的每一对点也是度量空间中的一对点,
因而把它们作为子集中的点的距离规定为它们作为度量空间中的点的距离当然是十分自然的.
我们把上述想法归纳成定义:
\begin{definition}
%@see: 《点集拓扑讲义(第四版)》(熊金城) P96 定义3.1.1
设\((X,\rho)\)是一个度量空间,
\(Y\)是\(X\)的一个子集,
% \(Y \times Y \subseteq X \times X\).
% \(\rho'\colon Y \times Y \to \mathbb{R}\)是\(Y\)的一个度量.
映射\(\rho'\)是\(\rho\)在\(Y \times Y\)上的限制\(\rho \SetRestrict (Y \times Y)\).
称“\(\rho'\)是由\(X\)的度量\(\rho\)诱导出来的”.
把\((Y,\rho)\)称为\((X,\rho)\)的一个\DefineConcept{度量子空间}.
\end{definition}

\begin{theorem}\label{theorem:子空间.度量子空间中的开集}
%@see: 《点集拓扑讲义(第四版)》(熊金城) P97 定理3.1.1
设\(Y\)是度量空间\(X\)的一个度量子空间,
则\(Y\)的子集\(U\)是\(Y\)中的一个开集,
当且仅当存在一个\(X\)中的开集\(V\),
使得\(U = V \cap Y\).
\begin{proof}
对于任意\(y \in Y\),
任取\(\epsilon>0\),
在度量空间\(X\)中\(y\)为中心、\(\epsilon\)为半径的球形邻域\(B_X(x,\epsilon)\),
和在度量空间\(Y\)中以\(y\)为中心、\(\epsilon\)为半径的球形邻域\(B_Y(y,\epsilon)\)
满足\begin{equation*}
	B_Y(y,\epsilon)
	= Y \cap B_X(y,\epsilon).
\end{equation*}
这是因为一个点\(x \in X\)属于\(B_Y(y,\epsilon)\),
当且仅当\(x\)是\(Y\)中的一个点,
并且它与\(y\)在\(Y\)中的距离(即它与\(y\)在\(X\)中的距离)小于\(\epsilon\).

现在设\(U\)是\(Y\)中的一个开集,
由于\(Y\)的所有球形邻域构成的集族是\(Y\)的拓扑的一个基,
\(U\)可以表示为\(Y\)中的一族球形邻域\(\mathscr{A}\)的并,
于是\begin{align*}
	U &= \bigcup_{B_Y(y,\epsilon) \in \mathscr{A}} B_Y(y,\epsilon)
	= \bigcup_{B_Y(y,\epsilon) \in \mathscr{A}} \left( Y \cap B_X(y,\epsilon) \right) \\
	&= Y \cap \left( \bigcup_{B_X(y,\epsilon) \in \mathscr{A}} B_X(y,\epsilon) \right).
\end{align*}
设\(V = \bigcup_{B_X(y,\epsilon) \in \mathscr{A}} B_X(y,\epsilon)\),
它是\(X\)中的一个开集,
并且我们有\(U = V \cap Y\).

反过来,假设\(U = V \cap Y\),其中\(V\)是\(X\)中的一个开集.
如果\(y \in U\),则有\(y \in Y\)和\(y \in V\).
于是在\(X\)中存在\(y\)的一个球形邻域\(B_X(y,\epsilon) \subseteq V\).
此时易见,\(B_Y(y,\epsilon) = Y \cap B_X(y,\epsilon) \subseteq U\).
这就证明\(U\)是\(Y\)中的一个开集.
\end{proof}
\end{theorem}

按照\cref{theorem:子空间.度量子空间中的开集} 的启示,
我们来逐步完成本节开始时所提出的任务.

\begin{definition}\label{definition:子空间.拓扑子空间中的集族的限制}
%@see: 《点集拓扑讲义(第四版)》(熊金城) P98 定义3.1.2
设\(\mathscr{A}\)是一个集族,\(Y\)是一个集合.
把集族\begin{equation*}
	\Set{ A \cap Y \given A \in \mathscr{A} }
\end{equation*}称为集族\(\mathscr{A}\)在集合\(Y\)上的\DefineConcept{限制},
记作\(\mathscr{A} \SetRestrict Y\).
%\cref{definition:映射.逆-复合-限制-像}
\end{definition}

\begin{lemma}
%@see: 《点集拓扑讲义(第四版)》(熊金城) P98 引理3.1.2
设\(Y\)是拓扑空间\((X,\T)\)的一个子集,
则集族\(\T \SetRestrict Y\)是\(Y\)的一个拓扑.
%TODO proof
\end{lemma}

\begin{definition}
%@see: 《点集拓扑讲义(第四版)》(熊金城) P98 定义3.1.3
设\(Y\)是拓扑空间\((X,\T)\)的一个子集,
\(Y\)的拓扑\(\T \SetRestrict Y\)称为
“(相对于\(X\)的拓扑\(\T\)而言的)\DefineConcept{相对拓扑}”,
“拓扑空间\((Y,\T \SetRestrict Y)\)称为拓扑空间\((X,\T)\)的一个\DefineConcept{拓扑子空间}”.
\end{definition}

假设\(Y\)是度量空间\(X\)的一个子空间.
现在有两个途径得到\(Y\)的拓扑:
一种途径是通过\(X\)的度量诱导出\(Y\)的度量,
然后考虑\(Y\)的这个度量诱导出来的拓扑;
另一种途径是先将\(X\)考虑成一个拓扑空间,
然后考虑\(Y\)的拓扑是\(X\)的拓扑在\(Y\)上引出来的相对拓扑.
事实上,\cref{theorem:子空间.度量子空间中的开集} 已经指出
经由这两种途径得到的\(Y\)的两个拓扑是一样的.
我们可以把这两种途径的等价性表述为\cref{theorem:子空间.引出子空间拓扑的两种途径的等价性}.
\begin{theorem}\label{theorem:子空间.引出子空间拓扑的两种途径的等价性}
%@see: 《点集拓扑讲义(第四版)》(熊金城) P99 定理3.1.3
设\(Y\)是度量空间\(X\)的一个度量子空间,
则\(X\)与\(Y\)都考虑作为拓扑空间时,
\(Y\)是\(X\)的一个拓扑子空间.
\end{theorem}

\begin{theorem}\label{theorem:子空间.亲子空间的传递性}
%@see: 《点集拓扑讲义(第四版)》(熊金城) P99 定理3.1.4
设\(X,Y,Z\)都是拓扑空间.
如果\(Y\)是\(X\)的一个拓扑子空间,
\(Z\)是\(Y\)的一个拓扑子空间,
则\(Z\)是\(X\)的一个拓扑子空间.
%TODO proof
\end{theorem}

\begin{theorem}
%@see: 《点集拓扑讲义(第四版)》(熊金城) P99 定理3.1.5
设\(Y\)是拓扑空间\(X\)的一个拓扑子空间,且\(y \in Y\).
\def\Uy{\mathscr{U}_y}
\def\wUy{\widetilde{\Uy}}
\begin{itemize}
	\item 分别记\(\T\)和\(\wT\)为\(X\)和\(Y\)的拓扑,
	则\(\wT = \T \SetRestrict Y\).
	\item 分别记\(\oT\)和\(\woT\)为\(X\)和\(Y\)的全体闭集,
	则\(\woT = \oT \SetRestrict Y\).
	\item 分别记\(\Uy\)和\(\wUy\)为点\(y\)在\(X\)和\(Y\)中的邻域系,
	则\(\wUy = \Uy \SetRestrict Y\).
\end{itemize}
%TODO proof
\end{theorem}

\begin{theorem}
%@see: 《点集拓扑讲义(第四版)》(熊金城) P100 定理3.1.6
设\(Y\)是拓扑空间\(X\)的一个拓扑子空间,\(A \subseteq Y\),
则\begin{itemize}
	\item \(A\)在\(Y\)中的导集是\(A\)在\(X\)中的导集与\(Y\)的交;
	\item \(A\)在\(Y\)中的闭包是\(A\)在\(X\)中的闭包与\(Y\)的交.
\end{itemize}
%TODO proof
\end{theorem}

\begin{theorem}
%@see: 《点集拓扑讲义(第四版)》(熊金城) P100 定理3.1.7
\def\Vy{\mathscr{V}_y}
设\(Y\)是拓扑空间\(X\)的一个拓扑子空间,且\(y \in Y\).
\begin{itemize}
	\item 如果\(\B\)是\(X\)的一个基,
	则\(\B \SetRestrict Y\)是\(Y\)的一个基.
	\item 如果\(\Vy\)是点\(y\)在\(X\)中的一个邻域基,
	则\(\Vy \SetRestrict Y\)是点\(y\)在\(Y\)中的一个邻域基.
\end{itemize}
%TODO proof
\end{theorem}

\subsection{拓扑空间之间的嵌入关系}
“子空间”事实上是从一个“大”拓扑空间中“切割”出来的一部分.
这里有一个反问题:
一个拓扑空间什么时候是另一个拓扑空间的子空间?
或者说,一个拓扑空间在什么条件下能够“镶嵌”到另一个拓扑空间中去?
当然,加入我们拘泥于某些细节,
例如涉及的拓扑空间是由什么样的点构成的,
那么问题会变得十分乏味;
然而我们在之前便提到过,
拓扑学的中心任务是研究拓扑不变性质,
也就是说我们不必区别同胚的两个拓扑空间.
于是,在这种意义下,以上问题可以精确地陈述如下:
\begin{definition}
%@see: 《点集拓扑讲义(第四版)》(熊金城) P101 定义3.1.4
设\(X,Y\)都是拓扑空间,
映射\(f\colon X \to Y\).
如果\(f\)是单射,
且\(f\)是\(X\)与\(f(X)\)之间的同胚\footnote{
	“\(f\)是\(X\)与\(f(X)\)之间的同胚”
	更准确地说应该是:
	映射\(\tilde{f}\colon X \to f(X), x \mapsto f(x)\)
	是\(X\)与\(f(X)\)之间的同胚.
},
则称“\(f\)是从\(X\)到\(Y\)的一个\DefineConcept{嵌入}”.
\end{definition}
\begin{definition}
%@see: 《点集拓扑讲义(第四版)》(熊金城) P101 定义3.1.4
设\(X,Y\)都是拓扑空间.
如果存在从\(X\)到\(Y\)的一个嵌入,
则称“拓扑空间\(X\)可以嵌入拓扑空间\(Y\)”.
\end{definition}
\begin{remark}
容易看出:
“拓扑空间\(X\)可以嵌入拓扑空间\(Y\)”的意思就是
“\(X\)与\(Y\)的某一个拓扑子空间同胚”.
因此,在同胚的意义下,\(X\)就是\(Y\)的一个拓扑子空间.
\end{remark}

\begin{example}
%@see: 《点集拓扑讲义(第四版)》(熊金城) P102
证明:一个离散空间,如果含有多于一个点,就绝不可能嵌入到任何一个平庸空间中去.
%TODO proof
\end{example}

\begin{example}
%@see: 《点集拓扑讲义(第四版)》(熊金城) P102
证明:一个平庸空间,如果含有多于一个点,就绝不可能嵌入到任意一个离散空间中去.
%TODO proof
\end{example}

\section{积空间(有限情形)}
\subsection{度量积空间}
给定两个拓扑空间,我们首先可以得到一个集合作为它们的笛卡尔积.
我们想要知道:如何按照某种自然的方式给定这个笛卡尔积一个拓扑,使之成为拓扑空间?

为此,我们先对度量空间中的同类问题进行研究.
首先回顾\cref{example:度量空间.欧氏空间} 中
\(n\)维欧氏空间\(\mathbb{R}^n\)中的度量是如何通过实数空间中的度量来定义的.
将欧氏空间的通常度量推广到有限个度量空间的笛卡尔积,不会产生任何困难.
\begin{definition}\label{definition:有限情形下的积空间.度量积空间}
%@see: 《点集拓扑讲义(第四版)》(熊金城) P104 定义3.2.1
\def\MatricSpaceCartesianProduct{(X_1,\rho_1),\allowbreak(X_2,\rho_2),\allowbreak\dotsc,\allowbreak(X_n,\rho_n)}
设\(\MatricSpaceCartesianProduct\)是\(n\ (n\geq1)\)个度量空间.
令\begin{gather*}
	X \defeq X_1 \times X_2 \times \dotsb \times X_n, \\
	\rho\colon X \times X \to \mathbb{R},
	(x,y) \mapsto \sqrt{\sum_{i=1}^n \rho_i(x_i,y_i)^2},
	\quad x=(\AutoTuple{x}{n}),y=(\AutoTuple{y}{n}) \in X.
\end{gather*}
把\(\rho\)称为“笛卡尔积\(X = X_1 \times X_2 \times \dotsb \times X_n\)的\DefineConcept{积度量}”.
把\((X,\rho)\)称为“度量空间\(\MatricSpaceCartesianProduct\)的\DefineConcept{度量积空间}”,
在不致混淆的情况下简称为\DefineConcept{积空间}.
\end{definition}

\begin{proposition}
\cref{definition:有限情形下的积空间.度量积空间} 中的积度量\(\rho\)是\(X\)的一个度量.
%TODO proof 需要用到施瓦茨引理
\end{proposition}

根据上述定义可知,\(n\)维欧氏空间\(\mathbb{R}^n\)就是\(n\)个实数空间\(\mathbb{R}\)的度量积空间.

\subsection{由积度量诱导出来的拓扑的性质}
现在来考察积度量所诱导出来的拓扑具有什么样的性质,
以便我们得到在拓扑空间中应该如何引出积空间中的概念的启示.
\begin{theorem}\label{theorem:有限情形下的积空间.由积度量诱导出来的拓扑的基}
%@see: 《点集拓扑讲义(第四版)》(熊金城) P104 定理3.2.1
\def\MatricSpaceCartesianProduct{(X_1,\rho_1),\allowbreak(X_2,\rho_2),\allowbreak\dotsc,\allowbreak(X_n,\rho_n)}
设\((X,\rho)\)是\(n\ (n\geq1)\)个度量空间\(\MatricSpaceCartesianProduct\)的度量积空间,
\(\T_i\ (i=1,2,\dotsc,n)\)是由\(\rho_i\)诱导出来的拓扑,
\(\T\)是由\(\rho\)诱导出来的拓扑,
则\(X\)的子集族\begin{equation*}
	\B \defeq \Set{
		U_1 \times U_2 \times \dotsb \times U_n
		\given
		U_i \in \T_i, i=1,2,\dotsc,n
	}
\end{equation*}
是\(X\)的拓扑\(\T\)的一个基.
%TODO proof
\end{theorem}

\subsection{拓扑积空间}
受到\cref{theorem:有限情形下的积空间.由积度量诱导出来的拓扑的基} 的启示,
我们因而引入有限个拓扑空间的积空间这一概念:
\begin{theorem}\label{theorem:有限情形下的积空间.积拓扑存在且唯一}
%@see: 《点集拓扑讲义(第四版)》(熊金城) P105 定理3.2.2
\def\MatricSpaceCartesianProduct{(X_1,\T_1),\allowbreak(X_2,\T_2),\allowbreak\dotsc,\allowbreak(X_n,\T_n)}
设\(\MatricSpaceCartesianProduct\)是\(n\ (n\geq1)\)个拓扑空间.
令\begin{gather*}
	X \defeq X_1 \times X_2 \times \dotsb \times X_n,
\end{gather*}
则\(X\)有唯一一个拓扑\(\T\)
以\(X\)的子集族\begin{equation*}
	\B \defeq \Set{
		U_1 \times U_2 \times \dotsb \times U_n
		\given
		U_i \in \T_i, i=1,2,\dotsc,n
	}
\end{equation*}
为它的一个基.
%TODO proof
\end{theorem}

\begin{definition}\label{definition:有限情形下的积空间.拓扑积空间}
%@see: 《点集拓扑讲义(第四版)》(熊金城) P106 定义3.2.2
\def\MatricSpaceCartesianProduct{(X_1,\T_1),\allowbreak(X_2,\T_2),\allowbreak\dotsc,\allowbreak(X_n,\T_n)}
设\(\MatricSpaceCartesianProduct\)是\(n\ (n\geq1)\)个拓扑空间.
令\begin{gather*}
	X \defeq X_1 \times X_2 \times \dotsb \times X_n.
\end{gather*}
把\cref{theorem:有限情形下的积空间.积拓扑存在且唯一} 中指出的拓扑\(\T\)
称为“笛卡尔积\(X = X_1 \times X_2 \times \dotsb \times X_n\)的\DefineConcept{积拓扑}”.
把\((X,\T)\)称为“拓扑空间\(\MatricSpaceCartesianProduct\)的\DefineConcept{拓扑积空间}”,
在不致混淆的情况下简称为\DefineConcept{积空间}.
\end{definition}

设\(\AutoTuple{X}{n}\)是\(n\ (n\geq1)\)个度量空间,
则笛卡尔积\(X = X_1 \times X_2 \times \dotsb \times X_n\)
可以由两种方式得到它的拓扑:
一种方式是先是求出\(X\)的积度量,再用这个积度量诱导出\(X\)的拓扑;
另一种方式是先用\(X_i\)的度量诱导出\(X_i\)的拓扑,再求出\(X\)的积拓扑.
\cref{theorem:有限情形下的积空间.由积度量诱导出来的拓扑的基} 实际上指明这两种拓扑是一致的,即
\begin{theorem}
%@see: 《点集拓扑讲义(第四版)》(熊金城) P106 定理3.2.3
设\(X\)是\(n\ (n\geq1)\)个度量空间\(\AutoTuple{X}{n}\)的度量积空间,
当把各个\(X_i\)以及\(X\)都视作拓扑空间时,
\(X\)就是\(\AutoTuple{X}{n}\)的拓扑积空间.
\end{theorem}

显然,\(n\)维欧氏空间\(\mathbb{R}^n\)就是\(n\)个实数空间\(\mathbb{R}\)的拓扑积空间.

\begin{theorem}\label{theorem:有限情形下的积空间.拓扑积空间的基}
%@see: 《点集拓扑讲义(第四版)》(熊金城) P106 定理3.2.4
设\(X\)是\(n\ (n\geq1)\)个拓扑空间\(\AutoTuple{X}{n}\)的拓扑积空间,
\(\B_i\)是\(X_i\)的一个基,
则\(X\)的子集族\begin{equation*}
	\tilde{\B} \defeq \Set{
		B_1 \times B_2 \times \dotsb \times B_n
		\given
		B_i \in \B_i, i=1,2,\dotsc,n
	}
\end{equation*}
是拓扑空间\(X\)的一个基.
%TODO proof
\end{theorem}

\begin{example}%\label{example:拓扑基.全体开方体是n维欧氏空间的一个基}
%@see: 《点集拓扑讲义(第四版)》(熊金城) P107 例3.2.1
证明:\(n\)维欧氏空间\(\mathbb{R}^n\)中所有开方体\begin{equation*}
	\tilde{\B} \defeq \Set{
		B_1 \times B_2 \times \dotsb \times B_n
		\given
		\text{$B_i = (a_i,b_i)$是开区间}, i=1,2,\dotsc,n
	}
\end{equation*}
构成\(\mathbb{R}^n\)的一个基.
\begin{proof}
由\cref{example:拓扑基.全体开区间是实数空间的一个基,theorem:有限情形下的积空间.拓扑积空间的基} 立即可得.
\end{proof}
\end{example}

\begin{theorem}\label{theorem:有限情形下的积空间.由各坐标空间中的开集在投射的逆下的像组成的子基}
%@see: 《点集拓扑讲义(第四版)》(熊金城) P107 定理3.2.5
\def\MatricSpaceCartesianProduct{(X_1,\T_1),\allowbreak(X_2,\T_2),\allowbreak\dotsc,\allowbreak(X_n,\T_n)}
设\((X,\T)\)是\(n\ (n\geq1)\)个拓扑空间\(\MatricSpaceCartesianProduct\)的拓扑积空间,
映射\(p_i\)是从\(X\)到\(X_i\)的投射,
则\(X\)的子集族\begin{equation*}
	\S \defeq \Set{
		p_i^{-1}(U_i)
		\given
		U_i \in \T_i, i=1,2,\dotsc,n
	}
\end{equation*}
是\(X\)的一个子基.
%TODO proof
\end{theorem}

\subsection{开映射,闭映射}
\begin{definition}
%@see: 《点集拓扑讲义(第四版)》(熊金城) P108 定义3.2.3
设\(X,Y\)都是拓扑空间,
映射\(f\colon X \to Y\).
\begin{itemize}
	\item 如果对于\(X\)中的任意一个开集\(U\),
	\(f(U)\)是\(Y\)中的开集,
	则称“\(f\)是一个\DefineConcept{开映射}”.
	\item 如果对于\(X\)中的任意一个闭集\(U\),
	\(f(U)\)是\(Y\)中的闭集,
	则称“\(f\)是一个\DefineConcept{闭映射}”.
\end{itemize}
\end{definition}

\begin{theorem}\label{theorem:有限情形下的积空间.投射是开映射}
%@see: 《点集拓扑讲义(第四版)》(熊金城) P108 定理3.2.6
设\(X\)是\(n\ (n\geq1)\)个拓扑空间\(\AutoTuple{X}{n}\)的拓扑积空间,
则从\(X\)到\(X_i\)的投射\(p_i\)是满的连续开映射.
%TODO proof
%\cref{theorem:一般情形下的积空间.投射是开映射}
\end{theorem}

\begin{theorem}\label{theorem:有限情形下的积空间.投射与映射的复合的连续性}
%@see: 《点集拓扑讲义(第四版)》(熊金城) P108 定理3.2.7
设\(X\)是\(n\ (n\geq1)\)个拓扑空间\(\AutoTuple{X}{n}\)的拓扑积空间,
\(Y\)是一个拓扑空间,
\(p_i\)是从\(X\)到\(X_i\)的投射,
则映射\(f\colon Y \to X\)连续,
当且仅当
复合映射\(p_i \circ f\ (i=1,2,\dotsc,n)\)都连续.
%TODO proof
%\cref{theorem:一般情形下的积空间.投射与映射的复合的连续性}
\end{theorem}

\begin{theorem}\label{theorem:有限情形下的积空间.积拓扑是使所有投射都连续的最小拓扑}
%@see: 《点集拓扑讲义(第四版)》(熊金城) P109 定理3.2.8
设\(X\)是\(n\ (n\geq1)\)个拓扑空间\(\AutoTuple{X}{n}\)的拓扑积空间,
\(\T\)是\(X\)的积拓扑.
如果对于\(X\)的一个拓扑\(\wT\)而言,
当\(i=1,2,\dotsc,n\)时,
从\(X\)到\(X_i\)的投射\(p_i\)总是连续映射,
那么\(\wT \supseteq \T\).
%TODO proof
%\cref{theorem:一般情形下的积空间.积拓扑是使所有投射都连续的最小拓扑}
\end{theorem}

\begin{theorem}
%@see: 《点集拓扑讲义(第四版)》(熊金城) P109 定理3.2.9
设\(\AutoTuple{X}{n}\)是\(n\ (n\geq2)\)个拓扑空间,
则拓扑积空间\(X_1 \times X_2 \times \dotsb \times X_{n-1} \times X_n\)
同胚于拓扑积空间\((X_1 \times X_2 \times \dotsb \times X_{n-1}) \times X_n\).
%TODO proof
\end{theorem}
上述定理说明:
尽管\(X_1 \times X_2 \times \dotsb \times X_{n-1} \times X_n\)
和\((X_1 \times X_2 \times \dotsb \times X_{n-1}) \times X_n\)
作为集合可以是完全不同的,但是,在同胚的意义下,两者却别无二致.
上述定理还说明:
在同胚的意义下,有限个拓扑空间的积空间可以通过归纳的方式予以定义.

\section{积空间(一般情形)}

\section{商空间}
将一条橡皮筋的两个端点“粘合”起来,便可得到一个橡皮圈.
将一块正方形橡皮的一对边“粘合”起来,便可得到一根橡皮管,
继续将橡皮管两端“粘合”起来,便可得到一个橡皮轮胎.
像这样,从一个给定的图形出发,构造出一个新图形的办法可以一般化.

\begin{proposition}
%@see: 《点集拓扑讲义(第四版)》(熊金城) P116 定义3.4.1
设\((X,\T)\)是一个拓扑空间,
\(Y\)是一个集合,
\(f\colon X \to Y\)是一个满射,
则\(Y\)的子集族\begin{equation*}
	\Set{
		U \subseteq Y
		\given
		f^{-1}(U) \in \T
	}
\end{equation*}
是\(Y\)的一个拓扑.
\end{proposition}

\begin{definition}
%@see: 《点集拓扑讲义(第四版)》(熊金城) P116 定义3.4.1
设\((X,\T)\)是一个拓扑空间,
\(Y\)是一个集合,
\(f\colon X \to Y\)是一个满射.
把\begin{equation*}
	\Set{
		U \subseteq Y
		\given
		f^{-1}(U) \in \T
	}
\end{equation*}
称为“\(Y\)的(相对于满射\(f\)而言的)\DefineConcept{商拓扑}”.
\end{definition}


\chapter{连通性}
本章讨论拓扑空间的几种拓扑不变性质,
包括连通性、局部连通性和道路连通性,
并且涉及某些简单的应用.
这些拓扑不变性质的研究也使我们能够区别一些互不同胚的空间.

\section{连通空间}
我们先通过直观的方式考察一个例子.
在实数空间\(\mathbb{R}\)中,
\((0,1)\)和\([1,2)\)这两个区间尽管互不相交,
但是它们的并\((0,1)\cup[1,2)=(0,2)\)却可以构成一个“整体”.
再看\((0,1)\)和\((1,2)\)这两个区间,
它们的并\((0,1)\cup(1,2)\)却明显是两个“部分”.
产生上述不同情况的原因在于,
在前一种情况中,区间\((0,1)\)有一个聚点\(1\)在\([1,2)\)中;
在后一种情况中,两个区间中的任何一个都没有聚点属于另一个区间.
于是我们可以抽象出如下概念:
\begin{definition}
%@see: 《点集拓扑讲义(第四版)》(熊金城) P122 定义4.1.1
设\(A,B\)都是拓扑空间\(X\)的子集.
如果\begin{equation*}
	(A \cap \overline{B}) \cup (B \cap \overline{A}) = \emptyset,
\end{equation*}
则称“\(A\)与\(B\)是\DefineConcept{隔离的}”.
\end{definition}

\section{连通性的某些简单应用}
我们知道,实数集\(\mathbb{R}\)中区间可以分为九类:\begin{gather*}
	(-\infty,+\infty),
	(a,+\infty),
	[a,+\infty),
	(-\infty,a),
	(-\infty,a],
	(a,b),
	(a,b],
	[a,b),
	[a,b].
\end{gather*}

在\cref{theorem:连通空间.实数空间是连通空间} 中
我们证明了实数空间\(\mathbb{R}\)是一个连通空间.
由于\((a,+\infty),(-\infty,a),(a,b)\)都同胚于\(\mathbb{R}\)
%TODO 写出对应的同胚映射
所以这3个区间也都是连通的.
由于\begin{gather*}
	\TopoClosureL{(a,+\infty)}
	= [a,+\infty), \\
	\TopoClosureL{(-\infty,a)}
	= (-\infty,a], \\
	(a,b) \subseteq [a,b) \subseteq [a,b], \\
	(a,b) \subseteq (a,b] \subseteq [a,b],
\end{gather*}
那么根据\cref{theorem:连通空间.连通性的夹逼准则} 可知
区间\([a,+\infty),(-\infty,a],[a,b),(a,b],[a,b]\)都是连通的.

假设\(E\)是\(\mathbb{R}\)的一个子集,
且\(E\)中至少有两个点.
如果\(E\)不是一个区间,
则存在\(a,b \in E\)满足\(a < b\),
使得\([a,b] \not\subseteq E\).
换言之,
如果\(E\)不是一个区间,
则存在\(a,b \in E\)满足\(a < c < b\),
使得\(c \notin E\),
若令\(A \defeq (-\infty,c) \cap E,
B \defeq (c,+\infty) \cap E\),
则有\(A,B\)都是\(E\)的非空的开集,
并且有\(A \cup B = E\)和\(A \cap B = \emptyset\)同时成立,
因此\(E\)不连通.
此外,当\(E\)是空集或单点集时,
\(E\)显然是一个区间.

综上以上两个方面,我们已经证明了:
\begin{theorem}
%@see: 《点集拓扑讲义(第四版)》(熊金城) P131 定理4.2.1
设\(E\)是实数空间\(\mathbb{R}\)的一个子集,
\(E\)是一个连通子集,
当且仅当\(E\)是一个区间.
%TODO proof
\end{theorem}

\begin{theorem}\label{theorem:连通空间.从连通空间到实数域的连续映射1}
%@see: 《点集拓扑讲义(第四版)》(熊金城) P131 定理4.2.2
设\(X\)是一个连通空间,
\(f\colon X \to \mathbb{R}\)是一个连续映射,
则\(f(X)\)是\(\mathbb{R}\)中的一个区间.
%TODO proof
\end{theorem}

\begin{theorem}\label{theorem:连通空间.从连通空间到实数域的连续映射2}
%@see: 《点集拓扑讲义(第四版)》(熊金城) P131 定理4.2.2
设\(X\)是一个连通空间,
\(f\colon X \to \mathbb{R}\)是一个连续映射,
则\begin{equation*}
	(\forall a,b \in X)
	(\forall t\in\mathbb{R})
	(\exists \xi \in X)
	[
		f(a) \leq t \leq f(b)
		\implies
		f(\xi) = t
	].
\end{equation*}
%TODO proof
\end{theorem}
根据\cref{theorem:连通空间.从连通空间到实数域的连续映射2} 立即可以推出介值定理和不动点定理.

%@see: 《点集拓扑讲义(第四版)》(熊金城) P131 定理4.2.3(介值定理)
%@see: 《点集拓扑讲义(第四版)》(熊金城) P131 定理4.2.4(不动点定理)

\begin{example}
%@see: 《点集拓扑讲义(第四版)》(熊金城) P131
证明:在欧氏平面\(\mathbb{R}^2\)中,单位圆周\(S^1\)是连通的.
%TODO proof
\end{example}

\begin{definition}
%@see: 《点集拓扑讲义(第四版)》(熊金城) P132
在欧氏空间\(\mathbb{R}^2\)中,
点\(x=(x_1,x_2) \in S^1\).
把\((-x_1,-x_2)\)称为“点\(x\)的\DefineConcept{对径点}”,
记作\(-x\).
把映射\begin{equation*}
	r\colon S^1 \to S^1, x \mapsto -x
\end{equation*}
称为\DefineConcept{对径映射}.
\end{definition}

\begin{example}\label{example:连通空间.从单位圆周到实数域的连续映射在一对对径点的值相等}
%@see: 《点集拓扑讲义(第四版)》(熊金城) P132 定理4.2.5(Borsuk-Ulam定理)
设\(f\colon S^1 \to \mathbb{R}\)是一个连续映射.
证明:在\(S^1\)中存在一对对径点\(x\)和\(-x\),使得\(f(x) = f(-x)\).
%TODO proof
\end{example}

\begin{example}
%@see: 《点集拓扑讲义(第四版)》(熊金城) P132 定理4.2.6
证明:\(n\ (n>1)\)维欧氏空间\(\mathbb{R}^n\)的子集\(\mathbb{R}^n-\{0\}\)是一个连通子集.
%TODO proof
\end{example}

下面再给出一个利用拓扑不变性质判定两个空间不同胚的例子:
\begin{example}
%@see: 《点集拓扑讲义(第四版)》(熊金城) P132 定理4.2.7
证明:欧氏平面\(\mathbb{R}^2\)和实数空间\(\mathbb{R}\)不同胚.
%TODO proof
\end{example}

%TODO 下面3个定理的证明需要用到代数拓扑学(例如同调论、同伦论)

\begin{theorem}
%@see: 《点集拓扑讲义(第四版)》(熊金城) P133 定理4.2.8(Brouwer不动点定理)
设\(E^n\)是\(n\)维闭球体,
\(f\colon E^n \to E^n\)是一个连续映射,
则存在\(z \in E^n\),使得\(f(z) = z\).
%TODO proof
\end{theorem}

\begin{theorem}
%@see: 《点集拓扑讲义(第四版)》(熊金城) P133 定理4.2.9(Borsuk-Ulam定理)
设\(f\colon S^n \to \mathbb{R}^m\ (n \geq m)\)是一个连续映射,
则存在\(x \in S^n\),使得\(f(x) = f(-x)\).
%TODO proof
\end{theorem}

\begin{theorem}
%@see: 《点集拓扑讲义(第四版)》(熊金城) P133 定理4.2.10
如果\(n \neq m\),则欧氏空间\(\mathbb{R}^n\)与\(\mathbb{R}^m\)不同胚.
%TODO proof
\end{theorem}

\section{连通分支}
\begin{definition}
%@see: 《点集拓扑讲义(第四版)》(熊金城) P134 定理4.3.1
设\(X\)是一个拓扑空间,点\(a,b \in X\).
如果\(X\)中有一个连通子集\(Y \supseteq \{a,b\}\),
则称“点\(a\)和\(b\)是\DefineConcept{连通的}”.
\end{definition}

容易证明,两点之间的连通关系,满足自反性、对称性、传递性,是一个等价关系.

\begin{definition}
%@see: 《点集拓扑讲义(第四版)》(熊金城) P134 定义4.3.2
设\(X\)是一个拓扑空间,点\(a \in X\),
把\(a\)在连通关系下的等价类\begin{equation*}
	\Set{
		x \in X
		\given
		\text{$x$和$a$是连通的}
	}
\end{equation*}
称为“拓扑空间\(X\)的一个\DefineConcept{连通分支}”.
\end{definition}

\begin{definition}
%@see: 《点集拓扑讲义(第四版)》(熊金城) P134 定义4.3.2
设\(X\)是一个拓扑空间,
\(Y\)是\(X\)的一个子集.
\(Y\)作为\(X\)的子空间的每一个连通分支,
称为“拓扑空间\(X\)的子集\(Y\)的一个\DefineConcept{连通分支}”.
\end{definition}

\begin{proposition}
%@see: 《点集拓扑讲义(第四版)》(熊金城) P135
拓扑空间\(X\ (\neq\emptyset)\)的每一个连通分支都不是空集.
%TODO proof
\end{proposition}

\begin{proposition}
%@see: 《点集拓扑讲义(第四版)》(熊金城) P135
拓扑空间\(X\ (\neq\emptyset)\)的不同的连通分支互斥.
%TODO proof
\end{proposition}

\begin{proposition}
%@see: 《点集拓扑讲义(第四版)》(熊金城) P135
拓扑空间\(X\ (\neq\emptyset)\)的所有连通分支之并就是\(X\)本身.
%TODO proof
\end{proposition}

\begin{proposition}
%@see: 《点集拓扑讲义(第四版)》(熊金城) P135
设\(X\)是一个拓扑空间,
点\(a,b \in X\),
则\(a,b\)属于\(X\)的同一个连通分支,
当且仅当\(a\)和\(b\)连通.
%TODO proof
\end{proposition}

\begin{proposition}
%@see: 《点集拓扑讲义(第四版)》(熊金城) P135
设\(X\)是一个拓扑空间,
\(Y\)是\(X\)的一个子集,
点\(a,b \in Y\),
则\(a,b\)属于\(Y\)的同一个连通分支,
当且仅当\(Y\)有一个连通子集同时包含\(a\)和\(b\).
%TODO proof
\end{proposition}

\begin{theorem}
%@see: 《点集拓扑讲义(第四版)》(熊金城) P135 定理4.3.1
设\(X\)是一个拓扑空间,
\(C\)是\(X\)的一个连通分支,
则\begin{itemize}
	\item 如果\(Y\)是\(X\)的一个连通子集,
	并且\(Y \cap C \neq \emptyset\),
	则\(Y \subseteq C\);

	\item \(C\)是一个连通子集;

	\item \(C\)是一个闭集.
\end{itemize}
%TODO proof
\end{theorem}

一般地说,连通分支可以不是开集.
例如,考虑有理数集\(\mathbb{Q}\)(作为实数空间\(\mathbb{R}\)的子空间),
假设\(x,y\in\mathbb{Q}\)满足\(x<y\),
如果\(\mathbb{Q}\)的一个子集\(E\)同时包含\(x,y\),
令\begin{equation*}
	A \defeq (-\infty,r) \cap E,
	B \defeq (r,+\infty) \cap E,
\end{equation*}
其中\(r\)是任何一个无理数,且\(x<r<y\),
此时易见\(A,B\)都是\(E\)的非空开集,并且\(E = A \cup B\),
因此\(E\)是非连通的.
以上论述说明:\(\mathbb{Q}\)中任意一个包含着多于一个点的集合都是不连通的,
或者说,\(\mathbb{Q}\)的连通分支都是单点集.
然而,\(\mathbb{Q}\)中每一个单点集都不是开集.

\begin{definition}
%@see: 《点集拓扑讲义(第四版)》(熊金城) P136
设\(P(x)\)是一个谓词公式,
\(X\)是拓扑空间族\(\{X_\gamma\}_{\gamma \in \Gamma}\)的拓扑积空间.
如果
	只要\(\{X_\gamma\}_{\gamma \in \Gamma}\)中每个拓扑空间\(X_\gamma\)都满足\(P(X_\gamma)\),
	就有\(X\)也满足\(P(X)\),
则称“\(P\)是一个\DefineConcept{可积性质}”.
\end{definition}
显然,有限可积性质是一类特殊的可积性质.
既然有限个拓扑空间的积空间是一族拓扑空间的积空间的特殊情形,
那么一个性质只要不是有限可积性质,那就一定不是可积性质.
但要注意到并非每一个有限可积性质都是可积性质.

\begin{theorem}
%@see: 《点集拓扑讲义(第四版)》(熊金城) P136 定理4.3.2
任何一族连通空间的积空间都是连通空间.
%TODO proof
\end{theorem}

\section{局部连通空间}
引进新的概念之前,我们先来考察一个例子.
\begin{example}
%@see: 《点集拓扑讲义(第四版)》(熊金城) P139 例4.4.1
在欧氏空间\(\mathbb{R}^2\)中,
令\begin{equation*}
	S \defeq \Set*{
		(x,\sin(1/x))
		\given
		x\in(0,1]
	},
	\qquad
	T \defeq \{0\}\times[-1,1],
\end{equation*}
把\(S\)称为\DefineConcept{拓扑学家的正弦曲线}.
显然\(S\)是区间\((0,1]\)在一个连续映射下的像,因此\(S\)是连通的.
此外,容易验证\(\TopoClosureL{S} = S \cup T\),
因此\(\TopoClosureL{S}\)也是连通的.
尽管如此,倘若我们查看\(\TopoClosureL{S}\)中的点,容易发现它们明显地分为两类:
\(S\)中每一个点的任何一个“较小的”邻域中都包含着一个连通的邻域,
而\(T\)中的每一个点的任何一个邻域都是不连通的.
\end{example}
我们用以下术语将这两个类型的点区别开来.
\begin{definition}
%@see: 《点集拓扑讲义(第四版)》(熊金城) P139 定义4.4.1
设\(X\)是一个拓扑空间,\(x \in X\).
如果\(x\)的每一个邻域\(U\)中,都包含着\(x\)的某一个连通的邻域\(V\),
则称“拓扑空间\(X\)在点\(x\)是\DefineConcept{局部连通的}”.
\end{definition}
\begin{definition}
%@see: 《点集拓扑讲义(第四版)》(熊金城) P139 定义4.4.1
设\(X\)是一个拓扑空间.
如果\(X\)在它的每一个点都是局部连通的,
则称“拓扑空间\(X\)是\DefineConcept{局部连通空间}”.
\end{definition}

根据上述定义,拓扑学家的正弦曲线\(S\)的闭包\(\TopoClosureL{S}\)
在其属于\(S\)的每一个点都是局部连通的,
而在其属于\(T\)的每一个点都不是局部连通的,
因此,尽管\(\TopoClosureL{S}\)是一个连通空间,但它却不是一个局部连通空间.

局部连通的拓扑空间也不必是连通的.
例如,每一个离散空间都是局部连通空间,但是包含多于一个点的离散空间却不是连通空间.
又例如,\(n\)维欧氏空间\(\mathbb{R}^n\)的任何一个开子空间都是局部连通的
(这是因为每一个球形邻域都同胚于整个欧氏空间\(\mathbb{R}^n\),因而是连通的),
特别地,欧氏空间\(\mathbb{R}^n\)本身是局部连通的;
另一方面,欧氏空间\(\mathbb{R}^n\)中两个互斥的非空开集的并作为子空间就一定不是连通的.

\begin{proposition}
%@see: 《点集拓扑讲义(第四版)》(熊金城) P140
拓扑空间\(X\)在点\(x \in X\)是局部连通的,
当且仅当\(x\)的所有连通邻域构成点\(x\)的一个邻域基.
%TODO proof
\end{proposition}

\begin{theorem}
%@see: 《点集拓扑讲义(第四版)》(熊金城) P140 定理4.4.1
设\(X\)是一个拓扑空间,则以下命题等价:\begin{itemize}
	\item \(X\)是一个局部连通空间;
	\item \(X\)的任何一个开集的任何一个连通分支都是开集;
	\item \(X\)有一个基,它的每一个元素都是连通的.
\end{itemize}
%TODO proof
\end{theorem}

\begin{corollary}
%@see: 《点集拓扑讲义(第四版)》(熊金城) P140
局部连通空间的每一个连通分支都是开集.
%TODO proof
\end{corollary}

\begin{theorem}
%@see: 《点集拓扑讲义(第四版)》(熊金城) P140 定理4.4.2
设\(X,Y\)都是拓扑空间,
\(X\)是局部连通的,
\(f\colon X \to Y\)是一个连续开映射,
则\(f(X)\)是一个局部连通空间.
%TODO proof
\end{theorem}
\begin{remark}
%@see: 《点集拓扑讲义(第四版)》(熊金城) P141
由上述定理可知,拓扑空间的局部连通性是一个拓扑不变性质.
\end{remark}

\begin{theorem}
%@see: 《点集拓扑讲义(第四版)》(熊金城) P141 定理4.4.3
设\(\AutoTuple{X}{n}\)是\(n\ (n\geq1)\)个局部连通空间,
则拓扑积空间\(\AutoTuple{X}{n}[\times]\)也是局部连通空间.
%TODO proof
\end{theorem}

应用这些定理,有些事情说起来就会简单得多.
例如,由于所有开区间构成实数空间\(\mathbb{R}\)的一个基,所以它是局部连通的.
又例如,\(n\)维欧氏空间\(\mathbb{R}^n\)是\(n\)个实数空间的积空间,所以它也是局部连通的.

\section{道路连通空间}
下面讨论\DefineConcept{道路连通性}(path connectedness).

\begin{definition}
%@see: 《点集拓扑讲义(第四版)》(熊金城) P142 定义4.5.1
设\(X\)是一个拓扑空间,
\(f\colon [0,1] \to X\)一个连续映射,
\(a=f(0),
b=f(1)\),
则称“\(f\)是拓扑空间\(X\)中从\(a\)到\(b\)的一条\DefineConcept{道路}”,
把\(f\)的像\(f([0,1])\)称为“拓扑空间\(X\)中的一条\DefineConcept{曲线}”
或“拓扑空间\(X\)中的一条\DefineConcept{弧}”,
把\(a\)称为“道路\(f\)的\DefineConcept{起点}”或“曲线\(f([0,1])\)的{起点}”,
把\(b\)称为“道路\(f\)的\DefineConcept{终点}”或“曲线\(f([0,1])\)的{终点}”.
\end{definition}

\begin{definition}
%@see: 《点集拓扑讲义(第四版)》(熊金城) P142 定义4.5.1
起点和终点相同的道路,称为\DefineConcept{回路},并且此时它的起点称为\DefineConcept{基点}.
\end{definition}

\begin{definition}
%@see: 《点集拓扑讲义(第四版)》(熊金城) P142 定义4.5.2
设\(X\)是一个拓扑空间.
如果对于任意\(a,b \in X\),\(X\)中存在一条从\(a\)到\(b\)的一条道路,
则称\(X\)是一个\DefineConcept{道路连通空间}.
\end{definition}

\begin{definition}
%@see: 《点集拓扑讲义(第四版)》(熊金城) P142 定义4.5.2
设\(Y\)是道路连通空间\(X\)的一个子集.
如果\(Y\)作为\(X\)的子空间是一个道路连通空间,
则称“\(Y\)是\(X\)的一个\DefineConcept{道路连通子集}”.
\end{definition}

实数空间\(\mathbb{R}\)是道路连通的.
这是因为任取\(a,b \in \mathbb{R}\),
则连续映射\(f\colon [0,1] \to \mathbb{R}, t \mapsto a+t(b-a)\)
便是\(\mathbb{R}\)中以\(a\)为起点、以\(b\)为终点的一条道路.
容易验证,\(\mathbb{R}\)中任何一个区间都是道路连通的.

\begin{theorem}\label{theorem:道路连通空间.道路连通空间一定是连通空间}
%@see: 《点集拓扑讲义(第四版)》(熊金城) P142 定理4.5.1
如果拓扑空间\(X\)是一个道路连通空间,
则\(X\)必然是一个连通空间.
%TODO proof
\end{theorem}

连通空间可以不是道路连通的.
例如,拓扑学家的正弦曲线\(S\)的闭包\(\TopoClosureL{S}\)是一个连通空间,但是它不是一个道路连通空间.

道路连通性与局部连通性之间更没有必然的蕴含关系.
例如,离散空间都是局部连通的,然而包含多于一个点的离散空间都不是连通空间,当然也就不是道路连通空间了.

\begin{theorem}
%@see: 《点集拓扑讲义(第四版)》(熊金城) P143 定理4.5.2
设\(X,Y\)都是拓扑空间,
\(X\)是道路连通的,
\(f\colon X \to Y\)是一个连续映射,
则\(f(X)\)是道路连通的.
%TODO proof
\end{theorem}
\begin{remark}
由上述定理可知,拓扑空间的道路连通性是一个拓扑不变性质,也是一个可商性质.
\end{remark}

\begin{theorem}
%@see: 《点集拓扑讲义(第四版)》(熊金城) P143 定理4.5.3
设\(\AutoTuple{X}{n}\)是\(n\ (n\geq1)\)个道路连通空间,
则拓扑积空间\(\AutoTuple{X}{n}[\times]\)也是道路连通空间.
%TODO proof
\end{theorem}

根据上述定理容易证明:\(n\)维欧氏空间\(\mathbb{R}^n\)是一个道路连通空间.

\begin{theorem}\label{theorem:道路连通空间.粘结引理}
%@see: 《点集拓扑讲义(第四版)》(熊金城) P144 定理4.5.4
设\(A,B\)是拓扑空间\(X\)中的两个开集(或闭集),
\(X = A \cup B\),
\(Y\)是一个拓扑空间,
\(f_1\colon A \to Y\)和\(f_2\colon B \to Y\)都是连续映射,
且满足\(f_1 \SetRestrict (A \cap B) = f_2 \SetRestrict (A \cap B)\),
定义映射\(f\colon X \to Y\)使之满足\begin{equation*}
	f(x) \defeq \left\{ \begin{array}{cl}
		f_1(x), & x \in A, \\
		f_2(x), & x \in B,
	\end{array} \right.
\end{equation*}
则\(f\)是一个连续映射.
%TODO proof
\end{theorem}

\begin{definition}
%@see: 《点集拓扑讲义(第四版)》(熊金城) P144 定义4.5.3
设\(X\)是一个拓扑空间,\(a,b \in X\).
如果\(X\)中有一条从\(a\)到\(b\)的道路,
则称“\(a\)和\(b\)是\DefineConcept{道路连通的}”.
\end{definition}

容易证明,两点之间的道路连通关系,满足自反性、对称性、传递性,是一个等价关系.

\begin{definition}
%@see: 《点集拓扑讲义(第四版)》(熊金城) P145 定义4.5.4
设\(X\)是一个拓扑空间,点\(a \in X\),
把\(a\)在道路连通关系下的等价类\begin{equation*}
	\Set{
		x \in X
		\given
		\text{$x$和$a$是道路连通的}
	}
\end{equation*}
称为“拓扑空间\(X\)的一个\DefineConcept{道路连通分支}”.
\end{definition}

\begin{definition}
%@see: 《点集拓扑讲义(第四版)》(熊金城) P145 定义4.5.4
设\(X\)是一个拓扑空间,
\(Y\)是\(X\)的一个子集.
\(Y\)作为\(X\)的子空间的每一个道路连通分支,
称为“拓扑空间\(X\)的子集\(Y\)的一个\DefineConcept{道路连通分支}”.
\end{definition}

\begin{proposition}
%@see: 《点集拓扑讲义(第四版)》(熊金城) P145
拓扑空间\(X\ (\neq\emptyset)\)的每一个道路连通分支都不是空集.
%TODO proof
\end{proposition}

\begin{proposition}
%@see: 《点集拓扑讲义(第四版)》(熊金城) P145
拓扑空间\(X\ (\neq\emptyset)\)的不同的道路连通分支互斥.
%TODO proof
\end{proposition}

\begin{proposition}
%@see: 《点集拓扑讲义(第四版)》(熊金城) P145
拓扑空间\(X\ (\neq\emptyset)\)的所有道路连通分支之并就是\(X\)本身.
%TODO proof
\end{proposition}

\begin{proposition}
%@see: 《点集拓扑讲义(第四版)》(熊金城) P145
设\(X\)是一个拓扑空间,
点\(a,b \in X\),
则\(a,b\)属于\(X\)的同一个道路连通分支,
当且仅当\(a\)和\(b\)道路连通.
%TODO proof
\end{proposition}

\begin{proposition}
%@see: 《点集拓扑讲义(第四版)》(熊金城) P145
设\(X\)是一个拓扑空间,
\(Y\)是\(X\)的一个子集,
点\(a,b \in Y\),
则\(a,b\)属于\(Y\)的同一个道路连通分支,
当且仅当\(Y\)中有一个从\(a\)到\(b\)的道路.
%TODO proof
\end{proposition}

由定义可知,拓扑空间中每一个道路连通分支\(A\),都是一个道路连通子集;
由\cref{theorem:道路连通空间.道路连通空间一定是连通空间} 可知
\(A\)也是一个连通子集;
由\cref{theorem:连通分支.连通分支的性质1} 可知,
\(A\)必然包含于某个连通分支.

作为\cref{theorem:道路连通空间.道路连通空间一定是连通空间} 在某种特定情形下的一个逆命题,
我们有如下命题:
\begin{proposition}
%@see: 《点集拓扑讲义(第四版)》(熊金城) P146 定理4.5.5
\(n\)维欧氏空间\(\mathbb{R}^n\)的任何一个连通开集都是道路连通的.
%TODO proof
\end{proposition}

\begin{proposition}
%@see: 《点集拓扑讲义(第四版)》(熊金城) P146 推论4.5.6
\(n\)维欧氏空间\(\mathbb{R}^n\)中任何开集的每一个道路连通分支同时也是它的一个连通分支.
%TODO proof
\end{proposition}

\begin{definition}
%@see: 《点集拓扑讲义(第四版)》(熊金城) P147 习题 5.
设\(X\)是一个拓扑空间,\(x \in X\).
如果\(x\)的每一个邻域\(U\)中,都包含着\(x\)的某一个道路连通的邻域\(V\),
则称“拓扑空间\(X\)在点\(x\)是\DefineConcept{局部道路连通的}”.
\end{definition}
\begin{definition}
%@see: 《点集拓扑讲义(第四版)》(熊金城) P147 习题 5.
设\(X\)是一个拓扑空间.
如果\(X\)在它的每一个点都是局部道路连通的,
则称“拓扑空间\(X\)是\DefineConcept{局部道路连通空间}”.
\end{definition}
\begin{definition}
%@see: 《点集拓扑讲义(第四版)》(熊金城) P147 习题 5.
设\(X\)是一个拓扑空间,
\(Y\)是\(X\)的一个子集.
如果\(Y\)作为\(X\)的子空间是一个道路连通空间,
则称“\(Y\)是\(X\)的一个\DefineConcept{局部道路连通子集}”.
\end{definition}


\chapter{有关可数性的公理}
\section{可数性公理}
我们知道,基和邻域基,对于确定拓扑空间的拓扑和验证映射的连续性,都有着重要的意义,
它们的元素的“个数”越少,讨论起来就越是方便.
因此我们试图对拓扑空间的基或邻域基的基数加以限制,
但又希望加了限制的拓扑空间仍能兼容绝大多数常见的拓扑空间,如欧氏空间、度量空间等.
以下的讨论表明,将基或邻域基限定为可数集是恰当的.

\subsection{第二可数性公理}
\begin{definition}
%@see: 《点集拓扑讲义(第四版)》(熊金城) P148
设\(\B\)是拓扑空间\(X\)的一个基.
如果\(\B\)是可数的,
则称“\(\B\)是拓扑空间\(X\)的一个\DefineConcept{可数基}”.
\end{definition}

\begin{definition}
%@see: 《点集拓扑讲义(第四版)》(熊金城) P148 定义5.1.1
设\(X\)是一个拓扑空间.
如果\(X\)有一个可数基,
则称“\(X\)满足\DefineConcept{第二可数性公理}”
“\(X\)是一个满足第二可数性公理的空间”
或“\(X\)是一个 \DefineConcept{\(A_2\)空间}”.
\end{definition}

\begin{example}
%@see: 《点集拓扑讲义(第四版)》(熊金城) P148 定理5.1.1
证明:实数空间\(\mathbb{R}\)满足第二可数性公理.
%TODO proof
\end{example}

\begin{example}
%@see: 《点集拓扑讲义(第四版)》(熊金城) P149
证明:含有不可数个点的离散空间不满足第二可数性公理.
\begin{proof}
因为离散空间的每一个单点子集都是开集,
而一个单点集不能表为异于自身的非空集合的并,
因此离散空间的每一个基必定包含它的所有单点子集.
\end{proof}
\end{example}

\subsection{第一可数性公理}
\begin{definition}
%@see: 《点集拓扑讲义(第四版)》(熊金城) P148
\def\Vx{\mathscr{V}_x}
设\(\Vx\)是点\(x\)的一个邻域基.
如果\(\Vx\)是可数的,
则称“\(\Vx\)是点\(x\)的一个\DefineConcept{可数邻域基}”.
\end{definition}

\begin{definition}
%@see: 《点集拓扑讲义(第四版)》(熊金城) P149 定义5.1.2
设\(X\)是一个拓扑空间.
如果\(X\)的每一个点都有一个可数邻域基,
则称“\(X\)满足\DefineConcept{第一可数性公理}”
“\(X\)是一个满足第一可数性公理的空间”
或“\(X\)是一个 \DefineConcept{\(A_1\)空间}”.
\end{definition}

\begin{theorem}
%@see: 《点集拓扑讲义(第四版)》(熊金城) P149 定理5.1.2
每一个度量空间都满足第一可数性公理.
%TODO proof
\end{theorem}

\begin{example}
%@see: 《点集拓扑讲义(第四版)》(熊金城) P149 例5.1.1
证明:含有不可数个点的可数补空间不满足第一可数性公理.
%TODO proof
\end{example}

\subsection{两个可数性公理的关系}
\begin{theorem}
%@see: 《点集拓扑讲义(第四版)》(熊金城) P149 定理5.1.3
每一个满足第二可数性公理的拓扑空间都满足第一可数性公理.
%TODO proof
\end{theorem}

\begin{example}
%@see: 《点集拓扑讲义(第四版)》(熊金城) P150
举例说明:满足第一可数性公理的拓扑空间不一定满足第二可数性公理.
\begin{solution}
显然任意一个离散空间均满足第一可数性公理,
但是含有不可数个点的离散空间不满足第二可数性公理.
\end{solution}
\end{example}

\subsection{可遗传性质}
\begin{theorem}
%@see: 《点集拓扑讲义(第四版)》(熊金城) P150 定理5.1.4
设\(X,Y\)都是拓扑空间,
\(f\colon X \to Y\)是一个满的连续开映射.
\begin{itemize}
	\item 如果\(X\)满足第二可数性公理,则\(Y\)也满足第二可数性公理.
	\item 如果\(X\)满足第一可数性公理,则\(Y\)也满足第一可数性公理.
\end{itemize}
%TODO proof
\end{theorem}
\begin{remark}
%@see: 《点集拓扑讲义(第四版)》(熊金城) P150
由上述定理可知,“拓扑空间是否满足第一可数性公理、第二可数性公理”是一个拓扑不变性质.
\end{remark}

\begin{definition}
%@see: 《点集拓扑讲义(第四版)》(熊金城) P150
设\(P(x)\)是一个谓词公式,
\(X\)是一个拓扑空间.
如果
	只要\(X\)满足\(P(X)\),
	就有\(X\)的任意一个子空间\(Y\)也满足\(P(Y)\),
则称“\(P\)是一个\DefineConcept{可遗传性质}”.
\end{definition}

\begin{proposition}
%@see: 《点集拓扑讲义(第四版)》(熊金城) P150
拓扑空间的离散性和平庸性都是可遗传性质.
\end{proposition}

\begin{proposition}
%@see: 《点集拓扑讲义(第四版)》(熊金城) P150
拓扑空间的连通性不是可遗传性质.
\end{proposition}

\begin{definition}
%@see: 《点集拓扑讲义(第四版)》(熊金城) P150
设\(P(x)\)是一个谓词公式,
\(X\)是一个拓扑空间.
如果
	只要\(X\)满足\(P(X)\),
	就有\(X\)的任意一个开子空间\(Y\)也满足\(P(Y)\),
则称“\(P\)是一个\DefineConcept{对于开子空间可遗传性质}”.
\end{definition}

\begin{definition}
%@see: 《点集拓扑讲义(第四版)》(熊金城) P150
设\(P(x)\)是一个谓词公式,
\(X\)是一个拓扑空间.
如果
	只要\(X\)满足\(P(X)\),
	就有\(X\)的任意一个闭子空间\(Y\)也满足\(P(Y)\),
则称“\(P\)是一个\DefineConcept{对于闭子空间可遗传性质}”.
\end{definition}

\begin{proposition}
%@see: 《点集拓扑讲义(第四版)》(熊金城) P151
拓扑空间的局部连通性是对于开子空间可遗传性质.
\end{proposition}

\begin{theorem}
%@see: 《点集拓扑讲义(第四版)》(熊金城) P151 定理5.1.5
设\(X\)是一个拓扑空间.
\begin{itemize}
	\item 如果\(X\)满足第二可数性公理,则\(X\)的任意一个子空间也满足第二可数性公理.
	\item 如果\(X\)满足第一可数性公理,则\(X\)的任意一个子空间也满足第一可数性公理.
\end{itemize}
%TODO proof
\end{theorem}
\begin{remark}
%@see: 《点集拓扑讲义(第四版)》(熊金城) P150
由上述定理可知,“拓扑空间是否满足第一可数性公理、第二可数性公理”是一个可遗传性质.
\end{remark}

\begin{theorem}
%@see: 《点集拓扑讲义(第四版)》(熊金城) P151 定理5.1.6
设\(\AutoTuple{X}{n}\)是一族拓扑空间,
\(X\)是\(\AutoTuple{X}{n}\)的积空间.
\begin{itemize}
	\item 如果\(\AutoTuple{X}{n}\)均满足第二可数性公理,则\(X\)也满足第二可数性公理.
	\item 如果\(\AutoTuple{X}{n}\)均满足第一可数性公理,则\(X\)也满足第一可数性公理.
\end{itemize}
%TODO proof
\end{theorem}
\begin{remark}
%@see: 《点集拓扑讲义(第四版)》(熊金城) P150
由上述定理可知,“拓扑空间是否满足第一可数性公理、第二可数性公理”是一个有限可积性质.
\end{remark}

\begin{example}
%@see: 《点集拓扑讲义(第四版)》(熊金城) P151 推论5.1.7
证明:\(n\)维欧氏空间\(\mathbb{R}^n\)的每一个子空间都满足第二可数性公理.
%TODO proof
\end{example}

\begin{theorem}
%@see: 《点集拓扑讲义(第四版)》(熊金城) P152 定理5.1.8
设\(X\)是一族非空拓扑空间\(\{X_\gamma\}_{\gamma \in \Gamma}\)的积空间,
则“\(X\)满足第二可数性公理”的充分必要条件是:
指标集\(\Gamma\)中有一个可数子集\(\Gamma_1\),
使得当\(\alpha \in \Gamma_1\)时\(X_\alpha\)满足第二可数性公理,
并且当\(\alpha \in \Gamma - \Gamma_1\)时\(X_\alpha\)是平庸空间.
%TODO proof
\end{theorem}

\begin{theorem}
%@see: 《点集拓扑讲义(第四版)》(熊金城) P155 习题 7.
设\(X\)是一族非空拓扑空间\(\{X_\gamma\}_{\gamma \in \Gamma}\)的积空间,
则“\(X\)满足第一可数性公理”的充分必要条件是:
指标集\(\Gamma\)中有一个可数子集\(\Gamma_1\),
使得当\(\alpha \in \Gamma_1\)时\(X_\alpha\)满足第一可数性公理,
并且当\(\alpha \in \Gamma - \Gamma_1\)时\(X_\alpha\)是平庸空间.
%TODO proof
\end{theorem}

\subsection{满足第一可数性公理的空间中序列的性质}
\begin{theorem}
%@see: 《点集拓扑讲义(第四版)》(熊金城) P153 定理5.1.9
设\(X\)是一个拓扑空间.
如果点\(x \in X\)有一个可数邻域基,
则点\(x \in X\)有一个可数邻域基\(\{U_n\}_{n\in\mathbb{Z}^+}\)
满足\(U_i \supseteq U_{i+1}\ (i=1,2,\dotsc)\).
%TODO proof
\end{theorem}

\begin{theorem}
%@see: 《点集拓扑讲义(第四版)》(熊金城) P153 定理5.1.10
设\(X\)是一个满足第一可数性公理的空间,\(A\)是\(X\)的一个子集,点\(x \in X\),
则\(x\)是\(A\)的一个聚点,
当且仅当在集合\(A-\{x\}\)中有一个序列收敛于\(x\).
%TODO proof
\end{theorem}

\begin{theorem}
%@see: 《点集拓扑讲义(第四版)》(熊金城) P153 定理5.1.11
设\(X,Y\)都是拓扑空间,
\(X\)满足第一可数性公理,
点\(x \in X\),
映射\(f\colon X \to Y\),
则“\(f\)在点\(x\)连续”的充分必要条件是:
如果\(X\)中的序列\(\{x_n\}\)收敛于\(x\),
则\(Y\)中的序列\(\{f(x_n)\}\)收敛于\(f(x)\).
%TODO proof
\end{theorem}

\begin{theorem}
%@see: 《点集拓扑讲义(第四版)》(熊金城) P153 定理5.1.12
设\(X,Y\)都是拓扑空间,
\(X\)满足第一可数性公理,
映射\(f\colon X \to Y\),
则“\(f\)是一个连续映射”的充分必要条件是:
对于任意一点\(x \in X\),
只要\(X\)中的序列\(\{x_n\}\)收敛于\(x\),
就有\(Y\)中的序列\(\{f(x_n)\}\)收敛于\(f(x)\).
%TODO proof
\end{theorem}

\section{可分空间}
\begin{definition}
%@see: 《Real Analysis Modern Techniques and Their Applications Second Edition》(Gerald B. Folland) P13
%@see: 《点集拓扑讲义(第四版)》(熊金城) P156 定义5.2.1
设\((X,\T)\)是拓扑空间,\(A \subseteq X\).
若\(\TopoClosureL{A}=X\),
则称“\(A\)是\(X\)的\DefineConcept{稠密子集}(dense subset)”
或“\(A\)在\(X\)中是\DefineConcept{稠密的}(\(A\) is \emph{dense} in \(X\))”.
\end{definition}

以下定理从一个侧面说明了讨论拓扑空间中稠密子集的意义.
\begin{theorem}
%@see: 《点集拓扑讲义(第四版)》(熊金城) P156 定理5.2.1
设\(X\)是一个拓扑空间,\(D\)是\(X\)中一个稠密子集,
\(f,g\)都是从\(X\)到\(\mathbb{R}\)的连续映射.
如果\(f \SetRestrict D = g \SetRestrict D\),
则\(f = g\).
%TODO proof
\end{theorem}

我们也希望讨论“有着较少点数的”稠密子集的拓扑空间,
例如具有有限稠密子集的拓扑空间.但这类拓扑空间过于简单,
大部分我们感兴趣的拓扑空间都不是这种情形,讨论起来意思不大.
例如一个度量空间如果有一个有限的稠密子集的话,那么这个空间一定就是一个离散空间.
相反,后继的讨论表明,许多重要的拓扑空间都有可数稠密子集.

\begin{definition}
%@see: 《Real Analysis Modern Techniques and Their Applications Second Edition》(Gerald B. Folland) P14
%@see: 《基础拓扑学讲义》(尤承业) P17
%@see: 《点集拓扑讲义(第四版)》(熊金城) P156 定义5.2.2
设\((X,\T)\)是拓扑空间.
若\(X\)存在一个可数稠密子集,
则称“拓扑空间\(X\)是\DefineConcept{可分的}(separable)”
或“\(X\)是\DefineConcept{可分空间}(separable space)”.
%@see: https://mathworld.wolfram.com/SeparableSpace.html
\end{definition}

\begin{theorem}\label{theorem:可分空间.满足第二可数性公理的空间都是可分空间}
%@see: 《点集拓扑讲义(第四版)》(熊金城) P157 定理5.2.2
每一个满足第二可数性公理的空间都是可分空间.
%TODO proof
\end{theorem}

\begin{example}
%@see: 《点集拓扑讲义(第四版)》(熊金城) P157
证明:含有不可数个点的离散空间一定不是可分的.
\begin{proof}
在含有不可数个点的离散空间中,任意一个可数子集的闭包都等于它自身,而不可能等于整个空间.
\end{proof}
\end{example}

\begin{corollary}
%@see: 《点集拓扑讲义(第四版)》(熊金城) P157 推论5.2.3
满足第二可数性公理的拓扑空间的每一个子空间都是可分空间.
\begin{proof}
因为“拓扑空间是否满足第二可数性公理”是一个可遗传性质,
所以由\cref{theorem:可分空间.满足第二可数性公理的空间都是可分空间} 可知,
满足第二可数性公理的拓扑空间的每一个子空间都是可分空间.
\end{proof}
\end{corollary}

\begin{example}
%@see: 《点集拓扑讲义(第四版)》(熊金城) P157
证明:\(n\)维欧氏空间\(\mathbb{R}^n\)中的每一个子空间(包括它自己)都是可分空间.
%TODO proof
\end{example}

\begin{example}
%@see: 《点集拓扑讲义(第四版)》(熊金城) P157 例5.2.1
设\((X,\T)\)是一个拓扑空间,
任取一个不属于\(X\)的元素\(\infty\)
(例如我们可以取\(\infty \defeq X\)),
令\begin{align*}
	X^* &\defeq X \cup \{\infty\}, \\
	\T^* &\defeq \Set{
		A \cup \{\infty\}
		\given
		A \in \T
	} \cup \{\emptyset\}.
\end{align*}
证明:\begin{itemize}
	\item \((X^*,\T^*)\)是一个拓扑空间;
	\item \((X^*,\T^*)\)是一个可分空间;
	\item \((X^*,\T^*)\)满足第二可数性公理,当且仅当\((X,\T)\)满足第二可数性公理;
	\item \(\T = \T^* \TopoRestrict X\);
	\item \((X,\T)\)是\((X^*,\T^*)\)的一个子空间.
\end{itemize}
%TODO proof
\end{example}
\begin{remark}
%@see: 《点集拓扑讲义(第四版)》(熊金城) P157
%@see: 《点集拓扑讲义(第四版)》(熊金城) P158
由上例可知:\begin{itemize}
	\item 可分空间可以不满足第二可数性公理;
	\item 可分空间的子空间可以不是可分空间,换言之,拓扑空间的可分性不是可遗传性质.
\end{itemize}
\end{remark}

\begin{theorem}
%@see: 《点集拓扑讲义(第四版)》(熊金城) P158 定理5.2.4
每一个可分度量空间都满足第二可数性公理.
%TODO proof
\end{theorem}

\begin{corollary}
%@see: 《点集拓扑讲义(第四版)》(熊金城) P159 推论5.2.5
可分度量空间的每一个子空间都是可分空间.
%TODO proof
\end{corollary}

\begin{example}
%@see: 《基础拓扑学讲义》(尤承业) P18
实数余有限拓扑空间\((\mathbb{R},\T_f)\)是可分的,
事实上它的任一无穷子集都是稠密的:
\(\mathbb{Q}\)就是它的一个可数稠密子集.
但是实数余可数拓扑空间\((\mathbb{R},\T_c)\)是不可分的,
因为它的任一可数集都是闭集,不可能稠密.
\end{example}

\begin{remark}
应当注意,当我们把一个度量空间看作拓扑空间时,
空间的拓扑是由度量诱导出来的拓扑,
而一个集合是不是一个某一个点的邻域,
无论是按\cref{definition:度量空间.邻域的概念},
还是按\cref{definition:拓扑学.点的分类},
都是一回事.
\end{remark}

\section{林德洛夫空间}
\begin{definition}
%@see: 《点集拓扑讲义(第四版)》(熊金城) P160 定义5.3.1
设\(\sfA\)是一个集族,\(B\)是一个集合.
如果\(\bigcup \sfA \supseteq B\),
则称“集族\(\sfA\)是集合\(B\)的一个\DefineConcept{覆盖}”.
\end{definition}
\begin{definition}
%@see: 《点集拓扑讲义(第四版)》(熊金城) P160 定义5.3.1
设集族\(\sfA\)是集合\(B\)的一个覆盖.
如果\(\sfA\)是可数集,
则称“集族\(\sfA\)是集合\(B\)的一个\DefineConcept{可数覆盖}”.
\end{definition}
\begin{definition}
%@see: 《点集拓扑讲义(第四版)》(熊金城) P160 定义5.3.1
设集族\(\sfA\)是集合\(B\)的一个覆盖.
如果\(\sfA\)是有限集,
则称“集族\(\sfA\)是集合\(B\)的一个\DefineConcept{有限覆盖}”.
\end{definition}
\begin{definition}
%@see: 《点集拓扑讲义(第四版)》(熊金城) P160 定义5.3.1
设集族\(\sfA\)是集合\(B\)的一个覆盖.
如果\(\sfA\)的一个子族\(\sfA_1\)也是集合\(B\)的覆盖,
则称“子集族\(\sfA_1\)是覆盖\(\sfA\)(关于集合\(B\))的一个\DefineConcept{子覆盖}”.
\end{definition}
\begin{definition}
%@see: 《点集拓扑讲义(第四版)》(熊金城) P160 定义5.3.1
设\(X\)是一个拓扑空间,
\(B\)是\(X\)的一个子集,
\(\sfA\)是\(X\)中的一个开集族.
如果\(\sfA\)是\(B\)的一个覆盖,
则称“\(\sfA\)是\(B\)的一个\DefineConcept{开覆盖}”.
\end{definition}
\begin{definition}
%@see: 《点集拓扑讲义(第四版)》(熊金城) P160 定义5.3.1
设\(X\)是一个拓扑空间,
\(B\)是\(X\)的一个子集,
\(\sfA\)是\(X\)中的一个闭集族.
如果\(\sfA\)是\(B\)的一个覆盖,
则称“\(\sfA\)是\(B\)的一个\DefineConcept{闭覆盖}”.
\end{definition}

\begin{definition}
%@see: 《点集拓扑讲义(第四版)》(熊金城) P161 定义5.3.2
设\(X\)是一个拓扑空间.
如果\(X\)的每一个开覆盖都有一个可数子覆盖,
则称“拓扑空间\(X\) \DefineConcept{具有林德洛夫性质}”
或“\(X\)是一个\DefineConcept{林德洛夫空间}”.
\end{definition}

\begin{example}
%@see: 《点集拓扑讲义(第四版)》(熊金城) P161
证明:含有不可数个点的离散空间一定不是林德洛夫空间.
\begin{proof}
设\(X\)是一个含有不可数个点的离散空间.
因为\(X\)中所有单点子集\begin{equation*}
	\sfA \defeq \Set{
		A
		\given
		A = \{a\},
		a \in X
	}
\end{equation*}
构成\(X\)的一个开覆盖,
而\(\sfA\)没有任何可数子覆盖.
\end{proof}
\end{example}

\begin{theorem}[林德洛夫定理]\label{theorem:林德洛夫空间.林德洛夫定理}
%@see: 《点集拓扑讲义(第四版)》(熊金城) P161 定理5.3.1(Lindelof定理)
任何一个满足第二可数性公理的拓扑空间都是林德洛夫空间.
%TODO proof
\end{theorem}

\begin{corollary}
%@see: 《点集拓扑讲义(第四版)》(熊金城) P162 推论5.3.2
满足第二可数性公理的拓扑空间的每一个子空间都是林德洛夫空间.
%TODO proof
\end{corollary}

\begin{example}
%@see: 《点集拓扑讲义(第四版)》(熊金城) P162
证明:\(n\)维欧氏空间\(\mathbb{R}^n\)的每一个子空间都是林德洛夫空间.
%TODO proof
\end{example}

\begin{example}
%@see: 《点集拓扑讲义(第四版)》(熊金城) P162
举例说明:林德洛夫空间可以不满足第二可数性公理.
%TODO proof
\end{example}

\begin{example}
%@see: 《点集拓扑讲义(第四版)》(熊金城) P162 例5.3.1
举例说明:林德洛夫空间可以不满足第二可数性公理.
%TODO proof
\end{example}

\begin{example}
%@see: 《点集拓扑讲义(第四版)》(熊金城) P162 例5.3.1
举例说明:即便拓扑空间\(X\)的每一个子空间都是林德洛夫空间,但是\(X\)不满足第二可数性公理.
%TODO proof
\end{example}

\begin{theorem}
%@see: 《点集拓扑讲义(第四版)》(熊金城) P162 定理5.3.3
设\(X\)是一个度量空间.
如果\(X\)是林德洛夫空间,
则\(X\)满足第二可数性公理.
%TODO proof
\end{theorem}

\begin{example}
%@see: 《点集拓扑讲义(第四版)》(熊金城) P163 例5.3.2
举例说明:林德洛夫空间的子空间可以不是林德洛夫空间.
%TODO
\end{example}
\begin{remark}
上例说明:拓扑空间的林德洛夫性质不是可遗传性质.
\end{remark}

\begin{example}
%@see: 《点集拓扑讲义(第四版)》(熊金城) P163
举例说明:两个林德洛夫空间的积空间可以不是林德洛夫空间.
%TODO
\end{example}

\begin{theorem}
%@see: 《点集拓扑讲义(第四版)》(熊金城) P163 定理5.3.4
林德洛夫空间的每一个闭子空间都是林德洛夫空间.
%TODO proof
\end{theorem}

\begin{theorem}
%@see: 《点集拓扑讲义(第四版)》(熊金城) P163 定理5.3.5
设拓扑空间\(X\)的任意一个子空间都是林德洛夫空间.
如果\(A\)是\(X\)的一个不可数子集,
则\(A\)中必定含有\(A\)的某个聚点.
%TODO proof
\end{theorem}

\begin{proposition}
%@see: 《点集拓扑讲义(第四版)》(熊金城) P164
设\(X\)是一个满足第二可数性公理的拓扑空间,
则\(X\)的每一个不可数子集\(A\)都含有\(A\)的某个聚点.
%TODO proof
\end{proposition}


\chapter{分离性公理}
\section{\texorpdfstring{\(T_0\)}{T0}空间,\texorpdfstring{\(T_1\)}{T1}空间,豪斯多夫空间}

\section{正则空间,正规空间,\texorpdfstring{\(T_3\)}{T3}空间,\texorpdfstring{\(T_4\)}{T4}空间}
\begin{definition}
%@see: 《点集拓扑讲义(第四版)》(熊金城) P171 定义6.2.1
设\(X\)是一个拓扑空间,
且\(A\)是\(X\)的子集.
如果集合\(Y\)包含于\(A\)的内部,
则称“\(A\)是集合\(Y\)的一个\DefineConcept{邻域}(neighborhood)”.
\end{definition}
\begin{definition}
%@see: 《点集拓扑讲义(第四版)》(熊金城) P171 定义6.2.1
设\(X\)是一个拓扑空间,
\(A\)是\(X\)的子集,
\(A\)是集合\(Y\)的一个邻域.
\begin{itemize}
	\item 如果\(A\)是一个开集,
	则称“\(A\)是集合\(Y\)的一个\DefineConcept{开邻域}(open neighborhood)”.
	\item 如果\(A\)是一个闭集,
	则称“\(A\)是集合\(Y\)的一个\DefineConcept{闭邻域}(open neighborhood)”.
\end{itemize}
\end{definition}

\begin{definition}
%@see: 《点集拓扑讲义(第四版)》(熊金城) P171 定义6.2.2
设\(X\)是一个拓扑空间.
如果\(X\)中任意一个点\(x\)和任意一个不含点\(x\)的闭集都各有一个开邻域,且两者互斥,
则称“拓扑空间\(X\)是\DefineConcept{正则的}”
“拓扑空间\(X\)具有\DefineConcept{正则性}”
或“\(X\)是一个\DefineConcept{正则空间}”.
\end{definition}

\begin{theorem}
%@see: 《点集拓扑讲义(第四版)》(熊金城) P171 定理6.2.1
设\(X\)是一个拓扑空间,
则\(X\)是一个正则空间,
当且仅当对于任意一点\(x \in X\)和\(x\)的任意一个开邻域\(U\),
存在\(x\)的一个开邻域\(V\),
使得\(V\)的闭包是\(U\)的一个子集.
%TODO proof
\end{theorem}

\begin{definition}
%@see: 《点集拓扑讲义(第四版)》(熊金城) P172 定义6.2.3
设\(X\)是一个拓扑空间.
如果\(X\)中任意两个互斥的闭集各有一个开邻域,并且这两个邻域也互斥,
则称“拓扑空间\(X\)是\DefineConcept{正规的}”
“拓扑空间\(X\)具有\DefineConcept{正规性}”
或“\(X\)是一个\DefineConcept{正规空间}”.
\end{definition}

\begin{theorem}
%@see: 《点集拓扑讲义(第四版)》(熊金城) P172 定理6.2.2
设\(X\)是一个拓扑空间,
则\(X\)是一个正规空间,
当且仅当对于任意一个闭集\(A \subseteq X\)和\(A\)的任意一个开邻域\(U\),
存在\(A\)的一个开邻域\(V\),
使得\(V\)的闭包是\(U\)的一个子集.
%TODO proof
\end{theorem}

%@see: 《点集拓扑讲义(第四版)》(熊金城) P172
“拓扑空间是否正则空间或正规空间”
与“拓扑空间是否\(T_0\)空间、\(T_1\)空间或\(T_2\)空间”
之间并无必然的蕴含关系.

\begin{example}
%@see: 《点集拓扑讲义(第四版)》(熊金城) P172 例6.2.1
举例说明:尽管拓扑空间\(X\)既是正则空间也是正规空间,但\(X\)不是\(T_0\)空间.
%TODO
\end{example}

\begin{example}
%@see: 《点集拓扑讲义(第四版)》(熊金城) P172 例6.2.2
举例说明:尽管拓扑空间\(X\)是豪斯多夫空间,但\(X\)既非正则空间亦非正规空间.
%TODO
\end{example}

%@see: 《点集拓扑讲义(第四版)》(熊金城) P174
“拓扑空间是否正则空间”
与“拓扑空间是否正规空间”
之间也没有必然的蕴含关系.

\begin{example}
%@see: 《点集拓扑讲义(第四版)》(熊金城) P174 例6.2.3
举例说明:尽管拓扑空间\(X\)是正规空间,但\(X\)不是正则空间.
%TODO
\end{example}

\begin{definition}
%@see: 《点集拓扑讲义(第四版)》(熊金城) P174 定义6.2.4
正则的\(T_1\)空间称为 \DefineConcept{\(T_3\)空间}.
\end{definition}

\begin{definition}
%@see: 《点集拓扑讲义(第四版)》(熊金城) P174 定义6.2.4
正规的\(T_1\)空间称为 \DefineConcept{\(T_4\)空间}.
\end{definition}

由于\(T_1\)空间中每一个单点集都是闭集,
因此\(T_4\)空间一定是\(T_3\)空间,
\(T_3\)空间一定是豪斯多夫空间.

\begin{example}
%@see: 《点集拓扑讲义(第四版)》(熊金城) P174
举例说明:尽管拓扑空间\(X\)是\(T_3\)空间,但\(X\)不是\(T_4\)空间.
%TODO
\end{example}

\begin{theorem}
%@see: 《点集拓扑讲义(第四版)》(熊金城) P174 定理6.2.3
每一个度量空间都是\(T_4\)空间.
%TODO proof
\end{theorem}

\section{乌瑞松引理,提彻扩张定理}
\begin{lemma}\label{theorem:扩张定理.乌瑞松引理}
%@see: 《点集拓扑讲义(第四版)》(熊金城) P176 定理6.3.1(Urysohn引理)
设\(X\)是一个拓扑空间,
则\(X\)是一个正规空间,
当且仅当对于\(X\)中任意两个互斥的\(A,B\),
存在一个连续映射\(f\colon X \to [a,b]\),
使得\begin{equation*}
	f(x) = \left\{ \begin{array}{cl}
		a, & x \in A, \\
		b, & x \in B.
	\end{array} \right.
\end{equation*}
%TODO proof
\end{lemma}

\begin{theorem}
%@see: 《点集拓扑讲义(第四版)》(熊金城) P179 定理6.3.2
\(T_4\)空间中任意一个连通子集,
只要含有多于一个点,
就一定是一个不可数集.
%TODO proof
\end{theorem}

\begin{lemma}
%@see: 《点集拓扑讲义(第四版)》(熊金城) P179 引理6.3.3
设\(X\)是一个正规空间,
\(A\)是\(X\)的一个闭子集,
\(\lambda\)是一个正实数,
\(k=1/3\),
则对于任意一个连续映射\(g\colon A \to [-\lambda,\lambda]\),
存在一个连续映射\(g^*\colon X \to [-k\lambda,k\lambda]\),
使得对于任意\(a \in A\)有\begin{equation*}
	\abs{g(a) - g^*(a)} \leq 2k\lambda.
\end{equation*}
%TODO proof
\end{lemma}

\begin{theorem}\label{theorem:扩张定理.提彻扩张定理}
%@see: 《点集拓扑讲义(第四版)》(熊金城) P180 定理6.3.4(Tietze扩张定理)
设\(X\)是一个拓扑空间,
则\(X\)是一个正规空间,
当且仅当对于\(X\)中任意一个闭集\(A\)和任意一个连续映射\(f\colon A \to [a,b]\),
有一个连续映射\(g\colon X \to [a,b]\)是\(f\)的扩张.
%TODO proof
\end{theorem}

\section{完全正则空间,提赫诺夫空间}

\section{分离性公理,子空间、积空间和商空间}

\section{可度量化空间}


\chapter{紧致性}
\section{紧致空间}
\begin{definition}
%@see: 《点集拓扑讲义(第四版)》(熊金城) P198 定义7.1.1
设\(X\)是一个拓扑空间.
如果\(X\)的每一个开覆盖都有一个有限子覆盖,
则称“拓扑空间\(X\)是\DefineConcept{紧致的}”
“拓扑空间\(X\)具有\DefineConcept{紧致性}”
或“\(X\)是一个\DefineConcept{紧致空间}”.
\end{definition}

\begin{proposition}
%@see: 《点集拓扑讲义(第四版)》(熊金城) P198
每一个紧致空间都是一个林德洛夫空间,反之不然.
%TODO proof
\end{proposition}

\begin{example}
%@see: 《点集拓扑讲义(第四版)》(熊金城) P198 例7.1.1
证明:实数空间\(\mathbb{R}\)不是一个紧致空间.
%TODO proof
\end{example}

\begin{definition}
%@see: 《点集拓扑讲义(第四版)》(熊金城) P198 定义7.1.2
设\(X\)是一个拓扑空间,
\(Y\)是\(X\)的一个子集.
如果\(Y\)作为\(X\)的子空间是一个紧致空间,
则称“\(Y\)是拓扑空间\(X\)的一个\DefineConcept{紧致子集}”.
\end{definition}

根据定义,拓扑空间\(X\)的一个子集\(Y\)是\(X\)的紧致子集,
意味着
\(Y\)的由子空间\(Y\)中的开集构成的每一个开覆盖都有有限子覆盖.
这并不明显地意味着
\(Y\)的由\(X\)中的开集构成的构成的每一个覆盖都有有限子覆盖.
\begin{theorem}
%@see: 《点集拓扑讲义(第四版)》(熊金城) P199 定理7.1.1
设\(X\)是一个拓扑空间,
\(Y\)是\(X\)的一个子集,
则\(Y\)是\(X\)的一个紧致子集,
当且仅当\(Y\)的由\(X\)中的开集构成的每一个覆盖都有有限子覆盖.
%TODO proof
\end{theorem}

下面介绍关于紧致性的一个等价说法.
\begin{definition}
%@see: 《点集拓扑讲义(第四版)》(熊金城) P199 定义7.1.3
设\(\sfA\)是一个集族.
如果\(\sfA\)的每一个有限子族都有非空的交,
即\begin{equation*}
	\text{$\sfA_1$是$\sfA$的有限子族}
	\implies
	\bigcap \sfA \neq \emptyset,
\end{equation*}
则称“\(\sfA\)是一个\DefineConcept{具有有限交性质的集族}”.
\end{definition}

\begin{theorem}
%@see: 《点集拓扑讲义(第四版)》(熊金城) P199 定理7.1.2
设\(X\)是一个拓扑空间,
则\(X\)是一个紧致空间,
当且仅当\(X\)中每一个具有有限交性质的闭集族都有非空的交.
%TODO proof
\end{theorem}

如果\(\B\)是紧致空间\(X\)的一个基,
那么由\(\B\)中的元素构成的\(X\)的一个覆盖当然是一个开覆盖,
因此\(\B\)有有限子覆盖.
下述定理指出,为了验证拓扑空间的紧致性,
只要验证由它的某一个基中的元素组成的覆盖有有限子覆盖.
\begin{theorem}
%@see: 《点集拓扑讲义(第四版)》(熊金城) P201 定理7.1.3
设\(\B\)是拓扑空间\(X\)的一个基,
并且\(X\)的由\(\B\)中的元素构成的每一个覆盖有一个有限子覆盖,
则\(X\)是一个紧致空间.
%TODO proof
\end{theorem}

\begin{theorem}
%@see: 《点集拓扑讲义(第四版)》(熊金城) P201 定理7.1.4
设\(X,Y\)都是拓扑空间,
\(f\colon X \to Y\)是一个连续映射.
如果\(A\)是\(X\)的一个紧致子集,
则\(f(A)\)是\(Y\)的一个紧致子集.
%TODO proof
\end{theorem}
%@see: 《点集拓扑讲义(第四版)》(熊金城) P202
上述定理说明,拓扑空间的紧致性是在连续映射下保持不变的性质,
因此它是拓扑不变性质,也是可商性质.

\begin{example}
%@see: 《点集拓扑讲义(第四版)》(熊金城) P202
证明:实数空间\(\mathbb{R}\)中的每一个开区间都不是紧致空间.
\begin{proof}
由于实数空间\(\mathbb{R}\)不是一个紧致空间,
而每一个开区间都是与之同配的,
所以每一个开区间作为子空间都不是紧致空间.
\end{proof}
\end{example}

\begin{theorem}
%@see: 《点集拓扑讲义(第四版)》(熊金城) P202 定理7.1.5
紧致空间中的每一个闭子集都是紧致子集.
%TODO proof
\end{theorem}

\begin{theorem}\label{theorem:紧致空间.拓扑空间的一点紧化}
%@see: 《点集拓扑讲义(第四版)》(熊金城) P202 定理7.1.6
任意一个拓扑空间必定是某个紧致空间的开子空间.
\begin{proof}
设\((X,\T)\)是一个拓扑空间,
任取一个不属于\(X\)的元素\(\infty\),
令\begin{align*}
	X^* &\defeq X \cup \{\infty\}, \\
	\T^* &\defeq \T \cup \T_1 \cup \{X^*\}, \\
	\T_1 &\defeq \Set{
		E \subseteq X^*
		\given
		\text{$X^* - E$是$X$中的一个紧致闭集}
	}.
\end{align*}

首先验证\(\T^*\)是\(X^*\)的一个拓扑.
\begin{itemize}
	\item 根据定义,立即有\(X^* \in \T^*\)和\(\emptyset \in \T \subseteq \T^*\).

	\item 任取\(A^*,B^* \in \T^*\).
	容易看出,要么\(A^*,B^*\)中至少有一个等于\(X^*\),
	要么\(A^*,B^*\)都不等于\(X^*\).
	假设\(A^*,B^*\)中至少有一个等于\(X^*\),
	则\(A^* \cap B^*\)等于\(A^* \cap X^* = B^*\)或\(X^* \cap B^* = A^*\),
	因而\(A^* \cap B^* \in \T^*\).
	假设\(A^*\)和\(B^*\)均不等于\(X^*\),则\(A^*,B^* \in \T \cup \T_1\).
	接下来进一步分为三种情况讨论.
	\begin{itemize}
		\item 如果\(A^*,B^* \in \T\),
		则有\begin{equation*}
			A^* \cap B^* \in \T \subseteq \T^*.
		\end{equation*}

		\item 如果\(A^*,B^* \in \T_1\),
		则\begin{equation*}
			X^* - (A^* \cap B^*)
			= (X^* - A^*) \cup (X^* - B^*)
		\end{equation*}
		作为\(X\)中的两个紧致闭集的并,也是一个紧致闭集,
		所以\(A^* \cap B^* \in \T_1 \subseteq \T^*\).

		\item 如果\(A^*,B^*\)既不同时属于\(\T\)也不同时属于\(\T_1\),
		不妨设\(A^* \in \T\)而\(B^* \in \T_1\),
		则\(X\)中的一个开集\(B\),
		使得\(B^* = B \cup \{\infty\}\),
		从而有\(A^* \cap B^*
		= A^* \cap B
		\in \T\).
	\end{itemize}
	总之,只要\(A^*,B^* \in \T^*\),便有\(A^* \cap B^* \in \T^*\).

	\item 取\(\T^*\)的一个子族\(\sfA\)使得\(\bigcup \sfA \notin \{\emptyset,X^*\}\).
	这时必有\(\sfA \neq \emptyset\)和\(X^* \notin \sfA\).
	\begin{itemize}
		\item 当\(\sfA \subseteq \T\)时,显然\(\bigcup \sfA \in \T \subseteq \T^*\).

		\item 当\(\sfA \subseteq \T_1\)时,则\begin{equation*}
			X^* - \bigcup \sfA = \bigcap_{A \in \sfA} (X^* - A)
		\end{equation*}
		是\(X\)中的一个闭集,
		并且对于任意\(A_0 \in \sfA\)有\begin{equation*}
			X^* - \bigcup \sfA \subseteq X^* - A_0.
		\end{equation*}
		因此\(x^* - \bigcup \sfA\)
		作为紧致空间\(X^* - A_0\)中的一个闭集也是紧致的,
		所以\begin{equation*}
			\bigcup \sfA \in \T_1 \subseteq \T^*.
		\end{equation*}

		\item 当\(\sfA\)既非\(\T\)的子集亦非\(\T_1\)的子集,则有\begin{equation*}
			\sfA_1 \defeq \sfA \cap \T \neq \emptyset,
			\qquad
			\sfA_2 \defeq \sfA \cap \T_1 \neq \emptyset.
		\end{equation*}
		令\begin{equation*}
			B_1 \defeq \bigcup \sfA_1,
			\qquad
			B_2 \defeq \bigcup \sfA_2,
		\end{equation*}
		则有\(\bigcup \sfA = B_1 \cup B_2\).
		此时必有\(B_1 \in \T\)和\(B_2 \in \T_1\),
		再由\(\infty \notin X^* - B_2\)
		有\begin{equation*}
			X^* - (B_1 \cup B_2)
			= (X^* - B_1) \cap (X^* - B_2)
			= (X - B_1) \cap (X^* - B_2),
		\end{equation*}
		它是紧致空间\(X^* - B_2\)的一个闭集,
		因此\begin{equation*}
			\bigcup \sfA
			= B_1 \cup B_2
			\in \T_1
			\subseteq \T^*.
		\end{equation*}
	\end{itemize}
	总之,只要\(\sfA \subseteq \T^*\),便有\(\bigcup \sfA  \in \T^*\).
\end{itemize}

接下来验证\((X^*,\T^*)\)是一个紧致空间.
设\(\sfC\)是\(X^*\)的一个开覆盖,
则存在\(C \in \sfC\)使得\(\infty \in C\).
不妨设\(C \neq X^*\),
于是\(C \in \T_1\),
因此\(X^* - C\)是紧致的,
并且\(\sfC-\{C\}\)是它的一个开覆盖.
于是\(\sfC-\{C\}\)有一个有限子族\(\wsfC\)覆盖\(X^* - C\).
易见\(\wsfC \cup \{C\}\)是\(\sfC\)的一个有限子族并且覆盖\(X^*\).

最后,显然有\(\T = \T^* \TopoRestrict X\),且\(X\)是\(X^*\)中的一个开集.
这就说明\((X,\T)\)是\((X^*,\T^*)\)的一个开子空间.
\end{proof}
\end{theorem}
\begin{definition}
%@see: 《点集拓扑讲义(第四版)》(熊金城) P202 定理7.1.6
在\cref{theorem:紧致空间.拓扑空间的一点紧化} 的证明过程中,
由拓扑空间\((X,\T)\)构造出来的紧致空间\((X^*,\T^*)\),
称为“拓扑空间\((X,\T)\)的\DefineConcept{一点紧化}”.
%@see: 《点集拓扑讲义(第四版)》(熊金城) P157 例5.2.1
\end{definition}

由于非紧致空间是它的一点紧化的一个子空间,
所以紧致性不是可遗传性质.

以下定理表明紧致性是可积性质.
\begin{theorem}
%@see: 《点集拓扑讲义(第四版)》(熊金城) P204 定理7.1.7
设\(\AutoTuple{X}{n}\)是\(n\ (n\geq1)\)个紧致空间,
则积空间\(\AutoTuple{X}{n}[\times]\)也是一个紧致空间.
\end{theorem}

\section{紧致性与分离性}
\begin{theorem}
%@see: 《点集拓扑讲义(第四版)》(熊金城) P206 定理7.2.1
设\(X\)是一个豪斯多夫空间.
如果\(A\)是\(X\)的一个不含点\(x \in X\)的紧致子集,
则\(x\)和\(A\)分别有开邻域\(U\)和\(V\)使得\(U \cap V = \emptyset\).
%TODO proof
\end{theorem}

\begin{corollary}
%@see: 《点集拓扑讲义(第四版)》(熊金城) P207 推论7.2.2
豪斯多夫空间中每一个紧致子集都是一个闭集.
%TODO proof
\end{corollary}

\begin{corollary}
%@see: 《点集拓扑讲义(第四版)》(熊金城) P207 推论7.2.3
在紧致的豪斯多夫空间中,
集合\(A\)是闭集,
当且仅当\(A\)是一个紧致子集.
%TODO proof
\end{corollary}

\begin{corollary}
%@see: 《点集拓扑讲义(第四版)》(熊金城) P207 推论7.2.4
任意一个紧致的豪斯多夫空间都是一个正则空间.
%TODO proof
\end{corollary}

\begin{theorem}
%@see: 《点集拓扑讲义(第四版)》(熊金城) P207 定理7.2.5
设\(X\)是一个豪斯多夫空间.
如果\(A\)和\(B\)是\(X\)的两个互斥的紧致子集,
则它们分别有开邻域\(U\)和\(V\)使得\(U \cap V = \emptyset\).
%TODO proof
\end{theorem}

\begin{corollary}
%@see: 《点集拓扑讲义(第四版)》(熊金城) P208 推论7.2.6
任意一个紧致的豪斯多夫空间都是一个\(T_4\)空间.
%TODO proof
\end{corollary}

\begin{theorem}
%@see: 《点集拓扑讲义(第四版)》(熊金城) P208 定理7.2.7
设\(X\)是一个正则空间.
如果\(A\)是\(X\)中的一个紧致子集,
\(U\)是\(A\)的一个开邻域,
则存在\(A\)的一个开邻域\(V\),
使得\(V\)的闭包是\(U\)的一个子集.
%TODO proof
\end{theorem}

\begin{proposition}
%@see: 《点集拓扑讲义(第四版)》(熊金城) P209
任意一个紧致的正则空间都是一个正规空间.
%TODO proof
\end{proposition}

\begin{example}
%@see: 《点集拓扑讲义(第四版)》(熊金城) P209
举例说明:尽管拓扑空间\(X\)是紧致的正规空间,但\(X\)不是正则空间.
%TODO
\end{example}

\begin{theorem}
%@see: 《点集拓扑讲义(第四版)》(熊金城) P208 定理7.2.8
从紧致空间到豪斯多夫空间的任意一个连续映射都是闭映射.
%TODO proof
\end{theorem}

\begin{corollary}
%@see: 《点集拓扑讲义(第四版)》(熊金城) P208 推论7.2.9
从紧致空间到豪斯多夫空间的任意一个连续双射都是同胚.
%TODO proof
\end{corollary}

\section{\texorpdfstring{$n$}{n}维欧氏空间中的紧致子集}

\section{几种紧致性以及其间的关系}

\section{度量空间中的紧致性}
\begin{definition}
%@see: 《点集拓扑讲义(第四版)》(熊金城) P219 定义7.5.2
设\((X,\rho)\)是一个度量空间,
\(\sfA\)是\(X\)的一个开覆盖,
实数\(\lambda>0\).
如果对于\(X\)中任意一个子集\(A\),
	只要\(\diam A < \lambda\)
	就有\(A\)包含于\(\sfA\)的某个元素之中,
则称“\(\lambda\)是\(\sfA\)的一个\DefineConcept{勒贝格数}”.
\end{definition}

勒贝格数不一定存在.
\begin{example}
%@see: 《点集拓扑讲义(第四版)》(熊金城) P219
证明:实数空间\(\mathbb{R}\)的开覆盖\begin{equation*}
	\Set{(-\infty,1)}
	\cup
	\Set{
		(n-1/n,n+1+1/n)
		\given
		n\in\mathbb{Z}^+
	}
\end{equation*}
没有勒贝格数.
%TODO proof
\end{example}

\begin{theorem}
%@see: 《点集拓扑讲义(第四版)》(熊金城) P219 定理7.5.1(Lebesgue数定理)
序列紧致的度量空间的每一个开覆盖都有一个勒贝格数.
%TODO proof
\end{theorem}

\begin{theorem}
%@see: 《点集拓扑讲义(第四版)》(熊金城) P220 定理7.5.2
任意一个序列紧致的度量空间都是一个紧致空间.
%TODO proof
\end{theorem}

\begin{theorem}
%@see: 《点集拓扑讲义(第四版)》(熊金城) P221 定理7.5.3
设\(X\)是一个度量空间,
则以下命题等价:\begin{itemize}
	\item \(X\)是一个紧致空间;
	\item \(X\)是一个列紧空间;
	\item \(X\)是一个序列紧致空间;
	\item \(X\)是一个可数紧致空间.
\end{itemize}
%TODO proof
\end{theorem}

\section{局部紧致空间,仿紧致空间}

\section{提赫诺夫乘积定理}

\section{拓扑空间在方体中的嵌入}
设\(X\)是一个拓扑空间,
\(\Gamma\)是一个集合.
根据定义,
从\(\Gamma\)到\(X\)的映射空间\(X^\Gamma\)
可以看作拓扑空间族\(\{X_\gamma\}_{\gamma \in \Gamma}\)的笛卡尔积
(其中每一个\(X_\gamma\)都等于\(X\)),
因此它必有一个积拓扑.

\begin{definition}
%@see: 《点集拓扑讲义(第四版)》(熊金城) P233
设\(X\)是一个拓扑空间,
\(\Gamma\)是一个集合.
把从\(\Gamma\)到\(X\)的映射空间\(X^\Gamma\)的积拓扑
称为“\(X^\Gamma\)的\DefineConcept{点式收敛拓扑}”.
\end{definition}

\begin{definition}
%@see: 《点集拓扑讲义(第四版)》(熊金城) P233 定义7.8.1
设\(\Gamma\)是一个集合,
从\(\Gamma\)到单位闭区间\([0,1]\)的映射空间\([0,1]^\Gamma\),连同它的点式收敛拓扑,
称为一个\DefineConcept{方体}.
\end{definition}

显然,\(n\)维欧氏空间\(\mathbb{R}^n\)中的单位方体是我们这里所说的方体的一个特例.

由于我们熟悉单位闭区间\([0,1]\)的拓扑特性,
所以方体\([0,1]^\Gamma\)的某些拓扑性质也容易得知.

\begin{proposition}
%@see: 《点集拓扑讲义(第四版)》(熊金城) P233
任意一个方体都是连通的紧致的提赫诺夫空间.
%TODO proof
\end{proposition}

\begin{proposition}
%@see: 《点集拓扑讲义(第四版)》(熊金城) P233
如果\(\Gamma\)是可数集,
那么方体\([0,1]^\Gamma\)是满足第二可数性公理的可度量化空间.
%TODO proof
\end{proposition}

然而,方体中究竟都包含了哪些类型的拓扑空间?
或者说,哪种拓扑空间可以嵌入到方体中去?
这是我们在本节要研究的问题.
可以注意到,
所有提赫诺夫空间,或者说所有紧致的豪斯多夫空间的每一个子空间,
都可以被“装”在某一个方体之中.

之前  % 《点集拓扑讲义(第四版)》(熊金城)第6章第6节
我们证明了
每一个满足第二可数性公理的\(T_3\)空间都可以嵌入希尔伯特空间.
那个定理的证明过程可以一般化,
这恰为我们解决现在的课题提供了一条有效的途径.

\begin{definition}
%@see: 《点集拓扑讲义(第四版)》(熊金城) P233 定义7.8.2
设\(X\)是一个拓扑空间,
\(F\)是一族映射,
\(F\)中的每一个元素都是从拓扑空间\(X\)到某一个拓扑空间的一个映射.
\begin{itemize}
	\item 如果对于任意\(a,b \in X\),
	只要满足\(a \neq b\),
	就存在\(f \in F\),
	使得\(f(a) \neq f(b)\),
	则称“\(F\)是一个\DefineConcept{区别点的映射族}”.

	\item 如果对于\(X\)中任意一个点\(x\)和\(X\)中任意一个不含点\(x\)的闭集\(B\),
	存在\(f \in F\),
	使得\(f(x) \notin \TopoClosureL{f(B)}\),
	则称“\(F\)是一个\DefineConcept{区别点和闭集的映射族}”.
\end{itemize}
\end{definition}

\begin{lemma}
%@see: 《点集拓扑讲义(第四版)》(熊金城) P234 引理7.8.1(嵌入引理)
设\(X\)是拓扑空间族\(\{X_\gamma\}_{\gamma \in \Gamma}\)的积空间,
\(Y\)是一个拓扑空间,
\(f\)是从\(Y\)到\(X\)的映射,
\(p_\gamma\)是从\(X\)到坐标集\(X_\gamma\)的投射,
令\begin{equation*}
	F \defeq \Set{
		p_\alpha \circ f
		\given
		\alpha \in \Gamma
	},
\end{equation*}
则\begin{itemize}
	\item \(f\)是一个连续映射,当且仅当\(F\)是一个连续映射族;
	\item \(f\)是一个单射,当且仅当\(F\)是一个区别点的映射族;
	\item 如果\(F\)是一个区别点和闭集的映射族,
	则\(g\colon Y \to f(Y), y \mapsto f(y)\)是一个开映射.
\end{itemize}
%TODO proof
\end{lemma}

\begin{theorem}
%@see: 《点集拓扑讲义(第四版)》(熊金城) P235 定理7.8.2(嵌入定理)
设\(X\)是一个拓扑空间,
则\(X\)是一个提赫诺夫空间,
当且仅当\(X\)可以嵌入某一个方体.
%TODO proof
\end{theorem}

\begin{theorem}
%@see: 《点集拓扑讲义(第四版)》(熊金城) P235 定理7.8.3
设\(X\)是一个拓扑空间,
则\(X\)是一个提赫诺夫空间,
当且仅当\(X\)可以嵌入某一个紧致的豪斯多夫空间.
%TODO proof
\end{theorem}


\chapter{完备度量空间}
本章介绍度量空间的一个重要的非拓扑性质.

\section{度量空间的完备化}
\subsection{度量空间的完备性}
度量空间的完备性是用关于度量空间中的点列的收敛的语言来刻画的.
由于度量空间本身便是拓扑空间,
所以我们在\cref{definition:序列.序列的聚点}
已经以拓扑的方式给出了度量空间中的点列收敛的定义,
并且可以通过度量的语言予以描述(参见\cref{theorem:序列.度量空间中的收敛序列}).
现在通过以下定义在度量空间中挑选出一类特殊的序列.

\begin{definition}\label{definition:度量空间的完备化.度量空间中的柯西序列}
%@see: 《点集拓扑讲义(第四版)》(熊金城) P237 定义8.1.1
设\((X,\rho)\)是一个度量空间,
\(\{x_n\}_{n\geq0}\)是\(X\)中的一个序列.
如果\begin{equation*}
	(\forall\epsilon>0)
	(\exists N\in\omega)
	(\forall m>N)
	(\forall n>N)
	[\rho(x_m,x_n)<\epsilon],
\end{equation*}
则称“\(\{x_n\}_{n\geq0}\)是\(X\)中的一个\DefineConcept{柯西序列}(Cauchy sequence)”.
\end{definition}

\begin{definition}\label{definition:度量空间的完备化.完备度量空间}
%@see: 《点集拓扑讲义(第四版)》(熊金城) P237 定义8.1.1
设\((X,\rho)\)是一个度量空间.
如果\(X\)中的每一个柯西序列都收敛,
则称“\((X,\rho)\)是一个\DefineConcept{完备度量空间}”.
\end{definition}

\begin{proposition}
%@see: 《点集拓扑讲义(第四版)》(熊金城) P237
度量空间中每一个收敛序列都是柯西序列,但反之不然.
\end{proposition}

\begin{example}
%@see: 《点集拓扑讲义(第四版)》(熊金城) P237 例8.1.1
实数空间\(\mathbb{R}\)是一个完备度量空间.
\end{example}

\begin{theorem}
%@see: 《点集拓扑讲义(第四版)》(熊金城) P238 定理8.1.1
完备度量空间中的每一个闭的度量子空间都是完备度量空间.
%TODO proof
\end{theorem}

\begin{lemma}
%@see: 《点集拓扑讲义(第四版)》(熊金城) P238 引理8.1.2
设\((X,\rho)\)是一个度量空间,\(Y \subseteq X\).
如果\(Y\)中的每一个柯西序列都在\(X\)中收敛,
则\(Y\)的闭包\(\overline{Y}\)中的每一个柯西序列也都在\(X\)中收敛.
%TODO proof
\end{lemma}

\begin{corollary}
%@see: 《点集拓扑讲义(第四版)》(熊金城) P239 推论8.1.3
设\((X,\rho)\)是一个度量空间,\(Y\)是\(X\)的一个稠密子集.
如果\(Y\)中的每一个柯西序列都在\(X\)中收敛,
则\(X\)是一个完备度量空间.
%TODO proof
\end{corollary}

\begin{theorem}
%@see: 《点集拓扑讲义(第四版)》(熊金城) P239 定理8.1.4
\(n\)维欧氏空间\(\mathbb{R}^n\)是完备度量空间.
\end{theorem}

\begin{theorem}
%@see: 《点集拓扑讲义(第四版)》(熊金城) P239 定理8.1.4
希尔伯特空间\(\mathbb{H}\)是完备度量空间.
\end{theorem}

\subsection{保距映射}
\begin{definition}
%@see: 《点集拓扑讲义(第四版)》(熊金城) P240 定义8.1.2
设\((X,\rho)\)和\((Y,\sigma)\)是两个度量空间,映射\(f\colon X \to Y\).
如果对于任意\(x_1,x_2 \in X\)有\(\sigma(f(x_1),f(x_2)) = \rho(x_1,x_2)\),
则称“\(f\)是\DefineConcept{保距的}”
“\(f\)是一个\DefineConcept{保距映射}”.
\end{definition}

\begin{proposition}
%@see: 《点集拓扑讲义(第四版)》(熊金城) P240
保距映射一定是一个单射.
\end{proposition}

\begin{proposition}
%@see: 《点集拓扑讲义(第四版)》(熊金城) P240
恒同映射是一个保距映射.
\end{proposition}

\begin{proposition}
%@see: 《点集拓扑讲义(第四版)》(熊金城) P240
两个保距映射的复合还是保距映射.
\end{proposition}

\begin{proposition}
%@see: 《点集拓扑讲义(第四版)》(熊金城) P240
保距满射的逆还是保距映射.
\end{proposition}

\begin{definition}
%@see: 《点集拓扑讲义(第四版)》(熊金城) P240 定义8.1.2
设\((X,\rho)\)和\((Y,\sigma)\)是两个度量空间.
如果存在一个从\(X\)到\(Y\)的保距满射,
则称“度量空间\((X,\rho)\)与度量空间\((Y,\sigma)\)~\DefineConcept{同距}”.
\end{definition}

\begin{proposition}
%@see: 《点集拓扑讲义(第四版)》(熊金城) P240
同距关系是等价关系.
对于度量空间\(X,Y,Z\)而言\begin{itemize}
	\item {\rm\bf 自反性} \(X\)与\(X\)同距;
	\item {\rm\bf 对称性} \(\text{$X$与$Y$同距} \implies \text{$Y$与$X$同距}\);
	\item {\rm\bf 传递性} \(\text{$X$与$Y$同距} \;\land\; \text{$Y$与$Z$同距} \implies \text{$X$与$Z$同距}\).
\end{itemize}
\end{proposition}

\begin{proposition}
%@see: 《点集拓扑讲义(第四版)》(熊金城) P240
保距满射一定是一个同胚.
\end{proposition}

\begin{proposition}
%@see: 《点集拓扑讲义(第四版)》(熊金城) P240
同距的度量空间是同胚的.
\end{proposition}

\subsection{度量空间的完备化}
\begin{definition}
设\(X\)是一个度量空间,\(X^*\)是一个完备度量空间.
如果\(X\)与\(X^*\)的一个稠密的度量子空间同距,
则称“完备度量空间\(X^*\)是度量空间\(X\)的一个\DefineConcept{完备化}”.
\end{definition}

\begin{proposition}
%@see: 《点集拓扑讲义(第四版)》(熊金城) P240
实数空间\(\mathbb{R}\)是有理数空间\(\mathbb{Q}\)的一个完备化.
\end{proposition}

\begin{theorem}
%@see: 《点集拓扑讲义(第四版)》(熊金城) P240 定理8.1.5
每一个度量空间都有完备化.
%TODO proof
\end{theorem}

一个度量空间可以有许多完备化.
但是,在同距的意义下,它的完备化是唯一的.

\begin{theorem}
%@see: 《点集拓扑讲义(第四版)》(熊金城) P242 定理8.1.6
每一个度量空间的任意两个完备化同距.
%TODO proof
\end{theorem}

\begin{corollary}
%@see: 《点集拓扑讲义(第四版)》(熊金城) P243 推论8.1.7
完备度量空间的任何一个完备化都与它本身同距.
\end{corollary}

\section{度量空间的完备性与紧致性}
\subsection{\texorpdfstring{$\epsilon$--网}{\textepsilon 网},完全有界度量空间}
\begin{definition}
%@see: 《点集拓扑讲义(第四版)》(熊金城) P244 定义8.2.1
设\((X,\rho)\)是一个度量空间,
实数\(\epsilon>0\),
\(A\)是\(X\)的有限子集.
如果\begin{equation*}
	(\forall x \in X)
	[\rho(x,A) < \epsilon],
\end{equation*}
则称“\(A\)是\(X\)的一个~\DefineConcept{\(\epsilon\)--网}(epsilon net)”.
%@see: https://ti.inf.ethz.ch/ew/courses/CG12/lecture/Chapter%2015.pdf
\end{definition}

\begin{definition}
%@see: 《点集拓扑讲义(第四版)》(熊金城) P244 定义8.2.1
设\((X,\rho)\)是一个度量空间.
如果对于任何实数\(\epsilon\),
\(X\)有一个\(\epsilon\)--网,
则称“度量空间\((X,\rho)\)是\DefineConcept{完全有界的}(totally bounded)”.
%@see: https://mathresearch.utsa.edu/wiki/index.php?title=Totally_Bounded_Metric_Spaces
\end{definition}

\begin{proposition}
%@see: 《点集拓扑讲义(第四版)》(熊金城) P244
设\(X\)是度量空间,
则“\(X\)是完全有界的”是“\(X\)是有界的”的充分不必要条件.
\begin{proof}
包含着无限多个点的离散度量空间是有界的,但不是完全有界的.
\end{proof}
\end{proposition}

\begin{theorem}
%@see: 《点集拓扑讲义(第四版)》(熊金城) P244 定理8.2.1
设\((X,\rho)\)是一个度量空间,
则\begin{equation*}
	\text{$(X,\rho)$是紧致的}
	\iff
	\text{$(X,\rho)$是完全有界的完备度量空间}.
\end{equation*}
%TODO proof
\end{theorem}

\begin{theorem}
%@see: 《点集拓扑讲义(第四版)》(熊金城) P245 定理8.2.2
设\((X,\rho)\)是一个完备度量空间.
如果\(\Powerset X\)中的一个单调减序列\(\{E_n\}_{n\geq1}\)满足\begin{equation*}
	\lim_{n\to\infty} \diam E_n = 0,
\end{equation*}
则\(\bigcap_{n=1}^\infty \TopoClosureL{E_n}\)是一个单点集.
%TODO proof
\end{theorem}

\subsection{贝尔定理}
\begin{theorem}[贝尔定理]\label{theorem:度量空间的完备性与紧致性.贝尔定理1}
%@see: 《点集拓扑讲义(第四版)》(熊金城) P246 定理8.2.3
设\((X,\rho)\)是一个完备度量空间.
如果\(\B\)是\(X\)中的一个稠密开集族,
\(\B\)是可数的,
则\(\bigcap \B\)是\(X\)中的一个稠密子集.
%TODO proof
\end{theorem}

下面的\cref{theorem:度量空间的完备性与紧致性.贝尔定理2}
是\cref{theorem:度量空间的完备性与紧致性.贝尔定理1} 的另一个常见的表达方式.
\begin{definition}
%@see: 《点集拓扑讲义(第四版)》(熊金城) P247 定义8.2.2
%@see: 《Real Analysis Modern Techniques and Their Applications Second Edition》(Folland) P13
设\(X\)是一个拓扑空间.
如果\(X\)的子集\(A\)的闭包的内部是空集,
即\(\TopoInterior{(\TopoClosureM{A})} = \emptyset\),
则称“\(A\)是\(X\)的一个\DefineConcept{无处稠密子集}(nowhere dense subset)”
或“\(A\)在\(X\)中是\DefineConcept{无处稠密的}(\(A\) is \emph{nowhere dense} in \(X\))”.
\end{definition}

\begin{definition}
%@see: 《点集拓扑讲义(第四版)》(熊金城) P247 定义8.2.2
设\(X\)是一个拓扑空间.
如果\(X\)的子集\(F\)可以表示为
\(X\)中可数个无处稠密子集的并,
则称“\(F\)是\DefineConcept{第一范畴集}”;
否则称“\(F\)是\DefineConcept{第二范畴集}”.
\end{definition}

\begin{theorem}[贝尔定理]\label{theorem:度量空间的完备性与紧致性.贝尔定理2}
%@see: 《点集拓扑讲义(第四版)》(熊金城) P247 定理8.2.4
完备度量空间中的任何一个非空开集都是第二范畴集.
%TODO proof
\end{theorem}

从\cref{theorem:度量空间的完备性与紧致性.贝尔定理2} 出发
也易于证明\cref{theorem:度量空间的完备性与紧致性.贝尔定理1}.


\chapter{映射空间}
\section{压缩映射原理}
%\cref{example:收敛准则.压缩映射原理1,example:收敛准则.压缩映射原理2}
%@see: https://www.math.cuhk.edu.hk/course_builder/1819/math3060/

\begin{definition}
%@see: 《数学分析(第7版 第二卷)》(卓里奇) P29 定义1
设映射\(f\colon X \to X\).
如果点\(a \in X\)满足\(f(a) = a\),
则称“点\(a\)是\(f\)的一个\DefineConcept{不动点}(fixed point)”.
\end{definition}

\begin{definition}
%@see: 《数学分析(第7版 第二卷)》(卓里奇) P29 定义2
设\((X,\rho)\)是一个度量空间,
映射\(f\colon X \to X\).
如果\begin{equation*}
	(\exists q\in\mathbb{R})
	(\forall x_1,x_2 \in X)
	[
		0 < q < 1
		\implies
		\rho(f(x_1),f(x_2))
		\leq
		q \cdot \rho(x_1,x_2)
	],
\end{equation*}
则称“\(f\)是\(X\)上的\DefineConcept{压缩映射}(contraction mapping)”.
\end{definition}

\begin{theorem}[皮卡--巴拿赫不动点原理]
%@see: 《数学分析(第7版 第二卷)》(卓里奇) P29 定理(皮卡--巴拿赫不动点原理)
设\((X,\rho)\)是一个完备度量空间,
则\(X\)上的压缩映射具有唯一的不动点.
%@see: https://wuli.wiki/online/ConMap.html
\end{theorem}

\begin{definition}
设\((X,\rho)\)是一个度量空间.
把集合\begin{equation*}
	\Set{ f \in X^X \given \text{$f$是$X$上的压缩映射} }
\end{equation*}称为“\(X\)上的\DefineConcept{压缩映射空间}”.
\end{definition}

\begin{definition}
%@see: 《数学分析(第7版 第二卷)》(卓里奇) P30 命题(关于不动点的稳定性)
设\((X,\rho)\)是一个度量空间,
\((\Omega,\T)\)是拓扑空间,
\(\{f_t\}_{t\in\Omega}\)是一个映射族.
如果\begin{equation*}
	(\exists q\in\mathbb{R})
	(\forall t\in\Omega)
	(\forall x_1,x_2 \in X)
	[
		0 < q < 1
		\implies
		\rho(f_t(x_1),f_t(x_2))
		\leq
		q \cdot \rho(x_1,x_2)
	],
\end{equation*}
则称“映射族\(\{f_t\}_{t\in\Omega}\)是\DefineConcept{一致压缩的}”.
\end{definition}

\begin{proposition}[关于不动点的稳定性]
%@see: 《数学分析(第7版 第二卷)》(卓里奇) P30 命题(关于不动点的稳定性)
设\((X,\rho)\)是一个完备度量空间,
\((\Omega,\T)\)是拓扑空间,
\(\tilde{X}\)是\(X\)上的压缩映射空间,
映射\(f\colon \Omega \to \tilde{X}\),
且\begin{itemize}
	\item 映射族\(\{f_t\}_{t\in\Omega}\)是一致压缩的,
	\item 对\(\forall x \in X\),
	映射\(g_x\colon \Omega \to X, t \mapsto f_t(x)\)在某个点\(t_0 \in \Omega\)连续,
	即\(\lim_{t \to t_0} g_x(t) = g_x(t_0)\),
\end{itemize}
那么方程\(x = f_t(x)\)的解\(a(t) \in X\)在点\(t_0\)连续地依赖于\(t\),
即\(\lim_{t \to t_0} a(t) = a(t_0)\).
\end{proposition}


\chapter{基本群}

\endgroup
