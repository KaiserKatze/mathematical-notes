\section{基数排序}
我们可以利用等势的概念来判定两个集合的大小是否一致,
但是我们要怎么说明一个集合比另一个集合更大呢?

\begin{definition}
%@see: 《Elements of Set Theory》 P145 Definition
设\(A,B\)都是集合.
当且仅当存在从\(A\)到\(B\)的单射,
称“\(A\)由\(B\)~\DefineConcept{主导}(\(A\) is \emph{dominated} by \(B\))”,
记作\(A \preceq B\)或\(B \succeq A\).
\end{definition}

\begin{definition}
设\(A,B\)都是集合.
定义:\begin{equation*}
	A \prec B
	\defiff
	B \succ A
	\defiff
	A \preceq B \land A \napprox B.
\end{equation*}
\end{definition}

\begin{example}
%@see: 《Elements of Set Theory》 P145
\(A \preceq A\).
\end{example}

\begin{example}
%@see: 《Elements of Set Theory》 P145
\(A \subseteq B \implies A \preceq B\).
\begin{proof}
显然\(A\)上的恒等映射\(1_A\)是从\(A\)到\(B\)的一个单射.
\end{proof}
\end{example}

\begin{theorem}\label{theorem:集合论.基数.集合主导关系的等价定义}
%@see: 《Elements of Set Theory》 P145
\(
	A \preceq B
	\iff
	(\exists b)[b \subseteq B \limp A \approx b]
\).
\begin{proof}
因为
	映射\(f\)是从\(A\)到\(B\)的一个映射
	当且仅当\(f\)是从\(A\)到\(B\)的某个子集上的一个映射,
所以
	\(A\)由\(B\)主导
	当且仅当\(A\)与\(B\)的某个子集等势.
\end{proof}
\end{theorem}

\begin{definition}
设\(A,B\)都是集合.
定义:\begin{gather}
	%@see: 《Elements of Set Theory》 P145
	\card A \leq \card B
	\defiff A \preceq B, \\
	\card A \geq \card B
	\defiff A \succeq B, \\
	\card A < \card B
	\defiff A \prec B, \\
	\card A > \card B
	\defiff A \succ B.
\end{gather}
\end{definition}

\begin{proposition}
设\(A,B,C\)都是集合,
则\begin{gather}
	\card A = \card B
	\implies
	\card A \leq \card B, \\
	\card A \leq \card B,
	\card B \leq \card C
	\implies
	\card A \leq \card C, \\
	\card A \leq \card B,
	\card B < \card C
	\implies
	\card A < \card C.
\end{gather}
\end{proposition}

\begin{proposition}
设映射\(f\colon A \to B\),
则\begin{align*}
	\text{$f$是单射}
	\implies
	\card A \leq \card B, \\
	\text{$f$是满射}
	\implies
	\card A \geq \card B, \\
	\text{$f$是双射}
	\implies
	\card A = \card B.
\end{align*}
\end{proposition}

\begin{proposition}
%@see: 《点集拓扑讲义(第四版)》(熊金城) P31 定理1.7.6
设\(X\)是一个集合,集合\(Y=\{0,1\}\).
从\(X\)到\(Y\)的映射空间\(Y^X\)满足\begin{equation*}
	X \prec Y^X.
\end{equation*}
\end{proposition}

\begin{lemma}\label{theorem:基数.集合在映射下的分解}
%@see: 《实变函数论(第三版)》(周民强) P15 引理1.4(集合在映射下的分解)
设映射\(f\colon A \to B,
g\colon B \to A\),
则存在集合\(X_1,X_2,Y_1,Y_2\),
使得\begin{itemize}
	\item \(\{X_1,X_2\}\)是\(A\)的一个划分,
	\(\{Y_1,Y_2\}\)是\(B\)的一个划分,
	\item \(X_1\)在\(f\)下的像为\(f\ImageOfSetUnderRelation{X_1}=Y_1\),
	\(Y_2\)在\(g\)下的像为\(g\ImageOfSetUnderRelation{Y_2}=X_2\).
\end{itemize}
\begin{proof}
我们首先作出如下定义:
设\(E\)是\(A\)的一个子集(不妨假定\(B - f\ImageOfSetUnderRelation{E} \neq \emptyset\)),
若\begin{equation*}
	E \cap g\ImageOfSetUnderRelation{B - f\ImageOfSetUnderRelation{E}} = \emptyset,
	\eqno(1)
\end{equation*}
则称“\(E\)是\(A\)中的\DefineConcept{分离集}”.

现将\(A\)中的分离集的全体记为\(\Gamma\).
令\(X_1 = \bigcup \Gamma\).
我们有\(X_1 \in \Gamma\).%TODO 为什么?
事实上,
对于\(\forall E \in \Gamma\),
由于\(X_1 \supseteq E\),
故从(1)式可知\(E \cap g\ImageOfSetUnderRelation{B - f\ImageOfSetUnderRelation{X_1}} = \emptyset\),
从而有\(X_1 \cap g\ImageOfSetUnderRelation{B - f\ImageOfSetUnderRelation{X_1}} = \emptyset\).
这说明\(X_1\)不光是\(A\)中的分离集,而且是\(\Gamma\)中对包含关系\(\subseteq\)的最大元.

现在令\(f\ImageOfSetUnderRelation{X_1}=Y_1,
B-Y_1=Y_2,
g\ImageOfSetUnderRelation{Y_2}=X_2\).
首先知道\(B=Y_1 \cup Y_2\).
其次,由于\(X_1 \cap X_2 = \emptyset\),
故又易得\(X_1 \cup X_2 = A\).
事实上,若不然,那么\(\exists a \in A\),
使得\(a \notin X_1 \cup X_2\).
现在取\(X_3 = X_1 \cup \{a\}\),
我们有\begin{equation*}
	Y_1 = f\ImageOfSetUnderRelation{X_1} \subseteq f\ImageOfSetUnderRelation{X_3},
	\qquad
	Y_2 \supseteq B - f\ImageOfSetUnderRelation{X_3},
\end{equation*}
从而知\(X_2 \supseteq g\ImageOfSetUnderRelation{B - f\ImageOfSetUnderRelation{X_3}}\).
这就是说\begin{equation*}
	X_1 \cap g\ImageOfSetUnderRelation{B - f\ImageOfSetUnderRelation{X_3}} = \emptyset.
\end{equation*}
由此可得\begin{equation*}
	X_3 \cap g\ImageOfSetUnderRelation{B - f\ImageOfSetUnderRelation{X_3}} = \emptyset.
\end{equation*}
这与“\(X_1\)是\(\Gamma\)的最大元”相矛盾!
\end{proof}
\end{lemma}
\hyperref[theorem:基数.集合在映射下的分解]{映射分解定理}是由巴拿赫建立的.

\begin{theorem}\label{theorem:集合论.施罗德--伯恩斯坦定理}
%@see: 《Elements of Set Theory》 P147 Schroder-Bernstein Theorem(a)
%@see: 《点集拓扑讲义(第四版)》(熊金城) P33 定理1.7.8
%@see: 《实变函数论(第三版)》(周民强) P16 定理1.5(Cantor-Bernstein定理)
设\(A,B\)都是集合,则\begin{equation*}
	A \preceq B,
	B \preceq A
	\implies
	A \approx B.
\end{equation*}
\begin{proof}
根据定义可知,
存在单射\(f\colon A \to B\)和\(g\colon B \to A\).
根据\hyperref[theorem:基数.集合在映射下的分解]{映射分解定理}知\begin{equation*}
	A = X_1 \cup X_2, \qquad
	B = Y_1 \cup Y_2, \qquad
	f\ImageOfSetUnderRelation{X_1} = Y_1, \qquad
	f\ImageOfSetUnderRelation{Y_2} = X_2.
\end{equation*}
因为\(f\)和\(g^{-1}\)是双射,
所以我们可以构造一个从\(A\)到\(B\)的双射:\begin{equation*}
	h(x) = \left\{ \begin{array}{cl}
		f(x), & x \in X_1, \\
		g^{-1}(x), & x \in X_2,
	\end{array} \right.
\end{equation*}
这就说明\(A \approx B\).
\end{proof}
\end{theorem}
\cref{theorem:集合论.施罗德--伯恩斯坦定理}
常被称为“施罗德--伯恩斯坦定理”.
不过有时候也把它称为“康托--伯恩斯坦定理”.

最先是康托于1897年证明了该定理,
但是他的证明利用了一条等价于选择公理的原理.
后来,施罗德在他于1896年撰写的一篇摘要中发表了这条定理,
但是他给出的证明是有缺陷的.
菲利克斯·伯恩斯坦是第一个为这条定理给出完全令人满意的证明的人,
波莱尔把他的证明发表在自己于1898年出版的
《函数论讲义(Le\c{c}ons sur la th\'eorie des fonctions)》上.
但是这里对\cref{theorem:集合论.施罗德--伯恩斯坦定理} 的证明是由巴拿赫给出的.

\begin{corollary}
%@see: 《Elements of Set Theory》 P147 Schroder-Bernstein Theorem(b)
设\(\kappa,\lambda\)都是基数.
如果\(\kappa \leq \lambda\)且\(\lambda \leq \kappa\),
则\(\kappa = \lambda\).
\end{corollary}

利用\cref{theorem:集合论.施罗德--伯恩斯坦定理} 可以得到如下推论.
\begin{corollary}\label{theorem:集合论.施罗德--伯恩斯坦定理.推论1}
%@see: 《Elements of Set Theory》 P148 Examples 1.
设\(A,B,C\)是集合,
且\(A \subseteq B \subseteq C\).
若\(A \approx C\),
则\(B \approx C\).
\end{corollary}

\begin{example}\label{example:基数.闭区间与全体实数等势1}
%@see: 《Elements of Set Theory》 P148 Examples 2.
%@see: 《实变函数论(第三版)》(周民强) P17 例5
因为\((-1,1)\subseteq[-1,1]\subseteq\mathbb{R}\),
且\((-1,1)\approx\mathbb{R}\),
所以\([-1,1]\approx\mathbb{R}\).

因为\((0,1)\subseteq[0,1]\subseteq\mathbb{R}\),
且\((0,1)\approx\mathbb{R}\),
所以\([0,1]\approx\mathbb{R}\).
\end{example}
在\cref{example:基数.闭区间与全体实数等势1} 中,
我们如果要直接建立\([-1,1]\)与\(\mathbb{R}\)之间的一一对应,就会比较繁琐.
这是因为想用一个连续函数来表示所需的一一对应是不可能的,
在以后我们会学到,闭区间上的连续函数的值域仍是闭区间(\cref{theorem:极限.最值定理}).
不过,只要运用\cref{theorem:集合论.施罗德--伯恩斯坦定理.推论1}
以及\cref{example:基数.开区间与全体实数等势2} 的结论,
就可以间接证明\([-1,1]\)与\(\mathbb{R}\)等势.

\begin{theorem}
%@see: 《Elements of Set Theory》 P149 Theorem 6L
设\(\kappa,\lambda,\mu\)都是基数,
则\begin{gather}
	\kappa \leq \lambda
	\implies
	\kappa + \mu \leq \kappa + \mu, \\
	\kappa \leq \lambda
	\implies
	\kappa \cdot \mu \leq \lambda \cdot \mu, \\
	\kappa \leq \lambda
	\implies
	\kappa^\mu \leq \lambda^\mu, \\
	\kappa \leq \lambda,
	\neg(\kappa = \mu = 0)
	\implies
	\mu^\kappa \leq \mu^\lambda.
\end{gather}
\end{theorem}

\begin{proposition}
%@see: 《Real Analysis Modern Techniques and Their Applications Second Edition》(Gerald B. Folland) P7 0.9 Proposition
设\(A\)是集合,
则\begin{equation*}
	\card A < \card(\Powerset A).
\end{equation*}
\end{proposition}

\begin{example}
%@see: 《Elements of Set Theory》 P148 Examples 4.
\(\mathbb{R}\approx2^\omega\),
\(\mathbb{R}\approx\Powerset\omega\).
\end{example}

\begin{example}
%@see: 《实变函数论(第三版)》(周民强) P17 思考题 1.
设\(A_1 \subseteq A_2,
B_1 \subseteq B_2,
A_1 \approx B_1,
A_2 \approx B_2\).
考察\((A_2 - A_2) \approx (B_2 - B_1)\)是否成立.
%TODO
\end{example}
\begin{example}
%@see: 《实变函数论(第三版)》(周民强) P17 思考题 2.
设\((A - B) \approx (B - A)\).
考察\(A \approx B\)是否成立.
%TODO
\end{example}
\begin{example}
%@see: 《实变函数论(第三版)》(周民强) P18 思考题 3.
设\(A \subseteq B,
A \approx (A \cup C)\).
考察\(B \approx (B \cup C)\)是否成立.
%TODO
\end{example}
