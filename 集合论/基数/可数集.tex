
\section{可数集,不可数集}
\begin{definition}
%@see: 《Real Analysis Modern Techniques and Their Applications Second Edition》(Gerald B. Folland) P7
%@see: 《点集拓扑讲义(第四版)》(熊金城) P29 定义1.7.1
%@see: 《实变函数论(第三版)》(周民强) P18
%@see: 《Elements of Set Theory》 P159 Definition
%@see: https://math.stackexchange.com/q/1431303/591741
设\(A\)是集合.

若\(
	% 存在一个从集合\(A\)到正整数集的单射
	\card A \leq \card\mathbb{N}
\),
则称“\(A\)是\DefineConcept{可数的}(countable)”
或“\(A\)是\DefineConcept{可列的}(denumerable)”
\footnote{
	% 《Real Analysis Modern Techniques and Their Applications Second Edition》(Gerald B. Folland)把可数集与可列集定义为相同的概念,即基数不大于\(\card\mathbb{N}\)的集合.
	% 《点集拓扑讲义(第四版)》(熊金城)没有定义“可列集”.
	% 《实变函数论(第三版)》(周民强)把“可列集”定义为基数恰好等于\(\card\mathbb{N}\)的集合.
	在有的书上,
	“可列集”特指那些基数恰好等于\(\card\mathbb{N}\)的集合.
}.

反之,若\(
	\card A > \card\mathbb{N}
\),
则称“\(A\)是\DefineConcept{不可数的}(uncountable)”
或“\(A\)是\DefineConcept{不可列的}(non-denumerable)”.
% 虽然在很多教材上都定义\(\aleph_0 \defeq \card\mathbb{N}\)和\(\aleph \defeq \card\mathbb{R}\),但是这里不采取该记法.
\end{definition}

\begin{definition}
%@see: 《Real Analysis Modern Techniques and Their Applications Second Edition》(Gerald B. Folland) P7
若集合\(A\)既是无限的又是可数的,
则称“\(A\)是\DefineConcept{可数无限的}(countably infinite)”.
\end{definition}

\begin{theorem}\label{theorem:集合论.基数.任一无限集必定包含一个可数无限子集}
%@see: 《实变函数论(第三版)》(周民强) P18 定理1.6
任一无限集必定包含一个可数无限子集.
\begin{proof}
设\(E\)是一个无限集,
任取\(a_1 \in E\),
再任取\(a_2 \in E-\{a_1\}\),
以此类推,直至选出\(\{\AutoTuple{a}{n}\}\).
因为\(E\)是无限的,所以\(E-\{\AutoTuple{a}{n}\}\neq\emptyset\).
于是我们还可以继续任取\(a_{n+1} \in E-\{\AutoTuple{a}{n}\}\).
这样,我们就得到一个集合\(\{\AutoTuple{a}{n+1},\dotsc\}\),
它既是\(E\)的一个子集,也是一个可数无限集.
\end{proof}
\end{theorem}
\begin{remark}
\cref{theorem:集合论.基数.任一无限集必定包含一个可数无限子集} 说明:
在众多的无限集中,最小的无限集的基数是\(\card\mathbb{N}\).
\end{remark}

\begin{proposition}
%@see: 《Real Analysis Modern Techniques and Their Applications Second Edition》(Gerald B. Folland) P7
有限集是可数的.
任一有限集\(A\)的基数\(\card A\)等于它的元素个数.
\end{proposition}

\begin{proposition}
%@see: 《点集拓扑讲义(第四版)》(熊金城) P30 定理1.7.1
%@see: 《Elements of Set Theory》 P159
可数集的任何子集都是可数集.
\end{proposition}

\begin{proposition}
%@see: 《点集拓扑讲义(第四版)》(熊金城) P30 定理1.7.2
设\(X\)和\(Y\)都是集合,映射\(f\colon X\to Y\).
如果\(X\)是可数集,
则\(f(X)\)也是可数集.
\end{proposition}

\begin{proposition}
%@see: 《点集拓扑讲义(第四版)》(熊金城) P30 定理1.7.3
集合\(X\)是可数集,当且仅当存在从自然数集\(\omega\)到\(X\)的一个满射.
\end{proposition}

\begin{proposition}
%@see: 《Elements of Set Theory》 P159
如果集合\(A,B\)都是可数的,
那么\(A \cup B\)也是可数的.
\end{proposition}

\begin{proposition}
%@see: 《Real Analysis Modern Techniques and Their Applications Second Edition》(Gerald B. Folland) P8 0.10 Proposition a.
%@see: 《点集拓扑讲义(第四版)》(熊金城) P30 定理1.7.4
%@see: 《Elements of Set Theory》 P159
如果集合\(A,B\)都是可数的,
那么\(A \times B\)也是可数的.
\end{proposition}

\begin{proposition}
%@see: 《实变函数论(第三版)》(周民强) P20 例9
如果\(n\)个集合\(\AutoTuple{A}{n}\)都是可数无限的,
那么\(\AutoTuple{A}{n}[\times]\)也是可数无限的.
\end{proposition}

\begin{example}
%@see: 《实变函数论(第三版)》(周民强) P18 例6
设\(A\)是有限集,\(B\)是可数无限集,则\(A \cup B\)是可数无限集.
\begin{proof}
不妨设\(A = \Set{\AutoTuple{a}{n}},
B = \Set{b_1,\dotsc}\).
若\(A \cap B = \emptyset\),
则由\begin{equation*}
	A \cup B = \Set{\AutoTuple{a}{n},b_1,\dotsc}
\end{equation*}
可知\(A \cup B\)是可数无限集;
若\(A \cap B = \emptyset\),
则由于\(A \cup B = (A-B) \cup B\),
易知\(A \cup B\)仍是可数无限集.
\end{proof}
\end{example}

\begin{proposition}
%@see: 《Real Analysis Modern Techniques and Their Applications Second Edition》(Gerald B. Folland) P8 0.10 Proposition b.
%@see: 《点集拓扑讲义(第四版)》(熊金城) P31 定理1.7.5
%@see: 《Elements of Set Theory》 P159 Theorem 6Q
如果\begin{itemize}
	\item 集合\(A\)是可数的,
	\item 对于每个\(a \in A\),\(X_a\)都是可数的,
\end{itemize}
那么\(\bigcup_{a \in A} X_a\)是可数的.
\end{proposition}

\begin{proposition}
%@see: 《实变函数论(第三版)》(周民强) P18 定理1.7
如果\begin{itemize}
	\item 集合\(A\)是可数无限的,
	\item 对于每个\(a \in A\),\(X_a\)都是可数无限的,
\end{itemize}
那么\(\bigcup_{a \in A} X_a\)是可数无限的.
\end{proposition}

\begin{proposition}
%@see: 《Real Analysis Modern Techniques and Their Applications Second Edition》(Gerald B. Folland) P8 0.10 Proposition c.
如果集合\(A\)是可数无限的,
那么\(\card A = \card{\mathbb{N}}\).
\end{proposition}

\begin{proposition}
如果集合\(A\)满足\(\card A = \card\mathbb{N}\),
则\(A\)是可数无限的.
\begin{proof}
假设\(\card A = \card\mathbb{N}\),
那么\(\card A \leq \card\mathbb{N}\),
由定义可知\(A\)是可数的.
因为\hyperref[theorem:集合论.鸽巢原理.推论3]{自然数集是无限的},
所以\(A\)是无限的.
因此\(A\)是可数无限的.
\end{proof}
\end{proposition}

\begin{corollary}
%@see: 《Real Analysis Modern Techniques and Their Applications Second Edition》(Gerald B. Folland) P8 0.11 Corollary
%@see: 《Elements of Set Theory》 P159
\(\mathbb{Z}\)是可数的.
\end{corollary}

\begin{corollary}
%@see: 《Real Analysis Modern Techniques and Their Applications Second Edition》(Gerald B. Folland) P8 0.11 Corollary
%@see: 《Elements of Set Theory》 P159
%@see: 《实变函数论(第三版)》(周民强) P19 例7
\(\mathbb{Q}\)是可数的.
\end{corollary}

\begin{example}
%@see: 《实变函数论(第三版)》(周民强) P20 例10
\(\mathbb{R}\)中的互斥开区间族是可数集.
\end{example}

\begin{example}
%@see: 《实变函数论(第三版)》(周民强) P20 例11
\(\mathbb{R}\)上单调函数的不连续点集是可数集.
\end{example}

\begin{proposition}
%@see: 《Real Analysis Modern Techniques and Their Applications Second Edition》(Gerald B. Folland) P8 0.12 Proposition
\(\card(\Powerset\mathbb{N}) = \card\mathbb{R}\).
\begin{proof}
定义映射\(f\colon \Powerset\mathbb{N}\to\mathbb{R}\),
使得\begin{equation*}
	f(A) = \left\{ \begin{array}{cl}
		\sum_{n \in A} 2^{-n}, & \text{集合$\mathbb{N}-A$是无限的}, \\
		1 + \sum_{n \in A} 2^{-n}, & \text{集合$\mathbb{N}-A$是有限的}.
	\end{array} \right.
\end{equation*}
易见\(f\)是单射.

再定义映射\(g\colon \Powerset\mathbb{Z}\to\mathbb{R}\),
使得\begin{equation*}
	g(A) = \left\{ \begin{array}{cl}
		\log\left(\sum_{n \in A} 2^{-n}\right), & \text{集合$A$有下界}, \\
		0, & \text{集合$A$没有下界}.
	\end{array} \right.
\end{equation*}
易见\(g\)是满射.

考虑到\(\card(\Powerset\mathbb{Z}) = \card(\Powerset\mathbb{N})\),
利用\hyperref[theorem:集合论.施罗德--伯恩斯坦定理]{施罗德--伯恩斯坦定理}便得结论.
\end{proof}
\end{proposition}

\begin{corollary}
%@see: 《Real Analysis Modern Techniques and Their Applications Second Edition》(Gerald B. Folland) P8 0.13 Corollary
设\(A\)是集合.
如果\(\card A \geq \card\mathbb{R}\),
那么\(A\)是不可数的.
\end{corollary}

\begin{theorem}
%@see: 《点集拓扑讲义(第四版)》(熊金城) P34 定理1.7.9
%@see: 《Elements of Set Theory》 P159
\(\mathbb{R}\)是不可数的.
\end{theorem}

\begin{proposition}
%@see: 《Elements of Set Theory》 P159
设\(A\)是无限集,
则\(\Powerset A\)是不可数的.
\end{proposition}
\begin{corollary}
%@see: 《Elements of Set Theory》 P159
设集合\(A\)的幂集\(\Powerset A\)是可数的,
则\(A\)是有限的.
\end{corollary}

\begin{proposition}
%@see: 《Real Analysis Modern Techniques and Their Applications Second Edition》(Gerald B. Folland) P9 0.14 Proposition
设\(A,B\)都是集合.
若\(\card A \leq \card\mathbb{R}\)且\(\card B \leq \card\mathbb{R}\),
则\(\card(A \times B) \leq \card\mathbb{R}\).
\end{proposition}

\begin{proposition}
%@see: 《Real Analysis Modern Techniques and Their Applications Second Edition》(Gerald B. Folland) P9 0.14 Proposition
如果有标集族\(\{X_a\}_{a \in A}\)同时满足\(
	\card A \leq \card\mathbb{R}
\)
和\(
	(\forall a \in A)
	[\card X_a \leq \card\mathbb{R}]
\),
那么\begin{equation*}
	\card\left( \bigcup_{a \in A} X_a \right) \leq \card\mathbb{R}.
\end{equation*}
\end{proposition}

证明集合等势的方法:\begin{enumerate}
	\item 利用\cref{definition:基数.等势的定义},直接证明双射的存在性.
	\item 根据\cref{theorem:集合论.等势相当于等价关系},运用中间集合过渡.
	\item 采用\hyperref[theorem:基数.集合在映射下的分解]{分解}、合并等思想.
	\item 参照\hyperref[theorem:集合论.施罗德--伯恩斯坦定理]{施罗德--伯恩斯坦定理},证明两个集合互相主导.
\end{enumerate}
