\section{等势}
我们时不时想要研究讨论下面的问题:
两个集合是否具有相同的大小?
或者说,这两个集合的元素的个数是否相同?
还是说,一个集合相比于另一个拥有更多的元素?

\begin{definition}\label{definition:基数.等势的定义}
%@see: 《Elements of Set Theory》 P129 Definition
%@see: 《实变函数论(第三版)》(周民强) P15 定义1.15
设\(A,B\)都是集合.
如果存在一个从\(A\)到\(B\)的双射,
那么称“\(A\)与\(B\) \DefineConcept{等势}(\(A\) is \emph{equinumerous} to \(B\))”,
% 《实变函数论(第三版)》(周民强)将这种情况称为“\(A\)与\(B\) \DefineConcept{对等}”,记为\(A \sim B\).
记作\(A \approx B\);
并把这个映射称为“\(A\)和\(B\)之间的\DefineConcept{一一对应}%
(\emph{one-to-one correspondence} between \(A\) and \(B\))”;
否则称“\(A\)与\(B\)不等势”,
记作\(A \napprox B\).
\end{definition}

\begin{example}
%@see: 《Elements of Set Theory》 P129 Example
%@see: 《实变函数论(第三版)》(周民强) P15 例3
自然数集\(\omega\)与\(\omega\times\omega\)等势.
%@see: 《Elements of Set Theory》 P133 Exercise 2.
这是因为映射\begin{equation*}
	J_1\colon \omega\times\omega\to\omega,
	\opair{m,n} \mapsto \frac12[(m+n)^2+3m+n]
	% 相当于将\(\opair{m,n}\)映成\((1 + 2 + \dotsb + (m+n)) + m\)
\end{equation*}是双射.
%@see: 《Elements of Set Theory》 P133 Exercise 1.
不止如此,映射\begin{equation*}
	J_2\colon \omega\times\omega\to\omega,
	\opair{m,n} \mapsto 2^m(2n+1)-1
\end{equation*}也是双射.
%TODO
%Mathematica: Table[1/2 ((m + n)^2 + 3 m + n), {m, 0, 3}, {n, 0, 3}] // TableForm
\end{example}

\begin{example}
%@see: 《Elements of Set Theory》 P130 Example
\(\omega\)与\(\mathbb{Q}\)等势.
%TODO
\end{example}

\begin{example}\label{example:基数.开区间与全体实数等势1}
%@see: 《Elements of Set Theory》 P130 Example
开区间\((0,1)\)与\(\mathbb{R}\)等势.
这是因为\begin{equation*}
	f(x) = \tan\frac{\pi(2x-1)}2
	\quad(0<x<1)
\end{equation*}是从\((0,1)\)到\(\mathbb{R}\)的双射.
\end{example}

\begin{example}\label{example:基数.开区间与全体实数等势2}
%@see: 《实变函数论(第三版)》(周民强) P15 例4
开区间\((-1,1)\)与\(\mathbb{R}\)等势.
这是因为\begin{equation*}
	f(x) = \frac{x}{1-x^2}
	\quad(-1<x<1)
\end{equation*}是从\((-1,1)\)到\(\mathbb{R}\)的双射.
\end{example}

\begin{example}
%@see: 《Elements of Set Theory》 P133 Exercise 3.
构造开区间\((0,1)\)与\(\mathbb{R}\)之间的一个一一对应,
使得有理数映成有理数,无理数映成无理数.
%TODO
\end{example}

\begin{example}
%@see: 《Elements of Set Theory》 P133 Exercise 4.
通过构造一一对应,证明:开区间\((0,1)\)与闭区间\([0,1]\)等势.
%TODO proof
\end{example}

\begin{example}\label{example:基数.幂集与特征函数空间等势}
%@see: 《Elements of Set Theory》 P131 Example
证明:对于任意集合\(A\),
总有\(A\)的幂集\(\Powerset A\)与映射空间\(2^A\)等势.
\begin{proof}
对于\(A\)的每个子集\(a\),
定义从\(A\)到\(\{0,1\}\)的映射\begin{equation*}
	f_a(x) = \left\{ \begin{array}{cl}
		1, & x \in a, \\
		0, & x \in A - a.
	\end{array} \right.
\end{equation*}
注意到\(2=\{0,1\}\),
于是\(f_a \in 2^A\).
再定义映射\begin{equation*}
	H\colon \Powerset A \to 2^A,
	a \mapsto f_a.
\end{equation*}
对于\(\forall a,b \in \Powerset A\),
只要\(a \neq b\),
就\(\exists x \in A\),
满足\(x \in a\),但不满足\(x \in b\),
这就使得\(f_a(x) = 1\)而\(f_b(x) = 0\),
于是\(f_a \neq f_b\),
由此可见\(H\)是单射.
又因为对于\(\forall g \in 2^A\),
若记\(G = \Set{ x \in A \given g(x) = 1 }\),
则必有\((\forall x \in A - G)[g(x) = 0]\),
于是\(H(G) = g\),
即\(g \in \ran H\).
由\(g\)的任意性可知\(\ran H = 2^A\),
因此\(H\)是满射.
既然\(H\)是双射,那么\(\Powerset A \approx 2^A\).
\end{proof}
\end{example}

\begin{example}
全体完全平方数组成的集合
\(
	\Set{
		n^2
		\given
		n \in \omega
	}
\)
与自然数集\(\omega\)等势.
\end{example}

\begin{example}
%@see: 《实变函数论(第三版)》(周民强) P15 例2
全体偶数组成的集合
\(
	\Set{
		2k
		\given
		k \in \mathbb{Z}
	}
\)
与整数集\(\mathbb{Z}\)等势.
\end{example}

\begin{theorem}\label{theorem:集合论.等势相当于等价关系}
%@see: 《Elements of Set Theory》 P132 Theorem 6A
%@see: 《Elements of Set Theory》 P133 Exercise 5.
%@see: 《实变函数论(第三版)》(周民强) P15
对于任意集合\(A,B,C\):\begin{itemize}
	\item \(A \approx A\).
	\item \(A \approx B \implies B \approx A\).
	\item \(A \approx B, B \approx C \implies A \approx C\).
\end{itemize}
\end{theorem}
\cref{theorem:集合论.等势相当于等价关系}
说明\DefineConcept{等势}(equinumerosity)
和每一个等价关系都一样,
也具有自反性、对称性、传递性,
但要注意,它不是一个等价关系,
这是因为它关注的是全体集合.

\begin{theorem}
%@see: 《Elements of Set Theory》 P132 Theorem 6B(a)
自然数集\(\omega\)与实数集\(\mathbb{R}\)不等势.
\end{theorem}

\begin{theorem}
%@see: 《Elements of Set Theory》 P132 Theorem 6B(b)
没有集合与它的幂集等势.
\end{theorem}
