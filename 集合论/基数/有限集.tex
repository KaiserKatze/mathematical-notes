\section{有限集}
\subsection{有限集,无限集}
\begin{definition}
%@see: 《Elements of Set Theory》 P133 Definition
设\(A\)是集合.
当且仅当存在自然数\(n\)与之等势时,
称“\(A\)是\DefineConcept{有限的}(finite)”
或“\(A\)是\DefineConcept{有限集}”;
否则称“\(A\)是\DefineConcept{无限的}(infinite)”
或“\(A\)是\DefineConcept{无限集}”,
即\begin{gather}
	\text{\(A\)是有限的}
	\defiff
	(\exists n\in\omega)[A \approx n]; \\
	\text{\(A\)是无限的}
	\defiff
	(\forall n\in\omega)[A \napprox n].
\end{gather}
\end{definition}

\begin{example}
任一自然数是有限集.
\end{example}

\subsection{鸽巢原理}
\begin{theorem}\label{theorem:集合论.鸽巢原理}
%@see: 《Elements of Set Theory》 P134 Pigeonhole Principle
没有自然数与它的真子集等势.
\begin{proof}
下面证明对于每一个自然数\(n \in \omega\),
任意一个从\(n\)到\(n\)的单射\(f\)都是满射.
用数学归纳法.
令\begin{equation*}
	T \defeq \Set{
		n \in \omega
		\given
		(\forall f \in n^n)
		[\text{$f$是单射} \implies \ran f = n]
	}.
\end{equation*}
显然\(0 \in T\),
% 映射空间\(\emptyset^\emptyset=\Set{\emptyset}\)
这是因为从\(0\)到\(0\)的唯一一个映射是\(\emptyset\),并且它的值域是\(0\).
假设自然数\(k \in T\),
\(k^+\)是\(k\)的后继,
且\(f\)是从\(k^+\)到\(k^+\)的单射.
要证\(k^+ \in T\),
只需证\(\ran f = k^+\).
注意到\(f\)在\(k\)上的限制\(f \SetRestrict k\)
将\(k\)映射到\(k^+\)里.
下面分情况讨论:\begin{itemize}
	\item 假设集合\(k\)对\(f\)封闭,
	也就是说,\(f \SetRestrict k\)将\(k\)映射到\(k\)里.
	根据归纳假设有\(k \in T\),
	而\(f \SetRestrict k\)作为一个从\(k\)到\(k\)的单射,
	必定成立\begin{equation*}
		\ran(f \SetRestrict k) = k.
	\end{equation*}
	既然\(f\)是单射,
	根据定义\begin{equation*}
		(\forall y \in \ran f)
		(\exists! x)
		[\opair{x,y} \in f],
	\end{equation*}
	而\begin{equation*}
		k^+ - \ran(f \SetRestrict k) = \{k\},
	\end{equation*}
	那么\(f(k)\)的取值只可能是\(k\),
	因此\(\ran f = k \cup \{k\} = k^+\).

	\item 假设集合\(k\)对\(f\)不封闭,
	也就是说,存在一个小于\(k\)的数\(p\)满足\(f(p) = k\).
	在这种情况下,我们可以交换\(p\)和\(k\)这两个数在映射\(f\)下的像,
	定义映射\(g\colon k^+ \to k^+\),
	令\begin{equation*}
		g(x) = \left\{ \begin{array}{cl}
			f(k), & x = p, \\
			f(p), & x = k, \\
			f(x), & \text{其他},
		\end{array} \right.
	\end{equation*}
	那么\(g\)可以将\(k^+\)映射到\(k^+\)里,
	且\(k\)对\(g\)封闭.
	于是由上一种情况可知,\(\ran g = k^+\).
	因为\(\ran f = \ran g\),
	所以\(\ran f = k^+\).
\end{itemize}
综上所述,不论是在哪种情况下,均有\(\ran f = k^+\).
于是\(T\)是归纳的,且\(T = \omega\).
\end{proof}
\end{theorem}

我们把\cref{theorem:集合论.鸽巢原理}
称为\DefineConcept{鸽巢原理}(Pigeonhole Principle).

\begin{corollary}\label{theorem:集合论.鸽巢原理.推论1}
%@see: 《Elements of Set Theory》 P135 Corollary 6C
没有有限集与它的真子集等势.
\begin{proof}
假设\(A\)是有限集,
根据定义,存在自然数\(n\),使得\(A\)与\(n\)等势,
不妨设\(g\)是\(A\)和\(n\)之间的一个一一对应.
用反证法.
假设\(A\)和它的某个真子集\(A'\)之间存在一个一一对应\(f\).

如\cref{figure:基数.没有有限集与它的真子集等势} 所示,
考虑复合映射\begin{equation*}
	h \defeq g \circ f \circ g^{-1}.
\end{equation*}
显然,\(h\)将\(n\)映射到\(n\)里.
因为\(g\)是双射,
%TODO “双射的逆是双射”缺乏依据
所以\(g^{-1}\)也是双射.
因为\(g,g^{-1},f\)都是单射,
所以由\cref{example:集合论.两个单根集的复合是单根的} 可知,
\(h\)是单射.
考虑到\((\forall a \in A - \ran f)[g(a) \in n - C]\),
%TODO 不知道这里翻译正不正确,原话说的是:Consider any a in \(A - \ran f\); then \(g(a) \in n - C\).
可以看出\(h\)的值域\(\ran h = C\)是\(n\)的真子集.
%TODO 怎么看出来的?
于是\(n\)与\(\ran h\)等势,
与鸽巢原理矛盾!
\begin{figure}[hbt]
	\centering
	\tikzset{
		myset/.style={x radius=2cm,y radius=1cm}
	}
	\def\labeloffset{1cm}
	\begin{tikzpicture}
		% \draw[help lines, color=gray!30, dashed] (0,0) grid (-6,-4);
		\draw(0,0)circle[myset]node[above=\labeloffset]{$n$};
		\draw(-6,0)circle[myset]node[above=\labeloffset]{$A$};
		\draw(-6,-4)circle[myset]node[below=\labeloffset]{$A$};
		\draw(0,-4)circle[myset]node[below=\labeloffset]{$n$};
		\draw({2*cos(120)-6},{sin(120)-4})--({2*cos(120)-6},{-sin(120)-4});
		\draw({2*cos(120)},{sin(120)-4})--({2*cos(120)},{-sin(120)-4});
		\fill(-7.3,-4.3)circle(2pt)node[above left]{$a$}
			++(5.8,0)circle(2pt)node[above]{$g(a)$};
		\draw(-5.7,-4)node{$\ran f$}
			++(6,0)node{$C$};
		\begin{scope}[-{Latex[length=3mm,width=0pt 10]}]
			\draw(-2,0)--(-4,0)node[midway,above]{$g^{-1}$};
			\draw(-6,-1)--(-6,-3)node[midway,left]{$f$};
			\draw(-4,-4)--(-2,-4)node[midway,below]{$g$};
		\end{scope}
	\end{tikzpicture}
	\caption{}
	\label{figure:基数.没有有限集与它的真子集等势}
\end{figure}
\end{proof}
\end{corollary}

\begin{corollary}\label{theorem:集合论.鸽巢原理.推论2}
%@see: 《Elements of Set Theory》 P136 Corollary 6D(a)
任一集合,如果与其真子集等势,就是无限的.
\begin{proof}
这是\cref{theorem:集合论.鸽巢原理.推论1} 的逆否命题.
\end{proof}
\end{corollary}

\begin{corollary}\label{theorem:集合论.鸽巢原理.推论3}
%@see: 《Elements of Set Theory》 P136 Corollary 6D(b)
自然数集\(\omega\)是无限的.
\begin{proof}
鉴于映射\(\sigma\colon \omega \to \omega-\{0\}, n \mapsto n^+\)是满射,
可知\(\omega \approx \omega-\{0\} \subset \omega\),
于是由\cref{theorem:集合论.鸽巢原理.推论2} 可知\(\omega\)是无限集.
\end{proof}
\end{corollary}

\begin{corollary}
%@see: 《Elements of Set Theory》 P136 Corollary 6E
任一有限集总与唯一一个自然数等势.
\begin{proof}
假设有限集\(A\)与自然数\(m\)等势,又与自然数\(n\)等势.
那么有\(m\)与\(n\)等势.
由\cref{theorem:集合论.自然数集的三一律,theorem:集合论.自然数集的三一律.推论1} 可知,
自然数的序关系只有以下三种情况:
要么成立\(m = n\),要么成立\(m \subset n\),要么成立\(n \subset m\).
由\(m \approx n\)可知\(m \subset n\)和\(n \subset m\)都是不可能的.
因此只能是\(m = n\),唯一性得证.
\end{proof}
\end{corollary}

\subsection{有限集的基数}
\begin{definition}
%@see: 《Elements of Set Theory》 P136
设\(A\)是有限集.
满足\(A \approx n\)的自然数\(n\),
称为“\(A\)的\DefineConcept{基数}(cardinal number)\footnote{%
有的地方也将其称为\DefineConcept{元数}或\DefineConcept{浓度}.}”,
记作\(\card A\)或\(\abs{A}\).
\end{definition}

\begin{example}
%@see: 《Elements of Set Theory》 P136
\((\forall n\in\omega)[\card n = n]\).
\end{example}

\begin{property}
\((\forall A)[A \approx \card A]\).
\end{property}

\begin{property}
\(A \approx B \iff \card A = \card B\).
\begin{proof}
假设\(\card A = \card B\),那么\(A \approx \card B\)和\(B \approx \card A\).
利用传递性,再由\(A \approx \card A\)和\(B \approx \card B\)便得\(A \approx B\).
\end{proof}
\end{property}

\begin{definition}
定义:\(\card A \neq \card B \defiff A \napprox B\).
\end{definition}

\begin{example}
设\(a,b,c,d\)各不相同,
则\(\card\{a,b,c,d\} = 4\),
这是因为\(\{a,b,c,d\} \approx 4\).
\end{example}

%@see: 《Elements of Set Theory》 P136
% 原话是:selecting a one-to-one correspondence is the process called `counting'.
挑选一个一一对应的过程,
称为\DefineConcept{计数}(counting).

\begin{property}
设\(A,B\)都是集合,且\(\card A = \card B\).
\begin{itemize}
	\item 如果\(A\)是有限集,则\(B\)也是有限集.
	\item 如果\(A\)是无限集,则\(B\)也是无限集.
\end{itemize}
\end{property}

从\cref{example:基数.幂集与特征函数空间等势} 可以推出任一有限集的所有子集的个数.
\begin{theorem}
设\(A\)是有限集,
则\begin{equation}
	\card \Powerset A
	= 2^{\card A}.
\end{equation}
\end{theorem}
\begin{remark}
换一套记号,可能更有利于记忆:\begin{equation*}
	\abs{2^A}
	= 2^{\abs{A}}.
\end{equation*}
\end{remark}

\begin{lemma}\label{theorem:基数.自然数n的真子集只能与小于n的自然数等势}
%@see: 《Elements of Set Theory》 P137 Lemma 6F
如果\(C\)是自然数\(n\)的真子集,
那么存在\(m<n\)使得\(C \approx m\).
\begin{proof}
下面我们证明每一个自然数\(n\)的任意一个真子集都与\(n\)的某个元素等势.
用数学归纳法.
令\begin{equation*}
	T \defeq \Set{
		n \in \omega
		\given
		(\forall c \subset n)
		(\exists m \in n)
		[c \approx m]
	}.
\end{equation*}
由于\(0 = \emptyset\)没有真子集,
所以\(0 \in T\).
假设自然数\(k \in T\),
考虑\(k\)的后继\(k^+\)的一个真子集\(C\),
分情况讨论.
\begin{itemize}
	\item 假设\(C = k\),
	那么\(C \approx k \in k^+\).

	\item 假设\(C \subset k\),
	由于\(k \in T\),于是存在\(m \in k \in k^+\),使得\(C \approx m\).

	\item 假设\(k \in C\),
	那么\(C = (C \cap k) \cup \{k\}\)
	且\(C \cap k \subset k\).
	因为\(k \in T\),
	所以存在\(m \in k\)使得\(C \cap k \approx m\).
	假如\(f\)是\(C \cap k\)和\(m\)之间的一一对应,
	那么\(f \cup \{\opair{k,m}\}\)就是\(C\)和\(m^+\)之间的一一对应.
	鉴于\(m \in k\),势必有\(m^+ \in k^+\).
	因此\(C \approx m^+ \in k^+ \in T\).
\end{itemize}
综上所述,\(T\)是归纳的,且\(T = \omega\).
\end{proof}
\end{lemma}

\begin{theorem}
%@see: 《Elements of Set Theory》 P138 Corollary 6G
任一有限集的任一子集也是有限的.
\begin{proof}
假设\(A \subseteq B\),
又假设\(f\)是\(B\)和某个自然数\(n\)之间的一一对应,
那么\begin{equation*}
	A \approx f\ImageOfSetUnderRelation{A} \subseteq n.
\end{equation*}
由\cref{theorem:基数.自然数n的真子集只能与小于n的自然数等势} 可知,
存在自然数\(m\)满足\(m \ineq n\),
使得\(f\ImageOfSetUnderRelation{A} \approx m\).
因此\(A \approx m \ineq n \in \omega\).
\end{proof}
\end{theorem}

\subsection{基数概念的推广}
我们可以看出,对于任意两个有限集\(A\)和\(B\),
我们有\begin{equation*}
	A \approx \card A
	\quad\text{和}\quad
	\card A = \card B
	\iff
	A \approx B.
\end{equation*}

这让我们不禁思考无限集是否具有同样的性质,希望能够找到某种“数字”用以衡量无限集的大小,
至于这种“数字”属于什么集合,就无足轻重了,
就好比我们不关心数字\(2\)属于哪一个集合一样.
因此我们需要扩展基数的定义,让它可以衡量无限集的大小,
其中的核心需求是:
对于任意集合\(A\)(不论它是有限的,还是有限的),
定义\(\card A\),
使得\begin{equation*}
	\card A = \card B
	\iff
	A \approx B
\end{equation*}恒成立.

考虑所有基数等于\(\alpha\)的集合的汇集\begin{equation*}
	\Ka \defeq \Set{ X \given \card X = \alpha }.
\end{equation*}
只要\(\Ka\)中有1个集合是有限集,
那么\(\Ka\)中的所有集合都是有限集,
这种情况下,我们将\(\alpha\)称为\DefineConcept{有限基数}(finite cardinal);
反之,则称其为\DefineConcept{无限基数}(infinite cardinal).
显然,所有自然数都是有限基数,
而\(\card\omega,
\card\mathbb{R},
\card\Powerset\omega,
\card\Powerset\Powerset\omega\)都是无限基数.
