\section{基数算术}
\begin{definition}\label{definition:基数.基数算术的定义}
%@see: 《Elements of Set Theory》 P139 Definition
设\(A,B\)都是集合,
\(\card A = \alpha\),
\(\card B = \beta\).
\begin{enumerate}
	\item 如果\(A\)和\(B\)互斥,那么\(\card(A \cup B)=\alpha+\beta\).
	\item \(\card(A \times B)=\alpha\cdot\beta\).
	\item \(\card(A^B)=\alpha^\beta\).
\end{enumerate}
\end{definition}

需要注意到的是,
如果给定的两个集合\(A,B\)不互斥,
那么可以令\begin{equation*}
	A' = A\times\{0\}, \qquad
	B' = B\times\{1\},
\end{equation*}
从而有\(A' \approx A\),\(B' \approx B\),
以及\(A'\)和\(B'\)互斥.

\begin{theorem}
%@see: 《Elements of Set Theory》 P139 Theorem 6H
设\(A \approx B\),\(C \approx D\).
\begin{enumerate}
	\item 如果\(A \cap C = B \cap D = \emptyset\),
	那么\(A \cup C \approx B \cup D\).
	\item \(A \times C \approx B \times D\).
	\item \(A^C \approx B^D\).
\end{enumerate}
\end{theorem}

\begin{theorem}
%@see: 《Elements of Set Theory》 P142 Theorem 6I
设\(\kappa,\lambda,\mu\)都是基数,
则\begin{gather}
	\kappa + \lambda = \lambda + \kappa, \\
	\kappa \cdot \lambda = \lambda \cdot \kappa, \\
	\kappa + (\lambda + \mu) = (\kappa + \lambda) + \mu, \\
	\kappa \cdot (\lambda \cdot \mu) = (\kappa \cdot \lambda) \cdot \mu, \\
	\kappa \cdot (\lambda + \mu) = \kappa \cdot \mu + \kappa \cdot \mu, \\
	\kappa^{\lambda+\mu} = \kappa^\lambda + \kappa^\mu, \\
	(\kappa \cdot \lambda)^\mu = \kappa^\mu \cdot \lambda^\mu, \\
	(\kappa^\lambda)^\mu = \kappa^{\lambda \cdot \mu}.
\end{gather}
\end{theorem}

\begin{theorem}
%@see: 《Elements of Set Theory》 P143 Theorem 6J
设\(m,n\)都是有限基数,
则\(
	m + n,
	m \cdot n,
	m^n
\)这三种基数算术均可看成自然数算术.
\end{theorem}

\begin{theorem}
%@see: 《Elements of Set Theory》 P144 Corollary 6K
如果\(A,B\)都是有限集,
那么\begin{equation*}
	A \cup B, \qquad
	A \times B, \qquad
	A^B
\end{equation*}也都是有限的.
\end{theorem}
