\section{基数}
\subsection{等势}
我们时不时想要研究讨论下面的问题:
两个集合是否具有相同的大小?
或者说,这两个集合的元素的个数是否相同?
还是说,一个集合相比于另一个拥有更多的元素?

\begin{definition}\label{definition:基数.等势的定义}
%@see: 《Elements of Set Theory》 P129 Definition
%@see: 《实变函数论(第三版)》(周民强) P15 定义1.15
设\(A,B\)都是集合.
如果存在一个从\(A\)到\(B\)的双射,
那么称“\(A\)与\(B\) \DefineConcept{等势}(\(A\) is \emph{equinumerous} to \(B\))”,
% 《实变函数论(第三版)》(周民强)将这种情况称为“\(A\)与\(B\) \DefineConcept{对等}”,记为\(A \sim B\).
记作\(A \approx B\);
并把这个映射称为“\(A\)和\(B\)之间的\DefineConcept{一一对应}%
(\emph{one-to-one correspondence} between \(A\) and \(B\))”;
否则称“\(A\)与\(B\)不等势”,
记作\(A \napprox B\).
\end{definition}

\begin{example}
%@see: 《实变函数论(第三版)》(周民强) P15 例2
自然数集\(\omega\)与\(\Set{ y \in \omega \given x \in \omega \land y = 2x }\)等势.
\end{example}

\begin{example}
%@see: 《Elements of Set Theory》 P129 Example
%@see: 《实变函数论(第三版)》(周民强) P15 例3
自然数集\(\omega\)与\(\omega\times\omega\)等势.
这是因为映射\[
	J_1\colon \omega\times\omega\to\omega,
	\opair{m,n} \mapsto \frac12[(m+n)^2+3m+n]
\]是双射.
%@see: 《Elements of Set Theory》 P133 Exercise 1.
不止如此,映射\[
	J_2\colon \omega\times\omega\to\omega,
	\opair{m,n} \mapsto 2^m(2n+1)-1
\]也是双射.
%TODO
\end{example}

\begin{example}
%@see: 《Elements of Set Theory》 P130 Example
\(\omega\)与\(\mathbb{Q}\)等势.
%TODO
\end{example}

\begin{example}\label{example:基数.开区间与全体实数等势1}
%@see: 《Elements of Set Theory》 P130 Example
开区间\((0,1)\)与\(\mathbb{R}\)等势.
这是因为\[
	f(x) = \tan\frac{\pi(2x-1)}2
	\quad(0<x<1)
\]是从\((0,1)\)到\(\mathbb{R}\)的双射.
\end{example}

\begin{example}\label{example:基数.开区间与全体实数等势2}
%@see: 《实变函数论(第三版)》(周民强) P15 例4
开区间\((-1,1)\)与\(\mathbb{R}\)等势.
这是因为\[
	f(x) = \frac{x}{1-x^2}
	\quad(-1<x<1)
\]是从\((-1,1)\)到\(\mathbb{R}\)的双射.
\end{example}

\begin{example}\label{example:基数.幂集与特征函数空间等势}
%@see: 《Elements of Set Theory》 P131 Example
证明:对于任意集合\(A\),
总有\(A\)的幂集\(\Powerset A\)与映射空间\(2^A\)等势.
\begin{proof}
对于\(A\)的每个子集\(a\),
定义从\(A\)到\(\{0,1\}\)的映射\[
	f_a(x) = \left\{ \begin{array}{cl}
		1, & x \in a, \\
		0, & x \in A - a.
	\end{array} \right.
\]
注意到\(2=\{0,1\}\),
于是\(f_a \in 2^A\).
再定义映射\[
	H\colon \Powerset A \to 2^A,
	a \mapsto f_a.
\]
对于\(\forall a,b \in \Powerset A\),
只要\(a \neq b\),
就\(\exists x \in A\),
满足\(x \in a\),但不满足\(x \in b\),
这就使得\(f_a(x) = 1\)而\(f_b(x) = 0\),
于是\(f_a \neq f_b\),
由此可见\(H\)是单射.
又因为对于\(\forall g \in 2^A\),
若记\(G = \Set{ x \in A \given g(x) = 1 }\),
则必有\((\forall x \in A - G)[g(x) = 0]\),
于是\(H(G) = g\),
即\(g \in \ran H\).
由\(g\)的任意性可知\(\ran H = 2^A\),
因此\(H\)是满射.
既然\(H\)是双射,那么\(\Powerset A \approx 2^A\).
\end{proof}
\end{example}

\begin{example}
全体完全平方数组成的集合与自然数集\(\omega\)等势.
\end{example}

\begin{example}
全体偶数组成的集合与整数集\(\mathbb{Z}\)等势.
\end{example}

\begin{theorem}\label{theorem:集合论.等势相当于等价关系}
%@see: 《Elements of Set Theory》 P132 Theorem 6A
%@see: 《实变函数论(第三版)》(周民强) P15
对于任意集合\(A,B,C\):\begin{itemize}
	\item \(A \approx A\).
	\item \(A \approx B \implies B \approx A\).
	\item \(A \approx B \land B \approx C \implies A \approx C\).
\end{itemize}
\end{theorem}
\cref{theorem:集合论.等势相当于等价关系}
说明\DefineConcept{等势}(equinumerosity)
和每一个等价关系都一样,
也具有自反性、对称性、传递性,
但要注意,它不是一个等价关系,
这是因为它关注的是全体集合.

\begin{theorem}
%@see: 《Elements of Set Theory》 P132 Theorem 6B(a)
自然数集\(\omega\)与实数集\(\mathbb{R}\)不等势.
\end{theorem}

\begin{theorem}
%@see: 《Elements of Set Theory》 P132 Theorem 6B(b)
没有集合与它的幂集等势.
\end{theorem}

\subsection{有限集,无限集}
\begin{definition}
%@see: 《Elements of Set Theory》 P133 Definition
设\(A\)是集合.
当且仅当存在自然数\(n\)与之等势时,
称“\(A\)是\DefineConcept{有限的}(finite)”
或“\(A\)是\DefineConcept{有限集}”;
否则称“\(A\)是\DefineConcept{无限的}(infinite)”
或“\(A\)是\DefineConcept{无限集}”,
即\begin{gather}
	\text{\(A\)是有限的}
	\defiff
	(\exists n\in\omega)[A \approx n]; \\
	\text{\(A\)是无限的}
	\defiff
	(\forall n\in\omega)[A \napprox n].
\end{gather}
\end{definition}

\begin{example}
任一自然数是有限集.
\end{example}

\subsection{鸽巢原理}
\begin{theorem}[鸽巢原理]
%@see: 《Elements of Set Theory》 P134 Pigeonhold Principle
没有自然数与它的真子集等势.
\end{theorem}

\begin{corollary}
%@see: 《Elements of Set Theory》 P135 Corollary 6C
没有有限集与它的真子集等势.
\end{corollary}

\begin{corollary}
%@see: 《Elements of Set Theory》 P136 Corollary 6D
任一集合,如果与其真子集等势,就是无限的.
\end{corollary}

\begin{corollary}
%@see: 《Elements of Set Theory》 P136 Corollary 6E
任一集合总与唯一一个自然数等势.
\end{corollary}

\subsection{基数}
\begin{definition}
%@see: 《Elements of Set Theory》 P136
设\(A\)是有限集.
满足\(A \approx n\)的自然数\(n\),
称为“\(A\)的\DefineConcept{基数}(cardinal number)\footnote{%
有的地方也将其称为\DefineConcept{元数}或\DefineConcept{浓度}.}”,
记作\(\card A\)或\(\abs{A}\).
\end{definition}

\begin{example}
%@see: 《Elements of Set Theory》 P136
\((\forall n\in\omega)[\card n = n]\).
\end{example}

我们可以看出,对于任意两个有限集\(A\)和\(B\),
我们有\[
	A \approx \card A
	\quad\text{和}\quad
	\card A = \card B
	\iff
	A \approx B.
\]

这让我们不禁思考无限集是否具有同样的性质.
因此我们需要扩展基数的定义,让它可以衡量无限集的大小.

从\cref{example:基数.幂集与特征函数空间等势} 可以推出任一有限集的所有子集的个数.
\begin{theorem}
设\(A\)是有限集,
则\begin{equation}
	\card \Powerset A
	= 2^{\card A}.
\end{equation}
\end{theorem}

\begin{lemma}
%@see: 《Elements of Set Theory》 P137 Lemma 6F
如果\(C\)是自然数\(n\)的真子集,
那么存在\(m<n\)使得\(C \approx m\).
\end{lemma}

\begin{theorem}
%@see: 《Elements of Set Theory》 P137 Corollary 6G
任一有限集的任一子集也是有限的.
\end{theorem}

\subsection{基数算术}
\begin{definition}
%@see: 《Elements of Set Theory》 P139 Definition
设\(A,B\)都是集合,
\(\card A = \alpha\),
\(\card B = \beta\).
\begin{enumerate}
	\item 如果\(A\)和\(B\)互斥,那么\(\card(A \cup B)=\alpha+\beta\).
	\item \(\card(A \times B)=\alpha\cdot\beta\).
	\item \(\card(A^B)=\alpha^\beta\).
\end{enumerate}
\end{definition}

需要注意到的是,
如果给定的两个集合\(A,B\)不互斥,
那么可以令\[
	A' = A\times\{0\}, \qquad
	B' = B\times\{1\},
\]
从而有\(A' \approx A\),\(B' \approx B\),
以及\(A'\)和\(B'\)互斥.

\begin{theorem}
%@see: 《Elements of Set Theory》 P139 Theorem 6H
设\(A \approx B\),\(C \approx D\).
\begin{enumerate}
	\item 如果\(A \cap C = B \cap D = \emptyset\),
	那么\(A \cup C \approx B \cup D\).
	\item \(A \times C \approx B \times D\).
	\item \(A^C \approx B^D\).
\end{enumerate}
\end{theorem}

\begin{theorem}
%@see: 《Elements of Set Theory》 P144 Corollary 6K
如果\(A,B\)都是有限集,
那么\[
	A \cup B, \qquad
	A \times B, \qquad
	A^B
\]也都是有限的.
\end{theorem}

\subsection{基数排序}
我们可以利用等势的概念来判定两个集合的大小是否一致,
但是我们要怎么说明一个集合比另一个集合更大呢?

\begin{definition}
%@see: 《Elements of Set Theory》 P145 Definition
设\(A,B\)都是集合.
当且仅当存在从\(A\)到\(B\)的单射,
称“\(A\)由\(B\)~\DefineConcept{主导}(\(A\) is \emph{dominated} by \(B\))”,
记作\(A \preceq B\)或\(B \succeq A\).
\end{definition}

\begin{definition}
设\(A,B\)都是集合.
定义:\[
	A \prec B
	\defiff
	B \succ A
	\defiff
	A \preceq B \land A \napprox B.
\]
\end{definition}

\begin{example}
\(A \preceq A\).
\end{example}

\begin{example}
\(A \subseteq B \implies A \preceq B\).
\end{example}

\begin{theorem}
\(A \preceq B
\iff
(\exists b)[b \subseteq B \implies A \approx b]\).
\end{theorem}

\begin{definition}
设\(A,B\)都是集合.
定义:\begin{gather}
	\card A \leq \card B
	\defiff A \preceq B, \\
	\card A \geq \card B
	\defiff A \succeq B, \\
	\card A < \card B
	\defiff A \prec B, \\
	\card A > \card B
	\defiff A \succ B.
\end{gather}
\end{definition}

\begin{proposition}
设映射\(f\colon A \to B\),
则\begin{align*}
	\text{$f$是单射}
	\implies
	\card A \leq \card B, \\
	\text{$f$是满射}
	\implies
	\card A \geq \card B, \\
	\text{$f$是双射}
	\implies
	\card A = \card B.
\end{align*}
\end{proposition}

\begin{proposition}
%@see: 《点集拓扑讲义(第四版)》(熊金城) P31 定理1.7.6
设\(X\)是一个集合,集合\(Y=\{0,1\}\).
从\(X\)到\(Y\)的映射空间\(Y^X\)满足\[
	X \prec Y^X.
\]
\end{proposition}

\begin{lemma}\label{theorem:基数.集合在映射下的分解}
%@see: 《实变函数论(第三版)》(周民强) P15 引理1.4(集合在映射下的分解)
设映射\(f\colon A \to B,
g\colon B \to A\),
则存在集合\(X_1,X_2,Y_1,Y_2\),
使得\begin{itemize}
	\item \(\{X_1,X_2\}\)是\(A\)的一个划分,
	\(\{Y_1,Y_2\}\)是\(B\)的一个划分,
	\item \(X_1\)在\(f\)下的像为\(f\ImageOfSetUnderRelation{X_1}=Y_1\),
	\(Y_2\)在\(g\)下的像为\(g\ImageOfSetUnderRelation{Y_2}=X_2\).
\end{itemize}
\begin{proof}
我们首先作出如下定义:
设\(E\)是\(A\)的一个子集(不妨假定\(B - f\ImageOfSetUnderRelation{E} \neq \emptyset\)),
若\[
	E \cap g\ImageOfSetUnderRelation{B - f\ImageOfSetUnderRelation{E}} = \emptyset,
	\eqno(1)
\]
则称“\(E\)是\(A\)中的\DefineConcept{分离集}”.

现将\(A\)中的分离集的全体记为\(\Gamma\).
令\(X_1 = \bigcup \Gamma\).
我们有\(X_1 \in \Gamma\).%TODO 为什么?
事实上,
对于\(\forall E \in \Gamma\),
由于\(X_1 \supseteq E\),
故从(1)式可知\(E \cap g\ImageOfSetUnderRelation{B - f\ImageOfSetUnderRelation{X_1}} = \emptyset\),
从而有\(X_1 \cap g\ImageOfSetUnderRelation{B - f\ImageOfSetUnderRelation{X_1}} = \emptyset\).
这说明\(X_1\)不光是\(A\)中的分离集,而且是\(\Gamma\)中对包含关系\(\subseteq\)的最大元.

现在令\(f\ImageOfSetUnderRelation{X_1}=Y_1,
B-Y_1=Y_2,
g\ImageOfSetUnderRelation{Y_2}=X_2\).
首先知道\(B=Y_1 \cup Y_2\).
其次,由于\(X_1 \cap X_2 = \emptyset\),
故又易得\(X_1 \cup X_2 = A\).
事实上,若不然,那么\(\exists a \in A\),
使得\(a \notin X_1 \cup X_2\).
现在取\(X_3 = X_1 \cup \{a\}\),
我们有\[
	Y_1 = f\ImageOfSetUnderRelation{X_1} \subseteq f\ImageOfSetUnderRelation{X_3},
	\qquad
	Y_2 \supseteq B - f\ImageOfSetUnderRelation{X_3},
\]
从而知\(X_2 \supseteq g\ImageOfSetUnderRelation{B - f\ImageOfSetUnderRelation{X_3}}\).
这就是说\[
	X_1 \cap g\ImageOfSetUnderRelation{B - f\ImageOfSetUnderRelation{X_3}} = \emptyset.
\]
由此可得\[
	X_3 \cap g\ImageOfSetUnderRelation{B - f\ImageOfSetUnderRelation{X_3}} = \emptyset.
\]
这与“\(X_1\)是\(\Gamma\)的最大元”相矛盾!
\end{proof}
\end{lemma}
\hyperref[theorem:基数.集合在映射下的分解]{映射分解定理}是由巴拿赫建立的.

\begin{theorem}\label{theorem:集合论.施罗德--伯恩斯坦定理}
%@see: 《Elements of Set Theory》 P147 Schroder-Bernstein Theorem(a)
%@see: 《点集拓扑讲义(第四版)》(熊金城) P33 定理1.7.8
%@see: 《实变函数论(第三版)》(周民强) P16 定理1.5(Cantor-Bernstein定理)
设\(A,B\)都是集合,则\[
	A \preceq B \land B \preceq A \implies A \approx B.
\]
\begin{proof}
根据定义可知,
存在单射\(f\colon A \to B\)和\(g\colon B \to A\).
根据\hyperref[theorem:基数.集合在映射下的分解]{映射分解定理}知\[
	A = X_1 \cup X_2, \qquad
	B = Y_1 \cup Y_2, \qquad
	f\ImageOfSetUnderRelation{X_1} = Y_1, \qquad
	f\ImageOfSetUnderRelation{Y_2} = X_2.
\]
因为\(f\)和\(g^{-1}\)是双射,
所以我们可以构造一个从\(A\)到\(B\)的双射:\[
	h(x) = \left\{ \begin{array}{cl}
		f(x), & x \in X_1, \\
		g^{-1}(x), & x \in X_2,
	\end{array} \right.
\]
这就说明\(A \approx B\).
\end{proof}
\end{theorem}
\cref{theorem:集合论.施罗德--伯恩斯坦定理}
常被称为“施罗德--伯恩斯坦定理”.
不过有时候也把它称为“康托--伯恩斯坦定理”.

最先是康托于1897年证明了该定理,
但是他的证明利用了一条等价于选择公理的原理.
后来,施罗德在他于1896年撰写的一篇摘要中发表了这条定理,
但是他给出的证明是有缺陷的.
菲利克斯·伯恩斯坦是第一个为这条定理给出完全令人满意的证明的人,
波莱尔把他的证明发表在自己于1898年出版的
《函数论讲义(Le\c{c}ons sur la th\'eorie des fonctions)》上.
但是这里对\cref{theorem:集合论.施罗德--伯恩斯坦定理} 的证明是由巴拿赫给出的.

利用\cref{theorem:集合论.施罗德--伯恩斯坦定理} 可以得到如下推论.
\begin{corollary}\label{theorem:集合论.施罗德--伯恩斯坦定理.推论1}
%@see: 《Elements of Set Theory》 P148 Examples 1.
设\(A,B,C\)是集合,
且\(A \subseteq B \subseteq C\).
若\(A \approx C\),
则\(B \approx C\).
\end{corollary}

\begin{example}\label{example:基数.闭区间与全体实数等势1}
%@see: 《Elements of Set Theory》 P148 Examples 2.
%@see: 《实变函数论(第三版)》(周民强) P17 例5
因为\((-1,1)\subseteq[-1,1]\subseteq\mathbb{R}\),
且\((-1,1)\approx\mathbb{R}\),
所以\([-1,1]\approx\mathbb{R}\).

因为\((0,1)\subseteq[0,1]\subseteq\mathbb{R}\),
且\((0,1)\approx\mathbb{R}\),
所以\([0,1]\approx\mathbb{R}\).
\end{example}
在\cref{example:基数.闭区间与全体实数等势1} 中,
我们如果要直接建立\([-1,1]\)与\(\mathbb{R}\)之间的一一对应,就会比较繁琐.
这是因为想用一个连续函数来表示所需的一一对应是不可能的,
在以后我们会学到,闭区间上的连续函数的值域仍是闭区间(\cref{theorem:极限.最值定理}).
不过,只要运用\cref{theorem:集合论.施罗德--伯恩斯坦定理.推论1}
以及\cref{example:基数.开区间与全体实数等势2} 的结论,
就可以间接证明\([-1,1]\)与\(\mathbb{R}\)等势.

\begin{proposition}
%@see: 《Real Analysis Modern Techniques and Their Applications Second Edition》 P7 0.9 Proposition
设\(A\)是集合,
则\[
	\card A < \card(\Powerset A).
\]
\end{proposition}

\begin{example}
%@see: 《Elements of Set Theory》 P148 Examples 4.
\(\mathbb{R}\approx2^\omega\),
\(\mathbb{R}\approx\Powerset\omega\).
\end{example}

\begin{example}
%@see: 《实变函数论(第三版)》(周民强) P17 思考题 1.
设\(A_1 \subseteq A_2,
B_1 \subseteq B_2,
A_1 \approx B_1,
A_2 \approx B_2\).
考察\((A_2 - A_2) \approx (B_2 - B_1)\)是否成立.
%TODO
\end{example}
\begin{example}
%@see: 《实变函数论(第三版)》(周民强) P17 思考题 2.
设\((A - B) \approx (B - A)\).
考察\(A \approx B\)是否成立.
%TODO
\end{example}
\begin{example}
%@see: 《实变函数论(第三版)》(周民强) P18 思考题 3.
设\(A \subseteq B,
A \approx (A \cup C)\).
考察\(B \approx (B \cup C)\)是否成立.
%TODO
\end{example}

\subsection{可数集,不可数集}
\begin{definition}
%@see: 《Real Analysis Modern Techniques and Their Applications Second Edition》 P7
%@see: 《点集拓扑讲义(第四版)》(熊金城) P29 定义1.7.1
设\(A\)是集合.
若\[
	% 存在一个从集合\(A\)到正整数集的单射
	\card A \leq \card\mathbb{N},
\]
则称“\(A\)是\DefineConcept{可数的}(countable)”
或“\(A\)是\DefineConcept{可列的}(denumerable)”.
反之,若\[
	\card A > \card\mathbb{N},
\]
则称“\(A\)是\DefineConcept{不可数的}(uncountable)”
或“\(A\)是\DefineConcept{不可列的}(non-denumerable)”.
% 虽然在很多教材上都定义\(\aleph_0 \defeq \card\mathbb{N}\),但是这里不采取该记法.
\end{definition}

\begin{theorem}
%@see: 《实变函数论(第三版)》(周民强) P18 定理1.6
任一无限集必定包含一个可列子集.
\begin{proof}
设\(E\)是一个无限集,
任取\(a_1 \in E\),
再任取\(a_2 \in E-\{a_1\}\),
以此类推,直至选出\(\{\AutoTuple{a}{n}\}\).
因为\(E\)是无限的,所以\(E-\{\AutoTuple{a}{n}\}\neq\emptyset\).
于是我们还可以继续任取\(a_{n+1} \in E-\{\AutoTuple{a}{n}\}\).
这样,我们就得到一个集合\(\{\AutoTuple{a}{n+1},\dotsc\}\),
它既是\(E\)的一个子集,也是一个可列集.
\end{proof}
\end{theorem}
\begin{remark}
这个定理说明:在众多的无限集中,最小的无限集的基数是\(\card\mathbb{N}\).
\end{remark}

\begin{proposition}
%@see: 《Real Analysis Modern Techniques and Their Applications Second Edition》 P7
有限集是可数的.
任一有限集\(A\)的基数\(\card A\)等于它的元素个数.
\end{proposition}

\begin{definition}
%@see: 《Real Analysis Modern Techniques and Their Applications Second Edition》 P7
若集合\(A\)既是无限的又是可数的,
则称“\(A\)是\DefineConcept{可数无限的}(countably infinite)”.
\end{definition}

\begin{proposition}
%@see: 《点集拓扑讲义(第四版)》(熊金城) P30 定理1.7.1
可数集的任何子集都是可数集.
\end{proposition}

\begin{proposition}
%@see: 《点集拓扑讲义(第四版)》(熊金城) P30 定理1.7.2
设\(X\)和\(Y\)都是集合,映射\(f\colon X\to Y\).
如果\(X\)是可数集,
则\(f(X)\)也是可数集.
\end{proposition}

\begin{proposition}
%@see: 《点集拓扑讲义(第四版)》(熊金城) P30 定理1.7.3
集合\(X\)是可数集,当且仅当存在从自然数集\(\omega\)到\(X\)的一个满射.
\end{proposition}

\begin{proposition}
%@see: 《Real Analysis Modern Techniques and Their Applications Second Edition》 P8 0.10 Proposition
%@see: 《点集拓扑讲义(第四版)》(熊金城) P30 定理1.7.4
如果集合\(A,B\)都是可数的,
那么\(A \times B\)也是可数的.
\end{proposition}

\begin{proposition}
%@see: 《Real Analysis Modern Techniques and Their Applications Second Edition》 P8 0.10 Proposition
%@see: 《点集拓扑讲义(第四版)》(熊金城) P31 定理1.7.5
如果\begin{itemize}
	\item 集合\(A\)是可数的;
	\item 对于每个\(a \in A\),\(X_a\)都是可数的,
\end{itemize}
那么\(\bigcup_{a \in A} X_a\)是可数的.
\end{proposition}

\begin{proposition}
%@see: 《Real Analysis Modern Techniques and Their Applications Second Edition》 P8 0.10 Proposition
如果集合\(A\)是可数无限的,
那么\(\card A = \card{\mathbb{N}}\).
\end{proposition}

\begin{corollary}
%@see: 《Real Analysis Modern Techniques and Their Applications Second Edition》 P8 0.11 Corollary
\(\mathbb{Z}\)是可数的.
\end{corollary}

\begin{corollary}
%@see: 《Real Analysis Modern Techniques and Their Applications Second Edition》 P8 0.11 Corollary
\(\mathbb{Q}\)是可数的.
\end{corollary}

\begin{proposition}
%@see: 《Real Analysis Modern Techniques and Their Applications Second Edition》 P8 0.12 Proposition
\(\card(\Powerset\mathbb{N}) = \card\mathbb{R}\).
\begin{proof}
定义映射\(f\colon \Powerset\mathbb{N}\to\mathbb{R}\),
使得\[
	f(A) = \left\{ \begin{array}{cl}
		\sum_{n \in A} 2^{-n}, & \text{集合$\mathbb{N}-A$是无限的}, \\
		1 + \sum_{n \in A} 2^{-n}, & \text{集合$\mathbb{N}-A$是有限的}.
	\end{array} \right.
\]
易见\(f\)是单射.

再定义映射\(g\colon \Powerset\mathbb{Z}\to\mathbb{R}\),
使得\[
	g(A) = \left\{ \begin{array}{cl}
		\log\left(\sum_{n \in A} 2^{-n}\right), & \text{集合$A$有下界}, \\
		0, & \text{集合$A$没有下界}.
	\end{array} \right.
\]
易见\(g\)是满射.

考虑到\(\card(\Powerset\mathbb{Z}) = \card(\Powerset\mathbb{N})\),
利用\hyperref[theorem:集合论.施罗德--伯恩斯坦定理]{施罗德--伯恩斯坦定理}便得结论.
\end{proof}
\end{proposition}

\begin{corollary}
%@see: 《Real Analysis Modern Techniques and Their Applications Second Edition》 P8 0.13 Corollary
设\(A\)是集合.
如果\(\card A \geq \card\mathbb{R}\),
那么\(A\)是不可数的.
\end{corollary}

\begin{theorem}
%@see: 《点集拓扑讲义(第四版)》(熊金城) P34 定理1.7.9
\(\mathbb{R}\)是不可数的.
\end{theorem}

\begin{proposition}
%@see: 《Real Analysis Modern Techniques and Their Applications Second Edition》 P9 0.14 Proposition
设\(A,B\)都是集合.
若\(\card A \leq \card\mathbb{R}\)且\(\card B \leq \card\mathbb{R}\),
则\(\card(A \times B) \leq \card\mathbb{R}\).
\end{proposition}

\begin{proposition}
%@see: 《Real Analysis Modern Techniques and Their Applications Second Edition》 P9 0.14 Proposition
如果\begin{itemize}
	\item \(\card A \leq \card\mathbb{R}\);
	\item \((\forall a \in A)
	[\card X_a \leq \card\mathbb{R}]\),
\end{itemize}
那么\[
	\card\left( \bigcup_{a \in A} X_a \right) \leq \card\mathbb{R}.
\]
\end{proposition}

证明集合等势的方法:\begin{enumerate}
	\item 利用\cref{definition:基数.等势的定义},直接证明双射的存在性.
	\item 根据\cref{theorem:集合论.等势相当于等价关系},运用中间集合过渡.
	\item 采用\hyperref[theorem:基数.集合在映射下的分解]{分解}、合并等思想.
	\item 参照\hyperref[theorem:集合论.施罗德--伯恩斯坦定理]{施罗德--伯恩斯坦定理},证明两个集合互相主导.
\end{enumerate}

\begin{example}
%@see: 《实变函数论(第三版)》(周民强) P18 例6
设\(A\)是有限集,\(B\)是可列集,则\(A \cup B\)是可列集.
\begin{proof}
不妨设\(A = \Set{\AutoTuple{a}{n}},
B = \Set{b_1,\dotsc}\).
若\(A \cap B = \emptyset\),
则由\begin{equation*}
	A \cup B = \Set{\AutoTuple{a}{n},b_1,\dotsc}
\end{equation*}
可知\(A \cup B\)是可列集;
若\(A \cap B = \emptyset\),
则由于\(A \cup B = (A-B) \cup B\),
易知\(A \cup B\)仍是可列集.
\end{proof}
\end{example}
