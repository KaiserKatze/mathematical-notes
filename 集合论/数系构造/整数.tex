\section{整数}
我们现在把自然数集扩张为整数集.
\subsection{整数集}
设\(a,b\in\omega\).
我们把数对\(\opair{a,b}\)称为“\(a\)与\(b\)的\DefineConcept{差}(difference)”,
记作\(a-b\)\footnote{读作“\(a\)减\(b\)”.},即\begin{equation*}
	a-b = \opair{a,b}.
\end{equation*}

\begin{definition}\label{definition:集合论.自然数的差的等价关系}
%@see: 《Elements of Set Theory》 P91 Definition
我们定义\(\omega\times\omega\)上的关系\(\sim\)如下:
\begin{equation*}
	(\forall a,b,c,d \in \omega)
	[
		\opair{a,b} \sim \opair{c,d}
		\defiff
		a+d=b+c
	].
\end{equation*}
\end{definition}

容易看出,关系\(\sim\)的定义域和值域都是自然数对.
除了通过命题公式给出\(\sim\)的定义以外,
我们还可以用更直接的形式给出它的定义:
\begin{equation*}
	\sim\ \defeq \Set*{
		\opair{\opair{a,b},\opair{c,d}}
		\given
		a+d=b+c \land a,b,c,d\in\omega
	}.
\end{equation*}

\begin{theorem}
%@see: 《Elements of Set Theory》 P91 Theorem 5ZA
关系\(\sim\)是\(\omega\times\omega\)上的等价关系.
\begin{proof}
容易看出,关系\(\sim\)具有自反性和对称性.
现在来证\(\sim\)具有传递性.
设\begin{equation*}
	\opair{m,n}\sim\opair{p,q}, \qquad
	\opair{p,q}\sim\opair{r,s}.
\end{equation*}
由定义有\begin{equation*}
	m+q=n+p, \qquad
	p+s=q+r;
\end{equation*}
相加得\begin{equation*}
	m+q+p+s=n+p+q+r;
\end{equation*}
再利用\hyperref[theorem:集合论.自然数的消去律]{消去律}便得\begin{equation*}
	m+s=n+r;
\end{equation*}
于是\(\opair{m,n}\sim\opair{r,s}\),关系\(\sim\)具有传递性.
综上所述,关系\(\sim\)是一个等价关系.
\end{proof}
\end{theorem}

既然关系\(\sim\)是一个等价关系,
那么对于任意给定的自然数对\(\opair{m,0}\),
我们可以取得它在关系\(\sim\)下唯一的等价类\begin{equation*}
	\EquivalenceClassB{\opair{m,0}}[\sim]
	= \Set{
		\opair{m,0},
		\opair{m+1,1},
		\opair{m+2,2},
		\dotsc
	};
\end{equation*}
类似地,
对于任意给定的自然数对\(\opair{0,n}\),
我们也可以取得它在关系\(\sim\)下唯一的等价类\begin{equation*}
	\EquivalenceClassB{\opair{0,n}}[\sim]
	= \Set{
		\opair{0,n},
		\opair{1,n+1},
		\opair{2,n+2},
		\dotsc
	}.
\end{equation*}
像这样,我们把任意两个自然数\(m\)和\(n\)的差\(\opair{m,n}\)%
在关系\(\sim\)下的等价类叫做一个\DefineConcept{整数}(integer).

于是我们可以构造出“整数集”.
\begin{definition}
%@see: 《Elements of Set Theory》 P92 Definition
称“\(\omega\times\omega\)对\(\sim\)的商集”为%
\DefineConcept{整数集}(the set of integers),
记作\(\mathbb{Z}\),即\begin{equation*}
	\mathbb{Z} \defeq \IntegerQuotient.
\end{equation*}
\end{definition}

例如,整数\(2\)是等价类\begin{equation*}
	\EquivalenceClassB{\opair{2,0}}[\sim]
	= \Set{\opair{2,0},\opair{3,1},\opair{4,2},\dotsc};
\end{equation*}
而整数\((-3)\)是等价类\begin{equation*}
	\EquivalenceClassB{\opair{0,3}}[\sim]
	= \Set{\opair{0,3},\opair{1,4},\opair{2,5},\dotsc}.
\end{equation*}

\begin{figure}[htb]
	%@see: 《Elements of Set Theory》 P92 Fig. 20. An integer is a line in \(\omega\times\omega\)
	\centering
	\begin{tikzpicture}[scale=.5]
		\draw[help lines, color=gray!30, dashed] (0,0)grid(9,9);
		\begin{scope}[>=Stealth,->]
			\draw(0,0)--(0,10);
			\draw(0,0)--(10,0);
		\end{scope}
		\foreach \i in {0,...,9} {
			\foreach \j in {0,...,9} {
				\fill (\i,\j)circle(2pt);
			}
		}
		\draw(0,4)--(5,9)node[above right]{\(-4\)};
		\draw(0,0)--(9,9)node[above right]{\(0\)};
		\draw(3,0)--(9,6)node[above right]{\(3\)};
	\end{tikzpicture}
	\caption{整数是\(\omega\times\omega\)上的“直线”}
\end{figure}

\subsection{整数集上的加法运算}
类比自然数,我们希望在整数集上也能定义加法和乘法.

利用我们从初等代数学到的知识,
容易验证下式是正确的:\begin{equation*}
	(m-n)+(p-q) = (m+p)-(n+q).
\end{equation*}
于是我们可以定义整数集上的加法运算:
\begin{equation}\label{equation:集合论.整数集上的加法运算}
	\EquivalenceClassB{\opair{m,n}}[\sim] + \EquivalenceClassB{\opair{p,q}}[\sim]
	\defeq
	\EquivalenceClassB{\opair{m+p,n+q}}[\sim].
\end{equation}
虽然已经有了整数加法的定义了,
但我们还不能确信它是良定的.
这和我们在讨论\cref{theorem:集合论.与等价关系兼容的映射的性质} 时遇到的情况是一样的.
我们想要为一对等价类给出加法运算结果的取值,就要分成下面三个步骤:
\begin{enumerate}
	\item 从两个等价类中各自取出一个代表\(\opair{m,n}\)和\(\opair{p,q}\);
	\item 在这两个代表上作运算,即\begin{equation*}
		\opair{m,n} + \opair{p,q}
		= \opair{m+p,n+q};
	\end{equation*}
	\item 构造运算结果的等价类\(\EquivalenceClassB{\opair{m+p,n+q}}[\sim]\).
\end{enumerate}

\begin{lemma}\label{theorem:集合论.整数集上的加法运算是良定的}
%@see: 《Elements of Set Theory》 P93 Lemma 5ZB
如果\(\opair{m,n}\sim\opair{m',n'}\)且\(\opair{p,q}\sim\opair{p',q'}\),
那么\begin{equation*}
	\opair{m+p,n+q}\sim\opair{m'+p',n'+q'}.
\end{equation*}
\begin{proof}
根据\cref{definition:集合论.自然数的差的等价关系},
\begin{equation*}
	m+n'=m'+n, \qquad
	p+q'=p'+q;
\end{equation*}
将上述两式相加得\begin{equation*}
	m+p+n'+q'=m'+p'+n+q;
\end{equation*}
于是\(\opair{m+p,n+q}\sim\opair{m'+p',n'+q'}\).
\end{proof}
\end{lemma}
\cref{theorem:集合论.整数集上的加法运算是良定的} 说明前面定义的%
\hyperref[equation:集合论.整数集上的加法运算]{整数集上的加法运算}是正当的.
换句话说,由于映射\begin{equation*}
	F\colon ((\omega\times\omega)\times(\omega\times\omega))\to(\omega\times\omega),
	\opair{\opair{m,n},\opair{p,q}} \mapsto \opair{m+p,n+q}
\end{equation*}和关系\(\sim\)兼容,
那么根据\cref{theorem:集合论.与等价关系兼容的映射的性质},
在商集\(\IntegerQuotient\)上,唯一存在映射\(\hat{F}\)(这就是整数集上的加法运算),满足\begin{equation*}
	\hat{F}(\EquivalenceClassB{\opair{m,n}}[\sim],\EquivalenceClassB{\opair{p,q}}[\sim]) = \EquivalenceClassB{\opair{m+p,n+q}}[\sim].
\end{equation*}

\begin{theorem}\label{theorem:集合论.整数加法的运算法则}
%@see: 《Elements of Set Theory》 P94 Theorem 5ZC
%@see: 《Elements of Set Theory》 P95 Theorem 5ZD
以下命题恒成立:
\begin{enumerate}
	\item 加法交换律
	\begin{equation}\label{equation:集合论.整数加法交换律}
		(\forall a,b\in\mathbb{Z})[a+b=b+a].
	\end{equation}
	\item 加法结合律
	\begin{equation}\label{equation:集合论.整数加法结合律}
		(\forall a,b,c\in\mathbb{Z})[(a+b)+c=a+(b+c)].
	\end{equation}
	\item 任意整数加上整数\(0\defeq\EquivalenceClassB{\opair{0,0}}[\sim]\)不变
	\begin{equation}\label{equation:集合论.任意整数加上零不变}
		(\forall a\in\mathbb{Z})[a+0=a].
	\end{equation}
	\item 加法可逆
	\begin{equation}\label{equation:集合论.整数加法可逆}
		(\forall a\in\mathbb{Z})(\exists! b\in\mathbb{Z})[a+b=0].
	\end{equation}
\end{enumerate}
\begin{proof}
首先证明 \labelcref{equation:集合论.整数加法交换律,equation:集合论.整数加法结合律} 这两个命题.
设\begin{equation*}
	a = \EquivalenceClassB{\opair{m,n}}[\sim], \qquad
	b = \EquivalenceClassB{\opair{p,q}}[\sim], \qquad
	c = \EquivalenceClassB{\opair{u,v}}[\sim],
\end{equation*}
其中\(m,n,p,q,u,v\)是自然数.
那么\begin{align*}
	a+b &= \EquivalenceClassB{\opair{m,n}}[\sim]+\EquivalenceClassB{\opair{p,q}}[\sim] \\
	&= \EquivalenceClassB{\opair{m+p,n+q}}[\sim]	\tag{\hyperref[equation:集合论.整数集上的加法运算]{整数加法的定义}} \\
	&= \EquivalenceClassB{\opair{p+m,q+n}}[\sim]	\tag{\hyperref[equation:集合论.自然数加法交换律]{自然数加法交换律}} \\
	&= \EquivalenceClassB{\opair{p,q}}[\sim]+\EquivalenceClassB{\opair{m,n}}[\sim] \\
	&= b+a; \\
	(a+b)+c &= \EquivalenceClassB{\opair{m+p,n+q}}[\sim]+\EquivalenceClassB{\opair{u,v}}[\sim] \\
	&= \EquivalenceClassB{\opair{(m+p)+u,(n+q)+v}}[\sim] \\
	&= \EquivalenceClassB{\opair{m+(p+u),n+(q+v)}}[\sim]	\tag{\hyperref[equation:集合论.自然数加法结合律]{自然数加法结合律}} \\
	&= \EquivalenceClassB{\opair{m,n}}[\sim]+\EquivalenceClassB{\opair{p+u,q+v}}[\sim] \\
	&= a+(b+c);
\end{align*}
因此整数加法满足交换律、结合律.

命题 \labelcref{equation:集合论.任意整数加上零不变} 显然成立,
这是因为对于任意给定的整数\(a = \EquivalenceClassB{\opair{m,n}}[\sim]\),总有\begin{equation*}
	a+0
	= \EquivalenceClassB{\opair{m,n}}[\sim]+\EquivalenceClassB{\opair{0,0}}[\sim]
	= \EquivalenceClassB{\opair{m+0,n+0}}[\sim]
	= \EquivalenceClassB{\opair{m,n}}[\sim]
	= a.
\end{equation*}

最后,我们来证明命题 \labelcref{equation:集合论.整数加法可逆}.
显然,对于任意取定整数\(a = \EquivalenceClassB{\opair{m,n}}[\sim]\),
整数\(b = \EquivalenceClassB{\opair{n,m}}[\sim]\)总是满足\begin{equation*}
	a+b
	= \EquivalenceClassB{\opair{m+n,n+m}}[\sim]
	= \EquivalenceClassB{\opair{0,0}}[\sim]
	= 0.
\end{equation*}
由此,我们不仅证明命题 \labelcref{equation:集合论.整数加法可逆} 中,
整数\(b\)的存在性,还给出了它的表达式.
我们再假设\(b,c\)这两个整数分别满足\begin{equation*}
	a+b=0, \qquad
	a+c=0;
\end{equation*}
那么\begin{equation*}
	b=b+(a+c)=(b+a)+c=c.
\end{equation*}
于是我们证得整数\(b\)的唯一性.
\end{proof}
\end{theorem}

从上面的证明过程我们得知,
对于任意给定的一个整数\(a\),
满足\(a+b=0\)的整数\(b\)存在且唯一.
于是,我们可以把\(b\)叫做“\(a\)的\DefineConcept{逆}(inverse)”,
记作\((-a)\).
那么有\begin{equation}
	-\EquivalenceClassB{\opair{m,n}}[\sim]=\EquivalenceClassB{\opair{n,m}}[\sim].
\end{equation}
整数的逆为我们提供了\DefineConcept{减法运算}(subtraction operation):
\begin{equation}
	b-a \defeq b+(-a).
\end{equation}

\cref{theorem:集合论.整数加法的运算法则}
说明整数集\(\mathbb{Z}\)配上加法运算就成为一个阿贝尔群!
在以后学到抽象代数的时候,我们会明白这意味着什么.

\subsection{整数集上的乘法运算}
我们遵循同样的思路,像在研究整数集上的加法运算那样,借助初等代数的知识,注意到\begin{equation*}
	(m-n)\cdot(p-q)=(mp+nq)-(mq+np).
\end{equation*}
于是我们可以定义整数集上的乘法运算:
\begin{equation}\label{equation:集合论.整数集上的乘法运算}
	\EquivalenceClassB{\opair{m,n}}[\sim]\cdot\EquivalenceClassB{\opair{p,q}}[\sim]
	\defeq \EquivalenceClassB{\opair{mp+nq,mq+np}}[\sim].
\end{equation}
同样地,我们还是要验证整数乘法的定义在等价类上是不是良定的.
也就是要验证映射\begin{equation*}
	G\colon ((\omega\times\omega)\times(\omega\times\omega))\to(\omega\times\omega),
	\opair{\opair{m,n},\opair{p,q}} \mapsto \opair{mp+nq,mq+np}
\end{equation*}和关系\(\sim\)兼容.
为此,我们有如下引理.
\begin{lemma}\label{theorem:集合论.整数集上的乘法运算是良定的}
%@see: 《Elements of Set Theory》 P96 Lemma 5ZE
如果\(\opair{m,n}\sim\opair{m',n'}\)且\(\opair{p,q}\sim\opair{p',q'}\),
那么\begin{equation*}
	\opair{mp+nq,mq+np}\sim\opair{m'p'+n'q',m'q'+n'p'}.
\end{equation*}
\begin{proof}
根据\cref{definition:集合论.自然数的差的等价关系},
\begin{equation*}
	m+n'=m'+n,
	\eqno(1)
\end{equation*}\begin{equation*}
	p+q'=p'+q;
	\eqno(2)
\end{equation*}
将(1)式乘以\(p\)倍,得\begin{equation*}
	mp+n'p=m'p+np;
	\eqno(3)
\end{equation*}
将(1)式乘以\(q\)倍,得\begin{equation*}
	m'q+nq=mq+n'q;
	\eqno(4)
\end{equation*}
将(2)式乘以\(m'\)倍,得\begin{equation*}
	m'p+m'q'=m'p'+m'q;
	\eqno(5)
\end{equation*}
将(2)式乘以\(n'\)倍,得\begin{equation*}
	n'p'+n'q=n'p+n'q';
	\eqno(6)
\end{equation*}
将(3)(4)(5)(6)这四个式子相加,
再利用\hyperref[theorem:集合论.自然数的消去律]{消去律}化简,便得\begin{equation*}
	mp+nq+m'q'+n'p'
	=np+mq+m'p'+n'q';
\end{equation*}
因此,\(\opair{mp+nq,mq+np}\sim\opair{m'p'+n'q',m'q'+n'p'}\).
\end{proof}
\end{lemma}

\begin{theorem}\label{theorem:集合论.整数乘法的运算法则1}
%@see: 《Elements of Set Theory》 P96 Theorem 5ZF
以下命题恒成立:
\begin{enumerate}
	\item 乘法交换律
	\begin{equation}\label{equation:集合论.整数乘法交换律}
		\forall a,b\in\mathbb{Z} \bigl(
			a \cdot b = b \cdot a
		\bigr).
	\end{equation}
	\item 乘法结合律
	\begin{equation}\label{equation:集合论.整数乘法结合律}
		\forall a,b,c\in\mathbb{Z} \bigl(
			(a \cdot b) \cdot c = a \cdot (b \cdot c)
		\bigr).
	\end{equation}
	\item 乘法分配律
	\begin{equation}\label{equation:集合论.整数乘法分配律}
		\forall a,b,c\in\mathbb{Z} \bigl(
			a \cdot (b + c) = (a \cdot b) + (a \cdot c)
		\bigr).
	\end{equation}
\end{enumerate}
\begin{proof}
设\(a=\EquivalenceClassB{\opair{m,n}}[\sim]\),\(b=\EquivalenceClassB{\opair{p,q}}[\sim]\),其中\(m,n,p,q\in\omega\).
那么根据\cref{equation:集合论.整数集上的乘法运算},\begin{equation*}
	a \cdot b = \EquivalenceClassB{\opair{mp+nq,mq+np}}[\sim],
\end{equation*}\begin{equation*}
	b \cdot a = \EquivalenceClassB{\opair{pm+qn,pn+qm}}[\sim].
\end{equation*}
根据自然数的\hyperref[equation:集合论.自然数加法交换律]{加法交换律}%
和\hyperref[equation:集合论.自然数乘法交换律]{乘法交换律},
有\(mp+nq=pm+qn\)和\(mq+np=pn+qm\),
于是\(a \cdot b = b \cdot a\)成立.

又设\(c=\EquivalenceClassB{\opair{r,s}}[\sim]\),其中\(r,s\in\omega\).
那么\begin{equation*}
	(a \cdot b) \cdot c
	= \EquivalenceClassB{\opair{(mp+nq)r+(mq+np)s,(mp+nq)s+(mq+np)r}}[\sim],
\end{equation*}\begin{equation*}
	a \cdot (b \cdot c)
	= \EquivalenceClassB{\opair{m(pr+qs)+n(ps+qr),m(ps+qr)+n(pr+qs)}}[\sim].
\end{equation*}
只要像上面一样,利用自然数的运算法则,
易证\((a \cdot b) \cdot c = a \cdot (b \cdot c)\).

最后,我们将\(a \cdot (b+c)\)展开,得\begin{equation*}
	\EquivalenceClassB{\opair{m(p+r)+n(q+s),m(q+s)+n(p+r)}}[\sim];
\end{equation*}
将\((a \cdot b) + (a \cdot c)\)展开,得\begin{equation*}
	\EquivalenceClassB{\opair{mp+nq+mr+ns,mq+np+ms+nr}}[\sim].
\end{equation*}
易证\(a \cdot (b + c) = (a \cdot b) + (a \cdot c)\)成立.
\end{proof}
\end{theorem}

\begin{theorem}\label{theorem:集合论.整数乘法的运算法则2}
%@see: 《Elements of Set Theory》 P97 Theorem 5ZG
以下命题恒成立:\begin{enumerate}
	\item 任意整数乘上整数\(1\defeq\EquivalenceClassB{\opair{1,0}}[\sim]\)不变,
	即\begin{equation}\label{equation:集合论.任意整数乘上一不变}
		(\forall a\in\mathbb{Z})
		[a \cdot 1 = a].
	\end{equation}
	\item 整数\(1\)与整数\(0\)不相等,即\(1\neq0\).
	\item 如果\(a \cdot b = 0\),那么\(a = 0 \lor b = 0\).
\end{enumerate}
\begin{proof}
设\(a=\EquivalenceClassB{\opair{m,n}}[\sim]\),那么
\begin{align*}
	a \cdot 1
	&= \EquivalenceClassB{\opair{m,n}}[\sim]\cdot\EquivalenceClassB{\opair{1,0}}[\sim] \\
	&= \EquivalenceClassB{\opair{m\cdot1+n\cdot0,m\cdot0+n\cdot1}}[\sim] \\
	&= \EquivalenceClassB{\opair{m,n}}[\sim] = a.
\end{align*}
于是命题 \labelcref{equation:集合论.任意整数乘上一不变} 成立.

要证整数\(1\)与整数\(0\)不相等,
即证\(\EquivalenceClassB{\opair{1,0}}[\sim]\neq\EquivalenceClassB{\opair{0,0}}[\sim]\),
那么只需证\begin{equation*}
	\opair{0,0}\not\sim(1,0);
\end{equation*}
即证\(0+0\neq0+1\)或\(0\neq1\),显然成立.

接下来,若要证\(a \cdot b = 0 \implies a = 0 \lor b = 0\),
可以用反证法,
假设\(a,b\)都是非零整数,只需证\(a \cdot b \neq 0\)制造矛盾.
设\(a=\EquivalenceClassB{\opair{m,n}}[\sim]\),\(b=\EquivalenceClassB{\opair{p,q}}[\sim]\),
其中\(m,n,p,q\)都是自然数.
那么\begin{equation*}
	a \cdot b = \EquivalenceClassB{\opair{mp+nq,mq+np}}[\sim].
\end{equation*}
既然\(a\neq0=\EquivalenceClassB{\opair{0,0}}[\sim]\),
那么\(m \neq n\),即有\begin{equation*}
	m < n \lor n < m;
	\eqno(1)
\end{equation*}
同理可得\begin{equation*}
	p < q \lor q < p.
	\eqno(2)
\end{equation*}
从(1)(2)两式可以得到\(m,n,p,q\)取值的四种不同情况,不过我们总有\begin{equation*}
	mp+nq \neq mq+np.
\end{equation*}
因此\(a \cdot b \neq 0\).
\end{proof}
\end{theorem}

我们可以说\(\opair{\mathbb{Z},+,\cdot}\)构成了%
\DefineConcept{整环}(integral domain),
这是因为\begin{enumerate}
	\item \(\opair{\mathbb{Z},+}\)构成阿贝尔群%
	(\cref{theorem:集合论.整数加法的运算法则}).
	\item 整数乘法遵守交换律和结合律,以及对加法的分配律%
	(\cref{theorem:集合论.整数乘法的运算法则1}).
	\item 整数\(1\)是乘法的单位元,且整数集中不存在零因子%
	(\cref{theorem:集合论.整数乘法的运算法则2}).
\end{enumerate}

\subsection{整数的序}
现在我们来为整数构造一个序.
利用我们从初等代数学到的知识,
容易验证下式是正确的:\begin{equation*}
	m-n<p-q \iff m+q<p+n.
\end{equation*}
于是我们可以定义整数集上的序:
\begin{equation}\label{equation:集合论.整数集上的序}
	\EquivalenceClassB{\opair{m,n}}[\sim] < \EquivalenceClassB{\opair{p,q}}[\sim]
	\iff
	m+q<p+n.
\end{equation}
接下来我们想要确认这里定义的序是否不是良定的关系.

\begin{lemma}\label{theorem:集合论.整数集上的序是良定的}
%@see: 《Elements of Set Theory》 P98 Theorem 5ZH
如果\(\opair{m,n}\sim\opair{m',n'}\)且\(\opair{p,q}\sim\opair{p',q'}\),
那么\begin{equation*}
	m+q<p+n \iff m'+q'<p'+n'.
\end{equation*}
\begin{proof}
因为\(\opair{m,n}\sim\opair{m',n'}\)且\(\opair{p,q}\sim\opair{p',q'}\),所以\begin{equation*}
	m+n'=m'+n, \qquad
	p+q'=p'+q.
\end{equation*}
于是\begin{align*}
	m+q<p+n
	&\iff
	m+q+n'+q'<p+n+n'+q'
		\tag{\cref{theorem:集合论.自然数的加法与乘法的保序性}} \\
	&\iff
	m'+n+q+q'<p'+q+n+n'
		\tag{代入上述两个等式} \\
	&\iff
	m'+q'<p'+n'
		\tag{\cref{theorem:集合论.自然数的加法与乘法的保序性}}.
\end{align*}
\end{proof}
\end{lemma}

\begin{theorem}\label{theorem:集合论.整数集上的序是线性序}
%@see: 《Elements of Set Theory》 P98 Theorem 5ZI
关系\(<\)是整数集上的线性序.
\begin{proof}
要证关系\(<\)是整数集上的线性序,
即证\(<\)具有传递性,且满足三一律.

先证\(<\)的传递性.
考虑整数\(a=\EquivalenceClassB{\opair{m,n}}[\sim]\),\(b=\EquivalenceClassB{\opair{p,q}}[\sim]\),\(c=\EquivalenceClassB{\opair{r,s}}[\sim]\),
那么\begin{align*}
	a<b \land b<c
	&\implies
	m+q<p+n \land p+s<r+q \\
	&\implies
	m+q+s<p+n+s \land p+s+n<r+q+n \\
	&\implies
	m+q+s<r+q+n \\
	&\implies
	m+s<r+n
		\tag{\cref{theorem:集合论.自然数的加法与乘法的保序性}} \\
	&\implies
	a < c.
\end{align*}

再证\(<\)服从三一律.
要说\begin{equation*}
	a < b, \qquad
	a = b, \qquad
	b < a
\end{equation*}这三个命题中有且仅有一个成立,
等于说\begin{equation*}
	m+q<p+n, \qquad
	m+q=p+n, \qquad
	p+n<m+q
\end{equation*}中有且仅有一个成立,
而这根据\cref{theorem:集合论.自然数集的三一律} 就能得到.
\end{proof}
\end{theorem}

\begin{definition}
设\(a\)是整数.
如果\begin{equation*}
	0 < a,
\end{equation*}
那么称“\(a\)是\DefineConcept{正的}(\(a\) is \emph{positive})”,
或称“\(a\)是\DefineConcept{正整数}(\(a\) is a \emph{positive integer})”.
反之,如果\begin{equation*}
	a < 0,
\end{equation*}
那么称“\(a\)是\DefineConcept{负的}(\(a\) is \emph{negative})”,
或称“\(a\)是\DefineConcept{负整数}(\(a\) is a \emph{negative integer})”.
\end{definition}

任给一个整数\(b\),
容易验证\begin{equation*}
	b < 0 \iff 0 < -b.
\end{equation*}
于是我们又有如下的三一律:
\begin{equation*}
	\text{\(b\)是正的}, \qquad
	\text{\(b\)是零}, \qquad
	\text{\(-b\)是正的}
\end{equation*}这三个命题有且仅有一个成立.

\begin{theorem}\label{theorem:集合论.整数的加法与乘法的保序性}
%@see: 《Elements of Set Theory》 P99 Theorem 5ZJ
设\(a,b,c\in\mathbb{Z}\).
那么\begin{gather}
	a<b \iff a+c<b+c,
	\label{equation:集合论.整数的加法的保序性} \\
	0<c \implies \bigl(
		a<b \iff a \cdot c < b \cdot c
	\bigr).
	\label{equation:集合论.整数的乘法的保序性}
\end{gather}
\begin{proof}
假设\begin{equation*}
	a=\EquivalenceClassB{\opair{m,n}}[\sim], \qquad
	b=\EquivalenceClassB{\opair{p,q}}[\sim], \qquad
	c=\EquivalenceClassB{\opair{r,s}}[\sim].
\end{equation*}
那么\cref{equation:集合论.整数的加法的保序性} 相当于\begin{equation*}
	m+q<p+n \iff m+r+q+s<p+r+n+s.
\end{equation*}
而这根据\cref{theorem:集合论.自然数的加法与乘法的保序性} 立即可得.

要证\cref{equation:集合论.整数的乘法的保序性},
根据\cref{theorem:集合论.自然数的加法与乘法的保序性},
只需证\begin{equation*}
	0<c \land a<b \implies a \cdot c < b \cdot c,
\end{equation*}
即证\begin{equation*}
	s<r \land m+q<p+n \implies mr+ns+ps+qr<pr+qs+ms+nr.
\end{equation*}
如果令\(k=m+q, l=p+n\),那么上式变为\begin{equation*}
	s<r \land k<l \implies kr+ls<ks+lr.
\end{equation*}
%TODO Exercise 25 of Chapter 4.
\end{proof}
\end{theorem}

\begin{corollary}\label{theorem:集合论.整数的消去律}
%@see: 《Elements of Set Theory》 P100 Corollary 5ZK
设\(a,b,c\in\mathbb{Z}\).
那么\begin{gather*}
	a+c=b+c \implies a=b, \\
	a \cdot c = b \cdot c \land c \neq 0 \implies a=b.
\end{gather*}
\end{corollary}
\cref{theorem:集合论.整数的消去律} 又称为“整数的消去律”.

最后,我们还可以给出减法:%@see: P101.
对于任意自然数\(m,n\),有\begin{equation*}
	\EquivalenceClassB{\opair{m,n}}[\sim]=E(m)-E(n).
\end{equation*}
