\section{关系}
\subsection{无序对,有序对}
\begin{definition}
由两个元素\(x_1\)和\(x_2\)组成的集合\begin{equation*}
\Set{x_1, x_2}
\end{equation*}称作\DefineConcept{无序对}(unordered pair).
\end{definition}

\begin{example}
%@see: 《Elements of Set Theory》 P38 Exercise 1.
定义:\begin{equation*}
	\opair{x,y,z}^* = \{\{x\},\{x,y\},\{x,y,z\}\}.
\end{equation*}
找出元素\(u,v,w,x,y,z\)使\begin{equation*}
	\opair{x,y,z}^* = \opair{u,v,w}^*
\end{equation*}成立,
但\(y \neq v\)或\(z \neq w\).
\begin{solution}
取\begin{equation*}
	\opair{1,1,2}^* = \{\{1\},\{1,1\},\{1,1,2\}\} = \{\{1\},\{1,2\}\},
\end{equation*}\begin{equation*}
	\opair{1,2,1}^* = \{\{1\},\{1,2\},\{1,2,1\}\} = \{\{1\},\{1,2\}\},
\end{equation*}
即可满足题设条件.
\end{solution}
\end{example}

\begin{definition}
两个元素\(x_1\)和\(x_2\)按一定顺序形成的排列称为\DefineConcept{有序对},
记作\begin{equation*}
	\opair{x_1,x_2}.
\end{equation*}

我们可以采用多种方式构造集合用于表记有序对\(\opair{x_1,x_2}\),
例如(由Norbert Wiener于1914年构造的)
\begin{equation*}
	\Set{ \Set{ \Set{x_1}, \emptyset }, \Set{ \Set{ x_2 } } }.
\end{equation*}
但我们更常用(由Kazimierz Kuratowski于1921年构造的)以下形式
\begin{equation*}
	\Set{ \Set{x_1}, \Set{x_1, x_2} }.
\end{equation*}
以后我们都采用第二种形式表记有序对,因此\begin{equation*}
	\opair{x_1,x_2}
	\defeq
	\Set{ \Set{x_1}, \Set{x_1, x_2} }.
\end{equation*}
\end{definition}

有序对具有以下性质.
\begin{property}\label{theorem:集合论.有序对的性质1}
%@see: 《Elements of Set Theory》 P36 Theorem 3A
\(\opair{u,v} = \opair{x,y}\)的充分必要条件是:\(u=x\)且\(v=y\).
\begin{proof}
充分性.
这个方向的证明是平凡的;
当\(u=x\)且\(v=y\)时,自然有\begin{equation*}
	\opair{u,v} = \opair{x,y}.
\end{equation*}

必要性.
当\(\opair{u,v} = \opair{x,y}\)时,根据定义有\begin{equation*}
	\Set{ \Set{u}, \Set{u, v} }
	= \Set{ \Set{x}, \Set{x, y} };
\end{equation*}于是\begin{equation*}
	\Set{u} \in \Set{ \Set{x}, \Set{x, y} },
	\eqno(1)
\end{equation*}且\begin{equation*}
	\Set{u,v} \in \Set{ \Set{x}, \Set{x, y} }.
	\eqno(2)
\end{equation*}
从(1)式我们知道,
要么等式\begin{equation*}
	\Set{u} = \Set{x}
	\eqno(3)
\end{equation*}成立,
要么等式\begin{equation*}
	\Set{u} = \Set{x,y}
	\eqno(4)
\end{equation*}成立;
从(2)式我们知道,
要么等式\begin{equation*}
	\Set{u,v} = \Set{x}
	\eqno(5)
\end{equation*}成立,
要么等式\begin{equation*}
	\Set{u,v} = \Set{x,y}
	\eqno(6)
\end{equation*}成立.

首先我们假设(4)式成立,那么有\(u = x = y\);
从而有(5)式等价于(6)式,且\(u = v = x = y\);
在这种情况下,必要性得证.
同理,假设(5)式成立,也可证明必要性.

然后我们假设(3)式、(6)式同时成立.
从(3)式成立我们知道\(u = x\).
从(6)式成立我们知道\(u = y\)或\(v = y\);
第一种情况,即\(u = x\)和\(u = y\)同时成立的情况,已经在(4)式成立时作了讨论.
第二种情况,即\(u = x\)和\(v = y\)同时成立的情况,立即保证了必要性.
\end{proof}
\end{property}
\cref{theorem:集合论.有序对的性质1} 让我们可以不含糊地将有序对\(\opair{x,y}\)
中的\(x\)和\(y\)分别称为%
有序对的\DefineConcept{第一坐标}(first coordinate)%
和\DefineConcept{第二坐标}(second coordinate).

%定义:
%\begin{equation*}
%	%TODO 需要找一个命名方法
%	\Set{ x \given (\exists u)(\exists v)[w=\opair{u,v} \land x \in u] },
%\end{equation*}\begin{equation*}
%	%TODO 需要找一个命名方法
%	\Set{ y \given (\exists u)(\exists v)[w=\opair{u,v} \land y \in v] }.
%\end{equation*}

\subsection{直积}
\begin{definition}\label{definition:集合论.直积}
设\(A,B\)都是集合.
在\(A\)中任取一个元素\(x\),
在\(B\)中任取一个元素\(y\),
组成一个有序对\(\opair{x,y}\),
把这样的有序对作为新的元素,
它们全体组成的类称为
“\(A\)与\(B\)的\DefineConcept{直积}(direct product)”,
或“\(A\)与\(B\)的\DefineConcept{笛卡尔乘积}(Cartesian product)”,
记为\(A \times B\),
即\begin{equation}
	A \times B
	\defeq
	\Set*{ \opair{x,y} \given x \in A \land y \in B }.
\end{equation}
%@see: https://mathworld.wolfram.com/DirectProduct.html
\end{definition}

在宣称\cref{definition:集合论.直积} 合法以前,
我们必须确认它给出的类\(A \times B\)是不是真的是一个集合.
而要证明\(A \times B\)是一个集合,
我们就必须找到一个足够大的集合,
它包括了我们需要的全部有序对,
然后利用子集公理证明\(A \times B\)是这个集合的子集.
为此我们给出下述引理.
\begin{lemma}\label{theorem:集合论.有序对是其坐标元素所在集合的二重幂集的元素}
%@see: 《Elements of Set Theory》 P37 Lemma 3B
如果\(x,y \in A\),那么\(\opair{x,y} \in \Powerset\Powerset A\).
\begin{proof}
由\(x,y \in A\)可知\(\{x\},\{x,y\} \subseteq A\),
即\(\{x\},\{x,y\} \in \Powerset A\),
那么\(\{\{x\},\{x,y\}\} \subseteq \Powerset A\),
进一步可得\(\opair{x,y} = \{\{x\},\{x,y\}\} \in \Powerset\Powerset A\).
\end{proof}
\end{lemma}

\begin{theorem}\label{theorem:集合论.直积存在定理}
%@see: 《Elements of Set Theory》 P38 Corollary 3C
设\(A,B\)都是集合,那么存在这样一个集合\(C\),使得\(C = A \times B\).
\begin{proof}
要使\(C = A \times B\),根据外延公理必有\begin{equation*}
	w \in C \iff w \in A \times B,
\end{equation*}
再根据直积的定义有\begin{equation*}
	w \in C \iff w = \opair{x,y} \land x \in A \land y \in B.
\end{equation*}
根据子集公理,我们可以构造集合\begin{equation*}
	C = \Set*{
		w \in \Powerset\Powerset(A \cup B)
		\given
		w = \opair{x,y} \land x \in A \land y \in B
	}.
\end{equation*}
显然,根据\cref{theorem:集合论.有序对是其坐标元素所在集合的二重幂集的元素},
\(C\)满足定理要求的条件.
\end{proof}
\end{theorem}
\cref{theorem:集合论.直积存在定理} 告诉我们,
任意两个集合\(A\)和\(B\)的直积\(A \times B\)也是集合.

容易发现,对于任意集合\(A\),总有\begin{equation}
	A \times \emptyset
	= \emptyset \times A
	= \emptyset \times \emptyset
	= \emptyset.
\end{equation}
这是因为\((\forall x)[x \notin \emptyset]\).

\begin{example}
设\(A,B,C\)是集合.
证明:\begin{equation}
	A \subseteq B
	\implies
	A \times C \subseteq B \times C.
\end{equation}
\begin{proof}
直接计算得
\begin{align*}
	A \subseteq B
	&\iff [x \in A \implies x \in B] \\
	&\implies [x \in A \land y \in C \implies x \in B \land y \in C] \\
	&\iff [\opair{x,y} \in A \times C \implies \opair{x,y} \in B \times C] \\
	&\iff [A \times C \subseteq B \times C].
	\qedhere
\end{align*}
\end{proof}
\end{example}

\begin{example}
%@see: 《Elements of Set Theory》 P38 Exercise 2.(a)
%@see: 《Elements of Set Theory》 P65 Exercise 54.
设\(A,B,C\)是集合.
证明:\begin{gather}
	A \times (B \cup C) = (A \times B) \cup (A \times C),
	\label{equation:集合论.直积分配律1} \\
	A \times (B \cap C) = (A \times B) \cap (A \times C),
	\label{equation:集合论.直积分配律2} \\
	A \times (B - C) = (A \times B) - (A \times C).
	\label{equation:集合论.直积分配律3}
\end{gather}
\begin{proof}
对\cref{equation:集合论.直积分配律1} 证明如下:
由于对于任意\(w = \opair{x,y}\)总有
\begin{align*}
	w \in A \times (B \cup C)
	&\iff x \in A \land y \in B \cup C \\
	&\iff x \in A \land (y \in B \lor y \in C) \\
	&\iff (x \in A \land y \in B) \lor (x \in A \land y \in C) \\
	&\iff (w \in A \times B) \lor (w \in A \times C),
\end{align*}
那么由外延公理可知\(A \times (B \cup C) = (A \times B) \cup (A \times C)\).

对\cref{equation:集合论.直积分配律2} 证明如下:
由于对于任意\(w = \opair{x,y}\)总有
\begin{align*}
	w \in A \times (B \cap C)
	&\iff x \in A \land y \in B \cap C \\
	&\iff x \in A \land (y \in B \land y \in C) \\
	&\iff (x \in A \land y \in B) \land (x \in A \land y \in C) \\
	&\iff (w \in A \times B) \land (w \in A \times C),
\end{align*}
那么由外延公理可知\(A \times (B \cap C) = (A \times B) \cap (A \times C)\).

对\cref{equation:集合论.直积分配律3} 证明如下:
由于对于任意\(w = \opair{x,y}\)总有
\begin{align*}
	w \in A \times (B - C)
	&\iff x \in A \land y \in (B - C) \\
	&\iff x \in A \land (y \in B \land y \notin C) \\
	&\iff (x \in A \land y \in B \land x \notin A)
		\lor (x \in A \land y \in B \land y \notin C) \\
	&\iff (x \in A \land y \in B) \land (x \notin A \lor y \notin C) \\
	&\iff (x \in A \land y \in B) \land \neg(x \in A \land y \in C) \\
	&\iff w \in A \times B \land w \notin A \times C,
\end{align*}
那么由外延公理可知\(A \times (B - C) = (A \times B) - (A \times C)\).
\end{proof}
\end{example}

\begin{example}
%@see: 《Elements of Set Theory》 P38 Exercise 2.(b)
证明:如果\(A \times B = A \times C\)且\(A \neq \emptyset\),则\(B = C\).
%TODO
\end{example}

\begin{example}
%@see: 《Elements of Set Theory》 P38 Exercise 3.
证明:\(A \times \bigcup B = \bigcup\Set{ A \times X \given X \in B }\).
%TODO
\end{example}

\begin{example}
%@see: 《Elements of Set Theory》 P38 Exercise 5.(a)
设\(A,B\)都是集合,试证:\(\Set{ \{x\} \times B \given x \in A }\)也是集合.
\begin{proof}
由于\(x \in A \implies \{x\} \subseteq A \implies \{x\} \times B \subseteq A \times B\),
所以我们可以利用子集公理构造集合\begin{equation*}
	\Set{ w \in A \times B \given w = \opair{u,v} \land u = x }.
	\qedhere
\end{equation*}
\end{proof}
\end{example}

\subsection{关系}
\begin{definition}
\DefineConcept{关系}(relation)是有序对的集合.
\end{definition}

如果有序对\(\opair{x,y}\)是关系\(\rel{R}\)的元素,即\begin{equation*}
	\opair{x,y} \in \rel{R},
\end{equation*}
那么称“\(x\)与\(y\)有\(\rel{R}\)关系”,
记作\(x\rel{R}y\);
反之,\begin{equation*}
	\opair{x,y} \notin \rel{R},
\end{equation*}
那么称“\(x\)与\(y\)没有\(\rel{R}\)关系”.

给定集合\(X\)和集合\(Y\),
如果关系\(\rel{R}\)满足\begin{equation*}
	\rel{R} \subseteq X \times Y,
\end{equation*}
我们就称“\(\rel{R}\)是\(X\)与\(Y\)之间的\DefineConcept{二元关系}(binary relation)”.
特别地,当\(X = Y\)时,就称“\(\rel{R}\)是\(X\)上的\DefineConcept{二元关系}”.

从关系的定义可以看出,关系是有序对的集合.
即便是看起来毫无意义的,只由一个有序对组成的集合,也可以看成是一个关系.
我们自然会认为有的关系比别的关系更有意义,
例如我们马上会提到的“映射”“等价关系”和“排序关系”.

\begin{definition}\label{definition:集合论.定义域与值域的定义}
设\(R\)是集合.

如果集合\(A\)满足\begin{equation*}
	x \in A \iff (\exists y)[\opair{x,y} \in R],
\end{equation*}
那么称\(A\)为“\(R\)的\DefineConcept{定义域}(domain)”,记作\(\dom R\).

如果集合\(B\)满足\begin{equation*}
	x \in B \iff (\exists t)[\opair{t,x} \in R],
\end{equation*}
那么称\(B\)为“\(R\)的\DefineConcept{值域}(range)”,记作\(\ran R\).

最后,我们把\(R\)的定义域与它的值域的并集称为
“\(R\)的\DefineConcept{域}(field)”,记作\(\fld R\),即\begin{equation*}
	\fld R \defeq \dom R \cup \ran R.
\end{equation*}
\end{definition}

为了正当化\cref{definition:集合论.定义域与值域的定义},
我们必须明确:对于任意集合\(R\),存在这样一个集合,
\(R\)中的有序对的第一坐标和第二坐标都是这个集合的元素.
这个问题类似于我们之前对于直积\(A \times B\)的定义的正当化,
当时我们证明了\cref{theorem:集合论.直积存在定理}.
为此,我们给出下述引理,
可以注意到它与\cref{theorem:集合论.有序对是其坐标元素所在集合的二重幂集的元素} 的关联.

\begin{lemma}
%@see: 《Elements of Set Theory》 P41 Lemma 3D
如果\(\opair{x,y} \in A\),那么\(x,y \in \bigcup\bigcup A\).
\end{lemma}
这条引理已经在\cref{example:集合论.有序对各坐标的取值范围} 得到证明.
利用这条引理,再加上子集公理,我们可以构造集合\(R\)的定义域和值域:
\begin{gather}
	\dom R
	\defeq
	\Set*{ x \in \bigcup\bigcup R \given (\exists y)[\opair{x,y} \in R] }, \\
	\ran R
	\defeq
	\Set*{ x \in \bigcup\bigcup R \given (\exists t)[\opair{t,x} \in R] }.
\end{gather}

\begin{example}
%@see: 《Elements of Set Theory》 P41 Exercise 6.
试证:集合\(A\)是一个关系的充分必要条件是\begin{equation*}
	A \subseteq \dom A \times \ran A.
\end{equation*}
%TODO
\end{example}

\begin{example}
%@see: 《Elements of Set Theory》 P41 Exercise 7.
试证:给定关系\(\rel{R}\),总有\begin{equation*}
	\fld \rel{R} = \bigcup\bigcup \rel{R}.
\end{equation*}
%TODO
\end{example}

\begin{example}
%@see: 《Elements of Set Theory》 P41 Exercise 8.
试证:对于任意集合\(A\),总有\begin{gather}
	\dom\bigcup A = \bigcup\Set{ \dom\rel{R} \given \rel{R} \in A }, \\
	\ran\bigcup A = \bigcup\Set{ \ran\rel{R} \given \rel{R} \in A }.
\end{gather}
%TODO
\end{example}

\subsection{多元关系}
我们可以将“有序对”和“二元关系”的概念分别扩展为“元组”和“多元关系”.

例如定义\begin{equation*}
	\opair{x_1,x_2,x_3}
	\defeq
	\opair{\opair{x_1,x_2},x_3},
\end{equation*}
称之为\DefineConcept{三元组}(triple);
定义\begin{equation*}
	\opair{x_1,x_2,x_3,x_4}
	\defeq
	\opair{\opair{x_1,x_2,x_3},x_4},
\end{equation*}
称之为\DefineConcept{四元组}(quadruple);
定义\begin{equation*}
	\opair{x_1,x_2,x_3,x_4,x_5}
	\defeq
	\opair{\opair{x_1,x_2,x_3,x_4},x_5},
\end{equation*}
称之为\DefineConcept{五元组}(quintuple);
以此类推,对于任意给定\(n\),可以定义\(n\)元组(\(n\)-tuple):\begin{equation*}
	\opair{x_1,x_2,\dotsc,x_n}
	\defeq
	\opair{\opair{x_1,x_2,\dotsc,x_{n-1}},x_n}.
\end{equation*}
为了让我们的定义看起来整齐划一,我们还可以补充定义\begin{equation*}
	\opair{x} \defeq x,
\end{equation*}
称之为\DefineConcept{一元组}(1-tuple).

我们把“在\(A\)上的\(n\)元\DefineConcept{关系}(n-ary \emph{relation} on \(A\))”
定义为由\(n\)元组构成的集合.
由于\(A\)上的二元关系是\(A \times A\)的子集,
以及\(A\)上的三元关系(ternary relation, 3-ary relation)是\((A \times A) \times A\)的子集,
所以三元关系也可归结为一种二元关系;
同理,其他\(n\)元关系,只要\(n>1\),也都可以归结为二元关系.
虽然在上面对\(n\)元关系的定义中,
我们也把由\(A\)中包括的一元组构成的集合称为
“\(A\)上的一元关系(unary relation, 1-ary relation)”,
但它只是\(A\)的一个子集,根本不满足关系的定义.
