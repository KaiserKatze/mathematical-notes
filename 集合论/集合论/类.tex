
\section{类}
公理化集合论的一个目标是规避经典悖论,其中之一就是“罗素悖论”.
这个悖论来源于我们天真地接受了这样一个想法:
任给一个性质,总存在一个集合,它的元素都满足上述性质.
例如,给定一个合式公式\(\phi\),它只含一个自由变元,存在一个集合,
所有满足\(\phi\)的对象都在这个集合里,
而所有不满足\(\phi\)的对象都不在这个集合里,也就是说
\begin{equation}\label{equation:集合论.类的逻辑表示}
	(\forall x)[x \in a \iff \phi(x)].
\end{equation}
这个原则被称为“抽象公理(Abstraction Axiom)”.
它曾在弗雷格于1893年所著的《算术基础(Grundgesetze der Arithmetik)》中出现过.
后来,罗素向弗雷格写信指出,抽象公理会导致悖论.

考虑到\(b \notin b\)是一个合式公式,
我们用受限变元\(x\)代替自由变元\(b\),
代入到“抽象公理”中,作为\(\phi(x)\).
然后,我们假设存在一个集合\(a\)使得\begin{equation*}
	(\forall x)[x \in a \iff x \notin x],
\end{equation*}
那么,特别地,有\begin{equation*}
	a \in a \iff a \notin a.
\end{equation*}
于是我们得到一个逻辑谬误,
这是因为根据排中律\(p \lor \neg p = 1\),
要么有\(a \in a\),要么有\(a \notin a\),不能二者兼得.
这就说明“抽象公理”是一个假命题.

我们想要取得“满足某种特定性质的全部对象的汇集”的愿望难以摈弃,
而如若不拒绝“抽象公理”,则会导致悖论,我们又该怎么办呢?
策梅洛与弗伦克尔提供了以下解决方案:
我们把命题公式 \labelcref{equation:集合论.类的逻辑表示} 中的\(a\)称为%
“具有性质\(\phi\)的各个事物的\DefineConcept{类}%
(the \emph{class} of individuals that have the property \(\phi\))”,
并且将它表示为
\begin{equation}
	\Set{ x \given \phi(x) }.
\end{equation}
以后我们可以证明,类是对集合这一概念的延伸,
每个集合都是一个类,但不是每个类都是一个集合.
这里,其实我们也把用于表示元素与集合之间从属关系的\(\in\)符号的使用范围拓宽到类上了:
“一个对象是某个类\(\Set{ x \given \phi(x) }\)的一个元素”%
当且仅当“这个对象是一个集合,且它具有这个类的定义性质\(\phi\)”.
接下来,要解决罗素悖论,
我们只需证明\(\Set{ x \given x \notin x }\)%
是一个\DefineConcept{真类}(proper class),
也就是证明这个类不是集合.
正因为这个类不是集合,所以它不从属于任何一个类,包括它自己.

现在我们来将形如\begin{equation*}
	\Set{ x \given \phi(x) }
\end{equation*}的符号加入我们的对象语言.
\begin{definition}
%@see: 《Introduction to Axiomatic Set Theory》 P12 Definition 4.1.
我们可以归纳地给出\DefineConcept{广义合式公式}(well-formed formula in the wider sense)的定义:
\begin{enumerate}
	\item 如果\(a\)和\(b\)都是自由变元,那么\(a \in b\)是广义合式公式.

	\item 如果\(\phi\)和\(\psi\)都是广义合式公式,且\(a\)和\(b\)都是自由变元,
	那么\begin{equation*}
		\def\A{\Set{ x \given \psi(x) }}
		\def\B{\Set{ x \given \phi(x) }}
		a \in \A, \qquad
		\B \in b, \qquad
		\B \in \A
	\end{equation*}都是广义合式公式.

	\item 如果\(\phi\)和\(\psi\)都是广义合式公式,
	那么\begin{equation*}
		\neg\phi,
		\qquad
		\phi\land\psi,
		\qquad
		\phi\lor\psi,
		\qquad
		\phi\limp\psi,
		\qquad
		\phi\liff\psi
	\end{equation*}
	都是广义合式公式.

	\item 如果\(\phi\)是广义合式公式,且\(x\)是受限变元,
	那么\begin{equation*}
		(\exists x)\phi(x)
		\quad\text{和}\quad
		(\forall x)\phi(x)
	\end{equation*}都是广义合式公式.
\end{enumerate}
任意一条公式,当且仅当它可以由上述三条规则演绎推得时,我们才称其为广义合式公式.
\end{definition}

\begin{definition}
%@see: 《Introduction to Axiomatic Set Theory》 P12 Definition 4.2.
设\(\phi\)和\(\psi\)都是广义合式公式.
\def\A{\Set{ x \given \phi(x) }}
\def\B{\Set{ x \given \psi(x) }}
那么\begin{enumerate}
	\item \(a \in \A \defiff \phi(a)\).
	\item \(\A \in a \defiff (\exists y)[y \in a \land (\forall z)[z \in y \iff \phi(z)]]\).
	\item \(\A \in \B \defiff (\exists y)[y \in \B \land (\forall z)[z \in y \iff \phi(z)]]\).
\end{enumerate}
\end{definition}

任意一条广义合式公式\(\phi\),
总\DefineConcept{可化简为}(is reducible to)一条唯一确定的合式公式\(\phi^*\).
\begin{definition}\label{definition:集合论.广义合式公式化简为合式公式的方法}
%@see: 《Introduction to Axiomatic Set Theory》 P12 Definition 4.3.
设\(\phi\)和\(\psi\)都是广义合式公式.
\def\A{\Set{ x \given \phi(x) }}
\def\B{\Set{ x \given \psi(x) }}
那么\begin{enumerate}
	\item \([a \in b]^* \defiff a \in b\).
	\item \([a \in \A]^* \defiff \phi^*(a) \defiff [\phi(a)]^*\).
	\item \([\A \in a]^* \defiff (\exists y)[y \in a \land (\forall z)[z \in y \iff \phi^*(z)]]\).
	\item \([\A \in \B]^* \defiff (\exists y)[\psi^*(y) \land (\forall z)[z \in y \iff \phi^*(z)]]\).
	\item \([\neg\phi]^* \defiff \neg\phi^*\).
	\item \([\phi\land\psi]^* \defiff \phi^*\land\psi^*\).
	\item \([(\forall x)\phi(x)]^* \defiff (\forall x)\phi^*(x)\).
\end{enumerate}
\end{definition}

\begin{theorem}
每一条广义合式公式\(\phi\),
总可按\cref{definition:集合论.广义合式公式化简为合式公式的方法} 约定的规则,
化简为唯一确定的合式公式\(\phi^*\).
\end{theorem}

我们把自由变元和类的符号统称为\DefineConcept{项}(term).
\begin{definition}
如果\(A\)和\(B\)是项,那么\begin{equation*}
	A=B \defiff (\forall x)[x \in A \iff x \in B].
\end{equation*}
\end{definition}

\begin{theorem}
\(A \in B \iff (\exists x)[x = A \land x \in B]\).
\end{theorem}

\begin{theorem}
如果\(A,B,C\)都是项,那么\begin{enumerate}
	\item \(A=A\).
	\item \(A=B \implies B=A\).
	\item \(A=B \land B=C \implies A=C\).
\end{enumerate}
\end{theorem}

\begin{theorem}
如果\(A\)和\(B\)是项,且\(\phi\)是广义合式公式,那么\begin{equation*}
	A=B \implies [\phi(A)\iff\phi(B)].
\end{equation*}
\end{theorem}

\begin{theorem}\label{theorem:集合论.集合是类}
\(a=\Set{ x \given x \in a }\).
\end{theorem}
\cref{theorem:集合论.集合是类} 说明“每个集合都是一个类”.
接下来,我们引入两个谓词:
用\(\mathfrak{M}(A)\)表示“\(A\)是集合”,
用\(\mathfrak{Pr}(A)\)表示“\(A\)是真类”,即\begin{equation*}
	\mathfrak{M}(A) \defiff (\exists x)[x = A],
\end{equation*}\begin{equation*}
	\mathfrak{Pr}(A) \defiff \neg\mathfrak{M}(A).
\end{equation*}

\begin{theorem}
设\(a\)是集合,那么\(\mathfrak{M}(a)\).
\end{theorem}

\begin{theorem}
\(A \in \Set{ x \given \phi(x) } \iff \mathfrak{M}(A) \land \phi(A)\).
\end{theorem}

\begin{theorem}
\def\Ru{\mathrm{Ru}}
把\begin{equation*}
	\Ru \defeq \Set{ x \given x \notin x }
\end{equation*}称为{\rm\DefineConcept{罗素类}(Russel class)},
那么\(\mathfrak{Pr}(\Ru)\).
\begin{proof}
由于\(\mathfrak{M}(\Ru)
\implies
[\Ru\in\Ru\iff\Ru\notin\Ru]\)矛盾,
故\(\Ru\)是真类.
\end{proof}
\end{theorem}
鉴于罗素类\(\mathrm{Ru}\)是真类,
罗素悖论旋即得以解决.

\begin{definition}
为了简化符号,
\def\x{x_1,\dotsc,x_n}
定义:\begin{gather}
	(\forall\x)\phi(\x)
	\defiff
	(\forall x_1)\dotso(\forall x_n)\phi(\x), \\
	(\exists\x)\phi(\x)
	\defiff
	(\exists x_1)\dotso(\forall x_n)\phi(\x), \\
	\begin{split}
		&\hspace{-20pt}
		(\forall\x\in A)\phi(\x) \\
		&\defiff
		(\forall\x)[x_1\in A \land \dotsb \land x_n\in A \implies \phi(\x)],
	\end{split} \\
	(\exists\x\in A)\phi(\x)
	\defiff
	(\exists\x)[x_1 \in A \land \dotsb \land x_n\in A \land \phi(\x)], \\
	a_1,\dotsc,a_n\in A \defiff a_1\in A \land \dotsb \land a_n\in A.
\end{gather}
\end{definition}

\begin{definition}
\def\x{x_1,\dotsc,x_n}
设\(\uptau\)是项,且\(\phi\)是合式公式,定义:\begin{equation}
	\Set{ \uptau(\x) \given \phi(\x) }
	\defeq
	\Set{ y \given (\exists\x)[y = \uptau(\x) \land \phi(\x)] }.
\end{equation}
\end{definition}
