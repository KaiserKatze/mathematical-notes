%@see: https://mathworld.wolfram.com/Zermelo-FraenkelAxioms.html
\section{公理化集合论}
虽然直到19世纪末德国数学家康托才创立了集合论,
但是其实人们很早就开始使用集合的概念并对集合进行运算了.
在《周易·系辞下》中有这么一句话:“上古结绳而治,后世圣人易之以书契.”
于是我们想到,古人在农业生产中始终有着对农产品分类和计数的需要,
即将同类的产品集中地放置,
再用一个符记(例如手势、绳结、算筹、文字)表示这类产品的数量.

在本节我们先来介绍集合和元素的概念、集合存在公理,以及集合的运算法则.

\subsection{外延公理}
要实现对事物的分类,我们应该首先明白这样一点:
在现实世界中,事物之间的差异与矛盾是根本存在的.
我们知道,人对事物的认识,是人脑对客观现实的反映.
如果我们在观测事物时,观测手段十分粗略,
那么想必很难找出事物之间的差异;
如果我们在观测事物时,观测手段十分细致,
那么必然能够找出事物之间的更多差异.

为了解决上述问题,一组相对的哲学概念被提出来了,这就是“逻辑同一性”和“实在同一性”.
简单地说,实在同一性和逻辑同一性都是对事物关系的一种描述,
实在同一性要求被比较的事物无法被区分,
而逻辑同一性是指两个物在部分的属性上具备某种相同、相等或相似的关系.
以实在同一性和逻辑同一性作为分类的标准,
我们就可以区分事物以构建对立关系,又可以混同事物以构建等价关系.

\begin{definition}
如果“\(a\)是\(A\)的元素(\(a\) is a \emph{member} of \(A\),
\(a\) is an \emph{element} of \(A\))”,
或“\(a\) \DefineConcept{属于} \(A\)
(\(a\) \emph{belongs to} \(A\))”,
记作\(a \in A\)
\footnote{
	用来表记“所属关系(membership)”的符号\(\in\)实际上是变形的小写希腊字母\(\epsilon\),
	它是由皮亚诺于 1889 年提出的.
}.
\end{definition}

\begin{definition}
如果“\(b\)不是\(A\)的元素(\(b\) is not an element of \(A\))”,
或“\(b\) \DefineConcept{不属于} \(A\)
(\(b\) does not belong to \(A\))”,
记作\(b \notin A\).
\end{definition}

\begin{axiom}[外延公理I]
%@see: 《Elements of Set Theory》 P17 Extensionality Axiom
如果集合\(A\)与\(B\)具有相同的元素,
则称“\(A\)与\(B\) \DefineConcept{相等}”
或“\(A\) \DefineConcept{等于} \(B\)”,
记作\(A=B\),即
\begin{equation*}
	(\forall A)(\forall B)[
		(\forall x)[x \in A \iff x \in B]
		\defiff
		A=B
	].
\end{equation*}
\end{axiom}
外延公理在英语中称作Extensionality Axiom,在德语中称作Axiom der Bestimmtheit.

我们还可以定义:\begin{equation*}
	[A \neq B] \defiff \neg(A=B).
\end{equation*}

\begin{property}
设\(a,b,c\)是集合.
那么有\begin{enumerate}
	\item\label{item:集合论.集合相等的自反性}
	自反性,即\(a=a\).
	\item\label{item:集合论.集合相等的对称性}
	对称性,即\(a=b \implies b=a\).
	\item\label{item:集合论.集合相等的传递性}
	传递性,即\(a=b \land b=c \implies a=c\).
\end{enumerate}
\begin{proof}
证明如下:
\begin{enumerate}
	\item \((\forall x)[x \in a \iff x \in a]\).
	\item \((\forall x)[x \in a \iff x \in b] \implies (\forall x)[x \in b \iff x \in a]\).
	\item \((\forall x)[x \in a \iff x \in b] \land (\forall x)[x \in b \iff x \in c]
	\implies (\forall x)[x \in a \iff x \in c]\).
	\qedhere
\end{enumerate}
\end{proof}
\end{property}

我们对相等性的直观理解源于同一性.
对于相等性,我们期望的一个基本性质是“两个相等的事物可以用另外两个相等的事物互相替代”.
也就是说,如果\(a=b\),那么对于任意事物,可以断言关于\(a\)的任何事情也可以断言关于\(b\)
特别地,如果某个合式公式对\(a\)成立,那么它也应对\(b\)成立,并且反之亦然,即\begin{equation*}
	(a=b)\implies[\phi(a)\iff\phi(b)],
\end{equation*}
其中\(\phi(a)\)和\(\phi(b)\)是%
用受限变元\(a\)和\(b\)代替合式公式\(\phi\)中的某个相同的自由变元而取得的.

\begin{axiom}[外延公理II]
%@see: 《Introduction to Axiomatic Set Theory》 P8 Extensionality Axiom
设\(A,B,C\)是集合,那么\begin{equation*}
	A=B \land A \in C \implies B \in C.
\end{equation*}
\end{axiom}

\begin{property}
设\(A,B,C\)是集合,那么\begin{equation*}
	A=B \implies [A \in C \iff B \in C].
\end{equation*}
\begin{proof}
根据外延公理II和\hyperref[item:集合论.集合相等的对称性]{集合相等的对称性}易得.
\end{proof}
\end{property}

\begin{theorem}
\(a=b \implies [\phi(a)\iff\phi(b)]\).
\end{theorem}

外延公理说明一个集合完全由它的元素确定.


\subsection{集合存在公理}
让我们想象这样的情景:
你是一个在原始部落中负责管理仓库的工人学徒.
今天,猎人们从森林里带回一些猎物,堆放在地面上,让你把它们先分好类再收进仓库.
今天是你上班的第一天,你不知道这些猎物是什么.
说要分类,你也只能一屁股坐在空地上,面朝这些码放在一起的猎物,一筹莫展.
正在此时,一位老人颤颤巍巍地走了过来,他正是你的师父.
你向他询问这些猎物应当如何分类.

“你应该首先了解这些猎物具有的特征,”
在望了一眼这堆猎物以后,你的师父说道,
“你看,和我们人类一样,这些猎物身上长着眼睛的部位也叫做头部.
在它们的头上也长有和我们的嘴巴一样用来进食的部位.
但是,你要注意,有的猎物的‘嘴’的形状与我们的嘴巴不同,它的‘嘴’是尖尖的,
我们把这类猎物的‘嘴’叫做‘喙’;
此外这类猎物也长着和我们的脚一样用来站立的部位,但它的‘脚’也是尖尖的,
我们把这类猎物的‘脚’叫做‘爪子’;
最后,我们把这类猎物叫做‘鸟’.”
他掏出水壶,喝上一口水以后,继续说:
“今天你就学习怎么把‘鸟’从这堆猎物中筛选出来,放在一起.
其余的猎物我让别人来收拾.”
紧接着,他缓缓蹲下,捡了一只长着尖喙尖爪的猎物,递给你,
“这就是‘鸟’,你照着这个样子分类吧!”
说完这句话,师父站起身来,去其他地方溜达了.

你伸出左手接住师父递给你的那只鸟,
再用空着的右手在猎物堆里扒拉,随意地拎出一只猎物.
你举起双臂,扭动翻转手腕、手肘,
仔细查看这两只猎物,比较它们的特征是否一致.
如果两者不同,右手上的猎物没有尖喙尖爪,
就把它放到身体右侧,和原本堆在地上的猎物隔开一段距离,
再用右手重新抓一只猎物进行下一轮比较;
如果两者相同,右手的猎物也有尖喙尖爪,
就把你左手提着的这只鸟放在身体左侧,也和原本堆在地上的猎物隔开一段距离,
再把右手抓着的鸟转移到左手上,
然后用右手重新抓一只猎物进行下一轮比较.

在过了一会儿后,你发现面前没有需要分类的猎物了,
于是你把左手拎着的那只鸟也放在身体左侧.
就这样,你把猎人们放在地上的那堆猎物分成了两小堆,左边这堆全是鸟.


我们经常会像上面这个故事一样,
把一些事物收集在一起,把这个整体叫做一个\DefineConcept{集合}(set),
简称为\DefineConcept{集};
然后把组成集合的这些事物叫做这个集合的\DefineConcept{元素}(member,或element),
简称为\DefineConcept{元}.
例如,在上面的故事中,在你举起双臂时,
你双手上抓着的两只鸟组成了一个集合,每只鸟都是这个集合的元素.
又例如,在你把猎物分成两堆以后,这两堆猎物中的每一堆都可以看作一个集合.
为了方便讨论,我们用大写拉丁字母(如\(A,B,C,X,Y\)等)表示集合,
用小写拉丁字母(如\(a,b,c,x,y\)等)表示元素.
由此,我们可以给出一个集合的粗糙的定义.

我们可以看出,集合具有三种特性:
\begin{enumerate}
	\item {\rm\bf 确定性},
	即对于任意一个元素,
	要么它属于某个指定集合,
	要么它不属于该集合,
	二者必居其一.

	\item {\rm\bf 互异性},
	即同一个集合中的元素是互不相同的,
	或者说,在以列举法表记的集合中,
	如果表记同一个元素的符号出现了多次,
	那么可以直接去除多余的符号而对集合的描述没有任何影响,即\begin{equation*}
		\Set{ x, x, y } = \Set{ x, y }.
	\end{equation*}

	\item {\rm\bf 无序性},
	即在以列举法表记的集合中,
	任意改变集合中元素的排列次序,
	它们仍然表示同一个集合,即\begin{equation*}
		\Set{ y, x } = \Set{ x, y }.
	\end{equation*}
\end{enumerate}



另外,根据上面这个故事,我们还可以想到:
一方面,原本空地上没有任何我们关心的东西,至少在这里我们并不关心空气和尘土;
另一方面,对于任意两个东西,如果我们觉得这两个东西能分类在一起,就可以把它们分类在一起.
我们可以说空地是一个空着的集合,或者说“空集”.
我们还说两只鸟组成一对鸟,或者说“对集”.

从上述生活经验出发,我们认定“空集”“对集”和“并集”是一定存在的,
于是我们可以给出如下的三个公理.

\begin{axiom}[空集公理]\label{axiom:集合论.空集公理}
%@see: 《Elements of Set Theory》 P18 Empty Set Axiom
总存在这样一个集合\(A\),没有任何元素属于它\footnote{%
我们也可以用命题公式\begin{equation*}
	(\exists A)(\forall x)[x \in A \iff x \neq x]
\end{equation*}表示“没有任何元素的集合”的存在性.
它主要利用了“\((x \neq x)\)一定是假命题”这一点.
},即\begin{equation*}
	(\exists A)(\forall x)[x \notin A].
\end{equation*}
\end{axiom}

\begin{axiom}[对集公理]\label{axiom:集合论.对集公理}
%@see: 《Elements of Set Theory》 P18 Pairing Axiom
对于任意两个元素\(u\)和\(v\),
总存在一个集合\(B\),它的元素只有\(u\)和\(v\),即\begin{equation*}
	(\forall u)(\forall v)(\exists B)(\forall x)
	[x \in B \iff x = u \lor x = v].
\end{equation*}
\end{axiom}

\begin{axiom}[并集公理I]
%@see: 《Elements of Set Theory》 P18 Union Axiom, Preliminary Form
对于任意两个集合\(a\)和\(b\),
总存在一个集合\(B\),它的元素要么属于\(a\),要么属于\(b\),即\begin{equation*}
	\forall a, \forall b, \exists B, \forall x
	\bigl(
		x \in B
		\iff
		x \in a \lor x \in b
	\bigr).
\end{equation*}
\end{axiom}

\begin{axiom}[幂集公理]
对于任意集合\(A\),总存在一个集合\(B\),\(B\)的全部元素恰好是集合\(A\)的全部子集,即\begin{equation*}
	(\forall A)(\exists B)(\forall x)
	[
		x \in B
		\iff
		(\forall t)[t \in x \implies t \in A]
		\defiff
		x \subseteq A
		\defiff
		A \supseteq x
	].
\end{equation*}
\end{axiom}


空集公理在英语中称作Empty Set Axiom.
对集公理在英语中称作Pairing Axiom.
并集公理在英语中称作Union Axiom, Preliminary Form,
在德语中称作Axiom der Vereinigung.
在德语中把对集公理和空集公理并称为Axiom der Elementarmengen.
幂集公理在英语中称作Power Set Axiom,在德语中称作Axiom der Potenzmenge.

下面我们给出我们对“空集”“对集”“并集”和“幂集”的正式定义.

\subsection{空集}
\begin{definition}
不含任何元素的集合称为\DefineConcept{空集}(empty set),
记作\(\emptyset\),即
\begin{equation}
	\emptyset \defeq \Set{},
\end{equation}
或
\begin{equation}
	\emptyset \defeq \Set{ x \given x \neq x }.
\end{equation}
\end{definition}
%定义空集\(\emptyset\)时一定要注意两点:
%一是“空集”是否存在,这已由空集公理确保;
%二是“空集”是否唯一,这则由外延公理确保.
%若是没有这两条公理的帮助,我们就不能说\(\emptyset\)是良定义的.

容易看出,当我们说“集合\(A\)不是空集”或“集合\(A\)是非空集合”时,
\(A\)中必定至少有一个元素,即\begin{equation*}
	(\exists x)[x \in A].
\end{equation*}

\subsection{对集}
\begin{definition}
对于任意给定的元素\(u\)和\(v\),
我们用符号\begin{equation*}
	\Set{ u, v }
\end{equation*}表示只有\(u,v\)元素的集合,
并把这个集合称为“\(u\)和\(v\)的\DefineConcept{对集}(pair set)”,
即\begin{equation*}
	\Set{ u, v } \defeq \Set{ x \given x = u \lor x = v }.
\end{equation*}
\end{definition}
在对集的定义中,我们没有明确说明元素\(u,v\)是否相同.
实际上,当\(u=v\)时,\(u\)和\(v\)的对集成为只含一个元素的集合\begin{equation*}
	\Set{ u } = \Set{ u, v };
\end{equation*}
我们称这种集合为\DefineConcept{单元素集}(singleton)或\DefineConcept{单点集}.
%@see: https://mathworld.wolfram.com/SingletonSet.html
单元素集是最简单的非空集合.

利用对集公理和并集公理,我们可以构造任意有限集合.
例如,给定任意的\(x\),我们可以定义“单元集”如下:\begin{equation*}
\{x\} \defeq \{x, x\}.
\end{equation*}
又例如,给定任意的\(\AutoTuple{x}{3}\),我们可以定义由这三个元素构成的集合
\begin{equation*}
	\{\AutoTuple{x}{3}\} \defeq \{x_1,x_2\}\cup\{x_3\};
\end{equation*}
同样地,我们还可以定义\(\{\AutoTuple{x}{4}\}\),以此类推.

%可以看出,对集公理和并集公理是保障我们采用“列举法”表示集合这一方法论的正当性的基础.

\subsection{并集}
\begin{definition}
称\begin{equation*}
	\Set{ x \given x \in A \lor x \in B }
\end{equation*}为“\(A\)和\(B\)的\DefineConcept{并}(the \emph{union} of \(A\) and \(B\))”,
记作\(A \cup B\)或\(A+B\),即\begin{equation*}
	A \cup B \defeq \Set{ x \given x \in A \lor x \in B }.
\end{equation*}
\end{definition}


\begin{definition}
设系\(A = \Set{\AutoTuple{a}{n}}\).
称集合\(\AutoTuple{a}{n}\)的并\begin{equation*}
	a_1 \cup a_2 \cup \dotsb \cup a_n
\end{equation*}为“\(A\)的\DefineConcept{并}”,
记作\(\bigcup A\)或\(\bigcup_i a_i\),即\begin{equation*}
	\bigcup A
	\defeq
	a_1 \cup a_2 \cup \dotsb \cup a_n
	= \Set*{ x \given (\exists a \in A)[x \in a] }.
\end{equation*}
\end{definition}

为了确保“系的并”这样的集合存在,我们需要改进并集公理,如下:
\begin{axiom}[并集公理II]
对于任意集合\(A\),总存在一个集合\(B\),
使得集合\(B\)中的元素恰好是集合\(A\)中的元素的元素,即\begin{equation*}
	(\forall x)[
		x \in B
		\iff
		(\exists b)[b \in A \implies x \in b]
	].
\end{equation*}
\end{axiom}

我们可以将\(\bigcup A\)的定义表述为如下形式:\begin{equation*}
	x \in \bigcup A
	\defiff
	(\exists b)[b \in A \implies x \in b].
\end{equation*}
原有的对于“集的并”的定义可以修改为\begin{equation*}
	a \cup b \defeq \bigcup\Set{a,b}.
\end{equation*}

\begin{example}
%@see: 《Elements of Set Theory》 P26 Example
如果\(b \in A\),那么必有\(b \subseteq \bigcup A\),
这是因为对于\(\forall x, \forall b\),总有\begin{equation*}
	x \in b \in A
	\implies
	x \in \bigcup A;
\end{equation*}
反之则不然,例如\begin{equation*}
	A = \Set{\Set{1,2,3},\Set{1,4}}
	\implies
	\bigcup A = \Set{1,2,3,4},
\end{equation*}
容易看出\(b = \Set{2,3,4} \subseteq \bigcup A\),但是\(b \notin A\).
\end{example}

\begin{example}\label{example:集合论.有序对各坐标的取值范围}
%@see: 《Elements of Set Theory》 P26 Example
%@see: 《Elements of Set Theory》 P38 Lemma 3D
如果有\(\Set{ \Set{x}, \Set{x,y} } \in A\),
考虑到\(\Set{x,y} \in \Set{ \Set{x}, \Set{x,y} }\),
那么有\begin{equation*}
	\Set{x,y} \in \bigcup A;
\end{equation*}
再考虑到\(x,y \in \Set{x,y}\),
进而有\begin{equation*}
	x,y \in \bigcup\bigcup A.
\end{equation*}
综上所述,我们有
\begin{equation}
	\Set{ \Set{x}, \Set{x,y} } \in A
	\implies
	x,y \in \bigcup\bigcup A.
\end{equation}
\end{example}

例如,
\begin{align*}
	&\bigcup\Set{a,b,c,d} = a \cup b \cup c \cup d. \\
	&\bigcup\Set{a} = a. \\
	&\bigcup\emptyset = \emptyset.
\end{align*}


\subsection{幂集}
\begin{definition}
由集合\(A\)的所有子集(包括空集和集合\(A\)本身)构成的集合,
称作“集合\(A\)的\DefineConcept{幂集}(power set)”,
记作\(\Powerset A\)或\(2^A\)或\(\mathcal{P}A\),
即\begin{equation*}
	\Powerset A
	\defeq
	\Set{ x \given x \subseteq A }.
\end{equation*}
\end{definition}

\begin{example}
%@see: 《Elements of Set Theory》 P26 Exercise 6(a).
证明:对任意集合\(A\),总有
\begin{equation}\label{equation:集合论.集的幂集的并等于集}
	\bigcup \Powerset A = A.
\end{equation}
\begin{proof}
不难得到
\begin{align*}
	x \in \bigcup \Powerset A
	&\iff
	(\exists b \in \Powerset A)
	[x \in b] \\
	&\iff
	(\exists b \subseteq A)
	[x \in b] \\
	&\iff
	x \in A.
	\qedhere
\end{align*}
\end{proof}
\end{example}

\subsection{子集公理}
\begin{axiom}[子集公理]
%@see: 《Elements of Set Theory》 P21 Subset Axiom
对于不涉及\(B\)的每一条命题公式\(\lambda\),命题\begin{equation*}
	(\forall A)(\exists B)(\forall x)
	[x \in B \iff x \in A \land \lambda]
\end{equation*}是一条公理.
\end{axiom}
子集公理有时候也称作\DefineConcept{分离公理},
它在英语中称作Subset Axioms,
在德语中称作Axiom der Aussonderung.
正如其名,该定理的作用就是从集合\(A\)中取出适合命题公式\(\lambda\)的元素,组合成新的集合\(B\).

可以看出,子集公理是保障我们采用“描述法”表示集合这一方法论的正当性的基础.
也就是说,\begin{equation*}
	(\forall A)(\exists B)(\forall x)
	[x \in B \iff x \in A \land \lambda]
	\quad\iff\quad
	B = \Set{ x \in A \given \lambda }.
\end{equation*}

\begin{example}
以空集为元素构成的集合\(\Set{\emptyset}\)不是空集,即\(\Set{\emptyset} \neq \emptyset\).
这是因为\(\emptyset \in \Set{\emptyset}\)但\(\emptyset \notin \emptyset\).
\end{example}

现在我们可以利用子集公理和罗素悖论的论据来证明%
“包括所有集合的类\(V\)本身不是一个集合”.
\begin{theorem}\label{theorem:集合论.以所有集合为元素组成的集合不存在}
%@see: 《Elements of Set Theory》 P22 Theorem 2A
以所有集合为元素组成的集合不存在.
\begin{proof}
设\(A\)是一个集合.
令\begin{equation*}
B = \Set{ x \in A \given x \notin x }.
\end{equation*}那么有\begin{equation*}
B \in B
\iff
B \in A \land B \notin B.
\eqno(1)
\end{equation*}

假设\(B \in A\),那么(1)式化为\begin{equation*}
B \in B \iff B \notin B,
\end{equation*}矛盾,故\(B \notin A\).

综上所述,我们总可构造出一个不属于\(A\)的集合\(B\),因此不存在“以所有集合为元素组成的集合”.
\end{proof}
\end{theorem}
有的人可能会对是否存在以其本身为元素的集合抱有疑问,我们将在后面证明这是不可能的.
而根据这个结论,在上面的证明中,集合\(B\)实际上与集合\(A\)完全相同.

利用子集定理,我们可以进一步定义子集、真子集、交集等概念.
\begin{definition}\label{definition:集合论.子集的定义}
设\(A\)、\(B\)是两个集合,
如果集合\(A\)的元素都是集合\(B\)的元素,
则称“\(A\)是\(B\)的\DefineConcept{子集}(\(A\) is a \emph{subset} of \(B\))”,
记作\(A \subseteq B\)\footnote{读作“\(A\)包含于\(B\)
(\(A\) is included in \(B\))”.},
或记作\(B \supseteq A\)\footnote{读作“\(B\)包含\(A\)
(\(B\) includes \(A\))”.},
即\begin{equation*}
	(\forall x \in A)[x \in B]
	\defiff
	A \subseteq B
	\defiff
	B \supseteq A.
\end{equation*}

若集合\(A \subseteq B\)且\(A \neq B\),
则称“\(A\)是\(B\)的\DefineConcept{真子集}%
(\(A\) is a \emph{proper subset} of \(B\))”,
记作\(A \subset B\)或\(B \supset A\),即\begin{equation*}
	A \subseteq B
	\land
	A \neq B
	\defiff
	A \subset B
	\defiff
	B \supset A.
\end{equation*}
\end{definition}

\begin{theorem}
任意集合都是其本身的子集,即\begin{equation*}
	(\forall A)[A \subseteq A].
\end{equation*}
\end{theorem}

\begin{theorem}
空集\(\emptyset\)是任何集合的子集,即\begin{equation*}
	(\forall A)[\emptyset \subseteq A];
\end{equation*}
还是任何非空集合的真子集,即\begin{equation*}
	(\forall A)[A \neq \emptyset \iff \emptyset \subset A].
\end{equation*}
\end{theorem}

在学习了从属关系(\(\in\))和包含关系(\(\subseteq\))以后,切莫将两者搞混.
在讨论\(A \in B\)是否成立时,我们是将\(A\)作为一个整体,看它是不是\(B\)中的一个元素.
在讨论\(A \subseteq B\)是否成立时,我们要将\(A\)打开,检查它里面的所有元素是不是都是\(B\)中的元素.

\begin{theorem}
如果集合\(A\)与集合\(B\)互为子集,那么集合\(A\)与集合\(B\)相等,即\begin{equation*}
	A \subseteq B \land B \subseteq A
	\iff
	A = B.
\end{equation*}
\end{theorem}

\begin{example}
\(\Powerset \emptyset = \{ \emptyset \},%
\Powerset \Set{ \emptyset } = \{ \emptyset, \{ \emptyset \} \}\).
\end{example}

\begin{example}
对于任意集合\(A,B\),试分析\(\Powerset(A-B)\)与\(\Powerset A - \Powerset B\)是否相等.
\begin{solution}
集合\(\Powerset(A-B)\)包含集合\(A-B\)的所有子集,
那么总有\(\emptyset\in\Powerset(A-B)\).
但是\(\emptyset\notin\Powerset A - \Powerset B\),
所以\(\Powerset(A-B) \neq \Powerset A - \Powerset B\).
\end{solution}
\end{example}

\begin{example}
求非空集合\(S\)的所有单元素子集.
\begin{solution}
我们知道\(\Powerset S\)中的元素是\(S\)的全部子集,
于是我们可以利用子集公理从中找出只有一个元素的集合,将它们重组为一个新的集合\begin{equation*}
	\Set{ a \in \Powerset S \given \text{\(a\)是单元集} }.
\end{equation*}
我们知道,单元素集中有且仅有一个元素,由此可知\begin{equation*}
	a \neq \emptyset
	\land
	\forall u,v \in a \bigl( u = v \bigr).
\end{equation*}
综上所述,非空集合\(S\)的所有单元素子集的集合为\begin{equation*}
	\Set*{ a \in \Powerset S \given a \neq \emptyset
	\land
	\forall u,v \in a \bigl( u = v \bigr) }.
\end{equation*}
\end{solution}
\end{example}


\subsection{交集}
\begin{definition}
称\begin{equation*}
	\Set{ x \given x \in A \land x \in B }
\end{equation*}为“\(A\)和\(B\)的\DefineConcept{交}(the \emph{intersection} of \(A\) and \(B\))”,
记作\(A \cap B\)或\(AB\).
\end{definition}

\begin{definition}
设\(A,B\)都是集合.
定义:\begin{equation*}
	A \cap B \neq \emptyset
	\defiff
	\text{$A$与$B$~\DefineConcept{相交}}.
\end{equation*}
\end{definition}

\begin{definition}
设\(A,B\)都是集合.
定义:\begin{equation*}
	A \cap B = \emptyset
	\defiff
	\text{$A$与$B$~\DefineConcept{互斥}}
	\defiff
	\text{$A$与$B$~\DefineConcept{不相交}}.
\end{equation*}
\end{definition}

\begin{theorem}\label{theorem:集合论.系的交的唯一存在性}
%@see: 《Elements of Set Theory》 P25 Theorem 2B
对于任意非空集合\(A\),存在唯一的集合\(B\),使得\begin{equation*}
	(\forall x)[x \in B \iff (\forall a)[a \in A \implies x \in a]].
\end{equation*}
\begin{proof}
先证集合\(B\)的存在性.
既然\(A\)是非空集,不妨取定\(c \in A\).
那么根据子集公理,存在集合\(B\),使得对于任意\(x\),都有\begin{equation*}
	x \in B
	\iff
	x \in c \land (\forall a)[a \in A \implies x \in a];
\end{equation*}
由于\(c\)是从\(A\)中任意取出的一个元素,所以\begin{equation*}
	(\forall a)[a \in A \implies x \in a]
	\implies
	x \in c,
\end{equation*}
那么就有\begin{equation*}
	x \in B
	\iff
	(\forall a)[a \in A \implies x \in a].
\end{equation*}

根据外延公理不难得出集合\(B\)的唯一性.
\end{proof}
\end{theorem}

\begin{example}
%@see: 《Elements of Set Theory》 P26 Example
因为\begin{equation*}
	\bigcap\Set{\Set{a},\Set{a,b}}
	= \Set{a}\cap\Set{a,b}
	= \Set{a},
\end{equation*}
所以\begin{equation*}
	\bigcup\bigcap\Set{\Set{a},\Set{a,b}}
	= \bigcup\Set{a}
	= a.
\end{equation*}

类似有\begin{equation*}
	\bigcap\bigcup\Set{\Set{a},\Set{a,b}}
	= \bigcap\Set{a,b}
	= a \cap b.
\end{equation*}
\end{example}

\begin{definition}
设系\(A = \Set{\AutoTuple{a}{n}}\).
称集合\(\AutoTuple{a}{n}\)的交\begin{equation*}
	a_1 \cap a_2 \cap \dotsb \cap a_n
\end{equation*}为\(A\)的\DefineConcept{交}\footnote{%
\cref{theorem:集合论.系的交的唯一存在性} 确保了\(\bigcap A\)可以定义为唯一存在的集合\(B\).
},
记作\(\bigcap A\)或\(\bigcap_i a_i\),即\begin{equation*}
	\bigcap A
	\defeq
	\Set*{ x \given (\forall a)[a \in A \implies x \in a] }.
\end{equation*}
\end{definition}

例如,
\begin{align*}
	&\bigcap\Set{a} = a. \\
	&\bigcap\Set{a,b} = a \cap b. \\
	&\bigcap\Set{a,b,c,d} = a \cap b \cap c \cap d.
\end{align*}

\begin{remark}
特别注意到“\(\bigcap\emptyset\)”是未定义的!
\end{remark}

\begin{example}
因为\(\bigcap\Set{ \Set{a}, \Set{a,b} } = \Set{a} \cap \Set{a,b} = \Set{a}\),
所以\begin{gather*}
	\bigcup \bigcap \Set{ \Set{a}, \Set{a,b} } = \bigcup \Set{a} = a; \\
	\bigcap \bigcup \Set{ \Set{a}, \Set{a,b} } = \bigcap \Set{a,b} = a \cap b.
\end{gather*}
\end{example}

\subsection{差集}
\begin{definition}
称集合\begin{equation*}
	\Set{ x \given x \in A \land x \notin B }
\end{equation*}为“\(A\)和\(B\)的\DefineConcept{差集}%
(the \emph{relative complement} of \(B\) in \(A\))”,
记作\(A - B\)或\(A \setminus B\).
\end{definition}

\begin{example}
证明:\(A - B \subseteq A\).
\begin{proof}
由定义有\(
	x \in (A - B)
	\iff  % 差集的定义
	x \in A \land x \notin B
	\implies  % 化简
	x \in A
\),
所以\(A - B \subseteq A\).
\end{proof}
\end{example}

\begin{example}
证明:当且仅当\(A \cap B = \emptyset\)时,成立\(A - B = A\).
\begin{proof}
首先证明“\(A \cap B = \emptyset \implies A - B = A\)”.
当\(A \cap B = \emptyset\)时,
显然\(
	x \in A - B
	\iff  % 定义
	x \in A \land x \notin B
	\implies  % 化简
	x \in A
\),
即\(A - B \subseteq A\);
同时还有\(
	x \in A
	\iff  % 同一律 (x \in B \lor x \notin B) = 1
	x \in A \land (x \in B \lor x \notin B)
	\iff  % 分配律
	(x \in A \land x \in B)
	\lor
	(x \in A \land x \notin B)
	\implies  % 同一律 (x \in A \land x \in B) = 0
	x \in A \land x \notin B
\),
即\(A \subseteq A - B\);
因此\(A = A - B\).

接着证明“\(A - B = A \implies A \cap B = \emptyset\)”.
当\(A - B = A\)时,
有\(
	(x \in A - B) \liff (x \in A)
	\iff  % 定义
	(x \in A \land x \notin B) \liff (x \in A)
	\iff  % 逻辑等值式 \cref{equation:数理逻辑.联结词等值式3}
	((x \in A \land x \notin B) \limp (x \in A))
	\land
	((x \in A) \limp (x \in A \land x \notin B))
	\iff  % 同一律 (x \in A \land x \notin B) \limp (x \in A) = 1
	(x \in A) \limp (x \in A \land x \notin B)
	\implies
	(x \in A \land x \in B) \limp (x \in A \land x \notin B \land x \in B)
	\iff  % 交集定义,空集公理
	(x \in A \cap B) \limp x \in \emptyset
	\iff
	A \cap B = \emptyset
\).
\end{proof}
\end{example}

\begin{example}
证明:\((A \cup C) - (B \cup C) = (A - B) - C\).
\begin{proof}
直接有\begin{align*}
	x \in (A \cup C) - (B \cup C)
	&\iff
	x \in (A \cup C)
	\land
	x \notin (B \cup C) \\
	&\iff
	(
		x \in A
		\lor
		x \in C
	)
	\land
	\neg(
		x \in B
		\lor
		x \in C
	) \\
	&\iff
	(
		x \in A
		\lor
		x \in C
	)
	\land
	(
		x \notin B
		\land
		x \notin C
	) \\
	&\iff
	(
		x \in A
		\land
		(
			x \notin B
			\land
			x \notin C
		)
	)
	\lor
	(
		x \in C
		\land
		(
			x \notin B
			\land
			x \notin C
		)
	) \\
	&\iff
	(
		x \in A
		\land
		x \notin B
	)
	\land
	x \notin C \\
	&\iff
	x \in (A - B)
	\land
	x \notin C \\
	&\iff
	x \in ((A - B) - C).
	\qedhere
\end{align*}
\end{proof}
\end{example}
\begin{remark}
由上例可知\((A \cup C) - (B \cup C) \subseteq A - B\).
\end{remark}

\begin{example}
证明:\((A \cap C) - (B \cap C) = (A - B) \cap C\).
\begin{proof}
直接有\begin{align*}
	x \in (A \cap C) - (B \cap C)
	&\iff
	x \in (A \cap C)
	\land
	x \notin (B \cap C) \\
	&\iff
	(x \in A \land x \in C)
	\land
	\neg(x \in B \land x \in C) \\
	&\iff
	(x \in A \land x \in C)
	\land
	(x \notin B \lor x \notin C) \\
	&\iff
	((x \in A \land x \in C) \land x \notin B)
	\lor
	((x \in A \land x \in C) \land x \notin C) \\
	&\iff
	(x \in A \land x \notin B) \land x \in C \\
	&\iff
	x \in (A - B) \land x \in C \\
	&\iff x \in (A - B) \cap C.
	\qedhere
\end{align*}
\end{proof}
\end{example}

\begin{example}
证明:\((A - B) \cap C = (A \cap C) - (B \cap C)\).
\begin{proof}
直接有\begin{align*}
	x \in ((A - B) \cap C)
	&\iff
	x \in (A - B)
	\land
	x \in C \\
	&\iff
	(
		x \in A
		\land
		x \notin B
	)
	\land
	x \in C \\
	&\iff
	(
		x \in A
		\land
		x \in C
		\land
		x \notin B
	)
	\lor
	(
		x \in A
		\land
		x \in C
		\land
		x \notin C
	) \\
	&\iff
	(
		x \in A
		\land
		x \in C
	)
	\land
	(
		x \notin B
		\lor
		x \notin C
	) \\
	&\iff
	(
		x \in A
		\land
		x \in C
	)
	\land
	\neg(
		x \in B
		\land
		x \in C
	) \\
	&\iff
	x \in (A \cap C)
	\land
	x \notin (B \cap C) \\
	&\iff
	x \in ((A \cap C) - (B \cap C)).
	\qedhere
\end{align*}
\end{proof}
\end{example}

\begin{example}
证明:\((A - B) \cup C \supseteq (A \cup C) - (B \cup C)\).
\begin{proof}
直接有\begin{align*}
	x \in ((A \cup C) - (B \cup C))
	&\iff
	x \in (A \cup C)
	\land
	x \notin (B \cup C) \\
	&\iff
	(
		x \in A
		\lor
		x \in C
	)
	\land
	\neg(
		x \in B
		\lor
		x \in C
	) \\
	&\iff
	(
		x \in A
		\lor
		x \in C
	)
	\land
	(
		x \notin B
		\land
		x \notin C
	) \\
	&\implies
	(
		x \in A
		\lor
		x \in C
	)
	\land
	(
		x \notin B
		\lor
		x \in C
	) \\
	&\iff
	(
		x \in A
		\land
		x \notin B
	)
	\lor
	x \in C \\
	&\iff
	x \in (A - B)
	\lor
	x \in C \\
	&\iff
	x \in ((A - B) \cup C).
	\qedhere
\end{align*}
\end{proof}
\end{example}

\begin{example}
举例说明:\((A - B) \cup C \neq (A \cup C) - (B \cup C)\).
\begin{solution}
取\(A = \{1,2,3\},
B = \{2,3\},
C = \{2\}\),
则\(A - B = \{1\},
(A - B) \cup C = \{1,2\},
A \cup C = A,				\allowbreak
B \cup C = B,
(A \cup C) - (B \cup C)
= A - B = \{1\}\),
显然\((A - B) \cup C \neq (A \cup C) - (B \cup C)\).
%@Mathematica: a = {1, 2, 3}
%@Mathematica: b = {2, 3}
%@Mathematica: c = {2}
%@Mathematica: Complement[a, b]
%@Mathematica: Complement[a, b] \[Union] c
%@Mathematica: a \[Union] c
%@Mathematica: b \[Union] c
%@Mathematica: Complement[a \[Union] c, b \[Union] c]
\end{solution}
\end{example}

\begin{definition}
设\(A,B\)都是集合.
称\begin{equation*}
	(A-B)\cup(B-A)
\end{equation*}为“\(A\)与\(B\)的\DefineConcept{对称差}(symmetric difference)”,
记作\(A \symdiff B\).
\end{definition}

\begin{theorem}
\(A \symdiff B = (A \cup B)-(A \cap B)\).
\begin{proof}
直接计算得
\begin{align*}
	x\in(A-B)\cup(B-A)
	&\iff x \in A-B \lor x \in B-A \\
	&\iff (x \in A \land x \notin B) \lor (x \in B \land x \notin A) \\
	&\iff (x \in A \lor (x \in B \land x \notin A)) \land (x \notin B \lor (x \in B \land x \notin A)) \\
	&\iff (x \in A \lor x \in B) \land (x \notin A \lor x \notin B) \\
	&\iff (x \in A \lor x \in B) \land \neg(x \in A \land x \in B) \\
	&\iff x \in A \cup B \land x \notin A \cap B \\
	&\iff x\in(A \cup B)-(A \cap B).
	\qedhere
\end{align*}
\end{proof}
\end{theorem}

\begin{example}
证明:\((A \cup C) \symdiff (B \cup C) = (A \symdiff B) - C\).
\begin{proof}
直接计算得\begin{align*}
	(A \cup C) \symdiff (B \cup C)
	&= ((A \cup C) \cup (B \cup C)) - ((A \cup C) \cap (B \cup C)) \\
	&= ((A \cup B) \cup C) - ((A \cap B) \cup C) \\
	&= ((A \cup B) - (A \cap B)) - C \\
	&= (A \symdiff B) - C.
	\qedhere
\end{align*}
\end{proof}
\end{example}
\begin{remark}
由上例可知\(
	(A \cup C) \symdiff (B \cup C)
	\subseteq
	A \symdiff B
	\subseteq
	(A \symdiff B) \cup C
\).
\end{remark}

\begin{example}
证明:\((A \cap C) \symdiff (B \cap C) = (A \symdiff B) \cap C\).
\begin{proof}
直接计算得\begin{align*}
	(A \cap C) \symdiff (B \cap C)
	&= ((A \cap C) \cup (B \cap C)) - ((A \cap C) \cap (B \cap C)) \\
	&= ((A \cup B) \cap C) - ((A \cap B) \cap C) \\
	&= ((A \cup B) - (A \cap B)) \cap C \\
	&= (A \symdiff B) \cap C.
	\qedhere
\end{align*}
\end{proof}
\end{example}

%@see: 《点集拓扑讲义(第四版)》(熊金城) P8 定义1.2.2
有时,我们研究某个问题限定在一个大的集合\(U\)中进行,所研究的其他集合都是\(U\)的子集;
此时我们称集合\(U\)为\DefineConcept{全集}(universe)或\DefineConcept{基本集}.
设集合\(A \subseteq U\),
则称\(U-A\)为“\(A\)的\DefineConcept{补集}”
或“\(A\)的\DefineConcept{余集}”,
记作\(\SetComplementaryL{A}[U]\),或\(\SetComplementaryC{A}[U]\).

\begin{example}
%@see: 《Elements of Set Theory》 P26 Exercise 6(b).
证明:对任意集合\(A\),总有
\begin{equation}\label{equation:集合论.系的并的幂集包含系}
	A \subseteq \Powerset \bigcup A;
\end{equation}
并找出使得\(A = \Powerset \bigcup A\)成立的条件.
\begin{proof}
因为对于\(\forall a\),
总有\begin{equation*}
	a \in A
	\implies
	a \subseteq \bigcup A
	\iff
	a \in \Powerset \bigcup A,
\end{equation*}
所以\(A \subseteq \Powerset \bigcup A\).

从上面的证明过程可以看出,
当\((\forall a)
[a \subseteq \bigcup A \iff a \in A]\)时,
就有\(A = \Powerset \bigcup A\)成立.

在\(a \subseteq \bigcup A \implies a \in A\)的前提下,
由于总有\(\emptyset \subseteq \bigcup A\)成立,
所以\(\emptyset \in A\)恒成立.
考虑当\(A = \Set{\emptyset}\)成立时,
必有\(\bigcup A = \bigcup\Set{\emptyset} = \emptyset\),
因此\(\Powerset \bigcup A = \Powerset \emptyset = \Set{\emptyset} = A\).
%TODO
\end{proof}
\end{example}
